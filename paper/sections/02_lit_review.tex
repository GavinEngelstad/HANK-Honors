This paper adds to a growing body of work that adds household heterogeneity and market incompleteness to workhorse New Keynesian models that have been used to inform governmental policy for decades \autocites{woodford2005interest}{smets2007shocks}. Specifically, it contributes to work examining transmission channels and business cycle dynamics within these models. 

HANK models have developed our understanding of the forces that affect the macroeconomy. Idiosyncratic household income risks give an additional motive for precautionary savings beyond the aggregate forces within representative agent models \autocites{mckay2016power}{auclert2020micro}{acharya2023optimal}. Heightened uncertainty from these risks explains parts of specific business cycle events, including the Great Recession \autocites{bayer2019precautionary}. Borrowing constrained households are more responsive to macroeconomic conditions, so their presence can exacerbate or dampen the consequences of macroeconomic shocks \autocites{bilbiie2020new}. After a monetary policy shock when heterogeneity is present, indirect, as opposed to direct, effects cause the aggregate household response to shocks \autocites{kaplan2018monetary}. In this paper, I extend the direct-indirect decomposition from \textcite{kaplan2018monetary} to all macroeconomic factors that directly effect household decisions. 

Cross-sectional variation in marginal propensity to consume plays an important role in HANK models. Transfers have a trickle-up effect since poorer households have a larger marginal propensity to consume \autocite{auclert2023trickling}. Wealthy households are self-insured against macroeconomic shocks, so the household response to the shock varies across the wealth distribution \autocite{gornemann2016doves}. Shocks have unequal effects on households since earnings, balance sheet, and interest rate exposure are not evenly distributed \autocite{auclert2019monetary}. Heterogeneous changes in savings behavior creates a ``redistribution channel'' that affects aggregates \autocite{auclert2019monetary}. My analysis examines the transmission channels for changes in decisions for households across the wealth distribution. I focus on the heterogeneous household outcomes that are driven by the different exposure channels.

Bayesian estimates for business cycles are similar for HANKs and representative agent models \autocites{smets2007shocks}{bayer2024shocks}. Investment, markups, and technology channels play the most important role in estimated business cycles \autocites{auclert2020micro}{bayer2024shocks}. Heterogeneous MPCs and precautionary motives drive economic outcomes in response to the estimates shocks \autocite{auclert2020micro}. The main obstacle to estimation is that the number of potential household states makes estimation slow. The estimation process in \textcite{bayer2024shocks} uses dimensionality reduction and parallelization to speed up the process and still takes days. Newer sequence-space methods make estimation much faster \autocite{auclert2021using}. Therefore, I use a sequence-space method in this paper to estimate a series of shocks, including a novel estimate of a government transfer shock.
