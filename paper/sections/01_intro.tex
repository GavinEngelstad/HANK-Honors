% The significant cross-sectional differences between households in the United States suggests business cycle fluctuations should have heterogeneous effects across households. At the same time, most macroeconomic research, both within representative agent and heterogeneous models, is interested in explaining changes in aggregates \autocites{smets2007shocks}{krusell1998income}{kaplan2018monetary}{auclert2019monetary}{mckay2016power}. I examine how business cycles affect households at different wealth and income levels. I also examine the important transmission channels for changes in behavior for different households. I find that wealthier and higher earning households are most affected by changes in the interest through precautionary saving mechanisms. Wages affect higher earning households through strong substitution effects, while direct transfers affect lower earning households through income effects.

% I analyze these effects within the framework of an estimated Heterogeneous Agent New Keynesian (HANK) model. HANKs add household heterogeneity to standard New Keynesian models that feature realistic price and market frictions \autocite{kaplan2018monetary}. Idiosyncratic changes in household states mean the model features incomplete markets and uninsurable risks that give households a strong precautionary motive that plays an important role in the economy \autocites{mckay2016power}{bayer2019precautionary}. My model follows the standard within HANK literature and features an idiosyncratic productivity process for households that determines their income potential \autocites{kaplan2018monetary}{mckay2016power}. Within the model, households decide how much to save, consume, and work, which determines their movements along the wealth distribution.

% To understand the effect of business cycles on the model, I perform a Bayesian estimation against US data from 1966 to 2019 of six macroeconomic shocks to the model: total factor productivity (TFP), price markups, government spending, monetary policy, government transfers to households, and tax progressivity. The first four shocks are chosen from representative agent literature as structural shocks to aggregates in the model \autocite{smets2007shocks}. The tax progressivity shock is unique to heterogeneous models and applies non-uniformly across the distribution of households \autocite{bayer2024shocks}. The household transfer shock does the opposite, applying a uniform shock to all households. My estimations suggest price markups and tax progressivity play the most important role explaining the changes in aggregate outcomes during business cycles.

% Given my estimates, I then examine the effect of business cycles on household decisions at different productivity and wealth levels. This goes one step farther than other Bayesian estimates of HANKs in the literature that typically focus on aggregates \autocites{auclert2020micro}{acharya2023estimating} or macro-level movements of the wealth distribution \autocites{bayer2024shocks}. I find that business cycle driven changes in household decisions are caused by a combination of price markups, tax progressivity, and household transfers. Price markups are especially important for low productivity and low wealth households while tax progressivity and transfers have larger effects on high productivity and high wealth households.

% Finally, I expand the direct-indirect effects decomposition from \textcite{kaplan2018monetary} to all factors that directly affect household decisions in my model: wages, interest rates, transfers, and taxes. Using this decomposition, I analyze which macroeconomic factors that directly play into household decisions are the most important for households at different locations along the wealth and productivity distributions. I find that interest rates are most important to higher income, net-saver households, which I attribute to precautionary saving factors. Wages are the most important for middle income households that display less precautionary saving but still have significant substitution effects as wages change. For low income households, I find that income effects dominate and transfers are the most important determinant of household decisions. Across all groups, there is comovement of aggregate variables that together cause opposite changes in household behavior that dampen each other's effects. 
