The significant cross-sectional differences between households in the United States suggests business cycle fluctuations should have heterogeneous effects across households. At the same time, most macroeconomic research, both within representative agent and heterogeneous models, is interested in explaining changes in aggregates \autocites{smets2007shocks}{krusell1998income}{kaplan2018monetary}{auclert2019monetary}{mckay2016power}. I examine how business cycles affect households at different wealth and income levels. I also examine the important transmission channels for changes in behavior for different households. I find the factors causing changes in consumption decisions vary substantially across the income distribution and the factors causing changes in savings decisions vary substantially across the wealth distribution.

I analyze these effects within the framework of an estimated Heterogeneous Agent New Keynesian (HANK) model. HANK models add household heterogeneity to standard New Keynesian models that feature price and market frictions \autocite{kaplan2018monetary}. The model features incomplete markets and uninsurable risks that give households a strong precautionary motive that plays an important role in the economy \autocites{mckay2016power}{bayer2019precautionary}. My model follows the standard with the HANK literature and features an idiosyncratic productivity process for households that determines their income \autocites{kaplan2018monetary}{mckay2016power}.

Within the model, households decide how much to consume and save, which determines their movements along the wealth distribution. In response to changes in aggregate macroeconomic conditions and individual incomes, households can adjust their decisions. A shock can affect both consumption and savings if it causes households to reallocate their income from consuming to saving or vice versa. It could also affect only one decision if it changes household income and the household responds by varying only their saving or consumption choice while keeping the other one fixed.

To understand the effect of business cycles on the model, I perform a Bayesian estimation with US data from 1966 to 2019. The model includes seven macroeconomic shocks to the model: total factor productivity (TFP), price markups, wage markups, government spending, monetary policy, government transfers to households, and tax progressivity. The first five shocks are chosen from the representative agent literature as structural shocks to aggregates in the model \autocite{smets2007shocks}. The tax progressivity shock is unique to heterogeneous agent models and applies non-uniformly across the distribution of households \autocite{bayer2024shocks}. The household transfer shock does the opposite, applying a uniform shock to the income of all households. This transfer shock is more impactful for lower-earning households, since the increase makes up a higher share of their income. My estimations suggest price and wage markups  as well as tax progressivity and transfers play the most important role explaining the changes in aggregate outcomes during business cycles.

Given my estimates, I then examine the effect of business cycles on household decisions at different productivity and wealth levels. This goes one step farther than other Bayesian estimates of HANKs in the literature that typically focus on aggregates \autocites{auclert2020micro}{acharya2023estimating} or macro-level movements of the wealth distribution \autocites{bayer2024shocks}. I find that all seven of the shocks are important causal factors for business cycle driven changes in household decisions. Price and wage markups are especially important for household consumption decisions, with wage markup shocks affecting lower income households the most and price markup shocks affecting higher income households the most. Monetary policy is most important for explaining higher wealth households savings decisions while government spending, tax progressivity, and transfers are the main drivers of variation in savings decisions for lower wealth households. 

Then, I expand the direct-indirect effects decomposition from \textcite{kaplan2018monetary} to all factors that directly affect household decisions in my model: labor supply, wages, interest rates, transfers, and taxes. Using this decomposition, I analyze which macroeconomic factors that directly play into household decisions are the most important for households at different locations along the wealth and productivity distributions. I find that interest rates are most important for high income households' saving and consumption decisions. Labor supply and transfers are the primary determinants of low and middle income household consumption decisions and the interest rate has the largest effects on their savings decisions.

Finally, I use a historical decomposition to analyze the important factors over time. Similar to the variance decompositions, the historical decomposition points to price and wage markups through their labor supply and transfer effects being the key drivers of household consumption decisions, with price markups being more important only for higher income households. Savings decisions are most affected by interest rates, markups, and taxes. Due to the progressive tax scheme in the model and inclusion of a tax progressivity shock, the tax channel especially highlights differences in how households across the wealth and income distribution make decisions.
