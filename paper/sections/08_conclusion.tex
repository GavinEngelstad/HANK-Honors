% This paper explores how cross-sectional household heterogeneity is associated with different household behavior during business cycles. I find that the business cycle determinants differ substantially at different points along the distribution for wealth and earnings potential (measured via productivity). In general, decisions for lower earning households are most affected by price markup shocks. In contrast, variation in higher earning household decisions is most explained through changes in transfers and tax progressivity. 

% I also examine the effects of different transmission channels for these shocks during business cycles. Looking at the relative importance of each aggregate economic factor that directly affect households decisions, I find that interest rates are most important for high-earning net-saver households. Median income households are most affected by wages, suggesting moderate earnings potential is high enough that substitution effects can dominate in response to macroeconomic changes. Transfers explain most of the variation in low earning households, meaning income effects dominate.

% This analysis focuses on changes to household decision rules at specific points on the steady-state wealth and income distribution. The results, therefore, should be interpreted as generalizations of household behavior at different points on the wealth and income distributions, not a specific household's response to business cycle shocks. This is especially important since components within the model (especially utility functions) do not perfectly represent any specific household. 

% My research suggests household responses to business cycles are very heterogeneous. Future work should explore these differences in behavior as a source of inequality. This is especially important given the important role business cycles play determining the levels of inequality \autocite{bayer2024shocks}.
