This paper explores how cross-sectional household heterogeneity is associated with different household behavior during business cycles. I find that the business cycle determinants differ substantially at different points along the distribution for wealth and earnings. In general, the factors driving consumption decisions vary substantially across the income distribution while household savings decisions vary most along the wealth distribution.

I also examine the effects of different transmission channels for these shocks during business cycles. Looking at the relative importance of each aggregate economic factor that directly affect households decisions, I find that interest rates are most important for high income households consumption and savings decisions. Low and median income households consumption decisions are most affected by changes in transfers and the labor supply and their savings decisions are most affected by changes in the interest rate. Factor comovement caused by macroeconomic shocks has clashing effects on low and middle income household consumption decisions and more consistent effects on high income household consumption decisions. In contrast, savings decisions have conflicting effects for high income households and more harmonious effects for low and middle income households.

Historical decompositions of the different shocks point to the 80s and Great Recession as key points where the effects on households from different shocks and factors flipped. In particular, before the 80s my historical decompositions suggest taxes increased consumption and saving for low income households and decreased consumption and saving for high income households. However, after the 80s the historical decompositions suggest taxes push low income households to consume and save less and high income households to consume and save more. This highlights the heterogeneity in how households are affected by macroeconomic shocks. More generally, since the historical decomposition was only fit to macro series and still replicates specific events, like the Reagan-era tax cuts, my estimates support the idea from \textcite{bayer2024shocks} and \textcite{iao2024estimating} that macrodata can be sufficient to perform accurate estimations of HANK models.

This analysis focuses on changes to household decision rules at specific points on the steady-state wealth and income distribution. Also, I used no microdata in my estimates. The results, therefore, should be interpreted as generalizations of household behavior at different points on the wealth and income distributions, not a specific household's response to business cycle shocks. This is especially important for the household paths in the historical decomposition, which are merely a simulation and should not be treated as true paths for the decision rules.

My research suggests household responses to business cycles are very heterogeneous. Future work should explore these differences in behavior as a source of inequality. This is especially important given the important role business cycles play determining the levels of inequality \autocite{bayer2024shocks}.
