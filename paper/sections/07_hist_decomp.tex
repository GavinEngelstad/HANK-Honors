The variance shares of the model give a good general idea of which factors contribute to changes in macroeconomic aggregates and household decisions, however they do not give any specific view of which shocks and factors have been important over time. To analyze this, I perform a historical decomposition of the shocks to the model using the same macroeconomic series in the Bayesian estimation.


\subsection{Decomposition Strategy}

To perform the historical decomposition, I use the process in \textcite{auclert2023estimating}. Using the deviation from trend in the observed data $d\mathbf{X}^\text{data}$ and the IRFs of the model $d \mathbf{X}$, I solve for a matrix of shocks $\boldsymbol{\epsilon}$ that create simulated paths for macroeconomic series $d \tilde{\mathbf{X}}$ to solve
\begin{align*}
    \min_{\boldsymbol{\epsilon}} \quad & \sum_{t = 0}^{T_\text{obs}} \left\| d\mathbf{X}_t^\text{data} - d \tilde{\mathbf{X}_t} \right\|^2 \\
    \text{subject to} \quad & d \tilde{\mathbf{X}}_t = \sum_{s = 0}^{T-1} d\mathbf{X}_s \boldsymbol{\epsilon}_{t-s}.
\end{align*}
Since I use seven data series to fit seven shocks that all have linearly independent IRFs, the sequences for shocks $\boldsymbol{\epsilon}$ when simulated $d \tilde{\mathbf{X}}_t$ perfectly match the data. Using the sequences for shocks in $\boldsymbol{\epsilon}$, the simulated $d \tilde{\mathbf{X}}_t$ can be decomposed as the sum of the effects from each individual shock. Figure \ref{fig:fit-hist-decomp} shows the decompositions for each of the fitted data series.

Then, I use the sequences of shocks to simulate the behavior of different households over time within the model. These series are pure simulation, and not fit to any microdata, so they should not be taken as true paths for the consumption and savings decisions for actual households. Therefore, I interpret these sequences more weakly to get insight into the factors affecting household decisions. In addition, I apply a moving average to better get at general trends and reduce noise.


\subsection{Historical Decompositions}

The historical decompositions for aggregate consumption and savings are presented in Figure \ref{fig:agg-hist-decomp}. Like in the variance decompositions, price and wage markup shocks have been the most important determinants of consumption. In contrast, many factors, including wage markups, monetary policy, and transfers, impact aggregate savings.

\begin{figure}[t]
    \centering
    \caption{Historical Decomposition: Household Aggregates}
    %% Creator: Matplotlib, PGF backend
%%
%% To include the figure in your LaTeX document, write
%%   \input{<filename>.pgf}
%%
%% Make sure the required packages are loaded in your preamble
%%   \usepackage{pgf}
%%
%% Also ensure that all the required font packages are loaded; for instance,
%% the lmodern package is sometimes necessary when using math font.
%%   \usepackage{lmodern}
%%
%% Figures using additional raster images can only be included by \input if
%% they are in the same directory as the main LaTeX file. For loading figures
%% from other directories you can use the `import` package
%%   \usepackage{import}
%%
%% and then include the figures with
%%   \import{<path to file>}{<filename>.pgf}
%%
%% Matplotlib used the following preamble
%%   \def\mathdefault#1{#1}
%%   \everymath=\expandafter{\the\everymath\displaystyle}
%%   
%%   \ifdefined\pdftexversion\else  % non-pdftex case.
%%     \usepackage{fontspec}
%%     \setmainfont{DejaVuSerif.ttf}[Path=\detokenize{/opt/miniconda3/lib/python3.12/site-packages/matplotlib/mpl-data/fonts/ttf/}]
%%     \setsansfont{DejaVuSans.ttf}[Path=\detokenize{/opt/miniconda3/lib/python3.12/site-packages/matplotlib/mpl-data/fonts/ttf/}]
%%     \setmonofont{DejaVuSansMono.ttf}[Path=\detokenize{/opt/miniconda3/lib/python3.12/site-packages/matplotlib/mpl-data/fonts/ttf/}]
%%   \fi
%%   \makeatletter\@ifpackageloaded{underscore}{}{\usepackage[strings]{underscore}}\makeatother
%%
\begingroup%
\makeatletter%
\begin{pgfpicture}%
\pgfpathrectangle{\pgfpointorigin}{\pgfqpoint{6.500000in}{3.000000in}}%
\pgfusepath{use as bounding box, clip}%
\begin{pgfscope}%
\pgfsetbuttcap%
\pgfsetmiterjoin%
\definecolor{currentfill}{rgb}{1.000000,1.000000,1.000000}%
\pgfsetfillcolor{currentfill}%
\pgfsetlinewidth{0.000000pt}%
\definecolor{currentstroke}{rgb}{1.000000,1.000000,1.000000}%
\pgfsetstrokecolor{currentstroke}%
\pgfsetdash{}{0pt}%
\pgfpathmoveto{\pgfqpoint{0.000000in}{0.000000in}}%
\pgfpathlineto{\pgfqpoint{6.500000in}{0.000000in}}%
\pgfpathlineto{\pgfqpoint{6.500000in}{3.000000in}}%
\pgfpathlineto{\pgfqpoint{0.000000in}{3.000000in}}%
\pgfpathlineto{\pgfqpoint{0.000000in}{0.000000in}}%
\pgfpathclose%
\pgfusepath{fill}%
\end{pgfscope}%
\begin{pgfscope}%
\pgfsetbuttcap%
\pgfsetmiterjoin%
\definecolor{currentfill}{rgb}{1.000000,1.000000,1.000000}%
\pgfsetfillcolor{currentfill}%
\pgfsetlinewidth{0.000000pt}%
\definecolor{currentstroke}{rgb}{0.000000,0.000000,0.000000}%
\pgfsetstrokecolor{currentstroke}%
\pgfsetstrokeopacity{0.000000}%
\pgfsetdash{}{0pt}%
\pgfpathmoveto{\pgfqpoint{0.697024in}{0.857143in}}%
\pgfpathlineto{\pgfqpoint{3.324127in}{0.857143in}}%
\pgfpathlineto{\pgfqpoint{3.324127in}{2.670576in}}%
\pgfpathlineto{\pgfqpoint{0.697024in}{2.670576in}}%
\pgfpathlineto{\pgfqpoint{0.697024in}{0.857143in}}%
\pgfpathclose%
\pgfusepath{fill}%
\end{pgfscope}%
\begin{pgfscope}%
\pgfpathrectangle{\pgfqpoint{0.697024in}{0.857143in}}{\pgfqpoint{2.627103in}{1.813434in}}%
\pgfusepath{clip}%
\pgfsetbuttcap%
\pgfsetroundjoin%
\definecolor{currentfill}{rgb}{0.827451,0.827451,0.827451}%
\pgfsetfillcolor{currentfill}%
\pgfsetfillopacity{0.500000}%
\pgfsetlinewidth{1.003750pt}%
\definecolor{currentstroke}{rgb}{0.827451,0.827451,0.827451}%
\pgfsetstrokecolor{currentstroke}%
\pgfsetstrokeopacity{0.500000}%
\pgfsetdash{}{0pt}%
\pgfpathmoveto{\pgfqpoint{0.966124in}{2.670576in}}%
\pgfpathlineto{\pgfqpoint{0.966124in}{0.857143in}}%
\pgfpathlineto{\pgfqpoint{0.977295in}{0.857143in}}%
\pgfpathlineto{\pgfqpoint{0.988465in}{0.857143in}}%
\pgfpathlineto{\pgfqpoint{0.999636in}{0.857143in}}%
\pgfpathlineto{\pgfqpoint{1.010807in}{0.857143in}}%
\pgfpathlineto{\pgfqpoint{1.010807in}{2.670576in}}%
\pgfpathlineto{\pgfqpoint{1.010807in}{2.670576in}}%
\pgfpathlineto{\pgfqpoint{0.999636in}{2.670576in}}%
\pgfpathlineto{\pgfqpoint{0.988465in}{2.670576in}}%
\pgfpathlineto{\pgfqpoint{0.977295in}{2.670576in}}%
\pgfpathlineto{\pgfqpoint{0.966124in}{2.670576in}}%
\pgfpathlineto{\pgfqpoint{0.966124in}{2.670576in}}%
\pgfpathclose%
\pgfusepath{stroke,fill}%
\end{pgfscope}%
\begin{pgfscope}%
\pgfpathrectangle{\pgfqpoint{0.697024in}{0.857143in}}{\pgfqpoint{2.627103in}{1.813434in}}%
\pgfusepath{clip}%
\pgfsetbuttcap%
\pgfsetroundjoin%
\definecolor{currentfill}{rgb}{0.827451,0.827451,0.827451}%
\pgfsetfillcolor{currentfill}%
\pgfsetfillopacity{0.500000}%
\pgfsetlinewidth{1.003750pt}%
\definecolor{currentstroke}{rgb}{0.827451,0.827451,0.827451}%
\pgfsetstrokecolor{currentstroke}%
\pgfsetstrokeopacity{0.500000}%
\pgfsetdash{}{0pt}%
\pgfpathmoveto{\pgfqpoint{1.144854in}{2.670576in}}%
\pgfpathlineto{\pgfqpoint{1.144854in}{0.857143in}}%
\pgfpathlineto{\pgfqpoint{1.156025in}{0.857143in}}%
\pgfpathlineto{\pgfqpoint{1.167195in}{0.857143in}}%
\pgfpathlineto{\pgfqpoint{1.178366in}{0.857143in}}%
\pgfpathlineto{\pgfqpoint{1.189536in}{0.857143in}}%
\pgfpathlineto{\pgfqpoint{1.200707in}{0.857143in}}%
\pgfpathlineto{\pgfqpoint{1.200707in}{2.670576in}}%
\pgfpathlineto{\pgfqpoint{1.200707in}{2.670576in}}%
\pgfpathlineto{\pgfqpoint{1.189536in}{2.670576in}}%
\pgfpathlineto{\pgfqpoint{1.178366in}{2.670576in}}%
\pgfpathlineto{\pgfqpoint{1.167195in}{2.670576in}}%
\pgfpathlineto{\pgfqpoint{1.156025in}{2.670576in}}%
\pgfpathlineto{\pgfqpoint{1.144854in}{2.670576in}}%
\pgfpathlineto{\pgfqpoint{1.144854in}{2.670576in}}%
\pgfpathclose%
\pgfusepath{stroke,fill}%
\end{pgfscope}%
\begin{pgfscope}%
\pgfpathrectangle{\pgfqpoint{0.697024in}{0.857143in}}{\pgfqpoint{2.627103in}{1.813434in}}%
\pgfusepath{clip}%
\pgfsetbuttcap%
\pgfsetroundjoin%
\definecolor{currentfill}{rgb}{0.827451,0.827451,0.827451}%
\pgfsetfillcolor{currentfill}%
\pgfsetfillopacity{0.500000}%
\pgfsetlinewidth{1.003750pt}%
\definecolor{currentstroke}{rgb}{0.827451,0.827451,0.827451}%
\pgfsetstrokecolor{currentstroke}%
\pgfsetstrokeopacity{0.500000}%
\pgfsetdash{}{0pt}%
\pgfpathmoveto{\pgfqpoint{1.424119in}{2.670576in}}%
\pgfpathlineto{\pgfqpoint{1.424119in}{0.857143in}}%
\pgfpathlineto{\pgfqpoint{1.435290in}{0.857143in}}%
\pgfpathlineto{\pgfqpoint{1.446460in}{0.857143in}}%
\pgfpathlineto{\pgfqpoint{1.446460in}{2.670576in}}%
\pgfpathlineto{\pgfqpoint{1.446460in}{2.670576in}}%
\pgfpathlineto{\pgfqpoint{1.435290in}{2.670576in}}%
\pgfpathlineto{\pgfqpoint{1.424119in}{2.670576in}}%
\pgfpathlineto{\pgfqpoint{1.424119in}{2.670576in}}%
\pgfpathclose%
\pgfusepath{stroke,fill}%
\end{pgfscope}%
\begin{pgfscope}%
\pgfpathrectangle{\pgfqpoint{0.697024in}{0.857143in}}{\pgfqpoint{2.627103in}{1.813434in}}%
\pgfusepath{clip}%
\pgfsetbuttcap%
\pgfsetroundjoin%
\definecolor{currentfill}{rgb}{0.827451,0.827451,0.827451}%
\pgfsetfillcolor{currentfill}%
\pgfsetfillopacity{0.500000}%
\pgfsetlinewidth{1.003750pt}%
\definecolor{currentstroke}{rgb}{0.827451,0.827451,0.827451}%
\pgfsetstrokecolor{currentstroke}%
\pgfsetstrokeopacity{0.500000}%
\pgfsetdash{}{0pt}%
\pgfpathmoveto{\pgfqpoint{1.491143in}{2.670576in}}%
\pgfpathlineto{\pgfqpoint{1.491143in}{0.857143in}}%
\pgfpathlineto{\pgfqpoint{1.502313in}{0.857143in}}%
\pgfpathlineto{\pgfqpoint{1.513484in}{0.857143in}}%
\pgfpathlineto{\pgfqpoint{1.524655in}{0.857143in}}%
\pgfpathlineto{\pgfqpoint{1.535825in}{0.857143in}}%
\pgfpathlineto{\pgfqpoint{1.546996in}{0.857143in}}%
\pgfpathlineto{\pgfqpoint{1.546996in}{2.670576in}}%
\pgfpathlineto{\pgfqpoint{1.546996in}{2.670576in}}%
\pgfpathlineto{\pgfqpoint{1.535825in}{2.670576in}}%
\pgfpathlineto{\pgfqpoint{1.524655in}{2.670576in}}%
\pgfpathlineto{\pgfqpoint{1.513484in}{2.670576in}}%
\pgfpathlineto{\pgfqpoint{1.502313in}{2.670576in}}%
\pgfpathlineto{\pgfqpoint{1.491143in}{2.670576in}}%
\pgfpathlineto{\pgfqpoint{1.491143in}{2.670576in}}%
\pgfpathclose%
\pgfusepath{stroke,fill}%
\end{pgfscope}%
\begin{pgfscope}%
\pgfpathrectangle{\pgfqpoint{0.697024in}{0.857143in}}{\pgfqpoint{2.627103in}{1.813434in}}%
\pgfusepath{clip}%
\pgfsetbuttcap%
\pgfsetroundjoin%
\definecolor{currentfill}{rgb}{0.827451,0.827451,0.827451}%
\pgfsetfillcolor{currentfill}%
\pgfsetfillopacity{0.500000}%
\pgfsetlinewidth{1.003750pt}%
\definecolor{currentstroke}{rgb}{0.827451,0.827451,0.827451}%
\pgfsetstrokecolor{currentstroke}%
\pgfsetstrokeopacity{0.500000}%
\pgfsetdash{}{0pt}%
\pgfpathmoveto{\pgfqpoint{1.893285in}{2.670576in}}%
\pgfpathlineto{\pgfqpoint{1.893285in}{0.857143in}}%
\pgfpathlineto{\pgfqpoint{1.904455in}{0.857143in}}%
\pgfpathlineto{\pgfqpoint{1.915626in}{0.857143in}}%
\pgfpathlineto{\pgfqpoint{1.915626in}{2.670576in}}%
\pgfpathlineto{\pgfqpoint{1.915626in}{2.670576in}}%
\pgfpathlineto{\pgfqpoint{1.904455in}{2.670576in}}%
\pgfpathlineto{\pgfqpoint{1.893285in}{2.670576in}}%
\pgfpathlineto{\pgfqpoint{1.893285in}{2.670576in}}%
\pgfpathclose%
\pgfusepath{stroke,fill}%
\end{pgfscope}%
\begin{pgfscope}%
\pgfpathrectangle{\pgfqpoint{0.697024in}{0.857143in}}{\pgfqpoint{2.627103in}{1.813434in}}%
\pgfusepath{clip}%
\pgfsetbuttcap%
\pgfsetroundjoin%
\definecolor{currentfill}{rgb}{0.827451,0.827451,0.827451}%
\pgfsetfillcolor{currentfill}%
\pgfsetfillopacity{0.500000}%
\pgfsetlinewidth{1.003750pt}%
\definecolor{currentstroke}{rgb}{0.827451,0.827451,0.827451}%
\pgfsetstrokecolor{currentstroke}%
\pgfsetstrokeopacity{0.500000}%
\pgfsetdash{}{0pt}%
\pgfpathmoveto{\pgfqpoint{2.362450in}{2.670576in}}%
\pgfpathlineto{\pgfqpoint{2.362450in}{0.857143in}}%
\pgfpathlineto{\pgfqpoint{2.373621in}{0.857143in}}%
\pgfpathlineto{\pgfqpoint{2.384791in}{0.857143in}}%
\pgfpathlineto{\pgfqpoint{2.395962in}{0.857143in}}%
\pgfpathlineto{\pgfqpoint{2.395962in}{2.670576in}}%
\pgfpathlineto{\pgfqpoint{2.395962in}{2.670576in}}%
\pgfpathlineto{\pgfqpoint{2.384791in}{2.670576in}}%
\pgfpathlineto{\pgfqpoint{2.373621in}{2.670576in}}%
\pgfpathlineto{\pgfqpoint{2.362450in}{2.670576in}}%
\pgfpathlineto{\pgfqpoint{2.362450in}{2.670576in}}%
\pgfpathclose%
\pgfusepath{stroke,fill}%
\end{pgfscope}%
\begin{pgfscope}%
\pgfpathrectangle{\pgfqpoint{0.697024in}{0.857143in}}{\pgfqpoint{2.627103in}{1.813434in}}%
\pgfusepath{clip}%
\pgfsetbuttcap%
\pgfsetroundjoin%
\definecolor{currentfill}{rgb}{0.827451,0.827451,0.827451}%
\pgfsetfillcolor{currentfill}%
\pgfsetfillopacity{0.500000}%
\pgfsetlinewidth{1.003750pt}%
\definecolor{currentstroke}{rgb}{0.827451,0.827451,0.827451}%
\pgfsetstrokecolor{currentstroke}%
\pgfsetstrokeopacity{0.500000}%
\pgfsetdash{}{0pt}%
\pgfpathmoveto{\pgfqpoint{2.664056in}{2.670576in}}%
\pgfpathlineto{\pgfqpoint{2.664056in}{0.857143in}}%
\pgfpathlineto{\pgfqpoint{2.675227in}{0.857143in}}%
\pgfpathlineto{\pgfqpoint{2.686398in}{0.857143in}}%
\pgfpathlineto{\pgfqpoint{2.697568in}{0.857143in}}%
\pgfpathlineto{\pgfqpoint{2.708739in}{0.857143in}}%
\pgfpathlineto{\pgfqpoint{2.719909in}{0.857143in}}%
\pgfpathlineto{\pgfqpoint{2.731080in}{0.857143in}}%
\pgfpathlineto{\pgfqpoint{2.731080in}{2.670576in}}%
\pgfpathlineto{\pgfqpoint{2.731080in}{2.670576in}}%
\pgfpathlineto{\pgfqpoint{2.719909in}{2.670576in}}%
\pgfpathlineto{\pgfqpoint{2.708739in}{2.670576in}}%
\pgfpathlineto{\pgfqpoint{2.697568in}{2.670576in}}%
\pgfpathlineto{\pgfqpoint{2.686398in}{2.670576in}}%
\pgfpathlineto{\pgfqpoint{2.675227in}{2.670576in}}%
\pgfpathlineto{\pgfqpoint{2.664056in}{2.670576in}}%
\pgfpathlineto{\pgfqpoint{2.664056in}{2.670576in}}%
\pgfpathclose%
\pgfusepath{stroke,fill}%
\end{pgfscope}%
\begin{pgfscope}%
\pgfpathrectangle{\pgfqpoint{0.697024in}{0.857143in}}{\pgfqpoint{2.627103in}{1.813434in}}%
\pgfusepath{clip}%
\pgfsetbuttcap%
\pgfsetroundjoin%
\definecolor{currentfill}{rgb}{0.827451,0.827451,0.827451}%
\pgfsetfillcolor{currentfill}%
\pgfsetfillopacity{0.500000}%
\pgfsetlinewidth{1.003750pt}%
\definecolor{currentstroke}{rgb}{0.827451,0.827451,0.827451}%
\pgfsetstrokecolor{currentstroke}%
\pgfsetstrokeopacity{0.500000}%
\pgfsetdash{}{0pt}%
\pgfpathmoveto{\pgfqpoint{3.200245in}{2.670576in}}%
\pgfpathlineto{\pgfqpoint{3.200245in}{0.857143in}}%
\pgfpathlineto{\pgfqpoint{3.200245in}{2.670576in}}%
\pgfpathlineto{\pgfqpoint{3.200245in}{2.670576in}}%
\pgfpathlineto{\pgfqpoint{3.200245in}{2.670576in}}%
\pgfpathclose%
\pgfusepath{stroke,fill}%
\end{pgfscope}%
\begin{pgfscope}%
\pgfpathrectangle{\pgfqpoint{0.697024in}{0.857143in}}{\pgfqpoint{2.627103in}{1.813434in}}%
\pgfusepath{clip}%
\pgfsetbuttcap%
\pgfsetmiterjoin%
\definecolor{currentfill}{rgb}{0.066899,0.263188,0.377594}%
\pgfsetfillcolor{currentfill}%
\pgfsetlinewidth{0.000000pt}%
\definecolor{currentstroke}{rgb}{0.000000,0.000000,0.000000}%
\pgfsetstrokecolor{currentstroke}%
\pgfsetstrokeopacity{0.000000}%
\pgfsetdash{}{0pt}%
\pgfpathmoveto{\pgfqpoint{0.816438in}{1.847071in}}%
\pgfpathlineto{\pgfqpoint{0.825375in}{1.847071in}}%
\pgfpathlineto{\pgfqpoint{0.825375in}{1.834354in}}%
\pgfpathlineto{\pgfqpoint{0.816438in}{1.834354in}}%
\pgfpathlineto{\pgfqpoint{0.816438in}{1.847071in}}%
\pgfpathclose%
\pgfusepath{fill}%
\end{pgfscope}%
\begin{pgfscope}%
\pgfpathrectangle{\pgfqpoint{0.697024in}{0.857143in}}{\pgfqpoint{2.627103in}{1.813434in}}%
\pgfusepath{clip}%
\pgfsetbuttcap%
\pgfsetmiterjoin%
\definecolor{currentfill}{rgb}{0.066899,0.263188,0.377594}%
\pgfsetfillcolor{currentfill}%
\pgfsetlinewidth{0.000000pt}%
\definecolor{currentstroke}{rgb}{0.000000,0.000000,0.000000}%
\pgfsetstrokecolor{currentstroke}%
\pgfsetstrokeopacity{0.000000}%
\pgfsetdash{}{0pt}%
\pgfpathmoveto{\pgfqpoint{0.827609in}{1.847071in}}%
\pgfpathlineto{\pgfqpoint{0.836545in}{1.847071in}}%
\pgfpathlineto{\pgfqpoint{0.836545in}{1.855665in}}%
\pgfpathlineto{\pgfqpoint{0.827609in}{1.855665in}}%
\pgfpathlineto{\pgfqpoint{0.827609in}{1.847071in}}%
\pgfpathclose%
\pgfusepath{fill}%
\end{pgfscope}%
\begin{pgfscope}%
\pgfpathrectangle{\pgfqpoint{0.697024in}{0.857143in}}{\pgfqpoint{2.627103in}{1.813434in}}%
\pgfusepath{clip}%
\pgfsetbuttcap%
\pgfsetmiterjoin%
\definecolor{currentfill}{rgb}{0.066899,0.263188,0.377594}%
\pgfsetfillcolor{currentfill}%
\pgfsetlinewidth{0.000000pt}%
\definecolor{currentstroke}{rgb}{0.000000,0.000000,0.000000}%
\pgfsetstrokecolor{currentstroke}%
\pgfsetstrokeopacity{0.000000}%
\pgfsetdash{}{0pt}%
\pgfpathmoveto{\pgfqpoint{0.838779in}{1.847071in}}%
\pgfpathlineto{\pgfqpoint{0.847716in}{1.847071in}}%
\pgfpathlineto{\pgfqpoint{0.847716in}{1.871273in}}%
\pgfpathlineto{\pgfqpoint{0.838779in}{1.871273in}}%
\pgfpathlineto{\pgfqpoint{0.838779in}{1.847071in}}%
\pgfpathclose%
\pgfusepath{fill}%
\end{pgfscope}%
\begin{pgfscope}%
\pgfpathrectangle{\pgfqpoint{0.697024in}{0.857143in}}{\pgfqpoint{2.627103in}{1.813434in}}%
\pgfusepath{clip}%
\pgfsetbuttcap%
\pgfsetmiterjoin%
\definecolor{currentfill}{rgb}{0.066899,0.263188,0.377594}%
\pgfsetfillcolor{currentfill}%
\pgfsetlinewidth{0.000000pt}%
\definecolor{currentstroke}{rgb}{0.000000,0.000000,0.000000}%
\pgfsetstrokecolor{currentstroke}%
\pgfsetstrokeopacity{0.000000}%
\pgfsetdash{}{0pt}%
\pgfpathmoveto{\pgfqpoint{0.849950in}{1.847071in}}%
\pgfpathlineto{\pgfqpoint{0.858886in}{1.847071in}}%
\pgfpathlineto{\pgfqpoint{0.858886in}{1.876187in}}%
\pgfpathlineto{\pgfqpoint{0.849950in}{1.876187in}}%
\pgfpathlineto{\pgfqpoint{0.849950in}{1.847071in}}%
\pgfpathclose%
\pgfusepath{fill}%
\end{pgfscope}%
\begin{pgfscope}%
\pgfpathrectangle{\pgfqpoint{0.697024in}{0.857143in}}{\pgfqpoint{2.627103in}{1.813434in}}%
\pgfusepath{clip}%
\pgfsetbuttcap%
\pgfsetmiterjoin%
\definecolor{currentfill}{rgb}{0.066899,0.263188,0.377594}%
\pgfsetfillcolor{currentfill}%
\pgfsetlinewidth{0.000000pt}%
\definecolor{currentstroke}{rgb}{0.000000,0.000000,0.000000}%
\pgfsetstrokecolor{currentstroke}%
\pgfsetstrokeopacity{0.000000}%
\pgfsetdash{}{0pt}%
\pgfpathmoveto{\pgfqpoint{0.861121in}{1.847071in}}%
\pgfpathlineto{\pgfqpoint{0.870057in}{1.847071in}}%
\pgfpathlineto{\pgfqpoint{0.870057in}{1.880253in}}%
\pgfpathlineto{\pgfqpoint{0.861121in}{1.880253in}}%
\pgfpathlineto{\pgfqpoint{0.861121in}{1.847071in}}%
\pgfpathclose%
\pgfusepath{fill}%
\end{pgfscope}%
\begin{pgfscope}%
\pgfpathrectangle{\pgfqpoint{0.697024in}{0.857143in}}{\pgfqpoint{2.627103in}{1.813434in}}%
\pgfusepath{clip}%
\pgfsetbuttcap%
\pgfsetmiterjoin%
\definecolor{currentfill}{rgb}{0.066899,0.263188,0.377594}%
\pgfsetfillcolor{currentfill}%
\pgfsetlinewidth{0.000000pt}%
\definecolor{currentstroke}{rgb}{0.000000,0.000000,0.000000}%
\pgfsetstrokecolor{currentstroke}%
\pgfsetstrokeopacity{0.000000}%
\pgfsetdash{}{0pt}%
\pgfpathmoveto{\pgfqpoint{0.872291in}{1.847071in}}%
\pgfpathlineto{\pgfqpoint{0.881228in}{1.847071in}}%
\pgfpathlineto{\pgfqpoint{0.881228in}{1.875523in}}%
\pgfpathlineto{\pgfqpoint{0.872291in}{1.875523in}}%
\pgfpathlineto{\pgfqpoint{0.872291in}{1.847071in}}%
\pgfpathclose%
\pgfusepath{fill}%
\end{pgfscope}%
\begin{pgfscope}%
\pgfpathrectangle{\pgfqpoint{0.697024in}{0.857143in}}{\pgfqpoint{2.627103in}{1.813434in}}%
\pgfusepath{clip}%
\pgfsetbuttcap%
\pgfsetmiterjoin%
\definecolor{currentfill}{rgb}{0.066899,0.263188,0.377594}%
\pgfsetfillcolor{currentfill}%
\pgfsetlinewidth{0.000000pt}%
\definecolor{currentstroke}{rgb}{0.000000,0.000000,0.000000}%
\pgfsetstrokecolor{currentstroke}%
\pgfsetstrokeopacity{0.000000}%
\pgfsetdash{}{0pt}%
\pgfpathmoveto{\pgfqpoint{0.883462in}{1.847071in}}%
\pgfpathlineto{\pgfqpoint{0.892398in}{1.847071in}}%
\pgfpathlineto{\pgfqpoint{0.892398in}{1.918564in}}%
\pgfpathlineto{\pgfqpoint{0.883462in}{1.918564in}}%
\pgfpathlineto{\pgfqpoint{0.883462in}{1.847071in}}%
\pgfpathclose%
\pgfusepath{fill}%
\end{pgfscope}%
\begin{pgfscope}%
\pgfpathrectangle{\pgfqpoint{0.697024in}{0.857143in}}{\pgfqpoint{2.627103in}{1.813434in}}%
\pgfusepath{clip}%
\pgfsetbuttcap%
\pgfsetmiterjoin%
\definecolor{currentfill}{rgb}{0.066899,0.263188,0.377594}%
\pgfsetfillcolor{currentfill}%
\pgfsetlinewidth{0.000000pt}%
\definecolor{currentstroke}{rgb}{0.000000,0.000000,0.000000}%
\pgfsetstrokecolor{currentstroke}%
\pgfsetstrokeopacity{0.000000}%
\pgfsetdash{}{0pt}%
\pgfpathmoveto{\pgfqpoint{0.894632in}{1.847071in}}%
\pgfpathlineto{\pgfqpoint{0.903569in}{1.847071in}}%
\pgfpathlineto{\pgfqpoint{0.903569in}{1.922325in}}%
\pgfpathlineto{\pgfqpoint{0.894632in}{1.922325in}}%
\pgfpathlineto{\pgfqpoint{0.894632in}{1.847071in}}%
\pgfpathclose%
\pgfusepath{fill}%
\end{pgfscope}%
\begin{pgfscope}%
\pgfpathrectangle{\pgfqpoint{0.697024in}{0.857143in}}{\pgfqpoint{2.627103in}{1.813434in}}%
\pgfusepath{clip}%
\pgfsetbuttcap%
\pgfsetmiterjoin%
\definecolor{currentfill}{rgb}{0.066899,0.263188,0.377594}%
\pgfsetfillcolor{currentfill}%
\pgfsetlinewidth{0.000000pt}%
\definecolor{currentstroke}{rgb}{0.000000,0.000000,0.000000}%
\pgfsetstrokecolor{currentstroke}%
\pgfsetstrokeopacity{0.000000}%
\pgfsetdash{}{0pt}%
\pgfpathmoveto{\pgfqpoint{0.905803in}{1.847071in}}%
\pgfpathlineto{\pgfqpoint{0.914739in}{1.847071in}}%
\pgfpathlineto{\pgfqpoint{0.914739in}{1.909239in}}%
\pgfpathlineto{\pgfqpoint{0.905803in}{1.909239in}}%
\pgfpathlineto{\pgfqpoint{0.905803in}{1.847071in}}%
\pgfpathclose%
\pgfusepath{fill}%
\end{pgfscope}%
\begin{pgfscope}%
\pgfpathrectangle{\pgfqpoint{0.697024in}{0.857143in}}{\pgfqpoint{2.627103in}{1.813434in}}%
\pgfusepath{clip}%
\pgfsetbuttcap%
\pgfsetmiterjoin%
\definecolor{currentfill}{rgb}{0.066899,0.263188,0.377594}%
\pgfsetfillcolor{currentfill}%
\pgfsetlinewidth{0.000000pt}%
\definecolor{currentstroke}{rgb}{0.000000,0.000000,0.000000}%
\pgfsetstrokecolor{currentstroke}%
\pgfsetstrokeopacity{0.000000}%
\pgfsetdash{}{0pt}%
\pgfpathmoveto{\pgfqpoint{0.916974in}{1.847071in}}%
\pgfpathlineto{\pgfqpoint{0.925910in}{1.847071in}}%
\pgfpathlineto{\pgfqpoint{0.925910in}{1.890481in}}%
\pgfpathlineto{\pgfqpoint{0.916974in}{1.890481in}}%
\pgfpathlineto{\pgfqpoint{0.916974in}{1.847071in}}%
\pgfpathclose%
\pgfusepath{fill}%
\end{pgfscope}%
\begin{pgfscope}%
\pgfpathrectangle{\pgfqpoint{0.697024in}{0.857143in}}{\pgfqpoint{2.627103in}{1.813434in}}%
\pgfusepath{clip}%
\pgfsetbuttcap%
\pgfsetmiterjoin%
\definecolor{currentfill}{rgb}{0.066899,0.263188,0.377594}%
\pgfsetfillcolor{currentfill}%
\pgfsetlinewidth{0.000000pt}%
\definecolor{currentstroke}{rgb}{0.000000,0.000000,0.000000}%
\pgfsetstrokecolor{currentstroke}%
\pgfsetstrokeopacity{0.000000}%
\pgfsetdash{}{0pt}%
\pgfpathmoveto{\pgfqpoint{0.928144in}{1.847071in}}%
\pgfpathlineto{\pgfqpoint{0.937081in}{1.847071in}}%
\pgfpathlineto{\pgfqpoint{0.937081in}{1.899510in}}%
\pgfpathlineto{\pgfqpoint{0.928144in}{1.899510in}}%
\pgfpathlineto{\pgfqpoint{0.928144in}{1.847071in}}%
\pgfpathclose%
\pgfusepath{fill}%
\end{pgfscope}%
\begin{pgfscope}%
\pgfpathrectangle{\pgfqpoint{0.697024in}{0.857143in}}{\pgfqpoint{2.627103in}{1.813434in}}%
\pgfusepath{clip}%
\pgfsetbuttcap%
\pgfsetmiterjoin%
\definecolor{currentfill}{rgb}{0.066899,0.263188,0.377594}%
\pgfsetfillcolor{currentfill}%
\pgfsetlinewidth{0.000000pt}%
\definecolor{currentstroke}{rgb}{0.000000,0.000000,0.000000}%
\pgfsetstrokecolor{currentstroke}%
\pgfsetstrokeopacity{0.000000}%
\pgfsetdash{}{0pt}%
\pgfpathmoveto{\pgfqpoint{0.939315in}{1.847071in}}%
\pgfpathlineto{\pgfqpoint{0.948251in}{1.847071in}}%
\pgfpathlineto{\pgfqpoint{0.948251in}{1.869062in}}%
\pgfpathlineto{\pgfqpoint{0.939315in}{1.869062in}}%
\pgfpathlineto{\pgfqpoint{0.939315in}{1.847071in}}%
\pgfpathclose%
\pgfusepath{fill}%
\end{pgfscope}%
\begin{pgfscope}%
\pgfpathrectangle{\pgfqpoint{0.697024in}{0.857143in}}{\pgfqpoint{2.627103in}{1.813434in}}%
\pgfusepath{clip}%
\pgfsetbuttcap%
\pgfsetmiterjoin%
\definecolor{currentfill}{rgb}{0.066899,0.263188,0.377594}%
\pgfsetfillcolor{currentfill}%
\pgfsetlinewidth{0.000000pt}%
\definecolor{currentstroke}{rgb}{0.000000,0.000000,0.000000}%
\pgfsetstrokecolor{currentstroke}%
\pgfsetstrokeopacity{0.000000}%
\pgfsetdash{}{0pt}%
\pgfpathmoveto{\pgfqpoint{0.950485in}{1.847071in}}%
\pgfpathlineto{\pgfqpoint{0.959422in}{1.847071in}}%
\pgfpathlineto{\pgfqpoint{0.959422in}{1.865855in}}%
\pgfpathlineto{\pgfqpoint{0.950485in}{1.865855in}}%
\pgfpathlineto{\pgfqpoint{0.950485in}{1.847071in}}%
\pgfpathclose%
\pgfusepath{fill}%
\end{pgfscope}%
\begin{pgfscope}%
\pgfpathrectangle{\pgfqpoint{0.697024in}{0.857143in}}{\pgfqpoint{2.627103in}{1.813434in}}%
\pgfusepath{clip}%
\pgfsetbuttcap%
\pgfsetmiterjoin%
\definecolor{currentfill}{rgb}{0.066899,0.263188,0.377594}%
\pgfsetfillcolor{currentfill}%
\pgfsetlinewidth{0.000000pt}%
\definecolor{currentstroke}{rgb}{0.000000,0.000000,0.000000}%
\pgfsetstrokecolor{currentstroke}%
\pgfsetstrokeopacity{0.000000}%
\pgfsetdash{}{0pt}%
\pgfpathmoveto{\pgfqpoint{0.961656in}{1.847071in}}%
\pgfpathlineto{\pgfqpoint{0.970593in}{1.847071in}}%
\pgfpathlineto{\pgfqpoint{0.970593in}{1.850046in}}%
\pgfpathlineto{\pgfqpoint{0.961656in}{1.850046in}}%
\pgfpathlineto{\pgfqpoint{0.961656in}{1.847071in}}%
\pgfpathclose%
\pgfusepath{fill}%
\end{pgfscope}%
\begin{pgfscope}%
\pgfpathrectangle{\pgfqpoint{0.697024in}{0.857143in}}{\pgfqpoint{2.627103in}{1.813434in}}%
\pgfusepath{clip}%
\pgfsetbuttcap%
\pgfsetmiterjoin%
\definecolor{currentfill}{rgb}{0.066899,0.263188,0.377594}%
\pgfsetfillcolor{currentfill}%
\pgfsetlinewidth{0.000000pt}%
\definecolor{currentstroke}{rgb}{0.000000,0.000000,0.000000}%
\pgfsetstrokecolor{currentstroke}%
\pgfsetstrokeopacity{0.000000}%
\pgfsetdash{}{0pt}%
\pgfpathmoveto{\pgfqpoint{0.972827in}{1.847071in}}%
\pgfpathlineto{\pgfqpoint{0.981763in}{1.847071in}}%
\pgfpathlineto{\pgfqpoint{0.981763in}{1.848356in}}%
\pgfpathlineto{\pgfqpoint{0.972827in}{1.848356in}}%
\pgfpathlineto{\pgfqpoint{0.972827in}{1.847071in}}%
\pgfpathclose%
\pgfusepath{fill}%
\end{pgfscope}%
\begin{pgfscope}%
\pgfpathrectangle{\pgfqpoint{0.697024in}{0.857143in}}{\pgfqpoint{2.627103in}{1.813434in}}%
\pgfusepath{clip}%
\pgfsetbuttcap%
\pgfsetmiterjoin%
\definecolor{currentfill}{rgb}{0.066899,0.263188,0.377594}%
\pgfsetfillcolor{currentfill}%
\pgfsetlinewidth{0.000000pt}%
\definecolor{currentstroke}{rgb}{0.000000,0.000000,0.000000}%
\pgfsetstrokecolor{currentstroke}%
\pgfsetstrokeopacity{0.000000}%
\pgfsetdash{}{0pt}%
\pgfpathmoveto{\pgfqpoint{0.983997in}{1.847071in}}%
\pgfpathlineto{\pgfqpoint{0.992934in}{1.847071in}}%
\pgfpathlineto{\pgfqpoint{0.992934in}{1.881730in}}%
\pgfpathlineto{\pgfqpoint{0.983997in}{1.881730in}}%
\pgfpathlineto{\pgfqpoint{0.983997in}{1.847071in}}%
\pgfpathclose%
\pgfusepath{fill}%
\end{pgfscope}%
\begin{pgfscope}%
\pgfpathrectangle{\pgfqpoint{0.697024in}{0.857143in}}{\pgfqpoint{2.627103in}{1.813434in}}%
\pgfusepath{clip}%
\pgfsetbuttcap%
\pgfsetmiterjoin%
\definecolor{currentfill}{rgb}{0.066899,0.263188,0.377594}%
\pgfsetfillcolor{currentfill}%
\pgfsetlinewidth{0.000000pt}%
\definecolor{currentstroke}{rgb}{0.000000,0.000000,0.000000}%
\pgfsetstrokecolor{currentstroke}%
\pgfsetstrokeopacity{0.000000}%
\pgfsetdash{}{0pt}%
\pgfpathmoveto{\pgfqpoint{0.995168in}{1.847071in}}%
\pgfpathlineto{\pgfqpoint{1.004104in}{1.847071in}}%
\pgfpathlineto{\pgfqpoint{1.004104in}{1.908802in}}%
\pgfpathlineto{\pgfqpoint{0.995168in}{1.908802in}}%
\pgfpathlineto{\pgfqpoint{0.995168in}{1.847071in}}%
\pgfpathclose%
\pgfusepath{fill}%
\end{pgfscope}%
\begin{pgfscope}%
\pgfpathrectangle{\pgfqpoint{0.697024in}{0.857143in}}{\pgfqpoint{2.627103in}{1.813434in}}%
\pgfusepath{clip}%
\pgfsetbuttcap%
\pgfsetmiterjoin%
\definecolor{currentfill}{rgb}{0.066899,0.263188,0.377594}%
\pgfsetfillcolor{currentfill}%
\pgfsetlinewidth{0.000000pt}%
\definecolor{currentstroke}{rgb}{0.000000,0.000000,0.000000}%
\pgfsetstrokecolor{currentstroke}%
\pgfsetstrokeopacity{0.000000}%
\pgfsetdash{}{0pt}%
\pgfpathmoveto{\pgfqpoint{1.006338in}{1.847071in}}%
\pgfpathlineto{\pgfqpoint{1.015275in}{1.847071in}}%
\pgfpathlineto{\pgfqpoint{1.015275in}{1.882192in}}%
\pgfpathlineto{\pgfqpoint{1.006338in}{1.882192in}}%
\pgfpathlineto{\pgfqpoint{1.006338in}{1.847071in}}%
\pgfpathclose%
\pgfusepath{fill}%
\end{pgfscope}%
\begin{pgfscope}%
\pgfpathrectangle{\pgfqpoint{0.697024in}{0.857143in}}{\pgfqpoint{2.627103in}{1.813434in}}%
\pgfusepath{clip}%
\pgfsetbuttcap%
\pgfsetmiterjoin%
\definecolor{currentfill}{rgb}{0.066899,0.263188,0.377594}%
\pgfsetfillcolor{currentfill}%
\pgfsetlinewidth{0.000000pt}%
\definecolor{currentstroke}{rgb}{0.000000,0.000000,0.000000}%
\pgfsetstrokecolor{currentstroke}%
\pgfsetstrokeopacity{0.000000}%
\pgfsetdash{}{0pt}%
\pgfpathmoveto{\pgfqpoint{1.017509in}{1.847071in}}%
\pgfpathlineto{\pgfqpoint{1.026446in}{1.847071in}}%
\pgfpathlineto{\pgfqpoint{1.026446in}{1.937559in}}%
\pgfpathlineto{\pgfqpoint{1.017509in}{1.937559in}}%
\pgfpathlineto{\pgfqpoint{1.017509in}{1.847071in}}%
\pgfpathclose%
\pgfusepath{fill}%
\end{pgfscope}%
\begin{pgfscope}%
\pgfpathrectangle{\pgfqpoint{0.697024in}{0.857143in}}{\pgfqpoint{2.627103in}{1.813434in}}%
\pgfusepath{clip}%
\pgfsetbuttcap%
\pgfsetmiterjoin%
\definecolor{currentfill}{rgb}{0.066899,0.263188,0.377594}%
\pgfsetfillcolor{currentfill}%
\pgfsetlinewidth{0.000000pt}%
\definecolor{currentstroke}{rgb}{0.000000,0.000000,0.000000}%
\pgfsetstrokecolor{currentstroke}%
\pgfsetstrokeopacity{0.000000}%
\pgfsetdash{}{0pt}%
\pgfpathmoveto{\pgfqpoint{1.028680in}{1.847071in}}%
\pgfpathlineto{\pgfqpoint{1.037616in}{1.847071in}}%
\pgfpathlineto{\pgfqpoint{1.037616in}{1.925336in}}%
\pgfpathlineto{\pgfqpoint{1.028680in}{1.925336in}}%
\pgfpathlineto{\pgfqpoint{1.028680in}{1.847071in}}%
\pgfpathclose%
\pgfusepath{fill}%
\end{pgfscope}%
\begin{pgfscope}%
\pgfpathrectangle{\pgfqpoint{0.697024in}{0.857143in}}{\pgfqpoint{2.627103in}{1.813434in}}%
\pgfusepath{clip}%
\pgfsetbuttcap%
\pgfsetmiterjoin%
\definecolor{currentfill}{rgb}{0.066899,0.263188,0.377594}%
\pgfsetfillcolor{currentfill}%
\pgfsetlinewidth{0.000000pt}%
\definecolor{currentstroke}{rgb}{0.000000,0.000000,0.000000}%
\pgfsetstrokecolor{currentstroke}%
\pgfsetstrokeopacity{0.000000}%
\pgfsetdash{}{0pt}%
\pgfpathmoveto{\pgfqpoint{1.039850in}{1.847071in}}%
\pgfpathlineto{\pgfqpoint{1.048787in}{1.847071in}}%
\pgfpathlineto{\pgfqpoint{1.048787in}{1.936114in}}%
\pgfpathlineto{\pgfqpoint{1.039850in}{1.936114in}}%
\pgfpathlineto{\pgfqpoint{1.039850in}{1.847071in}}%
\pgfpathclose%
\pgfusepath{fill}%
\end{pgfscope}%
\begin{pgfscope}%
\pgfpathrectangle{\pgfqpoint{0.697024in}{0.857143in}}{\pgfqpoint{2.627103in}{1.813434in}}%
\pgfusepath{clip}%
\pgfsetbuttcap%
\pgfsetmiterjoin%
\definecolor{currentfill}{rgb}{0.066899,0.263188,0.377594}%
\pgfsetfillcolor{currentfill}%
\pgfsetlinewidth{0.000000pt}%
\definecolor{currentstroke}{rgb}{0.000000,0.000000,0.000000}%
\pgfsetstrokecolor{currentstroke}%
\pgfsetstrokeopacity{0.000000}%
\pgfsetdash{}{0pt}%
\pgfpathmoveto{\pgfqpoint{1.051021in}{1.847071in}}%
\pgfpathlineto{\pgfqpoint{1.059957in}{1.847071in}}%
\pgfpathlineto{\pgfqpoint{1.059957in}{1.897086in}}%
\pgfpathlineto{\pgfqpoint{1.051021in}{1.897086in}}%
\pgfpathlineto{\pgfqpoint{1.051021in}{1.847071in}}%
\pgfpathclose%
\pgfusepath{fill}%
\end{pgfscope}%
\begin{pgfscope}%
\pgfpathrectangle{\pgfqpoint{0.697024in}{0.857143in}}{\pgfqpoint{2.627103in}{1.813434in}}%
\pgfusepath{clip}%
\pgfsetbuttcap%
\pgfsetmiterjoin%
\definecolor{currentfill}{rgb}{0.066899,0.263188,0.377594}%
\pgfsetfillcolor{currentfill}%
\pgfsetlinewidth{0.000000pt}%
\definecolor{currentstroke}{rgb}{0.000000,0.000000,0.000000}%
\pgfsetstrokecolor{currentstroke}%
\pgfsetstrokeopacity{0.000000}%
\pgfsetdash{}{0pt}%
\pgfpathmoveto{\pgfqpoint{1.062191in}{1.847071in}}%
\pgfpathlineto{\pgfqpoint{1.071128in}{1.847071in}}%
\pgfpathlineto{\pgfqpoint{1.071128in}{1.909754in}}%
\pgfpathlineto{\pgfqpoint{1.062191in}{1.909754in}}%
\pgfpathlineto{\pgfqpoint{1.062191in}{1.847071in}}%
\pgfpathclose%
\pgfusepath{fill}%
\end{pgfscope}%
\begin{pgfscope}%
\pgfpathrectangle{\pgfqpoint{0.697024in}{0.857143in}}{\pgfqpoint{2.627103in}{1.813434in}}%
\pgfusepath{clip}%
\pgfsetbuttcap%
\pgfsetmiterjoin%
\definecolor{currentfill}{rgb}{0.066899,0.263188,0.377594}%
\pgfsetfillcolor{currentfill}%
\pgfsetlinewidth{0.000000pt}%
\definecolor{currentstroke}{rgb}{0.000000,0.000000,0.000000}%
\pgfsetstrokecolor{currentstroke}%
\pgfsetstrokeopacity{0.000000}%
\pgfsetdash{}{0pt}%
\pgfpathmoveto{\pgfqpoint{1.073362in}{1.847071in}}%
\pgfpathlineto{\pgfqpoint{1.082299in}{1.847071in}}%
\pgfpathlineto{\pgfqpoint{1.082299in}{1.943778in}}%
\pgfpathlineto{\pgfqpoint{1.073362in}{1.943778in}}%
\pgfpathlineto{\pgfqpoint{1.073362in}{1.847071in}}%
\pgfpathclose%
\pgfusepath{fill}%
\end{pgfscope}%
\begin{pgfscope}%
\pgfpathrectangle{\pgfqpoint{0.697024in}{0.857143in}}{\pgfqpoint{2.627103in}{1.813434in}}%
\pgfusepath{clip}%
\pgfsetbuttcap%
\pgfsetmiterjoin%
\definecolor{currentfill}{rgb}{0.066899,0.263188,0.377594}%
\pgfsetfillcolor{currentfill}%
\pgfsetlinewidth{0.000000pt}%
\definecolor{currentstroke}{rgb}{0.000000,0.000000,0.000000}%
\pgfsetstrokecolor{currentstroke}%
\pgfsetstrokeopacity{0.000000}%
\pgfsetdash{}{0pt}%
\pgfpathmoveto{\pgfqpoint{1.084533in}{1.847071in}}%
\pgfpathlineto{\pgfqpoint{1.093469in}{1.847071in}}%
\pgfpathlineto{\pgfqpoint{1.093469in}{1.939020in}}%
\pgfpathlineto{\pgfqpoint{1.084533in}{1.939020in}}%
\pgfpathlineto{\pgfqpoint{1.084533in}{1.847071in}}%
\pgfpathclose%
\pgfusepath{fill}%
\end{pgfscope}%
\begin{pgfscope}%
\pgfpathrectangle{\pgfqpoint{0.697024in}{0.857143in}}{\pgfqpoint{2.627103in}{1.813434in}}%
\pgfusepath{clip}%
\pgfsetbuttcap%
\pgfsetmiterjoin%
\definecolor{currentfill}{rgb}{0.066899,0.263188,0.377594}%
\pgfsetfillcolor{currentfill}%
\pgfsetlinewidth{0.000000pt}%
\definecolor{currentstroke}{rgb}{0.000000,0.000000,0.000000}%
\pgfsetstrokecolor{currentstroke}%
\pgfsetstrokeopacity{0.000000}%
\pgfsetdash{}{0pt}%
\pgfpathmoveto{\pgfqpoint{1.095703in}{1.847071in}}%
\pgfpathlineto{\pgfqpoint{1.104640in}{1.847071in}}%
\pgfpathlineto{\pgfqpoint{1.104640in}{1.944085in}}%
\pgfpathlineto{\pgfqpoint{1.095703in}{1.944085in}}%
\pgfpathlineto{\pgfqpoint{1.095703in}{1.847071in}}%
\pgfpathclose%
\pgfusepath{fill}%
\end{pgfscope}%
\begin{pgfscope}%
\pgfpathrectangle{\pgfqpoint{0.697024in}{0.857143in}}{\pgfqpoint{2.627103in}{1.813434in}}%
\pgfusepath{clip}%
\pgfsetbuttcap%
\pgfsetmiterjoin%
\definecolor{currentfill}{rgb}{0.066899,0.263188,0.377594}%
\pgfsetfillcolor{currentfill}%
\pgfsetlinewidth{0.000000pt}%
\definecolor{currentstroke}{rgb}{0.000000,0.000000,0.000000}%
\pgfsetstrokecolor{currentstroke}%
\pgfsetstrokeopacity{0.000000}%
\pgfsetdash{}{0pt}%
\pgfpathmoveto{\pgfqpoint{1.106874in}{1.847071in}}%
\pgfpathlineto{\pgfqpoint{1.115810in}{1.847071in}}%
\pgfpathlineto{\pgfqpoint{1.115810in}{1.959548in}}%
\pgfpathlineto{\pgfqpoint{1.106874in}{1.959548in}}%
\pgfpathlineto{\pgfqpoint{1.106874in}{1.847071in}}%
\pgfpathclose%
\pgfusepath{fill}%
\end{pgfscope}%
\begin{pgfscope}%
\pgfpathrectangle{\pgfqpoint{0.697024in}{0.857143in}}{\pgfqpoint{2.627103in}{1.813434in}}%
\pgfusepath{clip}%
\pgfsetbuttcap%
\pgfsetmiterjoin%
\definecolor{currentfill}{rgb}{0.066899,0.263188,0.377594}%
\pgfsetfillcolor{currentfill}%
\pgfsetlinewidth{0.000000pt}%
\definecolor{currentstroke}{rgb}{0.000000,0.000000,0.000000}%
\pgfsetstrokecolor{currentstroke}%
\pgfsetstrokeopacity{0.000000}%
\pgfsetdash{}{0pt}%
\pgfpathmoveto{\pgfqpoint{1.118045in}{1.847071in}}%
\pgfpathlineto{\pgfqpoint{1.126981in}{1.847071in}}%
\pgfpathlineto{\pgfqpoint{1.126981in}{1.946302in}}%
\pgfpathlineto{\pgfqpoint{1.118045in}{1.946302in}}%
\pgfpathlineto{\pgfqpoint{1.118045in}{1.847071in}}%
\pgfpathclose%
\pgfusepath{fill}%
\end{pgfscope}%
\begin{pgfscope}%
\pgfpathrectangle{\pgfqpoint{0.697024in}{0.857143in}}{\pgfqpoint{2.627103in}{1.813434in}}%
\pgfusepath{clip}%
\pgfsetbuttcap%
\pgfsetmiterjoin%
\definecolor{currentfill}{rgb}{0.066899,0.263188,0.377594}%
\pgfsetfillcolor{currentfill}%
\pgfsetlinewidth{0.000000pt}%
\definecolor{currentstroke}{rgb}{0.000000,0.000000,0.000000}%
\pgfsetstrokecolor{currentstroke}%
\pgfsetstrokeopacity{0.000000}%
\pgfsetdash{}{0pt}%
\pgfpathmoveto{\pgfqpoint{1.129215in}{1.847071in}}%
\pgfpathlineto{\pgfqpoint{1.138152in}{1.847071in}}%
\pgfpathlineto{\pgfqpoint{1.138152in}{1.898715in}}%
\pgfpathlineto{\pgfqpoint{1.129215in}{1.898715in}}%
\pgfpathlineto{\pgfqpoint{1.129215in}{1.847071in}}%
\pgfpathclose%
\pgfusepath{fill}%
\end{pgfscope}%
\begin{pgfscope}%
\pgfpathrectangle{\pgfqpoint{0.697024in}{0.857143in}}{\pgfqpoint{2.627103in}{1.813434in}}%
\pgfusepath{clip}%
\pgfsetbuttcap%
\pgfsetmiterjoin%
\definecolor{currentfill}{rgb}{0.066899,0.263188,0.377594}%
\pgfsetfillcolor{currentfill}%
\pgfsetlinewidth{0.000000pt}%
\definecolor{currentstroke}{rgb}{0.000000,0.000000,0.000000}%
\pgfsetstrokecolor{currentstroke}%
\pgfsetstrokeopacity{0.000000}%
\pgfsetdash{}{0pt}%
\pgfpathmoveto{\pgfqpoint{1.140386in}{1.847071in}}%
\pgfpathlineto{\pgfqpoint{1.149322in}{1.847071in}}%
\pgfpathlineto{\pgfqpoint{1.149322in}{1.906149in}}%
\pgfpathlineto{\pgfqpoint{1.140386in}{1.906149in}}%
\pgfpathlineto{\pgfqpoint{1.140386in}{1.847071in}}%
\pgfpathclose%
\pgfusepath{fill}%
\end{pgfscope}%
\begin{pgfscope}%
\pgfpathrectangle{\pgfqpoint{0.697024in}{0.857143in}}{\pgfqpoint{2.627103in}{1.813434in}}%
\pgfusepath{clip}%
\pgfsetbuttcap%
\pgfsetmiterjoin%
\definecolor{currentfill}{rgb}{0.066899,0.263188,0.377594}%
\pgfsetfillcolor{currentfill}%
\pgfsetlinewidth{0.000000pt}%
\definecolor{currentstroke}{rgb}{0.000000,0.000000,0.000000}%
\pgfsetstrokecolor{currentstroke}%
\pgfsetstrokeopacity{0.000000}%
\pgfsetdash{}{0pt}%
\pgfpathmoveto{\pgfqpoint{1.151556in}{1.847071in}}%
\pgfpathlineto{\pgfqpoint{1.160493in}{1.847071in}}%
\pgfpathlineto{\pgfqpoint{1.160493in}{1.882857in}}%
\pgfpathlineto{\pgfqpoint{1.151556in}{1.882857in}}%
\pgfpathlineto{\pgfqpoint{1.151556in}{1.847071in}}%
\pgfpathclose%
\pgfusepath{fill}%
\end{pgfscope}%
\begin{pgfscope}%
\pgfpathrectangle{\pgfqpoint{0.697024in}{0.857143in}}{\pgfqpoint{2.627103in}{1.813434in}}%
\pgfusepath{clip}%
\pgfsetbuttcap%
\pgfsetmiterjoin%
\definecolor{currentfill}{rgb}{0.066899,0.263188,0.377594}%
\pgfsetfillcolor{currentfill}%
\pgfsetlinewidth{0.000000pt}%
\definecolor{currentstroke}{rgb}{0.000000,0.000000,0.000000}%
\pgfsetstrokecolor{currentstroke}%
\pgfsetstrokeopacity{0.000000}%
\pgfsetdash{}{0pt}%
\pgfpathmoveto{\pgfqpoint{1.162727in}{1.847071in}}%
\pgfpathlineto{\pgfqpoint{1.171663in}{1.847071in}}%
\pgfpathlineto{\pgfqpoint{1.171663in}{1.873787in}}%
\pgfpathlineto{\pgfqpoint{1.162727in}{1.873787in}}%
\pgfpathlineto{\pgfqpoint{1.162727in}{1.847071in}}%
\pgfpathclose%
\pgfusepath{fill}%
\end{pgfscope}%
\begin{pgfscope}%
\pgfpathrectangle{\pgfqpoint{0.697024in}{0.857143in}}{\pgfqpoint{2.627103in}{1.813434in}}%
\pgfusepath{clip}%
\pgfsetbuttcap%
\pgfsetmiterjoin%
\definecolor{currentfill}{rgb}{0.066899,0.263188,0.377594}%
\pgfsetfillcolor{currentfill}%
\pgfsetlinewidth{0.000000pt}%
\definecolor{currentstroke}{rgb}{0.000000,0.000000,0.000000}%
\pgfsetstrokecolor{currentstroke}%
\pgfsetstrokeopacity{0.000000}%
\pgfsetdash{}{0pt}%
\pgfpathmoveto{\pgfqpoint{1.173898in}{1.847071in}}%
\pgfpathlineto{\pgfqpoint{1.182834in}{1.847071in}}%
\pgfpathlineto{\pgfqpoint{1.182834in}{1.846531in}}%
\pgfpathlineto{\pgfqpoint{1.173898in}{1.846531in}}%
\pgfpathlineto{\pgfqpoint{1.173898in}{1.847071in}}%
\pgfpathclose%
\pgfusepath{fill}%
\end{pgfscope}%
\begin{pgfscope}%
\pgfpathrectangle{\pgfqpoint{0.697024in}{0.857143in}}{\pgfqpoint{2.627103in}{1.813434in}}%
\pgfusepath{clip}%
\pgfsetbuttcap%
\pgfsetmiterjoin%
\definecolor{currentfill}{rgb}{0.066899,0.263188,0.377594}%
\pgfsetfillcolor{currentfill}%
\pgfsetlinewidth{0.000000pt}%
\definecolor{currentstroke}{rgb}{0.000000,0.000000,0.000000}%
\pgfsetstrokecolor{currentstroke}%
\pgfsetstrokeopacity{0.000000}%
\pgfsetdash{}{0pt}%
\pgfpathmoveto{\pgfqpoint{1.185068in}{1.847071in}}%
\pgfpathlineto{\pgfqpoint{1.194005in}{1.847071in}}%
\pgfpathlineto{\pgfqpoint{1.194005in}{1.877233in}}%
\pgfpathlineto{\pgfqpoint{1.185068in}{1.877233in}}%
\pgfpathlineto{\pgfqpoint{1.185068in}{1.847071in}}%
\pgfpathclose%
\pgfusepath{fill}%
\end{pgfscope}%
\begin{pgfscope}%
\pgfpathrectangle{\pgfqpoint{0.697024in}{0.857143in}}{\pgfqpoint{2.627103in}{1.813434in}}%
\pgfusepath{clip}%
\pgfsetbuttcap%
\pgfsetmiterjoin%
\definecolor{currentfill}{rgb}{0.066899,0.263188,0.377594}%
\pgfsetfillcolor{currentfill}%
\pgfsetlinewidth{0.000000pt}%
\definecolor{currentstroke}{rgb}{0.000000,0.000000,0.000000}%
\pgfsetstrokecolor{currentstroke}%
\pgfsetstrokeopacity{0.000000}%
\pgfsetdash{}{0pt}%
\pgfpathmoveto{\pgfqpoint{1.196239in}{1.847071in}}%
\pgfpathlineto{\pgfqpoint{1.205175in}{1.847071in}}%
\pgfpathlineto{\pgfqpoint{1.205175in}{1.921473in}}%
\pgfpathlineto{\pgfqpoint{1.196239in}{1.921473in}}%
\pgfpathlineto{\pgfqpoint{1.196239in}{1.847071in}}%
\pgfpathclose%
\pgfusepath{fill}%
\end{pgfscope}%
\begin{pgfscope}%
\pgfpathrectangle{\pgfqpoint{0.697024in}{0.857143in}}{\pgfqpoint{2.627103in}{1.813434in}}%
\pgfusepath{clip}%
\pgfsetbuttcap%
\pgfsetmiterjoin%
\definecolor{currentfill}{rgb}{0.066899,0.263188,0.377594}%
\pgfsetfillcolor{currentfill}%
\pgfsetlinewidth{0.000000pt}%
\definecolor{currentstroke}{rgb}{0.000000,0.000000,0.000000}%
\pgfsetstrokecolor{currentstroke}%
\pgfsetstrokeopacity{0.000000}%
\pgfsetdash{}{0pt}%
\pgfpathmoveto{\pgfqpoint{1.207409in}{1.847071in}}%
\pgfpathlineto{\pgfqpoint{1.216346in}{1.847071in}}%
\pgfpathlineto{\pgfqpoint{1.216346in}{1.952259in}}%
\pgfpathlineto{\pgfqpoint{1.207409in}{1.952259in}}%
\pgfpathlineto{\pgfqpoint{1.207409in}{1.847071in}}%
\pgfpathclose%
\pgfusepath{fill}%
\end{pgfscope}%
\begin{pgfscope}%
\pgfpathrectangle{\pgfqpoint{0.697024in}{0.857143in}}{\pgfqpoint{2.627103in}{1.813434in}}%
\pgfusepath{clip}%
\pgfsetbuttcap%
\pgfsetmiterjoin%
\definecolor{currentfill}{rgb}{0.066899,0.263188,0.377594}%
\pgfsetfillcolor{currentfill}%
\pgfsetlinewidth{0.000000pt}%
\definecolor{currentstroke}{rgb}{0.000000,0.000000,0.000000}%
\pgfsetstrokecolor{currentstroke}%
\pgfsetstrokeopacity{0.000000}%
\pgfsetdash{}{0pt}%
\pgfpathmoveto{\pgfqpoint{1.218580in}{1.847071in}}%
\pgfpathlineto{\pgfqpoint{1.227516in}{1.847071in}}%
\pgfpathlineto{\pgfqpoint{1.227516in}{1.957059in}}%
\pgfpathlineto{\pgfqpoint{1.218580in}{1.957059in}}%
\pgfpathlineto{\pgfqpoint{1.218580in}{1.847071in}}%
\pgfpathclose%
\pgfusepath{fill}%
\end{pgfscope}%
\begin{pgfscope}%
\pgfpathrectangle{\pgfqpoint{0.697024in}{0.857143in}}{\pgfqpoint{2.627103in}{1.813434in}}%
\pgfusepath{clip}%
\pgfsetbuttcap%
\pgfsetmiterjoin%
\definecolor{currentfill}{rgb}{0.066899,0.263188,0.377594}%
\pgfsetfillcolor{currentfill}%
\pgfsetlinewidth{0.000000pt}%
\definecolor{currentstroke}{rgb}{0.000000,0.000000,0.000000}%
\pgfsetstrokecolor{currentstroke}%
\pgfsetstrokeopacity{0.000000}%
\pgfsetdash{}{0pt}%
\pgfpathmoveto{\pgfqpoint{1.229751in}{1.847071in}}%
\pgfpathlineto{\pgfqpoint{1.238687in}{1.847071in}}%
\pgfpathlineto{\pgfqpoint{1.238687in}{1.930348in}}%
\pgfpathlineto{\pgfqpoint{1.229751in}{1.930348in}}%
\pgfpathlineto{\pgfqpoint{1.229751in}{1.847071in}}%
\pgfpathclose%
\pgfusepath{fill}%
\end{pgfscope}%
\begin{pgfscope}%
\pgfpathrectangle{\pgfqpoint{0.697024in}{0.857143in}}{\pgfqpoint{2.627103in}{1.813434in}}%
\pgfusepath{clip}%
\pgfsetbuttcap%
\pgfsetmiterjoin%
\definecolor{currentfill}{rgb}{0.066899,0.263188,0.377594}%
\pgfsetfillcolor{currentfill}%
\pgfsetlinewidth{0.000000pt}%
\definecolor{currentstroke}{rgb}{0.000000,0.000000,0.000000}%
\pgfsetstrokecolor{currentstroke}%
\pgfsetstrokeopacity{0.000000}%
\pgfsetdash{}{0pt}%
\pgfpathmoveto{\pgfqpoint{1.240921in}{1.847071in}}%
\pgfpathlineto{\pgfqpoint{1.249858in}{1.847071in}}%
\pgfpathlineto{\pgfqpoint{1.249858in}{1.937419in}}%
\pgfpathlineto{\pgfqpoint{1.240921in}{1.937419in}}%
\pgfpathlineto{\pgfqpoint{1.240921in}{1.847071in}}%
\pgfpathclose%
\pgfusepath{fill}%
\end{pgfscope}%
\begin{pgfscope}%
\pgfpathrectangle{\pgfqpoint{0.697024in}{0.857143in}}{\pgfqpoint{2.627103in}{1.813434in}}%
\pgfusepath{clip}%
\pgfsetbuttcap%
\pgfsetmiterjoin%
\definecolor{currentfill}{rgb}{0.066899,0.263188,0.377594}%
\pgfsetfillcolor{currentfill}%
\pgfsetlinewidth{0.000000pt}%
\definecolor{currentstroke}{rgb}{0.000000,0.000000,0.000000}%
\pgfsetstrokecolor{currentstroke}%
\pgfsetstrokeopacity{0.000000}%
\pgfsetdash{}{0pt}%
\pgfpathmoveto{\pgfqpoint{1.252092in}{1.847071in}}%
\pgfpathlineto{\pgfqpoint{1.261028in}{1.847071in}}%
\pgfpathlineto{\pgfqpoint{1.261028in}{1.945503in}}%
\pgfpathlineto{\pgfqpoint{1.252092in}{1.945503in}}%
\pgfpathlineto{\pgfqpoint{1.252092in}{1.847071in}}%
\pgfpathclose%
\pgfusepath{fill}%
\end{pgfscope}%
\begin{pgfscope}%
\pgfpathrectangle{\pgfqpoint{0.697024in}{0.857143in}}{\pgfqpoint{2.627103in}{1.813434in}}%
\pgfusepath{clip}%
\pgfsetbuttcap%
\pgfsetmiterjoin%
\definecolor{currentfill}{rgb}{0.066899,0.263188,0.377594}%
\pgfsetfillcolor{currentfill}%
\pgfsetlinewidth{0.000000pt}%
\definecolor{currentstroke}{rgb}{0.000000,0.000000,0.000000}%
\pgfsetstrokecolor{currentstroke}%
\pgfsetstrokeopacity{0.000000}%
\pgfsetdash{}{0pt}%
\pgfpathmoveto{\pgfqpoint{1.263262in}{1.847071in}}%
\pgfpathlineto{\pgfqpoint{1.272199in}{1.847071in}}%
\pgfpathlineto{\pgfqpoint{1.272199in}{1.936817in}}%
\pgfpathlineto{\pgfqpoint{1.263262in}{1.936817in}}%
\pgfpathlineto{\pgfqpoint{1.263262in}{1.847071in}}%
\pgfpathclose%
\pgfusepath{fill}%
\end{pgfscope}%
\begin{pgfscope}%
\pgfpathrectangle{\pgfqpoint{0.697024in}{0.857143in}}{\pgfqpoint{2.627103in}{1.813434in}}%
\pgfusepath{clip}%
\pgfsetbuttcap%
\pgfsetmiterjoin%
\definecolor{currentfill}{rgb}{0.066899,0.263188,0.377594}%
\pgfsetfillcolor{currentfill}%
\pgfsetlinewidth{0.000000pt}%
\definecolor{currentstroke}{rgb}{0.000000,0.000000,0.000000}%
\pgfsetstrokecolor{currentstroke}%
\pgfsetstrokeopacity{0.000000}%
\pgfsetdash{}{0pt}%
\pgfpathmoveto{\pgfqpoint{1.274433in}{1.847071in}}%
\pgfpathlineto{\pgfqpoint{1.283369in}{1.847071in}}%
\pgfpathlineto{\pgfqpoint{1.283369in}{1.930941in}}%
\pgfpathlineto{\pgfqpoint{1.274433in}{1.930941in}}%
\pgfpathlineto{\pgfqpoint{1.274433in}{1.847071in}}%
\pgfpathclose%
\pgfusepath{fill}%
\end{pgfscope}%
\begin{pgfscope}%
\pgfpathrectangle{\pgfqpoint{0.697024in}{0.857143in}}{\pgfqpoint{2.627103in}{1.813434in}}%
\pgfusepath{clip}%
\pgfsetbuttcap%
\pgfsetmiterjoin%
\definecolor{currentfill}{rgb}{0.066899,0.263188,0.377594}%
\pgfsetfillcolor{currentfill}%
\pgfsetlinewidth{0.000000pt}%
\definecolor{currentstroke}{rgb}{0.000000,0.000000,0.000000}%
\pgfsetstrokecolor{currentstroke}%
\pgfsetstrokeopacity{0.000000}%
\pgfsetdash{}{0pt}%
\pgfpathmoveto{\pgfqpoint{1.285604in}{1.847071in}}%
\pgfpathlineto{\pgfqpoint{1.294540in}{1.847071in}}%
\pgfpathlineto{\pgfqpoint{1.294540in}{1.923735in}}%
\pgfpathlineto{\pgfqpoint{1.285604in}{1.923735in}}%
\pgfpathlineto{\pgfqpoint{1.285604in}{1.847071in}}%
\pgfpathclose%
\pgfusepath{fill}%
\end{pgfscope}%
\begin{pgfscope}%
\pgfpathrectangle{\pgfqpoint{0.697024in}{0.857143in}}{\pgfqpoint{2.627103in}{1.813434in}}%
\pgfusepath{clip}%
\pgfsetbuttcap%
\pgfsetmiterjoin%
\definecolor{currentfill}{rgb}{0.066899,0.263188,0.377594}%
\pgfsetfillcolor{currentfill}%
\pgfsetlinewidth{0.000000pt}%
\definecolor{currentstroke}{rgb}{0.000000,0.000000,0.000000}%
\pgfsetstrokecolor{currentstroke}%
\pgfsetstrokeopacity{0.000000}%
\pgfsetdash{}{0pt}%
\pgfpathmoveto{\pgfqpoint{1.296774in}{1.847071in}}%
\pgfpathlineto{\pgfqpoint{1.305711in}{1.847071in}}%
\pgfpathlineto{\pgfqpoint{1.305711in}{1.914558in}}%
\pgfpathlineto{\pgfqpoint{1.296774in}{1.914558in}}%
\pgfpathlineto{\pgfqpoint{1.296774in}{1.847071in}}%
\pgfpathclose%
\pgfusepath{fill}%
\end{pgfscope}%
\begin{pgfscope}%
\pgfpathrectangle{\pgfqpoint{0.697024in}{0.857143in}}{\pgfqpoint{2.627103in}{1.813434in}}%
\pgfusepath{clip}%
\pgfsetbuttcap%
\pgfsetmiterjoin%
\definecolor{currentfill}{rgb}{0.066899,0.263188,0.377594}%
\pgfsetfillcolor{currentfill}%
\pgfsetlinewidth{0.000000pt}%
\definecolor{currentstroke}{rgb}{0.000000,0.000000,0.000000}%
\pgfsetstrokecolor{currentstroke}%
\pgfsetstrokeopacity{0.000000}%
\pgfsetdash{}{0pt}%
\pgfpathmoveto{\pgfqpoint{1.307945in}{1.847071in}}%
\pgfpathlineto{\pgfqpoint{1.316881in}{1.847071in}}%
\pgfpathlineto{\pgfqpoint{1.316881in}{1.926030in}}%
\pgfpathlineto{\pgfqpoint{1.307945in}{1.926030in}}%
\pgfpathlineto{\pgfqpoint{1.307945in}{1.847071in}}%
\pgfpathclose%
\pgfusepath{fill}%
\end{pgfscope}%
\begin{pgfscope}%
\pgfpathrectangle{\pgfqpoint{0.697024in}{0.857143in}}{\pgfqpoint{2.627103in}{1.813434in}}%
\pgfusepath{clip}%
\pgfsetbuttcap%
\pgfsetmiterjoin%
\definecolor{currentfill}{rgb}{0.066899,0.263188,0.377594}%
\pgfsetfillcolor{currentfill}%
\pgfsetlinewidth{0.000000pt}%
\definecolor{currentstroke}{rgb}{0.000000,0.000000,0.000000}%
\pgfsetstrokecolor{currentstroke}%
\pgfsetstrokeopacity{0.000000}%
\pgfsetdash{}{0pt}%
\pgfpathmoveto{\pgfqpoint{1.319115in}{1.847071in}}%
\pgfpathlineto{\pgfqpoint{1.328052in}{1.847071in}}%
\pgfpathlineto{\pgfqpoint{1.328052in}{1.884029in}}%
\pgfpathlineto{\pgfqpoint{1.319115in}{1.884029in}}%
\pgfpathlineto{\pgfqpoint{1.319115in}{1.847071in}}%
\pgfpathclose%
\pgfusepath{fill}%
\end{pgfscope}%
\begin{pgfscope}%
\pgfpathrectangle{\pgfqpoint{0.697024in}{0.857143in}}{\pgfqpoint{2.627103in}{1.813434in}}%
\pgfusepath{clip}%
\pgfsetbuttcap%
\pgfsetmiterjoin%
\definecolor{currentfill}{rgb}{0.066899,0.263188,0.377594}%
\pgfsetfillcolor{currentfill}%
\pgfsetlinewidth{0.000000pt}%
\definecolor{currentstroke}{rgb}{0.000000,0.000000,0.000000}%
\pgfsetstrokecolor{currentstroke}%
\pgfsetstrokeopacity{0.000000}%
\pgfsetdash{}{0pt}%
\pgfpathmoveto{\pgfqpoint{1.330286in}{1.847071in}}%
\pgfpathlineto{\pgfqpoint{1.339222in}{1.847071in}}%
\pgfpathlineto{\pgfqpoint{1.339222in}{1.872646in}}%
\pgfpathlineto{\pgfqpoint{1.330286in}{1.872646in}}%
\pgfpathlineto{\pgfqpoint{1.330286in}{1.847071in}}%
\pgfpathclose%
\pgfusepath{fill}%
\end{pgfscope}%
\begin{pgfscope}%
\pgfpathrectangle{\pgfqpoint{0.697024in}{0.857143in}}{\pgfqpoint{2.627103in}{1.813434in}}%
\pgfusepath{clip}%
\pgfsetbuttcap%
\pgfsetmiterjoin%
\definecolor{currentfill}{rgb}{0.066899,0.263188,0.377594}%
\pgfsetfillcolor{currentfill}%
\pgfsetlinewidth{0.000000pt}%
\definecolor{currentstroke}{rgb}{0.000000,0.000000,0.000000}%
\pgfsetstrokecolor{currentstroke}%
\pgfsetstrokeopacity{0.000000}%
\pgfsetdash{}{0pt}%
\pgfpathmoveto{\pgfqpoint{1.341457in}{1.847071in}}%
\pgfpathlineto{\pgfqpoint{1.350393in}{1.847071in}}%
\pgfpathlineto{\pgfqpoint{1.350393in}{1.892155in}}%
\pgfpathlineto{\pgfqpoint{1.341457in}{1.892155in}}%
\pgfpathlineto{\pgfqpoint{1.341457in}{1.847071in}}%
\pgfpathclose%
\pgfusepath{fill}%
\end{pgfscope}%
\begin{pgfscope}%
\pgfpathrectangle{\pgfqpoint{0.697024in}{0.857143in}}{\pgfqpoint{2.627103in}{1.813434in}}%
\pgfusepath{clip}%
\pgfsetbuttcap%
\pgfsetmiterjoin%
\definecolor{currentfill}{rgb}{0.066899,0.263188,0.377594}%
\pgfsetfillcolor{currentfill}%
\pgfsetlinewidth{0.000000pt}%
\definecolor{currentstroke}{rgb}{0.000000,0.000000,0.000000}%
\pgfsetstrokecolor{currentstroke}%
\pgfsetstrokeopacity{0.000000}%
\pgfsetdash{}{0pt}%
\pgfpathmoveto{\pgfqpoint{1.352627in}{1.847071in}}%
\pgfpathlineto{\pgfqpoint{1.361564in}{1.847071in}}%
\pgfpathlineto{\pgfqpoint{1.361564in}{1.884328in}}%
\pgfpathlineto{\pgfqpoint{1.352627in}{1.884328in}}%
\pgfpathlineto{\pgfqpoint{1.352627in}{1.847071in}}%
\pgfpathclose%
\pgfusepath{fill}%
\end{pgfscope}%
\begin{pgfscope}%
\pgfpathrectangle{\pgfqpoint{0.697024in}{0.857143in}}{\pgfqpoint{2.627103in}{1.813434in}}%
\pgfusepath{clip}%
\pgfsetbuttcap%
\pgfsetmiterjoin%
\definecolor{currentfill}{rgb}{0.066899,0.263188,0.377594}%
\pgfsetfillcolor{currentfill}%
\pgfsetlinewidth{0.000000pt}%
\definecolor{currentstroke}{rgb}{0.000000,0.000000,0.000000}%
\pgfsetstrokecolor{currentstroke}%
\pgfsetstrokeopacity{0.000000}%
\pgfsetdash{}{0pt}%
\pgfpathmoveto{\pgfqpoint{1.363798in}{1.847071in}}%
\pgfpathlineto{\pgfqpoint{1.372734in}{1.847071in}}%
\pgfpathlineto{\pgfqpoint{1.372734in}{1.877043in}}%
\pgfpathlineto{\pgfqpoint{1.363798in}{1.877043in}}%
\pgfpathlineto{\pgfqpoint{1.363798in}{1.847071in}}%
\pgfpathclose%
\pgfusepath{fill}%
\end{pgfscope}%
\begin{pgfscope}%
\pgfpathrectangle{\pgfqpoint{0.697024in}{0.857143in}}{\pgfqpoint{2.627103in}{1.813434in}}%
\pgfusepath{clip}%
\pgfsetbuttcap%
\pgfsetmiterjoin%
\definecolor{currentfill}{rgb}{0.066899,0.263188,0.377594}%
\pgfsetfillcolor{currentfill}%
\pgfsetlinewidth{0.000000pt}%
\definecolor{currentstroke}{rgb}{0.000000,0.000000,0.000000}%
\pgfsetstrokecolor{currentstroke}%
\pgfsetstrokeopacity{0.000000}%
\pgfsetdash{}{0pt}%
\pgfpathmoveto{\pgfqpoint{1.374968in}{1.847071in}}%
\pgfpathlineto{\pgfqpoint{1.383905in}{1.847071in}}%
\pgfpathlineto{\pgfqpoint{1.383905in}{1.849961in}}%
\pgfpathlineto{\pgfqpoint{1.374968in}{1.849961in}}%
\pgfpathlineto{\pgfqpoint{1.374968in}{1.847071in}}%
\pgfpathclose%
\pgfusepath{fill}%
\end{pgfscope}%
\begin{pgfscope}%
\pgfpathrectangle{\pgfqpoint{0.697024in}{0.857143in}}{\pgfqpoint{2.627103in}{1.813434in}}%
\pgfusepath{clip}%
\pgfsetbuttcap%
\pgfsetmiterjoin%
\definecolor{currentfill}{rgb}{0.066899,0.263188,0.377594}%
\pgfsetfillcolor{currentfill}%
\pgfsetlinewidth{0.000000pt}%
\definecolor{currentstroke}{rgb}{0.000000,0.000000,0.000000}%
\pgfsetstrokecolor{currentstroke}%
\pgfsetstrokeopacity{0.000000}%
\pgfsetdash{}{0pt}%
\pgfpathmoveto{\pgfqpoint{1.386139in}{1.847071in}}%
\pgfpathlineto{\pgfqpoint{1.395076in}{1.847071in}}%
\pgfpathlineto{\pgfqpoint{1.395076in}{1.835467in}}%
\pgfpathlineto{\pgfqpoint{1.386139in}{1.835467in}}%
\pgfpathlineto{\pgfqpoint{1.386139in}{1.847071in}}%
\pgfpathclose%
\pgfusepath{fill}%
\end{pgfscope}%
\begin{pgfscope}%
\pgfpathrectangle{\pgfqpoint{0.697024in}{0.857143in}}{\pgfqpoint{2.627103in}{1.813434in}}%
\pgfusepath{clip}%
\pgfsetbuttcap%
\pgfsetmiterjoin%
\definecolor{currentfill}{rgb}{0.066899,0.263188,0.377594}%
\pgfsetfillcolor{currentfill}%
\pgfsetlinewidth{0.000000pt}%
\definecolor{currentstroke}{rgb}{0.000000,0.000000,0.000000}%
\pgfsetstrokecolor{currentstroke}%
\pgfsetstrokeopacity{0.000000}%
\pgfsetdash{}{0pt}%
\pgfpathmoveto{\pgfqpoint{1.397310in}{1.847071in}}%
\pgfpathlineto{\pgfqpoint{1.406246in}{1.847071in}}%
\pgfpathlineto{\pgfqpoint{1.406246in}{1.821999in}}%
\pgfpathlineto{\pgfqpoint{1.397310in}{1.821999in}}%
\pgfpathlineto{\pgfqpoint{1.397310in}{1.847071in}}%
\pgfpathclose%
\pgfusepath{fill}%
\end{pgfscope}%
\begin{pgfscope}%
\pgfpathrectangle{\pgfqpoint{0.697024in}{0.857143in}}{\pgfqpoint{2.627103in}{1.813434in}}%
\pgfusepath{clip}%
\pgfsetbuttcap%
\pgfsetmiterjoin%
\definecolor{currentfill}{rgb}{0.066899,0.263188,0.377594}%
\pgfsetfillcolor{currentfill}%
\pgfsetlinewidth{0.000000pt}%
\definecolor{currentstroke}{rgb}{0.000000,0.000000,0.000000}%
\pgfsetstrokecolor{currentstroke}%
\pgfsetstrokeopacity{0.000000}%
\pgfsetdash{}{0pt}%
\pgfpathmoveto{\pgfqpoint{1.408480in}{1.847071in}}%
\pgfpathlineto{\pgfqpoint{1.417417in}{1.847071in}}%
\pgfpathlineto{\pgfqpoint{1.417417in}{1.814040in}}%
\pgfpathlineto{\pgfqpoint{1.408480in}{1.814040in}}%
\pgfpathlineto{\pgfqpoint{1.408480in}{1.847071in}}%
\pgfpathclose%
\pgfusepath{fill}%
\end{pgfscope}%
\begin{pgfscope}%
\pgfpathrectangle{\pgfqpoint{0.697024in}{0.857143in}}{\pgfqpoint{2.627103in}{1.813434in}}%
\pgfusepath{clip}%
\pgfsetbuttcap%
\pgfsetmiterjoin%
\definecolor{currentfill}{rgb}{0.066899,0.263188,0.377594}%
\pgfsetfillcolor{currentfill}%
\pgfsetlinewidth{0.000000pt}%
\definecolor{currentstroke}{rgb}{0.000000,0.000000,0.000000}%
\pgfsetstrokecolor{currentstroke}%
\pgfsetstrokeopacity{0.000000}%
\pgfsetdash{}{0pt}%
\pgfpathmoveto{\pgfqpoint{1.419651in}{1.847071in}}%
\pgfpathlineto{\pgfqpoint{1.428587in}{1.847071in}}%
\pgfpathlineto{\pgfqpoint{1.428587in}{1.821295in}}%
\pgfpathlineto{\pgfqpoint{1.419651in}{1.821295in}}%
\pgfpathlineto{\pgfqpoint{1.419651in}{1.847071in}}%
\pgfpathclose%
\pgfusepath{fill}%
\end{pgfscope}%
\begin{pgfscope}%
\pgfpathrectangle{\pgfqpoint{0.697024in}{0.857143in}}{\pgfqpoint{2.627103in}{1.813434in}}%
\pgfusepath{clip}%
\pgfsetbuttcap%
\pgfsetmiterjoin%
\definecolor{currentfill}{rgb}{0.066899,0.263188,0.377594}%
\pgfsetfillcolor{currentfill}%
\pgfsetlinewidth{0.000000pt}%
\definecolor{currentstroke}{rgb}{0.000000,0.000000,0.000000}%
\pgfsetstrokecolor{currentstroke}%
\pgfsetstrokeopacity{0.000000}%
\pgfsetdash{}{0pt}%
\pgfpathmoveto{\pgfqpoint{1.430821in}{1.847071in}}%
\pgfpathlineto{\pgfqpoint{1.439758in}{1.847071in}}%
\pgfpathlineto{\pgfqpoint{1.439758in}{1.801756in}}%
\pgfpathlineto{\pgfqpoint{1.430821in}{1.801756in}}%
\pgfpathlineto{\pgfqpoint{1.430821in}{1.847071in}}%
\pgfpathclose%
\pgfusepath{fill}%
\end{pgfscope}%
\begin{pgfscope}%
\pgfpathrectangle{\pgfqpoint{0.697024in}{0.857143in}}{\pgfqpoint{2.627103in}{1.813434in}}%
\pgfusepath{clip}%
\pgfsetbuttcap%
\pgfsetmiterjoin%
\definecolor{currentfill}{rgb}{0.066899,0.263188,0.377594}%
\pgfsetfillcolor{currentfill}%
\pgfsetlinewidth{0.000000pt}%
\definecolor{currentstroke}{rgb}{0.000000,0.000000,0.000000}%
\pgfsetstrokecolor{currentstroke}%
\pgfsetstrokeopacity{0.000000}%
\pgfsetdash{}{0pt}%
\pgfpathmoveto{\pgfqpoint{1.441992in}{1.847071in}}%
\pgfpathlineto{\pgfqpoint{1.450929in}{1.847071in}}%
\pgfpathlineto{\pgfqpoint{1.450929in}{1.802324in}}%
\pgfpathlineto{\pgfqpoint{1.441992in}{1.802324in}}%
\pgfpathlineto{\pgfqpoint{1.441992in}{1.847071in}}%
\pgfpathclose%
\pgfusepath{fill}%
\end{pgfscope}%
\begin{pgfscope}%
\pgfpathrectangle{\pgfqpoint{0.697024in}{0.857143in}}{\pgfqpoint{2.627103in}{1.813434in}}%
\pgfusepath{clip}%
\pgfsetbuttcap%
\pgfsetmiterjoin%
\definecolor{currentfill}{rgb}{0.066899,0.263188,0.377594}%
\pgfsetfillcolor{currentfill}%
\pgfsetlinewidth{0.000000pt}%
\definecolor{currentstroke}{rgb}{0.000000,0.000000,0.000000}%
\pgfsetstrokecolor{currentstroke}%
\pgfsetstrokeopacity{0.000000}%
\pgfsetdash{}{0pt}%
\pgfpathmoveto{\pgfqpoint{1.453163in}{1.847071in}}%
\pgfpathlineto{\pgfqpoint{1.462099in}{1.847071in}}%
\pgfpathlineto{\pgfqpoint{1.462099in}{1.806726in}}%
\pgfpathlineto{\pgfqpoint{1.453163in}{1.806726in}}%
\pgfpathlineto{\pgfqpoint{1.453163in}{1.847071in}}%
\pgfpathclose%
\pgfusepath{fill}%
\end{pgfscope}%
\begin{pgfscope}%
\pgfpathrectangle{\pgfqpoint{0.697024in}{0.857143in}}{\pgfqpoint{2.627103in}{1.813434in}}%
\pgfusepath{clip}%
\pgfsetbuttcap%
\pgfsetmiterjoin%
\definecolor{currentfill}{rgb}{0.066899,0.263188,0.377594}%
\pgfsetfillcolor{currentfill}%
\pgfsetlinewidth{0.000000pt}%
\definecolor{currentstroke}{rgb}{0.000000,0.000000,0.000000}%
\pgfsetstrokecolor{currentstroke}%
\pgfsetstrokeopacity{0.000000}%
\pgfsetdash{}{0pt}%
\pgfpathmoveto{\pgfqpoint{1.464333in}{1.847071in}}%
\pgfpathlineto{\pgfqpoint{1.473270in}{1.847071in}}%
\pgfpathlineto{\pgfqpoint{1.473270in}{1.835521in}}%
\pgfpathlineto{\pgfqpoint{1.464333in}{1.835521in}}%
\pgfpathlineto{\pgfqpoint{1.464333in}{1.847071in}}%
\pgfpathclose%
\pgfusepath{fill}%
\end{pgfscope}%
\begin{pgfscope}%
\pgfpathrectangle{\pgfqpoint{0.697024in}{0.857143in}}{\pgfqpoint{2.627103in}{1.813434in}}%
\pgfusepath{clip}%
\pgfsetbuttcap%
\pgfsetmiterjoin%
\definecolor{currentfill}{rgb}{0.066899,0.263188,0.377594}%
\pgfsetfillcolor{currentfill}%
\pgfsetlinewidth{0.000000pt}%
\definecolor{currentstroke}{rgb}{0.000000,0.000000,0.000000}%
\pgfsetstrokecolor{currentstroke}%
\pgfsetstrokeopacity{0.000000}%
\pgfsetdash{}{0pt}%
\pgfpathmoveto{\pgfqpoint{1.475504in}{1.847071in}}%
\pgfpathlineto{\pgfqpoint{1.484440in}{1.847071in}}%
\pgfpathlineto{\pgfqpoint{1.484440in}{1.795076in}}%
\pgfpathlineto{\pgfqpoint{1.475504in}{1.795076in}}%
\pgfpathlineto{\pgfqpoint{1.475504in}{1.847071in}}%
\pgfpathclose%
\pgfusepath{fill}%
\end{pgfscope}%
\begin{pgfscope}%
\pgfpathrectangle{\pgfqpoint{0.697024in}{0.857143in}}{\pgfqpoint{2.627103in}{1.813434in}}%
\pgfusepath{clip}%
\pgfsetbuttcap%
\pgfsetmiterjoin%
\definecolor{currentfill}{rgb}{0.066899,0.263188,0.377594}%
\pgfsetfillcolor{currentfill}%
\pgfsetlinewidth{0.000000pt}%
\definecolor{currentstroke}{rgb}{0.000000,0.000000,0.000000}%
\pgfsetstrokecolor{currentstroke}%
\pgfsetstrokeopacity{0.000000}%
\pgfsetdash{}{0pt}%
\pgfpathmoveto{\pgfqpoint{1.486674in}{1.847071in}}%
\pgfpathlineto{\pgfqpoint{1.495611in}{1.847071in}}%
\pgfpathlineto{\pgfqpoint{1.495611in}{1.823435in}}%
\pgfpathlineto{\pgfqpoint{1.486674in}{1.823435in}}%
\pgfpathlineto{\pgfqpoint{1.486674in}{1.847071in}}%
\pgfpathclose%
\pgfusepath{fill}%
\end{pgfscope}%
\begin{pgfscope}%
\pgfpathrectangle{\pgfqpoint{0.697024in}{0.857143in}}{\pgfqpoint{2.627103in}{1.813434in}}%
\pgfusepath{clip}%
\pgfsetbuttcap%
\pgfsetmiterjoin%
\definecolor{currentfill}{rgb}{0.066899,0.263188,0.377594}%
\pgfsetfillcolor{currentfill}%
\pgfsetlinewidth{0.000000pt}%
\definecolor{currentstroke}{rgb}{0.000000,0.000000,0.000000}%
\pgfsetstrokecolor{currentstroke}%
\pgfsetstrokeopacity{0.000000}%
\pgfsetdash{}{0pt}%
\pgfpathmoveto{\pgfqpoint{1.497845in}{1.847071in}}%
\pgfpathlineto{\pgfqpoint{1.506782in}{1.847071in}}%
\pgfpathlineto{\pgfqpoint{1.506782in}{1.789478in}}%
\pgfpathlineto{\pgfqpoint{1.497845in}{1.789478in}}%
\pgfpathlineto{\pgfqpoint{1.497845in}{1.847071in}}%
\pgfpathclose%
\pgfusepath{fill}%
\end{pgfscope}%
\begin{pgfscope}%
\pgfpathrectangle{\pgfqpoint{0.697024in}{0.857143in}}{\pgfqpoint{2.627103in}{1.813434in}}%
\pgfusepath{clip}%
\pgfsetbuttcap%
\pgfsetmiterjoin%
\definecolor{currentfill}{rgb}{0.066899,0.263188,0.377594}%
\pgfsetfillcolor{currentfill}%
\pgfsetlinewidth{0.000000pt}%
\definecolor{currentstroke}{rgb}{0.000000,0.000000,0.000000}%
\pgfsetstrokecolor{currentstroke}%
\pgfsetstrokeopacity{0.000000}%
\pgfsetdash{}{0pt}%
\pgfpathmoveto{\pgfqpoint{1.509016in}{1.847071in}}%
\pgfpathlineto{\pgfqpoint{1.517952in}{1.847071in}}%
\pgfpathlineto{\pgfqpoint{1.517952in}{1.784441in}}%
\pgfpathlineto{\pgfqpoint{1.509016in}{1.784441in}}%
\pgfpathlineto{\pgfqpoint{1.509016in}{1.847071in}}%
\pgfpathclose%
\pgfusepath{fill}%
\end{pgfscope}%
\begin{pgfscope}%
\pgfpathrectangle{\pgfqpoint{0.697024in}{0.857143in}}{\pgfqpoint{2.627103in}{1.813434in}}%
\pgfusepath{clip}%
\pgfsetbuttcap%
\pgfsetmiterjoin%
\definecolor{currentfill}{rgb}{0.066899,0.263188,0.377594}%
\pgfsetfillcolor{currentfill}%
\pgfsetlinewidth{0.000000pt}%
\definecolor{currentstroke}{rgb}{0.000000,0.000000,0.000000}%
\pgfsetstrokecolor{currentstroke}%
\pgfsetstrokeopacity{0.000000}%
\pgfsetdash{}{0pt}%
\pgfpathmoveto{\pgfqpoint{1.520186in}{1.847071in}}%
\pgfpathlineto{\pgfqpoint{1.529123in}{1.847071in}}%
\pgfpathlineto{\pgfqpoint{1.529123in}{1.774299in}}%
\pgfpathlineto{\pgfqpoint{1.520186in}{1.774299in}}%
\pgfpathlineto{\pgfqpoint{1.520186in}{1.847071in}}%
\pgfpathclose%
\pgfusepath{fill}%
\end{pgfscope}%
\begin{pgfscope}%
\pgfpathrectangle{\pgfqpoint{0.697024in}{0.857143in}}{\pgfqpoint{2.627103in}{1.813434in}}%
\pgfusepath{clip}%
\pgfsetbuttcap%
\pgfsetmiterjoin%
\definecolor{currentfill}{rgb}{0.066899,0.263188,0.377594}%
\pgfsetfillcolor{currentfill}%
\pgfsetlinewidth{0.000000pt}%
\definecolor{currentstroke}{rgb}{0.000000,0.000000,0.000000}%
\pgfsetstrokecolor{currentstroke}%
\pgfsetstrokeopacity{0.000000}%
\pgfsetdash{}{0pt}%
\pgfpathmoveto{\pgfqpoint{1.531357in}{1.847071in}}%
\pgfpathlineto{\pgfqpoint{1.540293in}{1.847071in}}%
\pgfpathlineto{\pgfqpoint{1.540293in}{1.774393in}}%
\pgfpathlineto{\pgfqpoint{1.531357in}{1.774393in}}%
\pgfpathlineto{\pgfqpoint{1.531357in}{1.847071in}}%
\pgfpathclose%
\pgfusepath{fill}%
\end{pgfscope}%
\begin{pgfscope}%
\pgfpathrectangle{\pgfqpoint{0.697024in}{0.857143in}}{\pgfqpoint{2.627103in}{1.813434in}}%
\pgfusepath{clip}%
\pgfsetbuttcap%
\pgfsetmiterjoin%
\definecolor{currentfill}{rgb}{0.066899,0.263188,0.377594}%
\pgfsetfillcolor{currentfill}%
\pgfsetlinewidth{0.000000pt}%
\definecolor{currentstroke}{rgb}{0.000000,0.000000,0.000000}%
\pgfsetstrokecolor{currentstroke}%
\pgfsetstrokeopacity{0.000000}%
\pgfsetdash{}{0pt}%
\pgfpathmoveto{\pgfqpoint{1.542528in}{1.847071in}}%
\pgfpathlineto{\pgfqpoint{1.551464in}{1.847071in}}%
\pgfpathlineto{\pgfqpoint{1.551464in}{1.792667in}}%
\pgfpathlineto{\pgfqpoint{1.542528in}{1.792667in}}%
\pgfpathlineto{\pgfqpoint{1.542528in}{1.847071in}}%
\pgfpathclose%
\pgfusepath{fill}%
\end{pgfscope}%
\begin{pgfscope}%
\pgfpathrectangle{\pgfqpoint{0.697024in}{0.857143in}}{\pgfqpoint{2.627103in}{1.813434in}}%
\pgfusepath{clip}%
\pgfsetbuttcap%
\pgfsetmiterjoin%
\definecolor{currentfill}{rgb}{0.066899,0.263188,0.377594}%
\pgfsetfillcolor{currentfill}%
\pgfsetlinewidth{0.000000pt}%
\definecolor{currentstroke}{rgb}{0.000000,0.000000,0.000000}%
\pgfsetstrokecolor{currentstroke}%
\pgfsetstrokeopacity{0.000000}%
\pgfsetdash{}{0pt}%
\pgfpathmoveto{\pgfqpoint{1.553698in}{1.847071in}}%
\pgfpathlineto{\pgfqpoint{1.562635in}{1.847071in}}%
\pgfpathlineto{\pgfqpoint{1.562635in}{1.799777in}}%
\pgfpathlineto{\pgfqpoint{1.553698in}{1.799777in}}%
\pgfpathlineto{\pgfqpoint{1.553698in}{1.847071in}}%
\pgfpathclose%
\pgfusepath{fill}%
\end{pgfscope}%
\begin{pgfscope}%
\pgfpathrectangle{\pgfqpoint{0.697024in}{0.857143in}}{\pgfqpoint{2.627103in}{1.813434in}}%
\pgfusepath{clip}%
\pgfsetbuttcap%
\pgfsetmiterjoin%
\definecolor{currentfill}{rgb}{0.066899,0.263188,0.377594}%
\pgfsetfillcolor{currentfill}%
\pgfsetlinewidth{0.000000pt}%
\definecolor{currentstroke}{rgb}{0.000000,0.000000,0.000000}%
\pgfsetstrokecolor{currentstroke}%
\pgfsetstrokeopacity{0.000000}%
\pgfsetdash{}{0pt}%
\pgfpathmoveto{\pgfqpoint{1.564869in}{1.847071in}}%
\pgfpathlineto{\pgfqpoint{1.573805in}{1.847071in}}%
\pgfpathlineto{\pgfqpoint{1.573805in}{1.818659in}}%
\pgfpathlineto{\pgfqpoint{1.564869in}{1.818659in}}%
\pgfpathlineto{\pgfqpoint{1.564869in}{1.847071in}}%
\pgfpathclose%
\pgfusepath{fill}%
\end{pgfscope}%
\begin{pgfscope}%
\pgfpathrectangle{\pgfqpoint{0.697024in}{0.857143in}}{\pgfqpoint{2.627103in}{1.813434in}}%
\pgfusepath{clip}%
\pgfsetbuttcap%
\pgfsetmiterjoin%
\definecolor{currentfill}{rgb}{0.066899,0.263188,0.377594}%
\pgfsetfillcolor{currentfill}%
\pgfsetlinewidth{0.000000pt}%
\definecolor{currentstroke}{rgb}{0.000000,0.000000,0.000000}%
\pgfsetstrokecolor{currentstroke}%
\pgfsetstrokeopacity{0.000000}%
\pgfsetdash{}{0pt}%
\pgfpathmoveto{\pgfqpoint{1.576039in}{1.847071in}}%
\pgfpathlineto{\pgfqpoint{1.584976in}{1.847071in}}%
\pgfpathlineto{\pgfqpoint{1.584976in}{1.803151in}}%
\pgfpathlineto{\pgfqpoint{1.576039in}{1.803151in}}%
\pgfpathlineto{\pgfqpoint{1.576039in}{1.847071in}}%
\pgfpathclose%
\pgfusepath{fill}%
\end{pgfscope}%
\begin{pgfscope}%
\pgfpathrectangle{\pgfqpoint{0.697024in}{0.857143in}}{\pgfqpoint{2.627103in}{1.813434in}}%
\pgfusepath{clip}%
\pgfsetbuttcap%
\pgfsetmiterjoin%
\definecolor{currentfill}{rgb}{0.066899,0.263188,0.377594}%
\pgfsetfillcolor{currentfill}%
\pgfsetlinewidth{0.000000pt}%
\definecolor{currentstroke}{rgb}{0.000000,0.000000,0.000000}%
\pgfsetstrokecolor{currentstroke}%
\pgfsetstrokeopacity{0.000000}%
\pgfsetdash{}{0pt}%
\pgfpathmoveto{\pgfqpoint{1.587210in}{1.847071in}}%
\pgfpathlineto{\pgfqpoint{1.596146in}{1.847071in}}%
\pgfpathlineto{\pgfqpoint{1.596146in}{1.797390in}}%
\pgfpathlineto{\pgfqpoint{1.587210in}{1.797390in}}%
\pgfpathlineto{\pgfqpoint{1.587210in}{1.847071in}}%
\pgfpathclose%
\pgfusepath{fill}%
\end{pgfscope}%
\begin{pgfscope}%
\pgfpathrectangle{\pgfqpoint{0.697024in}{0.857143in}}{\pgfqpoint{2.627103in}{1.813434in}}%
\pgfusepath{clip}%
\pgfsetbuttcap%
\pgfsetmiterjoin%
\definecolor{currentfill}{rgb}{0.066899,0.263188,0.377594}%
\pgfsetfillcolor{currentfill}%
\pgfsetlinewidth{0.000000pt}%
\definecolor{currentstroke}{rgb}{0.000000,0.000000,0.000000}%
\pgfsetstrokecolor{currentstroke}%
\pgfsetstrokeopacity{0.000000}%
\pgfsetdash{}{0pt}%
\pgfpathmoveto{\pgfqpoint{1.598381in}{1.847071in}}%
\pgfpathlineto{\pgfqpoint{1.607317in}{1.847071in}}%
\pgfpathlineto{\pgfqpoint{1.607317in}{1.787213in}}%
\pgfpathlineto{\pgfqpoint{1.598381in}{1.787213in}}%
\pgfpathlineto{\pgfqpoint{1.598381in}{1.847071in}}%
\pgfpathclose%
\pgfusepath{fill}%
\end{pgfscope}%
\begin{pgfscope}%
\pgfpathrectangle{\pgfqpoint{0.697024in}{0.857143in}}{\pgfqpoint{2.627103in}{1.813434in}}%
\pgfusepath{clip}%
\pgfsetbuttcap%
\pgfsetmiterjoin%
\definecolor{currentfill}{rgb}{0.066899,0.263188,0.377594}%
\pgfsetfillcolor{currentfill}%
\pgfsetlinewidth{0.000000pt}%
\definecolor{currentstroke}{rgb}{0.000000,0.000000,0.000000}%
\pgfsetstrokecolor{currentstroke}%
\pgfsetstrokeopacity{0.000000}%
\pgfsetdash{}{0pt}%
\pgfpathmoveto{\pgfqpoint{1.609551in}{1.847071in}}%
\pgfpathlineto{\pgfqpoint{1.618488in}{1.847071in}}%
\pgfpathlineto{\pgfqpoint{1.618488in}{1.789342in}}%
\pgfpathlineto{\pgfqpoint{1.609551in}{1.789342in}}%
\pgfpathlineto{\pgfqpoint{1.609551in}{1.847071in}}%
\pgfpathclose%
\pgfusepath{fill}%
\end{pgfscope}%
\begin{pgfscope}%
\pgfpathrectangle{\pgfqpoint{0.697024in}{0.857143in}}{\pgfqpoint{2.627103in}{1.813434in}}%
\pgfusepath{clip}%
\pgfsetbuttcap%
\pgfsetmiterjoin%
\definecolor{currentfill}{rgb}{0.066899,0.263188,0.377594}%
\pgfsetfillcolor{currentfill}%
\pgfsetlinewidth{0.000000pt}%
\definecolor{currentstroke}{rgb}{0.000000,0.000000,0.000000}%
\pgfsetstrokecolor{currentstroke}%
\pgfsetstrokeopacity{0.000000}%
\pgfsetdash{}{0pt}%
\pgfpathmoveto{\pgfqpoint{1.620722in}{1.847071in}}%
\pgfpathlineto{\pgfqpoint{1.629658in}{1.847071in}}%
\pgfpathlineto{\pgfqpoint{1.629658in}{1.793552in}}%
\pgfpathlineto{\pgfqpoint{1.620722in}{1.793552in}}%
\pgfpathlineto{\pgfqpoint{1.620722in}{1.847071in}}%
\pgfpathclose%
\pgfusepath{fill}%
\end{pgfscope}%
\begin{pgfscope}%
\pgfpathrectangle{\pgfqpoint{0.697024in}{0.857143in}}{\pgfqpoint{2.627103in}{1.813434in}}%
\pgfusepath{clip}%
\pgfsetbuttcap%
\pgfsetmiterjoin%
\definecolor{currentfill}{rgb}{0.066899,0.263188,0.377594}%
\pgfsetfillcolor{currentfill}%
\pgfsetlinewidth{0.000000pt}%
\definecolor{currentstroke}{rgb}{0.000000,0.000000,0.000000}%
\pgfsetstrokecolor{currentstroke}%
\pgfsetstrokeopacity{0.000000}%
\pgfsetdash{}{0pt}%
\pgfpathmoveto{\pgfqpoint{1.631892in}{1.847071in}}%
\pgfpathlineto{\pgfqpoint{1.640829in}{1.847071in}}%
\pgfpathlineto{\pgfqpoint{1.640829in}{1.786157in}}%
\pgfpathlineto{\pgfqpoint{1.631892in}{1.786157in}}%
\pgfpathlineto{\pgfqpoint{1.631892in}{1.847071in}}%
\pgfpathclose%
\pgfusepath{fill}%
\end{pgfscope}%
\begin{pgfscope}%
\pgfpathrectangle{\pgfqpoint{0.697024in}{0.857143in}}{\pgfqpoint{2.627103in}{1.813434in}}%
\pgfusepath{clip}%
\pgfsetbuttcap%
\pgfsetmiterjoin%
\definecolor{currentfill}{rgb}{0.066899,0.263188,0.377594}%
\pgfsetfillcolor{currentfill}%
\pgfsetlinewidth{0.000000pt}%
\definecolor{currentstroke}{rgb}{0.000000,0.000000,0.000000}%
\pgfsetstrokecolor{currentstroke}%
\pgfsetstrokeopacity{0.000000}%
\pgfsetdash{}{0pt}%
\pgfpathmoveto{\pgfqpoint{1.643063in}{1.847071in}}%
\pgfpathlineto{\pgfqpoint{1.651999in}{1.847071in}}%
\pgfpathlineto{\pgfqpoint{1.651999in}{1.779810in}}%
\pgfpathlineto{\pgfqpoint{1.643063in}{1.779810in}}%
\pgfpathlineto{\pgfqpoint{1.643063in}{1.847071in}}%
\pgfpathclose%
\pgfusepath{fill}%
\end{pgfscope}%
\begin{pgfscope}%
\pgfpathrectangle{\pgfqpoint{0.697024in}{0.857143in}}{\pgfqpoint{2.627103in}{1.813434in}}%
\pgfusepath{clip}%
\pgfsetbuttcap%
\pgfsetmiterjoin%
\definecolor{currentfill}{rgb}{0.066899,0.263188,0.377594}%
\pgfsetfillcolor{currentfill}%
\pgfsetlinewidth{0.000000pt}%
\definecolor{currentstroke}{rgb}{0.000000,0.000000,0.000000}%
\pgfsetstrokecolor{currentstroke}%
\pgfsetstrokeopacity{0.000000}%
\pgfsetdash{}{0pt}%
\pgfpathmoveto{\pgfqpoint{1.654234in}{1.847071in}}%
\pgfpathlineto{\pgfqpoint{1.663170in}{1.847071in}}%
\pgfpathlineto{\pgfqpoint{1.663170in}{1.774748in}}%
\pgfpathlineto{\pgfqpoint{1.654234in}{1.774748in}}%
\pgfpathlineto{\pgfqpoint{1.654234in}{1.847071in}}%
\pgfpathclose%
\pgfusepath{fill}%
\end{pgfscope}%
\begin{pgfscope}%
\pgfpathrectangle{\pgfqpoint{0.697024in}{0.857143in}}{\pgfqpoint{2.627103in}{1.813434in}}%
\pgfusepath{clip}%
\pgfsetbuttcap%
\pgfsetmiterjoin%
\definecolor{currentfill}{rgb}{0.066899,0.263188,0.377594}%
\pgfsetfillcolor{currentfill}%
\pgfsetlinewidth{0.000000pt}%
\definecolor{currentstroke}{rgb}{0.000000,0.000000,0.000000}%
\pgfsetstrokecolor{currentstroke}%
\pgfsetstrokeopacity{0.000000}%
\pgfsetdash{}{0pt}%
\pgfpathmoveto{\pgfqpoint{1.665404in}{1.847071in}}%
\pgfpathlineto{\pgfqpoint{1.674341in}{1.847071in}}%
\pgfpathlineto{\pgfqpoint{1.674341in}{1.800862in}}%
\pgfpathlineto{\pgfqpoint{1.665404in}{1.800862in}}%
\pgfpathlineto{\pgfqpoint{1.665404in}{1.847071in}}%
\pgfpathclose%
\pgfusepath{fill}%
\end{pgfscope}%
\begin{pgfscope}%
\pgfpathrectangle{\pgfqpoint{0.697024in}{0.857143in}}{\pgfqpoint{2.627103in}{1.813434in}}%
\pgfusepath{clip}%
\pgfsetbuttcap%
\pgfsetmiterjoin%
\definecolor{currentfill}{rgb}{0.066899,0.263188,0.377594}%
\pgfsetfillcolor{currentfill}%
\pgfsetlinewidth{0.000000pt}%
\definecolor{currentstroke}{rgb}{0.000000,0.000000,0.000000}%
\pgfsetstrokecolor{currentstroke}%
\pgfsetstrokeopacity{0.000000}%
\pgfsetdash{}{0pt}%
\pgfpathmoveto{\pgfqpoint{1.676575in}{1.847071in}}%
\pgfpathlineto{\pgfqpoint{1.685511in}{1.847071in}}%
\pgfpathlineto{\pgfqpoint{1.685511in}{1.793289in}}%
\pgfpathlineto{\pgfqpoint{1.676575in}{1.793289in}}%
\pgfpathlineto{\pgfqpoint{1.676575in}{1.847071in}}%
\pgfpathclose%
\pgfusepath{fill}%
\end{pgfscope}%
\begin{pgfscope}%
\pgfpathrectangle{\pgfqpoint{0.697024in}{0.857143in}}{\pgfqpoint{2.627103in}{1.813434in}}%
\pgfusepath{clip}%
\pgfsetbuttcap%
\pgfsetmiterjoin%
\definecolor{currentfill}{rgb}{0.066899,0.263188,0.377594}%
\pgfsetfillcolor{currentfill}%
\pgfsetlinewidth{0.000000pt}%
\definecolor{currentstroke}{rgb}{0.000000,0.000000,0.000000}%
\pgfsetstrokecolor{currentstroke}%
\pgfsetstrokeopacity{0.000000}%
\pgfsetdash{}{0pt}%
\pgfpathmoveto{\pgfqpoint{1.687745in}{1.847071in}}%
\pgfpathlineto{\pgfqpoint{1.696682in}{1.847071in}}%
\pgfpathlineto{\pgfqpoint{1.696682in}{1.812967in}}%
\pgfpathlineto{\pgfqpoint{1.687745in}{1.812967in}}%
\pgfpathlineto{\pgfqpoint{1.687745in}{1.847071in}}%
\pgfpathclose%
\pgfusepath{fill}%
\end{pgfscope}%
\begin{pgfscope}%
\pgfpathrectangle{\pgfqpoint{0.697024in}{0.857143in}}{\pgfqpoint{2.627103in}{1.813434in}}%
\pgfusepath{clip}%
\pgfsetbuttcap%
\pgfsetmiterjoin%
\definecolor{currentfill}{rgb}{0.066899,0.263188,0.377594}%
\pgfsetfillcolor{currentfill}%
\pgfsetlinewidth{0.000000pt}%
\definecolor{currentstroke}{rgb}{0.000000,0.000000,0.000000}%
\pgfsetstrokecolor{currentstroke}%
\pgfsetstrokeopacity{0.000000}%
\pgfsetdash{}{0pt}%
\pgfpathmoveto{\pgfqpoint{1.698916in}{1.847071in}}%
\pgfpathlineto{\pgfqpoint{1.707852in}{1.847071in}}%
\pgfpathlineto{\pgfqpoint{1.707852in}{1.819826in}}%
\pgfpathlineto{\pgfqpoint{1.698916in}{1.819826in}}%
\pgfpathlineto{\pgfqpoint{1.698916in}{1.847071in}}%
\pgfpathclose%
\pgfusepath{fill}%
\end{pgfscope}%
\begin{pgfscope}%
\pgfpathrectangle{\pgfqpoint{0.697024in}{0.857143in}}{\pgfqpoint{2.627103in}{1.813434in}}%
\pgfusepath{clip}%
\pgfsetbuttcap%
\pgfsetmiterjoin%
\definecolor{currentfill}{rgb}{0.066899,0.263188,0.377594}%
\pgfsetfillcolor{currentfill}%
\pgfsetlinewidth{0.000000pt}%
\definecolor{currentstroke}{rgb}{0.000000,0.000000,0.000000}%
\pgfsetstrokecolor{currentstroke}%
\pgfsetstrokeopacity{0.000000}%
\pgfsetdash{}{0pt}%
\pgfpathmoveto{\pgfqpoint{1.710087in}{1.847071in}}%
\pgfpathlineto{\pgfqpoint{1.719023in}{1.847071in}}%
\pgfpathlineto{\pgfqpoint{1.719023in}{1.818994in}}%
\pgfpathlineto{\pgfqpoint{1.710087in}{1.818994in}}%
\pgfpathlineto{\pgfqpoint{1.710087in}{1.847071in}}%
\pgfpathclose%
\pgfusepath{fill}%
\end{pgfscope}%
\begin{pgfscope}%
\pgfpathrectangle{\pgfqpoint{0.697024in}{0.857143in}}{\pgfqpoint{2.627103in}{1.813434in}}%
\pgfusepath{clip}%
\pgfsetbuttcap%
\pgfsetmiterjoin%
\definecolor{currentfill}{rgb}{0.066899,0.263188,0.377594}%
\pgfsetfillcolor{currentfill}%
\pgfsetlinewidth{0.000000pt}%
\definecolor{currentstroke}{rgb}{0.000000,0.000000,0.000000}%
\pgfsetstrokecolor{currentstroke}%
\pgfsetstrokeopacity{0.000000}%
\pgfsetdash{}{0pt}%
\pgfpathmoveto{\pgfqpoint{1.721257in}{1.847071in}}%
\pgfpathlineto{\pgfqpoint{1.730194in}{1.847071in}}%
\pgfpathlineto{\pgfqpoint{1.730194in}{1.799383in}}%
\pgfpathlineto{\pgfqpoint{1.721257in}{1.799383in}}%
\pgfpathlineto{\pgfqpoint{1.721257in}{1.847071in}}%
\pgfpathclose%
\pgfusepath{fill}%
\end{pgfscope}%
\begin{pgfscope}%
\pgfpathrectangle{\pgfqpoint{0.697024in}{0.857143in}}{\pgfqpoint{2.627103in}{1.813434in}}%
\pgfusepath{clip}%
\pgfsetbuttcap%
\pgfsetmiterjoin%
\definecolor{currentfill}{rgb}{0.066899,0.263188,0.377594}%
\pgfsetfillcolor{currentfill}%
\pgfsetlinewidth{0.000000pt}%
\definecolor{currentstroke}{rgb}{0.000000,0.000000,0.000000}%
\pgfsetstrokecolor{currentstroke}%
\pgfsetstrokeopacity{0.000000}%
\pgfsetdash{}{0pt}%
\pgfpathmoveto{\pgfqpoint{1.732428in}{1.847071in}}%
\pgfpathlineto{\pgfqpoint{1.741364in}{1.847071in}}%
\pgfpathlineto{\pgfqpoint{1.741364in}{1.774187in}}%
\pgfpathlineto{\pgfqpoint{1.732428in}{1.774187in}}%
\pgfpathlineto{\pgfqpoint{1.732428in}{1.847071in}}%
\pgfpathclose%
\pgfusepath{fill}%
\end{pgfscope}%
\begin{pgfscope}%
\pgfpathrectangle{\pgfqpoint{0.697024in}{0.857143in}}{\pgfqpoint{2.627103in}{1.813434in}}%
\pgfusepath{clip}%
\pgfsetbuttcap%
\pgfsetmiterjoin%
\definecolor{currentfill}{rgb}{0.066899,0.263188,0.377594}%
\pgfsetfillcolor{currentfill}%
\pgfsetlinewidth{0.000000pt}%
\definecolor{currentstroke}{rgb}{0.000000,0.000000,0.000000}%
\pgfsetstrokecolor{currentstroke}%
\pgfsetstrokeopacity{0.000000}%
\pgfsetdash{}{0pt}%
\pgfpathmoveto{\pgfqpoint{1.743598in}{1.847071in}}%
\pgfpathlineto{\pgfqpoint{1.752535in}{1.847071in}}%
\pgfpathlineto{\pgfqpoint{1.752535in}{1.781250in}}%
\pgfpathlineto{\pgfqpoint{1.743598in}{1.781250in}}%
\pgfpathlineto{\pgfqpoint{1.743598in}{1.847071in}}%
\pgfpathclose%
\pgfusepath{fill}%
\end{pgfscope}%
\begin{pgfscope}%
\pgfpathrectangle{\pgfqpoint{0.697024in}{0.857143in}}{\pgfqpoint{2.627103in}{1.813434in}}%
\pgfusepath{clip}%
\pgfsetbuttcap%
\pgfsetmiterjoin%
\definecolor{currentfill}{rgb}{0.066899,0.263188,0.377594}%
\pgfsetfillcolor{currentfill}%
\pgfsetlinewidth{0.000000pt}%
\definecolor{currentstroke}{rgb}{0.000000,0.000000,0.000000}%
\pgfsetstrokecolor{currentstroke}%
\pgfsetstrokeopacity{0.000000}%
\pgfsetdash{}{0pt}%
\pgfpathmoveto{\pgfqpoint{1.754769in}{1.847071in}}%
\pgfpathlineto{\pgfqpoint{1.763705in}{1.847071in}}%
\pgfpathlineto{\pgfqpoint{1.763705in}{1.775403in}}%
\pgfpathlineto{\pgfqpoint{1.754769in}{1.775403in}}%
\pgfpathlineto{\pgfqpoint{1.754769in}{1.847071in}}%
\pgfpathclose%
\pgfusepath{fill}%
\end{pgfscope}%
\begin{pgfscope}%
\pgfpathrectangle{\pgfqpoint{0.697024in}{0.857143in}}{\pgfqpoint{2.627103in}{1.813434in}}%
\pgfusepath{clip}%
\pgfsetbuttcap%
\pgfsetmiterjoin%
\definecolor{currentfill}{rgb}{0.066899,0.263188,0.377594}%
\pgfsetfillcolor{currentfill}%
\pgfsetlinewidth{0.000000pt}%
\definecolor{currentstroke}{rgb}{0.000000,0.000000,0.000000}%
\pgfsetstrokecolor{currentstroke}%
\pgfsetstrokeopacity{0.000000}%
\pgfsetdash{}{0pt}%
\pgfpathmoveto{\pgfqpoint{1.765940in}{1.847071in}}%
\pgfpathlineto{\pgfqpoint{1.774876in}{1.847071in}}%
\pgfpathlineto{\pgfqpoint{1.774876in}{1.784828in}}%
\pgfpathlineto{\pgfqpoint{1.765940in}{1.784828in}}%
\pgfpathlineto{\pgfqpoint{1.765940in}{1.847071in}}%
\pgfpathclose%
\pgfusepath{fill}%
\end{pgfscope}%
\begin{pgfscope}%
\pgfpathrectangle{\pgfqpoint{0.697024in}{0.857143in}}{\pgfqpoint{2.627103in}{1.813434in}}%
\pgfusepath{clip}%
\pgfsetbuttcap%
\pgfsetmiterjoin%
\definecolor{currentfill}{rgb}{0.066899,0.263188,0.377594}%
\pgfsetfillcolor{currentfill}%
\pgfsetlinewidth{0.000000pt}%
\definecolor{currentstroke}{rgb}{0.000000,0.000000,0.000000}%
\pgfsetstrokecolor{currentstroke}%
\pgfsetstrokeopacity{0.000000}%
\pgfsetdash{}{0pt}%
\pgfpathmoveto{\pgfqpoint{1.777110in}{1.847071in}}%
\pgfpathlineto{\pgfqpoint{1.786047in}{1.847071in}}%
\pgfpathlineto{\pgfqpoint{1.786047in}{1.784930in}}%
\pgfpathlineto{\pgfqpoint{1.777110in}{1.784930in}}%
\pgfpathlineto{\pgfqpoint{1.777110in}{1.847071in}}%
\pgfpathclose%
\pgfusepath{fill}%
\end{pgfscope}%
\begin{pgfscope}%
\pgfpathrectangle{\pgfqpoint{0.697024in}{0.857143in}}{\pgfqpoint{2.627103in}{1.813434in}}%
\pgfusepath{clip}%
\pgfsetbuttcap%
\pgfsetmiterjoin%
\definecolor{currentfill}{rgb}{0.066899,0.263188,0.377594}%
\pgfsetfillcolor{currentfill}%
\pgfsetlinewidth{0.000000pt}%
\definecolor{currentstroke}{rgb}{0.000000,0.000000,0.000000}%
\pgfsetstrokecolor{currentstroke}%
\pgfsetstrokeopacity{0.000000}%
\pgfsetdash{}{0pt}%
\pgfpathmoveto{\pgfqpoint{1.788281in}{1.847071in}}%
\pgfpathlineto{\pgfqpoint{1.797217in}{1.847071in}}%
\pgfpathlineto{\pgfqpoint{1.797217in}{1.769753in}}%
\pgfpathlineto{\pgfqpoint{1.788281in}{1.769753in}}%
\pgfpathlineto{\pgfqpoint{1.788281in}{1.847071in}}%
\pgfpathclose%
\pgfusepath{fill}%
\end{pgfscope}%
\begin{pgfscope}%
\pgfpathrectangle{\pgfqpoint{0.697024in}{0.857143in}}{\pgfqpoint{2.627103in}{1.813434in}}%
\pgfusepath{clip}%
\pgfsetbuttcap%
\pgfsetmiterjoin%
\definecolor{currentfill}{rgb}{0.066899,0.263188,0.377594}%
\pgfsetfillcolor{currentfill}%
\pgfsetlinewidth{0.000000pt}%
\definecolor{currentstroke}{rgb}{0.000000,0.000000,0.000000}%
\pgfsetstrokecolor{currentstroke}%
\pgfsetstrokeopacity{0.000000}%
\pgfsetdash{}{0pt}%
\pgfpathmoveto{\pgfqpoint{1.799451in}{1.847071in}}%
\pgfpathlineto{\pgfqpoint{1.808388in}{1.847071in}}%
\pgfpathlineto{\pgfqpoint{1.808388in}{1.768824in}}%
\pgfpathlineto{\pgfqpoint{1.799451in}{1.768824in}}%
\pgfpathlineto{\pgfqpoint{1.799451in}{1.847071in}}%
\pgfpathclose%
\pgfusepath{fill}%
\end{pgfscope}%
\begin{pgfscope}%
\pgfpathrectangle{\pgfqpoint{0.697024in}{0.857143in}}{\pgfqpoint{2.627103in}{1.813434in}}%
\pgfusepath{clip}%
\pgfsetbuttcap%
\pgfsetmiterjoin%
\definecolor{currentfill}{rgb}{0.066899,0.263188,0.377594}%
\pgfsetfillcolor{currentfill}%
\pgfsetlinewidth{0.000000pt}%
\definecolor{currentstroke}{rgb}{0.000000,0.000000,0.000000}%
\pgfsetstrokecolor{currentstroke}%
\pgfsetstrokeopacity{0.000000}%
\pgfsetdash{}{0pt}%
\pgfpathmoveto{\pgfqpoint{1.810622in}{1.847071in}}%
\pgfpathlineto{\pgfqpoint{1.819559in}{1.847071in}}%
\pgfpathlineto{\pgfqpoint{1.819559in}{1.763873in}}%
\pgfpathlineto{\pgfqpoint{1.810622in}{1.763873in}}%
\pgfpathlineto{\pgfqpoint{1.810622in}{1.847071in}}%
\pgfpathclose%
\pgfusepath{fill}%
\end{pgfscope}%
\begin{pgfscope}%
\pgfpathrectangle{\pgfqpoint{0.697024in}{0.857143in}}{\pgfqpoint{2.627103in}{1.813434in}}%
\pgfusepath{clip}%
\pgfsetbuttcap%
\pgfsetmiterjoin%
\definecolor{currentfill}{rgb}{0.066899,0.263188,0.377594}%
\pgfsetfillcolor{currentfill}%
\pgfsetlinewidth{0.000000pt}%
\definecolor{currentstroke}{rgb}{0.000000,0.000000,0.000000}%
\pgfsetstrokecolor{currentstroke}%
\pgfsetstrokeopacity{0.000000}%
\pgfsetdash{}{0pt}%
\pgfpathmoveto{\pgfqpoint{1.821793in}{1.847071in}}%
\pgfpathlineto{\pgfqpoint{1.830729in}{1.847071in}}%
\pgfpathlineto{\pgfqpoint{1.830729in}{1.753261in}}%
\pgfpathlineto{\pgfqpoint{1.821793in}{1.753261in}}%
\pgfpathlineto{\pgfqpoint{1.821793in}{1.847071in}}%
\pgfpathclose%
\pgfusepath{fill}%
\end{pgfscope}%
\begin{pgfscope}%
\pgfpathrectangle{\pgfqpoint{0.697024in}{0.857143in}}{\pgfqpoint{2.627103in}{1.813434in}}%
\pgfusepath{clip}%
\pgfsetbuttcap%
\pgfsetmiterjoin%
\definecolor{currentfill}{rgb}{0.066899,0.263188,0.377594}%
\pgfsetfillcolor{currentfill}%
\pgfsetlinewidth{0.000000pt}%
\definecolor{currentstroke}{rgb}{0.000000,0.000000,0.000000}%
\pgfsetstrokecolor{currentstroke}%
\pgfsetstrokeopacity{0.000000}%
\pgfsetdash{}{0pt}%
\pgfpathmoveto{\pgfqpoint{1.832963in}{1.847071in}}%
\pgfpathlineto{\pgfqpoint{1.841900in}{1.847071in}}%
\pgfpathlineto{\pgfqpoint{1.841900in}{1.754186in}}%
\pgfpathlineto{\pgfqpoint{1.832963in}{1.754186in}}%
\pgfpathlineto{\pgfqpoint{1.832963in}{1.847071in}}%
\pgfpathclose%
\pgfusepath{fill}%
\end{pgfscope}%
\begin{pgfscope}%
\pgfpathrectangle{\pgfqpoint{0.697024in}{0.857143in}}{\pgfqpoint{2.627103in}{1.813434in}}%
\pgfusepath{clip}%
\pgfsetbuttcap%
\pgfsetmiterjoin%
\definecolor{currentfill}{rgb}{0.066899,0.263188,0.377594}%
\pgfsetfillcolor{currentfill}%
\pgfsetlinewidth{0.000000pt}%
\definecolor{currentstroke}{rgb}{0.000000,0.000000,0.000000}%
\pgfsetstrokecolor{currentstroke}%
\pgfsetstrokeopacity{0.000000}%
\pgfsetdash{}{0pt}%
\pgfpathmoveto{\pgfqpoint{1.844134in}{1.847071in}}%
\pgfpathlineto{\pgfqpoint{1.853070in}{1.847071in}}%
\pgfpathlineto{\pgfqpoint{1.853070in}{1.758191in}}%
\pgfpathlineto{\pgfqpoint{1.844134in}{1.758191in}}%
\pgfpathlineto{\pgfqpoint{1.844134in}{1.847071in}}%
\pgfpathclose%
\pgfusepath{fill}%
\end{pgfscope}%
\begin{pgfscope}%
\pgfpathrectangle{\pgfqpoint{0.697024in}{0.857143in}}{\pgfqpoint{2.627103in}{1.813434in}}%
\pgfusepath{clip}%
\pgfsetbuttcap%
\pgfsetmiterjoin%
\definecolor{currentfill}{rgb}{0.066899,0.263188,0.377594}%
\pgfsetfillcolor{currentfill}%
\pgfsetlinewidth{0.000000pt}%
\definecolor{currentstroke}{rgb}{0.000000,0.000000,0.000000}%
\pgfsetstrokecolor{currentstroke}%
\pgfsetstrokeopacity{0.000000}%
\pgfsetdash{}{0pt}%
\pgfpathmoveto{\pgfqpoint{1.855304in}{1.847071in}}%
\pgfpathlineto{\pgfqpoint{1.864241in}{1.847071in}}%
\pgfpathlineto{\pgfqpoint{1.864241in}{1.751687in}}%
\pgfpathlineto{\pgfqpoint{1.855304in}{1.751687in}}%
\pgfpathlineto{\pgfqpoint{1.855304in}{1.847071in}}%
\pgfpathclose%
\pgfusepath{fill}%
\end{pgfscope}%
\begin{pgfscope}%
\pgfpathrectangle{\pgfqpoint{0.697024in}{0.857143in}}{\pgfqpoint{2.627103in}{1.813434in}}%
\pgfusepath{clip}%
\pgfsetbuttcap%
\pgfsetmiterjoin%
\definecolor{currentfill}{rgb}{0.066899,0.263188,0.377594}%
\pgfsetfillcolor{currentfill}%
\pgfsetlinewidth{0.000000pt}%
\definecolor{currentstroke}{rgb}{0.000000,0.000000,0.000000}%
\pgfsetstrokecolor{currentstroke}%
\pgfsetstrokeopacity{0.000000}%
\pgfsetdash{}{0pt}%
\pgfpathmoveto{\pgfqpoint{1.866475in}{1.847071in}}%
\pgfpathlineto{\pgfqpoint{1.875412in}{1.847071in}}%
\pgfpathlineto{\pgfqpoint{1.875412in}{1.764080in}}%
\pgfpathlineto{\pgfqpoint{1.866475in}{1.764080in}}%
\pgfpathlineto{\pgfqpoint{1.866475in}{1.847071in}}%
\pgfpathclose%
\pgfusepath{fill}%
\end{pgfscope}%
\begin{pgfscope}%
\pgfpathrectangle{\pgfqpoint{0.697024in}{0.857143in}}{\pgfqpoint{2.627103in}{1.813434in}}%
\pgfusepath{clip}%
\pgfsetbuttcap%
\pgfsetmiterjoin%
\definecolor{currentfill}{rgb}{0.066899,0.263188,0.377594}%
\pgfsetfillcolor{currentfill}%
\pgfsetlinewidth{0.000000pt}%
\definecolor{currentstroke}{rgb}{0.000000,0.000000,0.000000}%
\pgfsetstrokecolor{currentstroke}%
\pgfsetstrokeopacity{0.000000}%
\pgfsetdash{}{0pt}%
\pgfpathmoveto{\pgfqpoint{1.877646in}{1.847071in}}%
\pgfpathlineto{\pgfqpoint{1.886582in}{1.847071in}}%
\pgfpathlineto{\pgfqpoint{1.886582in}{1.774717in}}%
\pgfpathlineto{\pgfqpoint{1.877646in}{1.774717in}}%
\pgfpathlineto{\pgfqpoint{1.877646in}{1.847071in}}%
\pgfpathclose%
\pgfusepath{fill}%
\end{pgfscope}%
\begin{pgfscope}%
\pgfpathrectangle{\pgfqpoint{0.697024in}{0.857143in}}{\pgfqpoint{2.627103in}{1.813434in}}%
\pgfusepath{clip}%
\pgfsetbuttcap%
\pgfsetmiterjoin%
\definecolor{currentfill}{rgb}{0.066899,0.263188,0.377594}%
\pgfsetfillcolor{currentfill}%
\pgfsetlinewidth{0.000000pt}%
\definecolor{currentstroke}{rgb}{0.000000,0.000000,0.000000}%
\pgfsetstrokecolor{currentstroke}%
\pgfsetstrokeopacity{0.000000}%
\pgfsetdash{}{0pt}%
\pgfpathmoveto{\pgfqpoint{1.888816in}{1.847071in}}%
\pgfpathlineto{\pgfqpoint{1.897753in}{1.847071in}}%
\pgfpathlineto{\pgfqpoint{1.897753in}{1.778016in}}%
\pgfpathlineto{\pgfqpoint{1.888816in}{1.778016in}}%
\pgfpathlineto{\pgfqpoint{1.888816in}{1.847071in}}%
\pgfpathclose%
\pgfusepath{fill}%
\end{pgfscope}%
\begin{pgfscope}%
\pgfpathrectangle{\pgfqpoint{0.697024in}{0.857143in}}{\pgfqpoint{2.627103in}{1.813434in}}%
\pgfusepath{clip}%
\pgfsetbuttcap%
\pgfsetmiterjoin%
\definecolor{currentfill}{rgb}{0.066899,0.263188,0.377594}%
\pgfsetfillcolor{currentfill}%
\pgfsetlinewidth{0.000000pt}%
\definecolor{currentstroke}{rgb}{0.000000,0.000000,0.000000}%
\pgfsetstrokecolor{currentstroke}%
\pgfsetstrokeopacity{0.000000}%
\pgfsetdash{}{0pt}%
\pgfpathmoveto{\pgfqpoint{1.899987in}{1.847071in}}%
\pgfpathlineto{\pgfqpoint{1.908923in}{1.847071in}}%
\pgfpathlineto{\pgfqpoint{1.908923in}{1.749890in}}%
\pgfpathlineto{\pgfqpoint{1.899987in}{1.749890in}}%
\pgfpathlineto{\pgfqpoint{1.899987in}{1.847071in}}%
\pgfpathclose%
\pgfusepath{fill}%
\end{pgfscope}%
\begin{pgfscope}%
\pgfpathrectangle{\pgfqpoint{0.697024in}{0.857143in}}{\pgfqpoint{2.627103in}{1.813434in}}%
\pgfusepath{clip}%
\pgfsetbuttcap%
\pgfsetmiterjoin%
\definecolor{currentfill}{rgb}{0.066899,0.263188,0.377594}%
\pgfsetfillcolor{currentfill}%
\pgfsetlinewidth{0.000000pt}%
\definecolor{currentstroke}{rgb}{0.000000,0.000000,0.000000}%
\pgfsetstrokecolor{currentstroke}%
\pgfsetstrokeopacity{0.000000}%
\pgfsetdash{}{0pt}%
\pgfpathmoveto{\pgfqpoint{1.911157in}{1.847071in}}%
\pgfpathlineto{\pgfqpoint{1.920094in}{1.847071in}}%
\pgfpathlineto{\pgfqpoint{1.920094in}{1.756017in}}%
\pgfpathlineto{\pgfqpoint{1.911157in}{1.756017in}}%
\pgfpathlineto{\pgfqpoint{1.911157in}{1.847071in}}%
\pgfpathclose%
\pgfusepath{fill}%
\end{pgfscope}%
\begin{pgfscope}%
\pgfpathrectangle{\pgfqpoint{0.697024in}{0.857143in}}{\pgfqpoint{2.627103in}{1.813434in}}%
\pgfusepath{clip}%
\pgfsetbuttcap%
\pgfsetmiterjoin%
\definecolor{currentfill}{rgb}{0.066899,0.263188,0.377594}%
\pgfsetfillcolor{currentfill}%
\pgfsetlinewidth{0.000000pt}%
\definecolor{currentstroke}{rgb}{0.000000,0.000000,0.000000}%
\pgfsetstrokecolor{currentstroke}%
\pgfsetstrokeopacity{0.000000}%
\pgfsetdash{}{0pt}%
\pgfpathmoveto{\pgfqpoint{1.922328in}{1.847071in}}%
\pgfpathlineto{\pgfqpoint{1.931265in}{1.847071in}}%
\pgfpathlineto{\pgfqpoint{1.931265in}{1.786759in}}%
\pgfpathlineto{\pgfqpoint{1.922328in}{1.786759in}}%
\pgfpathlineto{\pgfqpoint{1.922328in}{1.847071in}}%
\pgfpathclose%
\pgfusepath{fill}%
\end{pgfscope}%
\begin{pgfscope}%
\pgfpathrectangle{\pgfqpoint{0.697024in}{0.857143in}}{\pgfqpoint{2.627103in}{1.813434in}}%
\pgfusepath{clip}%
\pgfsetbuttcap%
\pgfsetmiterjoin%
\definecolor{currentfill}{rgb}{0.066899,0.263188,0.377594}%
\pgfsetfillcolor{currentfill}%
\pgfsetlinewidth{0.000000pt}%
\definecolor{currentstroke}{rgb}{0.000000,0.000000,0.000000}%
\pgfsetstrokecolor{currentstroke}%
\pgfsetstrokeopacity{0.000000}%
\pgfsetdash{}{0pt}%
\pgfpathmoveto{\pgfqpoint{1.933499in}{1.847071in}}%
\pgfpathlineto{\pgfqpoint{1.942435in}{1.847071in}}%
\pgfpathlineto{\pgfqpoint{1.942435in}{1.789748in}}%
\pgfpathlineto{\pgfqpoint{1.933499in}{1.789748in}}%
\pgfpathlineto{\pgfqpoint{1.933499in}{1.847071in}}%
\pgfpathclose%
\pgfusepath{fill}%
\end{pgfscope}%
\begin{pgfscope}%
\pgfpathrectangle{\pgfqpoint{0.697024in}{0.857143in}}{\pgfqpoint{2.627103in}{1.813434in}}%
\pgfusepath{clip}%
\pgfsetbuttcap%
\pgfsetmiterjoin%
\definecolor{currentfill}{rgb}{0.066899,0.263188,0.377594}%
\pgfsetfillcolor{currentfill}%
\pgfsetlinewidth{0.000000pt}%
\definecolor{currentstroke}{rgb}{0.000000,0.000000,0.000000}%
\pgfsetstrokecolor{currentstroke}%
\pgfsetstrokeopacity{0.000000}%
\pgfsetdash{}{0pt}%
\pgfpathmoveto{\pgfqpoint{1.944669in}{1.847071in}}%
\pgfpathlineto{\pgfqpoint{1.953606in}{1.847071in}}%
\pgfpathlineto{\pgfqpoint{1.953606in}{1.791044in}}%
\pgfpathlineto{\pgfqpoint{1.944669in}{1.791044in}}%
\pgfpathlineto{\pgfqpoint{1.944669in}{1.847071in}}%
\pgfpathclose%
\pgfusepath{fill}%
\end{pgfscope}%
\begin{pgfscope}%
\pgfpathrectangle{\pgfqpoint{0.697024in}{0.857143in}}{\pgfqpoint{2.627103in}{1.813434in}}%
\pgfusepath{clip}%
\pgfsetbuttcap%
\pgfsetmiterjoin%
\definecolor{currentfill}{rgb}{0.066899,0.263188,0.377594}%
\pgfsetfillcolor{currentfill}%
\pgfsetlinewidth{0.000000pt}%
\definecolor{currentstroke}{rgb}{0.000000,0.000000,0.000000}%
\pgfsetstrokecolor{currentstroke}%
\pgfsetstrokeopacity{0.000000}%
\pgfsetdash{}{0pt}%
\pgfpathmoveto{\pgfqpoint{1.955840in}{1.847071in}}%
\pgfpathlineto{\pgfqpoint{1.964776in}{1.847071in}}%
\pgfpathlineto{\pgfqpoint{1.964776in}{1.831225in}}%
\pgfpathlineto{\pgfqpoint{1.955840in}{1.831225in}}%
\pgfpathlineto{\pgfqpoint{1.955840in}{1.847071in}}%
\pgfpathclose%
\pgfusepath{fill}%
\end{pgfscope}%
\begin{pgfscope}%
\pgfpathrectangle{\pgfqpoint{0.697024in}{0.857143in}}{\pgfqpoint{2.627103in}{1.813434in}}%
\pgfusepath{clip}%
\pgfsetbuttcap%
\pgfsetmiterjoin%
\definecolor{currentfill}{rgb}{0.066899,0.263188,0.377594}%
\pgfsetfillcolor{currentfill}%
\pgfsetlinewidth{0.000000pt}%
\definecolor{currentstroke}{rgb}{0.000000,0.000000,0.000000}%
\pgfsetstrokecolor{currentstroke}%
\pgfsetstrokeopacity{0.000000}%
\pgfsetdash{}{0pt}%
\pgfpathmoveto{\pgfqpoint{1.967011in}{1.847071in}}%
\pgfpathlineto{\pgfqpoint{1.975947in}{1.847071in}}%
\pgfpathlineto{\pgfqpoint{1.975947in}{1.834262in}}%
\pgfpathlineto{\pgfqpoint{1.967011in}{1.834262in}}%
\pgfpathlineto{\pgfqpoint{1.967011in}{1.847071in}}%
\pgfpathclose%
\pgfusepath{fill}%
\end{pgfscope}%
\begin{pgfscope}%
\pgfpathrectangle{\pgfqpoint{0.697024in}{0.857143in}}{\pgfqpoint{2.627103in}{1.813434in}}%
\pgfusepath{clip}%
\pgfsetbuttcap%
\pgfsetmiterjoin%
\definecolor{currentfill}{rgb}{0.066899,0.263188,0.377594}%
\pgfsetfillcolor{currentfill}%
\pgfsetlinewidth{0.000000pt}%
\definecolor{currentstroke}{rgb}{0.000000,0.000000,0.000000}%
\pgfsetstrokecolor{currentstroke}%
\pgfsetstrokeopacity{0.000000}%
\pgfsetdash{}{0pt}%
\pgfpathmoveto{\pgfqpoint{1.978181in}{1.847071in}}%
\pgfpathlineto{\pgfqpoint{1.987118in}{1.847071in}}%
\pgfpathlineto{\pgfqpoint{1.987118in}{1.843485in}}%
\pgfpathlineto{\pgfqpoint{1.978181in}{1.843485in}}%
\pgfpathlineto{\pgfqpoint{1.978181in}{1.847071in}}%
\pgfpathclose%
\pgfusepath{fill}%
\end{pgfscope}%
\begin{pgfscope}%
\pgfpathrectangle{\pgfqpoint{0.697024in}{0.857143in}}{\pgfqpoint{2.627103in}{1.813434in}}%
\pgfusepath{clip}%
\pgfsetbuttcap%
\pgfsetmiterjoin%
\definecolor{currentfill}{rgb}{0.066899,0.263188,0.377594}%
\pgfsetfillcolor{currentfill}%
\pgfsetlinewidth{0.000000pt}%
\definecolor{currentstroke}{rgb}{0.000000,0.000000,0.000000}%
\pgfsetstrokecolor{currentstroke}%
\pgfsetstrokeopacity{0.000000}%
\pgfsetdash{}{0pt}%
\pgfpathmoveto{\pgfqpoint{1.989352in}{1.847071in}}%
\pgfpathlineto{\pgfqpoint{1.998288in}{1.847071in}}%
\pgfpathlineto{\pgfqpoint{1.998288in}{1.840641in}}%
\pgfpathlineto{\pgfqpoint{1.989352in}{1.840641in}}%
\pgfpathlineto{\pgfqpoint{1.989352in}{1.847071in}}%
\pgfpathclose%
\pgfusepath{fill}%
\end{pgfscope}%
\begin{pgfscope}%
\pgfpathrectangle{\pgfqpoint{0.697024in}{0.857143in}}{\pgfqpoint{2.627103in}{1.813434in}}%
\pgfusepath{clip}%
\pgfsetbuttcap%
\pgfsetmiterjoin%
\definecolor{currentfill}{rgb}{0.066899,0.263188,0.377594}%
\pgfsetfillcolor{currentfill}%
\pgfsetlinewidth{0.000000pt}%
\definecolor{currentstroke}{rgb}{0.000000,0.000000,0.000000}%
\pgfsetstrokecolor{currentstroke}%
\pgfsetstrokeopacity{0.000000}%
\pgfsetdash{}{0pt}%
\pgfpathmoveto{\pgfqpoint{2.000522in}{1.847071in}}%
\pgfpathlineto{\pgfqpoint{2.009459in}{1.847071in}}%
\pgfpathlineto{\pgfqpoint{2.009459in}{1.806091in}}%
\pgfpathlineto{\pgfqpoint{2.000522in}{1.806091in}}%
\pgfpathlineto{\pgfqpoint{2.000522in}{1.847071in}}%
\pgfpathclose%
\pgfusepath{fill}%
\end{pgfscope}%
\begin{pgfscope}%
\pgfpathrectangle{\pgfqpoint{0.697024in}{0.857143in}}{\pgfqpoint{2.627103in}{1.813434in}}%
\pgfusepath{clip}%
\pgfsetbuttcap%
\pgfsetmiterjoin%
\definecolor{currentfill}{rgb}{0.066899,0.263188,0.377594}%
\pgfsetfillcolor{currentfill}%
\pgfsetlinewidth{0.000000pt}%
\definecolor{currentstroke}{rgb}{0.000000,0.000000,0.000000}%
\pgfsetstrokecolor{currentstroke}%
\pgfsetstrokeopacity{0.000000}%
\pgfsetdash{}{0pt}%
\pgfpathmoveto{\pgfqpoint{2.011693in}{1.847071in}}%
\pgfpathlineto{\pgfqpoint{2.020629in}{1.847071in}}%
\pgfpathlineto{\pgfqpoint{2.020629in}{1.781540in}}%
\pgfpathlineto{\pgfqpoint{2.011693in}{1.781540in}}%
\pgfpathlineto{\pgfqpoint{2.011693in}{1.847071in}}%
\pgfpathclose%
\pgfusepath{fill}%
\end{pgfscope}%
\begin{pgfscope}%
\pgfpathrectangle{\pgfqpoint{0.697024in}{0.857143in}}{\pgfqpoint{2.627103in}{1.813434in}}%
\pgfusepath{clip}%
\pgfsetbuttcap%
\pgfsetmiterjoin%
\definecolor{currentfill}{rgb}{0.066899,0.263188,0.377594}%
\pgfsetfillcolor{currentfill}%
\pgfsetlinewidth{0.000000pt}%
\definecolor{currentstroke}{rgb}{0.000000,0.000000,0.000000}%
\pgfsetstrokecolor{currentstroke}%
\pgfsetstrokeopacity{0.000000}%
\pgfsetdash{}{0pt}%
\pgfpathmoveto{\pgfqpoint{2.022864in}{1.847071in}}%
\pgfpathlineto{\pgfqpoint{2.031800in}{1.847071in}}%
\pgfpathlineto{\pgfqpoint{2.031800in}{1.770905in}}%
\pgfpathlineto{\pgfqpoint{2.022864in}{1.770905in}}%
\pgfpathlineto{\pgfqpoint{2.022864in}{1.847071in}}%
\pgfpathclose%
\pgfusepath{fill}%
\end{pgfscope}%
\begin{pgfscope}%
\pgfpathrectangle{\pgfqpoint{0.697024in}{0.857143in}}{\pgfqpoint{2.627103in}{1.813434in}}%
\pgfusepath{clip}%
\pgfsetbuttcap%
\pgfsetmiterjoin%
\definecolor{currentfill}{rgb}{0.066899,0.263188,0.377594}%
\pgfsetfillcolor{currentfill}%
\pgfsetlinewidth{0.000000pt}%
\definecolor{currentstroke}{rgb}{0.000000,0.000000,0.000000}%
\pgfsetstrokecolor{currentstroke}%
\pgfsetstrokeopacity{0.000000}%
\pgfsetdash{}{0pt}%
\pgfpathmoveto{\pgfqpoint{2.034034in}{1.847071in}}%
\pgfpathlineto{\pgfqpoint{2.042971in}{1.847071in}}%
\pgfpathlineto{\pgfqpoint{2.042971in}{1.774830in}}%
\pgfpathlineto{\pgfqpoint{2.034034in}{1.774830in}}%
\pgfpathlineto{\pgfqpoint{2.034034in}{1.847071in}}%
\pgfpathclose%
\pgfusepath{fill}%
\end{pgfscope}%
\begin{pgfscope}%
\pgfpathrectangle{\pgfqpoint{0.697024in}{0.857143in}}{\pgfqpoint{2.627103in}{1.813434in}}%
\pgfusepath{clip}%
\pgfsetbuttcap%
\pgfsetmiterjoin%
\definecolor{currentfill}{rgb}{0.066899,0.263188,0.377594}%
\pgfsetfillcolor{currentfill}%
\pgfsetlinewidth{0.000000pt}%
\definecolor{currentstroke}{rgb}{0.000000,0.000000,0.000000}%
\pgfsetstrokecolor{currentstroke}%
\pgfsetstrokeopacity{0.000000}%
\pgfsetdash{}{0pt}%
\pgfpathmoveto{\pgfqpoint{2.045205in}{1.847071in}}%
\pgfpathlineto{\pgfqpoint{2.054141in}{1.847071in}}%
\pgfpathlineto{\pgfqpoint{2.054141in}{1.774470in}}%
\pgfpathlineto{\pgfqpoint{2.045205in}{1.774470in}}%
\pgfpathlineto{\pgfqpoint{2.045205in}{1.847071in}}%
\pgfpathclose%
\pgfusepath{fill}%
\end{pgfscope}%
\begin{pgfscope}%
\pgfpathrectangle{\pgfqpoint{0.697024in}{0.857143in}}{\pgfqpoint{2.627103in}{1.813434in}}%
\pgfusepath{clip}%
\pgfsetbuttcap%
\pgfsetmiterjoin%
\definecolor{currentfill}{rgb}{0.066899,0.263188,0.377594}%
\pgfsetfillcolor{currentfill}%
\pgfsetlinewidth{0.000000pt}%
\definecolor{currentstroke}{rgb}{0.000000,0.000000,0.000000}%
\pgfsetstrokecolor{currentstroke}%
\pgfsetstrokeopacity{0.000000}%
\pgfsetdash{}{0pt}%
\pgfpathmoveto{\pgfqpoint{2.056375in}{1.847071in}}%
\pgfpathlineto{\pgfqpoint{2.065312in}{1.847071in}}%
\pgfpathlineto{\pgfqpoint{2.065312in}{1.755294in}}%
\pgfpathlineto{\pgfqpoint{2.056375in}{1.755294in}}%
\pgfpathlineto{\pgfqpoint{2.056375in}{1.847071in}}%
\pgfpathclose%
\pgfusepath{fill}%
\end{pgfscope}%
\begin{pgfscope}%
\pgfpathrectangle{\pgfqpoint{0.697024in}{0.857143in}}{\pgfqpoint{2.627103in}{1.813434in}}%
\pgfusepath{clip}%
\pgfsetbuttcap%
\pgfsetmiterjoin%
\definecolor{currentfill}{rgb}{0.066899,0.263188,0.377594}%
\pgfsetfillcolor{currentfill}%
\pgfsetlinewidth{0.000000pt}%
\definecolor{currentstroke}{rgb}{0.000000,0.000000,0.000000}%
\pgfsetstrokecolor{currentstroke}%
\pgfsetstrokeopacity{0.000000}%
\pgfsetdash{}{0pt}%
\pgfpathmoveto{\pgfqpoint{2.067546in}{1.847071in}}%
\pgfpathlineto{\pgfqpoint{2.076482in}{1.847071in}}%
\pgfpathlineto{\pgfqpoint{2.076482in}{1.728389in}}%
\pgfpathlineto{\pgfqpoint{2.067546in}{1.728389in}}%
\pgfpathlineto{\pgfqpoint{2.067546in}{1.847071in}}%
\pgfpathclose%
\pgfusepath{fill}%
\end{pgfscope}%
\begin{pgfscope}%
\pgfpathrectangle{\pgfqpoint{0.697024in}{0.857143in}}{\pgfqpoint{2.627103in}{1.813434in}}%
\pgfusepath{clip}%
\pgfsetbuttcap%
\pgfsetmiterjoin%
\definecolor{currentfill}{rgb}{0.066899,0.263188,0.377594}%
\pgfsetfillcolor{currentfill}%
\pgfsetlinewidth{0.000000pt}%
\definecolor{currentstroke}{rgb}{0.000000,0.000000,0.000000}%
\pgfsetstrokecolor{currentstroke}%
\pgfsetstrokeopacity{0.000000}%
\pgfsetdash{}{0pt}%
\pgfpathmoveto{\pgfqpoint{2.078717in}{1.847071in}}%
\pgfpathlineto{\pgfqpoint{2.087653in}{1.847071in}}%
\pgfpathlineto{\pgfqpoint{2.087653in}{1.742542in}}%
\pgfpathlineto{\pgfqpoint{2.078717in}{1.742542in}}%
\pgfpathlineto{\pgfqpoint{2.078717in}{1.847071in}}%
\pgfpathclose%
\pgfusepath{fill}%
\end{pgfscope}%
\begin{pgfscope}%
\pgfpathrectangle{\pgfqpoint{0.697024in}{0.857143in}}{\pgfqpoint{2.627103in}{1.813434in}}%
\pgfusepath{clip}%
\pgfsetbuttcap%
\pgfsetmiterjoin%
\definecolor{currentfill}{rgb}{0.066899,0.263188,0.377594}%
\pgfsetfillcolor{currentfill}%
\pgfsetlinewidth{0.000000pt}%
\definecolor{currentstroke}{rgb}{0.000000,0.000000,0.000000}%
\pgfsetstrokecolor{currentstroke}%
\pgfsetstrokeopacity{0.000000}%
\pgfsetdash{}{0pt}%
\pgfpathmoveto{\pgfqpoint{2.089887in}{1.847071in}}%
\pgfpathlineto{\pgfqpoint{2.098824in}{1.847071in}}%
\pgfpathlineto{\pgfqpoint{2.098824in}{1.727330in}}%
\pgfpathlineto{\pgfqpoint{2.089887in}{1.727330in}}%
\pgfpathlineto{\pgfqpoint{2.089887in}{1.847071in}}%
\pgfpathclose%
\pgfusepath{fill}%
\end{pgfscope}%
\begin{pgfscope}%
\pgfpathrectangle{\pgfqpoint{0.697024in}{0.857143in}}{\pgfqpoint{2.627103in}{1.813434in}}%
\pgfusepath{clip}%
\pgfsetbuttcap%
\pgfsetmiterjoin%
\definecolor{currentfill}{rgb}{0.066899,0.263188,0.377594}%
\pgfsetfillcolor{currentfill}%
\pgfsetlinewidth{0.000000pt}%
\definecolor{currentstroke}{rgb}{0.000000,0.000000,0.000000}%
\pgfsetstrokecolor{currentstroke}%
\pgfsetstrokeopacity{0.000000}%
\pgfsetdash{}{0pt}%
\pgfpathmoveto{\pgfqpoint{2.101058in}{1.847071in}}%
\pgfpathlineto{\pgfqpoint{2.109994in}{1.847071in}}%
\pgfpathlineto{\pgfqpoint{2.109994in}{1.730596in}}%
\pgfpathlineto{\pgfqpoint{2.101058in}{1.730596in}}%
\pgfpathlineto{\pgfqpoint{2.101058in}{1.847071in}}%
\pgfpathclose%
\pgfusepath{fill}%
\end{pgfscope}%
\begin{pgfscope}%
\pgfpathrectangle{\pgfqpoint{0.697024in}{0.857143in}}{\pgfqpoint{2.627103in}{1.813434in}}%
\pgfusepath{clip}%
\pgfsetbuttcap%
\pgfsetmiterjoin%
\definecolor{currentfill}{rgb}{0.066899,0.263188,0.377594}%
\pgfsetfillcolor{currentfill}%
\pgfsetlinewidth{0.000000pt}%
\definecolor{currentstroke}{rgb}{0.000000,0.000000,0.000000}%
\pgfsetstrokecolor{currentstroke}%
\pgfsetstrokeopacity{0.000000}%
\pgfsetdash{}{0pt}%
\pgfpathmoveto{\pgfqpoint{2.112228in}{1.847071in}}%
\pgfpathlineto{\pgfqpoint{2.121165in}{1.847071in}}%
\pgfpathlineto{\pgfqpoint{2.121165in}{1.710249in}}%
\pgfpathlineto{\pgfqpoint{2.112228in}{1.710249in}}%
\pgfpathlineto{\pgfqpoint{2.112228in}{1.847071in}}%
\pgfpathclose%
\pgfusepath{fill}%
\end{pgfscope}%
\begin{pgfscope}%
\pgfpathrectangle{\pgfqpoint{0.697024in}{0.857143in}}{\pgfqpoint{2.627103in}{1.813434in}}%
\pgfusepath{clip}%
\pgfsetbuttcap%
\pgfsetmiterjoin%
\definecolor{currentfill}{rgb}{0.066899,0.263188,0.377594}%
\pgfsetfillcolor{currentfill}%
\pgfsetlinewidth{0.000000pt}%
\definecolor{currentstroke}{rgb}{0.000000,0.000000,0.000000}%
\pgfsetstrokecolor{currentstroke}%
\pgfsetstrokeopacity{0.000000}%
\pgfsetdash{}{0pt}%
\pgfpathmoveto{\pgfqpoint{2.123399in}{1.847071in}}%
\pgfpathlineto{\pgfqpoint{2.132335in}{1.847071in}}%
\pgfpathlineto{\pgfqpoint{2.132335in}{1.713586in}}%
\pgfpathlineto{\pgfqpoint{2.123399in}{1.713586in}}%
\pgfpathlineto{\pgfqpoint{2.123399in}{1.847071in}}%
\pgfpathclose%
\pgfusepath{fill}%
\end{pgfscope}%
\begin{pgfscope}%
\pgfpathrectangle{\pgfqpoint{0.697024in}{0.857143in}}{\pgfqpoint{2.627103in}{1.813434in}}%
\pgfusepath{clip}%
\pgfsetbuttcap%
\pgfsetmiterjoin%
\definecolor{currentfill}{rgb}{0.066899,0.263188,0.377594}%
\pgfsetfillcolor{currentfill}%
\pgfsetlinewidth{0.000000pt}%
\definecolor{currentstroke}{rgb}{0.000000,0.000000,0.000000}%
\pgfsetstrokecolor{currentstroke}%
\pgfsetstrokeopacity{0.000000}%
\pgfsetdash{}{0pt}%
\pgfpathmoveto{\pgfqpoint{2.134570in}{1.847071in}}%
\pgfpathlineto{\pgfqpoint{2.143506in}{1.847071in}}%
\pgfpathlineto{\pgfqpoint{2.143506in}{1.718886in}}%
\pgfpathlineto{\pgfqpoint{2.134570in}{1.718886in}}%
\pgfpathlineto{\pgfqpoint{2.134570in}{1.847071in}}%
\pgfpathclose%
\pgfusepath{fill}%
\end{pgfscope}%
\begin{pgfscope}%
\pgfpathrectangle{\pgfqpoint{0.697024in}{0.857143in}}{\pgfqpoint{2.627103in}{1.813434in}}%
\pgfusepath{clip}%
\pgfsetbuttcap%
\pgfsetmiterjoin%
\definecolor{currentfill}{rgb}{0.066899,0.263188,0.377594}%
\pgfsetfillcolor{currentfill}%
\pgfsetlinewidth{0.000000pt}%
\definecolor{currentstroke}{rgb}{0.000000,0.000000,0.000000}%
\pgfsetstrokecolor{currentstroke}%
\pgfsetstrokeopacity{0.000000}%
\pgfsetdash{}{0pt}%
\pgfpathmoveto{\pgfqpoint{2.145740in}{1.847071in}}%
\pgfpathlineto{\pgfqpoint{2.154677in}{1.847071in}}%
\pgfpathlineto{\pgfqpoint{2.154677in}{1.728207in}}%
\pgfpathlineto{\pgfqpoint{2.145740in}{1.728207in}}%
\pgfpathlineto{\pgfqpoint{2.145740in}{1.847071in}}%
\pgfpathclose%
\pgfusepath{fill}%
\end{pgfscope}%
\begin{pgfscope}%
\pgfpathrectangle{\pgfqpoint{0.697024in}{0.857143in}}{\pgfqpoint{2.627103in}{1.813434in}}%
\pgfusepath{clip}%
\pgfsetbuttcap%
\pgfsetmiterjoin%
\definecolor{currentfill}{rgb}{0.066899,0.263188,0.377594}%
\pgfsetfillcolor{currentfill}%
\pgfsetlinewidth{0.000000pt}%
\definecolor{currentstroke}{rgb}{0.000000,0.000000,0.000000}%
\pgfsetstrokecolor{currentstroke}%
\pgfsetstrokeopacity{0.000000}%
\pgfsetdash{}{0pt}%
\pgfpathmoveto{\pgfqpoint{2.156911in}{1.847071in}}%
\pgfpathlineto{\pgfqpoint{2.165847in}{1.847071in}}%
\pgfpathlineto{\pgfqpoint{2.165847in}{1.716533in}}%
\pgfpathlineto{\pgfqpoint{2.156911in}{1.716533in}}%
\pgfpathlineto{\pgfqpoint{2.156911in}{1.847071in}}%
\pgfpathclose%
\pgfusepath{fill}%
\end{pgfscope}%
\begin{pgfscope}%
\pgfpathrectangle{\pgfqpoint{0.697024in}{0.857143in}}{\pgfqpoint{2.627103in}{1.813434in}}%
\pgfusepath{clip}%
\pgfsetbuttcap%
\pgfsetmiterjoin%
\definecolor{currentfill}{rgb}{0.066899,0.263188,0.377594}%
\pgfsetfillcolor{currentfill}%
\pgfsetlinewidth{0.000000pt}%
\definecolor{currentstroke}{rgb}{0.000000,0.000000,0.000000}%
\pgfsetstrokecolor{currentstroke}%
\pgfsetstrokeopacity{0.000000}%
\pgfsetdash{}{0pt}%
\pgfpathmoveto{\pgfqpoint{2.168081in}{1.847071in}}%
\pgfpathlineto{\pgfqpoint{2.177018in}{1.847071in}}%
\pgfpathlineto{\pgfqpoint{2.177018in}{1.702186in}}%
\pgfpathlineto{\pgfqpoint{2.168081in}{1.702186in}}%
\pgfpathlineto{\pgfqpoint{2.168081in}{1.847071in}}%
\pgfpathclose%
\pgfusepath{fill}%
\end{pgfscope}%
\begin{pgfscope}%
\pgfpathrectangle{\pgfqpoint{0.697024in}{0.857143in}}{\pgfqpoint{2.627103in}{1.813434in}}%
\pgfusepath{clip}%
\pgfsetbuttcap%
\pgfsetmiterjoin%
\definecolor{currentfill}{rgb}{0.066899,0.263188,0.377594}%
\pgfsetfillcolor{currentfill}%
\pgfsetlinewidth{0.000000pt}%
\definecolor{currentstroke}{rgb}{0.000000,0.000000,0.000000}%
\pgfsetstrokecolor{currentstroke}%
\pgfsetstrokeopacity{0.000000}%
\pgfsetdash{}{0pt}%
\pgfpathmoveto{\pgfqpoint{2.179252in}{1.847071in}}%
\pgfpathlineto{\pgfqpoint{2.188189in}{1.847071in}}%
\pgfpathlineto{\pgfqpoint{2.188189in}{1.686026in}}%
\pgfpathlineto{\pgfqpoint{2.179252in}{1.686026in}}%
\pgfpathlineto{\pgfqpoint{2.179252in}{1.847071in}}%
\pgfpathclose%
\pgfusepath{fill}%
\end{pgfscope}%
\begin{pgfscope}%
\pgfpathrectangle{\pgfqpoint{0.697024in}{0.857143in}}{\pgfqpoint{2.627103in}{1.813434in}}%
\pgfusepath{clip}%
\pgfsetbuttcap%
\pgfsetmiterjoin%
\definecolor{currentfill}{rgb}{0.066899,0.263188,0.377594}%
\pgfsetfillcolor{currentfill}%
\pgfsetlinewidth{0.000000pt}%
\definecolor{currentstroke}{rgb}{0.000000,0.000000,0.000000}%
\pgfsetstrokecolor{currentstroke}%
\pgfsetstrokeopacity{0.000000}%
\pgfsetdash{}{0pt}%
\pgfpathmoveto{\pgfqpoint{2.190423in}{1.847071in}}%
\pgfpathlineto{\pgfqpoint{2.199359in}{1.847071in}}%
\pgfpathlineto{\pgfqpoint{2.199359in}{1.712007in}}%
\pgfpathlineto{\pgfqpoint{2.190423in}{1.712007in}}%
\pgfpathlineto{\pgfqpoint{2.190423in}{1.847071in}}%
\pgfpathclose%
\pgfusepath{fill}%
\end{pgfscope}%
\begin{pgfscope}%
\pgfpathrectangle{\pgfqpoint{0.697024in}{0.857143in}}{\pgfqpoint{2.627103in}{1.813434in}}%
\pgfusepath{clip}%
\pgfsetbuttcap%
\pgfsetmiterjoin%
\definecolor{currentfill}{rgb}{0.066899,0.263188,0.377594}%
\pgfsetfillcolor{currentfill}%
\pgfsetlinewidth{0.000000pt}%
\definecolor{currentstroke}{rgb}{0.000000,0.000000,0.000000}%
\pgfsetstrokecolor{currentstroke}%
\pgfsetstrokeopacity{0.000000}%
\pgfsetdash{}{0pt}%
\pgfpathmoveto{\pgfqpoint{2.201593in}{1.847071in}}%
\pgfpathlineto{\pgfqpoint{2.210530in}{1.847071in}}%
\pgfpathlineto{\pgfqpoint{2.210530in}{1.717555in}}%
\pgfpathlineto{\pgfqpoint{2.201593in}{1.717555in}}%
\pgfpathlineto{\pgfqpoint{2.201593in}{1.847071in}}%
\pgfpathclose%
\pgfusepath{fill}%
\end{pgfscope}%
\begin{pgfscope}%
\pgfpathrectangle{\pgfqpoint{0.697024in}{0.857143in}}{\pgfqpoint{2.627103in}{1.813434in}}%
\pgfusepath{clip}%
\pgfsetbuttcap%
\pgfsetmiterjoin%
\definecolor{currentfill}{rgb}{0.066899,0.263188,0.377594}%
\pgfsetfillcolor{currentfill}%
\pgfsetlinewidth{0.000000pt}%
\definecolor{currentstroke}{rgb}{0.000000,0.000000,0.000000}%
\pgfsetstrokecolor{currentstroke}%
\pgfsetstrokeopacity{0.000000}%
\pgfsetdash{}{0pt}%
\pgfpathmoveto{\pgfqpoint{2.212764in}{1.847071in}}%
\pgfpathlineto{\pgfqpoint{2.221700in}{1.847071in}}%
\pgfpathlineto{\pgfqpoint{2.221700in}{1.715099in}}%
\pgfpathlineto{\pgfqpoint{2.212764in}{1.715099in}}%
\pgfpathlineto{\pgfqpoint{2.212764in}{1.847071in}}%
\pgfpathclose%
\pgfusepath{fill}%
\end{pgfscope}%
\begin{pgfscope}%
\pgfpathrectangle{\pgfqpoint{0.697024in}{0.857143in}}{\pgfqpoint{2.627103in}{1.813434in}}%
\pgfusepath{clip}%
\pgfsetbuttcap%
\pgfsetmiterjoin%
\definecolor{currentfill}{rgb}{0.066899,0.263188,0.377594}%
\pgfsetfillcolor{currentfill}%
\pgfsetlinewidth{0.000000pt}%
\definecolor{currentstroke}{rgb}{0.000000,0.000000,0.000000}%
\pgfsetstrokecolor{currentstroke}%
\pgfsetstrokeopacity{0.000000}%
\pgfsetdash{}{0pt}%
\pgfpathmoveto{\pgfqpoint{2.223934in}{1.847071in}}%
\pgfpathlineto{\pgfqpoint{2.232871in}{1.847071in}}%
\pgfpathlineto{\pgfqpoint{2.232871in}{1.713613in}}%
\pgfpathlineto{\pgfqpoint{2.223934in}{1.713613in}}%
\pgfpathlineto{\pgfqpoint{2.223934in}{1.847071in}}%
\pgfpathclose%
\pgfusepath{fill}%
\end{pgfscope}%
\begin{pgfscope}%
\pgfpathrectangle{\pgfqpoint{0.697024in}{0.857143in}}{\pgfqpoint{2.627103in}{1.813434in}}%
\pgfusepath{clip}%
\pgfsetbuttcap%
\pgfsetmiterjoin%
\definecolor{currentfill}{rgb}{0.066899,0.263188,0.377594}%
\pgfsetfillcolor{currentfill}%
\pgfsetlinewidth{0.000000pt}%
\definecolor{currentstroke}{rgb}{0.000000,0.000000,0.000000}%
\pgfsetstrokecolor{currentstroke}%
\pgfsetstrokeopacity{0.000000}%
\pgfsetdash{}{0pt}%
\pgfpathmoveto{\pgfqpoint{2.235105in}{1.847071in}}%
\pgfpathlineto{\pgfqpoint{2.244042in}{1.847071in}}%
\pgfpathlineto{\pgfqpoint{2.244042in}{1.713644in}}%
\pgfpathlineto{\pgfqpoint{2.235105in}{1.713644in}}%
\pgfpathlineto{\pgfqpoint{2.235105in}{1.847071in}}%
\pgfpathclose%
\pgfusepath{fill}%
\end{pgfscope}%
\begin{pgfscope}%
\pgfpathrectangle{\pgfqpoint{0.697024in}{0.857143in}}{\pgfqpoint{2.627103in}{1.813434in}}%
\pgfusepath{clip}%
\pgfsetbuttcap%
\pgfsetmiterjoin%
\definecolor{currentfill}{rgb}{0.066899,0.263188,0.377594}%
\pgfsetfillcolor{currentfill}%
\pgfsetlinewidth{0.000000pt}%
\definecolor{currentstroke}{rgb}{0.000000,0.000000,0.000000}%
\pgfsetstrokecolor{currentstroke}%
\pgfsetstrokeopacity{0.000000}%
\pgfsetdash{}{0pt}%
\pgfpathmoveto{\pgfqpoint{2.246276in}{1.847071in}}%
\pgfpathlineto{\pgfqpoint{2.255212in}{1.847071in}}%
\pgfpathlineto{\pgfqpoint{2.255212in}{1.734829in}}%
\pgfpathlineto{\pgfqpoint{2.246276in}{1.734829in}}%
\pgfpathlineto{\pgfqpoint{2.246276in}{1.847071in}}%
\pgfpathclose%
\pgfusepath{fill}%
\end{pgfscope}%
\begin{pgfscope}%
\pgfpathrectangle{\pgfqpoint{0.697024in}{0.857143in}}{\pgfqpoint{2.627103in}{1.813434in}}%
\pgfusepath{clip}%
\pgfsetbuttcap%
\pgfsetmiterjoin%
\definecolor{currentfill}{rgb}{0.066899,0.263188,0.377594}%
\pgfsetfillcolor{currentfill}%
\pgfsetlinewidth{0.000000pt}%
\definecolor{currentstroke}{rgb}{0.000000,0.000000,0.000000}%
\pgfsetstrokecolor{currentstroke}%
\pgfsetstrokeopacity{0.000000}%
\pgfsetdash{}{0pt}%
\pgfpathmoveto{\pgfqpoint{2.257446in}{1.847071in}}%
\pgfpathlineto{\pgfqpoint{2.266383in}{1.847071in}}%
\pgfpathlineto{\pgfqpoint{2.266383in}{1.731222in}}%
\pgfpathlineto{\pgfqpoint{2.257446in}{1.731222in}}%
\pgfpathlineto{\pgfqpoint{2.257446in}{1.847071in}}%
\pgfpathclose%
\pgfusepath{fill}%
\end{pgfscope}%
\begin{pgfscope}%
\pgfpathrectangle{\pgfqpoint{0.697024in}{0.857143in}}{\pgfqpoint{2.627103in}{1.813434in}}%
\pgfusepath{clip}%
\pgfsetbuttcap%
\pgfsetmiterjoin%
\definecolor{currentfill}{rgb}{0.066899,0.263188,0.377594}%
\pgfsetfillcolor{currentfill}%
\pgfsetlinewidth{0.000000pt}%
\definecolor{currentstroke}{rgb}{0.000000,0.000000,0.000000}%
\pgfsetstrokecolor{currentstroke}%
\pgfsetstrokeopacity{0.000000}%
\pgfsetdash{}{0pt}%
\pgfpathmoveto{\pgfqpoint{2.268617in}{1.847071in}}%
\pgfpathlineto{\pgfqpoint{2.277553in}{1.847071in}}%
\pgfpathlineto{\pgfqpoint{2.277553in}{1.750385in}}%
\pgfpathlineto{\pgfqpoint{2.268617in}{1.750385in}}%
\pgfpathlineto{\pgfqpoint{2.268617in}{1.847071in}}%
\pgfpathclose%
\pgfusepath{fill}%
\end{pgfscope}%
\begin{pgfscope}%
\pgfpathrectangle{\pgfqpoint{0.697024in}{0.857143in}}{\pgfqpoint{2.627103in}{1.813434in}}%
\pgfusepath{clip}%
\pgfsetbuttcap%
\pgfsetmiterjoin%
\definecolor{currentfill}{rgb}{0.066899,0.263188,0.377594}%
\pgfsetfillcolor{currentfill}%
\pgfsetlinewidth{0.000000pt}%
\definecolor{currentstroke}{rgb}{0.000000,0.000000,0.000000}%
\pgfsetstrokecolor{currentstroke}%
\pgfsetstrokeopacity{0.000000}%
\pgfsetdash{}{0pt}%
\pgfpathmoveto{\pgfqpoint{2.279787in}{1.847071in}}%
\pgfpathlineto{\pgfqpoint{2.288724in}{1.847071in}}%
\pgfpathlineto{\pgfqpoint{2.288724in}{1.743937in}}%
\pgfpathlineto{\pgfqpoint{2.279787in}{1.743937in}}%
\pgfpathlineto{\pgfqpoint{2.279787in}{1.847071in}}%
\pgfpathclose%
\pgfusepath{fill}%
\end{pgfscope}%
\begin{pgfscope}%
\pgfpathrectangle{\pgfqpoint{0.697024in}{0.857143in}}{\pgfqpoint{2.627103in}{1.813434in}}%
\pgfusepath{clip}%
\pgfsetbuttcap%
\pgfsetmiterjoin%
\definecolor{currentfill}{rgb}{0.066899,0.263188,0.377594}%
\pgfsetfillcolor{currentfill}%
\pgfsetlinewidth{0.000000pt}%
\definecolor{currentstroke}{rgb}{0.000000,0.000000,0.000000}%
\pgfsetstrokecolor{currentstroke}%
\pgfsetstrokeopacity{0.000000}%
\pgfsetdash{}{0pt}%
\pgfpathmoveto{\pgfqpoint{2.290958in}{1.847071in}}%
\pgfpathlineto{\pgfqpoint{2.299895in}{1.847071in}}%
\pgfpathlineto{\pgfqpoint{2.299895in}{1.751497in}}%
\pgfpathlineto{\pgfqpoint{2.290958in}{1.751497in}}%
\pgfpathlineto{\pgfqpoint{2.290958in}{1.847071in}}%
\pgfpathclose%
\pgfusepath{fill}%
\end{pgfscope}%
\begin{pgfscope}%
\pgfpathrectangle{\pgfqpoint{0.697024in}{0.857143in}}{\pgfqpoint{2.627103in}{1.813434in}}%
\pgfusepath{clip}%
\pgfsetbuttcap%
\pgfsetmiterjoin%
\definecolor{currentfill}{rgb}{0.066899,0.263188,0.377594}%
\pgfsetfillcolor{currentfill}%
\pgfsetlinewidth{0.000000pt}%
\definecolor{currentstroke}{rgb}{0.000000,0.000000,0.000000}%
\pgfsetstrokecolor{currentstroke}%
\pgfsetstrokeopacity{0.000000}%
\pgfsetdash{}{0pt}%
\pgfpathmoveto{\pgfqpoint{2.302129in}{1.847071in}}%
\pgfpathlineto{\pgfqpoint{2.311065in}{1.847071in}}%
\pgfpathlineto{\pgfqpoint{2.311065in}{1.774356in}}%
\pgfpathlineto{\pgfqpoint{2.302129in}{1.774356in}}%
\pgfpathlineto{\pgfqpoint{2.302129in}{1.847071in}}%
\pgfpathclose%
\pgfusepath{fill}%
\end{pgfscope}%
\begin{pgfscope}%
\pgfpathrectangle{\pgfqpoint{0.697024in}{0.857143in}}{\pgfqpoint{2.627103in}{1.813434in}}%
\pgfusepath{clip}%
\pgfsetbuttcap%
\pgfsetmiterjoin%
\definecolor{currentfill}{rgb}{0.066899,0.263188,0.377594}%
\pgfsetfillcolor{currentfill}%
\pgfsetlinewidth{0.000000pt}%
\definecolor{currentstroke}{rgb}{0.000000,0.000000,0.000000}%
\pgfsetstrokecolor{currentstroke}%
\pgfsetstrokeopacity{0.000000}%
\pgfsetdash{}{0pt}%
\pgfpathmoveto{\pgfqpoint{2.313299in}{1.847071in}}%
\pgfpathlineto{\pgfqpoint{2.322236in}{1.847071in}}%
\pgfpathlineto{\pgfqpoint{2.322236in}{1.756040in}}%
\pgfpathlineto{\pgfqpoint{2.313299in}{1.756040in}}%
\pgfpathlineto{\pgfqpoint{2.313299in}{1.847071in}}%
\pgfpathclose%
\pgfusepath{fill}%
\end{pgfscope}%
\begin{pgfscope}%
\pgfpathrectangle{\pgfqpoint{0.697024in}{0.857143in}}{\pgfqpoint{2.627103in}{1.813434in}}%
\pgfusepath{clip}%
\pgfsetbuttcap%
\pgfsetmiterjoin%
\definecolor{currentfill}{rgb}{0.066899,0.263188,0.377594}%
\pgfsetfillcolor{currentfill}%
\pgfsetlinewidth{0.000000pt}%
\definecolor{currentstroke}{rgb}{0.000000,0.000000,0.000000}%
\pgfsetstrokecolor{currentstroke}%
\pgfsetstrokeopacity{0.000000}%
\pgfsetdash{}{0pt}%
\pgfpathmoveto{\pgfqpoint{2.324470in}{1.847071in}}%
\pgfpathlineto{\pgfqpoint{2.333406in}{1.847071in}}%
\pgfpathlineto{\pgfqpoint{2.333406in}{1.790197in}}%
\pgfpathlineto{\pgfqpoint{2.324470in}{1.790197in}}%
\pgfpathlineto{\pgfqpoint{2.324470in}{1.847071in}}%
\pgfpathclose%
\pgfusepath{fill}%
\end{pgfscope}%
\begin{pgfscope}%
\pgfpathrectangle{\pgfqpoint{0.697024in}{0.857143in}}{\pgfqpoint{2.627103in}{1.813434in}}%
\pgfusepath{clip}%
\pgfsetbuttcap%
\pgfsetmiterjoin%
\definecolor{currentfill}{rgb}{0.066899,0.263188,0.377594}%
\pgfsetfillcolor{currentfill}%
\pgfsetlinewidth{0.000000pt}%
\definecolor{currentstroke}{rgb}{0.000000,0.000000,0.000000}%
\pgfsetstrokecolor{currentstroke}%
\pgfsetstrokeopacity{0.000000}%
\pgfsetdash{}{0pt}%
\pgfpathmoveto{\pgfqpoint{2.335640in}{1.847071in}}%
\pgfpathlineto{\pgfqpoint{2.344577in}{1.847071in}}%
\pgfpathlineto{\pgfqpoint{2.344577in}{1.777902in}}%
\pgfpathlineto{\pgfqpoint{2.335640in}{1.777902in}}%
\pgfpathlineto{\pgfqpoint{2.335640in}{1.847071in}}%
\pgfpathclose%
\pgfusepath{fill}%
\end{pgfscope}%
\begin{pgfscope}%
\pgfpathrectangle{\pgfqpoint{0.697024in}{0.857143in}}{\pgfqpoint{2.627103in}{1.813434in}}%
\pgfusepath{clip}%
\pgfsetbuttcap%
\pgfsetmiterjoin%
\definecolor{currentfill}{rgb}{0.066899,0.263188,0.377594}%
\pgfsetfillcolor{currentfill}%
\pgfsetlinewidth{0.000000pt}%
\definecolor{currentstroke}{rgb}{0.000000,0.000000,0.000000}%
\pgfsetstrokecolor{currentstroke}%
\pgfsetstrokeopacity{0.000000}%
\pgfsetdash{}{0pt}%
\pgfpathmoveto{\pgfqpoint{2.346811in}{1.847071in}}%
\pgfpathlineto{\pgfqpoint{2.355748in}{1.847071in}}%
\pgfpathlineto{\pgfqpoint{2.355748in}{1.794305in}}%
\pgfpathlineto{\pgfqpoint{2.346811in}{1.794305in}}%
\pgfpathlineto{\pgfqpoint{2.346811in}{1.847071in}}%
\pgfpathclose%
\pgfusepath{fill}%
\end{pgfscope}%
\begin{pgfscope}%
\pgfpathrectangle{\pgfqpoint{0.697024in}{0.857143in}}{\pgfqpoint{2.627103in}{1.813434in}}%
\pgfusepath{clip}%
\pgfsetbuttcap%
\pgfsetmiterjoin%
\definecolor{currentfill}{rgb}{0.066899,0.263188,0.377594}%
\pgfsetfillcolor{currentfill}%
\pgfsetlinewidth{0.000000pt}%
\definecolor{currentstroke}{rgb}{0.000000,0.000000,0.000000}%
\pgfsetstrokecolor{currentstroke}%
\pgfsetstrokeopacity{0.000000}%
\pgfsetdash{}{0pt}%
\pgfpathmoveto{\pgfqpoint{2.357982in}{1.847071in}}%
\pgfpathlineto{\pgfqpoint{2.366918in}{1.847071in}}%
\pgfpathlineto{\pgfqpoint{2.366918in}{1.771542in}}%
\pgfpathlineto{\pgfqpoint{2.357982in}{1.771542in}}%
\pgfpathlineto{\pgfqpoint{2.357982in}{1.847071in}}%
\pgfpathclose%
\pgfusepath{fill}%
\end{pgfscope}%
\begin{pgfscope}%
\pgfpathrectangle{\pgfqpoint{0.697024in}{0.857143in}}{\pgfqpoint{2.627103in}{1.813434in}}%
\pgfusepath{clip}%
\pgfsetbuttcap%
\pgfsetmiterjoin%
\definecolor{currentfill}{rgb}{0.066899,0.263188,0.377594}%
\pgfsetfillcolor{currentfill}%
\pgfsetlinewidth{0.000000pt}%
\definecolor{currentstroke}{rgb}{0.000000,0.000000,0.000000}%
\pgfsetstrokecolor{currentstroke}%
\pgfsetstrokeopacity{0.000000}%
\pgfsetdash{}{0pt}%
\pgfpathmoveto{\pgfqpoint{2.369152in}{1.847071in}}%
\pgfpathlineto{\pgfqpoint{2.378089in}{1.847071in}}%
\pgfpathlineto{\pgfqpoint{2.378089in}{1.809295in}}%
\pgfpathlineto{\pgfqpoint{2.369152in}{1.809295in}}%
\pgfpathlineto{\pgfqpoint{2.369152in}{1.847071in}}%
\pgfpathclose%
\pgfusepath{fill}%
\end{pgfscope}%
\begin{pgfscope}%
\pgfpathrectangle{\pgfqpoint{0.697024in}{0.857143in}}{\pgfqpoint{2.627103in}{1.813434in}}%
\pgfusepath{clip}%
\pgfsetbuttcap%
\pgfsetmiterjoin%
\definecolor{currentfill}{rgb}{0.066899,0.263188,0.377594}%
\pgfsetfillcolor{currentfill}%
\pgfsetlinewidth{0.000000pt}%
\definecolor{currentstroke}{rgb}{0.000000,0.000000,0.000000}%
\pgfsetstrokecolor{currentstroke}%
\pgfsetstrokeopacity{0.000000}%
\pgfsetdash{}{0pt}%
\pgfpathmoveto{\pgfqpoint{2.380323in}{1.847071in}}%
\pgfpathlineto{\pgfqpoint{2.389259in}{1.847071in}}%
\pgfpathlineto{\pgfqpoint{2.389259in}{1.817593in}}%
\pgfpathlineto{\pgfqpoint{2.380323in}{1.817593in}}%
\pgfpathlineto{\pgfqpoint{2.380323in}{1.847071in}}%
\pgfpathclose%
\pgfusepath{fill}%
\end{pgfscope}%
\begin{pgfscope}%
\pgfpathrectangle{\pgfqpoint{0.697024in}{0.857143in}}{\pgfqpoint{2.627103in}{1.813434in}}%
\pgfusepath{clip}%
\pgfsetbuttcap%
\pgfsetmiterjoin%
\definecolor{currentfill}{rgb}{0.066899,0.263188,0.377594}%
\pgfsetfillcolor{currentfill}%
\pgfsetlinewidth{0.000000pt}%
\definecolor{currentstroke}{rgb}{0.000000,0.000000,0.000000}%
\pgfsetstrokecolor{currentstroke}%
\pgfsetstrokeopacity{0.000000}%
\pgfsetdash{}{0pt}%
\pgfpathmoveto{\pgfqpoint{2.391494in}{1.847071in}}%
\pgfpathlineto{\pgfqpoint{2.400430in}{1.847071in}}%
\pgfpathlineto{\pgfqpoint{2.400430in}{1.845214in}}%
\pgfpathlineto{\pgfqpoint{2.391494in}{1.845214in}}%
\pgfpathlineto{\pgfqpoint{2.391494in}{1.847071in}}%
\pgfpathclose%
\pgfusepath{fill}%
\end{pgfscope}%
\begin{pgfscope}%
\pgfpathrectangle{\pgfqpoint{0.697024in}{0.857143in}}{\pgfqpoint{2.627103in}{1.813434in}}%
\pgfusepath{clip}%
\pgfsetbuttcap%
\pgfsetmiterjoin%
\definecolor{currentfill}{rgb}{0.066899,0.263188,0.377594}%
\pgfsetfillcolor{currentfill}%
\pgfsetlinewidth{0.000000pt}%
\definecolor{currentstroke}{rgb}{0.000000,0.000000,0.000000}%
\pgfsetstrokecolor{currentstroke}%
\pgfsetstrokeopacity{0.000000}%
\pgfsetdash{}{0pt}%
\pgfpathmoveto{\pgfqpoint{2.402664in}{1.847071in}}%
\pgfpathlineto{\pgfqpoint{2.411601in}{1.847071in}}%
\pgfpathlineto{\pgfqpoint{2.411601in}{1.878459in}}%
\pgfpathlineto{\pgfqpoint{2.402664in}{1.878459in}}%
\pgfpathlineto{\pgfqpoint{2.402664in}{1.847071in}}%
\pgfpathclose%
\pgfusepath{fill}%
\end{pgfscope}%
\begin{pgfscope}%
\pgfpathrectangle{\pgfqpoint{0.697024in}{0.857143in}}{\pgfqpoint{2.627103in}{1.813434in}}%
\pgfusepath{clip}%
\pgfsetbuttcap%
\pgfsetmiterjoin%
\definecolor{currentfill}{rgb}{0.066899,0.263188,0.377594}%
\pgfsetfillcolor{currentfill}%
\pgfsetlinewidth{0.000000pt}%
\definecolor{currentstroke}{rgb}{0.000000,0.000000,0.000000}%
\pgfsetstrokecolor{currentstroke}%
\pgfsetstrokeopacity{0.000000}%
\pgfsetdash{}{0pt}%
\pgfpathmoveto{\pgfqpoint{2.413835in}{1.847071in}}%
\pgfpathlineto{\pgfqpoint{2.422771in}{1.847071in}}%
\pgfpathlineto{\pgfqpoint{2.422771in}{1.871756in}}%
\pgfpathlineto{\pgfqpoint{2.413835in}{1.871756in}}%
\pgfpathlineto{\pgfqpoint{2.413835in}{1.847071in}}%
\pgfpathclose%
\pgfusepath{fill}%
\end{pgfscope}%
\begin{pgfscope}%
\pgfpathrectangle{\pgfqpoint{0.697024in}{0.857143in}}{\pgfqpoint{2.627103in}{1.813434in}}%
\pgfusepath{clip}%
\pgfsetbuttcap%
\pgfsetmiterjoin%
\definecolor{currentfill}{rgb}{0.066899,0.263188,0.377594}%
\pgfsetfillcolor{currentfill}%
\pgfsetlinewidth{0.000000pt}%
\definecolor{currentstroke}{rgb}{0.000000,0.000000,0.000000}%
\pgfsetstrokecolor{currentstroke}%
\pgfsetstrokeopacity{0.000000}%
\pgfsetdash{}{0pt}%
\pgfpathmoveto{\pgfqpoint{2.425005in}{1.847071in}}%
\pgfpathlineto{\pgfqpoint{2.433942in}{1.847071in}}%
\pgfpathlineto{\pgfqpoint{2.433942in}{1.880722in}}%
\pgfpathlineto{\pgfqpoint{2.425005in}{1.880722in}}%
\pgfpathlineto{\pgfqpoint{2.425005in}{1.847071in}}%
\pgfpathclose%
\pgfusepath{fill}%
\end{pgfscope}%
\begin{pgfscope}%
\pgfpathrectangle{\pgfqpoint{0.697024in}{0.857143in}}{\pgfqpoint{2.627103in}{1.813434in}}%
\pgfusepath{clip}%
\pgfsetbuttcap%
\pgfsetmiterjoin%
\definecolor{currentfill}{rgb}{0.066899,0.263188,0.377594}%
\pgfsetfillcolor{currentfill}%
\pgfsetlinewidth{0.000000pt}%
\definecolor{currentstroke}{rgb}{0.000000,0.000000,0.000000}%
\pgfsetstrokecolor{currentstroke}%
\pgfsetstrokeopacity{0.000000}%
\pgfsetdash{}{0pt}%
\pgfpathmoveto{\pgfqpoint{2.436176in}{1.847071in}}%
\pgfpathlineto{\pgfqpoint{2.445112in}{1.847071in}}%
\pgfpathlineto{\pgfqpoint{2.445112in}{1.865053in}}%
\pgfpathlineto{\pgfqpoint{2.436176in}{1.865053in}}%
\pgfpathlineto{\pgfqpoint{2.436176in}{1.847071in}}%
\pgfpathclose%
\pgfusepath{fill}%
\end{pgfscope}%
\begin{pgfscope}%
\pgfpathrectangle{\pgfqpoint{0.697024in}{0.857143in}}{\pgfqpoint{2.627103in}{1.813434in}}%
\pgfusepath{clip}%
\pgfsetbuttcap%
\pgfsetmiterjoin%
\definecolor{currentfill}{rgb}{0.066899,0.263188,0.377594}%
\pgfsetfillcolor{currentfill}%
\pgfsetlinewidth{0.000000pt}%
\definecolor{currentstroke}{rgb}{0.000000,0.000000,0.000000}%
\pgfsetstrokecolor{currentstroke}%
\pgfsetstrokeopacity{0.000000}%
\pgfsetdash{}{0pt}%
\pgfpathmoveto{\pgfqpoint{2.447347in}{1.847071in}}%
\pgfpathlineto{\pgfqpoint{2.456283in}{1.847071in}}%
\pgfpathlineto{\pgfqpoint{2.456283in}{1.885329in}}%
\pgfpathlineto{\pgfqpoint{2.447347in}{1.885329in}}%
\pgfpathlineto{\pgfqpoint{2.447347in}{1.847071in}}%
\pgfpathclose%
\pgfusepath{fill}%
\end{pgfscope}%
\begin{pgfscope}%
\pgfpathrectangle{\pgfqpoint{0.697024in}{0.857143in}}{\pgfqpoint{2.627103in}{1.813434in}}%
\pgfusepath{clip}%
\pgfsetbuttcap%
\pgfsetmiterjoin%
\definecolor{currentfill}{rgb}{0.066899,0.263188,0.377594}%
\pgfsetfillcolor{currentfill}%
\pgfsetlinewidth{0.000000pt}%
\definecolor{currentstroke}{rgb}{0.000000,0.000000,0.000000}%
\pgfsetstrokecolor{currentstroke}%
\pgfsetstrokeopacity{0.000000}%
\pgfsetdash{}{0pt}%
\pgfpathmoveto{\pgfqpoint{2.458517in}{1.847071in}}%
\pgfpathlineto{\pgfqpoint{2.467454in}{1.847071in}}%
\pgfpathlineto{\pgfqpoint{2.467454in}{1.907319in}}%
\pgfpathlineto{\pgfqpoint{2.458517in}{1.907319in}}%
\pgfpathlineto{\pgfqpoint{2.458517in}{1.847071in}}%
\pgfpathclose%
\pgfusepath{fill}%
\end{pgfscope}%
\begin{pgfscope}%
\pgfpathrectangle{\pgfqpoint{0.697024in}{0.857143in}}{\pgfqpoint{2.627103in}{1.813434in}}%
\pgfusepath{clip}%
\pgfsetbuttcap%
\pgfsetmiterjoin%
\definecolor{currentfill}{rgb}{0.066899,0.263188,0.377594}%
\pgfsetfillcolor{currentfill}%
\pgfsetlinewidth{0.000000pt}%
\definecolor{currentstroke}{rgb}{0.000000,0.000000,0.000000}%
\pgfsetstrokecolor{currentstroke}%
\pgfsetstrokeopacity{0.000000}%
\pgfsetdash{}{0pt}%
\pgfpathmoveto{\pgfqpoint{2.469688in}{1.847071in}}%
\pgfpathlineto{\pgfqpoint{2.478624in}{1.847071in}}%
\pgfpathlineto{\pgfqpoint{2.478624in}{1.937461in}}%
\pgfpathlineto{\pgfqpoint{2.469688in}{1.937461in}}%
\pgfpathlineto{\pgfqpoint{2.469688in}{1.847071in}}%
\pgfpathclose%
\pgfusepath{fill}%
\end{pgfscope}%
\begin{pgfscope}%
\pgfpathrectangle{\pgfqpoint{0.697024in}{0.857143in}}{\pgfqpoint{2.627103in}{1.813434in}}%
\pgfusepath{clip}%
\pgfsetbuttcap%
\pgfsetmiterjoin%
\definecolor{currentfill}{rgb}{0.066899,0.263188,0.377594}%
\pgfsetfillcolor{currentfill}%
\pgfsetlinewidth{0.000000pt}%
\definecolor{currentstroke}{rgb}{0.000000,0.000000,0.000000}%
\pgfsetstrokecolor{currentstroke}%
\pgfsetstrokeopacity{0.000000}%
\pgfsetdash{}{0pt}%
\pgfpathmoveto{\pgfqpoint{2.480858in}{1.847071in}}%
\pgfpathlineto{\pgfqpoint{2.489795in}{1.847071in}}%
\pgfpathlineto{\pgfqpoint{2.489795in}{1.940235in}}%
\pgfpathlineto{\pgfqpoint{2.480858in}{1.940235in}}%
\pgfpathlineto{\pgfqpoint{2.480858in}{1.847071in}}%
\pgfpathclose%
\pgfusepath{fill}%
\end{pgfscope}%
\begin{pgfscope}%
\pgfpathrectangle{\pgfqpoint{0.697024in}{0.857143in}}{\pgfqpoint{2.627103in}{1.813434in}}%
\pgfusepath{clip}%
\pgfsetbuttcap%
\pgfsetmiterjoin%
\definecolor{currentfill}{rgb}{0.066899,0.263188,0.377594}%
\pgfsetfillcolor{currentfill}%
\pgfsetlinewidth{0.000000pt}%
\definecolor{currentstroke}{rgb}{0.000000,0.000000,0.000000}%
\pgfsetstrokecolor{currentstroke}%
\pgfsetstrokeopacity{0.000000}%
\pgfsetdash{}{0pt}%
\pgfpathmoveto{\pgfqpoint{2.492029in}{1.847071in}}%
\pgfpathlineto{\pgfqpoint{2.500965in}{1.847071in}}%
\pgfpathlineto{\pgfqpoint{2.500965in}{1.929004in}}%
\pgfpathlineto{\pgfqpoint{2.492029in}{1.929004in}}%
\pgfpathlineto{\pgfqpoint{2.492029in}{1.847071in}}%
\pgfpathclose%
\pgfusepath{fill}%
\end{pgfscope}%
\begin{pgfscope}%
\pgfpathrectangle{\pgfqpoint{0.697024in}{0.857143in}}{\pgfqpoint{2.627103in}{1.813434in}}%
\pgfusepath{clip}%
\pgfsetbuttcap%
\pgfsetmiterjoin%
\definecolor{currentfill}{rgb}{0.066899,0.263188,0.377594}%
\pgfsetfillcolor{currentfill}%
\pgfsetlinewidth{0.000000pt}%
\definecolor{currentstroke}{rgb}{0.000000,0.000000,0.000000}%
\pgfsetstrokecolor{currentstroke}%
\pgfsetstrokeopacity{0.000000}%
\pgfsetdash{}{0pt}%
\pgfpathmoveto{\pgfqpoint{2.503200in}{1.847071in}}%
\pgfpathlineto{\pgfqpoint{2.512136in}{1.847071in}}%
\pgfpathlineto{\pgfqpoint{2.512136in}{1.937025in}}%
\pgfpathlineto{\pgfqpoint{2.503200in}{1.937025in}}%
\pgfpathlineto{\pgfqpoint{2.503200in}{1.847071in}}%
\pgfpathclose%
\pgfusepath{fill}%
\end{pgfscope}%
\begin{pgfscope}%
\pgfpathrectangle{\pgfqpoint{0.697024in}{0.857143in}}{\pgfqpoint{2.627103in}{1.813434in}}%
\pgfusepath{clip}%
\pgfsetbuttcap%
\pgfsetmiterjoin%
\definecolor{currentfill}{rgb}{0.066899,0.263188,0.377594}%
\pgfsetfillcolor{currentfill}%
\pgfsetlinewidth{0.000000pt}%
\definecolor{currentstroke}{rgb}{0.000000,0.000000,0.000000}%
\pgfsetstrokecolor{currentstroke}%
\pgfsetstrokeopacity{0.000000}%
\pgfsetdash{}{0pt}%
\pgfpathmoveto{\pgfqpoint{2.514370in}{1.847071in}}%
\pgfpathlineto{\pgfqpoint{2.523307in}{1.847071in}}%
\pgfpathlineto{\pgfqpoint{2.523307in}{1.935810in}}%
\pgfpathlineto{\pgfqpoint{2.514370in}{1.935810in}}%
\pgfpathlineto{\pgfqpoint{2.514370in}{1.847071in}}%
\pgfpathclose%
\pgfusepath{fill}%
\end{pgfscope}%
\begin{pgfscope}%
\pgfpathrectangle{\pgfqpoint{0.697024in}{0.857143in}}{\pgfqpoint{2.627103in}{1.813434in}}%
\pgfusepath{clip}%
\pgfsetbuttcap%
\pgfsetmiterjoin%
\definecolor{currentfill}{rgb}{0.066899,0.263188,0.377594}%
\pgfsetfillcolor{currentfill}%
\pgfsetlinewidth{0.000000pt}%
\definecolor{currentstroke}{rgb}{0.000000,0.000000,0.000000}%
\pgfsetstrokecolor{currentstroke}%
\pgfsetstrokeopacity{0.000000}%
\pgfsetdash{}{0pt}%
\pgfpathmoveto{\pgfqpoint{2.525541in}{1.847071in}}%
\pgfpathlineto{\pgfqpoint{2.534477in}{1.847071in}}%
\pgfpathlineto{\pgfqpoint{2.534477in}{1.936951in}}%
\pgfpathlineto{\pgfqpoint{2.525541in}{1.936951in}}%
\pgfpathlineto{\pgfqpoint{2.525541in}{1.847071in}}%
\pgfpathclose%
\pgfusepath{fill}%
\end{pgfscope}%
\begin{pgfscope}%
\pgfpathrectangle{\pgfqpoint{0.697024in}{0.857143in}}{\pgfqpoint{2.627103in}{1.813434in}}%
\pgfusepath{clip}%
\pgfsetbuttcap%
\pgfsetmiterjoin%
\definecolor{currentfill}{rgb}{0.066899,0.263188,0.377594}%
\pgfsetfillcolor{currentfill}%
\pgfsetlinewidth{0.000000pt}%
\definecolor{currentstroke}{rgb}{0.000000,0.000000,0.000000}%
\pgfsetstrokecolor{currentstroke}%
\pgfsetstrokeopacity{0.000000}%
\pgfsetdash{}{0pt}%
\pgfpathmoveto{\pgfqpoint{2.536711in}{1.847071in}}%
\pgfpathlineto{\pgfqpoint{2.545648in}{1.847071in}}%
\pgfpathlineto{\pgfqpoint{2.545648in}{1.950197in}}%
\pgfpathlineto{\pgfqpoint{2.536711in}{1.950197in}}%
\pgfpathlineto{\pgfqpoint{2.536711in}{1.847071in}}%
\pgfpathclose%
\pgfusepath{fill}%
\end{pgfscope}%
\begin{pgfscope}%
\pgfpathrectangle{\pgfqpoint{0.697024in}{0.857143in}}{\pgfqpoint{2.627103in}{1.813434in}}%
\pgfusepath{clip}%
\pgfsetbuttcap%
\pgfsetmiterjoin%
\definecolor{currentfill}{rgb}{0.066899,0.263188,0.377594}%
\pgfsetfillcolor{currentfill}%
\pgfsetlinewidth{0.000000pt}%
\definecolor{currentstroke}{rgb}{0.000000,0.000000,0.000000}%
\pgfsetstrokecolor{currentstroke}%
\pgfsetstrokeopacity{0.000000}%
\pgfsetdash{}{0pt}%
\pgfpathmoveto{\pgfqpoint{2.547882in}{1.847071in}}%
\pgfpathlineto{\pgfqpoint{2.556818in}{1.847071in}}%
\pgfpathlineto{\pgfqpoint{2.556818in}{1.935488in}}%
\pgfpathlineto{\pgfqpoint{2.547882in}{1.935488in}}%
\pgfpathlineto{\pgfqpoint{2.547882in}{1.847071in}}%
\pgfpathclose%
\pgfusepath{fill}%
\end{pgfscope}%
\begin{pgfscope}%
\pgfpathrectangle{\pgfqpoint{0.697024in}{0.857143in}}{\pgfqpoint{2.627103in}{1.813434in}}%
\pgfusepath{clip}%
\pgfsetbuttcap%
\pgfsetmiterjoin%
\definecolor{currentfill}{rgb}{0.066899,0.263188,0.377594}%
\pgfsetfillcolor{currentfill}%
\pgfsetlinewidth{0.000000pt}%
\definecolor{currentstroke}{rgb}{0.000000,0.000000,0.000000}%
\pgfsetstrokecolor{currentstroke}%
\pgfsetstrokeopacity{0.000000}%
\pgfsetdash{}{0pt}%
\pgfpathmoveto{\pgfqpoint{2.559053in}{1.847071in}}%
\pgfpathlineto{\pgfqpoint{2.567989in}{1.847071in}}%
\pgfpathlineto{\pgfqpoint{2.567989in}{1.941032in}}%
\pgfpathlineto{\pgfqpoint{2.559053in}{1.941032in}}%
\pgfpathlineto{\pgfqpoint{2.559053in}{1.847071in}}%
\pgfpathclose%
\pgfusepath{fill}%
\end{pgfscope}%
\begin{pgfscope}%
\pgfpathrectangle{\pgfqpoint{0.697024in}{0.857143in}}{\pgfqpoint{2.627103in}{1.813434in}}%
\pgfusepath{clip}%
\pgfsetbuttcap%
\pgfsetmiterjoin%
\definecolor{currentfill}{rgb}{0.066899,0.263188,0.377594}%
\pgfsetfillcolor{currentfill}%
\pgfsetlinewidth{0.000000pt}%
\definecolor{currentstroke}{rgb}{0.000000,0.000000,0.000000}%
\pgfsetstrokecolor{currentstroke}%
\pgfsetstrokeopacity{0.000000}%
\pgfsetdash{}{0pt}%
\pgfpathmoveto{\pgfqpoint{2.570223in}{1.847071in}}%
\pgfpathlineto{\pgfqpoint{2.579160in}{1.847071in}}%
\pgfpathlineto{\pgfqpoint{2.579160in}{1.925871in}}%
\pgfpathlineto{\pgfqpoint{2.570223in}{1.925871in}}%
\pgfpathlineto{\pgfqpoint{2.570223in}{1.847071in}}%
\pgfpathclose%
\pgfusepath{fill}%
\end{pgfscope}%
\begin{pgfscope}%
\pgfpathrectangle{\pgfqpoint{0.697024in}{0.857143in}}{\pgfqpoint{2.627103in}{1.813434in}}%
\pgfusepath{clip}%
\pgfsetbuttcap%
\pgfsetmiterjoin%
\definecolor{currentfill}{rgb}{0.066899,0.263188,0.377594}%
\pgfsetfillcolor{currentfill}%
\pgfsetlinewidth{0.000000pt}%
\definecolor{currentstroke}{rgb}{0.000000,0.000000,0.000000}%
\pgfsetstrokecolor{currentstroke}%
\pgfsetstrokeopacity{0.000000}%
\pgfsetdash{}{0pt}%
\pgfpathmoveto{\pgfqpoint{2.581394in}{1.847071in}}%
\pgfpathlineto{\pgfqpoint{2.590330in}{1.847071in}}%
\pgfpathlineto{\pgfqpoint{2.590330in}{1.924111in}}%
\pgfpathlineto{\pgfqpoint{2.581394in}{1.924111in}}%
\pgfpathlineto{\pgfqpoint{2.581394in}{1.847071in}}%
\pgfpathclose%
\pgfusepath{fill}%
\end{pgfscope}%
\begin{pgfscope}%
\pgfpathrectangle{\pgfqpoint{0.697024in}{0.857143in}}{\pgfqpoint{2.627103in}{1.813434in}}%
\pgfusepath{clip}%
\pgfsetbuttcap%
\pgfsetmiterjoin%
\definecolor{currentfill}{rgb}{0.066899,0.263188,0.377594}%
\pgfsetfillcolor{currentfill}%
\pgfsetlinewidth{0.000000pt}%
\definecolor{currentstroke}{rgb}{0.000000,0.000000,0.000000}%
\pgfsetstrokecolor{currentstroke}%
\pgfsetstrokeopacity{0.000000}%
\pgfsetdash{}{0pt}%
\pgfpathmoveto{\pgfqpoint{2.592564in}{1.847071in}}%
\pgfpathlineto{\pgfqpoint{2.601501in}{1.847071in}}%
\pgfpathlineto{\pgfqpoint{2.601501in}{1.914299in}}%
\pgfpathlineto{\pgfqpoint{2.592564in}{1.914299in}}%
\pgfpathlineto{\pgfqpoint{2.592564in}{1.847071in}}%
\pgfpathclose%
\pgfusepath{fill}%
\end{pgfscope}%
\begin{pgfscope}%
\pgfpathrectangle{\pgfqpoint{0.697024in}{0.857143in}}{\pgfqpoint{2.627103in}{1.813434in}}%
\pgfusepath{clip}%
\pgfsetbuttcap%
\pgfsetmiterjoin%
\definecolor{currentfill}{rgb}{0.066899,0.263188,0.377594}%
\pgfsetfillcolor{currentfill}%
\pgfsetlinewidth{0.000000pt}%
\definecolor{currentstroke}{rgb}{0.000000,0.000000,0.000000}%
\pgfsetstrokecolor{currentstroke}%
\pgfsetstrokeopacity{0.000000}%
\pgfsetdash{}{0pt}%
\pgfpathmoveto{\pgfqpoint{2.603735in}{1.847071in}}%
\pgfpathlineto{\pgfqpoint{2.612672in}{1.847071in}}%
\pgfpathlineto{\pgfqpoint{2.612672in}{1.896998in}}%
\pgfpathlineto{\pgfqpoint{2.603735in}{1.896998in}}%
\pgfpathlineto{\pgfqpoint{2.603735in}{1.847071in}}%
\pgfpathclose%
\pgfusepath{fill}%
\end{pgfscope}%
\begin{pgfscope}%
\pgfpathrectangle{\pgfqpoint{0.697024in}{0.857143in}}{\pgfqpoint{2.627103in}{1.813434in}}%
\pgfusepath{clip}%
\pgfsetbuttcap%
\pgfsetmiterjoin%
\definecolor{currentfill}{rgb}{0.066899,0.263188,0.377594}%
\pgfsetfillcolor{currentfill}%
\pgfsetlinewidth{0.000000pt}%
\definecolor{currentstroke}{rgb}{0.000000,0.000000,0.000000}%
\pgfsetstrokecolor{currentstroke}%
\pgfsetstrokeopacity{0.000000}%
\pgfsetdash{}{0pt}%
\pgfpathmoveto{\pgfqpoint{2.614906in}{1.847071in}}%
\pgfpathlineto{\pgfqpoint{2.623842in}{1.847071in}}%
\pgfpathlineto{\pgfqpoint{2.623842in}{1.899754in}}%
\pgfpathlineto{\pgfqpoint{2.614906in}{1.899754in}}%
\pgfpathlineto{\pgfqpoint{2.614906in}{1.847071in}}%
\pgfpathclose%
\pgfusepath{fill}%
\end{pgfscope}%
\begin{pgfscope}%
\pgfpathrectangle{\pgfqpoint{0.697024in}{0.857143in}}{\pgfqpoint{2.627103in}{1.813434in}}%
\pgfusepath{clip}%
\pgfsetbuttcap%
\pgfsetmiterjoin%
\definecolor{currentfill}{rgb}{0.066899,0.263188,0.377594}%
\pgfsetfillcolor{currentfill}%
\pgfsetlinewidth{0.000000pt}%
\definecolor{currentstroke}{rgb}{0.000000,0.000000,0.000000}%
\pgfsetstrokecolor{currentstroke}%
\pgfsetstrokeopacity{0.000000}%
\pgfsetdash{}{0pt}%
\pgfpathmoveto{\pgfqpoint{2.626076in}{1.847071in}}%
\pgfpathlineto{\pgfqpoint{2.635013in}{1.847071in}}%
\pgfpathlineto{\pgfqpoint{2.635013in}{1.899544in}}%
\pgfpathlineto{\pgfqpoint{2.626076in}{1.899544in}}%
\pgfpathlineto{\pgfqpoint{2.626076in}{1.847071in}}%
\pgfpathclose%
\pgfusepath{fill}%
\end{pgfscope}%
\begin{pgfscope}%
\pgfpathrectangle{\pgfqpoint{0.697024in}{0.857143in}}{\pgfqpoint{2.627103in}{1.813434in}}%
\pgfusepath{clip}%
\pgfsetbuttcap%
\pgfsetmiterjoin%
\definecolor{currentfill}{rgb}{0.066899,0.263188,0.377594}%
\pgfsetfillcolor{currentfill}%
\pgfsetlinewidth{0.000000pt}%
\definecolor{currentstroke}{rgb}{0.000000,0.000000,0.000000}%
\pgfsetstrokecolor{currentstroke}%
\pgfsetstrokeopacity{0.000000}%
\pgfsetdash{}{0pt}%
\pgfpathmoveto{\pgfqpoint{2.637247in}{1.847071in}}%
\pgfpathlineto{\pgfqpoint{2.646183in}{1.847071in}}%
\pgfpathlineto{\pgfqpoint{2.646183in}{1.893783in}}%
\pgfpathlineto{\pgfqpoint{2.637247in}{1.893783in}}%
\pgfpathlineto{\pgfqpoint{2.637247in}{1.847071in}}%
\pgfpathclose%
\pgfusepath{fill}%
\end{pgfscope}%
\begin{pgfscope}%
\pgfpathrectangle{\pgfqpoint{0.697024in}{0.857143in}}{\pgfqpoint{2.627103in}{1.813434in}}%
\pgfusepath{clip}%
\pgfsetbuttcap%
\pgfsetmiterjoin%
\definecolor{currentfill}{rgb}{0.066899,0.263188,0.377594}%
\pgfsetfillcolor{currentfill}%
\pgfsetlinewidth{0.000000pt}%
\definecolor{currentstroke}{rgb}{0.000000,0.000000,0.000000}%
\pgfsetstrokecolor{currentstroke}%
\pgfsetstrokeopacity{0.000000}%
\pgfsetdash{}{0pt}%
\pgfpathmoveto{\pgfqpoint{2.648417in}{1.847071in}}%
\pgfpathlineto{\pgfqpoint{2.657354in}{1.847071in}}%
\pgfpathlineto{\pgfqpoint{2.657354in}{1.908001in}}%
\pgfpathlineto{\pgfqpoint{2.648417in}{1.908001in}}%
\pgfpathlineto{\pgfqpoint{2.648417in}{1.847071in}}%
\pgfpathclose%
\pgfusepath{fill}%
\end{pgfscope}%
\begin{pgfscope}%
\pgfpathrectangle{\pgfqpoint{0.697024in}{0.857143in}}{\pgfqpoint{2.627103in}{1.813434in}}%
\pgfusepath{clip}%
\pgfsetbuttcap%
\pgfsetmiterjoin%
\definecolor{currentfill}{rgb}{0.066899,0.263188,0.377594}%
\pgfsetfillcolor{currentfill}%
\pgfsetlinewidth{0.000000pt}%
\definecolor{currentstroke}{rgb}{0.000000,0.000000,0.000000}%
\pgfsetstrokecolor{currentstroke}%
\pgfsetstrokeopacity{0.000000}%
\pgfsetdash{}{0pt}%
\pgfpathmoveto{\pgfqpoint{2.659588in}{1.847071in}}%
\pgfpathlineto{\pgfqpoint{2.668525in}{1.847071in}}%
\pgfpathlineto{\pgfqpoint{2.668525in}{1.921300in}}%
\pgfpathlineto{\pgfqpoint{2.659588in}{1.921300in}}%
\pgfpathlineto{\pgfqpoint{2.659588in}{1.847071in}}%
\pgfpathclose%
\pgfusepath{fill}%
\end{pgfscope}%
\begin{pgfscope}%
\pgfpathrectangle{\pgfqpoint{0.697024in}{0.857143in}}{\pgfqpoint{2.627103in}{1.813434in}}%
\pgfusepath{clip}%
\pgfsetbuttcap%
\pgfsetmiterjoin%
\definecolor{currentfill}{rgb}{0.066899,0.263188,0.377594}%
\pgfsetfillcolor{currentfill}%
\pgfsetlinewidth{0.000000pt}%
\definecolor{currentstroke}{rgb}{0.000000,0.000000,0.000000}%
\pgfsetstrokecolor{currentstroke}%
\pgfsetstrokeopacity{0.000000}%
\pgfsetdash{}{0pt}%
\pgfpathmoveto{\pgfqpoint{2.670759in}{1.847071in}}%
\pgfpathlineto{\pgfqpoint{2.679695in}{1.847071in}}%
\pgfpathlineto{\pgfqpoint{2.679695in}{1.904614in}}%
\pgfpathlineto{\pgfqpoint{2.670759in}{1.904614in}}%
\pgfpathlineto{\pgfqpoint{2.670759in}{1.847071in}}%
\pgfpathclose%
\pgfusepath{fill}%
\end{pgfscope}%
\begin{pgfscope}%
\pgfpathrectangle{\pgfqpoint{0.697024in}{0.857143in}}{\pgfqpoint{2.627103in}{1.813434in}}%
\pgfusepath{clip}%
\pgfsetbuttcap%
\pgfsetmiterjoin%
\definecolor{currentfill}{rgb}{0.066899,0.263188,0.377594}%
\pgfsetfillcolor{currentfill}%
\pgfsetlinewidth{0.000000pt}%
\definecolor{currentstroke}{rgb}{0.000000,0.000000,0.000000}%
\pgfsetstrokecolor{currentstroke}%
\pgfsetstrokeopacity{0.000000}%
\pgfsetdash{}{0pt}%
\pgfpathmoveto{\pgfqpoint{2.681929in}{1.847071in}}%
\pgfpathlineto{\pgfqpoint{2.690866in}{1.847071in}}%
\pgfpathlineto{\pgfqpoint{2.690866in}{1.929050in}}%
\pgfpathlineto{\pgfqpoint{2.681929in}{1.929050in}}%
\pgfpathlineto{\pgfqpoint{2.681929in}{1.847071in}}%
\pgfpathclose%
\pgfusepath{fill}%
\end{pgfscope}%
\begin{pgfscope}%
\pgfpathrectangle{\pgfqpoint{0.697024in}{0.857143in}}{\pgfqpoint{2.627103in}{1.813434in}}%
\pgfusepath{clip}%
\pgfsetbuttcap%
\pgfsetmiterjoin%
\definecolor{currentfill}{rgb}{0.066899,0.263188,0.377594}%
\pgfsetfillcolor{currentfill}%
\pgfsetlinewidth{0.000000pt}%
\definecolor{currentstroke}{rgb}{0.000000,0.000000,0.000000}%
\pgfsetstrokecolor{currentstroke}%
\pgfsetstrokeopacity{0.000000}%
\pgfsetdash{}{0pt}%
\pgfpathmoveto{\pgfqpoint{2.693100in}{1.847071in}}%
\pgfpathlineto{\pgfqpoint{2.702036in}{1.847071in}}%
\pgfpathlineto{\pgfqpoint{2.702036in}{1.931966in}}%
\pgfpathlineto{\pgfqpoint{2.693100in}{1.931966in}}%
\pgfpathlineto{\pgfqpoint{2.693100in}{1.847071in}}%
\pgfpathclose%
\pgfusepath{fill}%
\end{pgfscope}%
\begin{pgfscope}%
\pgfpathrectangle{\pgfqpoint{0.697024in}{0.857143in}}{\pgfqpoint{2.627103in}{1.813434in}}%
\pgfusepath{clip}%
\pgfsetbuttcap%
\pgfsetmiterjoin%
\definecolor{currentfill}{rgb}{0.066899,0.263188,0.377594}%
\pgfsetfillcolor{currentfill}%
\pgfsetlinewidth{0.000000pt}%
\definecolor{currentstroke}{rgb}{0.000000,0.000000,0.000000}%
\pgfsetstrokecolor{currentstroke}%
\pgfsetstrokeopacity{0.000000}%
\pgfsetdash{}{0pt}%
\pgfpathmoveto{\pgfqpoint{2.704270in}{1.847071in}}%
\pgfpathlineto{\pgfqpoint{2.713207in}{1.847071in}}%
\pgfpathlineto{\pgfqpoint{2.713207in}{1.929741in}}%
\pgfpathlineto{\pgfqpoint{2.704270in}{1.929741in}}%
\pgfpathlineto{\pgfqpoint{2.704270in}{1.847071in}}%
\pgfpathclose%
\pgfusepath{fill}%
\end{pgfscope}%
\begin{pgfscope}%
\pgfpathrectangle{\pgfqpoint{0.697024in}{0.857143in}}{\pgfqpoint{2.627103in}{1.813434in}}%
\pgfusepath{clip}%
\pgfsetbuttcap%
\pgfsetmiterjoin%
\definecolor{currentfill}{rgb}{0.066899,0.263188,0.377594}%
\pgfsetfillcolor{currentfill}%
\pgfsetlinewidth{0.000000pt}%
\definecolor{currentstroke}{rgb}{0.000000,0.000000,0.000000}%
\pgfsetstrokecolor{currentstroke}%
\pgfsetstrokeopacity{0.000000}%
\pgfsetdash{}{0pt}%
\pgfpathmoveto{\pgfqpoint{2.715441in}{1.847071in}}%
\pgfpathlineto{\pgfqpoint{2.724378in}{1.847071in}}%
\pgfpathlineto{\pgfqpoint{2.724378in}{1.964849in}}%
\pgfpathlineto{\pgfqpoint{2.715441in}{1.964849in}}%
\pgfpathlineto{\pgfqpoint{2.715441in}{1.847071in}}%
\pgfpathclose%
\pgfusepath{fill}%
\end{pgfscope}%
\begin{pgfscope}%
\pgfpathrectangle{\pgfqpoint{0.697024in}{0.857143in}}{\pgfqpoint{2.627103in}{1.813434in}}%
\pgfusepath{clip}%
\pgfsetbuttcap%
\pgfsetmiterjoin%
\definecolor{currentfill}{rgb}{0.066899,0.263188,0.377594}%
\pgfsetfillcolor{currentfill}%
\pgfsetlinewidth{0.000000pt}%
\definecolor{currentstroke}{rgb}{0.000000,0.000000,0.000000}%
\pgfsetstrokecolor{currentstroke}%
\pgfsetstrokeopacity{0.000000}%
\pgfsetdash{}{0pt}%
\pgfpathmoveto{\pgfqpoint{2.726612in}{1.847071in}}%
\pgfpathlineto{\pgfqpoint{2.735548in}{1.847071in}}%
\pgfpathlineto{\pgfqpoint{2.735548in}{2.017346in}}%
\pgfpathlineto{\pgfqpoint{2.726612in}{2.017346in}}%
\pgfpathlineto{\pgfqpoint{2.726612in}{1.847071in}}%
\pgfpathclose%
\pgfusepath{fill}%
\end{pgfscope}%
\begin{pgfscope}%
\pgfpathrectangle{\pgfqpoint{0.697024in}{0.857143in}}{\pgfqpoint{2.627103in}{1.813434in}}%
\pgfusepath{clip}%
\pgfsetbuttcap%
\pgfsetmiterjoin%
\definecolor{currentfill}{rgb}{0.066899,0.263188,0.377594}%
\pgfsetfillcolor{currentfill}%
\pgfsetlinewidth{0.000000pt}%
\definecolor{currentstroke}{rgb}{0.000000,0.000000,0.000000}%
\pgfsetstrokecolor{currentstroke}%
\pgfsetstrokeopacity{0.000000}%
\pgfsetdash{}{0pt}%
\pgfpathmoveto{\pgfqpoint{2.737782in}{1.847071in}}%
\pgfpathlineto{\pgfqpoint{2.746719in}{1.847071in}}%
\pgfpathlineto{\pgfqpoint{2.746719in}{2.044550in}}%
\pgfpathlineto{\pgfqpoint{2.737782in}{2.044550in}}%
\pgfpathlineto{\pgfqpoint{2.737782in}{1.847071in}}%
\pgfpathclose%
\pgfusepath{fill}%
\end{pgfscope}%
\begin{pgfscope}%
\pgfpathrectangle{\pgfqpoint{0.697024in}{0.857143in}}{\pgfqpoint{2.627103in}{1.813434in}}%
\pgfusepath{clip}%
\pgfsetbuttcap%
\pgfsetmiterjoin%
\definecolor{currentfill}{rgb}{0.066899,0.263188,0.377594}%
\pgfsetfillcolor{currentfill}%
\pgfsetlinewidth{0.000000pt}%
\definecolor{currentstroke}{rgb}{0.000000,0.000000,0.000000}%
\pgfsetstrokecolor{currentstroke}%
\pgfsetstrokeopacity{0.000000}%
\pgfsetdash{}{0pt}%
\pgfpathmoveto{\pgfqpoint{2.748953in}{1.847071in}}%
\pgfpathlineto{\pgfqpoint{2.757889in}{1.847071in}}%
\pgfpathlineto{\pgfqpoint{2.757889in}{2.057299in}}%
\pgfpathlineto{\pgfqpoint{2.748953in}{2.057299in}}%
\pgfpathlineto{\pgfqpoint{2.748953in}{1.847071in}}%
\pgfpathclose%
\pgfusepath{fill}%
\end{pgfscope}%
\begin{pgfscope}%
\pgfpathrectangle{\pgfqpoint{0.697024in}{0.857143in}}{\pgfqpoint{2.627103in}{1.813434in}}%
\pgfusepath{clip}%
\pgfsetbuttcap%
\pgfsetmiterjoin%
\definecolor{currentfill}{rgb}{0.066899,0.263188,0.377594}%
\pgfsetfillcolor{currentfill}%
\pgfsetlinewidth{0.000000pt}%
\definecolor{currentstroke}{rgb}{0.000000,0.000000,0.000000}%
\pgfsetstrokecolor{currentstroke}%
\pgfsetstrokeopacity{0.000000}%
\pgfsetdash{}{0pt}%
\pgfpathmoveto{\pgfqpoint{2.760124in}{1.847071in}}%
\pgfpathlineto{\pgfqpoint{2.769060in}{1.847071in}}%
\pgfpathlineto{\pgfqpoint{2.769060in}{2.041499in}}%
\pgfpathlineto{\pgfqpoint{2.760124in}{2.041499in}}%
\pgfpathlineto{\pgfqpoint{2.760124in}{1.847071in}}%
\pgfpathclose%
\pgfusepath{fill}%
\end{pgfscope}%
\begin{pgfscope}%
\pgfpathrectangle{\pgfqpoint{0.697024in}{0.857143in}}{\pgfqpoint{2.627103in}{1.813434in}}%
\pgfusepath{clip}%
\pgfsetbuttcap%
\pgfsetmiterjoin%
\definecolor{currentfill}{rgb}{0.066899,0.263188,0.377594}%
\pgfsetfillcolor{currentfill}%
\pgfsetlinewidth{0.000000pt}%
\definecolor{currentstroke}{rgb}{0.000000,0.000000,0.000000}%
\pgfsetstrokecolor{currentstroke}%
\pgfsetstrokeopacity{0.000000}%
\pgfsetdash{}{0pt}%
\pgfpathmoveto{\pgfqpoint{2.771294in}{1.847071in}}%
\pgfpathlineto{\pgfqpoint{2.780231in}{1.847071in}}%
\pgfpathlineto{\pgfqpoint{2.780231in}{2.034910in}}%
\pgfpathlineto{\pgfqpoint{2.771294in}{2.034910in}}%
\pgfpathlineto{\pgfqpoint{2.771294in}{1.847071in}}%
\pgfpathclose%
\pgfusepath{fill}%
\end{pgfscope}%
\begin{pgfscope}%
\pgfpathrectangle{\pgfqpoint{0.697024in}{0.857143in}}{\pgfqpoint{2.627103in}{1.813434in}}%
\pgfusepath{clip}%
\pgfsetbuttcap%
\pgfsetmiterjoin%
\definecolor{currentfill}{rgb}{0.066899,0.263188,0.377594}%
\pgfsetfillcolor{currentfill}%
\pgfsetlinewidth{0.000000pt}%
\definecolor{currentstroke}{rgb}{0.000000,0.000000,0.000000}%
\pgfsetstrokecolor{currentstroke}%
\pgfsetstrokeopacity{0.000000}%
\pgfsetdash{}{0pt}%
\pgfpathmoveto{\pgfqpoint{2.782465in}{1.847071in}}%
\pgfpathlineto{\pgfqpoint{2.791401in}{1.847071in}}%
\pgfpathlineto{\pgfqpoint{2.791401in}{2.032331in}}%
\pgfpathlineto{\pgfqpoint{2.782465in}{2.032331in}}%
\pgfpathlineto{\pgfqpoint{2.782465in}{1.847071in}}%
\pgfpathclose%
\pgfusepath{fill}%
\end{pgfscope}%
\begin{pgfscope}%
\pgfpathrectangle{\pgfqpoint{0.697024in}{0.857143in}}{\pgfqpoint{2.627103in}{1.813434in}}%
\pgfusepath{clip}%
\pgfsetbuttcap%
\pgfsetmiterjoin%
\definecolor{currentfill}{rgb}{0.066899,0.263188,0.377594}%
\pgfsetfillcolor{currentfill}%
\pgfsetlinewidth{0.000000pt}%
\definecolor{currentstroke}{rgb}{0.000000,0.000000,0.000000}%
\pgfsetstrokecolor{currentstroke}%
\pgfsetstrokeopacity{0.000000}%
\pgfsetdash{}{0pt}%
\pgfpathmoveto{\pgfqpoint{2.793635in}{1.847071in}}%
\pgfpathlineto{\pgfqpoint{2.802572in}{1.847071in}}%
\pgfpathlineto{\pgfqpoint{2.802572in}{2.022229in}}%
\pgfpathlineto{\pgfqpoint{2.793635in}{2.022229in}}%
\pgfpathlineto{\pgfqpoint{2.793635in}{1.847071in}}%
\pgfpathclose%
\pgfusepath{fill}%
\end{pgfscope}%
\begin{pgfscope}%
\pgfpathrectangle{\pgfqpoint{0.697024in}{0.857143in}}{\pgfqpoint{2.627103in}{1.813434in}}%
\pgfusepath{clip}%
\pgfsetbuttcap%
\pgfsetmiterjoin%
\definecolor{currentfill}{rgb}{0.066899,0.263188,0.377594}%
\pgfsetfillcolor{currentfill}%
\pgfsetlinewidth{0.000000pt}%
\definecolor{currentstroke}{rgb}{0.000000,0.000000,0.000000}%
\pgfsetstrokecolor{currentstroke}%
\pgfsetstrokeopacity{0.000000}%
\pgfsetdash{}{0pt}%
\pgfpathmoveto{\pgfqpoint{2.804806in}{1.847071in}}%
\pgfpathlineto{\pgfqpoint{2.813742in}{1.847071in}}%
\pgfpathlineto{\pgfqpoint{2.813742in}{1.999940in}}%
\pgfpathlineto{\pgfqpoint{2.804806in}{1.999940in}}%
\pgfpathlineto{\pgfqpoint{2.804806in}{1.847071in}}%
\pgfpathclose%
\pgfusepath{fill}%
\end{pgfscope}%
\begin{pgfscope}%
\pgfpathrectangle{\pgfqpoint{0.697024in}{0.857143in}}{\pgfqpoint{2.627103in}{1.813434in}}%
\pgfusepath{clip}%
\pgfsetbuttcap%
\pgfsetmiterjoin%
\definecolor{currentfill}{rgb}{0.066899,0.263188,0.377594}%
\pgfsetfillcolor{currentfill}%
\pgfsetlinewidth{0.000000pt}%
\definecolor{currentstroke}{rgb}{0.000000,0.000000,0.000000}%
\pgfsetstrokecolor{currentstroke}%
\pgfsetstrokeopacity{0.000000}%
\pgfsetdash{}{0pt}%
\pgfpathmoveto{\pgfqpoint{2.815977in}{1.847071in}}%
\pgfpathlineto{\pgfqpoint{2.824913in}{1.847071in}}%
\pgfpathlineto{\pgfqpoint{2.824913in}{1.982260in}}%
\pgfpathlineto{\pgfqpoint{2.815977in}{1.982260in}}%
\pgfpathlineto{\pgfqpoint{2.815977in}{1.847071in}}%
\pgfpathclose%
\pgfusepath{fill}%
\end{pgfscope}%
\begin{pgfscope}%
\pgfpathrectangle{\pgfqpoint{0.697024in}{0.857143in}}{\pgfqpoint{2.627103in}{1.813434in}}%
\pgfusepath{clip}%
\pgfsetbuttcap%
\pgfsetmiterjoin%
\definecolor{currentfill}{rgb}{0.066899,0.263188,0.377594}%
\pgfsetfillcolor{currentfill}%
\pgfsetlinewidth{0.000000pt}%
\definecolor{currentstroke}{rgb}{0.000000,0.000000,0.000000}%
\pgfsetstrokecolor{currentstroke}%
\pgfsetstrokeopacity{0.000000}%
\pgfsetdash{}{0pt}%
\pgfpathmoveto{\pgfqpoint{2.827147in}{1.847071in}}%
\pgfpathlineto{\pgfqpoint{2.836084in}{1.847071in}}%
\pgfpathlineto{\pgfqpoint{2.836084in}{1.965698in}}%
\pgfpathlineto{\pgfqpoint{2.827147in}{1.965698in}}%
\pgfpathlineto{\pgfqpoint{2.827147in}{1.847071in}}%
\pgfpathclose%
\pgfusepath{fill}%
\end{pgfscope}%
\begin{pgfscope}%
\pgfpathrectangle{\pgfqpoint{0.697024in}{0.857143in}}{\pgfqpoint{2.627103in}{1.813434in}}%
\pgfusepath{clip}%
\pgfsetbuttcap%
\pgfsetmiterjoin%
\definecolor{currentfill}{rgb}{0.066899,0.263188,0.377594}%
\pgfsetfillcolor{currentfill}%
\pgfsetlinewidth{0.000000pt}%
\definecolor{currentstroke}{rgb}{0.000000,0.000000,0.000000}%
\pgfsetstrokecolor{currentstroke}%
\pgfsetstrokeopacity{0.000000}%
\pgfsetdash{}{0pt}%
\pgfpathmoveto{\pgfqpoint{2.838318in}{1.847071in}}%
\pgfpathlineto{\pgfqpoint{2.847254in}{1.847071in}}%
\pgfpathlineto{\pgfqpoint{2.847254in}{1.963374in}}%
\pgfpathlineto{\pgfqpoint{2.838318in}{1.963374in}}%
\pgfpathlineto{\pgfqpoint{2.838318in}{1.847071in}}%
\pgfpathclose%
\pgfusepath{fill}%
\end{pgfscope}%
\begin{pgfscope}%
\pgfpathrectangle{\pgfqpoint{0.697024in}{0.857143in}}{\pgfqpoint{2.627103in}{1.813434in}}%
\pgfusepath{clip}%
\pgfsetbuttcap%
\pgfsetmiterjoin%
\definecolor{currentfill}{rgb}{0.066899,0.263188,0.377594}%
\pgfsetfillcolor{currentfill}%
\pgfsetlinewidth{0.000000pt}%
\definecolor{currentstroke}{rgb}{0.000000,0.000000,0.000000}%
\pgfsetstrokecolor{currentstroke}%
\pgfsetstrokeopacity{0.000000}%
\pgfsetdash{}{0pt}%
\pgfpathmoveto{\pgfqpoint{2.849488in}{1.847071in}}%
\pgfpathlineto{\pgfqpoint{2.858425in}{1.847071in}}%
\pgfpathlineto{\pgfqpoint{2.858425in}{1.956662in}}%
\pgfpathlineto{\pgfqpoint{2.849488in}{1.956662in}}%
\pgfpathlineto{\pgfqpoint{2.849488in}{1.847071in}}%
\pgfpathclose%
\pgfusepath{fill}%
\end{pgfscope}%
\begin{pgfscope}%
\pgfpathrectangle{\pgfqpoint{0.697024in}{0.857143in}}{\pgfqpoint{2.627103in}{1.813434in}}%
\pgfusepath{clip}%
\pgfsetbuttcap%
\pgfsetmiterjoin%
\definecolor{currentfill}{rgb}{0.066899,0.263188,0.377594}%
\pgfsetfillcolor{currentfill}%
\pgfsetlinewidth{0.000000pt}%
\definecolor{currentstroke}{rgb}{0.000000,0.000000,0.000000}%
\pgfsetstrokecolor{currentstroke}%
\pgfsetstrokeopacity{0.000000}%
\pgfsetdash{}{0pt}%
\pgfpathmoveto{\pgfqpoint{2.860659in}{1.847071in}}%
\pgfpathlineto{\pgfqpoint{2.869595in}{1.847071in}}%
\pgfpathlineto{\pgfqpoint{2.869595in}{1.955717in}}%
\pgfpathlineto{\pgfqpoint{2.860659in}{1.955717in}}%
\pgfpathlineto{\pgfqpoint{2.860659in}{1.847071in}}%
\pgfpathclose%
\pgfusepath{fill}%
\end{pgfscope}%
\begin{pgfscope}%
\pgfpathrectangle{\pgfqpoint{0.697024in}{0.857143in}}{\pgfqpoint{2.627103in}{1.813434in}}%
\pgfusepath{clip}%
\pgfsetbuttcap%
\pgfsetmiterjoin%
\definecolor{currentfill}{rgb}{0.066899,0.263188,0.377594}%
\pgfsetfillcolor{currentfill}%
\pgfsetlinewidth{0.000000pt}%
\definecolor{currentstroke}{rgb}{0.000000,0.000000,0.000000}%
\pgfsetstrokecolor{currentstroke}%
\pgfsetstrokeopacity{0.000000}%
\pgfsetdash{}{0pt}%
\pgfpathmoveto{\pgfqpoint{2.871830in}{1.847071in}}%
\pgfpathlineto{\pgfqpoint{2.880766in}{1.847071in}}%
\pgfpathlineto{\pgfqpoint{2.880766in}{1.932239in}}%
\pgfpathlineto{\pgfqpoint{2.871830in}{1.932239in}}%
\pgfpathlineto{\pgfqpoint{2.871830in}{1.847071in}}%
\pgfpathclose%
\pgfusepath{fill}%
\end{pgfscope}%
\begin{pgfscope}%
\pgfpathrectangle{\pgfqpoint{0.697024in}{0.857143in}}{\pgfqpoint{2.627103in}{1.813434in}}%
\pgfusepath{clip}%
\pgfsetbuttcap%
\pgfsetmiterjoin%
\definecolor{currentfill}{rgb}{0.066899,0.263188,0.377594}%
\pgfsetfillcolor{currentfill}%
\pgfsetlinewidth{0.000000pt}%
\definecolor{currentstroke}{rgb}{0.000000,0.000000,0.000000}%
\pgfsetstrokecolor{currentstroke}%
\pgfsetstrokeopacity{0.000000}%
\pgfsetdash{}{0pt}%
\pgfpathmoveto{\pgfqpoint{2.883000in}{1.847071in}}%
\pgfpathlineto{\pgfqpoint{2.891937in}{1.847071in}}%
\pgfpathlineto{\pgfqpoint{2.891937in}{1.910011in}}%
\pgfpathlineto{\pgfqpoint{2.883000in}{1.910011in}}%
\pgfpathlineto{\pgfqpoint{2.883000in}{1.847071in}}%
\pgfpathclose%
\pgfusepath{fill}%
\end{pgfscope}%
\begin{pgfscope}%
\pgfpathrectangle{\pgfqpoint{0.697024in}{0.857143in}}{\pgfqpoint{2.627103in}{1.813434in}}%
\pgfusepath{clip}%
\pgfsetbuttcap%
\pgfsetmiterjoin%
\definecolor{currentfill}{rgb}{0.066899,0.263188,0.377594}%
\pgfsetfillcolor{currentfill}%
\pgfsetlinewidth{0.000000pt}%
\definecolor{currentstroke}{rgb}{0.000000,0.000000,0.000000}%
\pgfsetstrokecolor{currentstroke}%
\pgfsetstrokeopacity{0.000000}%
\pgfsetdash{}{0pt}%
\pgfpathmoveto{\pgfqpoint{2.894171in}{1.847071in}}%
\pgfpathlineto{\pgfqpoint{2.903107in}{1.847071in}}%
\pgfpathlineto{\pgfqpoint{2.903107in}{1.911618in}}%
\pgfpathlineto{\pgfqpoint{2.894171in}{1.911618in}}%
\pgfpathlineto{\pgfqpoint{2.894171in}{1.847071in}}%
\pgfpathclose%
\pgfusepath{fill}%
\end{pgfscope}%
\begin{pgfscope}%
\pgfpathrectangle{\pgfqpoint{0.697024in}{0.857143in}}{\pgfqpoint{2.627103in}{1.813434in}}%
\pgfusepath{clip}%
\pgfsetbuttcap%
\pgfsetmiterjoin%
\definecolor{currentfill}{rgb}{0.066899,0.263188,0.377594}%
\pgfsetfillcolor{currentfill}%
\pgfsetlinewidth{0.000000pt}%
\definecolor{currentstroke}{rgb}{0.000000,0.000000,0.000000}%
\pgfsetstrokecolor{currentstroke}%
\pgfsetstrokeopacity{0.000000}%
\pgfsetdash{}{0pt}%
\pgfpathmoveto{\pgfqpoint{2.905341in}{1.847071in}}%
\pgfpathlineto{\pgfqpoint{2.914278in}{1.847071in}}%
\pgfpathlineto{\pgfqpoint{2.914278in}{1.897722in}}%
\pgfpathlineto{\pgfqpoint{2.905341in}{1.897722in}}%
\pgfpathlineto{\pgfqpoint{2.905341in}{1.847071in}}%
\pgfpathclose%
\pgfusepath{fill}%
\end{pgfscope}%
\begin{pgfscope}%
\pgfpathrectangle{\pgfqpoint{0.697024in}{0.857143in}}{\pgfqpoint{2.627103in}{1.813434in}}%
\pgfusepath{clip}%
\pgfsetbuttcap%
\pgfsetmiterjoin%
\definecolor{currentfill}{rgb}{0.066899,0.263188,0.377594}%
\pgfsetfillcolor{currentfill}%
\pgfsetlinewidth{0.000000pt}%
\definecolor{currentstroke}{rgb}{0.000000,0.000000,0.000000}%
\pgfsetstrokecolor{currentstroke}%
\pgfsetstrokeopacity{0.000000}%
\pgfsetdash{}{0pt}%
\pgfpathmoveto{\pgfqpoint{2.916512in}{1.847071in}}%
\pgfpathlineto{\pgfqpoint{2.925448in}{1.847071in}}%
\pgfpathlineto{\pgfqpoint{2.925448in}{1.901874in}}%
\pgfpathlineto{\pgfqpoint{2.916512in}{1.901874in}}%
\pgfpathlineto{\pgfqpoint{2.916512in}{1.847071in}}%
\pgfpathclose%
\pgfusepath{fill}%
\end{pgfscope}%
\begin{pgfscope}%
\pgfpathrectangle{\pgfqpoint{0.697024in}{0.857143in}}{\pgfqpoint{2.627103in}{1.813434in}}%
\pgfusepath{clip}%
\pgfsetbuttcap%
\pgfsetmiterjoin%
\definecolor{currentfill}{rgb}{0.066899,0.263188,0.377594}%
\pgfsetfillcolor{currentfill}%
\pgfsetlinewidth{0.000000pt}%
\definecolor{currentstroke}{rgb}{0.000000,0.000000,0.000000}%
\pgfsetstrokecolor{currentstroke}%
\pgfsetstrokeopacity{0.000000}%
\pgfsetdash{}{0pt}%
\pgfpathmoveto{\pgfqpoint{2.927683in}{1.847071in}}%
\pgfpathlineto{\pgfqpoint{2.936619in}{1.847071in}}%
\pgfpathlineto{\pgfqpoint{2.936619in}{1.908171in}}%
\pgfpathlineto{\pgfqpoint{2.927683in}{1.908171in}}%
\pgfpathlineto{\pgfqpoint{2.927683in}{1.847071in}}%
\pgfpathclose%
\pgfusepath{fill}%
\end{pgfscope}%
\begin{pgfscope}%
\pgfpathrectangle{\pgfqpoint{0.697024in}{0.857143in}}{\pgfqpoint{2.627103in}{1.813434in}}%
\pgfusepath{clip}%
\pgfsetbuttcap%
\pgfsetmiterjoin%
\definecolor{currentfill}{rgb}{0.066899,0.263188,0.377594}%
\pgfsetfillcolor{currentfill}%
\pgfsetlinewidth{0.000000pt}%
\definecolor{currentstroke}{rgb}{0.000000,0.000000,0.000000}%
\pgfsetstrokecolor{currentstroke}%
\pgfsetstrokeopacity{0.000000}%
\pgfsetdash{}{0pt}%
\pgfpathmoveto{\pgfqpoint{2.938853in}{1.847071in}}%
\pgfpathlineto{\pgfqpoint{2.947790in}{1.847071in}}%
\pgfpathlineto{\pgfqpoint{2.947790in}{1.873308in}}%
\pgfpathlineto{\pgfqpoint{2.938853in}{1.873308in}}%
\pgfpathlineto{\pgfqpoint{2.938853in}{1.847071in}}%
\pgfpathclose%
\pgfusepath{fill}%
\end{pgfscope}%
\begin{pgfscope}%
\pgfpathrectangle{\pgfqpoint{0.697024in}{0.857143in}}{\pgfqpoint{2.627103in}{1.813434in}}%
\pgfusepath{clip}%
\pgfsetbuttcap%
\pgfsetmiterjoin%
\definecolor{currentfill}{rgb}{0.066899,0.263188,0.377594}%
\pgfsetfillcolor{currentfill}%
\pgfsetlinewidth{0.000000pt}%
\definecolor{currentstroke}{rgb}{0.000000,0.000000,0.000000}%
\pgfsetstrokecolor{currentstroke}%
\pgfsetstrokeopacity{0.000000}%
\pgfsetdash{}{0pt}%
\pgfpathmoveto{\pgfqpoint{2.950024in}{1.847071in}}%
\pgfpathlineto{\pgfqpoint{2.958960in}{1.847071in}}%
\pgfpathlineto{\pgfqpoint{2.958960in}{1.872685in}}%
\pgfpathlineto{\pgfqpoint{2.950024in}{1.872685in}}%
\pgfpathlineto{\pgfqpoint{2.950024in}{1.847071in}}%
\pgfpathclose%
\pgfusepath{fill}%
\end{pgfscope}%
\begin{pgfscope}%
\pgfpathrectangle{\pgfqpoint{0.697024in}{0.857143in}}{\pgfqpoint{2.627103in}{1.813434in}}%
\pgfusepath{clip}%
\pgfsetbuttcap%
\pgfsetmiterjoin%
\definecolor{currentfill}{rgb}{0.066899,0.263188,0.377594}%
\pgfsetfillcolor{currentfill}%
\pgfsetlinewidth{0.000000pt}%
\definecolor{currentstroke}{rgb}{0.000000,0.000000,0.000000}%
\pgfsetstrokecolor{currentstroke}%
\pgfsetstrokeopacity{0.000000}%
\pgfsetdash{}{0pt}%
\pgfpathmoveto{\pgfqpoint{2.961194in}{1.847071in}}%
\pgfpathlineto{\pgfqpoint{2.970131in}{1.847071in}}%
\pgfpathlineto{\pgfqpoint{2.970131in}{1.874790in}}%
\pgfpathlineto{\pgfqpoint{2.961194in}{1.874790in}}%
\pgfpathlineto{\pgfqpoint{2.961194in}{1.847071in}}%
\pgfpathclose%
\pgfusepath{fill}%
\end{pgfscope}%
\begin{pgfscope}%
\pgfpathrectangle{\pgfqpoint{0.697024in}{0.857143in}}{\pgfqpoint{2.627103in}{1.813434in}}%
\pgfusepath{clip}%
\pgfsetbuttcap%
\pgfsetmiterjoin%
\definecolor{currentfill}{rgb}{0.066899,0.263188,0.377594}%
\pgfsetfillcolor{currentfill}%
\pgfsetlinewidth{0.000000pt}%
\definecolor{currentstroke}{rgb}{0.000000,0.000000,0.000000}%
\pgfsetstrokecolor{currentstroke}%
\pgfsetstrokeopacity{0.000000}%
\pgfsetdash{}{0pt}%
\pgfpathmoveto{\pgfqpoint{2.972365in}{1.847071in}}%
\pgfpathlineto{\pgfqpoint{2.981301in}{1.847071in}}%
\pgfpathlineto{\pgfqpoint{2.981301in}{1.848178in}}%
\pgfpathlineto{\pgfqpoint{2.972365in}{1.848178in}}%
\pgfpathlineto{\pgfqpoint{2.972365in}{1.847071in}}%
\pgfpathclose%
\pgfusepath{fill}%
\end{pgfscope}%
\begin{pgfscope}%
\pgfpathrectangle{\pgfqpoint{0.697024in}{0.857143in}}{\pgfqpoint{2.627103in}{1.813434in}}%
\pgfusepath{clip}%
\pgfsetbuttcap%
\pgfsetmiterjoin%
\definecolor{currentfill}{rgb}{0.066899,0.263188,0.377594}%
\pgfsetfillcolor{currentfill}%
\pgfsetlinewidth{0.000000pt}%
\definecolor{currentstroke}{rgb}{0.000000,0.000000,0.000000}%
\pgfsetstrokecolor{currentstroke}%
\pgfsetstrokeopacity{0.000000}%
\pgfsetdash{}{0pt}%
\pgfpathmoveto{\pgfqpoint{2.983536in}{1.847071in}}%
\pgfpathlineto{\pgfqpoint{2.992472in}{1.847071in}}%
\pgfpathlineto{\pgfqpoint{2.992472in}{1.851318in}}%
\pgfpathlineto{\pgfqpoint{2.983536in}{1.851318in}}%
\pgfpathlineto{\pgfqpoint{2.983536in}{1.847071in}}%
\pgfpathclose%
\pgfusepath{fill}%
\end{pgfscope}%
\begin{pgfscope}%
\pgfpathrectangle{\pgfqpoint{0.697024in}{0.857143in}}{\pgfqpoint{2.627103in}{1.813434in}}%
\pgfusepath{clip}%
\pgfsetbuttcap%
\pgfsetmiterjoin%
\definecolor{currentfill}{rgb}{0.066899,0.263188,0.377594}%
\pgfsetfillcolor{currentfill}%
\pgfsetlinewidth{0.000000pt}%
\definecolor{currentstroke}{rgb}{0.000000,0.000000,0.000000}%
\pgfsetstrokecolor{currentstroke}%
\pgfsetstrokeopacity{0.000000}%
\pgfsetdash{}{0pt}%
\pgfpathmoveto{\pgfqpoint{2.994706in}{1.847071in}}%
\pgfpathlineto{\pgfqpoint{3.003643in}{1.847071in}}%
\pgfpathlineto{\pgfqpoint{3.003643in}{1.851047in}}%
\pgfpathlineto{\pgfqpoint{2.994706in}{1.851047in}}%
\pgfpathlineto{\pgfqpoint{2.994706in}{1.847071in}}%
\pgfpathclose%
\pgfusepath{fill}%
\end{pgfscope}%
\begin{pgfscope}%
\pgfpathrectangle{\pgfqpoint{0.697024in}{0.857143in}}{\pgfqpoint{2.627103in}{1.813434in}}%
\pgfusepath{clip}%
\pgfsetbuttcap%
\pgfsetmiterjoin%
\definecolor{currentfill}{rgb}{0.066899,0.263188,0.377594}%
\pgfsetfillcolor{currentfill}%
\pgfsetlinewidth{0.000000pt}%
\definecolor{currentstroke}{rgb}{0.000000,0.000000,0.000000}%
\pgfsetstrokecolor{currentstroke}%
\pgfsetstrokeopacity{0.000000}%
\pgfsetdash{}{0pt}%
\pgfpathmoveto{\pgfqpoint{3.005877in}{1.847071in}}%
\pgfpathlineto{\pgfqpoint{3.014813in}{1.847071in}}%
\pgfpathlineto{\pgfqpoint{3.014813in}{1.847595in}}%
\pgfpathlineto{\pgfqpoint{3.005877in}{1.847595in}}%
\pgfpathlineto{\pgfqpoint{3.005877in}{1.847071in}}%
\pgfpathclose%
\pgfusepath{fill}%
\end{pgfscope}%
\begin{pgfscope}%
\pgfpathrectangle{\pgfqpoint{0.697024in}{0.857143in}}{\pgfqpoint{2.627103in}{1.813434in}}%
\pgfusepath{clip}%
\pgfsetbuttcap%
\pgfsetmiterjoin%
\definecolor{currentfill}{rgb}{0.066899,0.263188,0.377594}%
\pgfsetfillcolor{currentfill}%
\pgfsetlinewidth{0.000000pt}%
\definecolor{currentstroke}{rgb}{0.000000,0.000000,0.000000}%
\pgfsetstrokecolor{currentstroke}%
\pgfsetstrokeopacity{0.000000}%
\pgfsetdash{}{0pt}%
\pgfpathmoveto{\pgfqpoint{3.017047in}{1.847071in}}%
\pgfpathlineto{\pgfqpoint{3.025984in}{1.847071in}}%
\pgfpathlineto{\pgfqpoint{3.025984in}{1.820073in}}%
\pgfpathlineto{\pgfqpoint{3.017047in}{1.820073in}}%
\pgfpathlineto{\pgfqpoint{3.017047in}{1.847071in}}%
\pgfpathclose%
\pgfusepath{fill}%
\end{pgfscope}%
\begin{pgfscope}%
\pgfpathrectangle{\pgfqpoint{0.697024in}{0.857143in}}{\pgfqpoint{2.627103in}{1.813434in}}%
\pgfusepath{clip}%
\pgfsetbuttcap%
\pgfsetmiterjoin%
\definecolor{currentfill}{rgb}{0.066899,0.263188,0.377594}%
\pgfsetfillcolor{currentfill}%
\pgfsetlinewidth{0.000000pt}%
\definecolor{currentstroke}{rgb}{0.000000,0.000000,0.000000}%
\pgfsetstrokecolor{currentstroke}%
\pgfsetstrokeopacity{0.000000}%
\pgfsetdash{}{0pt}%
\pgfpathmoveto{\pgfqpoint{3.028218in}{1.847071in}}%
\pgfpathlineto{\pgfqpoint{3.037155in}{1.847071in}}%
\pgfpathlineto{\pgfqpoint{3.037155in}{1.821362in}}%
\pgfpathlineto{\pgfqpoint{3.028218in}{1.821362in}}%
\pgfpathlineto{\pgfqpoint{3.028218in}{1.847071in}}%
\pgfpathclose%
\pgfusepath{fill}%
\end{pgfscope}%
\begin{pgfscope}%
\pgfpathrectangle{\pgfqpoint{0.697024in}{0.857143in}}{\pgfqpoint{2.627103in}{1.813434in}}%
\pgfusepath{clip}%
\pgfsetbuttcap%
\pgfsetmiterjoin%
\definecolor{currentfill}{rgb}{0.066899,0.263188,0.377594}%
\pgfsetfillcolor{currentfill}%
\pgfsetlinewidth{0.000000pt}%
\definecolor{currentstroke}{rgb}{0.000000,0.000000,0.000000}%
\pgfsetstrokecolor{currentstroke}%
\pgfsetstrokeopacity{0.000000}%
\pgfsetdash{}{0pt}%
\pgfpathmoveto{\pgfqpoint{3.039389in}{1.847071in}}%
\pgfpathlineto{\pgfqpoint{3.048325in}{1.847071in}}%
\pgfpathlineto{\pgfqpoint{3.048325in}{1.812546in}}%
\pgfpathlineto{\pgfqpoint{3.039389in}{1.812546in}}%
\pgfpathlineto{\pgfqpoint{3.039389in}{1.847071in}}%
\pgfpathclose%
\pgfusepath{fill}%
\end{pgfscope}%
\begin{pgfscope}%
\pgfpathrectangle{\pgfqpoint{0.697024in}{0.857143in}}{\pgfqpoint{2.627103in}{1.813434in}}%
\pgfusepath{clip}%
\pgfsetbuttcap%
\pgfsetmiterjoin%
\definecolor{currentfill}{rgb}{0.066899,0.263188,0.377594}%
\pgfsetfillcolor{currentfill}%
\pgfsetlinewidth{0.000000pt}%
\definecolor{currentstroke}{rgb}{0.000000,0.000000,0.000000}%
\pgfsetstrokecolor{currentstroke}%
\pgfsetstrokeopacity{0.000000}%
\pgfsetdash{}{0pt}%
\pgfpathmoveto{\pgfqpoint{3.050559in}{1.847071in}}%
\pgfpathlineto{\pgfqpoint{3.059496in}{1.847071in}}%
\pgfpathlineto{\pgfqpoint{3.059496in}{1.809118in}}%
\pgfpathlineto{\pgfqpoint{3.050559in}{1.809118in}}%
\pgfpathlineto{\pgfqpoint{3.050559in}{1.847071in}}%
\pgfpathclose%
\pgfusepath{fill}%
\end{pgfscope}%
\begin{pgfscope}%
\pgfpathrectangle{\pgfqpoint{0.697024in}{0.857143in}}{\pgfqpoint{2.627103in}{1.813434in}}%
\pgfusepath{clip}%
\pgfsetbuttcap%
\pgfsetmiterjoin%
\definecolor{currentfill}{rgb}{0.066899,0.263188,0.377594}%
\pgfsetfillcolor{currentfill}%
\pgfsetlinewidth{0.000000pt}%
\definecolor{currentstroke}{rgb}{0.000000,0.000000,0.000000}%
\pgfsetstrokecolor{currentstroke}%
\pgfsetstrokeopacity{0.000000}%
\pgfsetdash{}{0pt}%
\pgfpathmoveto{\pgfqpoint{3.061730in}{1.847071in}}%
\pgfpathlineto{\pgfqpoint{3.070666in}{1.847071in}}%
\pgfpathlineto{\pgfqpoint{3.070666in}{1.811954in}}%
\pgfpathlineto{\pgfqpoint{3.061730in}{1.811954in}}%
\pgfpathlineto{\pgfqpoint{3.061730in}{1.847071in}}%
\pgfpathclose%
\pgfusepath{fill}%
\end{pgfscope}%
\begin{pgfscope}%
\pgfpathrectangle{\pgfqpoint{0.697024in}{0.857143in}}{\pgfqpoint{2.627103in}{1.813434in}}%
\pgfusepath{clip}%
\pgfsetbuttcap%
\pgfsetmiterjoin%
\definecolor{currentfill}{rgb}{0.066899,0.263188,0.377594}%
\pgfsetfillcolor{currentfill}%
\pgfsetlinewidth{0.000000pt}%
\definecolor{currentstroke}{rgb}{0.000000,0.000000,0.000000}%
\pgfsetstrokecolor{currentstroke}%
\pgfsetstrokeopacity{0.000000}%
\pgfsetdash{}{0pt}%
\pgfpathmoveto{\pgfqpoint{3.072900in}{1.847071in}}%
\pgfpathlineto{\pgfqpoint{3.081837in}{1.847071in}}%
\pgfpathlineto{\pgfqpoint{3.081837in}{1.804264in}}%
\pgfpathlineto{\pgfqpoint{3.072900in}{1.804264in}}%
\pgfpathlineto{\pgfqpoint{3.072900in}{1.847071in}}%
\pgfpathclose%
\pgfusepath{fill}%
\end{pgfscope}%
\begin{pgfscope}%
\pgfpathrectangle{\pgfqpoint{0.697024in}{0.857143in}}{\pgfqpoint{2.627103in}{1.813434in}}%
\pgfusepath{clip}%
\pgfsetbuttcap%
\pgfsetmiterjoin%
\definecolor{currentfill}{rgb}{0.066899,0.263188,0.377594}%
\pgfsetfillcolor{currentfill}%
\pgfsetlinewidth{0.000000pt}%
\definecolor{currentstroke}{rgb}{0.000000,0.000000,0.000000}%
\pgfsetstrokecolor{currentstroke}%
\pgfsetstrokeopacity{0.000000}%
\pgfsetdash{}{0pt}%
\pgfpathmoveto{\pgfqpoint{3.084071in}{1.847071in}}%
\pgfpathlineto{\pgfqpoint{3.093008in}{1.847071in}}%
\pgfpathlineto{\pgfqpoint{3.093008in}{1.790921in}}%
\pgfpathlineto{\pgfqpoint{3.084071in}{1.790921in}}%
\pgfpathlineto{\pgfqpoint{3.084071in}{1.847071in}}%
\pgfpathclose%
\pgfusepath{fill}%
\end{pgfscope}%
\begin{pgfscope}%
\pgfpathrectangle{\pgfqpoint{0.697024in}{0.857143in}}{\pgfqpoint{2.627103in}{1.813434in}}%
\pgfusepath{clip}%
\pgfsetbuttcap%
\pgfsetmiterjoin%
\definecolor{currentfill}{rgb}{0.066899,0.263188,0.377594}%
\pgfsetfillcolor{currentfill}%
\pgfsetlinewidth{0.000000pt}%
\definecolor{currentstroke}{rgb}{0.000000,0.000000,0.000000}%
\pgfsetstrokecolor{currentstroke}%
\pgfsetstrokeopacity{0.000000}%
\pgfsetdash{}{0pt}%
\pgfpathmoveto{\pgfqpoint{3.095242in}{1.847071in}}%
\pgfpathlineto{\pgfqpoint{3.104178in}{1.847071in}}%
\pgfpathlineto{\pgfqpoint{3.104178in}{1.797072in}}%
\pgfpathlineto{\pgfqpoint{3.095242in}{1.797072in}}%
\pgfpathlineto{\pgfqpoint{3.095242in}{1.847071in}}%
\pgfpathclose%
\pgfusepath{fill}%
\end{pgfscope}%
\begin{pgfscope}%
\pgfpathrectangle{\pgfqpoint{0.697024in}{0.857143in}}{\pgfqpoint{2.627103in}{1.813434in}}%
\pgfusepath{clip}%
\pgfsetbuttcap%
\pgfsetmiterjoin%
\definecolor{currentfill}{rgb}{0.066899,0.263188,0.377594}%
\pgfsetfillcolor{currentfill}%
\pgfsetlinewidth{0.000000pt}%
\definecolor{currentstroke}{rgb}{0.000000,0.000000,0.000000}%
\pgfsetstrokecolor{currentstroke}%
\pgfsetstrokeopacity{0.000000}%
\pgfsetdash{}{0pt}%
\pgfpathmoveto{\pgfqpoint{3.106412in}{1.847071in}}%
\pgfpathlineto{\pgfqpoint{3.115349in}{1.847071in}}%
\pgfpathlineto{\pgfqpoint{3.115349in}{1.799896in}}%
\pgfpathlineto{\pgfqpoint{3.106412in}{1.799896in}}%
\pgfpathlineto{\pgfqpoint{3.106412in}{1.847071in}}%
\pgfpathclose%
\pgfusepath{fill}%
\end{pgfscope}%
\begin{pgfscope}%
\pgfpathrectangle{\pgfqpoint{0.697024in}{0.857143in}}{\pgfqpoint{2.627103in}{1.813434in}}%
\pgfusepath{clip}%
\pgfsetbuttcap%
\pgfsetmiterjoin%
\definecolor{currentfill}{rgb}{0.066899,0.263188,0.377594}%
\pgfsetfillcolor{currentfill}%
\pgfsetlinewidth{0.000000pt}%
\definecolor{currentstroke}{rgb}{0.000000,0.000000,0.000000}%
\pgfsetstrokecolor{currentstroke}%
\pgfsetstrokeopacity{0.000000}%
\pgfsetdash{}{0pt}%
\pgfpathmoveto{\pgfqpoint{3.117583in}{1.847071in}}%
\pgfpathlineto{\pgfqpoint{3.126519in}{1.847071in}}%
\pgfpathlineto{\pgfqpoint{3.126519in}{1.795845in}}%
\pgfpathlineto{\pgfqpoint{3.117583in}{1.795845in}}%
\pgfpathlineto{\pgfqpoint{3.117583in}{1.847071in}}%
\pgfpathclose%
\pgfusepath{fill}%
\end{pgfscope}%
\begin{pgfscope}%
\pgfpathrectangle{\pgfqpoint{0.697024in}{0.857143in}}{\pgfqpoint{2.627103in}{1.813434in}}%
\pgfusepath{clip}%
\pgfsetbuttcap%
\pgfsetmiterjoin%
\definecolor{currentfill}{rgb}{0.066899,0.263188,0.377594}%
\pgfsetfillcolor{currentfill}%
\pgfsetlinewidth{0.000000pt}%
\definecolor{currentstroke}{rgb}{0.000000,0.000000,0.000000}%
\pgfsetstrokecolor{currentstroke}%
\pgfsetstrokeopacity{0.000000}%
\pgfsetdash{}{0pt}%
\pgfpathmoveto{\pgfqpoint{3.128753in}{1.847071in}}%
\pgfpathlineto{\pgfqpoint{3.137690in}{1.847071in}}%
\pgfpathlineto{\pgfqpoint{3.137690in}{1.784619in}}%
\pgfpathlineto{\pgfqpoint{3.128753in}{1.784619in}}%
\pgfpathlineto{\pgfqpoint{3.128753in}{1.847071in}}%
\pgfpathclose%
\pgfusepath{fill}%
\end{pgfscope}%
\begin{pgfscope}%
\pgfpathrectangle{\pgfqpoint{0.697024in}{0.857143in}}{\pgfqpoint{2.627103in}{1.813434in}}%
\pgfusepath{clip}%
\pgfsetbuttcap%
\pgfsetmiterjoin%
\definecolor{currentfill}{rgb}{0.066899,0.263188,0.377594}%
\pgfsetfillcolor{currentfill}%
\pgfsetlinewidth{0.000000pt}%
\definecolor{currentstroke}{rgb}{0.000000,0.000000,0.000000}%
\pgfsetstrokecolor{currentstroke}%
\pgfsetstrokeopacity{0.000000}%
\pgfsetdash{}{0pt}%
\pgfpathmoveto{\pgfqpoint{3.139924in}{1.847071in}}%
\pgfpathlineto{\pgfqpoint{3.148861in}{1.847071in}}%
\pgfpathlineto{\pgfqpoint{3.148861in}{1.779588in}}%
\pgfpathlineto{\pgfqpoint{3.139924in}{1.779588in}}%
\pgfpathlineto{\pgfqpoint{3.139924in}{1.847071in}}%
\pgfpathclose%
\pgfusepath{fill}%
\end{pgfscope}%
\begin{pgfscope}%
\pgfpathrectangle{\pgfqpoint{0.697024in}{0.857143in}}{\pgfqpoint{2.627103in}{1.813434in}}%
\pgfusepath{clip}%
\pgfsetbuttcap%
\pgfsetmiterjoin%
\definecolor{currentfill}{rgb}{0.066899,0.263188,0.377594}%
\pgfsetfillcolor{currentfill}%
\pgfsetlinewidth{0.000000pt}%
\definecolor{currentstroke}{rgb}{0.000000,0.000000,0.000000}%
\pgfsetstrokecolor{currentstroke}%
\pgfsetstrokeopacity{0.000000}%
\pgfsetdash{}{0pt}%
\pgfpathmoveto{\pgfqpoint{3.151095in}{1.847071in}}%
\pgfpathlineto{\pgfqpoint{3.160031in}{1.847071in}}%
\pgfpathlineto{\pgfqpoint{3.160031in}{1.763640in}}%
\pgfpathlineto{\pgfqpoint{3.151095in}{1.763640in}}%
\pgfpathlineto{\pgfqpoint{3.151095in}{1.847071in}}%
\pgfpathclose%
\pgfusepath{fill}%
\end{pgfscope}%
\begin{pgfscope}%
\pgfpathrectangle{\pgfqpoint{0.697024in}{0.857143in}}{\pgfqpoint{2.627103in}{1.813434in}}%
\pgfusepath{clip}%
\pgfsetbuttcap%
\pgfsetmiterjoin%
\definecolor{currentfill}{rgb}{0.066899,0.263188,0.377594}%
\pgfsetfillcolor{currentfill}%
\pgfsetlinewidth{0.000000pt}%
\definecolor{currentstroke}{rgb}{0.000000,0.000000,0.000000}%
\pgfsetstrokecolor{currentstroke}%
\pgfsetstrokeopacity{0.000000}%
\pgfsetdash{}{0pt}%
\pgfpathmoveto{\pgfqpoint{3.162265in}{1.847071in}}%
\pgfpathlineto{\pgfqpoint{3.171202in}{1.847071in}}%
\pgfpathlineto{\pgfqpoint{3.171202in}{1.772042in}}%
\pgfpathlineto{\pgfqpoint{3.162265in}{1.772042in}}%
\pgfpathlineto{\pgfqpoint{3.162265in}{1.847071in}}%
\pgfpathclose%
\pgfusepath{fill}%
\end{pgfscope}%
\begin{pgfscope}%
\pgfpathrectangle{\pgfqpoint{0.697024in}{0.857143in}}{\pgfqpoint{2.627103in}{1.813434in}}%
\pgfusepath{clip}%
\pgfsetbuttcap%
\pgfsetmiterjoin%
\definecolor{currentfill}{rgb}{0.066899,0.263188,0.377594}%
\pgfsetfillcolor{currentfill}%
\pgfsetlinewidth{0.000000pt}%
\definecolor{currentstroke}{rgb}{0.000000,0.000000,0.000000}%
\pgfsetstrokecolor{currentstroke}%
\pgfsetstrokeopacity{0.000000}%
\pgfsetdash{}{0pt}%
\pgfpathmoveto{\pgfqpoint{3.173436in}{1.847071in}}%
\pgfpathlineto{\pgfqpoint{3.182372in}{1.847071in}}%
\pgfpathlineto{\pgfqpoint{3.182372in}{1.775087in}}%
\pgfpathlineto{\pgfqpoint{3.173436in}{1.775087in}}%
\pgfpathlineto{\pgfqpoint{3.173436in}{1.847071in}}%
\pgfpathclose%
\pgfusepath{fill}%
\end{pgfscope}%
\begin{pgfscope}%
\pgfpathrectangle{\pgfqpoint{0.697024in}{0.857143in}}{\pgfqpoint{2.627103in}{1.813434in}}%
\pgfusepath{clip}%
\pgfsetbuttcap%
\pgfsetmiterjoin%
\definecolor{currentfill}{rgb}{0.066899,0.263188,0.377594}%
\pgfsetfillcolor{currentfill}%
\pgfsetlinewidth{0.000000pt}%
\definecolor{currentstroke}{rgb}{0.000000,0.000000,0.000000}%
\pgfsetstrokecolor{currentstroke}%
\pgfsetstrokeopacity{0.000000}%
\pgfsetdash{}{0pt}%
\pgfpathmoveto{\pgfqpoint{3.184607in}{1.847071in}}%
\pgfpathlineto{\pgfqpoint{3.193543in}{1.847071in}}%
\pgfpathlineto{\pgfqpoint{3.193543in}{1.789171in}}%
\pgfpathlineto{\pgfqpoint{3.184607in}{1.789171in}}%
\pgfpathlineto{\pgfqpoint{3.184607in}{1.847071in}}%
\pgfpathclose%
\pgfusepath{fill}%
\end{pgfscope}%
\begin{pgfscope}%
\pgfpathrectangle{\pgfqpoint{0.697024in}{0.857143in}}{\pgfqpoint{2.627103in}{1.813434in}}%
\pgfusepath{clip}%
\pgfsetbuttcap%
\pgfsetmiterjoin%
\definecolor{currentfill}{rgb}{0.066899,0.263188,0.377594}%
\pgfsetfillcolor{currentfill}%
\pgfsetlinewidth{0.000000pt}%
\definecolor{currentstroke}{rgb}{0.000000,0.000000,0.000000}%
\pgfsetstrokecolor{currentstroke}%
\pgfsetstrokeopacity{0.000000}%
\pgfsetdash{}{0pt}%
\pgfpathmoveto{\pgfqpoint{3.195777in}{1.847071in}}%
\pgfpathlineto{\pgfqpoint{3.204714in}{1.847071in}}%
\pgfpathlineto{\pgfqpoint{3.204714in}{1.802627in}}%
\pgfpathlineto{\pgfqpoint{3.195777in}{1.802627in}}%
\pgfpathlineto{\pgfqpoint{3.195777in}{1.847071in}}%
\pgfpathclose%
\pgfusepath{fill}%
\end{pgfscope}%
\begin{pgfscope}%
\pgfpathrectangle{\pgfqpoint{0.697024in}{0.857143in}}{\pgfqpoint{2.627103in}{1.813434in}}%
\pgfusepath{clip}%
\pgfsetbuttcap%
\pgfsetmiterjoin%
\definecolor{currentfill}{rgb}{0.133298,0.375282,0.379395}%
\pgfsetfillcolor{currentfill}%
\pgfsetlinewidth{0.000000pt}%
\definecolor{currentstroke}{rgb}{0.000000,0.000000,0.000000}%
\pgfsetstrokecolor{currentstroke}%
\pgfsetstrokeopacity{0.000000}%
\pgfsetdash{}{0pt}%
\pgfpathmoveto{\pgfqpoint{0.816438in}{1.847071in}}%
\pgfpathlineto{\pgfqpoint{0.825375in}{1.847071in}}%
\pgfpathlineto{\pgfqpoint{0.825375in}{1.878420in}}%
\pgfpathlineto{\pgfqpoint{0.816438in}{1.878420in}}%
\pgfpathlineto{\pgfqpoint{0.816438in}{1.847071in}}%
\pgfpathclose%
\pgfusepath{fill}%
\end{pgfscope}%
\begin{pgfscope}%
\pgfpathrectangle{\pgfqpoint{0.697024in}{0.857143in}}{\pgfqpoint{2.627103in}{1.813434in}}%
\pgfusepath{clip}%
\pgfsetbuttcap%
\pgfsetmiterjoin%
\definecolor{currentfill}{rgb}{0.133298,0.375282,0.379395}%
\pgfsetfillcolor{currentfill}%
\pgfsetlinewidth{0.000000pt}%
\definecolor{currentstroke}{rgb}{0.000000,0.000000,0.000000}%
\pgfsetstrokecolor{currentstroke}%
\pgfsetstrokeopacity{0.000000}%
\pgfsetdash{}{0pt}%
\pgfpathmoveto{\pgfqpoint{0.827609in}{1.855665in}}%
\pgfpathlineto{\pgfqpoint{0.836545in}{1.855665in}}%
\pgfpathlineto{\pgfqpoint{0.836545in}{1.922926in}}%
\pgfpathlineto{\pgfqpoint{0.827609in}{1.922926in}}%
\pgfpathlineto{\pgfqpoint{0.827609in}{1.855665in}}%
\pgfpathclose%
\pgfusepath{fill}%
\end{pgfscope}%
\begin{pgfscope}%
\pgfpathrectangle{\pgfqpoint{0.697024in}{0.857143in}}{\pgfqpoint{2.627103in}{1.813434in}}%
\pgfusepath{clip}%
\pgfsetbuttcap%
\pgfsetmiterjoin%
\definecolor{currentfill}{rgb}{0.133298,0.375282,0.379395}%
\pgfsetfillcolor{currentfill}%
\pgfsetlinewidth{0.000000pt}%
\definecolor{currentstroke}{rgb}{0.000000,0.000000,0.000000}%
\pgfsetstrokecolor{currentstroke}%
\pgfsetstrokeopacity{0.000000}%
\pgfsetdash{}{0pt}%
\pgfpathmoveto{\pgfqpoint{0.838779in}{1.871273in}}%
\pgfpathlineto{\pgfqpoint{0.847716in}{1.871273in}}%
\pgfpathlineto{\pgfqpoint{0.847716in}{2.053228in}}%
\pgfpathlineto{\pgfqpoint{0.838779in}{2.053228in}}%
\pgfpathlineto{\pgfqpoint{0.838779in}{1.871273in}}%
\pgfpathclose%
\pgfusepath{fill}%
\end{pgfscope}%
\begin{pgfscope}%
\pgfpathrectangle{\pgfqpoint{0.697024in}{0.857143in}}{\pgfqpoint{2.627103in}{1.813434in}}%
\pgfusepath{clip}%
\pgfsetbuttcap%
\pgfsetmiterjoin%
\definecolor{currentfill}{rgb}{0.133298,0.375282,0.379395}%
\pgfsetfillcolor{currentfill}%
\pgfsetlinewidth{0.000000pt}%
\definecolor{currentstroke}{rgb}{0.000000,0.000000,0.000000}%
\pgfsetstrokecolor{currentstroke}%
\pgfsetstrokeopacity{0.000000}%
\pgfsetdash{}{0pt}%
\pgfpathmoveto{\pgfqpoint{0.849950in}{1.876187in}}%
\pgfpathlineto{\pgfqpoint{0.858886in}{1.876187in}}%
\pgfpathlineto{\pgfqpoint{0.858886in}{2.008949in}}%
\pgfpathlineto{\pgfqpoint{0.849950in}{2.008949in}}%
\pgfpathlineto{\pgfqpoint{0.849950in}{1.876187in}}%
\pgfpathclose%
\pgfusepath{fill}%
\end{pgfscope}%
\begin{pgfscope}%
\pgfpathrectangle{\pgfqpoint{0.697024in}{0.857143in}}{\pgfqpoint{2.627103in}{1.813434in}}%
\pgfusepath{clip}%
\pgfsetbuttcap%
\pgfsetmiterjoin%
\definecolor{currentfill}{rgb}{0.133298,0.375282,0.379395}%
\pgfsetfillcolor{currentfill}%
\pgfsetlinewidth{0.000000pt}%
\definecolor{currentstroke}{rgb}{0.000000,0.000000,0.000000}%
\pgfsetstrokecolor{currentstroke}%
\pgfsetstrokeopacity{0.000000}%
\pgfsetdash{}{0pt}%
\pgfpathmoveto{\pgfqpoint{0.861121in}{1.880253in}}%
\pgfpathlineto{\pgfqpoint{0.870057in}{1.880253in}}%
\pgfpathlineto{\pgfqpoint{0.870057in}{1.918433in}}%
\pgfpathlineto{\pgfqpoint{0.861121in}{1.918433in}}%
\pgfpathlineto{\pgfqpoint{0.861121in}{1.880253in}}%
\pgfpathclose%
\pgfusepath{fill}%
\end{pgfscope}%
\begin{pgfscope}%
\pgfpathrectangle{\pgfqpoint{0.697024in}{0.857143in}}{\pgfqpoint{2.627103in}{1.813434in}}%
\pgfusepath{clip}%
\pgfsetbuttcap%
\pgfsetmiterjoin%
\definecolor{currentfill}{rgb}{0.133298,0.375282,0.379395}%
\pgfsetfillcolor{currentfill}%
\pgfsetlinewidth{0.000000pt}%
\definecolor{currentstroke}{rgb}{0.000000,0.000000,0.000000}%
\pgfsetstrokecolor{currentstroke}%
\pgfsetstrokeopacity{0.000000}%
\pgfsetdash{}{0pt}%
\pgfpathmoveto{\pgfqpoint{0.872291in}{1.847071in}}%
\pgfpathlineto{\pgfqpoint{0.881228in}{1.847071in}}%
\pgfpathlineto{\pgfqpoint{0.881228in}{1.838065in}}%
\pgfpathlineto{\pgfqpoint{0.872291in}{1.838065in}}%
\pgfpathlineto{\pgfqpoint{0.872291in}{1.847071in}}%
\pgfpathclose%
\pgfusepath{fill}%
\end{pgfscope}%
\begin{pgfscope}%
\pgfpathrectangle{\pgfqpoint{0.697024in}{0.857143in}}{\pgfqpoint{2.627103in}{1.813434in}}%
\pgfusepath{clip}%
\pgfsetbuttcap%
\pgfsetmiterjoin%
\definecolor{currentfill}{rgb}{0.133298,0.375282,0.379395}%
\pgfsetfillcolor{currentfill}%
\pgfsetlinewidth{0.000000pt}%
\definecolor{currentstroke}{rgb}{0.000000,0.000000,0.000000}%
\pgfsetstrokecolor{currentstroke}%
\pgfsetstrokeopacity{0.000000}%
\pgfsetdash{}{0pt}%
\pgfpathmoveto{\pgfqpoint{0.883462in}{1.918564in}}%
\pgfpathlineto{\pgfqpoint{0.892398in}{1.918564in}}%
\pgfpathlineto{\pgfqpoint{0.892398in}{1.987277in}}%
\pgfpathlineto{\pgfqpoint{0.883462in}{1.987277in}}%
\pgfpathlineto{\pgfqpoint{0.883462in}{1.918564in}}%
\pgfpathclose%
\pgfusepath{fill}%
\end{pgfscope}%
\begin{pgfscope}%
\pgfpathrectangle{\pgfqpoint{0.697024in}{0.857143in}}{\pgfqpoint{2.627103in}{1.813434in}}%
\pgfusepath{clip}%
\pgfsetbuttcap%
\pgfsetmiterjoin%
\definecolor{currentfill}{rgb}{0.133298,0.375282,0.379395}%
\pgfsetfillcolor{currentfill}%
\pgfsetlinewidth{0.000000pt}%
\definecolor{currentstroke}{rgb}{0.000000,0.000000,0.000000}%
\pgfsetstrokecolor{currentstroke}%
\pgfsetstrokeopacity{0.000000}%
\pgfsetdash{}{0pt}%
\pgfpathmoveto{\pgfqpoint{0.894632in}{1.922325in}}%
\pgfpathlineto{\pgfqpoint{0.903569in}{1.922325in}}%
\pgfpathlineto{\pgfqpoint{0.903569in}{2.073701in}}%
\pgfpathlineto{\pgfqpoint{0.894632in}{2.073701in}}%
\pgfpathlineto{\pgfqpoint{0.894632in}{1.922325in}}%
\pgfpathclose%
\pgfusepath{fill}%
\end{pgfscope}%
\begin{pgfscope}%
\pgfpathrectangle{\pgfqpoint{0.697024in}{0.857143in}}{\pgfqpoint{2.627103in}{1.813434in}}%
\pgfusepath{clip}%
\pgfsetbuttcap%
\pgfsetmiterjoin%
\definecolor{currentfill}{rgb}{0.133298,0.375282,0.379395}%
\pgfsetfillcolor{currentfill}%
\pgfsetlinewidth{0.000000pt}%
\definecolor{currentstroke}{rgb}{0.000000,0.000000,0.000000}%
\pgfsetstrokecolor{currentstroke}%
\pgfsetstrokeopacity{0.000000}%
\pgfsetdash{}{0pt}%
\pgfpathmoveto{\pgfqpoint{0.905803in}{1.909239in}}%
\pgfpathlineto{\pgfqpoint{0.914739in}{1.909239in}}%
\pgfpathlineto{\pgfqpoint{0.914739in}{2.060983in}}%
\pgfpathlineto{\pgfqpoint{0.905803in}{2.060983in}}%
\pgfpathlineto{\pgfqpoint{0.905803in}{1.909239in}}%
\pgfpathclose%
\pgfusepath{fill}%
\end{pgfscope}%
\begin{pgfscope}%
\pgfpathrectangle{\pgfqpoint{0.697024in}{0.857143in}}{\pgfqpoint{2.627103in}{1.813434in}}%
\pgfusepath{clip}%
\pgfsetbuttcap%
\pgfsetmiterjoin%
\definecolor{currentfill}{rgb}{0.133298,0.375282,0.379395}%
\pgfsetfillcolor{currentfill}%
\pgfsetlinewidth{0.000000pt}%
\definecolor{currentstroke}{rgb}{0.000000,0.000000,0.000000}%
\pgfsetstrokecolor{currentstroke}%
\pgfsetstrokeopacity{0.000000}%
\pgfsetdash{}{0pt}%
\pgfpathmoveto{\pgfqpoint{0.916974in}{1.890481in}}%
\pgfpathlineto{\pgfqpoint{0.925910in}{1.890481in}}%
\pgfpathlineto{\pgfqpoint{0.925910in}{1.993207in}}%
\pgfpathlineto{\pgfqpoint{0.916974in}{1.993207in}}%
\pgfpathlineto{\pgfqpoint{0.916974in}{1.890481in}}%
\pgfpathclose%
\pgfusepath{fill}%
\end{pgfscope}%
\begin{pgfscope}%
\pgfpathrectangle{\pgfqpoint{0.697024in}{0.857143in}}{\pgfqpoint{2.627103in}{1.813434in}}%
\pgfusepath{clip}%
\pgfsetbuttcap%
\pgfsetmiterjoin%
\definecolor{currentfill}{rgb}{0.133298,0.375282,0.379395}%
\pgfsetfillcolor{currentfill}%
\pgfsetlinewidth{0.000000pt}%
\definecolor{currentstroke}{rgb}{0.000000,0.000000,0.000000}%
\pgfsetstrokecolor{currentstroke}%
\pgfsetstrokeopacity{0.000000}%
\pgfsetdash{}{0pt}%
\pgfpathmoveto{\pgfqpoint{0.928144in}{1.899510in}}%
\pgfpathlineto{\pgfqpoint{0.937081in}{1.899510in}}%
\pgfpathlineto{\pgfqpoint{0.937081in}{2.084203in}}%
\pgfpathlineto{\pgfqpoint{0.928144in}{2.084203in}}%
\pgfpathlineto{\pgfqpoint{0.928144in}{1.899510in}}%
\pgfpathclose%
\pgfusepath{fill}%
\end{pgfscope}%
\begin{pgfscope}%
\pgfpathrectangle{\pgfqpoint{0.697024in}{0.857143in}}{\pgfqpoint{2.627103in}{1.813434in}}%
\pgfusepath{clip}%
\pgfsetbuttcap%
\pgfsetmiterjoin%
\definecolor{currentfill}{rgb}{0.133298,0.375282,0.379395}%
\pgfsetfillcolor{currentfill}%
\pgfsetlinewidth{0.000000pt}%
\definecolor{currentstroke}{rgb}{0.000000,0.000000,0.000000}%
\pgfsetstrokecolor{currentstroke}%
\pgfsetstrokeopacity{0.000000}%
\pgfsetdash{}{0pt}%
\pgfpathmoveto{\pgfqpoint{0.939315in}{1.869062in}}%
\pgfpathlineto{\pgfqpoint{0.948251in}{1.869062in}}%
\pgfpathlineto{\pgfqpoint{0.948251in}{2.121289in}}%
\pgfpathlineto{\pgfqpoint{0.939315in}{2.121289in}}%
\pgfpathlineto{\pgfqpoint{0.939315in}{1.869062in}}%
\pgfpathclose%
\pgfusepath{fill}%
\end{pgfscope}%
\begin{pgfscope}%
\pgfpathrectangle{\pgfqpoint{0.697024in}{0.857143in}}{\pgfqpoint{2.627103in}{1.813434in}}%
\pgfusepath{clip}%
\pgfsetbuttcap%
\pgfsetmiterjoin%
\definecolor{currentfill}{rgb}{0.133298,0.375282,0.379395}%
\pgfsetfillcolor{currentfill}%
\pgfsetlinewidth{0.000000pt}%
\definecolor{currentstroke}{rgb}{0.000000,0.000000,0.000000}%
\pgfsetstrokecolor{currentstroke}%
\pgfsetstrokeopacity{0.000000}%
\pgfsetdash{}{0pt}%
\pgfpathmoveto{\pgfqpoint{0.950485in}{1.865855in}}%
\pgfpathlineto{\pgfqpoint{0.959422in}{1.865855in}}%
\pgfpathlineto{\pgfqpoint{0.959422in}{2.017884in}}%
\pgfpathlineto{\pgfqpoint{0.950485in}{2.017884in}}%
\pgfpathlineto{\pgfqpoint{0.950485in}{1.865855in}}%
\pgfpathclose%
\pgfusepath{fill}%
\end{pgfscope}%
\begin{pgfscope}%
\pgfpathrectangle{\pgfqpoint{0.697024in}{0.857143in}}{\pgfqpoint{2.627103in}{1.813434in}}%
\pgfusepath{clip}%
\pgfsetbuttcap%
\pgfsetmiterjoin%
\definecolor{currentfill}{rgb}{0.133298,0.375282,0.379395}%
\pgfsetfillcolor{currentfill}%
\pgfsetlinewidth{0.000000pt}%
\definecolor{currentstroke}{rgb}{0.000000,0.000000,0.000000}%
\pgfsetstrokecolor{currentstroke}%
\pgfsetstrokeopacity{0.000000}%
\pgfsetdash{}{0pt}%
\pgfpathmoveto{\pgfqpoint{0.961656in}{1.850046in}}%
\pgfpathlineto{\pgfqpoint{0.970593in}{1.850046in}}%
\pgfpathlineto{\pgfqpoint{0.970593in}{2.121360in}}%
\pgfpathlineto{\pgfqpoint{0.961656in}{2.121360in}}%
\pgfpathlineto{\pgfqpoint{0.961656in}{1.850046in}}%
\pgfpathclose%
\pgfusepath{fill}%
\end{pgfscope}%
\begin{pgfscope}%
\pgfpathrectangle{\pgfqpoint{0.697024in}{0.857143in}}{\pgfqpoint{2.627103in}{1.813434in}}%
\pgfusepath{clip}%
\pgfsetbuttcap%
\pgfsetmiterjoin%
\definecolor{currentfill}{rgb}{0.133298,0.375282,0.379395}%
\pgfsetfillcolor{currentfill}%
\pgfsetlinewidth{0.000000pt}%
\definecolor{currentstroke}{rgb}{0.000000,0.000000,0.000000}%
\pgfsetstrokecolor{currentstroke}%
\pgfsetstrokeopacity{0.000000}%
\pgfsetdash{}{0pt}%
\pgfpathmoveto{\pgfqpoint{0.972827in}{1.848356in}}%
\pgfpathlineto{\pgfqpoint{0.981763in}{1.848356in}}%
\pgfpathlineto{\pgfqpoint{0.981763in}{1.991450in}}%
\pgfpathlineto{\pgfqpoint{0.972827in}{1.991450in}}%
\pgfpathlineto{\pgfqpoint{0.972827in}{1.848356in}}%
\pgfpathclose%
\pgfusepath{fill}%
\end{pgfscope}%
\begin{pgfscope}%
\pgfpathrectangle{\pgfqpoint{0.697024in}{0.857143in}}{\pgfqpoint{2.627103in}{1.813434in}}%
\pgfusepath{clip}%
\pgfsetbuttcap%
\pgfsetmiterjoin%
\definecolor{currentfill}{rgb}{0.133298,0.375282,0.379395}%
\pgfsetfillcolor{currentfill}%
\pgfsetlinewidth{0.000000pt}%
\definecolor{currentstroke}{rgb}{0.000000,0.000000,0.000000}%
\pgfsetstrokecolor{currentstroke}%
\pgfsetstrokeopacity{0.000000}%
\pgfsetdash{}{0pt}%
\pgfpathmoveto{\pgfqpoint{0.983997in}{1.881730in}}%
\pgfpathlineto{\pgfqpoint{0.992934in}{1.881730in}}%
\pgfpathlineto{\pgfqpoint{0.992934in}{2.038470in}}%
\pgfpathlineto{\pgfqpoint{0.983997in}{2.038470in}}%
\pgfpathlineto{\pgfqpoint{0.983997in}{1.881730in}}%
\pgfpathclose%
\pgfusepath{fill}%
\end{pgfscope}%
\begin{pgfscope}%
\pgfpathrectangle{\pgfqpoint{0.697024in}{0.857143in}}{\pgfqpoint{2.627103in}{1.813434in}}%
\pgfusepath{clip}%
\pgfsetbuttcap%
\pgfsetmiterjoin%
\definecolor{currentfill}{rgb}{0.133298,0.375282,0.379395}%
\pgfsetfillcolor{currentfill}%
\pgfsetlinewidth{0.000000pt}%
\definecolor{currentstroke}{rgb}{0.000000,0.000000,0.000000}%
\pgfsetstrokecolor{currentstroke}%
\pgfsetstrokeopacity{0.000000}%
\pgfsetdash{}{0pt}%
\pgfpathmoveto{\pgfqpoint{0.995168in}{1.908802in}}%
\pgfpathlineto{\pgfqpoint{1.004104in}{1.908802in}}%
\pgfpathlineto{\pgfqpoint{1.004104in}{2.175233in}}%
\pgfpathlineto{\pgfqpoint{0.995168in}{2.175233in}}%
\pgfpathlineto{\pgfqpoint{0.995168in}{1.908802in}}%
\pgfpathclose%
\pgfusepath{fill}%
\end{pgfscope}%
\begin{pgfscope}%
\pgfpathrectangle{\pgfqpoint{0.697024in}{0.857143in}}{\pgfqpoint{2.627103in}{1.813434in}}%
\pgfusepath{clip}%
\pgfsetbuttcap%
\pgfsetmiterjoin%
\definecolor{currentfill}{rgb}{0.133298,0.375282,0.379395}%
\pgfsetfillcolor{currentfill}%
\pgfsetlinewidth{0.000000pt}%
\definecolor{currentstroke}{rgb}{0.000000,0.000000,0.000000}%
\pgfsetstrokecolor{currentstroke}%
\pgfsetstrokeopacity{0.000000}%
\pgfsetdash{}{0pt}%
\pgfpathmoveto{\pgfqpoint{1.006338in}{1.882192in}}%
\pgfpathlineto{\pgfqpoint{1.015275in}{1.882192in}}%
\pgfpathlineto{\pgfqpoint{1.015275in}{1.979833in}}%
\pgfpathlineto{\pgfqpoint{1.006338in}{1.979833in}}%
\pgfpathlineto{\pgfqpoint{1.006338in}{1.882192in}}%
\pgfpathclose%
\pgfusepath{fill}%
\end{pgfscope}%
\begin{pgfscope}%
\pgfpathrectangle{\pgfqpoint{0.697024in}{0.857143in}}{\pgfqpoint{2.627103in}{1.813434in}}%
\pgfusepath{clip}%
\pgfsetbuttcap%
\pgfsetmiterjoin%
\definecolor{currentfill}{rgb}{0.133298,0.375282,0.379395}%
\pgfsetfillcolor{currentfill}%
\pgfsetlinewidth{0.000000pt}%
\definecolor{currentstroke}{rgb}{0.000000,0.000000,0.000000}%
\pgfsetstrokecolor{currentstroke}%
\pgfsetstrokeopacity{0.000000}%
\pgfsetdash{}{0pt}%
\pgfpathmoveto{\pgfqpoint{1.017509in}{1.937559in}}%
\pgfpathlineto{\pgfqpoint{1.026446in}{1.937559in}}%
\pgfpathlineto{\pgfqpoint{1.026446in}{2.069529in}}%
\pgfpathlineto{\pgfqpoint{1.017509in}{2.069529in}}%
\pgfpathlineto{\pgfqpoint{1.017509in}{1.937559in}}%
\pgfpathclose%
\pgfusepath{fill}%
\end{pgfscope}%
\begin{pgfscope}%
\pgfpathrectangle{\pgfqpoint{0.697024in}{0.857143in}}{\pgfqpoint{2.627103in}{1.813434in}}%
\pgfusepath{clip}%
\pgfsetbuttcap%
\pgfsetmiterjoin%
\definecolor{currentfill}{rgb}{0.133298,0.375282,0.379395}%
\pgfsetfillcolor{currentfill}%
\pgfsetlinewidth{0.000000pt}%
\definecolor{currentstroke}{rgb}{0.000000,0.000000,0.000000}%
\pgfsetstrokecolor{currentstroke}%
\pgfsetstrokeopacity{0.000000}%
\pgfsetdash{}{0pt}%
\pgfpathmoveto{\pgfqpoint{1.028680in}{1.925336in}}%
\pgfpathlineto{\pgfqpoint{1.037616in}{1.925336in}}%
\pgfpathlineto{\pgfqpoint{1.037616in}{2.043038in}}%
\pgfpathlineto{\pgfqpoint{1.028680in}{2.043038in}}%
\pgfpathlineto{\pgfqpoint{1.028680in}{1.925336in}}%
\pgfpathclose%
\pgfusepath{fill}%
\end{pgfscope}%
\begin{pgfscope}%
\pgfpathrectangle{\pgfqpoint{0.697024in}{0.857143in}}{\pgfqpoint{2.627103in}{1.813434in}}%
\pgfusepath{clip}%
\pgfsetbuttcap%
\pgfsetmiterjoin%
\definecolor{currentfill}{rgb}{0.133298,0.375282,0.379395}%
\pgfsetfillcolor{currentfill}%
\pgfsetlinewidth{0.000000pt}%
\definecolor{currentstroke}{rgb}{0.000000,0.000000,0.000000}%
\pgfsetstrokecolor{currentstroke}%
\pgfsetstrokeopacity{0.000000}%
\pgfsetdash{}{0pt}%
\pgfpathmoveto{\pgfqpoint{1.039850in}{1.936114in}}%
\pgfpathlineto{\pgfqpoint{1.048787in}{1.936114in}}%
\pgfpathlineto{\pgfqpoint{1.048787in}{2.125576in}}%
\pgfpathlineto{\pgfqpoint{1.039850in}{2.125576in}}%
\pgfpathlineto{\pgfqpoint{1.039850in}{1.936114in}}%
\pgfpathclose%
\pgfusepath{fill}%
\end{pgfscope}%
\begin{pgfscope}%
\pgfpathrectangle{\pgfqpoint{0.697024in}{0.857143in}}{\pgfqpoint{2.627103in}{1.813434in}}%
\pgfusepath{clip}%
\pgfsetbuttcap%
\pgfsetmiterjoin%
\definecolor{currentfill}{rgb}{0.133298,0.375282,0.379395}%
\pgfsetfillcolor{currentfill}%
\pgfsetlinewidth{0.000000pt}%
\definecolor{currentstroke}{rgb}{0.000000,0.000000,0.000000}%
\pgfsetstrokecolor{currentstroke}%
\pgfsetstrokeopacity{0.000000}%
\pgfsetdash{}{0pt}%
\pgfpathmoveto{\pgfqpoint{1.051021in}{1.897086in}}%
\pgfpathlineto{\pgfqpoint{1.059957in}{1.897086in}}%
\pgfpathlineto{\pgfqpoint{1.059957in}{2.189946in}}%
\pgfpathlineto{\pgfqpoint{1.051021in}{2.189946in}}%
\pgfpathlineto{\pgfqpoint{1.051021in}{1.897086in}}%
\pgfpathclose%
\pgfusepath{fill}%
\end{pgfscope}%
\begin{pgfscope}%
\pgfpathrectangle{\pgfqpoint{0.697024in}{0.857143in}}{\pgfqpoint{2.627103in}{1.813434in}}%
\pgfusepath{clip}%
\pgfsetbuttcap%
\pgfsetmiterjoin%
\definecolor{currentfill}{rgb}{0.133298,0.375282,0.379395}%
\pgfsetfillcolor{currentfill}%
\pgfsetlinewidth{0.000000pt}%
\definecolor{currentstroke}{rgb}{0.000000,0.000000,0.000000}%
\pgfsetstrokecolor{currentstroke}%
\pgfsetstrokeopacity{0.000000}%
\pgfsetdash{}{0pt}%
\pgfpathmoveto{\pgfqpoint{1.062191in}{1.909754in}}%
\pgfpathlineto{\pgfqpoint{1.071128in}{1.909754in}}%
\pgfpathlineto{\pgfqpoint{1.071128in}{2.255207in}}%
\pgfpathlineto{\pgfqpoint{1.062191in}{2.255207in}}%
\pgfpathlineto{\pgfqpoint{1.062191in}{1.909754in}}%
\pgfpathclose%
\pgfusepath{fill}%
\end{pgfscope}%
\begin{pgfscope}%
\pgfpathrectangle{\pgfqpoint{0.697024in}{0.857143in}}{\pgfqpoint{2.627103in}{1.813434in}}%
\pgfusepath{clip}%
\pgfsetbuttcap%
\pgfsetmiterjoin%
\definecolor{currentfill}{rgb}{0.133298,0.375282,0.379395}%
\pgfsetfillcolor{currentfill}%
\pgfsetlinewidth{0.000000pt}%
\definecolor{currentstroke}{rgb}{0.000000,0.000000,0.000000}%
\pgfsetstrokecolor{currentstroke}%
\pgfsetstrokeopacity{0.000000}%
\pgfsetdash{}{0pt}%
\pgfpathmoveto{\pgfqpoint{1.073362in}{1.943778in}}%
\pgfpathlineto{\pgfqpoint{1.082299in}{1.943778in}}%
\pgfpathlineto{\pgfqpoint{1.082299in}{2.439075in}}%
\pgfpathlineto{\pgfqpoint{1.073362in}{2.439075in}}%
\pgfpathlineto{\pgfqpoint{1.073362in}{1.943778in}}%
\pgfpathclose%
\pgfusepath{fill}%
\end{pgfscope}%
\begin{pgfscope}%
\pgfpathrectangle{\pgfqpoint{0.697024in}{0.857143in}}{\pgfqpoint{2.627103in}{1.813434in}}%
\pgfusepath{clip}%
\pgfsetbuttcap%
\pgfsetmiterjoin%
\definecolor{currentfill}{rgb}{0.133298,0.375282,0.379395}%
\pgfsetfillcolor{currentfill}%
\pgfsetlinewidth{0.000000pt}%
\definecolor{currentstroke}{rgb}{0.000000,0.000000,0.000000}%
\pgfsetstrokecolor{currentstroke}%
\pgfsetstrokeopacity{0.000000}%
\pgfsetdash{}{0pt}%
\pgfpathmoveto{\pgfqpoint{1.084533in}{1.939020in}}%
\pgfpathlineto{\pgfqpoint{1.093469in}{1.939020in}}%
\pgfpathlineto{\pgfqpoint{1.093469in}{2.303466in}}%
\pgfpathlineto{\pgfqpoint{1.084533in}{2.303466in}}%
\pgfpathlineto{\pgfqpoint{1.084533in}{1.939020in}}%
\pgfpathclose%
\pgfusepath{fill}%
\end{pgfscope}%
\begin{pgfscope}%
\pgfpathrectangle{\pgfqpoint{0.697024in}{0.857143in}}{\pgfqpoint{2.627103in}{1.813434in}}%
\pgfusepath{clip}%
\pgfsetbuttcap%
\pgfsetmiterjoin%
\definecolor{currentfill}{rgb}{0.133298,0.375282,0.379395}%
\pgfsetfillcolor{currentfill}%
\pgfsetlinewidth{0.000000pt}%
\definecolor{currentstroke}{rgb}{0.000000,0.000000,0.000000}%
\pgfsetstrokecolor{currentstroke}%
\pgfsetstrokeopacity{0.000000}%
\pgfsetdash{}{0pt}%
\pgfpathmoveto{\pgfqpoint{1.095703in}{1.944085in}}%
\pgfpathlineto{\pgfqpoint{1.104640in}{1.944085in}}%
\pgfpathlineto{\pgfqpoint{1.104640in}{2.309442in}}%
\pgfpathlineto{\pgfqpoint{1.095703in}{2.309442in}}%
\pgfpathlineto{\pgfqpoint{1.095703in}{1.944085in}}%
\pgfpathclose%
\pgfusepath{fill}%
\end{pgfscope}%
\begin{pgfscope}%
\pgfpathrectangle{\pgfqpoint{0.697024in}{0.857143in}}{\pgfqpoint{2.627103in}{1.813434in}}%
\pgfusepath{clip}%
\pgfsetbuttcap%
\pgfsetmiterjoin%
\definecolor{currentfill}{rgb}{0.133298,0.375282,0.379395}%
\pgfsetfillcolor{currentfill}%
\pgfsetlinewidth{0.000000pt}%
\definecolor{currentstroke}{rgb}{0.000000,0.000000,0.000000}%
\pgfsetstrokecolor{currentstroke}%
\pgfsetstrokeopacity{0.000000}%
\pgfsetdash{}{0pt}%
\pgfpathmoveto{\pgfqpoint{1.106874in}{1.959548in}}%
\pgfpathlineto{\pgfqpoint{1.115810in}{1.959548in}}%
\pgfpathlineto{\pgfqpoint{1.115810in}{2.371492in}}%
\pgfpathlineto{\pgfqpoint{1.106874in}{2.371492in}}%
\pgfpathlineto{\pgfqpoint{1.106874in}{1.959548in}}%
\pgfpathclose%
\pgfusepath{fill}%
\end{pgfscope}%
\begin{pgfscope}%
\pgfpathrectangle{\pgfqpoint{0.697024in}{0.857143in}}{\pgfqpoint{2.627103in}{1.813434in}}%
\pgfusepath{clip}%
\pgfsetbuttcap%
\pgfsetmiterjoin%
\definecolor{currentfill}{rgb}{0.133298,0.375282,0.379395}%
\pgfsetfillcolor{currentfill}%
\pgfsetlinewidth{0.000000pt}%
\definecolor{currentstroke}{rgb}{0.000000,0.000000,0.000000}%
\pgfsetstrokecolor{currentstroke}%
\pgfsetstrokeopacity{0.000000}%
\pgfsetdash{}{0pt}%
\pgfpathmoveto{\pgfqpoint{1.118045in}{1.946302in}}%
\pgfpathlineto{\pgfqpoint{1.126981in}{1.946302in}}%
\pgfpathlineto{\pgfqpoint{1.126981in}{2.301290in}}%
\pgfpathlineto{\pgfqpoint{1.118045in}{2.301290in}}%
\pgfpathlineto{\pgfqpoint{1.118045in}{1.946302in}}%
\pgfpathclose%
\pgfusepath{fill}%
\end{pgfscope}%
\begin{pgfscope}%
\pgfpathrectangle{\pgfqpoint{0.697024in}{0.857143in}}{\pgfqpoint{2.627103in}{1.813434in}}%
\pgfusepath{clip}%
\pgfsetbuttcap%
\pgfsetmiterjoin%
\definecolor{currentfill}{rgb}{0.133298,0.375282,0.379395}%
\pgfsetfillcolor{currentfill}%
\pgfsetlinewidth{0.000000pt}%
\definecolor{currentstroke}{rgb}{0.000000,0.000000,0.000000}%
\pgfsetstrokecolor{currentstroke}%
\pgfsetstrokeopacity{0.000000}%
\pgfsetdash{}{0pt}%
\pgfpathmoveto{\pgfqpoint{1.129215in}{1.898715in}}%
\pgfpathlineto{\pgfqpoint{1.138152in}{1.898715in}}%
\pgfpathlineto{\pgfqpoint{1.138152in}{2.190834in}}%
\pgfpathlineto{\pgfqpoint{1.129215in}{2.190834in}}%
\pgfpathlineto{\pgfqpoint{1.129215in}{1.898715in}}%
\pgfpathclose%
\pgfusepath{fill}%
\end{pgfscope}%
\begin{pgfscope}%
\pgfpathrectangle{\pgfqpoint{0.697024in}{0.857143in}}{\pgfqpoint{2.627103in}{1.813434in}}%
\pgfusepath{clip}%
\pgfsetbuttcap%
\pgfsetmiterjoin%
\definecolor{currentfill}{rgb}{0.133298,0.375282,0.379395}%
\pgfsetfillcolor{currentfill}%
\pgfsetlinewidth{0.000000pt}%
\definecolor{currentstroke}{rgb}{0.000000,0.000000,0.000000}%
\pgfsetstrokecolor{currentstroke}%
\pgfsetstrokeopacity{0.000000}%
\pgfsetdash{}{0pt}%
\pgfpathmoveto{\pgfqpoint{1.140386in}{1.906149in}}%
\pgfpathlineto{\pgfqpoint{1.149322in}{1.906149in}}%
\pgfpathlineto{\pgfqpoint{1.149322in}{2.147188in}}%
\pgfpathlineto{\pgfqpoint{1.140386in}{2.147188in}}%
\pgfpathlineto{\pgfqpoint{1.140386in}{1.906149in}}%
\pgfpathclose%
\pgfusepath{fill}%
\end{pgfscope}%
\begin{pgfscope}%
\pgfpathrectangle{\pgfqpoint{0.697024in}{0.857143in}}{\pgfqpoint{2.627103in}{1.813434in}}%
\pgfusepath{clip}%
\pgfsetbuttcap%
\pgfsetmiterjoin%
\definecolor{currentfill}{rgb}{0.133298,0.375282,0.379395}%
\pgfsetfillcolor{currentfill}%
\pgfsetlinewidth{0.000000pt}%
\definecolor{currentstroke}{rgb}{0.000000,0.000000,0.000000}%
\pgfsetstrokecolor{currentstroke}%
\pgfsetstrokeopacity{0.000000}%
\pgfsetdash{}{0pt}%
\pgfpathmoveto{\pgfqpoint{1.151556in}{1.882857in}}%
\pgfpathlineto{\pgfqpoint{1.160493in}{1.882857in}}%
\pgfpathlineto{\pgfqpoint{1.160493in}{2.117154in}}%
\pgfpathlineto{\pgfqpoint{1.151556in}{2.117154in}}%
\pgfpathlineto{\pgfqpoint{1.151556in}{1.882857in}}%
\pgfpathclose%
\pgfusepath{fill}%
\end{pgfscope}%
\begin{pgfscope}%
\pgfpathrectangle{\pgfqpoint{0.697024in}{0.857143in}}{\pgfqpoint{2.627103in}{1.813434in}}%
\pgfusepath{clip}%
\pgfsetbuttcap%
\pgfsetmiterjoin%
\definecolor{currentfill}{rgb}{0.133298,0.375282,0.379395}%
\pgfsetfillcolor{currentfill}%
\pgfsetlinewidth{0.000000pt}%
\definecolor{currentstroke}{rgb}{0.000000,0.000000,0.000000}%
\pgfsetstrokecolor{currentstroke}%
\pgfsetstrokeopacity{0.000000}%
\pgfsetdash{}{0pt}%
\pgfpathmoveto{\pgfqpoint{1.162727in}{1.873787in}}%
\pgfpathlineto{\pgfqpoint{1.171663in}{1.873787in}}%
\pgfpathlineto{\pgfqpoint{1.171663in}{1.944442in}}%
\pgfpathlineto{\pgfqpoint{1.162727in}{1.944442in}}%
\pgfpathlineto{\pgfqpoint{1.162727in}{1.873787in}}%
\pgfpathclose%
\pgfusepath{fill}%
\end{pgfscope}%
\begin{pgfscope}%
\pgfpathrectangle{\pgfqpoint{0.697024in}{0.857143in}}{\pgfqpoint{2.627103in}{1.813434in}}%
\pgfusepath{clip}%
\pgfsetbuttcap%
\pgfsetmiterjoin%
\definecolor{currentfill}{rgb}{0.133298,0.375282,0.379395}%
\pgfsetfillcolor{currentfill}%
\pgfsetlinewidth{0.000000pt}%
\definecolor{currentstroke}{rgb}{0.000000,0.000000,0.000000}%
\pgfsetstrokecolor{currentstroke}%
\pgfsetstrokeopacity{0.000000}%
\pgfsetdash{}{0pt}%
\pgfpathmoveto{\pgfqpoint{1.173898in}{1.847071in}}%
\pgfpathlineto{\pgfqpoint{1.182834in}{1.847071in}}%
\pgfpathlineto{\pgfqpoint{1.182834in}{1.905116in}}%
\pgfpathlineto{\pgfqpoint{1.173898in}{1.905116in}}%
\pgfpathlineto{\pgfqpoint{1.173898in}{1.847071in}}%
\pgfpathclose%
\pgfusepath{fill}%
\end{pgfscope}%
\begin{pgfscope}%
\pgfpathrectangle{\pgfqpoint{0.697024in}{0.857143in}}{\pgfqpoint{2.627103in}{1.813434in}}%
\pgfusepath{clip}%
\pgfsetbuttcap%
\pgfsetmiterjoin%
\definecolor{currentfill}{rgb}{0.133298,0.375282,0.379395}%
\pgfsetfillcolor{currentfill}%
\pgfsetlinewidth{0.000000pt}%
\definecolor{currentstroke}{rgb}{0.000000,0.000000,0.000000}%
\pgfsetstrokecolor{currentstroke}%
\pgfsetstrokeopacity{0.000000}%
\pgfsetdash{}{0pt}%
\pgfpathmoveto{\pgfqpoint{1.185068in}{1.847071in}}%
\pgfpathlineto{\pgfqpoint{1.194005in}{1.847071in}}%
\pgfpathlineto{\pgfqpoint{1.194005in}{1.805687in}}%
\pgfpathlineto{\pgfqpoint{1.185068in}{1.805687in}}%
\pgfpathlineto{\pgfqpoint{1.185068in}{1.847071in}}%
\pgfpathclose%
\pgfusepath{fill}%
\end{pgfscope}%
\begin{pgfscope}%
\pgfpathrectangle{\pgfqpoint{0.697024in}{0.857143in}}{\pgfqpoint{2.627103in}{1.813434in}}%
\pgfusepath{clip}%
\pgfsetbuttcap%
\pgfsetmiterjoin%
\definecolor{currentfill}{rgb}{0.133298,0.375282,0.379395}%
\pgfsetfillcolor{currentfill}%
\pgfsetlinewidth{0.000000pt}%
\definecolor{currentstroke}{rgb}{0.000000,0.000000,0.000000}%
\pgfsetstrokecolor{currentstroke}%
\pgfsetstrokeopacity{0.000000}%
\pgfsetdash{}{0pt}%
\pgfpathmoveto{\pgfqpoint{1.196239in}{1.847071in}}%
\pgfpathlineto{\pgfqpoint{1.205175in}{1.847071in}}%
\pgfpathlineto{\pgfqpoint{1.205175in}{1.682902in}}%
\pgfpathlineto{\pgfqpoint{1.196239in}{1.682902in}}%
\pgfpathlineto{\pgfqpoint{1.196239in}{1.847071in}}%
\pgfpathclose%
\pgfusepath{fill}%
\end{pgfscope}%
\begin{pgfscope}%
\pgfpathrectangle{\pgfqpoint{0.697024in}{0.857143in}}{\pgfqpoint{2.627103in}{1.813434in}}%
\pgfusepath{clip}%
\pgfsetbuttcap%
\pgfsetmiterjoin%
\definecolor{currentfill}{rgb}{0.133298,0.375282,0.379395}%
\pgfsetfillcolor{currentfill}%
\pgfsetlinewidth{0.000000pt}%
\definecolor{currentstroke}{rgb}{0.000000,0.000000,0.000000}%
\pgfsetstrokecolor{currentstroke}%
\pgfsetstrokeopacity{0.000000}%
\pgfsetdash{}{0pt}%
\pgfpathmoveto{\pgfqpoint{1.207409in}{1.847071in}}%
\pgfpathlineto{\pgfqpoint{1.216346in}{1.847071in}}%
\pgfpathlineto{\pgfqpoint{1.216346in}{1.778759in}}%
\pgfpathlineto{\pgfqpoint{1.207409in}{1.778759in}}%
\pgfpathlineto{\pgfqpoint{1.207409in}{1.847071in}}%
\pgfpathclose%
\pgfusepath{fill}%
\end{pgfscope}%
\begin{pgfscope}%
\pgfpathrectangle{\pgfqpoint{0.697024in}{0.857143in}}{\pgfqpoint{2.627103in}{1.813434in}}%
\pgfusepath{clip}%
\pgfsetbuttcap%
\pgfsetmiterjoin%
\definecolor{currentfill}{rgb}{0.133298,0.375282,0.379395}%
\pgfsetfillcolor{currentfill}%
\pgfsetlinewidth{0.000000pt}%
\definecolor{currentstroke}{rgb}{0.000000,0.000000,0.000000}%
\pgfsetstrokecolor{currentstroke}%
\pgfsetstrokeopacity{0.000000}%
\pgfsetdash{}{0pt}%
\pgfpathmoveto{\pgfqpoint{1.218580in}{1.847071in}}%
\pgfpathlineto{\pgfqpoint{1.227516in}{1.847071in}}%
\pgfpathlineto{\pgfqpoint{1.227516in}{1.694675in}}%
\pgfpathlineto{\pgfqpoint{1.218580in}{1.694675in}}%
\pgfpathlineto{\pgfqpoint{1.218580in}{1.847071in}}%
\pgfpathclose%
\pgfusepath{fill}%
\end{pgfscope}%
\begin{pgfscope}%
\pgfpathrectangle{\pgfqpoint{0.697024in}{0.857143in}}{\pgfqpoint{2.627103in}{1.813434in}}%
\pgfusepath{clip}%
\pgfsetbuttcap%
\pgfsetmiterjoin%
\definecolor{currentfill}{rgb}{0.133298,0.375282,0.379395}%
\pgfsetfillcolor{currentfill}%
\pgfsetlinewidth{0.000000pt}%
\definecolor{currentstroke}{rgb}{0.000000,0.000000,0.000000}%
\pgfsetstrokecolor{currentstroke}%
\pgfsetstrokeopacity{0.000000}%
\pgfsetdash{}{0pt}%
\pgfpathmoveto{\pgfqpoint{1.229751in}{1.847071in}}%
\pgfpathlineto{\pgfqpoint{1.238687in}{1.847071in}}%
\pgfpathlineto{\pgfqpoint{1.238687in}{1.739811in}}%
\pgfpathlineto{\pgfqpoint{1.229751in}{1.739811in}}%
\pgfpathlineto{\pgfqpoint{1.229751in}{1.847071in}}%
\pgfpathclose%
\pgfusepath{fill}%
\end{pgfscope}%
\begin{pgfscope}%
\pgfpathrectangle{\pgfqpoint{0.697024in}{0.857143in}}{\pgfqpoint{2.627103in}{1.813434in}}%
\pgfusepath{clip}%
\pgfsetbuttcap%
\pgfsetmiterjoin%
\definecolor{currentfill}{rgb}{0.133298,0.375282,0.379395}%
\pgfsetfillcolor{currentfill}%
\pgfsetlinewidth{0.000000pt}%
\definecolor{currentstroke}{rgb}{0.000000,0.000000,0.000000}%
\pgfsetstrokecolor{currentstroke}%
\pgfsetstrokeopacity{0.000000}%
\pgfsetdash{}{0pt}%
\pgfpathmoveto{\pgfqpoint{1.240921in}{1.937419in}}%
\pgfpathlineto{\pgfqpoint{1.249858in}{1.937419in}}%
\pgfpathlineto{\pgfqpoint{1.249858in}{1.988223in}}%
\pgfpathlineto{\pgfqpoint{1.240921in}{1.988223in}}%
\pgfpathlineto{\pgfqpoint{1.240921in}{1.937419in}}%
\pgfpathclose%
\pgfusepath{fill}%
\end{pgfscope}%
\begin{pgfscope}%
\pgfpathrectangle{\pgfqpoint{0.697024in}{0.857143in}}{\pgfqpoint{2.627103in}{1.813434in}}%
\pgfusepath{clip}%
\pgfsetbuttcap%
\pgfsetmiterjoin%
\definecolor{currentfill}{rgb}{0.133298,0.375282,0.379395}%
\pgfsetfillcolor{currentfill}%
\pgfsetlinewidth{0.000000pt}%
\definecolor{currentstroke}{rgb}{0.000000,0.000000,0.000000}%
\pgfsetstrokecolor{currentstroke}%
\pgfsetstrokeopacity{0.000000}%
\pgfsetdash{}{0pt}%
\pgfpathmoveto{\pgfqpoint{1.252092in}{1.945503in}}%
\pgfpathlineto{\pgfqpoint{1.261028in}{1.945503in}}%
\pgfpathlineto{\pgfqpoint{1.261028in}{2.017163in}}%
\pgfpathlineto{\pgfqpoint{1.252092in}{2.017163in}}%
\pgfpathlineto{\pgfqpoint{1.252092in}{1.945503in}}%
\pgfpathclose%
\pgfusepath{fill}%
\end{pgfscope}%
\begin{pgfscope}%
\pgfpathrectangle{\pgfqpoint{0.697024in}{0.857143in}}{\pgfqpoint{2.627103in}{1.813434in}}%
\pgfusepath{clip}%
\pgfsetbuttcap%
\pgfsetmiterjoin%
\definecolor{currentfill}{rgb}{0.133298,0.375282,0.379395}%
\pgfsetfillcolor{currentfill}%
\pgfsetlinewidth{0.000000pt}%
\definecolor{currentstroke}{rgb}{0.000000,0.000000,0.000000}%
\pgfsetstrokecolor{currentstroke}%
\pgfsetstrokeopacity{0.000000}%
\pgfsetdash{}{0pt}%
\pgfpathmoveto{\pgfqpoint{1.263262in}{1.936817in}}%
\pgfpathlineto{\pgfqpoint{1.272199in}{1.936817in}}%
\pgfpathlineto{\pgfqpoint{1.272199in}{1.966378in}}%
\pgfpathlineto{\pgfqpoint{1.263262in}{1.966378in}}%
\pgfpathlineto{\pgfqpoint{1.263262in}{1.936817in}}%
\pgfpathclose%
\pgfusepath{fill}%
\end{pgfscope}%
\begin{pgfscope}%
\pgfpathrectangle{\pgfqpoint{0.697024in}{0.857143in}}{\pgfqpoint{2.627103in}{1.813434in}}%
\pgfusepath{clip}%
\pgfsetbuttcap%
\pgfsetmiterjoin%
\definecolor{currentfill}{rgb}{0.133298,0.375282,0.379395}%
\pgfsetfillcolor{currentfill}%
\pgfsetlinewidth{0.000000pt}%
\definecolor{currentstroke}{rgb}{0.000000,0.000000,0.000000}%
\pgfsetstrokecolor{currentstroke}%
\pgfsetstrokeopacity{0.000000}%
\pgfsetdash{}{0pt}%
\pgfpathmoveto{\pgfqpoint{1.274433in}{1.847071in}}%
\pgfpathlineto{\pgfqpoint{1.283369in}{1.847071in}}%
\pgfpathlineto{\pgfqpoint{1.283369in}{1.792622in}}%
\pgfpathlineto{\pgfqpoint{1.274433in}{1.792622in}}%
\pgfpathlineto{\pgfqpoint{1.274433in}{1.847071in}}%
\pgfpathclose%
\pgfusepath{fill}%
\end{pgfscope}%
\begin{pgfscope}%
\pgfpathrectangle{\pgfqpoint{0.697024in}{0.857143in}}{\pgfqpoint{2.627103in}{1.813434in}}%
\pgfusepath{clip}%
\pgfsetbuttcap%
\pgfsetmiterjoin%
\definecolor{currentfill}{rgb}{0.133298,0.375282,0.379395}%
\pgfsetfillcolor{currentfill}%
\pgfsetlinewidth{0.000000pt}%
\definecolor{currentstroke}{rgb}{0.000000,0.000000,0.000000}%
\pgfsetstrokecolor{currentstroke}%
\pgfsetstrokeopacity{0.000000}%
\pgfsetdash{}{0pt}%
\pgfpathmoveto{\pgfqpoint{1.285604in}{1.923735in}}%
\pgfpathlineto{\pgfqpoint{1.294540in}{1.923735in}}%
\pgfpathlineto{\pgfqpoint{1.294540in}{1.938897in}}%
\pgfpathlineto{\pgfqpoint{1.285604in}{1.938897in}}%
\pgfpathlineto{\pgfqpoint{1.285604in}{1.923735in}}%
\pgfpathclose%
\pgfusepath{fill}%
\end{pgfscope}%
\begin{pgfscope}%
\pgfpathrectangle{\pgfqpoint{0.697024in}{0.857143in}}{\pgfqpoint{2.627103in}{1.813434in}}%
\pgfusepath{clip}%
\pgfsetbuttcap%
\pgfsetmiterjoin%
\definecolor{currentfill}{rgb}{0.133298,0.375282,0.379395}%
\pgfsetfillcolor{currentfill}%
\pgfsetlinewidth{0.000000pt}%
\definecolor{currentstroke}{rgb}{0.000000,0.000000,0.000000}%
\pgfsetstrokecolor{currentstroke}%
\pgfsetstrokeopacity{0.000000}%
\pgfsetdash{}{0pt}%
\pgfpathmoveto{\pgfqpoint{1.296774in}{1.914558in}}%
\pgfpathlineto{\pgfqpoint{1.305711in}{1.914558in}}%
\pgfpathlineto{\pgfqpoint{1.305711in}{2.017940in}}%
\pgfpathlineto{\pgfqpoint{1.296774in}{2.017940in}}%
\pgfpathlineto{\pgfqpoint{1.296774in}{1.914558in}}%
\pgfpathclose%
\pgfusepath{fill}%
\end{pgfscope}%
\begin{pgfscope}%
\pgfpathrectangle{\pgfqpoint{0.697024in}{0.857143in}}{\pgfqpoint{2.627103in}{1.813434in}}%
\pgfusepath{clip}%
\pgfsetbuttcap%
\pgfsetmiterjoin%
\definecolor{currentfill}{rgb}{0.133298,0.375282,0.379395}%
\pgfsetfillcolor{currentfill}%
\pgfsetlinewidth{0.000000pt}%
\definecolor{currentstroke}{rgb}{0.000000,0.000000,0.000000}%
\pgfsetstrokecolor{currentstroke}%
\pgfsetstrokeopacity{0.000000}%
\pgfsetdash{}{0pt}%
\pgfpathmoveto{\pgfqpoint{1.307945in}{1.926030in}}%
\pgfpathlineto{\pgfqpoint{1.316881in}{1.926030in}}%
\pgfpathlineto{\pgfqpoint{1.316881in}{2.084846in}}%
\pgfpathlineto{\pgfqpoint{1.307945in}{2.084846in}}%
\pgfpathlineto{\pgfqpoint{1.307945in}{1.926030in}}%
\pgfpathclose%
\pgfusepath{fill}%
\end{pgfscope}%
\begin{pgfscope}%
\pgfpathrectangle{\pgfqpoint{0.697024in}{0.857143in}}{\pgfqpoint{2.627103in}{1.813434in}}%
\pgfusepath{clip}%
\pgfsetbuttcap%
\pgfsetmiterjoin%
\definecolor{currentfill}{rgb}{0.133298,0.375282,0.379395}%
\pgfsetfillcolor{currentfill}%
\pgfsetlinewidth{0.000000pt}%
\definecolor{currentstroke}{rgb}{0.000000,0.000000,0.000000}%
\pgfsetstrokecolor{currentstroke}%
\pgfsetstrokeopacity{0.000000}%
\pgfsetdash{}{0pt}%
\pgfpathmoveto{\pgfqpoint{1.319115in}{1.847071in}}%
\pgfpathlineto{\pgfqpoint{1.328052in}{1.847071in}}%
\pgfpathlineto{\pgfqpoint{1.328052in}{1.779246in}}%
\pgfpathlineto{\pgfqpoint{1.319115in}{1.779246in}}%
\pgfpathlineto{\pgfqpoint{1.319115in}{1.847071in}}%
\pgfpathclose%
\pgfusepath{fill}%
\end{pgfscope}%
\begin{pgfscope}%
\pgfpathrectangle{\pgfqpoint{0.697024in}{0.857143in}}{\pgfqpoint{2.627103in}{1.813434in}}%
\pgfusepath{clip}%
\pgfsetbuttcap%
\pgfsetmiterjoin%
\definecolor{currentfill}{rgb}{0.133298,0.375282,0.379395}%
\pgfsetfillcolor{currentfill}%
\pgfsetlinewidth{0.000000pt}%
\definecolor{currentstroke}{rgb}{0.000000,0.000000,0.000000}%
\pgfsetstrokecolor{currentstroke}%
\pgfsetstrokeopacity{0.000000}%
\pgfsetdash{}{0pt}%
\pgfpathmoveto{\pgfqpoint{1.330286in}{1.872646in}}%
\pgfpathlineto{\pgfqpoint{1.339222in}{1.872646in}}%
\pgfpathlineto{\pgfqpoint{1.339222in}{2.042928in}}%
\pgfpathlineto{\pgfqpoint{1.330286in}{2.042928in}}%
\pgfpathlineto{\pgfqpoint{1.330286in}{1.872646in}}%
\pgfpathclose%
\pgfusepath{fill}%
\end{pgfscope}%
\begin{pgfscope}%
\pgfpathrectangle{\pgfqpoint{0.697024in}{0.857143in}}{\pgfqpoint{2.627103in}{1.813434in}}%
\pgfusepath{clip}%
\pgfsetbuttcap%
\pgfsetmiterjoin%
\definecolor{currentfill}{rgb}{0.133298,0.375282,0.379395}%
\pgfsetfillcolor{currentfill}%
\pgfsetlinewidth{0.000000pt}%
\definecolor{currentstroke}{rgb}{0.000000,0.000000,0.000000}%
\pgfsetstrokecolor{currentstroke}%
\pgfsetstrokeopacity{0.000000}%
\pgfsetdash{}{0pt}%
\pgfpathmoveto{\pgfqpoint{1.341457in}{1.892155in}}%
\pgfpathlineto{\pgfqpoint{1.350393in}{1.892155in}}%
\pgfpathlineto{\pgfqpoint{1.350393in}{1.996859in}}%
\pgfpathlineto{\pgfqpoint{1.341457in}{1.996859in}}%
\pgfpathlineto{\pgfqpoint{1.341457in}{1.892155in}}%
\pgfpathclose%
\pgfusepath{fill}%
\end{pgfscope}%
\begin{pgfscope}%
\pgfpathrectangle{\pgfqpoint{0.697024in}{0.857143in}}{\pgfqpoint{2.627103in}{1.813434in}}%
\pgfusepath{clip}%
\pgfsetbuttcap%
\pgfsetmiterjoin%
\definecolor{currentfill}{rgb}{0.133298,0.375282,0.379395}%
\pgfsetfillcolor{currentfill}%
\pgfsetlinewidth{0.000000pt}%
\definecolor{currentstroke}{rgb}{0.000000,0.000000,0.000000}%
\pgfsetstrokecolor{currentstroke}%
\pgfsetstrokeopacity{0.000000}%
\pgfsetdash{}{0pt}%
\pgfpathmoveto{\pgfqpoint{1.352627in}{1.884328in}}%
\pgfpathlineto{\pgfqpoint{1.361564in}{1.884328in}}%
\pgfpathlineto{\pgfqpoint{1.361564in}{2.026321in}}%
\pgfpathlineto{\pgfqpoint{1.352627in}{2.026321in}}%
\pgfpathlineto{\pgfqpoint{1.352627in}{1.884328in}}%
\pgfpathclose%
\pgfusepath{fill}%
\end{pgfscope}%
\begin{pgfscope}%
\pgfpathrectangle{\pgfqpoint{0.697024in}{0.857143in}}{\pgfqpoint{2.627103in}{1.813434in}}%
\pgfusepath{clip}%
\pgfsetbuttcap%
\pgfsetmiterjoin%
\definecolor{currentfill}{rgb}{0.133298,0.375282,0.379395}%
\pgfsetfillcolor{currentfill}%
\pgfsetlinewidth{0.000000pt}%
\definecolor{currentstroke}{rgb}{0.000000,0.000000,0.000000}%
\pgfsetstrokecolor{currentstroke}%
\pgfsetstrokeopacity{0.000000}%
\pgfsetdash{}{0pt}%
\pgfpathmoveto{\pgfqpoint{1.363798in}{1.877043in}}%
\pgfpathlineto{\pgfqpoint{1.372734in}{1.877043in}}%
\pgfpathlineto{\pgfqpoint{1.372734in}{1.966854in}}%
\pgfpathlineto{\pgfqpoint{1.363798in}{1.966854in}}%
\pgfpathlineto{\pgfqpoint{1.363798in}{1.877043in}}%
\pgfpathclose%
\pgfusepath{fill}%
\end{pgfscope}%
\begin{pgfscope}%
\pgfpathrectangle{\pgfqpoint{0.697024in}{0.857143in}}{\pgfqpoint{2.627103in}{1.813434in}}%
\pgfusepath{clip}%
\pgfsetbuttcap%
\pgfsetmiterjoin%
\definecolor{currentfill}{rgb}{0.133298,0.375282,0.379395}%
\pgfsetfillcolor{currentfill}%
\pgfsetlinewidth{0.000000pt}%
\definecolor{currentstroke}{rgb}{0.000000,0.000000,0.000000}%
\pgfsetstrokecolor{currentstroke}%
\pgfsetstrokeopacity{0.000000}%
\pgfsetdash{}{0pt}%
\pgfpathmoveto{\pgfqpoint{1.374968in}{1.849961in}}%
\pgfpathlineto{\pgfqpoint{1.383905in}{1.849961in}}%
\pgfpathlineto{\pgfqpoint{1.383905in}{2.016884in}}%
\pgfpathlineto{\pgfqpoint{1.374968in}{2.016884in}}%
\pgfpathlineto{\pgfqpoint{1.374968in}{1.849961in}}%
\pgfpathclose%
\pgfusepath{fill}%
\end{pgfscope}%
\begin{pgfscope}%
\pgfpathrectangle{\pgfqpoint{0.697024in}{0.857143in}}{\pgfqpoint{2.627103in}{1.813434in}}%
\pgfusepath{clip}%
\pgfsetbuttcap%
\pgfsetmiterjoin%
\definecolor{currentfill}{rgb}{0.133298,0.375282,0.379395}%
\pgfsetfillcolor{currentfill}%
\pgfsetlinewidth{0.000000pt}%
\definecolor{currentstroke}{rgb}{0.000000,0.000000,0.000000}%
\pgfsetstrokecolor{currentstroke}%
\pgfsetstrokeopacity{0.000000}%
\pgfsetdash{}{0pt}%
\pgfpathmoveto{\pgfqpoint{1.386139in}{1.835467in}}%
\pgfpathlineto{\pgfqpoint{1.395076in}{1.835467in}}%
\pgfpathlineto{\pgfqpoint{1.395076in}{1.748359in}}%
\pgfpathlineto{\pgfqpoint{1.386139in}{1.748359in}}%
\pgfpathlineto{\pgfqpoint{1.386139in}{1.835467in}}%
\pgfpathclose%
\pgfusepath{fill}%
\end{pgfscope}%
\begin{pgfscope}%
\pgfpathrectangle{\pgfqpoint{0.697024in}{0.857143in}}{\pgfqpoint{2.627103in}{1.813434in}}%
\pgfusepath{clip}%
\pgfsetbuttcap%
\pgfsetmiterjoin%
\definecolor{currentfill}{rgb}{0.133298,0.375282,0.379395}%
\pgfsetfillcolor{currentfill}%
\pgfsetlinewidth{0.000000pt}%
\definecolor{currentstroke}{rgb}{0.000000,0.000000,0.000000}%
\pgfsetstrokecolor{currentstroke}%
\pgfsetstrokeopacity{0.000000}%
\pgfsetdash{}{0pt}%
\pgfpathmoveto{\pgfqpoint{1.397310in}{1.847071in}}%
\pgfpathlineto{\pgfqpoint{1.406246in}{1.847071in}}%
\pgfpathlineto{\pgfqpoint{1.406246in}{1.909782in}}%
\pgfpathlineto{\pgfqpoint{1.397310in}{1.909782in}}%
\pgfpathlineto{\pgfqpoint{1.397310in}{1.847071in}}%
\pgfpathclose%
\pgfusepath{fill}%
\end{pgfscope}%
\begin{pgfscope}%
\pgfpathrectangle{\pgfqpoint{0.697024in}{0.857143in}}{\pgfqpoint{2.627103in}{1.813434in}}%
\pgfusepath{clip}%
\pgfsetbuttcap%
\pgfsetmiterjoin%
\definecolor{currentfill}{rgb}{0.133298,0.375282,0.379395}%
\pgfsetfillcolor{currentfill}%
\pgfsetlinewidth{0.000000pt}%
\definecolor{currentstroke}{rgb}{0.000000,0.000000,0.000000}%
\pgfsetstrokecolor{currentstroke}%
\pgfsetstrokeopacity{0.000000}%
\pgfsetdash{}{0pt}%
\pgfpathmoveto{\pgfqpoint{1.408480in}{1.847071in}}%
\pgfpathlineto{\pgfqpoint{1.417417in}{1.847071in}}%
\pgfpathlineto{\pgfqpoint{1.417417in}{1.949761in}}%
\pgfpathlineto{\pgfqpoint{1.408480in}{1.949761in}}%
\pgfpathlineto{\pgfqpoint{1.408480in}{1.847071in}}%
\pgfpathclose%
\pgfusepath{fill}%
\end{pgfscope}%
\begin{pgfscope}%
\pgfpathrectangle{\pgfqpoint{0.697024in}{0.857143in}}{\pgfqpoint{2.627103in}{1.813434in}}%
\pgfusepath{clip}%
\pgfsetbuttcap%
\pgfsetmiterjoin%
\definecolor{currentfill}{rgb}{0.133298,0.375282,0.379395}%
\pgfsetfillcolor{currentfill}%
\pgfsetlinewidth{0.000000pt}%
\definecolor{currentstroke}{rgb}{0.000000,0.000000,0.000000}%
\pgfsetstrokecolor{currentstroke}%
\pgfsetstrokeopacity{0.000000}%
\pgfsetdash{}{0pt}%
\pgfpathmoveto{\pgfqpoint{1.419651in}{1.847071in}}%
\pgfpathlineto{\pgfqpoint{1.428587in}{1.847071in}}%
\pgfpathlineto{\pgfqpoint{1.428587in}{1.869869in}}%
\pgfpathlineto{\pgfqpoint{1.419651in}{1.869869in}}%
\pgfpathlineto{\pgfqpoint{1.419651in}{1.847071in}}%
\pgfpathclose%
\pgfusepath{fill}%
\end{pgfscope}%
\begin{pgfscope}%
\pgfpathrectangle{\pgfqpoint{0.697024in}{0.857143in}}{\pgfqpoint{2.627103in}{1.813434in}}%
\pgfusepath{clip}%
\pgfsetbuttcap%
\pgfsetmiterjoin%
\definecolor{currentfill}{rgb}{0.133298,0.375282,0.379395}%
\pgfsetfillcolor{currentfill}%
\pgfsetlinewidth{0.000000pt}%
\definecolor{currentstroke}{rgb}{0.000000,0.000000,0.000000}%
\pgfsetstrokecolor{currentstroke}%
\pgfsetstrokeopacity{0.000000}%
\pgfsetdash{}{0pt}%
\pgfpathmoveto{\pgfqpoint{1.430821in}{1.847071in}}%
\pgfpathlineto{\pgfqpoint{1.439758in}{1.847071in}}%
\pgfpathlineto{\pgfqpoint{1.439758in}{1.872725in}}%
\pgfpathlineto{\pgfqpoint{1.430821in}{1.872725in}}%
\pgfpathlineto{\pgfqpoint{1.430821in}{1.847071in}}%
\pgfpathclose%
\pgfusepath{fill}%
\end{pgfscope}%
\begin{pgfscope}%
\pgfpathrectangle{\pgfqpoint{0.697024in}{0.857143in}}{\pgfqpoint{2.627103in}{1.813434in}}%
\pgfusepath{clip}%
\pgfsetbuttcap%
\pgfsetmiterjoin%
\definecolor{currentfill}{rgb}{0.133298,0.375282,0.379395}%
\pgfsetfillcolor{currentfill}%
\pgfsetlinewidth{0.000000pt}%
\definecolor{currentstroke}{rgb}{0.000000,0.000000,0.000000}%
\pgfsetstrokecolor{currentstroke}%
\pgfsetstrokeopacity{0.000000}%
\pgfsetdash{}{0pt}%
\pgfpathmoveto{\pgfqpoint{1.441992in}{1.802324in}}%
\pgfpathlineto{\pgfqpoint{1.450929in}{1.802324in}}%
\pgfpathlineto{\pgfqpoint{1.450929in}{1.742013in}}%
\pgfpathlineto{\pgfqpoint{1.441992in}{1.742013in}}%
\pgfpathlineto{\pgfqpoint{1.441992in}{1.802324in}}%
\pgfpathclose%
\pgfusepath{fill}%
\end{pgfscope}%
\begin{pgfscope}%
\pgfpathrectangle{\pgfqpoint{0.697024in}{0.857143in}}{\pgfqpoint{2.627103in}{1.813434in}}%
\pgfusepath{clip}%
\pgfsetbuttcap%
\pgfsetmiterjoin%
\definecolor{currentfill}{rgb}{0.133298,0.375282,0.379395}%
\pgfsetfillcolor{currentfill}%
\pgfsetlinewidth{0.000000pt}%
\definecolor{currentstroke}{rgb}{0.000000,0.000000,0.000000}%
\pgfsetstrokecolor{currentstroke}%
\pgfsetstrokeopacity{0.000000}%
\pgfsetdash{}{0pt}%
\pgfpathmoveto{\pgfqpoint{1.453163in}{1.806726in}}%
\pgfpathlineto{\pgfqpoint{1.462099in}{1.806726in}}%
\pgfpathlineto{\pgfqpoint{1.462099in}{1.653772in}}%
\pgfpathlineto{\pgfqpoint{1.453163in}{1.653772in}}%
\pgfpathlineto{\pgfqpoint{1.453163in}{1.806726in}}%
\pgfpathclose%
\pgfusepath{fill}%
\end{pgfscope}%
\begin{pgfscope}%
\pgfpathrectangle{\pgfqpoint{0.697024in}{0.857143in}}{\pgfqpoint{2.627103in}{1.813434in}}%
\pgfusepath{clip}%
\pgfsetbuttcap%
\pgfsetmiterjoin%
\definecolor{currentfill}{rgb}{0.133298,0.375282,0.379395}%
\pgfsetfillcolor{currentfill}%
\pgfsetlinewidth{0.000000pt}%
\definecolor{currentstroke}{rgb}{0.000000,0.000000,0.000000}%
\pgfsetstrokecolor{currentstroke}%
\pgfsetstrokeopacity{0.000000}%
\pgfsetdash{}{0pt}%
\pgfpathmoveto{\pgfqpoint{1.464333in}{1.835521in}}%
\pgfpathlineto{\pgfqpoint{1.473270in}{1.835521in}}%
\pgfpathlineto{\pgfqpoint{1.473270in}{1.719917in}}%
\pgfpathlineto{\pgfqpoint{1.464333in}{1.719917in}}%
\pgfpathlineto{\pgfqpoint{1.464333in}{1.835521in}}%
\pgfpathclose%
\pgfusepath{fill}%
\end{pgfscope}%
\begin{pgfscope}%
\pgfpathrectangle{\pgfqpoint{0.697024in}{0.857143in}}{\pgfqpoint{2.627103in}{1.813434in}}%
\pgfusepath{clip}%
\pgfsetbuttcap%
\pgfsetmiterjoin%
\definecolor{currentfill}{rgb}{0.133298,0.375282,0.379395}%
\pgfsetfillcolor{currentfill}%
\pgfsetlinewidth{0.000000pt}%
\definecolor{currentstroke}{rgb}{0.000000,0.000000,0.000000}%
\pgfsetstrokecolor{currentstroke}%
\pgfsetstrokeopacity{0.000000}%
\pgfsetdash{}{0pt}%
\pgfpathmoveto{\pgfqpoint{1.475504in}{1.847071in}}%
\pgfpathlineto{\pgfqpoint{1.484440in}{1.847071in}}%
\pgfpathlineto{\pgfqpoint{1.484440in}{1.905264in}}%
\pgfpathlineto{\pgfqpoint{1.475504in}{1.905264in}}%
\pgfpathlineto{\pgfqpoint{1.475504in}{1.847071in}}%
\pgfpathclose%
\pgfusepath{fill}%
\end{pgfscope}%
\begin{pgfscope}%
\pgfpathrectangle{\pgfqpoint{0.697024in}{0.857143in}}{\pgfqpoint{2.627103in}{1.813434in}}%
\pgfusepath{clip}%
\pgfsetbuttcap%
\pgfsetmiterjoin%
\definecolor{currentfill}{rgb}{0.133298,0.375282,0.379395}%
\pgfsetfillcolor{currentfill}%
\pgfsetlinewidth{0.000000pt}%
\definecolor{currentstroke}{rgb}{0.000000,0.000000,0.000000}%
\pgfsetstrokecolor{currentstroke}%
\pgfsetstrokeopacity{0.000000}%
\pgfsetdash{}{0pt}%
\pgfpathmoveto{\pgfqpoint{1.486674in}{1.823435in}}%
\pgfpathlineto{\pgfqpoint{1.495611in}{1.823435in}}%
\pgfpathlineto{\pgfqpoint{1.495611in}{1.810404in}}%
\pgfpathlineto{\pgfqpoint{1.486674in}{1.810404in}}%
\pgfpathlineto{\pgfqpoint{1.486674in}{1.823435in}}%
\pgfpathclose%
\pgfusepath{fill}%
\end{pgfscope}%
\begin{pgfscope}%
\pgfpathrectangle{\pgfqpoint{0.697024in}{0.857143in}}{\pgfqpoint{2.627103in}{1.813434in}}%
\pgfusepath{clip}%
\pgfsetbuttcap%
\pgfsetmiterjoin%
\definecolor{currentfill}{rgb}{0.133298,0.375282,0.379395}%
\pgfsetfillcolor{currentfill}%
\pgfsetlinewidth{0.000000pt}%
\definecolor{currentstroke}{rgb}{0.000000,0.000000,0.000000}%
\pgfsetstrokecolor{currentstroke}%
\pgfsetstrokeopacity{0.000000}%
\pgfsetdash{}{0pt}%
\pgfpathmoveto{\pgfqpoint{1.497845in}{1.847071in}}%
\pgfpathlineto{\pgfqpoint{1.506782in}{1.847071in}}%
\pgfpathlineto{\pgfqpoint{1.506782in}{1.867380in}}%
\pgfpathlineto{\pgfqpoint{1.497845in}{1.867380in}}%
\pgfpathlineto{\pgfqpoint{1.497845in}{1.847071in}}%
\pgfpathclose%
\pgfusepath{fill}%
\end{pgfscope}%
\begin{pgfscope}%
\pgfpathrectangle{\pgfqpoint{0.697024in}{0.857143in}}{\pgfqpoint{2.627103in}{1.813434in}}%
\pgfusepath{clip}%
\pgfsetbuttcap%
\pgfsetmiterjoin%
\definecolor{currentfill}{rgb}{0.133298,0.375282,0.379395}%
\pgfsetfillcolor{currentfill}%
\pgfsetlinewidth{0.000000pt}%
\definecolor{currentstroke}{rgb}{0.000000,0.000000,0.000000}%
\pgfsetstrokecolor{currentstroke}%
\pgfsetstrokeopacity{0.000000}%
\pgfsetdash{}{0pt}%
\pgfpathmoveto{\pgfqpoint{1.509016in}{1.847071in}}%
\pgfpathlineto{\pgfqpoint{1.517952in}{1.847071in}}%
\pgfpathlineto{\pgfqpoint{1.517952in}{1.893161in}}%
\pgfpathlineto{\pgfqpoint{1.509016in}{1.893161in}}%
\pgfpathlineto{\pgfqpoint{1.509016in}{1.847071in}}%
\pgfpathclose%
\pgfusepath{fill}%
\end{pgfscope}%
\begin{pgfscope}%
\pgfpathrectangle{\pgfqpoint{0.697024in}{0.857143in}}{\pgfqpoint{2.627103in}{1.813434in}}%
\pgfusepath{clip}%
\pgfsetbuttcap%
\pgfsetmiterjoin%
\definecolor{currentfill}{rgb}{0.133298,0.375282,0.379395}%
\pgfsetfillcolor{currentfill}%
\pgfsetlinewidth{0.000000pt}%
\definecolor{currentstroke}{rgb}{0.000000,0.000000,0.000000}%
\pgfsetstrokecolor{currentstroke}%
\pgfsetstrokeopacity{0.000000}%
\pgfsetdash{}{0pt}%
\pgfpathmoveto{\pgfqpoint{1.520186in}{1.774299in}}%
\pgfpathlineto{\pgfqpoint{1.529123in}{1.774299in}}%
\pgfpathlineto{\pgfqpoint{1.529123in}{1.735874in}}%
\pgfpathlineto{\pgfqpoint{1.520186in}{1.735874in}}%
\pgfpathlineto{\pgfqpoint{1.520186in}{1.774299in}}%
\pgfpathclose%
\pgfusepath{fill}%
\end{pgfscope}%
\begin{pgfscope}%
\pgfpathrectangle{\pgfqpoint{0.697024in}{0.857143in}}{\pgfqpoint{2.627103in}{1.813434in}}%
\pgfusepath{clip}%
\pgfsetbuttcap%
\pgfsetmiterjoin%
\definecolor{currentfill}{rgb}{0.133298,0.375282,0.379395}%
\pgfsetfillcolor{currentfill}%
\pgfsetlinewidth{0.000000pt}%
\definecolor{currentstroke}{rgb}{0.000000,0.000000,0.000000}%
\pgfsetstrokecolor{currentstroke}%
\pgfsetstrokeopacity{0.000000}%
\pgfsetdash{}{0pt}%
\pgfpathmoveto{\pgfqpoint{1.531357in}{1.774393in}}%
\pgfpathlineto{\pgfqpoint{1.540293in}{1.774393in}}%
\pgfpathlineto{\pgfqpoint{1.540293in}{1.731622in}}%
\pgfpathlineto{\pgfqpoint{1.531357in}{1.731622in}}%
\pgfpathlineto{\pgfqpoint{1.531357in}{1.774393in}}%
\pgfpathclose%
\pgfusepath{fill}%
\end{pgfscope}%
\begin{pgfscope}%
\pgfpathrectangle{\pgfqpoint{0.697024in}{0.857143in}}{\pgfqpoint{2.627103in}{1.813434in}}%
\pgfusepath{clip}%
\pgfsetbuttcap%
\pgfsetmiterjoin%
\definecolor{currentfill}{rgb}{0.133298,0.375282,0.379395}%
\pgfsetfillcolor{currentfill}%
\pgfsetlinewidth{0.000000pt}%
\definecolor{currentstroke}{rgb}{0.000000,0.000000,0.000000}%
\pgfsetstrokecolor{currentstroke}%
\pgfsetstrokeopacity{0.000000}%
\pgfsetdash{}{0pt}%
\pgfpathmoveto{\pgfqpoint{1.542528in}{1.792667in}}%
\pgfpathlineto{\pgfqpoint{1.551464in}{1.792667in}}%
\pgfpathlineto{\pgfqpoint{1.551464in}{1.697441in}}%
\pgfpathlineto{\pgfqpoint{1.542528in}{1.697441in}}%
\pgfpathlineto{\pgfqpoint{1.542528in}{1.792667in}}%
\pgfpathclose%
\pgfusepath{fill}%
\end{pgfscope}%
\begin{pgfscope}%
\pgfpathrectangle{\pgfqpoint{0.697024in}{0.857143in}}{\pgfqpoint{2.627103in}{1.813434in}}%
\pgfusepath{clip}%
\pgfsetbuttcap%
\pgfsetmiterjoin%
\definecolor{currentfill}{rgb}{0.133298,0.375282,0.379395}%
\pgfsetfillcolor{currentfill}%
\pgfsetlinewidth{0.000000pt}%
\definecolor{currentstroke}{rgb}{0.000000,0.000000,0.000000}%
\pgfsetstrokecolor{currentstroke}%
\pgfsetstrokeopacity{0.000000}%
\pgfsetdash{}{0pt}%
\pgfpathmoveto{\pgfqpoint{1.553698in}{1.799777in}}%
\pgfpathlineto{\pgfqpoint{1.562635in}{1.799777in}}%
\pgfpathlineto{\pgfqpoint{1.562635in}{1.796508in}}%
\pgfpathlineto{\pgfqpoint{1.553698in}{1.796508in}}%
\pgfpathlineto{\pgfqpoint{1.553698in}{1.799777in}}%
\pgfpathclose%
\pgfusepath{fill}%
\end{pgfscope}%
\begin{pgfscope}%
\pgfpathrectangle{\pgfqpoint{0.697024in}{0.857143in}}{\pgfqpoint{2.627103in}{1.813434in}}%
\pgfusepath{clip}%
\pgfsetbuttcap%
\pgfsetmiterjoin%
\definecolor{currentfill}{rgb}{0.133298,0.375282,0.379395}%
\pgfsetfillcolor{currentfill}%
\pgfsetlinewidth{0.000000pt}%
\definecolor{currentstroke}{rgb}{0.000000,0.000000,0.000000}%
\pgfsetstrokecolor{currentstroke}%
\pgfsetstrokeopacity{0.000000}%
\pgfsetdash{}{0pt}%
\pgfpathmoveto{\pgfqpoint{1.564869in}{1.818659in}}%
\pgfpathlineto{\pgfqpoint{1.573805in}{1.818659in}}%
\pgfpathlineto{\pgfqpoint{1.573805in}{1.761882in}}%
\pgfpathlineto{\pgfqpoint{1.564869in}{1.761882in}}%
\pgfpathlineto{\pgfqpoint{1.564869in}{1.818659in}}%
\pgfpathclose%
\pgfusepath{fill}%
\end{pgfscope}%
\begin{pgfscope}%
\pgfpathrectangle{\pgfqpoint{0.697024in}{0.857143in}}{\pgfqpoint{2.627103in}{1.813434in}}%
\pgfusepath{clip}%
\pgfsetbuttcap%
\pgfsetmiterjoin%
\definecolor{currentfill}{rgb}{0.133298,0.375282,0.379395}%
\pgfsetfillcolor{currentfill}%
\pgfsetlinewidth{0.000000pt}%
\definecolor{currentstroke}{rgb}{0.000000,0.000000,0.000000}%
\pgfsetstrokecolor{currentstroke}%
\pgfsetstrokeopacity{0.000000}%
\pgfsetdash{}{0pt}%
\pgfpathmoveto{\pgfqpoint{1.576039in}{1.803151in}}%
\pgfpathlineto{\pgfqpoint{1.584976in}{1.803151in}}%
\pgfpathlineto{\pgfqpoint{1.584976in}{1.718566in}}%
\pgfpathlineto{\pgfqpoint{1.576039in}{1.718566in}}%
\pgfpathlineto{\pgfqpoint{1.576039in}{1.803151in}}%
\pgfpathclose%
\pgfusepath{fill}%
\end{pgfscope}%
\begin{pgfscope}%
\pgfpathrectangle{\pgfqpoint{0.697024in}{0.857143in}}{\pgfqpoint{2.627103in}{1.813434in}}%
\pgfusepath{clip}%
\pgfsetbuttcap%
\pgfsetmiterjoin%
\definecolor{currentfill}{rgb}{0.133298,0.375282,0.379395}%
\pgfsetfillcolor{currentfill}%
\pgfsetlinewidth{0.000000pt}%
\definecolor{currentstroke}{rgb}{0.000000,0.000000,0.000000}%
\pgfsetstrokecolor{currentstroke}%
\pgfsetstrokeopacity{0.000000}%
\pgfsetdash{}{0pt}%
\pgfpathmoveto{\pgfqpoint{1.587210in}{1.847071in}}%
\pgfpathlineto{\pgfqpoint{1.596146in}{1.847071in}}%
\pgfpathlineto{\pgfqpoint{1.596146in}{1.876087in}}%
\pgfpathlineto{\pgfqpoint{1.587210in}{1.876087in}}%
\pgfpathlineto{\pgfqpoint{1.587210in}{1.847071in}}%
\pgfpathclose%
\pgfusepath{fill}%
\end{pgfscope}%
\begin{pgfscope}%
\pgfpathrectangle{\pgfqpoint{0.697024in}{0.857143in}}{\pgfqpoint{2.627103in}{1.813434in}}%
\pgfusepath{clip}%
\pgfsetbuttcap%
\pgfsetmiterjoin%
\definecolor{currentfill}{rgb}{0.133298,0.375282,0.379395}%
\pgfsetfillcolor{currentfill}%
\pgfsetlinewidth{0.000000pt}%
\definecolor{currentstroke}{rgb}{0.000000,0.000000,0.000000}%
\pgfsetstrokecolor{currentstroke}%
\pgfsetstrokeopacity{0.000000}%
\pgfsetdash{}{0pt}%
\pgfpathmoveto{\pgfqpoint{1.598381in}{1.787213in}}%
\pgfpathlineto{\pgfqpoint{1.607317in}{1.787213in}}%
\pgfpathlineto{\pgfqpoint{1.607317in}{1.717820in}}%
\pgfpathlineto{\pgfqpoint{1.598381in}{1.717820in}}%
\pgfpathlineto{\pgfqpoint{1.598381in}{1.787213in}}%
\pgfpathclose%
\pgfusepath{fill}%
\end{pgfscope}%
\begin{pgfscope}%
\pgfpathrectangle{\pgfqpoint{0.697024in}{0.857143in}}{\pgfqpoint{2.627103in}{1.813434in}}%
\pgfusepath{clip}%
\pgfsetbuttcap%
\pgfsetmiterjoin%
\definecolor{currentfill}{rgb}{0.133298,0.375282,0.379395}%
\pgfsetfillcolor{currentfill}%
\pgfsetlinewidth{0.000000pt}%
\definecolor{currentstroke}{rgb}{0.000000,0.000000,0.000000}%
\pgfsetstrokecolor{currentstroke}%
\pgfsetstrokeopacity{0.000000}%
\pgfsetdash{}{0pt}%
\pgfpathmoveto{\pgfqpoint{1.609551in}{1.789342in}}%
\pgfpathlineto{\pgfqpoint{1.618488in}{1.789342in}}%
\pgfpathlineto{\pgfqpoint{1.618488in}{1.759276in}}%
\pgfpathlineto{\pgfqpoint{1.609551in}{1.759276in}}%
\pgfpathlineto{\pgfqpoint{1.609551in}{1.789342in}}%
\pgfpathclose%
\pgfusepath{fill}%
\end{pgfscope}%
\begin{pgfscope}%
\pgfpathrectangle{\pgfqpoint{0.697024in}{0.857143in}}{\pgfqpoint{2.627103in}{1.813434in}}%
\pgfusepath{clip}%
\pgfsetbuttcap%
\pgfsetmiterjoin%
\definecolor{currentfill}{rgb}{0.133298,0.375282,0.379395}%
\pgfsetfillcolor{currentfill}%
\pgfsetlinewidth{0.000000pt}%
\definecolor{currentstroke}{rgb}{0.000000,0.000000,0.000000}%
\pgfsetstrokecolor{currentstroke}%
\pgfsetstrokeopacity{0.000000}%
\pgfsetdash{}{0pt}%
\pgfpathmoveto{\pgfqpoint{1.620722in}{1.793552in}}%
\pgfpathlineto{\pgfqpoint{1.629658in}{1.793552in}}%
\pgfpathlineto{\pgfqpoint{1.629658in}{1.729882in}}%
\pgfpathlineto{\pgfqpoint{1.620722in}{1.729882in}}%
\pgfpathlineto{\pgfqpoint{1.620722in}{1.793552in}}%
\pgfpathclose%
\pgfusepath{fill}%
\end{pgfscope}%
\begin{pgfscope}%
\pgfpathrectangle{\pgfqpoint{0.697024in}{0.857143in}}{\pgfqpoint{2.627103in}{1.813434in}}%
\pgfusepath{clip}%
\pgfsetbuttcap%
\pgfsetmiterjoin%
\definecolor{currentfill}{rgb}{0.133298,0.375282,0.379395}%
\pgfsetfillcolor{currentfill}%
\pgfsetlinewidth{0.000000pt}%
\definecolor{currentstroke}{rgb}{0.000000,0.000000,0.000000}%
\pgfsetstrokecolor{currentstroke}%
\pgfsetstrokeopacity{0.000000}%
\pgfsetdash{}{0pt}%
\pgfpathmoveto{\pgfqpoint{1.631892in}{1.786157in}}%
\pgfpathlineto{\pgfqpoint{1.640829in}{1.786157in}}%
\pgfpathlineto{\pgfqpoint{1.640829in}{1.692417in}}%
\pgfpathlineto{\pgfqpoint{1.631892in}{1.692417in}}%
\pgfpathlineto{\pgfqpoint{1.631892in}{1.786157in}}%
\pgfpathclose%
\pgfusepath{fill}%
\end{pgfscope}%
\begin{pgfscope}%
\pgfpathrectangle{\pgfqpoint{0.697024in}{0.857143in}}{\pgfqpoint{2.627103in}{1.813434in}}%
\pgfusepath{clip}%
\pgfsetbuttcap%
\pgfsetmiterjoin%
\definecolor{currentfill}{rgb}{0.133298,0.375282,0.379395}%
\pgfsetfillcolor{currentfill}%
\pgfsetlinewidth{0.000000pt}%
\definecolor{currentstroke}{rgb}{0.000000,0.000000,0.000000}%
\pgfsetstrokecolor{currentstroke}%
\pgfsetstrokeopacity{0.000000}%
\pgfsetdash{}{0pt}%
\pgfpathmoveto{\pgfqpoint{1.643063in}{1.779810in}}%
\pgfpathlineto{\pgfqpoint{1.651999in}{1.779810in}}%
\pgfpathlineto{\pgfqpoint{1.651999in}{1.615715in}}%
\pgfpathlineto{\pgfqpoint{1.643063in}{1.615715in}}%
\pgfpathlineto{\pgfqpoint{1.643063in}{1.779810in}}%
\pgfpathclose%
\pgfusepath{fill}%
\end{pgfscope}%
\begin{pgfscope}%
\pgfpathrectangle{\pgfqpoint{0.697024in}{0.857143in}}{\pgfqpoint{2.627103in}{1.813434in}}%
\pgfusepath{clip}%
\pgfsetbuttcap%
\pgfsetmiterjoin%
\definecolor{currentfill}{rgb}{0.133298,0.375282,0.379395}%
\pgfsetfillcolor{currentfill}%
\pgfsetlinewidth{0.000000pt}%
\definecolor{currentstroke}{rgb}{0.000000,0.000000,0.000000}%
\pgfsetstrokecolor{currentstroke}%
\pgfsetstrokeopacity{0.000000}%
\pgfsetdash{}{0pt}%
\pgfpathmoveto{\pgfqpoint{1.654234in}{1.774748in}}%
\pgfpathlineto{\pgfqpoint{1.663170in}{1.774748in}}%
\pgfpathlineto{\pgfqpoint{1.663170in}{1.705720in}}%
\pgfpathlineto{\pgfqpoint{1.654234in}{1.705720in}}%
\pgfpathlineto{\pgfqpoint{1.654234in}{1.774748in}}%
\pgfpathclose%
\pgfusepath{fill}%
\end{pgfscope}%
\begin{pgfscope}%
\pgfpathrectangle{\pgfqpoint{0.697024in}{0.857143in}}{\pgfqpoint{2.627103in}{1.813434in}}%
\pgfusepath{clip}%
\pgfsetbuttcap%
\pgfsetmiterjoin%
\definecolor{currentfill}{rgb}{0.133298,0.375282,0.379395}%
\pgfsetfillcolor{currentfill}%
\pgfsetlinewidth{0.000000pt}%
\definecolor{currentstroke}{rgb}{0.000000,0.000000,0.000000}%
\pgfsetstrokecolor{currentstroke}%
\pgfsetstrokeopacity{0.000000}%
\pgfsetdash{}{0pt}%
\pgfpathmoveto{\pgfqpoint{1.665404in}{1.800862in}}%
\pgfpathlineto{\pgfqpoint{1.674341in}{1.800862in}}%
\pgfpathlineto{\pgfqpoint{1.674341in}{1.689069in}}%
\pgfpathlineto{\pgfqpoint{1.665404in}{1.689069in}}%
\pgfpathlineto{\pgfqpoint{1.665404in}{1.800862in}}%
\pgfpathclose%
\pgfusepath{fill}%
\end{pgfscope}%
\begin{pgfscope}%
\pgfpathrectangle{\pgfqpoint{0.697024in}{0.857143in}}{\pgfqpoint{2.627103in}{1.813434in}}%
\pgfusepath{clip}%
\pgfsetbuttcap%
\pgfsetmiterjoin%
\definecolor{currentfill}{rgb}{0.133298,0.375282,0.379395}%
\pgfsetfillcolor{currentfill}%
\pgfsetlinewidth{0.000000pt}%
\definecolor{currentstroke}{rgb}{0.000000,0.000000,0.000000}%
\pgfsetstrokecolor{currentstroke}%
\pgfsetstrokeopacity{0.000000}%
\pgfsetdash{}{0pt}%
\pgfpathmoveto{\pgfqpoint{1.676575in}{1.793289in}}%
\pgfpathlineto{\pgfqpoint{1.685511in}{1.793289in}}%
\pgfpathlineto{\pgfqpoint{1.685511in}{1.656798in}}%
\pgfpathlineto{\pgfqpoint{1.676575in}{1.656798in}}%
\pgfpathlineto{\pgfqpoint{1.676575in}{1.793289in}}%
\pgfpathclose%
\pgfusepath{fill}%
\end{pgfscope}%
\begin{pgfscope}%
\pgfpathrectangle{\pgfqpoint{0.697024in}{0.857143in}}{\pgfqpoint{2.627103in}{1.813434in}}%
\pgfusepath{clip}%
\pgfsetbuttcap%
\pgfsetmiterjoin%
\definecolor{currentfill}{rgb}{0.133298,0.375282,0.379395}%
\pgfsetfillcolor{currentfill}%
\pgfsetlinewidth{0.000000pt}%
\definecolor{currentstroke}{rgb}{0.000000,0.000000,0.000000}%
\pgfsetstrokecolor{currentstroke}%
\pgfsetstrokeopacity{0.000000}%
\pgfsetdash{}{0pt}%
\pgfpathmoveto{\pgfqpoint{1.687745in}{1.812967in}}%
\pgfpathlineto{\pgfqpoint{1.696682in}{1.812967in}}%
\pgfpathlineto{\pgfqpoint{1.696682in}{1.722210in}}%
\pgfpathlineto{\pgfqpoint{1.687745in}{1.722210in}}%
\pgfpathlineto{\pgfqpoint{1.687745in}{1.812967in}}%
\pgfpathclose%
\pgfusepath{fill}%
\end{pgfscope}%
\begin{pgfscope}%
\pgfpathrectangle{\pgfqpoint{0.697024in}{0.857143in}}{\pgfqpoint{2.627103in}{1.813434in}}%
\pgfusepath{clip}%
\pgfsetbuttcap%
\pgfsetmiterjoin%
\definecolor{currentfill}{rgb}{0.133298,0.375282,0.379395}%
\pgfsetfillcolor{currentfill}%
\pgfsetlinewidth{0.000000pt}%
\definecolor{currentstroke}{rgb}{0.000000,0.000000,0.000000}%
\pgfsetstrokecolor{currentstroke}%
\pgfsetstrokeopacity{0.000000}%
\pgfsetdash{}{0pt}%
\pgfpathmoveto{\pgfqpoint{1.698916in}{1.819826in}}%
\pgfpathlineto{\pgfqpoint{1.707852in}{1.819826in}}%
\pgfpathlineto{\pgfqpoint{1.707852in}{1.685914in}}%
\pgfpathlineto{\pgfqpoint{1.698916in}{1.685914in}}%
\pgfpathlineto{\pgfqpoint{1.698916in}{1.819826in}}%
\pgfpathclose%
\pgfusepath{fill}%
\end{pgfscope}%
\begin{pgfscope}%
\pgfpathrectangle{\pgfqpoint{0.697024in}{0.857143in}}{\pgfqpoint{2.627103in}{1.813434in}}%
\pgfusepath{clip}%
\pgfsetbuttcap%
\pgfsetmiterjoin%
\definecolor{currentfill}{rgb}{0.133298,0.375282,0.379395}%
\pgfsetfillcolor{currentfill}%
\pgfsetlinewidth{0.000000pt}%
\definecolor{currentstroke}{rgb}{0.000000,0.000000,0.000000}%
\pgfsetstrokecolor{currentstroke}%
\pgfsetstrokeopacity{0.000000}%
\pgfsetdash{}{0pt}%
\pgfpathmoveto{\pgfqpoint{1.710087in}{1.818994in}}%
\pgfpathlineto{\pgfqpoint{1.719023in}{1.818994in}}%
\pgfpathlineto{\pgfqpoint{1.719023in}{1.654197in}}%
\pgfpathlineto{\pgfqpoint{1.710087in}{1.654197in}}%
\pgfpathlineto{\pgfqpoint{1.710087in}{1.818994in}}%
\pgfpathclose%
\pgfusepath{fill}%
\end{pgfscope}%
\begin{pgfscope}%
\pgfpathrectangle{\pgfqpoint{0.697024in}{0.857143in}}{\pgfqpoint{2.627103in}{1.813434in}}%
\pgfusepath{clip}%
\pgfsetbuttcap%
\pgfsetmiterjoin%
\definecolor{currentfill}{rgb}{0.133298,0.375282,0.379395}%
\pgfsetfillcolor{currentfill}%
\pgfsetlinewidth{0.000000pt}%
\definecolor{currentstroke}{rgb}{0.000000,0.000000,0.000000}%
\pgfsetstrokecolor{currentstroke}%
\pgfsetstrokeopacity{0.000000}%
\pgfsetdash{}{0pt}%
\pgfpathmoveto{\pgfqpoint{1.721257in}{1.799383in}}%
\pgfpathlineto{\pgfqpoint{1.730194in}{1.799383in}}%
\pgfpathlineto{\pgfqpoint{1.730194in}{1.608018in}}%
\pgfpathlineto{\pgfqpoint{1.721257in}{1.608018in}}%
\pgfpathlineto{\pgfqpoint{1.721257in}{1.799383in}}%
\pgfpathclose%
\pgfusepath{fill}%
\end{pgfscope}%
\begin{pgfscope}%
\pgfpathrectangle{\pgfqpoint{0.697024in}{0.857143in}}{\pgfqpoint{2.627103in}{1.813434in}}%
\pgfusepath{clip}%
\pgfsetbuttcap%
\pgfsetmiterjoin%
\definecolor{currentfill}{rgb}{0.133298,0.375282,0.379395}%
\pgfsetfillcolor{currentfill}%
\pgfsetlinewidth{0.000000pt}%
\definecolor{currentstroke}{rgb}{0.000000,0.000000,0.000000}%
\pgfsetstrokecolor{currentstroke}%
\pgfsetstrokeopacity{0.000000}%
\pgfsetdash{}{0pt}%
\pgfpathmoveto{\pgfqpoint{1.732428in}{1.774187in}}%
\pgfpathlineto{\pgfqpoint{1.741364in}{1.774187in}}%
\pgfpathlineto{\pgfqpoint{1.741364in}{1.603706in}}%
\pgfpathlineto{\pgfqpoint{1.732428in}{1.603706in}}%
\pgfpathlineto{\pgfqpoint{1.732428in}{1.774187in}}%
\pgfpathclose%
\pgfusepath{fill}%
\end{pgfscope}%
\begin{pgfscope}%
\pgfpathrectangle{\pgfqpoint{0.697024in}{0.857143in}}{\pgfqpoint{2.627103in}{1.813434in}}%
\pgfusepath{clip}%
\pgfsetbuttcap%
\pgfsetmiterjoin%
\definecolor{currentfill}{rgb}{0.133298,0.375282,0.379395}%
\pgfsetfillcolor{currentfill}%
\pgfsetlinewidth{0.000000pt}%
\definecolor{currentstroke}{rgb}{0.000000,0.000000,0.000000}%
\pgfsetstrokecolor{currentstroke}%
\pgfsetstrokeopacity{0.000000}%
\pgfsetdash{}{0pt}%
\pgfpathmoveto{\pgfqpoint{1.743598in}{1.781250in}}%
\pgfpathlineto{\pgfqpoint{1.752535in}{1.781250in}}%
\pgfpathlineto{\pgfqpoint{1.752535in}{1.582261in}}%
\pgfpathlineto{\pgfqpoint{1.743598in}{1.582261in}}%
\pgfpathlineto{\pgfqpoint{1.743598in}{1.781250in}}%
\pgfpathclose%
\pgfusepath{fill}%
\end{pgfscope}%
\begin{pgfscope}%
\pgfpathrectangle{\pgfqpoint{0.697024in}{0.857143in}}{\pgfqpoint{2.627103in}{1.813434in}}%
\pgfusepath{clip}%
\pgfsetbuttcap%
\pgfsetmiterjoin%
\definecolor{currentfill}{rgb}{0.133298,0.375282,0.379395}%
\pgfsetfillcolor{currentfill}%
\pgfsetlinewidth{0.000000pt}%
\definecolor{currentstroke}{rgb}{0.000000,0.000000,0.000000}%
\pgfsetstrokecolor{currentstroke}%
\pgfsetstrokeopacity{0.000000}%
\pgfsetdash{}{0pt}%
\pgfpathmoveto{\pgfqpoint{1.754769in}{1.775403in}}%
\pgfpathlineto{\pgfqpoint{1.763705in}{1.775403in}}%
\pgfpathlineto{\pgfqpoint{1.763705in}{1.532107in}}%
\pgfpathlineto{\pgfqpoint{1.754769in}{1.532107in}}%
\pgfpathlineto{\pgfqpoint{1.754769in}{1.775403in}}%
\pgfpathclose%
\pgfusepath{fill}%
\end{pgfscope}%
\begin{pgfscope}%
\pgfpathrectangle{\pgfqpoint{0.697024in}{0.857143in}}{\pgfqpoint{2.627103in}{1.813434in}}%
\pgfusepath{clip}%
\pgfsetbuttcap%
\pgfsetmiterjoin%
\definecolor{currentfill}{rgb}{0.133298,0.375282,0.379395}%
\pgfsetfillcolor{currentfill}%
\pgfsetlinewidth{0.000000pt}%
\definecolor{currentstroke}{rgb}{0.000000,0.000000,0.000000}%
\pgfsetstrokecolor{currentstroke}%
\pgfsetstrokeopacity{0.000000}%
\pgfsetdash{}{0pt}%
\pgfpathmoveto{\pgfqpoint{1.765940in}{1.784828in}}%
\pgfpathlineto{\pgfqpoint{1.774876in}{1.784828in}}%
\pgfpathlineto{\pgfqpoint{1.774876in}{1.580443in}}%
\pgfpathlineto{\pgfqpoint{1.765940in}{1.580443in}}%
\pgfpathlineto{\pgfqpoint{1.765940in}{1.784828in}}%
\pgfpathclose%
\pgfusepath{fill}%
\end{pgfscope}%
\begin{pgfscope}%
\pgfpathrectangle{\pgfqpoint{0.697024in}{0.857143in}}{\pgfqpoint{2.627103in}{1.813434in}}%
\pgfusepath{clip}%
\pgfsetbuttcap%
\pgfsetmiterjoin%
\definecolor{currentfill}{rgb}{0.133298,0.375282,0.379395}%
\pgfsetfillcolor{currentfill}%
\pgfsetlinewidth{0.000000pt}%
\definecolor{currentstroke}{rgb}{0.000000,0.000000,0.000000}%
\pgfsetstrokecolor{currentstroke}%
\pgfsetstrokeopacity{0.000000}%
\pgfsetdash{}{0pt}%
\pgfpathmoveto{\pgfqpoint{1.777110in}{1.784930in}}%
\pgfpathlineto{\pgfqpoint{1.786047in}{1.784930in}}%
\pgfpathlineto{\pgfqpoint{1.786047in}{1.559847in}}%
\pgfpathlineto{\pgfqpoint{1.777110in}{1.559847in}}%
\pgfpathlineto{\pgfqpoint{1.777110in}{1.784930in}}%
\pgfpathclose%
\pgfusepath{fill}%
\end{pgfscope}%
\begin{pgfscope}%
\pgfpathrectangle{\pgfqpoint{0.697024in}{0.857143in}}{\pgfqpoint{2.627103in}{1.813434in}}%
\pgfusepath{clip}%
\pgfsetbuttcap%
\pgfsetmiterjoin%
\definecolor{currentfill}{rgb}{0.133298,0.375282,0.379395}%
\pgfsetfillcolor{currentfill}%
\pgfsetlinewidth{0.000000pt}%
\definecolor{currentstroke}{rgb}{0.000000,0.000000,0.000000}%
\pgfsetstrokecolor{currentstroke}%
\pgfsetstrokeopacity{0.000000}%
\pgfsetdash{}{0pt}%
\pgfpathmoveto{\pgfqpoint{1.788281in}{1.769753in}}%
\pgfpathlineto{\pgfqpoint{1.797217in}{1.769753in}}%
\pgfpathlineto{\pgfqpoint{1.797217in}{1.485741in}}%
\pgfpathlineto{\pgfqpoint{1.788281in}{1.485741in}}%
\pgfpathlineto{\pgfqpoint{1.788281in}{1.769753in}}%
\pgfpathclose%
\pgfusepath{fill}%
\end{pgfscope}%
\begin{pgfscope}%
\pgfpathrectangle{\pgfqpoint{0.697024in}{0.857143in}}{\pgfqpoint{2.627103in}{1.813434in}}%
\pgfusepath{clip}%
\pgfsetbuttcap%
\pgfsetmiterjoin%
\definecolor{currentfill}{rgb}{0.133298,0.375282,0.379395}%
\pgfsetfillcolor{currentfill}%
\pgfsetlinewidth{0.000000pt}%
\definecolor{currentstroke}{rgb}{0.000000,0.000000,0.000000}%
\pgfsetstrokecolor{currentstroke}%
\pgfsetstrokeopacity{0.000000}%
\pgfsetdash{}{0pt}%
\pgfpathmoveto{\pgfqpoint{1.799451in}{1.768824in}}%
\pgfpathlineto{\pgfqpoint{1.808388in}{1.768824in}}%
\pgfpathlineto{\pgfqpoint{1.808388in}{1.439748in}}%
\pgfpathlineto{\pgfqpoint{1.799451in}{1.439748in}}%
\pgfpathlineto{\pgfqpoint{1.799451in}{1.768824in}}%
\pgfpathclose%
\pgfusepath{fill}%
\end{pgfscope}%
\begin{pgfscope}%
\pgfpathrectangle{\pgfqpoint{0.697024in}{0.857143in}}{\pgfqpoint{2.627103in}{1.813434in}}%
\pgfusepath{clip}%
\pgfsetbuttcap%
\pgfsetmiterjoin%
\definecolor{currentfill}{rgb}{0.133298,0.375282,0.379395}%
\pgfsetfillcolor{currentfill}%
\pgfsetlinewidth{0.000000pt}%
\definecolor{currentstroke}{rgb}{0.000000,0.000000,0.000000}%
\pgfsetstrokecolor{currentstroke}%
\pgfsetstrokeopacity{0.000000}%
\pgfsetdash{}{0pt}%
\pgfpathmoveto{\pgfqpoint{1.810622in}{1.763873in}}%
\pgfpathlineto{\pgfqpoint{1.819559in}{1.763873in}}%
\pgfpathlineto{\pgfqpoint{1.819559in}{1.550517in}}%
\pgfpathlineto{\pgfqpoint{1.810622in}{1.550517in}}%
\pgfpathlineto{\pgfqpoint{1.810622in}{1.763873in}}%
\pgfpathclose%
\pgfusepath{fill}%
\end{pgfscope}%
\begin{pgfscope}%
\pgfpathrectangle{\pgfqpoint{0.697024in}{0.857143in}}{\pgfqpoint{2.627103in}{1.813434in}}%
\pgfusepath{clip}%
\pgfsetbuttcap%
\pgfsetmiterjoin%
\definecolor{currentfill}{rgb}{0.133298,0.375282,0.379395}%
\pgfsetfillcolor{currentfill}%
\pgfsetlinewidth{0.000000pt}%
\definecolor{currentstroke}{rgb}{0.000000,0.000000,0.000000}%
\pgfsetstrokecolor{currentstroke}%
\pgfsetstrokeopacity{0.000000}%
\pgfsetdash{}{0pt}%
\pgfpathmoveto{\pgfqpoint{1.821793in}{1.753261in}}%
\pgfpathlineto{\pgfqpoint{1.830729in}{1.753261in}}%
\pgfpathlineto{\pgfqpoint{1.830729in}{1.491584in}}%
\pgfpathlineto{\pgfqpoint{1.821793in}{1.491584in}}%
\pgfpathlineto{\pgfqpoint{1.821793in}{1.753261in}}%
\pgfpathclose%
\pgfusepath{fill}%
\end{pgfscope}%
\begin{pgfscope}%
\pgfpathrectangle{\pgfqpoint{0.697024in}{0.857143in}}{\pgfqpoint{2.627103in}{1.813434in}}%
\pgfusepath{clip}%
\pgfsetbuttcap%
\pgfsetmiterjoin%
\definecolor{currentfill}{rgb}{0.133298,0.375282,0.379395}%
\pgfsetfillcolor{currentfill}%
\pgfsetlinewidth{0.000000pt}%
\definecolor{currentstroke}{rgb}{0.000000,0.000000,0.000000}%
\pgfsetstrokecolor{currentstroke}%
\pgfsetstrokeopacity{0.000000}%
\pgfsetdash{}{0pt}%
\pgfpathmoveto{\pgfqpoint{1.832963in}{1.754186in}}%
\pgfpathlineto{\pgfqpoint{1.841900in}{1.754186in}}%
\pgfpathlineto{\pgfqpoint{1.841900in}{1.493008in}}%
\pgfpathlineto{\pgfqpoint{1.832963in}{1.493008in}}%
\pgfpathlineto{\pgfqpoint{1.832963in}{1.754186in}}%
\pgfpathclose%
\pgfusepath{fill}%
\end{pgfscope}%
\begin{pgfscope}%
\pgfpathrectangle{\pgfqpoint{0.697024in}{0.857143in}}{\pgfqpoint{2.627103in}{1.813434in}}%
\pgfusepath{clip}%
\pgfsetbuttcap%
\pgfsetmiterjoin%
\definecolor{currentfill}{rgb}{0.133298,0.375282,0.379395}%
\pgfsetfillcolor{currentfill}%
\pgfsetlinewidth{0.000000pt}%
\definecolor{currentstroke}{rgb}{0.000000,0.000000,0.000000}%
\pgfsetstrokecolor{currentstroke}%
\pgfsetstrokeopacity{0.000000}%
\pgfsetdash{}{0pt}%
\pgfpathmoveto{\pgfqpoint{1.844134in}{1.758191in}}%
\pgfpathlineto{\pgfqpoint{1.853070in}{1.758191in}}%
\pgfpathlineto{\pgfqpoint{1.853070in}{1.549358in}}%
\pgfpathlineto{\pgfqpoint{1.844134in}{1.549358in}}%
\pgfpathlineto{\pgfqpoint{1.844134in}{1.758191in}}%
\pgfpathclose%
\pgfusepath{fill}%
\end{pgfscope}%
\begin{pgfscope}%
\pgfpathrectangle{\pgfqpoint{0.697024in}{0.857143in}}{\pgfqpoint{2.627103in}{1.813434in}}%
\pgfusepath{clip}%
\pgfsetbuttcap%
\pgfsetmiterjoin%
\definecolor{currentfill}{rgb}{0.133298,0.375282,0.379395}%
\pgfsetfillcolor{currentfill}%
\pgfsetlinewidth{0.000000pt}%
\definecolor{currentstroke}{rgb}{0.000000,0.000000,0.000000}%
\pgfsetstrokecolor{currentstroke}%
\pgfsetstrokeopacity{0.000000}%
\pgfsetdash{}{0pt}%
\pgfpathmoveto{\pgfqpoint{1.855304in}{1.751687in}}%
\pgfpathlineto{\pgfqpoint{1.864241in}{1.751687in}}%
\pgfpathlineto{\pgfqpoint{1.864241in}{1.534914in}}%
\pgfpathlineto{\pgfqpoint{1.855304in}{1.534914in}}%
\pgfpathlineto{\pgfqpoint{1.855304in}{1.751687in}}%
\pgfpathclose%
\pgfusepath{fill}%
\end{pgfscope}%
\begin{pgfscope}%
\pgfpathrectangle{\pgfqpoint{0.697024in}{0.857143in}}{\pgfqpoint{2.627103in}{1.813434in}}%
\pgfusepath{clip}%
\pgfsetbuttcap%
\pgfsetmiterjoin%
\definecolor{currentfill}{rgb}{0.133298,0.375282,0.379395}%
\pgfsetfillcolor{currentfill}%
\pgfsetlinewidth{0.000000pt}%
\definecolor{currentstroke}{rgb}{0.000000,0.000000,0.000000}%
\pgfsetstrokecolor{currentstroke}%
\pgfsetstrokeopacity{0.000000}%
\pgfsetdash{}{0pt}%
\pgfpathmoveto{\pgfqpoint{1.866475in}{1.764080in}}%
\pgfpathlineto{\pgfqpoint{1.875412in}{1.764080in}}%
\pgfpathlineto{\pgfqpoint{1.875412in}{1.437789in}}%
\pgfpathlineto{\pgfqpoint{1.866475in}{1.437789in}}%
\pgfpathlineto{\pgfqpoint{1.866475in}{1.764080in}}%
\pgfpathclose%
\pgfusepath{fill}%
\end{pgfscope}%
\begin{pgfscope}%
\pgfpathrectangle{\pgfqpoint{0.697024in}{0.857143in}}{\pgfqpoint{2.627103in}{1.813434in}}%
\pgfusepath{clip}%
\pgfsetbuttcap%
\pgfsetmiterjoin%
\definecolor{currentfill}{rgb}{0.133298,0.375282,0.379395}%
\pgfsetfillcolor{currentfill}%
\pgfsetlinewidth{0.000000pt}%
\definecolor{currentstroke}{rgb}{0.000000,0.000000,0.000000}%
\pgfsetstrokecolor{currentstroke}%
\pgfsetstrokeopacity{0.000000}%
\pgfsetdash{}{0pt}%
\pgfpathmoveto{\pgfqpoint{1.877646in}{1.774717in}}%
\pgfpathlineto{\pgfqpoint{1.886582in}{1.774717in}}%
\pgfpathlineto{\pgfqpoint{1.886582in}{1.416682in}}%
\pgfpathlineto{\pgfqpoint{1.877646in}{1.416682in}}%
\pgfpathlineto{\pgfqpoint{1.877646in}{1.774717in}}%
\pgfpathclose%
\pgfusepath{fill}%
\end{pgfscope}%
\begin{pgfscope}%
\pgfpathrectangle{\pgfqpoint{0.697024in}{0.857143in}}{\pgfqpoint{2.627103in}{1.813434in}}%
\pgfusepath{clip}%
\pgfsetbuttcap%
\pgfsetmiterjoin%
\definecolor{currentfill}{rgb}{0.133298,0.375282,0.379395}%
\pgfsetfillcolor{currentfill}%
\pgfsetlinewidth{0.000000pt}%
\definecolor{currentstroke}{rgb}{0.000000,0.000000,0.000000}%
\pgfsetstrokecolor{currentstroke}%
\pgfsetstrokeopacity{0.000000}%
\pgfsetdash{}{0pt}%
\pgfpathmoveto{\pgfqpoint{1.888816in}{1.778016in}}%
\pgfpathlineto{\pgfqpoint{1.897753in}{1.778016in}}%
\pgfpathlineto{\pgfqpoint{1.897753in}{1.478087in}}%
\pgfpathlineto{\pgfqpoint{1.888816in}{1.478087in}}%
\pgfpathlineto{\pgfqpoint{1.888816in}{1.778016in}}%
\pgfpathclose%
\pgfusepath{fill}%
\end{pgfscope}%
\begin{pgfscope}%
\pgfpathrectangle{\pgfqpoint{0.697024in}{0.857143in}}{\pgfqpoint{2.627103in}{1.813434in}}%
\pgfusepath{clip}%
\pgfsetbuttcap%
\pgfsetmiterjoin%
\definecolor{currentfill}{rgb}{0.133298,0.375282,0.379395}%
\pgfsetfillcolor{currentfill}%
\pgfsetlinewidth{0.000000pt}%
\definecolor{currentstroke}{rgb}{0.000000,0.000000,0.000000}%
\pgfsetstrokecolor{currentstroke}%
\pgfsetstrokeopacity{0.000000}%
\pgfsetdash{}{0pt}%
\pgfpathmoveto{\pgfqpoint{1.899987in}{1.749890in}}%
\pgfpathlineto{\pgfqpoint{1.908923in}{1.749890in}}%
\pgfpathlineto{\pgfqpoint{1.908923in}{1.456659in}}%
\pgfpathlineto{\pgfqpoint{1.899987in}{1.456659in}}%
\pgfpathlineto{\pgfqpoint{1.899987in}{1.749890in}}%
\pgfpathclose%
\pgfusepath{fill}%
\end{pgfscope}%
\begin{pgfscope}%
\pgfpathrectangle{\pgfqpoint{0.697024in}{0.857143in}}{\pgfqpoint{2.627103in}{1.813434in}}%
\pgfusepath{clip}%
\pgfsetbuttcap%
\pgfsetmiterjoin%
\definecolor{currentfill}{rgb}{0.133298,0.375282,0.379395}%
\pgfsetfillcolor{currentfill}%
\pgfsetlinewidth{0.000000pt}%
\definecolor{currentstroke}{rgb}{0.000000,0.000000,0.000000}%
\pgfsetstrokecolor{currentstroke}%
\pgfsetstrokeopacity{0.000000}%
\pgfsetdash{}{0pt}%
\pgfpathmoveto{\pgfqpoint{1.911157in}{1.756017in}}%
\pgfpathlineto{\pgfqpoint{1.920094in}{1.756017in}}%
\pgfpathlineto{\pgfqpoint{1.920094in}{1.350132in}}%
\pgfpathlineto{\pgfqpoint{1.911157in}{1.350132in}}%
\pgfpathlineto{\pgfqpoint{1.911157in}{1.756017in}}%
\pgfpathclose%
\pgfusepath{fill}%
\end{pgfscope}%
\begin{pgfscope}%
\pgfpathrectangle{\pgfqpoint{0.697024in}{0.857143in}}{\pgfqpoint{2.627103in}{1.813434in}}%
\pgfusepath{clip}%
\pgfsetbuttcap%
\pgfsetmiterjoin%
\definecolor{currentfill}{rgb}{0.133298,0.375282,0.379395}%
\pgfsetfillcolor{currentfill}%
\pgfsetlinewidth{0.000000pt}%
\definecolor{currentstroke}{rgb}{0.000000,0.000000,0.000000}%
\pgfsetstrokecolor{currentstroke}%
\pgfsetstrokeopacity{0.000000}%
\pgfsetdash{}{0pt}%
\pgfpathmoveto{\pgfqpoint{1.922328in}{1.786759in}}%
\pgfpathlineto{\pgfqpoint{1.931265in}{1.786759in}}%
\pgfpathlineto{\pgfqpoint{1.931265in}{1.417943in}}%
\pgfpathlineto{\pgfqpoint{1.922328in}{1.417943in}}%
\pgfpathlineto{\pgfqpoint{1.922328in}{1.786759in}}%
\pgfpathclose%
\pgfusepath{fill}%
\end{pgfscope}%
\begin{pgfscope}%
\pgfpathrectangle{\pgfqpoint{0.697024in}{0.857143in}}{\pgfqpoint{2.627103in}{1.813434in}}%
\pgfusepath{clip}%
\pgfsetbuttcap%
\pgfsetmiterjoin%
\definecolor{currentfill}{rgb}{0.133298,0.375282,0.379395}%
\pgfsetfillcolor{currentfill}%
\pgfsetlinewidth{0.000000pt}%
\definecolor{currentstroke}{rgb}{0.000000,0.000000,0.000000}%
\pgfsetstrokecolor{currentstroke}%
\pgfsetstrokeopacity{0.000000}%
\pgfsetdash{}{0pt}%
\pgfpathmoveto{\pgfqpoint{1.933499in}{1.789748in}}%
\pgfpathlineto{\pgfqpoint{1.942435in}{1.789748in}}%
\pgfpathlineto{\pgfqpoint{1.942435in}{1.370585in}}%
\pgfpathlineto{\pgfqpoint{1.933499in}{1.370585in}}%
\pgfpathlineto{\pgfqpoint{1.933499in}{1.789748in}}%
\pgfpathclose%
\pgfusepath{fill}%
\end{pgfscope}%
\begin{pgfscope}%
\pgfpathrectangle{\pgfqpoint{0.697024in}{0.857143in}}{\pgfqpoint{2.627103in}{1.813434in}}%
\pgfusepath{clip}%
\pgfsetbuttcap%
\pgfsetmiterjoin%
\definecolor{currentfill}{rgb}{0.133298,0.375282,0.379395}%
\pgfsetfillcolor{currentfill}%
\pgfsetlinewidth{0.000000pt}%
\definecolor{currentstroke}{rgb}{0.000000,0.000000,0.000000}%
\pgfsetstrokecolor{currentstroke}%
\pgfsetstrokeopacity{0.000000}%
\pgfsetdash{}{0pt}%
\pgfpathmoveto{\pgfqpoint{1.944669in}{1.791044in}}%
\pgfpathlineto{\pgfqpoint{1.953606in}{1.791044in}}%
\pgfpathlineto{\pgfqpoint{1.953606in}{1.367968in}}%
\pgfpathlineto{\pgfqpoint{1.944669in}{1.367968in}}%
\pgfpathlineto{\pgfqpoint{1.944669in}{1.791044in}}%
\pgfpathclose%
\pgfusepath{fill}%
\end{pgfscope}%
\begin{pgfscope}%
\pgfpathrectangle{\pgfqpoint{0.697024in}{0.857143in}}{\pgfqpoint{2.627103in}{1.813434in}}%
\pgfusepath{clip}%
\pgfsetbuttcap%
\pgfsetmiterjoin%
\definecolor{currentfill}{rgb}{0.133298,0.375282,0.379395}%
\pgfsetfillcolor{currentfill}%
\pgfsetlinewidth{0.000000pt}%
\definecolor{currentstroke}{rgb}{0.000000,0.000000,0.000000}%
\pgfsetstrokecolor{currentstroke}%
\pgfsetstrokeopacity{0.000000}%
\pgfsetdash{}{0pt}%
\pgfpathmoveto{\pgfqpoint{1.955840in}{1.831225in}}%
\pgfpathlineto{\pgfqpoint{1.964776in}{1.831225in}}%
\pgfpathlineto{\pgfqpoint{1.964776in}{1.433285in}}%
\pgfpathlineto{\pgfqpoint{1.955840in}{1.433285in}}%
\pgfpathlineto{\pgfqpoint{1.955840in}{1.831225in}}%
\pgfpathclose%
\pgfusepath{fill}%
\end{pgfscope}%
\begin{pgfscope}%
\pgfpathrectangle{\pgfqpoint{0.697024in}{0.857143in}}{\pgfqpoint{2.627103in}{1.813434in}}%
\pgfusepath{clip}%
\pgfsetbuttcap%
\pgfsetmiterjoin%
\definecolor{currentfill}{rgb}{0.133298,0.375282,0.379395}%
\pgfsetfillcolor{currentfill}%
\pgfsetlinewidth{0.000000pt}%
\definecolor{currentstroke}{rgb}{0.000000,0.000000,0.000000}%
\pgfsetstrokecolor{currentstroke}%
\pgfsetstrokeopacity{0.000000}%
\pgfsetdash{}{0pt}%
\pgfpathmoveto{\pgfqpoint{1.967011in}{1.834262in}}%
\pgfpathlineto{\pgfqpoint{1.975947in}{1.834262in}}%
\pgfpathlineto{\pgfqpoint{1.975947in}{1.363036in}}%
\pgfpathlineto{\pgfqpoint{1.967011in}{1.363036in}}%
\pgfpathlineto{\pgfqpoint{1.967011in}{1.834262in}}%
\pgfpathclose%
\pgfusepath{fill}%
\end{pgfscope}%
\begin{pgfscope}%
\pgfpathrectangle{\pgfqpoint{0.697024in}{0.857143in}}{\pgfqpoint{2.627103in}{1.813434in}}%
\pgfusepath{clip}%
\pgfsetbuttcap%
\pgfsetmiterjoin%
\definecolor{currentfill}{rgb}{0.133298,0.375282,0.379395}%
\pgfsetfillcolor{currentfill}%
\pgfsetlinewidth{0.000000pt}%
\definecolor{currentstroke}{rgb}{0.000000,0.000000,0.000000}%
\pgfsetstrokecolor{currentstroke}%
\pgfsetstrokeopacity{0.000000}%
\pgfsetdash{}{0pt}%
\pgfpathmoveto{\pgfqpoint{1.978181in}{1.843485in}}%
\pgfpathlineto{\pgfqpoint{1.987118in}{1.843485in}}%
\pgfpathlineto{\pgfqpoint{1.987118in}{1.383499in}}%
\pgfpathlineto{\pgfqpoint{1.978181in}{1.383499in}}%
\pgfpathlineto{\pgfqpoint{1.978181in}{1.843485in}}%
\pgfpathclose%
\pgfusepath{fill}%
\end{pgfscope}%
\begin{pgfscope}%
\pgfpathrectangle{\pgfqpoint{0.697024in}{0.857143in}}{\pgfqpoint{2.627103in}{1.813434in}}%
\pgfusepath{clip}%
\pgfsetbuttcap%
\pgfsetmiterjoin%
\definecolor{currentfill}{rgb}{0.133298,0.375282,0.379395}%
\pgfsetfillcolor{currentfill}%
\pgfsetlinewidth{0.000000pt}%
\definecolor{currentstroke}{rgb}{0.000000,0.000000,0.000000}%
\pgfsetstrokecolor{currentstroke}%
\pgfsetstrokeopacity{0.000000}%
\pgfsetdash{}{0pt}%
\pgfpathmoveto{\pgfqpoint{1.989352in}{1.840641in}}%
\pgfpathlineto{\pgfqpoint{1.998288in}{1.840641in}}%
\pgfpathlineto{\pgfqpoint{1.998288in}{1.322762in}}%
\pgfpathlineto{\pgfqpoint{1.989352in}{1.322762in}}%
\pgfpathlineto{\pgfqpoint{1.989352in}{1.840641in}}%
\pgfpathclose%
\pgfusepath{fill}%
\end{pgfscope}%
\begin{pgfscope}%
\pgfpathrectangle{\pgfqpoint{0.697024in}{0.857143in}}{\pgfqpoint{2.627103in}{1.813434in}}%
\pgfusepath{clip}%
\pgfsetbuttcap%
\pgfsetmiterjoin%
\definecolor{currentfill}{rgb}{0.133298,0.375282,0.379395}%
\pgfsetfillcolor{currentfill}%
\pgfsetlinewidth{0.000000pt}%
\definecolor{currentstroke}{rgb}{0.000000,0.000000,0.000000}%
\pgfsetstrokecolor{currentstroke}%
\pgfsetstrokeopacity{0.000000}%
\pgfsetdash{}{0pt}%
\pgfpathmoveto{\pgfqpoint{2.000522in}{1.806091in}}%
\pgfpathlineto{\pgfqpoint{2.009459in}{1.806091in}}%
\pgfpathlineto{\pgfqpoint{2.009459in}{1.362698in}}%
\pgfpathlineto{\pgfqpoint{2.000522in}{1.362698in}}%
\pgfpathlineto{\pgfqpoint{2.000522in}{1.806091in}}%
\pgfpathclose%
\pgfusepath{fill}%
\end{pgfscope}%
\begin{pgfscope}%
\pgfpathrectangle{\pgfqpoint{0.697024in}{0.857143in}}{\pgfqpoint{2.627103in}{1.813434in}}%
\pgfusepath{clip}%
\pgfsetbuttcap%
\pgfsetmiterjoin%
\definecolor{currentfill}{rgb}{0.133298,0.375282,0.379395}%
\pgfsetfillcolor{currentfill}%
\pgfsetlinewidth{0.000000pt}%
\definecolor{currentstroke}{rgb}{0.000000,0.000000,0.000000}%
\pgfsetstrokecolor{currentstroke}%
\pgfsetstrokeopacity{0.000000}%
\pgfsetdash{}{0pt}%
\pgfpathmoveto{\pgfqpoint{2.011693in}{1.781540in}}%
\pgfpathlineto{\pgfqpoint{2.020629in}{1.781540in}}%
\pgfpathlineto{\pgfqpoint{2.020629in}{1.309497in}}%
\pgfpathlineto{\pgfqpoint{2.011693in}{1.309497in}}%
\pgfpathlineto{\pgfqpoint{2.011693in}{1.781540in}}%
\pgfpathclose%
\pgfusepath{fill}%
\end{pgfscope}%
\begin{pgfscope}%
\pgfpathrectangle{\pgfqpoint{0.697024in}{0.857143in}}{\pgfqpoint{2.627103in}{1.813434in}}%
\pgfusepath{clip}%
\pgfsetbuttcap%
\pgfsetmiterjoin%
\definecolor{currentfill}{rgb}{0.133298,0.375282,0.379395}%
\pgfsetfillcolor{currentfill}%
\pgfsetlinewidth{0.000000pt}%
\definecolor{currentstroke}{rgb}{0.000000,0.000000,0.000000}%
\pgfsetstrokecolor{currentstroke}%
\pgfsetstrokeopacity{0.000000}%
\pgfsetdash{}{0pt}%
\pgfpathmoveto{\pgfqpoint{2.022864in}{1.770905in}}%
\pgfpathlineto{\pgfqpoint{2.031800in}{1.770905in}}%
\pgfpathlineto{\pgfqpoint{2.031800in}{1.306402in}}%
\pgfpathlineto{\pgfqpoint{2.022864in}{1.306402in}}%
\pgfpathlineto{\pgfqpoint{2.022864in}{1.770905in}}%
\pgfpathclose%
\pgfusepath{fill}%
\end{pgfscope}%
\begin{pgfscope}%
\pgfpathrectangle{\pgfqpoint{0.697024in}{0.857143in}}{\pgfqpoint{2.627103in}{1.813434in}}%
\pgfusepath{clip}%
\pgfsetbuttcap%
\pgfsetmiterjoin%
\definecolor{currentfill}{rgb}{0.133298,0.375282,0.379395}%
\pgfsetfillcolor{currentfill}%
\pgfsetlinewidth{0.000000pt}%
\definecolor{currentstroke}{rgb}{0.000000,0.000000,0.000000}%
\pgfsetstrokecolor{currentstroke}%
\pgfsetstrokeopacity{0.000000}%
\pgfsetdash{}{0pt}%
\pgfpathmoveto{\pgfqpoint{2.034034in}{1.774830in}}%
\pgfpathlineto{\pgfqpoint{2.042971in}{1.774830in}}%
\pgfpathlineto{\pgfqpoint{2.042971in}{1.323674in}}%
\pgfpathlineto{\pgfqpoint{2.034034in}{1.323674in}}%
\pgfpathlineto{\pgfqpoint{2.034034in}{1.774830in}}%
\pgfpathclose%
\pgfusepath{fill}%
\end{pgfscope}%
\begin{pgfscope}%
\pgfpathrectangle{\pgfqpoint{0.697024in}{0.857143in}}{\pgfqpoint{2.627103in}{1.813434in}}%
\pgfusepath{clip}%
\pgfsetbuttcap%
\pgfsetmiterjoin%
\definecolor{currentfill}{rgb}{0.133298,0.375282,0.379395}%
\pgfsetfillcolor{currentfill}%
\pgfsetlinewidth{0.000000pt}%
\definecolor{currentstroke}{rgb}{0.000000,0.000000,0.000000}%
\pgfsetstrokecolor{currentstroke}%
\pgfsetstrokeopacity{0.000000}%
\pgfsetdash{}{0pt}%
\pgfpathmoveto{\pgfqpoint{2.045205in}{1.774470in}}%
\pgfpathlineto{\pgfqpoint{2.054141in}{1.774470in}}%
\pgfpathlineto{\pgfqpoint{2.054141in}{1.358504in}}%
\pgfpathlineto{\pgfqpoint{2.045205in}{1.358504in}}%
\pgfpathlineto{\pgfqpoint{2.045205in}{1.774470in}}%
\pgfpathclose%
\pgfusepath{fill}%
\end{pgfscope}%
\begin{pgfscope}%
\pgfpathrectangle{\pgfqpoint{0.697024in}{0.857143in}}{\pgfqpoint{2.627103in}{1.813434in}}%
\pgfusepath{clip}%
\pgfsetbuttcap%
\pgfsetmiterjoin%
\definecolor{currentfill}{rgb}{0.133298,0.375282,0.379395}%
\pgfsetfillcolor{currentfill}%
\pgfsetlinewidth{0.000000pt}%
\definecolor{currentstroke}{rgb}{0.000000,0.000000,0.000000}%
\pgfsetstrokecolor{currentstroke}%
\pgfsetstrokeopacity{0.000000}%
\pgfsetdash{}{0pt}%
\pgfpathmoveto{\pgfqpoint{2.056375in}{1.755294in}}%
\pgfpathlineto{\pgfqpoint{2.065312in}{1.755294in}}%
\pgfpathlineto{\pgfqpoint{2.065312in}{1.336991in}}%
\pgfpathlineto{\pgfqpoint{2.056375in}{1.336991in}}%
\pgfpathlineto{\pgfqpoint{2.056375in}{1.755294in}}%
\pgfpathclose%
\pgfusepath{fill}%
\end{pgfscope}%
\begin{pgfscope}%
\pgfpathrectangle{\pgfqpoint{0.697024in}{0.857143in}}{\pgfqpoint{2.627103in}{1.813434in}}%
\pgfusepath{clip}%
\pgfsetbuttcap%
\pgfsetmiterjoin%
\definecolor{currentfill}{rgb}{0.133298,0.375282,0.379395}%
\pgfsetfillcolor{currentfill}%
\pgfsetlinewidth{0.000000pt}%
\definecolor{currentstroke}{rgb}{0.000000,0.000000,0.000000}%
\pgfsetstrokecolor{currentstroke}%
\pgfsetstrokeopacity{0.000000}%
\pgfsetdash{}{0pt}%
\pgfpathmoveto{\pgfqpoint{2.067546in}{1.728389in}}%
\pgfpathlineto{\pgfqpoint{2.076482in}{1.728389in}}%
\pgfpathlineto{\pgfqpoint{2.076482in}{1.321007in}}%
\pgfpathlineto{\pgfqpoint{2.067546in}{1.321007in}}%
\pgfpathlineto{\pgfqpoint{2.067546in}{1.728389in}}%
\pgfpathclose%
\pgfusepath{fill}%
\end{pgfscope}%
\begin{pgfscope}%
\pgfpathrectangle{\pgfqpoint{0.697024in}{0.857143in}}{\pgfqpoint{2.627103in}{1.813434in}}%
\pgfusepath{clip}%
\pgfsetbuttcap%
\pgfsetmiterjoin%
\definecolor{currentfill}{rgb}{0.133298,0.375282,0.379395}%
\pgfsetfillcolor{currentfill}%
\pgfsetlinewidth{0.000000pt}%
\definecolor{currentstroke}{rgb}{0.000000,0.000000,0.000000}%
\pgfsetstrokecolor{currentstroke}%
\pgfsetstrokeopacity{0.000000}%
\pgfsetdash{}{0pt}%
\pgfpathmoveto{\pgfqpoint{2.078717in}{1.742542in}}%
\pgfpathlineto{\pgfqpoint{2.087653in}{1.742542in}}%
\pgfpathlineto{\pgfqpoint{2.087653in}{1.341642in}}%
\pgfpathlineto{\pgfqpoint{2.078717in}{1.341642in}}%
\pgfpathlineto{\pgfqpoint{2.078717in}{1.742542in}}%
\pgfpathclose%
\pgfusepath{fill}%
\end{pgfscope}%
\begin{pgfscope}%
\pgfpathrectangle{\pgfqpoint{0.697024in}{0.857143in}}{\pgfqpoint{2.627103in}{1.813434in}}%
\pgfusepath{clip}%
\pgfsetbuttcap%
\pgfsetmiterjoin%
\definecolor{currentfill}{rgb}{0.133298,0.375282,0.379395}%
\pgfsetfillcolor{currentfill}%
\pgfsetlinewidth{0.000000pt}%
\definecolor{currentstroke}{rgb}{0.000000,0.000000,0.000000}%
\pgfsetstrokecolor{currentstroke}%
\pgfsetstrokeopacity{0.000000}%
\pgfsetdash{}{0pt}%
\pgfpathmoveto{\pgfqpoint{2.089887in}{1.727330in}}%
\pgfpathlineto{\pgfqpoint{2.098824in}{1.727330in}}%
\pgfpathlineto{\pgfqpoint{2.098824in}{1.329493in}}%
\pgfpathlineto{\pgfqpoint{2.089887in}{1.329493in}}%
\pgfpathlineto{\pgfqpoint{2.089887in}{1.727330in}}%
\pgfpathclose%
\pgfusepath{fill}%
\end{pgfscope}%
\begin{pgfscope}%
\pgfpathrectangle{\pgfqpoint{0.697024in}{0.857143in}}{\pgfqpoint{2.627103in}{1.813434in}}%
\pgfusepath{clip}%
\pgfsetbuttcap%
\pgfsetmiterjoin%
\definecolor{currentfill}{rgb}{0.133298,0.375282,0.379395}%
\pgfsetfillcolor{currentfill}%
\pgfsetlinewidth{0.000000pt}%
\definecolor{currentstroke}{rgb}{0.000000,0.000000,0.000000}%
\pgfsetstrokecolor{currentstroke}%
\pgfsetstrokeopacity{0.000000}%
\pgfsetdash{}{0pt}%
\pgfpathmoveto{\pgfqpoint{2.101058in}{1.730596in}}%
\pgfpathlineto{\pgfqpoint{2.109994in}{1.730596in}}%
\pgfpathlineto{\pgfqpoint{2.109994in}{1.363601in}}%
\pgfpathlineto{\pgfqpoint{2.101058in}{1.363601in}}%
\pgfpathlineto{\pgfqpoint{2.101058in}{1.730596in}}%
\pgfpathclose%
\pgfusepath{fill}%
\end{pgfscope}%
\begin{pgfscope}%
\pgfpathrectangle{\pgfqpoint{0.697024in}{0.857143in}}{\pgfqpoint{2.627103in}{1.813434in}}%
\pgfusepath{clip}%
\pgfsetbuttcap%
\pgfsetmiterjoin%
\definecolor{currentfill}{rgb}{0.133298,0.375282,0.379395}%
\pgfsetfillcolor{currentfill}%
\pgfsetlinewidth{0.000000pt}%
\definecolor{currentstroke}{rgb}{0.000000,0.000000,0.000000}%
\pgfsetstrokecolor{currentstroke}%
\pgfsetstrokeopacity{0.000000}%
\pgfsetdash{}{0pt}%
\pgfpathmoveto{\pgfqpoint{2.112228in}{1.710249in}}%
\pgfpathlineto{\pgfqpoint{2.121165in}{1.710249in}}%
\pgfpathlineto{\pgfqpoint{2.121165in}{1.402872in}}%
\pgfpathlineto{\pgfqpoint{2.112228in}{1.402872in}}%
\pgfpathlineto{\pgfqpoint{2.112228in}{1.710249in}}%
\pgfpathclose%
\pgfusepath{fill}%
\end{pgfscope}%
\begin{pgfscope}%
\pgfpathrectangle{\pgfqpoint{0.697024in}{0.857143in}}{\pgfqpoint{2.627103in}{1.813434in}}%
\pgfusepath{clip}%
\pgfsetbuttcap%
\pgfsetmiterjoin%
\definecolor{currentfill}{rgb}{0.133298,0.375282,0.379395}%
\pgfsetfillcolor{currentfill}%
\pgfsetlinewidth{0.000000pt}%
\definecolor{currentstroke}{rgb}{0.000000,0.000000,0.000000}%
\pgfsetstrokecolor{currentstroke}%
\pgfsetstrokeopacity{0.000000}%
\pgfsetdash{}{0pt}%
\pgfpathmoveto{\pgfqpoint{2.123399in}{1.713586in}}%
\pgfpathlineto{\pgfqpoint{2.132335in}{1.713586in}}%
\pgfpathlineto{\pgfqpoint{2.132335in}{1.389063in}}%
\pgfpathlineto{\pgfqpoint{2.123399in}{1.389063in}}%
\pgfpathlineto{\pgfqpoint{2.123399in}{1.713586in}}%
\pgfpathclose%
\pgfusepath{fill}%
\end{pgfscope}%
\begin{pgfscope}%
\pgfpathrectangle{\pgfqpoint{0.697024in}{0.857143in}}{\pgfqpoint{2.627103in}{1.813434in}}%
\pgfusepath{clip}%
\pgfsetbuttcap%
\pgfsetmiterjoin%
\definecolor{currentfill}{rgb}{0.133298,0.375282,0.379395}%
\pgfsetfillcolor{currentfill}%
\pgfsetlinewidth{0.000000pt}%
\definecolor{currentstroke}{rgb}{0.000000,0.000000,0.000000}%
\pgfsetstrokecolor{currentstroke}%
\pgfsetstrokeopacity{0.000000}%
\pgfsetdash{}{0pt}%
\pgfpathmoveto{\pgfqpoint{2.134570in}{1.718886in}}%
\pgfpathlineto{\pgfqpoint{2.143506in}{1.718886in}}%
\pgfpathlineto{\pgfqpoint{2.143506in}{1.408360in}}%
\pgfpathlineto{\pgfqpoint{2.134570in}{1.408360in}}%
\pgfpathlineto{\pgfqpoint{2.134570in}{1.718886in}}%
\pgfpathclose%
\pgfusepath{fill}%
\end{pgfscope}%
\begin{pgfscope}%
\pgfpathrectangle{\pgfqpoint{0.697024in}{0.857143in}}{\pgfqpoint{2.627103in}{1.813434in}}%
\pgfusepath{clip}%
\pgfsetbuttcap%
\pgfsetmiterjoin%
\definecolor{currentfill}{rgb}{0.133298,0.375282,0.379395}%
\pgfsetfillcolor{currentfill}%
\pgfsetlinewidth{0.000000pt}%
\definecolor{currentstroke}{rgb}{0.000000,0.000000,0.000000}%
\pgfsetstrokecolor{currentstroke}%
\pgfsetstrokeopacity{0.000000}%
\pgfsetdash{}{0pt}%
\pgfpathmoveto{\pgfqpoint{2.145740in}{1.728207in}}%
\pgfpathlineto{\pgfqpoint{2.154677in}{1.728207in}}%
\pgfpathlineto{\pgfqpoint{2.154677in}{1.441896in}}%
\pgfpathlineto{\pgfqpoint{2.145740in}{1.441896in}}%
\pgfpathlineto{\pgfqpoint{2.145740in}{1.728207in}}%
\pgfpathclose%
\pgfusepath{fill}%
\end{pgfscope}%
\begin{pgfscope}%
\pgfpathrectangle{\pgfqpoint{0.697024in}{0.857143in}}{\pgfqpoint{2.627103in}{1.813434in}}%
\pgfusepath{clip}%
\pgfsetbuttcap%
\pgfsetmiterjoin%
\definecolor{currentfill}{rgb}{0.133298,0.375282,0.379395}%
\pgfsetfillcolor{currentfill}%
\pgfsetlinewidth{0.000000pt}%
\definecolor{currentstroke}{rgb}{0.000000,0.000000,0.000000}%
\pgfsetstrokecolor{currentstroke}%
\pgfsetstrokeopacity{0.000000}%
\pgfsetdash{}{0pt}%
\pgfpathmoveto{\pgfqpoint{2.156911in}{1.716533in}}%
\pgfpathlineto{\pgfqpoint{2.165847in}{1.716533in}}%
\pgfpathlineto{\pgfqpoint{2.165847in}{1.484292in}}%
\pgfpathlineto{\pgfqpoint{2.156911in}{1.484292in}}%
\pgfpathlineto{\pgfqpoint{2.156911in}{1.716533in}}%
\pgfpathclose%
\pgfusepath{fill}%
\end{pgfscope}%
\begin{pgfscope}%
\pgfpathrectangle{\pgfqpoint{0.697024in}{0.857143in}}{\pgfqpoint{2.627103in}{1.813434in}}%
\pgfusepath{clip}%
\pgfsetbuttcap%
\pgfsetmiterjoin%
\definecolor{currentfill}{rgb}{0.133298,0.375282,0.379395}%
\pgfsetfillcolor{currentfill}%
\pgfsetlinewidth{0.000000pt}%
\definecolor{currentstroke}{rgb}{0.000000,0.000000,0.000000}%
\pgfsetstrokecolor{currentstroke}%
\pgfsetstrokeopacity{0.000000}%
\pgfsetdash{}{0pt}%
\pgfpathmoveto{\pgfqpoint{2.168081in}{1.702186in}}%
\pgfpathlineto{\pgfqpoint{2.177018in}{1.702186in}}%
\pgfpathlineto{\pgfqpoint{2.177018in}{1.437732in}}%
\pgfpathlineto{\pgfqpoint{2.168081in}{1.437732in}}%
\pgfpathlineto{\pgfqpoint{2.168081in}{1.702186in}}%
\pgfpathclose%
\pgfusepath{fill}%
\end{pgfscope}%
\begin{pgfscope}%
\pgfpathrectangle{\pgfqpoint{0.697024in}{0.857143in}}{\pgfqpoint{2.627103in}{1.813434in}}%
\pgfusepath{clip}%
\pgfsetbuttcap%
\pgfsetmiterjoin%
\definecolor{currentfill}{rgb}{0.133298,0.375282,0.379395}%
\pgfsetfillcolor{currentfill}%
\pgfsetlinewidth{0.000000pt}%
\definecolor{currentstroke}{rgb}{0.000000,0.000000,0.000000}%
\pgfsetstrokecolor{currentstroke}%
\pgfsetstrokeopacity{0.000000}%
\pgfsetdash{}{0pt}%
\pgfpathmoveto{\pgfqpoint{2.179252in}{1.686026in}}%
\pgfpathlineto{\pgfqpoint{2.188189in}{1.686026in}}%
\pgfpathlineto{\pgfqpoint{2.188189in}{1.476099in}}%
\pgfpathlineto{\pgfqpoint{2.179252in}{1.476099in}}%
\pgfpathlineto{\pgfqpoint{2.179252in}{1.686026in}}%
\pgfpathclose%
\pgfusepath{fill}%
\end{pgfscope}%
\begin{pgfscope}%
\pgfpathrectangle{\pgfqpoint{0.697024in}{0.857143in}}{\pgfqpoint{2.627103in}{1.813434in}}%
\pgfusepath{clip}%
\pgfsetbuttcap%
\pgfsetmiterjoin%
\definecolor{currentfill}{rgb}{0.133298,0.375282,0.379395}%
\pgfsetfillcolor{currentfill}%
\pgfsetlinewidth{0.000000pt}%
\definecolor{currentstroke}{rgb}{0.000000,0.000000,0.000000}%
\pgfsetstrokecolor{currentstroke}%
\pgfsetstrokeopacity{0.000000}%
\pgfsetdash{}{0pt}%
\pgfpathmoveto{\pgfqpoint{2.190423in}{1.712007in}}%
\pgfpathlineto{\pgfqpoint{2.199359in}{1.712007in}}%
\pgfpathlineto{\pgfqpoint{2.199359in}{1.589298in}}%
\pgfpathlineto{\pgfqpoint{2.190423in}{1.589298in}}%
\pgfpathlineto{\pgfqpoint{2.190423in}{1.712007in}}%
\pgfpathclose%
\pgfusepath{fill}%
\end{pgfscope}%
\begin{pgfscope}%
\pgfpathrectangle{\pgfqpoint{0.697024in}{0.857143in}}{\pgfqpoint{2.627103in}{1.813434in}}%
\pgfusepath{clip}%
\pgfsetbuttcap%
\pgfsetmiterjoin%
\definecolor{currentfill}{rgb}{0.133298,0.375282,0.379395}%
\pgfsetfillcolor{currentfill}%
\pgfsetlinewidth{0.000000pt}%
\definecolor{currentstroke}{rgb}{0.000000,0.000000,0.000000}%
\pgfsetstrokecolor{currentstroke}%
\pgfsetstrokeopacity{0.000000}%
\pgfsetdash{}{0pt}%
\pgfpathmoveto{\pgfqpoint{2.201593in}{1.717555in}}%
\pgfpathlineto{\pgfqpoint{2.210530in}{1.717555in}}%
\pgfpathlineto{\pgfqpoint{2.210530in}{1.568136in}}%
\pgfpathlineto{\pgfqpoint{2.201593in}{1.568136in}}%
\pgfpathlineto{\pgfqpoint{2.201593in}{1.717555in}}%
\pgfpathclose%
\pgfusepath{fill}%
\end{pgfscope}%
\begin{pgfscope}%
\pgfpathrectangle{\pgfqpoint{0.697024in}{0.857143in}}{\pgfqpoint{2.627103in}{1.813434in}}%
\pgfusepath{clip}%
\pgfsetbuttcap%
\pgfsetmiterjoin%
\definecolor{currentfill}{rgb}{0.133298,0.375282,0.379395}%
\pgfsetfillcolor{currentfill}%
\pgfsetlinewidth{0.000000pt}%
\definecolor{currentstroke}{rgb}{0.000000,0.000000,0.000000}%
\pgfsetstrokecolor{currentstroke}%
\pgfsetstrokeopacity{0.000000}%
\pgfsetdash{}{0pt}%
\pgfpathmoveto{\pgfqpoint{2.212764in}{1.715099in}}%
\pgfpathlineto{\pgfqpoint{2.221700in}{1.715099in}}%
\pgfpathlineto{\pgfqpoint{2.221700in}{1.662130in}}%
\pgfpathlineto{\pgfqpoint{2.212764in}{1.662130in}}%
\pgfpathlineto{\pgfqpoint{2.212764in}{1.715099in}}%
\pgfpathclose%
\pgfusepath{fill}%
\end{pgfscope}%
\begin{pgfscope}%
\pgfpathrectangle{\pgfqpoint{0.697024in}{0.857143in}}{\pgfqpoint{2.627103in}{1.813434in}}%
\pgfusepath{clip}%
\pgfsetbuttcap%
\pgfsetmiterjoin%
\definecolor{currentfill}{rgb}{0.133298,0.375282,0.379395}%
\pgfsetfillcolor{currentfill}%
\pgfsetlinewidth{0.000000pt}%
\definecolor{currentstroke}{rgb}{0.000000,0.000000,0.000000}%
\pgfsetstrokecolor{currentstroke}%
\pgfsetstrokeopacity{0.000000}%
\pgfsetdash{}{0pt}%
\pgfpathmoveto{\pgfqpoint{2.223934in}{1.847071in}}%
\pgfpathlineto{\pgfqpoint{2.232871in}{1.847071in}}%
\pgfpathlineto{\pgfqpoint{2.232871in}{1.879406in}}%
\pgfpathlineto{\pgfqpoint{2.223934in}{1.879406in}}%
\pgfpathlineto{\pgfqpoint{2.223934in}{1.847071in}}%
\pgfpathclose%
\pgfusepath{fill}%
\end{pgfscope}%
\begin{pgfscope}%
\pgfpathrectangle{\pgfqpoint{0.697024in}{0.857143in}}{\pgfqpoint{2.627103in}{1.813434in}}%
\pgfusepath{clip}%
\pgfsetbuttcap%
\pgfsetmiterjoin%
\definecolor{currentfill}{rgb}{0.133298,0.375282,0.379395}%
\pgfsetfillcolor{currentfill}%
\pgfsetlinewidth{0.000000pt}%
\definecolor{currentstroke}{rgb}{0.000000,0.000000,0.000000}%
\pgfsetstrokecolor{currentstroke}%
\pgfsetstrokeopacity{0.000000}%
\pgfsetdash{}{0pt}%
\pgfpathmoveto{\pgfqpoint{2.235105in}{1.847071in}}%
\pgfpathlineto{\pgfqpoint{2.244042in}{1.847071in}}%
\pgfpathlineto{\pgfqpoint{2.244042in}{1.900467in}}%
\pgfpathlineto{\pgfqpoint{2.235105in}{1.900467in}}%
\pgfpathlineto{\pgfqpoint{2.235105in}{1.847071in}}%
\pgfpathclose%
\pgfusepath{fill}%
\end{pgfscope}%
\begin{pgfscope}%
\pgfpathrectangle{\pgfqpoint{0.697024in}{0.857143in}}{\pgfqpoint{2.627103in}{1.813434in}}%
\pgfusepath{clip}%
\pgfsetbuttcap%
\pgfsetmiterjoin%
\definecolor{currentfill}{rgb}{0.133298,0.375282,0.379395}%
\pgfsetfillcolor{currentfill}%
\pgfsetlinewidth{0.000000pt}%
\definecolor{currentstroke}{rgb}{0.000000,0.000000,0.000000}%
\pgfsetstrokecolor{currentstroke}%
\pgfsetstrokeopacity{0.000000}%
\pgfsetdash{}{0pt}%
\pgfpathmoveto{\pgfqpoint{2.246276in}{1.734829in}}%
\pgfpathlineto{\pgfqpoint{2.255212in}{1.734829in}}%
\pgfpathlineto{\pgfqpoint{2.255212in}{1.713106in}}%
\pgfpathlineto{\pgfqpoint{2.246276in}{1.713106in}}%
\pgfpathlineto{\pgfqpoint{2.246276in}{1.734829in}}%
\pgfpathclose%
\pgfusepath{fill}%
\end{pgfscope}%
\begin{pgfscope}%
\pgfpathrectangle{\pgfqpoint{0.697024in}{0.857143in}}{\pgfqpoint{2.627103in}{1.813434in}}%
\pgfusepath{clip}%
\pgfsetbuttcap%
\pgfsetmiterjoin%
\definecolor{currentfill}{rgb}{0.133298,0.375282,0.379395}%
\pgfsetfillcolor{currentfill}%
\pgfsetlinewidth{0.000000pt}%
\definecolor{currentstroke}{rgb}{0.000000,0.000000,0.000000}%
\pgfsetstrokecolor{currentstroke}%
\pgfsetstrokeopacity{0.000000}%
\pgfsetdash{}{0pt}%
\pgfpathmoveto{\pgfqpoint{2.257446in}{1.847071in}}%
\pgfpathlineto{\pgfqpoint{2.266383in}{1.847071in}}%
\pgfpathlineto{\pgfqpoint{2.266383in}{1.922830in}}%
\pgfpathlineto{\pgfqpoint{2.257446in}{1.922830in}}%
\pgfpathlineto{\pgfqpoint{2.257446in}{1.847071in}}%
\pgfpathclose%
\pgfusepath{fill}%
\end{pgfscope}%
\begin{pgfscope}%
\pgfpathrectangle{\pgfqpoint{0.697024in}{0.857143in}}{\pgfqpoint{2.627103in}{1.813434in}}%
\pgfusepath{clip}%
\pgfsetbuttcap%
\pgfsetmiterjoin%
\definecolor{currentfill}{rgb}{0.133298,0.375282,0.379395}%
\pgfsetfillcolor{currentfill}%
\pgfsetlinewidth{0.000000pt}%
\definecolor{currentstroke}{rgb}{0.000000,0.000000,0.000000}%
\pgfsetstrokecolor{currentstroke}%
\pgfsetstrokeopacity{0.000000}%
\pgfsetdash{}{0pt}%
\pgfpathmoveto{\pgfqpoint{2.268617in}{1.847071in}}%
\pgfpathlineto{\pgfqpoint{2.277553in}{1.847071in}}%
\pgfpathlineto{\pgfqpoint{2.277553in}{1.926951in}}%
\pgfpathlineto{\pgfqpoint{2.268617in}{1.926951in}}%
\pgfpathlineto{\pgfqpoint{2.268617in}{1.847071in}}%
\pgfpathclose%
\pgfusepath{fill}%
\end{pgfscope}%
\begin{pgfscope}%
\pgfpathrectangle{\pgfqpoint{0.697024in}{0.857143in}}{\pgfqpoint{2.627103in}{1.813434in}}%
\pgfusepath{clip}%
\pgfsetbuttcap%
\pgfsetmiterjoin%
\definecolor{currentfill}{rgb}{0.133298,0.375282,0.379395}%
\pgfsetfillcolor{currentfill}%
\pgfsetlinewidth{0.000000pt}%
\definecolor{currentstroke}{rgb}{0.000000,0.000000,0.000000}%
\pgfsetstrokecolor{currentstroke}%
\pgfsetstrokeopacity{0.000000}%
\pgfsetdash{}{0pt}%
\pgfpathmoveto{\pgfqpoint{2.279787in}{1.847071in}}%
\pgfpathlineto{\pgfqpoint{2.288724in}{1.847071in}}%
\pgfpathlineto{\pgfqpoint{2.288724in}{1.938065in}}%
\pgfpathlineto{\pgfqpoint{2.279787in}{1.938065in}}%
\pgfpathlineto{\pgfqpoint{2.279787in}{1.847071in}}%
\pgfpathclose%
\pgfusepath{fill}%
\end{pgfscope}%
\begin{pgfscope}%
\pgfpathrectangle{\pgfqpoint{0.697024in}{0.857143in}}{\pgfqpoint{2.627103in}{1.813434in}}%
\pgfusepath{clip}%
\pgfsetbuttcap%
\pgfsetmiterjoin%
\definecolor{currentfill}{rgb}{0.133298,0.375282,0.379395}%
\pgfsetfillcolor{currentfill}%
\pgfsetlinewidth{0.000000pt}%
\definecolor{currentstroke}{rgb}{0.000000,0.000000,0.000000}%
\pgfsetstrokecolor{currentstroke}%
\pgfsetstrokeopacity{0.000000}%
\pgfsetdash{}{0pt}%
\pgfpathmoveto{\pgfqpoint{2.290958in}{1.847071in}}%
\pgfpathlineto{\pgfqpoint{2.299895in}{1.847071in}}%
\pgfpathlineto{\pgfqpoint{2.299895in}{1.982655in}}%
\pgfpathlineto{\pgfqpoint{2.290958in}{1.982655in}}%
\pgfpathlineto{\pgfqpoint{2.290958in}{1.847071in}}%
\pgfpathclose%
\pgfusepath{fill}%
\end{pgfscope}%
\begin{pgfscope}%
\pgfpathrectangle{\pgfqpoint{0.697024in}{0.857143in}}{\pgfqpoint{2.627103in}{1.813434in}}%
\pgfusepath{clip}%
\pgfsetbuttcap%
\pgfsetmiterjoin%
\definecolor{currentfill}{rgb}{0.133298,0.375282,0.379395}%
\pgfsetfillcolor{currentfill}%
\pgfsetlinewidth{0.000000pt}%
\definecolor{currentstroke}{rgb}{0.000000,0.000000,0.000000}%
\pgfsetstrokecolor{currentstroke}%
\pgfsetstrokeopacity{0.000000}%
\pgfsetdash{}{0pt}%
\pgfpathmoveto{\pgfqpoint{2.302129in}{1.847071in}}%
\pgfpathlineto{\pgfqpoint{2.311065in}{1.847071in}}%
\pgfpathlineto{\pgfqpoint{2.311065in}{1.931558in}}%
\pgfpathlineto{\pgfqpoint{2.302129in}{1.931558in}}%
\pgfpathlineto{\pgfqpoint{2.302129in}{1.847071in}}%
\pgfpathclose%
\pgfusepath{fill}%
\end{pgfscope}%
\begin{pgfscope}%
\pgfpathrectangle{\pgfqpoint{0.697024in}{0.857143in}}{\pgfqpoint{2.627103in}{1.813434in}}%
\pgfusepath{clip}%
\pgfsetbuttcap%
\pgfsetmiterjoin%
\definecolor{currentfill}{rgb}{0.133298,0.375282,0.379395}%
\pgfsetfillcolor{currentfill}%
\pgfsetlinewidth{0.000000pt}%
\definecolor{currentstroke}{rgb}{0.000000,0.000000,0.000000}%
\pgfsetstrokecolor{currentstroke}%
\pgfsetstrokeopacity{0.000000}%
\pgfsetdash{}{0pt}%
\pgfpathmoveto{\pgfqpoint{2.313299in}{1.847071in}}%
\pgfpathlineto{\pgfqpoint{2.322236in}{1.847071in}}%
\pgfpathlineto{\pgfqpoint{2.322236in}{1.933288in}}%
\pgfpathlineto{\pgfqpoint{2.313299in}{1.933288in}}%
\pgfpathlineto{\pgfqpoint{2.313299in}{1.847071in}}%
\pgfpathclose%
\pgfusepath{fill}%
\end{pgfscope}%
\begin{pgfscope}%
\pgfpathrectangle{\pgfqpoint{0.697024in}{0.857143in}}{\pgfqpoint{2.627103in}{1.813434in}}%
\pgfusepath{clip}%
\pgfsetbuttcap%
\pgfsetmiterjoin%
\definecolor{currentfill}{rgb}{0.133298,0.375282,0.379395}%
\pgfsetfillcolor{currentfill}%
\pgfsetlinewidth{0.000000pt}%
\definecolor{currentstroke}{rgb}{0.000000,0.000000,0.000000}%
\pgfsetstrokecolor{currentstroke}%
\pgfsetstrokeopacity{0.000000}%
\pgfsetdash{}{0pt}%
\pgfpathmoveto{\pgfqpoint{2.324470in}{1.847071in}}%
\pgfpathlineto{\pgfqpoint{2.333406in}{1.847071in}}%
\pgfpathlineto{\pgfqpoint{2.333406in}{1.973105in}}%
\pgfpathlineto{\pgfqpoint{2.324470in}{1.973105in}}%
\pgfpathlineto{\pgfqpoint{2.324470in}{1.847071in}}%
\pgfpathclose%
\pgfusepath{fill}%
\end{pgfscope}%
\begin{pgfscope}%
\pgfpathrectangle{\pgfqpoint{0.697024in}{0.857143in}}{\pgfqpoint{2.627103in}{1.813434in}}%
\pgfusepath{clip}%
\pgfsetbuttcap%
\pgfsetmiterjoin%
\definecolor{currentfill}{rgb}{0.133298,0.375282,0.379395}%
\pgfsetfillcolor{currentfill}%
\pgfsetlinewidth{0.000000pt}%
\definecolor{currentstroke}{rgb}{0.000000,0.000000,0.000000}%
\pgfsetstrokecolor{currentstroke}%
\pgfsetstrokeopacity{0.000000}%
\pgfsetdash{}{0pt}%
\pgfpathmoveto{\pgfqpoint{2.335640in}{1.847071in}}%
\pgfpathlineto{\pgfqpoint{2.344577in}{1.847071in}}%
\pgfpathlineto{\pgfqpoint{2.344577in}{1.996254in}}%
\pgfpathlineto{\pgfqpoint{2.335640in}{1.996254in}}%
\pgfpathlineto{\pgfqpoint{2.335640in}{1.847071in}}%
\pgfpathclose%
\pgfusepath{fill}%
\end{pgfscope}%
\begin{pgfscope}%
\pgfpathrectangle{\pgfqpoint{0.697024in}{0.857143in}}{\pgfqpoint{2.627103in}{1.813434in}}%
\pgfusepath{clip}%
\pgfsetbuttcap%
\pgfsetmiterjoin%
\definecolor{currentfill}{rgb}{0.133298,0.375282,0.379395}%
\pgfsetfillcolor{currentfill}%
\pgfsetlinewidth{0.000000pt}%
\definecolor{currentstroke}{rgb}{0.000000,0.000000,0.000000}%
\pgfsetstrokecolor{currentstroke}%
\pgfsetstrokeopacity{0.000000}%
\pgfsetdash{}{0pt}%
\pgfpathmoveto{\pgfqpoint{2.346811in}{1.847071in}}%
\pgfpathlineto{\pgfqpoint{2.355748in}{1.847071in}}%
\pgfpathlineto{\pgfqpoint{2.355748in}{2.031300in}}%
\pgfpathlineto{\pgfqpoint{2.346811in}{2.031300in}}%
\pgfpathlineto{\pgfqpoint{2.346811in}{1.847071in}}%
\pgfpathclose%
\pgfusepath{fill}%
\end{pgfscope}%
\begin{pgfscope}%
\pgfpathrectangle{\pgfqpoint{0.697024in}{0.857143in}}{\pgfqpoint{2.627103in}{1.813434in}}%
\pgfusepath{clip}%
\pgfsetbuttcap%
\pgfsetmiterjoin%
\definecolor{currentfill}{rgb}{0.133298,0.375282,0.379395}%
\pgfsetfillcolor{currentfill}%
\pgfsetlinewidth{0.000000pt}%
\definecolor{currentstroke}{rgb}{0.000000,0.000000,0.000000}%
\pgfsetstrokecolor{currentstroke}%
\pgfsetstrokeopacity{0.000000}%
\pgfsetdash{}{0pt}%
\pgfpathmoveto{\pgfqpoint{2.357982in}{1.847071in}}%
\pgfpathlineto{\pgfqpoint{2.366918in}{1.847071in}}%
\pgfpathlineto{\pgfqpoint{2.366918in}{1.981780in}}%
\pgfpathlineto{\pgfqpoint{2.357982in}{1.981780in}}%
\pgfpathlineto{\pgfqpoint{2.357982in}{1.847071in}}%
\pgfpathclose%
\pgfusepath{fill}%
\end{pgfscope}%
\begin{pgfscope}%
\pgfpathrectangle{\pgfqpoint{0.697024in}{0.857143in}}{\pgfqpoint{2.627103in}{1.813434in}}%
\pgfusepath{clip}%
\pgfsetbuttcap%
\pgfsetmiterjoin%
\definecolor{currentfill}{rgb}{0.133298,0.375282,0.379395}%
\pgfsetfillcolor{currentfill}%
\pgfsetlinewidth{0.000000pt}%
\definecolor{currentstroke}{rgb}{0.000000,0.000000,0.000000}%
\pgfsetstrokecolor{currentstroke}%
\pgfsetstrokeopacity{0.000000}%
\pgfsetdash{}{0pt}%
\pgfpathmoveto{\pgfqpoint{2.369152in}{1.847071in}}%
\pgfpathlineto{\pgfqpoint{2.378089in}{1.847071in}}%
\pgfpathlineto{\pgfqpoint{2.378089in}{2.014650in}}%
\pgfpathlineto{\pgfqpoint{2.369152in}{2.014650in}}%
\pgfpathlineto{\pgfqpoint{2.369152in}{1.847071in}}%
\pgfpathclose%
\pgfusepath{fill}%
\end{pgfscope}%
\begin{pgfscope}%
\pgfpathrectangle{\pgfqpoint{0.697024in}{0.857143in}}{\pgfqpoint{2.627103in}{1.813434in}}%
\pgfusepath{clip}%
\pgfsetbuttcap%
\pgfsetmiterjoin%
\definecolor{currentfill}{rgb}{0.133298,0.375282,0.379395}%
\pgfsetfillcolor{currentfill}%
\pgfsetlinewidth{0.000000pt}%
\definecolor{currentstroke}{rgb}{0.000000,0.000000,0.000000}%
\pgfsetstrokecolor{currentstroke}%
\pgfsetstrokeopacity{0.000000}%
\pgfsetdash{}{0pt}%
\pgfpathmoveto{\pgfqpoint{2.380323in}{1.847071in}}%
\pgfpathlineto{\pgfqpoint{2.389259in}{1.847071in}}%
\pgfpathlineto{\pgfqpoint{2.389259in}{2.000964in}}%
\pgfpathlineto{\pgfqpoint{2.380323in}{2.000964in}}%
\pgfpathlineto{\pgfqpoint{2.380323in}{1.847071in}}%
\pgfpathclose%
\pgfusepath{fill}%
\end{pgfscope}%
\begin{pgfscope}%
\pgfpathrectangle{\pgfqpoint{0.697024in}{0.857143in}}{\pgfqpoint{2.627103in}{1.813434in}}%
\pgfusepath{clip}%
\pgfsetbuttcap%
\pgfsetmiterjoin%
\definecolor{currentfill}{rgb}{0.133298,0.375282,0.379395}%
\pgfsetfillcolor{currentfill}%
\pgfsetlinewidth{0.000000pt}%
\definecolor{currentstroke}{rgb}{0.000000,0.000000,0.000000}%
\pgfsetstrokecolor{currentstroke}%
\pgfsetstrokeopacity{0.000000}%
\pgfsetdash{}{0pt}%
\pgfpathmoveto{\pgfqpoint{2.391494in}{1.847071in}}%
\pgfpathlineto{\pgfqpoint{2.400430in}{1.847071in}}%
\pgfpathlineto{\pgfqpoint{2.400430in}{1.994649in}}%
\pgfpathlineto{\pgfqpoint{2.391494in}{1.994649in}}%
\pgfpathlineto{\pgfqpoint{2.391494in}{1.847071in}}%
\pgfpathclose%
\pgfusepath{fill}%
\end{pgfscope}%
\begin{pgfscope}%
\pgfpathrectangle{\pgfqpoint{0.697024in}{0.857143in}}{\pgfqpoint{2.627103in}{1.813434in}}%
\pgfusepath{clip}%
\pgfsetbuttcap%
\pgfsetmiterjoin%
\definecolor{currentfill}{rgb}{0.133298,0.375282,0.379395}%
\pgfsetfillcolor{currentfill}%
\pgfsetlinewidth{0.000000pt}%
\definecolor{currentstroke}{rgb}{0.000000,0.000000,0.000000}%
\pgfsetstrokecolor{currentstroke}%
\pgfsetstrokeopacity{0.000000}%
\pgfsetdash{}{0pt}%
\pgfpathmoveto{\pgfqpoint{2.402664in}{1.878459in}}%
\pgfpathlineto{\pgfqpoint{2.411601in}{1.878459in}}%
\pgfpathlineto{\pgfqpoint{2.411601in}{2.002555in}}%
\pgfpathlineto{\pgfqpoint{2.402664in}{2.002555in}}%
\pgfpathlineto{\pgfqpoint{2.402664in}{1.878459in}}%
\pgfpathclose%
\pgfusepath{fill}%
\end{pgfscope}%
\begin{pgfscope}%
\pgfpathrectangle{\pgfqpoint{0.697024in}{0.857143in}}{\pgfqpoint{2.627103in}{1.813434in}}%
\pgfusepath{clip}%
\pgfsetbuttcap%
\pgfsetmiterjoin%
\definecolor{currentfill}{rgb}{0.133298,0.375282,0.379395}%
\pgfsetfillcolor{currentfill}%
\pgfsetlinewidth{0.000000pt}%
\definecolor{currentstroke}{rgb}{0.000000,0.000000,0.000000}%
\pgfsetstrokecolor{currentstroke}%
\pgfsetstrokeopacity{0.000000}%
\pgfsetdash{}{0pt}%
\pgfpathmoveto{\pgfqpoint{2.413835in}{1.871756in}}%
\pgfpathlineto{\pgfqpoint{2.422771in}{1.871756in}}%
\pgfpathlineto{\pgfqpoint{2.422771in}{1.978986in}}%
\pgfpathlineto{\pgfqpoint{2.413835in}{1.978986in}}%
\pgfpathlineto{\pgfqpoint{2.413835in}{1.871756in}}%
\pgfpathclose%
\pgfusepath{fill}%
\end{pgfscope}%
\begin{pgfscope}%
\pgfpathrectangle{\pgfqpoint{0.697024in}{0.857143in}}{\pgfqpoint{2.627103in}{1.813434in}}%
\pgfusepath{clip}%
\pgfsetbuttcap%
\pgfsetmiterjoin%
\definecolor{currentfill}{rgb}{0.133298,0.375282,0.379395}%
\pgfsetfillcolor{currentfill}%
\pgfsetlinewidth{0.000000pt}%
\definecolor{currentstroke}{rgb}{0.000000,0.000000,0.000000}%
\pgfsetstrokecolor{currentstroke}%
\pgfsetstrokeopacity{0.000000}%
\pgfsetdash{}{0pt}%
\pgfpathmoveto{\pgfqpoint{2.425005in}{1.880722in}}%
\pgfpathlineto{\pgfqpoint{2.433942in}{1.880722in}}%
\pgfpathlineto{\pgfqpoint{2.433942in}{1.994111in}}%
\pgfpathlineto{\pgfqpoint{2.425005in}{1.994111in}}%
\pgfpathlineto{\pgfqpoint{2.425005in}{1.880722in}}%
\pgfpathclose%
\pgfusepath{fill}%
\end{pgfscope}%
\begin{pgfscope}%
\pgfpathrectangle{\pgfqpoint{0.697024in}{0.857143in}}{\pgfqpoint{2.627103in}{1.813434in}}%
\pgfusepath{clip}%
\pgfsetbuttcap%
\pgfsetmiterjoin%
\definecolor{currentfill}{rgb}{0.133298,0.375282,0.379395}%
\pgfsetfillcolor{currentfill}%
\pgfsetlinewidth{0.000000pt}%
\definecolor{currentstroke}{rgb}{0.000000,0.000000,0.000000}%
\pgfsetstrokecolor{currentstroke}%
\pgfsetstrokeopacity{0.000000}%
\pgfsetdash{}{0pt}%
\pgfpathmoveto{\pgfqpoint{2.436176in}{1.865053in}}%
\pgfpathlineto{\pgfqpoint{2.445112in}{1.865053in}}%
\pgfpathlineto{\pgfqpoint{2.445112in}{1.978156in}}%
\pgfpathlineto{\pgfqpoint{2.436176in}{1.978156in}}%
\pgfpathlineto{\pgfqpoint{2.436176in}{1.865053in}}%
\pgfpathclose%
\pgfusepath{fill}%
\end{pgfscope}%
\begin{pgfscope}%
\pgfpathrectangle{\pgfqpoint{0.697024in}{0.857143in}}{\pgfqpoint{2.627103in}{1.813434in}}%
\pgfusepath{clip}%
\pgfsetbuttcap%
\pgfsetmiterjoin%
\definecolor{currentfill}{rgb}{0.133298,0.375282,0.379395}%
\pgfsetfillcolor{currentfill}%
\pgfsetlinewidth{0.000000pt}%
\definecolor{currentstroke}{rgb}{0.000000,0.000000,0.000000}%
\pgfsetstrokecolor{currentstroke}%
\pgfsetstrokeopacity{0.000000}%
\pgfsetdash{}{0pt}%
\pgfpathmoveto{\pgfqpoint{2.447347in}{1.885329in}}%
\pgfpathlineto{\pgfqpoint{2.456283in}{1.885329in}}%
\pgfpathlineto{\pgfqpoint{2.456283in}{2.008935in}}%
\pgfpathlineto{\pgfqpoint{2.447347in}{2.008935in}}%
\pgfpathlineto{\pgfqpoint{2.447347in}{1.885329in}}%
\pgfpathclose%
\pgfusepath{fill}%
\end{pgfscope}%
\begin{pgfscope}%
\pgfpathrectangle{\pgfqpoint{0.697024in}{0.857143in}}{\pgfqpoint{2.627103in}{1.813434in}}%
\pgfusepath{clip}%
\pgfsetbuttcap%
\pgfsetmiterjoin%
\definecolor{currentfill}{rgb}{0.133298,0.375282,0.379395}%
\pgfsetfillcolor{currentfill}%
\pgfsetlinewidth{0.000000pt}%
\definecolor{currentstroke}{rgb}{0.000000,0.000000,0.000000}%
\pgfsetstrokecolor{currentstroke}%
\pgfsetstrokeopacity{0.000000}%
\pgfsetdash{}{0pt}%
\pgfpathmoveto{\pgfqpoint{2.458517in}{1.907319in}}%
\pgfpathlineto{\pgfqpoint{2.467454in}{1.907319in}}%
\pgfpathlineto{\pgfqpoint{2.467454in}{1.996707in}}%
\pgfpathlineto{\pgfqpoint{2.458517in}{1.996707in}}%
\pgfpathlineto{\pgfqpoint{2.458517in}{1.907319in}}%
\pgfpathclose%
\pgfusepath{fill}%
\end{pgfscope}%
\begin{pgfscope}%
\pgfpathrectangle{\pgfqpoint{0.697024in}{0.857143in}}{\pgfqpoint{2.627103in}{1.813434in}}%
\pgfusepath{clip}%
\pgfsetbuttcap%
\pgfsetmiterjoin%
\definecolor{currentfill}{rgb}{0.133298,0.375282,0.379395}%
\pgfsetfillcolor{currentfill}%
\pgfsetlinewidth{0.000000pt}%
\definecolor{currentstroke}{rgb}{0.000000,0.000000,0.000000}%
\pgfsetstrokecolor{currentstroke}%
\pgfsetstrokeopacity{0.000000}%
\pgfsetdash{}{0pt}%
\pgfpathmoveto{\pgfqpoint{2.469688in}{1.937461in}}%
\pgfpathlineto{\pgfqpoint{2.478624in}{1.937461in}}%
\pgfpathlineto{\pgfqpoint{2.478624in}{1.994211in}}%
\pgfpathlineto{\pgfqpoint{2.469688in}{1.994211in}}%
\pgfpathlineto{\pgfqpoint{2.469688in}{1.937461in}}%
\pgfpathclose%
\pgfusepath{fill}%
\end{pgfscope}%
\begin{pgfscope}%
\pgfpathrectangle{\pgfqpoint{0.697024in}{0.857143in}}{\pgfqpoint{2.627103in}{1.813434in}}%
\pgfusepath{clip}%
\pgfsetbuttcap%
\pgfsetmiterjoin%
\definecolor{currentfill}{rgb}{0.133298,0.375282,0.379395}%
\pgfsetfillcolor{currentfill}%
\pgfsetlinewidth{0.000000pt}%
\definecolor{currentstroke}{rgb}{0.000000,0.000000,0.000000}%
\pgfsetstrokecolor{currentstroke}%
\pgfsetstrokeopacity{0.000000}%
\pgfsetdash{}{0pt}%
\pgfpathmoveto{\pgfqpoint{2.480858in}{1.847071in}}%
\pgfpathlineto{\pgfqpoint{2.489795in}{1.847071in}}%
\pgfpathlineto{\pgfqpoint{2.489795in}{1.800812in}}%
\pgfpathlineto{\pgfqpoint{2.480858in}{1.800812in}}%
\pgfpathlineto{\pgfqpoint{2.480858in}{1.847071in}}%
\pgfpathclose%
\pgfusepath{fill}%
\end{pgfscope}%
\begin{pgfscope}%
\pgfpathrectangle{\pgfqpoint{0.697024in}{0.857143in}}{\pgfqpoint{2.627103in}{1.813434in}}%
\pgfusepath{clip}%
\pgfsetbuttcap%
\pgfsetmiterjoin%
\definecolor{currentfill}{rgb}{0.133298,0.375282,0.379395}%
\pgfsetfillcolor{currentfill}%
\pgfsetlinewidth{0.000000pt}%
\definecolor{currentstroke}{rgb}{0.000000,0.000000,0.000000}%
\pgfsetstrokecolor{currentstroke}%
\pgfsetstrokeopacity{0.000000}%
\pgfsetdash{}{0pt}%
\pgfpathmoveto{\pgfqpoint{2.492029in}{1.847071in}}%
\pgfpathlineto{\pgfqpoint{2.500965in}{1.847071in}}%
\pgfpathlineto{\pgfqpoint{2.500965in}{1.807603in}}%
\pgfpathlineto{\pgfqpoint{2.492029in}{1.807603in}}%
\pgfpathlineto{\pgfqpoint{2.492029in}{1.847071in}}%
\pgfpathclose%
\pgfusepath{fill}%
\end{pgfscope}%
\begin{pgfscope}%
\pgfpathrectangle{\pgfqpoint{0.697024in}{0.857143in}}{\pgfqpoint{2.627103in}{1.813434in}}%
\pgfusepath{clip}%
\pgfsetbuttcap%
\pgfsetmiterjoin%
\definecolor{currentfill}{rgb}{0.133298,0.375282,0.379395}%
\pgfsetfillcolor{currentfill}%
\pgfsetlinewidth{0.000000pt}%
\definecolor{currentstroke}{rgb}{0.000000,0.000000,0.000000}%
\pgfsetstrokecolor{currentstroke}%
\pgfsetstrokeopacity{0.000000}%
\pgfsetdash{}{0pt}%
\pgfpathmoveto{\pgfqpoint{2.503200in}{1.847071in}}%
\pgfpathlineto{\pgfqpoint{2.512136in}{1.847071in}}%
\pgfpathlineto{\pgfqpoint{2.512136in}{1.748004in}}%
\pgfpathlineto{\pgfqpoint{2.503200in}{1.748004in}}%
\pgfpathlineto{\pgfqpoint{2.503200in}{1.847071in}}%
\pgfpathclose%
\pgfusepath{fill}%
\end{pgfscope}%
\begin{pgfscope}%
\pgfpathrectangle{\pgfqpoint{0.697024in}{0.857143in}}{\pgfqpoint{2.627103in}{1.813434in}}%
\pgfusepath{clip}%
\pgfsetbuttcap%
\pgfsetmiterjoin%
\definecolor{currentfill}{rgb}{0.133298,0.375282,0.379395}%
\pgfsetfillcolor{currentfill}%
\pgfsetlinewidth{0.000000pt}%
\definecolor{currentstroke}{rgb}{0.000000,0.000000,0.000000}%
\pgfsetstrokecolor{currentstroke}%
\pgfsetstrokeopacity{0.000000}%
\pgfsetdash{}{0pt}%
\pgfpathmoveto{\pgfqpoint{2.514370in}{1.847071in}}%
\pgfpathlineto{\pgfqpoint{2.523307in}{1.847071in}}%
\pgfpathlineto{\pgfqpoint{2.523307in}{1.798627in}}%
\pgfpathlineto{\pgfqpoint{2.514370in}{1.798627in}}%
\pgfpathlineto{\pgfqpoint{2.514370in}{1.847071in}}%
\pgfpathclose%
\pgfusepath{fill}%
\end{pgfscope}%
\begin{pgfscope}%
\pgfpathrectangle{\pgfqpoint{0.697024in}{0.857143in}}{\pgfqpoint{2.627103in}{1.813434in}}%
\pgfusepath{clip}%
\pgfsetbuttcap%
\pgfsetmiterjoin%
\definecolor{currentfill}{rgb}{0.133298,0.375282,0.379395}%
\pgfsetfillcolor{currentfill}%
\pgfsetlinewidth{0.000000pt}%
\definecolor{currentstroke}{rgb}{0.000000,0.000000,0.000000}%
\pgfsetstrokecolor{currentstroke}%
\pgfsetstrokeopacity{0.000000}%
\pgfsetdash{}{0pt}%
\pgfpathmoveto{\pgfqpoint{2.525541in}{1.847071in}}%
\pgfpathlineto{\pgfqpoint{2.534477in}{1.847071in}}%
\pgfpathlineto{\pgfqpoint{2.534477in}{1.758929in}}%
\pgfpathlineto{\pgfqpoint{2.525541in}{1.758929in}}%
\pgfpathlineto{\pgfqpoint{2.525541in}{1.847071in}}%
\pgfpathclose%
\pgfusepath{fill}%
\end{pgfscope}%
\begin{pgfscope}%
\pgfpathrectangle{\pgfqpoint{0.697024in}{0.857143in}}{\pgfqpoint{2.627103in}{1.813434in}}%
\pgfusepath{clip}%
\pgfsetbuttcap%
\pgfsetmiterjoin%
\definecolor{currentfill}{rgb}{0.133298,0.375282,0.379395}%
\pgfsetfillcolor{currentfill}%
\pgfsetlinewidth{0.000000pt}%
\definecolor{currentstroke}{rgb}{0.000000,0.000000,0.000000}%
\pgfsetstrokecolor{currentstroke}%
\pgfsetstrokeopacity{0.000000}%
\pgfsetdash{}{0pt}%
\pgfpathmoveto{\pgfqpoint{2.536711in}{1.847071in}}%
\pgfpathlineto{\pgfqpoint{2.545648in}{1.847071in}}%
\pgfpathlineto{\pgfqpoint{2.545648in}{1.720409in}}%
\pgfpathlineto{\pgfqpoint{2.536711in}{1.720409in}}%
\pgfpathlineto{\pgfqpoint{2.536711in}{1.847071in}}%
\pgfpathclose%
\pgfusepath{fill}%
\end{pgfscope}%
\begin{pgfscope}%
\pgfpathrectangle{\pgfqpoint{0.697024in}{0.857143in}}{\pgfqpoint{2.627103in}{1.813434in}}%
\pgfusepath{clip}%
\pgfsetbuttcap%
\pgfsetmiterjoin%
\definecolor{currentfill}{rgb}{0.133298,0.375282,0.379395}%
\pgfsetfillcolor{currentfill}%
\pgfsetlinewidth{0.000000pt}%
\definecolor{currentstroke}{rgb}{0.000000,0.000000,0.000000}%
\pgfsetstrokecolor{currentstroke}%
\pgfsetstrokeopacity{0.000000}%
\pgfsetdash{}{0pt}%
\pgfpathmoveto{\pgfqpoint{2.547882in}{1.847071in}}%
\pgfpathlineto{\pgfqpoint{2.556818in}{1.847071in}}%
\pgfpathlineto{\pgfqpoint{2.556818in}{1.766469in}}%
\pgfpathlineto{\pgfqpoint{2.547882in}{1.766469in}}%
\pgfpathlineto{\pgfqpoint{2.547882in}{1.847071in}}%
\pgfpathclose%
\pgfusepath{fill}%
\end{pgfscope}%
\begin{pgfscope}%
\pgfpathrectangle{\pgfqpoint{0.697024in}{0.857143in}}{\pgfqpoint{2.627103in}{1.813434in}}%
\pgfusepath{clip}%
\pgfsetbuttcap%
\pgfsetmiterjoin%
\definecolor{currentfill}{rgb}{0.133298,0.375282,0.379395}%
\pgfsetfillcolor{currentfill}%
\pgfsetlinewidth{0.000000pt}%
\definecolor{currentstroke}{rgb}{0.000000,0.000000,0.000000}%
\pgfsetstrokecolor{currentstroke}%
\pgfsetstrokeopacity{0.000000}%
\pgfsetdash{}{0pt}%
\pgfpathmoveto{\pgfqpoint{2.559053in}{1.847071in}}%
\pgfpathlineto{\pgfqpoint{2.567989in}{1.847071in}}%
\pgfpathlineto{\pgfqpoint{2.567989in}{1.721051in}}%
\pgfpathlineto{\pgfqpoint{2.559053in}{1.721051in}}%
\pgfpathlineto{\pgfqpoint{2.559053in}{1.847071in}}%
\pgfpathclose%
\pgfusepath{fill}%
\end{pgfscope}%
\begin{pgfscope}%
\pgfpathrectangle{\pgfqpoint{0.697024in}{0.857143in}}{\pgfqpoint{2.627103in}{1.813434in}}%
\pgfusepath{clip}%
\pgfsetbuttcap%
\pgfsetmiterjoin%
\definecolor{currentfill}{rgb}{0.133298,0.375282,0.379395}%
\pgfsetfillcolor{currentfill}%
\pgfsetlinewidth{0.000000pt}%
\definecolor{currentstroke}{rgb}{0.000000,0.000000,0.000000}%
\pgfsetstrokecolor{currentstroke}%
\pgfsetstrokeopacity{0.000000}%
\pgfsetdash{}{0pt}%
\pgfpathmoveto{\pgfqpoint{2.570223in}{1.847071in}}%
\pgfpathlineto{\pgfqpoint{2.579160in}{1.847071in}}%
\pgfpathlineto{\pgfqpoint{2.579160in}{1.734664in}}%
\pgfpathlineto{\pgfqpoint{2.570223in}{1.734664in}}%
\pgfpathlineto{\pgfqpoint{2.570223in}{1.847071in}}%
\pgfpathclose%
\pgfusepath{fill}%
\end{pgfscope}%
\begin{pgfscope}%
\pgfpathrectangle{\pgfqpoint{0.697024in}{0.857143in}}{\pgfqpoint{2.627103in}{1.813434in}}%
\pgfusepath{clip}%
\pgfsetbuttcap%
\pgfsetmiterjoin%
\definecolor{currentfill}{rgb}{0.133298,0.375282,0.379395}%
\pgfsetfillcolor{currentfill}%
\pgfsetlinewidth{0.000000pt}%
\definecolor{currentstroke}{rgb}{0.000000,0.000000,0.000000}%
\pgfsetstrokecolor{currentstroke}%
\pgfsetstrokeopacity{0.000000}%
\pgfsetdash{}{0pt}%
\pgfpathmoveto{\pgfqpoint{2.581394in}{1.847071in}}%
\pgfpathlineto{\pgfqpoint{2.590330in}{1.847071in}}%
\pgfpathlineto{\pgfqpoint{2.590330in}{1.771486in}}%
\pgfpathlineto{\pgfqpoint{2.581394in}{1.771486in}}%
\pgfpathlineto{\pgfqpoint{2.581394in}{1.847071in}}%
\pgfpathclose%
\pgfusepath{fill}%
\end{pgfscope}%
\begin{pgfscope}%
\pgfpathrectangle{\pgfqpoint{0.697024in}{0.857143in}}{\pgfqpoint{2.627103in}{1.813434in}}%
\pgfusepath{clip}%
\pgfsetbuttcap%
\pgfsetmiterjoin%
\definecolor{currentfill}{rgb}{0.133298,0.375282,0.379395}%
\pgfsetfillcolor{currentfill}%
\pgfsetlinewidth{0.000000pt}%
\definecolor{currentstroke}{rgb}{0.000000,0.000000,0.000000}%
\pgfsetstrokecolor{currentstroke}%
\pgfsetstrokeopacity{0.000000}%
\pgfsetdash{}{0pt}%
\pgfpathmoveto{\pgfqpoint{2.592564in}{1.847071in}}%
\pgfpathlineto{\pgfqpoint{2.601501in}{1.847071in}}%
\pgfpathlineto{\pgfqpoint{2.601501in}{1.814524in}}%
\pgfpathlineto{\pgfqpoint{2.592564in}{1.814524in}}%
\pgfpathlineto{\pgfqpoint{2.592564in}{1.847071in}}%
\pgfpathclose%
\pgfusepath{fill}%
\end{pgfscope}%
\begin{pgfscope}%
\pgfpathrectangle{\pgfqpoint{0.697024in}{0.857143in}}{\pgfqpoint{2.627103in}{1.813434in}}%
\pgfusepath{clip}%
\pgfsetbuttcap%
\pgfsetmiterjoin%
\definecolor{currentfill}{rgb}{0.133298,0.375282,0.379395}%
\pgfsetfillcolor{currentfill}%
\pgfsetlinewidth{0.000000pt}%
\definecolor{currentstroke}{rgb}{0.000000,0.000000,0.000000}%
\pgfsetstrokecolor{currentstroke}%
\pgfsetstrokeopacity{0.000000}%
\pgfsetdash{}{0pt}%
\pgfpathmoveto{\pgfqpoint{2.603735in}{1.847071in}}%
\pgfpathlineto{\pgfqpoint{2.612672in}{1.847071in}}%
\pgfpathlineto{\pgfqpoint{2.612672in}{1.838805in}}%
\pgfpathlineto{\pgfqpoint{2.603735in}{1.838805in}}%
\pgfpathlineto{\pgfqpoint{2.603735in}{1.847071in}}%
\pgfpathclose%
\pgfusepath{fill}%
\end{pgfscope}%
\begin{pgfscope}%
\pgfpathrectangle{\pgfqpoint{0.697024in}{0.857143in}}{\pgfqpoint{2.627103in}{1.813434in}}%
\pgfusepath{clip}%
\pgfsetbuttcap%
\pgfsetmiterjoin%
\definecolor{currentfill}{rgb}{0.133298,0.375282,0.379395}%
\pgfsetfillcolor{currentfill}%
\pgfsetlinewidth{0.000000pt}%
\definecolor{currentstroke}{rgb}{0.000000,0.000000,0.000000}%
\pgfsetstrokecolor{currentstroke}%
\pgfsetstrokeopacity{0.000000}%
\pgfsetdash{}{0pt}%
\pgfpathmoveto{\pgfqpoint{2.614906in}{1.899754in}}%
\pgfpathlineto{\pgfqpoint{2.623842in}{1.899754in}}%
\pgfpathlineto{\pgfqpoint{2.623842in}{1.995964in}}%
\pgfpathlineto{\pgfqpoint{2.614906in}{1.995964in}}%
\pgfpathlineto{\pgfqpoint{2.614906in}{1.899754in}}%
\pgfpathclose%
\pgfusepath{fill}%
\end{pgfscope}%
\begin{pgfscope}%
\pgfpathrectangle{\pgfqpoint{0.697024in}{0.857143in}}{\pgfqpoint{2.627103in}{1.813434in}}%
\pgfusepath{clip}%
\pgfsetbuttcap%
\pgfsetmiterjoin%
\definecolor{currentfill}{rgb}{0.133298,0.375282,0.379395}%
\pgfsetfillcolor{currentfill}%
\pgfsetlinewidth{0.000000pt}%
\definecolor{currentstroke}{rgb}{0.000000,0.000000,0.000000}%
\pgfsetstrokecolor{currentstroke}%
\pgfsetstrokeopacity{0.000000}%
\pgfsetdash{}{0pt}%
\pgfpathmoveto{\pgfqpoint{2.626076in}{1.847071in}}%
\pgfpathlineto{\pgfqpoint{2.635013in}{1.847071in}}%
\pgfpathlineto{\pgfqpoint{2.635013in}{1.788001in}}%
\pgfpathlineto{\pgfqpoint{2.626076in}{1.788001in}}%
\pgfpathlineto{\pgfqpoint{2.626076in}{1.847071in}}%
\pgfpathclose%
\pgfusepath{fill}%
\end{pgfscope}%
\begin{pgfscope}%
\pgfpathrectangle{\pgfqpoint{0.697024in}{0.857143in}}{\pgfqpoint{2.627103in}{1.813434in}}%
\pgfusepath{clip}%
\pgfsetbuttcap%
\pgfsetmiterjoin%
\definecolor{currentfill}{rgb}{0.133298,0.375282,0.379395}%
\pgfsetfillcolor{currentfill}%
\pgfsetlinewidth{0.000000pt}%
\definecolor{currentstroke}{rgb}{0.000000,0.000000,0.000000}%
\pgfsetstrokecolor{currentstroke}%
\pgfsetstrokeopacity{0.000000}%
\pgfsetdash{}{0pt}%
\pgfpathmoveto{\pgfqpoint{2.637247in}{1.893783in}}%
\pgfpathlineto{\pgfqpoint{2.646183in}{1.893783in}}%
\pgfpathlineto{\pgfqpoint{2.646183in}{1.939508in}}%
\pgfpathlineto{\pgfqpoint{2.637247in}{1.939508in}}%
\pgfpathlineto{\pgfqpoint{2.637247in}{1.893783in}}%
\pgfpathclose%
\pgfusepath{fill}%
\end{pgfscope}%
\begin{pgfscope}%
\pgfpathrectangle{\pgfqpoint{0.697024in}{0.857143in}}{\pgfqpoint{2.627103in}{1.813434in}}%
\pgfusepath{clip}%
\pgfsetbuttcap%
\pgfsetmiterjoin%
\definecolor{currentfill}{rgb}{0.133298,0.375282,0.379395}%
\pgfsetfillcolor{currentfill}%
\pgfsetlinewidth{0.000000pt}%
\definecolor{currentstroke}{rgb}{0.000000,0.000000,0.000000}%
\pgfsetstrokecolor{currentstroke}%
\pgfsetstrokeopacity{0.000000}%
\pgfsetdash{}{0pt}%
\pgfpathmoveto{\pgfqpoint{2.648417in}{1.908001in}}%
\pgfpathlineto{\pgfqpoint{2.657354in}{1.908001in}}%
\pgfpathlineto{\pgfqpoint{2.657354in}{1.994920in}}%
\pgfpathlineto{\pgfqpoint{2.648417in}{1.994920in}}%
\pgfpathlineto{\pgfqpoint{2.648417in}{1.908001in}}%
\pgfpathclose%
\pgfusepath{fill}%
\end{pgfscope}%
\begin{pgfscope}%
\pgfpathrectangle{\pgfqpoint{0.697024in}{0.857143in}}{\pgfqpoint{2.627103in}{1.813434in}}%
\pgfusepath{clip}%
\pgfsetbuttcap%
\pgfsetmiterjoin%
\definecolor{currentfill}{rgb}{0.133298,0.375282,0.379395}%
\pgfsetfillcolor{currentfill}%
\pgfsetlinewidth{0.000000pt}%
\definecolor{currentstroke}{rgb}{0.000000,0.000000,0.000000}%
\pgfsetstrokecolor{currentstroke}%
\pgfsetstrokeopacity{0.000000}%
\pgfsetdash{}{0pt}%
\pgfpathmoveto{\pgfqpoint{2.659588in}{1.921300in}}%
\pgfpathlineto{\pgfqpoint{2.668525in}{1.921300in}}%
\pgfpathlineto{\pgfqpoint{2.668525in}{1.998017in}}%
\pgfpathlineto{\pgfqpoint{2.659588in}{1.998017in}}%
\pgfpathlineto{\pgfqpoint{2.659588in}{1.921300in}}%
\pgfpathclose%
\pgfusepath{fill}%
\end{pgfscope}%
\begin{pgfscope}%
\pgfpathrectangle{\pgfqpoint{0.697024in}{0.857143in}}{\pgfqpoint{2.627103in}{1.813434in}}%
\pgfusepath{clip}%
\pgfsetbuttcap%
\pgfsetmiterjoin%
\definecolor{currentfill}{rgb}{0.133298,0.375282,0.379395}%
\pgfsetfillcolor{currentfill}%
\pgfsetlinewidth{0.000000pt}%
\definecolor{currentstroke}{rgb}{0.000000,0.000000,0.000000}%
\pgfsetstrokecolor{currentstroke}%
\pgfsetstrokeopacity{0.000000}%
\pgfsetdash{}{0pt}%
\pgfpathmoveto{\pgfqpoint{2.670759in}{1.904614in}}%
\pgfpathlineto{\pgfqpoint{2.679695in}{1.904614in}}%
\pgfpathlineto{\pgfqpoint{2.679695in}{2.017368in}}%
\pgfpathlineto{\pgfqpoint{2.670759in}{2.017368in}}%
\pgfpathlineto{\pgfqpoint{2.670759in}{1.904614in}}%
\pgfpathclose%
\pgfusepath{fill}%
\end{pgfscope}%
\begin{pgfscope}%
\pgfpathrectangle{\pgfqpoint{0.697024in}{0.857143in}}{\pgfqpoint{2.627103in}{1.813434in}}%
\pgfusepath{clip}%
\pgfsetbuttcap%
\pgfsetmiterjoin%
\definecolor{currentfill}{rgb}{0.133298,0.375282,0.379395}%
\pgfsetfillcolor{currentfill}%
\pgfsetlinewidth{0.000000pt}%
\definecolor{currentstroke}{rgb}{0.000000,0.000000,0.000000}%
\pgfsetstrokecolor{currentstroke}%
\pgfsetstrokeopacity{0.000000}%
\pgfsetdash{}{0pt}%
\pgfpathmoveto{\pgfqpoint{2.681929in}{1.929050in}}%
\pgfpathlineto{\pgfqpoint{2.690866in}{1.929050in}}%
\pgfpathlineto{\pgfqpoint{2.690866in}{2.020398in}}%
\pgfpathlineto{\pgfqpoint{2.681929in}{2.020398in}}%
\pgfpathlineto{\pgfqpoint{2.681929in}{1.929050in}}%
\pgfpathclose%
\pgfusepath{fill}%
\end{pgfscope}%
\begin{pgfscope}%
\pgfpathrectangle{\pgfqpoint{0.697024in}{0.857143in}}{\pgfqpoint{2.627103in}{1.813434in}}%
\pgfusepath{clip}%
\pgfsetbuttcap%
\pgfsetmiterjoin%
\definecolor{currentfill}{rgb}{0.133298,0.375282,0.379395}%
\pgfsetfillcolor{currentfill}%
\pgfsetlinewidth{0.000000pt}%
\definecolor{currentstroke}{rgb}{0.000000,0.000000,0.000000}%
\pgfsetstrokecolor{currentstroke}%
\pgfsetstrokeopacity{0.000000}%
\pgfsetdash{}{0pt}%
\pgfpathmoveto{\pgfqpoint{2.693100in}{1.847071in}}%
\pgfpathlineto{\pgfqpoint{2.702036in}{1.847071in}}%
\pgfpathlineto{\pgfqpoint{2.702036in}{1.840324in}}%
\pgfpathlineto{\pgfqpoint{2.693100in}{1.840324in}}%
\pgfpathlineto{\pgfqpoint{2.693100in}{1.847071in}}%
\pgfpathclose%
\pgfusepath{fill}%
\end{pgfscope}%
\begin{pgfscope}%
\pgfpathrectangle{\pgfqpoint{0.697024in}{0.857143in}}{\pgfqpoint{2.627103in}{1.813434in}}%
\pgfusepath{clip}%
\pgfsetbuttcap%
\pgfsetmiterjoin%
\definecolor{currentfill}{rgb}{0.133298,0.375282,0.379395}%
\pgfsetfillcolor{currentfill}%
\pgfsetlinewidth{0.000000pt}%
\definecolor{currentstroke}{rgb}{0.000000,0.000000,0.000000}%
\pgfsetstrokecolor{currentstroke}%
\pgfsetstrokeopacity{0.000000}%
\pgfsetdash{}{0pt}%
\pgfpathmoveto{\pgfqpoint{2.704270in}{1.929741in}}%
\pgfpathlineto{\pgfqpoint{2.713207in}{1.929741in}}%
\pgfpathlineto{\pgfqpoint{2.713207in}{2.033013in}}%
\pgfpathlineto{\pgfqpoint{2.704270in}{2.033013in}}%
\pgfpathlineto{\pgfqpoint{2.704270in}{1.929741in}}%
\pgfpathclose%
\pgfusepath{fill}%
\end{pgfscope}%
\begin{pgfscope}%
\pgfpathrectangle{\pgfqpoint{0.697024in}{0.857143in}}{\pgfqpoint{2.627103in}{1.813434in}}%
\pgfusepath{clip}%
\pgfsetbuttcap%
\pgfsetmiterjoin%
\definecolor{currentfill}{rgb}{0.133298,0.375282,0.379395}%
\pgfsetfillcolor{currentfill}%
\pgfsetlinewidth{0.000000pt}%
\definecolor{currentstroke}{rgb}{0.000000,0.000000,0.000000}%
\pgfsetstrokecolor{currentstroke}%
\pgfsetstrokeopacity{0.000000}%
\pgfsetdash{}{0pt}%
\pgfpathmoveto{\pgfqpoint{2.715441in}{1.964849in}}%
\pgfpathlineto{\pgfqpoint{2.724378in}{1.964849in}}%
\pgfpathlineto{\pgfqpoint{2.724378in}{2.147082in}}%
\pgfpathlineto{\pgfqpoint{2.715441in}{2.147082in}}%
\pgfpathlineto{\pgfqpoint{2.715441in}{1.964849in}}%
\pgfpathclose%
\pgfusepath{fill}%
\end{pgfscope}%
\begin{pgfscope}%
\pgfpathrectangle{\pgfqpoint{0.697024in}{0.857143in}}{\pgfqpoint{2.627103in}{1.813434in}}%
\pgfusepath{clip}%
\pgfsetbuttcap%
\pgfsetmiterjoin%
\definecolor{currentfill}{rgb}{0.133298,0.375282,0.379395}%
\pgfsetfillcolor{currentfill}%
\pgfsetlinewidth{0.000000pt}%
\definecolor{currentstroke}{rgb}{0.000000,0.000000,0.000000}%
\pgfsetstrokecolor{currentstroke}%
\pgfsetstrokeopacity{0.000000}%
\pgfsetdash{}{0pt}%
\pgfpathmoveto{\pgfqpoint{2.726612in}{2.017346in}}%
\pgfpathlineto{\pgfqpoint{2.735548in}{2.017346in}}%
\pgfpathlineto{\pgfqpoint{2.735548in}{2.209477in}}%
\pgfpathlineto{\pgfqpoint{2.726612in}{2.209477in}}%
\pgfpathlineto{\pgfqpoint{2.726612in}{2.017346in}}%
\pgfpathclose%
\pgfusepath{fill}%
\end{pgfscope}%
\begin{pgfscope}%
\pgfpathrectangle{\pgfqpoint{0.697024in}{0.857143in}}{\pgfqpoint{2.627103in}{1.813434in}}%
\pgfusepath{clip}%
\pgfsetbuttcap%
\pgfsetmiterjoin%
\definecolor{currentfill}{rgb}{0.133298,0.375282,0.379395}%
\pgfsetfillcolor{currentfill}%
\pgfsetlinewidth{0.000000pt}%
\definecolor{currentstroke}{rgb}{0.000000,0.000000,0.000000}%
\pgfsetstrokecolor{currentstroke}%
\pgfsetstrokeopacity{0.000000}%
\pgfsetdash{}{0pt}%
\pgfpathmoveto{\pgfqpoint{2.737782in}{2.044550in}}%
\pgfpathlineto{\pgfqpoint{2.746719in}{2.044550in}}%
\pgfpathlineto{\pgfqpoint{2.746719in}{2.137425in}}%
\pgfpathlineto{\pgfqpoint{2.737782in}{2.137425in}}%
\pgfpathlineto{\pgfqpoint{2.737782in}{2.044550in}}%
\pgfpathclose%
\pgfusepath{fill}%
\end{pgfscope}%
\begin{pgfscope}%
\pgfpathrectangle{\pgfqpoint{0.697024in}{0.857143in}}{\pgfqpoint{2.627103in}{1.813434in}}%
\pgfusepath{clip}%
\pgfsetbuttcap%
\pgfsetmiterjoin%
\definecolor{currentfill}{rgb}{0.133298,0.375282,0.379395}%
\pgfsetfillcolor{currentfill}%
\pgfsetlinewidth{0.000000pt}%
\definecolor{currentstroke}{rgb}{0.000000,0.000000,0.000000}%
\pgfsetstrokecolor{currentstroke}%
\pgfsetstrokeopacity{0.000000}%
\pgfsetdash{}{0pt}%
\pgfpathmoveto{\pgfqpoint{2.748953in}{2.057299in}}%
\pgfpathlineto{\pgfqpoint{2.757889in}{2.057299in}}%
\pgfpathlineto{\pgfqpoint{2.757889in}{2.150328in}}%
\pgfpathlineto{\pgfqpoint{2.748953in}{2.150328in}}%
\pgfpathlineto{\pgfqpoint{2.748953in}{2.057299in}}%
\pgfpathclose%
\pgfusepath{fill}%
\end{pgfscope}%
\begin{pgfscope}%
\pgfpathrectangle{\pgfqpoint{0.697024in}{0.857143in}}{\pgfqpoint{2.627103in}{1.813434in}}%
\pgfusepath{clip}%
\pgfsetbuttcap%
\pgfsetmiterjoin%
\definecolor{currentfill}{rgb}{0.133298,0.375282,0.379395}%
\pgfsetfillcolor{currentfill}%
\pgfsetlinewidth{0.000000pt}%
\definecolor{currentstroke}{rgb}{0.000000,0.000000,0.000000}%
\pgfsetstrokecolor{currentstroke}%
\pgfsetstrokeopacity{0.000000}%
\pgfsetdash{}{0pt}%
\pgfpathmoveto{\pgfqpoint{2.760124in}{2.041499in}}%
\pgfpathlineto{\pgfqpoint{2.769060in}{2.041499in}}%
\pgfpathlineto{\pgfqpoint{2.769060in}{2.172579in}}%
\pgfpathlineto{\pgfqpoint{2.760124in}{2.172579in}}%
\pgfpathlineto{\pgfqpoint{2.760124in}{2.041499in}}%
\pgfpathclose%
\pgfusepath{fill}%
\end{pgfscope}%
\begin{pgfscope}%
\pgfpathrectangle{\pgfqpoint{0.697024in}{0.857143in}}{\pgfqpoint{2.627103in}{1.813434in}}%
\pgfusepath{clip}%
\pgfsetbuttcap%
\pgfsetmiterjoin%
\definecolor{currentfill}{rgb}{0.133298,0.375282,0.379395}%
\pgfsetfillcolor{currentfill}%
\pgfsetlinewidth{0.000000pt}%
\definecolor{currentstroke}{rgb}{0.000000,0.000000,0.000000}%
\pgfsetstrokecolor{currentstroke}%
\pgfsetstrokeopacity{0.000000}%
\pgfsetdash{}{0pt}%
\pgfpathmoveto{\pgfqpoint{2.771294in}{2.034910in}}%
\pgfpathlineto{\pgfqpoint{2.780231in}{2.034910in}}%
\pgfpathlineto{\pgfqpoint{2.780231in}{2.095956in}}%
\pgfpathlineto{\pgfqpoint{2.771294in}{2.095956in}}%
\pgfpathlineto{\pgfqpoint{2.771294in}{2.034910in}}%
\pgfpathclose%
\pgfusepath{fill}%
\end{pgfscope}%
\begin{pgfscope}%
\pgfpathrectangle{\pgfqpoint{0.697024in}{0.857143in}}{\pgfqpoint{2.627103in}{1.813434in}}%
\pgfusepath{clip}%
\pgfsetbuttcap%
\pgfsetmiterjoin%
\definecolor{currentfill}{rgb}{0.133298,0.375282,0.379395}%
\pgfsetfillcolor{currentfill}%
\pgfsetlinewidth{0.000000pt}%
\definecolor{currentstroke}{rgb}{0.000000,0.000000,0.000000}%
\pgfsetstrokecolor{currentstroke}%
\pgfsetstrokeopacity{0.000000}%
\pgfsetdash{}{0pt}%
\pgfpathmoveto{\pgfqpoint{2.782465in}{2.032331in}}%
\pgfpathlineto{\pgfqpoint{2.791401in}{2.032331in}}%
\pgfpathlineto{\pgfqpoint{2.791401in}{2.143905in}}%
\pgfpathlineto{\pgfqpoint{2.782465in}{2.143905in}}%
\pgfpathlineto{\pgfqpoint{2.782465in}{2.032331in}}%
\pgfpathclose%
\pgfusepath{fill}%
\end{pgfscope}%
\begin{pgfscope}%
\pgfpathrectangle{\pgfqpoint{0.697024in}{0.857143in}}{\pgfqpoint{2.627103in}{1.813434in}}%
\pgfusepath{clip}%
\pgfsetbuttcap%
\pgfsetmiterjoin%
\definecolor{currentfill}{rgb}{0.133298,0.375282,0.379395}%
\pgfsetfillcolor{currentfill}%
\pgfsetlinewidth{0.000000pt}%
\definecolor{currentstroke}{rgb}{0.000000,0.000000,0.000000}%
\pgfsetstrokecolor{currentstroke}%
\pgfsetstrokeopacity{0.000000}%
\pgfsetdash{}{0pt}%
\pgfpathmoveto{\pgfqpoint{2.793635in}{2.022229in}}%
\pgfpathlineto{\pgfqpoint{2.802572in}{2.022229in}}%
\pgfpathlineto{\pgfqpoint{2.802572in}{2.105041in}}%
\pgfpathlineto{\pgfqpoint{2.793635in}{2.105041in}}%
\pgfpathlineto{\pgfqpoint{2.793635in}{2.022229in}}%
\pgfpathclose%
\pgfusepath{fill}%
\end{pgfscope}%
\begin{pgfscope}%
\pgfpathrectangle{\pgfqpoint{0.697024in}{0.857143in}}{\pgfqpoint{2.627103in}{1.813434in}}%
\pgfusepath{clip}%
\pgfsetbuttcap%
\pgfsetmiterjoin%
\definecolor{currentfill}{rgb}{0.133298,0.375282,0.379395}%
\pgfsetfillcolor{currentfill}%
\pgfsetlinewidth{0.000000pt}%
\definecolor{currentstroke}{rgb}{0.000000,0.000000,0.000000}%
\pgfsetstrokecolor{currentstroke}%
\pgfsetstrokeopacity{0.000000}%
\pgfsetdash{}{0pt}%
\pgfpathmoveto{\pgfqpoint{2.804806in}{1.999940in}}%
\pgfpathlineto{\pgfqpoint{2.813742in}{1.999940in}}%
\pgfpathlineto{\pgfqpoint{2.813742in}{2.119431in}}%
\pgfpathlineto{\pgfqpoint{2.804806in}{2.119431in}}%
\pgfpathlineto{\pgfqpoint{2.804806in}{1.999940in}}%
\pgfpathclose%
\pgfusepath{fill}%
\end{pgfscope}%
\begin{pgfscope}%
\pgfpathrectangle{\pgfqpoint{0.697024in}{0.857143in}}{\pgfqpoint{2.627103in}{1.813434in}}%
\pgfusepath{clip}%
\pgfsetbuttcap%
\pgfsetmiterjoin%
\definecolor{currentfill}{rgb}{0.133298,0.375282,0.379395}%
\pgfsetfillcolor{currentfill}%
\pgfsetlinewidth{0.000000pt}%
\definecolor{currentstroke}{rgb}{0.000000,0.000000,0.000000}%
\pgfsetstrokecolor{currentstroke}%
\pgfsetstrokeopacity{0.000000}%
\pgfsetdash{}{0pt}%
\pgfpathmoveto{\pgfqpoint{2.815977in}{1.982260in}}%
\pgfpathlineto{\pgfqpoint{2.824913in}{1.982260in}}%
\pgfpathlineto{\pgfqpoint{2.824913in}{2.009777in}}%
\pgfpathlineto{\pgfqpoint{2.815977in}{2.009777in}}%
\pgfpathlineto{\pgfqpoint{2.815977in}{1.982260in}}%
\pgfpathclose%
\pgfusepath{fill}%
\end{pgfscope}%
\begin{pgfscope}%
\pgfpathrectangle{\pgfqpoint{0.697024in}{0.857143in}}{\pgfqpoint{2.627103in}{1.813434in}}%
\pgfusepath{clip}%
\pgfsetbuttcap%
\pgfsetmiterjoin%
\definecolor{currentfill}{rgb}{0.133298,0.375282,0.379395}%
\pgfsetfillcolor{currentfill}%
\pgfsetlinewidth{0.000000pt}%
\definecolor{currentstroke}{rgb}{0.000000,0.000000,0.000000}%
\pgfsetstrokecolor{currentstroke}%
\pgfsetstrokeopacity{0.000000}%
\pgfsetdash{}{0pt}%
\pgfpathmoveto{\pgfqpoint{2.827147in}{1.965698in}}%
\pgfpathlineto{\pgfqpoint{2.836084in}{1.965698in}}%
\pgfpathlineto{\pgfqpoint{2.836084in}{2.013588in}}%
\pgfpathlineto{\pgfqpoint{2.827147in}{2.013588in}}%
\pgfpathlineto{\pgfqpoint{2.827147in}{1.965698in}}%
\pgfpathclose%
\pgfusepath{fill}%
\end{pgfscope}%
\begin{pgfscope}%
\pgfpathrectangle{\pgfqpoint{0.697024in}{0.857143in}}{\pgfqpoint{2.627103in}{1.813434in}}%
\pgfusepath{clip}%
\pgfsetbuttcap%
\pgfsetmiterjoin%
\definecolor{currentfill}{rgb}{0.133298,0.375282,0.379395}%
\pgfsetfillcolor{currentfill}%
\pgfsetlinewidth{0.000000pt}%
\definecolor{currentstroke}{rgb}{0.000000,0.000000,0.000000}%
\pgfsetstrokecolor{currentstroke}%
\pgfsetstrokeopacity{0.000000}%
\pgfsetdash{}{0pt}%
\pgfpathmoveto{\pgfqpoint{2.838318in}{1.963374in}}%
\pgfpathlineto{\pgfqpoint{2.847254in}{1.963374in}}%
\pgfpathlineto{\pgfqpoint{2.847254in}{2.133892in}}%
\pgfpathlineto{\pgfqpoint{2.838318in}{2.133892in}}%
\pgfpathlineto{\pgfqpoint{2.838318in}{1.963374in}}%
\pgfpathclose%
\pgfusepath{fill}%
\end{pgfscope}%
\begin{pgfscope}%
\pgfpathrectangle{\pgfqpoint{0.697024in}{0.857143in}}{\pgfqpoint{2.627103in}{1.813434in}}%
\pgfusepath{clip}%
\pgfsetbuttcap%
\pgfsetmiterjoin%
\definecolor{currentfill}{rgb}{0.133298,0.375282,0.379395}%
\pgfsetfillcolor{currentfill}%
\pgfsetlinewidth{0.000000pt}%
\definecolor{currentstroke}{rgb}{0.000000,0.000000,0.000000}%
\pgfsetstrokecolor{currentstroke}%
\pgfsetstrokeopacity{0.000000}%
\pgfsetdash{}{0pt}%
\pgfpathmoveto{\pgfqpoint{2.849488in}{1.956662in}}%
\pgfpathlineto{\pgfqpoint{2.858425in}{1.956662in}}%
\pgfpathlineto{\pgfqpoint{2.858425in}{1.957877in}}%
\pgfpathlineto{\pgfqpoint{2.849488in}{1.957877in}}%
\pgfpathlineto{\pgfqpoint{2.849488in}{1.956662in}}%
\pgfpathclose%
\pgfusepath{fill}%
\end{pgfscope}%
\begin{pgfscope}%
\pgfpathrectangle{\pgfqpoint{0.697024in}{0.857143in}}{\pgfqpoint{2.627103in}{1.813434in}}%
\pgfusepath{clip}%
\pgfsetbuttcap%
\pgfsetmiterjoin%
\definecolor{currentfill}{rgb}{0.133298,0.375282,0.379395}%
\pgfsetfillcolor{currentfill}%
\pgfsetlinewidth{0.000000pt}%
\definecolor{currentstroke}{rgb}{0.000000,0.000000,0.000000}%
\pgfsetstrokecolor{currentstroke}%
\pgfsetstrokeopacity{0.000000}%
\pgfsetdash{}{0pt}%
\pgfpathmoveto{\pgfqpoint{2.860659in}{1.955717in}}%
\pgfpathlineto{\pgfqpoint{2.869595in}{1.955717in}}%
\pgfpathlineto{\pgfqpoint{2.869595in}{2.022127in}}%
\pgfpathlineto{\pgfqpoint{2.860659in}{2.022127in}}%
\pgfpathlineto{\pgfqpoint{2.860659in}{1.955717in}}%
\pgfpathclose%
\pgfusepath{fill}%
\end{pgfscope}%
\begin{pgfscope}%
\pgfpathrectangle{\pgfqpoint{0.697024in}{0.857143in}}{\pgfqpoint{2.627103in}{1.813434in}}%
\pgfusepath{clip}%
\pgfsetbuttcap%
\pgfsetmiterjoin%
\definecolor{currentfill}{rgb}{0.133298,0.375282,0.379395}%
\pgfsetfillcolor{currentfill}%
\pgfsetlinewidth{0.000000pt}%
\definecolor{currentstroke}{rgb}{0.000000,0.000000,0.000000}%
\pgfsetstrokecolor{currentstroke}%
\pgfsetstrokeopacity{0.000000}%
\pgfsetdash{}{0pt}%
\pgfpathmoveto{\pgfqpoint{2.871830in}{1.932239in}}%
\pgfpathlineto{\pgfqpoint{2.880766in}{1.932239in}}%
\pgfpathlineto{\pgfqpoint{2.880766in}{1.945439in}}%
\pgfpathlineto{\pgfqpoint{2.871830in}{1.945439in}}%
\pgfpathlineto{\pgfqpoint{2.871830in}{1.932239in}}%
\pgfpathclose%
\pgfusepath{fill}%
\end{pgfscope}%
\begin{pgfscope}%
\pgfpathrectangle{\pgfqpoint{0.697024in}{0.857143in}}{\pgfqpoint{2.627103in}{1.813434in}}%
\pgfusepath{clip}%
\pgfsetbuttcap%
\pgfsetmiterjoin%
\definecolor{currentfill}{rgb}{0.133298,0.375282,0.379395}%
\pgfsetfillcolor{currentfill}%
\pgfsetlinewidth{0.000000pt}%
\definecolor{currentstroke}{rgb}{0.000000,0.000000,0.000000}%
\pgfsetstrokecolor{currentstroke}%
\pgfsetstrokeopacity{0.000000}%
\pgfsetdash{}{0pt}%
\pgfpathmoveto{\pgfqpoint{2.883000in}{1.910011in}}%
\pgfpathlineto{\pgfqpoint{2.891937in}{1.910011in}}%
\pgfpathlineto{\pgfqpoint{2.891937in}{1.914174in}}%
\pgfpathlineto{\pgfqpoint{2.883000in}{1.914174in}}%
\pgfpathlineto{\pgfqpoint{2.883000in}{1.910011in}}%
\pgfpathclose%
\pgfusepath{fill}%
\end{pgfscope}%
\begin{pgfscope}%
\pgfpathrectangle{\pgfqpoint{0.697024in}{0.857143in}}{\pgfqpoint{2.627103in}{1.813434in}}%
\pgfusepath{clip}%
\pgfsetbuttcap%
\pgfsetmiterjoin%
\definecolor{currentfill}{rgb}{0.133298,0.375282,0.379395}%
\pgfsetfillcolor{currentfill}%
\pgfsetlinewidth{0.000000pt}%
\definecolor{currentstroke}{rgb}{0.000000,0.000000,0.000000}%
\pgfsetstrokecolor{currentstroke}%
\pgfsetstrokeopacity{0.000000}%
\pgfsetdash{}{0pt}%
\pgfpathmoveto{\pgfqpoint{2.894171in}{1.911618in}}%
\pgfpathlineto{\pgfqpoint{2.903107in}{1.911618in}}%
\pgfpathlineto{\pgfqpoint{2.903107in}{1.988232in}}%
\pgfpathlineto{\pgfqpoint{2.894171in}{1.988232in}}%
\pgfpathlineto{\pgfqpoint{2.894171in}{1.911618in}}%
\pgfpathclose%
\pgfusepath{fill}%
\end{pgfscope}%
\begin{pgfscope}%
\pgfpathrectangle{\pgfqpoint{0.697024in}{0.857143in}}{\pgfqpoint{2.627103in}{1.813434in}}%
\pgfusepath{clip}%
\pgfsetbuttcap%
\pgfsetmiterjoin%
\definecolor{currentfill}{rgb}{0.133298,0.375282,0.379395}%
\pgfsetfillcolor{currentfill}%
\pgfsetlinewidth{0.000000pt}%
\definecolor{currentstroke}{rgb}{0.000000,0.000000,0.000000}%
\pgfsetstrokecolor{currentstroke}%
\pgfsetstrokeopacity{0.000000}%
\pgfsetdash{}{0pt}%
\pgfpathmoveto{\pgfqpoint{2.905341in}{1.897722in}}%
\pgfpathlineto{\pgfqpoint{2.914278in}{1.897722in}}%
\pgfpathlineto{\pgfqpoint{2.914278in}{2.030786in}}%
\pgfpathlineto{\pgfqpoint{2.905341in}{2.030786in}}%
\pgfpathlineto{\pgfqpoint{2.905341in}{1.897722in}}%
\pgfpathclose%
\pgfusepath{fill}%
\end{pgfscope}%
\begin{pgfscope}%
\pgfpathrectangle{\pgfqpoint{0.697024in}{0.857143in}}{\pgfqpoint{2.627103in}{1.813434in}}%
\pgfusepath{clip}%
\pgfsetbuttcap%
\pgfsetmiterjoin%
\definecolor{currentfill}{rgb}{0.133298,0.375282,0.379395}%
\pgfsetfillcolor{currentfill}%
\pgfsetlinewidth{0.000000pt}%
\definecolor{currentstroke}{rgb}{0.000000,0.000000,0.000000}%
\pgfsetstrokecolor{currentstroke}%
\pgfsetstrokeopacity{0.000000}%
\pgfsetdash{}{0pt}%
\pgfpathmoveto{\pgfqpoint{2.916512in}{1.901874in}}%
\pgfpathlineto{\pgfqpoint{2.925448in}{1.901874in}}%
\pgfpathlineto{\pgfqpoint{2.925448in}{1.951917in}}%
\pgfpathlineto{\pgfqpoint{2.916512in}{1.951917in}}%
\pgfpathlineto{\pgfqpoint{2.916512in}{1.901874in}}%
\pgfpathclose%
\pgfusepath{fill}%
\end{pgfscope}%
\begin{pgfscope}%
\pgfpathrectangle{\pgfqpoint{0.697024in}{0.857143in}}{\pgfqpoint{2.627103in}{1.813434in}}%
\pgfusepath{clip}%
\pgfsetbuttcap%
\pgfsetmiterjoin%
\definecolor{currentfill}{rgb}{0.133298,0.375282,0.379395}%
\pgfsetfillcolor{currentfill}%
\pgfsetlinewidth{0.000000pt}%
\definecolor{currentstroke}{rgb}{0.000000,0.000000,0.000000}%
\pgfsetstrokecolor{currentstroke}%
\pgfsetstrokeopacity{0.000000}%
\pgfsetdash{}{0pt}%
\pgfpathmoveto{\pgfqpoint{2.927683in}{1.908171in}}%
\pgfpathlineto{\pgfqpoint{2.936619in}{1.908171in}}%
\pgfpathlineto{\pgfqpoint{2.936619in}{1.925300in}}%
\pgfpathlineto{\pgfqpoint{2.927683in}{1.925300in}}%
\pgfpathlineto{\pgfqpoint{2.927683in}{1.908171in}}%
\pgfpathclose%
\pgfusepath{fill}%
\end{pgfscope}%
\begin{pgfscope}%
\pgfpathrectangle{\pgfqpoint{0.697024in}{0.857143in}}{\pgfqpoint{2.627103in}{1.813434in}}%
\pgfusepath{clip}%
\pgfsetbuttcap%
\pgfsetmiterjoin%
\definecolor{currentfill}{rgb}{0.133298,0.375282,0.379395}%
\pgfsetfillcolor{currentfill}%
\pgfsetlinewidth{0.000000pt}%
\definecolor{currentstroke}{rgb}{0.000000,0.000000,0.000000}%
\pgfsetstrokecolor{currentstroke}%
\pgfsetstrokeopacity{0.000000}%
\pgfsetdash{}{0pt}%
\pgfpathmoveto{\pgfqpoint{2.938853in}{1.873308in}}%
\pgfpathlineto{\pgfqpoint{2.947790in}{1.873308in}}%
\pgfpathlineto{\pgfqpoint{2.947790in}{1.970502in}}%
\pgfpathlineto{\pgfqpoint{2.938853in}{1.970502in}}%
\pgfpathlineto{\pgfqpoint{2.938853in}{1.873308in}}%
\pgfpathclose%
\pgfusepath{fill}%
\end{pgfscope}%
\begin{pgfscope}%
\pgfpathrectangle{\pgfqpoint{0.697024in}{0.857143in}}{\pgfqpoint{2.627103in}{1.813434in}}%
\pgfusepath{clip}%
\pgfsetbuttcap%
\pgfsetmiterjoin%
\definecolor{currentfill}{rgb}{0.133298,0.375282,0.379395}%
\pgfsetfillcolor{currentfill}%
\pgfsetlinewidth{0.000000pt}%
\definecolor{currentstroke}{rgb}{0.000000,0.000000,0.000000}%
\pgfsetstrokecolor{currentstroke}%
\pgfsetstrokeopacity{0.000000}%
\pgfsetdash{}{0pt}%
\pgfpathmoveto{\pgfqpoint{2.950024in}{1.872685in}}%
\pgfpathlineto{\pgfqpoint{2.958960in}{1.872685in}}%
\pgfpathlineto{\pgfqpoint{2.958960in}{1.931238in}}%
\pgfpathlineto{\pgfqpoint{2.950024in}{1.931238in}}%
\pgfpathlineto{\pgfqpoint{2.950024in}{1.872685in}}%
\pgfpathclose%
\pgfusepath{fill}%
\end{pgfscope}%
\begin{pgfscope}%
\pgfpathrectangle{\pgfqpoint{0.697024in}{0.857143in}}{\pgfqpoint{2.627103in}{1.813434in}}%
\pgfusepath{clip}%
\pgfsetbuttcap%
\pgfsetmiterjoin%
\definecolor{currentfill}{rgb}{0.133298,0.375282,0.379395}%
\pgfsetfillcolor{currentfill}%
\pgfsetlinewidth{0.000000pt}%
\definecolor{currentstroke}{rgb}{0.000000,0.000000,0.000000}%
\pgfsetstrokecolor{currentstroke}%
\pgfsetstrokeopacity{0.000000}%
\pgfsetdash{}{0pt}%
\pgfpathmoveto{\pgfqpoint{2.961194in}{1.874790in}}%
\pgfpathlineto{\pgfqpoint{2.970131in}{1.874790in}}%
\pgfpathlineto{\pgfqpoint{2.970131in}{1.952906in}}%
\pgfpathlineto{\pgfqpoint{2.961194in}{1.952906in}}%
\pgfpathlineto{\pgfqpoint{2.961194in}{1.874790in}}%
\pgfpathclose%
\pgfusepath{fill}%
\end{pgfscope}%
\begin{pgfscope}%
\pgfpathrectangle{\pgfqpoint{0.697024in}{0.857143in}}{\pgfqpoint{2.627103in}{1.813434in}}%
\pgfusepath{clip}%
\pgfsetbuttcap%
\pgfsetmiterjoin%
\definecolor{currentfill}{rgb}{0.133298,0.375282,0.379395}%
\pgfsetfillcolor{currentfill}%
\pgfsetlinewidth{0.000000pt}%
\definecolor{currentstroke}{rgb}{0.000000,0.000000,0.000000}%
\pgfsetstrokecolor{currentstroke}%
\pgfsetstrokeopacity{0.000000}%
\pgfsetdash{}{0pt}%
\pgfpathmoveto{\pgfqpoint{2.972365in}{1.848178in}}%
\pgfpathlineto{\pgfqpoint{2.981301in}{1.848178in}}%
\pgfpathlineto{\pgfqpoint{2.981301in}{2.020457in}}%
\pgfpathlineto{\pgfqpoint{2.972365in}{2.020457in}}%
\pgfpathlineto{\pgfqpoint{2.972365in}{1.848178in}}%
\pgfpathclose%
\pgfusepath{fill}%
\end{pgfscope}%
\begin{pgfscope}%
\pgfpathrectangle{\pgfqpoint{0.697024in}{0.857143in}}{\pgfqpoint{2.627103in}{1.813434in}}%
\pgfusepath{clip}%
\pgfsetbuttcap%
\pgfsetmiterjoin%
\definecolor{currentfill}{rgb}{0.133298,0.375282,0.379395}%
\pgfsetfillcolor{currentfill}%
\pgfsetlinewidth{0.000000pt}%
\definecolor{currentstroke}{rgb}{0.000000,0.000000,0.000000}%
\pgfsetstrokecolor{currentstroke}%
\pgfsetstrokeopacity{0.000000}%
\pgfsetdash{}{0pt}%
\pgfpathmoveto{\pgfqpoint{2.983536in}{1.851318in}}%
\pgfpathlineto{\pgfqpoint{2.992472in}{1.851318in}}%
\pgfpathlineto{\pgfqpoint{2.992472in}{2.090520in}}%
\pgfpathlineto{\pgfqpoint{2.983536in}{2.090520in}}%
\pgfpathlineto{\pgfqpoint{2.983536in}{1.851318in}}%
\pgfpathclose%
\pgfusepath{fill}%
\end{pgfscope}%
\begin{pgfscope}%
\pgfpathrectangle{\pgfqpoint{0.697024in}{0.857143in}}{\pgfqpoint{2.627103in}{1.813434in}}%
\pgfusepath{clip}%
\pgfsetbuttcap%
\pgfsetmiterjoin%
\definecolor{currentfill}{rgb}{0.133298,0.375282,0.379395}%
\pgfsetfillcolor{currentfill}%
\pgfsetlinewidth{0.000000pt}%
\definecolor{currentstroke}{rgb}{0.000000,0.000000,0.000000}%
\pgfsetstrokecolor{currentstroke}%
\pgfsetstrokeopacity{0.000000}%
\pgfsetdash{}{0pt}%
\pgfpathmoveto{\pgfqpoint{2.994706in}{1.851047in}}%
\pgfpathlineto{\pgfqpoint{3.003643in}{1.851047in}}%
\pgfpathlineto{\pgfqpoint{3.003643in}{1.936448in}}%
\pgfpathlineto{\pgfqpoint{2.994706in}{1.936448in}}%
\pgfpathlineto{\pgfqpoint{2.994706in}{1.851047in}}%
\pgfpathclose%
\pgfusepath{fill}%
\end{pgfscope}%
\begin{pgfscope}%
\pgfpathrectangle{\pgfqpoint{0.697024in}{0.857143in}}{\pgfqpoint{2.627103in}{1.813434in}}%
\pgfusepath{clip}%
\pgfsetbuttcap%
\pgfsetmiterjoin%
\definecolor{currentfill}{rgb}{0.133298,0.375282,0.379395}%
\pgfsetfillcolor{currentfill}%
\pgfsetlinewidth{0.000000pt}%
\definecolor{currentstroke}{rgb}{0.000000,0.000000,0.000000}%
\pgfsetstrokecolor{currentstroke}%
\pgfsetstrokeopacity{0.000000}%
\pgfsetdash{}{0pt}%
\pgfpathmoveto{\pgfqpoint{3.005877in}{1.847595in}}%
\pgfpathlineto{\pgfqpoint{3.014813in}{1.847595in}}%
\pgfpathlineto{\pgfqpoint{3.014813in}{2.021404in}}%
\pgfpathlineto{\pgfqpoint{3.005877in}{2.021404in}}%
\pgfpathlineto{\pgfqpoint{3.005877in}{1.847595in}}%
\pgfpathclose%
\pgfusepath{fill}%
\end{pgfscope}%
\begin{pgfscope}%
\pgfpathrectangle{\pgfqpoint{0.697024in}{0.857143in}}{\pgfqpoint{2.627103in}{1.813434in}}%
\pgfusepath{clip}%
\pgfsetbuttcap%
\pgfsetmiterjoin%
\definecolor{currentfill}{rgb}{0.133298,0.375282,0.379395}%
\pgfsetfillcolor{currentfill}%
\pgfsetlinewidth{0.000000pt}%
\definecolor{currentstroke}{rgb}{0.000000,0.000000,0.000000}%
\pgfsetstrokecolor{currentstroke}%
\pgfsetstrokeopacity{0.000000}%
\pgfsetdash{}{0pt}%
\pgfpathmoveto{\pgfqpoint{3.017047in}{1.847071in}}%
\pgfpathlineto{\pgfqpoint{3.025984in}{1.847071in}}%
\pgfpathlineto{\pgfqpoint{3.025984in}{2.125699in}}%
\pgfpathlineto{\pgfqpoint{3.017047in}{2.125699in}}%
\pgfpathlineto{\pgfqpoint{3.017047in}{1.847071in}}%
\pgfpathclose%
\pgfusepath{fill}%
\end{pgfscope}%
\begin{pgfscope}%
\pgfpathrectangle{\pgfqpoint{0.697024in}{0.857143in}}{\pgfqpoint{2.627103in}{1.813434in}}%
\pgfusepath{clip}%
\pgfsetbuttcap%
\pgfsetmiterjoin%
\definecolor{currentfill}{rgb}{0.133298,0.375282,0.379395}%
\pgfsetfillcolor{currentfill}%
\pgfsetlinewidth{0.000000pt}%
\definecolor{currentstroke}{rgb}{0.000000,0.000000,0.000000}%
\pgfsetstrokecolor{currentstroke}%
\pgfsetstrokeopacity{0.000000}%
\pgfsetdash{}{0pt}%
\pgfpathmoveto{\pgfqpoint{3.028218in}{1.847071in}}%
\pgfpathlineto{\pgfqpoint{3.037155in}{1.847071in}}%
\pgfpathlineto{\pgfqpoint{3.037155in}{2.180838in}}%
\pgfpathlineto{\pgfqpoint{3.028218in}{2.180838in}}%
\pgfpathlineto{\pgfqpoint{3.028218in}{1.847071in}}%
\pgfpathclose%
\pgfusepath{fill}%
\end{pgfscope}%
\begin{pgfscope}%
\pgfpathrectangle{\pgfqpoint{0.697024in}{0.857143in}}{\pgfqpoint{2.627103in}{1.813434in}}%
\pgfusepath{clip}%
\pgfsetbuttcap%
\pgfsetmiterjoin%
\definecolor{currentfill}{rgb}{0.133298,0.375282,0.379395}%
\pgfsetfillcolor{currentfill}%
\pgfsetlinewidth{0.000000pt}%
\definecolor{currentstroke}{rgb}{0.000000,0.000000,0.000000}%
\pgfsetstrokecolor{currentstroke}%
\pgfsetstrokeopacity{0.000000}%
\pgfsetdash{}{0pt}%
\pgfpathmoveto{\pgfqpoint{3.039389in}{1.847071in}}%
\pgfpathlineto{\pgfqpoint{3.048325in}{1.847071in}}%
\pgfpathlineto{\pgfqpoint{3.048325in}{2.016049in}}%
\pgfpathlineto{\pgfqpoint{3.039389in}{2.016049in}}%
\pgfpathlineto{\pgfqpoint{3.039389in}{1.847071in}}%
\pgfpathclose%
\pgfusepath{fill}%
\end{pgfscope}%
\begin{pgfscope}%
\pgfpathrectangle{\pgfqpoint{0.697024in}{0.857143in}}{\pgfqpoint{2.627103in}{1.813434in}}%
\pgfusepath{clip}%
\pgfsetbuttcap%
\pgfsetmiterjoin%
\definecolor{currentfill}{rgb}{0.133298,0.375282,0.379395}%
\pgfsetfillcolor{currentfill}%
\pgfsetlinewidth{0.000000pt}%
\definecolor{currentstroke}{rgb}{0.000000,0.000000,0.000000}%
\pgfsetstrokecolor{currentstroke}%
\pgfsetstrokeopacity{0.000000}%
\pgfsetdash{}{0pt}%
\pgfpathmoveto{\pgfqpoint{3.050559in}{1.847071in}}%
\pgfpathlineto{\pgfqpoint{3.059496in}{1.847071in}}%
\pgfpathlineto{\pgfqpoint{3.059496in}{2.128011in}}%
\pgfpathlineto{\pgfqpoint{3.050559in}{2.128011in}}%
\pgfpathlineto{\pgfqpoint{3.050559in}{1.847071in}}%
\pgfpathclose%
\pgfusepath{fill}%
\end{pgfscope}%
\begin{pgfscope}%
\pgfpathrectangle{\pgfqpoint{0.697024in}{0.857143in}}{\pgfqpoint{2.627103in}{1.813434in}}%
\pgfusepath{clip}%
\pgfsetbuttcap%
\pgfsetmiterjoin%
\definecolor{currentfill}{rgb}{0.133298,0.375282,0.379395}%
\pgfsetfillcolor{currentfill}%
\pgfsetlinewidth{0.000000pt}%
\definecolor{currentstroke}{rgb}{0.000000,0.000000,0.000000}%
\pgfsetstrokecolor{currentstroke}%
\pgfsetstrokeopacity{0.000000}%
\pgfsetdash{}{0pt}%
\pgfpathmoveto{\pgfqpoint{3.061730in}{1.847071in}}%
\pgfpathlineto{\pgfqpoint{3.070666in}{1.847071in}}%
\pgfpathlineto{\pgfqpoint{3.070666in}{2.062674in}}%
\pgfpathlineto{\pgfqpoint{3.061730in}{2.062674in}}%
\pgfpathlineto{\pgfqpoint{3.061730in}{1.847071in}}%
\pgfpathclose%
\pgfusepath{fill}%
\end{pgfscope}%
\begin{pgfscope}%
\pgfpathrectangle{\pgfqpoint{0.697024in}{0.857143in}}{\pgfqpoint{2.627103in}{1.813434in}}%
\pgfusepath{clip}%
\pgfsetbuttcap%
\pgfsetmiterjoin%
\definecolor{currentfill}{rgb}{0.133298,0.375282,0.379395}%
\pgfsetfillcolor{currentfill}%
\pgfsetlinewidth{0.000000pt}%
\definecolor{currentstroke}{rgb}{0.000000,0.000000,0.000000}%
\pgfsetstrokecolor{currentstroke}%
\pgfsetstrokeopacity{0.000000}%
\pgfsetdash{}{0pt}%
\pgfpathmoveto{\pgfqpoint{3.072900in}{1.847071in}}%
\pgfpathlineto{\pgfqpoint{3.081837in}{1.847071in}}%
\pgfpathlineto{\pgfqpoint{3.081837in}{2.065984in}}%
\pgfpathlineto{\pgfqpoint{3.072900in}{2.065984in}}%
\pgfpathlineto{\pgfqpoint{3.072900in}{1.847071in}}%
\pgfpathclose%
\pgfusepath{fill}%
\end{pgfscope}%
\begin{pgfscope}%
\pgfpathrectangle{\pgfqpoint{0.697024in}{0.857143in}}{\pgfqpoint{2.627103in}{1.813434in}}%
\pgfusepath{clip}%
\pgfsetbuttcap%
\pgfsetmiterjoin%
\definecolor{currentfill}{rgb}{0.133298,0.375282,0.379395}%
\pgfsetfillcolor{currentfill}%
\pgfsetlinewidth{0.000000pt}%
\definecolor{currentstroke}{rgb}{0.000000,0.000000,0.000000}%
\pgfsetstrokecolor{currentstroke}%
\pgfsetstrokeopacity{0.000000}%
\pgfsetdash{}{0pt}%
\pgfpathmoveto{\pgfqpoint{3.084071in}{1.847071in}}%
\pgfpathlineto{\pgfqpoint{3.093008in}{1.847071in}}%
\pgfpathlineto{\pgfqpoint{3.093008in}{2.153858in}}%
\pgfpathlineto{\pgfqpoint{3.084071in}{2.153858in}}%
\pgfpathlineto{\pgfqpoint{3.084071in}{1.847071in}}%
\pgfpathclose%
\pgfusepath{fill}%
\end{pgfscope}%
\begin{pgfscope}%
\pgfpathrectangle{\pgfqpoint{0.697024in}{0.857143in}}{\pgfqpoint{2.627103in}{1.813434in}}%
\pgfusepath{clip}%
\pgfsetbuttcap%
\pgfsetmiterjoin%
\definecolor{currentfill}{rgb}{0.133298,0.375282,0.379395}%
\pgfsetfillcolor{currentfill}%
\pgfsetlinewidth{0.000000pt}%
\definecolor{currentstroke}{rgb}{0.000000,0.000000,0.000000}%
\pgfsetstrokecolor{currentstroke}%
\pgfsetstrokeopacity{0.000000}%
\pgfsetdash{}{0pt}%
\pgfpathmoveto{\pgfqpoint{3.095242in}{1.847071in}}%
\pgfpathlineto{\pgfqpoint{3.104178in}{1.847071in}}%
\pgfpathlineto{\pgfqpoint{3.104178in}{2.077878in}}%
\pgfpathlineto{\pgfqpoint{3.095242in}{2.077878in}}%
\pgfpathlineto{\pgfqpoint{3.095242in}{1.847071in}}%
\pgfpathclose%
\pgfusepath{fill}%
\end{pgfscope}%
\begin{pgfscope}%
\pgfpathrectangle{\pgfqpoint{0.697024in}{0.857143in}}{\pgfqpoint{2.627103in}{1.813434in}}%
\pgfusepath{clip}%
\pgfsetbuttcap%
\pgfsetmiterjoin%
\definecolor{currentfill}{rgb}{0.133298,0.375282,0.379395}%
\pgfsetfillcolor{currentfill}%
\pgfsetlinewidth{0.000000pt}%
\definecolor{currentstroke}{rgb}{0.000000,0.000000,0.000000}%
\pgfsetstrokecolor{currentstroke}%
\pgfsetstrokeopacity{0.000000}%
\pgfsetdash{}{0pt}%
\pgfpathmoveto{\pgfqpoint{3.106412in}{1.847071in}}%
\pgfpathlineto{\pgfqpoint{3.115349in}{1.847071in}}%
\pgfpathlineto{\pgfqpoint{3.115349in}{2.032722in}}%
\pgfpathlineto{\pgfqpoint{3.106412in}{2.032722in}}%
\pgfpathlineto{\pgfqpoint{3.106412in}{1.847071in}}%
\pgfpathclose%
\pgfusepath{fill}%
\end{pgfscope}%
\begin{pgfscope}%
\pgfpathrectangle{\pgfqpoint{0.697024in}{0.857143in}}{\pgfqpoint{2.627103in}{1.813434in}}%
\pgfusepath{clip}%
\pgfsetbuttcap%
\pgfsetmiterjoin%
\definecolor{currentfill}{rgb}{0.133298,0.375282,0.379395}%
\pgfsetfillcolor{currentfill}%
\pgfsetlinewidth{0.000000pt}%
\definecolor{currentstroke}{rgb}{0.000000,0.000000,0.000000}%
\pgfsetstrokecolor{currentstroke}%
\pgfsetstrokeopacity{0.000000}%
\pgfsetdash{}{0pt}%
\pgfpathmoveto{\pgfqpoint{3.117583in}{1.847071in}}%
\pgfpathlineto{\pgfqpoint{3.126519in}{1.847071in}}%
\pgfpathlineto{\pgfqpoint{3.126519in}{2.063725in}}%
\pgfpathlineto{\pgfqpoint{3.117583in}{2.063725in}}%
\pgfpathlineto{\pgfqpoint{3.117583in}{1.847071in}}%
\pgfpathclose%
\pgfusepath{fill}%
\end{pgfscope}%
\begin{pgfscope}%
\pgfpathrectangle{\pgfqpoint{0.697024in}{0.857143in}}{\pgfqpoint{2.627103in}{1.813434in}}%
\pgfusepath{clip}%
\pgfsetbuttcap%
\pgfsetmiterjoin%
\definecolor{currentfill}{rgb}{0.133298,0.375282,0.379395}%
\pgfsetfillcolor{currentfill}%
\pgfsetlinewidth{0.000000pt}%
\definecolor{currentstroke}{rgb}{0.000000,0.000000,0.000000}%
\pgfsetstrokecolor{currentstroke}%
\pgfsetstrokeopacity{0.000000}%
\pgfsetdash{}{0pt}%
\pgfpathmoveto{\pgfqpoint{3.128753in}{1.847071in}}%
\pgfpathlineto{\pgfqpoint{3.137690in}{1.847071in}}%
\pgfpathlineto{\pgfqpoint{3.137690in}{2.057803in}}%
\pgfpathlineto{\pgfqpoint{3.128753in}{2.057803in}}%
\pgfpathlineto{\pgfqpoint{3.128753in}{1.847071in}}%
\pgfpathclose%
\pgfusepath{fill}%
\end{pgfscope}%
\begin{pgfscope}%
\pgfpathrectangle{\pgfqpoint{0.697024in}{0.857143in}}{\pgfqpoint{2.627103in}{1.813434in}}%
\pgfusepath{clip}%
\pgfsetbuttcap%
\pgfsetmiterjoin%
\definecolor{currentfill}{rgb}{0.133298,0.375282,0.379395}%
\pgfsetfillcolor{currentfill}%
\pgfsetlinewidth{0.000000pt}%
\definecolor{currentstroke}{rgb}{0.000000,0.000000,0.000000}%
\pgfsetstrokecolor{currentstroke}%
\pgfsetstrokeopacity{0.000000}%
\pgfsetdash{}{0pt}%
\pgfpathmoveto{\pgfqpoint{3.139924in}{1.847071in}}%
\pgfpathlineto{\pgfqpoint{3.148861in}{1.847071in}}%
\pgfpathlineto{\pgfqpoint{3.148861in}{2.130247in}}%
\pgfpathlineto{\pgfqpoint{3.139924in}{2.130247in}}%
\pgfpathlineto{\pgfqpoint{3.139924in}{1.847071in}}%
\pgfpathclose%
\pgfusepath{fill}%
\end{pgfscope}%
\begin{pgfscope}%
\pgfpathrectangle{\pgfqpoint{0.697024in}{0.857143in}}{\pgfqpoint{2.627103in}{1.813434in}}%
\pgfusepath{clip}%
\pgfsetbuttcap%
\pgfsetmiterjoin%
\definecolor{currentfill}{rgb}{0.133298,0.375282,0.379395}%
\pgfsetfillcolor{currentfill}%
\pgfsetlinewidth{0.000000pt}%
\definecolor{currentstroke}{rgb}{0.000000,0.000000,0.000000}%
\pgfsetstrokecolor{currentstroke}%
\pgfsetstrokeopacity{0.000000}%
\pgfsetdash{}{0pt}%
\pgfpathmoveto{\pgfqpoint{3.151095in}{1.847071in}}%
\pgfpathlineto{\pgfqpoint{3.160031in}{1.847071in}}%
\pgfpathlineto{\pgfqpoint{3.160031in}{2.166955in}}%
\pgfpathlineto{\pgfqpoint{3.151095in}{2.166955in}}%
\pgfpathlineto{\pgfqpoint{3.151095in}{1.847071in}}%
\pgfpathclose%
\pgfusepath{fill}%
\end{pgfscope}%
\begin{pgfscope}%
\pgfpathrectangle{\pgfqpoint{0.697024in}{0.857143in}}{\pgfqpoint{2.627103in}{1.813434in}}%
\pgfusepath{clip}%
\pgfsetbuttcap%
\pgfsetmiterjoin%
\definecolor{currentfill}{rgb}{0.133298,0.375282,0.379395}%
\pgfsetfillcolor{currentfill}%
\pgfsetlinewidth{0.000000pt}%
\definecolor{currentstroke}{rgb}{0.000000,0.000000,0.000000}%
\pgfsetstrokecolor{currentstroke}%
\pgfsetstrokeopacity{0.000000}%
\pgfsetdash{}{0pt}%
\pgfpathmoveto{\pgfqpoint{3.162265in}{1.847071in}}%
\pgfpathlineto{\pgfqpoint{3.171202in}{1.847071in}}%
\pgfpathlineto{\pgfqpoint{3.171202in}{2.255901in}}%
\pgfpathlineto{\pgfqpoint{3.162265in}{2.255901in}}%
\pgfpathlineto{\pgfqpoint{3.162265in}{1.847071in}}%
\pgfpathclose%
\pgfusepath{fill}%
\end{pgfscope}%
\begin{pgfscope}%
\pgfpathrectangle{\pgfqpoint{0.697024in}{0.857143in}}{\pgfqpoint{2.627103in}{1.813434in}}%
\pgfusepath{clip}%
\pgfsetbuttcap%
\pgfsetmiterjoin%
\definecolor{currentfill}{rgb}{0.133298,0.375282,0.379395}%
\pgfsetfillcolor{currentfill}%
\pgfsetlinewidth{0.000000pt}%
\definecolor{currentstroke}{rgb}{0.000000,0.000000,0.000000}%
\pgfsetstrokecolor{currentstroke}%
\pgfsetstrokeopacity{0.000000}%
\pgfsetdash{}{0pt}%
\pgfpathmoveto{\pgfqpoint{3.173436in}{1.847071in}}%
\pgfpathlineto{\pgfqpoint{3.182372in}{1.847071in}}%
\pgfpathlineto{\pgfqpoint{3.182372in}{2.200243in}}%
\pgfpathlineto{\pgfqpoint{3.173436in}{2.200243in}}%
\pgfpathlineto{\pgfqpoint{3.173436in}{1.847071in}}%
\pgfpathclose%
\pgfusepath{fill}%
\end{pgfscope}%
\begin{pgfscope}%
\pgfpathrectangle{\pgfqpoint{0.697024in}{0.857143in}}{\pgfqpoint{2.627103in}{1.813434in}}%
\pgfusepath{clip}%
\pgfsetbuttcap%
\pgfsetmiterjoin%
\definecolor{currentfill}{rgb}{0.133298,0.375282,0.379395}%
\pgfsetfillcolor{currentfill}%
\pgfsetlinewidth{0.000000pt}%
\definecolor{currentstroke}{rgb}{0.000000,0.000000,0.000000}%
\pgfsetstrokecolor{currentstroke}%
\pgfsetstrokeopacity{0.000000}%
\pgfsetdash{}{0pt}%
\pgfpathmoveto{\pgfqpoint{3.184607in}{1.847071in}}%
\pgfpathlineto{\pgfqpoint{3.193543in}{1.847071in}}%
\pgfpathlineto{\pgfqpoint{3.193543in}{2.258040in}}%
\pgfpathlineto{\pgfqpoint{3.184607in}{2.258040in}}%
\pgfpathlineto{\pgfqpoint{3.184607in}{1.847071in}}%
\pgfpathclose%
\pgfusepath{fill}%
\end{pgfscope}%
\begin{pgfscope}%
\pgfpathrectangle{\pgfqpoint{0.697024in}{0.857143in}}{\pgfqpoint{2.627103in}{1.813434in}}%
\pgfusepath{clip}%
\pgfsetbuttcap%
\pgfsetmiterjoin%
\definecolor{currentfill}{rgb}{0.133298,0.375282,0.379395}%
\pgfsetfillcolor{currentfill}%
\pgfsetlinewidth{0.000000pt}%
\definecolor{currentstroke}{rgb}{0.000000,0.000000,0.000000}%
\pgfsetstrokecolor{currentstroke}%
\pgfsetstrokeopacity{0.000000}%
\pgfsetdash{}{0pt}%
\pgfpathmoveto{\pgfqpoint{3.195777in}{1.847071in}}%
\pgfpathlineto{\pgfqpoint{3.204714in}{1.847071in}}%
\pgfpathlineto{\pgfqpoint{3.204714in}{2.268961in}}%
\pgfpathlineto{\pgfqpoint{3.195777in}{2.268961in}}%
\pgfpathlineto{\pgfqpoint{3.195777in}{1.847071in}}%
\pgfpathclose%
\pgfusepath{fill}%
\end{pgfscope}%
\begin{pgfscope}%
\pgfpathrectangle{\pgfqpoint{0.697024in}{0.857143in}}{\pgfqpoint{2.627103in}{1.813434in}}%
\pgfusepath{clip}%
\pgfsetbuttcap%
\pgfsetmiterjoin%
\definecolor{currentfill}{rgb}{0.302379,0.450282,0.300122}%
\pgfsetfillcolor{currentfill}%
\pgfsetlinewidth{0.000000pt}%
\definecolor{currentstroke}{rgb}{0.000000,0.000000,0.000000}%
\pgfsetstrokecolor{currentstroke}%
\pgfsetstrokeopacity{0.000000}%
\pgfsetdash{}{0pt}%
\pgfpathmoveto{\pgfqpoint{0.816438in}{1.834354in}}%
\pgfpathlineto{\pgfqpoint{0.825375in}{1.834354in}}%
\pgfpathlineto{\pgfqpoint{0.825375in}{1.600305in}}%
\pgfpathlineto{\pgfqpoint{0.816438in}{1.600305in}}%
\pgfpathlineto{\pgfqpoint{0.816438in}{1.834354in}}%
\pgfpathclose%
\pgfusepath{fill}%
\end{pgfscope}%
\begin{pgfscope}%
\pgfpathrectangle{\pgfqpoint{0.697024in}{0.857143in}}{\pgfqpoint{2.627103in}{1.813434in}}%
\pgfusepath{clip}%
\pgfsetbuttcap%
\pgfsetmiterjoin%
\definecolor{currentfill}{rgb}{0.302379,0.450282,0.300122}%
\pgfsetfillcolor{currentfill}%
\pgfsetlinewidth{0.000000pt}%
\definecolor{currentstroke}{rgb}{0.000000,0.000000,0.000000}%
\pgfsetstrokecolor{currentstroke}%
\pgfsetstrokeopacity{0.000000}%
\pgfsetdash{}{0pt}%
\pgfpathmoveto{\pgfqpoint{0.827609in}{1.847071in}}%
\pgfpathlineto{\pgfqpoint{0.836545in}{1.847071in}}%
\pgfpathlineto{\pgfqpoint{0.836545in}{1.600680in}}%
\pgfpathlineto{\pgfqpoint{0.827609in}{1.600680in}}%
\pgfpathlineto{\pgfqpoint{0.827609in}{1.847071in}}%
\pgfpathclose%
\pgfusepath{fill}%
\end{pgfscope}%
\begin{pgfscope}%
\pgfpathrectangle{\pgfqpoint{0.697024in}{0.857143in}}{\pgfqpoint{2.627103in}{1.813434in}}%
\pgfusepath{clip}%
\pgfsetbuttcap%
\pgfsetmiterjoin%
\definecolor{currentfill}{rgb}{0.302379,0.450282,0.300122}%
\pgfsetfillcolor{currentfill}%
\pgfsetlinewidth{0.000000pt}%
\definecolor{currentstroke}{rgb}{0.000000,0.000000,0.000000}%
\pgfsetstrokecolor{currentstroke}%
\pgfsetstrokeopacity{0.000000}%
\pgfsetdash{}{0pt}%
\pgfpathmoveto{\pgfqpoint{0.838779in}{1.847071in}}%
\pgfpathlineto{\pgfqpoint{0.847716in}{1.847071in}}%
\pgfpathlineto{\pgfqpoint{0.847716in}{1.539362in}}%
\pgfpathlineto{\pgfqpoint{0.838779in}{1.539362in}}%
\pgfpathlineto{\pgfqpoint{0.838779in}{1.847071in}}%
\pgfpathclose%
\pgfusepath{fill}%
\end{pgfscope}%
\begin{pgfscope}%
\pgfpathrectangle{\pgfqpoint{0.697024in}{0.857143in}}{\pgfqpoint{2.627103in}{1.813434in}}%
\pgfusepath{clip}%
\pgfsetbuttcap%
\pgfsetmiterjoin%
\definecolor{currentfill}{rgb}{0.302379,0.450282,0.300122}%
\pgfsetfillcolor{currentfill}%
\pgfsetlinewidth{0.000000pt}%
\definecolor{currentstroke}{rgb}{0.000000,0.000000,0.000000}%
\pgfsetstrokecolor{currentstroke}%
\pgfsetstrokeopacity{0.000000}%
\pgfsetdash{}{0pt}%
\pgfpathmoveto{\pgfqpoint{0.849950in}{1.847071in}}%
\pgfpathlineto{\pgfqpoint{0.858886in}{1.847071in}}%
\pgfpathlineto{\pgfqpoint{0.858886in}{1.568557in}}%
\pgfpathlineto{\pgfqpoint{0.849950in}{1.568557in}}%
\pgfpathlineto{\pgfqpoint{0.849950in}{1.847071in}}%
\pgfpathclose%
\pgfusepath{fill}%
\end{pgfscope}%
\begin{pgfscope}%
\pgfpathrectangle{\pgfqpoint{0.697024in}{0.857143in}}{\pgfqpoint{2.627103in}{1.813434in}}%
\pgfusepath{clip}%
\pgfsetbuttcap%
\pgfsetmiterjoin%
\definecolor{currentfill}{rgb}{0.302379,0.450282,0.300122}%
\pgfsetfillcolor{currentfill}%
\pgfsetlinewidth{0.000000pt}%
\definecolor{currentstroke}{rgb}{0.000000,0.000000,0.000000}%
\pgfsetstrokecolor{currentstroke}%
\pgfsetstrokeopacity{0.000000}%
\pgfsetdash{}{0pt}%
\pgfpathmoveto{\pgfqpoint{0.861121in}{1.847071in}}%
\pgfpathlineto{\pgfqpoint{0.870057in}{1.847071in}}%
\pgfpathlineto{\pgfqpoint{0.870057in}{1.470702in}}%
\pgfpathlineto{\pgfqpoint{0.861121in}{1.470702in}}%
\pgfpathlineto{\pgfqpoint{0.861121in}{1.847071in}}%
\pgfpathclose%
\pgfusepath{fill}%
\end{pgfscope}%
\begin{pgfscope}%
\pgfpathrectangle{\pgfqpoint{0.697024in}{0.857143in}}{\pgfqpoint{2.627103in}{1.813434in}}%
\pgfusepath{clip}%
\pgfsetbuttcap%
\pgfsetmiterjoin%
\definecolor{currentfill}{rgb}{0.302379,0.450282,0.300122}%
\pgfsetfillcolor{currentfill}%
\pgfsetlinewidth{0.000000pt}%
\definecolor{currentstroke}{rgb}{0.000000,0.000000,0.000000}%
\pgfsetstrokecolor{currentstroke}%
\pgfsetstrokeopacity{0.000000}%
\pgfsetdash{}{0pt}%
\pgfpathmoveto{\pgfqpoint{0.872291in}{1.838065in}}%
\pgfpathlineto{\pgfqpoint{0.881228in}{1.838065in}}%
\pgfpathlineto{\pgfqpoint{0.881228in}{1.476686in}}%
\pgfpathlineto{\pgfqpoint{0.872291in}{1.476686in}}%
\pgfpathlineto{\pgfqpoint{0.872291in}{1.838065in}}%
\pgfpathclose%
\pgfusepath{fill}%
\end{pgfscope}%
\begin{pgfscope}%
\pgfpathrectangle{\pgfqpoint{0.697024in}{0.857143in}}{\pgfqpoint{2.627103in}{1.813434in}}%
\pgfusepath{clip}%
\pgfsetbuttcap%
\pgfsetmiterjoin%
\definecolor{currentfill}{rgb}{0.302379,0.450282,0.300122}%
\pgfsetfillcolor{currentfill}%
\pgfsetlinewidth{0.000000pt}%
\definecolor{currentstroke}{rgb}{0.000000,0.000000,0.000000}%
\pgfsetstrokecolor{currentstroke}%
\pgfsetstrokeopacity{0.000000}%
\pgfsetdash{}{0pt}%
\pgfpathmoveto{\pgfqpoint{0.883462in}{1.847071in}}%
\pgfpathlineto{\pgfqpoint{0.892398in}{1.847071in}}%
\pgfpathlineto{\pgfqpoint{0.892398in}{1.500594in}}%
\pgfpathlineto{\pgfqpoint{0.883462in}{1.500594in}}%
\pgfpathlineto{\pgfqpoint{0.883462in}{1.847071in}}%
\pgfpathclose%
\pgfusepath{fill}%
\end{pgfscope}%
\begin{pgfscope}%
\pgfpathrectangle{\pgfqpoint{0.697024in}{0.857143in}}{\pgfqpoint{2.627103in}{1.813434in}}%
\pgfusepath{clip}%
\pgfsetbuttcap%
\pgfsetmiterjoin%
\definecolor{currentfill}{rgb}{0.302379,0.450282,0.300122}%
\pgfsetfillcolor{currentfill}%
\pgfsetlinewidth{0.000000pt}%
\definecolor{currentstroke}{rgb}{0.000000,0.000000,0.000000}%
\pgfsetstrokecolor{currentstroke}%
\pgfsetstrokeopacity{0.000000}%
\pgfsetdash{}{0pt}%
\pgfpathmoveto{\pgfqpoint{0.894632in}{1.847071in}}%
\pgfpathlineto{\pgfqpoint{0.903569in}{1.847071in}}%
\pgfpathlineto{\pgfqpoint{0.903569in}{1.537253in}}%
\pgfpathlineto{\pgfqpoint{0.894632in}{1.537253in}}%
\pgfpathlineto{\pgfqpoint{0.894632in}{1.847071in}}%
\pgfpathclose%
\pgfusepath{fill}%
\end{pgfscope}%
\begin{pgfscope}%
\pgfpathrectangle{\pgfqpoint{0.697024in}{0.857143in}}{\pgfqpoint{2.627103in}{1.813434in}}%
\pgfusepath{clip}%
\pgfsetbuttcap%
\pgfsetmiterjoin%
\definecolor{currentfill}{rgb}{0.302379,0.450282,0.300122}%
\pgfsetfillcolor{currentfill}%
\pgfsetlinewidth{0.000000pt}%
\definecolor{currentstroke}{rgb}{0.000000,0.000000,0.000000}%
\pgfsetstrokecolor{currentstroke}%
\pgfsetstrokeopacity{0.000000}%
\pgfsetdash{}{0pt}%
\pgfpathmoveto{\pgfqpoint{0.905803in}{1.847071in}}%
\pgfpathlineto{\pgfqpoint{0.914739in}{1.847071in}}%
\pgfpathlineto{\pgfqpoint{0.914739in}{1.608726in}}%
\pgfpathlineto{\pgfqpoint{0.905803in}{1.608726in}}%
\pgfpathlineto{\pgfqpoint{0.905803in}{1.847071in}}%
\pgfpathclose%
\pgfusepath{fill}%
\end{pgfscope}%
\begin{pgfscope}%
\pgfpathrectangle{\pgfqpoint{0.697024in}{0.857143in}}{\pgfqpoint{2.627103in}{1.813434in}}%
\pgfusepath{clip}%
\pgfsetbuttcap%
\pgfsetmiterjoin%
\definecolor{currentfill}{rgb}{0.302379,0.450282,0.300122}%
\pgfsetfillcolor{currentfill}%
\pgfsetlinewidth{0.000000pt}%
\definecolor{currentstroke}{rgb}{0.000000,0.000000,0.000000}%
\pgfsetstrokecolor{currentstroke}%
\pgfsetstrokeopacity{0.000000}%
\pgfsetdash{}{0pt}%
\pgfpathmoveto{\pgfqpoint{0.916974in}{1.847071in}}%
\pgfpathlineto{\pgfqpoint{0.925910in}{1.847071in}}%
\pgfpathlineto{\pgfqpoint{0.925910in}{1.593487in}}%
\pgfpathlineto{\pgfqpoint{0.916974in}{1.593487in}}%
\pgfpathlineto{\pgfqpoint{0.916974in}{1.847071in}}%
\pgfpathclose%
\pgfusepath{fill}%
\end{pgfscope}%
\begin{pgfscope}%
\pgfpathrectangle{\pgfqpoint{0.697024in}{0.857143in}}{\pgfqpoint{2.627103in}{1.813434in}}%
\pgfusepath{clip}%
\pgfsetbuttcap%
\pgfsetmiterjoin%
\definecolor{currentfill}{rgb}{0.302379,0.450282,0.300122}%
\pgfsetfillcolor{currentfill}%
\pgfsetlinewidth{0.000000pt}%
\definecolor{currentstroke}{rgb}{0.000000,0.000000,0.000000}%
\pgfsetstrokecolor{currentstroke}%
\pgfsetstrokeopacity{0.000000}%
\pgfsetdash{}{0pt}%
\pgfpathmoveto{\pgfqpoint{0.928144in}{1.847071in}}%
\pgfpathlineto{\pgfqpoint{0.937081in}{1.847071in}}%
\pgfpathlineto{\pgfqpoint{0.937081in}{1.609445in}}%
\pgfpathlineto{\pgfqpoint{0.928144in}{1.609445in}}%
\pgfpathlineto{\pgfqpoint{0.928144in}{1.847071in}}%
\pgfpathclose%
\pgfusepath{fill}%
\end{pgfscope}%
\begin{pgfscope}%
\pgfpathrectangle{\pgfqpoint{0.697024in}{0.857143in}}{\pgfqpoint{2.627103in}{1.813434in}}%
\pgfusepath{clip}%
\pgfsetbuttcap%
\pgfsetmiterjoin%
\definecolor{currentfill}{rgb}{0.302379,0.450282,0.300122}%
\pgfsetfillcolor{currentfill}%
\pgfsetlinewidth{0.000000pt}%
\definecolor{currentstroke}{rgb}{0.000000,0.000000,0.000000}%
\pgfsetstrokecolor{currentstroke}%
\pgfsetstrokeopacity{0.000000}%
\pgfsetdash{}{0pt}%
\pgfpathmoveto{\pgfqpoint{0.939315in}{1.847071in}}%
\pgfpathlineto{\pgfqpoint{0.948251in}{1.847071in}}%
\pgfpathlineto{\pgfqpoint{0.948251in}{1.583232in}}%
\pgfpathlineto{\pgfqpoint{0.939315in}{1.583232in}}%
\pgfpathlineto{\pgfqpoint{0.939315in}{1.847071in}}%
\pgfpathclose%
\pgfusepath{fill}%
\end{pgfscope}%
\begin{pgfscope}%
\pgfpathrectangle{\pgfqpoint{0.697024in}{0.857143in}}{\pgfqpoint{2.627103in}{1.813434in}}%
\pgfusepath{clip}%
\pgfsetbuttcap%
\pgfsetmiterjoin%
\definecolor{currentfill}{rgb}{0.302379,0.450282,0.300122}%
\pgfsetfillcolor{currentfill}%
\pgfsetlinewidth{0.000000pt}%
\definecolor{currentstroke}{rgb}{0.000000,0.000000,0.000000}%
\pgfsetstrokecolor{currentstroke}%
\pgfsetstrokeopacity{0.000000}%
\pgfsetdash{}{0pt}%
\pgfpathmoveto{\pgfqpoint{0.950485in}{1.847071in}}%
\pgfpathlineto{\pgfqpoint{0.959422in}{1.847071in}}%
\pgfpathlineto{\pgfqpoint{0.959422in}{1.624813in}}%
\pgfpathlineto{\pgfqpoint{0.950485in}{1.624813in}}%
\pgfpathlineto{\pgfqpoint{0.950485in}{1.847071in}}%
\pgfpathclose%
\pgfusepath{fill}%
\end{pgfscope}%
\begin{pgfscope}%
\pgfpathrectangle{\pgfqpoint{0.697024in}{0.857143in}}{\pgfqpoint{2.627103in}{1.813434in}}%
\pgfusepath{clip}%
\pgfsetbuttcap%
\pgfsetmiterjoin%
\definecolor{currentfill}{rgb}{0.302379,0.450282,0.300122}%
\pgfsetfillcolor{currentfill}%
\pgfsetlinewidth{0.000000pt}%
\definecolor{currentstroke}{rgb}{0.000000,0.000000,0.000000}%
\pgfsetstrokecolor{currentstroke}%
\pgfsetstrokeopacity{0.000000}%
\pgfsetdash{}{0pt}%
\pgfpathmoveto{\pgfqpoint{0.961656in}{1.847071in}}%
\pgfpathlineto{\pgfqpoint{0.970593in}{1.847071in}}%
\pgfpathlineto{\pgfqpoint{0.970593in}{1.577425in}}%
\pgfpathlineto{\pgfqpoint{0.961656in}{1.577425in}}%
\pgfpathlineto{\pgfqpoint{0.961656in}{1.847071in}}%
\pgfpathclose%
\pgfusepath{fill}%
\end{pgfscope}%
\begin{pgfscope}%
\pgfpathrectangle{\pgfqpoint{0.697024in}{0.857143in}}{\pgfqpoint{2.627103in}{1.813434in}}%
\pgfusepath{clip}%
\pgfsetbuttcap%
\pgfsetmiterjoin%
\definecolor{currentfill}{rgb}{0.302379,0.450282,0.300122}%
\pgfsetfillcolor{currentfill}%
\pgfsetlinewidth{0.000000pt}%
\definecolor{currentstroke}{rgb}{0.000000,0.000000,0.000000}%
\pgfsetstrokecolor{currentstroke}%
\pgfsetstrokeopacity{0.000000}%
\pgfsetdash{}{0pt}%
\pgfpathmoveto{\pgfqpoint{0.972827in}{1.847071in}}%
\pgfpathlineto{\pgfqpoint{0.981763in}{1.847071in}}%
\pgfpathlineto{\pgfqpoint{0.981763in}{1.659143in}}%
\pgfpathlineto{\pgfqpoint{0.972827in}{1.659143in}}%
\pgfpathlineto{\pgfqpoint{0.972827in}{1.847071in}}%
\pgfpathclose%
\pgfusepath{fill}%
\end{pgfscope}%
\begin{pgfscope}%
\pgfpathrectangle{\pgfqpoint{0.697024in}{0.857143in}}{\pgfqpoint{2.627103in}{1.813434in}}%
\pgfusepath{clip}%
\pgfsetbuttcap%
\pgfsetmiterjoin%
\definecolor{currentfill}{rgb}{0.302379,0.450282,0.300122}%
\pgfsetfillcolor{currentfill}%
\pgfsetlinewidth{0.000000pt}%
\definecolor{currentstroke}{rgb}{0.000000,0.000000,0.000000}%
\pgfsetstrokecolor{currentstroke}%
\pgfsetstrokeopacity{0.000000}%
\pgfsetdash{}{0pt}%
\pgfpathmoveto{\pgfqpoint{0.983997in}{1.847071in}}%
\pgfpathlineto{\pgfqpoint{0.992934in}{1.847071in}}%
\pgfpathlineto{\pgfqpoint{0.992934in}{1.562055in}}%
\pgfpathlineto{\pgfqpoint{0.983997in}{1.562055in}}%
\pgfpathlineto{\pgfqpoint{0.983997in}{1.847071in}}%
\pgfpathclose%
\pgfusepath{fill}%
\end{pgfscope}%
\begin{pgfscope}%
\pgfpathrectangle{\pgfqpoint{0.697024in}{0.857143in}}{\pgfqpoint{2.627103in}{1.813434in}}%
\pgfusepath{clip}%
\pgfsetbuttcap%
\pgfsetmiterjoin%
\definecolor{currentfill}{rgb}{0.302379,0.450282,0.300122}%
\pgfsetfillcolor{currentfill}%
\pgfsetlinewidth{0.000000pt}%
\definecolor{currentstroke}{rgb}{0.000000,0.000000,0.000000}%
\pgfsetstrokecolor{currentstroke}%
\pgfsetstrokeopacity{0.000000}%
\pgfsetdash{}{0pt}%
\pgfpathmoveto{\pgfqpoint{0.995168in}{1.847071in}}%
\pgfpathlineto{\pgfqpoint{1.004104in}{1.847071in}}%
\pgfpathlineto{\pgfqpoint{1.004104in}{1.472361in}}%
\pgfpathlineto{\pgfqpoint{0.995168in}{1.472361in}}%
\pgfpathlineto{\pgfqpoint{0.995168in}{1.847071in}}%
\pgfpathclose%
\pgfusepath{fill}%
\end{pgfscope}%
\begin{pgfscope}%
\pgfpathrectangle{\pgfqpoint{0.697024in}{0.857143in}}{\pgfqpoint{2.627103in}{1.813434in}}%
\pgfusepath{clip}%
\pgfsetbuttcap%
\pgfsetmiterjoin%
\definecolor{currentfill}{rgb}{0.302379,0.450282,0.300122}%
\pgfsetfillcolor{currentfill}%
\pgfsetlinewidth{0.000000pt}%
\definecolor{currentstroke}{rgb}{0.000000,0.000000,0.000000}%
\pgfsetstrokecolor{currentstroke}%
\pgfsetstrokeopacity{0.000000}%
\pgfsetdash{}{0pt}%
\pgfpathmoveto{\pgfqpoint{1.006338in}{1.847071in}}%
\pgfpathlineto{\pgfqpoint{1.015275in}{1.847071in}}%
\pgfpathlineto{\pgfqpoint{1.015275in}{1.482814in}}%
\pgfpathlineto{\pgfqpoint{1.006338in}{1.482814in}}%
\pgfpathlineto{\pgfqpoint{1.006338in}{1.847071in}}%
\pgfpathclose%
\pgfusepath{fill}%
\end{pgfscope}%
\begin{pgfscope}%
\pgfpathrectangle{\pgfqpoint{0.697024in}{0.857143in}}{\pgfqpoint{2.627103in}{1.813434in}}%
\pgfusepath{clip}%
\pgfsetbuttcap%
\pgfsetmiterjoin%
\definecolor{currentfill}{rgb}{0.302379,0.450282,0.300122}%
\pgfsetfillcolor{currentfill}%
\pgfsetlinewidth{0.000000pt}%
\definecolor{currentstroke}{rgb}{0.000000,0.000000,0.000000}%
\pgfsetstrokecolor{currentstroke}%
\pgfsetstrokeopacity{0.000000}%
\pgfsetdash{}{0pt}%
\pgfpathmoveto{\pgfqpoint{1.017509in}{1.847071in}}%
\pgfpathlineto{\pgfqpoint{1.026446in}{1.847071in}}%
\pgfpathlineto{\pgfqpoint{1.026446in}{1.382679in}}%
\pgfpathlineto{\pgfqpoint{1.017509in}{1.382679in}}%
\pgfpathlineto{\pgfqpoint{1.017509in}{1.847071in}}%
\pgfpathclose%
\pgfusepath{fill}%
\end{pgfscope}%
\begin{pgfscope}%
\pgfpathrectangle{\pgfqpoint{0.697024in}{0.857143in}}{\pgfqpoint{2.627103in}{1.813434in}}%
\pgfusepath{clip}%
\pgfsetbuttcap%
\pgfsetmiterjoin%
\definecolor{currentfill}{rgb}{0.302379,0.450282,0.300122}%
\pgfsetfillcolor{currentfill}%
\pgfsetlinewidth{0.000000pt}%
\definecolor{currentstroke}{rgb}{0.000000,0.000000,0.000000}%
\pgfsetstrokecolor{currentstroke}%
\pgfsetstrokeopacity{0.000000}%
\pgfsetdash{}{0pt}%
\pgfpathmoveto{\pgfqpoint{1.028680in}{1.847071in}}%
\pgfpathlineto{\pgfqpoint{1.037616in}{1.847071in}}%
\pgfpathlineto{\pgfqpoint{1.037616in}{1.405412in}}%
\pgfpathlineto{\pgfqpoint{1.028680in}{1.405412in}}%
\pgfpathlineto{\pgfqpoint{1.028680in}{1.847071in}}%
\pgfpathclose%
\pgfusepath{fill}%
\end{pgfscope}%
\begin{pgfscope}%
\pgfpathrectangle{\pgfqpoint{0.697024in}{0.857143in}}{\pgfqpoint{2.627103in}{1.813434in}}%
\pgfusepath{clip}%
\pgfsetbuttcap%
\pgfsetmiterjoin%
\definecolor{currentfill}{rgb}{0.302379,0.450282,0.300122}%
\pgfsetfillcolor{currentfill}%
\pgfsetlinewidth{0.000000pt}%
\definecolor{currentstroke}{rgb}{0.000000,0.000000,0.000000}%
\pgfsetstrokecolor{currentstroke}%
\pgfsetstrokeopacity{0.000000}%
\pgfsetdash{}{0pt}%
\pgfpathmoveto{\pgfqpoint{1.039850in}{1.847071in}}%
\pgfpathlineto{\pgfqpoint{1.048787in}{1.847071in}}%
\pgfpathlineto{\pgfqpoint{1.048787in}{1.355810in}}%
\pgfpathlineto{\pgfqpoint{1.039850in}{1.355810in}}%
\pgfpathlineto{\pgfqpoint{1.039850in}{1.847071in}}%
\pgfpathclose%
\pgfusepath{fill}%
\end{pgfscope}%
\begin{pgfscope}%
\pgfpathrectangle{\pgfqpoint{0.697024in}{0.857143in}}{\pgfqpoint{2.627103in}{1.813434in}}%
\pgfusepath{clip}%
\pgfsetbuttcap%
\pgfsetmiterjoin%
\definecolor{currentfill}{rgb}{0.302379,0.450282,0.300122}%
\pgfsetfillcolor{currentfill}%
\pgfsetlinewidth{0.000000pt}%
\definecolor{currentstroke}{rgb}{0.000000,0.000000,0.000000}%
\pgfsetstrokecolor{currentstroke}%
\pgfsetstrokeopacity{0.000000}%
\pgfsetdash{}{0pt}%
\pgfpathmoveto{\pgfqpoint{1.051021in}{1.847071in}}%
\pgfpathlineto{\pgfqpoint{1.059957in}{1.847071in}}%
\pgfpathlineto{\pgfqpoint{1.059957in}{1.362162in}}%
\pgfpathlineto{\pgfqpoint{1.051021in}{1.362162in}}%
\pgfpathlineto{\pgfqpoint{1.051021in}{1.847071in}}%
\pgfpathclose%
\pgfusepath{fill}%
\end{pgfscope}%
\begin{pgfscope}%
\pgfpathrectangle{\pgfqpoint{0.697024in}{0.857143in}}{\pgfqpoint{2.627103in}{1.813434in}}%
\pgfusepath{clip}%
\pgfsetbuttcap%
\pgfsetmiterjoin%
\definecolor{currentfill}{rgb}{0.302379,0.450282,0.300122}%
\pgfsetfillcolor{currentfill}%
\pgfsetlinewidth{0.000000pt}%
\definecolor{currentstroke}{rgb}{0.000000,0.000000,0.000000}%
\pgfsetstrokecolor{currentstroke}%
\pgfsetstrokeopacity{0.000000}%
\pgfsetdash{}{0pt}%
\pgfpathmoveto{\pgfqpoint{1.062191in}{1.847071in}}%
\pgfpathlineto{\pgfqpoint{1.071128in}{1.847071in}}%
\pgfpathlineto{\pgfqpoint{1.071128in}{1.194426in}}%
\pgfpathlineto{\pgfqpoint{1.062191in}{1.194426in}}%
\pgfpathlineto{\pgfqpoint{1.062191in}{1.847071in}}%
\pgfpathclose%
\pgfusepath{fill}%
\end{pgfscope}%
\begin{pgfscope}%
\pgfpathrectangle{\pgfqpoint{0.697024in}{0.857143in}}{\pgfqpoint{2.627103in}{1.813434in}}%
\pgfusepath{clip}%
\pgfsetbuttcap%
\pgfsetmiterjoin%
\definecolor{currentfill}{rgb}{0.302379,0.450282,0.300122}%
\pgfsetfillcolor{currentfill}%
\pgfsetlinewidth{0.000000pt}%
\definecolor{currentstroke}{rgb}{0.000000,0.000000,0.000000}%
\pgfsetstrokecolor{currentstroke}%
\pgfsetstrokeopacity{0.000000}%
\pgfsetdash{}{0pt}%
\pgfpathmoveto{\pgfqpoint{1.073362in}{1.847071in}}%
\pgfpathlineto{\pgfqpoint{1.082299in}{1.847071in}}%
\pgfpathlineto{\pgfqpoint{1.082299in}{1.268238in}}%
\pgfpathlineto{\pgfqpoint{1.073362in}{1.268238in}}%
\pgfpathlineto{\pgfqpoint{1.073362in}{1.847071in}}%
\pgfpathclose%
\pgfusepath{fill}%
\end{pgfscope}%
\begin{pgfscope}%
\pgfpathrectangle{\pgfqpoint{0.697024in}{0.857143in}}{\pgfqpoint{2.627103in}{1.813434in}}%
\pgfusepath{clip}%
\pgfsetbuttcap%
\pgfsetmiterjoin%
\definecolor{currentfill}{rgb}{0.302379,0.450282,0.300122}%
\pgfsetfillcolor{currentfill}%
\pgfsetlinewidth{0.000000pt}%
\definecolor{currentstroke}{rgb}{0.000000,0.000000,0.000000}%
\pgfsetstrokecolor{currentstroke}%
\pgfsetstrokeopacity{0.000000}%
\pgfsetdash{}{0pt}%
\pgfpathmoveto{\pgfqpoint{1.084533in}{1.847071in}}%
\pgfpathlineto{\pgfqpoint{1.093469in}{1.847071in}}%
\pgfpathlineto{\pgfqpoint{1.093469in}{1.353715in}}%
\pgfpathlineto{\pgfqpoint{1.084533in}{1.353715in}}%
\pgfpathlineto{\pgfqpoint{1.084533in}{1.847071in}}%
\pgfpathclose%
\pgfusepath{fill}%
\end{pgfscope}%
\begin{pgfscope}%
\pgfpathrectangle{\pgfqpoint{0.697024in}{0.857143in}}{\pgfqpoint{2.627103in}{1.813434in}}%
\pgfusepath{clip}%
\pgfsetbuttcap%
\pgfsetmiterjoin%
\definecolor{currentfill}{rgb}{0.302379,0.450282,0.300122}%
\pgfsetfillcolor{currentfill}%
\pgfsetlinewidth{0.000000pt}%
\definecolor{currentstroke}{rgb}{0.000000,0.000000,0.000000}%
\pgfsetstrokecolor{currentstroke}%
\pgfsetstrokeopacity{0.000000}%
\pgfsetdash{}{0pt}%
\pgfpathmoveto{\pgfqpoint{1.095703in}{1.847071in}}%
\pgfpathlineto{\pgfqpoint{1.104640in}{1.847071in}}%
\pgfpathlineto{\pgfqpoint{1.104640in}{1.358196in}}%
\pgfpathlineto{\pgfqpoint{1.095703in}{1.358196in}}%
\pgfpathlineto{\pgfqpoint{1.095703in}{1.847071in}}%
\pgfpathclose%
\pgfusepath{fill}%
\end{pgfscope}%
\begin{pgfscope}%
\pgfpathrectangle{\pgfqpoint{0.697024in}{0.857143in}}{\pgfqpoint{2.627103in}{1.813434in}}%
\pgfusepath{clip}%
\pgfsetbuttcap%
\pgfsetmiterjoin%
\definecolor{currentfill}{rgb}{0.302379,0.450282,0.300122}%
\pgfsetfillcolor{currentfill}%
\pgfsetlinewidth{0.000000pt}%
\definecolor{currentstroke}{rgb}{0.000000,0.000000,0.000000}%
\pgfsetstrokecolor{currentstroke}%
\pgfsetstrokeopacity{0.000000}%
\pgfsetdash{}{0pt}%
\pgfpathmoveto{\pgfqpoint{1.106874in}{1.847071in}}%
\pgfpathlineto{\pgfqpoint{1.115810in}{1.847071in}}%
\pgfpathlineto{\pgfqpoint{1.115810in}{1.416930in}}%
\pgfpathlineto{\pgfqpoint{1.106874in}{1.416930in}}%
\pgfpathlineto{\pgfqpoint{1.106874in}{1.847071in}}%
\pgfpathclose%
\pgfusepath{fill}%
\end{pgfscope}%
\begin{pgfscope}%
\pgfpathrectangle{\pgfqpoint{0.697024in}{0.857143in}}{\pgfqpoint{2.627103in}{1.813434in}}%
\pgfusepath{clip}%
\pgfsetbuttcap%
\pgfsetmiterjoin%
\definecolor{currentfill}{rgb}{0.302379,0.450282,0.300122}%
\pgfsetfillcolor{currentfill}%
\pgfsetlinewidth{0.000000pt}%
\definecolor{currentstroke}{rgb}{0.000000,0.000000,0.000000}%
\pgfsetstrokecolor{currentstroke}%
\pgfsetstrokeopacity{0.000000}%
\pgfsetdash{}{0pt}%
\pgfpathmoveto{\pgfqpoint{1.118045in}{1.847071in}}%
\pgfpathlineto{\pgfqpoint{1.126981in}{1.847071in}}%
\pgfpathlineto{\pgfqpoint{1.126981in}{1.496484in}}%
\pgfpathlineto{\pgfqpoint{1.118045in}{1.496484in}}%
\pgfpathlineto{\pgfqpoint{1.118045in}{1.847071in}}%
\pgfpathclose%
\pgfusepath{fill}%
\end{pgfscope}%
\begin{pgfscope}%
\pgfpathrectangle{\pgfqpoint{0.697024in}{0.857143in}}{\pgfqpoint{2.627103in}{1.813434in}}%
\pgfusepath{clip}%
\pgfsetbuttcap%
\pgfsetmiterjoin%
\definecolor{currentfill}{rgb}{0.302379,0.450282,0.300122}%
\pgfsetfillcolor{currentfill}%
\pgfsetlinewidth{0.000000pt}%
\definecolor{currentstroke}{rgb}{0.000000,0.000000,0.000000}%
\pgfsetstrokecolor{currentstroke}%
\pgfsetstrokeopacity{0.000000}%
\pgfsetdash{}{0pt}%
\pgfpathmoveto{\pgfqpoint{1.129215in}{1.847071in}}%
\pgfpathlineto{\pgfqpoint{1.138152in}{1.847071in}}%
\pgfpathlineto{\pgfqpoint{1.138152in}{1.582381in}}%
\pgfpathlineto{\pgfqpoint{1.129215in}{1.582381in}}%
\pgfpathlineto{\pgfqpoint{1.129215in}{1.847071in}}%
\pgfpathclose%
\pgfusepath{fill}%
\end{pgfscope}%
\begin{pgfscope}%
\pgfpathrectangle{\pgfqpoint{0.697024in}{0.857143in}}{\pgfqpoint{2.627103in}{1.813434in}}%
\pgfusepath{clip}%
\pgfsetbuttcap%
\pgfsetmiterjoin%
\definecolor{currentfill}{rgb}{0.302379,0.450282,0.300122}%
\pgfsetfillcolor{currentfill}%
\pgfsetlinewidth{0.000000pt}%
\definecolor{currentstroke}{rgb}{0.000000,0.000000,0.000000}%
\pgfsetstrokecolor{currentstroke}%
\pgfsetstrokeopacity{0.000000}%
\pgfsetdash{}{0pt}%
\pgfpathmoveto{\pgfqpoint{1.140386in}{1.847071in}}%
\pgfpathlineto{\pgfqpoint{1.149322in}{1.847071in}}%
\pgfpathlineto{\pgfqpoint{1.149322in}{1.585019in}}%
\pgfpathlineto{\pgfqpoint{1.140386in}{1.585019in}}%
\pgfpathlineto{\pgfqpoint{1.140386in}{1.847071in}}%
\pgfpathclose%
\pgfusepath{fill}%
\end{pgfscope}%
\begin{pgfscope}%
\pgfpathrectangle{\pgfqpoint{0.697024in}{0.857143in}}{\pgfqpoint{2.627103in}{1.813434in}}%
\pgfusepath{clip}%
\pgfsetbuttcap%
\pgfsetmiterjoin%
\definecolor{currentfill}{rgb}{0.302379,0.450282,0.300122}%
\pgfsetfillcolor{currentfill}%
\pgfsetlinewidth{0.000000pt}%
\definecolor{currentstroke}{rgb}{0.000000,0.000000,0.000000}%
\pgfsetstrokecolor{currentstroke}%
\pgfsetstrokeopacity{0.000000}%
\pgfsetdash{}{0pt}%
\pgfpathmoveto{\pgfqpoint{1.151556in}{1.847071in}}%
\pgfpathlineto{\pgfqpoint{1.160493in}{1.847071in}}%
\pgfpathlineto{\pgfqpoint{1.160493in}{1.572621in}}%
\pgfpathlineto{\pgfqpoint{1.151556in}{1.572621in}}%
\pgfpathlineto{\pgfqpoint{1.151556in}{1.847071in}}%
\pgfpathclose%
\pgfusepath{fill}%
\end{pgfscope}%
\begin{pgfscope}%
\pgfpathrectangle{\pgfqpoint{0.697024in}{0.857143in}}{\pgfqpoint{2.627103in}{1.813434in}}%
\pgfusepath{clip}%
\pgfsetbuttcap%
\pgfsetmiterjoin%
\definecolor{currentfill}{rgb}{0.302379,0.450282,0.300122}%
\pgfsetfillcolor{currentfill}%
\pgfsetlinewidth{0.000000pt}%
\definecolor{currentstroke}{rgb}{0.000000,0.000000,0.000000}%
\pgfsetstrokecolor{currentstroke}%
\pgfsetstrokeopacity{0.000000}%
\pgfsetdash{}{0pt}%
\pgfpathmoveto{\pgfqpoint{1.162727in}{1.847071in}}%
\pgfpathlineto{\pgfqpoint{1.171663in}{1.847071in}}%
\pgfpathlineto{\pgfqpoint{1.171663in}{1.674919in}}%
\pgfpathlineto{\pgfqpoint{1.162727in}{1.674919in}}%
\pgfpathlineto{\pgfqpoint{1.162727in}{1.847071in}}%
\pgfpathclose%
\pgfusepath{fill}%
\end{pgfscope}%
\begin{pgfscope}%
\pgfpathrectangle{\pgfqpoint{0.697024in}{0.857143in}}{\pgfqpoint{2.627103in}{1.813434in}}%
\pgfusepath{clip}%
\pgfsetbuttcap%
\pgfsetmiterjoin%
\definecolor{currentfill}{rgb}{0.302379,0.450282,0.300122}%
\pgfsetfillcolor{currentfill}%
\pgfsetlinewidth{0.000000pt}%
\definecolor{currentstroke}{rgb}{0.000000,0.000000,0.000000}%
\pgfsetstrokecolor{currentstroke}%
\pgfsetstrokeopacity{0.000000}%
\pgfsetdash{}{0pt}%
\pgfpathmoveto{\pgfqpoint{1.173898in}{1.846531in}}%
\pgfpathlineto{\pgfqpoint{1.182834in}{1.846531in}}%
\pgfpathlineto{\pgfqpoint{1.182834in}{1.633167in}}%
\pgfpathlineto{\pgfqpoint{1.173898in}{1.633167in}}%
\pgfpathlineto{\pgfqpoint{1.173898in}{1.846531in}}%
\pgfpathclose%
\pgfusepath{fill}%
\end{pgfscope}%
\begin{pgfscope}%
\pgfpathrectangle{\pgfqpoint{0.697024in}{0.857143in}}{\pgfqpoint{2.627103in}{1.813434in}}%
\pgfusepath{clip}%
\pgfsetbuttcap%
\pgfsetmiterjoin%
\definecolor{currentfill}{rgb}{0.302379,0.450282,0.300122}%
\pgfsetfillcolor{currentfill}%
\pgfsetlinewidth{0.000000pt}%
\definecolor{currentstroke}{rgb}{0.000000,0.000000,0.000000}%
\pgfsetstrokecolor{currentstroke}%
\pgfsetstrokeopacity{0.000000}%
\pgfsetdash{}{0pt}%
\pgfpathmoveto{\pgfqpoint{1.185068in}{1.805687in}}%
\pgfpathlineto{\pgfqpoint{1.194005in}{1.805687in}}%
\pgfpathlineto{\pgfqpoint{1.194005in}{1.540863in}}%
\pgfpathlineto{\pgfqpoint{1.185068in}{1.540863in}}%
\pgfpathlineto{\pgfqpoint{1.185068in}{1.805687in}}%
\pgfpathclose%
\pgfusepath{fill}%
\end{pgfscope}%
\begin{pgfscope}%
\pgfpathrectangle{\pgfqpoint{0.697024in}{0.857143in}}{\pgfqpoint{2.627103in}{1.813434in}}%
\pgfusepath{clip}%
\pgfsetbuttcap%
\pgfsetmiterjoin%
\definecolor{currentfill}{rgb}{0.302379,0.450282,0.300122}%
\pgfsetfillcolor{currentfill}%
\pgfsetlinewidth{0.000000pt}%
\definecolor{currentstroke}{rgb}{0.000000,0.000000,0.000000}%
\pgfsetstrokecolor{currentstroke}%
\pgfsetstrokeopacity{0.000000}%
\pgfsetdash{}{0pt}%
\pgfpathmoveto{\pgfqpoint{1.196239in}{1.682902in}}%
\pgfpathlineto{\pgfqpoint{1.205175in}{1.682902in}}%
\pgfpathlineto{\pgfqpoint{1.205175in}{1.394849in}}%
\pgfpathlineto{\pgfqpoint{1.196239in}{1.394849in}}%
\pgfpathlineto{\pgfqpoint{1.196239in}{1.682902in}}%
\pgfpathclose%
\pgfusepath{fill}%
\end{pgfscope}%
\begin{pgfscope}%
\pgfpathrectangle{\pgfqpoint{0.697024in}{0.857143in}}{\pgfqpoint{2.627103in}{1.813434in}}%
\pgfusepath{clip}%
\pgfsetbuttcap%
\pgfsetmiterjoin%
\definecolor{currentfill}{rgb}{0.302379,0.450282,0.300122}%
\pgfsetfillcolor{currentfill}%
\pgfsetlinewidth{0.000000pt}%
\definecolor{currentstroke}{rgb}{0.000000,0.000000,0.000000}%
\pgfsetstrokecolor{currentstroke}%
\pgfsetstrokeopacity{0.000000}%
\pgfsetdash{}{0pt}%
\pgfpathmoveto{\pgfqpoint{1.207409in}{1.778759in}}%
\pgfpathlineto{\pgfqpoint{1.216346in}{1.778759in}}%
\pgfpathlineto{\pgfqpoint{1.216346in}{1.380634in}}%
\pgfpathlineto{\pgfqpoint{1.207409in}{1.380634in}}%
\pgfpathlineto{\pgfqpoint{1.207409in}{1.778759in}}%
\pgfpathclose%
\pgfusepath{fill}%
\end{pgfscope}%
\begin{pgfscope}%
\pgfpathrectangle{\pgfqpoint{0.697024in}{0.857143in}}{\pgfqpoint{2.627103in}{1.813434in}}%
\pgfusepath{clip}%
\pgfsetbuttcap%
\pgfsetmiterjoin%
\definecolor{currentfill}{rgb}{0.302379,0.450282,0.300122}%
\pgfsetfillcolor{currentfill}%
\pgfsetlinewidth{0.000000pt}%
\definecolor{currentstroke}{rgb}{0.000000,0.000000,0.000000}%
\pgfsetstrokecolor{currentstroke}%
\pgfsetstrokeopacity{0.000000}%
\pgfsetdash{}{0pt}%
\pgfpathmoveto{\pgfqpoint{1.218580in}{1.694675in}}%
\pgfpathlineto{\pgfqpoint{1.227516in}{1.694675in}}%
\pgfpathlineto{\pgfqpoint{1.227516in}{1.408064in}}%
\pgfpathlineto{\pgfqpoint{1.218580in}{1.408064in}}%
\pgfpathlineto{\pgfqpoint{1.218580in}{1.694675in}}%
\pgfpathclose%
\pgfusepath{fill}%
\end{pgfscope}%
\begin{pgfscope}%
\pgfpathrectangle{\pgfqpoint{0.697024in}{0.857143in}}{\pgfqpoint{2.627103in}{1.813434in}}%
\pgfusepath{clip}%
\pgfsetbuttcap%
\pgfsetmiterjoin%
\definecolor{currentfill}{rgb}{0.302379,0.450282,0.300122}%
\pgfsetfillcolor{currentfill}%
\pgfsetlinewidth{0.000000pt}%
\definecolor{currentstroke}{rgb}{0.000000,0.000000,0.000000}%
\pgfsetstrokecolor{currentstroke}%
\pgfsetstrokeopacity{0.000000}%
\pgfsetdash{}{0pt}%
\pgfpathmoveto{\pgfqpoint{1.229751in}{1.739811in}}%
\pgfpathlineto{\pgfqpoint{1.238687in}{1.739811in}}%
\pgfpathlineto{\pgfqpoint{1.238687in}{1.431378in}}%
\pgfpathlineto{\pgfqpoint{1.229751in}{1.431378in}}%
\pgfpathlineto{\pgfqpoint{1.229751in}{1.739811in}}%
\pgfpathclose%
\pgfusepath{fill}%
\end{pgfscope}%
\begin{pgfscope}%
\pgfpathrectangle{\pgfqpoint{0.697024in}{0.857143in}}{\pgfqpoint{2.627103in}{1.813434in}}%
\pgfusepath{clip}%
\pgfsetbuttcap%
\pgfsetmiterjoin%
\definecolor{currentfill}{rgb}{0.302379,0.450282,0.300122}%
\pgfsetfillcolor{currentfill}%
\pgfsetlinewidth{0.000000pt}%
\definecolor{currentstroke}{rgb}{0.000000,0.000000,0.000000}%
\pgfsetstrokecolor{currentstroke}%
\pgfsetstrokeopacity{0.000000}%
\pgfsetdash{}{0pt}%
\pgfpathmoveto{\pgfqpoint{1.240921in}{1.847071in}}%
\pgfpathlineto{\pgfqpoint{1.249858in}{1.847071in}}%
\pgfpathlineto{\pgfqpoint{1.249858in}{1.573725in}}%
\pgfpathlineto{\pgfqpoint{1.240921in}{1.573725in}}%
\pgfpathlineto{\pgfqpoint{1.240921in}{1.847071in}}%
\pgfpathclose%
\pgfusepath{fill}%
\end{pgfscope}%
\begin{pgfscope}%
\pgfpathrectangle{\pgfqpoint{0.697024in}{0.857143in}}{\pgfqpoint{2.627103in}{1.813434in}}%
\pgfusepath{clip}%
\pgfsetbuttcap%
\pgfsetmiterjoin%
\definecolor{currentfill}{rgb}{0.302379,0.450282,0.300122}%
\pgfsetfillcolor{currentfill}%
\pgfsetlinewidth{0.000000pt}%
\definecolor{currentstroke}{rgb}{0.000000,0.000000,0.000000}%
\pgfsetstrokecolor{currentstroke}%
\pgfsetstrokeopacity{0.000000}%
\pgfsetdash{}{0pt}%
\pgfpathmoveto{\pgfqpoint{1.252092in}{1.847071in}}%
\pgfpathlineto{\pgfqpoint{1.261028in}{1.847071in}}%
\pgfpathlineto{\pgfqpoint{1.261028in}{1.539634in}}%
\pgfpathlineto{\pgfqpoint{1.252092in}{1.539634in}}%
\pgfpathlineto{\pgfqpoint{1.252092in}{1.847071in}}%
\pgfpathclose%
\pgfusepath{fill}%
\end{pgfscope}%
\begin{pgfscope}%
\pgfpathrectangle{\pgfqpoint{0.697024in}{0.857143in}}{\pgfqpoint{2.627103in}{1.813434in}}%
\pgfusepath{clip}%
\pgfsetbuttcap%
\pgfsetmiterjoin%
\definecolor{currentfill}{rgb}{0.302379,0.450282,0.300122}%
\pgfsetfillcolor{currentfill}%
\pgfsetlinewidth{0.000000pt}%
\definecolor{currentstroke}{rgb}{0.000000,0.000000,0.000000}%
\pgfsetstrokecolor{currentstroke}%
\pgfsetstrokeopacity{0.000000}%
\pgfsetdash{}{0pt}%
\pgfpathmoveto{\pgfqpoint{1.263262in}{1.847071in}}%
\pgfpathlineto{\pgfqpoint{1.272199in}{1.847071in}}%
\pgfpathlineto{\pgfqpoint{1.272199in}{1.566953in}}%
\pgfpathlineto{\pgfqpoint{1.263262in}{1.566953in}}%
\pgfpathlineto{\pgfqpoint{1.263262in}{1.847071in}}%
\pgfpathclose%
\pgfusepath{fill}%
\end{pgfscope}%
\begin{pgfscope}%
\pgfpathrectangle{\pgfqpoint{0.697024in}{0.857143in}}{\pgfqpoint{2.627103in}{1.813434in}}%
\pgfusepath{clip}%
\pgfsetbuttcap%
\pgfsetmiterjoin%
\definecolor{currentfill}{rgb}{0.302379,0.450282,0.300122}%
\pgfsetfillcolor{currentfill}%
\pgfsetlinewidth{0.000000pt}%
\definecolor{currentstroke}{rgb}{0.000000,0.000000,0.000000}%
\pgfsetstrokecolor{currentstroke}%
\pgfsetstrokeopacity{0.000000}%
\pgfsetdash{}{0pt}%
\pgfpathmoveto{\pgfqpoint{1.274433in}{1.792622in}}%
\pgfpathlineto{\pgfqpoint{1.283369in}{1.792622in}}%
\pgfpathlineto{\pgfqpoint{1.283369in}{1.508640in}}%
\pgfpathlineto{\pgfqpoint{1.274433in}{1.508640in}}%
\pgfpathlineto{\pgfqpoint{1.274433in}{1.792622in}}%
\pgfpathclose%
\pgfusepath{fill}%
\end{pgfscope}%
\begin{pgfscope}%
\pgfpathrectangle{\pgfqpoint{0.697024in}{0.857143in}}{\pgfqpoint{2.627103in}{1.813434in}}%
\pgfusepath{clip}%
\pgfsetbuttcap%
\pgfsetmiterjoin%
\definecolor{currentfill}{rgb}{0.302379,0.450282,0.300122}%
\pgfsetfillcolor{currentfill}%
\pgfsetlinewidth{0.000000pt}%
\definecolor{currentstroke}{rgb}{0.000000,0.000000,0.000000}%
\pgfsetstrokecolor{currentstroke}%
\pgfsetstrokeopacity{0.000000}%
\pgfsetdash{}{0pt}%
\pgfpathmoveto{\pgfqpoint{1.285604in}{1.847071in}}%
\pgfpathlineto{\pgfqpoint{1.294540in}{1.847071in}}%
\pgfpathlineto{\pgfqpoint{1.294540in}{1.581121in}}%
\pgfpathlineto{\pgfqpoint{1.285604in}{1.581121in}}%
\pgfpathlineto{\pgfqpoint{1.285604in}{1.847071in}}%
\pgfpathclose%
\pgfusepath{fill}%
\end{pgfscope}%
\begin{pgfscope}%
\pgfpathrectangle{\pgfqpoint{0.697024in}{0.857143in}}{\pgfqpoint{2.627103in}{1.813434in}}%
\pgfusepath{clip}%
\pgfsetbuttcap%
\pgfsetmiterjoin%
\definecolor{currentfill}{rgb}{0.302379,0.450282,0.300122}%
\pgfsetfillcolor{currentfill}%
\pgfsetlinewidth{0.000000pt}%
\definecolor{currentstroke}{rgb}{0.000000,0.000000,0.000000}%
\pgfsetstrokecolor{currentstroke}%
\pgfsetstrokeopacity{0.000000}%
\pgfsetdash{}{0pt}%
\pgfpathmoveto{\pgfqpoint{1.296774in}{1.847071in}}%
\pgfpathlineto{\pgfqpoint{1.305711in}{1.847071in}}%
\pgfpathlineto{\pgfqpoint{1.305711in}{1.594696in}}%
\pgfpathlineto{\pgfqpoint{1.296774in}{1.594696in}}%
\pgfpathlineto{\pgfqpoint{1.296774in}{1.847071in}}%
\pgfpathclose%
\pgfusepath{fill}%
\end{pgfscope}%
\begin{pgfscope}%
\pgfpathrectangle{\pgfqpoint{0.697024in}{0.857143in}}{\pgfqpoint{2.627103in}{1.813434in}}%
\pgfusepath{clip}%
\pgfsetbuttcap%
\pgfsetmiterjoin%
\definecolor{currentfill}{rgb}{0.302379,0.450282,0.300122}%
\pgfsetfillcolor{currentfill}%
\pgfsetlinewidth{0.000000pt}%
\definecolor{currentstroke}{rgb}{0.000000,0.000000,0.000000}%
\pgfsetstrokecolor{currentstroke}%
\pgfsetstrokeopacity{0.000000}%
\pgfsetdash{}{0pt}%
\pgfpathmoveto{\pgfqpoint{1.307945in}{1.847071in}}%
\pgfpathlineto{\pgfqpoint{1.316881in}{1.847071in}}%
\pgfpathlineto{\pgfqpoint{1.316881in}{1.586350in}}%
\pgfpathlineto{\pgfqpoint{1.307945in}{1.586350in}}%
\pgfpathlineto{\pgfqpoint{1.307945in}{1.847071in}}%
\pgfpathclose%
\pgfusepath{fill}%
\end{pgfscope}%
\begin{pgfscope}%
\pgfpathrectangle{\pgfqpoint{0.697024in}{0.857143in}}{\pgfqpoint{2.627103in}{1.813434in}}%
\pgfusepath{clip}%
\pgfsetbuttcap%
\pgfsetmiterjoin%
\definecolor{currentfill}{rgb}{0.302379,0.450282,0.300122}%
\pgfsetfillcolor{currentfill}%
\pgfsetlinewidth{0.000000pt}%
\definecolor{currentstroke}{rgb}{0.000000,0.000000,0.000000}%
\pgfsetstrokecolor{currentstroke}%
\pgfsetstrokeopacity{0.000000}%
\pgfsetdash{}{0pt}%
\pgfpathmoveto{\pgfqpoint{1.319115in}{1.779246in}}%
\pgfpathlineto{\pgfqpoint{1.328052in}{1.779246in}}%
\pgfpathlineto{\pgfqpoint{1.328052in}{1.632570in}}%
\pgfpathlineto{\pgfqpoint{1.319115in}{1.632570in}}%
\pgfpathlineto{\pgfqpoint{1.319115in}{1.779246in}}%
\pgfpathclose%
\pgfusepath{fill}%
\end{pgfscope}%
\begin{pgfscope}%
\pgfpathrectangle{\pgfqpoint{0.697024in}{0.857143in}}{\pgfqpoint{2.627103in}{1.813434in}}%
\pgfusepath{clip}%
\pgfsetbuttcap%
\pgfsetmiterjoin%
\definecolor{currentfill}{rgb}{0.302379,0.450282,0.300122}%
\pgfsetfillcolor{currentfill}%
\pgfsetlinewidth{0.000000pt}%
\definecolor{currentstroke}{rgb}{0.000000,0.000000,0.000000}%
\pgfsetstrokecolor{currentstroke}%
\pgfsetstrokeopacity{0.000000}%
\pgfsetdash{}{0pt}%
\pgfpathmoveto{\pgfqpoint{1.330286in}{1.847071in}}%
\pgfpathlineto{\pgfqpoint{1.339222in}{1.847071in}}%
\pgfpathlineto{\pgfqpoint{1.339222in}{1.584374in}}%
\pgfpathlineto{\pgfqpoint{1.330286in}{1.584374in}}%
\pgfpathlineto{\pgfqpoint{1.330286in}{1.847071in}}%
\pgfpathclose%
\pgfusepath{fill}%
\end{pgfscope}%
\begin{pgfscope}%
\pgfpathrectangle{\pgfqpoint{0.697024in}{0.857143in}}{\pgfqpoint{2.627103in}{1.813434in}}%
\pgfusepath{clip}%
\pgfsetbuttcap%
\pgfsetmiterjoin%
\definecolor{currentfill}{rgb}{0.302379,0.450282,0.300122}%
\pgfsetfillcolor{currentfill}%
\pgfsetlinewidth{0.000000pt}%
\definecolor{currentstroke}{rgb}{0.000000,0.000000,0.000000}%
\pgfsetstrokecolor{currentstroke}%
\pgfsetstrokeopacity{0.000000}%
\pgfsetdash{}{0pt}%
\pgfpathmoveto{\pgfqpoint{1.341457in}{1.847071in}}%
\pgfpathlineto{\pgfqpoint{1.350393in}{1.847071in}}%
\pgfpathlineto{\pgfqpoint{1.350393in}{1.710613in}}%
\pgfpathlineto{\pgfqpoint{1.341457in}{1.710613in}}%
\pgfpathlineto{\pgfqpoint{1.341457in}{1.847071in}}%
\pgfpathclose%
\pgfusepath{fill}%
\end{pgfscope}%
\begin{pgfscope}%
\pgfpathrectangle{\pgfqpoint{0.697024in}{0.857143in}}{\pgfqpoint{2.627103in}{1.813434in}}%
\pgfusepath{clip}%
\pgfsetbuttcap%
\pgfsetmiterjoin%
\definecolor{currentfill}{rgb}{0.302379,0.450282,0.300122}%
\pgfsetfillcolor{currentfill}%
\pgfsetlinewidth{0.000000pt}%
\definecolor{currentstroke}{rgb}{0.000000,0.000000,0.000000}%
\pgfsetstrokecolor{currentstroke}%
\pgfsetstrokeopacity{0.000000}%
\pgfsetdash{}{0pt}%
\pgfpathmoveto{\pgfqpoint{1.352627in}{1.847071in}}%
\pgfpathlineto{\pgfqpoint{1.361564in}{1.847071in}}%
\pgfpathlineto{\pgfqpoint{1.361564in}{1.716181in}}%
\pgfpathlineto{\pgfqpoint{1.352627in}{1.716181in}}%
\pgfpathlineto{\pgfqpoint{1.352627in}{1.847071in}}%
\pgfpathclose%
\pgfusepath{fill}%
\end{pgfscope}%
\begin{pgfscope}%
\pgfpathrectangle{\pgfqpoint{0.697024in}{0.857143in}}{\pgfqpoint{2.627103in}{1.813434in}}%
\pgfusepath{clip}%
\pgfsetbuttcap%
\pgfsetmiterjoin%
\definecolor{currentfill}{rgb}{0.302379,0.450282,0.300122}%
\pgfsetfillcolor{currentfill}%
\pgfsetlinewidth{0.000000pt}%
\definecolor{currentstroke}{rgb}{0.000000,0.000000,0.000000}%
\pgfsetstrokecolor{currentstroke}%
\pgfsetstrokeopacity{0.000000}%
\pgfsetdash{}{0pt}%
\pgfpathmoveto{\pgfqpoint{1.363798in}{1.847071in}}%
\pgfpathlineto{\pgfqpoint{1.372734in}{1.847071in}}%
\pgfpathlineto{\pgfqpoint{1.372734in}{1.766379in}}%
\pgfpathlineto{\pgfqpoint{1.363798in}{1.766379in}}%
\pgfpathlineto{\pgfqpoint{1.363798in}{1.847071in}}%
\pgfpathclose%
\pgfusepath{fill}%
\end{pgfscope}%
\begin{pgfscope}%
\pgfpathrectangle{\pgfqpoint{0.697024in}{0.857143in}}{\pgfqpoint{2.627103in}{1.813434in}}%
\pgfusepath{clip}%
\pgfsetbuttcap%
\pgfsetmiterjoin%
\definecolor{currentfill}{rgb}{0.302379,0.450282,0.300122}%
\pgfsetfillcolor{currentfill}%
\pgfsetlinewidth{0.000000pt}%
\definecolor{currentstroke}{rgb}{0.000000,0.000000,0.000000}%
\pgfsetstrokecolor{currentstroke}%
\pgfsetstrokeopacity{0.000000}%
\pgfsetdash{}{0pt}%
\pgfpathmoveto{\pgfqpoint{1.374968in}{1.847071in}}%
\pgfpathlineto{\pgfqpoint{1.383905in}{1.847071in}}%
\pgfpathlineto{\pgfqpoint{1.383905in}{1.793829in}}%
\pgfpathlineto{\pgfqpoint{1.374968in}{1.793829in}}%
\pgfpathlineto{\pgfqpoint{1.374968in}{1.847071in}}%
\pgfpathclose%
\pgfusepath{fill}%
\end{pgfscope}%
\begin{pgfscope}%
\pgfpathrectangle{\pgfqpoint{0.697024in}{0.857143in}}{\pgfqpoint{2.627103in}{1.813434in}}%
\pgfusepath{clip}%
\pgfsetbuttcap%
\pgfsetmiterjoin%
\definecolor{currentfill}{rgb}{0.302379,0.450282,0.300122}%
\pgfsetfillcolor{currentfill}%
\pgfsetlinewidth{0.000000pt}%
\definecolor{currentstroke}{rgb}{0.000000,0.000000,0.000000}%
\pgfsetstrokecolor{currentstroke}%
\pgfsetstrokeopacity{0.000000}%
\pgfsetdash{}{0pt}%
\pgfpathmoveto{\pgfqpoint{1.386139in}{1.847071in}}%
\pgfpathlineto{\pgfqpoint{1.395076in}{1.847071in}}%
\pgfpathlineto{\pgfqpoint{1.395076in}{1.912109in}}%
\pgfpathlineto{\pgfqpoint{1.386139in}{1.912109in}}%
\pgfpathlineto{\pgfqpoint{1.386139in}{1.847071in}}%
\pgfpathclose%
\pgfusepath{fill}%
\end{pgfscope}%
\begin{pgfscope}%
\pgfpathrectangle{\pgfqpoint{0.697024in}{0.857143in}}{\pgfqpoint{2.627103in}{1.813434in}}%
\pgfusepath{clip}%
\pgfsetbuttcap%
\pgfsetmiterjoin%
\definecolor{currentfill}{rgb}{0.302379,0.450282,0.300122}%
\pgfsetfillcolor{currentfill}%
\pgfsetlinewidth{0.000000pt}%
\definecolor{currentstroke}{rgb}{0.000000,0.000000,0.000000}%
\pgfsetstrokecolor{currentstroke}%
\pgfsetstrokeopacity{0.000000}%
\pgfsetdash{}{0pt}%
\pgfpathmoveto{\pgfqpoint{1.397310in}{1.821999in}}%
\pgfpathlineto{\pgfqpoint{1.406246in}{1.821999in}}%
\pgfpathlineto{\pgfqpoint{1.406246in}{1.800678in}}%
\pgfpathlineto{\pgfqpoint{1.397310in}{1.800678in}}%
\pgfpathlineto{\pgfqpoint{1.397310in}{1.821999in}}%
\pgfpathclose%
\pgfusepath{fill}%
\end{pgfscope}%
\begin{pgfscope}%
\pgfpathrectangle{\pgfqpoint{0.697024in}{0.857143in}}{\pgfqpoint{2.627103in}{1.813434in}}%
\pgfusepath{clip}%
\pgfsetbuttcap%
\pgfsetmiterjoin%
\definecolor{currentfill}{rgb}{0.302379,0.450282,0.300122}%
\pgfsetfillcolor{currentfill}%
\pgfsetlinewidth{0.000000pt}%
\definecolor{currentstroke}{rgb}{0.000000,0.000000,0.000000}%
\pgfsetstrokecolor{currentstroke}%
\pgfsetstrokeopacity{0.000000}%
\pgfsetdash{}{0pt}%
\pgfpathmoveto{\pgfqpoint{1.408480in}{1.949761in}}%
\pgfpathlineto{\pgfqpoint{1.417417in}{1.949761in}}%
\pgfpathlineto{\pgfqpoint{1.417417in}{2.059029in}}%
\pgfpathlineto{\pgfqpoint{1.408480in}{2.059029in}}%
\pgfpathlineto{\pgfqpoint{1.408480in}{1.949761in}}%
\pgfpathclose%
\pgfusepath{fill}%
\end{pgfscope}%
\begin{pgfscope}%
\pgfpathrectangle{\pgfqpoint{0.697024in}{0.857143in}}{\pgfqpoint{2.627103in}{1.813434in}}%
\pgfusepath{clip}%
\pgfsetbuttcap%
\pgfsetmiterjoin%
\definecolor{currentfill}{rgb}{0.302379,0.450282,0.300122}%
\pgfsetfillcolor{currentfill}%
\pgfsetlinewidth{0.000000pt}%
\definecolor{currentstroke}{rgb}{0.000000,0.000000,0.000000}%
\pgfsetstrokecolor{currentstroke}%
\pgfsetstrokeopacity{0.000000}%
\pgfsetdash{}{0pt}%
\pgfpathmoveto{\pgfqpoint{1.419651in}{1.869869in}}%
\pgfpathlineto{\pgfqpoint{1.428587in}{1.869869in}}%
\pgfpathlineto{\pgfqpoint{1.428587in}{2.047687in}}%
\pgfpathlineto{\pgfqpoint{1.419651in}{2.047687in}}%
\pgfpathlineto{\pgfqpoint{1.419651in}{1.869869in}}%
\pgfpathclose%
\pgfusepath{fill}%
\end{pgfscope}%
\begin{pgfscope}%
\pgfpathrectangle{\pgfqpoint{0.697024in}{0.857143in}}{\pgfqpoint{2.627103in}{1.813434in}}%
\pgfusepath{clip}%
\pgfsetbuttcap%
\pgfsetmiterjoin%
\definecolor{currentfill}{rgb}{0.302379,0.450282,0.300122}%
\pgfsetfillcolor{currentfill}%
\pgfsetlinewidth{0.000000pt}%
\definecolor{currentstroke}{rgb}{0.000000,0.000000,0.000000}%
\pgfsetstrokecolor{currentstroke}%
\pgfsetstrokeopacity{0.000000}%
\pgfsetdash{}{0pt}%
\pgfpathmoveto{\pgfqpoint{1.430821in}{1.872725in}}%
\pgfpathlineto{\pgfqpoint{1.439758in}{1.872725in}}%
\pgfpathlineto{\pgfqpoint{1.439758in}{1.941022in}}%
\pgfpathlineto{\pgfqpoint{1.430821in}{1.941022in}}%
\pgfpathlineto{\pgfqpoint{1.430821in}{1.872725in}}%
\pgfpathclose%
\pgfusepath{fill}%
\end{pgfscope}%
\begin{pgfscope}%
\pgfpathrectangle{\pgfqpoint{0.697024in}{0.857143in}}{\pgfqpoint{2.627103in}{1.813434in}}%
\pgfusepath{clip}%
\pgfsetbuttcap%
\pgfsetmiterjoin%
\definecolor{currentfill}{rgb}{0.302379,0.450282,0.300122}%
\pgfsetfillcolor{currentfill}%
\pgfsetlinewidth{0.000000pt}%
\definecolor{currentstroke}{rgb}{0.000000,0.000000,0.000000}%
\pgfsetstrokecolor{currentstroke}%
\pgfsetstrokeopacity{0.000000}%
\pgfsetdash{}{0pt}%
\pgfpathmoveto{\pgfqpoint{1.441992in}{1.742013in}}%
\pgfpathlineto{\pgfqpoint{1.450929in}{1.742013in}}%
\pgfpathlineto{\pgfqpoint{1.450929in}{1.726231in}}%
\pgfpathlineto{\pgfqpoint{1.441992in}{1.726231in}}%
\pgfpathlineto{\pgfqpoint{1.441992in}{1.742013in}}%
\pgfpathclose%
\pgfusepath{fill}%
\end{pgfscope}%
\begin{pgfscope}%
\pgfpathrectangle{\pgfqpoint{0.697024in}{0.857143in}}{\pgfqpoint{2.627103in}{1.813434in}}%
\pgfusepath{clip}%
\pgfsetbuttcap%
\pgfsetmiterjoin%
\definecolor{currentfill}{rgb}{0.302379,0.450282,0.300122}%
\pgfsetfillcolor{currentfill}%
\pgfsetlinewidth{0.000000pt}%
\definecolor{currentstroke}{rgb}{0.000000,0.000000,0.000000}%
\pgfsetstrokecolor{currentstroke}%
\pgfsetstrokeopacity{0.000000}%
\pgfsetdash{}{0pt}%
\pgfpathmoveto{\pgfqpoint{1.453163in}{1.847071in}}%
\pgfpathlineto{\pgfqpoint{1.462099in}{1.847071in}}%
\pgfpathlineto{\pgfqpoint{1.462099in}{1.907816in}}%
\pgfpathlineto{\pgfqpoint{1.453163in}{1.907816in}}%
\pgfpathlineto{\pgfqpoint{1.453163in}{1.847071in}}%
\pgfpathclose%
\pgfusepath{fill}%
\end{pgfscope}%
\begin{pgfscope}%
\pgfpathrectangle{\pgfqpoint{0.697024in}{0.857143in}}{\pgfqpoint{2.627103in}{1.813434in}}%
\pgfusepath{clip}%
\pgfsetbuttcap%
\pgfsetmiterjoin%
\definecolor{currentfill}{rgb}{0.302379,0.450282,0.300122}%
\pgfsetfillcolor{currentfill}%
\pgfsetlinewidth{0.000000pt}%
\definecolor{currentstroke}{rgb}{0.000000,0.000000,0.000000}%
\pgfsetstrokecolor{currentstroke}%
\pgfsetstrokeopacity{0.000000}%
\pgfsetdash{}{0pt}%
\pgfpathmoveto{\pgfqpoint{1.464333in}{1.847071in}}%
\pgfpathlineto{\pgfqpoint{1.473270in}{1.847071in}}%
\pgfpathlineto{\pgfqpoint{1.473270in}{2.011678in}}%
\pgfpathlineto{\pgfqpoint{1.464333in}{2.011678in}}%
\pgfpathlineto{\pgfqpoint{1.464333in}{1.847071in}}%
\pgfpathclose%
\pgfusepath{fill}%
\end{pgfscope}%
\begin{pgfscope}%
\pgfpathrectangle{\pgfqpoint{0.697024in}{0.857143in}}{\pgfqpoint{2.627103in}{1.813434in}}%
\pgfusepath{clip}%
\pgfsetbuttcap%
\pgfsetmiterjoin%
\definecolor{currentfill}{rgb}{0.302379,0.450282,0.300122}%
\pgfsetfillcolor{currentfill}%
\pgfsetlinewidth{0.000000pt}%
\definecolor{currentstroke}{rgb}{0.000000,0.000000,0.000000}%
\pgfsetstrokecolor{currentstroke}%
\pgfsetstrokeopacity{0.000000}%
\pgfsetdash{}{0pt}%
\pgfpathmoveto{\pgfqpoint{1.475504in}{1.905264in}}%
\pgfpathlineto{\pgfqpoint{1.484440in}{1.905264in}}%
\pgfpathlineto{\pgfqpoint{1.484440in}{2.004638in}}%
\pgfpathlineto{\pgfqpoint{1.475504in}{2.004638in}}%
\pgfpathlineto{\pgfqpoint{1.475504in}{1.905264in}}%
\pgfpathclose%
\pgfusepath{fill}%
\end{pgfscope}%
\begin{pgfscope}%
\pgfpathrectangle{\pgfqpoint{0.697024in}{0.857143in}}{\pgfqpoint{2.627103in}{1.813434in}}%
\pgfusepath{clip}%
\pgfsetbuttcap%
\pgfsetmiterjoin%
\definecolor{currentfill}{rgb}{0.302379,0.450282,0.300122}%
\pgfsetfillcolor{currentfill}%
\pgfsetlinewidth{0.000000pt}%
\definecolor{currentstroke}{rgb}{0.000000,0.000000,0.000000}%
\pgfsetstrokecolor{currentstroke}%
\pgfsetstrokeopacity{0.000000}%
\pgfsetdash{}{0pt}%
\pgfpathmoveto{\pgfqpoint{1.486674in}{1.847071in}}%
\pgfpathlineto{\pgfqpoint{1.495611in}{1.847071in}}%
\pgfpathlineto{\pgfqpoint{1.495611in}{1.977692in}}%
\pgfpathlineto{\pgfqpoint{1.486674in}{1.977692in}}%
\pgfpathlineto{\pgfqpoint{1.486674in}{1.847071in}}%
\pgfpathclose%
\pgfusepath{fill}%
\end{pgfscope}%
\begin{pgfscope}%
\pgfpathrectangle{\pgfqpoint{0.697024in}{0.857143in}}{\pgfqpoint{2.627103in}{1.813434in}}%
\pgfusepath{clip}%
\pgfsetbuttcap%
\pgfsetmiterjoin%
\definecolor{currentfill}{rgb}{0.302379,0.450282,0.300122}%
\pgfsetfillcolor{currentfill}%
\pgfsetlinewidth{0.000000pt}%
\definecolor{currentstroke}{rgb}{0.000000,0.000000,0.000000}%
\pgfsetstrokecolor{currentstroke}%
\pgfsetstrokeopacity{0.000000}%
\pgfsetdash{}{0pt}%
\pgfpathmoveto{\pgfqpoint{1.497845in}{1.867380in}}%
\pgfpathlineto{\pgfqpoint{1.506782in}{1.867380in}}%
\pgfpathlineto{\pgfqpoint{1.506782in}{1.870941in}}%
\pgfpathlineto{\pgfqpoint{1.497845in}{1.870941in}}%
\pgfpathlineto{\pgfqpoint{1.497845in}{1.867380in}}%
\pgfpathclose%
\pgfusepath{fill}%
\end{pgfscope}%
\begin{pgfscope}%
\pgfpathrectangle{\pgfqpoint{0.697024in}{0.857143in}}{\pgfqpoint{2.627103in}{1.813434in}}%
\pgfusepath{clip}%
\pgfsetbuttcap%
\pgfsetmiterjoin%
\definecolor{currentfill}{rgb}{0.302379,0.450282,0.300122}%
\pgfsetfillcolor{currentfill}%
\pgfsetlinewidth{0.000000pt}%
\definecolor{currentstroke}{rgb}{0.000000,0.000000,0.000000}%
\pgfsetstrokecolor{currentstroke}%
\pgfsetstrokeopacity{0.000000}%
\pgfsetdash{}{0pt}%
\pgfpathmoveto{\pgfqpoint{1.509016in}{1.784441in}}%
\pgfpathlineto{\pgfqpoint{1.517952in}{1.784441in}}%
\pgfpathlineto{\pgfqpoint{1.517952in}{1.687074in}}%
\pgfpathlineto{\pgfqpoint{1.509016in}{1.687074in}}%
\pgfpathlineto{\pgfqpoint{1.509016in}{1.784441in}}%
\pgfpathclose%
\pgfusepath{fill}%
\end{pgfscope}%
\begin{pgfscope}%
\pgfpathrectangle{\pgfqpoint{0.697024in}{0.857143in}}{\pgfqpoint{2.627103in}{1.813434in}}%
\pgfusepath{clip}%
\pgfsetbuttcap%
\pgfsetmiterjoin%
\definecolor{currentfill}{rgb}{0.302379,0.450282,0.300122}%
\pgfsetfillcolor{currentfill}%
\pgfsetlinewidth{0.000000pt}%
\definecolor{currentstroke}{rgb}{0.000000,0.000000,0.000000}%
\pgfsetstrokecolor{currentstroke}%
\pgfsetstrokeopacity{0.000000}%
\pgfsetdash{}{0pt}%
\pgfpathmoveto{\pgfqpoint{1.520186in}{1.847071in}}%
\pgfpathlineto{\pgfqpoint{1.529123in}{1.847071in}}%
\pgfpathlineto{\pgfqpoint{1.529123in}{1.853767in}}%
\pgfpathlineto{\pgfqpoint{1.520186in}{1.853767in}}%
\pgfpathlineto{\pgfqpoint{1.520186in}{1.847071in}}%
\pgfpathclose%
\pgfusepath{fill}%
\end{pgfscope}%
\begin{pgfscope}%
\pgfpathrectangle{\pgfqpoint{0.697024in}{0.857143in}}{\pgfqpoint{2.627103in}{1.813434in}}%
\pgfusepath{clip}%
\pgfsetbuttcap%
\pgfsetmiterjoin%
\definecolor{currentfill}{rgb}{0.302379,0.450282,0.300122}%
\pgfsetfillcolor{currentfill}%
\pgfsetlinewidth{0.000000pt}%
\definecolor{currentstroke}{rgb}{0.000000,0.000000,0.000000}%
\pgfsetstrokecolor{currentstroke}%
\pgfsetstrokeopacity{0.000000}%
\pgfsetdash{}{0pt}%
\pgfpathmoveto{\pgfqpoint{1.531357in}{1.731622in}}%
\pgfpathlineto{\pgfqpoint{1.540293in}{1.731622in}}%
\pgfpathlineto{\pgfqpoint{1.540293in}{1.625250in}}%
\pgfpathlineto{\pgfqpoint{1.531357in}{1.625250in}}%
\pgfpathlineto{\pgfqpoint{1.531357in}{1.731622in}}%
\pgfpathclose%
\pgfusepath{fill}%
\end{pgfscope}%
\begin{pgfscope}%
\pgfpathrectangle{\pgfqpoint{0.697024in}{0.857143in}}{\pgfqpoint{2.627103in}{1.813434in}}%
\pgfusepath{clip}%
\pgfsetbuttcap%
\pgfsetmiterjoin%
\definecolor{currentfill}{rgb}{0.302379,0.450282,0.300122}%
\pgfsetfillcolor{currentfill}%
\pgfsetlinewidth{0.000000pt}%
\definecolor{currentstroke}{rgb}{0.000000,0.000000,0.000000}%
\pgfsetstrokecolor{currentstroke}%
\pgfsetstrokeopacity{0.000000}%
\pgfsetdash{}{0pt}%
\pgfpathmoveto{\pgfqpoint{1.542528in}{1.697441in}}%
\pgfpathlineto{\pgfqpoint{1.551464in}{1.697441in}}%
\pgfpathlineto{\pgfqpoint{1.551464in}{1.622144in}}%
\pgfpathlineto{\pgfqpoint{1.542528in}{1.622144in}}%
\pgfpathlineto{\pgfqpoint{1.542528in}{1.697441in}}%
\pgfpathclose%
\pgfusepath{fill}%
\end{pgfscope}%
\begin{pgfscope}%
\pgfpathrectangle{\pgfqpoint{0.697024in}{0.857143in}}{\pgfqpoint{2.627103in}{1.813434in}}%
\pgfusepath{clip}%
\pgfsetbuttcap%
\pgfsetmiterjoin%
\definecolor{currentfill}{rgb}{0.302379,0.450282,0.300122}%
\pgfsetfillcolor{currentfill}%
\pgfsetlinewidth{0.000000pt}%
\definecolor{currentstroke}{rgb}{0.000000,0.000000,0.000000}%
\pgfsetstrokecolor{currentstroke}%
\pgfsetstrokeopacity{0.000000}%
\pgfsetdash{}{0pt}%
\pgfpathmoveto{\pgfqpoint{1.553698in}{1.796508in}}%
\pgfpathlineto{\pgfqpoint{1.562635in}{1.796508in}}%
\pgfpathlineto{\pgfqpoint{1.562635in}{1.660883in}}%
\pgfpathlineto{\pgfqpoint{1.553698in}{1.660883in}}%
\pgfpathlineto{\pgfqpoint{1.553698in}{1.796508in}}%
\pgfpathclose%
\pgfusepath{fill}%
\end{pgfscope}%
\begin{pgfscope}%
\pgfpathrectangle{\pgfqpoint{0.697024in}{0.857143in}}{\pgfqpoint{2.627103in}{1.813434in}}%
\pgfusepath{clip}%
\pgfsetbuttcap%
\pgfsetmiterjoin%
\definecolor{currentfill}{rgb}{0.302379,0.450282,0.300122}%
\pgfsetfillcolor{currentfill}%
\pgfsetlinewidth{0.000000pt}%
\definecolor{currentstroke}{rgb}{0.000000,0.000000,0.000000}%
\pgfsetstrokecolor{currentstroke}%
\pgfsetstrokeopacity{0.000000}%
\pgfsetdash{}{0pt}%
\pgfpathmoveto{\pgfqpoint{1.564869in}{1.761882in}}%
\pgfpathlineto{\pgfqpoint{1.573805in}{1.761882in}}%
\pgfpathlineto{\pgfqpoint{1.573805in}{1.682518in}}%
\pgfpathlineto{\pgfqpoint{1.564869in}{1.682518in}}%
\pgfpathlineto{\pgfqpoint{1.564869in}{1.761882in}}%
\pgfpathclose%
\pgfusepath{fill}%
\end{pgfscope}%
\begin{pgfscope}%
\pgfpathrectangle{\pgfqpoint{0.697024in}{0.857143in}}{\pgfqpoint{2.627103in}{1.813434in}}%
\pgfusepath{clip}%
\pgfsetbuttcap%
\pgfsetmiterjoin%
\definecolor{currentfill}{rgb}{0.302379,0.450282,0.300122}%
\pgfsetfillcolor{currentfill}%
\pgfsetlinewidth{0.000000pt}%
\definecolor{currentstroke}{rgb}{0.000000,0.000000,0.000000}%
\pgfsetstrokecolor{currentstroke}%
\pgfsetstrokeopacity{0.000000}%
\pgfsetdash{}{0pt}%
\pgfpathmoveto{\pgfqpoint{1.576039in}{1.847071in}}%
\pgfpathlineto{\pgfqpoint{1.584976in}{1.847071in}}%
\pgfpathlineto{\pgfqpoint{1.584976in}{1.874835in}}%
\pgfpathlineto{\pgfqpoint{1.576039in}{1.874835in}}%
\pgfpathlineto{\pgfqpoint{1.576039in}{1.847071in}}%
\pgfpathclose%
\pgfusepath{fill}%
\end{pgfscope}%
\begin{pgfscope}%
\pgfpathrectangle{\pgfqpoint{0.697024in}{0.857143in}}{\pgfqpoint{2.627103in}{1.813434in}}%
\pgfusepath{clip}%
\pgfsetbuttcap%
\pgfsetmiterjoin%
\definecolor{currentfill}{rgb}{0.302379,0.450282,0.300122}%
\pgfsetfillcolor{currentfill}%
\pgfsetlinewidth{0.000000pt}%
\definecolor{currentstroke}{rgb}{0.000000,0.000000,0.000000}%
\pgfsetstrokecolor{currentstroke}%
\pgfsetstrokeopacity{0.000000}%
\pgfsetdash{}{0pt}%
\pgfpathmoveto{\pgfqpoint{1.587210in}{1.876087in}}%
\pgfpathlineto{\pgfqpoint{1.596146in}{1.876087in}}%
\pgfpathlineto{\pgfqpoint{1.596146in}{1.931603in}}%
\pgfpathlineto{\pgfqpoint{1.587210in}{1.931603in}}%
\pgfpathlineto{\pgfqpoint{1.587210in}{1.876087in}}%
\pgfpathclose%
\pgfusepath{fill}%
\end{pgfscope}%
\begin{pgfscope}%
\pgfpathrectangle{\pgfqpoint{0.697024in}{0.857143in}}{\pgfqpoint{2.627103in}{1.813434in}}%
\pgfusepath{clip}%
\pgfsetbuttcap%
\pgfsetmiterjoin%
\definecolor{currentfill}{rgb}{0.302379,0.450282,0.300122}%
\pgfsetfillcolor{currentfill}%
\pgfsetlinewidth{0.000000pt}%
\definecolor{currentstroke}{rgb}{0.000000,0.000000,0.000000}%
\pgfsetstrokecolor{currentstroke}%
\pgfsetstrokeopacity{0.000000}%
\pgfsetdash{}{0pt}%
\pgfpathmoveto{\pgfqpoint{1.598381in}{1.847071in}}%
\pgfpathlineto{\pgfqpoint{1.607317in}{1.847071in}}%
\pgfpathlineto{\pgfqpoint{1.607317in}{1.999484in}}%
\pgfpathlineto{\pgfqpoint{1.598381in}{1.999484in}}%
\pgfpathlineto{\pgfqpoint{1.598381in}{1.847071in}}%
\pgfpathclose%
\pgfusepath{fill}%
\end{pgfscope}%
\begin{pgfscope}%
\pgfpathrectangle{\pgfqpoint{0.697024in}{0.857143in}}{\pgfqpoint{2.627103in}{1.813434in}}%
\pgfusepath{clip}%
\pgfsetbuttcap%
\pgfsetmiterjoin%
\definecolor{currentfill}{rgb}{0.302379,0.450282,0.300122}%
\pgfsetfillcolor{currentfill}%
\pgfsetlinewidth{0.000000pt}%
\definecolor{currentstroke}{rgb}{0.000000,0.000000,0.000000}%
\pgfsetstrokecolor{currentstroke}%
\pgfsetstrokeopacity{0.000000}%
\pgfsetdash{}{0pt}%
\pgfpathmoveto{\pgfqpoint{1.609551in}{1.847071in}}%
\pgfpathlineto{\pgfqpoint{1.618488in}{1.847071in}}%
\pgfpathlineto{\pgfqpoint{1.618488in}{2.018391in}}%
\pgfpathlineto{\pgfqpoint{1.609551in}{2.018391in}}%
\pgfpathlineto{\pgfqpoint{1.609551in}{1.847071in}}%
\pgfpathclose%
\pgfusepath{fill}%
\end{pgfscope}%
\begin{pgfscope}%
\pgfpathrectangle{\pgfqpoint{0.697024in}{0.857143in}}{\pgfqpoint{2.627103in}{1.813434in}}%
\pgfusepath{clip}%
\pgfsetbuttcap%
\pgfsetmiterjoin%
\definecolor{currentfill}{rgb}{0.302379,0.450282,0.300122}%
\pgfsetfillcolor{currentfill}%
\pgfsetlinewidth{0.000000pt}%
\definecolor{currentstroke}{rgb}{0.000000,0.000000,0.000000}%
\pgfsetstrokecolor{currentstroke}%
\pgfsetstrokeopacity{0.000000}%
\pgfsetdash{}{0pt}%
\pgfpathmoveto{\pgfqpoint{1.620722in}{1.847071in}}%
\pgfpathlineto{\pgfqpoint{1.629658in}{1.847071in}}%
\pgfpathlineto{\pgfqpoint{1.629658in}{2.053922in}}%
\pgfpathlineto{\pgfqpoint{1.620722in}{2.053922in}}%
\pgfpathlineto{\pgfqpoint{1.620722in}{1.847071in}}%
\pgfpathclose%
\pgfusepath{fill}%
\end{pgfscope}%
\begin{pgfscope}%
\pgfpathrectangle{\pgfqpoint{0.697024in}{0.857143in}}{\pgfqpoint{2.627103in}{1.813434in}}%
\pgfusepath{clip}%
\pgfsetbuttcap%
\pgfsetmiterjoin%
\definecolor{currentfill}{rgb}{0.302379,0.450282,0.300122}%
\pgfsetfillcolor{currentfill}%
\pgfsetlinewidth{0.000000pt}%
\definecolor{currentstroke}{rgb}{0.000000,0.000000,0.000000}%
\pgfsetstrokecolor{currentstroke}%
\pgfsetstrokeopacity{0.000000}%
\pgfsetdash{}{0pt}%
\pgfpathmoveto{\pgfqpoint{1.631892in}{1.847071in}}%
\pgfpathlineto{\pgfqpoint{1.640829in}{1.847071in}}%
\pgfpathlineto{\pgfqpoint{1.640829in}{2.061452in}}%
\pgfpathlineto{\pgfqpoint{1.631892in}{2.061452in}}%
\pgfpathlineto{\pgfqpoint{1.631892in}{1.847071in}}%
\pgfpathclose%
\pgfusepath{fill}%
\end{pgfscope}%
\begin{pgfscope}%
\pgfpathrectangle{\pgfqpoint{0.697024in}{0.857143in}}{\pgfqpoint{2.627103in}{1.813434in}}%
\pgfusepath{clip}%
\pgfsetbuttcap%
\pgfsetmiterjoin%
\definecolor{currentfill}{rgb}{0.302379,0.450282,0.300122}%
\pgfsetfillcolor{currentfill}%
\pgfsetlinewidth{0.000000pt}%
\definecolor{currentstroke}{rgb}{0.000000,0.000000,0.000000}%
\pgfsetstrokecolor{currentstroke}%
\pgfsetstrokeopacity{0.000000}%
\pgfsetdash{}{0pt}%
\pgfpathmoveto{\pgfqpoint{1.643063in}{1.847071in}}%
\pgfpathlineto{\pgfqpoint{1.651999in}{1.847071in}}%
\pgfpathlineto{\pgfqpoint{1.651999in}{2.148291in}}%
\pgfpathlineto{\pgfqpoint{1.643063in}{2.148291in}}%
\pgfpathlineto{\pgfqpoint{1.643063in}{1.847071in}}%
\pgfpathclose%
\pgfusepath{fill}%
\end{pgfscope}%
\begin{pgfscope}%
\pgfpathrectangle{\pgfqpoint{0.697024in}{0.857143in}}{\pgfqpoint{2.627103in}{1.813434in}}%
\pgfusepath{clip}%
\pgfsetbuttcap%
\pgfsetmiterjoin%
\definecolor{currentfill}{rgb}{0.302379,0.450282,0.300122}%
\pgfsetfillcolor{currentfill}%
\pgfsetlinewidth{0.000000pt}%
\definecolor{currentstroke}{rgb}{0.000000,0.000000,0.000000}%
\pgfsetstrokecolor{currentstroke}%
\pgfsetstrokeopacity{0.000000}%
\pgfsetdash{}{0pt}%
\pgfpathmoveto{\pgfqpoint{1.654234in}{1.847071in}}%
\pgfpathlineto{\pgfqpoint{1.663170in}{1.847071in}}%
\pgfpathlineto{\pgfqpoint{1.663170in}{2.101637in}}%
\pgfpathlineto{\pgfqpoint{1.654234in}{2.101637in}}%
\pgfpathlineto{\pgfqpoint{1.654234in}{1.847071in}}%
\pgfpathclose%
\pgfusepath{fill}%
\end{pgfscope}%
\begin{pgfscope}%
\pgfpathrectangle{\pgfqpoint{0.697024in}{0.857143in}}{\pgfqpoint{2.627103in}{1.813434in}}%
\pgfusepath{clip}%
\pgfsetbuttcap%
\pgfsetmiterjoin%
\definecolor{currentfill}{rgb}{0.302379,0.450282,0.300122}%
\pgfsetfillcolor{currentfill}%
\pgfsetlinewidth{0.000000pt}%
\definecolor{currentstroke}{rgb}{0.000000,0.000000,0.000000}%
\pgfsetstrokecolor{currentstroke}%
\pgfsetstrokeopacity{0.000000}%
\pgfsetdash{}{0pt}%
\pgfpathmoveto{\pgfqpoint{1.665404in}{1.847071in}}%
\pgfpathlineto{\pgfqpoint{1.674341in}{1.847071in}}%
\pgfpathlineto{\pgfqpoint{1.674341in}{2.162134in}}%
\pgfpathlineto{\pgfqpoint{1.665404in}{2.162134in}}%
\pgfpathlineto{\pgfqpoint{1.665404in}{1.847071in}}%
\pgfpathclose%
\pgfusepath{fill}%
\end{pgfscope}%
\begin{pgfscope}%
\pgfpathrectangle{\pgfqpoint{0.697024in}{0.857143in}}{\pgfqpoint{2.627103in}{1.813434in}}%
\pgfusepath{clip}%
\pgfsetbuttcap%
\pgfsetmiterjoin%
\definecolor{currentfill}{rgb}{0.302379,0.450282,0.300122}%
\pgfsetfillcolor{currentfill}%
\pgfsetlinewidth{0.000000pt}%
\definecolor{currentstroke}{rgb}{0.000000,0.000000,0.000000}%
\pgfsetstrokecolor{currentstroke}%
\pgfsetstrokeopacity{0.000000}%
\pgfsetdash{}{0pt}%
\pgfpathmoveto{\pgfqpoint{1.676575in}{1.847071in}}%
\pgfpathlineto{\pgfqpoint{1.685511in}{1.847071in}}%
\pgfpathlineto{\pgfqpoint{1.685511in}{2.151922in}}%
\pgfpathlineto{\pgfqpoint{1.676575in}{2.151922in}}%
\pgfpathlineto{\pgfqpoint{1.676575in}{1.847071in}}%
\pgfpathclose%
\pgfusepath{fill}%
\end{pgfscope}%
\begin{pgfscope}%
\pgfpathrectangle{\pgfqpoint{0.697024in}{0.857143in}}{\pgfqpoint{2.627103in}{1.813434in}}%
\pgfusepath{clip}%
\pgfsetbuttcap%
\pgfsetmiterjoin%
\definecolor{currentfill}{rgb}{0.302379,0.450282,0.300122}%
\pgfsetfillcolor{currentfill}%
\pgfsetlinewidth{0.000000pt}%
\definecolor{currentstroke}{rgb}{0.000000,0.000000,0.000000}%
\pgfsetstrokecolor{currentstroke}%
\pgfsetstrokeopacity{0.000000}%
\pgfsetdash{}{0pt}%
\pgfpathmoveto{\pgfqpoint{1.687745in}{1.847071in}}%
\pgfpathlineto{\pgfqpoint{1.696682in}{1.847071in}}%
\pgfpathlineto{\pgfqpoint{1.696682in}{2.212756in}}%
\pgfpathlineto{\pgfqpoint{1.687745in}{2.212756in}}%
\pgfpathlineto{\pgfqpoint{1.687745in}{1.847071in}}%
\pgfpathclose%
\pgfusepath{fill}%
\end{pgfscope}%
\begin{pgfscope}%
\pgfpathrectangle{\pgfqpoint{0.697024in}{0.857143in}}{\pgfqpoint{2.627103in}{1.813434in}}%
\pgfusepath{clip}%
\pgfsetbuttcap%
\pgfsetmiterjoin%
\definecolor{currentfill}{rgb}{0.302379,0.450282,0.300122}%
\pgfsetfillcolor{currentfill}%
\pgfsetlinewidth{0.000000pt}%
\definecolor{currentstroke}{rgb}{0.000000,0.000000,0.000000}%
\pgfsetstrokecolor{currentstroke}%
\pgfsetstrokeopacity{0.000000}%
\pgfsetdash{}{0pt}%
\pgfpathmoveto{\pgfqpoint{1.698916in}{1.847071in}}%
\pgfpathlineto{\pgfqpoint{1.707852in}{1.847071in}}%
\pgfpathlineto{\pgfqpoint{1.707852in}{2.178891in}}%
\pgfpathlineto{\pgfqpoint{1.698916in}{2.178891in}}%
\pgfpathlineto{\pgfqpoint{1.698916in}{1.847071in}}%
\pgfpathclose%
\pgfusepath{fill}%
\end{pgfscope}%
\begin{pgfscope}%
\pgfpathrectangle{\pgfqpoint{0.697024in}{0.857143in}}{\pgfqpoint{2.627103in}{1.813434in}}%
\pgfusepath{clip}%
\pgfsetbuttcap%
\pgfsetmiterjoin%
\definecolor{currentfill}{rgb}{0.302379,0.450282,0.300122}%
\pgfsetfillcolor{currentfill}%
\pgfsetlinewidth{0.000000pt}%
\definecolor{currentstroke}{rgb}{0.000000,0.000000,0.000000}%
\pgfsetstrokecolor{currentstroke}%
\pgfsetstrokeopacity{0.000000}%
\pgfsetdash{}{0pt}%
\pgfpathmoveto{\pgfqpoint{1.710087in}{1.847071in}}%
\pgfpathlineto{\pgfqpoint{1.719023in}{1.847071in}}%
\pgfpathlineto{\pgfqpoint{1.719023in}{2.204658in}}%
\pgfpathlineto{\pgfqpoint{1.710087in}{2.204658in}}%
\pgfpathlineto{\pgfqpoint{1.710087in}{1.847071in}}%
\pgfpathclose%
\pgfusepath{fill}%
\end{pgfscope}%
\begin{pgfscope}%
\pgfpathrectangle{\pgfqpoint{0.697024in}{0.857143in}}{\pgfqpoint{2.627103in}{1.813434in}}%
\pgfusepath{clip}%
\pgfsetbuttcap%
\pgfsetmiterjoin%
\definecolor{currentfill}{rgb}{0.302379,0.450282,0.300122}%
\pgfsetfillcolor{currentfill}%
\pgfsetlinewidth{0.000000pt}%
\definecolor{currentstroke}{rgb}{0.000000,0.000000,0.000000}%
\pgfsetstrokecolor{currentstroke}%
\pgfsetstrokeopacity{0.000000}%
\pgfsetdash{}{0pt}%
\pgfpathmoveto{\pgfqpoint{1.721257in}{1.847071in}}%
\pgfpathlineto{\pgfqpoint{1.730194in}{1.847071in}}%
\pgfpathlineto{\pgfqpoint{1.730194in}{2.212595in}}%
\pgfpathlineto{\pgfqpoint{1.721257in}{2.212595in}}%
\pgfpathlineto{\pgfqpoint{1.721257in}{1.847071in}}%
\pgfpathclose%
\pgfusepath{fill}%
\end{pgfscope}%
\begin{pgfscope}%
\pgfpathrectangle{\pgfqpoint{0.697024in}{0.857143in}}{\pgfqpoint{2.627103in}{1.813434in}}%
\pgfusepath{clip}%
\pgfsetbuttcap%
\pgfsetmiterjoin%
\definecolor{currentfill}{rgb}{0.302379,0.450282,0.300122}%
\pgfsetfillcolor{currentfill}%
\pgfsetlinewidth{0.000000pt}%
\definecolor{currentstroke}{rgb}{0.000000,0.000000,0.000000}%
\pgfsetstrokecolor{currentstroke}%
\pgfsetstrokeopacity{0.000000}%
\pgfsetdash{}{0pt}%
\pgfpathmoveto{\pgfqpoint{1.732428in}{1.847071in}}%
\pgfpathlineto{\pgfqpoint{1.741364in}{1.847071in}}%
\pgfpathlineto{\pgfqpoint{1.741364in}{2.190952in}}%
\pgfpathlineto{\pgfqpoint{1.732428in}{2.190952in}}%
\pgfpathlineto{\pgfqpoint{1.732428in}{1.847071in}}%
\pgfpathclose%
\pgfusepath{fill}%
\end{pgfscope}%
\begin{pgfscope}%
\pgfpathrectangle{\pgfqpoint{0.697024in}{0.857143in}}{\pgfqpoint{2.627103in}{1.813434in}}%
\pgfusepath{clip}%
\pgfsetbuttcap%
\pgfsetmiterjoin%
\definecolor{currentfill}{rgb}{0.302379,0.450282,0.300122}%
\pgfsetfillcolor{currentfill}%
\pgfsetlinewidth{0.000000pt}%
\definecolor{currentstroke}{rgb}{0.000000,0.000000,0.000000}%
\pgfsetstrokecolor{currentstroke}%
\pgfsetstrokeopacity{0.000000}%
\pgfsetdash{}{0pt}%
\pgfpathmoveto{\pgfqpoint{1.743598in}{1.847071in}}%
\pgfpathlineto{\pgfqpoint{1.752535in}{1.847071in}}%
\pgfpathlineto{\pgfqpoint{1.752535in}{2.288257in}}%
\pgfpathlineto{\pgfqpoint{1.743598in}{2.288257in}}%
\pgfpathlineto{\pgfqpoint{1.743598in}{1.847071in}}%
\pgfpathclose%
\pgfusepath{fill}%
\end{pgfscope}%
\begin{pgfscope}%
\pgfpathrectangle{\pgfqpoint{0.697024in}{0.857143in}}{\pgfqpoint{2.627103in}{1.813434in}}%
\pgfusepath{clip}%
\pgfsetbuttcap%
\pgfsetmiterjoin%
\definecolor{currentfill}{rgb}{0.302379,0.450282,0.300122}%
\pgfsetfillcolor{currentfill}%
\pgfsetlinewidth{0.000000pt}%
\definecolor{currentstroke}{rgb}{0.000000,0.000000,0.000000}%
\pgfsetstrokecolor{currentstroke}%
\pgfsetstrokeopacity{0.000000}%
\pgfsetdash{}{0pt}%
\pgfpathmoveto{\pgfqpoint{1.754769in}{1.847071in}}%
\pgfpathlineto{\pgfqpoint{1.763705in}{1.847071in}}%
\pgfpathlineto{\pgfqpoint{1.763705in}{2.357787in}}%
\pgfpathlineto{\pgfqpoint{1.754769in}{2.357787in}}%
\pgfpathlineto{\pgfqpoint{1.754769in}{1.847071in}}%
\pgfpathclose%
\pgfusepath{fill}%
\end{pgfscope}%
\begin{pgfscope}%
\pgfpathrectangle{\pgfqpoint{0.697024in}{0.857143in}}{\pgfqpoint{2.627103in}{1.813434in}}%
\pgfusepath{clip}%
\pgfsetbuttcap%
\pgfsetmiterjoin%
\definecolor{currentfill}{rgb}{0.302379,0.450282,0.300122}%
\pgfsetfillcolor{currentfill}%
\pgfsetlinewidth{0.000000pt}%
\definecolor{currentstroke}{rgb}{0.000000,0.000000,0.000000}%
\pgfsetstrokecolor{currentstroke}%
\pgfsetstrokeopacity{0.000000}%
\pgfsetdash{}{0pt}%
\pgfpathmoveto{\pgfqpoint{1.765940in}{1.847071in}}%
\pgfpathlineto{\pgfqpoint{1.774876in}{1.847071in}}%
\pgfpathlineto{\pgfqpoint{1.774876in}{2.306304in}}%
\pgfpathlineto{\pgfqpoint{1.765940in}{2.306304in}}%
\pgfpathlineto{\pgfqpoint{1.765940in}{1.847071in}}%
\pgfpathclose%
\pgfusepath{fill}%
\end{pgfscope}%
\begin{pgfscope}%
\pgfpathrectangle{\pgfqpoint{0.697024in}{0.857143in}}{\pgfqpoint{2.627103in}{1.813434in}}%
\pgfusepath{clip}%
\pgfsetbuttcap%
\pgfsetmiterjoin%
\definecolor{currentfill}{rgb}{0.302379,0.450282,0.300122}%
\pgfsetfillcolor{currentfill}%
\pgfsetlinewidth{0.000000pt}%
\definecolor{currentstroke}{rgb}{0.000000,0.000000,0.000000}%
\pgfsetstrokecolor{currentstroke}%
\pgfsetstrokeopacity{0.000000}%
\pgfsetdash{}{0pt}%
\pgfpathmoveto{\pgfqpoint{1.777110in}{1.847071in}}%
\pgfpathlineto{\pgfqpoint{1.786047in}{1.847071in}}%
\pgfpathlineto{\pgfqpoint{1.786047in}{2.379179in}}%
\pgfpathlineto{\pgfqpoint{1.777110in}{2.379179in}}%
\pgfpathlineto{\pgfqpoint{1.777110in}{1.847071in}}%
\pgfpathclose%
\pgfusepath{fill}%
\end{pgfscope}%
\begin{pgfscope}%
\pgfpathrectangle{\pgfqpoint{0.697024in}{0.857143in}}{\pgfqpoint{2.627103in}{1.813434in}}%
\pgfusepath{clip}%
\pgfsetbuttcap%
\pgfsetmiterjoin%
\definecolor{currentfill}{rgb}{0.302379,0.450282,0.300122}%
\pgfsetfillcolor{currentfill}%
\pgfsetlinewidth{0.000000pt}%
\definecolor{currentstroke}{rgb}{0.000000,0.000000,0.000000}%
\pgfsetstrokecolor{currentstroke}%
\pgfsetstrokeopacity{0.000000}%
\pgfsetdash{}{0pt}%
\pgfpathmoveto{\pgfqpoint{1.788281in}{1.847071in}}%
\pgfpathlineto{\pgfqpoint{1.797217in}{1.847071in}}%
\pgfpathlineto{\pgfqpoint{1.797217in}{2.415145in}}%
\pgfpathlineto{\pgfqpoint{1.788281in}{2.415145in}}%
\pgfpathlineto{\pgfqpoint{1.788281in}{1.847071in}}%
\pgfpathclose%
\pgfusepath{fill}%
\end{pgfscope}%
\begin{pgfscope}%
\pgfpathrectangle{\pgfqpoint{0.697024in}{0.857143in}}{\pgfqpoint{2.627103in}{1.813434in}}%
\pgfusepath{clip}%
\pgfsetbuttcap%
\pgfsetmiterjoin%
\definecolor{currentfill}{rgb}{0.302379,0.450282,0.300122}%
\pgfsetfillcolor{currentfill}%
\pgfsetlinewidth{0.000000pt}%
\definecolor{currentstroke}{rgb}{0.000000,0.000000,0.000000}%
\pgfsetstrokecolor{currentstroke}%
\pgfsetstrokeopacity{0.000000}%
\pgfsetdash{}{0pt}%
\pgfpathmoveto{\pgfqpoint{1.799451in}{1.847071in}}%
\pgfpathlineto{\pgfqpoint{1.808388in}{1.847071in}}%
\pgfpathlineto{\pgfqpoint{1.808388in}{2.469976in}}%
\pgfpathlineto{\pgfqpoint{1.799451in}{2.469976in}}%
\pgfpathlineto{\pgfqpoint{1.799451in}{1.847071in}}%
\pgfpathclose%
\pgfusepath{fill}%
\end{pgfscope}%
\begin{pgfscope}%
\pgfpathrectangle{\pgfqpoint{0.697024in}{0.857143in}}{\pgfqpoint{2.627103in}{1.813434in}}%
\pgfusepath{clip}%
\pgfsetbuttcap%
\pgfsetmiterjoin%
\definecolor{currentfill}{rgb}{0.302379,0.450282,0.300122}%
\pgfsetfillcolor{currentfill}%
\pgfsetlinewidth{0.000000pt}%
\definecolor{currentstroke}{rgb}{0.000000,0.000000,0.000000}%
\pgfsetstrokecolor{currentstroke}%
\pgfsetstrokeopacity{0.000000}%
\pgfsetdash{}{0pt}%
\pgfpathmoveto{\pgfqpoint{1.810622in}{1.847071in}}%
\pgfpathlineto{\pgfqpoint{1.819559in}{1.847071in}}%
\pgfpathlineto{\pgfqpoint{1.819559in}{2.458478in}}%
\pgfpathlineto{\pgfqpoint{1.810622in}{2.458478in}}%
\pgfpathlineto{\pgfqpoint{1.810622in}{1.847071in}}%
\pgfpathclose%
\pgfusepath{fill}%
\end{pgfscope}%
\begin{pgfscope}%
\pgfpathrectangle{\pgfqpoint{0.697024in}{0.857143in}}{\pgfqpoint{2.627103in}{1.813434in}}%
\pgfusepath{clip}%
\pgfsetbuttcap%
\pgfsetmiterjoin%
\definecolor{currentfill}{rgb}{0.302379,0.450282,0.300122}%
\pgfsetfillcolor{currentfill}%
\pgfsetlinewidth{0.000000pt}%
\definecolor{currentstroke}{rgb}{0.000000,0.000000,0.000000}%
\pgfsetstrokecolor{currentstroke}%
\pgfsetstrokeopacity{0.000000}%
\pgfsetdash{}{0pt}%
\pgfpathmoveto{\pgfqpoint{1.821793in}{1.847071in}}%
\pgfpathlineto{\pgfqpoint{1.830729in}{1.847071in}}%
\pgfpathlineto{\pgfqpoint{1.830729in}{2.534609in}}%
\pgfpathlineto{\pgfqpoint{1.821793in}{2.534609in}}%
\pgfpathlineto{\pgfqpoint{1.821793in}{1.847071in}}%
\pgfpathclose%
\pgfusepath{fill}%
\end{pgfscope}%
\begin{pgfscope}%
\pgfpathrectangle{\pgfqpoint{0.697024in}{0.857143in}}{\pgfqpoint{2.627103in}{1.813434in}}%
\pgfusepath{clip}%
\pgfsetbuttcap%
\pgfsetmiterjoin%
\definecolor{currentfill}{rgb}{0.302379,0.450282,0.300122}%
\pgfsetfillcolor{currentfill}%
\pgfsetlinewidth{0.000000pt}%
\definecolor{currentstroke}{rgb}{0.000000,0.000000,0.000000}%
\pgfsetstrokecolor{currentstroke}%
\pgfsetstrokeopacity{0.000000}%
\pgfsetdash{}{0pt}%
\pgfpathmoveto{\pgfqpoint{1.832963in}{1.847071in}}%
\pgfpathlineto{\pgfqpoint{1.841900in}{1.847071in}}%
\pgfpathlineto{\pgfqpoint{1.841900in}{2.550636in}}%
\pgfpathlineto{\pgfqpoint{1.832963in}{2.550636in}}%
\pgfpathlineto{\pgfqpoint{1.832963in}{1.847071in}}%
\pgfpathclose%
\pgfusepath{fill}%
\end{pgfscope}%
\begin{pgfscope}%
\pgfpathrectangle{\pgfqpoint{0.697024in}{0.857143in}}{\pgfqpoint{2.627103in}{1.813434in}}%
\pgfusepath{clip}%
\pgfsetbuttcap%
\pgfsetmiterjoin%
\definecolor{currentfill}{rgb}{0.302379,0.450282,0.300122}%
\pgfsetfillcolor{currentfill}%
\pgfsetlinewidth{0.000000pt}%
\definecolor{currentstroke}{rgb}{0.000000,0.000000,0.000000}%
\pgfsetstrokecolor{currentstroke}%
\pgfsetstrokeopacity{0.000000}%
\pgfsetdash{}{0pt}%
\pgfpathmoveto{\pgfqpoint{1.844134in}{1.847071in}}%
\pgfpathlineto{\pgfqpoint{1.853070in}{1.847071in}}%
\pgfpathlineto{\pgfqpoint{1.853070in}{2.529543in}}%
\pgfpathlineto{\pgfqpoint{1.844134in}{2.529543in}}%
\pgfpathlineto{\pgfqpoint{1.844134in}{1.847071in}}%
\pgfpathclose%
\pgfusepath{fill}%
\end{pgfscope}%
\begin{pgfscope}%
\pgfpathrectangle{\pgfqpoint{0.697024in}{0.857143in}}{\pgfqpoint{2.627103in}{1.813434in}}%
\pgfusepath{clip}%
\pgfsetbuttcap%
\pgfsetmiterjoin%
\definecolor{currentfill}{rgb}{0.302379,0.450282,0.300122}%
\pgfsetfillcolor{currentfill}%
\pgfsetlinewidth{0.000000pt}%
\definecolor{currentstroke}{rgb}{0.000000,0.000000,0.000000}%
\pgfsetstrokecolor{currentstroke}%
\pgfsetstrokeopacity{0.000000}%
\pgfsetdash{}{0pt}%
\pgfpathmoveto{\pgfqpoint{1.855304in}{1.847071in}}%
\pgfpathlineto{\pgfqpoint{1.864241in}{1.847071in}}%
\pgfpathlineto{\pgfqpoint{1.864241in}{2.494926in}}%
\pgfpathlineto{\pgfqpoint{1.855304in}{2.494926in}}%
\pgfpathlineto{\pgfqpoint{1.855304in}{1.847071in}}%
\pgfpathclose%
\pgfusepath{fill}%
\end{pgfscope}%
\begin{pgfscope}%
\pgfpathrectangle{\pgfqpoint{0.697024in}{0.857143in}}{\pgfqpoint{2.627103in}{1.813434in}}%
\pgfusepath{clip}%
\pgfsetbuttcap%
\pgfsetmiterjoin%
\definecolor{currentfill}{rgb}{0.302379,0.450282,0.300122}%
\pgfsetfillcolor{currentfill}%
\pgfsetlinewidth{0.000000pt}%
\definecolor{currentstroke}{rgb}{0.000000,0.000000,0.000000}%
\pgfsetstrokecolor{currentstroke}%
\pgfsetstrokeopacity{0.000000}%
\pgfsetdash{}{0pt}%
\pgfpathmoveto{\pgfqpoint{1.866475in}{1.847071in}}%
\pgfpathlineto{\pgfqpoint{1.875412in}{1.847071in}}%
\pgfpathlineto{\pgfqpoint{1.875412in}{2.550434in}}%
\pgfpathlineto{\pgfqpoint{1.866475in}{2.550434in}}%
\pgfpathlineto{\pgfqpoint{1.866475in}{1.847071in}}%
\pgfpathclose%
\pgfusepath{fill}%
\end{pgfscope}%
\begin{pgfscope}%
\pgfpathrectangle{\pgfqpoint{0.697024in}{0.857143in}}{\pgfqpoint{2.627103in}{1.813434in}}%
\pgfusepath{clip}%
\pgfsetbuttcap%
\pgfsetmiterjoin%
\definecolor{currentfill}{rgb}{0.302379,0.450282,0.300122}%
\pgfsetfillcolor{currentfill}%
\pgfsetlinewidth{0.000000pt}%
\definecolor{currentstroke}{rgb}{0.000000,0.000000,0.000000}%
\pgfsetstrokecolor{currentstroke}%
\pgfsetstrokeopacity{0.000000}%
\pgfsetdash{}{0pt}%
\pgfpathmoveto{\pgfqpoint{1.877646in}{1.847071in}}%
\pgfpathlineto{\pgfqpoint{1.886582in}{1.847071in}}%
\pgfpathlineto{\pgfqpoint{1.886582in}{2.525627in}}%
\pgfpathlineto{\pgfqpoint{1.877646in}{2.525627in}}%
\pgfpathlineto{\pgfqpoint{1.877646in}{1.847071in}}%
\pgfpathclose%
\pgfusepath{fill}%
\end{pgfscope}%
\begin{pgfscope}%
\pgfpathrectangle{\pgfqpoint{0.697024in}{0.857143in}}{\pgfqpoint{2.627103in}{1.813434in}}%
\pgfusepath{clip}%
\pgfsetbuttcap%
\pgfsetmiterjoin%
\definecolor{currentfill}{rgb}{0.302379,0.450282,0.300122}%
\pgfsetfillcolor{currentfill}%
\pgfsetlinewidth{0.000000pt}%
\definecolor{currentstroke}{rgb}{0.000000,0.000000,0.000000}%
\pgfsetstrokecolor{currentstroke}%
\pgfsetstrokeopacity{0.000000}%
\pgfsetdash{}{0pt}%
\pgfpathmoveto{\pgfqpoint{1.888816in}{1.847071in}}%
\pgfpathlineto{\pgfqpoint{1.897753in}{1.847071in}}%
\pgfpathlineto{\pgfqpoint{1.897753in}{2.485916in}}%
\pgfpathlineto{\pgfqpoint{1.888816in}{2.485916in}}%
\pgfpathlineto{\pgfqpoint{1.888816in}{1.847071in}}%
\pgfpathclose%
\pgfusepath{fill}%
\end{pgfscope}%
\begin{pgfscope}%
\pgfpathrectangle{\pgfqpoint{0.697024in}{0.857143in}}{\pgfqpoint{2.627103in}{1.813434in}}%
\pgfusepath{clip}%
\pgfsetbuttcap%
\pgfsetmiterjoin%
\definecolor{currentfill}{rgb}{0.302379,0.450282,0.300122}%
\pgfsetfillcolor{currentfill}%
\pgfsetlinewidth{0.000000pt}%
\definecolor{currentstroke}{rgb}{0.000000,0.000000,0.000000}%
\pgfsetstrokecolor{currentstroke}%
\pgfsetstrokeopacity{0.000000}%
\pgfsetdash{}{0pt}%
\pgfpathmoveto{\pgfqpoint{1.899987in}{1.847071in}}%
\pgfpathlineto{\pgfqpoint{1.908923in}{1.847071in}}%
\pgfpathlineto{\pgfqpoint{1.908923in}{2.466340in}}%
\pgfpathlineto{\pgfqpoint{1.899987in}{2.466340in}}%
\pgfpathlineto{\pgfqpoint{1.899987in}{1.847071in}}%
\pgfpathclose%
\pgfusepath{fill}%
\end{pgfscope}%
\begin{pgfscope}%
\pgfpathrectangle{\pgfqpoint{0.697024in}{0.857143in}}{\pgfqpoint{2.627103in}{1.813434in}}%
\pgfusepath{clip}%
\pgfsetbuttcap%
\pgfsetmiterjoin%
\definecolor{currentfill}{rgb}{0.302379,0.450282,0.300122}%
\pgfsetfillcolor{currentfill}%
\pgfsetlinewidth{0.000000pt}%
\definecolor{currentstroke}{rgb}{0.000000,0.000000,0.000000}%
\pgfsetstrokecolor{currentstroke}%
\pgfsetstrokeopacity{0.000000}%
\pgfsetdash{}{0pt}%
\pgfpathmoveto{\pgfqpoint{1.911157in}{1.847071in}}%
\pgfpathlineto{\pgfqpoint{1.920094in}{1.847071in}}%
\pgfpathlineto{\pgfqpoint{1.920094in}{2.382578in}}%
\pgfpathlineto{\pgfqpoint{1.911157in}{2.382578in}}%
\pgfpathlineto{\pgfqpoint{1.911157in}{1.847071in}}%
\pgfpathclose%
\pgfusepath{fill}%
\end{pgfscope}%
\begin{pgfscope}%
\pgfpathrectangle{\pgfqpoint{0.697024in}{0.857143in}}{\pgfqpoint{2.627103in}{1.813434in}}%
\pgfusepath{clip}%
\pgfsetbuttcap%
\pgfsetmiterjoin%
\definecolor{currentfill}{rgb}{0.302379,0.450282,0.300122}%
\pgfsetfillcolor{currentfill}%
\pgfsetlinewidth{0.000000pt}%
\definecolor{currentstroke}{rgb}{0.000000,0.000000,0.000000}%
\pgfsetstrokecolor{currentstroke}%
\pgfsetstrokeopacity{0.000000}%
\pgfsetdash{}{0pt}%
\pgfpathmoveto{\pgfqpoint{1.922328in}{1.847071in}}%
\pgfpathlineto{\pgfqpoint{1.931265in}{1.847071in}}%
\pgfpathlineto{\pgfqpoint{1.931265in}{2.337795in}}%
\pgfpathlineto{\pgfqpoint{1.922328in}{2.337795in}}%
\pgfpathlineto{\pgfqpoint{1.922328in}{1.847071in}}%
\pgfpathclose%
\pgfusepath{fill}%
\end{pgfscope}%
\begin{pgfscope}%
\pgfpathrectangle{\pgfqpoint{0.697024in}{0.857143in}}{\pgfqpoint{2.627103in}{1.813434in}}%
\pgfusepath{clip}%
\pgfsetbuttcap%
\pgfsetmiterjoin%
\definecolor{currentfill}{rgb}{0.302379,0.450282,0.300122}%
\pgfsetfillcolor{currentfill}%
\pgfsetlinewidth{0.000000pt}%
\definecolor{currentstroke}{rgb}{0.000000,0.000000,0.000000}%
\pgfsetstrokecolor{currentstroke}%
\pgfsetstrokeopacity{0.000000}%
\pgfsetdash{}{0pt}%
\pgfpathmoveto{\pgfqpoint{1.933499in}{1.847071in}}%
\pgfpathlineto{\pgfqpoint{1.942435in}{1.847071in}}%
\pgfpathlineto{\pgfqpoint{1.942435in}{2.332237in}}%
\pgfpathlineto{\pgfqpoint{1.933499in}{2.332237in}}%
\pgfpathlineto{\pgfqpoint{1.933499in}{1.847071in}}%
\pgfpathclose%
\pgfusepath{fill}%
\end{pgfscope}%
\begin{pgfscope}%
\pgfpathrectangle{\pgfqpoint{0.697024in}{0.857143in}}{\pgfqpoint{2.627103in}{1.813434in}}%
\pgfusepath{clip}%
\pgfsetbuttcap%
\pgfsetmiterjoin%
\definecolor{currentfill}{rgb}{0.302379,0.450282,0.300122}%
\pgfsetfillcolor{currentfill}%
\pgfsetlinewidth{0.000000pt}%
\definecolor{currentstroke}{rgb}{0.000000,0.000000,0.000000}%
\pgfsetstrokecolor{currentstroke}%
\pgfsetstrokeopacity{0.000000}%
\pgfsetdash{}{0pt}%
\pgfpathmoveto{\pgfqpoint{1.944669in}{1.847071in}}%
\pgfpathlineto{\pgfqpoint{1.953606in}{1.847071in}}%
\pgfpathlineto{\pgfqpoint{1.953606in}{2.285655in}}%
\pgfpathlineto{\pgfqpoint{1.944669in}{2.285655in}}%
\pgfpathlineto{\pgfqpoint{1.944669in}{1.847071in}}%
\pgfpathclose%
\pgfusepath{fill}%
\end{pgfscope}%
\begin{pgfscope}%
\pgfpathrectangle{\pgfqpoint{0.697024in}{0.857143in}}{\pgfqpoint{2.627103in}{1.813434in}}%
\pgfusepath{clip}%
\pgfsetbuttcap%
\pgfsetmiterjoin%
\definecolor{currentfill}{rgb}{0.302379,0.450282,0.300122}%
\pgfsetfillcolor{currentfill}%
\pgfsetlinewidth{0.000000pt}%
\definecolor{currentstroke}{rgb}{0.000000,0.000000,0.000000}%
\pgfsetstrokecolor{currentstroke}%
\pgfsetstrokeopacity{0.000000}%
\pgfsetdash{}{0pt}%
\pgfpathmoveto{\pgfqpoint{1.955840in}{1.847071in}}%
\pgfpathlineto{\pgfqpoint{1.964776in}{1.847071in}}%
\pgfpathlineto{\pgfqpoint{1.964776in}{2.305523in}}%
\pgfpathlineto{\pgfqpoint{1.955840in}{2.305523in}}%
\pgfpathlineto{\pgfqpoint{1.955840in}{1.847071in}}%
\pgfpathclose%
\pgfusepath{fill}%
\end{pgfscope}%
\begin{pgfscope}%
\pgfpathrectangle{\pgfqpoint{0.697024in}{0.857143in}}{\pgfqpoint{2.627103in}{1.813434in}}%
\pgfusepath{clip}%
\pgfsetbuttcap%
\pgfsetmiterjoin%
\definecolor{currentfill}{rgb}{0.302379,0.450282,0.300122}%
\pgfsetfillcolor{currentfill}%
\pgfsetlinewidth{0.000000pt}%
\definecolor{currentstroke}{rgb}{0.000000,0.000000,0.000000}%
\pgfsetstrokecolor{currentstroke}%
\pgfsetstrokeopacity{0.000000}%
\pgfsetdash{}{0pt}%
\pgfpathmoveto{\pgfqpoint{1.967011in}{1.847071in}}%
\pgfpathlineto{\pgfqpoint{1.975947in}{1.847071in}}%
\pgfpathlineto{\pgfqpoint{1.975947in}{2.296482in}}%
\pgfpathlineto{\pgfqpoint{1.967011in}{2.296482in}}%
\pgfpathlineto{\pgfqpoint{1.967011in}{1.847071in}}%
\pgfpathclose%
\pgfusepath{fill}%
\end{pgfscope}%
\begin{pgfscope}%
\pgfpathrectangle{\pgfqpoint{0.697024in}{0.857143in}}{\pgfqpoint{2.627103in}{1.813434in}}%
\pgfusepath{clip}%
\pgfsetbuttcap%
\pgfsetmiterjoin%
\definecolor{currentfill}{rgb}{0.302379,0.450282,0.300122}%
\pgfsetfillcolor{currentfill}%
\pgfsetlinewidth{0.000000pt}%
\definecolor{currentstroke}{rgb}{0.000000,0.000000,0.000000}%
\pgfsetstrokecolor{currentstroke}%
\pgfsetstrokeopacity{0.000000}%
\pgfsetdash{}{0pt}%
\pgfpathmoveto{\pgfqpoint{1.978181in}{1.847071in}}%
\pgfpathlineto{\pgfqpoint{1.987118in}{1.847071in}}%
\pgfpathlineto{\pgfqpoint{1.987118in}{2.308989in}}%
\pgfpathlineto{\pgfqpoint{1.978181in}{2.308989in}}%
\pgfpathlineto{\pgfqpoint{1.978181in}{1.847071in}}%
\pgfpathclose%
\pgfusepath{fill}%
\end{pgfscope}%
\begin{pgfscope}%
\pgfpathrectangle{\pgfqpoint{0.697024in}{0.857143in}}{\pgfqpoint{2.627103in}{1.813434in}}%
\pgfusepath{clip}%
\pgfsetbuttcap%
\pgfsetmiterjoin%
\definecolor{currentfill}{rgb}{0.302379,0.450282,0.300122}%
\pgfsetfillcolor{currentfill}%
\pgfsetlinewidth{0.000000pt}%
\definecolor{currentstroke}{rgb}{0.000000,0.000000,0.000000}%
\pgfsetstrokecolor{currentstroke}%
\pgfsetstrokeopacity{0.000000}%
\pgfsetdash{}{0pt}%
\pgfpathmoveto{\pgfqpoint{1.989352in}{1.847071in}}%
\pgfpathlineto{\pgfqpoint{1.998288in}{1.847071in}}%
\pgfpathlineto{\pgfqpoint{1.998288in}{2.333588in}}%
\pgfpathlineto{\pgfqpoint{1.989352in}{2.333588in}}%
\pgfpathlineto{\pgfqpoint{1.989352in}{1.847071in}}%
\pgfpathclose%
\pgfusepath{fill}%
\end{pgfscope}%
\begin{pgfscope}%
\pgfpathrectangle{\pgfqpoint{0.697024in}{0.857143in}}{\pgfqpoint{2.627103in}{1.813434in}}%
\pgfusepath{clip}%
\pgfsetbuttcap%
\pgfsetmiterjoin%
\definecolor{currentfill}{rgb}{0.302379,0.450282,0.300122}%
\pgfsetfillcolor{currentfill}%
\pgfsetlinewidth{0.000000pt}%
\definecolor{currentstroke}{rgb}{0.000000,0.000000,0.000000}%
\pgfsetstrokecolor{currentstroke}%
\pgfsetstrokeopacity{0.000000}%
\pgfsetdash{}{0pt}%
\pgfpathmoveto{\pgfqpoint{2.000522in}{1.847071in}}%
\pgfpathlineto{\pgfqpoint{2.009459in}{1.847071in}}%
\pgfpathlineto{\pgfqpoint{2.009459in}{2.322637in}}%
\pgfpathlineto{\pgfqpoint{2.000522in}{2.322637in}}%
\pgfpathlineto{\pgfqpoint{2.000522in}{1.847071in}}%
\pgfpathclose%
\pgfusepath{fill}%
\end{pgfscope}%
\begin{pgfscope}%
\pgfpathrectangle{\pgfqpoint{0.697024in}{0.857143in}}{\pgfqpoint{2.627103in}{1.813434in}}%
\pgfusepath{clip}%
\pgfsetbuttcap%
\pgfsetmiterjoin%
\definecolor{currentfill}{rgb}{0.302379,0.450282,0.300122}%
\pgfsetfillcolor{currentfill}%
\pgfsetlinewidth{0.000000pt}%
\definecolor{currentstroke}{rgb}{0.000000,0.000000,0.000000}%
\pgfsetstrokecolor{currentstroke}%
\pgfsetstrokeopacity{0.000000}%
\pgfsetdash{}{0pt}%
\pgfpathmoveto{\pgfqpoint{2.011693in}{1.847071in}}%
\pgfpathlineto{\pgfqpoint{2.020629in}{1.847071in}}%
\pgfpathlineto{\pgfqpoint{2.020629in}{2.336750in}}%
\pgfpathlineto{\pgfqpoint{2.011693in}{2.336750in}}%
\pgfpathlineto{\pgfqpoint{2.011693in}{1.847071in}}%
\pgfpathclose%
\pgfusepath{fill}%
\end{pgfscope}%
\begin{pgfscope}%
\pgfpathrectangle{\pgfqpoint{0.697024in}{0.857143in}}{\pgfqpoint{2.627103in}{1.813434in}}%
\pgfusepath{clip}%
\pgfsetbuttcap%
\pgfsetmiterjoin%
\definecolor{currentfill}{rgb}{0.302379,0.450282,0.300122}%
\pgfsetfillcolor{currentfill}%
\pgfsetlinewidth{0.000000pt}%
\definecolor{currentstroke}{rgb}{0.000000,0.000000,0.000000}%
\pgfsetstrokecolor{currentstroke}%
\pgfsetstrokeopacity{0.000000}%
\pgfsetdash{}{0pt}%
\pgfpathmoveto{\pgfqpoint{2.022864in}{1.847071in}}%
\pgfpathlineto{\pgfqpoint{2.031800in}{1.847071in}}%
\pgfpathlineto{\pgfqpoint{2.031800in}{2.367532in}}%
\pgfpathlineto{\pgfqpoint{2.022864in}{2.367532in}}%
\pgfpathlineto{\pgfqpoint{2.022864in}{1.847071in}}%
\pgfpathclose%
\pgfusepath{fill}%
\end{pgfscope}%
\begin{pgfscope}%
\pgfpathrectangle{\pgfqpoint{0.697024in}{0.857143in}}{\pgfqpoint{2.627103in}{1.813434in}}%
\pgfusepath{clip}%
\pgfsetbuttcap%
\pgfsetmiterjoin%
\definecolor{currentfill}{rgb}{0.302379,0.450282,0.300122}%
\pgfsetfillcolor{currentfill}%
\pgfsetlinewidth{0.000000pt}%
\definecolor{currentstroke}{rgb}{0.000000,0.000000,0.000000}%
\pgfsetstrokecolor{currentstroke}%
\pgfsetstrokeopacity{0.000000}%
\pgfsetdash{}{0pt}%
\pgfpathmoveto{\pgfqpoint{2.034034in}{1.847071in}}%
\pgfpathlineto{\pgfqpoint{2.042971in}{1.847071in}}%
\pgfpathlineto{\pgfqpoint{2.042971in}{2.348993in}}%
\pgfpathlineto{\pgfqpoint{2.034034in}{2.348993in}}%
\pgfpathlineto{\pgfqpoint{2.034034in}{1.847071in}}%
\pgfpathclose%
\pgfusepath{fill}%
\end{pgfscope}%
\begin{pgfscope}%
\pgfpathrectangle{\pgfqpoint{0.697024in}{0.857143in}}{\pgfqpoint{2.627103in}{1.813434in}}%
\pgfusepath{clip}%
\pgfsetbuttcap%
\pgfsetmiterjoin%
\definecolor{currentfill}{rgb}{0.302379,0.450282,0.300122}%
\pgfsetfillcolor{currentfill}%
\pgfsetlinewidth{0.000000pt}%
\definecolor{currentstroke}{rgb}{0.000000,0.000000,0.000000}%
\pgfsetstrokecolor{currentstroke}%
\pgfsetstrokeopacity{0.000000}%
\pgfsetdash{}{0pt}%
\pgfpathmoveto{\pgfqpoint{2.045205in}{1.847071in}}%
\pgfpathlineto{\pgfqpoint{2.054141in}{1.847071in}}%
\pgfpathlineto{\pgfqpoint{2.054141in}{2.346516in}}%
\pgfpathlineto{\pgfqpoint{2.045205in}{2.346516in}}%
\pgfpathlineto{\pgfqpoint{2.045205in}{1.847071in}}%
\pgfpathclose%
\pgfusepath{fill}%
\end{pgfscope}%
\begin{pgfscope}%
\pgfpathrectangle{\pgfqpoint{0.697024in}{0.857143in}}{\pgfqpoint{2.627103in}{1.813434in}}%
\pgfusepath{clip}%
\pgfsetbuttcap%
\pgfsetmiterjoin%
\definecolor{currentfill}{rgb}{0.302379,0.450282,0.300122}%
\pgfsetfillcolor{currentfill}%
\pgfsetlinewidth{0.000000pt}%
\definecolor{currentstroke}{rgb}{0.000000,0.000000,0.000000}%
\pgfsetstrokecolor{currentstroke}%
\pgfsetstrokeopacity{0.000000}%
\pgfsetdash{}{0pt}%
\pgfpathmoveto{\pgfqpoint{2.056375in}{1.847071in}}%
\pgfpathlineto{\pgfqpoint{2.065312in}{1.847071in}}%
\pgfpathlineto{\pgfqpoint{2.065312in}{2.403056in}}%
\pgfpathlineto{\pgfqpoint{2.056375in}{2.403056in}}%
\pgfpathlineto{\pgfqpoint{2.056375in}{1.847071in}}%
\pgfpathclose%
\pgfusepath{fill}%
\end{pgfscope}%
\begin{pgfscope}%
\pgfpathrectangle{\pgfqpoint{0.697024in}{0.857143in}}{\pgfqpoint{2.627103in}{1.813434in}}%
\pgfusepath{clip}%
\pgfsetbuttcap%
\pgfsetmiterjoin%
\definecolor{currentfill}{rgb}{0.302379,0.450282,0.300122}%
\pgfsetfillcolor{currentfill}%
\pgfsetlinewidth{0.000000pt}%
\definecolor{currentstroke}{rgb}{0.000000,0.000000,0.000000}%
\pgfsetstrokecolor{currentstroke}%
\pgfsetstrokeopacity{0.000000}%
\pgfsetdash{}{0pt}%
\pgfpathmoveto{\pgfqpoint{2.067546in}{1.847071in}}%
\pgfpathlineto{\pgfqpoint{2.076482in}{1.847071in}}%
\pgfpathlineto{\pgfqpoint{2.076482in}{2.415344in}}%
\pgfpathlineto{\pgfqpoint{2.067546in}{2.415344in}}%
\pgfpathlineto{\pgfqpoint{2.067546in}{1.847071in}}%
\pgfpathclose%
\pgfusepath{fill}%
\end{pgfscope}%
\begin{pgfscope}%
\pgfpathrectangle{\pgfqpoint{0.697024in}{0.857143in}}{\pgfqpoint{2.627103in}{1.813434in}}%
\pgfusepath{clip}%
\pgfsetbuttcap%
\pgfsetmiterjoin%
\definecolor{currentfill}{rgb}{0.302379,0.450282,0.300122}%
\pgfsetfillcolor{currentfill}%
\pgfsetlinewidth{0.000000pt}%
\definecolor{currentstroke}{rgb}{0.000000,0.000000,0.000000}%
\pgfsetstrokecolor{currentstroke}%
\pgfsetstrokeopacity{0.000000}%
\pgfsetdash{}{0pt}%
\pgfpathmoveto{\pgfqpoint{2.078717in}{1.847071in}}%
\pgfpathlineto{\pgfqpoint{2.087653in}{1.847071in}}%
\pgfpathlineto{\pgfqpoint{2.087653in}{2.433772in}}%
\pgfpathlineto{\pgfqpoint{2.078717in}{2.433772in}}%
\pgfpathlineto{\pgfqpoint{2.078717in}{1.847071in}}%
\pgfpathclose%
\pgfusepath{fill}%
\end{pgfscope}%
\begin{pgfscope}%
\pgfpathrectangle{\pgfqpoint{0.697024in}{0.857143in}}{\pgfqpoint{2.627103in}{1.813434in}}%
\pgfusepath{clip}%
\pgfsetbuttcap%
\pgfsetmiterjoin%
\definecolor{currentfill}{rgb}{0.302379,0.450282,0.300122}%
\pgfsetfillcolor{currentfill}%
\pgfsetlinewidth{0.000000pt}%
\definecolor{currentstroke}{rgb}{0.000000,0.000000,0.000000}%
\pgfsetstrokecolor{currentstroke}%
\pgfsetstrokeopacity{0.000000}%
\pgfsetdash{}{0pt}%
\pgfpathmoveto{\pgfqpoint{2.089887in}{1.847071in}}%
\pgfpathlineto{\pgfqpoint{2.098824in}{1.847071in}}%
\pgfpathlineto{\pgfqpoint{2.098824in}{2.449224in}}%
\pgfpathlineto{\pgfqpoint{2.089887in}{2.449224in}}%
\pgfpathlineto{\pgfqpoint{2.089887in}{1.847071in}}%
\pgfpathclose%
\pgfusepath{fill}%
\end{pgfscope}%
\begin{pgfscope}%
\pgfpathrectangle{\pgfqpoint{0.697024in}{0.857143in}}{\pgfqpoint{2.627103in}{1.813434in}}%
\pgfusepath{clip}%
\pgfsetbuttcap%
\pgfsetmiterjoin%
\definecolor{currentfill}{rgb}{0.302379,0.450282,0.300122}%
\pgfsetfillcolor{currentfill}%
\pgfsetlinewidth{0.000000pt}%
\definecolor{currentstroke}{rgb}{0.000000,0.000000,0.000000}%
\pgfsetstrokecolor{currentstroke}%
\pgfsetstrokeopacity{0.000000}%
\pgfsetdash{}{0pt}%
\pgfpathmoveto{\pgfqpoint{2.101058in}{1.847071in}}%
\pgfpathlineto{\pgfqpoint{2.109994in}{1.847071in}}%
\pgfpathlineto{\pgfqpoint{2.109994in}{2.413895in}}%
\pgfpathlineto{\pgfqpoint{2.101058in}{2.413895in}}%
\pgfpathlineto{\pgfqpoint{2.101058in}{1.847071in}}%
\pgfpathclose%
\pgfusepath{fill}%
\end{pgfscope}%
\begin{pgfscope}%
\pgfpathrectangle{\pgfqpoint{0.697024in}{0.857143in}}{\pgfqpoint{2.627103in}{1.813434in}}%
\pgfusepath{clip}%
\pgfsetbuttcap%
\pgfsetmiterjoin%
\definecolor{currentfill}{rgb}{0.302379,0.450282,0.300122}%
\pgfsetfillcolor{currentfill}%
\pgfsetlinewidth{0.000000pt}%
\definecolor{currentstroke}{rgb}{0.000000,0.000000,0.000000}%
\pgfsetstrokecolor{currentstroke}%
\pgfsetstrokeopacity{0.000000}%
\pgfsetdash{}{0pt}%
\pgfpathmoveto{\pgfqpoint{2.112228in}{1.847071in}}%
\pgfpathlineto{\pgfqpoint{2.121165in}{1.847071in}}%
\pgfpathlineto{\pgfqpoint{2.121165in}{2.407727in}}%
\pgfpathlineto{\pgfqpoint{2.112228in}{2.407727in}}%
\pgfpathlineto{\pgfqpoint{2.112228in}{1.847071in}}%
\pgfpathclose%
\pgfusepath{fill}%
\end{pgfscope}%
\begin{pgfscope}%
\pgfpathrectangle{\pgfqpoint{0.697024in}{0.857143in}}{\pgfqpoint{2.627103in}{1.813434in}}%
\pgfusepath{clip}%
\pgfsetbuttcap%
\pgfsetmiterjoin%
\definecolor{currentfill}{rgb}{0.302379,0.450282,0.300122}%
\pgfsetfillcolor{currentfill}%
\pgfsetlinewidth{0.000000pt}%
\definecolor{currentstroke}{rgb}{0.000000,0.000000,0.000000}%
\pgfsetstrokecolor{currentstroke}%
\pgfsetstrokeopacity{0.000000}%
\pgfsetdash{}{0pt}%
\pgfpathmoveto{\pgfqpoint{2.123399in}{1.847071in}}%
\pgfpathlineto{\pgfqpoint{2.132335in}{1.847071in}}%
\pgfpathlineto{\pgfqpoint{2.132335in}{2.401973in}}%
\pgfpathlineto{\pgfqpoint{2.123399in}{2.401973in}}%
\pgfpathlineto{\pgfqpoint{2.123399in}{1.847071in}}%
\pgfpathclose%
\pgfusepath{fill}%
\end{pgfscope}%
\begin{pgfscope}%
\pgfpathrectangle{\pgfqpoint{0.697024in}{0.857143in}}{\pgfqpoint{2.627103in}{1.813434in}}%
\pgfusepath{clip}%
\pgfsetbuttcap%
\pgfsetmiterjoin%
\definecolor{currentfill}{rgb}{0.302379,0.450282,0.300122}%
\pgfsetfillcolor{currentfill}%
\pgfsetlinewidth{0.000000pt}%
\definecolor{currentstroke}{rgb}{0.000000,0.000000,0.000000}%
\pgfsetstrokecolor{currentstroke}%
\pgfsetstrokeopacity{0.000000}%
\pgfsetdash{}{0pt}%
\pgfpathmoveto{\pgfqpoint{2.134570in}{1.847071in}}%
\pgfpathlineto{\pgfqpoint{2.143506in}{1.847071in}}%
\pgfpathlineto{\pgfqpoint{2.143506in}{2.360711in}}%
\pgfpathlineto{\pgfqpoint{2.134570in}{2.360711in}}%
\pgfpathlineto{\pgfqpoint{2.134570in}{1.847071in}}%
\pgfpathclose%
\pgfusepath{fill}%
\end{pgfscope}%
\begin{pgfscope}%
\pgfpathrectangle{\pgfqpoint{0.697024in}{0.857143in}}{\pgfqpoint{2.627103in}{1.813434in}}%
\pgfusepath{clip}%
\pgfsetbuttcap%
\pgfsetmiterjoin%
\definecolor{currentfill}{rgb}{0.302379,0.450282,0.300122}%
\pgfsetfillcolor{currentfill}%
\pgfsetlinewidth{0.000000pt}%
\definecolor{currentstroke}{rgb}{0.000000,0.000000,0.000000}%
\pgfsetstrokecolor{currentstroke}%
\pgfsetstrokeopacity{0.000000}%
\pgfsetdash{}{0pt}%
\pgfpathmoveto{\pgfqpoint{2.145740in}{1.847071in}}%
\pgfpathlineto{\pgfqpoint{2.154677in}{1.847071in}}%
\pgfpathlineto{\pgfqpoint{2.154677in}{2.414896in}}%
\pgfpathlineto{\pgfqpoint{2.145740in}{2.414896in}}%
\pgfpathlineto{\pgfqpoint{2.145740in}{1.847071in}}%
\pgfpathclose%
\pgfusepath{fill}%
\end{pgfscope}%
\begin{pgfscope}%
\pgfpathrectangle{\pgfqpoint{0.697024in}{0.857143in}}{\pgfqpoint{2.627103in}{1.813434in}}%
\pgfusepath{clip}%
\pgfsetbuttcap%
\pgfsetmiterjoin%
\definecolor{currentfill}{rgb}{0.302379,0.450282,0.300122}%
\pgfsetfillcolor{currentfill}%
\pgfsetlinewidth{0.000000pt}%
\definecolor{currentstroke}{rgb}{0.000000,0.000000,0.000000}%
\pgfsetstrokecolor{currentstroke}%
\pgfsetstrokeopacity{0.000000}%
\pgfsetdash{}{0pt}%
\pgfpathmoveto{\pgfqpoint{2.156911in}{1.847071in}}%
\pgfpathlineto{\pgfqpoint{2.165847in}{1.847071in}}%
\pgfpathlineto{\pgfqpoint{2.165847in}{2.374657in}}%
\pgfpathlineto{\pgfqpoint{2.156911in}{2.374657in}}%
\pgfpathlineto{\pgfqpoint{2.156911in}{1.847071in}}%
\pgfpathclose%
\pgfusepath{fill}%
\end{pgfscope}%
\begin{pgfscope}%
\pgfpathrectangle{\pgfqpoint{0.697024in}{0.857143in}}{\pgfqpoint{2.627103in}{1.813434in}}%
\pgfusepath{clip}%
\pgfsetbuttcap%
\pgfsetmiterjoin%
\definecolor{currentfill}{rgb}{0.302379,0.450282,0.300122}%
\pgfsetfillcolor{currentfill}%
\pgfsetlinewidth{0.000000pt}%
\definecolor{currentstroke}{rgb}{0.000000,0.000000,0.000000}%
\pgfsetstrokecolor{currentstroke}%
\pgfsetstrokeopacity{0.000000}%
\pgfsetdash{}{0pt}%
\pgfpathmoveto{\pgfqpoint{2.168081in}{1.847071in}}%
\pgfpathlineto{\pgfqpoint{2.177018in}{1.847071in}}%
\pgfpathlineto{\pgfqpoint{2.177018in}{2.385756in}}%
\pgfpathlineto{\pgfqpoint{2.168081in}{2.385756in}}%
\pgfpathlineto{\pgfqpoint{2.168081in}{1.847071in}}%
\pgfpathclose%
\pgfusepath{fill}%
\end{pgfscope}%
\begin{pgfscope}%
\pgfpathrectangle{\pgfqpoint{0.697024in}{0.857143in}}{\pgfqpoint{2.627103in}{1.813434in}}%
\pgfusepath{clip}%
\pgfsetbuttcap%
\pgfsetmiterjoin%
\definecolor{currentfill}{rgb}{0.302379,0.450282,0.300122}%
\pgfsetfillcolor{currentfill}%
\pgfsetlinewidth{0.000000pt}%
\definecolor{currentstroke}{rgb}{0.000000,0.000000,0.000000}%
\pgfsetstrokecolor{currentstroke}%
\pgfsetstrokeopacity{0.000000}%
\pgfsetdash{}{0pt}%
\pgfpathmoveto{\pgfqpoint{2.179252in}{1.847071in}}%
\pgfpathlineto{\pgfqpoint{2.188189in}{1.847071in}}%
\pgfpathlineto{\pgfqpoint{2.188189in}{2.360398in}}%
\pgfpathlineto{\pgfqpoint{2.179252in}{2.360398in}}%
\pgfpathlineto{\pgfqpoint{2.179252in}{1.847071in}}%
\pgfpathclose%
\pgfusepath{fill}%
\end{pgfscope}%
\begin{pgfscope}%
\pgfpathrectangle{\pgfqpoint{0.697024in}{0.857143in}}{\pgfqpoint{2.627103in}{1.813434in}}%
\pgfusepath{clip}%
\pgfsetbuttcap%
\pgfsetmiterjoin%
\definecolor{currentfill}{rgb}{0.302379,0.450282,0.300122}%
\pgfsetfillcolor{currentfill}%
\pgfsetlinewidth{0.000000pt}%
\definecolor{currentstroke}{rgb}{0.000000,0.000000,0.000000}%
\pgfsetstrokecolor{currentstroke}%
\pgfsetstrokeopacity{0.000000}%
\pgfsetdash{}{0pt}%
\pgfpathmoveto{\pgfqpoint{2.190423in}{1.847071in}}%
\pgfpathlineto{\pgfqpoint{2.199359in}{1.847071in}}%
\pgfpathlineto{\pgfqpoint{2.199359in}{2.346952in}}%
\pgfpathlineto{\pgfqpoint{2.190423in}{2.346952in}}%
\pgfpathlineto{\pgfqpoint{2.190423in}{1.847071in}}%
\pgfpathclose%
\pgfusepath{fill}%
\end{pgfscope}%
\begin{pgfscope}%
\pgfpathrectangle{\pgfqpoint{0.697024in}{0.857143in}}{\pgfqpoint{2.627103in}{1.813434in}}%
\pgfusepath{clip}%
\pgfsetbuttcap%
\pgfsetmiterjoin%
\definecolor{currentfill}{rgb}{0.302379,0.450282,0.300122}%
\pgfsetfillcolor{currentfill}%
\pgfsetlinewidth{0.000000pt}%
\definecolor{currentstroke}{rgb}{0.000000,0.000000,0.000000}%
\pgfsetstrokecolor{currentstroke}%
\pgfsetstrokeopacity{0.000000}%
\pgfsetdash{}{0pt}%
\pgfpathmoveto{\pgfqpoint{2.201593in}{1.847071in}}%
\pgfpathlineto{\pgfqpoint{2.210530in}{1.847071in}}%
\pgfpathlineto{\pgfqpoint{2.210530in}{2.342837in}}%
\pgfpathlineto{\pgfqpoint{2.201593in}{2.342837in}}%
\pgfpathlineto{\pgfqpoint{2.201593in}{1.847071in}}%
\pgfpathclose%
\pgfusepath{fill}%
\end{pgfscope}%
\begin{pgfscope}%
\pgfpathrectangle{\pgfqpoint{0.697024in}{0.857143in}}{\pgfqpoint{2.627103in}{1.813434in}}%
\pgfusepath{clip}%
\pgfsetbuttcap%
\pgfsetmiterjoin%
\definecolor{currentfill}{rgb}{0.302379,0.450282,0.300122}%
\pgfsetfillcolor{currentfill}%
\pgfsetlinewidth{0.000000pt}%
\definecolor{currentstroke}{rgb}{0.000000,0.000000,0.000000}%
\pgfsetstrokecolor{currentstroke}%
\pgfsetstrokeopacity{0.000000}%
\pgfsetdash{}{0pt}%
\pgfpathmoveto{\pgfqpoint{2.212764in}{1.847071in}}%
\pgfpathlineto{\pgfqpoint{2.221700in}{1.847071in}}%
\pgfpathlineto{\pgfqpoint{2.221700in}{2.284647in}}%
\pgfpathlineto{\pgfqpoint{2.212764in}{2.284647in}}%
\pgfpathlineto{\pgfqpoint{2.212764in}{1.847071in}}%
\pgfpathclose%
\pgfusepath{fill}%
\end{pgfscope}%
\begin{pgfscope}%
\pgfpathrectangle{\pgfqpoint{0.697024in}{0.857143in}}{\pgfqpoint{2.627103in}{1.813434in}}%
\pgfusepath{clip}%
\pgfsetbuttcap%
\pgfsetmiterjoin%
\definecolor{currentfill}{rgb}{0.302379,0.450282,0.300122}%
\pgfsetfillcolor{currentfill}%
\pgfsetlinewidth{0.000000pt}%
\definecolor{currentstroke}{rgb}{0.000000,0.000000,0.000000}%
\pgfsetstrokecolor{currentstroke}%
\pgfsetstrokeopacity{0.000000}%
\pgfsetdash{}{0pt}%
\pgfpathmoveto{\pgfqpoint{2.223934in}{1.879406in}}%
\pgfpathlineto{\pgfqpoint{2.232871in}{1.879406in}}%
\pgfpathlineto{\pgfqpoint{2.232871in}{2.286950in}}%
\pgfpathlineto{\pgfqpoint{2.223934in}{2.286950in}}%
\pgfpathlineto{\pgfqpoint{2.223934in}{1.879406in}}%
\pgfpathclose%
\pgfusepath{fill}%
\end{pgfscope}%
\begin{pgfscope}%
\pgfpathrectangle{\pgfqpoint{0.697024in}{0.857143in}}{\pgfqpoint{2.627103in}{1.813434in}}%
\pgfusepath{clip}%
\pgfsetbuttcap%
\pgfsetmiterjoin%
\definecolor{currentfill}{rgb}{0.302379,0.450282,0.300122}%
\pgfsetfillcolor{currentfill}%
\pgfsetlinewidth{0.000000pt}%
\definecolor{currentstroke}{rgb}{0.000000,0.000000,0.000000}%
\pgfsetstrokecolor{currentstroke}%
\pgfsetstrokeopacity{0.000000}%
\pgfsetdash{}{0pt}%
\pgfpathmoveto{\pgfqpoint{2.235105in}{1.900467in}}%
\pgfpathlineto{\pgfqpoint{2.244042in}{1.900467in}}%
\pgfpathlineto{\pgfqpoint{2.244042in}{2.314063in}}%
\pgfpathlineto{\pgfqpoint{2.235105in}{2.314063in}}%
\pgfpathlineto{\pgfqpoint{2.235105in}{1.900467in}}%
\pgfpathclose%
\pgfusepath{fill}%
\end{pgfscope}%
\begin{pgfscope}%
\pgfpathrectangle{\pgfqpoint{0.697024in}{0.857143in}}{\pgfqpoint{2.627103in}{1.813434in}}%
\pgfusepath{clip}%
\pgfsetbuttcap%
\pgfsetmiterjoin%
\definecolor{currentfill}{rgb}{0.302379,0.450282,0.300122}%
\pgfsetfillcolor{currentfill}%
\pgfsetlinewidth{0.000000pt}%
\definecolor{currentstroke}{rgb}{0.000000,0.000000,0.000000}%
\pgfsetstrokecolor{currentstroke}%
\pgfsetstrokeopacity{0.000000}%
\pgfsetdash{}{0pt}%
\pgfpathmoveto{\pgfqpoint{2.246276in}{1.847071in}}%
\pgfpathlineto{\pgfqpoint{2.255212in}{1.847071in}}%
\pgfpathlineto{\pgfqpoint{2.255212in}{2.320810in}}%
\pgfpathlineto{\pgfqpoint{2.246276in}{2.320810in}}%
\pgfpathlineto{\pgfqpoint{2.246276in}{1.847071in}}%
\pgfpathclose%
\pgfusepath{fill}%
\end{pgfscope}%
\begin{pgfscope}%
\pgfpathrectangle{\pgfqpoint{0.697024in}{0.857143in}}{\pgfqpoint{2.627103in}{1.813434in}}%
\pgfusepath{clip}%
\pgfsetbuttcap%
\pgfsetmiterjoin%
\definecolor{currentfill}{rgb}{0.302379,0.450282,0.300122}%
\pgfsetfillcolor{currentfill}%
\pgfsetlinewidth{0.000000pt}%
\definecolor{currentstroke}{rgb}{0.000000,0.000000,0.000000}%
\pgfsetstrokecolor{currentstroke}%
\pgfsetstrokeopacity{0.000000}%
\pgfsetdash{}{0pt}%
\pgfpathmoveto{\pgfqpoint{2.257446in}{1.922830in}}%
\pgfpathlineto{\pgfqpoint{2.266383in}{1.922830in}}%
\pgfpathlineto{\pgfqpoint{2.266383in}{2.344326in}}%
\pgfpathlineto{\pgfqpoint{2.257446in}{2.344326in}}%
\pgfpathlineto{\pgfqpoint{2.257446in}{1.922830in}}%
\pgfpathclose%
\pgfusepath{fill}%
\end{pgfscope}%
\begin{pgfscope}%
\pgfpathrectangle{\pgfqpoint{0.697024in}{0.857143in}}{\pgfqpoint{2.627103in}{1.813434in}}%
\pgfusepath{clip}%
\pgfsetbuttcap%
\pgfsetmiterjoin%
\definecolor{currentfill}{rgb}{0.302379,0.450282,0.300122}%
\pgfsetfillcolor{currentfill}%
\pgfsetlinewidth{0.000000pt}%
\definecolor{currentstroke}{rgb}{0.000000,0.000000,0.000000}%
\pgfsetstrokecolor{currentstroke}%
\pgfsetstrokeopacity{0.000000}%
\pgfsetdash{}{0pt}%
\pgfpathmoveto{\pgfqpoint{2.268617in}{1.926951in}}%
\pgfpathlineto{\pgfqpoint{2.277553in}{1.926951in}}%
\pgfpathlineto{\pgfqpoint{2.277553in}{2.327182in}}%
\pgfpathlineto{\pgfqpoint{2.268617in}{2.327182in}}%
\pgfpathlineto{\pgfqpoint{2.268617in}{1.926951in}}%
\pgfpathclose%
\pgfusepath{fill}%
\end{pgfscope}%
\begin{pgfscope}%
\pgfpathrectangle{\pgfqpoint{0.697024in}{0.857143in}}{\pgfqpoint{2.627103in}{1.813434in}}%
\pgfusepath{clip}%
\pgfsetbuttcap%
\pgfsetmiterjoin%
\definecolor{currentfill}{rgb}{0.302379,0.450282,0.300122}%
\pgfsetfillcolor{currentfill}%
\pgfsetlinewidth{0.000000pt}%
\definecolor{currentstroke}{rgb}{0.000000,0.000000,0.000000}%
\pgfsetstrokecolor{currentstroke}%
\pgfsetstrokeopacity{0.000000}%
\pgfsetdash{}{0pt}%
\pgfpathmoveto{\pgfqpoint{2.279787in}{1.938065in}}%
\pgfpathlineto{\pgfqpoint{2.288724in}{1.938065in}}%
\pgfpathlineto{\pgfqpoint{2.288724in}{2.379490in}}%
\pgfpathlineto{\pgfqpoint{2.279787in}{2.379490in}}%
\pgfpathlineto{\pgfqpoint{2.279787in}{1.938065in}}%
\pgfpathclose%
\pgfusepath{fill}%
\end{pgfscope}%
\begin{pgfscope}%
\pgfpathrectangle{\pgfqpoint{0.697024in}{0.857143in}}{\pgfqpoint{2.627103in}{1.813434in}}%
\pgfusepath{clip}%
\pgfsetbuttcap%
\pgfsetmiterjoin%
\definecolor{currentfill}{rgb}{0.302379,0.450282,0.300122}%
\pgfsetfillcolor{currentfill}%
\pgfsetlinewidth{0.000000pt}%
\definecolor{currentstroke}{rgb}{0.000000,0.000000,0.000000}%
\pgfsetstrokecolor{currentstroke}%
\pgfsetstrokeopacity{0.000000}%
\pgfsetdash{}{0pt}%
\pgfpathmoveto{\pgfqpoint{2.290958in}{1.982655in}}%
\pgfpathlineto{\pgfqpoint{2.299895in}{1.982655in}}%
\pgfpathlineto{\pgfqpoint{2.299895in}{2.401806in}}%
\pgfpathlineto{\pgfqpoint{2.290958in}{2.401806in}}%
\pgfpathlineto{\pgfqpoint{2.290958in}{1.982655in}}%
\pgfpathclose%
\pgfusepath{fill}%
\end{pgfscope}%
\begin{pgfscope}%
\pgfpathrectangle{\pgfqpoint{0.697024in}{0.857143in}}{\pgfqpoint{2.627103in}{1.813434in}}%
\pgfusepath{clip}%
\pgfsetbuttcap%
\pgfsetmiterjoin%
\definecolor{currentfill}{rgb}{0.302379,0.450282,0.300122}%
\pgfsetfillcolor{currentfill}%
\pgfsetlinewidth{0.000000pt}%
\definecolor{currentstroke}{rgb}{0.000000,0.000000,0.000000}%
\pgfsetstrokecolor{currentstroke}%
\pgfsetstrokeopacity{0.000000}%
\pgfsetdash{}{0pt}%
\pgfpathmoveto{\pgfqpoint{2.302129in}{1.931558in}}%
\pgfpathlineto{\pgfqpoint{2.311065in}{1.931558in}}%
\pgfpathlineto{\pgfqpoint{2.311065in}{2.369665in}}%
\pgfpathlineto{\pgfqpoint{2.302129in}{2.369665in}}%
\pgfpathlineto{\pgfqpoint{2.302129in}{1.931558in}}%
\pgfpathclose%
\pgfusepath{fill}%
\end{pgfscope}%
\begin{pgfscope}%
\pgfpathrectangle{\pgfqpoint{0.697024in}{0.857143in}}{\pgfqpoint{2.627103in}{1.813434in}}%
\pgfusepath{clip}%
\pgfsetbuttcap%
\pgfsetmiterjoin%
\definecolor{currentfill}{rgb}{0.302379,0.450282,0.300122}%
\pgfsetfillcolor{currentfill}%
\pgfsetlinewidth{0.000000pt}%
\definecolor{currentstroke}{rgb}{0.000000,0.000000,0.000000}%
\pgfsetstrokecolor{currentstroke}%
\pgfsetstrokeopacity{0.000000}%
\pgfsetdash{}{0pt}%
\pgfpathmoveto{\pgfqpoint{2.313299in}{1.933288in}}%
\pgfpathlineto{\pgfqpoint{2.322236in}{1.933288in}}%
\pgfpathlineto{\pgfqpoint{2.322236in}{2.428362in}}%
\pgfpathlineto{\pgfqpoint{2.313299in}{2.428362in}}%
\pgfpathlineto{\pgfqpoint{2.313299in}{1.933288in}}%
\pgfpathclose%
\pgfusepath{fill}%
\end{pgfscope}%
\begin{pgfscope}%
\pgfpathrectangle{\pgfqpoint{0.697024in}{0.857143in}}{\pgfqpoint{2.627103in}{1.813434in}}%
\pgfusepath{clip}%
\pgfsetbuttcap%
\pgfsetmiterjoin%
\definecolor{currentfill}{rgb}{0.302379,0.450282,0.300122}%
\pgfsetfillcolor{currentfill}%
\pgfsetlinewidth{0.000000pt}%
\definecolor{currentstroke}{rgb}{0.000000,0.000000,0.000000}%
\pgfsetstrokecolor{currentstroke}%
\pgfsetstrokeopacity{0.000000}%
\pgfsetdash{}{0pt}%
\pgfpathmoveto{\pgfqpoint{2.324470in}{1.973105in}}%
\pgfpathlineto{\pgfqpoint{2.333406in}{1.973105in}}%
\pgfpathlineto{\pgfqpoint{2.333406in}{2.484676in}}%
\pgfpathlineto{\pgfqpoint{2.324470in}{2.484676in}}%
\pgfpathlineto{\pgfqpoint{2.324470in}{1.973105in}}%
\pgfpathclose%
\pgfusepath{fill}%
\end{pgfscope}%
\begin{pgfscope}%
\pgfpathrectangle{\pgfqpoint{0.697024in}{0.857143in}}{\pgfqpoint{2.627103in}{1.813434in}}%
\pgfusepath{clip}%
\pgfsetbuttcap%
\pgfsetmiterjoin%
\definecolor{currentfill}{rgb}{0.302379,0.450282,0.300122}%
\pgfsetfillcolor{currentfill}%
\pgfsetlinewidth{0.000000pt}%
\definecolor{currentstroke}{rgb}{0.000000,0.000000,0.000000}%
\pgfsetstrokecolor{currentstroke}%
\pgfsetstrokeopacity{0.000000}%
\pgfsetdash{}{0pt}%
\pgfpathmoveto{\pgfqpoint{2.335640in}{1.996254in}}%
\pgfpathlineto{\pgfqpoint{2.344577in}{1.996254in}}%
\pgfpathlineto{\pgfqpoint{2.344577in}{2.499490in}}%
\pgfpathlineto{\pgfqpoint{2.335640in}{2.499490in}}%
\pgfpathlineto{\pgfqpoint{2.335640in}{1.996254in}}%
\pgfpathclose%
\pgfusepath{fill}%
\end{pgfscope}%
\begin{pgfscope}%
\pgfpathrectangle{\pgfqpoint{0.697024in}{0.857143in}}{\pgfqpoint{2.627103in}{1.813434in}}%
\pgfusepath{clip}%
\pgfsetbuttcap%
\pgfsetmiterjoin%
\definecolor{currentfill}{rgb}{0.302379,0.450282,0.300122}%
\pgfsetfillcolor{currentfill}%
\pgfsetlinewidth{0.000000pt}%
\definecolor{currentstroke}{rgb}{0.000000,0.000000,0.000000}%
\pgfsetstrokecolor{currentstroke}%
\pgfsetstrokeopacity{0.000000}%
\pgfsetdash{}{0pt}%
\pgfpathmoveto{\pgfqpoint{2.346811in}{2.031300in}}%
\pgfpathlineto{\pgfqpoint{2.355748in}{2.031300in}}%
\pgfpathlineto{\pgfqpoint{2.355748in}{2.520198in}}%
\pgfpathlineto{\pgfqpoint{2.346811in}{2.520198in}}%
\pgfpathlineto{\pgfqpoint{2.346811in}{2.031300in}}%
\pgfpathclose%
\pgfusepath{fill}%
\end{pgfscope}%
\begin{pgfscope}%
\pgfpathrectangle{\pgfqpoint{0.697024in}{0.857143in}}{\pgfqpoint{2.627103in}{1.813434in}}%
\pgfusepath{clip}%
\pgfsetbuttcap%
\pgfsetmiterjoin%
\definecolor{currentfill}{rgb}{0.302379,0.450282,0.300122}%
\pgfsetfillcolor{currentfill}%
\pgfsetlinewidth{0.000000pt}%
\definecolor{currentstroke}{rgb}{0.000000,0.000000,0.000000}%
\pgfsetstrokecolor{currentstroke}%
\pgfsetstrokeopacity{0.000000}%
\pgfsetdash{}{0pt}%
\pgfpathmoveto{\pgfqpoint{2.357982in}{1.981780in}}%
\pgfpathlineto{\pgfqpoint{2.366918in}{1.981780in}}%
\pgfpathlineto{\pgfqpoint{2.366918in}{2.467524in}}%
\pgfpathlineto{\pgfqpoint{2.357982in}{2.467524in}}%
\pgfpathlineto{\pgfqpoint{2.357982in}{1.981780in}}%
\pgfpathclose%
\pgfusepath{fill}%
\end{pgfscope}%
\begin{pgfscope}%
\pgfpathrectangle{\pgfqpoint{0.697024in}{0.857143in}}{\pgfqpoint{2.627103in}{1.813434in}}%
\pgfusepath{clip}%
\pgfsetbuttcap%
\pgfsetmiterjoin%
\definecolor{currentfill}{rgb}{0.302379,0.450282,0.300122}%
\pgfsetfillcolor{currentfill}%
\pgfsetlinewidth{0.000000pt}%
\definecolor{currentstroke}{rgb}{0.000000,0.000000,0.000000}%
\pgfsetstrokecolor{currentstroke}%
\pgfsetstrokeopacity{0.000000}%
\pgfsetdash{}{0pt}%
\pgfpathmoveto{\pgfqpoint{2.369152in}{2.014650in}}%
\pgfpathlineto{\pgfqpoint{2.378089in}{2.014650in}}%
\pgfpathlineto{\pgfqpoint{2.378089in}{2.399199in}}%
\pgfpathlineto{\pgfqpoint{2.369152in}{2.399199in}}%
\pgfpathlineto{\pgfqpoint{2.369152in}{2.014650in}}%
\pgfpathclose%
\pgfusepath{fill}%
\end{pgfscope}%
\begin{pgfscope}%
\pgfpathrectangle{\pgfqpoint{0.697024in}{0.857143in}}{\pgfqpoint{2.627103in}{1.813434in}}%
\pgfusepath{clip}%
\pgfsetbuttcap%
\pgfsetmiterjoin%
\definecolor{currentfill}{rgb}{0.302379,0.450282,0.300122}%
\pgfsetfillcolor{currentfill}%
\pgfsetlinewidth{0.000000pt}%
\definecolor{currentstroke}{rgb}{0.000000,0.000000,0.000000}%
\pgfsetstrokecolor{currentstroke}%
\pgfsetstrokeopacity{0.000000}%
\pgfsetdash{}{0pt}%
\pgfpathmoveto{\pgfqpoint{2.380323in}{2.000964in}}%
\pgfpathlineto{\pgfqpoint{2.389259in}{2.000964in}}%
\pgfpathlineto{\pgfqpoint{2.389259in}{2.345994in}}%
\pgfpathlineto{\pgfqpoint{2.380323in}{2.345994in}}%
\pgfpathlineto{\pgfqpoint{2.380323in}{2.000964in}}%
\pgfpathclose%
\pgfusepath{fill}%
\end{pgfscope}%
\begin{pgfscope}%
\pgfpathrectangle{\pgfqpoint{0.697024in}{0.857143in}}{\pgfqpoint{2.627103in}{1.813434in}}%
\pgfusepath{clip}%
\pgfsetbuttcap%
\pgfsetmiterjoin%
\definecolor{currentfill}{rgb}{0.302379,0.450282,0.300122}%
\pgfsetfillcolor{currentfill}%
\pgfsetlinewidth{0.000000pt}%
\definecolor{currentstroke}{rgb}{0.000000,0.000000,0.000000}%
\pgfsetstrokecolor{currentstroke}%
\pgfsetstrokeopacity{0.000000}%
\pgfsetdash{}{0pt}%
\pgfpathmoveto{\pgfqpoint{2.391494in}{1.994649in}}%
\pgfpathlineto{\pgfqpoint{2.400430in}{1.994649in}}%
\pgfpathlineto{\pgfqpoint{2.400430in}{2.304443in}}%
\pgfpathlineto{\pgfqpoint{2.391494in}{2.304443in}}%
\pgfpathlineto{\pgfqpoint{2.391494in}{1.994649in}}%
\pgfpathclose%
\pgfusepath{fill}%
\end{pgfscope}%
\begin{pgfscope}%
\pgfpathrectangle{\pgfqpoint{0.697024in}{0.857143in}}{\pgfqpoint{2.627103in}{1.813434in}}%
\pgfusepath{clip}%
\pgfsetbuttcap%
\pgfsetmiterjoin%
\definecolor{currentfill}{rgb}{0.302379,0.450282,0.300122}%
\pgfsetfillcolor{currentfill}%
\pgfsetlinewidth{0.000000pt}%
\definecolor{currentstroke}{rgb}{0.000000,0.000000,0.000000}%
\pgfsetstrokecolor{currentstroke}%
\pgfsetstrokeopacity{0.000000}%
\pgfsetdash{}{0pt}%
\pgfpathmoveto{\pgfqpoint{2.402664in}{2.002555in}}%
\pgfpathlineto{\pgfqpoint{2.411601in}{2.002555in}}%
\pgfpathlineto{\pgfqpoint{2.411601in}{2.224687in}}%
\pgfpathlineto{\pgfqpoint{2.402664in}{2.224687in}}%
\pgfpathlineto{\pgfqpoint{2.402664in}{2.002555in}}%
\pgfpathclose%
\pgfusepath{fill}%
\end{pgfscope}%
\begin{pgfscope}%
\pgfpathrectangle{\pgfqpoint{0.697024in}{0.857143in}}{\pgfqpoint{2.627103in}{1.813434in}}%
\pgfusepath{clip}%
\pgfsetbuttcap%
\pgfsetmiterjoin%
\definecolor{currentfill}{rgb}{0.302379,0.450282,0.300122}%
\pgfsetfillcolor{currentfill}%
\pgfsetlinewidth{0.000000pt}%
\definecolor{currentstroke}{rgb}{0.000000,0.000000,0.000000}%
\pgfsetstrokecolor{currentstroke}%
\pgfsetstrokeopacity{0.000000}%
\pgfsetdash{}{0pt}%
\pgfpathmoveto{\pgfqpoint{2.413835in}{1.978986in}}%
\pgfpathlineto{\pgfqpoint{2.422771in}{1.978986in}}%
\pgfpathlineto{\pgfqpoint{2.422771in}{2.227591in}}%
\pgfpathlineto{\pgfqpoint{2.413835in}{2.227591in}}%
\pgfpathlineto{\pgfqpoint{2.413835in}{1.978986in}}%
\pgfpathclose%
\pgfusepath{fill}%
\end{pgfscope}%
\begin{pgfscope}%
\pgfpathrectangle{\pgfqpoint{0.697024in}{0.857143in}}{\pgfqpoint{2.627103in}{1.813434in}}%
\pgfusepath{clip}%
\pgfsetbuttcap%
\pgfsetmiterjoin%
\definecolor{currentfill}{rgb}{0.302379,0.450282,0.300122}%
\pgfsetfillcolor{currentfill}%
\pgfsetlinewidth{0.000000pt}%
\definecolor{currentstroke}{rgb}{0.000000,0.000000,0.000000}%
\pgfsetstrokecolor{currentstroke}%
\pgfsetstrokeopacity{0.000000}%
\pgfsetdash{}{0pt}%
\pgfpathmoveto{\pgfqpoint{2.425005in}{1.994111in}}%
\pgfpathlineto{\pgfqpoint{2.433942in}{1.994111in}}%
\pgfpathlineto{\pgfqpoint{2.433942in}{2.200153in}}%
\pgfpathlineto{\pgfqpoint{2.425005in}{2.200153in}}%
\pgfpathlineto{\pgfqpoint{2.425005in}{1.994111in}}%
\pgfpathclose%
\pgfusepath{fill}%
\end{pgfscope}%
\begin{pgfscope}%
\pgfpathrectangle{\pgfqpoint{0.697024in}{0.857143in}}{\pgfqpoint{2.627103in}{1.813434in}}%
\pgfusepath{clip}%
\pgfsetbuttcap%
\pgfsetmiterjoin%
\definecolor{currentfill}{rgb}{0.302379,0.450282,0.300122}%
\pgfsetfillcolor{currentfill}%
\pgfsetlinewidth{0.000000pt}%
\definecolor{currentstroke}{rgb}{0.000000,0.000000,0.000000}%
\pgfsetstrokecolor{currentstroke}%
\pgfsetstrokeopacity{0.000000}%
\pgfsetdash{}{0pt}%
\pgfpathmoveto{\pgfqpoint{2.436176in}{1.978156in}}%
\pgfpathlineto{\pgfqpoint{2.445112in}{1.978156in}}%
\pgfpathlineto{\pgfqpoint{2.445112in}{2.136130in}}%
\pgfpathlineto{\pgfqpoint{2.436176in}{2.136130in}}%
\pgfpathlineto{\pgfqpoint{2.436176in}{1.978156in}}%
\pgfpathclose%
\pgfusepath{fill}%
\end{pgfscope}%
\begin{pgfscope}%
\pgfpathrectangle{\pgfqpoint{0.697024in}{0.857143in}}{\pgfqpoint{2.627103in}{1.813434in}}%
\pgfusepath{clip}%
\pgfsetbuttcap%
\pgfsetmiterjoin%
\definecolor{currentfill}{rgb}{0.302379,0.450282,0.300122}%
\pgfsetfillcolor{currentfill}%
\pgfsetlinewidth{0.000000pt}%
\definecolor{currentstroke}{rgb}{0.000000,0.000000,0.000000}%
\pgfsetstrokecolor{currentstroke}%
\pgfsetstrokeopacity{0.000000}%
\pgfsetdash{}{0pt}%
\pgfpathmoveto{\pgfqpoint{2.447347in}{2.008935in}}%
\pgfpathlineto{\pgfqpoint{2.456283in}{2.008935in}}%
\pgfpathlineto{\pgfqpoint{2.456283in}{2.171115in}}%
\pgfpathlineto{\pgfqpoint{2.447347in}{2.171115in}}%
\pgfpathlineto{\pgfqpoint{2.447347in}{2.008935in}}%
\pgfpathclose%
\pgfusepath{fill}%
\end{pgfscope}%
\begin{pgfscope}%
\pgfpathrectangle{\pgfqpoint{0.697024in}{0.857143in}}{\pgfqpoint{2.627103in}{1.813434in}}%
\pgfusepath{clip}%
\pgfsetbuttcap%
\pgfsetmiterjoin%
\definecolor{currentfill}{rgb}{0.302379,0.450282,0.300122}%
\pgfsetfillcolor{currentfill}%
\pgfsetlinewidth{0.000000pt}%
\definecolor{currentstroke}{rgb}{0.000000,0.000000,0.000000}%
\pgfsetstrokecolor{currentstroke}%
\pgfsetstrokeopacity{0.000000}%
\pgfsetdash{}{0pt}%
\pgfpathmoveto{\pgfqpoint{2.458517in}{1.996707in}}%
\pgfpathlineto{\pgfqpoint{2.467454in}{1.996707in}}%
\pgfpathlineto{\pgfqpoint{2.467454in}{2.161834in}}%
\pgfpathlineto{\pgfqpoint{2.458517in}{2.161834in}}%
\pgfpathlineto{\pgfqpoint{2.458517in}{1.996707in}}%
\pgfpathclose%
\pgfusepath{fill}%
\end{pgfscope}%
\begin{pgfscope}%
\pgfpathrectangle{\pgfqpoint{0.697024in}{0.857143in}}{\pgfqpoint{2.627103in}{1.813434in}}%
\pgfusepath{clip}%
\pgfsetbuttcap%
\pgfsetmiterjoin%
\definecolor{currentfill}{rgb}{0.302379,0.450282,0.300122}%
\pgfsetfillcolor{currentfill}%
\pgfsetlinewidth{0.000000pt}%
\definecolor{currentstroke}{rgb}{0.000000,0.000000,0.000000}%
\pgfsetstrokecolor{currentstroke}%
\pgfsetstrokeopacity{0.000000}%
\pgfsetdash{}{0pt}%
\pgfpathmoveto{\pgfqpoint{2.469688in}{1.994211in}}%
\pgfpathlineto{\pgfqpoint{2.478624in}{1.994211in}}%
\pgfpathlineto{\pgfqpoint{2.478624in}{2.157094in}}%
\pgfpathlineto{\pgfqpoint{2.469688in}{2.157094in}}%
\pgfpathlineto{\pgfqpoint{2.469688in}{1.994211in}}%
\pgfpathclose%
\pgfusepath{fill}%
\end{pgfscope}%
\begin{pgfscope}%
\pgfpathrectangle{\pgfqpoint{0.697024in}{0.857143in}}{\pgfqpoint{2.627103in}{1.813434in}}%
\pgfusepath{clip}%
\pgfsetbuttcap%
\pgfsetmiterjoin%
\definecolor{currentfill}{rgb}{0.302379,0.450282,0.300122}%
\pgfsetfillcolor{currentfill}%
\pgfsetlinewidth{0.000000pt}%
\definecolor{currentstroke}{rgb}{0.000000,0.000000,0.000000}%
\pgfsetstrokecolor{currentstroke}%
\pgfsetstrokeopacity{0.000000}%
\pgfsetdash{}{0pt}%
\pgfpathmoveto{\pgfqpoint{2.480858in}{1.940235in}}%
\pgfpathlineto{\pgfqpoint{2.489795in}{1.940235in}}%
\pgfpathlineto{\pgfqpoint{2.489795in}{2.174912in}}%
\pgfpathlineto{\pgfqpoint{2.480858in}{2.174912in}}%
\pgfpathlineto{\pgfqpoint{2.480858in}{1.940235in}}%
\pgfpathclose%
\pgfusepath{fill}%
\end{pgfscope}%
\begin{pgfscope}%
\pgfpathrectangle{\pgfqpoint{0.697024in}{0.857143in}}{\pgfqpoint{2.627103in}{1.813434in}}%
\pgfusepath{clip}%
\pgfsetbuttcap%
\pgfsetmiterjoin%
\definecolor{currentfill}{rgb}{0.302379,0.450282,0.300122}%
\pgfsetfillcolor{currentfill}%
\pgfsetlinewidth{0.000000pt}%
\definecolor{currentstroke}{rgb}{0.000000,0.000000,0.000000}%
\pgfsetstrokecolor{currentstroke}%
\pgfsetstrokeopacity{0.000000}%
\pgfsetdash{}{0pt}%
\pgfpathmoveto{\pgfqpoint{2.492029in}{1.929004in}}%
\pgfpathlineto{\pgfqpoint{2.500965in}{1.929004in}}%
\pgfpathlineto{\pgfqpoint{2.500965in}{2.174223in}}%
\pgfpathlineto{\pgfqpoint{2.492029in}{2.174223in}}%
\pgfpathlineto{\pgfqpoint{2.492029in}{1.929004in}}%
\pgfpathclose%
\pgfusepath{fill}%
\end{pgfscope}%
\begin{pgfscope}%
\pgfpathrectangle{\pgfqpoint{0.697024in}{0.857143in}}{\pgfqpoint{2.627103in}{1.813434in}}%
\pgfusepath{clip}%
\pgfsetbuttcap%
\pgfsetmiterjoin%
\definecolor{currentfill}{rgb}{0.302379,0.450282,0.300122}%
\pgfsetfillcolor{currentfill}%
\pgfsetlinewidth{0.000000pt}%
\definecolor{currentstroke}{rgb}{0.000000,0.000000,0.000000}%
\pgfsetstrokecolor{currentstroke}%
\pgfsetstrokeopacity{0.000000}%
\pgfsetdash{}{0pt}%
\pgfpathmoveto{\pgfqpoint{2.503200in}{1.937025in}}%
\pgfpathlineto{\pgfqpoint{2.512136in}{1.937025in}}%
\pgfpathlineto{\pgfqpoint{2.512136in}{2.201661in}}%
\pgfpathlineto{\pgfqpoint{2.503200in}{2.201661in}}%
\pgfpathlineto{\pgfqpoint{2.503200in}{1.937025in}}%
\pgfpathclose%
\pgfusepath{fill}%
\end{pgfscope}%
\begin{pgfscope}%
\pgfpathrectangle{\pgfqpoint{0.697024in}{0.857143in}}{\pgfqpoint{2.627103in}{1.813434in}}%
\pgfusepath{clip}%
\pgfsetbuttcap%
\pgfsetmiterjoin%
\definecolor{currentfill}{rgb}{0.302379,0.450282,0.300122}%
\pgfsetfillcolor{currentfill}%
\pgfsetlinewidth{0.000000pt}%
\definecolor{currentstroke}{rgb}{0.000000,0.000000,0.000000}%
\pgfsetstrokecolor{currentstroke}%
\pgfsetstrokeopacity{0.000000}%
\pgfsetdash{}{0pt}%
\pgfpathmoveto{\pgfqpoint{2.514370in}{1.935810in}}%
\pgfpathlineto{\pgfqpoint{2.523307in}{1.935810in}}%
\pgfpathlineto{\pgfqpoint{2.523307in}{2.203340in}}%
\pgfpathlineto{\pgfqpoint{2.514370in}{2.203340in}}%
\pgfpathlineto{\pgfqpoint{2.514370in}{1.935810in}}%
\pgfpathclose%
\pgfusepath{fill}%
\end{pgfscope}%
\begin{pgfscope}%
\pgfpathrectangle{\pgfqpoint{0.697024in}{0.857143in}}{\pgfqpoint{2.627103in}{1.813434in}}%
\pgfusepath{clip}%
\pgfsetbuttcap%
\pgfsetmiterjoin%
\definecolor{currentfill}{rgb}{0.302379,0.450282,0.300122}%
\pgfsetfillcolor{currentfill}%
\pgfsetlinewidth{0.000000pt}%
\definecolor{currentstroke}{rgb}{0.000000,0.000000,0.000000}%
\pgfsetstrokecolor{currentstroke}%
\pgfsetstrokeopacity{0.000000}%
\pgfsetdash{}{0pt}%
\pgfpathmoveto{\pgfqpoint{2.525541in}{1.936951in}}%
\pgfpathlineto{\pgfqpoint{2.534477in}{1.936951in}}%
\pgfpathlineto{\pgfqpoint{2.534477in}{2.231373in}}%
\pgfpathlineto{\pgfqpoint{2.525541in}{2.231373in}}%
\pgfpathlineto{\pgfqpoint{2.525541in}{1.936951in}}%
\pgfpathclose%
\pgfusepath{fill}%
\end{pgfscope}%
\begin{pgfscope}%
\pgfpathrectangle{\pgfqpoint{0.697024in}{0.857143in}}{\pgfqpoint{2.627103in}{1.813434in}}%
\pgfusepath{clip}%
\pgfsetbuttcap%
\pgfsetmiterjoin%
\definecolor{currentfill}{rgb}{0.302379,0.450282,0.300122}%
\pgfsetfillcolor{currentfill}%
\pgfsetlinewidth{0.000000pt}%
\definecolor{currentstroke}{rgb}{0.000000,0.000000,0.000000}%
\pgfsetstrokecolor{currentstroke}%
\pgfsetstrokeopacity{0.000000}%
\pgfsetdash{}{0pt}%
\pgfpathmoveto{\pgfqpoint{2.536711in}{1.950197in}}%
\pgfpathlineto{\pgfqpoint{2.545648in}{1.950197in}}%
\pgfpathlineto{\pgfqpoint{2.545648in}{2.274566in}}%
\pgfpathlineto{\pgfqpoint{2.536711in}{2.274566in}}%
\pgfpathlineto{\pgfqpoint{2.536711in}{1.950197in}}%
\pgfpathclose%
\pgfusepath{fill}%
\end{pgfscope}%
\begin{pgfscope}%
\pgfpathrectangle{\pgfqpoint{0.697024in}{0.857143in}}{\pgfqpoint{2.627103in}{1.813434in}}%
\pgfusepath{clip}%
\pgfsetbuttcap%
\pgfsetmiterjoin%
\definecolor{currentfill}{rgb}{0.302379,0.450282,0.300122}%
\pgfsetfillcolor{currentfill}%
\pgfsetlinewidth{0.000000pt}%
\definecolor{currentstroke}{rgb}{0.000000,0.000000,0.000000}%
\pgfsetstrokecolor{currentstroke}%
\pgfsetstrokeopacity{0.000000}%
\pgfsetdash{}{0pt}%
\pgfpathmoveto{\pgfqpoint{2.547882in}{1.935488in}}%
\pgfpathlineto{\pgfqpoint{2.556818in}{1.935488in}}%
\pgfpathlineto{\pgfqpoint{2.556818in}{2.301594in}}%
\pgfpathlineto{\pgfqpoint{2.547882in}{2.301594in}}%
\pgfpathlineto{\pgfqpoint{2.547882in}{1.935488in}}%
\pgfpathclose%
\pgfusepath{fill}%
\end{pgfscope}%
\begin{pgfscope}%
\pgfpathrectangle{\pgfqpoint{0.697024in}{0.857143in}}{\pgfqpoint{2.627103in}{1.813434in}}%
\pgfusepath{clip}%
\pgfsetbuttcap%
\pgfsetmiterjoin%
\definecolor{currentfill}{rgb}{0.302379,0.450282,0.300122}%
\pgfsetfillcolor{currentfill}%
\pgfsetlinewidth{0.000000pt}%
\definecolor{currentstroke}{rgb}{0.000000,0.000000,0.000000}%
\pgfsetstrokecolor{currentstroke}%
\pgfsetstrokeopacity{0.000000}%
\pgfsetdash{}{0pt}%
\pgfpathmoveto{\pgfqpoint{2.559053in}{1.941032in}}%
\pgfpathlineto{\pgfqpoint{2.567989in}{1.941032in}}%
\pgfpathlineto{\pgfqpoint{2.567989in}{2.338646in}}%
\pgfpathlineto{\pgfqpoint{2.559053in}{2.338646in}}%
\pgfpathlineto{\pgfqpoint{2.559053in}{1.941032in}}%
\pgfpathclose%
\pgfusepath{fill}%
\end{pgfscope}%
\begin{pgfscope}%
\pgfpathrectangle{\pgfqpoint{0.697024in}{0.857143in}}{\pgfqpoint{2.627103in}{1.813434in}}%
\pgfusepath{clip}%
\pgfsetbuttcap%
\pgfsetmiterjoin%
\definecolor{currentfill}{rgb}{0.302379,0.450282,0.300122}%
\pgfsetfillcolor{currentfill}%
\pgfsetlinewidth{0.000000pt}%
\definecolor{currentstroke}{rgb}{0.000000,0.000000,0.000000}%
\pgfsetstrokecolor{currentstroke}%
\pgfsetstrokeopacity{0.000000}%
\pgfsetdash{}{0pt}%
\pgfpathmoveto{\pgfqpoint{2.570223in}{1.925871in}}%
\pgfpathlineto{\pgfqpoint{2.579160in}{1.925871in}}%
\pgfpathlineto{\pgfqpoint{2.579160in}{2.306020in}}%
\pgfpathlineto{\pgfqpoint{2.570223in}{2.306020in}}%
\pgfpathlineto{\pgfqpoint{2.570223in}{1.925871in}}%
\pgfpathclose%
\pgfusepath{fill}%
\end{pgfscope}%
\begin{pgfscope}%
\pgfpathrectangle{\pgfqpoint{0.697024in}{0.857143in}}{\pgfqpoint{2.627103in}{1.813434in}}%
\pgfusepath{clip}%
\pgfsetbuttcap%
\pgfsetmiterjoin%
\definecolor{currentfill}{rgb}{0.302379,0.450282,0.300122}%
\pgfsetfillcolor{currentfill}%
\pgfsetlinewidth{0.000000pt}%
\definecolor{currentstroke}{rgb}{0.000000,0.000000,0.000000}%
\pgfsetstrokecolor{currentstroke}%
\pgfsetstrokeopacity{0.000000}%
\pgfsetdash{}{0pt}%
\pgfpathmoveto{\pgfqpoint{2.581394in}{1.924111in}}%
\pgfpathlineto{\pgfqpoint{2.590330in}{1.924111in}}%
\pgfpathlineto{\pgfqpoint{2.590330in}{2.317750in}}%
\pgfpathlineto{\pgfqpoint{2.581394in}{2.317750in}}%
\pgfpathlineto{\pgfqpoint{2.581394in}{1.924111in}}%
\pgfpathclose%
\pgfusepath{fill}%
\end{pgfscope}%
\begin{pgfscope}%
\pgfpathrectangle{\pgfqpoint{0.697024in}{0.857143in}}{\pgfqpoint{2.627103in}{1.813434in}}%
\pgfusepath{clip}%
\pgfsetbuttcap%
\pgfsetmiterjoin%
\definecolor{currentfill}{rgb}{0.302379,0.450282,0.300122}%
\pgfsetfillcolor{currentfill}%
\pgfsetlinewidth{0.000000pt}%
\definecolor{currentstroke}{rgb}{0.000000,0.000000,0.000000}%
\pgfsetstrokecolor{currentstroke}%
\pgfsetstrokeopacity{0.000000}%
\pgfsetdash{}{0pt}%
\pgfpathmoveto{\pgfqpoint{2.592564in}{1.914299in}}%
\pgfpathlineto{\pgfqpoint{2.601501in}{1.914299in}}%
\pgfpathlineto{\pgfqpoint{2.601501in}{2.271715in}}%
\pgfpathlineto{\pgfqpoint{2.592564in}{2.271715in}}%
\pgfpathlineto{\pgfqpoint{2.592564in}{1.914299in}}%
\pgfpathclose%
\pgfusepath{fill}%
\end{pgfscope}%
\begin{pgfscope}%
\pgfpathrectangle{\pgfqpoint{0.697024in}{0.857143in}}{\pgfqpoint{2.627103in}{1.813434in}}%
\pgfusepath{clip}%
\pgfsetbuttcap%
\pgfsetmiterjoin%
\definecolor{currentfill}{rgb}{0.302379,0.450282,0.300122}%
\pgfsetfillcolor{currentfill}%
\pgfsetlinewidth{0.000000pt}%
\definecolor{currentstroke}{rgb}{0.000000,0.000000,0.000000}%
\pgfsetstrokecolor{currentstroke}%
\pgfsetstrokeopacity{0.000000}%
\pgfsetdash{}{0pt}%
\pgfpathmoveto{\pgfqpoint{2.603735in}{1.896998in}}%
\pgfpathlineto{\pgfqpoint{2.612672in}{1.896998in}}%
\pgfpathlineto{\pgfqpoint{2.612672in}{2.289466in}}%
\pgfpathlineto{\pgfqpoint{2.603735in}{2.289466in}}%
\pgfpathlineto{\pgfqpoint{2.603735in}{1.896998in}}%
\pgfpathclose%
\pgfusepath{fill}%
\end{pgfscope}%
\begin{pgfscope}%
\pgfpathrectangle{\pgfqpoint{0.697024in}{0.857143in}}{\pgfqpoint{2.627103in}{1.813434in}}%
\pgfusepath{clip}%
\pgfsetbuttcap%
\pgfsetmiterjoin%
\definecolor{currentfill}{rgb}{0.302379,0.450282,0.300122}%
\pgfsetfillcolor{currentfill}%
\pgfsetlinewidth{0.000000pt}%
\definecolor{currentstroke}{rgb}{0.000000,0.000000,0.000000}%
\pgfsetstrokecolor{currentstroke}%
\pgfsetstrokeopacity{0.000000}%
\pgfsetdash{}{0pt}%
\pgfpathmoveto{\pgfqpoint{2.614906in}{1.995964in}}%
\pgfpathlineto{\pgfqpoint{2.623842in}{1.995964in}}%
\pgfpathlineto{\pgfqpoint{2.623842in}{2.306475in}}%
\pgfpathlineto{\pgfqpoint{2.614906in}{2.306475in}}%
\pgfpathlineto{\pgfqpoint{2.614906in}{1.995964in}}%
\pgfpathclose%
\pgfusepath{fill}%
\end{pgfscope}%
\begin{pgfscope}%
\pgfpathrectangle{\pgfqpoint{0.697024in}{0.857143in}}{\pgfqpoint{2.627103in}{1.813434in}}%
\pgfusepath{clip}%
\pgfsetbuttcap%
\pgfsetmiterjoin%
\definecolor{currentfill}{rgb}{0.302379,0.450282,0.300122}%
\pgfsetfillcolor{currentfill}%
\pgfsetlinewidth{0.000000pt}%
\definecolor{currentstroke}{rgb}{0.000000,0.000000,0.000000}%
\pgfsetstrokecolor{currentstroke}%
\pgfsetstrokeopacity{0.000000}%
\pgfsetdash{}{0pt}%
\pgfpathmoveto{\pgfqpoint{2.626076in}{1.899544in}}%
\pgfpathlineto{\pgfqpoint{2.635013in}{1.899544in}}%
\pgfpathlineto{\pgfqpoint{2.635013in}{2.249544in}}%
\pgfpathlineto{\pgfqpoint{2.626076in}{2.249544in}}%
\pgfpathlineto{\pgfqpoint{2.626076in}{1.899544in}}%
\pgfpathclose%
\pgfusepath{fill}%
\end{pgfscope}%
\begin{pgfscope}%
\pgfpathrectangle{\pgfqpoint{0.697024in}{0.857143in}}{\pgfqpoint{2.627103in}{1.813434in}}%
\pgfusepath{clip}%
\pgfsetbuttcap%
\pgfsetmiterjoin%
\definecolor{currentfill}{rgb}{0.302379,0.450282,0.300122}%
\pgfsetfillcolor{currentfill}%
\pgfsetlinewidth{0.000000pt}%
\definecolor{currentstroke}{rgb}{0.000000,0.000000,0.000000}%
\pgfsetstrokecolor{currentstroke}%
\pgfsetstrokeopacity{0.000000}%
\pgfsetdash{}{0pt}%
\pgfpathmoveto{\pgfqpoint{2.637247in}{1.939508in}}%
\pgfpathlineto{\pgfqpoint{2.646183in}{1.939508in}}%
\pgfpathlineto{\pgfqpoint{2.646183in}{2.249603in}}%
\pgfpathlineto{\pgfqpoint{2.637247in}{2.249603in}}%
\pgfpathlineto{\pgfqpoint{2.637247in}{1.939508in}}%
\pgfpathclose%
\pgfusepath{fill}%
\end{pgfscope}%
\begin{pgfscope}%
\pgfpathrectangle{\pgfqpoint{0.697024in}{0.857143in}}{\pgfqpoint{2.627103in}{1.813434in}}%
\pgfusepath{clip}%
\pgfsetbuttcap%
\pgfsetmiterjoin%
\definecolor{currentfill}{rgb}{0.302379,0.450282,0.300122}%
\pgfsetfillcolor{currentfill}%
\pgfsetlinewidth{0.000000pt}%
\definecolor{currentstroke}{rgb}{0.000000,0.000000,0.000000}%
\pgfsetstrokecolor{currentstroke}%
\pgfsetstrokeopacity{0.000000}%
\pgfsetdash{}{0pt}%
\pgfpathmoveto{\pgfqpoint{2.648417in}{1.994920in}}%
\pgfpathlineto{\pgfqpoint{2.657354in}{1.994920in}}%
\pgfpathlineto{\pgfqpoint{2.657354in}{2.250273in}}%
\pgfpathlineto{\pgfqpoint{2.648417in}{2.250273in}}%
\pgfpathlineto{\pgfqpoint{2.648417in}{1.994920in}}%
\pgfpathclose%
\pgfusepath{fill}%
\end{pgfscope}%
\begin{pgfscope}%
\pgfpathrectangle{\pgfqpoint{0.697024in}{0.857143in}}{\pgfqpoint{2.627103in}{1.813434in}}%
\pgfusepath{clip}%
\pgfsetbuttcap%
\pgfsetmiterjoin%
\definecolor{currentfill}{rgb}{0.302379,0.450282,0.300122}%
\pgfsetfillcolor{currentfill}%
\pgfsetlinewidth{0.000000pt}%
\definecolor{currentstroke}{rgb}{0.000000,0.000000,0.000000}%
\pgfsetstrokecolor{currentstroke}%
\pgfsetstrokeopacity{0.000000}%
\pgfsetdash{}{0pt}%
\pgfpathmoveto{\pgfqpoint{2.659588in}{1.998017in}}%
\pgfpathlineto{\pgfqpoint{2.668525in}{1.998017in}}%
\pgfpathlineto{\pgfqpoint{2.668525in}{2.252971in}}%
\pgfpathlineto{\pgfqpoint{2.659588in}{2.252971in}}%
\pgfpathlineto{\pgfqpoint{2.659588in}{1.998017in}}%
\pgfpathclose%
\pgfusepath{fill}%
\end{pgfscope}%
\begin{pgfscope}%
\pgfpathrectangle{\pgfqpoint{0.697024in}{0.857143in}}{\pgfqpoint{2.627103in}{1.813434in}}%
\pgfusepath{clip}%
\pgfsetbuttcap%
\pgfsetmiterjoin%
\definecolor{currentfill}{rgb}{0.302379,0.450282,0.300122}%
\pgfsetfillcolor{currentfill}%
\pgfsetlinewidth{0.000000pt}%
\definecolor{currentstroke}{rgb}{0.000000,0.000000,0.000000}%
\pgfsetstrokecolor{currentstroke}%
\pgfsetstrokeopacity{0.000000}%
\pgfsetdash{}{0pt}%
\pgfpathmoveto{\pgfqpoint{2.670759in}{2.017368in}}%
\pgfpathlineto{\pgfqpoint{2.679695in}{2.017368in}}%
\pgfpathlineto{\pgfqpoint{2.679695in}{2.222230in}}%
\pgfpathlineto{\pgfqpoint{2.670759in}{2.222230in}}%
\pgfpathlineto{\pgfqpoint{2.670759in}{2.017368in}}%
\pgfpathclose%
\pgfusepath{fill}%
\end{pgfscope}%
\begin{pgfscope}%
\pgfpathrectangle{\pgfqpoint{0.697024in}{0.857143in}}{\pgfqpoint{2.627103in}{1.813434in}}%
\pgfusepath{clip}%
\pgfsetbuttcap%
\pgfsetmiterjoin%
\definecolor{currentfill}{rgb}{0.302379,0.450282,0.300122}%
\pgfsetfillcolor{currentfill}%
\pgfsetlinewidth{0.000000pt}%
\definecolor{currentstroke}{rgb}{0.000000,0.000000,0.000000}%
\pgfsetstrokecolor{currentstroke}%
\pgfsetstrokeopacity{0.000000}%
\pgfsetdash{}{0pt}%
\pgfpathmoveto{\pgfqpoint{2.681929in}{2.020398in}}%
\pgfpathlineto{\pgfqpoint{2.690866in}{2.020398in}}%
\pgfpathlineto{\pgfqpoint{2.690866in}{2.158131in}}%
\pgfpathlineto{\pgfqpoint{2.681929in}{2.158131in}}%
\pgfpathlineto{\pgfqpoint{2.681929in}{2.020398in}}%
\pgfpathclose%
\pgfusepath{fill}%
\end{pgfscope}%
\begin{pgfscope}%
\pgfpathrectangle{\pgfqpoint{0.697024in}{0.857143in}}{\pgfqpoint{2.627103in}{1.813434in}}%
\pgfusepath{clip}%
\pgfsetbuttcap%
\pgfsetmiterjoin%
\definecolor{currentfill}{rgb}{0.302379,0.450282,0.300122}%
\pgfsetfillcolor{currentfill}%
\pgfsetlinewidth{0.000000pt}%
\definecolor{currentstroke}{rgb}{0.000000,0.000000,0.000000}%
\pgfsetstrokecolor{currentstroke}%
\pgfsetstrokeopacity{0.000000}%
\pgfsetdash{}{0pt}%
\pgfpathmoveto{\pgfqpoint{2.693100in}{1.931966in}}%
\pgfpathlineto{\pgfqpoint{2.702036in}{1.931966in}}%
\pgfpathlineto{\pgfqpoint{2.702036in}{2.010036in}}%
\pgfpathlineto{\pgfqpoint{2.693100in}{2.010036in}}%
\pgfpathlineto{\pgfqpoint{2.693100in}{1.931966in}}%
\pgfpathclose%
\pgfusepath{fill}%
\end{pgfscope}%
\begin{pgfscope}%
\pgfpathrectangle{\pgfqpoint{0.697024in}{0.857143in}}{\pgfqpoint{2.627103in}{1.813434in}}%
\pgfusepath{clip}%
\pgfsetbuttcap%
\pgfsetmiterjoin%
\definecolor{currentfill}{rgb}{0.302379,0.450282,0.300122}%
\pgfsetfillcolor{currentfill}%
\pgfsetlinewidth{0.000000pt}%
\definecolor{currentstroke}{rgb}{0.000000,0.000000,0.000000}%
\pgfsetstrokecolor{currentstroke}%
\pgfsetstrokeopacity{0.000000}%
\pgfsetdash{}{0pt}%
\pgfpathmoveto{\pgfqpoint{2.704270in}{1.847071in}}%
\pgfpathlineto{\pgfqpoint{2.713207in}{1.847071in}}%
\pgfpathlineto{\pgfqpoint{2.713207in}{1.723040in}}%
\pgfpathlineto{\pgfqpoint{2.704270in}{1.723040in}}%
\pgfpathlineto{\pgfqpoint{2.704270in}{1.847071in}}%
\pgfpathclose%
\pgfusepath{fill}%
\end{pgfscope}%
\begin{pgfscope}%
\pgfpathrectangle{\pgfqpoint{0.697024in}{0.857143in}}{\pgfqpoint{2.627103in}{1.813434in}}%
\pgfusepath{clip}%
\pgfsetbuttcap%
\pgfsetmiterjoin%
\definecolor{currentfill}{rgb}{0.302379,0.450282,0.300122}%
\pgfsetfillcolor{currentfill}%
\pgfsetlinewidth{0.000000pt}%
\definecolor{currentstroke}{rgb}{0.000000,0.000000,0.000000}%
\pgfsetstrokecolor{currentstroke}%
\pgfsetstrokeopacity{0.000000}%
\pgfsetdash{}{0pt}%
\pgfpathmoveto{\pgfqpoint{2.715441in}{1.847071in}}%
\pgfpathlineto{\pgfqpoint{2.724378in}{1.847071in}}%
\pgfpathlineto{\pgfqpoint{2.724378in}{1.524854in}}%
\pgfpathlineto{\pgfqpoint{2.715441in}{1.524854in}}%
\pgfpathlineto{\pgfqpoint{2.715441in}{1.847071in}}%
\pgfpathclose%
\pgfusepath{fill}%
\end{pgfscope}%
\begin{pgfscope}%
\pgfpathrectangle{\pgfqpoint{0.697024in}{0.857143in}}{\pgfqpoint{2.627103in}{1.813434in}}%
\pgfusepath{clip}%
\pgfsetbuttcap%
\pgfsetmiterjoin%
\definecolor{currentfill}{rgb}{0.302379,0.450282,0.300122}%
\pgfsetfillcolor{currentfill}%
\pgfsetlinewidth{0.000000pt}%
\definecolor{currentstroke}{rgb}{0.000000,0.000000,0.000000}%
\pgfsetstrokecolor{currentstroke}%
\pgfsetstrokeopacity{0.000000}%
\pgfsetdash{}{0pt}%
\pgfpathmoveto{\pgfqpoint{2.726612in}{1.847071in}}%
\pgfpathlineto{\pgfqpoint{2.735548in}{1.847071in}}%
\pgfpathlineto{\pgfqpoint{2.735548in}{1.394884in}}%
\pgfpathlineto{\pgfqpoint{2.726612in}{1.394884in}}%
\pgfpathlineto{\pgfqpoint{2.726612in}{1.847071in}}%
\pgfpathclose%
\pgfusepath{fill}%
\end{pgfscope}%
\begin{pgfscope}%
\pgfpathrectangle{\pgfqpoint{0.697024in}{0.857143in}}{\pgfqpoint{2.627103in}{1.813434in}}%
\pgfusepath{clip}%
\pgfsetbuttcap%
\pgfsetmiterjoin%
\definecolor{currentfill}{rgb}{0.302379,0.450282,0.300122}%
\pgfsetfillcolor{currentfill}%
\pgfsetlinewidth{0.000000pt}%
\definecolor{currentstroke}{rgb}{0.000000,0.000000,0.000000}%
\pgfsetstrokecolor{currentstroke}%
\pgfsetstrokeopacity{0.000000}%
\pgfsetdash{}{0pt}%
\pgfpathmoveto{\pgfqpoint{2.737782in}{1.847071in}}%
\pgfpathlineto{\pgfqpoint{2.746719in}{1.847071in}}%
\pgfpathlineto{\pgfqpoint{2.746719in}{1.425271in}}%
\pgfpathlineto{\pgfqpoint{2.737782in}{1.425271in}}%
\pgfpathlineto{\pgfqpoint{2.737782in}{1.847071in}}%
\pgfpathclose%
\pgfusepath{fill}%
\end{pgfscope}%
\begin{pgfscope}%
\pgfpathrectangle{\pgfqpoint{0.697024in}{0.857143in}}{\pgfqpoint{2.627103in}{1.813434in}}%
\pgfusepath{clip}%
\pgfsetbuttcap%
\pgfsetmiterjoin%
\definecolor{currentfill}{rgb}{0.302379,0.450282,0.300122}%
\pgfsetfillcolor{currentfill}%
\pgfsetlinewidth{0.000000pt}%
\definecolor{currentstroke}{rgb}{0.000000,0.000000,0.000000}%
\pgfsetstrokecolor{currentstroke}%
\pgfsetstrokeopacity{0.000000}%
\pgfsetdash{}{0pt}%
\pgfpathmoveto{\pgfqpoint{2.748953in}{1.847071in}}%
\pgfpathlineto{\pgfqpoint{2.757889in}{1.847071in}}%
\pgfpathlineto{\pgfqpoint{2.757889in}{1.375225in}}%
\pgfpathlineto{\pgfqpoint{2.748953in}{1.375225in}}%
\pgfpathlineto{\pgfqpoint{2.748953in}{1.847071in}}%
\pgfpathclose%
\pgfusepath{fill}%
\end{pgfscope}%
\begin{pgfscope}%
\pgfpathrectangle{\pgfqpoint{0.697024in}{0.857143in}}{\pgfqpoint{2.627103in}{1.813434in}}%
\pgfusepath{clip}%
\pgfsetbuttcap%
\pgfsetmiterjoin%
\definecolor{currentfill}{rgb}{0.302379,0.450282,0.300122}%
\pgfsetfillcolor{currentfill}%
\pgfsetlinewidth{0.000000pt}%
\definecolor{currentstroke}{rgb}{0.000000,0.000000,0.000000}%
\pgfsetstrokecolor{currentstroke}%
\pgfsetstrokeopacity{0.000000}%
\pgfsetdash{}{0pt}%
\pgfpathmoveto{\pgfqpoint{2.760124in}{1.847071in}}%
\pgfpathlineto{\pgfqpoint{2.769060in}{1.847071in}}%
\pgfpathlineto{\pgfqpoint{2.769060in}{1.343198in}}%
\pgfpathlineto{\pgfqpoint{2.760124in}{1.343198in}}%
\pgfpathlineto{\pgfqpoint{2.760124in}{1.847071in}}%
\pgfpathclose%
\pgfusepath{fill}%
\end{pgfscope}%
\begin{pgfscope}%
\pgfpathrectangle{\pgfqpoint{0.697024in}{0.857143in}}{\pgfqpoint{2.627103in}{1.813434in}}%
\pgfusepath{clip}%
\pgfsetbuttcap%
\pgfsetmiterjoin%
\definecolor{currentfill}{rgb}{0.302379,0.450282,0.300122}%
\pgfsetfillcolor{currentfill}%
\pgfsetlinewidth{0.000000pt}%
\definecolor{currentstroke}{rgb}{0.000000,0.000000,0.000000}%
\pgfsetstrokecolor{currentstroke}%
\pgfsetstrokeopacity{0.000000}%
\pgfsetdash{}{0pt}%
\pgfpathmoveto{\pgfqpoint{2.771294in}{1.847071in}}%
\pgfpathlineto{\pgfqpoint{2.780231in}{1.847071in}}%
\pgfpathlineto{\pgfqpoint{2.780231in}{1.392840in}}%
\pgfpathlineto{\pgfqpoint{2.771294in}{1.392840in}}%
\pgfpathlineto{\pgfqpoint{2.771294in}{1.847071in}}%
\pgfpathclose%
\pgfusepath{fill}%
\end{pgfscope}%
\begin{pgfscope}%
\pgfpathrectangle{\pgfqpoint{0.697024in}{0.857143in}}{\pgfqpoint{2.627103in}{1.813434in}}%
\pgfusepath{clip}%
\pgfsetbuttcap%
\pgfsetmiterjoin%
\definecolor{currentfill}{rgb}{0.302379,0.450282,0.300122}%
\pgfsetfillcolor{currentfill}%
\pgfsetlinewidth{0.000000pt}%
\definecolor{currentstroke}{rgb}{0.000000,0.000000,0.000000}%
\pgfsetstrokecolor{currentstroke}%
\pgfsetstrokeopacity{0.000000}%
\pgfsetdash{}{0pt}%
\pgfpathmoveto{\pgfqpoint{2.782465in}{1.847071in}}%
\pgfpathlineto{\pgfqpoint{2.791401in}{1.847071in}}%
\pgfpathlineto{\pgfqpoint{2.791401in}{1.373620in}}%
\pgfpathlineto{\pgfqpoint{2.782465in}{1.373620in}}%
\pgfpathlineto{\pgfqpoint{2.782465in}{1.847071in}}%
\pgfpathclose%
\pgfusepath{fill}%
\end{pgfscope}%
\begin{pgfscope}%
\pgfpathrectangle{\pgfqpoint{0.697024in}{0.857143in}}{\pgfqpoint{2.627103in}{1.813434in}}%
\pgfusepath{clip}%
\pgfsetbuttcap%
\pgfsetmiterjoin%
\definecolor{currentfill}{rgb}{0.302379,0.450282,0.300122}%
\pgfsetfillcolor{currentfill}%
\pgfsetlinewidth{0.000000pt}%
\definecolor{currentstroke}{rgb}{0.000000,0.000000,0.000000}%
\pgfsetstrokecolor{currentstroke}%
\pgfsetstrokeopacity{0.000000}%
\pgfsetdash{}{0pt}%
\pgfpathmoveto{\pgfqpoint{2.793635in}{1.847071in}}%
\pgfpathlineto{\pgfqpoint{2.802572in}{1.847071in}}%
\pgfpathlineto{\pgfqpoint{2.802572in}{1.344882in}}%
\pgfpathlineto{\pgfqpoint{2.793635in}{1.344882in}}%
\pgfpathlineto{\pgfqpoint{2.793635in}{1.847071in}}%
\pgfpathclose%
\pgfusepath{fill}%
\end{pgfscope}%
\begin{pgfscope}%
\pgfpathrectangle{\pgfqpoint{0.697024in}{0.857143in}}{\pgfqpoint{2.627103in}{1.813434in}}%
\pgfusepath{clip}%
\pgfsetbuttcap%
\pgfsetmiterjoin%
\definecolor{currentfill}{rgb}{0.302379,0.450282,0.300122}%
\pgfsetfillcolor{currentfill}%
\pgfsetlinewidth{0.000000pt}%
\definecolor{currentstroke}{rgb}{0.000000,0.000000,0.000000}%
\pgfsetstrokecolor{currentstroke}%
\pgfsetstrokeopacity{0.000000}%
\pgfsetdash{}{0pt}%
\pgfpathmoveto{\pgfqpoint{2.804806in}{1.847071in}}%
\pgfpathlineto{\pgfqpoint{2.813742in}{1.847071in}}%
\pgfpathlineto{\pgfqpoint{2.813742in}{1.351510in}}%
\pgfpathlineto{\pgfqpoint{2.804806in}{1.351510in}}%
\pgfpathlineto{\pgfqpoint{2.804806in}{1.847071in}}%
\pgfpathclose%
\pgfusepath{fill}%
\end{pgfscope}%
\begin{pgfscope}%
\pgfpathrectangle{\pgfqpoint{0.697024in}{0.857143in}}{\pgfqpoint{2.627103in}{1.813434in}}%
\pgfusepath{clip}%
\pgfsetbuttcap%
\pgfsetmiterjoin%
\definecolor{currentfill}{rgb}{0.302379,0.450282,0.300122}%
\pgfsetfillcolor{currentfill}%
\pgfsetlinewidth{0.000000pt}%
\definecolor{currentstroke}{rgb}{0.000000,0.000000,0.000000}%
\pgfsetstrokecolor{currentstroke}%
\pgfsetstrokeopacity{0.000000}%
\pgfsetdash{}{0pt}%
\pgfpathmoveto{\pgfqpoint{2.815977in}{1.847071in}}%
\pgfpathlineto{\pgfqpoint{2.824913in}{1.847071in}}%
\pgfpathlineto{\pgfqpoint{2.824913in}{1.438439in}}%
\pgfpathlineto{\pgfqpoint{2.815977in}{1.438439in}}%
\pgfpathlineto{\pgfqpoint{2.815977in}{1.847071in}}%
\pgfpathclose%
\pgfusepath{fill}%
\end{pgfscope}%
\begin{pgfscope}%
\pgfpathrectangle{\pgfqpoint{0.697024in}{0.857143in}}{\pgfqpoint{2.627103in}{1.813434in}}%
\pgfusepath{clip}%
\pgfsetbuttcap%
\pgfsetmiterjoin%
\definecolor{currentfill}{rgb}{0.302379,0.450282,0.300122}%
\pgfsetfillcolor{currentfill}%
\pgfsetlinewidth{0.000000pt}%
\definecolor{currentstroke}{rgb}{0.000000,0.000000,0.000000}%
\pgfsetstrokecolor{currentstroke}%
\pgfsetstrokeopacity{0.000000}%
\pgfsetdash{}{0pt}%
\pgfpathmoveto{\pgfqpoint{2.827147in}{1.847071in}}%
\pgfpathlineto{\pgfqpoint{2.836084in}{1.847071in}}%
\pgfpathlineto{\pgfqpoint{2.836084in}{1.376137in}}%
\pgfpathlineto{\pgfqpoint{2.827147in}{1.376137in}}%
\pgfpathlineto{\pgfqpoint{2.827147in}{1.847071in}}%
\pgfpathclose%
\pgfusepath{fill}%
\end{pgfscope}%
\begin{pgfscope}%
\pgfpathrectangle{\pgfqpoint{0.697024in}{0.857143in}}{\pgfqpoint{2.627103in}{1.813434in}}%
\pgfusepath{clip}%
\pgfsetbuttcap%
\pgfsetmiterjoin%
\definecolor{currentfill}{rgb}{0.302379,0.450282,0.300122}%
\pgfsetfillcolor{currentfill}%
\pgfsetlinewidth{0.000000pt}%
\definecolor{currentstroke}{rgb}{0.000000,0.000000,0.000000}%
\pgfsetstrokecolor{currentstroke}%
\pgfsetstrokeopacity{0.000000}%
\pgfsetdash{}{0pt}%
\pgfpathmoveto{\pgfqpoint{2.838318in}{1.847071in}}%
\pgfpathlineto{\pgfqpoint{2.847254in}{1.847071in}}%
\pgfpathlineto{\pgfqpoint{2.847254in}{1.332645in}}%
\pgfpathlineto{\pgfqpoint{2.838318in}{1.332645in}}%
\pgfpathlineto{\pgfqpoint{2.838318in}{1.847071in}}%
\pgfpathclose%
\pgfusepath{fill}%
\end{pgfscope}%
\begin{pgfscope}%
\pgfpathrectangle{\pgfqpoint{0.697024in}{0.857143in}}{\pgfqpoint{2.627103in}{1.813434in}}%
\pgfusepath{clip}%
\pgfsetbuttcap%
\pgfsetmiterjoin%
\definecolor{currentfill}{rgb}{0.302379,0.450282,0.300122}%
\pgfsetfillcolor{currentfill}%
\pgfsetlinewidth{0.000000pt}%
\definecolor{currentstroke}{rgb}{0.000000,0.000000,0.000000}%
\pgfsetstrokecolor{currentstroke}%
\pgfsetstrokeopacity{0.000000}%
\pgfsetdash{}{0pt}%
\pgfpathmoveto{\pgfqpoint{2.849488in}{1.847071in}}%
\pgfpathlineto{\pgfqpoint{2.858425in}{1.847071in}}%
\pgfpathlineto{\pgfqpoint{2.858425in}{1.401482in}}%
\pgfpathlineto{\pgfqpoint{2.849488in}{1.401482in}}%
\pgfpathlineto{\pgfqpoint{2.849488in}{1.847071in}}%
\pgfpathclose%
\pgfusepath{fill}%
\end{pgfscope}%
\begin{pgfscope}%
\pgfpathrectangle{\pgfqpoint{0.697024in}{0.857143in}}{\pgfqpoint{2.627103in}{1.813434in}}%
\pgfusepath{clip}%
\pgfsetbuttcap%
\pgfsetmiterjoin%
\definecolor{currentfill}{rgb}{0.302379,0.450282,0.300122}%
\pgfsetfillcolor{currentfill}%
\pgfsetlinewidth{0.000000pt}%
\definecolor{currentstroke}{rgb}{0.000000,0.000000,0.000000}%
\pgfsetstrokecolor{currentstroke}%
\pgfsetstrokeopacity{0.000000}%
\pgfsetdash{}{0pt}%
\pgfpathmoveto{\pgfqpoint{2.860659in}{1.847071in}}%
\pgfpathlineto{\pgfqpoint{2.869595in}{1.847071in}}%
\pgfpathlineto{\pgfqpoint{2.869595in}{1.376314in}}%
\pgfpathlineto{\pgfqpoint{2.860659in}{1.376314in}}%
\pgfpathlineto{\pgfqpoint{2.860659in}{1.847071in}}%
\pgfpathclose%
\pgfusepath{fill}%
\end{pgfscope}%
\begin{pgfscope}%
\pgfpathrectangle{\pgfqpoint{0.697024in}{0.857143in}}{\pgfqpoint{2.627103in}{1.813434in}}%
\pgfusepath{clip}%
\pgfsetbuttcap%
\pgfsetmiterjoin%
\definecolor{currentfill}{rgb}{0.302379,0.450282,0.300122}%
\pgfsetfillcolor{currentfill}%
\pgfsetlinewidth{0.000000pt}%
\definecolor{currentstroke}{rgb}{0.000000,0.000000,0.000000}%
\pgfsetstrokecolor{currentstroke}%
\pgfsetstrokeopacity{0.000000}%
\pgfsetdash{}{0pt}%
\pgfpathmoveto{\pgfqpoint{2.871830in}{1.847071in}}%
\pgfpathlineto{\pgfqpoint{2.880766in}{1.847071in}}%
\pgfpathlineto{\pgfqpoint{2.880766in}{1.365917in}}%
\pgfpathlineto{\pgfqpoint{2.871830in}{1.365917in}}%
\pgfpathlineto{\pgfqpoint{2.871830in}{1.847071in}}%
\pgfpathclose%
\pgfusepath{fill}%
\end{pgfscope}%
\begin{pgfscope}%
\pgfpathrectangle{\pgfqpoint{0.697024in}{0.857143in}}{\pgfqpoint{2.627103in}{1.813434in}}%
\pgfusepath{clip}%
\pgfsetbuttcap%
\pgfsetmiterjoin%
\definecolor{currentfill}{rgb}{0.302379,0.450282,0.300122}%
\pgfsetfillcolor{currentfill}%
\pgfsetlinewidth{0.000000pt}%
\definecolor{currentstroke}{rgb}{0.000000,0.000000,0.000000}%
\pgfsetstrokecolor{currentstroke}%
\pgfsetstrokeopacity{0.000000}%
\pgfsetdash{}{0pt}%
\pgfpathmoveto{\pgfqpoint{2.883000in}{1.847071in}}%
\pgfpathlineto{\pgfqpoint{2.891937in}{1.847071in}}%
\pgfpathlineto{\pgfqpoint{2.891937in}{1.375979in}}%
\pgfpathlineto{\pgfqpoint{2.883000in}{1.375979in}}%
\pgfpathlineto{\pgfqpoint{2.883000in}{1.847071in}}%
\pgfpathclose%
\pgfusepath{fill}%
\end{pgfscope}%
\begin{pgfscope}%
\pgfpathrectangle{\pgfqpoint{0.697024in}{0.857143in}}{\pgfqpoint{2.627103in}{1.813434in}}%
\pgfusepath{clip}%
\pgfsetbuttcap%
\pgfsetmiterjoin%
\definecolor{currentfill}{rgb}{0.302379,0.450282,0.300122}%
\pgfsetfillcolor{currentfill}%
\pgfsetlinewidth{0.000000pt}%
\definecolor{currentstroke}{rgb}{0.000000,0.000000,0.000000}%
\pgfsetstrokecolor{currentstroke}%
\pgfsetstrokeopacity{0.000000}%
\pgfsetdash{}{0pt}%
\pgfpathmoveto{\pgfqpoint{2.894171in}{1.847071in}}%
\pgfpathlineto{\pgfqpoint{2.903107in}{1.847071in}}%
\pgfpathlineto{\pgfqpoint{2.903107in}{1.323175in}}%
\pgfpathlineto{\pgfqpoint{2.894171in}{1.323175in}}%
\pgfpathlineto{\pgfqpoint{2.894171in}{1.847071in}}%
\pgfpathclose%
\pgfusepath{fill}%
\end{pgfscope}%
\begin{pgfscope}%
\pgfpathrectangle{\pgfqpoint{0.697024in}{0.857143in}}{\pgfqpoint{2.627103in}{1.813434in}}%
\pgfusepath{clip}%
\pgfsetbuttcap%
\pgfsetmiterjoin%
\definecolor{currentfill}{rgb}{0.302379,0.450282,0.300122}%
\pgfsetfillcolor{currentfill}%
\pgfsetlinewidth{0.000000pt}%
\definecolor{currentstroke}{rgb}{0.000000,0.000000,0.000000}%
\pgfsetstrokecolor{currentstroke}%
\pgfsetstrokeopacity{0.000000}%
\pgfsetdash{}{0pt}%
\pgfpathmoveto{\pgfqpoint{2.905341in}{1.847071in}}%
\pgfpathlineto{\pgfqpoint{2.914278in}{1.847071in}}%
\pgfpathlineto{\pgfqpoint{2.914278in}{1.304297in}}%
\pgfpathlineto{\pgfqpoint{2.905341in}{1.304297in}}%
\pgfpathlineto{\pgfqpoint{2.905341in}{1.847071in}}%
\pgfpathclose%
\pgfusepath{fill}%
\end{pgfscope}%
\begin{pgfscope}%
\pgfpathrectangle{\pgfqpoint{0.697024in}{0.857143in}}{\pgfqpoint{2.627103in}{1.813434in}}%
\pgfusepath{clip}%
\pgfsetbuttcap%
\pgfsetmiterjoin%
\definecolor{currentfill}{rgb}{0.302379,0.450282,0.300122}%
\pgfsetfillcolor{currentfill}%
\pgfsetlinewidth{0.000000pt}%
\definecolor{currentstroke}{rgb}{0.000000,0.000000,0.000000}%
\pgfsetstrokecolor{currentstroke}%
\pgfsetstrokeopacity{0.000000}%
\pgfsetdash{}{0pt}%
\pgfpathmoveto{\pgfqpoint{2.916512in}{1.847071in}}%
\pgfpathlineto{\pgfqpoint{2.925448in}{1.847071in}}%
\pgfpathlineto{\pgfqpoint{2.925448in}{1.314515in}}%
\pgfpathlineto{\pgfqpoint{2.916512in}{1.314515in}}%
\pgfpathlineto{\pgfqpoint{2.916512in}{1.847071in}}%
\pgfpathclose%
\pgfusepath{fill}%
\end{pgfscope}%
\begin{pgfscope}%
\pgfpathrectangle{\pgfqpoint{0.697024in}{0.857143in}}{\pgfqpoint{2.627103in}{1.813434in}}%
\pgfusepath{clip}%
\pgfsetbuttcap%
\pgfsetmiterjoin%
\definecolor{currentfill}{rgb}{0.302379,0.450282,0.300122}%
\pgfsetfillcolor{currentfill}%
\pgfsetlinewidth{0.000000pt}%
\definecolor{currentstroke}{rgb}{0.000000,0.000000,0.000000}%
\pgfsetstrokecolor{currentstroke}%
\pgfsetstrokeopacity{0.000000}%
\pgfsetdash{}{0pt}%
\pgfpathmoveto{\pgfqpoint{2.927683in}{1.847071in}}%
\pgfpathlineto{\pgfqpoint{2.936619in}{1.847071in}}%
\pgfpathlineto{\pgfqpoint{2.936619in}{1.300121in}}%
\pgfpathlineto{\pgfqpoint{2.927683in}{1.300121in}}%
\pgfpathlineto{\pgfqpoint{2.927683in}{1.847071in}}%
\pgfpathclose%
\pgfusepath{fill}%
\end{pgfscope}%
\begin{pgfscope}%
\pgfpathrectangle{\pgfqpoint{0.697024in}{0.857143in}}{\pgfqpoint{2.627103in}{1.813434in}}%
\pgfusepath{clip}%
\pgfsetbuttcap%
\pgfsetmiterjoin%
\definecolor{currentfill}{rgb}{0.302379,0.450282,0.300122}%
\pgfsetfillcolor{currentfill}%
\pgfsetlinewidth{0.000000pt}%
\definecolor{currentstroke}{rgb}{0.000000,0.000000,0.000000}%
\pgfsetstrokecolor{currentstroke}%
\pgfsetstrokeopacity{0.000000}%
\pgfsetdash{}{0pt}%
\pgfpathmoveto{\pgfqpoint{2.938853in}{1.847071in}}%
\pgfpathlineto{\pgfqpoint{2.947790in}{1.847071in}}%
\pgfpathlineto{\pgfqpoint{2.947790in}{1.251291in}}%
\pgfpathlineto{\pgfqpoint{2.938853in}{1.251291in}}%
\pgfpathlineto{\pgfqpoint{2.938853in}{1.847071in}}%
\pgfpathclose%
\pgfusepath{fill}%
\end{pgfscope}%
\begin{pgfscope}%
\pgfpathrectangle{\pgfqpoint{0.697024in}{0.857143in}}{\pgfqpoint{2.627103in}{1.813434in}}%
\pgfusepath{clip}%
\pgfsetbuttcap%
\pgfsetmiterjoin%
\definecolor{currentfill}{rgb}{0.302379,0.450282,0.300122}%
\pgfsetfillcolor{currentfill}%
\pgfsetlinewidth{0.000000pt}%
\definecolor{currentstroke}{rgb}{0.000000,0.000000,0.000000}%
\pgfsetstrokecolor{currentstroke}%
\pgfsetstrokeopacity{0.000000}%
\pgfsetdash{}{0pt}%
\pgfpathmoveto{\pgfqpoint{2.950024in}{1.847071in}}%
\pgfpathlineto{\pgfqpoint{2.958960in}{1.847071in}}%
\pgfpathlineto{\pgfqpoint{2.958960in}{1.308135in}}%
\pgfpathlineto{\pgfqpoint{2.950024in}{1.308135in}}%
\pgfpathlineto{\pgfqpoint{2.950024in}{1.847071in}}%
\pgfpathclose%
\pgfusepath{fill}%
\end{pgfscope}%
\begin{pgfscope}%
\pgfpathrectangle{\pgfqpoint{0.697024in}{0.857143in}}{\pgfqpoint{2.627103in}{1.813434in}}%
\pgfusepath{clip}%
\pgfsetbuttcap%
\pgfsetmiterjoin%
\definecolor{currentfill}{rgb}{0.302379,0.450282,0.300122}%
\pgfsetfillcolor{currentfill}%
\pgfsetlinewidth{0.000000pt}%
\definecolor{currentstroke}{rgb}{0.000000,0.000000,0.000000}%
\pgfsetstrokecolor{currentstroke}%
\pgfsetstrokeopacity{0.000000}%
\pgfsetdash{}{0pt}%
\pgfpathmoveto{\pgfqpoint{2.961194in}{1.847071in}}%
\pgfpathlineto{\pgfqpoint{2.970131in}{1.847071in}}%
\pgfpathlineto{\pgfqpoint{2.970131in}{1.311019in}}%
\pgfpathlineto{\pgfqpoint{2.961194in}{1.311019in}}%
\pgfpathlineto{\pgfqpoint{2.961194in}{1.847071in}}%
\pgfpathclose%
\pgfusepath{fill}%
\end{pgfscope}%
\begin{pgfscope}%
\pgfpathrectangle{\pgfqpoint{0.697024in}{0.857143in}}{\pgfqpoint{2.627103in}{1.813434in}}%
\pgfusepath{clip}%
\pgfsetbuttcap%
\pgfsetmiterjoin%
\definecolor{currentfill}{rgb}{0.302379,0.450282,0.300122}%
\pgfsetfillcolor{currentfill}%
\pgfsetlinewidth{0.000000pt}%
\definecolor{currentstroke}{rgb}{0.000000,0.000000,0.000000}%
\pgfsetstrokecolor{currentstroke}%
\pgfsetstrokeopacity{0.000000}%
\pgfsetdash{}{0pt}%
\pgfpathmoveto{\pgfqpoint{2.972365in}{1.847071in}}%
\pgfpathlineto{\pgfqpoint{2.981301in}{1.847071in}}%
\pgfpathlineto{\pgfqpoint{2.981301in}{1.303696in}}%
\pgfpathlineto{\pgfqpoint{2.972365in}{1.303696in}}%
\pgfpathlineto{\pgfqpoint{2.972365in}{1.847071in}}%
\pgfpathclose%
\pgfusepath{fill}%
\end{pgfscope}%
\begin{pgfscope}%
\pgfpathrectangle{\pgfqpoint{0.697024in}{0.857143in}}{\pgfqpoint{2.627103in}{1.813434in}}%
\pgfusepath{clip}%
\pgfsetbuttcap%
\pgfsetmiterjoin%
\definecolor{currentfill}{rgb}{0.302379,0.450282,0.300122}%
\pgfsetfillcolor{currentfill}%
\pgfsetlinewidth{0.000000pt}%
\definecolor{currentstroke}{rgb}{0.000000,0.000000,0.000000}%
\pgfsetstrokecolor{currentstroke}%
\pgfsetstrokeopacity{0.000000}%
\pgfsetdash{}{0pt}%
\pgfpathmoveto{\pgfqpoint{2.983536in}{1.847071in}}%
\pgfpathlineto{\pgfqpoint{2.992472in}{1.847071in}}%
\pgfpathlineto{\pgfqpoint{2.992472in}{1.252522in}}%
\pgfpathlineto{\pgfqpoint{2.983536in}{1.252522in}}%
\pgfpathlineto{\pgfqpoint{2.983536in}{1.847071in}}%
\pgfpathclose%
\pgfusepath{fill}%
\end{pgfscope}%
\begin{pgfscope}%
\pgfpathrectangle{\pgfqpoint{0.697024in}{0.857143in}}{\pgfqpoint{2.627103in}{1.813434in}}%
\pgfusepath{clip}%
\pgfsetbuttcap%
\pgfsetmiterjoin%
\definecolor{currentfill}{rgb}{0.302379,0.450282,0.300122}%
\pgfsetfillcolor{currentfill}%
\pgfsetlinewidth{0.000000pt}%
\definecolor{currentstroke}{rgb}{0.000000,0.000000,0.000000}%
\pgfsetstrokecolor{currentstroke}%
\pgfsetstrokeopacity{0.000000}%
\pgfsetdash{}{0pt}%
\pgfpathmoveto{\pgfqpoint{2.994706in}{1.847071in}}%
\pgfpathlineto{\pgfqpoint{3.003643in}{1.847071in}}%
\pgfpathlineto{\pgfqpoint{3.003643in}{1.295211in}}%
\pgfpathlineto{\pgfqpoint{2.994706in}{1.295211in}}%
\pgfpathlineto{\pgfqpoint{2.994706in}{1.847071in}}%
\pgfpathclose%
\pgfusepath{fill}%
\end{pgfscope}%
\begin{pgfscope}%
\pgfpathrectangle{\pgfqpoint{0.697024in}{0.857143in}}{\pgfqpoint{2.627103in}{1.813434in}}%
\pgfusepath{clip}%
\pgfsetbuttcap%
\pgfsetmiterjoin%
\definecolor{currentfill}{rgb}{0.302379,0.450282,0.300122}%
\pgfsetfillcolor{currentfill}%
\pgfsetlinewidth{0.000000pt}%
\definecolor{currentstroke}{rgb}{0.000000,0.000000,0.000000}%
\pgfsetstrokecolor{currentstroke}%
\pgfsetstrokeopacity{0.000000}%
\pgfsetdash{}{0pt}%
\pgfpathmoveto{\pgfqpoint{3.005877in}{1.847071in}}%
\pgfpathlineto{\pgfqpoint{3.014813in}{1.847071in}}%
\pgfpathlineto{\pgfqpoint{3.014813in}{1.281940in}}%
\pgfpathlineto{\pgfqpoint{3.005877in}{1.281940in}}%
\pgfpathlineto{\pgfqpoint{3.005877in}{1.847071in}}%
\pgfpathclose%
\pgfusepath{fill}%
\end{pgfscope}%
\begin{pgfscope}%
\pgfpathrectangle{\pgfqpoint{0.697024in}{0.857143in}}{\pgfqpoint{2.627103in}{1.813434in}}%
\pgfusepath{clip}%
\pgfsetbuttcap%
\pgfsetmiterjoin%
\definecolor{currentfill}{rgb}{0.302379,0.450282,0.300122}%
\pgfsetfillcolor{currentfill}%
\pgfsetlinewidth{0.000000pt}%
\definecolor{currentstroke}{rgb}{0.000000,0.000000,0.000000}%
\pgfsetstrokecolor{currentstroke}%
\pgfsetstrokeopacity{0.000000}%
\pgfsetdash{}{0pt}%
\pgfpathmoveto{\pgfqpoint{3.017047in}{1.820073in}}%
\pgfpathlineto{\pgfqpoint{3.025984in}{1.820073in}}%
\pgfpathlineto{\pgfqpoint{3.025984in}{1.146754in}}%
\pgfpathlineto{\pgfqpoint{3.017047in}{1.146754in}}%
\pgfpathlineto{\pgfqpoint{3.017047in}{1.820073in}}%
\pgfpathclose%
\pgfusepath{fill}%
\end{pgfscope}%
\begin{pgfscope}%
\pgfpathrectangle{\pgfqpoint{0.697024in}{0.857143in}}{\pgfqpoint{2.627103in}{1.813434in}}%
\pgfusepath{clip}%
\pgfsetbuttcap%
\pgfsetmiterjoin%
\definecolor{currentfill}{rgb}{0.302379,0.450282,0.300122}%
\pgfsetfillcolor{currentfill}%
\pgfsetlinewidth{0.000000pt}%
\definecolor{currentstroke}{rgb}{0.000000,0.000000,0.000000}%
\pgfsetstrokecolor{currentstroke}%
\pgfsetstrokeopacity{0.000000}%
\pgfsetdash{}{0pt}%
\pgfpathmoveto{\pgfqpoint{3.028218in}{1.821362in}}%
\pgfpathlineto{\pgfqpoint{3.037155in}{1.821362in}}%
\pgfpathlineto{\pgfqpoint{3.037155in}{1.139195in}}%
\pgfpathlineto{\pgfqpoint{3.028218in}{1.139195in}}%
\pgfpathlineto{\pgfqpoint{3.028218in}{1.821362in}}%
\pgfpathclose%
\pgfusepath{fill}%
\end{pgfscope}%
\begin{pgfscope}%
\pgfpathrectangle{\pgfqpoint{0.697024in}{0.857143in}}{\pgfqpoint{2.627103in}{1.813434in}}%
\pgfusepath{clip}%
\pgfsetbuttcap%
\pgfsetmiterjoin%
\definecolor{currentfill}{rgb}{0.302379,0.450282,0.300122}%
\pgfsetfillcolor{currentfill}%
\pgfsetlinewidth{0.000000pt}%
\definecolor{currentstroke}{rgb}{0.000000,0.000000,0.000000}%
\pgfsetstrokecolor{currentstroke}%
\pgfsetstrokeopacity{0.000000}%
\pgfsetdash{}{0pt}%
\pgfpathmoveto{\pgfqpoint{3.039389in}{1.812546in}}%
\pgfpathlineto{\pgfqpoint{3.048325in}{1.812546in}}%
\pgfpathlineto{\pgfqpoint{3.048325in}{1.174176in}}%
\pgfpathlineto{\pgfqpoint{3.039389in}{1.174176in}}%
\pgfpathlineto{\pgfqpoint{3.039389in}{1.812546in}}%
\pgfpathclose%
\pgfusepath{fill}%
\end{pgfscope}%
\begin{pgfscope}%
\pgfpathrectangle{\pgfqpoint{0.697024in}{0.857143in}}{\pgfqpoint{2.627103in}{1.813434in}}%
\pgfusepath{clip}%
\pgfsetbuttcap%
\pgfsetmiterjoin%
\definecolor{currentfill}{rgb}{0.302379,0.450282,0.300122}%
\pgfsetfillcolor{currentfill}%
\pgfsetlinewidth{0.000000pt}%
\definecolor{currentstroke}{rgb}{0.000000,0.000000,0.000000}%
\pgfsetstrokecolor{currentstroke}%
\pgfsetstrokeopacity{0.000000}%
\pgfsetdash{}{0pt}%
\pgfpathmoveto{\pgfqpoint{3.050559in}{1.809118in}}%
\pgfpathlineto{\pgfqpoint{3.059496in}{1.809118in}}%
\pgfpathlineto{\pgfqpoint{3.059496in}{1.148843in}}%
\pgfpathlineto{\pgfqpoint{3.050559in}{1.148843in}}%
\pgfpathlineto{\pgfqpoint{3.050559in}{1.809118in}}%
\pgfpathclose%
\pgfusepath{fill}%
\end{pgfscope}%
\begin{pgfscope}%
\pgfpathrectangle{\pgfqpoint{0.697024in}{0.857143in}}{\pgfqpoint{2.627103in}{1.813434in}}%
\pgfusepath{clip}%
\pgfsetbuttcap%
\pgfsetmiterjoin%
\definecolor{currentfill}{rgb}{0.302379,0.450282,0.300122}%
\pgfsetfillcolor{currentfill}%
\pgfsetlinewidth{0.000000pt}%
\definecolor{currentstroke}{rgb}{0.000000,0.000000,0.000000}%
\pgfsetstrokecolor{currentstroke}%
\pgfsetstrokeopacity{0.000000}%
\pgfsetdash{}{0pt}%
\pgfpathmoveto{\pgfqpoint{3.061730in}{1.811954in}}%
\pgfpathlineto{\pgfqpoint{3.070666in}{1.811954in}}%
\pgfpathlineto{\pgfqpoint{3.070666in}{1.143266in}}%
\pgfpathlineto{\pgfqpoint{3.061730in}{1.143266in}}%
\pgfpathlineto{\pgfqpoint{3.061730in}{1.811954in}}%
\pgfpathclose%
\pgfusepath{fill}%
\end{pgfscope}%
\begin{pgfscope}%
\pgfpathrectangle{\pgfqpoint{0.697024in}{0.857143in}}{\pgfqpoint{2.627103in}{1.813434in}}%
\pgfusepath{clip}%
\pgfsetbuttcap%
\pgfsetmiterjoin%
\definecolor{currentfill}{rgb}{0.302379,0.450282,0.300122}%
\pgfsetfillcolor{currentfill}%
\pgfsetlinewidth{0.000000pt}%
\definecolor{currentstroke}{rgb}{0.000000,0.000000,0.000000}%
\pgfsetstrokecolor{currentstroke}%
\pgfsetstrokeopacity{0.000000}%
\pgfsetdash{}{0pt}%
\pgfpathmoveto{\pgfqpoint{3.072900in}{1.804264in}}%
\pgfpathlineto{\pgfqpoint{3.081837in}{1.804264in}}%
\pgfpathlineto{\pgfqpoint{3.081837in}{1.186931in}}%
\pgfpathlineto{\pgfqpoint{3.072900in}{1.186931in}}%
\pgfpathlineto{\pgfqpoint{3.072900in}{1.804264in}}%
\pgfpathclose%
\pgfusepath{fill}%
\end{pgfscope}%
\begin{pgfscope}%
\pgfpathrectangle{\pgfqpoint{0.697024in}{0.857143in}}{\pgfqpoint{2.627103in}{1.813434in}}%
\pgfusepath{clip}%
\pgfsetbuttcap%
\pgfsetmiterjoin%
\definecolor{currentfill}{rgb}{0.302379,0.450282,0.300122}%
\pgfsetfillcolor{currentfill}%
\pgfsetlinewidth{0.000000pt}%
\definecolor{currentstroke}{rgb}{0.000000,0.000000,0.000000}%
\pgfsetstrokecolor{currentstroke}%
\pgfsetstrokeopacity{0.000000}%
\pgfsetdash{}{0pt}%
\pgfpathmoveto{\pgfqpoint{3.084071in}{1.790921in}}%
\pgfpathlineto{\pgfqpoint{3.093008in}{1.790921in}}%
\pgfpathlineto{\pgfqpoint{3.093008in}{1.137192in}}%
\pgfpathlineto{\pgfqpoint{3.084071in}{1.137192in}}%
\pgfpathlineto{\pgfqpoint{3.084071in}{1.790921in}}%
\pgfpathclose%
\pgfusepath{fill}%
\end{pgfscope}%
\begin{pgfscope}%
\pgfpathrectangle{\pgfqpoint{0.697024in}{0.857143in}}{\pgfqpoint{2.627103in}{1.813434in}}%
\pgfusepath{clip}%
\pgfsetbuttcap%
\pgfsetmiterjoin%
\definecolor{currentfill}{rgb}{0.302379,0.450282,0.300122}%
\pgfsetfillcolor{currentfill}%
\pgfsetlinewidth{0.000000pt}%
\definecolor{currentstroke}{rgb}{0.000000,0.000000,0.000000}%
\pgfsetstrokecolor{currentstroke}%
\pgfsetstrokeopacity{0.000000}%
\pgfsetdash{}{0pt}%
\pgfpathmoveto{\pgfqpoint{3.095242in}{1.797072in}}%
\pgfpathlineto{\pgfqpoint{3.104178in}{1.797072in}}%
\pgfpathlineto{\pgfqpoint{3.104178in}{1.132166in}}%
\pgfpathlineto{\pgfqpoint{3.095242in}{1.132166in}}%
\pgfpathlineto{\pgfqpoint{3.095242in}{1.797072in}}%
\pgfpathclose%
\pgfusepath{fill}%
\end{pgfscope}%
\begin{pgfscope}%
\pgfpathrectangle{\pgfqpoint{0.697024in}{0.857143in}}{\pgfqpoint{2.627103in}{1.813434in}}%
\pgfusepath{clip}%
\pgfsetbuttcap%
\pgfsetmiterjoin%
\definecolor{currentfill}{rgb}{0.302379,0.450282,0.300122}%
\pgfsetfillcolor{currentfill}%
\pgfsetlinewidth{0.000000pt}%
\definecolor{currentstroke}{rgb}{0.000000,0.000000,0.000000}%
\pgfsetstrokecolor{currentstroke}%
\pgfsetstrokeopacity{0.000000}%
\pgfsetdash{}{0pt}%
\pgfpathmoveto{\pgfqpoint{3.106412in}{1.799896in}}%
\pgfpathlineto{\pgfqpoint{3.115349in}{1.799896in}}%
\pgfpathlineto{\pgfqpoint{3.115349in}{1.180329in}}%
\pgfpathlineto{\pgfqpoint{3.106412in}{1.180329in}}%
\pgfpathlineto{\pgfqpoint{3.106412in}{1.799896in}}%
\pgfpathclose%
\pgfusepath{fill}%
\end{pgfscope}%
\begin{pgfscope}%
\pgfpathrectangle{\pgfqpoint{0.697024in}{0.857143in}}{\pgfqpoint{2.627103in}{1.813434in}}%
\pgfusepath{clip}%
\pgfsetbuttcap%
\pgfsetmiterjoin%
\definecolor{currentfill}{rgb}{0.302379,0.450282,0.300122}%
\pgfsetfillcolor{currentfill}%
\pgfsetlinewidth{0.000000pt}%
\definecolor{currentstroke}{rgb}{0.000000,0.000000,0.000000}%
\pgfsetstrokecolor{currentstroke}%
\pgfsetstrokeopacity{0.000000}%
\pgfsetdash{}{0pt}%
\pgfpathmoveto{\pgfqpoint{3.117583in}{1.795845in}}%
\pgfpathlineto{\pgfqpoint{3.126519in}{1.795845in}}%
\pgfpathlineto{\pgfqpoint{3.126519in}{1.162073in}}%
\pgfpathlineto{\pgfqpoint{3.117583in}{1.162073in}}%
\pgfpathlineto{\pgfqpoint{3.117583in}{1.795845in}}%
\pgfpathclose%
\pgfusepath{fill}%
\end{pgfscope}%
\begin{pgfscope}%
\pgfpathrectangle{\pgfqpoint{0.697024in}{0.857143in}}{\pgfqpoint{2.627103in}{1.813434in}}%
\pgfusepath{clip}%
\pgfsetbuttcap%
\pgfsetmiterjoin%
\definecolor{currentfill}{rgb}{0.302379,0.450282,0.300122}%
\pgfsetfillcolor{currentfill}%
\pgfsetlinewidth{0.000000pt}%
\definecolor{currentstroke}{rgb}{0.000000,0.000000,0.000000}%
\pgfsetstrokecolor{currentstroke}%
\pgfsetstrokeopacity{0.000000}%
\pgfsetdash{}{0pt}%
\pgfpathmoveto{\pgfqpoint{3.128753in}{1.784619in}}%
\pgfpathlineto{\pgfqpoint{3.137690in}{1.784619in}}%
\pgfpathlineto{\pgfqpoint{3.137690in}{1.177771in}}%
\pgfpathlineto{\pgfqpoint{3.128753in}{1.177771in}}%
\pgfpathlineto{\pgfqpoint{3.128753in}{1.784619in}}%
\pgfpathclose%
\pgfusepath{fill}%
\end{pgfscope}%
\begin{pgfscope}%
\pgfpathrectangle{\pgfqpoint{0.697024in}{0.857143in}}{\pgfqpoint{2.627103in}{1.813434in}}%
\pgfusepath{clip}%
\pgfsetbuttcap%
\pgfsetmiterjoin%
\definecolor{currentfill}{rgb}{0.302379,0.450282,0.300122}%
\pgfsetfillcolor{currentfill}%
\pgfsetlinewidth{0.000000pt}%
\definecolor{currentstroke}{rgb}{0.000000,0.000000,0.000000}%
\pgfsetstrokecolor{currentstroke}%
\pgfsetstrokeopacity{0.000000}%
\pgfsetdash{}{0pt}%
\pgfpathmoveto{\pgfqpoint{3.139924in}{1.779588in}}%
\pgfpathlineto{\pgfqpoint{3.148861in}{1.779588in}}%
\pgfpathlineto{\pgfqpoint{3.148861in}{1.142039in}}%
\pgfpathlineto{\pgfqpoint{3.139924in}{1.142039in}}%
\pgfpathlineto{\pgfqpoint{3.139924in}{1.779588in}}%
\pgfpathclose%
\pgfusepath{fill}%
\end{pgfscope}%
\begin{pgfscope}%
\pgfpathrectangle{\pgfqpoint{0.697024in}{0.857143in}}{\pgfqpoint{2.627103in}{1.813434in}}%
\pgfusepath{clip}%
\pgfsetbuttcap%
\pgfsetmiterjoin%
\definecolor{currentfill}{rgb}{0.302379,0.450282,0.300122}%
\pgfsetfillcolor{currentfill}%
\pgfsetlinewidth{0.000000pt}%
\definecolor{currentstroke}{rgb}{0.000000,0.000000,0.000000}%
\pgfsetstrokecolor{currentstroke}%
\pgfsetstrokeopacity{0.000000}%
\pgfsetdash{}{0pt}%
\pgfpathmoveto{\pgfqpoint{3.151095in}{1.763640in}}%
\pgfpathlineto{\pgfqpoint{3.160031in}{1.763640in}}%
\pgfpathlineto{\pgfqpoint{3.160031in}{1.084308in}}%
\pgfpathlineto{\pgfqpoint{3.151095in}{1.084308in}}%
\pgfpathlineto{\pgfqpoint{3.151095in}{1.763640in}}%
\pgfpathclose%
\pgfusepath{fill}%
\end{pgfscope}%
\begin{pgfscope}%
\pgfpathrectangle{\pgfqpoint{0.697024in}{0.857143in}}{\pgfqpoint{2.627103in}{1.813434in}}%
\pgfusepath{clip}%
\pgfsetbuttcap%
\pgfsetmiterjoin%
\definecolor{currentfill}{rgb}{0.302379,0.450282,0.300122}%
\pgfsetfillcolor{currentfill}%
\pgfsetlinewidth{0.000000pt}%
\definecolor{currentstroke}{rgb}{0.000000,0.000000,0.000000}%
\pgfsetstrokecolor{currentstroke}%
\pgfsetstrokeopacity{0.000000}%
\pgfsetdash{}{0pt}%
\pgfpathmoveto{\pgfqpoint{3.162265in}{1.772042in}}%
\pgfpathlineto{\pgfqpoint{3.171202in}{1.772042in}}%
\pgfpathlineto{\pgfqpoint{3.171202in}{1.011978in}}%
\pgfpathlineto{\pgfqpoint{3.162265in}{1.011978in}}%
\pgfpathlineto{\pgfqpoint{3.162265in}{1.772042in}}%
\pgfpathclose%
\pgfusepath{fill}%
\end{pgfscope}%
\begin{pgfscope}%
\pgfpathrectangle{\pgfqpoint{0.697024in}{0.857143in}}{\pgfqpoint{2.627103in}{1.813434in}}%
\pgfusepath{clip}%
\pgfsetbuttcap%
\pgfsetmiterjoin%
\definecolor{currentfill}{rgb}{0.302379,0.450282,0.300122}%
\pgfsetfillcolor{currentfill}%
\pgfsetlinewidth{0.000000pt}%
\definecolor{currentstroke}{rgb}{0.000000,0.000000,0.000000}%
\pgfsetstrokecolor{currentstroke}%
\pgfsetstrokeopacity{0.000000}%
\pgfsetdash{}{0pt}%
\pgfpathmoveto{\pgfqpoint{3.173436in}{1.775087in}}%
\pgfpathlineto{\pgfqpoint{3.182372in}{1.775087in}}%
\pgfpathlineto{\pgfqpoint{3.182372in}{1.052972in}}%
\pgfpathlineto{\pgfqpoint{3.173436in}{1.052972in}}%
\pgfpathlineto{\pgfqpoint{3.173436in}{1.775087in}}%
\pgfpathclose%
\pgfusepath{fill}%
\end{pgfscope}%
\begin{pgfscope}%
\pgfpathrectangle{\pgfqpoint{0.697024in}{0.857143in}}{\pgfqpoint{2.627103in}{1.813434in}}%
\pgfusepath{clip}%
\pgfsetbuttcap%
\pgfsetmiterjoin%
\definecolor{currentfill}{rgb}{0.302379,0.450282,0.300122}%
\pgfsetfillcolor{currentfill}%
\pgfsetlinewidth{0.000000pt}%
\definecolor{currentstroke}{rgb}{0.000000,0.000000,0.000000}%
\pgfsetstrokecolor{currentstroke}%
\pgfsetstrokeopacity{0.000000}%
\pgfsetdash{}{0pt}%
\pgfpathmoveto{\pgfqpoint{3.184607in}{1.789171in}}%
\pgfpathlineto{\pgfqpoint{3.193543in}{1.789171in}}%
\pgfpathlineto{\pgfqpoint{3.193543in}{1.021959in}}%
\pgfpathlineto{\pgfqpoint{3.184607in}{1.021959in}}%
\pgfpathlineto{\pgfqpoint{3.184607in}{1.789171in}}%
\pgfpathclose%
\pgfusepath{fill}%
\end{pgfscope}%
\begin{pgfscope}%
\pgfpathrectangle{\pgfqpoint{0.697024in}{0.857143in}}{\pgfqpoint{2.627103in}{1.813434in}}%
\pgfusepath{clip}%
\pgfsetbuttcap%
\pgfsetmiterjoin%
\definecolor{currentfill}{rgb}{0.302379,0.450282,0.300122}%
\pgfsetfillcolor{currentfill}%
\pgfsetlinewidth{0.000000pt}%
\definecolor{currentstroke}{rgb}{0.000000,0.000000,0.000000}%
\pgfsetstrokecolor{currentstroke}%
\pgfsetstrokeopacity{0.000000}%
\pgfsetdash{}{0pt}%
\pgfpathmoveto{\pgfqpoint{3.195777in}{1.802627in}}%
\pgfpathlineto{\pgfqpoint{3.204714in}{1.802627in}}%
\pgfpathlineto{\pgfqpoint{3.204714in}{1.004448in}}%
\pgfpathlineto{\pgfqpoint{3.195777in}{1.004448in}}%
\pgfpathlineto{\pgfqpoint{3.195777in}{1.802627in}}%
\pgfpathclose%
\pgfusepath{fill}%
\end{pgfscope}%
\begin{pgfscope}%
\pgfpathrectangle{\pgfqpoint{0.697024in}{0.857143in}}{\pgfqpoint{2.627103in}{1.813434in}}%
\pgfusepath{clip}%
\pgfsetbuttcap%
\pgfsetmiterjoin%
\definecolor{currentfill}{rgb}{0.511253,0.510898,0.193296}%
\pgfsetfillcolor{currentfill}%
\pgfsetlinewidth{0.000000pt}%
\definecolor{currentstroke}{rgb}{0.000000,0.000000,0.000000}%
\pgfsetstrokecolor{currentstroke}%
\pgfsetstrokeopacity{0.000000}%
\pgfsetdash{}{0pt}%
\pgfpathmoveto{\pgfqpoint{0.816438in}{1.878420in}}%
\pgfpathlineto{\pgfqpoint{0.825375in}{1.878420in}}%
\pgfpathlineto{\pgfqpoint{0.825375in}{1.881732in}}%
\pgfpathlineto{\pgfqpoint{0.816438in}{1.881732in}}%
\pgfpathlineto{\pgfqpoint{0.816438in}{1.878420in}}%
\pgfpathclose%
\pgfusepath{fill}%
\end{pgfscope}%
\begin{pgfscope}%
\pgfpathrectangle{\pgfqpoint{0.697024in}{0.857143in}}{\pgfqpoint{2.627103in}{1.813434in}}%
\pgfusepath{clip}%
\pgfsetbuttcap%
\pgfsetmiterjoin%
\definecolor{currentfill}{rgb}{0.511253,0.510898,0.193296}%
\pgfsetfillcolor{currentfill}%
\pgfsetlinewidth{0.000000pt}%
\definecolor{currentstroke}{rgb}{0.000000,0.000000,0.000000}%
\pgfsetstrokecolor{currentstroke}%
\pgfsetstrokeopacity{0.000000}%
\pgfsetdash{}{0pt}%
\pgfpathmoveto{\pgfqpoint{0.827609in}{1.922926in}}%
\pgfpathlineto{\pgfqpoint{0.836545in}{1.922926in}}%
\pgfpathlineto{\pgfqpoint{0.836545in}{1.923483in}}%
\pgfpathlineto{\pgfqpoint{0.827609in}{1.923483in}}%
\pgfpathlineto{\pgfqpoint{0.827609in}{1.922926in}}%
\pgfpathclose%
\pgfusepath{fill}%
\end{pgfscope}%
\begin{pgfscope}%
\pgfpathrectangle{\pgfqpoint{0.697024in}{0.857143in}}{\pgfqpoint{2.627103in}{1.813434in}}%
\pgfusepath{clip}%
\pgfsetbuttcap%
\pgfsetmiterjoin%
\definecolor{currentfill}{rgb}{0.511253,0.510898,0.193296}%
\pgfsetfillcolor{currentfill}%
\pgfsetlinewidth{0.000000pt}%
\definecolor{currentstroke}{rgb}{0.000000,0.000000,0.000000}%
\pgfsetstrokecolor{currentstroke}%
\pgfsetstrokeopacity{0.000000}%
\pgfsetdash{}{0pt}%
\pgfpathmoveto{\pgfqpoint{0.838779in}{1.539362in}}%
\pgfpathlineto{\pgfqpoint{0.847716in}{1.539362in}}%
\pgfpathlineto{\pgfqpoint{0.847716in}{1.538643in}}%
\pgfpathlineto{\pgfqpoint{0.838779in}{1.538643in}}%
\pgfpathlineto{\pgfqpoint{0.838779in}{1.539362in}}%
\pgfpathclose%
\pgfusepath{fill}%
\end{pgfscope}%
\begin{pgfscope}%
\pgfpathrectangle{\pgfqpoint{0.697024in}{0.857143in}}{\pgfqpoint{2.627103in}{1.813434in}}%
\pgfusepath{clip}%
\pgfsetbuttcap%
\pgfsetmiterjoin%
\definecolor{currentfill}{rgb}{0.511253,0.510898,0.193296}%
\pgfsetfillcolor{currentfill}%
\pgfsetlinewidth{0.000000pt}%
\definecolor{currentstroke}{rgb}{0.000000,0.000000,0.000000}%
\pgfsetstrokecolor{currentstroke}%
\pgfsetstrokeopacity{0.000000}%
\pgfsetdash{}{0pt}%
\pgfpathmoveto{\pgfqpoint{0.849950in}{1.568557in}}%
\pgfpathlineto{\pgfqpoint{0.858886in}{1.568557in}}%
\pgfpathlineto{\pgfqpoint{0.858886in}{1.565867in}}%
\pgfpathlineto{\pgfqpoint{0.849950in}{1.565867in}}%
\pgfpathlineto{\pgfqpoint{0.849950in}{1.568557in}}%
\pgfpathclose%
\pgfusepath{fill}%
\end{pgfscope}%
\begin{pgfscope}%
\pgfpathrectangle{\pgfqpoint{0.697024in}{0.857143in}}{\pgfqpoint{2.627103in}{1.813434in}}%
\pgfusepath{clip}%
\pgfsetbuttcap%
\pgfsetmiterjoin%
\definecolor{currentfill}{rgb}{0.511253,0.510898,0.193296}%
\pgfsetfillcolor{currentfill}%
\pgfsetlinewidth{0.000000pt}%
\definecolor{currentstroke}{rgb}{0.000000,0.000000,0.000000}%
\pgfsetstrokecolor{currentstroke}%
\pgfsetstrokeopacity{0.000000}%
\pgfsetdash{}{0pt}%
\pgfpathmoveto{\pgfqpoint{0.861121in}{1.918433in}}%
\pgfpathlineto{\pgfqpoint{0.870057in}{1.918433in}}%
\pgfpathlineto{\pgfqpoint{0.870057in}{1.919282in}}%
\pgfpathlineto{\pgfqpoint{0.861121in}{1.919282in}}%
\pgfpathlineto{\pgfqpoint{0.861121in}{1.918433in}}%
\pgfpathclose%
\pgfusepath{fill}%
\end{pgfscope}%
\begin{pgfscope}%
\pgfpathrectangle{\pgfqpoint{0.697024in}{0.857143in}}{\pgfqpoint{2.627103in}{1.813434in}}%
\pgfusepath{clip}%
\pgfsetbuttcap%
\pgfsetmiterjoin%
\definecolor{currentfill}{rgb}{0.511253,0.510898,0.193296}%
\pgfsetfillcolor{currentfill}%
\pgfsetlinewidth{0.000000pt}%
\definecolor{currentstroke}{rgb}{0.000000,0.000000,0.000000}%
\pgfsetstrokecolor{currentstroke}%
\pgfsetstrokeopacity{0.000000}%
\pgfsetdash{}{0pt}%
\pgfpathmoveto{\pgfqpoint{0.872291in}{1.476686in}}%
\pgfpathlineto{\pgfqpoint{0.881228in}{1.476686in}}%
\pgfpathlineto{\pgfqpoint{0.881228in}{1.476461in}}%
\pgfpathlineto{\pgfqpoint{0.872291in}{1.476461in}}%
\pgfpathlineto{\pgfqpoint{0.872291in}{1.476686in}}%
\pgfpathclose%
\pgfusepath{fill}%
\end{pgfscope}%
\begin{pgfscope}%
\pgfpathrectangle{\pgfqpoint{0.697024in}{0.857143in}}{\pgfqpoint{2.627103in}{1.813434in}}%
\pgfusepath{clip}%
\pgfsetbuttcap%
\pgfsetmiterjoin%
\definecolor{currentfill}{rgb}{0.511253,0.510898,0.193296}%
\pgfsetfillcolor{currentfill}%
\pgfsetlinewidth{0.000000pt}%
\definecolor{currentstroke}{rgb}{0.000000,0.000000,0.000000}%
\pgfsetstrokecolor{currentstroke}%
\pgfsetstrokeopacity{0.000000}%
\pgfsetdash{}{0pt}%
\pgfpathmoveto{\pgfqpoint{0.883462in}{1.500594in}}%
\pgfpathlineto{\pgfqpoint{0.892398in}{1.500594in}}%
\pgfpathlineto{\pgfqpoint{0.892398in}{1.496876in}}%
\pgfpathlineto{\pgfqpoint{0.883462in}{1.496876in}}%
\pgfpathlineto{\pgfqpoint{0.883462in}{1.500594in}}%
\pgfpathclose%
\pgfusepath{fill}%
\end{pgfscope}%
\begin{pgfscope}%
\pgfpathrectangle{\pgfqpoint{0.697024in}{0.857143in}}{\pgfqpoint{2.627103in}{1.813434in}}%
\pgfusepath{clip}%
\pgfsetbuttcap%
\pgfsetmiterjoin%
\definecolor{currentfill}{rgb}{0.511253,0.510898,0.193296}%
\pgfsetfillcolor{currentfill}%
\pgfsetlinewidth{0.000000pt}%
\definecolor{currentstroke}{rgb}{0.000000,0.000000,0.000000}%
\pgfsetstrokecolor{currentstroke}%
\pgfsetstrokeopacity{0.000000}%
\pgfsetdash{}{0pt}%
\pgfpathmoveto{\pgfqpoint{0.894632in}{1.537253in}}%
\pgfpathlineto{\pgfqpoint{0.903569in}{1.537253in}}%
\pgfpathlineto{\pgfqpoint{0.903569in}{1.535240in}}%
\pgfpathlineto{\pgfqpoint{0.894632in}{1.535240in}}%
\pgfpathlineto{\pgfqpoint{0.894632in}{1.537253in}}%
\pgfpathclose%
\pgfusepath{fill}%
\end{pgfscope}%
\begin{pgfscope}%
\pgfpathrectangle{\pgfqpoint{0.697024in}{0.857143in}}{\pgfqpoint{2.627103in}{1.813434in}}%
\pgfusepath{clip}%
\pgfsetbuttcap%
\pgfsetmiterjoin%
\definecolor{currentfill}{rgb}{0.511253,0.510898,0.193296}%
\pgfsetfillcolor{currentfill}%
\pgfsetlinewidth{0.000000pt}%
\definecolor{currentstroke}{rgb}{0.000000,0.000000,0.000000}%
\pgfsetstrokecolor{currentstroke}%
\pgfsetstrokeopacity{0.000000}%
\pgfsetdash{}{0pt}%
\pgfpathmoveto{\pgfqpoint{0.905803in}{1.608726in}}%
\pgfpathlineto{\pgfqpoint{0.914739in}{1.608726in}}%
\pgfpathlineto{\pgfqpoint{0.914739in}{1.604699in}}%
\pgfpathlineto{\pgfqpoint{0.905803in}{1.604699in}}%
\pgfpathlineto{\pgfqpoint{0.905803in}{1.608726in}}%
\pgfpathclose%
\pgfusepath{fill}%
\end{pgfscope}%
\begin{pgfscope}%
\pgfpathrectangle{\pgfqpoint{0.697024in}{0.857143in}}{\pgfqpoint{2.627103in}{1.813434in}}%
\pgfusepath{clip}%
\pgfsetbuttcap%
\pgfsetmiterjoin%
\definecolor{currentfill}{rgb}{0.511253,0.510898,0.193296}%
\pgfsetfillcolor{currentfill}%
\pgfsetlinewidth{0.000000pt}%
\definecolor{currentstroke}{rgb}{0.000000,0.000000,0.000000}%
\pgfsetstrokecolor{currentstroke}%
\pgfsetstrokeopacity{0.000000}%
\pgfsetdash{}{0pt}%
\pgfpathmoveto{\pgfqpoint{0.916974in}{1.993207in}}%
\pgfpathlineto{\pgfqpoint{0.925910in}{1.993207in}}%
\pgfpathlineto{\pgfqpoint{0.925910in}{1.995758in}}%
\pgfpathlineto{\pgfqpoint{0.916974in}{1.995758in}}%
\pgfpathlineto{\pgfqpoint{0.916974in}{1.993207in}}%
\pgfpathclose%
\pgfusepath{fill}%
\end{pgfscope}%
\begin{pgfscope}%
\pgfpathrectangle{\pgfqpoint{0.697024in}{0.857143in}}{\pgfqpoint{2.627103in}{1.813434in}}%
\pgfusepath{clip}%
\pgfsetbuttcap%
\pgfsetmiterjoin%
\definecolor{currentfill}{rgb}{0.511253,0.510898,0.193296}%
\pgfsetfillcolor{currentfill}%
\pgfsetlinewidth{0.000000pt}%
\definecolor{currentstroke}{rgb}{0.000000,0.000000,0.000000}%
\pgfsetstrokecolor{currentstroke}%
\pgfsetstrokeopacity{0.000000}%
\pgfsetdash{}{0pt}%
\pgfpathmoveto{\pgfqpoint{0.928144in}{2.084203in}}%
\pgfpathlineto{\pgfqpoint{0.937081in}{2.084203in}}%
\pgfpathlineto{\pgfqpoint{0.937081in}{2.088169in}}%
\pgfpathlineto{\pgfqpoint{0.928144in}{2.088169in}}%
\pgfpathlineto{\pgfqpoint{0.928144in}{2.084203in}}%
\pgfpathclose%
\pgfusepath{fill}%
\end{pgfscope}%
\begin{pgfscope}%
\pgfpathrectangle{\pgfqpoint{0.697024in}{0.857143in}}{\pgfqpoint{2.627103in}{1.813434in}}%
\pgfusepath{clip}%
\pgfsetbuttcap%
\pgfsetmiterjoin%
\definecolor{currentfill}{rgb}{0.511253,0.510898,0.193296}%
\pgfsetfillcolor{currentfill}%
\pgfsetlinewidth{0.000000pt}%
\definecolor{currentstroke}{rgb}{0.000000,0.000000,0.000000}%
\pgfsetstrokecolor{currentstroke}%
\pgfsetstrokeopacity{0.000000}%
\pgfsetdash{}{0pt}%
\pgfpathmoveto{\pgfqpoint{0.939315in}{2.121289in}}%
\pgfpathlineto{\pgfqpoint{0.948251in}{2.121289in}}%
\pgfpathlineto{\pgfqpoint{0.948251in}{2.123375in}}%
\pgfpathlineto{\pgfqpoint{0.939315in}{2.123375in}}%
\pgfpathlineto{\pgfqpoint{0.939315in}{2.121289in}}%
\pgfpathclose%
\pgfusepath{fill}%
\end{pgfscope}%
\begin{pgfscope}%
\pgfpathrectangle{\pgfqpoint{0.697024in}{0.857143in}}{\pgfqpoint{2.627103in}{1.813434in}}%
\pgfusepath{clip}%
\pgfsetbuttcap%
\pgfsetmiterjoin%
\definecolor{currentfill}{rgb}{0.511253,0.510898,0.193296}%
\pgfsetfillcolor{currentfill}%
\pgfsetlinewidth{0.000000pt}%
\definecolor{currentstroke}{rgb}{0.000000,0.000000,0.000000}%
\pgfsetstrokecolor{currentstroke}%
\pgfsetstrokeopacity{0.000000}%
\pgfsetdash{}{0pt}%
\pgfpathmoveto{\pgfqpoint{0.950485in}{2.017884in}}%
\pgfpathlineto{\pgfqpoint{0.959422in}{2.017884in}}%
\pgfpathlineto{\pgfqpoint{0.959422in}{2.020992in}}%
\pgfpathlineto{\pgfqpoint{0.950485in}{2.020992in}}%
\pgfpathlineto{\pgfqpoint{0.950485in}{2.017884in}}%
\pgfpathclose%
\pgfusepath{fill}%
\end{pgfscope}%
\begin{pgfscope}%
\pgfpathrectangle{\pgfqpoint{0.697024in}{0.857143in}}{\pgfqpoint{2.627103in}{1.813434in}}%
\pgfusepath{clip}%
\pgfsetbuttcap%
\pgfsetmiterjoin%
\definecolor{currentfill}{rgb}{0.511253,0.510898,0.193296}%
\pgfsetfillcolor{currentfill}%
\pgfsetlinewidth{0.000000pt}%
\definecolor{currentstroke}{rgb}{0.000000,0.000000,0.000000}%
\pgfsetstrokecolor{currentstroke}%
\pgfsetstrokeopacity{0.000000}%
\pgfsetdash{}{0pt}%
\pgfpathmoveto{\pgfqpoint{0.961656in}{1.577425in}}%
\pgfpathlineto{\pgfqpoint{0.970593in}{1.577425in}}%
\pgfpathlineto{\pgfqpoint{0.970593in}{1.576558in}}%
\pgfpathlineto{\pgfqpoint{0.961656in}{1.576558in}}%
\pgfpathlineto{\pgfqpoint{0.961656in}{1.577425in}}%
\pgfpathclose%
\pgfusepath{fill}%
\end{pgfscope}%
\begin{pgfscope}%
\pgfpathrectangle{\pgfqpoint{0.697024in}{0.857143in}}{\pgfqpoint{2.627103in}{1.813434in}}%
\pgfusepath{clip}%
\pgfsetbuttcap%
\pgfsetmiterjoin%
\definecolor{currentfill}{rgb}{0.511253,0.510898,0.193296}%
\pgfsetfillcolor{currentfill}%
\pgfsetlinewidth{0.000000pt}%
\definecolor{currentstroke}{rgb}{0.000000,0.000000,0.000000}%
\pgfsetstrokecolor{currentstroke}%
\pgfsetstrokeopacity{0.000000}%
\pgfsetdash{}{0pt}%
\pgfpathmoveto{\pgfqpoint{0.972827in}{1.991450in}}%
\pgfpathlineto{\pgfqpoint{0.981763in}{1.991450in}}%
\pgfpathlineto{\pgfqpoint{0.981763in}{1.995558in}}%
\pgfpathlineto{\pgfqpoint{0.972827in}{1.995558in}}%
\pgfpathlineto{\pgfqpoint{0.972827in}{1.991450in}}%
\pgfpathclose%
\pgfusepath{fill}%
\end{pgfscope}%
\begin{pgfscope}%
\pgfpathrectangle{\pgfqpoint{0.697024in}{0.857143in}}{\pgfqpoint{2.627103in}{1.813434in}}%
\pgfusepath{clip}%
\pgfsetbuttcap%
\pgfsetmiterjoin%
\definecolor{currentfill}{rgb}{0.511253,0.510898,0.193296}%
\pgfsetfillcolor{currentfill}%
\pgfsetlinewidth{0.000000pt}%
\definecolor{currentstroke}{rgb}{0.000000,0.000000,0.000000}%
\pgfsetstrokecolor{currentstroke}%
\pgfsetstrokeopacity{0.000000}%
\pgfsetdash{}{0pt}%
\pgfpathmoveto{\pgfqpoint{0.983997in}{2.038470in}}%
\pgfpathlineto{\pgfqpoint{0.992934in}{2.038470in}}%
\pgfpathlineto{\pgfqpoint{0.992934in}{2.045412in}}%
\pgfpathlineto{\pgfqpoint{0.983997in}{2.045412in}}%
\pgfpathlineto{\pgfqpoint{0.983997in}{2.038470in}}%
\pgfpathclose%
\pgfusepath{fill}%
\end{pgfscope}%
\begin{pgfscope}%
\pgfpathrectangle{\pgfqpoint{0.697024in}{0.857143in}}{\pgfqpoint{2.627103in}{1.813434in}}%
\pgfusepath{clip}%
\pgfsetbuttcap%
\pgfsetmiterjoin%
\definecolor{currentfill}{rgb}{0.511253,0.510898,0.193296}%
\pgfsetfillcolor{currentfill}%
\pgfsetlinewidth{0.000000pt}%
\definecolor{currentstroke}{rgb}{0.000000,0.000000,0.000000}%
\pgfsetstrokecolor{currentstroke}%
\pgfsetstrokeopacity{0.000000}%
\pgfsetdash{}{0pt}%
\pgfpathmoveto{\pgfqpoint{0.995168in}{2.175233in}}%
\pgfpathlineto{\pgfqpoint{1.004104in}{2.175233in}}%
\pgfpathlineto{\pgfqpoint{1.004104in}{2.180483in}}%
\pgfpathlineto{\pgfqpoint{0.995168in}{2.180483in}}%
\pgfpathlineto{\pgfqpoint{0.995168in}{2.175233in}}%
\pgfpathclose%
\pgfusepath{fill}%
\end{pgfscope}%
\begin{pgfscope}%
\pgfpathrectangle{\pgfqpoint{0.697024in}{0.857143in}}{\pgfqpoint{2.627103in}{1.813434in}}%
\pgfusepath{clip}%
\pgfsetbuttcap%
\pgfsetmiterjoin%
\definecolor{currentfill}{rgb}{0.511253,0.510898,0.193296}%
\pgfsetfillcolor{currentfill}%
\pgfsetlinewidth{0.000000pt}%
\definecolor{currentstroke}{rgb}{0.000000,0.000000,0.000000}%
\pgfsetstrokecolor{currentstroke}%
\pgfsetstrokeopacity{0.000000}%
\pgfsetdash{}{0pt}%
\pgfpathmoveto{\pgfqpoint{1.006338in}{1.979833in}}%
\pgfpathlineto{\pgfqpoint{1.015275in}{1.979833in}}%
\pgfpathlineto{\pgfqpoint{1.015275in}{1.982765in}}%
\pgfpathlineto{\pgfqpoint{1.006338in}{1.982765in}}%
\pgfpathlineto{\pgfqpoint{1.006338in}{1.979833in}}%
\pgfpathclose%
\pgfusepath{fill}%
\end{pgfscope}%
\begin{pgfscope}%
\pgfpathrectangle{\pgfqpoint{0.697024in}{0.857143in}}{\pgfqpoint{2.627103in}{1.813434in}}%
\pgfusepath{clip}%
\pgfsetbuttcap%
\pgfsetmiterjoin%
\definecolor{currentfill}{rgb}{0.511253,0.510898,0.193296}%
\pgfsetfillcolor{currentfill}%
\pgfsetlinewidth{0.000000pt}%
\definecolor{currentstroke}{rgb}{0.000000,0.000000,0.000000}%
\pgfsetstrokecolor{currentstroke}%
\pgfsetstrokeopacity{0.000000}%
\pgfsetdash{}{0pt}%
\pgfpathmoveto{\pgfqpoint{1.017509in}{2.069529in}}%
\pgfpathlineto{\pgfqpoint{1.026446in}{2.069529in}}%
\pgfpathlineto{\pgfqpoint{1.026446in}{2.077298in}}%
\pgfpathlineto{\pgfqpoint{1.017509in}{2.077298in}}%
\pgfpathlineto{\pgfqpoint{1.017509in}{2.069529in}}%
\pgfpathclose%
\pgfusepath{fill}%
\end{pgfscope}%
\begin{pgfscope}%
\pgfpathrectangle{\pgfqpoint{0.697024in}{0.857143in}}{\pgfqpoint{2.627103in}{1.813434in}}%
\pgfusepath{clip}%
\pgfsetbuttcap%
\pgfsetmiterjoin%
\definecolor{currentfill}{rgb}{0.511253,0.510898,0.193296}%
\pgfsetfillcolor{currentfill}%
\pgfsetlinewidth{0.000000pt}%
\definecolor{currentstroke}{rgb}{0.000000,0.000000,0.000000}%
\pgfsetstrokecolor{currentstroke}%
\pgfsetstrokeopacity{0.000000}%
\pgfsetdash{}{0pt}%
\pgfpathmoveto{\pgfqpoint{1.028680in}{2.043038in}}%
\pgfpathlineto{\pgfqpoint{1.037616in}{2.043038in}}%
\pgfpathlineto{\pgfqpoint{1.037616in}{2.045452in}}%
\pgfpathlineto{\pgfqpoint{1.028680in}{2.045452in}}%
\pgfpathlineto{\pgfqpoint{1.028680in}{2.043038in}}%
\pgfpathclose%
\pgfusepath{fill}%
\end{pgfscope}%
\begin{pgfscope}%
\pgfpathrectangle{\pgfqpoint{0.697024in}{0.857143in}}{\pgfqpoint{2.627103in}{1.813434in}}%
\pgfusepath{clip}%
\pgfsetbuttcap%
\pgfsetmiterjoin%
\definecolor{currentfill}{rgb}{0.511253,0.510898,0.193296}%
\pgfsetfillcolor{currentfill}%
\pgfsetlinewidth{0.000000pt}%
\definecolor{currentstroke}{rgb}{0.000000,0.000000,0.000000}%
\pgfsetstrokecolor{currentstroke}%
\pgfsetstrokeopacity{0.000000}%
\pgfsetdash{}{0pt}%
\pgfpathmoveto{\pgfqpoint{1.039850in}{2.125576in}}%
\pgfpathlineto{\pgfqpoint{1.048787in}{2.125576in}}%
\pgfpathlineto{\pgfqpoint{1.048787in}{2.126257in}}%
\pgfpathlineto{\pgfqpoint{1.039850in}{2.126257in}}%
\pgfpathlineto{\pgfqpoint{1.039850in}{2.125576in}}%
\pgfpathclose%
\pgfusepath{fill}%
\end{pgfscope}%
\begin{pgfscope}%
\pgfpathrectangle{\pgfqpoint{0.697024in}{0.857143in}}{\pgfqpoint{2.627103in}{1.813434in}}%
\pgfusepath{clip}%
\pgfsetbuttcap%
\pgfsetmiterjoin%
\definecolor{currentfill}{rgb}{0.511253,0.510898,0.193296}%
\pgfsetfillcolor{currentfill}%
\pgfsetlinewidth{0.000000pt}%
\definecolor{currentstroke}{rgb}{0.000000,0.000000,0.000000}%
\pgfsetstrokecolor{currentstroke}%
\pgfsetstrokeopacity{0.000000}%
\pgfsetdash{}{0pt}%
\pgfpathmoveto{\pgfqpoint{1.051021in}{1.362162in}}%
\pgfpathlineto{\pgfqpoint{1.059957in}{1.362162in}}%
\pgfpathlineto{\pgfqpoint{1.059957in}{1.358871in}}%
\pgfpathlineto{\pgfqpoint{1.051021in}{1.358871in}}%
\pgfpathlineto{\pgfqpoint{1.051021in}{1.362162in}}%
\pgfpathclose%
\pgfusepath{fill}%
\end{pgfscope}%
\begin{pgfscope}%
\pgfpathrectangle{\pgfqpoint{0.697024in}{0.857143in}}{\pgfqpoint{2.627103in}{1.813434in}}%
\pgfusepath{clip}%
\pgfsetbuttcap%
\pgfsetmiterjoin%
\definecolor{currentfill}{rgb}{0.511253,0.510898,0.193296}%
\pgfsetfillcolor{currentfill}%
\pgfsetlinewidth{0.000000pt}%
\definecolor{currentstroke}{rgb}{0.000000,0.000000,0.000000}%
\pgfsetstrokecolor{currentstroke}%
\pgfsetstrokeopacity{0.000000}%
\pgfsetdash{}{0pt}%
\pgfpathmoveto{\pgfqpoint{1.062191in}{2.255207in}}%
\pgfpathlineto{\pgfqpoint{1.071128in}{2.255207in}}%
\pgfpathlineto{\pgfqpoint{1.071128in}{2.256855in}}%
\pgfpathlineto{\pgfqpoint{1.062191in}{2.256855in}}%
\pgfpathlineto{\pgfqpoint{1.062191in}{2.255207in}}%
\pgfpathclose%
\pgfusepath{fill}%
\end{pgfscope}%
\begin{pgfscope}%
\pgfpathrectangle{\pgfqpoint{0.697024in}{0.857143in}}{\pgfqpoint{2.627103in}{1.813434in}}%
\pgfusepath{clip}%
\pgfsetbuttcap%
\pgfsetmiterjoin%
\definecolor{currentfill}{rgb}{0.511253,0.510898,0.193296}%
\pgfsetfillcolor{currentfill}%
\pgfsetlinewidth{0.000000pt}%
\definecolor{currentstroke}{rgb}{0.000000,0.000000,0.000000}%
\pgfsetstrokecolor{currentstroke}%
\pgfsetstrokeopacity{0.000000}%
\pgfsetdash{}{0pt}%
\pgfpathmoveto{\pgfqpoint{1.073362in}{1.268238in}}%
\pgfpathlineto{\pgfqpoint{1.082299in}{1.268238in}}%
\pgfpathlineto{\pgfqpoint{1.082299in}{1.266800in}}%
\pgfpathlineto{\pgfqpoint{1.073362in}{1.266800in}}%
\pgfpathlineto{\pgfqpoint{1.073362in}{1.268238in}}%
\pgfpathclose%
\pgfusepath{fill}%
\end{pgfscope}%
\begin{pgfscope}%
\pgfpathrectangle{\pgfqpoint{0.697024in}{0.857143in}}{\pgfqpoint{2.627103in}{1.813434in}}%
\pgfusepath{clip}%
\pgfsetbuttcap%
\pgfsetmiterjoin%
\definecolor{currentfill}{rgb}{0.511253,0.510898,0.193296}%
\pgfsetfillcolor{currentfill}%
\pgfsetlinewidth{0.000000pt}%
\definecolor{currentstroke}{rgb}{0.000000,0.000000,0.000000}%
\pgfsetstrokecolor{currentstroke}%
\pgfsetstrokeopacity{0.000000}%
\pgfsetdash{}{0pt}%
\pgfpathmoveto{\pgfqpoint{1.084533in}{1.353715in}}%
\pgfpathlineto{\pgfqpoint{1.093469in}{1.353715in}}%
\pgfpathlineto{\pgfqpoint{1.093469in}{1.350686in}}%
\pgfpathlineto{\pgfqpoint{1.084533in}{1.350686in}}%
\pgfpathlineto{\pgfqpoint{1.084533in}{1.353715in}}%
\pgfpathclose%
\pgfusepath{fill}%
\end{pgfscope}%
\begin{pgfscope}%
\pgfpathrectangle{\pgfqpoint{0.697024in}{0.857143in}}{\pgfqpoint{2.627103in}{1.813434in}}%
\pgfusepath{clip}%
\pgfsetbuttcap%
\pgfsetmiterjoin%
\definecolor{currentfill}{rgb}{0.511253,0.510898,0.193296}%
\pgfsetfillcolor{currentfill}%
\pgfsetlinewidth{0.000000pt}%
\definecolor{currentstroke}{rgb}{0.000000,0.000000,0.000000}%
\pgfsetstrokecolor{currentstroke}%
\pgfsetstrokeopacity{0.000000}%
\pgfsetdash{}{0pt}%
\pgfpathmoveto{\pgfqpoint{1.095703in}{1.358196in}}%
\pgfpathlineto{\pgfqpoint{1.104640in}{1.358196in}}%
\pgfpathlineto{\pgfqpoint{1.104640in}{1.355907in}}%
\pgfpathlineto{\pgfqpoint{1.095703in}{1.355907in}}%
\pgfpathlineto{\pgfqpoint{1.095703in}{1.358196in}}%
\pgfpathclose%
\pgfusepath{fill}%
\end{pgfscope}%
\begin{pgfscope}%
\pgfpathrectangle{\pgfqpoint{0.697024in}{0.857143in}}{\pgfqpoint{2.627103in}{1.813434in}}%
\pgfusepath{clip}%
\pgfsetbuttcap%
\pgfsetmiterjoin%
\definecolor{currentfill}{rgb}{0.511253,0.510898,0.193296}%
\pgfsetfillcolor{currentfill}%
\pgfsetlinewidth{0.000000pt}%
\definecolor{currentstroke}{rgb}{0.000000,0.000000,0.000000}%
\pgfsetstrokecolor{currentstroke}%
\pgfsetstrokeopacity{0.000000}%
\pgfsetdash{}{0pt}%
\pgfpathmoveto{\pgfqpoint{1.106874in}{2.371492in}}%
\pgfpathlineto{\pgfqpoint{1.115810in}{2.371492in}}%
\pgfpathlineto{\pgfqpoint{1.115810in}{2.371664in}}%
\pgfpathlineto{\pgfqpoint{1.106874in}{2.371664in}}%
\pgfpathlineto{\pgfqpoint{1.106874in}{2.371492in}}%
\pgfpathclose%
\pgfusepath{fill}%
\end{pgfscope}%
\begin{pgfscope}%
\pgfpathrectangle{\pgfqpoint{0.697024in}{0.857143in}}{\pgfqpoint{2.627103in}{1.813434in}}%
\pgfusepath{clip}%
\pgfsetbuttcap%
\pgfsetmiterjoin%
\definecolor{currentfill}{rgb}{0.511253,0.510898,0.193296}%
\pgfsetfillcolor{currentfill}%
\pgfsetlinewidth{0.000000pt}%
\definecolor{currentstroke}{rgb}{0.000000,0.000000,0.000000}%
\pgfsetstrokecolor{currentstroke}%
\pgfsetstrokeopacity{0.000000}%
\pgfsetdash{}{0pt}%
\pgfpathmoveto{\pgfqpoint{1.118045in}{1.496484in}}%
\pgfpathlineto{\pgfqpoint{1.126981in}{1.496484in}}%
\pgfpathlineto{\pgfqpoint{1.126981in}{1.494598in}}%
\pgfpathlineto{\pgfqpoint{1.118045in}{1.494598in}}%
\pgfpathlineto{\pgfqpoint{1.118045in}{1.496484in}}%
\pgfpathclose%
\pgfusepath{fill}%
\end{pgfscope}%
\begin{pgfscope}%
\pgfpathrectangle{\pgfqpoint{0.697024in}{0.857143in}}{\pgfqpoint{2.627103in}{1.813434in}}%
\pgfusepath{clip}%
\pgfsetbuttcap%
\pgfsetmiterjoin%
\definecolor{currentfill}{rgb}{0.511253,0.510898,0.193296}%
\pgfsetfillcolor{currentfill}%
\pgfsetlinewidth{0.000000pt}%
\definecolor{currentstroke}{rgb}{0.000000,0.000000,0.000000}%
\pgfsetstrokecolor{currentstroke}%
\pgfsetstrokeopacity{0.000000}%
\pgfsetdash{}{0pt}%
\pgfpathmoveto{\pgfqpoint{1.129215in}{1.582381in}}%
\pgfpathlineto{\pgfqpoint{1.138152in}{1.582381in}}%
\pgfpathlineto{\pgfqpoint{1.138152in}{1.578386in}}%
\pgfpathlineto{\pgfqpoint{1.129215in}{1.578386in}}%
\pgfpathlineto{\pgfqpoint{1.129215in}{1.582381in}}%
\pgfpathclose%
\pgfusepath{fill}%
\end{pgfscope}%
\begin{pgfscope}%
\pgfpathrectangle{\pgfqpoint{0.697024in}{0.857143in}}{\pgfqpoint{2.627103in}{1.813434in}}%
\pgfusepath{clip}%
\pgfsetbuttcap%
\pgfsetmiterjoin%
\definecolor{currentfill}{rgb}{0.511253,0.510898,0.193296}%
\pgfsetfillcolor{currentfill}%
\pgfsetlinewidth{0.000000pt}%
\definecolor{currentstroke}{rgb}{0.000000,0.000000,0.000000}%
\pgfsetstrokecolor{currentstroke}%
\pgfsetstrokeopacity{0.000000}%
\pgfsetdash{}{0pt}%
\pgfpathmoveto{\pgfqpoint{1.140386in}{2.147188in}}%
\pgfpathlineto{\pgfqpoint{1.149322in}{2.147188in}}%
\pgfpathlineto{\pgfqpoint{1.149322in}{2.147862in}}%
\pgfpathlineto{\pgfqpoint{1.140386in}{2.147862in}}%
\pgfpathlineto{\pgfqpoint{1.140386in}{2.147188in}}%
\pgfpathclose%
\pgfusepath{fill}%
\end{pgfscope}%
\begin{pgfscope}%
\pgfpathrectangle{\pgfqpoint{0.697024in}{0.857143in}}{\pgfqpoint{2.627103in}{1.813434in}}%
\pgfusepath{clip}%
\pgfsetbuttcap%
\pgfsetmiterjoin%
\definecolor{currentfill}{rgb}{0.511253,0.510898,0.193296}%
\pgfsetfillcolor{currentfill}%
\pgfsetlinewidth{0.000000pt}%
\definecolor{currentstroke}{rgb}{0.000000,0.000000,0.000000}%
\pgfsetstrokecolor{currentstroke}%
\pgfsetstrokeopacity{0.000000}%
\pgfsetdash{}{0pt}%
\pgfpathmoveto{\pgfqpoint{1.151556in}{1.572621in}}%
\pgfpathlineto{\pgfqpoint{1.160493in}{1.572621in}}%
\pgfpathlineto{\pgfqpoint{1.160493in}{1.569715in}}%
\pgfpathlineto{\pgfqpoint{1.151556in}{1.569715in}}%
\pgfpathlineto{\pgfqpoint{1.151556in}{1.572621in}}%
\pgfpathclose%
\pgfusepath{fill}%
\end{pgfscope}%
\begin{pgfscope}%
\pgfpathrectangle{\pgfqpoint{0.697024in}{0.857143in}}{\pgfqpoint{2.627103in}{1.813434in}}%
\pgfusepath{clip}%
\pgfsetbuttcap%
\pgfsetmiterjoin%
\definecolor{currentfill}{rgb}{0.511253,0.510898,0.193296}%
\pgfsetfillcolor{currentfill}%
\pgfsetlinewidth{0.000000pt}%
\definecolor{currentstroke}{rgb}{0.000000,0.000000,0.000000}%
\pgfsetstrokecolor{currentstroke}%
\pgfsetstrokeopacity{0.000000}%
\pgfsetdash{}{0pt}%
\pgfpathmoveto{\pgfqpoint{1.162727in}{1.674919in}}%
\pgfpathlineto{\pgfqpoint{1.171663in}{1.674919in}}%
\pgfpathlineto{\pgfqpoint{1.171663in}{1.674676in}}%
\pgfpathlineto{\pgfqpoint{1.162727in}{1.674676in}}%
\pgfpathlineto{\pgfqpoint{1.162727in}{1.674919in}}%
\pgfpathclose%
\pgfusepath{fill}%
\end{pgfscope}%
\begin{pgfscope}%
\pgfpathrectangle{\pgfqpoint{0.697024in}{0.857143in}}{\pgfqpoint{2.627103in}{1.813434in}}%
\pgfusepath{clip}%
\pgfsetbuttcap%
\pgfsetmiterjoin%
\definecolor{currentfill}{rgb}{0.511253,0.510898,0.193296}%
\pgfsetfillcolor{currentfill}%
\pgfsetlinewidth{0.000000pt}%
\definecolor{currentstroke}{rgb}{0.000000,0.000000,0.000000}%
\pgfsetstrokecolor{currentstroke}%
\pgfsetstrokeopacity{0.000000}%
\pgfsetdash{}{0pt}%
\pgfpathmoveto{\pgfqpoint{1.173898in}{1.905116in}}%
\pgfpathlineto{\pgfqpoint{1.182834in}{1.905116in}}%
\pgfpathlineto{\pgfqpoint{1.182834in}{1.906503in}}%
\pgfpathlineto{\pgfqpoint{1.173898in}{1.906503in}}%
\pgfpathlineto{\pgfqpoint{1.173898in}{1.905116in}}%
\pgfpathclose%
\pgfusepath{fill}%
\end{pgfscope}%
\begin{pgfscope}%
\pgfpathrectangle{\pgfqpoint{0.697024in}{0.857143in}}{\pgfqpoint{2.627103in}{1.813434in}}%
\pgfusepath{clip}%
\pgfsetbuttcap%
\pgfsetmiterjoin%
\definecolor{currentfill}{rgb}{0.511253,0.510898,0.193296}%
\pgfsetfillcolor{currentfill}%
\pgfsetlinewidth{0.000000pt}%
\definecolor{currentstroke}{rgb}{0.000000,0.000000,0.000000}%
\pgfsetstrokecolor{currentstroke}%
\pgfsetstrokeopacity{0.000000}%
\pgfsetdash{}{0pt}%
\pgfpathmoveto{\pgfqpoint{1.185068in}{1.877233in}}%
\pgfpathlineto{\pgfqpoint{1.194005in}{1.877233in}}%
\pgfpathlineto{\pgfqpoint{1.194005in}{1.883462in}}%
\pgfpathlineto{\pgfqpoint{1.185068in}{1.883462in}}%
\pgfpathlineto{\pgfqpoint{1.185068in}{1.877233in}}%
\pgfpathclose%
\pgfusepath{fill}%
\end{pgfscope}%
\begin{pgfscope}%
\pgfpathrectangle{\pgfqpoint{0.697024in}{0.857143in}}{\pgfqpoint{2.627103in}{1.813434in}}%
\pgfusepath{clip}%
\pgfsetbuttcap%
\pgfsetmiterjoin%
\definecolor{currentfill}{rgb}{0.511253,0.510898,0.193296}%
\pgfsetfillcolor{currentfill}%
\pgfsetlinewidth{0.000000pt}%
\definecolor{currentstroke}{rgb}{0.000000,0.000000,0.000000}%
\pgfsetstrokecolor{currentstroke}%
\pgfsetstrokeopacity{0.000000}%
\pgfsetdash{}{0pt}%
\pgfpathmoveto{\pgfqpoint{1.196239in}{1.921473in}}%
\pgfpathlineto{\pgfqpoint{1.205175in}{1.921473in}}%
\pgfpathlineto{\pgfqpoint{1.205175in}{1.927207in}}%
\pgfpathlineto{\pgfqpoint{1.196239in}{1.927207in}}%
\pgfpathlineto{\pgfqpoint{1.196239in}{1.921473in}}%
\pgfpathclose%
\pgfusepath{fill}%
\end{pgfscope}%
\begin{pgfscope}%
\pgfpathrectangle{\pgfqpoint{0.697024in}{0.857143in}}{\pgfqpoint{2.627103in}{1.813434in}}%
\pgfusepath{clip}%
\pgfsetbuttcap%
\pgfsetmiterjoin%
\definecolor{currentfill}{rgb}{0.511253,0.510898,0.193296}%
\pgfsetfillcolor{currentfill}%
\pgfsetlinewidth{0.000000pt}%
\definecolor{currentstroke}{rgb}{0.000000,0.000000,0.000000}%
\pgfsetstrokecolor{currentstroke}%
\pgfsetstrokeopacity{0.000000}%
\pgfsetdash{}{0pt}%
\pgfpathmoveto{\pgfqpoint{1.207409in}{1.952259in}}%
\pgfpathlineto{\pgfqpoint{1.216346in}{1.952259in}}%
\pgfpathlineto{\pgfqpoint{1.216346in}{1.960291in}}%
\pgfpathlineto{\pgfqpoint{1.207409in}{1.960291in}}%
\pgfpathlineto{\pgfqpoint{1.207409in}{1.952259in}}%
\pgfpathclose%
\pgfusepath{fill}%
\end{pgfscope}%
\begin{pgfscope}%
\pgfpathrectangle{\pgfqpoint{0.697024in}{0.857143in}}{\pgfqpoint{2.627103in}{1.813434in}}%
\pgfusepath{clip}%
\pgfsetbuttcap%
\pgfsetmiterjoin%
\definecolor{currentfill}{rgb}{0.511253,0.510898,0.193296}%
\pgfsetfillcolor{currentfill}%
\pgfsetlinewidth{0.000000pt}%
\definecolor{currentstroke}{rgb}{0.000000,0.000000,0.000000}%
\pgfsetstrokecolor{currentstroke}%
\pgfsetstrokeopacity{0.000000}%
\pgfsetdash{}{0pt}%
\pgfpathmoveto{\pgfqpoint{1.218580in}{1.957059in}}%
\pgfpathlineto{\pgfqpoint{1.227516in}{1.957059in}}%
\pgfpathlineto{\pgfqpoint{1.227516in}{1.963193in}}%
\pgfpathlineto{\pgfqpoint{1.218580in}{1.963193in}}%
\pgfpathlineto{\pgfqpoint{1.218580in}{1.957059in}}%
\pgfpathclose%
\pgfusepath{fill}%
\end{pgfscope}%
\begin{pgfscope}%
\pgfpathrectangle{\pgfqpoint{0.697024in}{0.857143in}}{\pgfqpoint{2.627103in}{1.813434in}}%
\pgfusepath{clip}%
\pgfsetbuttcap%
\pgfsetmiterjoin%
\definecolor{currentfill}{rgb}{0.511253,0.510898,0.193296}%
\pgfsetfillcolor{currentfill}%
\pgfsetlinewidth{0.000000pt}%
\definecolor{currentstroke}{rgb}{0.000000,0.000000,0.000000}%
\pgfsetstrokecolor{currentstroke}%
\pgfsetstrokeopacity{0.000000}%
\pgfsetdash{}{0pt}%
\pgfpathmoveto{\pgfqpoint{1.229751in}{1.930348in}}%
\pgfpathlineto{\pgfqpoint{1.238687in}{1.930348in}}%
\pgfpathlineto{\pgfqpoint{1.238687in}{1.940041in}}%
\pgfpathlineto{\pgfqpoint{1.229751in}{1.940041in}}%
\pgfpathlineto{\pgfqpoint{1.229751in}{1.930348in}}%
\pgfpathclose%
\pgfusepath{fill}%
\end{pgfscope}%
\begin{pgfscope}%
\pgfpathrectangle{\pgfqpoint{0.697024in}{0.857143in}}{\pgfqpoint{2.627103in}{1.813434in}}%
\pgfusepath{clip}%
\pgfsetbuttcap%
\pgfsetmiterjoin%
\definecolor{currentfill}{rgb}{0.511253,0.510898,0.193296}%
\pgfsetfillcolor{currentfill}%
\pgfsetlinewidth{0.000000pt}%
\definecolor{currentstroke}{rgb}{0.000000,0.000000,0.000000}%
\pgfsetstrokecolor{currentstroke}%
\pgfsetstrokeopacity{0.000000}%
\pgfsetdash{}{0pt}%
\pgfpathmoveto{\pgfqpoint{1.240921in}{1.988223in}}%
\pgfpathlineto{\pgfqpoint{1.249858in}{1.988223in}}%
\pgfpathlineto{\pgfqpoint{1.249858in}{1.993269in}}%
\pgfpathlineto{\pgfqpoint{1.240921in}{1.993269in}}%
\pgfpathlineto{\pgfqpoint{1.240921in}{1.988223in}}%
\pgfpathclose%
\pgfusepath{fill}%
\end{pgfscope}%
\begin{pgfscope}%
\pgfpathrectangle{\pgfqpoint{0.697024in}{0.857143in}}{\pgfqpoint{2.627103in}{1.813434in}}%
\pgfusepath{clip}%
\pgfsetbuttcap%
\pgfsetmiterjoin%
\definecolor{currentfill}{rgb}{0.511253,0.510898,0.193296}%
\pgfsetfillcolor{currentfill}%
\pgfsetlinewidth{0.000000pt}%
\definecolor{currentstroke}{rgb}{0.000000,0.000000,0.000000}%
\pgfsetstrokecolor{currentstroke}%
\pgfsetstrokeopacity{0.000000}%
\pgfsetdash{}{0pt}%
\pgfpathmoveto{\pgfqpoint{1.252092in}{2.017163in}}%
\pgfpathlineto{\pgfqpoint{1.261028in}{2.017163in}}%
\pgfpathlineto{\pgfqpoint{1.261028in}{2.023881in}}%
\pgfpathlineto{\pgfqpoint{1.252092in}{2.023881in}}%
\pgfpathlineto{\pgfqpoint{1.252092in}{2.017163in}}%
\pgfpathclose%
\pgfusepath{fill}%
\end{pgfscope}%
\begin{pgfscope}%
\pgfpathrectangle{\pgfqpoint{0.697024in}{0.857143in}}{\pgfqpoint{2.627103in}{1.813434in}}%
\pgfusepath{clip}%
\pgfsetbuttcap%
\pgfsetmiterjoin%
\definecolor{currentfill}{rgb}{0.511253,0.510898,0.193296}%
\pgfsetfillcolor{currentfill}%
\pgfsetlinewidth{0.000000pt}%
\definecolor{currentstroke}{rgb}{0.000000,0.000000,0.000000}%
\pgfsetstrokecolor{currentstroke}%
\pgfsetstrokeopacity{0.000000}%
\pgfsetdash{}{0pt}%
\pgfpathmoveto{\pgfqpoint{1.263262in}{1.966378in}}%
\pgfpathlineto{\pgfqpoint{1.272199in}{1.966378in}}%
\pgfpathlineto{\pgfqpoint{1.272199in}{1.972016in}}%
\pgfpathlineto{\pgfqpoint{1.263262in}{1.972016in}}%
\pgfpathlineto{\pgfqpoint{1.263262in}{1.966378in}}%
\pgfpathclose%
\pgfusepath{fill}%
\end{pgfscope}%
\begin{pgfscope}%
\pgfpathrectangle{\pgfqpoint{0.697024in}{0.857143in}}{\pgfqpoint{2.627103in}{1.813434in}}%
\pgfusepath{clip}%
\pgfsetbuttcap%
\pgfsetmiterjoin%
\definecolor{currentfill}{rgb}{0.511253,0.510898,0.193296}%
\pgfsetfillcolor{currentfill}%
\pgfsetlinewidth{0.000000pt}%
\definecolor{currentstroke}{rgb}{0.000000,0.000000,0.000000}%
\pgfsetstrokecolor{currentstroke}%
\pgfsetstrokeopacity{0.000000}%
\pgfsetdash{}{0pt}%
\pgfpathmoveto{\pgfqpoint{1.274433in}{1.930941in}}%
\pgfpathlineto{\pgfqpoint{1.283369in}{1.930941in}}%
\pgfpathlineto{\pgfqpoint{1.283369in}{1.936647in}}%
\pgfpathlineto{\pgfqpoint{1.274433in}{1.936647in}}%
\pgfpathlineto{\pgfqpoint{1.274433in}{1.930941in}}%
\pgfpathclose%
\pgfusepath{fill}%
\end{pgfscope}%
\begin{pgfscope}%
\pgfpathrectangle{\pgfqpoint{0.697024in}{0.857143in}}{\pgfqpoint{2.627103in}{1.813434in}}%
\pgfusepath{clip}%
\pgfsetbuttcap%
\pgfsetmiterjoin%
\definecolor{currentfill}{rgb}{0.511253,0.510898,0.193296}%
\pgfsetfillcolor{currentfill}%
\pgfsetlinewidth{0.000000pt}%
\definecolor{currentstroke}{rgb}{0.000000,0.000000,0.000000}%
\pgfsetstrokecolor{currentstroke}%
\pgfsetstrokeopacity{0.000000}%
\pgfsetdash{}{0pt}%
\pgfpathmoveto{\pgfqpoint{1.285604in}{1.938897in}}%
\pgfpathlineto{\pgfqpoint{1.294540in}{1.938897in}}%
\pgfpathlineto{\pgfqpoint{1.294540in}{1.942557in}}%
\pgfpathlineto{\pgfqpoint{1.285604in}{1.942557in}}%
\pgfpathlineto{\pgfqpoint{1.285604in}{1.938897in}}%
\pgfpathclose%
\pgfusepath{fill}%
\end{pgfscope}%
\begin{pgfscope}%
\pgfpathrectangle{\pgfqpoint{0.697024in}{0.857143in}}{\pgfqpoint{2.627103in}{1.813434in}}%
\pgfusepath{clip}%
\pgfsetbuttcap%
\pgfsetmiterjoin%
\definecolor{currentfill}{rgb}{0.511253,0.510898,0.193296}%
\pgfsetfillcolor{currentfill}%
\pgfsetlinewidth{0.000000pt}%
\definecolor{currentstroke}{rgb}{0.000000,0.000000,0.000000}%
\pgfsetstrokecolor{currentstroke}%
\pgfsetstrokeopacity{0.000000}%
\pgfsetdash{}{0pt}%
\pgfpathmoveto{\pgfqpoint{1.296774in}{2.017940in}}%
\pgfpathlineto{\pgfqpoint{1.305711in}{2.017940in}}%
\pgfpathlineto{\pgfqpoint{1.305711in}{2.022580in}}%
\pgfpathlineto{\pgfqpoint{1.296774in}{2.022580in}}%
\pgfpathlineto{\pgfqpoint{1.296774in}{2.017940in}}%
\pgfpathclose%
\pgfusepath{fill}%
\end{pgfscope}%
\begin{pgfscope}%
\pgfpathrectangle{\pgfqpoint{0.697024in}{0.857143in}}{\pgfqpoint{2.627103in}{1.813434in}}%
\pgfusepath{clip}%
\pgfsetbuttcap%
\pgfsetmiterjoin%
\definecolor{currentfill}{rgb}{0.511253,0.510898,0.193296}%
\pgfsetfillcolor{currentfill}%
\pgfsetlinewidth{0.000000pt}%
\definecolor{currentstroke}{rgb}{0.000000,0.000000,0.000000}%
\pgfsetstrokecolor{currentstroke}%
\pgfsetstrokeopacity{0.000000}%
\pgfsetdash{}{0pt}%
\pgfpathmoveto{\pgfqpoint{1.307945in}{2.084846in}}%
\pgfpathlineto{\pgfqpoint{1.316881in}{2.084846in}}%
\pgfpathlineto{\pgfqpoint{1.316881in}{2.086937in}}%
\pgfpathlineto{\pgfqpoint{1.307945in}{2.086937in}}%
\pgfpathlineto{\pgfqpoint{1.307945in}{2.084846in}}%
\pgfpathclose%
\pgfusepath{fill}%
\end{pgfscope}%
\begin{pgfscope}%
\pgfpathrectangle{\pgfqpoint{0.697024in}{0.857143in}}{\pgfqpoint{2.627103in}{1.813434in}}%
\pgfusepath{clip}%
\pgfsetbuttcap%
\pgfsetmiterjoin%
\definecolor{currentfill}{rgb}{0.511253,0.510898,0.193296}%
\pgfsetfillcolor{currentfill}%
\pgfsetlinewidth{0.000000pt}%
\definecolor{currentstroke}{rgb}{0.000000,0.000000,0.000000}%
\pgfsetstrokecolor{currentstroke}%
\pgfsetstrokeopacity{0.000000}%
\pgfsetdash{}{0pt}%
\pgfpathmoveto{\pgfqpoint{1.319115in}{1.632570in}}%
\pgfpathlineto{\pgfqpoint{1.328052in}{1.632570in}}%
\pgfpathlineto{\pgfqpoint{1.328052in}{1.631863in}}%
\pgfpathlineto{\pgfqpoint{1.319115in}{1.631863in}}%
\pgfpathlineto{\pgfqpoint{1.319115in}{1.632570in}}%
\pgfpathclose%
\pgfusepath{fill}%
\end{pgfscope}%
\begin{pgfscope}%
\pgfpathrectangle{\pgfqpoint{0.697024in}{0.857143in}}{\pgfqpoint{2.627103in}{1.813434in}}%
\pgfusepath{clip}%
\pgfsetbuttcap%
\pgfsetmiterjoin%
\definecolor{currentfill}{rgb}{0.511253,0.510898,0.193296}%
\pgfsetfillcolor{currentfill}%
\pgfsetlinewidth{0.000000pt}%
\definecolor{currentstroke}{rgb}{0.000000,0.000000,0.000000}%
\pgfsetstrokecolor{currentstroke}%
\pgfsetstrokeopacity{0.000000}%
\pgfsetdash{}{0pt}%
\pgfpathmoveto{\pgfqpoint{1.330286in}{1.584374in}}%
\pgfpathlineto{\pgfqpoint{1.339222in}{1.584374in}}%
\pgfpathlineto{\pgfqpoint{1.339222in}{1.584227in}}%
\pgfpathlineto{\pgfqpoint{1.330286in}{1.584227in}}%
\pgfpathlineto{\pgfqpoint{1.330286in}{1.584374in}}%
\pgfpathclose%
\pgfusepath{fill}%
\end{pgfscope}%
\begin{pgfscope}%
\pgfpathrectangle{\pgfqpoint{0.697024in}{0.857143in}}{\pgfqpoint{2.627103in}{1.813434in}}%
\pgfusepath{clip}%
\pgfsetbuttcap%
\pgfsetmiterjoin%
\definecolor{currentfill}{rgb}{0.511253,0.510898,0.193296}%
\pgfsetfillcolor{currentfill}%
\pgfsetlinewidth{0.000000pt}%
\definecolor{currentstroke}{rgb}{0.000000,0.000000,0.000000}%
\pgfsetstrokecolor{currentstroke}%
\pgfsetstrokeopacity{0.000000}%
\pgfsetdash{}{0pt}%
\pgfpathmoveto{\pgfqpoint{1.341457in}{1.710613in}}%
\pgfpathlineto{\pgfqpoint{1.350393in}{1.710613in}}%
\pgfpathlineto{\pgfqpoint{1.350393in}{1.710410in}}%
\pgfpathlineto{\pgfqpoint{1.341457in}{1.710410in}}%
\pgfpathlineto{\pgfqpoint{1.341457in}{1.710613in}}%
\pgfpathclose%
\pgfusepath{fill}%
\end{pgfscope}%
\begin{pgfscope}%
\pgfpathrectangle{\pgfqpoint{0.697024in}{0.857143in}}{\pgfqpoint{2.627103in}{1.813434in}}%
\pgfusepath{clip}%
\pgfsetbuttcap%
\pgfsetmiterjoin%
\definecolor{currentfill}{rgb}{0.511253,0.510898,0.193296}%
\pgfsetfillcolor{currentfill}%
\pgfsetlinewidth{0.000000pt}%
\definecolor{currentstroke}{rgb}{0.000000,0.000000,0.000000}%
\pgfsetstrokecolor{currentstroke}%
\pgfsetstrokeopacity{0.000000}%
\pgfsetdash{}{0pt}%
\pgfpathmoveto{\pgfqpoint{1.352627in}{1.716181in}}%
\pgfpathlineto{\pgfqpoint{1.361564in}{1.716181in}}%
\pgfpathlineto{\pgfqpoint{1.361564in}{1.714419in}}%
\pgfpathlineto{\pgfqpoint{1.352627in}{1.714419in}}%
\pgfpathlineto{\pgfqpoint{1.352627in}{1.716181in}}%
\pgfpathclose%
\pgfusepath{fill}%
\end{pgfscope}%
\begin{pgfscope}%
\pgfpathrectangle{\pgfqpoint{0.697024in}{0.857143in}}{\pgfqpoint{2.627103in}{1.813434in}}%
\pgfusepath{clip}%
\pgfsetbuttcap%
\pgfsetmiterjoin%
\definecolor{currentfill}{rgb}{0.511253,0.510898,0.193296}%
\pgfsetfillcolor{currentfill}%
\pgfsetlinewidth{0.000000pt}%
\definecolor{currentstroke}{rgb}{0.000000,0.000000,0.000000}%
\pgfsetstrokecolor{currentstroke}%
\pgfsetstrokeopacity{0.000000}%
\pgfsetdash{}{0pt}%
\pgfpathmoveto{\pgfqpoint{1.363798in}{1.766379in}}%
\pgfpathlineto{\pgfqpoint{1.372734in}{1.766379in}}%
\pgfpathlineto{\pgfqpoint{1.372734in}{1.764183in}}%
\pgfpathlineto{\pgfqpoint{1.363798in}{1.764183in}}%
\pgfpathlineto{\pgfqpoint{1.363798in}{1.766379in}}%
\pgfpathclose%
\pgfusepath{fill}%
\end{pgfscope}%
\begin{pgfscope}%
\pgfpathrectangle{\pgfqpoint{0.697024in}{0.857143in}}{\pgfqpoint{2.627103in}{1.813434in}}%
\pgfusepath{clip}%
\pgfsetbuttcap%
\pgfsetmiterjoin%
\definecolor{currentfill}{rgb}{0.511253,0.510898,0.193296}%
\pgfsetfillcolor{currentfill}%
\pgfsetlinewidth{0.000000pt}%
\definecolor{currentstroke}{rgb}{0.000000,0.000000,0.000000}%
\pgfsetstrokecolor{currentstroke}%
\pgfsetstrokeopacity{0.000000}%
\pgfsetdash{}{0pt}%
\pgfpathmoveto{\pgfqpoint{1.374968in}{1.793829in}}%
\pgfpathlineto{\pgfqpoint{1.383905in}{1.793829in}}%
\pgfpathlineto{\pgfqpoint{1.383905in}{1.790545in}}%
\pgfpathlineto{\pgfqpoint{1.374968in}{1.790545in}}%
\pgfpathlineto{\pgfqpoint{1.374968in}{1.793829in}}%
\pgfpathclose%
\pgfusepath{fill}%
\end{pgfscope}%
\begin{pgfscope}%
\pgfpathrectangle{\pgfqpoint{0.697024in}{0.857143in}}{\pgfqpoint{2.627103in}{1.813434in}}%
\pgfusepath{clip}%
\pgfsetbuttcap%
\pgfsetmiterjoin%
\definecolor{currentfill}{rgb}{0.511253,0.510898,0.193296}%
\pgfsetfillcolor{currentfill}%
\pgfsetlinewidth{0.000000pt}%
\definecolor{currentstroke}{rgb}{0.000000,0.000000,0.000000}%
\pgfsetstrokecolor{currentstroke}%
\pgfsetstrokeopacity{0.000000}%
\pgfsetdash{}{0pt}%
\pgfpathmoveto{\pgfqpoint{1.386139in}{1.912109in}}%
\pgfpathlineto{\pgfqpoint{1.395076in}{1.912109in}}%
\pgfpathlineto{\pgfqpoint{1.395076in}{1.912253in}}%
\pgfpathlineto{\pgfqpoint{1.386139in}{1.912253in}}%
\pgfpathlineto{\pgfqpoint{1.386139in}{1.912109in}}%
\pgfpathclose%
\pgfusepath{fill}%
\end{pgfscope}%
\begin{pgfscope}%
\pgfpathrectangle{\pgfqpoint{0.697024in}{0.857143in}}{\pgfqpoint{2.627103in}{1.813434in}}%
\pgfusepath{clip}%
\pgfsetbuttcap%
\pgfsetmiterjoin%
\definecolor{currentfill}{rgb}{0.511253,0.510898,0.193296}%
\pgfsetfillcolor{currentfill}%
\pgfsetlinewidth{0.000000pt}%
\definecolor{currentstroke}{rgb}{0.000000,0.000000,0.000000}%
\pgfsetstrokecolor{currentstroke}%
\pgfsetstrokeopacity{0.000000}%
\pgfsetdash{}{0pt}%
\pgfpathmoveto{\pgfqpoint{1.397310in}{1.909782in}}%
\pgfpathlineto{\pgfqpoint{1.406246in}{1.909782in}}%
\pgfpathlineto{\pgfqpoint{1.406246in}{1.912227in}}%
\pgfpathlineto{\pgfqpoint{1.397310in}{1.912227in}}%
\pgfpathlineto{\pgfqpoint{1.397310in}{1.909782in}}%
\pgfpathclose%
\pgfusepath{fill}%
\end{pgfscope}%
\begin{pgfscope}%
\pgfpathrectangle{\pgfqpoint{0.697024in}{0.857143in}}{\pgfqpoint{2.627103in}{1.813434in}}%
\pgfusepath{clip}%
\pgfsetbuttcap%
\pgfsetmiterjoin%
\definecolor{currentfill}{rgb}{0.511253,0.510898,0.193296}%
\pgfsetfillcolor{currentfill}%
\pgfsetlinewidth{0.000000pt}%
\definecolor{currentstroke}{rgb}{0.000000,0.000000,0.000000}%
\pgfsetstrokecolor{currentstroke}%
\pgfsetstrokeopacity{0.000000}%
\pgfsetdash{}{0pt}%
\pgfpathmoveto{\pgfqpoint{1.408480in}{2.059029in}}%
\pgfpathlineto{\pgfqpoint{1.417417in}{2.059029in}}%
\pgfpathlineto{\pgfqpoint{1.417417in}{2.062895in}}%
\pgfpathlineto{\pgfqpoint{1.408480in}{2.062895in}}%
\pgfpathlineto{\pgfqpoint{1.408480in}{2.059029in}}%
\pgfpathclose%
\pgfusepath{fill}%
\end{pgfscope}%
\begin{pgfscope}%
\pgfpathrectangle{\pgfqpoint{0.697024in}{0.857143in}}{\pgfqpoint{2.627103in}{1.813434in}}%
\pgfusepath{clip}%
\pgfsetbuttcap%
\pgfsetmiterjoin%
\definecolor{currentfill}{rgb}{0.511253,0.510898,0.193296}%
\pgfsetfillcolor{currentfill}%
\pgfsetlinewidth{0.000000pt}%
\definecolor{currentstroke}{rgb}{0.000000,0.000000,0.000000}%
\pgfsetstrokecolor{currentstroke}%
\pgfsetstrokeopacity{0.000000}%
\pgfsetdash{}{0pt}%
\pgfpathmoveto{\pgfqpoint{1.419651in}{2.047687in}}%
\pgfpathlineto{\pgfqpoint{1.428587in}{2.047687in}}%
\pgfpathlineto{\pgfqpoint{1.428587in}{2.057210in}}%
\pgfpathlineto{\pgfqpoint{1.419651in}{2.057210in}}%
\pgfpathlineto{\pgfqpoint{1.419651in}{2.047687in}}%
\pgfpathclose%
\pgfusepath{fill}%
\end{pgfscope}%
\begin{pgfscope}%
\pgfpathrectangle{\pgfqpoint{0.697024in}{0.857143in}}{\pgfqpoint{2.627103in}{1.813434in}}%
\pgfusepath{clip}%
\pgfsetbuttcap%
\pgfsetmiterjoin%
\definecolor{currentfill}{rgb}{0.511253,0.510898,0.193296}%
\pgfsetfillcolor{currentfill}%
\pgfsetlinewidth{0.000000pt}%
\definecolor{currentstroke}{rgb}{0.000000,0.000000,0.000000}%
\pgfsetstrokecolor{currentstroke}%
\pgfsetstrokeopacity{0.000000}%
\pgfsetdash{}{0pt}%
\pgfpathmoveto{\pgfqpoint{1.430821in}{1.941022in}}%
\pgfpathlineto{\pgfqpoint{1.439758in}{1.941022in}}%
\pgfpathlineto{\pgfqpoint{1.439758in}{1.956910in}}%
\pgfpathlineto{\pgfqpoint{1.430821in}{1.956910in}}%
\pgfpathlineto{\pgfqpoint{1.430821in}{1.941022in}}%
\pgfpathclose%
\pgfusepath{fill}%
\end{pgfscope}%
\begin{pgfscope}%
\pgfpathrectangle{\pgfqpoint{0.697024in}{0.857143in}}{\pgfqpoint{2.627103in}{1.813434in}}%
\pgfusepath{clip}%
\pgfsetbuttcap%
\pgfsetmiterjoin%
\definecolor{currentfill}{rgb}{0.511253,0.510898,0.193296}%
\pgfsetfillcolor{currentfill}%
\pgfsetlinewidth{0.000000pt}%
\definecolor{currentstroke}{rgb}{0.000000,0.000000,0.000000}%
\pgfsetstrokecolor{currentstroke}%
\pgfsetstrokeopacity{0.000000}%
\pgfsetdash{}{0pt}%
\pgfpathmoveto{\pgfqpoint{1.441992in}{1.847071in}}%
\pgfpathlineto{\pgfqpoint{1.450929in}{1.847071in}}%
\pgfpathlineto{\pgfqpoint{1.450929in}{1.862274in}}%
\pgfpathlineto{\pgfqpoint{1.441992in}{1.862274in}}%
\pgfpathlineto{\pgfqpoint{1.441992in}{1.847071in}}%
\pgfpathclose%
\pgfusepath{fill}%
\end{pgfscope}%
\begin{pgfscope}%
\pgfpathrectangle{\pgfqpoint{0.697024in}{0.857143in}}{\pgfqpoint{2.627103in}{1.813434in}}%
\pgfusepath{clip}%
\pgfsetbuttcap%
\pgfsetmiterjoin%
\definecolor{currentfill}{rgb}{0.511253,0.510898,0.193296}%
\pgfsetfillcolor{currentfill}%
\pgfsetlinewidth{0.000000pt}%
\definecolor{currentstroke}{rgb}{0.000000,0.000000,0.000000}%
\pgfsetstrokecolor{currentstroke}%
\pgfsetstrokeopacity{0.000000}%
\pgfsetdash{}{0pt}%
\pgfpathmoveto{\pgfqpoint{1.453163in}{1.907816in}}%
\pgfpathlineto{\pgfqpoint{1.462099in}{1.907816in}}%
\pgfpathlineto{\pgfqpoint{1.462099in}{1.926639in}}%
\pgfpathlineto{\pgfqpoint{1.453163in}{1.926639in}}%
\pgfpathlineto{\pgfqpoint{1.453163in}{1.907816in}}%
\pgfpathclose%
\pgfusepath{fill}%
\end{pgfscope}%
\begin{pgfscope}%
\pgfpathrectangle{\pgfqpoint{0.697024in}{0.857143in}}{\pgfqpoint{2.627103in}{1.813434in}}%
\pgfusepath{clip}%
\pgfsetbuttcap%
\pgfsetmiterjoin%
\definecolor{currentfill}{rgb}{0.511253,0.510898,0.193296}%
\pgfsetfillcolor{currentfill}%
\pgfsetlinewidth{0.000000pt}%
\definecolor{currentstroke}{rgb}{0.000000,0.000000,0.000000}%
\pgfsetstrokecolor{currentstroke}%
\pgfsetstrokeopacity{0.000000}%
\pgfsetdash{}{0pt}%
\pgfpathmoveto{\pgfqpoint{1.464333in}{2.011678in}}%
\pgfpathlineto{\pgfqpoint{1.473270in}{2.011678in}}%
\pgfpathlineto{\pgfqpoint{1.473270in}{2.031609in}}%
\pgfpathlineto{\pgfqpoint{1.464333in}{2.031609in}}%
\pgfpathlineto{\pgfqpoint{1.464333in}{2.011678in}}%
\pgfpathclose%
\pgfusepath{fill}%
\end{pgfscope}%
\begin{pgfscope}%
\pgfpathrectangle{\pgfqpoint{0.697024in}{0.857143in}}{\pgfqpoint{2.627103in}{1.813434in}}%
\pgfusepath{clip}%
\pgfsetbuttcap%
\pgfsetmiterjoin%
\definecolor{currentfill}{rgb}{0.511253,0.510898,0.193296}%
\pgfsetfillcolor{currentfill}%
\pgfsetlinewidth{0.000000pt}%
\definecolor{currentstroke}{rgb}{0.000000,0.000000,0.000000}%
\pgfsetstrokecolor{currentstroke}%
\pgfsetstrokeopacity{0.000000}%
\pgfsetdash{}{0pt}%
\pgfpathmoveto{\pgfqpoint{1.475504in}{2.004638in}}%
\pgfpathlineto{\pgfqpoint{1.484440in}{2.004638in}}%
\pgfpathlineto{\pgfqpoint{1.484440in}{2.018063in}}%
\pgfpathlineto{\pgfqpoint{1.475504in}{2.018063in}}%
\pgfpathlineto{\pgfqpoint{1.475504in}{2.004638in}}%
\pgfpathclose%
\pgfusepath{fill}%
\end{pgfscope}%
\begin{pgfscope}%
\pgfpathrectangle{\pgfqpoint{0.697024in}{0.857143in}}{\pgfqpoint{2.627103in}{1.813434in}}%
\pgfusepath{clip}%
\pgfsetbuttcap%
\pgfsetmiterjoin%
\definecolor{currentfill}{rgb}{0.511253,0.510898,0.193296}%
\pgfsetfillcolor{currentfill}%
\pgfsetlinewidth{0.000000pt}%
\definecolor{currentstroke}{rgb}{0.000000,0.000000,0.000000}%
\pgfsetstrokecolor{currentstroke}%
\pgfsetstrokeopacity{0.000000}%
\pgfsetdash{}{0pt}%
\pgfpathmoveto{\pgfqpoint{1.486674in}{1.977692in}}%
\pgfpathlineto{\pgfqpoint{1.495611in}{1.977692in}}%
\pgfpathlineto{\pgfqpoint{1.495611in}{1.993621in}}%
\pgfpathlineto{\pgfqpoint{1.486674in}{1.993621in}}%
\pgfpathlineto{\pgfqpoint{1.486674in}{1.977692in}}%
\pgfpathclose%
\pgfusepath{fill}%
\end{pgfscope}%
\begin{pgfscope}%
\pgfpathrectangle{\pgfqpoint{0.697024in}{0.857143in}}{\pgfqpoint{2.627103in}{1.813434in}}%
\pgfusepath{clip}%
\pgfsetbuttcap%
\pgfsetmiterjoin%
\definecolor{currentfill}{rgb}{0.511253,0.510898,0.193296}%
\pgfsetfillcolor{currentfill}%
\pgfsetlinewidth{0.000000pt}%
\definecolor{currentstroke}{rgb}{0.000000,0.000000,0.000000}%
\pgfsetstrokecolor{currentstroke}%
\pgfsetstrokeopacity{0.000000}%
\pgfsetdash{}{0pt}%
\pgfpathmoveto{\pgfqpoint{1.497845in}{1.870941in}}%
\pgfpathlineto{\pgfqpoint{1.506782in}{1.870941in}}%
\pgfpathlineto{\pgfqpoint{1.506782in}{1.884127in}}%
\pgfpathlineto{\pgfqpoint{1.497845in}{1.884127in}}%
\pgfpathlineto{\pgfqpoint{1.497845in}{1.870941in}}%
\pgfpathclose%
\pgfusepath{fill}%
\end{pgfscope}%
\begin{pgfscope}%
\pgfpathrectangle{\pgfqpoint{0.697024in}{0.857143in}}{\pgfqpoint{2.627103in}{1.813434in}}%
\pgfusepath{clip}%
\pgfsetbuttcap%
\pgfsetmiterjoin%
\definecolor{currentfill}{rgb}{0.511253,0.510898,0.193296}%
\pgfsetfillcolor{currentfill}%
\pgfsetlinewidth{0.000000pt}%
\definecolor{currentstroke}{rgb}{0.000000,0.000000,0.000000}%
\pgfsetstrokecolor{currentstroke}%
\pgfsetstrokeopacity{0.000000}%
\pgfsetdash{}{0pt}%
\pgfpathmoveto{\pgfqpoint{1.509016in}{1.893161in}}%
\pgfpathlineto{\pgfqpoint{1.517952in}{1.893161in}}%
\pgfpathlineto{\pgfqpoint{1.517952in}{1.906343in}}%
\pgfpathlineto{\pgfqpoint{1.509016in}{1.906343in}}%
\pgfpathlineto{\pgfqpoint{1.509016in}{1.893161in}}%
\pgfpathclose%
\pgfusepath{fill}%
\end{pgfscope}%
\begin{pgfscope}%
\pgfpathrectangle{\pgfqpoint{0.697024in}{0.857143in}}{\pgfqpoint{2.627103in}{1.813434in}}%
\pgfusepath{clip}%
\pgfsetbuttcap%
\pgfsetmiterjoin%
\definecolor{currentfill}{rgb}{0.511253,0.510898,0.193296}%
\pgfsetfillcolor{currentfill}%
\pgfsetlinewidth{0.000000pt}%
\definecolor{currentstroke}{rgb}{0.000000,0.000000,0.000000}%
\pgfsetstrokecolor{currentstroke}%
\pgfsetstrokeopacity{0.000000}%
\pgfsetdash{}{0pt}%
\pgfpathmoveto{\pgfqpoint{1.520186in}{1.853767in}}%
\pgfpathlineto{\pgfqpoint{1.529123in}{1.853767in}}%
\pgfpathlineto{\pgfqpoint{1.529123in}{1.875260in}}%
\pgfpathlineto{\pgfqpoint{1.520186in}{1.875260in}}%
\pgfpathlineto{\pgfqpoint{1.520186in}{1.853767in}}%
\pgfpathclose%
\pgfusepath{fill}%
\end{pgfscope}%
\begin{pgfscope}%
\pgfpathrectangle{\pgfqpoint{0.697024in}{0.857143in}}{\pgfqpoint{2.627103in}{1.813434in}}%
\pgfusepath{clip}%
\pgfsetbuttcap%
\pgfsetmiterjoin%
\definecolor{currentfill}{rgb}{0.511253,0.510898,0.193296}%
\pgfsetfillcolor{currentfill}%
\pgfsetlinewidth{0.000000pt}%
\definecolor{currentstroke}{rgb}{0.000000,0.000000,0.000000}%
\pgfsetstrokecolor{currentstroke}%
\pgfsetstrokeopacity{0.000000}%
\pgfsetdash{}{0pt}%
\pgfpathmoveto{\pgfqpoint{1.531357in}{1.847071in}}%
\pgfpathlineto{\pgfqpoint{1.540293in}{1.847071in}}%
\pgfpathlineto{\pgfqpoint{1.540293in}{1.868622in}}%
\pgfpathlineto{\pgfqpoint{1.531357in}{1.868622in}}%
\pgfpathlineto{\pgfqpoint{1.531357in}{1.847071in}}%
\pgfpathclose%
\pgfusepath{fill}%
\end{pgfscope}%
\begin{pgfscope}%
\pgfpathrectangle{\pgfqpoint{0.697024in}{0.857143in}}{\pgfqpoint{2.627103in}{1.813434in}}%
\pgfusepath{clip}%
\pgfsetbuttcap%
\pgfsetmiterjoin%
\definecolor{currentfill}{rgb}{0.511253,0.510898,0.193296}%
\pgfsetfillcolor{currentfill}%
\pgfsetlinewidth{0.000000pt}%
\definecolor{currentstroke}{rgb}{0.000000,0.000000,0.000000}%
\pgfsetstrokecolor{currentstroke}%
\pgfsetstrokeopacity{0.000000}%
\pgfsetdash{}{0pt}%
\pgfpathmoveto{\pgfqpoint{1.542528in}{1.847071in}}%
\pgfpathlineto{\pgfqpoint{1.551464in}{1.847071in}}%
\pgfpathlineto{\pgfqpoint{1.551464in}{1.870810in}}%
\pgfpathlineto{\pgfqpoint{1.542528in}{1.870810in}}%
\pgfpathlineto{\pgfqpoint{1.542528in}{1.847071in}}%
\pgfpathclose%
\pgfusepath{fill}%
\end{pgfscope}%
\begin{pgfscope}%
\pgfpathrectangle{\pgfqpoint{0.697024in}{0.857143in}}{\pgfqpoint{2.627103in}{1.813434in}}%
\pgfusepath{clip}%
\pgfsetbuttcap%
\pgfsetmiterjoin%
\definecolor{currentfill}{rgb}{0.511253,0.510898,0.193296}%
\pgfsetfillcolor{currentfill}%
\pgfsetlinewidth{0.000000pt}%
\definecolor{currentstroke}{rgb}{0.000000,0.000000,0.000000}%
\pgfsetstrokecolor{currentstroke}%
\pgfsetstrokeopacity{0.000000}%
\pgfsetdash{}{0pt}%
\pgfpathmoveto{\pgfqpoint{1.553698in}{1.847071in}}%
\pgfpathlineto{\pgfqpoint{1.562635in}{1.847071in}}%
\pgfpathlineto{\pgfqpoint{1.562635in}{1.875473in}}%
\pgfpathlineto{\pgfqpoint{1.553698in}{1.875473in}}%
\pgfpathlineto{\pgfqpoint{1.553698in}{1.847071in}}%
\pgfpathclose%
\pgfusepath{fill}%
\end{pgfscope}%
\begin{pgfscope}%
\pgfpathrectangle{\pgfqpoint{0.697024in}{0.857143in}}{\pgfqpoint{2.627103in}{1.813434in}}%
\pgfusepath{clip}%
\pgfsetbuttcap%
\pgfsetmiterjoin%
\definecolor{currentfill}{rgb}{0.511253,0.510898,0.193296}%
\pgfsetfillcolor{currentfill}%
\pgfsetlinewidth{0.000000pt}%
\definecolor{currentstroke}{rgb}{0.000000,0.000000,0.000000}%
\pgfsetstrokecolor{currentstroke}%
\pgfsetstrokeopacity{0.000000}%
\pgfsetdash{}{0pt}%
\pgfpathmoveto{\pgfqpoint{1.564869in}{1.847071in}}%
\pgfpathlineto{\pgfqpoint{1.573805in}{1.847071in}}%
\pgfpathlineto{\pgfqpoint{1.573805in}{1.876334in}}%
\pgfpathlineto{\pgfqpoint{1.564869in}{1.876334in}}%
\pgfpathlineto{\pgfqpoint{1.564869in}{1.847071in}}%
\pgfpathclose%
\pgfusepath{fill}%
\end{pgfscope}%
\begin{pgfscope}%
\pgfpathrectangle{\pgfqpoint{0.697024in}{0.857143in}}{\pgfqpoint{2.627103in}{1.813434in}}%
\pgfusepath{clip}%
\pgfsetbuttcap%
\pgfsetmiterjoin%
\definecolor{currentfill}{rgb}{0.511253,0.510898,0.193296}%
\pgfsetfillcolor{currentfill}%
\pgfsetlinewidth{0.000000pt}%
\definecolor{currentstroke}{rgb}{0.000000,0.000000,0.000000}%
\pgfsetstrokecolor{currentstroke}%
\pgfsetstrokeopacity{0.000000}%
\pgfsetdash{}{0pt}%
\pgfpathmoveto{\pgfqpoint{1.576039in}{1.874835in}}%
\pgfpathlineto{\pgfqpoint{1.584976in}{1.874835in}}%
\pgfpathlineto{\pgfqpoint{1.584976in}{1.899399in}}%
\pgfpathlineto{\pgfqpoint{1.576039in}{1.899399in}}%
\pgfpathlineto{\pgfqpoint{1.576039in}{1.874835in}}%
\pgfpathclose%
\pgfusepath{fill}%
\end{pgfscope}%
\begin{pgfscope}%
\pgfpathrectangle{\pgfqpoint{0.697024in}{0.857143in}}{\pgfqpoint{2.627103in}{1.813434in}}%
\pgfusepath{clip}%
\pgfsetbuttcap%
\pgfsetmiterjoin%
\definecolor{currentfill}{rgb}{0.511253,0.510898,0.193296}%
\pgfsetfillcolor{currentfill}%
\pgfsetlinewidth{0.000000pt}%
\definecolor{currentstroke}{rgb}{0.000000,0.000000,0.000000}%
\pgfsetstrokecolor{currentstroke}%
\pgfsetstrokeopacity{0.000000}%
\pgfsetdash{}{0pt}%
\pgfpathmoveto{\pgfqpoint{1.587210in}{1.931603in}}%
\pgfpathlineto{\pgfqpoint{1.596146in}{1.931603in}}%
\pgfpathlineto{\pgfqpoint{1.596146in}{1.954975in}}%
\pgfpathlineto{\pgfqpoint{1.587210in}{1.954975in}}%
\pgfpathlineto{\pgfqpoint{1.587210in}{1.931603in}}%
\pgfpathclose%
\pgfusepath{fill}%
\end{pgfscope}%
\begin{pgfscope}%
\pgfpathrectangle{\pgfqpoint{0.697024in}{0.857143in}}{\pgfqpoint{2.627103in}{1.813434in}}%
\pgfusepath{clip}%
\pgfsetbuttcap%
\pgfsetmiterjoin%
\definecolor{currentfill}{rgb}{0.511253,0.510898,0.193296}%
\pgfsetfillcolor{currentfill}%
\pgfsetlinewidth{0.000000pt}%
\definecolor{currentstroke}{rgb}{0.000000,0.000000,0.000000}%
\pgfsetstrokecolor{currentstroke}%
\pgfsetstrokeopacity{0.000000}%
\pgfsetdash{}{0pt}%
\pgfpathmoveto{\pgfqpoint{1.598381in}{1.999484in}}%
\pgfpathlineto{\pgfqpoint{1.607317in}{1.999484in}}%
\pgfpathlineto{\pgfqpoint{1.607317in}{2.019231in}}%
\pgfpathlineto{\pgfqpoint{1.598381in}{2.019231in}}%
\pgfpathlineto{\pgfqpoint{1.598381in}{1.999484in}}%
\pgfpathclose%
\pgfusepath{fill}%
\end{pgfscope}%
\begin{pgfscope}%
\pgfpathrectangle{\pgfqpoint{0.697024in}{0.857143in}}{\pgfqpoint{2.627103in}{1.813434in}}%
\pgfusepath{clip}%
\pgfsetbuttcap%
\pgfsetmiterjoin%
\definecolor{currentfill}{rgb}{0.511253,0.510898,0.193296}%
\pgfsetfillcolor{currentfill}%
\pgfsetlinewidth{0.000000pt}%
\definecolor{currentstroke}{rgb}{0.000000,0.000000,0.000000}%
\pgfsetstrokecolor{currentstroke}%
\pgfsetstrokeopacity{0.000000}%
\pgfsetdash{}{0pt}%
\pgfpathmoveto{\pgfqpoint{1.609551in}{2.018391in}}%
\pgfpathlineto{\pgfqpoint{1.618488in}{2.018391in}}%
\pgfpathlineto{\pgfqpoint{1.618488in}{2.035773in}}%
\pgfpathlineto{\pgfqpoint{1.609551in}{2.035773in}}%
\pgfpathlineto{\pgfqpoint{1.609551in}{2.018391in}}%
\pgfpathclose%
\pgfusepath{fill}%
\end{pgfscope}%
\begin{pgfscope}%
\pgfpathrectangle{\pgfqpoint{0.697024in}{0.857143in}}{\pgfqpoint{2.627103in}{1.813434in}}%
\pgfusepath{clip}%
\pgfsetbuttcap%
\pgfsetmiterjoin%
\definecolor{currentfill}{rgb}{0.511253,0.510898,0.193296}%
\pgfsetfillcolor{currentfill}%
\pgfsetlinewidth{0.000000pt}%
\definecolor{currentstroke}{rgb}{0.000000,0.000000,0.000000}%
\pgfsetstrokecolor{currentstroke}%
\pgfsetstrokeopacity{0.000000}%
\pgfsetdash{}{0pt}%
\pgfpathmoveto{\pgfqpoint{1.620722in}{2.053922in}}%
\pgfpathlineto{\pgfqpoint{1.629658in}{2.053922in}}%
\pgfpathlineto{\pgfqpoint{1.629658in}{2.066723in}}%
\pgfpathlineto{\pgfqpoint{1.620722in}{2.066723in}}%
\pgfpathlineto{\pgfqpoint{1.620722in}{2.053922in}}%
\pgfpathclose%
\pgfusepath{fill}%
\end{pgfscope}%
\begin{pgfscope}%
\pgfpathrectangle{\pgfqpoint{0.697024in}{0.857143in}}{\pgfqpoint{2.627103in}{1.813434in}}%
\pgfusepath{clip}%
\pgfsetbuttcap%
\pgfsetmiterjoin%
\definecolor{currentfill}{rgb}{0.511253,0.510898,0.193296}%
\pgfsetfillcolor{currentfill}%
\pgfsetlinewidth{0.000000pt}%
\definecolor{currentstroke}{rgb}{0.000000,0.000000,0.000000}%
\pgfsetstrokecolor{currentstroke}%
\pgfsetstrokeopacity{0.000000}%
\pgfsetdash{}{0pt}%
\pgfpathmoveto{\pgfqpoint{1.631892in}{2.061452in}}%
\pgfpathlineto{\pgfqpoint{1.640829in}{2.061452in}}%
\pgfpathlineto{\pgfqpoint{1.640829in}{2.071831in}}%
\pgfpathlineto{\pgfqpoint{1.631892in}{2.071831in}}%
\pgfpathlineto{\pgfqpoint{1.631892in}{2.061452in}}%
\pgfpathclose%
\pgfusepath{fill}%
\end{pgfscope}%
\begin{pgfscope}%
\pgfpathrectangle{\pgfqpoint{0.697024in}{0.857143in}}{\pgfqpoint{2.627103in}{1.813434in}}%
\pgfusepath{clip}%
\pgfsetbuttcap%
\pgfsetmiterjoin%
\definecolor{currentfill}{rgb}{0.511253,0.510898,0.193296}%
\pgfsetfillcolor{currentfill}%
\pgfsetlinewidth{0.000000pt}%
\definecolor{currentstroke}{rgb}{0.000000,0.000000,0.000000}%
\pgfsetstrokecolor{currentstroke}%
\pgfsetstrokeopacity{0.000000}%
\pgfsetdash{}{0pt}%
\pgfpathmoveto{\pgfqpoint{1.643063in}{2.148291in}}%
\pgfpathlineto{\pgfqpoint{1.651999in}{2.148291in}}%
\pgfpathlineto{\pgfqpoint{1.651999in}{2.156191in}}%
\pgfpathlineto{\pgfqpoint{1.643063in}{2.156191in}}%
\pgfpathlineto{\pgfqpoint{1.643063in}{2.148291in}}%
\pgfpathclose%
\pgfusepath{fill}%
\end{pgfscope}%
\begin{pgfscope}%
\pgfpathrectangle{\pgfqpoint{0.697024in}{0.857143in}}{\pgfqpoint{2.627103in}{1.813434in}}%
\pgfusepath{clip}%
\pgfsetbuttcap%
\pgfsetmiterjoin%
\definecolor{currentfill}{rgb}{0.511253,0.510898,0.193296}%
\pgfsetfillcolor{currentfill}%
\pgfsetlinewidth{0.000000pt}%
\definecolor{currentstroke}{rgb}{0.000000,0.000000,0.000000}%
\pgfsetstrokecolor{currentstroke}%
\pgfsetstrokeopacity{0.000000}%
\pgfsetdash{}{0pt}%
\pgfpathmoveto{\pgfqpoint{1.654234in}{2.101637in}}%
\pgfpathlineto{\pgfqpoint{1.663170in}{2.101637in}}%
\pgfpathlineto{\pgfqpoint{1.663170in}{2.114525in}}%
\pgfpathlineto{\pgfqpoint{1.654234in}{2.114525in}}%
\pgfpathlineto{\pgfqpoint{1.654234in}{2.101637in}}%
\pgfpathclose%
\pgfusepath{fill}%
\end{pgfscope}%
\begin{pgfscope}%
\pgfpathrectangle{\pgfqpoint{0.697024in}{0.857143in}}{\pgfqpoint{2.627103in}{1.813434in}}%
\pgfusepath{clip}%
\pgfsetbuttcap%
\pgfsetmiterjoin%
\definecolor{currentfill}{rgb}{0.511253,0.510898,0.193296}%
\pgfsetfillcolor{currentfill}%
\pgfsetlinewidth{0.000000pt}%
\definecolor{currentstroke}{rgb}{0.000000,0.000000,0.000000}%
\pgfsetstrokecolor{currentstroke}%
\pgfsetstrokeopacity{0.000000}%
\pgfsetdash{}{0pt}%
\pgfpathmoveto{\pgfqpoint{1.665404in}{2.162134in}}%
\pgfpathlineto{\pgfqpoint{1.674341in}{2.162134in}}%
\pgfpathlineto{\pgfqpoint{1.674341in}{2.176579in}}%
\pgfpathlineto{\pgfqpoint{1.665404in}{2.176579in}}%
\pgfpathlineto{\pgfqpoint{1.665404in}{2.162134in}}%
\pgfpathclose%
\pgfusepath{fill}%
\end{pgfscope}%
\begin{pgfscope}%
\pgfpathrectangle{\pgfqpoint{0.697024in}{0.857143in}}{\pgfqpoint{2.627103in}{1.813434in}}%
\pgfusepath{clip}%
\pgfsetbuttcap%
\pgfsetmiterjoin%
\definecolor{currentfill}{rgb}{0.511253,0.510898,0.193296}%
\pgfsetfillcolor{currentfill}%
\pgfsetlinewidth{0.000000pt}%
\definecolor{currentstroke}{rgb}{0.000000,0.000000,0.000000}%
\pgfsetstrokecolor{currentstroke}%
\pgfsetstrokeopacity{0.000000}%
\pgfsetdash{}{0pt}%
\pgfpathmoveto{\pgfqpoint{1.676575in}{2.151922in}}%
\pgfpathlineto{\pgfqpoint{1.685511in}{2.151922in}}%
\pgfpathlineto{\pgfqpoint{1.685511in}{2.166234in}}%
\pgfpathlineto{\pgfqpoint{1.676575in}{2.166234in}}%
\pgfpathlineto{\pgfqpoint{1.676575in}{2.151922in}}%
\pgfpathclose%
\pgfusepath{fill}%
\end{pgfscope}%
\begin{pgfscope}%
\pgfpathrectangle{\pgfqpoint{0.697024in}{0.857143in}}{\pgfqpoint{2.627103in}{1.813434in}}%
\pgfusepath{clip}%
\pgfsetbuttcap%
\pgfsetmiterjoin%
\definecolor{currentfill}{rgb}{0.511253,0.510898,0.193296}%
\pgfsetfillcolor{currentfill}%
\pgfsetlinewidth{0.000000pt}%
\definecolor{currentstroke}{rgb}{0.000000,0.000000,0.000000}%
\pgfsetstrokecolor{currentstroke}%
\pgfsetstrokeopacity{0.000000}%
\pgfsetdash{}{0pt}%
\pgfpathmoveto{\pgfqpoint{1.687745in}{2.212756in}}%
\pgfpathlineto{\pgfqpoint{1.696682in}{2.212756in}}%
\pgfpathlineto{\pgfqpoint{1.696682in}{2.225778in}}%
\pgfpathlineto{\pgfqpoint{1.687745in}{2.225778in}}%
\pgfpathlineto{\pgfqpoint{1.687745in}{2.212756in}}%
\pgfpathclose%
\pgfusepath{fill}%
\end{pgfscope}%
\begin{pgfscope}%
\pgfpathrectangle{\pgfqpoint{0.697024in}{0.857143in}}{\pgfqpoint{2.627103in}{1.813434in}}%
\pgfusepath{clip}%
\pgfsetbuttcap%
\pgfsetmiterjoin%
\definecolor{currentfill}{rgb}{0.511253,0.510898,0.193296}%
\pgfsetfillcolor{currentfill}%
\pgfsetlinewidth{0.000000pt}%
\definecolor{currentstroke}{rgb}{0.000000,0.000000,0.000000}%
\pgfsetstrokecolor{currentstroke}%
\pgfsetstrokeopacity{0.000000}%
\pgfsetdash{}{0pt}%
\pgfpathmoveto{\pgfqpoint{1.698916in}{2.178891in}}%
\pgfpathlineto{\pgfqpoint{1.707852in}{2.178891in}}%
\pgfpathlineto{\pgfqpoint{1.707852in}{2.195750in}}%
\pgfpathlineto{\pgfqpoint{1.698916in}{2.195750in}}%
\pgfpathlineto{\pgfqpoint{1.698916in}{2.178891in}}%
\pgfpathclose%
\pgfusepath{fill}%
\end{pgfscope}%
\begin{pgfscope}%
\pgfpathrectangle{\pgfqpoint{0.697024in}{0.857143in}}{\pgfqpoint{2.627103in}{1.813434in}}%
\pgfusepath{clip}%
\pgfsetbuttcap%
\pgfsetmiterjoin%
\definecolor{currentfill}{rgb}{0.511253,0.510898,0.193296}%
\pgfsetfillcolor{currentfill}%
\pgfsetlinewidth{0.000000pt}%
\definecolor{currentstroke}{rgb}{0.000000,0.000000,0.000000}%
\pgfsetstrokecolor{currentstroke}%
\pgfsetstrokeopacity{0.000000}%
\pgfsetdash{}{0pt}%
\pgfpathmoveto{\pgfqpoint{1.710087in}{2.204658in}}%
\pgfpathlineto{\pgfqpoint{1.719023in}{2.204658in}}%
\pgfpathlineto{\pgfqpoint{1.719023in}{2.219420in}}%
\pgfpathlineto{\pgfqpoint{1.710087in}{2.219420in}}%
\pgfpathlineto{\pgfqpoint{1.710087in}{2.204658in}}%
\pgfpathclose%
\pgfusepath{fill}%
\end{pgfscope}%
\begin{pgfscope}%
\pgfpathrectangle{\pgfqpoint{0.697024in}{0.857143in}}{\pgfqpoint{2.627103in}{1.813434in}}%
\pgfusepath{clip}%
\pgfsetbuttcap%
\pgfsetmiterjoin%
\definecolor{currentfill}{rgb}{0.511253,0.510898,0.193296}%
\pgfsetfillcolor{currentfill}%
\pgfsetlinewidth{0.000000pt}%
\definecolor{currentstroke}{rgb}{0.000000,0.000000,0.000000}%
\pgfsetstrokecolor{currentstroke}%
\pgfsetstrokeopacity{0.000000}%
\pgfsetdash{}{0pt}%
\pgfpathmoveto{\pgfqpoint{1.721257in}{2.212595in}}%
\pgfpathlineto{\pgfqpoint{1.730194in}{2.212595in}}%
\pgfpathlineto{\pgfqpoint{1.730194in}{2.225791in}}%
\pgfpathlineto{\pgfqpoint{1.721257in}{2.225791in}}%
\pgfpathlineto{\pgfqpoint{1.721257in}{2.212595in}}%
\pgfpathclose%
\pgfusepath{fill}%
\end{pgfscope}%
\begin{pgfscope}%
\pgfpathrectangle{\pgfqpoint{0.697024in}{0.857143in}}{\pgfqpoint{2.627103in}{1.813434in}}%
\pgfusepath{clip}%
\pgfsetbuttcap%
\pgfsetmiterjoin%
\definecolor{currentfill}{rgb}{0.511253,0.510898,0.193296}%
\pgfsetfillcolor{currentfill}%
\pgfsetlinewidth{0.000000pt}%
\definecolor{currentstroke}{rgb}{0.000000,0.000000,0.000000}%
\pgfsetstrokecolor{currentstroke}%
\pgfsetstrokeopacity{0.000000}%
\pgfsetdash{}{0pt}%
\pgfpathmoveto{\pgfqpoint{1.732428in}{2.190952in}}%
\pgfpathlineto{\pgfqpoint{1.741364in}{2.190952in}}%
\pgfpathlineto{\pgfqpoint{1.741364in}{2.206952in}}%
\pgfpathlineto{\pgfqpoint{1.732428in}{2.206952in}}%
\pgfpathlineto{\pgfqpoint{1.732428in}{2.190952in}}%
\pgfpathclose%
\pgfusepath{fill}%
\end{pgfscope}%
\begin{pgfscope}%
\pgfpathrectangle{\pgfqpoint{0.697024in}{0.857143in}}{\pgfqpoint{2.627103in}{1.813434in}}%
\pgfusepath{clip}%
\pgfsetbuttcap%
\pgfsetmiterjoin%
\definecolor{currentfill}{rgb}{0.511253,0.510898,0.193296}%
\pgfsetfillcolor{currentfill}%
\pgfsetlinewidth{0.000000pt}%
\definecolor{currentstroke}{rgb}{0.000000,0.000000,0.000000}%
\pgfsetstrokecolor{currentstroke}%
\pgfsetstrokeopacity{0.000000}%
\pgfsetdash{}{0pt}%
\pgfpathmoveto{\pgfqpoint{1.743598in}{2.288257in}}%
\pgfpathlineto{\pgfqpoint{1.752535in}{2.288257in}}%
\pgfpathlineto{\pgfqpoint{1.752535in}{2.297173in}}%
\pgfpathlineto{\pgfqpoint{1.743598in}{2.297173in}}%
\pgfpathlineto{\pgfqpoint{1.743598in}{2.288257in}}%
\pgfpathclose%
\pgfusepath{fill}%
\end{pgfscope}%
\begin{pgfscope}%
\pgfpathrectangle{\pgfqpoint{0.697024in}{0.857143in}}{\pgfqpoint{2.627103in}{1.813434in}}%
\pgfusepath{clip}%
\pgfsetbuttcap%
\pgfsetmiterjoin%
\definecolor{currentfill}{rgb}{0.511253,0.510898,0.193296}%
\pgfsetfillcolor{currentfill}%
\pgfsetlinewidth{0.000000pt}%
\definecolor{currentstroke}{rgb}{0.000000,0.000000,0.000000}%
\pgfsetstrokecolor{currentstroke}%
\pgfsetstrokeopacity{0.000000}%
\pgfsetdash{}{0pt}%
\pgfpathmoveto{\pgfqpoint{1.754769in}{2.357787in}}%
\pgfpathlineto{\pgfqpoint{1.763705in}{2.357787in}}%
\pgfpathlineto{\pgfqpoint{1.763705in}{2.369671in}}%
\pgfpathlineto{\pgfqpoint{1.754769in}{2.369671in}}%
\pgfpathlineto{\pgfqpoint{1.754769in}{2.357787in}}%
\pgfpathclose%
\pgfusepath{fill}%
\end{pgfscope}%
\begin{pgfscope}%
\pgfpathrectangle{\pgfqpoint{0.697024in}{0.857143in}}{\pgfqpoint{2.627103in}{1.813434in}}%
\pgfusepath{clip}%
\pgfsetbuttcap%
\pgfsetmiterjoin%
\definecolor{currentfill}{rgb}{0.511253,0.510898,0.193296}%
\pgfsetfillcolor{currentfill}%
\pgfsetlinewidth{0.000000pt}%
\definecolor{currentstroke}{rgb}{0.000000,0.000000,0.000000}%
\pgfsetstrokecolor{currentstroke}%
\pgfsetstrokeopacity{0.000000}%
\pgfsetdash{}{0pt}%
\pgfpathmoveto{\pgfqpoint{1.765940in}{2.306304in}}%
\pgfpathlineto{\pgfqpoint{1.774876in}{2.306304in}}%
\pgfpathlineto{\pgfqpoint{1.774876in}{2.320194in}}%
\pgfpathlineto{\pgfqpoint{1.765940in}{2.320194in}}%
\pgfpathlineto{\pgfqpoint{1.765940in}{2.306304in}}%
\pgfpathclose%
\pgfusepath{fill}%
\end{pgfscope}%
\begin{pgfscope}%
\pgfpathrectangle{\pgfqpoint{0.697024in}{0.857143in}}{\pgfqpoint{2.627103in}{1.813434in}}%
\pgfusepath{clip}%
\pgfsetbuttcap%
\pgfsetmiterjoin%
\definecolor{currentfill}{rgb}{0.511253,0.510898,0.193296}%
\pgfsetfillcolor{currentfill}%
\pgfsetlinewidth{0.000000pt}%
\definecolor{currentstroke}{rgb}{0.000000,0.000000,0.000000}%
\pgfsetstrokecolor{currentstroke}%
\pgfsetstrokeopacity{0.000000}%
\pgfsetdash{}{0pt}%
\pgfpathmoveto{\pgfqpoint{1.777110in}{2.379179in}}%
\pgfpathlineto{\pgfqpoint{1.786047in}{2.379179in}}%
\pgfpathlineto{\pgfqpoint{1.786047in}{2.386386in}}%
\pgfpathlineto{\pgfqpoint{1.777110in}{2.386386in}}%
\pgfpathlineto{\pgfqpoint{1.777110in}{2.379179in}}%
\pgfpathclose%
\pgfusepath{fill}%
\end{pgfscope}%
\begin{pgfscope}%
\pgfpathrectangle{\pgfqpoint{0.697024in}{0.857143in}}{\pgfqpoint{2.627103in}{1.813434in}}%
\pgfusepath{clip}%
\pgfsetbuttcap%
\pgfsetmiterjoin%
\definecolor{currentfill}{rgb}{0.511253,0.510898,0.193296}%
\pgfsetfillcolor{currentfill}%
\pgfsetlinewidth{0.000000pt}%
\definecolor{currentstroke}{rgb}{0.000000,0.000000,0.000000}%
\pgfsetstrokecolor{currentstroke}%
\pgfsetstrokeopacity{0.000000}%
\pgfsetdash{}{0pt}%
\pgfpathmoveto{\pgfqpoint{1.788281in}{2.415145in}}%
\pgfpathlineto{\pgfqpoint{1.797217in}{2.415145in}}%
\pgfpathlineto{\pgfqpoint{1.797217in}{2.424768in}}%
\pgfpathlineto{\pgfqpoint{1.788281in}{2.424768in}}%
\pgfpathlineto{\pgfqpoint{1.788281in}{2.415145in}}%
\pgfpathclose%
\pgfusepath{fill}%
\end{pgfscope}%
\begin{pgfscope}%
\pgfpathrectangle{\pgfqpoint{0.697024in}{0.857143in}}{\pgfqpoint{2.627103in}{1.813434in}}%
\pgfusepath{clip}%
\pgfsetbuttcap%
\pgfsetmiterjoin%
\definecolor{currentfill}{rgb}{0.511253,0.510898,0.193296}%
\pgfsetfillcolor{currentfill}%
\pgfsetlinewidth{0.000000pt}%
\definecolor{currentstroke}{rgb}{0.000000,0.000000,0.000000}%
\pgfsetstrokecolor{currentstroke}%
\pgfsetstrokeopacity{0.000000}%
\pgfsetdash{}{0pt}%
\pgfpathmoveto{\pgfqpoint{1.799451in}{2.469976in}}%
\pgfpathlineto{\pgfqpoint{1.808388in}{2.469976in}}%
\pgfpathlineto{\pgfqpoint{1.808388in}{2.475733in}}%
\pgfpathlineto{\pgfqpoint{1.799451in}{2.475733in}}%
\pgfpathlineto{\pgfqpoint{1.799451in}{2.469976in}}%
\pgfpathclose%
\pgfusepath{fill}%
\end{pgfscope}%
\begin{pgfscope}%
\pgfpathrectangle{\pgfqpoint{0.697024in}{0.857143in}}{\pgfqpoint{2.627103in}{1.813434in}}%
\pgfusepath{clip}%
\pgfsetbuttcap%
\pgfsetmiterjoin%
\definecolor{currentfill}{rgb}{0.511253,0.510898,0.193296}%
\pgfsetfillcolor{currentfill}%
\pgfsetlinewidth{0.000000pt}%
\definecolor{currentstroke}{rgb}{0.000000,0.000000,0.000000}%
\pgfsetstrokecolor{currentstroke}%
\pgfsetstrokeopacity{0.000000}%
\pgfsetdash{}{0pt}%
\pgfpathmoveto{\pgfqpoint{1.810622in}{2.458478in}}%
\pgfpathlineto{\pgfqpoint{1.819559in}{2.458478in}}%
\pgfpathlineto{\pgfqpoint{1.819559in}{2.464668in}}%
\pgfpathlineto{\pgfqpoint{1.810622in}{2.464668in}}%
\pgfpathlineto{\pgfqpoint{1.810622in}{2.458478in}}%
\pgfpathclose%
\pgfusepath{fill}%
\end{pgfscope}%
\begin{pgfscope}%
\pgfpathrectangle{\pgfqpoint{0.697024in}{0.857143in}}{\pgfqpoint{2.627103in}{1.813434in}}%
\pgfusepath{clip}%
\pgfsetbuttcap%
\pgfsetmiterjoin%
\definecolor{currentfill}{rgb}{0.511253,0.510898,0.193296}%
\pgfsetfillcolor{currentfill}%
\pgfsetlinewidth{0.000000pt}%
\definecolor{currentstroke}{rgb}{0.000000,0.000000,0.000000}%
\pgfsetstrokecolor{currentstroke}%
\pgfsetstrokeopacity{0.000000}%
\pgfsetdash{}{0pt}%
\pgfpathmoveto{\pgfqpoint{1.821793in}{2.534609in}}%
\pgfpathlineto{\pgfqpoint{1.830729in}{2.534609in}}%
\pgfpathlineto{\pgfqpoint{1.830729in}{2.539751in}}%
\pgfpathlineto{\pgfqpoint{1.821793in}{2.539751in}}%
\pgfpathlineto{\pgfqpoint{1.821793in}{2.534609in}}%
\pgfpathclose%
\pgfusepath{fill}%
\end{pgfscope}%
\begin{pgfscope}%
\pgfpathrectangle{\pgfqpoint{0.697024in}{0.857143in}}{\pgfqpoint{2.627103in}{1.813434in}}%
\pgfusepath{clip}%
\pgfsetbuttcap%
\pgfsetmiterjoin%
\definecolor{currentfill}{rgb}{0.511253,0.510898,0.193296}%
\pgfsetfillcolor{currentfill}%
\pgfsetlinewidth{0.000000pt}%
\definecolor{currentstroke}{rgb}{0.000000,0.000000,0.000000}%
\pgfsetstrokecolor{currentstroke}%
\pgfsetstrokeopacity{0.000000}%
\pgfsetdash{}{0pt}%
\pgfpathmoveto{\pgfqpoint{1.832963in}{2.550636in}}%
\pgfpathlineto{\pgfqpoint{1.841900in}{2.550636in}}%
\pgfpathlineto{\pgfqpoint{1.841900in}{2.554731in}}%
\pgfpathlineto{\pgfqpoint{1.832963in}{2.554731in}}%
\pgfpathlineto{\pgfqpoint{1.832963in}{2.550636in}}%
\pgfpathclose%
\pgfusepath{fill}%
\end{pgfscope}%
\begin{pgfscope}%
\pgfpathrectangle{\pgfqpoint{0.697024in}{0.857143in}}{\pgfqpoint{2.627103in}{1.813434in}}%
\pgfusepath{clip}%
\pgfsetbuttcap%
\pgfsetmiterjoin%
\definecolor{currentfill}{rgb}{0.511253,0.510898,0.193296}%
\pgfsetfillcolor{currentfill}%
\pgfsetlinewidth{0.000000pt}%
\definecolor{currentstroke}{rgb}{0.000000,0.000000,0.000000}%
\pgfsetstrokecolor{currentstroke}%
\pgfsetstrokeopacity{0.000000}%
\pgfsetdash{}{0pt}%
\pgfpathmoveto{\pgfqpoint{1.844134in}{2.529543in}}%
\pgfpathlineto{\pgfqpoint{1.853070in}{2.529543in}}%
\pgfpathlineto{\pgfqpoint{1.853070in}{2.532507in}}%
\pgfpathlineto{\pgfqpoint{1.844134in}{2.532507in}}%
\pgfpathlineto{\pgfqpoint{1.844134in}{2.529543in}}%
\pgfpathclose%
\pgfusepath{fill}%
\end{pgfscope}%
\begin{pgfscope}%
\pgfpathrectangle{\pgfqpoint{0.697024in}{0.857143in}}{\pgfqpoint{2.627103in}{1.813434in}}%
\pgfusepath{clip}%
\pgfsetbuttcap%
\pgfsetmiterjoin%
\definecolor{currentfill}{rgb}{0.511253,0.510898,0.193296}%
\pgfsetfillcolor{currentfill}%
\pgfsetlinewidth{0.000000pt}%
\definecolor{currentstroke}{rgb}{0.000000,0.000000,0.000000}%
\pgfsetstrokecolor{currentstroke}%
\pgfsetstrokeopacity{0.000000}%
\pgfsetdash{}{0pt}%
\pgfpathmoveto{\pgfqpoint{1.855304in}{2.494926in}}%
\pgfpathlineto{\pgfqpoint{1.864241in}{2.494926in}}%
\pgfpathlineto{\pgfqpoint{1.864241in}{2.495537in}}%
\pgfpathlineto{\pgfqpoint{1.855304in}{2.495537in}}%
\pgfpathlineto{\pgfqpoint{1.855304in}{2.494926in}}%
\pgfpathclose%
\pgfusepath{fill}%
\end{pgfscope}%
\begin{pgfscope}%
\pgfpathrectangle{\pgfqpoint{0.697024in}{0.857143in}}{\pgfqpoint{2.627103in}{1.813434in}}%
\pgfusepath{clip}%
\pgfsetbuttcap%
\pgfsetmiterjoin%
\definecolor{currentfill}{rgb}{0.511253,0.510898,0.193296}%
\pgfsetfillcolor{currentfill}%
\pgfsetlinewidth{0.000000pt}%
\definecolor{currentstroke}{rgb}{0.000000,0.000000,0.000000}%
\pgfsetstrokecolor{currentstroke}%
\pgfsetstrokeopacity{0.000000}%
\pgfsetdash{}{0pt}%
\pgfpathmoveto{\pgfqpoint{1.866475in}{2.550434in}}%
\pgfpathlineto{\pgfqpoint{1.875412in}{2.550434in}}%
\pgfpathlineto{\pgfqpoint{1.875412in}{2.550901in}}%
\pgfpathlineto{\pgfqpoint{1.866475in}{2.550901in}}%
\pgfpathlineto{\pgfqpoint{1.866475in}{2.550434in}}%
\pgfpathclose%
\pgfusepath{fill}%
\end{pgfscope}%
\begin{pgfscope}%
\pgfpathrectangle{\pgfqpoint{0.697024in}{0.857143in}}{\pgfqpoint{2.627103in}{1.813434in}}%
\pgfusepath{clip}%
\pgfsetbuttcap%
\pgfsetmiterjoin%
\definecolor{currentfill}{rgb}{0.511253,0.510898,0.193296}%
\pgfsetfillcolor{currentfill}%
\pgfsetlinewidth{0.000000pt}%
\definecolor{currentstroke}{rgb}{0.000000,0.000000,0.000000}%
\pgfsetstrokecolor{currentstroke}%
\pgfsetstrokeopacity{0.000000}%
\pgfsetdash{}{0pt}%
\pgfpathmoveto{\pgfqpoint{1.877646in}{2.525627in}}%
\pgfpathlineto{\pgfqpoint{1.886582in}{2.525627in}}%
\pgfpathlineto{\pgfqpoint{1.886582in}{2.527902in}}%
\pgfpathlineto{\pgfqpoint{1.877646in}{2.527902in}}%
\pgfpathlineto{\pgfqpoint{1.877646in}{2.525627in}}%
\pgfpathclose%
\pgfusepath{fill}%
\end{pgfscope}%
\begin{pgfscope}%
\pgfpathrectangle{\pgfqpoint{0.697024in}{0.857143in}}{\pgfqpoint{2.627103in}{1.813434in}}%
\pgfusepath{clip}%
\pgfsetbuttcap%
\pgfsetmiterjoin%
\definecolor{currentfill}{rgb}{0.511253,0.510898,0.193296}%
\pgfsetfillcolor{currentfill}%
\pgfsetlinewidth{0.000000pt}%
\definecolor{currentstroke}{rgb}{0.000000,0.000000,0.000000}%
\pgfsetstrokecolor{currentstroke}%
\pgfsetstrokeopacity{0.000000}%
\pgfsetdash{}{0pt}%
\pgfpathmoveto{\pgfqpoint{1.888816in}{2.485916in}}%
\pgfpathlineto{\pgfqpoint{1.897753in}{2.485916in}}%
\pgfpathlineto{\pgfqpoint{1.897753in}{2.486022in}}%
\pgfpathlineto{\pgfqpoint{1.888816in}{2.486022in}}%
\pgfpathlineto{\pgfqpoint{1.888816in}{2.485916in}}%
\pgfpathclose%
\pgfusepath{fill}%
\end{pgfscope}%
\begin{pgfscope}%
\pgfpathrectangle{\pgfqpoint{0.697024in}{0.857143in}}{\pgfqpoint{2.627103in}{1.813434in}}%
\pgfusepath{clip}%
\pgfsetbuttcap%
\pgfsetmiterjoin%
\definecolor{currentfill}{rgb}{0.511253,0.510898,0.193296}%
\pgfsetfillcolor{currentfill}%
\pgfsetlinewidth{0.000000pt}%
\definecolor{currentstroke}{rgb}{0.000000,0.000000,0.000000}%
\pgfsetstrokecolor{currentstroke}%
\pgfsetstrokeopacity{0.000000}%
\pgfsetdash{}{0pt}%
\pgfpathmoveto{\pgfqpoint{1.899987in}{1.456659in}}%
\pgfpathlineto{\pgfqpoint{1.908923in}{1.456659in}}%
\pgfpathlineto{\pgfqpoint{1.908923in}{1.455278in}}%
\pgfpathlineto{\pgfqpoint{1.899987in}{1.455278in}}%
\pgfpathlineto{\pgfqpoint{1.899987in}{1.456659in}}%
\pgfpathclose%
\pgfusepath{fill}%
\end{pgfscope}%
\begin{pgfscope}%
\pgfpathrectangle{\pgfqpoint{0.697024in}{0.857143in}}{\pgfqpoint{2.627103in}{1.813434in}}%
\pgfusepath{clip}%
\pgfsetbuttcap%
\pgfsetmiterjoin%
\definecolor{currentfill}{rgb}{0.511253,0.510898,0.193296}%
\pgfsetfillcolor{currentfill}%
\pgfsetlinewidth{0.000000pt}%
\definecolor{currentstroke}{rgb}{0.000000,0.000000,0.000000}%
\pgfsetstrokecolor{currentstroke}%
\pgfsetstrokeopacity{0.000000}%
\pgfsetdash{}{0pt}%
\pgfpathmoveto{\pgfqpoint{1.911157in}{2.382578in}}%
\pgfpathlineto{\pgfqpoint{1.920094in}{2.382578in}}%
\pgfpathlineto{\pgfqpoint{1.920094in}{2.386845in}}%
\pgfpathlineto{\pgfqpoint{1.911157in}{2.386845in}}%
\pgfpathlineto{\pgfqpoint{1.911157in}{2.382578in}}%
\pgfpathclose%
\pgfusepath{fill}%
\end{pgfscope}%
\begin{pgfscope}%
\pgfpathrectangle{\pgfqpoint{0.697024in}{0.857143in}}{\pgfqpoint{2.627103in}{1.813434in}}%
\pgfusepath{clip}%
\pgfsetbuttcap%
\pgfsetmiterjoin%
\definecolor{currentfill}{rgb}{0.511253,0.510898,0.193296}%
\pgfsetfillcolor{currentfill}%
\pgfsetlinewidth{0.000000pt}%
\definecolor{currentstroke}{rgb}{0.000000,0.000000,0.000000}%
\pgfsetstrokecolor{currentstroke}%
\pgfsetstrokeopacity{0.000000}%
\pgfsetdash{}{0pt}%
\pgfpathmoveto{\pgfqpoint{1.922328in}{1.417943in}}%
\pgfpathlineto{\pgfqpoint{1.931265in}{1.417943in}}%
\pgfpathlineto{\pgfqpoint{1.931265in}{1.417799in}}%
\pgfpathlineto{\pgfqpoint{1.922328in}{1.417799in}}%
\pgfpathlineto{\pgfqpoint{1.922328in}{1.417943in}}%
\pgfpathclose%
\pgfusepath{fill}%
\end{pgfscope}%
\begin{pgfscope}%
\pgfpathrectangle{\pgfqpoint{0.697024in}{0.857143in}}{\pgfqpoint{2.627103in}{1.813434in}}%
\pgfusepath{clip}%
\pgfsetbuttcap%
\pgfsetmiterjoin%
\definecolor{currentfill}{rgb}{0.511253,0.510898,0.193296}%
\pgfsetfillcolor{currentfill}%
\pgfsetlinewidth{0.000000pt}%
\definecolor{currentstroke}{rgb}{0.000000,0.000000,0.000000}%
\pgfsetstrokecolor{currentstroke}%
\pgfsetstrokeopacity{0.000000}%
\pgfsetdash{}{0pt}%
\pgfpathmoveto{\pgfqpoint{1.933499in}{1.370585in}}%
\pgfpathlineto{\pgfqpoint{1.942435in}{1.370585in}}%
\pgfpathlineto{\pgfqpoint{1.942435in}{1.368682in}}%
\pgfpathlineto{\pgfqpoint{1.933499in}{1.368682in}}%
\pgfpathlineto{\pgfqpoint{1.933499in}{1.370585in}}%
\pgfpathclose%
\pgfusepath{fill}%
\end{pgfscope}%
\begin{pgfscope}%
\pgfpathrectangle{\pgfqpoint{0.697024in}{0.857143in}}{\pgfqpoint{2.627103in}{1.813434in}}%
\pgfusepath{clip}%
\pgfsetbuttcap%
\pgfsetmiterjoin%
\definecolor{currentfill}{rgb}{0.511253,0.510898,0.193296}%
\pgfsetfillcolor{currentfill}%
\pgfsetlinewidth{0.000000pt}%
\definecolor{currentstroke}{rgb}{0.000000,0.000000,0.000000}%
\pgfsetstrokecolor{currentstroke}%
\pgfsetstrokeopacity{0.000000}%
\pgfsetdash{}{0pt}%
\pgfpathmoveto{\pgfqpoint{1.944669in}{1.367968in}}%
\pgfpathlineto{\pgfqpoint{1.953606in}{1.367968in}}%
\pgfpathlineto{\pgfqpoint{1.953606in}{1.366518in}}%
\pgfpathlineto{\pgfqpoint{1.944669in}{1.366518in}}%
\pgfpathlineto{\pgfqpoint{1.944669in}{1.367968in}}%
\pgfpathclose%
\pgfusepath{fill}%
\end{pgfscope}%
\begin{pgfscope}%
\pgfpathrectangle{\pgfqpoint{0.697024in}{0.857143in}}{\pgfqpoint{2.627103in}{1.813434in}}%
\pgfusepath{clip}%
\pgfsetbuttcap%
\pgfsetmiterjoin%
\definecolor{currentfill}{rgb}{0.511253,0.510898,0.193296}%
\pgfsetfillcolor{currentfill}%
\pgfsetlinewidth{0.000000pt}%
\definecolor{currentstroke}{rgb}{0.000000,0.000000,0.000000}%
\pgfsetstrokecolor{currentstroke}%
\pgfsetstrokeopacity{0.000000}%
\pgfsetdash{}{0pt}%
\pgfpathmoveto{\pgfqpoint{1.955840in}{1.433285in}}%
\pgfpathlineto{\pgfqpoint{1.964776in}{1.433285in}}%
\pgfpathlineto{\pgfqpoint{1.964776in}{1.427387in}}%
\pgfpathlineto{\pgfqpoint{1.955840in}{1.427387in}}%
\pgfpathlineto{\pgfqpoint{1.955840in}{1.433285in}}%
\pgfpathclose%
\pgfusepath{fill}%
\end{pgfscope}%
\begin{pgfscope}%
\pgfpathrectangle{\pgfqpoint{0.697024in}{0.857143in}}{\pgfqpoint{2.627103in}{1.813434in}}%
\pgfusepath{clip}%
\pgfsetbuttcap%
\pgfsetmiterjoin%
\definecolor{currentfill}{rgb}{0.511253,0.510898,0.193296}%
\pgfsetfillcolor{currentfill}%
\pgfsetlinewidth{0.000000pt}%
\definecolor{currentstroke}{rgb}{0.000000,0.000000,0.000000}%
\pgfsetstrokecolor{currentstroke}%
\pgfsetstrokeopacity{0.000000}%
\pgfsetdash{}{0pt}%
\pgfpathmoveto{\pgfqpoint{1.967011in}{1.363036in}}%
\pgfpathlineto{\pgfqpoint{1.975947in}{1.363036in}}%
\pgfpathlineto{\pgfqpoint{1.975947in}{1.360234in}}%
\pgfpathlineto{\pgfqpoint{1.967011in}{1.360234in}}%
\pgfpathlineto{\pgfqpoint{1.967011in}{1.363036in}}%
\pgfpathclose%
\pgfusepath{fill}%
\end{pgfscope}%
\begin{pgfscope}%
\pgfpathrectangle{\pgfqpoint{0.697024in}{0.857143in}}{\pgfqpoint{2.627103in}{1.813434in}}%
\pgfusepath{clip}%
\pgfsetbuttcap%
\pgfsetmiterjoin%
\definecolor{currentfill}{rgb}{0.511253,0.510898,0.193296}%
\pgfsetfillcolor{currentfill}%
\pgfsetlinewidth{0.000000pt}%
\definecolor{currentstroke}{rgb}{0.000000,0.000000,0.000000}%
\pgfsetstrokecolor{currentstroke}%
\pgfsetstrokeopacity{0.000000}%
\pgfsetdash{}{0pt}%
\pgfpathmoveto{\pgfqpoint{1.978181in}{1.383499in}}%
\pgfpathlineto{\pgfqpoint{1.987118in}{1.383499in}}%
\pgfpathlineto{\pgfqpoint{1.987118in}{1.378682in}}%
\pgfpathlineto{\pgfqpoint{1.978181in}{1.378682in}}%
\pgfpathlineto{\pgfqpoint{1.978181in}{1.383499in}}%
\pgfpathclose%
\pgfusepath{fill}%
\end{pgfscope}%
\begin{pgfscope}%
\pgfpathrectangle{\pgfqpoint{0.697024in}{0.857143in}}{\pgfqpoint{2.627103in}{1.813434in}}%
\pgfusepath{clip}%
\pgfsetbuttcap%
\pgfsetmiterjoin%
\definecolor{currentfill}{rgb}{0.511253,0.510898,0.193296}%
\pgfsetfillcolor{currentfill}%
\pgfsetlinewidth{0.000000pt}%
\definecolor{currentstroke}{rgb}{0.000000,0.000000,0.000000}%
\pgfsetstrokecolor{currentstroke}%
\pgfsetstrokeopacity{0.000000}%
\pgfsetdash{}{0pt}%
\pgfpathmoveto{\pgfqpoint{1.989352in}{1.322762in}}%
\pgfpathlineto{\pgfqpoint{1.998288in}{1.322762in}}%
\pgfpathlineto{\pgfqpoint{1.998288in}{1.317660in}}%
\pgfpathlineto{\pgfqpoint{1.989352in}{1.317660in}}%
\pgfpathlineto{\pgfqpoint{1.989352in}{1.322762in}}%
\pgfpathclose%
\pgfusepath{fill}%
\end{pgfscope}%
\begin{pgfscope}%
\pgfpathrectangle{\pgfqpoint{0.697024in}{0.857143in}}{\pgfqpoint{2.627103in}{1.813434in}}%
\pgfusepath{clip}%
\pgfsetbuttcap%
\pgfsetmiterjoin%
\definecolor{currentfill}{rgb}{0.511253,0.510898,0.193296}%
\pgfsetfillcolor{currentfill}%
\pgfsetlinewidth{0.000000pt}%
\definecolor{currentstroke}{rgb}{0.000000,0.000000,0.000000}%
\pgfsetstrokecolor{currentstroke}%
\pgfsetstrokeopacity{0.000000}%
\pgfsetdash{}{0pt}%
\pgfpathmoveto{\pgfqpoint{2.000522in}{1.362698in}}%
\pgfpathlineto{\pgfqpoint{2.009459in}{1.362698in}}%
\pgfpathlineto{\pgfqpoint{2.009459in}{1.356778in}}%
\pgfpathlineto{\pgfqpoint{2.000522in}{1.356778in}}%
\pgfpathlineto{\pgfqpoint{2.000522in}{1.362698in}}%
\pgfpathclose%
\pgfusepath{fill}%
\end{pgfscope}%
\begin{pgfscope}%
\pgfpathrectangle{\pgfqpoint{0.697024in}{0.857143in}}{\pgfqpoint{2.627103in}{1.813434in}}%
\pgfusepath{clip}%
\pgfsetbuttcap%
\pgfsetmiterjoin%
\definecolor{currentfill}{rgb}{0.511253,0.510898,0.193296}%
\pgfsetfillcolor{currentfill}%
\pgfsetlinewidth{0.000000pt}%
\definecolor{currentstroke}{rgb}{0.000000,0.000000,0.000000}%
\pgfsetstrokecolor{currentstroke}%
\pgfsetstrokeopacity{0.000000}%
\pgfsetdash{}{0pt}%
\pgfpathmoveto{\pgfqpoint{2.011693in}{1.309497in}}%
\pgfpathlineto{\pgfqpoint{2.020629in}{1.309497in}}%
\pgfpathlineto{\pgfqpoint{2.020629in}{1.304682in}}%
\pgfpathlineto{\pgfqpoint{2.011693in}{1.304682in}}%
\pgfpathlineto{\pgfqpoint{2.011693in}{1.309497in}}%
\pgfpathclose%
\pgfusepath{fill}%
\end{pgfscope}%
\begin{pgfscope}%
\pgfpathrectangle{\pgfqpoint{0.697024in}{0.857143in}}{\pgfqpoint{2.627103in}{1.813434in}}%
\pgfusepath{clip}%
\pgfsetbuttcap%
\pgfsetmiterjoin%
\definecolor{currentfill}{rgb}{0.511253,0.510898,0.193296}%
\pgfsetfillcolor{currentfill}%
\pgfsetlinewidth{0.000000pt}%
\definecolor{currentstroke}{rgb}{0.000000,0.000000,0.000000}%
\pgfsetstrokecolor{currentstroke}%
\pgfsetstrokeopacity{0.000000}%
\pgfsetdash{}{0pt}%
\pgfpathmoveto{\pgfqpoint{2.022864in}{1.306402in}}%
\pgfpathlineto{\pgfqpoint{2.031800in}{1.306402in}}%
\pgfpathlineto{\pgfqpoint{2.031800in}{1.300277in}}%
\pgfpathlineto{\pgfqpoint{2.022864in}{1.300277in}}%
\pgfpathlineto{\pgfqpoint{2.022864in}{1.306402in}}%
\pgfpathclose%
\pgfusepath{fill}%
\end{pgfscope}%
\begin{pgfscope}%
\pgfpathrectangle{\pgfqpoint{0.697024in}{0.857143in}}{\pgfqpoint{2.627103in}{1.813434in}}%
\pgfusepath{clip}%
\pgfsetbuttcap%
\pgfsetmiterjoin%
\definecolor{currentfill}{rgb}{0.511253,0.510898,0.193296}%
\pgfsetfillcolor{currentfill}%
\pgfsetlinewidth{0.000000pt}%
\definecolor{currentstroke}{rgb}{0.000000,0.000000,0.000000}%
\pgfsetstrokecolor{currentstroke}%
\pgfsetstrokeopacity{0.000000}%
\pgfsetdash{}{0pt}%
\pgfpathmoveto{\pgfqpoint{2.034034in}{1.323674in}}%
\pgfpathlineto{\pgfqpoint{2.042971in}{1.323674in}}%
\pgfpathlineto{\pgfqpoint{2.042971in}{1.319027in}}%
\pgfpathlineto{\pgfqpoint{2.034034in}{1.319027in}}%
\pgfpathlineto{\pgfqpoint{2.034034in}{1.323674in}}%
\pgfpathclose%
\pgfusepath{fill}%
\end{pgfscope}%
\begin{pgfscope}%
\pgfpathrectangle{\pgfqpoint{0.697024in}{0.857143in}}{\pgfqpoint{2.627103in}{1.813434in}}%
\pgfusepath{clip}%
\pgfsetbuttcap%
\pgfsetmiterjoin%
\definecolor{currentfill}{rgb}{0.511253,0.510898,0.193296}%
\pgfsetfillcolor{currentfill}%
\pgfsetlinewidth{0.000000pt}%
\definecolor{currentstroke}{rgb}{0.000000,0.000000,0.000000}%
\pgfsetstrokecolor{currentstroke}%
\pgfsetstrokeopacity{0.000000}%
\pgfsetdash{}{0pt}%
\pgfpathmoveto{\pgfqpoint{2.045205in}{1.358504in}}%
\pgfpathlineto{\pgfqpoint{2.054141in}{1.358504in}}%
\pgfpathlineto{\pgfqpoint{2.054141in}{1.351788in}}%
\pgfpathlineto{\pgfqpoint{2.045205in}{1.351788in}}%
\pgfpathlineto{\pgfqpoint{2.045205in}{1.358504in}}%
\pgfpathclose%
\pgfusepath{fill}%
\end{pgfscope}%
\begin{pgfscope}%
\pgfpathrectangle{\pgfqpoint{0.697024in}{0.857143in}}{\pgfqpoint{2.627103in}{1.813434in}}%
\pgfusepath{clip}%
\pgfsetbuttcap%
\pgfsetmiterjoin%
\definecolor{currentfill}{rgb}{0.511253,0.510898,0.193296}%
\pgfsetfillcolor{currentfill}%
\pgfsetlinewidth{0.000000pt}%
\definecolor{currentstroke}{rgb}{0.000000,0.000000,0.000000}%
\pgfsetstrokecolor{currentstroke}%
\pgfsetstrokeopacity{0.000000}%
\pgfsetdash{}{0pt}%
\pgfpathmoveto{\pgfqpoint{2.056375in}{1.336991in}}%
\pgfpathlineto{\pgfqpoint{2.065312in}{1.336991in}}%
\pgfpathlineto{\pgfqpoint{2.065312in}{1.327622in}}%
\pgfpathlineto{\pgfqpoint{2.056375in}{1.327622in}}%
\pgfpathlineto{\pgfqpoint{2.056375in}{1.336991in}}%
\pgfpathclose%
\pgfusepath{fill}%
\end{pgfscope}%
\begin{pgfscope}%
\pgfpathrectangle{\pgfqpoint{0.697024in}{0.857143in}}{\pgfqpoint{2.627103in}{1.813434in}}%
\pgfusepath{clip}%
\pgfsetbuttcap%
\pgfsetmiterjoin%
\definecolor{currentfill}{rgb}{0.511253,0.510898,0.193296}%
\pgfsetfillcolor{currentfill}%
\pgfsetlinewidth{0.000000pt}%
\definecolor{currentstroke}{rgb}{0.000000,0.000000,0.000000}%
\pgfsetstrokecolor{currentstroke}%
\pgfsetstrokeopacity{0.000000}%
\pgfsetdash{}{0pt}%
\pgfpathmoveto{\pgfqpoint{2.067546in}{1.321007in}}%
\pgfpathlineto{\pgfqpoint{2.076482in}{1.321007in}}%
\pgfpathlineto{\pgfqpoint{2.076482in}{1.310472in}}%
\pgfpathlineto{\pgfqpoint{2.067546in}{1.310472in}}%
\pgfpathlineto{\pgfqpoint{2.067546in}{1.321007in}}%
\pgfpathclose%
\pgfusepath{fill}%
\end{pgfscope}%
\begin{pgfscope}%
\pgfpathrectangle{\pgfqpoint{0.697024in}{0.857143in}}{\pgfqpoint{2.627103in}{1.813434in}}%
\pgfusepath{clip}%
\pgfsetbuttcap%
\pgfsetmiterjoin%
\definecolor{currentfill}{rgb}{0.511253,0.510898,0.193296}%
\pgfsetfillcolor{currentfill}%
\pgfsetlinewidth{0.000000pt}%
\definecolor{currentstroke}{rgb}{0.000000,0.000000,0.000000}%
\pgfsetstrokecolor{currentstroke}%
\pgfsetstrokeopacity{0.000000}%
\pgfsetdash{}{0pt}%
\pgfpathmoveto{\pgfqpoint{2.078717in}{1.341642in}}%
\pgfpathlineto{\pgfqpoint{2.087653in}{1.341642in}}%
\pgfpathlineto{\pgfqpoint{2.087653in}{1.329628in}}%
\pgfpathlineto{\pgfqpoint{2.078717in}{1.329628in}}%
\pgfpathlineto{\pgfqpoint{2.078717in}{1.341642in}}%
\pgfpathclose%
\pgfusepath{fill}%
\end{pgfscope}%
\begin{pgfscope}%
\pgfpathrectangle{\pgfqpoint{0.697024in}{0.857143in}}{\pgfqpoint{2.627103in}{1.813434in}}%
\pgfusepath{clip}%
\pgfsetbuttcap%
\pgfsetmiterjoin%
\definecolor{currentfill}{rgb}{0.511253,0.510898,0.193296}%
\pgfsetfillcolor{currentfill}%
\pgfsetlinewidth{0.000000pt}%
\definecolor{currentstroke}{rgb}{0.000000,0.000000,0.000000}%
\pgfsetstrokecolor{currentstroke}%
\pgfsetstrokeopacity{0.000000}%
\pgfsetdash{}{0pt}%
\pgfpathmoveto{\pgfqpoint{2.089887in}{1.329493in}}%
\pgfpathlineto{\pgfqpoint{2.098824in}{1.329493in}}%
\pgfpathlineto{\pgfqpoint{2.098824in}{1.317791in}}%
\pgfpathlineto{\pgfqpoint{2.089887in}{1.317791in}}%
\pgfpathlineto{\pgfqpoint{2.089887in}{1.329493in}}%
\pgfpathclose%
\pgfusepath{fill}%
\end{pgfscope}%
\begin{pgfscope}%
\pgfpathrectangle{\pgfqpoint{0.697024in}{0.857143in}}{\pgfqpoint{2.627103in}{1.813434in}}%
\pgfusepath{clip}%
\pgfsetbuttcap%
\pgfsetmiterjoin%
\definecolor{currentfill}{rgb}{0.511253,0.510898,0.193296}%
\pgfsetfillcolor{currentfill}%
\pgfsetlinewidth{0.000000pt}%
\definecolor{currentstroke}{rgb}{0.000000,0.000000,0.000000}%
\pgfsetstrokecolor{currentstroke}%
\pgfsetstrokeopacity{0.000000}%
\pgfsetdash{}{0pt}%
\pgfpathmoveto{\pgfqpoint{2.101058in}{1.363601in}}%
\pgfpathlineto{\pgfqpoint{2.109994in}{1.363601in}}%
\pgfpathlineto{\pgfqpoint{2.109994in}{1.351299in}}%
\pgfpathlineto{\pgfqpoint{2.101058in}{1.351299in}}%
\pgfpathlineto{\pgfqpoint{2.101058in}{1.363601in}}%
\pgfpathclose%
\pgfusepath{fill}%
\end{pgfscope}%
\begin{pgfscope}%
\pgfpathrectangle{\pgfqpoint{0.697024in}{0.857143in}}{\pgfqpoint{2.627103in}{1.813434in}}%
\pgfusepath{clip}%
\pgfsetbuttcap%
\pgfsetmiterjoin%
\definecolor{currentfill}{rgb}{0.511253,0.510898,0.193296}%
\pgfsetfillcolor{currentfill}%
\pgfsetlinewidth{0.000000pt}%
\definecolor{currentstroke}{rgb}{0.000000,0.000000,0.000000}%
\pgfsetstrokecolor{currentstroke}%
\pgfsetstrokeopacity{0.000000}%
\pgfsetdash{}{0pt}%
\pgfpathmoveto{\pgfqpoint{2.112228in}{1.402872in}}%
\pgfpathlineto{\pgfqpoint{2.121165in}{1.402872in}}%
\pgfpathlineto{\pgfqpoint{2.121165in}{1.389882in}}%
\pgfpathlineto{\pgfqpoint{2.112228in}{1.389882in}}%
\pgfpathlineto{\pgfqpoint{2.112228in}{1.402872in}}%
\pgfpathclose%
\pgfusepath{fill}%
\end{pgfscope}%
\begin{pgfscope}%
\pgfpathrectangle{\pgfqpoint{0.697024in}{0.857143in}}{\pgfqpoint{2.627103in}{1.813434in}}%
\pgfusepath{clip}%
\pgfsetbuttcap%
\pgfsetmiterjoin%
\definecolor{currentfill}{rgb}{0.511253,0.510898,0.193296}%
\pgfsetfillcolor{currentfill}%
\pgfsetlinewidth{0.000000pt}%
\definecolor{currentstroke}{rgb}{0.000000,0.000000,0.000000}%
\pgfsetstrokecolor{currentstroke}%
\pgfsetstrokeopacity{0.000000}%
\pgfsetdash{}{0pt}%
\pgfpathmoveto{\pgfqpoint{2.123399in}{1.389063in}}%
\pgfpathlineto{\pgfqpoint{2.132335in}{1.389063in}}%
\pgfpathlineto{\pgfqpoint{2.132335in}{1.377083in}}%
\pgfpathlineto{\pgfqpoint{2.123399in}{1.377083in}}%
\pgfpathlineto{\pgfqpoint{2.123399in}{1.389063in}}%
\pgfpathclose%
\pgfusepath{fill}%
\end{pgfscope}%
\begin{pgfscope}%
\pgfpathrectangle{\pgfqpoint{0.697024in}{0.857143in}}{\pgfqpoint{2.627103in}{1.813434in}}%
\pgfusepath{clip}%
\pgfsetbuttcap%
\pgfsetmiterjoin%
\definecolor{currentfill}{rgb}{0.511253,0.510898,0.193296}%
\pgfsetfillcolor{currentfill}%
\pgfsetlinewidth{0.000000pt}%
\definecolor{currentstroke}{rgb}{0.000000,0.000000,0.000000}%
\pgfsetstrokecolor{currentstroke}%
\pgfsetstrokeopacity{0.000000}%
\pgfsetdash{}{0pt}%
\pgfpathmoveto{\pgfqpoint{2.134570in}{1.408360in}}%
\pgfpathlineto{\pgfqpoint{2.143506in}{1.408360in}}%
\pgfpathlineto{\pgfqpoint{2.143506in}{1.395204in}}%
\pgfpathlineto{\pgfqpoint{2.134570in}{1.395204in}}%
\pgfpathlineto{\pgfqpoint{2.134570in}{1.408360in}}%
\pgfpathclose%
\pgfusepath{fill}%
\end{pgfscope}%
\begin{pgfscope}%
\pgfpathrectangle{\pgfqpoint{0.697024in}{0.857143in}}{\pgfqpoint{2.627103in}{1.813434in}}%
\pgfusepath{clip}%
\pgfsetbuttcap%
\pgfsetmiterjoin%
\definecolor{currentfill}{rgb}{0.511253,0.510898,0.193296}%
\pgfsetfillcolor{currentfill}%
\pgfsetlinewidth{0.000000pt}%
\definecolor{currentstroke}{rgb}{0.000000,0.000000,0.000000}%
\pgfsetstrokecolor{currentstroke}%
\pgfsetstrokeopacity{0.000000}%
\pgfsetdash{}{0pt}%
\pgfpathmoveto{\pgfqpoint{2.145740in}{1.441896in}}%
\pgfpathlineto{\pgfqpoint{2.154677in}{1.441896in}}%
\pgfpathlineto{\pgfqpoint{2.154677in}{1.428593in}}%
\pgfpathlineto{\pgfqpoint{2.145740in}{1.428593in}}%
\pgfpathlineto{\pgfqpoint{2.145740in}{1.441896in}}%
\pgfpathclose%
\pgfusepath{fill}%
\end{pgfscope}%
\begin{pgfscope}%
\pgfpathrectangle{\pgfqpoint{0.697024in}{0.857143in}}{\pgfqpoint{2.627103in}{1.813434in}}%
\pgfusepath{clip}%
\pgfsetbuttcap%
\pgfsetmiterjoin%
\definecolor{currentfill}{rgb}{0.511253,0.510898,0.193296}%
\pgfsetfillcolor{currentfill}%
\pgfsetlinewidth{0.000000pt}%
\definecolor{currentstroke}{rgb}{0.000000,0.000000,0.000000}%
\pgfsetstrokecolor{currentstroke}%
\pgfsetstrokeopacity{0.000000}%
\pgfsetdash{}{0pt}%
\pgfpathmoveto{\pgfqpoint{2.156911in}{1.484292in}}%
\pgfpathlineto{\pgfqpoint{2.165847in}{1.484292in}}%
\pgfpathlineto{\pgfqpoint{2.165847in}{1.472447in}}%
\pgfpathlineto{\pgfqpoint{2.156911in}{1.472447in}}%
\pgfpathlineto{\pgfqpoint{2.156911in}{1.484292in}}%
\pgfpathclose%
\pgfusepath{fill}%
\end{pgfscope}%
\begin{pgfscope}%
\pgfpathrectangle{\pgfqpoint{0.697024in}{0.857143in}}{\pgfqpoint{2.627103in}{1.813434in}}%
\pgfusepath{clip}%
\pgfsetbuttcap%
\pgfsetmiterjoin%
\definecolor{currentfill}{rgb}{0.511253,0.510898,0.193296}%
\pgfsetfillcolor{currentfill}%
\pgfsetlinewidth{0.000000pt}%
\definecolor{currentstroke}{rgb}{0.000000,0.000000,0.000000}%
\pgfsetstrokecolor{currentstroke}%
\pgfsetstrokeopacity{0.000000}%
\pgfsetdash{}{0pt}%
\pgfpathmoveto{\pgfqpoint{2.168081in}{1.437732in}}%
\pgfpathlineto{\pgfqpoint{2.177018in}{1.437732in}}%
\pgfpathlineto{\pgfqpoint{2.177018in}{1.424961in}}%
\pgfpathlineto{\pgfqpoint{2.168081in}{1.424961in}}%
\pgfpathlineto{\pgfqpoint{2.168081in}{1.437732in}}%
\pgfpathclose%
\pgfusepath{fill}%
\end{pgfscope}%
\begin{pgfscope}%
\pgfpathrectangle{\pgfqpoint{0.697024in}{0.857143in}}{\pgfqpoint{2.627103in}{1.813434in}}%
\pgfusepath{clip}%
\pgfsetbuttcap%
\pgfsetmiterjoin%
\definecolor{currentfill}{rgb}{0.511253,0.510898,0.193296}%
\pgfsetfillcolor{currentfill}%
\pgfsetlinewidth{0.000000pt}%
\definecolor{currentstroke}{rgb}{0.000000,0.000000,0.000000}%
\pgfsetstrokecolor{currentstroke}%
\pgfsetstrokeopacity{0.000000}%
\pgfsetdash{}{0pt}%
\pgfpathmoveto{\pgfqpoint{2.179252in}{1.476099in}}%
\pgfpathlineto{\pgfqpoint{2.188189in}{1.476099in}}%
\pgfpathlineto{\pgfqpoint{2.188189in}{1.462273in}}%
\pgfpathlineto{\pgfqpoint{2.179252in}{1.462273in}}%
\pgfpathlineto{\pgfqpoint{2.179252in}{1.476099in}}%
\pgfpathclose%
\pgfusepath{fill}%
\end{pgfscope}%
\begin{pgfscope}%
\pgfpathrectangle{\pgfqpoint{0.697024in}{0.857143in}}{\pgfqpoint{2.627103in}{1.813434in}}%
\pgfusepath{clip}%
\pgfsetbuttcap%
\pgfsetmiterjoin%
\definecolor{currentfill}{rgb}{0.511253,0.510898,0.193296}%
\pgfsetfillcolor{currentfill}%
\pgfsetlinewidth{0.000000pt}%
\definecolor{currentstroke}{rgb}{0.000000,0.000000,0.000000}%
\pgfsetstrokecolor{currentstroke}%
\pgfsetstrokeopacity{0.000000}%
\pgfsetdash{}{0pt}%
\pgfpathmoveto{\pgfqpoint{2.190423in}{1.589298in}}%
\pgfpathlineto{\pgfqpoint{2.199359in}{1.589298in}}%
\pgfpathlineto{\pgfqpoint{2.199359in}{1.577718in}}%
\pgfpathlineto{\pgfqpoint{2.190423in}{1.577718in}}%
\pgfpathlineto{\pgfqpoint{2.190423in}{1.589298in}}%
\pgfpathclose%
\pgfusepath{fill}%
\end{pgfscope}%
\begin{pgfscope}%
\pgfpathrectangle{\pgfqpoint{0.697024in}{0.857143in}}{\pgfqpoint{2.627103in}{1.813434in}}%
\pgfusepath{clip}%
\pgfsetbuttcap%
\pgfsetmiterjoin%
\definecolor{currentfill}{rgb}{0.511253,0.510898,0.193296}%
\pgfsetfillcolor{currentfill}%
\pgfsetlinewidth{0.000000pt}%
\definecolor{currentstroke}{rgb}{0.000000,0.000000,0.000000}%
\pgfsetstrokecolor{currentstroke}%
\pgfsetstrokeopacity{0.000000}%
\pgfsetdash{}{0pt}%
\pgfpathmoveto{\pgfqpoint{2.201593in}{1.568136in}}%
\pgfpathlineto{\pgfqpoint{2.210530in}{1.568136in}}%
\pgfpathlineto{\pgfqpoint{2.210530in}{1.554149in}}%
\pgfpathlineto{\pgfqpoint{2.201593in}{1.554149in}}%
\pgfpathlineto{\pgfqpoint{2.201593in}{1.568136in}}%
\pgfpathclose%
\pgfusepath{fill}%
\end{pgfscope}%
\begin{pgfscope}%
\pgfpathrectangle{\pgfqpoint{0.697024in}{0.857143in}}{\pgfqpoint{2.627103in}{1.813434in}}%
\pgfusepath{clip}%
\pgfsetbuttcap%
\pgfsetmiterjoin%
\definecolor{currentfill}{rgb}{0.511253,0.510898,0.193296}%
\pgfsetfillcolor{currentfill}%
\pgfsetlinewidth{0.000000pt}%
\definecolor{currentstroke}{rgb}{0.000000,0.000000,0.000000}%
\pgfsetstrokecolor{currentstroke}%
\pgfsetstrokeopacity{0.000000}%
\pgfsetdash{}{0pt}%
\pgfpathmoveto{\pgfqpoint{2.212764in}{1.662130in}}%
\pgfpathlineto{\pgfqpoint{2.221700in}{1.662130in}}%
\pgfpathlineto{\pgfqpoint{2.221700in}{1.647496in}}%
\pgfpathlineto{\pgfqpoint{2.212764in}{1.647496in}}%
\pgfpathlineto{\pgfqpoint{2.212764in}{1.662130in}}%
\pgfpathclose%
\pgfusepath{fill}%
\end{pgfscope}%
\begin{pgfscope}%
\pgfpathrectangle{\pgfqpoint{0.697024in}{0.857143in}}{\pgfqpoint{2.627103in}{1.813434in}}%
\pgfusepath{clip}%
\pgfsetbuttcap%
\pgfsetmiterjoin%
\definecolor{currentfill}{rgb}{0.511253,0.510898,0.193296}%
\pgfsetfillcolor{currentfill}%
\pgfsetlinewidth{0.000000pt}%
\definecolor{currentstroke}{rgb}{0.000000,0.000000,0.000000}%
\pgfsetstrokecolor{currentstroke}%
\pgfsetstrokeopacity{0.000000}%
\pgfsetdash{}{0pt}%
\pgfpathmoveto{\pgfqpoint{2.223934in}{1.713613in}}%
\pgfpathlineto{\pgfqpoint{2.232871in}{1.713613in}}%
\pgfpathlineto{\pgfqpoint{2.232871in}{1.701506in}}%
\pgfpathlineto{\pgfqpoint{2.223934in}{1.701506in}}%
\pgfpathlineto{\pgfqpoint{2.223934in}{1.713613in}}%
\pgfpathclose%
\pgfusepath{fill}%
\end{pgfscope}%
\begin{pgfscope}%
\pgfpathrectangle{\pgfqpoint{0.697024in}{0.857143in}}{\pgfqpoint{2.627103in}{1.813434in}}%
\pgfusepath{clip}%
\pgfsetbuttcap%
\pgfsetmiterjoin%
\definecolor{currentfill}{rgb}{0.511253,0.510898,0.193296}%
\pgfsetfillcolor{currentfill}%
\pgfsetlinewidth{0.000000pt}%
\definecolor{currentstroke}{rgb}{0.000000,0.000000,0.000000}%
\pgfsetstrokecolor{currentstroke}%
\pgfsetstrokeopacity{0.000000}%
\pgfsetdash{}{0pt}%
\pgfpathmoveto{\pgfqpoint{2.235105in}{1.713644in}}%
\pgfpathlineto{\pgfqpoint{2.244042in}{1.713644in}}%
\pgfpathlineto{\pgfqpoint{2.244042in}{1.701649in}}%
\pgfpathlineto{\pgfqpoint{2.235105in}{1.701649in}}%
\pgfpathlineto{\pgfqpoint{2.235105in}{1.713644in}}%
\pgfpathclose%
\pgfusepath{fill}%
\end{pgfscope}%
\begin{pgfscope}%
\pgfpathrectangle{\pgfqpoint{0.697024in}{0.857143in}}{\pgfqpoint{2.627103in}{1.813434in}}%
\pgfusepath{clip}%
\pgfsetbuttcap%
\pgfsetmiterjoin%
\definecolor{currentfill}{rgb}{0.511253,0.510898,0.193296}%
\pgfsetfillcolor{currentfill}%
\pgfsetlinewidth{0.000000pt}%
\definecolor{currentstroke}{rgb}{0.000000,0.000000,0.000000}%
\pgfsetstrokecolor{currentstroke}%
\pgfsetstrokeopacity{0.000000}%
\pgfsetdash{}{0pt}%
\pgfpathmoveto{\pgfqpoint{2.246276in}{1.713106in}}%
\pgfpathlineto{\pgfqpoint{2.255212in}{1.713106in}}%
\pgfpathlineto{\pgfqpoint{2.255212in}{1.703886in}}%
\pgfpathlineto{\pgfqpoint{2.246276in}{1.703886in}}%
\pgfpathlineto{\pgfqpoint{2.246276in}{1.713106in}}%
\pgfpathclose%
\pgfusepath{fill}%
\end{pgfscope}%
\begin{pgfscope}%
\pgfpathrectangle{\pgfqpoint{0.697024in}{0.857143in}}{\pgfqpoint{2.627103in}{1.813434in}}%
\pgfusepath{clip}%
\pgfsetbuttcap%
\pgfsetmiterjoin%
\definecolor{currentfill}{rgb}{0.511253,0.510898,0.193296}%
\pgfsetfillcolor{currentfill}%
\pgfsetlinewidth{0.000000pt}%
\definecolor{currentstroke}{rgb}{0.000000,0.000000,0.000000}%
\pgfsetstrokecolor{currentstroke}%
\pgfsetstrokeopacity{0.000000}%
\pgfsetdash{}{0pt}%
\pgfpathmoveto{\pgfqpoint{2.257446in}{1.731222in}}%
\pgfpathlineto{\pgfqpoint{2.266383in}{1.731222in}}%
\pgfpathlineto{\pgfqpoint{2.266383in}{1.723756in}}%
\pgfpathlineto{\pgfqpoint{2.257446in}{1.723756in}}%
\pgfpathlineto{\pgfqpoint{2.257446in}{1.731222in}}%
\pgfpathclose%
\pgfusepath{fill}%
\end{pgfscope}%
\begin{pgfscope}%
\pgfpathrectangle{\pgfqpoint{0.697024in}{0.857143in}}{\pgfqpoint{2.627103in}{1.813434in}}%
\pgfusepath{clip}%
\pgfsetbuttcap%
\pgfsetmiterjoin%
\definecolor{currentfill}{rgb}{0.511253,0.510898,0.193296}%
\pgfsetfillcolor{currentfill}%
\pgfsetlinewidth{0.000000pt}%
\definecolor{currentstroke}{rgb}{0.000000,0.000000,0.000000}%
\pgfsetstrokecolor{currentstroke}%
\pgfsetstrokeopacity{0.000000}%
\pgfsetdash{}{0pt}%
\pgfpathmoveto{\pgfqpoint{2.268617in}{1.750385in}}%
\pgfpathlineto{\pgfqpoint{2.277553in}{1.750385in}}%
\pgfpathlineto{\pgfqpoint{2.277553in}{1.743785in}}%
\pgfpathlineto{\pgfqpoint{2.268617in}{1.743785in}}%
\pgfpathlineto{\pgfqpoint{2.268617in}{1.750385in}}%
\pgfpathclose%
\pgfusepath{fill}%
\end{pgfscope}%
\begin{pgfscope}%
\pgfpathrectangle{\pgfqpoint{0.697024in}{0.857143in}}{\pgfqpoint{2.627103in}{1.813434in}}%
\pgfusepath{clip}%
\pgfsetbuttcap%
\pgfsetmiterjoin%
\definecolor{currentfill}{rgb}{0.511253,0.510898,0.193296}%
\pgfsetfillcolor{currentfill}%
\pgfsetlinewidth{0.000000pt}%
\definecolor{currentstroke}{rgb}{0.000000,0.000000,0.000000}%
\pgfsetstrokecolor{currentstroke}%
\pgfsetstrokeopacity{0.000000}%
\pgfsetdash{}{0pt}%
\pgfpathmoveto{\pgfqpoint{2.279787in}{1.743937in}}%
\pgfpathlineto{\pgfqpoint{2.288724in}{1.743937in}}%
\pgfpathlineto{\pgfqpoint{2.288724in}{1.736913in}}%
\pgfpathlineto{\pgfqpoint{2.279787in}{1.736913in}}%
\pgfpathlineto{\pgfqpoint{2.279787in}{1.743937in}}%
\pgfpathclose%
\pgfusepath{fill}%
\end{pgfscope}%
\begin{pgfscope}%
\pgfpathrectangle{\pgfqpoint{0.697024in}{0.857143in}}{\pgfqpoint{2.627103in}{1.813434in}}%
\pgfusepath{clip}%
\pgfsetbuttcap%
\pgfsetmiterjoin%
\definecolor{currentfill}{rgb}{0.511253,0.510898,0.193296}%
\pgfsetfillcolor{currentfill}%
\pgfsetlinewidth{0.000000pt}%
\definecolor{currentstroke}{rgb}{0.000000,0.000000,0.000000}%
\pgfsetstrokecolor{currentstroke}%
\pgfsetstrokeopacity{0.000000}%
\pgfsetdash{}{0pt}%
\pgfpathmoveto{\pgfqpoint{2.290958in}{1.751497in}}%
\pgfpathlineto{\pgfqpoint{2.299895in}{1.751497in}}%
\pgfpathlineto{\pgfqpoint{2.299895in}{1.748082in}}%
\pgfpathlineto{\pgfqpoint{2.290958in}{1.748082in}}%
\pgfpathlineto{\pgfqpoint{2.290958in}{1.751497in}}%
\pgfpathclose%
\pgfusepath{fill}%
\end{pgfscope}%
\begin{pgfscope}%
\pgfpathrectangle{\pgfqpoint{0.697024in}{0.857143in}}{\pgfqpoint{2.627103in}{1.813434in}}%
\pgfusepath{clip}%
\pgfsetbuttcap%
\pgfsetmiterjoin%
\definecolor{currentfill}{rgb}{0.511253,0.510898,0.193296}%
\pgfsetfillcolor{currentfill}%
\pgfsetlinewidth{0.000000pt}%
\definecolor{currentstroke}{rgb}{0.000000,0.000000,0.000000}%
\pgfsetstrokecolor{currentstroke}%
\pgfsetstrokeopacity{0.000000}%
\pgfsetdash{}{0pt}%
\pgfpathmoveto{\pgfqpoint{2.302129in}{1.774356in}}%
\pgfpathlineto{\pgfqpoint{2.311065in}{1.774356in}}%
\pgfpathlineto{\pgfqpoint{2.311065in}{1.771461in}}%
\pgfpathlineto{\pgfqpoint{2.302129in}{1.771461in}}%
\pgfpathlineto{\pgfqpoint{2.302129in}{1.774356in}}%
\pgfpathclose%
\pgfusepath{fill}%
\end{pgfscope}%
\begin{pgfscope}%
\pgfpathrectangle{\pgfqpoint{0.697024in}{0.857143in}}{\pgfqpoint{2.627103in}{1.813434in}}%
\pgfusepath{clip}%
\pgfsetbuttcap%
\pgfsetmiterjoin%
\definecolor{currentfill}{rgb}{0.511253,0.510898,0.193296}%
\pgfsetfillcolor{currentfill}%
\pgfsetlinewidth{0.000000pt}%
\definecolor{currentstroke}{rgb}{0.000000,0.000000,0.000000}%
\pgfsetstrokecolor{currentstroke}%
\pgfsetstrokeopacity{0.000000}%
\pgfsetdash{}{0pt}%
\pgfpathmoveto{\pgfqpoint{2.313299in}{1.756040in}}%
\pgfpathlineto{\pgfqpoint{2.322236in}{1.756040in}}%
\pgfpathlineto{\pgfqpoint{2.322236in}{1.751222in}}%
\pgfpathlineto{\pgfqpoint{2.313299in}{1.751222in}}%
\pgfpathlineto{\pgfqpoint{2.313299in}{1.756040in}}%
\pgfpathclose%
\pgfusepath{fill}%
\end{pgfscope}%
\begin{pgfscope}%
\pgfpathrectangle{\pgfqpoint{0.697024in}{0.857143in}}{\pgfqpoint{2.627103in}{1.813434in}}%
\pgfusepath{clip}%
\pgfsetbuttcap%
\pgfsetmiterjoin%
\definecolor{currentfill}{rgb}{0.511253,0.510898,0.193296}%
\pgfsetfillcolor{currentfill}%
\pgfsetlinewidth{0.000000pt}%
\definecolor{currentstroke}{rgb}{0.000000,0.000000,0.000000}%
\pgfsetstrokecolor{currentstroke}%
\pgfsetstrokeopacity{0.000000}%
\pgfsetdash{}{0pt}%
\pgfpathmoveto{\pgfqpoint{2.324470in}{1.790197in}}%
\pgfpathlineto{\pgfqpoint{2.333406in}{1.790197in}}%
\pgfpathlineto{\pgfqpoint{2.333406in}{1.788999in}}%
\pgfpathlineto{\pgfqpoint{2.324470in}{1.788999in}}%
\pgfpathlineto{\pgfqpoint{2.324470in}{1.790197in}}%
\pgfpathclose%
\pgfusepath{fill}%
\end{pgfscope}%
\begin{pgfscope}%
\pgfpathrectangle{\pgfqpoint{0.697024in}{0.857143in}}{\pgfqpoint{2.627103in}{1.813434in}}%
\pgfusepath{clip}%
\pgfsetbuttcap%
\pgfsetmiterjoin%
\definecolor{currentfill}{rgb}{0.511253,0.510898,0.193296}%
\pgfsetfillcolor{currentfill}%
\pgfsetlinewidth{0.000000pt}%
\definecolor{currentstroke}{rgb}{0.000000,0.000000,0.000000}%
\pgfsetstrokecolor{currentstroke}%
\pgfsetstrokeopacity{0.000000}%
\pgfsetdash{}{0pt}%
\pgfpathmoveto{\pgfqpoint{2.335640in}{1.777902in}}%
\pgfpathlineto{\pgfqpoint{2.344577in}{1.777902in}}%
\pgfpathlineto{\pgfqpoint{2.344577in}{1.775373in}}%
\pgfpathlineto{\pgfqpoint{2.335640in}{1.775373in}}%
\pgfpathlineto{\pgfqpoint{2.335640in}{1.777902in}}%
\pgfpathclose%
\pgfusepath{fill}%
\end{pgfscope}%
\begin{pgfscope}%
\pgfpathrectangle{\pgfqpoint{0.697024in}{0.857143in}}{\pgfqpoint{2.627103in}{1.813434in}}%
\pgfusepath{clip}%
\pgfsetbuttcap%
\pgfsetmiterjoin%
\definecolor{currentfill}{rgb}{0.511253,0.510898,0.193296}%
\pgfsetfillcolor{currentfill}%
\pgfsetlinewidth{0.000000pt}%
\definecolor{currentstroke}{rgb}{0.000000,0.000000,0.000000}%
\pgfsetstrokecolor{currentstroke}%
\pgfsetstrokeopacity{0.000000}%
\pgfsetdash{}{0pt}%
\pgfpathmoveto{\pgfqpoint{2.346811in}{1.794305in}}%
\pgfpathlineto{\pgfqpoint{2.355748in}{1.794305in}}%
\pgfpathlineto{\pgfqpoint{2.355748in}{1.792451in}}%
\pgfpathlineto{\pgfqpoint{2.346811in}{1.792451in}}%
\pgfpathlineto{\pgfqpoint{2.346811in}{1.794305in}}%
\pgfpathclose%
\pgfusepath{fill}%
\end{pgfscope}%
\begin{pgfscope}%
\pgfpathrectangle{\pgfqpoint{0.697024in}{0.857143in}}{\pgfqpoint{2.627103in}{1.813434in}}%
\pgfusepath{clip}%
\pgfsetbuttcap%
\pgfsetmiterjoin%
\definecolor{currentfill}{rgb}{0.511253,0.510898,0.193296}%
\pgfsetfillcolor{currentfill}%
\pgfsetlinewidth{0.000000pt}%
\definecolor{currentstroke}{rgb}{0.000000,0.000000,0.000000}%
\pgfsetstrokecolor{currentstroke}%
\pgfsetstrokeopacity{0.000000}%
\pgfsetdash{}{0pt}%
\pgfpathmoveto{\pgfqpoint{2.357982in}{1.771542in}}%
\pgfpathlineto{\pgfqpoint{2.366918in}{1.771542in}}%
\pgfpathlineto{\pgfqpoint{2.366918in}{1.770380in}}%
\pgfpathlineto{\pgfqpoint{2.357982in}{1.770380in}}%
\pgfpathlineto{\pgfqpoint{2.357982in}{1.771542in}}%
\pgfpathclose%
\pgfusepath{fill}%
\end{pgfscope}%
\begin{pgfscope}%
\pgfpathrectangle{\pgfqpoint{0.697024in}{0.857143in}}{\pgfqpoint{2.627103in}{1.813434in}}%
\pgfusepath{clip}%
\pgfsetbuttcap%
\pgfsetmiterjoin%
\definecolor{currentfill}{rgb}{0.511253,0.510898,0.193296}%
\pgfsetfillcolor{currentfill}%
\pgfsetlinewidth{0.000000pt}%
\definecolor{currentstroke}{rgb}{0.000000,0.000000,0.000000}%
\pgfsetstrokecolor{currentstroke}%
\pgfsetstrokeopacity{0.000000}%
\pgfsetdash{}{0pt}%
\pgfpathmoveto{\pgfqpoint{2.369152in}{2.399199in}}%
\pgfpathlineto{\pgfqpoint{2.378089in}{2.399199in}}%
\pgfpathlineto{\pgfqpoint{2.378089in}{2.402012in}}%
\pgfpathlineto{\pgfqpoint{2.369152in}{2.402012in}}%
\pgfpathlineto{\pgfqpoint{2.369152in}{2.399199in}}%
\pgfpathclose%
\pgfusepath{fill}%
\end{pgfscope}%
\begin{pgfscope}%
\pgfpathrectangle{\pgfqpoint{0.697024in}{0.857143in}}{\pgfqpoint{2.627103in}{1.813434in}}%
\pgfusepath{clip}%
\pgfsetbuttcap%
\pgfsetmiterjoin%
\definecolor{currentfill}{rgb}{0.511253,0.510898,0.193296}%
\pgfsetfillcolor{currentfill}%
\pgfsetlinewidth{0.000000pt}%
\definecolor{currentstroke}{rgb}{0.000000,0.000000,0.000000}%
\pgfsetstrokecolor{currentstroke}%
\pgfsetstrokeopacity{0.000000}%
\pgfsetdash{}{0pt}%
\pgfpathmoveto{\pgfqpoint{2.380323in}{1.817593in}}%
\pgfpathlineto{\pgfqpoint{2.389259in}{1.817593in}}%
\pgfpathlineto{\pgfqpoint{2.389259in}{1.814951in}}%
\pgfpathlineto{\pgfqpoint{2.380323in}{1.814951in}}%
\pgfpathlineto{\pgfqpoint{2.380323in}{1.817593in}}%
\pgfpathclose%
\pgfusepath{fill}%
\end{pgfscope}%
\begin{pgfscope}%
\pgfpathrectangle{\pgfqpoint{0.697024in}{0.857143in}}{\pgfqpoint{2.627103in}{1.813434in}}%
\pgfusepath{clip}%
\pgfsetbuttcap%
\pgfsetmiterjoin%
\definecolor{currentfill}{rgb}{0.511253,0.510898,0.193296}%
\pgfsetfillcolor{currentfill}%
\pgfsetlinewidth{0.000000pt}%
\definecolor{currentstroke}{rgb}{0.000000,0.000000,0.000000}%
\pgfsetstrokecolor{currentstroke}%
\pgfsetstrokeopacity{0.000000}%
\pgfsetdash{}{0pt}%
\pgfpathmoveto{\pgfqpoint{2.391494in}{1.845214in}}%
\pgfpathlineto{\pgfqpoint{2.400430in}{1.845214in}}%
\pgfpathlineto{\pgfqpoint{2.400430in}{1.841825in}}%
\pgfpathlineto{\pgfqpoint{2.391494in}{1.841825in}}%
\pgfpathlineto{\pgfqpoint{2.391494in}{1.845214in}}%
\pgfpathclose%
\pgfusepath{fill}%
\end{pgfscope}%
\begin{pgfscope}%
\pgfpathrectangle{\pgfqpoint{0.697024in}{0.857143in}}{\pgfqpoint{2.627103in}{1.813434in}}%
\pgfusepath{clip}%
\pgfsetbuttcap%
\pgfsetmiterjoin%
\definecolor{currentfill}{rgb}{0.511253,0.510898,0.193296}%
\pgfsetfillcolor{currentfill}%
\pgfsetlinewidth{0.000000pt}%
\definecolor{currentstroke}{rgb}{0.000000,0.000000,0.000000}%
\pgfsetstrokecolor{currentstroke}%
\pgfsetstrokeopacity{0.000000}%
\pgfsetdash{}{0pt}%
\pgfpathmoveto{\pgfqpoint{2.402664in}{2.224687in}}%
\pgfpathlineto{\pgfqpoint{2.411601in}{2.224687in}}%
\pgfpathlineto{\pgfqpoint{2.411601in}{2.225639in}}%
\pgfpathlineto{\pgfqpoint{2.402664in}{2.225639in}}%
\pgfpathlineto{\pgfqpoint{2.402664in}{2.224687in}}%
\pgfpathclose%
\pgfusepath{fill}%
\end{pgfscope}%
\begin{pgfscope}%
\pgfpathrectangle{\pgfqpoint{0.697024in}{0.857143in}}{\pgfqpoint{2.627103in}{1.813434in}}%
\pgfusepath{clip}%
\pgfsetbuttcap%
\pgfsetmiterjoin%
\definecolor{currentfill}{rgb}{0.511253,0.510898,0.193296}%
\pgfsetfillcolor{currentfill}%
\pgfsetlinewidth{0.000000pt}%
\definecolor{currentstroke}{rgb}{0.000000,0.000000,0.000000}%
\pgfsetstrokecolor{currentstroke}%
\pgfsetstrokeopacity{0.000000}%
\pgfsetdash{}{0pt}%
\pgfpathmoveto{\pgfqpoint{2.413835in}{1.847071in}}%
\pgfpathlineto{\pgfqpoint{2.422771in}{1.847071in}}%
\pgfpathlineto{\pgfqpoint{2.422771in}{1.840384in}}%
\pgfpathlineto{\pgfqpoint{2.413835in}{1.840384in}}%
\pgfpathlineto{\pgfqpoint{2.413835in}{1.847071in}}%
\pgfpathclose%
\pgfusepath{fill}%
\end{pgfscope}%
\begin{pgfscope}%
\pgfpathrectangle{\pgfqpoint{0.697024in}{0.857143in}}{\pgfqpoint{2.627103in}{1.813434in}}%
\pgfusepath{clip}%
\pgfsetbuttcap%
\pgfsetmiterjoin%
\definecolor{currentfill}{rgb}{0.511253,0.510898,0.193296}%
\pgfsetfillcolor{currentfill}%
\pgfsetlinewidth{0.000000pt}%
\definecolor{currentstroke}{rgb}{0.000000,0.000000,0.000000}%
\pgfsetstrokecolor{currentstroke}%
\pgfsetstrokeopacity{0.000000}%
\pgfsetdash{}{0pt}%
\pgfpathmoveto{\pgfqpoint{2.425005in}{1.847071in}}%
\pgfpathlineto{\pgfqpoint{2.433942in}{1.847071in}}%
\pgfpathlineto{\pgfqpoint{2.433942in}{1.838978in}}%
\pgfpathlineto{\pgfqpoint{2.425005in}{1.838978in}}%
\pgfpathlineto{\pgfqpoint{2.425005in}{1.847071in}}%
\pgfpathclose%
\pgfusepath{fill}%
\end{pgfscope}%
\begin{pgfscope}%
\pgfpathrectangle{\pgfqpoint{0.697024in}{0.857143in}}{\pgfqpoint{2.627103in}{1.813434in}}%
\pgfusepath{clip}%
\pgfsetbuttcap%
\pgfsetmiterjoin%
\definecolor{currentfill}{rgb}{0.511253,0.510898,0.193296}%
\pgfsetfillcolor{currentfill}%
\pgfsetlinewidth{0.000000pt}%
\definecolor{currentstroke}{rgb}{0.000000,0.000000,0.000000}%
\pgfsetstrokecolor{currentstroke}%
\pgfsetstrokeopacity{0.000000}%
\pgfsetdash{}{0pt}%
\pgfpathmoveto{\pgfqpoint{2.436176in}{1.847071in}}%
\pgfpathlineto{\pgfqpoint{2.445112in}{1.847071in}}%
\pgfpathlineto{\pgfqpoint{2.445112in}{1.837258in}}%
\pgfpathlineto{\pgfqpoint{2.436176in}{1.837258in}}%
\pgfpathlineto{\pgfqpoint{2.436176in}{1.847071in}}%
\pgfpathclose%
\pgfusepath{fill}%
\end{pgfscope}%
\begin{pgfscope}%
\pgfpathrectangle{\pgfqpoint{0.697024in}{0.857143in}}{\pgfqpoint{2.627103in}{1.813434in}}%
\pgfusepath{clip}%
\pgfsetbuttcap%
\pgfsetmiterjoin%
\definecolor{currentfill}{rgb}{0.511253,0.510898,0.193296}%
\pgfsetfillcolor{currentfill}%
\pgfsetlinewidth{0.000000pt}%
\definecolor{currentstroke}{rgb}{0.000000,0.000000,0.000000}%
\pgfsetstrokecolor{currentstroke}%
\pgfsetstrokeopacity{0.000000}%
\pgfsetdash{}{0pt}%
\pgfpathmoveto{\pgfqpoint{2.447347in}{1.847071in}}%
\pgfpathlineto{\pgfqpoint{2.456283in}{1.847071in}}%
\pgfpathlineto{\pgfqpoint{2.456283in}{1.834166in}}%
\pgfpathlineto{\pgfqpoint{2.447347in}{1.834166in}}%
\pgfpathlineto{\pgfqpoint{2.447347in}{1.847071in}}%
\pgfpathclose%
\pgfusepath{fill}%
\end{pgfscope}%
\begin{pgfscope}%
\pgfpathrectangle{\pgfqpoint{0.697024in}{0.857143in}}{\pgfqpoint{2.627103in}{1.813434in}}%
\pgfusepath{clip}%
\pgfsetbuttcap%
\pgfsetmiterjoin%
\definecolor{currentfill}{rgb}{0.511253,0.510898,0.193296}%
\pgfsetfillcolor{currentfill}%
\pgfsetlinewidth{0.000000pt}%
\definecolor{currentstroke}{rgb}{0.000000,0.000000,0.000000}%
\pgfsetstrokecolor{currentstroke}%
\pgfsetstrokeopacity{0.000000}%
\pgfsetdash{}{0pt}%
\pgfpathmoveto{\pgfqpoint{2.458517in}{1.847071in}}%
\pgfpathlineto{\pgfqpoint{2.467454in}{1.847071in}}%
\pgfpathlineto{\pgfqpoint{2.467454in}{1.832937in}}%
\pgfpathlineto{\pgfqpoint{2.458517in}{1.832937in}}%
\pgfpathlineto{\pgfqpoint{2.458517in}{1.847071in}}%
\pgfpathclose%
\pgfusepath{fill}%
\end{pgfscope}%
\begin{pgfscope}%
\pgfpathrectangle{\pgfqpoint{0.697024in}{0.857143in}}{\pgfqpoint{2.627103in}{1.813434in}}%
\pgfusepath{clip}%
\pgfsetbuttcap%
\pgfsetmiterjoin%
\definecolor{currentfill}{rgb}{0.511253,0.510898,0.193296}%
\pgfsetfillcolor{currentfill}%
\pgfsetlinewidth{0.000000pt}%
\definecolor{currentstroke}{rgb}{0.000000,0.000000,0.000000}%
\pgfsetstrokecolor{currentstroke}%
\pgfsetstrokeopacity{0.000000}%
\pgfsetdash{}{0pt}%
\pgfpathmoveto{\pgfqpoint{2.469688in}{1.847071in}}%
\pgfpathlineto{\pgfqpoint{2.478624in}{1.847071in}}%
\pgfpathlineto{\pgfqpoint{2.478624in}{1.831610in}}%
\pgfpathlineto{\pgfqpoint{2.469688in}{1.831610in}}%
\pgfpathlineto{\pgfqpoint{2.469688in}{1.847071in}}%
\pgfpathclose%
\pgfusepath{fill}%
\end{pgfscope}%
\begin{pgfscope}%
\pgfpathrectangle{\pgfqpoint{0.697024in}{0.857143in}}{\pgfqpoint{2.627103in}{1.813434in}}%
\pgfusepath{clip}%
\pgfsetbuttcap%
\pgfsetmiterjoin%
\definecolor{currentfill}{rgb}{0.511253,0.510898,0.193296}%
\pgfsetfillcolor{currentfill}%
\pgfsetlinewidth{0.000000pt}%
\definecolor{currentstroke}{rgb}{0.000000,0.000000,0.000000}%
\pgfsetstrokecolor{currentstroke}%
\pgfsetstrokeopacity{0.000000}%
\pgfsetdash{}{0pt}%
\pgfpathmoveto{\pgfqpoint{2.480858in}{1.800812in}}%
\pgfpathlineto{\pgfqpoint{2.489795in}{1.800812in}}%
\pgfpathlineto{\pgfqpoint{2.489795in}{1.781588in}}%
\pgfpathlineto{\pgfqpoint{2.480858in}{1.781588in}}%
\pgfpathlineto{\pgfqpoint{2.480858in}{1.800812in}}%
\pgfpathclose%
\pgfusepath{fill}%
\end{pgfscope}%
\begin{pgfscope}%
\pgfpathrectangle{\pgfqpoint{0.697024in}{0.857143in}}{\pgfqpoint{2.627103in}{1.813434in}}%
\pgfusepath{clip}%
\pgfsetbuttcap%
\pgfsetmiterjoin%
\definecolor{currentfill}{rgb}{0.511253,0.510898,0.193296}%
\pgfsetfillcolor{currentfill}%
\pgfsetlinewidth{0.000000pt}%
\definecolor{currentstroke}{rgb}{0.000000,0.000000,0.000000}%
\pgfsetstrokecolor{currentstroke}%
\pgfsetstrokeopacity{0.000000}%
\pgfsetdash{}{0pt}%
\pgfpathmoveto{\pgfqpoint{2.492029in}{1.807603in}}%
\pgfpathlineto{\pgfqpoint{2.500965in}{1.807603in}}%
\pgfpathlineto{\pgfqpoint{2.500965in}{1.784277in}}%
\pgfpathlineto{\pgfqpoint{2.492029in}{1.784277in}}%
\pgfpathlineto{\pgfqpoint{2.492029in}{1.807603in}}%
\pgfpathclose%
\pgfusepath{fill}%
\end{pgfscope}%
\begin{pgfscope}%
\pgfpathrectangle{\pgfqpoint{0.697024in}{0.857143in}}{\pgfqpoint{2.627103in}{1.813434in}}%
\pgfusepath{clip}%
\pgfsetbuttcap%
\pgfsetmiterjoin%
\definecolor{currentfill}{rgb}{0.511253,0.510898,0.193296}%
\pgfsetfillcolor{currentfill}%
\pgfsetlinewidth{0.000000pt}%
\definecolor{currentstroke}{rgb}{0.000000,0.000000,0.000000}%
\pgfsetstrokecolor{currentstroke}%
\pgfsetstrokeopacity{0.000000}%
\pgfsetdash{}{0pt}%
\pgfpathmoveto{\pgfqpoint{2.503200in}{1.748004in}}%
\pgfpathlineto{\pgfqpoint{2.512136in}{1.748004in}}%
\pgfpathlineto{\pgfqpoint{2.512136in}{1.724916in}}%
\pgfpathlineto{\pgfqpoint{2.503200in}{1.724916in}}%
\pgfpathlineto{\pgfqpoint{2.503200in}{1.748004in}}%
\pgfpathclose%
\pgfusepath{fill}%
\end{pgfscope}%
\begin{pgfscope}%
\pgfpathrectangle{\pgfqpoint{0.697024in}{0.857143in}}{\pgfqpoint{2.627103in}{1.813434in}}%
\pgfusepath{clip}%
\pgfsetbuttcap%
\pgfsetmiterjoin%
\definecolor{currentfill}{rgb}{0.511253,0.510898,0.193296}%
\pgfsetfillcolor{currentfill}%
\pgfsetlinewidth{0.000000pt}%
\definecolor{currentstroke}{rgb}{0.000000,0.000000,0.000000}%
\pgfsetstrokecolor{currentstroke}%
\pgfsetstrokeopacity{0.000000}%
\pgfsetdash{}{0pt}%
\pgfpathmoveto{\pgfqpoint{2.514370in}{1.798627in}}%
\pgfpathlineto{\pgfqpoint{2.523307in}{1.798627in}}%
\pgfpathlineto{\pgfqpoint{2.523307in}{1.773730in}}%
\pgfpathlineto{\pgfqpoint{2.514370in}{1.773730in}}%
\pgfpathlineto{\pgfqpoint{2.514370in}{1.798627in}}%
\pgfpathclose%
\pgfusepath{fill}%
\end{pgfscope}%
\begin{pgfscope}%
\pgfpathrectangle{\pgfqpoint{0.697024in}{0.857143in}}{\pgfqpoint{2.627103in}{1.813434in}}%
\pgfusepath{clip}%
\pgfsetbuttcap%
\pgfsetmiterjoin%
\definecolor{currentfill}{rgb}{0.511253,0.510898,0.193296}%
\pgfsetfillcolor{currentfill}%
\pgfsetlinewidth{0.000000pt}%
\definecolor{currentstroke}{rgb}{0.000000,0.000000,0.000000}%
\pgfsetstrokecolor{currentstroke}%
\pgfsetstrokeopacity{0.000000}%
\pgfsetdash{}{0pt}%
\pgfpathmoveto{\pgfqpoint{2.525541in}{1.758929in}}%
\pgfpathlineto{\pgfqpoint{2.534477in}{1.758929in}}%
\pgfpathlineto{\pgfqpoint{2.534477in}{1.732887in}}%
\pgfpathlineto{\pgfqpoint{2.525541in}{1.732887in}}%
\pgfpathlineto{\pgfqpoint{2.525541in}{1.758929in}}%
\pgfpathclose%
\pgfusepath{fill}%
\end{pgfscope}%
\begin{pgfscope}%
\pgfpathrectangle{\pgfqpoint{0.697024in}{0.857143in}}{\pgfqpoint{2.627103in}{1.813434in}}%
\pgfusepath{clip}%
\pgfsetbuttcap%
\pgfsetmiterjoin%
\definecolor{currentfill}{rgb}{0.511253,0.510898,0.193296}%
\pgfsetfillcolor{currentfill}%
\pgfsetlinewidth{0.000000pt}%
\definecolor{currentstroke}{rgb}{0.000000,0.000000,0.000000}%
\pgfsetstrokecolor{currentstroke}%
\pgfsetstrokeopacity{0.000000}%
\pgfsetdash{}{0pt}%
\pgfpathmoveto{\pgfqpoint{2.536711in}{1.720409in}}%
\pgfpathlineto{\pgfqpoint{2.545648in}{1.720409in}}%
\pgfpathlineto{\pgfqpoint{2.545648in}{1.695142in}}%
\pgfpathlineto{\pgfqpoint{2.536711in}{1.695142in}}%
\pgfpathlineto{\pgfqpoint{2.536711in}{1.720409in}}%
\pgfpathclose%
\pgfusepath{fill}%
\end{pgfscope}%
\begin{pgfscope}%
\pgfpathrectangle{\pgfqpoint{0.697024in}{0.857143in}}{\pgfqpoint{2.627103in}{1.813434in}}%
\pgfusepath{clip}%
\pgfsetbuttcap%
\pgfsetmiterjoin%
\definecolor{currentfill}{rgb}{0.511253,0.510898,0.193296}%
\pgfsetfillcolor{currentfill}%
\pgfsetlinewidth{0.000000pt}%
\definecolor{currentstroke}{rgb}{0.000000,0.000000,0.000000}%
\pgfsetstrokecolor{currentstroke}%
\pgfsetstrokeopacity{0.000000}%
\pgfsetdash{}{0pt}%
\pgfpathmoveto{\pgfqpoint{2.547882in}{1.766469in}}%
\pgfpathlineto{\pgfqpoint{2.556818in}{1.766469in}}%
\pgfpathlineto{\pgfqpoint{2.556818in}{1.737511in}}%
\pgfpathlineto{\pgfqpoint{2.547882in}{1.737511in}}%
\pgfpathlineto{\pgfqpoint{2.547882in}{1.766469in}}%
\pgfpathclose%
\pgfusepath{fill}%
\end{pgfscope}%
\begin{pgfscope}%
\pgfpathrectangle{\pgfqpoint{0.697024in}{0.857143in}}{\pgfqpoint{2.627103in}{1.813434in}}%
\pgfusepath{clip}%
\pgfsetbuttcap%
\pgfsetmiterjoin%
\definecolor{currentfill}{rgb}{0.511253,0.510898,0.193296}%
\pgfsetfillcolor{currentfill}%
\pgfsetlinewidth{0.000000pt}%
\definecolor{currentstroke}{rgb}{0.000000,0.000000,0.000000}%
\pgfsetstrokecolor{currentstroke}%
\pgfsetstrokeopacity{0.000000}%
\pgfsetdash{}{0pt}%
\pgfpathmoveto{\pgfqpoint{2.559053in}{1.721051in}}%
\pgfpathlineto{\pgfqpoint{2.567989in}{1.721051in}}%
\pgfpathlineto{\pgfqpoint{2.567989in}{1.693844in}}%
\pgfpathlineto{\pgfqpoint{2.559053in}{1.693844in}}%
\pgfpathlineto{\pgfqpoint{2.559053in}{1.721051in}}%
\pgfpathclose%
\pgfusepath{fill}%
\end{pgfscope}%
\begin{pgfscope}%
\pgfpathrectangle{\pgfqpoint{0.697024in}{0.857143in}}{\pgfqpoint{2.627103in}{1.813434in}}%
\pgfusepath{clip}%
\pgfsetbuttcap%
\pgfsetmiterjoin%
\definecolor{currentfill}{rgb}{0.511253,0.510898,0.193296}%
\pgfsetfillcolor{currentfill}%
\pgfsetlinewidth{0.000000pt}%
\definecolor{currentstroke}{rgb}{0.000000,0.000000,0.000000}%
\pgfsetstrokecolor{currentstroke}%
\pgfsetstrokeopacity{0.000000}%
\pgfsetdash{}{0pt}%
\pgfpathmoveto{\pgfqpoint{2.570223in}{1.734664in}}%
\pgfpathlineto{\pgfqpoint{2.579160in}{1.734664in}}%
\pgfpathlineto{\pgfqpoint{2.579160in}{1.709446in}}%
\pgfpathlineto{\pgfqpoint{2.570223in}{1.709446in}}%
\pgfpathlineto{\pgfqpoint{2.570223in}{1.734664in}}%
\pgfpathclose%
\pgfusepath{fill}%
\end{pgfscope}%
\begin{pgfscope}%
\pgfpathrectangle{\pgfqpoint{0.697024in}{0.857143in}}{\pgfqpoint{2.627103in}{1.813434in}}%
\pgfusepath{clip}%
\pgfsetbuttcap%
\pgfsetmiterjoin%
\definecolor{currentfill}{rgb}{0.511253,0.510898,0.193296}%
\pgfsetfillcolor{currentfill}%
\pgfsetlinewidth{0.000000pt}%
\definecolor{currentstroke}{rgb}{0.000000,0.000000,0.000000}%
\pgfsetstrokecolor{currentstroke}%
\pgfsetstrokeopacity{0.000000}%
\pgfsetdash{}{0pt}%
\pgfpathmoveto{\pgfqpoint{2.581394in}{1.771486in}}%
\pgfpathlineto{\pgfqpoint{2.590330in}{1.771486in}}%
\pgfpathlineto{\pgfqpoint{2.590330in}{1.745753in}}%
\pgfpathlineto{\pgfqpoint{2.581394in}{1.745753in}}%
\pgfpathlineto{\pgfqpoint{2.581394in}{1.771486in}}%
\pgfpathclose%
\pgfusepath{fill}%
\end{pgfscope}%
\begin{pgfscope}%
\pgfpathrectangle{\pgfqpoint{0.697024in}{0.857143in}}{\pgfqpoint{2.627103in}{1.813434in}}%
\pgfusepath{clip}%
\pgfsetbuttcap%
\pgfsetmiterjoin%
\definecolor{currentfill}{rgb}{0.511253,0.510898,0.193296}%
\pgfsetfillcolor{currentfill}%
\pgfsetlinewidth{0.000000pt}%
\definecolor{currentstroke}{rgb}{0.000000,0.000000,0.000000}%
\pgfsetstrokecolor{currentstroke}%
\pgfsetstrokeopacity{0.000000}%
\pgfsetdash{}{0pt}%
\pgfpathmoveto{\pgfqpoint{2.592564in}{1.814524in}}%
\pgfpathlineto{\pgfqpoint{2.601501in}{1.814524in}}%
\pgfpathlineto{\pgfqpoint{2.601501in}{1.786277in}}%
\pgfpathlineto{\pgfqpoint{2.592564in}{1.786277in}}%
\pgfpathlineto{\pgfqpoint{2.592564in}{1.814524in}}%
\pgfpathclose%
\pgfusepath{fill}%
\end{pgfscope}%
\begin{pgfscope}%
\pgfpathrectangle{\pgfqpoint{0.697024in}{0.857143in}}{\pgfqpoint{2.627103in}{1.813434in}}%
\pgfusepath{clip}%
\pgfsetbuttcap%
\pgfsetmiterjoin%
\definecolor{currentfill}{rgb}{0.511253,0.510898,0.193296}%
\pgfsetfillcolor{currentfill}%
\pgfsetlinewidth{0.000000pt}%
\definecolor{currentstroke}{rgb}{0.000000,0.000000,0.000000}%
\pgfsetstrokecolor{currentstroke}%
\pgfsetstrokeopacity{0.000000}%
\pgfsetdash{}{0pt}%
\pgfpathmoveto{\pgfqpoint{2.603735in}{1.838805in}}%
\pgfpathlineto{\pgfqpoint{2.612672in}{1.838805in}}%
\pgfpathlineto{\pgfqpoint{2.612672in}{1.809682in}}%
\pgfpathlineto{\pgfqpoint{2.603735in}{1.809682in}}%
\pgfpathlineto{\pgfqpoint{2.603735in}{1.838805in}}%
\pgfpathclose%
\pgfusepath{fill}%
\end{pgfscope}%
\begin{pgfscope}%
\pgfpathrectangle{\pgfqpoint{0.697024in}{0.857143in}}{\pgfqpoint{2.627103in}{1.813434in}}%
\pgfusepath{clip}%
\pgfsetbuttcap%
\pgfsetmiterjoin%
\definecolor{currentfill}{rgb}{0.511253,0.510898,0.193296}%
\pgfsetfillcolor{currentfill}%
\pgfsetlinewidth{0.000000pt}%
\definecolor{currentstroke}{rgb}{0.000000,0.000000,0.000000}%
\pgfsetstrokecolor{currentstroke}%
\pgfsetstrokeopacity{0.000000}%
\pgfsetdash{}{0pt}%
\pgfpathmoveto{\pgfqpoint{2.614906in}{1.847071in}}%
\pgfpathlineto{\pgfqpoint{2.623842in}{1.847071in}}%
\pgfpathlineto{\pgfqpoint{2.623842in}{1.821708in}}%
\pgfpathlineto{\pgfqpoint{2.614906in}{1.821708in}}%
\pgfpathlineto{\pgfqpoint{2.614906in}{1.847071in}}%
\pgfpathclose%
\pgfusepath{fill}%
\end{pgfscope}%
\begin{pgfscope}%
\pgfpathrectangle{\pgfqpoint{0.697024in}{0.857143in}}{\pgfqpoint{2.627103in}{1.813434in}}%
\pgfusepath{clip}%
\pgfsetbuttcap%
\pgfsetmiterjoin%
\definecolor{currentfill}{rgb}{0.511253,0.510898,0.193296}%
\pgfsetfillcolor{currentfill}%
\pgfsetlinewidth{0.000000pt}%
\definecolor{currentstroke}{rgb}{0.000000,0.000000,0.000000}%
\pgfsetstrokecolor{currentstroke}%
\pgfsetstrokeopacity{0.000000}%
\pgfsetdash{}{0pt}%
\pgfpathmoveto{\pgfqpoint{2.626076in}{1.788001in}}%
\pgfpathlineto{\pgfqpoint{2.635013in}{1.788001in}}%
\pgfpathlineto{\pgfqpoint{2.635013in}{1.760957in}}%
\pgfpathlineto{\pgfqpoint{2.626076in}{1.760957in}}%
\pgfpathlineto{\pgfqpoint{2.626076in}{1.788001in}}%
\pgfpathclose%
\pgfusepath{fill}%
\end{pgfscope}%
\begin{pgfscope}%
\pgfpathrectangle{\pgfqpoint{0.697024in}{0.857143in}}{\pgfqpoint{2.627103in}{1.813434in}}%
\pgfusepath{clip}%
\pgfsetbuttcap%
\pgfsetmiterjoin%
\definecolor{currentfill}{rgb}{0.511253,0.510898,0.193296}%
\pgfsetfillcolor{currentfill}%
\pgfsetlinewidth{0.000000pt}%
\definecolor{currentstroke}{rgb}{0.000000,0.000000,0.000000}%
\pgfsetstrokecolor{currentstroke}%
\pgfsetstrokeopacity{0.000000}%
\pgfsetdash{}{0pt}%
\pgfpathmoveto{\pgfqpoint{2.637247in}{1.847071in}}%
\pgfpathlineto{\pgfqpoint{2.646183in}{1.847071in}}%
\pgfpathlineto{\pgfqpoint{2.646183in}{1.821048in}}%
\pgfpathlineto{\pgfqpoint{2.637247in}{1.821048in}}%
\pgfpathlineto{\pgfqpoint{2.637247in}{1.847071in}}%
\pgfpathclose%
\pgfusepath{fill}%
\end{pgfscope}%
\begin{pgfscope}%
\pgfpathrectangle{\pgfqpoint{0.697024in}{0.857143in}}{\pgfqpoint{2.627103in}{1.813434in}}%
\pgfusepath{clip}%
\pgfsetbuttcap%
\pgfsetmiterjoin%
\definecolor{currentfill}{rgb}{0.511253,0.510898,0.193296}%
\pgfsetfillcolor{currentfill}%
\pgfsetlinewidth{0.000000pt}%
\definecolor{currentstroke}{rgb}{0.000000,0.000000,0.000000}%
\pgfsetstrokecolor{currentstroke}%
\pgfsetstrokeopacity{0.000000}%
\pgfsetdash{}{0pt}%
\pgfpathmoveto{\pgfqpoint{2.648417in}{1.847071in}}%
\pgfpathlineto{\pgfqpoint{2.657354in}{1.847071in}}%
\pgfpathlineto{\pgfqpoint{2.657354in}{1.820868in}}%
\pgfpathlineto{\pgfqpoint{2.648417in}{1.820868in}}%
\pgfpathlineto{\pgfqpoint{2.648417in}{1.847071in}}%
\pgfpathclose%
\pgfusepath{fill}%
\end{pgfscope}%
\begin{pgfscope}%
\pgfpathrectangle{\pgfqpoint{0.697024in}{0.857143in}}{\pgfqpoint{2.627103in}{1.813434in}}%
\pgfusepath{clip}%
\pgfsetbuttcap%
\pgfsetmiterjoin%
\definecolor{currentfill}{rgb}{0.511253,0.510898,0.193296}%
\pgfsetfillcolor{currentfill}%
\pgfsetlinewidth{0.000000pt}%
\definecolor{currentstroke}{rgb}{0.000000,0.000000,0.000000}%
\pgfsetstrokecolor{currentstroke}%
\pgfsetstrokeopacity{0.000000}%
\pgfsetdash{}{0pt}%
\pgfpathmoveto{\pgfqpoint{2.659588in}{1.847071in}}%
\pgfpathlineto{\pgfqpoint{2.668525in}{1.847071in}}%
\pgfpathlineto{\pgfqpoint{2.668525in}{1.820736in}}%
\pgfpathlineto{\pgfqpoint{2.659588in}{1.820736in}}%
\pgfpathlineto{\pgfqpoint{2.659588in}{1.847071in}}%
\pgfpathclose%
\pgfusepath{fill}%
\end{pgfscope}%
\begin{pgfscope}%
\pgfpathrectangle{\pgfqpoint{0.697024in}{0.857143in}}{\pgfqpoint{2.627103in}{1.813434in}}%
\pgfusepath{clip}%
\pgfsetbuttcap%
\pgfsetmiterjoin%
\definecolor{currentfill}{rgb}{0.511253,0.510898,0.193296}%
\pgfsetfillcolor{currentfill}%
\pgfsetlinewidth{0.000000pt}%
\definecolor{currentstroke}{rgb}{0.000000,0.000000,0.000000}%
\pgfsetstrokecolor{currentstroke}%
\pgfsetstrokeopacity{0.000000}%
\pgfsetdash{}{0pt}%
\pgfpathmoveto{\pgfqpoint{2.670759in}{1.847071in}}%
\pgfpathlineto{\pgfqpoint{2.679695in}{1.847071in}}%
\pgfpathlineto{\pgfqpoint{2.679695in}{1.820584in}}%
\pgfpathlineto{\pgfqpoint{2.670759in}{1.820584in}}%
\pgfpathlineto{\pgfqpoint{2.670759in}{1.847071in}}%
\pgfpathclose%
\pgfusepath{fill}%
\end{pgfscope}%
\begin{pgfscope}%
\pgfpathrectangle{\pgfqpoint{0.697024in}{0.857143in}}{\pgfqpoint{2.627103in}{1.813434in}}%
\pgfusepath{clip}%
\pgfsetbuttcap%
\pgfsetmiterjoin%
\definecolor{currentfill}{rgb}{0.511253,0.510898,0.193296}%
\pgfsetfillcolor{currentfill}%
\pgfsetlinewidth{0.000000pt}%
\definecolor{currentstroke}{rgb}{0.000000,0.000000,0.000000}%
\pgfsetstrokecolor{currentstroke}%
\pgfsetstrokeopacity{0.000000}%
\pgfsetdash{}{0pt}%
\pgfpathmoveto{\pgfqpoint{2.681929in}{1.847071in}}%
\pgfpathlineto{\pgfqpoint{2.690866in}{1.847071in}}%
\pgfpathlineto{\pgfqpoint{2.690866in}{1.825396in}}%
\pgfpathlineto{\pgfqpoint{2.681929in}{1.825396in}}%
\pgfpathlineto{\pgfqpoint{2.681929in}{1.847071in}}%
\pgfpathclose%
\pgfusepath{fill}%
\end{pgfscope}%
\begin{pgfscope}%
\pgfpathrectangle{\pgfqpoint{0.697024in}{0.857143in}}{\pgfqpoint{2.627103in}{1.813434in}}%
\pgfusepath{clip}%
\pgfsetbuttcap%
\pgfsetmiterjoin%
\definecolor{currentfill}{rgb}{0.511253,0.510898,0.193296}%
\pgfsetfillcolor{currentfill}%
\pgfsetlinewidth{0.000000pt}%
\definecolor{currentstroke}{rgb}{0.000000,0.000000,0.000000}%
\pgfsetstrokecolor{currentstroke}%
\pgfsetstrokeopacity{0.000000}%
\pgfsetdash{}{0pt}%
\pgfpathmoveto{\pgfqpoint{2.693100in}{1.840324in}}%
\pgfpathlineto{\pgfqpoint{2.702036in}{1.840324in}}%
\pgfpathlineto{\pgfqpoint{2.702036in}{1.817181in}}%
\pgfpathlineto{\pgfqpoint{2.693100in}{1.817181in}}%
\pgfpathlineto{\pgfqpoint{2.693100in}{1.840324in}}%
\pgfpathclose%
\pgfusepath{fill}%
\end{pgfscope}%
\begin{pgfscope}%
\pgfpathrectangle{\pgfqpoint{0.697024in}{0.857143in}}{\pgfqpoint{2.627103in}{1.813434in}}%
\pgfusepath{clip}%
\pgfsetbuttcap%
\pgfsetmiterjoin%
\definecolor{currentfill}{rgb}{0.511253,0.510898,0.193296}%
\pgfsetfillcolor{currentfill}%
\pgfsetlinewidth{0.000000pt}%
\definecolor{currentstroke}{rgb}{0.000000,0.000000,0.000000}%
\pgfsetstrokecolor{currentstroke}%
\pgfsetstrokeopacity{0.000000}%
\pgfsetdash{}{0pt}%
\pgfpathmoveto{\pgfqpoint{2.704270in}{1.723040in}}%
\pgfpathlineto{\pgfqpoint{2.713207in}{1.723040in}}%
\pgfpathlineto{\pgfqpoint{2.713207in}{1.701429in}}%
\pgfpathlineto{\pgfqpoint{2.704270in}{1.701429in}}%
\pgfpathlineto{\pgfqpoint{2.704270in}{1.723040in}}%
\pgfpathclose%
\pgfusepath{fill}%
\end{pgfscope}%
\begin{pgfscope}%
\pgfpathrectangle{\pgfqpoint{0.697024in}{0.857143in}}{\pgfqpoint{2.627103in}{1.813434in}}%
\pgfusepath{clip}%
\pgfsetbuttcap%
\pgfsetmiterjoin%
\definecolor{currentfill}{rgb}{0.511253,0.510898,0.193296}%
\pgfsetfillcolor{currentfill}%
\pgfsetlinewidth{0.000000pt}%
\definecolor{currentstroke}{rgb}{0.000000,0.000000,0.000000}%
\pgfsetstrokecolor{currentstroke}%
\pgfsetstrokeopacity{0.000000}%
\pgfsetdash{}{0pt}%
\pgfpathmoveto{\pgfqpoint{2.715441in}{1.524854in}}%
\pgfpathlineto{\pgfqpoint{2.724378in}{1.524854in}}%
\pgfpathlineto{\pgfqpoint{2.724378in}{1.505197in}}%
\pgfpathlineto{\pgfqpoint{2.715441in}{1.505197in}}%
\pgfpathlineto{\pgfqpoint{2.715441in}{1.524854in}}%
\pgfpathclose%
\pgfusepath{fill}%
\end{pgfscope}%
\begin{pgfscope}%
\pgfpathrectangle{\pgfqpoint{0.697024in}{0.857143in}}{\pgfqpoint{2.627103in}{1.813434in}}%
\pgfusepath{clip}%
\pgfsetbuttcap%
\pgfsetmiterjoin%
\definecolor{currentfill}{rgb}{0.511253,0.510898,0.193296}%
\pgfsetfillcolor{currentfill}%
\pgfsetlinewidth{0.000000pt}%
\definecolor{currentstroke}{rgb}{0.000000,0.000000,0.000000}%
\pgfsetstrokecolor{currentstroke}%
\pgfsetstrokeopacity{0.000000}%
\pgfsetdash{}{0pt}%
\pgfpathmoveto{\pgfqpoint{2.726612in}{1.394884in}}%
\pgfpathlineto{\pgfqpoint{2.735548in}{1.394884in}}%
\pgfpathlineto{\pgfqpoint{2.735548in}{1.372710in}}%
\pgfpathlineto{\pgfqpoint{2.726612in}{1.372710in}}%
\pgfpathlineto{\pgfqpoint{2.726612in}{1.394884in}}%
\pgfpathclose%
\pgfusepath{fill}%
\end{pgfscope}%
\begin{pgfscope}%
\pgfpathrectangle{\pgfqpoint{0.697024in}{0.857143in}}{\pgfqpoint{2.627103in}{1.813434in}}%
\pgfusepath{clip}%
\pgfsetbuttcap%
\pgfsetmiterjoin%
\definecolor{currentfill}{rgb}{0.511253,0.510898,0.193296}%
\pgfsetfillcolor{currentfill}%
\pgfsetlinewidth{0.000000pt}%
\definecolor{currentstroke}{rgb}{0.000000,0.000000,0.000000}%
\pgfsetstrokecolor{currentstroke}%
\pgfsetstrokeopacity{0.000000}%
\pgfsetdash{}{0pt}%
\pgfpathmoveto{\pgfqpoint{2.737782in}{1.425271in}}%
\pgfpathlineto{\pgfqpoint{2.746719in}{1.425271in}}%
\pgfpathlineto{\pgfqpoint{2.746719in}{1.398994in}}%
\pgfpathlineto{\pgfqpoint{2.737782in}{1.398994in}}%
\pgfpathlineto{\pgfqpoint{2.737782in}{1.425271in}}%
\pgfpathclose%
\pgfusepath{fill}%
\end{pgfscope}%
\begin{pgfscope}%
\pgfpathrectangle{\pgfqpoint{0.697024in}{0.857143in}}{\pgfqpoint{2.627103in}{1.813434in}}%
\pgfusepath{clip}%
\pgfsetbuttcap%
\pgfsetmiterjoin%
\definecolor{currentfill}{rgb}{0.511253,0.510898,0.193296}%
\pgfsetfillcolor{currentfill}%
\pgfsetlinewidth{0.000000pt}%
\definecolor{currentstroke}{rgb}{0.000000,0.000000,0.000000}%
\pgfsetstrokecolor{currentstroke}%
\pgfsetstrokeopacity{0.000000}%
\pgfsetdash{}{0pt}%
\pgfpathmoveto{\pgfqpoint{2.748953in}{1.375225in}}%
\pgfpathlineto{\pgfqpoint{2.757889in}{1.375225in}}%
\pgfpathlineto{\pgfqpoint{2.757889in}{1.350207in}}%
\pgfpathlineto{\pgfqpoint{2.748953in}{1.350207in}}%
\pgfpathlineto{\pgfqpoint{2.748953in}{1.375225in}}%
\pgfpathclose%
\pgfusepath{fill}%
\end{pgfscope}%
\begin{pgfscope}%
\pgfpathrectangle{\pgfqpoint{0.697024in}{0.857143in}}{\pgfqpoint{2.627103in}{1.813434in}}%
\pgfusepath{clip}%
\pgfsetbuttcap%
\pgfsetmiterjoin%
\definecolor{currentfill}{rgb}{0.511253,0.510898,0.193296}%
\pgfsetfillcolor{currentfill}%
\pgfsetlinewidth{0.000000pt}%
\definecolor{currentstroke}{rgb}{0.000000,0.000000,0.000000}%
\pgfsetstrokecolor{currentstroke}%
\pgfsetstrokeopacity{0.000000}%
\pgfsetdash{}{0pt}%
\pgfpathmoveto{\pgfqpoint{2.760124in}{1.343198in}}%
\pgfpathlineto{\pgfqpoint{2.769060in}{1.343198in}}%
\pgfpathlineto{\pgfqpoint{2.769060in}{1.316142in}}%
\pgfpathlineto{\pgfqpoint{2.760124in}{1.316142in}}%
\pgfpathlineto{\pgfqpoint{2.760124in}{1.343198in}}%
\pgfpathclose%
\pgfusepath{fill}%
\end{pgfscope}%
\begin{pgfscope}%
\pgfpathrectangle{\pgfqpoint{0.697024in}{0.857143in}}{\pgfqpoint{2.627103in}{1.813434in}}%
\pgfusepath{clip}%
\pgfsetbuttcap%
\pgfsetmiterjoin%
\definecolor{currentfill}{rgb}{0.511253,0.510898,0.193296}%
\pgfsetfillcolor{currentfill}%
\pgfsetlinewidth{0.000000pt}%
\definecolor{currentstroke}{rgb}{0.000000,0.000000,0.000000}%
\pgfsetstrokecolor{currentstroke}%
\pgfsetstrokeopacity{0.000000}%
\pgfsetdash{}{0pt}%
\pgfpathmoveto{\pgfqpoint{2.771294in}{1.392840in}}%
\pgfpathlineto{\pgfqpoint{2.780231in}{1.392840in}}%
\pgfpathlineto{\pgfqpoint{2.780231in}{1.365107in}}%
\pgfpathlineto{\pgfqpoint{2.771294in}{1.365107in}}%
\pgfpathlineto{\pgfqpoint{2.771294in}{1.392840in}}%
\pgfpathclose%
\pgfusepath{fill}%
\end{pgfscope}%
\begin{pgfscope}%
\pgfpathrectangle{\pgfqpoint{0.697024in}{0.857143in}}{\pgfqpoint{2.627103in}{1.813434in}}%
\pgfusepath{clip}%
\pgfsetbuttcap%
\pgfsetmiterjoin%
\definecolor{currentfill}{rgb}{0.511253,0.510898,0.193296}%
\pgfsetfillcolor{currentfill}%
\pgfsetlinewidth{0.000000pt}%
\definecolor{currentstroke}{rgb}{0.000000,0.000000,0.000000}%
\pgfsetstrokecolor{currentstroke}%
\pgfsetstrokeopacity{0.000000}%
\pgfsetdash{}{0pt}%
\pgfpathmoveto{\pgfqpoint{2.782465in}{1.373620in}}%
\pgfpathlineto{\pgfqpoint{2.791401in}{1.373620in}}%
\pgfpathlineto{\pgfqpoint{2.791401in}{1.345043in}}%
\pgfpathlineto{\pgfqpoint{2.782465in}{1.345043in}}%
\pgfpathlineto{\pgfqpoint{2.782465in}{1.373620in}}%
\pgfpathclose%
\pgfusepath{fill}%
\end{pgfscope}%
\begin{pgfscope}%
\pgfpathrectangle{\pgfqpoint{0.697024in}{0.857143in}}{\pgfqpoint{2.627103in}{1.813434in}}%
\pgfusepath{clip}%
\pgfsetbuttcap%
\pgfsetmiterjoin%
\definecolor{currentfill}{rgb}{0.511253,0.510898,0.193296}%
\pgfsetfillcolor{currentfill}%
\pgfsetlinewidth{0.000000pt}%
\definecolor{currentstroke}{rgb}{0.000000,0.000000,0.000000}%
\pgfsetstrokecolor{currentstroke}%
\pgfsetstrokeopacity{0.000000}%
\pgfsetdash{}{0pt}%
\pgfpathmoveto{\pgfqpoint{2.793635in}{1.344882in}}%
\pgfpathlineto{\pgfqpoint{2.802572in}{1.344882in}}%
\pgfpathlineto{\pgfqpoint{2.802572in}{1.315173in}}%
\pgfpathlineto{\pgfqpoint{2.793635in}{1.315173in}}%
\pgfpathlineto{\pgfqpoint{2.793635in}{1.344882in}}%
\pgfpathclose%
\pgfusepath{fill}%
\end{pgfscope}%
\begin{pgfscope}%
\pgfpathrectangle{\pgfqpoint{0.697024in}{0.857143in}}{\pgfqpoint{2.627103in}{1.813434in}}%
\pgfusepath{clip}%
\pgfsetbuttcap%
\pgfsetmiterjoin%
\definecolor{currentfill}{rgb}{0.511253,0.510898,0.193296}%
\pgfsetfillcolor{currentfill}%
\pgfsetlinewidth{0.000000pt}%
\definecolor{currentstroke}{rgb}{0.000000,0.000000,0.000000}%
\pgfsetstrokecolor{currentstroke}%
\pgfsetstrokeopacity{0.000000}%
\pgfsetdash{}{0pt}%
\pgfpathmoveto{\pgfqpoint{2.804806in}{1.351510in}}%
\pgfpathlineto{\pgfqpoint{2.813742in}{1.351510in}}%
\pgfpathlineto{\pgfqpoint{2.813742in}{1.320573in}}%
\pgfpathlineto{\pgfqpoint{2.804806in}{1.320573in}}%
\pgfpathlineto{\pgfqpoint{2.804806in}{1.351510in}}%
\pgfpathclose%
\pgfusepath{fill}%
\end{pgfscope}%
\begin{pgfscope}%
\pgfpathrectangle{\pgfqpoint{0.697024in}{0.857143in}}{\pgfqpoint{2.627103in}{1.813434in}}%
\pgfusepath{clip}%
\pgfsetbuttcap%
\pgfsetmiterjoin%
\definecolor{currentfill}{rgb}{0.511253,0.510898,0.193296}%
\pgfsetfillcolor{currentfill}%
\pgfsetlinewidth{0.000000pt}%
\definecolor{currentstroke}{rgb}{0.000000,0.000000,0.000000}%
\pgfsetstrokecolor{currentstroke}%
\pgfsetstrokeopacity{0.000000}%
\pgfsetdash{}{0pt}%
\pgfpathmoveto{\pgfqpoint{2.815977in}{1.438439in}}%
\pgfpathlineto{\pgfqpoint{2.824913in}{1.438439in}}%
\pgfpathlineto{\pgfqpoint{2.824913in}{1.410753in}}%
\pgfpathlineto{\pgfqpoint{2.815977in}{1.410753in}}%
\pgfpathlineto{\pgfqpoint{2.815977in}{1.438439in}}%
\pgfpathclose%
\pgfusepath{fill}%
\end{pgfscope}%
\begin{pgfscope}%
\pgfpathrectangle{\pgfqpoint{0.697024in}{0.857143in}}{\pgfqpoint{2.627103in}{1.813434in}}%
\pgfusepath{clip}%
\pgfsetbuttcap%
\pgfsetmiterjoin%
\definecolor{currentfill}{rgb}{0.511253,0.510898,0.193296}%
\pgfsetfillcolor{currentfill}%
\pgfsetlinewidth{0.000000pt}%
\definecolor{currentstroke}{rgb}{0.000000,0.000000,0.000000}%
\pgfsetstrokecolor{currentstroke}%
\pgfsetstrokeopacity{0.000000}%
\pgfsetdash{}{0pt}%
\pgfpathmoveto{\pgfqpoint{2.827147in}{1.376137in}}%
\pgfpathlineto{\pgfqpoint{2.836084in}{1.376137in}}%
\pgfpathlineto{\pgfqpoint{2.836084in}{1.348856in}}%
\pgfpathlineto{\pgfqpoint{2.827147in}{1.348856in}}%
\pgfpathlineto{\pgfqpoint{2.827147in}{1.376137in}}%
\pgfpathclose%
\pgfusepath{fill}%
\end{pgfscope}%
\begin{pgfscope}%
\pgfpathrectangle{\pgfqpoint{0.697024in}{0.857143in}}{\pgfqpoint{2.627103in}{1.813434in}}%
\pgfusepath{clip}%
\pgfsetbuttcap%
\pgfsetmiterjoin%
\definecolor{currentfill}{rgb}{0.511253,0.510898,0.193296}%
\pgfsetfillcolor{currentfill}%
\pgfsetlinewidth{0.000000pt}%
\definecolor{currentstroke}{rgb}{0.000000,0.000000,0.000000}%
\pgfsetstrokecolor{currentstroke}%
\pgfsetstrokeopacity{0.000000}%
\pgfsetdash{}{0pt}%
\pgfpathmoveto{\pgfqpoint{2.838318in}{1.332645in}}%
\pgfpathlineto{\pgfqpoint{2.847254in}{1.332645in}}%
\pgfpathlineto{\pgfqpoint{2.847254in}{1.306492in}}%
\pgfpathlineto{\pgfqpoint{2.838318in}{1.306492in}}%
\pgfpathlineto{\pgfqpoint{2.838318in}{1.332645in}}%
\pgfpathclose%
\pgfusepath{fill}%
\end{pgfscope}%
\begin{pgfscope}%
\pgfpathrectangle{\pgfqpoint{0.697024in}{0.857143in}}{\pgfqpoint{2.627103in}{1.813434in}}%
\pgfusepath{clip}%
\pgfsetbuttcap%
\pgfsetmiterjoin%
\definecolor{currentfill}{rgb}{0.511253,0.510898,0.193296}%
\pgfsetfillcolor{currentfill}%
\pgfsetlinewidth{0.000000pt}%
\definecolor{currentstroke}{rgb}{0.000000,0.000000,0.000000}%
\pgfsetstrokecolor{currentstroke}%
\pgfsetstrokeopacity{0.000000}%
\pgfsetdash{}{0pt}%
\pgfpathmoveto{\pgfqpoint{2.849488in}{1.401482in}}%
\pgfpathlineto{\pgfqpoint{2.858425in}{1.401482in}}%
\pgfpathlineto{\pgfqpoint{2.858425in}{1.374042in}}%
\pgfpathlineto{\pgfqpoint{2.849488in}{1.374042in}}%
\pgfpathlineto{\pgfqpoint{2.849488in}{1.401482in}}%
\pgfpathclose%
\pgfusepath{fill}%
\end{pgfscope}%
\begin{pgfscope}%
\pgfpathrectangle{\pgfqpoint{0.697024in}{0.857143in}}{\pgfqpoint{2.627103in}{1.813434in}}%
\pgfusepath{clip}%
\pgfsetbuttcap%
\pgfsetmiterjoin%
\definecolor{currentfill}{rgb}{0.511253,0.510898,0.193296}%
\pgfsetfillcolor{currentfill}%
\pgfsetlinewidth{0.000000pt}%
\definecolor{currentstroke}{rgb}{0.000000,0.000000,0.000000}%
\pgfsetstrokecolor{currentstroke}%
\pgfsetstrokeopacity{0.000000}%
\pgfsetdash{}{0pt}%
\pgfpathmoveto{\pgfqpoint{2.860659in}{1.376314in}}%
\pgfpathlineto{\pgfqpoint{2.869595in}{1.376314in}}%
\pgfpathlineto{\pgfqpoint{2.869595in}{1.346259in}}%
\pgfpathlineto{\pgfqpoint{2.860659in}{1.346259in}}%
\pgfpathlineto{\pgfqpoint{2.860659in}{1.376314in}}%
\pgfpathclose%
\pgfusepath{fill}%
\end{pgfscope}%
\begin{pgfscope}%
\pgfpathrectangle{\pgfqpoint{0.697024in}{0.857143in}}{\pgfqpoint{2.627103in}{1.813434in}}%
\pgfusepath{clip}%
\pgfsetbuttcap%
\pgfsetmiterjoin%
\definecolor{currentfill}{rgb}{0.511253,0.510898,0.193296}%
\pgfsetfillcolor{currentfill}%
\pgfsetlinewidth{0.000000pt}%
\definecolor{currentstroke}{rgb}{0.000000,0.000000,0.000000}%
\pgfsetstrokecolor{currentstroke}%
\pgfsetstrokeopacity{0.000000}%
\pgfsetdash{}{0pt}%
\pgfpathmoveto{\pgfqpoint{2.871830in}{1.365917in}}%
\pgfpathlineto{\pgfqpoint{2.880766in}{1.365917in}}%
\pgfpathlineto{\pgfqpoint{2.880766in}{1.337449in}}%
\pgfpathlineto{\pgfqpoint{2.871830in}{1.337449in}}%
\pgfpathlineto{\pgfqpoint{2.871830in}{1.365917in}}%
\pgfpathclose%
\pgfusepath{fill}%
\end{pgfscope}%
\begin{pgfscope}%
\pgfpathrectangle{\pgfqpoint{0.697024in}{0.857143in}}{\pgfqpoint{2.627103in}{1.813434in}}%
\pgfusepath{clip}%
\pgfsetbuttcap%
\pgfsetmiterjoin%
\definecolor{currentfill}{rgb}{0.511253,0.510898,0.193296}%
\pgfsetfillcolor{currentfill}%
\pgfsetlinewidth{0.000000pt}%
\definecolor{currentstroke}{rgb}{0.000000,0.000000,0.000000}%
\pgfsetstrokecolor{currentstroke}%
\pgfsetstrokeopacity{0.000000}%
\pgfsetdash{}{0pt}%
\pgfpathmoveto{\pgfqpoint{2.883000in}{1.375979in}}%
\pgfpathlineto{\pgfqpoint{2.891937in}{1.375979in}}%
\pgfpathlineto{\pgfqpoint{2.891937in}{1.345516in}}%
\pgfpathlineto{\pgfqpoint{2.883000in}{1.345516in}}%
\pgfpathlineto{\pgfqpoint{2.883000in}{1.375979in}}%
\pgfpathclose%
\pgfusepath{fill}%
\end{pgfscope}%
\begin{pgfscope}%
\pgfpathrectangle{\pgfqpoint{0.697024in}{0.857143in}}{\pgfqpoint{2.627103in}{1.813434in}}%
\pgfusepath{clip}%
\pgfsetbuttcap%
\pgfsetmiterjoin%
\definecolor{currentfill}{rgb}{0.511253,0.510898,0.193296}%
\pgfsetfillcolor{currentfill}%
\pgfsetlinewidth{0.000000pt}%
\definecolor{currentstroke}{rgb}{0.000000,0.000000,0.000000}%
\pgfsetstrokecolor{currentstroke}%
\pgfsetstrokeopacity{0.000000}%
\pgfsetdash{}{0pt}%
\pgfpathmoveto{\pgfqpoint{2.894171in}{1.323175in}}%
\pgfpathlineto{\pgfqpoint{2.903107in}{1.323175in}}%
\pgfpathlineto{\pgfqpoint{2.903107in}{1.295284in}}%
\pgfpathlineto{\pgfqpoint{2.894171in}{1.295284in}}%
\pgfpathlineto{\pgfqpoint{2.894171in}{1.323175in}}%
\pgfpathclose%
\pgfusepath{fill}%
\end{pgfscope}%
\begin{pgfscope}%
\pgfpathrectangle{\pgfqpoint{0.697024in}{0.857143in}}{\pgfqpoint{2.627103in}{1.813434in}}%
\pgfusepath{clip}%
\pgfsetbuttcap%
\pgfsetmiterjoin%
\definecolor{currentfill}{rgb}{0.511253,0.510898,0.193296}%
\pgfsetfillcolor{currentfill}%
\pgfsetlinewidth{0.000000pt}%
\definecolor{currentstroke}{rgb}{0.000000,0.000000,0.000000}%
\pgfsetstrokecolor{currentstroke}%
\pgfsetstrokeopacity{0.000000}%
\pgfsetdash{}{0pt}%
\pgfpathmoveto{\pgfqpoint{2.905341in}{1.304297in}}%
\pgfpathlineto{\pgfqpoint{2.914278in}{1.304297in}}%
\pgfpathlineto{\pgfqpoint{2.914278in}{1.277219in}}%
\pgfpathlineto{\pgfqpoint{2.905341in}{1.277219in}}%
\pgfpathlineto{\pgfqpoint{2.905341in}{1.304297in}}%
\pgfpathclose%
\pgfusepath{fill}%
\end{pgfscope}%
\begin{pgfscope}%
\pgfpathrectangle{\pgfqpoint{0.697024in}{0.857143in}}{\pgfqpoint{2.627103in}{1.813434in}}%
\pgfusepath{clip}%
\pgfsetbuttcap%
\pgfsetmiterjoin%
\definecolor{currentfill}{rgb}{0.511253,0.510898,0.193296}%
\pgfsetfillcolor{currentfill}%
\pgfsetlinewidth{0.000000pt}%
\definecolor{currentstroke}{rgb}{0.000000,0.000000,0.000000}%
\pgfsetstrokecolor{currentstroke}%
\pgfsetstrokeopacity{0.000000}%
\pgfsetdash{}{0pt}%
\pgfpathmoveto{\pgfqpoint{2.916512in}{1.314515in}}%
\pgfpathlineto{\pgfqpoint{2.925448in}{1.314515in}}%
\pgfpathlineto{\pgfqpoint{2.925448in}{1.288748in}}%
\pgfpathlineto{\pgfqpoint{2.916512in}{1.288748in}}%
\pgfpathlineto{\pgfqpoint{2.916512in}{1.314515in}}%
\pgfpathclose%
\pgfusepath{fill}%
\end{pgfscope}%
\begin{pgfscope}%
\pgfpathrectangle{\pgfqpoint{0.697024in}{0.857143in}}{\pgfqpoint{2.627103in}{1.813434in}}%
\pgfusepath{clip}%
\pgfsetbuttcap%
\pgfsetmiterjoin%
\definecolor{currentfill}{rgb}{0.511253,0.510898,0.193296}%
\pgfsetfillcolor{currentfill}%
\pgfsetlinewidth{0.000000pt}%
\definecolor{currentstroke}{rgb}{0.000000,0.000000,0.000000}%
\pgfsetstrokecolor{currentstroke}%
\pgfsetstrokeopacity{0.000000}%
\pgfsetdash{}{0pt}%
\pgfpathmoveto{\pgfqpoint{2.927683in}{1.300121in}}%
\pgfpathlineto{\pgfqpoint{2.936619in}{1.300121in}}%
\pgfpathlineto{\pgfqpoint{2.936619in}{1.274691in}}%
\pgfpathlineto{\pgfqpoint{2.927683in}{1.274691in}}%
\pgfpathlineto{\pgfqpoint{2.927683in}{1.300121in}}%
\pgfpathclose%
\pgfusepath{fill}%
\end{pgfscope}%
\begin{pgfscope}%
\pgfpathrectangle{\pgfqpoint{0.697024in}{0.857143in}}{\pgfqpoint{2.627103in}{1.813434in}}%
\pgfusepath{clip}%
\pgfsetbuttcap%
\pgfsetmiterjoin%
\definecolor{currentfill}{rgb}{0.511253,0.510898,0.193296}%
\pgfsetfillcolor{currentfill}%
\pgfsetlinewidth{0.000000pt}%
\definecolor{currentstroke}{rgb}{0.000000,0.000000,0.000000}%
\pgfsetstrokecolor{currentstroke}%
\pgfsetstrokeopacity{0.000000}%
\pgfsetdash{}{0pt}%
\pgfpathmoveto{\pgfqpoint{2.938853in}{1.251291in}}%
\pgfpathlineto{\pgfqpoint{2.947790in}{1.251291in}}%
\pgfpathlineto{\pgfqpoint{2.947790in}{1.225797in}}%
\pgfpathlineto{\pgfqpoint{2.938853in}{1.225797in}}%
\pgfpathlineto{\pgfqpoint{2.938853in}{1.251291in}}%
\pgfpathclose%
\pgfusepath{fill}%
\end{pgfscope}%
\begin{pgfscope}%
\pgfpathrectangle{\pgfqpoint{0.697024in}{0.857143in}}{\pgfqpoint{2.627103in}{1.813434in}}%
\pgfusepath{clip}%
\pgfsetbuttcap%
\pgfsetmiterjoin%
\definecolor{currentfill}{rgb}{0.511253,0.510898,0.193296}%
\pgfsetfillcolor{currentfill}%
\pgfsetlinewidth{0.000000pt}%
\definecolor{currentstroke}{rgb}{0.000000,0.000000,0.000000}%
\pgfsetstrokecolor{currentstroke}%
\pgfsetstrokeopacity{0.000000}%
\pgfsetdash{}{0pt}%
\pgfpathmoveto{\pgfqpoint{2.950024in}{1.308135in}}%
\pgfpathlineto{\pgfqpoint{2.958960in}{1.308135in}}%
\pgfpathlineto{\pgfqpoint{2.958960in}{1.287323in}}%
\pgfpathlineto{\pgfqpoint{2.950024in}{1.287323in}}%
\pgfpathlineto{\pgfqpoint{2.950024in}{1.308135in}}%
\pgfpathclose%
\pgfusepath{fill}%
\end{pgfscope}%
\begin{pgfscope}%
\pgfpathrectangle{\pgfqpoint{0.697024in}{0.857143in}}{\pgfqpoint{2.627103in}{1.813434in}}%
\pgfusepath{clip}%
\pgfsetbuttcap%
\pgfsetmiterjoin%
\definecolor{currentfill}{rgb}{0.511253,0.510898,0.193296}%
\pgfsetfillcolor{currentfill}%
\pgfsetlinewidth{0.000000pt}%
\definecolor{currentstroke}{rgb}{0.000000,0.000000,0.000000}%
\pgfsetstrokecolor{currentstroke}%
\pgfsetstrokeopacity{0.000000}%
\pgfsetdash{}{0pt}%
\pgfpathmoveto{\pgfqpoint{2.961194in}{1.311019in}}%
\pgfpathlineto{\pgfqpoint{2.970131in}{1.311019in}}%
\pgfpathlineto{\pgfqpoint{2.970131in}{1.292902in}}%
\pgfpathlineto{\pgfqpoint{2.961194in}{1.292902in}}%
\pgfpathlineto{\pgfqpoint{2.961194in}{1.311019in}}%
\pgfpathclose%
\pgfusepath{fill}%
\end{pgfscope}%
\begin{pgfscope}%
\pgfpathrectangle{\pgfqpoint{0.697024in}{0.857143in}}{\pgfqpoint{2.627103in}{1.813434in}}%
\pgfusepath{clip}%
\pgfsetbuttcap%
\pgfsetmiterjoin%
\definecolor{currentfill}{rgb}{0.511253,0.510898,0.193296}%
\pgfsetfillcolor{currentfill}%
\pgfsetlinewidth{0.000000pt}%
\definecolor{currentstroke}{rgb}{0.000000,0.000000,0.000000}%
\pgfsetstrokecolor{currentstroke}%
\pgfsetstrokeopacity{0.000000}%
\pgfsetdash{}{0pt}%
\pgfpathmoveto{\pgfqpoint{2.972365in}{1.303696in}}%
\pgfpathlineto{\pgfqpoint{2.981301in}{1.303696in}}%
\pgfpathlineto{\pgfqpoint{2.981301in}{1.285688in}}%
\pgfpathlineto{\pgfqpoint{2.972365in}{1.285688in}}%
\pgfpathlineto{\pgfqpoint{2.972365in}{1.303696in}}%
\pgfpathclose%
\pgfusepath{fill}%
\end{pgfscope}%
\begin{pgfscope}%
\pgfpathrectangle{\pgfqpoint{0.697024in}{0.857143in}}{\pgfqpoint{2.627103in}{1.813434in}}%
\pgfusepath{clip}%
\pgfsetbuttcap%
\pgfsetmiterjoin%
\definecolor{currentfill}{rgb}{0.511253,0.510898,0.193296}%
\pgfsetfillcolor{currentfill}%
\pgfsetlinewidth{0.000000pt}%
\definecolor{currentstroke}{rgb}{0.000000,0.000000,0.000000}%
\pgfsetstrokecolor{currentstroke}%
\pgfsetstrokeopacity{0.000000}%
\pgfsetdash{}{0pt}%
\pgfpathmoveto{\pgfqpoint{2.983536in}{1.252522in}}%
\pgfpathlineto{\pgfqpoint{2.992472in}{1.252522in}}%
\pgfpathlineto{\pgfqpoint{2.992472in}{1.240144in}}%
\pgfpathlineto{\pgfqpoint{2.983536in}{1.240144in}}%
\pgfpathlineto{\pgfqpoint{2.983536in}{1.252522in}}%
\pgfpathclose%
\pgfusepath{fill}%
\end{pgfscope}%
\begin{pgfscope}%
\pgfpathrectangle{\pgfqpoint{0.697024in}{0.857143in}}{\pgfqpoint{2.627103in}{1.813434in}}%
\pgfusepath{clip}%
\pgfsetbuttcap%
\pgfsetmiterjoin%
\definecolor{currentfill}{rgb}{0.511253,0.510898,0.193296}%
\pgfsetfillcolor{currentfill}%
\pgfsetlinewidth{0.000000pt}%
\definecolor{currentstroke}{rgb}{0.000000,0.000000,0.000000}%
\pgfsetstrokecolor{currentstroke}%
\pgfsetstrokeopacity{0.000000}%
\pgfsetdash{}{0pt}%
\pgfpathmoveto{\pgfqpoint{2.994706in}{1.295211in}}%
\pgfpathlineto{\pgfqpoint{3.003643in}{1.295211in}}%
\pgfpathlineto{\pgfqpoint{3.003643in}{1.283266in}}%
\pgfpathlineto{\pgfqpoint{2.994706in}{1.283266in}}%
\pgfpathlineto{\pgfqpoint{2.994706in}{1.295211in}}%
\pgfpathclose%
\pgfusepath{fill}%
\end{pgfscope}%
\begin{pgfscope}%
\pgfpathrectangle{\pgfqpoint{0.697024in}{0.857143in}}{\pgfqpoint{2.627103in}{1.813434in}}%
\pgfusepath{clip}%
\pgfsetbuttcap%
\pgfsetmiterjoin%
\definecolor{currentfill}{rgb}{0.511253,0.510898,0.193296}%
\pgfsetfillcolor{currentfill}%
\pgfsetlinewidth{0.000000pt}%
\definecolor{currentstroke}{rgb}{0.000000,0.000000,0.000000}%
\pgfsetstrokecolor{currentstroke}%
\pgfsetstrokeopacity{0.000000}%
\pgfsetdash{}{0pt}%
\pgfpathmoveto{\pgfqpoint{3.005877in}{1.281940in}}%
\pgfpathlineto{\pgfqpoint{3.014813in}{1.281940in}}%
\pgfpathlineto{\pgfqpoint{3.014813in}{1.271191in}}%
\pgfpathlineto{\pgfqpoint{3.005877in}{1.271191in}}%
\pgfpathlineto{\pgfqpoint{3.005877in}{1.281940in}}%
\pgfpathclose%
\pgfusepath{fill}%
\end{pgfscope}%
\begin{pgfscope}%
\pgfpathrectangle{\pgfqpoint{0.697024in}{0.857143in}}{\pgfqpoint{2.627103in}{1.813434in}}%
\pgfusepath{clip}%
\pgfsetbuttcap%
\pgfsetmiterjoin%
\definecolor{currentfill}{rgb}{0.511253,0.510898,0.193296}%
\pgfsetfillcolor{currentfill}%
\pgfsetlinewidth{0.000000pt}%
\definecolor{currentstroke}{rgb}{0.000000,0.000000,0.000000}%
\pgfsetstrokecolor{currentstroke}%
\pgfsetstrokeopacity{0.000000}%
\pgfsetdash{}{0pt}%
\pgfpathmoveto{\pgfqpoint{3.017047in}{1.146754in}}%
\pgfpathlineto{\pgfqpoint{3.025984in}{1.146754in}}%
\pgfpathlineto{\pgfqpoint{3.025984in}{1.140376in}}%
\pgfpathlineto{\pgfqpoint{3.017047in}{1.140376in}}%
\pgfpathlineto{\pgfqpoint{3.017047in}{1.146754in}}%
\pgfpathclose%
\pgfusepath{fill}%
\end{pgfscope}%
\begin{pgfscope}%
\pgfpathrectangle{\pgfqpoint{0.697024in}{0.857143in}}{\pgfqpoint{2.627103in}{1.813434in}}%
\pgfusepath{clip}%
\pgfsetbuttcap%
\pgfsetmiterjoin%
\definecolor{currentfill}{rgb}{0.511253,0.510898,0.193296}%
\pgfsetfillcolor{currentfill}%
\pgfsetlinewidth{0.000000pt}%
\definecolor{currentstroke}{rgb}{0.000000,0.000000,0.000000}%
\pgfsetstrokecolor{currentstroke}%
\pgfsetstrokeopacity{0.000000}%
\pgfsetdash{}{0pt}%
\pgfpathmoveto{\pgfqpoint{3.028218in}{1.139195in}}%
\pgfpathlineto{\pgfqpoint{3.037155in}{1.139195in}}%
\pgfpathlineto{\pgfqpoint{3.037155in}{1.135721in}}%
\pgfpathlineto{\pgfqpoint{3.028218in}{1.135721in}}%
\pgfpathlineto{\pgfqpoint{3.028218in}{1.139195in}}%
\pgfpathclose%
\pgfusepath{fill}%
\end{pgfscope}%
\begin{pgfscope}%
\pgfpathrectangle{\pgfqpoint{0.697024in}{0.857143in}}{\pgfqpoint{2.627103in}{1.813434in}}%
\pgfusepath{clip}%
\pgfsetbuttcap%
\pgfsetmiterjoin%
\definecolor{currentfill}{rgb}{0.511253,0.510898,0.193296}%
\pgfsetfillcolor{currentfill}%
\pgfsetlinewidth{0.000000pt}%
\definecolor{currentstroke}{rgb}{0.000000,0.000000,0.000000}%
\pgfsetstrokecolor{currentstroke}%
\pgfsetstrokeopacity{0.000000}%
\pgfsetdash{}{0pt}%
\pgfpathmoveto{\pgfqpoint{3.039389in}{1.174176in}}%
\pgfpathlineto{\pgfqpoint{3.048325in}{1.174176in}}%
\pgfpathlineto{\pgfqpoint{3.048325in}{1.173527in}}%
\pgfpathlineto{\pgfqpoint{3.039389in}{1.173527in}}%
\pgfpathlineto{\pgfqpoint{3.039389in}{1.174176in}}%
\pgfpathclose%
\pgfusepath{fill}%
\end{pgfscope}%
\begin{pgfscope}%
\pgfpathrectangle{\pgfqpoint{0.697024in}{0.857143in}}{\pgfqpoint{2.627103in}{1.813434in}}%
\pgfusepath{clip}%
\pgfsetbuttcap%
\pgfsetmiterjoin%
\definecolor{currentfill}{rgb}{0.511253,0.510898,0.193296}%
\pgfsetfillcolor{currentfill}%
\pgfsetlinewidth{0.000000pt}%
\definecolor{currentstroke}{rgb}{0.000000,0.000000,0.000000}%
\pgfsetstrokecolor{currentstroke}%
\pgfsetstrokeopacity{0.000000}%
\pgfsetdash{}{0pt}%
\pgfpathmoveto{\pgfqpoint{3.050559in}{2.128011in}}%
\pgfpathlineto{\pgfqpoint{3.059496in}{2.128011in}}%
\pgfpathlineto{\pgfqpoint{3.059496in}{2.129440in}}%
\pgfpathlineto{\pgfqpoint{3.050559in}{2.129440in}}%
\pgfpathlineto{\pgfqpoint{3.050559in}{2.128011in}}%
\pgfpathclose%
\pgfusepath{fill}%
\end{pgfscope}%
\begin{pgfscope}%
\pgfpathrectangle{\pgfqpoint{0.697024in}{0.857143in}}{\pgfqpoint{2.627103in}{1.813434in}}%
\pgfusepath{clip}%
\pgfsetbuttcap%
\pgfsetmiterjoin%
\definecolor{currentfill}{rgb}{0.511253,0.510898,0.193296}%
\pgfsetfillcolor{currentfill}%
\pgfsetlinewidth{0.000000pt}%
\definecolor{currentstroke}{rgb}{0.000000,0.000000,0.000000}%
\pgfsetstrokecolor{currentstroke}%
\pgfsetstrokeopacity{0.000000}%
\pgfsetdash{}{0pt}%
\pgfpathmoveto{\pgfqpoint{3.061730in}{2.062674in}}%
\pgfpathlineto{\pgfqpoint{3.070666in}{2.062674in}}%
\pgfpathlineto{\pgfqpoint{3.070666in}{2.066726in}}%
\pgfpathlineto{\pgfqpoint{3.061730in}{2.066726in}}%
\pgfpathlineto{\pgfqpoint{3.061730in}{2.062674in}}%
\pgfpathclose%
\pgfusepath{fill}%
\end{pgfscope}%
\begin{pgfscope}%
\pgfpathrectangle{\pgfqpoint{0.697024in}{0.857143in}}{\pgfqpoint{2.627103in}{1.813434in}}%
\pgfusepath{clip}%
\pgfsetbuttcap%
\pgfsetmiterjoin%
\definecolor{currentfill}{rgb}{0.511253,0.510898,0.193296}%
\pgfsetfillcolor{currentfill}%
\pgfsetlinewidth{0.000000pt}%
\definecolor{currentstroke}{rgb}{0.000000,0.000000,0.000000}%
\pgfsetstrokecolor{currentstroke}%
\pgfsetstrokeopacity{0.000000}%
\pgfsetdash{}{0pt}%
\pgfpathmoveto{\pgfqpoint{3.072900in}{2.065984in}}%
\pgfpathlineto{\pgfqpoint{3.081837in}{2.065984in}}%
\pgfpathlineto{\pgfqpoint{3.081837in}{2.070502in}}%
\pgfpathlineto{\pgfqpoint{3.072900in}{2.070502in}}%
\pgfpathlineto{\pgfqpoint{3.072900in}{2.065984in}}%
\pgfpathclose%
\pgfusepath{fill}%
\end{pgfscope}%
\begin{pgfscope}%
\pgfpathrectangle{\pgfqpoint{0.697024in}{0.857143in}}{\pgfqpoint{2.627103in}{1.813434in}}%
\pgfusepath{clip}%
\pgfsetbuttcap%
\pgfsetmiterjoin%
\definecolor{currentfill}{rgb}{0.511253,0.510898,0.193296}%
\pgfsetfillcolor{currentfill}%
\pgfsetlinewidth{0.000000pt}%
\definecolor{currentstroke}{rgb}{0.000000,0.000000,0.000000}%
\pgfsetstrokecolor{currentstroke}%
\pgfsetstrokeopacity{0.000000}%
\pgfsetdash{}{0pt}%
\pgfpathmoveto{\pgfqpoint{3.084071in}{2.153858in}}%
\pgfpathlineto{\pgfqpoint{3.093008in}{2.153858in}}%
\pgfpathlineto{\pgfqpoint{3.093008in}{2.161378in}}%
\pgfpathlineto{\pgfqpoint{3.084071in}{2.161378in}}%
\pgfpathlineto{\pgfqpoint{3.084071in}{2.153858in}}%
\pgfpathclose%
\pgfusepath{fill}%
\end{pgfscope}%
\begin{pgfscope}%
\pgfpathrectangle{\pgfqpoint{0.697024in}{0.857143in}}{\pgfqpoint{2.627103in}{1.813434in}}%
\pgfusepath{clip}%
\pgfsetbuttcap%
\pgfsetmiterjoin%
\definecolor{currentfill}{rgb}{0.511253,0.510898,0.193296}%
\pgfsetfillcolor{currentfill}%
\pgfsetlinewidth{0.000000pt}%
\definecolor{currentstroke}{rgb}{0.000000,0.000000,0.000000}%
\pgfsetstrokecolor{currentstroke}%
\pgfsetstrokeopacity{0.000000}%
\pgfsetdash{}{0pt}%
\pgfpathmoveto{\pgfqpoint{3.095242in}{2.077878in}}%
\pgfpathlineto{\pgfqpoint{3.104178in}{2.077878in}}%
\pgfpathlineto{\pgfqpoint{3.104178in}{2.087776in}}%
\pgfpathlineto{\pgfqpoint{3.095242in}{2.087776in}}%
\pgfpathlineto{\pgfqpoint{3.095242in}{2.077878in}}%
\pgfpathclose%
\pgfusepath{fill}%
\end{pgfscope}%
\begin{pgfscope}%
\pgfpathrectangle{\pgfqpoint{0.697024in}{0.857143in}}{\pgfqpoint{2.627103in}{1.813434in}}%
\pgfusepath{clip}%
\pgfsetbuttcap%
\pgfsetmiterjoin%
\definecolor{currentfill}{rgb}{0.511253,0.510898,0.193296}%
\pgfsetfillcolor{currentfill}%
\pgfsetlinewidth{0.000000pt}%
\definecolor{currentstroke}{rgb}{0.000000,0.000000,0.000000}%
\pgfsetstrokecolor{currentstroke}%
\pgfsetstrokeopacity{0.000000}%
\pgfsetdash{}{0pt}%
\pgfpathmoveto{\pgfqpoint{3.106412in}{2.032722in}}%
\pgfpathlineto{\pgfqpoint{3.115349in}{2.032722in}}%
\pgfpathlineto{\pgfqpoint{3.115349in}{2.042483in}}%
\pgfpathlineto{\pgfqpoint{3.106412in}{2.042483in}}%
\pgfpathlineto{\pgfqpoint{3.106412in}{2.032722in}}%
\pgfpathclose%
\pgfusepath{fill}%
\end{pgfscope}%
\begin{pgfscope}%
\pgfpathrectangle{\pgfqpoint{0.697024in}{0.857143in}}{\pgfqpoint{2.627103in}{1.813434in}}%
\pgfusepath{clip}%
\pgfsetbuttcap%
\pgfsetmiterjoin%
\definecolor{currentfill}{rgb}{0.511253,0.510898,0.193296}%
\pgfsetfillcolor{currentfill}%
\pgfsetlinewidth{0.000000pt}%
\definecolor{currentstroke}{rgb}{0.000000,0.000000,0.000000}%
\pgfsetstrokecolor{currentstroke}%
\pgfsetstrokeopacity{0.000000}%
\pgfsetdash{}{0pt}%
\pgfpathmoveto{\pgfqpoint{3.117583in}{2.063725in}}%
\pgfpathlineto{\pgfqpoint{3.126519in}{2.063725in}}%
\pgfpathlineto{\pgfqpoint{3.126519in}{2.072700in}}%
\pgfpathlineto{\pgfqpoint{3.117583in}{2.072700in}}%
\pgfpathlineto{\pgfqpoint{3.117583in}{2.063725in}}%
\pgfpathclose%
\pgfusepath{fill}%
\end{pgfscope}%
\begin{pgfscope}%
\pgfpathrectangle{\pgfqpoint{0.697024in}{0.857143in}}{\pgfqpoint{2.627103in}{1.813434in}}%
\pgfusepath{clip}%
\pgfsetbuttcap%
\pgfsetmiterjoin%
\definecolor{currentfill}{rgb}{0.511253,0.510898,0.193296}%
\pgfsetfillcolor{currentfill}%
\pgfsetlinewidth{0.000000pt}%
\definecolor{currentstroke}{rgb}{0.000000,0.000000,0.000000}%
\pgfsetstrokecolor{currentstroke}%
\pgfsetstrokeopacity{0.000000}%
\pgfsetdash{}{0pt}%
\pgfpathmoveto{\pgfqpoint{3.128753in}{2.057803in}}%
\pgfpathlineto{\pgfqpoint{3.137690in}{2.057803in}}%
\pgfpathlineto{\pgfqpoint{3.137690in}{2.069804in}}%
\pgfpathlineto{\pgfqpoint{3.128753in}{2.069804in}}%
\pgfpathlineto{\pgfqpoint{3.128753in}{2.057803in}}%
\pgfpathclose%
\pgfusepath{fill}%
\end{pgfscope}%
\begin{pgfscope}%
\pgfpathrectangle{\pgfqpoint{0.697024in}{0.857143in}}{\pgfqpoint{2.627103in}{1.813434in}}%
\pgfusepath{clip}%
\pgfsetbuttcap%
\pgfsetmiterjoin%
\definecolor{currentfill}{rgb}{0.511253,0.510898,0.193296}%
\pgfsetfillcolor{currentfill}%
\pgfsetlinewidth{0.000000pt}%
\definecolor{currentstroke}{rgb}{0.000000,0.000000,0.000000}%
\pgfsetstrokecolor{currentstroke}%
\pgfsetstrokeopacity{0.000000}%
\pgfsetdash{}{0pt}%
\pgfpathmoveto{\pgfqpoint{3.139924in}{2.130247in}}%
\pgfpathlineto{\pgfqpoint{3.148861in}{2.130247in}}%
\pgfpathlineto{\pgfqpoint{3.148861in}{2.143621in}}%
\pgfpathlineto{\pgfqpoint{3.139924in}{2.143621in}}%
\pgfpathlineto{\pgfqpoint{3.139924in}{2.130247in}}%
\pgfpathclose%
\pgfusepath{fill}%
\end{pgfscope}%
\begin{pgfscope}%
\pgfpathrectangle{\pgfqpoint{0.697024in}{0.857143in}}{\pgfqpoint{2.627103in}{1.813434in}}%
\pgfusepath{clip}%
\pgfsetbuttcap%
\pgfsetmiterjoin%
\definecolor{currentfill}{rgb}{0.511253,0.510898,0.193296}%
\pgfsetfillcolor{currentfill}%
\pgfsetlinewidth{0.000000pt}%
\definecolor{currentstroke}{rgb}{0.000000,0.000000,0.000000}%
\pgfsetstrokecolor{currentstroke}%
\pgfsetstrokeopacity{0.000000}%
\pgfsetdash{}{0pt}%
\pgfpathmoveto{\pgfqpoint{3.151095in}{2.166955in}}%
\pgfpathlineto{\pgfqpoint{3.160031in}{2.166955in}}%
\pgfpathlineto{\pgfqpoint{3.160031in}{2.181567in}}%
\pgfpathlineto{\pgfqpoint{3.151095in}{2.181567in}}%
\pgfpathlineto{\pgfqpoint{3.151095in}{2.166955in}}%
\pgfpathclose%
\pgfusepath{fill}%
\end{pgfscope}%
\begin{pgfscope}%
\pgfpathrectangle{\pgfqpoint{0.697024in}{0.857143in}}{\pgfqpoint{2.627103in}{1.813434in}}%
\pgfusepath{clip}%
\pgfsetbuttcap%
\pgfsetmiterjoin%
\definecolor{currentfill}{rgb}{0.511253,0.510898,0.193296}%
\pgfsetfillcolor{currentfill}%
\pgfsetlinewidth{0.000000pt}%
\definecolor{currentstroke}{rgb}{0.000000,0.000000,0.000000}%
\pgfsetstrokecolor{currentstroke}%
\pgfsetstrokeopacity{0.000000}%
\pgfsetdash{}{0pt}%
\pgfpathmoveto{\pgfqpoint{3.162265in}{2.255901in}}%
\pgfpathlineto{\pgfqpoint{3.171202in}{2.255901in}}%
\pgfpathlineto{\pgfqpoint{3.171202in}{2.274219in}}%
\pgfpathlineto{\pgfqpoint{3.162265in}{2.274219in}}%
\pgfpathlineto{\pgfqpoint{3.162265in}{2.255901in}}%
\pgfpathclose%
\pgfusepath{fill}%
\end{pgfscope}%
\begin{pgfscope}%
\pgfpathrectangle{\pgfqpoint{0.697024in}{0.857143in}}{\pgfqpoint{2.627103in}{1.813434in}}%
\pgfusepath{clip}%
\pgfsetbuttcap%
\pgfsetmiterjoin%
\definecolor{currentfill}{rgb}{0.511253,0.510898,0.193296}%
\pgfsetfillcolor{currentfill}%
\pgfsetlinewidth{0.000000pt}%
\definecolor{currentstroke}{rgb}{0.000000,0.000000,0.000000}%
\pgfsetstrokecolor{currentstroke}%
\pgfsetstrokeopacity{0.000000}%
\pgfsetdash{}{0pt}%
\pgfpathmoveto{\pgfqpoint{3.173436in}{2.200243in}}%
\pgfpathlineto{\pgfqpoint{3.182372in}{2.200243in}}%
\pgfpathlineto{\pgfqpoint{3.182372in}{2.217730in}}%
\pgfpathlineto{\pgfqpoint{3.173436in}{2.217730in}}%
\pgfpathlineto{\pgfqpoint{3.173436in}{2.200243in}}%
\pgfpathclose%
\pgfusepath{fill}%
\end{pgfscope}%
\begin{pgfscope}%
\pgfpathrectangle{\pgfqpoint{0.697024in}{0.857143in}}{\pgfqpoint{2.627103in}{1.813434in}}%
\pgfusepath{clip}%
\pgfsetbuttcap%
\pgfsetmiterjoin%
\definecolor{currentfill}{rgb}{0.511253,0.510898,0.193296}%
\pgfsetfillcolor{currentfill}%
\pgfsetlinewidth{0.000000pt}%
\definecolor{currentstroke}{rgb}{0.000000,0.000000,0.000000}%
\pgfsetstrokecolor{currentstroke}%
\pgfsetstrokeopacity{0.000000}%
\pgfsetdash{}{0pt}%
\pgfpathmoveto{\pgfqpoint{3.184607in}{2.258040in}}%
\pgfpathlineto{\pgfqpoint{3.193543in}{2.258040in}}%
\pgfpathlineto{\pgfqpoint{3.193543in}{2.277559in}}%
\pgfpathlineto{\pgfqpoint{3.184607in}{2.277559in}}%
\pgfpathlineto{\pgfqpoint{3.184607in}{2.258040in}}%
\pgfpathclose%
\pgfusepath{fill}%
\end{pgfscope}%
\begin{pgfscope}%
\pgfpathrectangle{\pgfqpoint{0.697024in}{0.857143in}}{\pgfqpoint{2.627103in}{1.813434in}}%
\pgfusepath{clip}%
\pgfsetbuttcap%
\pgfsetmiterjoin%
\definecolor{currentfill}{rgb}{0.511253,0.510898,0.193296}%
\pgfsetfillcolor{currentfill}%
\pgfsetlinewidth{0.000000pt}%
\definecolor{currentstroke}{rgb}{0.000000,0.000000,0.000000}%
\pgfsetstrokecolor{currentstroke}%
\pgfsetstrokeopacity{0.000000}%
\pgfsetdash{}{0pt}%
\pgfpathmoveto{\pgfqpoint{3.195777in}{2.268961in}}%
\pgfpathlineto{\pgfqpoint{3.204714in}{2.268961in}}%
\pgfpathlineto{\pgfqpoint{3.204714in}{2.291081in}}%
\pgfpathlineto{\pgfqpoint{3.195777in}{2.291081in}}%
\pgfpathlineto{\pgfqpoint{3.195777in}{2.268961in}}%
\pgfpathclose%
\pgfusepath{fill}%
\end{pgfscope}%
\begin{pgfscope}%
\pgfpathrectangle{\pgfqpoint{0.697024in}{0.857143in}}{\pgfqpoint{2.627103in}{1.813434in}}%
\pgfusepath{clip}%
\pgfsetbuttcap%
\pgfsetmiterjoin%
\definecolor{currentfill}{rgb}{0.754268,0.565033,0.211761}%
\pgfsetfillcolor{currentfill}%
\pgfsetlinewidth{0.000000pt}%
\definecolor{currentstroke}{rgb}{0.000000,0.000000,0.000000}%
\pgfsetstrokecolor{currentstroke}%
\pgfsetstrokeopacity{0.000000}%
\pgfsetdash{}{0pt}%
\pgfpathmoveto{\pgfqpoint{0.816438in}{1.881732in}}%
\pgfpathlineto{\pgfqpoint{0.825375in}{1.881732in}}%
\pgfpathlineto{\pgfqpoint{0.825375in}{1.894532in}}%
\pgfpathlineto{\pgfqpoint{0.816438in}{1.894532in}}%
\pgfpathlineto{\pgfqpoint{0.816438in}{1.881732in}}%
\pgfpathclose%
\pgfusepath{fill}%
\end{pgfscope}%
\begin{pgfscope}%
\pgfpathrectangle{\pgfqpoint{0.697024in}{0.857143in}}{\pgfqpoint{2.627103in}{1.813434in}}%
\pgfusepath{clip}%
\pgfsetbuttcap%
\pgfsetmiterjoin%
\definecolor{currentfill}{rgb}{0.754268,0.565033,0.211761}%
\pgfsetfillcolor{currentfill}%
\pgfsetlinewidth{0.000000pt}%
\definecolor{currentstroke}{rgb}{0.000000,0.000000,0.000000}%
\pgfsetstrokecolor{currentstroke}%
\pgfsetstrokeopacity{0.000000}%
\pgfsetdash{}{0pt}%
\pgfpathmoveto{\pgfqpoint{0.827609in}{1.600680in}}%
\pgfpathlineto{\pgfqpoint{0.836545in}{1.600680in}}%
\pgfpathlineto{\pgfqpoint{0.836545in}{1.589723in}}%
\pgfpathlineto{\pgfqpoint{0.827609in}{1.589723in}}%
\pgfpathlineto{\pgfqpoint{0.827609in}{1.600680in}}%
\pgfpathclose%
\pgfusepath{fill}%
\end{pgfscope}%
\begin{pgfscope}%
\pgfpathrectangle{\pgfqpoint{0.697024in}{0.857143in}}{\pgfqpoint{2.627103in}{1.813434in}}%
\pgfusepath{clip}%
\pgfsetbuttcap%
\pgfsetmiterjoin%
\definecolor{currentfill}{rgb}{0.754268,0.565033,0.211761}%
\pgfsetfillcolor{currentfill}%
\pgfsetlinewidth{0.000000pt}%
\definecolor{currentstroke}{rgb}{0.000000,0.000000,0.000000}%
\pgfsetstrokecolor{currentstroke}%
\pgfsetstrokeopacity{0.000000}%
\pgfsetdash{}{0pt}%
\pgfpathmoveto{\pgfqpoint{0.838779in}{1.538643in}}%
\pgfpathlineto{\pgfqpoint{0.847716in}{1.538643in}}%
\pgfpathlineto{\pgfqpoint{0.847716in}{1.488412in}}%
\pgfpathlineto{\pgfqpoint{0.838779in}{1.488412in}}%
\pgfpathlineto{\pgfqpoint{0.838779in}{1.538643in}}%
\pgfpathclose%
\pgfusepath{fill}%
\end{pgfscope}%
\begin{pgfscope}%
\pgfpathrectangle{\pgfqpoint{0.697024in}{0.857143in}}{\pgfqpoint{2.627103in}{1.813434in}}%
\pgfusepath{clip}%
\pgfsetbuttcap%
\pgfsetmiterjoin%
\definecolor{currentfill}{rgb}{0.754268,0.565033,0.211761}%
\pgfsetfillcolor{currentfill}%
\pgfsetlinewidth{0.000000pt}%
\definecolor{currentstroke}{rgb}{0.000000,0.000000,0.000000}%
\pgfsetstrokecolor{currentstroke}%
\pgfsetstrokeopacity{0.000000}%
\pgfsetdash{}{0pt}%
\pgfpathmoveto{\pgfqpoint{0.849950in}{1.565867in}}%
\pgfpathlineto{\pgfqpoint{0.858886in}{1.565867in}}%
\pgfpathlineto{\pgfqpoint{0.858886in}{1.553819in}}%
\pgfpathlineto{\pgfqpoint{0.849950in}{1.553819in}}%
\pgfpathlineto{\pgfqpoint{0.849950in}{1.565867in}}%
\pgfpathclose%
\pgfusepath{fill}%
\end{pgfscope}%
\begin{pgfscope}%
\pgfpathrectangle{\pgfqpoint{0.697024in}{0.857143in}}{\pgfqpoint{2.627103in}{1.813434in}}%
\pgfusepath{clip}%
\pgfsetbuttcap%
\pgfsetmiterjoin%
\definecolor{currentfill}{rgb}{0.754268,0.565033,0.211761}%
\pgfsetfillcolor{currentfill}%
\pgfsetlinewidth{0.000000pt}%
\definecolor{currentstroke}{rgb}{0.000000,0.000000,0.000000}%
\pgfsetstrokecolor{currentstroke}%
\pgfsetstrokeopacity{0.000000}%
\pgfsetdash{}{0pt}%
\pgfpathmoveto{\pgfqpoint{0.861121in}{1.919282in}}%
\pgfpathlineto{\pgfqpoint{0.870057in}{1.919282in}}%
\pgfpathlineto{\pgfqpoint{0.870057in}{1.970601in}}%
\pgfpathlineto{\pgfqpoint{0.861121in}{1.970601in}}%
\pgfpathlineto{\pgfqpoint{0.861121in}{1.919282in}}%
\pgfpathclose%
\pgfusepath{fill}%
\end{pgfscope}%
\begin{pgfscope}%
\pgfpathrectangle{\pgfqpoint{0.697024in}{0.857143in}}{\pgfqpoint{2.627103in}{1.813434in}}%
\pgfusepath{clip}%
\pgfsetbuttcap%
\pgfsetmiterjoin%
\definecolor{currentfill}{rgb}{0.754268,0.565033,0.211761}%
\pgfsetfillcolor{currentfill}%
\pgfsetlinewidth{0.000000pt}%
\definecolor{currentstroke}{rgb}{0.000000,0.000000,0.000000}%
\pgfsetstrokecolor{currentstroke}%
\pgfsetstrokeopacity{0.000000}%
\pgfsetdash{}{0pt}%
\pgfpathmoveto{\pgfqpoint{0.872291in}{1.875523in}}%
\pgfpathlineto{\pgfqpoint{0.881228in}{1.875523in}}%
\pgfpathlineto{\pgfqpoint{0.881228in}{1.935839in}}%
\pgfpathlineto{\pgfqpoint{0.872291in}{1.935839in}}%
\pgfpathlineto{\pgfqpoint{0.872291in}{1.875523in}}%
\pgfpathclose%
\pgfusepath{fill}%
\end{pgfscope}%
\begin{pgfscope}%
\pgfpathrectangle{\pgfqpoint{0.697024in}{0.857143in}}{\pgfqpoint{2.627103in}{1.813434in}}%
\pgfusepath{clip}%
\pgfsetbuttcap%
\pgfsetmiterjoin%
\definecolor{currentfill}{rgb}{0.754268,0.565033,0.211761}%
\pgfsetfillcolor{currentfill}%
\pgfsetlinewidth{0.000000pt}%
\definecolor{currentstroke}{rgb}{0.000000,0.000000,0.000000}%
\pgfsetstrokecolor{currentstroke}%
\pgfsetstrokeopacity{0.000000}%
\pgfsetdash{}{0pt}%
\pgfpathmoveto{\pgfqpoint{0.883462in}{1.987277in}}%
\pgfpathlineto{\pgfqpoint{0.892398in}{1.987277in}}%
\pgfpathlineto{\pgfqpoint{0.892398in}{2.023885in}}%
\pgfpathlineto{\pgfqpoint{0.883462in}{2.023885in}}%
\pgfpathlineto{\pgfqpoint{0.883462in}{1.987277in}}%
\pgfpathclose%
\pgfusepath{fill}%
\end{pgfscope}%
\begin{pgfscope}%
\pgfpathrectangle{\pgfqpoint{0.697024in}{0.857143in}}{\pgfqpoint{2.627103in}{1.813434in}}%
\pgfusepath{clip}%
\pgfsetbuttcap%
\pgfsetmiterjoin%
\definecolor{currentfill}{rgb}{0.754268,0.565033,0.211761}%
\pgfsetfillcolor{currentfill}%
\pgfsetlinewidth{0.000000pt}%
\definecolor{currentstroke}{rgb}{0.000000,0.000000,0.000000}%
\pgfsetstrokecolor{currentstroke}%
\pgfsetstrokeopacity{0.000000}%
\pgfsetdash{}{0pt}%
\pgfpathmoveto{\pgfqpoint{0.894632in}{2.073701in}}%
\pgfpathlineto{\pgfqpoint{0.903569in}{2.073701in}}%
\pgfpathlineto{\pgfqpoint{0.903569in}{2.077324in}}%
\pgfpathlineto{\pgfqpoint{0.894632in}{2.077324in}}%
\pgfpathlineto{\pgfqpoint{0.894632in}{2.073701in}}%
\pgfpathclose%
\pgfusepath{fill}%
\end{pgfscope}%
\begin{pgfscope}%
\pgfpathrectangle{\pgfqpoint{0.697024in}{0.857143in}}{\pgfqpoint{2.627103in}{1.813434in}}%
\pgfusepath{clip}%
\pgfsetbuttcap%
\pgfsetmiterjoin%
\definecolor{currentfill}{rgb}{0.754268,0.565033,0.211761}%
\pgfsetfillcolor{currentfill}%
\pgfsetlinewidth{0.000000pt}%
\definecolor{currentstroke}{rgb}{0.000000,0.000000,0.000000}%
\pgfsetstrokecolor{currentstroke}%
\pgfsetstrokeopacity{0.000000}%
\pgfsetdash{}{0pt}%
\pgfpathmoveto{\pgfqpoint{0.905803in}{1.604699in}}%
\pgfpathlineto{\pgfqpoint{0.914739in}{1.604699in}}%
\pgfpathlineto{\pgfqpoint{0.914739in}{1.595804in}}%
\pgfpathlineto{\pgfqpoint{0.905803in}{1.595804in}}%
\pgfpathlineto{\pgfqpoint{0.905803in}{1.604699in}}%
\pgfpathclose%
\pgfusepath{fill}%
\end{pgfscope}%
\begin{pgfscope}%
\pgfpathrectangle{\pgfqpoint{0.697024in}{0.857143in}}{\pgfqpoint{2.627103in}{1.813434in}}%
\pgfusepath{clip}%
\pgfsetbuttcap%
\pgfsetmiterjoin%
\definecolor{currentfill}{rgb}{0.754268,0.565033,0.211761}%
\pgfsetfillcolor{currentfill}%
\pgfsetlinewidth{0.000000pt}%
\definecolor{currentstroke}{rgb}{0.000000,0.000000,0.000000}%
\pgfsetstrokecolor{currentstroke}%
\pgfsetstrokeopacity{0.000000}%
\pgfsetdash{}{0pt}%
\pgfpathmoveto{\pgfqpoint{0.916974in}{1.995758in}}%
\pgfpathlineto{\pgfqpoint{0.925910in}{1.995758in}}%
\pgfpathlineto{\pgfqpoint{0.925910in}{2.048880in}}%
\pgfpathlineto{\pgfqpoint{0.916974in}{2.048880in}}%
\pgfpathlineto{\pgfqpoint{0.916974in}{1.995758in}}%
\pgfpathclose%
\pgfusepath{fill}%
\end{pgfscope}%
\begin{pgfscope}%
\pgfpathrectangle{\pgfqpoint{0.697024in}{0.857143in}}{\pgfqpoint{2.627103in}{1.813434in}}%
\pgfusepath{clip}%
\pgfsetbuttcap%
\pgfsetmiterjoin%
\definecolor{currentfill}{rgb}{0.754268,0.565033,0.211761}%
\pgfsetfillcolor{currentfill}%
\pgfsetlinewidth{0.000000pt}%
\definecolor{currentstroke}{rgb}{0.000000,0.000000,0.000000}%
\pgfsetstrokecolor{currentstroke}%
\pgfsetstrokeopacity{0.000000}%
\pgfsetdash{}{0pt}%
\pgfpathmoveto{\pgfqpoint{0.928144in}{1.609445in}}%
\pgfpathlineto{\pgfqpoint{0.937081in}{1.609445in}}%
\pgfpathlineto{\pgfqpoint{0.937081in}{1.598060in}}%
\pgfpathlineto{\pgfqpoint{0.928144in}{1.598060in}}%
\pgfpathlineto{\pgfqpoint{0.928144in}{1.609445in}}%
\pgfpathclose%
\pgfusepath{fill}%
\end{pgfscope}%
\begin{pgfscope}%
\pgfpathrectangle{\pgfqpoint{0.697024in}{0.857143in}}{\pgfqpoint{2.627103in}{1.813434in}}%
\pgfusepath{clip}%
\pgfsetbuttcap%
\pgfsetmiterjoin%
\definecolor{currentfill}{rgb}{0.754268,0.565033,0.211761}%
\pgfsetfillcolor{currentfill}%
\pgfsetlinewidth{0.000000pt}%
\definecolor{currentstroke}{rgb}{0.000000,0.000000,0.000000}%
\pgfsetstrokecolor{currentstroke}%
\pgfsetstrokeopacity{0.000000}%
\pgfsetdash{}{0pt}%
\pgfpathmoveto{\pgfqpoint{0.939315in}{2.123375in}}%
\pgfpathlineto{\pgfqpoint{0.948251in}{2.123375in}}%
\pgfpathlineto{\pgfqpoint{0.948251in}{2.123526in}}%
\pgfpathlineto{\pgfqpoint{0.939315in}{2.123526in}}%
\pgfpathlineto{\pgfqpoint{0.939315in}{2.123375in}}%
\pgfpathclose%
\pgfusepath{fill}%
\end{pgfscope}%
\begin{pgfscope}%
\pgfpathrectangle{\pgfqpoint{0.697024in}{0.857143in}}{\pgfqpoint{2.627103in}{1.813434in}}%
\pgfusepath{clip}%
\pgfsetbuttcap%
\pgfsetmiterjoin%
\definecolor{currentfill}{rgb}{0.754268,0.565033,0.211761}%
\pgfsetfillcolor{currentfill}%
\pgfsetlinewidth{0.000000pt}%
\definecolor{currentstroke}{rgb}{0.000000,0.000000,0.000000}%
\pgfsetstrokecolor{currentstroke}%
\pgfsetstrokeopacity{0.000000}%
\pgfsetdash{}{0pt}%
\pgfpathmoveto{\pgfqpoint{0.950485in}{1.624813in}}%
\pgfpathlineto{\pgfqpoint{0.959422in}{1.624813in}}%
\pgfpathlineto{\pgfqpoint{0.959422in}{1.617428in}}%
\pgfpathlineto{\pgfqpoint{0.950485in}{1.617428in}}%
\pgfpathlineto{\pgfqpoint{0.950485in}{1.624813in}}%
\pgfpathclose%
\pgfusepath{fill}%
\end{pgfscope}%
\begin{pgfscope}%
\pgfpathrectangle{\pgfqpoint{0.697024in}{0.857143in}}{\pgfqpoint{2.627103in}{1.813434in}}%
\pgfusepath{clip}%
\pgfsetbuttcap%
\pgfsetmiterjoin%
\definecolor{currentfill}{rgb}{0.754268,0.565033,0.211761}%
\pgfsetfillcolor{currentfill}%
\pgfsetlinewidth{0.000000pt}%
\definecolor{currentstroke}{rgb}{0.000000,0.000000,0.000000}%
\pgfsetstrokecolor{currentstroke}%
\pgfsetstrokeopacity{0.000000}%
\pgfsetdash{}{0pt}%
\pgfpathmoveto{\pgfqpoint{0.961656in}{1.576558in}}%
\pgfpathlineto{\pgfqpoint{0.970593in}{1.576558in}}%
\pgfpathlineto{\pgfqpoint{0.970593in}{1.547054in}}%
\pgfpathlineto{\pgfqpoint{0.961656in}{1.547054in}}%
\pgfpathlineto{\pgfqpoint{0.961656in}{1.576558in}}%
\pgfpathclose%
\pgfusepath{fill}%
\end{pgfscope}%
\begin{pgfscope}%
\pgfpathrectangle{\pgfqpoint{0.697024in}{0.857143in}}{\pgfqpoint{2.627103in}{1.813434in}}%
\pgfusepath{clip}%
\pgfsetbuttcap%
\pgfsetmiterjoin%
\definecolor{currentfill}{rgb}{0.754268,0.565033,0.211761}%
\pgfsetfillcolor{currentfill}%
\pgfsetlinewidth{0.000000pt}%
\definecolor{currentstroke}{rgb}{0.000000,0.000000,0.000000}%
\pgfsetstrokecolor{currentstroke}%
\pgfsetstrokeopacity{0.000000}%
\pgfsetdash{}{0pt}%
\pgfpathmoveto{\pgfqpoint{0.972827in}{1.659143in}}%
\pgfpathlineto{\pgfqpoint{0.981763in}{1.659143in}}%
\pgfpathlineto{\pgfqpoint{0.981763in}{1.649970in}}%
\pgfpathlineto{\pgfqpoint{0.972827in}{1.649970in}}%
\pgfpathlineto{\pgfqpoint{0.972827in}{1.659143in}}%
\pgfpathclose%
\pgfusepath{fill}%
\end{pgfscope}%
\begin{pgfscope}%
\pgfpathrectangle{\pgfqpoint{0.697024in}{0.857143in}}{\pgfqpoint{2.627103in}{1.813434in}}%
\pgfusepath{clip}%
\pgfsetbuttcap%
\pgfsetmiterjoin%
\definecolor{currentfill}{rgb}{0.754268,0.565033,0.211761}%
\pgfsetfillcolor{currentfill}%
\pgfsetlinewidth{0.000000pt}%
\definecolor{currentstroke}{rgb}{0.000000,0.000000,0.000000}%
\pgfsetstrokecolor{currentstroke}%
\pgfsetstrokeopacity{0.000000}%
\pgfsetdash{}{0pt}%
\pgfpathmoveto{\pgfqpoint{0.983997in}{2.045412in}}%
\pgfpathlineto{\pgfqpoint{0.992934in}{2.045412in}}%
\pgfpathlineto{\pgfqpoint{0.992934in}{2.052750in}}%
\pgfpathlineto{\pgfqpoint{0.983997in}{2.052750in}}%
\pgfpathlineto{\pgfqpoint{0.983997in}{2.045412in}}%
\pgfpathclose%
\pgfusepath{fill}%
\end{pgfscope}%
\begin{pgfscope}%
\pgfpathrectangle{\pgfqpoint{0.697024in}{0.857143in}}{\pgfqpoint{2.627103in}{1.813434in}}%
\pgfusepath{clip}%
\pgfsetbuttcap%
\pgfsetmiterjoin%
\definecolor{currentfill}{rgb}{0.754268,0.565033,0.211761}%
\pgfsetfillcolor{currentfill}%
\pgfsetlinewidth{0.000000pt}%
\definecolor{currentstroke}{rgb}{0.000000,0.000000,0.000000}%
\pgfsetstrokecolor{currentstroke}%
\pgfsetstrokeopacity{0.000000}%
\pgfsetdash{}{0pt}%
\pgfpathmoveto{\pgfqpoint{0.995168in}{1.472361in}}%
\pgfpathlineto{\pgfqpoint{1.004104in}{1.472361in}}%
\pgfpathlineto{\pgfqpoint{1.004104in}{1.422896in}}%
\pgfpathlineto{\pgfqpoint{0.995168in}{1.422896in}}%
\pgfpathlineto{\pgfqpoint{0.995168in}{1.472361in}}%
\pgfpathclose%
\pgfusepath{fill}%
\end{pgfscope}%
\begin{pgfscope}%
\pgfpathrectangle{\pgfqpoint{0.697024in}{0.857143in}}{\pgfqpoint{2.627103in}{1.813434in}}%
\pgfusepath{clip}%
\pgfsetbuttcap%
\pgfsetmiterjoin%
\definecolor{currentfill}{rgb}{0.754268,0.565033,0.211761}%
\pgfsetfillcolor{currentfill}%
\pgfsetlinewidth{0.000000pt}%
\definecolor{currentstroke}{rgb}{0.000000,0.000000,0.000000}%
\pgfsetstrokecolor{currentstroke}%
\pgfsetstrokeopacity{0.000000}%
\pgfsetdash{}{0pt}%
\pgfpathmoveto{\pgfqpoint{1.006338in}{1.982765in}}%
\pgfpathlineto{\pgfqpoint{1.015275in}{1.982765in}}%
\pgfpathlineto{\pgfqpoint{1.015275in}{2.027208in}}%
\pgfpathlineto{\pgfqpoint{1.006338in}{2.027208in}}%
\pgfpathlineto{\pgfqpoint{1.006338in}{1.982765in}}%
\pgfpathclose%
\pgfusepath{fill}%
\end{pgfscope}%
\begin{pgfscope}%
\pgfpathrectangle{\pgfqpoint{0.697024in}{0.857143in}}{\pgfqpoint{2.627103in}{1.813434in}}%
\pgfusepath{clip}%
\pgfsetbuttcap%
\pgfsetmiterjoin%
\definecolor{currentfill}{rgb}{0.754268,0.565033,0.211761}%
\pgfsetfillcolor{currentfill}%
\pgfsetlinewidth{0.000000pt}%
\definecolor{currentstroke}{rgb}{0.000000,0.000000,0.000000}%
\pgfsetstrokecolor{currentstroke}%
\pgfsetstrokeopacity{0.000000}%
\pgfsetdash{}{0pt}%
\pgfpathmoveto{\pgfqpoint{1.017509in}{2.077298in}}%
\pgfpathlineto{\pgfqpoint{1.026446in}{2.077298in}}%
\pgfpathlineto{\pgfqpoint{1.026446in}{2.183309in}}%
\pgfpathlineto{\pgfqpoint{1.017509in}{2.183309in}}%
\pgfpathlineto{\pgfqpoint{1.017509in}{2.077298in}}%
\pgfpathclose%
\pgfusepath{fill}%
\end{pgfscope}%
\begin{pgfscope}%
\pgfpathrectangle{\pgfqpoint{0.697024in}{0.857143in}}{\pgfqpoint{2.627103in}{1.813434in}}%
\pgfusepath{clip}%
\pgfsetbuttcap%
\pgfsetmiterjoin%
\definecolor{currentfill}{rgb}{0.754268,0.565033,0.211761}%
\pgfsetfillcolor{currentfill}%
\pgfsetlinewidth{0.000000pt}%
\definecolor{currentstroke}{rgb}{0.000000,0.000000,0.000000}%
\pgfsetstrokecolor{currentstroke}%
\pgfsetstrokeopacity{0.000000}%
\pgfsetdash{}{0pt}%
\pgfpathmoveto{\pgfqpoint{1.028680in}{2.045452in}}%
\pgfpathlineto{\pgfqpoint{1.037616in}{2.045452in}}%
\pgfpathlineto{\pgfqpoint{1.037616in}{2.110563in}}%
\pgfpathlineto{\pgfqpoint{1.028680in}{2.110563in}}%
\pgfpathlineto{\pgfqpoint{1.028680in}{2.045452in}}%
\pgfpathclose%
\pgfusepath{fill}%
\end{pgfscope}%
\begin{pgfscope}%
\pgfpathrectangle{\pgfqpoint{0.697024in}{0.857143in}}{\pgfqpoint{2.627103in}{1.813434in}}%
\pgfusepath{clip}%
\pgfsetbuttcap%
\pgfsetmiterjoin%
\definecolor{currentfill}{rgb}{0.754268,0.565033,0.211761}%
\pgfsetfillcolor{currentfill}%
\pgfsetlinewidth{0.000000pt}%
\definecolor{currentstroke}{rgb}{0.000000,0.000000,0.000000}%
\pgfsetstrokecolor{currentstroke}%
\pgfsetstrokeopacity{0.000000}%
\pgfsetdash{}{0pt}%
\pgfpathmoveto{\pgfqpoint{1.039850in}{1.355810in}}%
\pgfpathlineto{\pgfqpoint{1.048787in}{1.355810in}}%
\pgfpathlineto{\pgfqpoint{1.048787in}{1.354230in}}%
\pgfpathlineto{\pgfqpoint{1.039850in}{1.354230in}}%
\pgfpathlineto{\pgfqpoint{1.039850in}{1.355810in}}%
\pgfpathclose%
\pgfusepath{fill}%
\end{pgfscope}%
\begin{pgfscope}%
\pgfpathrectangle{\pgfqpoint{0.697024in}{0.857143in}}{\pgfqpoint{2.627103in}{1.813434in}}%
\pgfusepath{clip}%
\pgfsetbuttcap%
\pgfsetmiterjoin%
\definecolor{currentfill}{rgb}{0.754268,0.565033,0.211761}%
\pgfsetfillcolor{currentfill}%
\pgfsetlinewidth{0.000000pt}%
\definecolor{currentstroke}{rgb}{0.000000,0.000000,0.000000}%
\pgfsetstrokecolor{currentstroke}%
\pgfsetstrokeopacity{0.000000}%
\pgfsetdash{}{0pt}%
\pgfpathmoveto{\pgfqpoint{1.051021in}{1.358871in}}%
\pgfpathlineto{\pgfqpoint{1.059957in}{1.358871in}}%
\pgfpathlineto{\pgfqpoint{1.059957in}{1.342546in}}%
\pgfpathlineto{\pgfqpoint{1.051021in}{1.342546in}}%
\pgfpathlineto{\pgfqpoint{1.051021in}{1.358871in}}%
\pgfpathclose%
\pgfusepath{fill}%
\end{pgfscope}%
\begin{pgfscope}%
\pgfpathrectangle{\pgfqpoint{0.697024in}{0.857143in}}{\pgfqpoint{2.627103in}{1.813434in}}%
\pgfusepath{clip}%
\pgfsetbuttcap%
\pgfsetmiterjoin%
\definecolor{currentfill}{rgb}{0.754268,0.565033,0.211761}%
\pgfsetfillcolor{currentfill}%
\pgfsetlinewidth{0.000000pt}%
\definecolor{currentstroke}{rgb}{0.000000,0.000000,0.000000}%
\pgfsetstrokecolor{currentstroke}%
\pgfsetstrokeopacity{0.000000}%
\pgfsetdash{}{0pt}%
\pgfpathmoveto{\pgfqpoint{1.062191in}{2.256855in}}%
\pgfpathlineto{\pgfqpoint{1.071128in}{2.256855in}}%
\pgfpathlineto{\pgfqpoint{1.071128in}{2.372450in}}%
\pgfpathlineto{\pgfqpoint{1.062191in}{2.372450in}}%
\pgfpathlineto{\pgfqpoint{1.062191in}{2.256855in}}%
\pgfpathclose%
\pgfusepath{fill}%
\end{pgfscope}%
\begin{pgfscope}%
\pgfpathrectangle{\pgfqpoint{0.697024in}{0.857143in}}{\pgfqpoint{2.627103in}{1.813434in}}%
\pgfusepath{clip}%
\pgfsetbuttcap%
\pgfsetmiterjoin%
\definecolor{currentfill}{rgb}{0.754268,0.565033,0.211761}%
\pgfsetfillcolor{currentfill}%
\pgfsetlinewidth{0.000000pt}%
\definecolor{currentstroke}{rgb}{0.000000,0.000000,0.000000}%
\pgfsetstrokecolor{currentstroke}%
\pgfsetstrokeopacity{0.000000}%
\pgfsetdash{}{0pt}%
\pgfpathmoveto{\pgfqpoint{1.073362in}{1.266800in}}%
\pgfpathlineto{\pgfqpoint{1.082299in}{1.266800in}}%
\pgfpathlineto{\pgfqpoint{1.082299in}{1.235633in}}%
\pgfpathlineto{\pgfqpoint{1.073362in}{1.235633in}}%
\pgfpathlineto{\pgfqpoint{1.073362in}{1.266800in}}%
\pgfpathclose%
\pgfusepath{fill}%
\end{pgfscope}%
\begin{pgfscope}%
\pgfpathrectangle{\pgfqpoint{0.697024in}{0.857143in}}{\pgfqpoint{2.627103in}{1.813434in}}%
\pgfusepath{clip}%
\pgfsetbuttcap%
\pgfsetmiterjoin%
\definecolor{currentfill}{rgb}{0.754268,0.565033,0.211761}%
\pgfsetfillcolor{currentfill}%
\pgfsetlinewidth{0.000000pt}%
\definecolor{currentstroke}{rgb}{0.000000,0.000000,0.000000}%
\pgfsetstrokecolor{currentstroke}%
\pgfsetstrokeopacity{0.000000}%
\pgfsetdash{}{0pt}%
\pgfpathmoveto{\pgfqpoint{1.084533in}{2.303466in}}%
\pgfpathlineto{\pgfqpoint{1.093469in}{2.303466in}}%
\pgfpathlineto{\pgfqpoint{1.093469in}{2.318242in}}%
\pgfpathlineto{\pgfqpoint{1.084533in}{2.318242in}}%
\pgfpathlineto{\pgfqpoint{1.084533in}{2.303466in}}%
\pgfpathclose%
\pgfusepath{fill}%
\end{pgfscope}%
\begin{pgfscope}%
\pgfpathrectangle{\pgfqpoint{0.697024in}{0.857143in}}{\pgfqpoint{2.627103in}{1.813434in}}%
\pgfusepath{clip}%
\pgfsetbuttcap%
\pgfsetmiterjoin%
\definecolor{currentfill}{rgb}{0.754268,0.565033,0.211761}%
\pgfsetfillcolor{currentfill}%
\pgfsetlinewidth{0.000000pt}%
\definecolor{currentstroke}{rgb}{0.000000,0.000000,0.000000}%
\pgfsetstrokecolor{currentstroke}%
\pgfsetstrokeopacity{0.000000}%
\pgfsetdash{}{0pt}%
\pgfpathmoveto{\pgfqpoint{1.095703in}{2.309442in}}%
\pgfpathlineto{\pgfqpoint{1.104640in}{2.309442in}}%
\pgfpathlineto{\pgfqpoint{1.104640in}{2.354518in}}%
\pgfpathlineto{\pgfqpoint{1.095703in}{2.354518in}}%
\pgfpathlineto{\pgfqpoint{1.095703in}{2.309442in}}%
\pgfpathclose%
\pgfusepath{fill}%
\end{pgfscope}%
\begin{pgfscope}%
\pgfpathrectangle{\pgfqpoint{0.697024in}{0.857143in}}{\pgfqpoint{2.627103in}{1.813434in}}%
\pgfusepath{clip}%
\pgfsetbuttcap%
\pgfsetmiterjoin%
\definecolor{currentfill}{rgb}{0.754268,0.565033,0.211761}%
\pgfsetfillcolor{currentfill}%
\pgfsetlinewidth{0.000000pt}%
\definecolor{currentstroke}{rgb}{0.000000,0.000000,0.000000}%
\pgfsetstrokecolor{currentstroke}%
\pgfsetstrokeopacity{0.000000}%
\pgfsetdash{}{0pt}%
\pgfpathmoveto{\pgfqpoint{1.106874in}{2.371664in}}%
\pgfpathlineto{\pgfqpoint{1.115810in}{2.371664in}}%
\pgfpathlineto{\pgfqpoint{1.115810in}{2.376869in}}%
\pgfpathlineto{\pgfqpoint{1.106874in}{2.376869in}}%
\pgfpathlineto{\pgfqpoint{1.106874in}{2.371664in}}%
\pgfpathclose%
\pgfusepath{fill}%
\end{pgfscope}%
\begin{pgfscope}%
\pgfpathrectangle{\pgfqpoint{0.697024in}{0.857143in}}{\pgfqpoint{2.627103in}{1.813434in}}%
\pgfusepath{clip}%
\pgfsetbuttcap%
\pgfsetmiterjoin%
\definecolor{currentfill}{rgb}{0.754268,0.565033,0.211761}%
\pgfsetfillcolor{currentfill}%
\pgfsetlinewidth{0.000000pt}%
\definecolor{currentstroke}{rgb}{0.000000,0.000000,0.000000}%
\pgfsetstrokecolor{currentstroke}%
\pgfsetstrokeopacity{0.000000}%
\pgfsetdash{}{0pt}%
\pgfpathmoveto{\pgfqpoint{1.118045in}{2.301290in}}%
\pgfpathlineto{\pgfqpoint{1.126981in}{2.301290in}}%
\pgfpathlineto{\pgfqpoint{1.126981in}{2.333471in}}%
\pgfpathlineto{\pgfqpoint{1.118045in}{2.333471in}}%
\pgfpathlineto{\pgfqpoint{1.118045in}{2.301290in}}%
\pgfpathclose%
\pgfusepath{fill}%
\end{pgfscope}%
\begin{pgfscope}%
\pgfpathrectangle{\pgfqpoint{0.697024in}{0.857143in}}{\pgfqpoint{2.627103in}{1.813434in}}%
\pgfusepath{clip}%
\pgfsetbuttcap%
\pgfsetmiterjoin%
\definecolor{currentfill}{rgb}{0.754268,0.565033,0.211761}%
\pgfsetfillcolor{currentfill}%
\pgfsetlinewidth{0.000000pt}%
\definecolor{currentstroke}{rgb}{0.000000,0.000000,0.000000}%
\pgfsetstrokecolor{currentstroke}%
\pgfsetstrokeopacity{0.000000}%
\pgfsetdash{}{0pt}%
\pgfpathmoveto{\pgfqpoint{1.129215in}{2.190834in}}%
\pgfpathlineto{\pgfqpoint{1.138152in}{2.190834in}}%
\pgfpathlineto{\pgfqpoint{1.138152in}{2.201212in}}%
\pgfpathlineto{\pgfqpoint{1.129215in}{2.201212in}}%
\pgfpathlineto{\pgfqpoint{1.129215in}{2.190834in}}%
\pgfpathclose%
\pgfusepath{fill}%
\end{pgfscope}%
\begin{pgfscope}%
\pgfpathrectangle{\pgfqpoint{0.697024in}{0.857143in}}{\pgfqpoint{2.627103in}{1.813434in}}%
\pgfusepath{clip}%
\pgfsetbuttcap%
\pgfsetmiterjoin%
\definecolor{currentfill}{rgb}{0.754268,0.565033,0.211761}%
\pgfsetfillcolor{currentfill}%
\pgfsetlinewidth{0.000000pt}%
\definecolor{currentstroke}{rgb}{0.000000,0.000000,0.000000}%
\pgfsetstrokecolor{currentstroke}%
\pgfsetstrokeopacity{0.000000}%
\pgfsetdash{}{0pt}%
\pgfpathmoveto{\pgfqpoint{1.140386in}{2.147862in}}%
\pgfpathlineto{\pgfqpoint{1.149322in}{2.147862in}}%
\pgfpathlineto{\pgfqpoint{1.149322in}{2.174375in}}%
\pgfpathlineto{\pgfqpoint{1.140386in}{2.174375in}}%
\pgfpathlineto{\pgfqpoint{1.140386in}{2.147862in}}%
\pgfpathclose%
\pgfusepath{fill}%
\end{pgfscope}%
\begin{pgfscope}%
\pgfpathrectangle{\pgfqpoint{0.697024in}{0.857143in}}{\pgfqpoint{2.627103in}{1.813434in}}%
\pgfusepath{clip}%
\pgfsetbuttcap%
\pgfsetmiterjoin%
\definecolor{currentfill}{rgb}{0.754268,0.565033,0.211761}%
\pgfsetfillcolor{currentfill}%
\pgfsetlinewidth{0.000000pt}%
\definecolor{currentstroke}{rgb}{0.000000,0.000000,0.000000}%
\pgfsetstrokecolor{currentstroke}%
\pgfsetstrokeopacity{0.000000}%
\pgfsetdash{}{0pt}%
\pgfpathmoveto{\pgfqpoint{1.151556in}{2.117154in}}%
\pgfpathlineto{\pgfqpoint{1.160493in}{2.117154in}}%
\pgfpathlineto{\pgfqpoint{1.160493in}{2.136401in}}%
\pgfpathlineto{\pgfqpoint{1.151556in}{2.136401in}}%
\pgfpathlineto{\pgfqpoint{1.151556in}{2.117154in}}%
\pgfpathclose%
\pgfusepath{fill}%
\end{pgfscope}%
\begin{pgfscope}%
\pgfpathrectangle{\pgfqpoint{0.697024in}{0.857143in}}{\pgfqpoint{2.627103in}{1.813434in}}%
\pgfusepath{clip}%
\pgfsetbuttcap%
\pgfsetmiterjoin%
\definecolor{currentfill}{rgb}{0.754268,0.565033,0.211761}%
\pgfsetfillcolor{currentfill}%
\pgfsetlinewidth{0.000000pt}%
\definecolor{currentstroke}{rgb}{0.000000,0.000000,0.000000}%
\pgfsetstrokecolor{currentstroke}%
\pgfsetstrokeopacity{0.000000}%
\pgfsetdash{}{0pt}%
\pgfpathmoveto{\pgfqpoint{1.162727in}{1.944442in}}%
\pgfpathlineto{\pgfqpoint{1.171663in}{1.944442in}}%
\pgfpathlineto{\pgfqpoint{1.171663in}{2.005955in}}%
\pgfpathlineto{\pgfqpoint{1.162727in}{2.005955in}}%
\pgfpathlineto{\pgfqpoint{1.162727in}{1.944442in}}%
\pgfpathclose%
\pgfusepath{fill}%
\end{pgfscope}%
\begin{pgfscope}%
\pgfpathrectangle{\pgfqpoint{0.697024in}{0.857143in}}{\pgfqpoint{2.627103in}{1.813434in}}%
\pgfusepath{clip}%
\pgfsetbuttcap%
\pgfsetmiterjoin%
\definecolor{currentfill}{rgb}{0.754268,0.565033,0.211761}%
\pgfsetfillcolor{currentfill}%
\pgfsetlinewidth{0.000000pt}%
\definecolor{currentstroke}{rgb}{0.000000,0.000000,0.000000}%
\pgfsetstrokecolor{currentstroke}%
\pgfsetstrokeopacity{0.000000}%
\pgfsetdash{}{0pt}%
\pgfpathmoveto{\pgfqpoint{1.173898in}{1.906503in}}%
\pgfpathlineto{\pgfqpoint{1.182834in}{1.906503in}}%
\pgfpathlineto{\pgfqpoint{1.182834in}{1.981808in}}%
\pgfpathlineto{\pgfqpoint{1.173898in}{1.981808in}}%
\pgfpathlineto{\pgfqpoint{1.173898in}{1.906503in}}%
\pgfpathclose%
\pgfusepath{fill}%
\end{pgfscope}%
\begin{pgfscope}%
\pgfpathrectangle{\pgfqpoint{0.697024in}{0.857143in}}{\pgfqpoint{2.627103in}{1.813434in}}%
\pgfusepath{clip}%
\pgfsetbuttcap%
\pgfsetmiterjoin%
\definecolor{currentfill}{rgb}{0.754268,0.565033,0.211761}%
\pgfsetfillcolor{currentfill}%
\pgfsetlinewidth{0.000000pt}%
\definecolor{currentstroke}{rgb}{0.000000,0.000000,0.000000}%
\pgfsetstrokecolor{currentstroke}%
\pgfsetstrokeopacity{0.000000}%
\pgfsetdash{}{0pt}%
\pgfpathmoveto{\pgfqpoint{1.185068in}{1.883462in}}%
\pgfpathlineto{\pgfqpoint{1.194005in}{1.883462in}}%
\pgfpathlineto{\pgfqpoint{1.194005in}{2.016490in}}%
\pgfpathlineto{\pgfqpoint{1.185068in}{2.016490in}}%
\pgfpathlineto{\pgfqpoint{1.185068in}{1.883462in}}%
\pgfpathclose%
\pgfusepath{fill}%
\end{pgfscope}%
\begin{pgfscope}%
\pgfpathrectangle{\pgfqpoint{0.697024in}{0.857143in}}{\pgfqpoint{2.627103in}{1.813434in}}%
\pgfusepath{clip}%
\pgfsetbuttcap%
\pgfsetmiterjoin%
\definecolor{currentfill}{rgb}{0.754268,0.565033,0.211761}%
\pgfsetfillcolor{currentfill}%
\pgfsetlinewidth{0.000000pt}%
\definecolor{currentstroke}{rgb}{0.000000,0.000000,0.000000}%
\pgfsetstrokecolor{currentstroke}%
\pgfsetstrokeopacity{0.000000}%
\pgfsetdash{}{0pt}%
\pgfpathmoveto{\pgfqpoint{1.196239in}{1.927207in}}%
\pgfpathlineto{\pgfqpoint{1.205175in}{1.927207in}}%
\pgfpathlineto{\pgfqpoint{1.205175in}{2.026533in}}%
\pgfpathlineto{\pgfqpoint{1.196239in}{2.026533in}}%
\pgfpathlineto{\pgfqpoint{1.196239in}{1.927207in}}%
\pgfpathclose%
\pgfusepath{fill}%
\end{pgfscope}%
\begin{pgfscope}%
\pgfpathrectangle{\pgfqpoint{0.697024in}{0.857143in}}{\pgfqpoint{2.627103in}{1.813434in}}%
\pgfusepath{clip}%
\pgfsetbuttcap%
\pgfsetmiterjoin%
\definecolor{currentfill}{rgb}{0.754268,0.565033,0.211761}%
\pgfsetfillcolor{currentfill}%
\pgfsetlinewidth{0.000000pt}%
\definecolor{currentstroke}{rgb}{0.000000,0.000000,0.000000}%
\pgfsetstrokecolor{currentstroke}%
\pgfsetstrokeopacity{0.000000}%
\pgfsetdash{}{0pt}%
\pgfpathmoveto{\pgfqpoint{1.207409in}{1.960291in}}%
\pgfpathlineto{\pgfqpoint{1.216346in}{1.960291in}}%
\pgfpathlineto{\pgfqpoint{1.216346in}{1.990461in}}%
\pgfpathlineto{\pgfqpoint{1.207409in}{1.990461in}}%
\pgfpathlineto{\pgfqpoint{1.207409in}{1.960291in}}%
\pgfpathclose%
\pgfusepath{fill}%
\end{pgfscope}%
\begin{pgfscope}%
\pgfpathrectangle{\pgfqpoint{0.697024in}{0.857143in}}{\pgfqpoint{2.627103in}{1.813434in}}%
\pgfusepath{clip}%
\pgfsetbuttcap%
\pgfsetmiterjoin%
\definecolor{currentfill}{rgb}{0.754268,0.565033,0.211761}%
\pgfsetfillcolor{currentfill}%
\pgfsetlinewidth{0.000000pt}%
\definecolor{currentstroke}{rgb}{0.000000,0.000000,0.000000}%
\pgfsetstrokecolor{currentstroke}%
\pgfsetstrokeopacity{0.000000}%
\pgfsetdash{}{0pt}%
\pgfpathmoveto{\pgfqpoint{1.218580in}{1.963193in}}%
\pgfpathlineto{\pgfqpoint{1.227516in}{1.963193in}}%
\pgfpathlineto{\pgfqpoint{1.227516in}{2.022538in}}%
\pgfpathlineto{\pgfqpoint{1.218580in}{2.022538in}}%
\pgfpathlineto{\pgfqpoint{1.218580in}{1.963193in}}%
\pgfpathclose%
\pgfusepath{fill}%
\end{pgfscope}%
\begin{pgfscope}%
\pgfpathrectangle{\pgfqpoint{0.697024in}{0.857143in}}{\pgfqpoint{2.627103in}{1.813434in}}%
\pgfusepath{clip}%
\pgfsetbuttcap%
\pgfsetmiterjoin%
\definecolor{currentfill}{rgb}{0.754268,0.565033,0.211761}%
\pgfsetfillcolor{currentfill}%
\pgfsetlinewidth{0.000000pt}%
\definecolor{currentstroke}{rgb}{0.000000,0.000000,0.000000}%
\pgfsetstrokecolor{currentstroke}%
\pgfsetstrokeopacity{0.000000}%
\pgfsetdash{}{0pt}%
\pgfpathmoveto{\pgfqpoint{1.229751in}{1.940041in}}%
\pgfpathlineto{\pgfqpoint{1.238687in}{1.940041in}}%
\pgfpathlineto{\pgfqpoint{1.238687in}{1.992655in}}%
\pgfpathlineto{\pgfqpoint{1.229751in}{1.992655in}}%
\pgfpathlineto{\pgfqpoint{1.229751in}{1.940041in}}%
\pgfpathclose%
\pgfusepath{fill}%
\end{pgfscope}%
\begin{pgfscope}%
\pgfpathrectangle{\pgfqpoint{0.697024in}{0.857143in}}{\pgfqpoint{2.627103in}{1.813434in}}%
\pgfusepath{clip}%
\pgfsetbuttcap%
\pgfsetmiterjoin%
\definecolor{currentfill}{rgb}{0.754268,0.565033,0.211761}%
\pgfsetfillcolor{currentfill}%
\pgfsetlinewidth{0.000000pt}%
\definecolor{currentstroke}{rgb}{0.000000,0.000000,0.000000}%
\pgfsetstrokecolor{currentstroke}%
\pgfsetstrokeopacity{0.000000}%
\pgfsetdash{}{0pt}%
\pgfpathmoveto{\pgfqpoint{1.240921in}{1.573725in}}%
\pgfpathlineto{\pgfqpoint{1.249858in}{1.573725in}}%
\pgfpathlineto{\pgfqpoint{1.249858in}{1.566963in}}%
\pgfpathlineto{\pgfqpoint{1.240921in}{1.566963in}}%
\pgfpathlineto{\pgfqpoint{1.240921in}{1.573725in}}%
\pgfpathclose%
\pgfusepath{fill}%
\end{pgfscope}%
\begin{pgfscope}%
\pgfpathrectangle{\pgfqpoint{0.697024in}{0.857143in}}{\pgfqpoint{2.627103in}{1.813434in}}%
\pgfusepath{clip}%
\pgfsetbuttcap%
\pgfsetmiterjoin%
\definecolor{currentfill}{rgb}{0.754268,0.565033,0.211761}%
\pgfsetfillcolor{currentfill}%
\pgfsetlinewidth{0.000000pt}%
\definecolor{currentstroke}{rgb}{0.000000,0.000000,0.000000}%
\pgfsetstrokecolor{currentstroke}%
\pgfsetstrokeopacity{0.000000}%
\pgfsetdash{}{0pt}%
\pgfpathmoveto{\pgfqpoint{1.252092in}{1.539634in}}%
\pgfpathlineto{\pgfqpoint{1.261028in}{1.539634in}}%
\pgfpathlineto{\pgfqpoint{1.261028in}{1.535239in}}%
\pgfpathlineto{\pgfqpoint{1.252092in}{1.535239in}}%
\pgfpathlineto{\pgfqpoint{1.252092in}{1.539634in}}%
\pgfpathclose%
\pgfusepath{fill}%
\end{pgfscope}%
\begin{pgfscope}%
\pgfpathrectangle{\pgfqpoint{0.697024in}{0.857143in}}{\pgfqpoint{2.627103in}{1.813434in}}%
\pgfusepath{clip}%
\pgfsetbuttcap%
\pgfsetmiterjoin%
\definecolor{currentfill}{rgb}{0.754268,0.565033,0.211761}%
\pgfsetfillcolor{currentfill}%
\pgfsetlinewidth{0.000000pt}%
\definecolor{currentstroke}{rgb}{0.000000,0.000000,0.000000}%
\pgfsetstrokecolor{currentstroke}%
\pgfsetstrokeopacity{0.000000}%
\pgfsetdash{}{0pt}%
\pgfpathmoveto{\pgfqpoint{1.263262in}{1.972016in}}%
\pgfpathlineto{\pgfqpoint{1.272199in}{1.972016in}}%
\pgfpathlineto{\pgfqpoint{1.272199in}{1.998006in}}%
\pgfpathlineto{\pgfqpoint{1.263262in}{1.998006in}}%
\pgfpathlineto{\pgfqpoint{1.263262in}{1.972016in}}%
\pgfpathclose%
\pgfusepath{fill}%
\end{pgfscope}%
\begin{pgfscope}%
\pgfpathrectangle{\pgfqpoint{0.697024in}{0.857143in}}{\pgfqpoint{2.627103in}{1.813434in}}%
\pgfusepath{clip}%
\pgfsetbuttcap%
\pgfsetmiterjoin%
\definecolor{currentfill}{rgb}{0.754268,0.565033,0.211761}%
\pgfsetfillcolor{currentfill}%
\pgfsetlinewidth{0.000000pt}%
\definecolor{currentstroke}{rgb}{0.000000,0.000000,0.000000}%
\pgfsetstrokecolor{currentstroke}%
\pgfsetstrokeopacity{0.000000}%
\pgfsetdash{}{0pt}%
\pgfpathmoveto{\pgfqpoint{1.274433in}{1.936647in}}%
\pgfpathlineto{\pgfqpoint{1.283369in}{1.936647in}}%
\pgfpathlineto{\pgfqpoint{1.283369in}{2.031703in}}%
\pgfpathlineto{\pgfqpoint{1.274433in}{2.031703in}}%
\pgfpathlineto{\pgfqpoint{1.274433in}{1.936647in}}%
\pgfpathclose%
\pgfusepath{fill}%
\end{pgfscope}%
\begin{pgfscope}%
\pgfpathrectangle{\pgfqpoint{0.697024in}{0.857143in}}{\pgfqpoint{2.627103in}{1.813434in}}%
\pgfusepath{clip}%
\pgfsetbuttcap%
\pgfsetmiterjoin%
\definecolor{currentfill}{rgb}{0.754268,0.565033,0.211761}%
\pgfsetfillcolor{currentfill}%
\pgfsetlinewidth{0.000000pt}%
\definecolor{currentstroke}{rgb}{0.000000,0.000000,0.000000}%
\pgfsetstrokecolor{currentstroke}%
\pgfsetstrokeopacity{0.000000}%
\pgfsetdash{}{0pt}%
\pgfpathmoveto{\pgfqpoint{1.285604in}{1.942557in}}%
\pgfpathlineto{\pgfqpoint{1.294540in}{1.942557in}}%
\pgfpathlineto{\pgfqpoint{1.294540in}{2.012008in}}%
\pgfpathlineto{\pgfqpoint{1.285604in}{2.012008in}}%
\pgfpathlineto{\pgfqpoint{1.285604in}{1.942557in}}%
\pgfpathclose%
\pgfusepath{fill}%
\end{pgfscope}%
\begin{pgfscope}%
\pgfpathrectangle{\pgfqpoint{0.697024in}{0.857143in}}{\pgfqpoint{2.627103in}{1.813434in}}%
\pgfusepath{clip}%
\pgfsetbuttcap%
\pgfsetmiterjoin%
\definecolor{currentfill}{rgb}{0.754268,0.565033,0.211761}%
\pgfsetfillcolor{currentfill}%
\pgfsetlinewidth{0.000000pt}%
\definecolor{currentstroke}{rgb}{0.000000,0.000000,0.000000}%
\pgfsetstrokecolor{currentstroke}%
\pgfsetstrokeopacity{0.000000}%
\pgfsetdash{}{0pt}%
\pgfpathmoveto{\pgfqpoint{1.296774in}{2.022580in}}%
\pgfpathlineto{\pgfqpoint{1.305711in}{2.022580in}}%
\pgfpathlineto{\pgfqpoint{1.305711in}{2.060957in}}%
\pgfpathlineto{\pgfqpoint{1.296774in}{2.060957in}}%
\pgfpathlineto{\pgfqpoint{1.296774in}{2.022580in}}%
\pgfpathclose%
\pgfusepath{fill}%
\end{pgfscope}%
\begin{pgfscope}%
\pgfpathrectangle{\pgfqpoint{0.697024in}{0.857143in}}{\pgfqpoint{2.627103in}{1.813434in}}%
\pgfusepath{clip}%
\pgfsetbuttcap%
\pgfsetmiterjoin%
\definecolor{currentfill}{rgb}{0.754268,0.565033,0.211761}%
\pgfsetfillcolor{currentfill}%
\pgfsetlinewidth{0.000000pt}%
\definecolor{currentstroke}{rgb}{0.000000,0.000000,0.000000}%
\pgfsetstrokecolor{currentstroke}%
\pgfsetstrokeopacity{0.000000}%
\pgfsetdash{}{0pt}%
\pgfpathmoveto{\pgfqpoint{1.307945in}{2.086937in}}%
\pgfpathlineto{\pgfqpoint{1.316881in}{2.086937in}}%
\pgfpathlineto{\pgfqpoint{1.316881in}{2.087047in}}%
\pgfpathlineto{\pgfqpoint{1.307945in}{2.087047in}}%
\pgfpathlineto{\pgfqpoint{1.307945in}{2.086937in}}%
\pgfpathclose%
\pgfusepath{fill}%
\end{pgfscope}%
\begin{pgfscope}%
\pgfpathrectangle{\pgfqpoint{0.697024in}{0.857143in}}{\pgfqpoint{2.627103in}{1.813434in}}%
\pgfusepath{clip}%
\pgfsetbuttcap%
\pgfsetmiterjoin%
\definecolor{currentfill}{rgb}{0.754268,0.565033,0.211761}%
\pgfsetfillcolor{currentfill}%
\pgfsetlinewidth{0.000000pt}%
\definecolor{currentstroke}{rgb}{0.000000,0.000000,0.000000}%
\pgfsetstrokecolor{currentstroke}%
\pgfsetstrokeopacity{0.000000}%
\pgfsetdash{}{0pt}%
\pgfpathmoveto{\pgfqpoint{1.319115in}{1.884029in}}%
\pgfpathlineto{\pgfqpoint{1.328052in}{1.884029in}}%
\pgfpathlineto{\pgfqpoint{1.328052in}{1.989111in}}%
\pgfpathlineto{\pgfqpoint{1.319115in}{1.989111in}}%
\pgfpathlineto{\pgfqpoint{1.319115in}{1.884029in}}%
\pgfpathclose%
\pgfusepath{fill}%
\end{pgfscope}%
\begin{pgfscope}%
\pgfpathrectangle{\pgfqpoint{0.697024in}{0.857143in}}{\pgfqpoint{2.627103in}{1.813434in}}%
\pgfusepath{clip}%
\pgfsetbuttcap%
\pgfsetmiterjoin%
\definecolor{currentfill}{rgb}{0.754268,0.565033,0.211761}%
\pgfsetfillcolor{currentfill}%
\pgfsetlinewidth{0.000000pt}%
\definecolor{currentstroke}{rgb}{0.000000,0.000000,0.000000}%
\pgfsetstrokecolor{currentstroke}%
\pgfsetstrokeopacity{0.000000}%
\pgfsetdash{}{0pt}%
\pgfpathmoveto{\pgfqpoint{1.330286in}{1.584227in}}%
\pgfpathlineto{\pgfqpoint{1.339222in}{1.584227in}}%
\pgfpathlineto{\pgfqpoint{1.339222in}{1.581666in}}%
\pgfpathlineto{\pgfqpoint{1.330286in}{1.581666in}}%
\pgfpathlineto{\pgfqpoint{1.330286in}{1.584227in}}%
\pgfpathclose%
\pgfusepath{fill}%
\end{pgfscope}%
\begin{pgfscope}%
\pgfpathrectangle{\pgfqpoint{0.697024in}{0.857143in}}{\pgfqpoint{2.627103in}{1.813434in}}%
\pgfusepath{clip}%
\pgfsetbuttcap%
\pgfsetmiterjoin%
\definecolor{currentfill}{rgb}{0.754268,0.565033,0.211761}%
\pgfsetfillcolor{currentfill}%
\pgfsetlinewidth{0.000000pt}%
\definecolor{currentstroke}{rgb}{0.000000,0.000000,0.000000}%
\pgfsetstrokecolor{currentstroke}%
\pgfsetstrokeopacity{0.000000}%
\pgfsetdash{}{0pt}%
\pgfpathmoveto{\pgfqpoint{1.341457in}{1.996859in}}%
\pgfpathlineto{\pgfqpoint{1.350393in}{1.996859in}}%
\pgfpathlineto{\pgfqpoint{1.350393in}{2.057559in}}%
\pgfpathlineto{\pgfqpoint{1.341457in}{2.057559in}}%
\pgfpathlineto{\pgfqpoint{1.341457in}{1.996859in}}%
\pgfpathclose%
\pgfusepath{fill}%
\end{pgfscope}%
\begin{pgfscope}%
\pgfpathrectangle{\pgfqpoint{0.697024in}{0.857143in}}{\pgfqpoint{2.627103in}{1.813434in}}%
\pgfusepath{clip}%
\pgfsetbuttcap%
\pgfsetmiterjoin%
\definecolor{currentfill}{rgb}{0.754268,0.565033,0.211761}%
\pgfsetfillcolor{currentfill}%
\pgfsetlinewidth{0.000000pt}%
\definecolor{currentstroke}{rgb}{0.000000,0.000000,0.000000}%
\pgfsetstrokecolor{currentstroke}%
\pgfsetstrokeopacity{0.000000}%
\pgfsetdash{}{0pt}%
\pgfpathmoveto{\pgfqpoint{1.352627in}{2.026321in}}%
\pgfpathlineto{\pgfqpoint{1.361564in}{2.026321in}}%
\pgfpathlineto{\pgfqpoint{1.361564in}{2.034936in}}%
\pgfpathlineto{\pgfqpoint{1.352627in}{2.034936in}}%
\pgfpathlineto{\pgfqpoint{1.352627in}{2.026321in}}%
\pgfpathclose%
\pgfusepath{fill}%
\end{pgfscope}%
\begin{pgfscope}%
\pgfpathrectangle{\pgfqpoint{0.697024in}{0.857143in}}{\pgfqpoint{2.627103in}{1.813434in}}%
\pgfusepath{clip}%
\pgfsetbuttcap%
\pgfsetmiterjoin%
\definecolor{currentfill}{rgb}{0.754268,0.565033,0.211761}%
\pgfsetfillcolor{currentfill}%
\pgfsetlinewidth{0.000000pt}%
\definecolor{currentstroke}{rgb}{0.000000,0.000000,0.000000}%
\pgfsetstrokecolor{currentstroke}%
\pgfsetstrokeopacity{0.000000}%
\pgfsetdash{}{0pt}%
\pgfpathmoveto{\pgfqpoint{1.363798in}{1.966854in}}%
\pgfpathlineto{\pgfqpoint{1.372734in}{1.966854in}}%
\pgfpathlineto{\pgfqpoint{1.372734in}{1.999482in}}%
\pgfpathlineto{\pgfqpoint{1.363798in}{1.999482in}}%
\pgfpathlineto{\pgfqpoint{1.363798in}{1.966854in}}%
\pgfpathclose%
\pgfusepath{fill}%
\end{pgfscope}%
\begin{pgfscope}%
\pgfpathrectangle{\pgfqpoint{0.697024in}{0.857143in}}{\pgfqpoint{2.627103in}{1.813434in}}%
\pgfusepath{clip}%
\pgfsetbuttcap%
\pgfsetmiterjoin%
\definecolor{currentfill}{rgb}{0.754268,0.565033,0.211761}%
\pgfsetfillcolor{currentfill}%
\pgfsetlinewidth{0.000000pt}%
\definecolor{currentstroke}{rgb}{0.000000,0.000000,0.000000}%
\pgfsetstrokecolor{currentstroke}%
\pgfsetstrokeopacity{0.000000}%
\pgfsetdash{}{0pt}%
\pgfpathmoveto{\pgfqpoint{1.374968in}{1.790545in}}%
\pgfpathlineto{\pgfqpoint{1.383905in}{1.790545in}}%
\pgfpathlineto{\pgfqpoint{1.383905in}{1.769604in}}%
\pgfpathlineto{\pgfqpoint{1.374968in}{1.769604in}}%
\pgfpathlineto{\pgfqpoint{1.374968in}{1.790545in}}%
\pgfpathclose%
\pgfusepath{fill}%
\end{pgfscope}%
\begin{pgfscope}%
\pgfpathrectangle{\pgfqpoint{0.697024in}{0.857143in}}{\pgfqpoint{2.627103in}{1.813434in}}%
\pgfusepath{clip}%
\pgfsetbuttcap%
\pgfsetmiterjoin%
\definecolor{currentfill}{rgb}{0.754268,0.565033,0.211761}%
\pgfsetfillcolor{currentfill}%
\pgfsetlinewidth{0.000000pt}%
\definecolor{currentstroke}{rgb}{0.000000,0.000000,0.000000}%
\pgfsetstrokecolor{currentstroke}%
\pgfsetstrokeopacity{0.000000}%
\pgfsetdash{}{0pt}%
\pgfpathmoveto{\pgfqpoint{1.386139in}{1.912253in}}%
\pgfpathlineto{\pgfqpoint{1.395076in}{1.912253in}}%
\pgfpathlineto{\pgfqpoint{1.395076in}{1.980052in}}%
\pgfpathlineto{\pgfqpoint{1.386139in}{1.980052in}}%
\pgfpathlineto{\pgfqpoint{1.386139in}{1.912253in}}%
\pgfpathclose%
\pgfusepath{fill}%
\end{pgfscope}%
\begin{pgfscope}%
\pgfpathrectangle{\pgfqpoint{0.697024in}{0.857143in}}{\pgfqpoint{2.627103in}{1.813434in}}%
\pgfusepath{clip}%
\pgfsetbuttcap%
\pgfsetmiterjoin%
\definecolor{currentfill}{rgb}{0.754268,0.565033,0.211761}%
\pgfsetfillcolor{currentfill}%
\pgfsetlinewidth{0.000000pt}%
\definecolor{currentstroke}{rgb}{0.000000,0.000000,0.000000}%
\pgfsetstrokecolor{currentstroke}%
\pgfsetstrokeopacity{0.000000}%
\pgfsetdash{}{0pt}%
\pgfpathmoveto{\pgfqpoint{1.397310in}{1.912227in}}%
\pgfpathlineto{\pgfqpoint{1.406246in}{1.912227in}}%
\pgfpathlineto{\pgfqpoint{1.406246in}{1.928196in}}%
\pgfpathlineto{\pgfqpoint{1.397310in}{1.928196in}}%
\pgfpathlineto{\pgfqpoint{1.397310in}{1.912227in}}%
\pgfpathclose%
\pgfusepath{fill}%
\end{pgfscope}%
\begin{pgfscope}%
\pgfpathrectangle{\pgfqpoint{0.697024in}{0.857143in}}{\pgfqpoint{2.627103in}{1.813434in}}%
\pgfusepath{clip}%
\pgfsetbuttcap%
\pgfsetmiterjoin%
\definecolor{currentfill}{rgb}{0.754268,0.565033,0.211761}%
\pgfsetfillcolor{currentfill}%
\pgfsetlinewidth{0.000000pt}%
\definecolor{currentstroke}{rgb}{0.000000,0.000000,0.000000}%
\pgfsetstrokecolor{currentstroke}%
\pgfsetstrokeopacity{0.000000}%
\pgfsetdash{}{0pt}%
\pgfpathmoveto{\pgfqpoint{1.408480in}{1.814040in}}%
\pgfpathlineto{\pgfqpoint{1.417417in}{1.814040in}}%
\pgfpathlineto{\pgfqpoint{1.417417in}{1.717782in}}%
\pgfpathlineto{\pgfqpoint{1.408480in}{1.717782in}}%
\pgfpathlineto{\pgfqpoint{1.408480in}{1.814040in}}%
\pgfpathclose%
\pgfusepath{fill}%
\end{pgfscope}%
\begin{pgfscope}%
\pgfpathrectangle{\pgfqpoint{0.697024in}{0.857143in}}{\pgfqpoint{2.627103in}{1.813434in}}%
\pgfusepath{clip}%
\pgfsetbuttcap%
\pgfsetmiterjoin%
\definecolor{currentfill}{rgb}{0.754268,0.565033,0.211761}%
\pgfsetfillcolor{currentfill}%
\pgfsetlinewidth{0.000000pt}%
\definecolor{currentstroke}{rgb}{0.000000,0.000000,0.000000}%
\pgfsetstrokecolor{currentstroke}%
\pgfsetstrokeopacity{0.000000}%
\pgfsetdash{}{0pt}%
\pgfpathmoveto{\pgfqpoint{1.419651in}{1.821295in}}%
\pgfpathlineto{\pgfqpoint{1.428587in}{1.821295in}}%
\pgfpathlineto{\pgfqpoint{1.428587in}{1.772776in}}%
\pgfpathlineto{\pgfqpoint{1.419651in}{1.772776in}}%
\pgfpathlineto{\pgfqpoint{1.419651in}{1.821295in}}%
\pgfpathclose%
\pgfusepath{fill}%
\end{pgfscope}%
\begin{pgfscope}%
\pgfpathrectangle{\pgfqpoint{0.697024in}{0.857143in}}{\pgfqpoint{2.627103in}{1.813434in}}%
\pgfusepath{clip}%
\pgfsetbuttcap%
\pgfsetmiterjoin%
\definecolor{currentfill}{rgb}{0.754268,0.565033,0.211761}%
\pgfsetfillcolor{currentfill}%
\pgfsetlinewidth{0.000000pt}%
\definecolor{currentstroke}{rgb}{0.000000,0.000000,0.000000}%
\pgfsetstrokecolor{currentstroke}%
\pgfsetstrokeopacity{0.000000}%
\pgfsetdash{}{0pt}%
\pgfpathmoveto{\pgfqpoint{1.430821in}{1.801756in}}%
\pgfpathlineto{\pgfqpoint{1.439758in}{1.801756in}}%
\pgfpathlineto{\pgfqpoint{1.439758in}{1.710228in}}%
\pgfpathlineto{\pgfqpoint{1.430821in}{1.710228in}}%
\pgfpathlineto{\pgfqpoint{1.430821in}{1.801756in}}%
\pgfpathclose%
\pgfusepath{fill}%
\end{pgfscope}%
\begin{pgfscope}%
\pgfpathrectangle{\pgfqpoint{0.697024in}{0.857143in}}{\pgfqpoint{2.627103in}{1.813434in}}%
\pgfusepath{clip}%
\pgfsetbuttcap%
\pgfsetmiterjoin%
\definecolor{currentfill}{rgb}{0.754268,0.565033,0.211761}%
\pgfsetfillcolor{currentfill}%
\pgfsetlinewidth{0.000000pt}%
\definecolor{currentstroke}{rgb}{0.000000,0.000000,0.000000}%
\pgfsetstrokecolor{currentstroke}%
\pgfsetstrokeopacity{0.000000}%
\pgfsetdash{}{0pt}%
\pgfpathmoveto{\pgfqpoint{1.441992in}{1.862274in}}%
\pgfpathlineto{\pgfqpoint{1.450929in}{1.862274in}}%
\pgfpathlineto{\pgfqpoint{1.450929in}{1.949059in}}%
\pgfpathlineto{\pgfqpoint{1.441992in}{1.949059in}}%
\pgfpathlineto{\pgfqpoint{1.441992in}{1.862274in}}%
\pgfpathclose%
\pgfusepath{fill}%
\end{pgfscope}%
\begin{pgfscope}%
\pgfpathrectangle{\pgfqpoint{0.697024in}{0.857143in}}{\pgfqpoint{2.627103in}{1.813434in}}%
\pgfusepath{clip}%
\pgfsetbuttcap%
\pgfsetmiterjoin%
\definecolor{currentfill}{rgb}{0.754268,0.565033,0.211761}%
\pgfsetfillcolor{currentfill}%
\pgfsetlinewidth{0.000000pt}%
\definecolor{currentstroke}{rgb}{0.000000,0.000000,0.000000}%
\pgfsetstrokecolor{currentstroke}%
\pgfsetstrokeopacity{0.000000}%
\pgfsetdash{}{0pt}%
\pgfpathmoveto{\pgfqpoint{1.453163in}{1.926639in}}%
\pgfpathlineto{\pgfqpoint{1.462099in}{1.926639in}}%
\pgfpathlineto{\pgfqpoint{1.462099in}{1.959958in}}%
\pgfpathlineto{\pgfqpoint{1.453163in}{1.959958in}}%
\pgfpathlineto{\pgfqpoint{1.453163in}{1.926639in}}%
\pgfpathclose%
\pgfusepath{fill}%
\end{pgfscope}%
\begin{pgfscope}%
\pgfpathrectangle{\pgfqpoint{0.697024in}{0.857143in}}{\pgfqpoint{2.627103in}{1.813434in}}%
\pgfusepath{clip}%
\pgfsetbuttcap%
\pgfsetmiterjoin%
\definecolor{currentfill}{rgb}{0.754268,0.565033,0.211761}%
\pgfsetfillcolor{currentfill}%
\pgfsetlinewidth{0.000000pt}%
\definecolor{currentstroke}{rgb}{0.000000,0.000000,0.000000}%
\pgfsetstrokecolor{currentstroke}%
\pgfsetstrokeopacity{0.000000}%
\pgfsetdash{}{0pt}%
\pgfpathmoveto{\pgfqpoint{1.464333in}{1.719917in}}%
\pgfpathlineto{\pgfqpoint{1.473270in}{1.719917in}}%
\pgfpathlineto{\pgfqpoint{1.473270in}{1.617038in}}%
\pgfpathlineto{\pgfqpoint{1.464333in}{1.617038in}}%
\pgfpathlineto{\pgfqpoint{1.464333in}{1.719917in}}%
\pgfpathclose%
\pgfusepath{fill}%
\end{pgfscope}%
\begin{pgfscope}%
\pgfpathrectangle{\pgfqpoint{0.697024in}{0.857143in}}{\pgfqpoint{2.627103in}{1.813434in}}%
\pgfusepath{clip}%
\pgfsetbuttcap%
\pgfsetmiterjoin%
\definecolor{currentfill}{rgb}{0.754268,0.565033,0.211761}%
\pgfsetfillcolor{currentfill}%
\pgfsetlinewidth{0.000000pt}%
\definecolor{currentstroke}{rgb}{0.000000,0.000000,0.000000}%
\pgfsetstrokecolor{currentstroke}%
\pgfsetstrokeopacity{0.000000}%
\pgfsetdash{}{0pt}%
\pgfpathmoveto{\pgfqpoint{1.475504in}{1.795076in}}%
\pgfpathlineto{\pgfqpoint{1.484440in}{1.795076in}}%
\pgfpathlineto{\pgfqpoint{1.484440in}{1.699439in}}%
\pgfpathlineto{\pgfqpoint{1.475504in}{1.699439in}}%
\pgfpathlineto{\pgfqpoint{1.475504in}{1.795076in}}%
\pgfpathclose%
\pgfusepath{fill}%
\end{pgfscope}%
\begin{pgfscope}%
\pgfpathrectangle{\pgfqpoint{0.697024in}{0.857143in}}{\pgfqpoint{2.627103in}{1.813434in}}%
\pgfusepath{clip}%
\pgfsetbuttcap%
\pgfsetmiterjoin%
\definecolor{currentfill}{rgb}{0.754268,0.565033,0.211761}%
\pgfsetfillcolor{currentfill}%
\pgfsetlinewidth{0.000000pt}%
\definecolor{currentstroke}{rgb}{0.000000,0.000000,0.000000}%
\pgfsetstrokecolor{currentstroke}%
\pgfsetstrokeopacity{0.000000}%
\pgfsetdash{}{0pt}%
\pgfpathmoveto{\pgfqpoint{1.486674in}{1.810404in}}%
\pgfpathlineto{\pgfqpoint{1.495611in}{1.810404in}}%
\pgfpathlineto{\pgfqpoint{1.495611in}{1.639420in}}%
\pgfpathlineto{\pgfqpoint{1.486674in}{1.639420in}}%
\pgfpathlineto{\pgfqpoint{1.486674in}{1.810404in}}%
\pgfpathclose%
\pgfusepath{fill}%
\end{pgfscope}%
\begin{pgfscope}%
\pgfpathrectangle{\pgfqpoint{0.697024in}{0.857143in}}{\pgfqpoint{2.627103in}{1.813434in}}%
\pgfusepath{clip}%
\pgfsetbuttcap%
\pgfsetmiterjoin%
\definecolor{currentfill}{rgb}{0.754268,0.565033,0.211761}%
\pgfsetfillcolor{currentfill}%
\pgfsetlinewidth{0.000000pt}%
\definecolor{currentstroke}{rgb}{0.000000,0.000000,0.000000}%
\pgfsetstrokecolor{currentstroke}%
\pgfsetstrokeopacity{0.000000}%
\pgfsetdash{}{0pt}%
\pgfpathmoveto{\pgfqpoint{1.497845in}{1.789478in}}%
\pgfpathlineto{\pgfqpoint{1.506782in}{1.789478in}}%
\pgfpathlineto{\pgfqpoint{1.506782in}{1.702827in}}%
\pgfpathlineto{\pgfqpoint{1.497845in}{1.702827in}}%
\pgfpathlineto{\pgfqpoint{1.497845in}{1.789478in}}%
\pgfpathclose%
\pgfusepath{fill}%
\end{pgfscope}%
\begin{pgfscope}%
\pgfpathrectangle{\pgfqpoint{0.697024in}{0.857143in}}{\pgfqpoint{2.627103in}{1.813434in}}%
\pgfusepath{clip}%
\pgfsetbuttcap%
\pgfsetmiterjoin%
\definecolor{currentfill}{rgb}{0.754268,0.565033,0.211761}%
\pgfsetfillcolor{currentfill}%
\pgfsetlinewidth{0.000000pt}%
\definecolor{currentstroke}{rgb}{0.000000,0.000000,0.000000}%
\pgfsetstrokecolor{currentstroke}%
\pgfsetstrokeopacity{0.000000}%
\pgfsetdash{}{0pt}%
\pgfpathmoveto{\pgfqpoint{1.509016in}{1.687074in}}%
\pgfpathlineto{\pgfqpoint{1.517952in}{1.687074in}}%
\pgfpathlineto{\pgfqpoint{1.517952in}{1.595905in}}%
\pgfpathlineto{\pgfqpoint{1.509016in}{1.595905in}}%
\pgfpathlineto{\pgfqpoint{1.509016in}{1.687074in}}%
\pgfpathclose%
\pgfusepath{fill}%
\end{pgfscope}%
\begin{pgfscope}%
\pgfpathrectangle{\pgfqpoint{0.697024in}{0.857143in}}{\pgfqpoint{2.627103in}{1.813434in}}%
\pgfusepath{clip}%
\pgfsetbuttcap%
\pgfsetmiterjoin%
\definecolor{currentfill}{rgb}{0.754268,0.565033,0.211761}%
\pgfsetfillcolor{currentfill}%
\pgfsetlinewidth{0.000000pt}%
\definecolor{currentstroke}{rgb}{0.000000,0.000000,0.000000}%
\pgfsetstrokecolor{currentstroke}%
\pgfsetstrokeopacity{0.000000}%
\pgfsetdash{}{0pt}%
\pgfpathmoveto{\pgfqpoint{1.520186in}{1.735874in}}%
\pgfpathlineto{\pgfqpoint{1.529123in}{1.735874in}}%
\pgfpathlineto{\pgfqpoint{1.529123in}{1.599730in}}%
\pgfpathlineto{\pgfqpoint{1.520186in}{1.599730in}}%
\pgfpathlineto{\pgfqpoint{1.520186in}{1.735874in}}%
\pgfpathclose%
\pgfusepath{fill}%
\end{pgfscope}%
\begin{pgfscope}%
\pgfpathrectangle{\pgfqpoint{0.697024in}{0.857143in}}{\pgfqpoint{2.627103in}{1.813434in}}%
\pgfusepath{clip}%
\pgfsetbuttcap%
\pgfsetmiterjoin%
\definecolor{currentfill}{rgb}{0.754268,0.565033,0.211761}%
\pgfsetfillcolor{currentfill}%
\pgfsetlinewidth{0.000000pt}%
\definecolor{currentstroke}{rgb}{0.000000,0.000000,0.000000}%
\pgfsetstrokecolor{currentstroke}%
\pgfsetstrokeopacity{0.000000}%
\pgfsetdash{}{0pt}%
\pgfpathmoveto{\pgfqpoint{1.531357in}{1.625250in}}%
\pgfpathlineto{\pgfqpoint{1.540293in}{1.625250in}}%
\pgfpathlineto{\pgfqpoint{1.540293in}{1.565269in}}%
\pgfpathlineto{\pgfqpoint{1.531357in}{1.565269in}}%
\pgfpathlineto{\pgfqpoint{1.531357in}{1.625250in}}%
\pgfpathclose%
\pgfusepath{fill}%
\end{pgfscope}%
\begin{pgfscope}%
\pgfpathrectangle{\pgfqpoint{0.697024in}{0.857143in}}{\pgfqpoint{2.627103in}{1.813434in}}%
\pgfusepath{clip}%
\pgfsetbuttcap%
\pgfsetmiterjoin%
\definecolor{currentfill}{rgb}{0.754268,0.565033,0.211761}%
\pgfsetfillcolor{currentfill}%
\pgfsetlinewidth{0.000000pt}%
\definecolor{currentstroke}{rgb}{0.000000,0.000000,0.000000}%
\pgfsetstrokecolor{currentstroke}%
\pgfsetstrokeopacity{0.000000}%
\pgfsetdash{}{0pt}%
\pgfpathmoveto{\pgfqpoint{1.542528in}{1.622144in}}%
\pgfpathlineto{\pgfqpoint{1.551464in}{1.622144in}}%
\pgfpathlineto{\pgfqpoint{1.551464in}{1.572041in}}%
\pgfpathlineto{\pgfqpoint{1.542528in}{1.572041in}}%
\pgfpathlineto{\pgfqpoint{1.542528in}{1.622144in}}%
\pgfpathclose%
\pgfusepath{fill}%
\end{pgfscope}%
\begin{pgfscope}%
\pgfpathrectangle{\pgfqpoint{0.697024in}{0.857143in}}{\pgfqpoint{2.627103in}{1.813434in}}%
\pgfusepath{clip}%
\pgfsetbuttcap%
\pgfsetmiterjoin%
\definecolor{currentfill}{rgb}{0.754268,0.565033,0.211761}%
\pgfsetfillcolor{currentfill}%
\pgfsetlinewidth{0.000000pt}%
\definecolor{currentstroke}{rgb}{0.000000,0.000000,0.000000}%
\pgfsetstrokecolor{currentstroke}%
\pgfsetstrokeopacity{0.000000}%
\pgfsetdash{}{0pt}%
\pgfpathmoveto{\pgfqpoint{1.553698in}{1.660883in}}%
\pgfpathlineto{\pgfqpoint{1.562635in}{1.660883in}}%
\pgfpathlineto{\pgfqpoint{1.562635in}{1.594586in}}%
\pgfpathlineto{\pgfqpoint{1.553698in}{1.594586in}}%
\pgfpathlineto{\pgfqpoint{1.553698in}{1.660883in}}%
\pgfpathclose%
\pgfusepath{fill}%
\end{pgfscope}%
\begin{pgfscope}%
\pgfpathrectangle{\pgfqpoint{0.697024in}{0.857143in}}{\pgfqpoint{2.627103in}{1.813434in}}%
\pgfusepath{clip}%
\pgfsetbuttcap%
\pgfsetmiterjoin%
\definecolor{currentfill}{rgb}{0.754268,0.565033,0.211761}%
\pgfsetfillcolor{currentfill}%
\pgfsetlinewidth{0.000000pt}%
\definecolor{currentstroke}{rgb}{0.000000,0.000000,0.000000}%
\pgfsetstrokecolor{currentstroke}%
\pgfsetstrokeopacity{0.000000}%
\pgfsetdash{}{0pt}%
\pgfpathmoveto{\pgfqpoint{1.564869in}{1.682518in}}%
\pgfpathlineto{\pgfqpoint{1.573805in}{1.682518in}}%
\pgfpathlineto{\pgfqpoint{1.573805in}{1.613734in}}%
\pgfpathlineto{\pgfqpoint{1.564869in}{1.613734in}}%
\pgfpathlineto{\pgfqpoint{1.564869in}{1.682518in}}%
\pgfpathclose%
\pgfusepath{fill}%
\end{pgfscope}%
\begin{pgfscope}%
\pgfpathrectangle{\pgfqpoint{0.697024in}{0.857143in}}{\pgfqpoint{2.627103in}{1.813434in}}%
\pgfusepath{clip}%
\pgfsetbuttcap%
\pgfsetmiterjoin%
\definecolor{currentfill}{rgb}{0.754268,0.565033,0.211761}%
\pgfsetfillcolor{currentfill}%
\pgfsetlinewidth{0.000000pt}%
\definecolor{currentstroke}{rgb}{0.000000,0.000000,0.000000}%
\pgfsetstrokecolor{currentstroke}%
\pgfsetstrokeopacity{0.000000}%
\pgfsetdash{}{0pt}%
\pgfpathmoveto{\pgfqpoint{1.576039in}{1.718566in}}%
\pgfpathlineto{\pgfqpoint{1.584976in}{1.718566in}}%
\pgfpathlineto{\pgfqpoint{1.584976in}{1.681604in}}%
\pgfpathlineto{\pgfqpoint{1.576039in}{1.681604in}}%
\pgfpathlineto{\pgfqpoint{1.576039in}{1.718566in}}%
\pgfpathclose%
\pgfusepath{fill}%
\end{pgfscope}%
\begin{pgfscope}%
\pgfpathrectangle{\pgfqpoint{0.697024in}{0.857143in}}{\pgfqpoint{2.627103in}{1.813434in}}%
\pgfusepath{clip}%
\pgfsetbuttcap%
\pgfsetmiterjoin%
\definecolor{currentfill}{rgb}{0.754268,0.565033,0.211761}%
\pgfsetfillcolor{currentfill}%
\pgfsetlinewidth{0.000000pt}%
\definecolor{currentstroke}{rgb}{0.000000,0.000000,0.000000}%
\pgfsetstrokecolor{currentstroke}%
\pgfsetstrokeopacity{0.000000}%
\pgfsetdash{}{0pt}%
\pgfpathmoveto{\pgfqpoint{1.587210in}{1.797390in}}%
\pgfpathlineto{\pgfqpoint{1.596146in}{1.797390in}}%
\pgfpathlineto{\pgfqpoint{1.596146in}{1.716025in}}%
\pgfpathlineto{\pgfqpoint{1.587210in}{1.716025in}}%
\pgfpathlineto{\pgfqpoint{1.587210in}{1.797390in}}%
\pgfpathclose%
\pgfusepath{fill}%
\end{pgfscope}%
\begin{pgfscope}%
\pgfpathrectangle{\pgfqpoint{0.697024in}{0.857143in}}{\pgfqpoint{2.627103in}{1.813434in}}%
\pgfusepath{clip}%
\pgfsetbuttcap%
\pgfsetmiterjoin%
\definecolor{currentfill}{rgb}{0.754268,0.565033,0.211761}%
\pgfsetfillcolor{currentfill}%
\pgfsetlinewidth{0.000000pt}%
\definecolor{currentstroke}{rgb}{0.000000,0.000000,0.000000}%
\pgfsetstrokecolor{currentstroke}%
\pgfsetstrokeopacity{0.000000}%
\pgfsetdash{}{0pt}%
\pgfpathmoveto{\pgfqpoint{1.598381in}{1.717820in}}%
\pgfpathlineto{\pgfqpoint{1.607317in}{1.717820in}}%
\pgfpathlineto{\pgfqpoint{1.607317in}{1.673571in}}%
\pgfpathlineto{\pgfqpoint{1.598381in}{1.673571in}}%
\pgfpathlineto{\pgfqpoint{1.598381in}{1.717820in}}%
\pgfpathclose%
\pgfusepath{fill}%
\end{pgfscope}%
\begin{pgfscope}%
\pgfpathrectangle{\pgfqpoint{0.697024in}{0.857143in}}{\pgfqpoint{2.627103in}{1.813434in}}%
\pgfusepath{clip}%
\pgfsetbuttcap%
\pgfsetmiterjoin%
\definecolor{currentfill}{rgb}{0.754268,0.565033,0.211761}%
\pgfsetfillcolor{currentfill}%
\pgfsetlinewidth{0.000000pt}%
\definecolor{currentstroke}{rgb}{0.000000,0.000000,0.000000}%
\pgfsetstrokecolor{currentstroke}%
\pgfsetstrokeopacity{0.000000}%
\pgfsetdash{}{0pt}%
\pgfpathmoveto{\pgfqpoint{1.609551in}{1.759276in}}%
\pgfpathlineto{\pgfqpoint{1.618488in}{1.759276in}}%
\pgfpathlineto{\pgfqpoint{1.618488in}{1.677358in}}%
\pgfpathlineto{\pgfqpoint{1.609551in}{1.677358in}}%
\pgfpathlineto{\pgfqpoint{1.609551in}{1.759276in}}%
\pgfpathclose%
\pgfusepath{fill}%
\end{pgfscope}%
\begin{pgfscope}%
\pgfpathrectangle{\pgfqpoint{0.697024in}{0.857143in}}{\pgfqpoint{2.627103in}{1.813434in}}%
\pgfusepath{clip}%
\pgfsetbuttcap%
\pgfsetmiterjoin%
\definecolor{currentfill}{rgb}{0.754268,0.565033,0.211761}%
\pgfsetfillcolor{currentfill}%
\pgfsetlinewidth{0.000000pt}%
\definecolor{currentstroke}{rgb}{0.000000,0.000000,0.000000}%
\pgfsetstrokecolor{currentstroke}%
\pgfsetstrokeopacity{0.000000}%
\pgfsetdash{}{0pt}%
\pgfpathmoveto{\pgfqpoint{1.620722in}{1.729882in}}%
\pgfpathlineto{\pgfqpoint{1.629658in}{1.729882in}}%
\pgfpathlineto{\pgfqpoint{1.629658in}{1.637049in}}%
\pgfpathlineto{\pgfqpoint{1.620722in}{1.637049in}}%
\pgfpathlineto{\pgfqpoint{1.620722in}{1.729882in}}%
\pgfpathclose%
\pgfusepath{fill}%
\end{pgfscope}%
\begin{pgfscope}%
\pgfpathrectangle{\pgfqpoint{0.697024in}{0.857143in}}{\pgfqpoint{2.627103in}{1.813434in}}%
\pgfusepath{clip}%
\pgfsetbuttcap%
\pgfsetmiterjoin%
\definecolor{currentfill}{rgb}{0.754268,0.565033,0.211761}%
\pgfsetfillcolor{currentfill}%
\pgfsetlinewidth{0.000000pt}%
\definecolor{currentstroke}{rgb}{0.000000,0.000000,0.000000}%
\pgfsetstrokecolor{currentstroke}%
\pgfsetstrokeopacity{0.000000}%
\pgfsetdash{}{0pt}%
\pgfpathmoveto{\pgfqpoint{1.631892in}{1.692417in}}%
\pgfpathlineto{\pgfqpoint{1.640829in}{1.692417in}}%
\pgfpathlineto{\pgfqpoint{1.640829in}{1.613716in}}%
\pgfpathlineto{\pgfqpoint{1.631892in}{1.613716in}}%
\pgfpathlineto{\pgfqpoint{1.631892in}{1.692417in}}%
\pgfpathclose%
\pgfusepath{fill}%
\end{pgfscope}%
\begin{pgfscope}%
\pgfpathrectangle{\pgfqpoint{0.697024in}{0.857143in}}{\pgfqpoint{2.627103in}{1.813434in}}%
\pgfusepath{clip}%
\pgfsetbuttcap%
\pgfsetmiterjoin%
\definecolor{currentfill}{rgb}{0.754268,0.565033,0.211761}%
\pgfsetfillcolor{currentfill}%
\pgfsetlinewidth{0.000000pt}%
\definecolor{currentstroke}{rgb}{0.000000,0.000000,0.000000}%
\pgfsetstrokecolor{currentstroke}%
\pgfsetstrokeopacity{0.000000}%
\pgfsetdash{}{0pt}%
\pgfpathmoveto{\pgfqpoint{1.643063in}{1.615715in}}%
\pgfpathlineto{\pgfqpoint{1.651999in}{1.615715in}}%
\pgfpathlineto{\pgfqpoint{1.651999in}{1.606278in}}%
\pgfpathlineto{\pgfqpoint{1.643063in}{1.606278in}}%
\pgfpathlineto{\pgfqpoint{1.643063in}{1.615715in}}%
\pgfpathclose%
\pgfusepath{fill}%
\end{pgfscope}%
\begin{pgfscope}%
\pgfpathrectangle{\pgfqpoint{0.697024in}{0.857143in}}{\pgfqpoint{2.627103in}{1.813434in}}%
\pgfusepath{clip}%
\pgfsetbuttcap%
\pgfsetmiterjoin%
\definecolor{currentfill}{rgb}{0.754268,0.565033,0.211761}%
\pgfsetfillcolor{currentfill}%
\pgfsetlinewidth{0.000000pt}%
\definecolor{currentstroke}{rgb}{0.000000,0.000000,0.000000}%
\pgfsetstrokecolor{currentstroke}%
\pgfsetstrokeopacity{0.000000}%
\pgfsetdash{}{0pt}%
\pgfpathmoveto{\pgfqpoint{1.654234in}{1.705720in}}%
\pgfpathlineto{\pgfqpoint{1.663170in}{1.705720in}}%
\pgfpathlineto{\pgfqpoint{1.663170in}{1.645488in}}%
\pgfpathlineto{\pgfqpoint{1.654234in}{1.645488in}}%
\pgfpathlineto{\pgfqpoint{1.654234in}{1.705720in}}%
\pgfpathclose%
\pgfusepath{fill}%
\end{pgfscope}%
\begin{pgfscope}%
\pgfpathrectangle{\pgfqpoint{0.697024in}{0.857143in}}{\pgfqpoint{2.627103in}{1.813434in}}%
\pgfusepath{clip}%
\pgfsetbuttcap%
\pgfsetmiterjoin%
\definecolor{currentfill}{rgb}{0.754268,0.565033,0.211761}%
\pgfsetfillcolor{currentfill}%
\pgfsetlinewidth{0.000000pt}%
\definecolor{currentstroke}{rgb}{0.000000,0.000000,0.000000}%
\pgfsetstrokecolor{currentstroke}%
\pgfsetstrokeopacity{0.000000}%
\pgfsetdash{}{0pt}%
\pgfpathmoveto{\pgfqpoint{1.665404in}{1.689069in}}%
\pgfpathlineto{\pgfqpoint{1.674341in}{1.689069in}}%
\pgfpathlineto{\pgfqpoint{1.674341in}{1.635301in}}%
\pgfpathlineto{\pgfqpoint{1.665404in}{1.635301in}}%
\pgfpathlineto{\pgfqpoint{1.665404in}{1.689069in}}%
\pgfpathclose%
\pgfusepath{fill}%
\end{pgfscope}%
\begin{pgfscope}%
\pgfpathrectangle{\pgfqpoint{0.697024in}{0.857143in}}{\pgfqpoint{2.627103in}{1.813434in}}%
\pgfusepath{clip}%
\pgfsetbuttcap%
\pgfsetmiterjoin%
\definecolor{currentfill}{rgb}{0.754268,0.565033,0.211761}%
\pgfsetfillcolor{currentfill}%
\pgfsetlinewidth{0.000000pt}%
\definecolor{currentstroke}{rgb}{0.000000,0.000000,0.000000}%
\pgfsetstrokecolor{currentstroke}%
\pgfsetstrokeopacity{0.000000}%
\pgfsetdash{}{0pt}%
\pgfpathmoveto{\pgfqpoint{1.676575in}{1.656798in}}%
\pgfpathlineto{\pgfqpoint{1.685511in}{1.656798in}}%
\pgfpathlineto{\pgfqpoint{1.685511in}{1.596146in}}%
\pgfpathlineto{\pgfqpoint{1.676575in}{1.596146in}}%
\pgfpathlineto{\pgfqpoint{1.676575in}{1.656798in}}%
\pgfpathclose%
\pgfusepath{fill}%
\end{pgfscope}%
\begin{pgfscope}%
\pgfpathrectangle{\pgfqpoint{0.697024in}{0.857143in}}{\pgfqpoint{2.627103in}{1.813434in}}%
\pgfusepath{clip}%
\pgfsetbuttcap%
\pgfsetmiterjoin%
\definecolor{currentfill}{rgb}{0.754268,0.565033,0.211761}%
\pgfsetfillcolor{currentfill}%
\pgfsetlinewidth{0.000000pt}%
\definecolor{currentstroke}{rgb}{0.000000,0.000000,0.000000}%
\pgfsetstrokecolor{currentstroke}%
\pgfsetstrokeopacity{0.000000}%
\pgfsetdash{}{0pt}%
\pgfpathmoveto{\pgfqpoint{1.687745in}{1.722210in}}%
\pgfpathlineto{\pgfqpoint{1.696682in}{1.722210in}}%
\pgfpathlineto{\pgfqpoint{1.696682in}{1.652480in}}%
\pgfpathlineto{\pgfqpoint{1.687745in}{1.652480in}}%
\pgfpathlineto{\pgfqpoint{1.687745in}{1.722210in}}%
\pgfpathclose%
\pgfusepath{fill}%
\end{pgfscope}%
\begin{pgfscope}%
\pgfpathrectangle{\pgfqpoint{0.697024in}{0.857143in}}{\pgfqpoint{2.627103in}{1.813434in}}%
\pgfusepath{clip}%
\pgfsetbuttcap%
\pgfsetmiterjoin%
\definecolor{currentfill}{rgb}{0.754268,0.565033,0.211761}%
\pgfsetfillcolor{currentfill}%
\pgfsetlinewidth{0.000000pt}%
\definecolor{currentstroke}{rgb}{0.000000,0.000000,0.000000}%
\pgfsetstrokecolor{currentstroke}%
\pgfsetstrokeopacity{0.000000}%
\pgfsetdash{}{0pt}%
\pgfpathmoveto{\pgfqpoint{1.698916in}{1.685914in}}%
\pgfpathlineto{\pgfqpoint{1.707852in}{1.685914in}}%
\pgfpathlineto{\pgfqpoint{1.707852in}{1.628883in}}%
\pgfpathlineto{\pgfqpoint{1.698916in}{1.628883in}}%
\pgfpathlineto{\pgfqpoint{1.698916in}{1.685914in}}%
\pgfpathclose%
\pgfusepath{fill}%
\end{pgfscope}%
\begin{pgfscope}%
\pgfpathrectangle{\pgfqpoint{0.697024in}{0.857143in}}{\pgfqpoint{2.627103in}{1.813434in}}%
\pgfusepath{clip}%
\pgfsetbuttcap%
\pgfsetmiterjoin%
\definecolor{currentfill}{rgb}{0.754268,0.565033,0.211761}%
\pgfsetfillcolor{currentfill}%
\pgfsetlinewidth{0.000000pt}%
\definecolor{currentstroke}{rgb}{0.000000,0.000000,0.000000}%
\pgfsetstrokecolor{currentstroke}%
\pgfsetstrokeopacity{0.000000}%
\pgfsetdash{}{0pt}%
\pgfpathmoveto{\pgfqpoint{1.710087in}{1.654197in}}%
\pgfpathlineto{\pgfqpoint{1.719023in}{1.654197in}}%
\pgfpathlineto{\pgfqpoint{1.719023in}{1.611267in}}%
\pgfpathlineto{\pgfqpoint{1.710087in}{1.611267in}}%
\pgfpathlineto{\pgfqpoint{1.710087in}{1.654197in}}%
\pgfpathclose%
\pgfusepath{fill}%
\end{pgfscope}%
\begin{pgfscope}%
\pgfpathrectangle{\pgfqpoint{0.697024in}{0.857143in}}{\pgfqpoint{2.627103in}{1.813434in}}%
\pgfusepath{clip}%
\pgfsetbuttcap%
\pgfsetmiterjoin%
\definecolor{currentfill}{rgb}{0.754268,0.565033,0.211761}%
\pgfsetfillcolor{currentfill}%
\pgfsetlinewidth{0.000000pt}%
\definecolor{currentstroke}{rgb}{0.000000,0.000000,0.000000}%
\pgfsetstrokecolor{currentstroke}%
\pgfsetstrokeopacity{0.000000}%
\pgfsetdash{}{0pt}%
\pgfpathmoveto{\pgfqpoint{1.721257in}{1.608018in}}%
\pgfpathlineto{\pgfqpoint{1.730194in}{1.608018in}}%
\pgfpathlineto{\pgfqpoint{1.730194in}{1.597114in}}%
\pgfpathlineto{\pgfqpoint{1.721257in}{1.597114in}}%
\pgfpathlineto{\pgfqpoint{1.721257in}{1.608018in}}%
\pgfpathclose%
\pgfusepath{fill}%
\end{pgfscope}%
\begin{pgfscope}%
\pgfpathrectangle{\pgfqpoint{0.697024in}{0.857143in}}{\pgfqpoint{2.627103in}{1.813434in}}%
\pgfusepath{clip}%
\pgfsetbuttcap%
\pgfsetmiterjoin%
\definecolor{currentfill}{rgb}{0.754268,0.565033,0.211761}%
\pgfsetfillcolor{currentfill}%
\pgfsetlinewidth{0.000000pt}%
\definecolor{currentstroke}{rgb}{0.000000,0.000000,0.000000}%
\pgfsetstrokecolor{currentstroke}%
\pgfsetstrokeopacity{0.000000}%
\pgfsetdash{}{0pt}%
\pgfpathmoveto{\pgfqpoint{1.732428in}{1.603706in}}%
\pgfpathlineto{\pgfqpoint{1.741364in}{1.603706in}}%
\pgfpathlineto{\pgfqpoint{1.741364in}{1.587705in}}%
\pgfpathlineto{\pgfqpoint{1.732428in}{1.587705in}}%
\pgfpathlineto{\pgfqpoint{1.732428in}{1.603706in}}%
\pgfpathclose%
\pgfusepath{fill}%
\end{pgfscope}%
\begin{pgfscope}%
\pgfpathrectangle{\pgfqpoint{0.697024in}{0.857143in}}{\pgfqpoint{2.627103in}{1.813434in}}%
\pgfusepath{clip}%
\pgfsetbuttcap%
\pgfsetmiterjoin%
\definecolor{currentfill}{rgb}{0.754268,0.565033,0.211761}%
\pgfsetfillcolor{currentfill}%
\pgfsetlinewidth{0.000000pt}%
\definecolor{currentstroke}{rgb}{0.000000,0.000000,0.000000}%
\pgfsetstrokecolor{currentstroke}%
\pgfsetstrokeopacity{0.000000}%
\pgfsetdash{}{0pt}%
\pgfpathmoveto{\pgfqpoint{1.743598in}{1.582261in}}%
\pgfpathlineto{\pgfqpoint{1.752535in}{1.582261in}}%
\pgfpathlineto{\pgfqpoint{1.752535in}{1.572511in}}%
\pgfpathlineto{\pgfqpoint{1.743598in}{1.572511in}}%
\pgfpathlineto{\pgfqpoint{1.743598in}{1.582261in}}%
\pgfpathclose%
\pgfusepath{fill}%
\end{pgfscope}%
\begin{pgfscope}%
\pgfpathrectangle{\pgfqpoint{0.697024in}{0.857143in}}{\pgfqpoint{2.627103in}{1.813434in}}%
\pgfusepath{clip}%
\pgfsetbuttcap%
\pgfsetmiterjoin%
\definecolor{currentfill}{rgb}{0.754268,0.565033,0.211761}%
\pgfsetfillcolor{currentfill}%
\pgfsetlinewidth{0.000000pt}%
\definecolor{currentstroke}{rgb}{0.000000,0.000000,0.000000}%
\pgfsetstrokecolor{currentstroke}%
\pgfsetstrokeopacity{0.000000}%
\pgfsetdash{}{0pt}%
\pgfpathmoveto{\pgfqpoint{1.754769in}{1.532107in}}%
\pgfpathlineto{\pgfqpoint{1.763705in}{1.532107in}}%
\pgfpathlineto{\pgfqpoint{1.763705in}{1.524193in}}%
\pgfpathlineto{\pgfqpoint{1.754769in}{1.524193in}}%
\pgfpathlineto{\pgfqpoint{1.754769in}{1.532107in}}%
\pgfpathclose%
\pgfusepath{fill}%
\end{pgfscope}%
\begin{pgfscope}%
\pgfpathrectangle{\pgfqpoint{0.697024in}{0.857143in}}{\pgfqpoint{2.627103in}{1.813434in}}%
\pgfusepath{clip}%
\pgfsetbuttcap%
\pgfsetmiterjoin%
\definecolor{currentfill}{rgb}{0.754268,0.565033,0.211761}%
\pgfsetfillcolor{currentfill}%
\pgfsetlinewidth{0.000000pt}%
\definecolor{currentstroke}{rgb}{0.000000,0.000000,0.000000}%
\pgfsetstrokecolor{currentstroke}%
\pgfsetstrokeopacity{0.000000}%
\pgfsetdash{}{0pt}%
\pgfpathmoveto{\pgfqpoint{1.765940in}{1.580443in}}%
\pgfpathlineto{\pgfqpoint{1.774876in}{1.580443in}}%
\pgfpathlineto{\pgfqpoint{1.774876in}{1.560547in}}%
\pgfpathlineto{\pgfqpoint{1.765940in}{1.560547in}}%
\pgfpathlineto{\pgfqpoint{1.765940in}{1.580443in}}%
\pgfpathclose%
\pgfusepath{fill}%
\end{pgfscope}%
\begin{pgfscope}%
\pgfpathrectangle{\pgfqpoint{0.697024in}{0.857143in}}{\pgfqpoint{2.627103in}{1.813434in}}%
\pgfusepath{clip}%
\pgfsetbuttcap%
\pgfsetmiterjoin%
\definecolor{currentfill}{rgb}{0.754268,0.565033,0.211761}%
\pgfsetfillcolor{currentfill}%
\pgfsetlinewidth{0.000000pt}%
\definecolor{currentstroke}{rgb}{0.000000,0.000000,0.000000}%
\pgfsetstrokecolor{currentstroke}%
\pgfsetstrokeopacity{0.000000}%
\pgfsetdash{}{0pt}%
\pgfpathmoveto{\pgfqpoint{1.777110in}{1.559847in}}%
\pgfpathlineto{\pgfqpoint{1.786047in}{1.559847in}}%
\pgfpathlineto{\pgfqpoint{1.786047in}{1.548795in}}%
\pgfpathlineto{\pgfqpoint{1.777110in}{1.548795in}}%
\pgfpathlineto{\pgfqpoint{1.777110in}{1.559847in}}%
\pgfpathclose%
\pgfusepath{fill}%
\end{pgfscope}%
\begin{pgfscope}%
\pgfpathrectangle{\pgfqpoint{0.697024in}{0.857143in}}{\pgfqpoint{2.627103in}{1.813434in}}%
\pgfusepath{clip}%
\pgfsetbuttcap%
\pgfsetmiterjoin%
\definecolor{currentfill}{rgb}{0.754268,0.565033,0.211761}%
\pgfsetfillcolor{currentfill}%
\pgfsetlinewidth{0.000000pt}%
\definecolor{currentstroke}{rgb}{0.000000,0.000000,0.000000}%
\pgfsetstrokecolor{currentstroke}%
\pgfsetstrokeopacity{0.000000}%
\pgfsetdash{}{0pt}%
\pgfpathmoveto{\pgfqpoint{1.788281in}{2.424768in}}%
\pgfpathlineto{\pgfqpoint{1.797217in}{2.424768in}}%
\pgfpathlineto{\pgfqpoint{1.797217in}{2.436448in}}%
\pgfpathlineto{\pgfqpoint{1.788281in}{2.436448in}}%
\pgfpathlineto{\pgfqpoint{1.788281in}{2.424768in}}%
\pgfpathclose%
\pgfusepath{fill}%
\end{pgfscope}%
\begin{pgfscope}%
\pgfpathrectangle{\pgfqpoint{0.697024in}{0.857143in}}{\pgfqpoint{2.627103in}{1.813434in}}%
\pgfusepath{clip}%
\pgfsetbuttcap%
\pgfsetmiterjoin%
\definecolor{currentfill}{rgb}{0.754268,0.565033,0.211761}%
\pgfsetfillcolor{currentfill}%
\pgfsetlinewidth{0.000000pt}%
\definecolor{currentstroke}{rgb}{0.000000,0.000000,0.000000}%
\pgfsetstrokecolor{currentstroke}%
\pgfsetstrokeopacity{0.000000}%
\pgfsetdash{}{0pt}%
\pgfpathmoveto{\pgfqpoint{1.799451in}{2.475733in}}%
\pgfpathlineto{\pgfqpoint{1.808388in}{2.475733in}}%
\pgfpathlineto{\pgfqpoint{1.808388in}{2.493124in}}%
\pgfpathlineto{\pgfqpoint{1.799451in}{2.493124in}}%
\pgfpathlineto{\pgfqpoint{1.799451in}{2.475733in}}%
\pgfpathclose%
\pgfusepath{fill}%
\end{pgfscope}%
\begin{pgfscope}%
\pgfpathrectangle{\pgfqpoint{0.697024in}{0.857143in}}{\pgfqpoint{2.627103in}{1.813434in}}%
\pgfusepath{clip}%
\pgfsetbuttcap%
\pgfsetmiterjoin%
\definecolor{currentfill}{rgb}{0.754268,0.565033,0.211761}%
\pgfsetfillcolor{currentfill}%
\pgfsetlinewidth{0.000000pt}%
\definecolor{currentstroke}{rgb}{0.000000,0.000000,0.000000}%
\pgfsetstrokecolor{currentstroke}%
\pgfsetstrokeopacity{0.000000}%
\pgfsetdash{}{0pt}%
\pgfpathmoveto{\pgfqpoint{1.810622in}{1.550517in}}%
\pgfpathlineto{\pgfqpoint{1.819559in}{1.550517in}}%
\pgfpathlineto{\pgfqpoint{1.819559in}{1.509357in}}%
\pgfpathlineto{\pgfqpoint{1.810622in}{1.509357in}}%
\pgfpathlineto{\pgfqpoint{1.810622in}{1.550517in}}%
\pgfpathclose%
\pgfusepath{fill}%
\end{pgfscope}%
\begin{pgfscope}%
\pgfpathrectangle{\pgfqpoint{0.697024in}{0.857143in}}{\pgfqpoint{2.627103in}{1.813434in}}%
\pgfusepath{clip}%
\pgfsetbuttcap%
\pgfsetmiterjoin%
\definecolor{currentfill}{rgb}{0.754268,0.565033,0.211761}%
\pgfsetfillcolor{currentfill}%
\pgfsetlinewidth{0.000000pt}%
\definecolor{currentstroke}{rgb}{0.000000,0.000000,0.000000}%
\pgfsetstrokecolor{currentstroke}%
\pgfsetstrokeopacity{0.000000}%
\pgfsetdash{}{0pt}%
\pgfpathmoveto{\pgfqpoint{1.821793in}{1.491584in}}%
\pgfpathlineto{\pgfqpoint{1.830729in}{1.491584in}}%
\pgfpathlineto{\pgfqpoint{1.830729in}{1.461562in}}%
\pgfpathlineto{\pgfqpoint{1.821793in}{1.461562in}}%
\pgfpathlineto{\pgfqpoint{1.821793in}{1.491584in}}%
\pgfpathclose%
\pgfusepath{fill}%
\end{pgfscope}%
\begin{pgfscope}%
\pgfpathrectangle{\pgfqpoint{0.697024in}{0.857143in}}{\pgfqpoint{2.627103in}{1.813434in}}%
\pgfusepath{clip}%
\pgfsetbuttcap%
\pgfsetmiterjoin%
\definecolor{currentfill}{rgb}{0.754268,0.565033,0.211761}%
\pgfsetfillcolor{currentfill}%
\pgfsetlinewidth{0.000000pt}%
\definecolor{currentstroke}{rgb}{0.000000,0.000000,0.000000}%
\pgfsetstrokecolor{currentstroke}%
\pgfsetstrokeopacity{0.000000}%
\pgfsetdash{}{0pt}%
\pgfpathmoveto{\pgfqpoint{1.832963in}{1.493008in}}%
\pgfpathlineto{\pgfqpoint{1.841900in}{1.493008in}}%
\pgfpathlineto{\pgfqpoint{1.841900in}{1.451964in}}%
\pgfpathlineto{\pgfqpoint{1.832963in}{1.451964in}}%
\pgfpathlineto{\pgfqpoint{1.832963in}{1.493008in}}%
\pgfpathclose%
\pgfusepath{fill}%
\end{pgfscope}%
\begin{pgfscope}%
\pgfpathrectangle{\pgfqpoint{0.697024in}{0.857143in}}{\pgfqpoint{2.627103in}{1.813434in}}%
\pgfusepath{clip}%
\pgfsetbuttcap%
\pgfsetmiterjoin%
\definecolor{currentfill}{rgb}{0.754268,0.565033,0.211761}%
\pgfsetfillcolor{currentfill}%
\pgfsetlinewidth{0.000000pt}%
\definecolor{currentstroke}{rgb}{0.000000,0.000000,0.000000}%
\pgfsetstrokecolor{currentstroke}%
\pgfsetstrokeopacity{0.000000}%
\pgfsetdash{}{0pt}%
\pgfpathmoveto{\pgfqpoint{1.844134in}{1.549358in}}%
\pgfpathlineto{\pgfqpoint{1.853070in}{1.549358in}}%
\pgfpathlineto{\pgfqpoint{1.853070in}{1.480404in}}%
\pgfpathlineto{\pgfqpoint{1.844134in}{1.480404in}}%
\pgfpathlineto{\pgfqpoint{1.844134in}{1.549358in}}%
\pgfpathclose%
\pgfusepath{fill}%
\end{pgfscope}%
\begin{pgfscope}%
\pgfpathrectangle{\pgfqpoint{0.697024in}{0.857143in}}{\pgfqpoint{2.627103in}{1.813434in}}%
\pgfusepath{clip}%
\pgfsetbuttcap%
\pgfsetmiterjoin%
\definecolor{currentfill}{rgb}{0.754268,0.565033,0.211761}%
\pgfsetfillcolor{currentfill}%
\pgfsetlinewidth{0.000000pt}%
\definecolor{currentstroke}{rgb}{0.000000,0.000000,0.000000}%
\pgfsetstrokecolor{currentstroke}%
\pgfsetstrokeopacity{0.000000}%
\pgfsetdash{}{0pt}%
\pgfpathmoveto{\pgfqpoint{1.855304in}{1.534914in}}%
\pgfpathlineto{\pgfqpoint{1.864241in}{1.534914in}}%
\pgfpathlineto{\pgfqpoint{1.864241in}{1.478054in}}%
\pgfpathlineto{\pgfqpoint{1.855304in}{1.478054in}}%
\pgfpathlineto{\pgfqpoint{1.855304in}{1.534914in}}%
\pgfpathclose%
\pgfusepath{fill}%
\end{pgfscope}%
\begin{pgfscope}%
\pgfpathrectangle{\pgfqpoint{0.697024in}{0.857143in}}{\pgfqpoint{2.627103in}{1.813434in}}%
\pgfusepath{clip}%
\pgfsetbuttcap%
\pgfsetmiterjoin%
\definecolor{currentfill}{rgb}{0.754268,0.565033,0.211761}%
\pgfsetfillcolor{currentfill}%
\pgfsetlinewidth{0.000000pt}%
\definecolor{currentstroke}{rgb}{0.000000,0.000000,0.000000}%
\pgfsetstrokecolor{currentstroke}%
\pgfsetstrokeopacity{0.000000}%
\pgfsetdash{}{0pt}%
\pgfpathmoveto{\pgfqpoint{1.866475in}{2.550901in}}%
\pgfpathlineto{\pgfqpoint{1.875412in}{2.550901in}}%
\pgfpathlineto{\pgfqpoint{1.875412in}{2.557023in}}%
\pgfpathlineto{\pgfqpoint{1.866475in}{2.557023in}}%
\pgfpathlineto{\pgfqpoint{1.866475in}{2.550901in}}%
\pgfpathclose%
\pgfusepath{fill}%
\end{pgfscope}%
\begin{pgfscope}%
\pgfpathrectangle{\pgfqpoint{0.697024in}{0.857143in}}{\pgfqpoint{2.627103in}{1.813434in}}%
\pgfusepath{clip}%
\pgfsetbuttcap%
\pgfsetmiterjoin%
\definecolor{currentfill}{rgb}{0.754268,0.565033,0.211761}%
\pgfsetfillcolor{currentfill}%
\pgfsetlinewidth{0.000000pt}%
\definecolor{currentstroke}{rgb}{0.000000,0.000000,0.000000}%
\pgfsetstrokecolor{currentstroke}%
\pgfsetstrokeopacity{0.000000}%
\pgfsetdash{}{0pt}%
\pgfpathmoveto{\pgfqpoint{1.877646in}{2.527902in}}%
\pgfpathlineto{\pgfqpoint{1.886582in}{2.527902in}}%
\pgfpathlineto{\pgfqpoint{1.886582in}{2.531505in}}%
\pgfpathlineto{\pgfqpoint{1.877646in}{2.531505in}}%
\pgfpathlineto{\pgfqpoint{1.877646in}{2.527902in}}%
\pgfpathclose%
\pgfusepath{fill}%
\end{pgfscope}%
\begin{pgfscope}%
\pgfpathrectangle{\pgfqpoint{0.697024in}{0.857143in}}{\pgfqpoint{2.627103in}{1.813434in}}%
\pgfusepath{clip}%
\pgfsetbuttcap%
\pgfsetmiterjoin%
\definecolor{currentfill}{rgb}{0.754268,0.565033,0.211761}%
\pgfsetfillcolor{currentfill}%
\pgfsetlinewidth{0.000000pt}%
\definecolor{currentstroke}{rgb}{0.000000,0.000000,0.000000}%
\pgfsetstrokecolor{currentstroke}%
\pgfsetstrokeopacity{0.000000}%
\pgfsetdash{}{0pt}%
\pgfpathmoveto{\pgfqpoint{1.888816in}{1.478087in}}%
\pgfpathlineto{\pgfqpoint{1.897753in}{1.478087in}}%
\pgfpathlineto{\pgfqpoint{1.897753in}{1.445935in}}%
\pgfpathlineto{\pgfqpoint{1.888816in}{1.445935in}}%
\pgfpathlineto{\pgfqpoint{1.888816in}{1.478087in}}%
\pgfpathclose%
\pgfusepath{fill}%
\end{pgfscope}%
\begin{pgfscope}%
\pgfpathrectangle{\pgfqpoint{0.697024in}{0.857143in}}{\pgfqpoint{2.627103in}{1.813434in}}%
\pgfusepath{clip}%
\pgfsetbuttcap%
\pgfsetmiterjoin%
\definecolor{currentfill}{rgb}{0.754268,0.565033,0.211761}%
\pgfsetfillcolor{currentfill}%
\pgfsetlinewidth{0.000000pt}%
\definecolor{currentstroke}{rgb}{0.000000,0.000000,0.000000}%
\pgfsetstrokecolor{currentstroke}%
\pgfsetstrokeopacity{0.000000}%
\pgfsetdash{}{0pt}%
\pgfpathmoveto{\pgfqpoint{1.899987in}{1.455278in}}%
\pgfpathlineto{\pgfqpoint{1.908923in}{1.455278in}}%
\pgfpathlineto{\pgfqpoint{1.908923in}{1.412176in}}%
\pgfpathlineto{\pgfqpoint{1.899987in}{1.412176in}}%
\pgfpathlineto{\pgfqpoint{1.899987in}{1.455278in}}%
\pgfpathclose%
\pgfusepath{fill}%
\end{pgfscope}%
\begin{pgfscope}%
\pgfpathrectangle{\pgfqpoint{0.697024in}{0.857143in}}{\pgfqpoint{2.627103in}{1.813434in}}%
\pgfusepath{clip}%
\pgfsetbuttcap%
\pgfsetmiterjoin%
\definecolor{currentfill}{rgb}{0.754268,0.565033,0.211761}%
\pgfsetfillcolor{currentfill}%
\pgfsetlinewidth{0.000000pt}%
\definecolor{currentstroke}{rgb}{0.000000,0.000000,0.000000}%
\pgfsetstrokecolor{currentstroke}%
\pgfsetstrokeopacity{0.000000}%
\pgfsetdash{}{0pt}%
\pgfpathmoveto{\pgfqpoint{1.911157in}{2.386845in}}%
\pgfpathlineto{\pgfqpoint{1.920094in}{2.386845in}}%
\pgfpathlineto{\pgfqpoint{1.920094in}{2.403472in}}%
\pgfpathlineto{\pgfqpoint{1.911157in}{2.403472in}}%
\pgfpathlineto{\pgfqpoint{1.911157in}{2.386845in}}%
\pgfpathclose%
\pgfusepath{fill}%
\end{pgfscope}%
\begin{pgfscope}%
\pgfpathrectangle{\pgfqpoint{0.697024in}{0.857143in}}{\pgfqpoint{2.627103in}{1.813434in}}%
\pgfusepath{clip}%
\pgfsetbuttcap%
\pgfsetmiterjoin%
\definecolor{currentfill}{rgb}{0.754268,0.565033,0.211761}%
\pgfsetfillcolor{currentfill}%
\pgfsetlinewidth{0.000000pt}%
\definecolor{currentstroke}{rgb}{0.000000,0.000000,0.000000}%
\pgfsetstrokecolor{currentstroke}%
\pgfsetstrokeopacity{0.000000}%
\pgfsetdash{}{0pt}%
\pgfpathmoveto{\pgfqpoint{1.922328in}{2.337795in}}%
\pgfpathlineto{\pgfqpoint{1.931265in}{2.337795in}}%
\pgfpathlineto{\pgfqpoint{1.931265in}{2.337884in}}%
\pgfpathlineto{\pgfqpoint{1.922328in}{2.337884in}}%
\pgfpathlineto{\pgfqpoint{1.922328in}{2.337795in}}%
\pgfpathclose%
\pgfusepath{fill}%
\end{pgfscope}%
\begin{pgfscope}%
\pgfpathrectangle{\pgfqpoint{0.697024in}{0.857143in}}{\pgfqpoint{2.627103in}{1.813434in}}%
\pgfusepath{clip}%
\pgfsetbuttcap%
\pgfsetmiterjoin%
\definecolor{currentfill}{rgb}{0.754268,0.565033,0.211761}%
\pgfsetfillcolor{currentfill}%
\pgfsetlinewidth{0.000000pt}%
\definecolor{currentstroke}{rgb}{0.000000,0.000000,0.000000}%
\pgfsetstrokecolor{currentstroke}%
\pgfsetstrokeopacity{0.000000}%
\pgfsetdash{}{0pt}%
\pgfpathmoveto{\pgfqpoint{1.933499in}{2.332237in}}%
\pgfpathlineto{\pgfqpoint{1.942435in}{2.332237in}}%
\pgfpathlineto{\pgfqpoint{1.942435in}{2.338974in}}%
\pgfpathlineto{\pgfqpoint{1.933499in}{2.338974in}}%
\pgfpathlineto{\pgfqpoint{1.933499in}{2.332237in}}%
\pgfpathclose%
\pgfusepath{fill}%
\end{pgfscope}%
\begin{pgfscope}%
\pgfpathrectangle{\pgfqpoint{0.697024in}{0.857143in}}{\pgfqpoint{2.627103in}{1.813434in}}%
\pgfusepath{clip}%
\pgfsetbuttcap%
\pgfsetmiterjoin%
\definecolor{currentfill}{rgb}{0.754268,0.565033,0.211761}%
\pgfsetfillcolor{currentfill}%
\pgfsetlinewidth{0.000000pt}%
\definecolor{currentstroke}{rgb}{0.000000,0.000000,0.000000}%
\pgfsetstrokecolor{currentstroke}%
\pgfsetstrokeopacity{0.000000}%
\pgfsetdash{}{0pt}%
\pgfpathmoveto{\pgfqpoint{1.944669in}{1.366518in}}%
\pgfpathlineto{\pgfqpoint{1.953606in}{1.366518in}}%
\pgfpathlineto{\pgfqpoint{1.953606in}{1.359730in}}%
\pgfpathlineto{\pgfqpoint{1.944669in}{1.359730in}}%
\pgfpathlineto{\pgfqpoint{1.944669in}{1.366518in}}%
\pgfpathclose%
\pgfusepath{fill}%
\end{pgfscope}%
\begin{pgfscope}%
\pgfpathrectangle{\pgfqpoint{0.697024in}{0.857143in}}{\pgfqpoint{2.627103in}{1.813434in}}%
\pgfusepath{clip}%
\pgfsetbuttcap%
\pgfsetmiterjoin%
\definecolor{currentfill}{rgb}{0.754268,0.565033,0.211761}%
\pgfsetfillcolor{currentfill}%
\pgfsetlinewidth{0.000000pt}%
\definecolor{currentstroke}{rgb}{0.000000,0.000000,0.000000}%
\pgfsetstrokecolor{currentstroke}%
\pgfsetstrokeopacity{0.000000}%
\pgfsetdash{}{0pt}%
\pgfpathmoveto{\pgfqpoint{1.955840in}{1.427387in}}%
\pgfpathlineto{\pgfqpoint{1.964776in}{1.427387in}}%
\pgfpathlineto{\pgfqpoint{1.964776in}{1.417806in}}%
\pgfpathlineto{\pgfqpoint{1.955840in}{1.417806in}}%
\pgfpathlineto{\pgfqpoint{1.955840in}{1.427387in}}%
\pgfpathclose%
\pgfusepath{fill}%
\end{pgfscope}%
\begin{pgfscope}%
\pgfpathrectangle{\pgfqpoint{0.697024in}{0.857143in}}{\pgfqpoint{2.627103in}{1.813434in}}%
\pgfusepath{clip}%
\pgfsetbuttcap%
\pgfsetmiterjoin%
\definecolor{currentfill}{rgb}{0.754268,0.565033,0.211761}%
\pgfsetfillcolor{currentfill}%
\pgfsetlinewidth{0.000000pt}%
\definecolor{currentstroke}{rgb}{0.000000,0.000000,0.000000}%
\pgfsetstrokecolor{currentstroke}%
\pgfsetstrokeopacity{0.000000}%
\pgfsetdash{}{0pt}%
\pgfpathmoveto{\pgfqpoint{1.967011in}{2.296482in}}%
\pgfpathlineto{\pgfqpoint{1.975947in}{2.296482in}}%
\pgfpathlineto{\pgfqpoint{1.975947in}{2.323811in}}%
\pgfpathlineto{\pgfqpoint{1.967011in}{2.323811in}}%
\pgfpathlineto{\pgfqpoint{1.967011in}{2.296482in}}%
\pgfpathclose%
\pgfusepath{fill}%
\end{pgfscope}%
\begin{pgfscope}%
\pgfpathrectangle{\pgfqpoint{0.697024in}{0.857143in}}{\pgfqpoint{2.627103in}{1.813434in}}%
\pgfusepath{clip}%
\pgfsetbuttcap%
\pgfsetmiterjoin%
\definecolor{currentfill}{rgb}{0.754268,0.565033,0.211761}%
\pgfsetfillcolor{currentfill}%
\pgfsetlinewidth{0.000000pt}%
\definecolor{currentstroke}{rgb}{0.000000,0.000000,0.000000}%
\pgfsetstrokecolor{currentstroke}%
\pgfsetstrokeopacity{0.000000}%
\pgfsetdash{}{0pt}%
\pgfpathmoveto{\pgfqpoint{1.978181in}{2.308989in}}%
\pgfpathlineto{\pgfqpoint{1.987118in}{2.308989in}}%
\pgfpathlineto{\pgfqpoint{1.987118in}{2.326929in}}%
\pgfpathlineto{\pgfqpoint{1.978181in}{2.326929in}}%
\pgfpathlineto{\pgfqpoint{1.978181in}{2.308989in}}%
\pgfpathclose%
\pgfusepath{fill}%
\end{pgfscope}%
\begin{pgfscope}%
\pgfpathrectangle{\pgfqpoint{0.697024in}{0.857143in}}{\pgfqpoint{2.627103in}{1.813434in}}%
\pgfusepath{clip}%
\pgfsetbuttcap%
\pgfsetmiterjoin%
\definecolor{currentfill}{rgb}{0.754268,0.565033,0.211761}%
\pgfsetfillcolor{currentfill}%
\pgfsetlinewidth{0.000000pt}%
\definecolor{currentstroke}{rgb}{0.000000,0.000000,0.000000}%
\pgfsetstrokecolor{currentstroke}%
\pgfsetstrokeopacity{0.000000}%
\pgfsetdash{}{0pt}%
\pgfpathmoveto{\pgfqpoint{1.989352in}{2.333588in}}%
\pgfpathlineto{\pgfqpoint{1.998288in}{2.333588in}}%
\pgfpathlineto{\pgfqpoint{1.998288in}{2.378473in}}%
\pgfpathlineto{\pgfqpoint{1.989352in}{2.378473in}}%
\pgfpathlineto{\pgfqpoint{1.989352in}{2.333588in}}%
\pgfpathclose%
\pgfusepath{fill}%
\end{pgfscope}%
\begin{pgfscope}%
\pgfpathrectangle{\pgfqpoint{0.697024in}{0.857143in}}{\pgfqpoint{2.627103in}{1.813434in}}%
\pgfusepath{clip}%
\pgfsetbuttcap%
\pgfsetmiterjoin%
\definecolor{currentfill}{rgb}{0.754268,0.565033,0.211761}%
\pgfsetfillcolor{currentfill}%
\pgfsetlinewidth{0.000000pt}%
\definecolor{currentstroke}{rgb}{0.000000,0.000000,0.000000}%
\pgfsetstrokecolor{currentstroke}%
\pgfsetstrokeopacity{0.000000}%
\pgfsetdash{}{0pt}%
\pgfpathmoveto{\pgfqpoint{2.000522in}{2.322637in}}%
\pgfpathlineto{\pgfqpoint{2.009459in}{2.322637in}}%
\pgfpathlineto{\pgfqpoint{2.009459in}{2.347268in}}%
\pgfpathlineto{\pgfqpoint{2.000522in}{2.347268in}}%
\pgfpathlineto{\pgfqpoint{2.000522in}{2.322637in}}%
\pgfpathclose%
\pgfusepath{fill}%
\end{pgfscope}%
\begin{pgfscope}%
\pgfpathrectangle{\pgfqpoint{0.697024in}{0.857143in}}{\pgfqpoint{2.627103in}{1.813434in}}%
\pgfusepath{clip}%
\pgfsetbuttcap%
\pgfsetmiterjoin%
\definecolor{currentfill}{rgb}{0.754268,0.565033,0.211761}%
\pgfsetfillcolor{currentfill}%
\pgfsetlinewidth{0.000000pt}%
\definecolor{currentstroke}{rgb}{0.000000,0.000000,0.000000}%
\pgfsetstrokecolor{currentstroke}%
\pgfsetstrokeopacity{0.000000}%
\pgfsetdash{}{0pt}%
\pgfpathmoveto{\pgfqpoint{2.011693in}{2.336750in}}%
\pgfpathlineto{\pgfqpoint{2.020629in}{2.336750in}}%
\pgfpathlineto{\pgfqpoint{2.020629in}{2.366153in}}%
\pgfpathlineto{\pgfqpoint{2.011693in}{2.366153in}}%
\pgfpathlineto{\pgfqpoint{2.011693in}{2.336750in}}%
\pgfpathclose%
\pgfusepath{fill}%
\end{pgfscope}%
\begin{pgfscope}%
\pgfpathrectangle{\pgfqpoint{0.697024in}{0.857143in}}{\pgfqpoint{2.627103in}{1.813434in}}%
\pgfusepath{clip}%
\pgfsetbuttcap%
\pgfsetmiterjoin%
\definecolor{currentfill}{rgb}{0.754268,0.565033,0.211761}%
\pgfsetfillcolor{currentfill}%
\pgfsetlinewidth{0.000000pt}%
\definecolor{currentstroke}{rgb}{0.000000,0.000000,0.000000}%
\pgfsetstrokecolor{currentstroke}%
\pgfsetstrokeopacity{0.000000}%
\pgfsetdash{}{0pt}%
\pgfpathmoveto{\pgfqpoint{2.022864in}{2.367532in}}%
\pgfpathlineto{\pgfqpoint{2.031800in}{2.367532in}}%
\pgfpathlineto{\pgfqpoint{2.031800in}{2.392575in}}%
\pgfpathlineto{\pgfqpoint{2.022864in}{2.392575in}}%
\pgfpathlineto{\pgfqpoint{2.022864in}{2.367532in}}%
\pgfpathclose%
\pgfusepath{fill}%
\end{pgfscope}%
\begin{pgfscope}%
\pgfpathrectangle{\pgfqpoint{0.697024in}{0.857143in}}{\pgfqpoint{2.627103in}{1.813434in}}%
\pgfusepath{clip}%
\pgfsetbuttcap%
\pgfsetmiterjoin%
\definecolor{currentfill}{rgb}{0.754268,0.565033,0.211761}%
\pgfsetfillcolor{currentfill}%
\pgfsetlinewidth{0.000000pt}%
\definecolor{currentstroke}{rgb}{0.000000,0.000000,0.000000}%
\pgfsetstrokecolor{currentstroke}%
\pgfsetstrokeopacity{0.000000}%
\pgfsetdash{}{0pt}%
\pgfpathmoveto{\pgfqpoint{2.034034in}{2.348993in}}%
\pgfpathlineto{\pgfqpoint{2.042971in}{2.348993in}}%
\pgfpathlineto{\pgfqpoint{2.042971in}{2.368110in}}%
\pgfpathlineto{\pgfqpoint{2.034034in}{2.368110in}}%
\pgfpathlineto{\pgfqpoint{2.034034in}{2.348993in}}%
\pgfpathclose%
\pgfusepath{fill}%
\end{pgfscope}%
\begin{pgfscope}%
\pgfpathrectangle{\pgfqpoint{0.697024in}{0.857143in}}{\pgfqpoint{2.627103in}{1.813434in}}%
\pgfusepath{clip}%
\pgfsetbuttcap%
\pgfsetmiterjoin%
\definecolor{currentfill}{rgb}{0.754268,0.565033,0.211761}%
\pgfsetfillcolor{currentfill}%
\pgfsetlinewidth{0.000000pt}%
\definecolor{currentstroke}{rgb}{0.000000,0.000000,0.000000}%
\pgfsetstrokecolor{currentstroke}%
\pgfsetstrokeopacity{0.000000}%
\pgfsetdash{}{0pt}%
\pgfpathmoveto{\pgfqpoint{2.045205in}{2.346516in}}%
\pgfpathlineto{\pgfqpoint{2.054141in}{2.346516in}}%
\pgfpathlineto{\pgfqpoint{2.054141in}{2.355274in}}%
\pgfpathlineto{\pgfqpoint{2.045205in}{2.355274in}}%
\pgfpathlineto{\pgfqpoint{2.045205in}{2.346516in}}%
\pgfpathclose%
\pgfusepath{fill}%
\end{pgfscope}%
\begin{pgfscope}%
\pgfpathrectangle{\pgfqpoint{0.697024in}{0.857143in}}{\pgfqpoint{2.627103in}{1.813434in}}%
\pgfusepath{clip}%
\pgfsetbuttcap%
\pgfsetmiterjoin%
\definecolor{currentfill}{rgb}{0.754268,0.565033,0.211761}%
\pgfsetfillcolor{currentfill}%
\pgfsetlinewidth{0.000000pt}%
\definecolor{currentstroke}{rgb}{0.000000,0.000000,0.000000}%
\pgfsetstrokecolor{currentstroke}%
\pgfsetstrokeopacity{0.000000}%
\pgfsetdash{}{0pt}%
\pgfpathmoveto{\pgfqpoint{2.056375in}{1.327622in}}%
\pgfpathlineto{\pgfqpoint{2.065312in}{1.327622in}}%
\pgfpathlineto{\pgfqpoint{2.065312in}{1.326463in}}%
\pgfpathlineto{\pgfqpoint{2.056375in}{1.326463in}}%
\pgfpathlineto{\pgfqpoint{2.056375in}{1.327622in}}%
\pgfpathclose%
\pgfusepath{fill}%
\end{pgfscope}%
\begin{pgfscope}%
\pgfpathrectangle{\pgfqpoint{0.697024in}{0.857143in}}{\pgfqpoint{2.627103in}{1.813434in}}%
\pgfusepath{clip}%
\pgfsetbuttcap%
\pgfsetmiterjoin%
\definecolor{currentfill}{rgb}{0.754268,0.565033,0.211761}%
\pgfsetfillcolor{currentfill}%
\pgfsetlinewidth{0.000000pt}%
\definecolor{currentstroke}{rgb}{0.000000,0.000000,0.000000}%
\pgfsetstrokecolor{currentstroke}%
\pgfsetstrokeopacity{0.000000}%
\pgfsetdash{}{0pt}%
\pgfpathmoveto{\pgfqpoint{2.067546in}{1.310472in}}%
\pgfpathlineto{\pgfqpoint{2.076482in}{1.310472in}}%
\pgfpathlineto{\pgfqpoint{2.076482in}{1.307319in}}%
\pgfpathlineto{\pgfqpoint{2.067546in}{1.307319in}}%
\pgfpathlineto{\pgfqpoint{2.067546in}{1.310472in}}%
\pgfpathclose%
\pgfusepath{fill}%
\end{pgfscope}%
\begin{pgfscope}%
\pgfpathrectangle{\pgfqpoint{0.697024in}{0.857143in}}{\pgfqpoint{2.627103in}{1.813434in}}%
\pgfusepath{clip}%
\pgfsetbuttcap%
\pgfsetmiterjoin%
\definecolor{currentfill}{rgb}{0.754268,0.565033,0.211761}%
\pgfsetfillcolor{currentfill}%
\pgfsetlinewidth{0.000000pt}%
\definecolor{currentstroke}{rgb}{0.000000,0.000000,0.000000}%
\pgfsetstrokecolor{currentstroke}%
\pgfsetstrokeopacity{0.000000}%
\pgfsetdash{}{0pt}%
\pgfpathmoveto{\pgfqpoint{2.078717in}{1.329628in}}%
\pgfpathlineto{\pgfqpoint{2.087653in}{1.329628in}}%
\pgfpathlineto{\pgfqpoint{2.087653in}{1.312225in}}%
\pgfpathlineto{\pgfqpoint{2.078717in}{1.312225in}}%
\pgfpathlineto{\pgfqpoint{2.078717in}{1.329628in}}%
\pgfpathclose%
\pgfusepath{fill}%
\end{pgfscope}%
\begin{pgfscope}%
\pgfpathrectangle{\pgfqpoint{0.697024in}{0.857143in}}{\pgfqpoint{2.627103in}{1.813434in}}%
\pgfusepath{clip}%
\pgfsetbuttcap%
\pgfsetmiterjoin%
\definecolor{currentfill}{rgb}{0.754268,0.565033,0.211761}%
\pgfsetfillcolor{currentfill}%
\pgfsetlinewidth{0.000000pt}%
\definecolor{currentstroke}{rgb}{0.000000,0.000000,0.000000}%
\pgfsetstrokecolor{currentstroke}%
\pgfsetstrokeopacity{0.000000}%
\pgfsetdash{}{0pt}%
\pgfpathmoveto{\pgfqpoint{2.089887in}{1.317791in}}%
\pgfpathlineto{\pgfqpoint{2.098824in}{1.317791in}}%
\pgfpathlineto{\pgfqpoint{2.098824in}{1.286134in}}%
\pgfpathlineto{\pgfqpoint{2.089887in}{1.286134in}}%
\pgfpathlineto{\pgfqpoint{2.089887in}{1.317791in}}%
\pgfpathclose%
\pgfusepath{fill}%
\end{pgfscope}%
\begin{pgfscope}%
\pgfpathrectangle{\pgfqpoint{0.697024in}{0.857143in}}{\pgfqpoint{2.627103in}{1.813434in}}%
\pgfusepath{clip}%
\pgfsetbuttcap%
\pgfsetmiterjoin%
\definecolor{currentfill}{rgb}{0.754268,0.565033,0.211761}%
\pgfsetfillcolor{currentfill}%
\pgfsetlinewidth{0.000000pt}%
\definecolor{currentstroke}{rgb}{0.000000,0.000000,0.000000}%
\pgfsetstrokecolor{currentstroke}%
\pgfsetstrokeopacity{0.000000}%
\pgfsetdash{}{0pt}%
\pgfpathmoveto{\pgfqpoint{2.101058in}{1.351299in}}%
\pgfpathlineto{\pgfqpoint{2.109994in}{1.351299in}}%
\pgfpathlineto{\pgfqpoint{2.109994in}{1.303423in}}%
\pgfpathlineto{\pgfqpoint{2.101058in}{1.303423in}}%
\pgfpathlineto{\pgfqpoint{2.101058in}{1.351299in}}%
\pgfpathclose%
\pgfusepath{fill}%
\end{pgfscope}%
\begin{pgfscope}%
\pgfpathrectangle{\pgfqpoint{0.697024in}{0.857143in}}{\pgfqpoint{2.627103in}{1.813434in}}%
\pgfusepath{clip}%
\pgfsetbuttcap%
\pgfsetmiterjoin%
\definecolor{currentfill}{rgb}{0.754268,0.565033,0.211761}%
\pgfsetfillcolor{currentfill}%
\pgfsetlinewidth{0.000000pt}%
\definecolor{currentstroke}{rgb}{0.000000,0.000000,0.000000}%
\pgfsetstrokecolor{currentstroke}%
\pgfsetstrokeopacity{0.000000}%
\pgfsetdash{}{0pt}%
\pgfpathmoveto{\pgfqpoint{2.112228in}{1.389882in}}%
\pgfpathlineto{\pgfqpoint{2.121165in}{1.389882in}}%
\pgfpathlineto{\pgfqpoint{2.121165in}{1.352449in}}%
\pgfpathlineto{\pgfqpoint{2.112228in}{1.352449in}}%
\pgfpathlineto{\pgfqpoint{2.112228in}{1.389882in}}%
\pgfpathclose%
\pgfusepath{fill}%
\end{pgfscope}%
\begin{pgfscope}%
\pgfpathrectangle{\pgfqpoint{0.697024in}{0.857143in}}{\pgfqpoint{2.627103in}{1.813434in}}%
\pgfusepath{clip}%
\pgfsetbuttcap%
\pgfsetmiterjoin%
\definecolor{currentfill}{rgb}{0.754268,0.565033,0.211761}%
\pgfsetfillcolor{currentfill}%
\pgfsetlinewidth{0.000000pt}%
\definecolor{currentstroke}{rgb}{0.000000,0.000000,0.000000}%
\pgfsetstrokecolor{currentstroke}%
\pgfsetstrokeopacity{0.000000}%
\pgfsetdash{}{0pt}%
\pgfpathmoveto{\pgfqpoint{2.123399in}{1.377083in}}%
\pgfpathlineto{\pgfqpoint{2.132335in}{1.377083in}}%
\pgfpathlineto{\pgfqpoint{2.132335in}{1.341601in}}%
\pgfpathlineto{\pgfqpoint{2.123399in}{1.341601in}}%
\pgfpathlineto{\pgfqpoint{2.123399in}{1.377083in}}%
\pgfpathclose%
\pgfusepath{fill}%
\end{pgfscope}%
\begin{pgfscope}%
\pgfpathrectangle{\pgfqpoint{0.697024in}{0.857143in}}{\pgfqpoint{2.627103in}{1.813434in}}%
\pgfusepath{clip}%
\pgfsetbuttcap%
\pgfsetmiterjoin%
\definecolor{currentfill}{rgb}{0.754268,0.565033,0.211761}%
\pgfsetfillcolor{currentfill}%
\pgfsetlinewidth{0.000000pt}%
\definecolor{currentstroke}{rgb}{0.000000,0.000000,0.000000}%
\pgfsetstrokecolor{currentstroke}%
\pgfsetstrokeopacity{0.000000}%
\pgfsetdash{}{0pt}%
\pgfpathmoveto{\pgfqpoint{2.134570in}{1.395204in}}%
\pgfpathlineto{\pgfqpoint{2.143506in}{1.395204in}}%
\pgfpathlineto{\pgfqpoint{2.143506in}{1.364951in}}%
\pgfpathlineto{\pgfqpoint{2.134570in}{1.364951in}}%
\pgfpathlineto{\pgfqpoint{2.134570in}{1.395204in}}%
\pgfpathclose%
\pgfusepath{fill}%
\end{pgfscope}%
\begin{pgfscope}%
\pgfpathrectangle{\pgfqpoint{0.697024in}{0.857143in}}{\pgfqpoint{2.627103in}{1.813434in}}%
\pgfusepath{clip}%
\pgfsetbuttcap%
\pgfsetmiterjoin%
\definecolor{currentfill}{rgb}{0.754268,0.565033,0.211761}%
\pgfsetfillcolor{currentfill}%
\pgfsetlinewidth{0.000000pt}%
\definecolor{currentstroke}{rgb}{0.000000,0.000000,0.000000}%
\pgfsetstrokecolor{currentstroke}%
\pgfsetstrokeopacity{0.000000}%
\pgfsetdash{}{0pt}%
\pgfpathmoveto{\pgfqpoint{2.145740in}{1.428593in}}%
\pgfpathlineto{\pgfqpoint{2.154677in}{1.428593in}}%
\pgfpathlineto{\pgfqpoint{2.154677in}{1.397383in}}%
\pgfpathlineto{\pgfqpoint{2.145740in}{1.397383in}}%
\pgfpathlineto{\pgfqpoint{2.145740in}{1.428593in}}%
\pgfpathclose%
\pgfusepath{fill}%
\end{pgfscope}%
\begin{pgfscope}%
\pgfpathrectangle{\pgfqpoint{0.697024in}{0.857143in}}{\pgfqpoint{2.627103in}{1.813434in}}%
\pgfusepath{clip}%
\pgfsetbuttcap%
\pgfsetmiterjoin%
\definecolor{currentfill}{rgb}{0.754268,0.565033,0.211761}%
\pgfsetfillcolor{currentfill}%
\pgfsetlinewidth{0.000000pt}%
\definecolor{currentstroke}{rgb}{0.000000,0.000000,0.000000}%
\pgfsetstrokecolor{currentstroke}%
\pgfsetstrokeopacity{0.000000}%
\pgfsetdash{}{0pt}%
\pgfpathmoveto{\pgfqpoint{2.156911in}{1.472447in}}%
\pgfpathlineto{\pgfqpoint{2.165847in}{1.472447in}}%
\pgfpathlineto{\pgfqpoint{2.165847in}{1.426518in}}%
\pgfpathlineto{\pgfqpoint{2.156911in}{1.426518in}}%
\pgfpathlineto{\pgfqpoint{2.156911in}{1.472447in}}%
\pgfpathclose%
\pgfusepath{fill}%
\end{pgfscope}%
\begin{pgfscope}%
\pgfpathrectangle{\pgfqpoint{0.697024in}{0.857143in}}{\pgfqpoint{2.627103in}{1.813434in}}%
\pgfusepath{clip}%
\pgfsetbuttcap%
\pgfsetmiterjoin%
\definecolor{currentfill}{rgb}{0.754268,0.565033,0.211761}%
\pgfsetfillcolor{currentfill}%
\pgfsetlinewidth{0.000000pt}%
\definecolor{currentstroke}{rgb}{0.000000,0.000000,0.000000}%
\pgfsetstrokecolor{currentstroke}%
\pgfsetstrokeopacity{0.000000}%
\pgfsetdash{}{0pt}%
\pgfpathmoveto{\pgfqpoint{2.168081in}{1.424961in}}%
\pgfpathlineto{\pgfqpoint{2.177018in}{1.424961in}}%
\pgfpathlineto{\pgfqpoint{2.177018in}{1.412059in}}%
\pgfpathlineto{\pgfqpoint{2.168081in}{1.412059in}}%
\pgfpathlineto{\pgfqpoint{2.168081in}{1.424961in}}%
\pgfpathclose%
\pgfusepath{fill}%
\end{pgfscope}%
\begin{pgfscope}%
\pgfpathrectangle{\pgfqpoint{0.697024in}{0.857143in}}{\pgfqpoint{2.627103in}{1.813434in}}%
\pgfusepath{clip}%
\pgfsetbuttcap%
\pgfsetmiterjoin%
\definecolor{currentfill}{rgb}{0.754268,0.565033,0.211761}%
\pgfsetfillcolor{currentfill}%
\pgfsetlinewidth{0.000000pt}%
\definecolor{currentstroke}{rgb}{0.000000,0.000000,0.000000}%
\pgfsetstrokecolor{currentstroke}%
\pgfsetstrokeopacity{0.000000}%
\pgfsetdash{}{0pt}%
\pgfpathmoveto{\pgfqpoint{2.179252in}{1.462273in}}%
\pgfpathlineto{\pgfqpoint{2.188189in}{1.462273in}}%
\pgfpathlineto{\pgfqpoint{2.188189in}{1.454816in}}%
\pgfpathlineto{\pgfqpoint{2.179252in}{1.454816in}}%
\pgfpathlineto{\pgfqpoint{2.179252in}{1.462273in}}%
\pgfpathclose%
\pgfusepath{fill}%
\end{pgfscope}%
\begin{pgfscope}%
\pgfpathrectangle{\pgfqpoint{0.697024in}{0.857143in}}{\pgfqpoint{2.627103in}{1.813434in}}%
\pgfusepath{clip}%
\pgfsetbuttcap%
\pgfsetmiterjoin%
\definecolor{currentfill}{rgb}{0.754268,0.565033,0.211761}%
\pgfsetfillcolor{currentfill}%
\pgfsetlinewidth{0.000000pt}%
\definecolor{currentstroke}{rgb}{0.000000,0.000000,0.000000}%
\pgfsetstrokecolor{currentstroke}%
\pgfsetstrokeopacity{0.000000}%
\pgfsetdash{}{0pt}%
\pgfpathmoveto{\pgfqpoint{2.190423in}{1.577718in}}%
\pgfpathlineto{\pgfqpoint{2.199359in}{1.577718in}}%
\pgfpathlineto{\pgfqpoint{2.199359in}{1.510320in}}%
\pgfpathlineto{\pgfqpoint{2.190423in}{1.510320in}}%
\pgfpathlineto{\pgfqpoint{2.190423in}{1.577718in}}%
\pgfpathclose%
\pgfusepath{fill}%
\end{pgfscope}%
\begin{pgfscope}%
\pgfpathrectangle{\pgfqpoint{0.697024in}{0.857143in}}{\pgfqpoint{2.627103in}{1.813434in}}%
\pgfusepath{clip}%
\pgfsetbuttcap%
\pgfsetmiterjoin%
\definecolor{currentfill}{rgb}{0.754268,0.565033,0.211761}%
\pgfsetfillcolor{currentfill}%
\pgfsetlinewidth{0.000000pt}%
\definecolor{currentstroke}{rgb}{0.000000,0.000000,0.000000}%
\pgfsetstrokecolor{currentstroke}%
\pgfsetstrokeopacity{0.000000}%
\pgfsetdash{}{0pt}%
\pgfpathmoveto{\pgfqpoint{2.201593in}{1.554149in}}%
\pgfpathlineto{\pgfqpoint{2.210530in}{1.554149in}}%
\pgfpathlineto{\pgfqpoint{2.210530in}{1.524901in}}%
\pgfpathlineto{\pgfqpoint{2.201593in}{1.524901in}}%
\pgfpathlineto{\pgfqpoint{2.201593in}{1.554149in}}%
\pgfpathclose%
\pgfusepath{fill}%
\end{pgfscope}%
\begin{pgfscope}%
\pgfpathrectangle{\pgfqpoint{0.697024in}{0.857143in}}{\pgfqpoint{2.627103in}{1.813434in}}%
\pgfusepath{clip}%
\pgfsetbuttcap%
\pgfsetmiterjoin%
\definecolor{currentfill}{rgb}{0.754268,0.565033,0.211761}%
\pgfsetfillcolor{currentfill}%
\pgfsetlinewidth{0.000000pt}%
\definecolor{currentstroke}{rgb}{0.000000,0.000000,0.000000}%
\pgfsetstrokecolor{currentstroke}%
\pgfsetstrokeopacity{0.000000}%
\pgfsetdash{}{0pt}%
\pgfpathmoveto{\pgfqpoint{2.212764in}{1.647496in}}%
\pgfpathlineto{\pgfqpoint{2.221700in}{1.647496in}}%
\pgfpathlineto{\pgfqpoint{2.221700in}{1.604314in}}%
\pgfpathlineto{\pgfqpoint{2.212764in}{1.604314in}}%
\pgfpathlineto{\pgfqpoint{2.212764in}{1.647496in}}%
\pgfpathclose%
\pgfusepath{fill}%
\end{pgfscope}%
\begin{pgfscope}%
\pgfpathrectangle{\pgfqpoint{0.697024in}{0.857143in}}{\pgfqpoint{2.627103in}{1.813434in}}%
\pgfusepath{clip}%
\pgfsetbuttcap%
\pgfsetmiterjoin%
\definecolor{currentfill}{rgb}{0.754268,0.565033,0.211761}%
\pgfsetfillcolor{currentfill}%
\pgfsetlinewidth{0.000000pt}%
\definecolor{currentstroke}{rgb}{0.000000,0.000000,0.000000}%
\pgfsetstrokecolor{currentstroke}%
\pgfsetstrokeopacity{0.000000}%
\pgfsetdash{}{0pt}%
\pgfpathmoveto{\pgfqpoint{2.223934in}{1.701506in}}%
\pgfpathlineto{\pgfqpoint{2.232871in}{1.701506in}}%
\pgfpathlineto{\pgfqpoint{2.232871in}{1.633924in}}%
\pgfpathlineto{\pgfqpoint{2.223934in}{1.633924in}}%
\pgfpathlineto{\pgfqpoint{2.223934in}{1.701506in}}%
\pgfpathclose%
\pgfusepath{fill}%
\end{pgfscope}%
\begin{pgfscope}%
\pgfpathrectangle{\pgfqpoint{0.697024in}{0.857143in}}{\pgfqpoint{2.627103in}{1.813434in}}%
\pgfusepath{clip}%
\pgfsetbuttcap%
\pgfsetmiterjoin%
\definecolor{currentfill}{rgb}{0.754268,0.565033,0.211761}%
\pgfsetfillcolor{currentfill}%
\pgfsetlinewidth{0.000000pt}%
\definecolor{currentstroke}{rgb}{0.000000,0.000000,0.000000}%
\pgfsetstrokecolor{currentstroke}%
\pgfsetstrokeopacity{0.000000}%
\pgfsetdash{}{0pt}%
\pgfpathmoveto{\pgfqpoint{2.235105in}{1.701649in}}%
\pgfpathlineto{\pgfqpoint{2.244042in}{1.701649in}}%
\pgfpathlineto{\pgfqpoint{2.244042in}{1.653666in}}%
\pgfpathlineto{\pgfqpoint{2.235105in}{1.653666in}}%
\pgfpathlineto{\pgfqpoint{2.235105in}{1.701649in}}%
\pgfpathclose%
\pgfusepath{fill}%
\end{pgfscope}%
\begin{pgfscope}%
\pgfpathrectangle{\pgfqpoint{0.697024in}{0.857143in}}{\pgfqpoint{2.627103in}{1.813434in}}%
\pgfusepath{clip}%
\pgfsetbuttcap%
\pgfsetmiterjoin%
\definecolor{currentfill}{rgb}{0.754268,0.565033,0.211761}%
\pgfsetfillcolor{currentfill}%
\pgfsetlinewidth{0.000000pt}%
\definecolor{currentstroke}{rgb}{0.000000,0.000000,0.000000}%
\pgfsetstrokecolor{currentstroke}%
\pgfsetstrokeopacity{0.000000}%
\pgfsetdash{}{0pt}%
\pgfpathmoveto{\pgfqpoint{2.246276in}{1.703886in}}%
\pgfpathlineto{\pgfqpoint{2.255212in}{1.703886in}}%
\pgfpathlineto{\pgfqpoint{2.255212in}{1.678920in}}%
\pgfpathlineto{\pgfqpoint{2.246276in}{1.678920in}}%
\pgfpathlineto{\pgfqpoint{2.246276in}{1.703886in}}%
\pgfpathclose%
\pgfusepath{fill}%
\end{pgfscope}%
\begin{pgfscope}%
\pgfpathrectangle{\pgfqpoint{0.697024in}{0.857143in}}{\pgfqpoint{2.627103in}{1.813434in}}%
\pgfusepath{clip}%
\pgfsetbuttcap%
\pgfsetmiterjoin%
\definecolor{currentfill}{rgb}{0.754268,0.565033,0.211761}%
\pgfsetfillcolor{currentfill}%
\pgfsetlinewidth{0.000000pt}%
\definecolor{currentstroke}{rgb}{0.000000,0.000000,0.000000}%
\pgfsetstrokecolor{currentstroke}%
\pgfsetstrokeopacity{0.000000}%
\pgfsetdash{}{0pt}%
\pgfpathmoveto{\pgfqpoint{2.257446in}{1.723756in}}%
\pgfpathlineto{\pgfqpoint{2.266383in}{1.723756in}}%
\pgfpathlineto{\pgfqpoint{2.266383in}{1.688186in}}%
\pgfpathlineto{\pgfqpoint{2.257446in}{1.688186in}}%
\pgfpathlineto{\pgfqpoint{2.257446in}{1.723756in}}%
\pgfpathclose%
\pgfusepath{fill}%
\end{pgfscope}%
\begin{pgfscope}%
\pgfpathrectangle{\pgfqpoint{0.697024in}{0.857143in}}{\pgfqpoint{2.627103in}{1.813434in}}%
\pgfusepath{clip}%
\pgfsetbuttcap%
\pgfsetmiterjoin%
\definecolor{currentfill}{rgb}{0.754268,0.565033,0.211761}%
\pgfsetfillcolor{currentfill}%
\pgfsetlinewidth{0.000000pt}%
\definecolor{currentstroke}{rgb}{0.000000,0.000000,0.000000}%
\pgfsetstrokecolor{currentstroke}%
\pgfsetstrokeopacity{0.000000}%
\pgfsetdash{}{0pt}%
\pgfpathmoveto{\pgfqpoint{2.268617in}{1.743785in}}%
\pgfpathlineto{\pgfqpoint{2.277553in}{1.743785in}}%
\pgfpathlineto{\pgfqpoint{2.277553in}{1.723848in}}%
\pgfpathlineto{\pgfqpoint{2.268617in}{1.723848in}}%
\pgfpathlineto{\pgfqpoint{2.268617in}{1.743785in}}%
\pgfpathclose%
\pgfusepath{fill}%
\end{pgfscope}%
\begin{pgfscope}%
\pgfpathrectangle{\pgfqpoint{0.697024in}{0.857143in}}{\pgfqpoint{2.627103in}{1.813434in}}%
\pgfusepath{clip}%
\pgfsetbuttcap%
\pgfsetmiterjoin%
\definecolor{currentfill}{rgb}{0.754268,0.565033,0.211761}%
\pgfsetfillcolor{currentfill}%
\pgfsetlinewidth{0.000000pt}%
\definecolor{currentstroke}{rgb}{0.000000,0.000000,0.000000}%
\pgfsetstrokecolor{currentstroke}%
\pgfsetstrokeopacity{0.000000}%
\pgfsetdash{}{0pt}%
\pgfpathmoveto{\pgfqpoint{2.279787in}{1.736913in}}%
\pgfpathlineto{\pgfqpoint{2.288724in}{1.736913in}}%
\pgfpathlineto{\pgfqpoint{2.288724in}{1.719717in}}%
\pgfpathlineto{\pgfqpoint{2.279787in}{1.719717in}}%
\pgfpathlineto{\pgfqpoint{2.279787in}{1.736913in}}%
\pgfpathclose%
\pgfusepath{fill}%
\end{pgfscope}%
\begin{pgfscope}%
\pgfpathrectangle{\pgfqpoint{0.697024in}{0.857143in}}{\pgfqpoint{2.627103in}{1.813434in}}%
\pgfusepath{clip}%
\pgfsetbuttcap%
\pgfsetmiterjoin%
\definecolor{currentfill}{rgb}{0.754268,0.565033,0.211761}%
\pgfsetfillcolor{currentfill}%
\pgfsetlinewidth{0.000000pt}%
\definecolor{currentstroke}{rgb}{0.000000,0.000000,0.000000}%
\pgfsetstrokecolor{currentstroke}%
\pgfsetstrokeopacity{0.000000}%
\pgfsetdash{}{0pt}%
\pgfpathmoveto{\pgfqpoint{2.290958in}{1.748082in}}%
\pgfpathlineto{\pgfqpoint{2.299895in}{1.748082in}}%
\pgfpathlineto{\pgfqpoint{2.299895in}{1.721015in}}%
\pgfpathlineto{\pgfqpoint{2.290958in}{1.721015in}}%
\pgfpathlineto{\pgfqpoint{2.290958in}{1.748082in}}%
\pgfpathclose%
\pgfusepath{fill}%
\end{pgfscope}%
\begin{pgfscope}%
\pgfpathrectangle{\pgfqpoint{0.697024in}{0.857143in}}{\pgfqpoint{2.627103in}{1.813434in}}%
\pgfusepath{clip}%
\pgfsetbuttcap%
\pgfsetmiterjoin%
\definecolor{currentfill}{rgb}{0.754268,0.565033,0.211761}%
\pgfsetfillcolor{currentfill}%
\pgfsetlinewidth{0.000000pt}%
\definecolor{currentstroke}{rgb}{0.000000,0.000000,0.000000}%
\pgfsetstrokecolor{currentstroke}%
\pgfsetstrokeopacity{0.000000}%
\pgfsetdash{}{0pt}%
\pgfpathmoveto{\pgfqpoint{2.302129in}{1.771461in}}%
\pgfpathlineto{\pgfqpoint{2.311065in}{1.771461in}}%
\pgfpathlineto{\pgfqpoint{2.311065in}{1.766714in}}%
\pgfpathlineto{\pgfqpoint{2.302129in}{1.766714in}}%
\pgfpathlineto{\pgfqpoint{2.302129in}{1.771461in}}%
\pgfpathclose%
\pgfusepath{fill}%
\end{pgfscope}%
\begin{pgfscope}%
\pgfpathrectangle{\pgfqpoint{0.697024in}{0.857143in}}{\pgfqpoint{2.627103in}{1.813434in}}%
\pgfusepath{clip}%
\pgfsetbuttcap%
\pgfsetmiterjoin%
\definecolor{currentfill}{rgb}{0.754268,0.565033,0.211761}%
\pgfsetfillcolor{currentfill}%
\pgfsetlinewidth{0.000000pt}%
\definecolor{currentstroke}{rgb}{0.000000,0.000000,0.000000}%
\pgfsetstrokecolor{currentstroke}%
\pgfsetstrokeopacity{0.000000}%
\pgfsetdash{}{0pt}%
\pgfpathmoveto{\pgfqpoint{2.313299in}{2.428362in}}%
\pgfpathlineto{\pgfqpoint{2.322236in}{2.428362in}}%
\pgfpathlineto{\pgfqpoint{2.322236in}{2.429798in}}%
\pgfpathlineto{\pgfqpoint{2.313299in}{2.429798in}}%
\pgfpathlineto{\pgfqpoint{2.313299in}{2.428362in}}%
\pgfpathclose%
\pgfusepath{fill}%
\end{pgfscope}%
\begin{pgfscope}%
\pgfpathrectangle{\pgfqpoint{0.697024in}{0.857143in}}{\pgfqpoint{2.627103in}{1.813434in}}%
\pgfusepath{clip}%
\pgfsetbuttcap%
\pgfsetmiterjoin%
\definecolor{currentfill}{rgb}{0.754268,0.565033,0.211761}%
\pgfsetfillcolor{currentfill}%
\pgfsetlinewidth{0.000000pt}%
\definecolor{currentstroke}{rgb}{0.000000,0.000000,0.000000}%
\pgfsetstrokecolor{currentstroke}%
\pgfsetstrokeopacity{0.000000}%
\pgfsetdash{}{0pt}%
\pgfpathmoveto{\pgfqpoint{2.324470in}{1.788999in}}%
\pgfpathlineto{\pgfqpoint{2.333406in}{1.788999in}}%
\pgfpathlineto{\pgfqpoint{2.333406in}{1.770001in}}%
\pgfpathlineto{\pgfqpoint{2.324470in}{1.770001in}}%
\pgfpathlineto{\pgfqpoint{2.324470in}{1.788999in}}%
\pgfpathclose%
\pgfusepath{fill}%
\end{pgfscope}%
\begin{pgfscope}%
\pgfpathrectangle{\pgfqpoint{0.697024in}{0.857143in}}{\pgfqpoint{2.627103in}{1.813434in}}%
\pgfusepath{clip}%
\pgfsetbuttcap%
\pgfsetmiterjoin%
\definecolor{currentfill}{rgb}{0.754268,0.565033,0.211761}%
\pgfsetfillcolor{currentfill}%
\pgfsetlinewidth{0.000000pt}%
\definecolor{currentstroke}{rgb}{0.000000,0.000000,0.000000}%
\pgfsetstrokecolor{currentstroke}%
\pgfsetstrokeopacity{0.000000}%
\pgfsetdash{}{0pt}%
\pgfpathmoveto{\pgfqpoint{2.335640in}{1.775373in}}%
\pgfpathlineto{\pgfqpoint{2.344577in}{1.775373in}}%
\pgfpathlineto{\pgfqpoint{2.344577in}{1.741229in}}%
\pgfpathlineto{\pgfqpoint{2.335640in}{1.741229in}}%
\pgfpathlineto{\pgfqpoint{2.335640in}{1.775373in}}%
\pgfpathclose%
\pgfusepath{fill}%
\end{pgfscope}%
\begin{pgfscope}%
\pgfpathrectangle{\pgfqpoint{0.697024in}{0.857143in}}{\pgfqpoint{2.627103in}{1.813434in}}%
\pgfusepath{clip}%
\pgfsetbuttcap%
\pgfsetmiterjoin%
\definecolor{currentfill}{rgb}{0.754268,0.565033,0.211761}%
\pgfsetfillcolor{currentfill}%
\pgfsetlinewidth{0.000000pt}%
\definecolor{currentstroke}{rgb}{0.000000,0.000000,0.000000}%
\pgfsetstrokecolor{currentstroke}%
\pgfsetstrokeopacity{0.000000}%
\pgfsetdash{}{0pt}%
\pgfpathmoveto{\pgfqpoint{2.346811in}{1.792451in}}%
\pgfpathlineto{\pgfqpoint{2.355748in}{1.792451in}}%
\pgfpathlineto{\pgfqpoint{2.355748in}{1.755711in}}%
\pgfpathlineto{\pgfqpoint{2.346811in}{1.755711in}}%
\pgfpathlineto{\pgfqpoint{2.346811in}{1.792451in}}%
\pgfpathclose%
\pgfusepath{fill}%
\end{pgfscope}%
\begin{pgfscope}%
\pgfpathrectangle{\pgfqpoint{0.697024in}{0.857143in}}{\pgfqpoint{2.627103in}{1.813434in}}%
\pgfusepath{clip}%
\pgfsetbuttcap%
\pgfsetmiterjoin%
\definecolor{currentfill}{rgb}{0.754268,0.565033,0.211761}%
\pgfsetfillcolor{currentfill}%
\pgfsetlinewidth{0.000000pt}%
\definecolor{currentstroke}{rgb}{0.000000,0.000000,0.000000}%
\pgfsetstrokecolor{currentstroke}%
\pgfsetstrokeopacity{0.000000}%
\pgfsetdash{}{0pt}%
\pgfpathmoveto{\pgfqpoint{2.357982in}{1.770380in}}%
\pgfpathlineto{\pgfqpoint{2.366918in}{1.770380in}}%
\pgfpathlineto{\pgfqpoint{2.366918in}{1.763193in}}%
\pgfpathlineto{\pgfqpoint{2.357982in}{1.763193in}}%
\pgfpathlineto{\pgfqpoint{2.357982in}{1.770380in}}%
\pgfpathclose%
\pgfusepath{fill}%
\end{pgfscope}%
\begin{pgfscope}%
\pgfpathrectangle{\pgfqpoint{0.697024in}{0.857143in}}{\pgfqpoint{2.627103in}{1.813434in}}%
\pgfusepath{clip}%
\pgfsetbuttcap%
\pgfsetmiterjoin%
\definecolor{currentfill}{rgb}{0.754268,0.565033,0.211761}%
\pgfsetfillcolor{currentfill}%
\pgfsetlinewidth{0.000000pt}%
\definecolor{currentstroke}{rgb}{0.000000,0.000000,0.000000}%
\pgfsetstrokecolor{currentstroke}%
\pgfsetstrokeopacity{0.000000}%
\pgfsetdash{}{0pt}%
\pgfpathmoveto{\pgfqpoint{2.369152in}{2.402012in}}%
\pgfpathlineto{\pgfqpoint{2.378089in}{2.402012in}}%
\pgfpathlineto{\pgfqpoint{2.378089in}{2.409946in}}%
\pgfpathlineto{\pgfqpoint{2.369152in}{2.409946in}}%
\pgfpathlineto{\pgfqpoint{2.369152in}{2.402012in}}%
\pgfpathclose%
\pgfusepath{fill}%
\end{pgfscope}%
\begin{pgfscope}%
\pgfpathrectangle{\pgfqpoint{0.697024in}{0.857143in}}{\pgfqpoint{2.627103in}{1.813434in}}%
\pgfusepath{clip}%
\pgfsetbuttcap%
\pgfsetmiterjoin%
\definecolor{currentfill}{rgb}{0.754268,0.565033,0.211761}%
\pgfsetfillcolor{currentfill}%
\pgfsetlinewidth{0.000000pt}%
\definecolor{currentstroke}{rgb}{0.000000,0.000000,0.000000}%
\pgfsetstrokecolor{currentstroke}%
\pgfsetstrokeopacity{0.000000}%
\pgfsetdash{}{0pt}%
\pgfpathmoveto{\pgfqpoint{2.380323in}{1.814951in}}%
\pgfpathlineto{\pgfqpoint{2.389259in}{1.814951in}}%
\pgfpathlineto{\pgfqpoint{2.389259in}{1.814409in}}%
\pgfpathlineto{\pgfqpoint{2.380323in}{1.814409in}}%
\pgfpathlineto{\pgfqpoint{2.380323in}{1.814951in}}%
\pgfpathclose%
\pgfusepath{fill}%
\end{pgfscope}%
\begin{pgfscope}%
\pgfpathrectangle{\pgfqpoint{0.697024in}{0.857143in}}{\pgfqpoint{2.627103in}{1.813434in}}%
\pgfusepath{clip}%
\pgfsetbuttcap%
\pgfsetmiterjoin%
\definecolor{currentfill}{rgb}{0.754268,0.565033,0.211761}%
\pgfsetfillcolor{currentfill}%
\pgfsetlinewidth{0.000000pt}%
\definecolor{currentstroke}{rgb}{0.000000,0.000000,0.000000}%
\pgfsetstrokecolor{currentstroke}%
\pgfsetstrokeopacity{0.000000}%
\pgfsetdash{}{0pt}%
\pgfpathmoveto{\pgfqpoint{2.391494in}{2.304443in}}%
\pgfpathlineto{\pgfqpoint{2.400430in}{2.304443in}}%
\pgfpathlineto{\pgfqpoint{2.400430in}{2.320266in}}%
\pgfpathlineto{\pgfqpoint{2.391494in}{2.320266in}}%
\pgfpathlineto{\pgfqpoint{2.391494in}{2.304443in}}%
\pgfpathclose%
\pgfusepath{fill}%
\end{pgfscope}%
\begin{pgfscope}%
\pgfpathrectangle{\pgfqpoint{0.697024in}{0.857143in}}{\pgfqpoint{2.627103in}{1.813434in}}%
\pgfusepath{clip}%
\pgfsetbuttcap%
\pgfsetmiterjoin%
\definecolor{currentfill}{rgb}{0.754268,0.565033,0.211761}%
\pgfsetfillcolor{currentfill}%
\pgfsetlinewidth{0.000000pt}%
\definecolor{currentstroke}{rgb}{0.000000,0.000000,0.000000}%
\pgfsetstrokecolor{currentstroke}%
\pgfsetstrokeopacity{0.000000}%
\pgfsetdash{}{0pt}%
\pgfpathmoveto{\pgfqpoint{2.402664in}{2.225639in}}%
\pgfpathlineto{\pgfqpoint{2.411601in}{2.225639in}}%
\pgfpathlineto{\pgfqpoint{2.411601in}{2.254833in}}%
\pgfpathlineto{\pgfqpoint{2.402664in}{2.254833in}}%
\pgfpathlineto{\pgfqpoint{2.402664in}{2.225639in}}%
\pgfpathclose%
\pgfusepath{fill}%
\end{pgfscope}%
\begin{pgfscope}%
\pgfpathrectangle{\pgfqpoint{0.697024in}{0.857143in}}{\pgfqpoint{2.627103in}{1.813434in}}%
\pgfusepath{clip}%
\pgfsetbuttcap%
\pgfsetmiterjoin%
\definecolor{currentfill}{rgb}{0.754268,0.565033,0.211761}%
\pgfsetfillcolor{currentfill}%
\pgfsetlinewidth{0.000000pt}%
\definecolor{currentstroke}{rgb}{0.000000,0.000000,0.000000}%
\pgfsetstrokecolor{currentstroke}%
\pgfsetstrokeopacity{0.000000}%
\pgfsetdash{}{0pt}%
\pgfpathmoveto{\pgfqpoint{2.413835in}{2.227591in}}%
\pgfpathlineto{\pgfqpoint{2.422771in}{2.227591in}}%
\pgfpathlineto{\pgfqpoint{2.422771in}{2.256749in}}%
\pgfpathlineto{\pgfqpoint{2.413835in}{2.256749in}}%
\pgfpathlineto{\pgfqpoint{2.413835in}{2.227591in}}%
\pgfpathclose%
\pgfusepath{fill}%
\end{pgfscope}%
\begin{pgfscope}%
\pgfpathrectangle{\pgfqpoint{0.697024in}{0.857143in}}{\pgfqpoint{2.627103in}{1.813434in}}%
\pgfusepath{clip}%
\pgfsetbuttcap%
\pgfsetmiterjoin%
\definecolor{currentfill}{rgb}{0.754268,0.565033,0.211761}%
\pgfsetfillcolor{currentfill}%
\pgfsetlinewidth{0.000000pt}%
\definecolor{currentstroke}{rgb}{0.000000,0.000000,0.000000}%
\pgfsetstrokecolor{currentstroke}%
\pgfsetstrokeopacity{0.000000}%
\pgfsetdash{}{0pt}%
\pgfpathmoveto{\pgfqpoint{2.425005in}{2.200153in}}%
\pgfpathlineto{\pgfqpoint{2.433942in}{2.200153in}}%
\pgfpathlineto{\pgfqpoint{2.433942in}{2.243799in}}%
\pgfpathlineto{\pgfqpoint{2.425005in}{2.243799in}}%
\pgfpathlineto{\pgfqpoint{2.425005in}{2.200153in}}%
\pgfpathclose%
\pgfusepath{fill}%
\end{pgfscope}%
\begin{pgfscope}%
\pgfpathrectangle{\pgfqpoint{0.697024in}{0.857143in}}{\pgfqpoint{2.627103in}{1.813434in}}%
\pgfusepath{clip}%
\pgfsetbuttcap%
\pgfsetmiterjoin%
\definecolor{currentfill}{rgb}{0.754268,0.565033,0.211761}%
\pgfsetfillcolor{currentfill}%
\pgfsetlinewidth{0.000000pt}%
\definecolor{currentstroke}{rgb}{0.000000,0.000000,0.000000}%
\pgfsetstrokecolor{currentstroke}%
\pgfsetstrokeopacity{0.000000}%
\pgfsetdash{}{0pt}%
\pgfpathmoveto{\pgfqpoint{2.436176in}{2.136130in}}%
\pgfpathlineto{\pgfqpoint{2.445112in}{2.136130in}}%
\pgfpathlineto{\pgfqpoint{2.445112in}{2.188117in}}%
\pgfpathlineto{\pgfqpoint{2.436176in}{2.188117in}}%
\pgfpathlineto{\pgfqpoint{2.436176in}{2.136130in}}%
\pgfpathclose%
\pgfusepath{fill}%
\end{pgfscope}%
\begin{pgfscope}%
\pgfpathrectangle{\pgfqpoint{0.697024in}{0.857143in}}{\pgfqpoint{2.627103in}{1.813434in}}%
\pgfusepath{clip}%
\pgfsetbuttcap%
\pgfsetmiterjoin%
\definecolor{currentfill}{rgb}{0.754268,0.565033,0.211761}%
\pgfsetfillcolor{currentfill}%
\pgfsetlinewidth{0.000000pt}%
\definecolor{currentstroke}{rgb}{0.000000,0.000000,0.000000}%
\pgfsetstrokecolor{currentstroke}%
\pgfsetstrokeopacity{0.000000}%
\pgfsetdash{}{0pt}%
\pgfpathmoveto{\pgfqpoint{2.447347in}{2.171115in}}%
\pgfpathlineto{\pgfqpoint{2.456283in}{2.171115in}}%
\pgfpathlineto{\pgfqpoint{2.456283in}{2.217877in}}%
\pgfpathlineto{\pgfqpoint{2.447347in}{2.217877in}}%
\pgfpathlineto{\pgfqpoint{2.447347in}{2.171115in}}%
\pgfpathclose%
\pgfusepath{fill}%
\end{pgfscope}%
\begin{pgfscope}%
\pgfpathrectangle{\pgfqpoint{0.697024in}{0.857143in}}{\pgfqpoint{2.627103in}{1.813434in}}%
\pgfusepath{clip}%
\pgfsetbuttcap%
\pgfsetmiterjoin%
\definecolor{currentfill}{rgb}{0.754268,0.565033,0.211761}%
\pgfsetfillcolor{currentfill}%
\pgfsetlinewidth{0.000000pt}%
\definecolor{currentstroke}{rgb}{0.000000,0.000000,0.000000}%
\pgfsetstrokecolor{currentstroke}%
\pgfsetstrokeopacity{0.000000}%
\pgfsetdash{}{0pt}%
\pgfpathmoveto{\pgfqpoint{2.458517in}{2.161834in}}%
\pgfpathlineto{\pgfqpoint{2.467454in}{2.161834in}}%
\pgfpathlineto{\pgfqpoint{2.467454in}{2.187571in}}%
\pgfpathlineto{\pgfqpoint{2.458517in}{2.187571in}}%
\pgfpathlineto{\pgfqpoint{2.458517in}{2.161834in}}%
\pgfpathclose%
\pgfusepath{fill}%
\end{pgfscope}%
\begin{pgfscope}%
\pgfpathrectangle{\pgfqpoint{0.697024in}{0.857143in}}{\pgfqpoint{2.627103in}{1.813434in}}%
\pgfusepath{clip}%
\pgfsetbuttcap%
\pgfsetmiterjoin%
\definecolor{currentfill}{rgb}{0.754268,0.565033,0.211761}%
\pgfsetfillcolor{currentfill}%
\pgfsetlinewidth{0.000000pt}%
\definecolor{currentstroke}{rgb}{0.000000,0.000000,0.000000}%
\pgfsetstrokecolor{currentstroke}%
\pgfsetstrokeopacity{0.000000}%
\pgfsetdash{}{0pt}%
\pgfpathmoveto{\pgfqpoint{2.469688in}{2.157094in}}%
\pgfpathlineto{\pgfqpoint{2.478624in}{2.157094in}}%
\pgfpathlineto{\pgfqpoint{2.478624in}{2.217593in}}%
\pgfpathlineto{\pgfqpoint{2.469688in}{2.217593in}}%
\pgfpathlineto{\pgfqpoint{2.469688in}{2.157094in}}%
\pgfpathclose%
\pgfusepath{fill}%
\end{pgfscope}%
\begin{pgfscope}%
\pgfpathrectangle{\pgfqpoint{0.697024in}{0.857143in}}{\pgfqpoint{2.627103in}{1.813434in}}%
\pgfusepath{clip}%
\pgfsetbuttcap%
\pgfsetmiterjoin%
\definecolor{currentfill}{rgb}{0.754268,0.565033,0.211761}%
\pgfsetfillcolor{currentfill}%
\pgfsetlinewidth{0.000000pt}%
\definecolor{currentstroke}{rgb}{0.000000,0.000000,0.000000}%
\pgfsetstrokecolor{currentstroke}%
\pgfsetstrokeopacity{0.000000}%
\pgfsetdash{}{0pt}%
\pgfpathmoveto{\pgfqpoint{2.480858in}{2.174912in}}%
\pgfpathlineto{\pgfqpoint{2.489795in}{2.174912in}}%
\pgfpathlineto{\pgfqpoint{2.489795in}{2.234952in}}%
\pgfpathlineto{\pgfqpoint{2.480858in}{2.234952in}}%
\pgfpathlineto{\pgfqpoint{2.480858in}{2.174912in}}%
\pgfpathclose%
\pgfusepath{fill}%
\end{pgfscope}%
\begin{pgfscope}%
\pgfpathrectangle{\pgfqpoint{0.697024in}{0.857143in}}{\pgfqpoint{2.627103in}{1.813434in}}%
\pgfusepath{clip}%
\pgfsetbuttcap%
\pgfsetmiterjoin%
\definecolor{currentfill}{rgb}{0.754268,0.565033,0.211761}%
\pgfsetfillcolor{currentfill}%
\pgfsetlinewidth{0.000000pt}%
\definecolor{currentstroke}{rgb}{0.000000,0.000000,0.000000}%
\pgfsetstrokecolor{currentstroke}%
\pgfsetstrokeopacity{0.000000}%
\pgfsetdash{}{0pt}%
\pgfpathmoveto{\pgfqpoint{2.492029in}{2.174223in}}%
\pgfpathlineto{\pgfqpoint{2.500965in}{2.174223in}}%
\pgfpathlineto{\pgfqpoint{2.500965in}{2.244252in}}%
\pgfpathlineto{\pgfqpoint{2.492029in}{2.244252in}}%
\pgfpathlineto{\pgfqpoint{2.492029in}{2.174223in}}%
\pgfpathclose%
\pgfusepath{fill}%
\end{pgfscope}%
\begin{pgfscope}%
\pgfpathrectangle{\pgfqpoint{0.697024in}{0.857143in}}{\pgfqpoint{2.627103in}{1.813434in}}%
\pgfusepath{clip}%
\pgfsetbuttcap%
\pgfsetmiterjoin%
\definecolor{currentfill}{rgb}{0.754268,0.565033,0.211761}%
\pgfsetfillcolor{currentfill}%
\pgfsetlinewidth{0.000000pt}%
\definecolor{currentstroke}{rgb}{0.000000,0.000000,0.000000}%
\pgfsetstrokecolor{currentstroke}%
\pgfsetstrokeopacity{0.000000}%
\pgfsetdash{}{0pt}%
\pgfpathmoveto{\pgfqpoint{2.503200in}{2.201661in}}%
\pgfpathlineto{\pgfqpoint{2.512136in}{2.201661in}}%
\pgfpathlineto{\pgfqpoint{2.512136in}{2.279294in}}%
\pgfpathlineto{\pgfqpoint{2.503200in}{2.279294in}}%
\pgfpathlineto{\pgfqpoint{2.503200in}{2.201661in}}%
\pgfpathclose%
\pgfusepath{fill}%
\end{pgfscope}%
\begin{pgfscope}%
\pgfpathrectangle{\pgfqpoint{0.697024in}{0.857143in}}{\pgfqpoint{2.627103in}{1.813434in}}%
\pgfusepath{clip}%
\pgfsetbuttcap%
\pgfsetmiterjoin%
\definecolor{currentfill}{rgb}{0.754268,0.565033,0.211761}%
\pgfsetfillcolor{currentfill}%
\pgfsetlinewidth{0.000000pt}%
\definecolor{currentstroke}{rgb}{0.000000,0.000000,0.000000}%
\pgfsetstrokecolor{currentstroke}%
\pgfsetstrokeopacity{0.000000}%
\pgfsetdash{}{0pt}%
\pgfpathmoveto{\pgfqpoint{2.514370in}{2.203340in}}%
\pgfpathlineto{\pgfqpoint{2.523307in}{2.203340in}}%
\pgfpathlineto{\pgfqpoint{2.523307in}{2.248110in}}%
\pgfpathlineto{\pgfqpoint{2.514370in}{2.248110in}}%
\pgfpathlineto{\pgfqpoint{2.514370in}{2.203340in}}%
\pgfpathclose%
\pgfusepath{fill}%
\end{pgfscope}%
\begin{pgfscope}%
\pgfpathrectangle{\pgfqpoint{0.697024in}{0.857143in}}{\pgfqpoint{2.627103in}{1.813434in}}%
\pgfusepath{clip}%
\pgfsetbuttcap%
\pgfsetmiterjoin%
\definecolor{currentfill}{rgb}{0.754268,0.565033,0.211761}%
\pgfsetfillcolor{currentfill}%
\pgfsetlinewidth{0.000000pt}%
\definecolor{currentstroke}{rgb}{0.000000,0.000000,0.000000}%
\pgfsetstrokecolor{currentstroke}%
\pgfsetstrokeopacity{0.000000}%
\pgfsetdash{}{0pt}%
\pgfpathmoveto{\pgfqpoint{2.525541in}{2.231373in}}%
\pgfpathlineto{\pgfqpoint{2.534477in}{2.231373in}}%
\pgfpathlineto{\pgfqpoint{2.534477in}{2.284391in}}%
\pgfpathlineto{\pgfqpoint{2.525541in}{2.284391in}}%
\pgfpathlineto{\pgfqpoint{2.525541in}{2.231373in}}%
\pgfpathclose%
\pgfusepath{fill}%
\end{pgfscope}%
\begin{pgfscope}%
\pgfpathrectangle{\pgfqpoint{0.697024in}{0.857143in}}{\pgfqpoint{2.627103in}{1.813434in}}%
\pgfusepath{clip}%
\pgfsetbuttcap%
\pgfsetmiterjoin%
\definecolor{currentfill}{rgb}{0.754268,0.565033,0.211761}%
\pgfsetfillcolor{currentfill}%
\pgfsetlinewidth{0.000000pt}%
\definecolor{currentstroke}{rgb}{0.000000,0.000000,0.000000}%
\pgfsetstrokecolor{currentstroke}%
\pgfsetstrokeopacity{0.000000}%
\pgfsetdash{}{0pt}%
\pgfpathmoveto{\pgfqpoint{2.536711in}{2.274566in}}%
\pgfpathlineto{\pgfqpoint{2.545648in}{2.274566in}}%
\pgfpathlineto{\pgfqpoint{2.545648in}{2.317801in}}%
\pgfpathlineto{\pgfqpoint{2.536711in}{2.317801in}}%
\pgfpathlineto{\pgfqpoint{2.536711in}{2.274566in}}%
\pgfpathclose%
\pgfusepath{fill}%
\end{pgfscope}%
\begin{pgfscope}%
\pgfpathrectangle{\pgfqpoint{0.697024in}{0.857143in}}{\pgfqpoint{2.627103in}{1.813434in}}%
\pgfusepath{clip}%
\pgfsetbuttcap%
\pgfsetmiterjoin%
\definecolor{currentfill}{rgb}{0.754268,0.565033,0.211761}%
\pgfsetfillcolor{currentfill}%
\pgfsetlinewidth{0.000000pt}%
\definecolor{currentstroke}{rgb}{0.000000,0.000000,0.000000}%
\pgfsetstrokecolor{currentstroke}%
\pgfsetstrokeopacity{0.000000}%
\pgfsetdash{}{0pt}%
\pgfpathmoveto{\pgfqpoint{2.547882in}{2.301594in}}%
\pgfpathlineto{\pgfqpoint{2.556818in}{2.301594in}}%
\pgfpathlineto{\pgfqpoint{2.556818in}{2.323016in}}%
\pgfpathlineto{\pgfqpoint{2.547882in}{2.323016in}}%
\pgfpathlineto{\pgfqpoint{2.547882in}{2.301594in}}%
\pgfpathclose%
\pgfusepath{fill}%
\end{pgfscope}%
\begin{pgfscope}%
\pgfpathrectangle{\pgfqpoint{0.697024in}{0.857143in}}{\pgfqpoint{2.627103in}{1.813434in}}%
\pgfusepath{clip}%
\pgfsetbuttcap%
\pgfsetmiterjoin%
\definecolor{currentfill}{rgb}{0.754268,0.565033,0.211761}%
\pgfsetfillcolor{currentfill}%
\pgfsetlinewidth{0.000000pt}%
\definecolor{currentstroke}{rgb}{0.000000,0.000000,0.000000}%
\pgfsetstrokecolor{currentstroke}%
\pgfsetstrokeopacity{0.000000}%
\pgfsetdash{}{0pt}%
\pgfpathmoveto{\pgfqpoint{2.559053in}{2.338646in}}%
\pgfpathlineto{\pgfqpoint{2.567989in}{2.338646in}}%
\pgfpathlineto{\pgfqpoint{2.567989in}{2.376971in}}%
\pgfpathlineto{\pgfqpoint{2.559053in}{2.376971in}}%
\pgfpathlineto{\pgfqpoint{2.559053in}{2.338646in}}%
\pgfpathclose%
\pgfusepath{fill}%
\end{pgfscope}%
\begin{pgfscope}%
\pgfpathrectangle{\pgfqpoint{0.697024in}{0.857143in}}{\pgfqpoint{2.627103in}{1.813434in}}%
\pgfusepath{clip}%
\pgfsetbuttcap%
\pgfsetmiterjoin%
\definecolor{currentfill}{rgb}{0.754268,0.565033,0.211761}%
\pgfsetfillcolor{currentfill}%
\pgfsetlinewidth{0.000000pt}%
\definecolor{currentstroke}{rgb}{0.000000,0.000000,0.000000}%
\pgfsetstrokecolor{currentstroke}%
\pgfsetstrokeopacity{0.000000}%
\pgfsetdash{}{0pt}%
\pgfpathmoveto{\pgfqpoint{2.570223in}{2.306020in}}%
\pgfpathlineto{\pgfqpoint{2.579160in}{2.306020in}}%
\pgfpathlineto{\pgfqpoint{2.579160in}{2.319717in}}%
\pgfpathlineto{\pgfqpoint{2.570223in}{2.319717in}}%
\pgfpathlineto{\pgfqpoint{2.570223in}{2.306020in}}%
\pgfpathclose%
\pgfusepath{fill}%
\end{pgfscope}%
\begin{pgfscope}%
\pgfpathrectangle{\pgfqpoint{0.697024in}{0.857143in}}{\pgfqpoint{2.627103in}{1.813434in}}%
\pgfusepath{clip}%
\pgfsetbuttcap%
\pgfsetmiterjoin%
\definecolor{currentfill}{rgb}{0.754268,0.565033,0.211761}%
\pgfsetfillcolor{currentfill}%
\pgfsetlinewidth{0.000000pt}%
\definecolor{currentstroke}{rgb}{0.000000,0.000000,0.000000}%
\pgfsetstrokecolor{currentstroke}%
\pgfsetstrokeopacity{0.000000}%
\pgfsetdash{}{0pt}%
\pgfpathmoveto{\pgfqpoint{2.581394in}{1.745753in}}%
\pgfpathlineto{\pgfqpoint{2.590330in}{1.745753in}}%
\pgfpathlineto{\pgfqpoint{2.590330in}{1.735680in}}%
\pgfpathlineto{\pgfqpoint{2.581394in}{1.735680in}}%
\pgfpathlineto{\pgfqpoint{2.581394in}{1.745753in}}%
\pgfpathclose%
\pgfusepath{fill}%
\end{pgfscope}%
\begin{pgfscope}%
\pgfpathrectangle{\pgfqpoint{0.697024in}{0.857143in}}{\pgfqpoint{2.627103in}{1.813434in}}%
\pgfusepath{clip}%
\pgfsetbuttcap%
\pgfsetmiterjoin%
\definecolor{currentfill}{rgb}{0.754268,0.565033,0.211761}%
\pgfsetfillcolor{currentfill}%
\pgfsetlinewidth{0.000000pt}%
\definecolor{currentstroke}{rgb}{0.000000,0.000000,0.000000}%
\pgfsetstrokecolor{currentstroke}%
\pgfsetstrokeopacity{0.000000}%
\pgfsetdash{}{0pt}%
\pgfpathmoveto{\pgfqpoint{2.592564in}{2.271715in}}%
\pgfpathlineto{\pgfqpoint{2.601501in}{2.271715in}}%
\pgfpathlineto{\pgfqpoint{2.601501in}{2.276421in}}%
\pgfpathlineto{\pgfqpoint{2.592564in}{2.276421in}}%
\pgfpathlineto{\pgfqpoint{2.592564in}{2.271715in}}%
\pgfpathclose%
\pgfusepath{fill}%
\end{pgfscope}%
\begin{pgfscope}%
\pgfpathrectangle{\pgfqpoint{0.697024in}{0.857143in}}{\pgfqpoint{2.627103in}{1.813434in}}%
\pgfusepath{clip}%
\pgfsetbuttcap%
\pgfsetmiterjoin%
\definecolor{currentfill}{rgb}{0.754268,0.565033,0.211761}%
\pgfsetfillcolor{currentfill}%
\pgfsetlinewidth{0.000000pt}%
\definecolor{currentstroke}{rgb}{0.000000,0.000000,0.000000}%
\pgfsetstrokecolor{currentstroke}%
\pgfsetstrokeopacity{0.000000}%
\pgfsetdash{}{0pt}%
\pgfpathmoveto{\pgfqpoint{2.603735in}{1.809682in}}%
\pgfpathlineto{\pgfqpoint{2.612672in}{1.809682in}}%
\pgfpathlineto{\pgfqpoint{2.612672in}{1.782268in}}%
\pgfpathlineto{\pgfqpoint{2.603735in}{1.782268in}}%
\pgfpathlineto{\pgfqpoint{2.603735in}{1.809682in}}%
\pgfpathclose%
\pgfusepath{fill}%
\end{pgfscope}%
\begin{pgfscope}%
\pgfpathrectangle{\pgfqpoint{0.697024in}{0.857143in}}{\pgfqpoint{2.627103in}{1.813434in}}%
\pgfusepath{clip}%
\pgfsetbuttcap%
\pgfsetmiterjoin%
\definecolor{currentfill}{rgb}{0.754268,0.565033,0.211761}%
\pgfsetfillcolor{currentfill}%
\pgfsetlinewidth{0.000000pt}%
\definecolor{currentstroke}{rgb}{0.000000,0.000000,0.000000}%
\pgfsetstrokecolor{currentstroke}%
\pgfsetstrokeopacity{0.000000}%
\pgfsetdash{}{0pt}%
\pgfpathmoveto{\pgfqpoint{2.614906in}{1.821708in}}%
\pgfpathlineto{\pgfqpoint{2.623842in}{1.821708in}}%
\pgfpathlineto{\pgfqpoint{2.623842in}{1.754386in}}%
\pgfpathlineto{\pgfqpoint{2.614906in}{1.754386in}}%
\pgfpathlineto{\pgfqpoint{2.614906in}{1.821708in}}%
\pgfpathclose%
\pgfusepath{fill}%
\end{pgfscope}%
\begin{pgfscope}%
\pgfpathrectangle{\pgfqpoint{0.697024in}{0.857143in}}{\pgfqpoint{2.627103in}{1.813434in}}%
\pgfusepath{clip}%
\pgfsetbuttcap%
\pgfsetmiterjoin%
\definecolor{currentfill}{rgb}{0.754268,0.565033,0.211761}%
\pgfsetfillcolor{currentfill}%
\pgfsetlinewidth{0.000000pt}%
\definecolor{currentstroke}{rgb}{0.000000,0.000000,0.000000}%
\pgfsetstrokecolor{currentstroke}%
\pgfsetstrokeopacity{0.000000}%
\pgfsetdash{}{0pt}%
\pgfpathmoveto{\pgfqpoint{2.626076in}{2.249544in}}%
\pgfpathlineto{\pgfqpoint{2.635013in}{2.249544in}}%
\pgfpathlineto{\pgfqpoint{2.635013in}{2.268156in}}%
\pgfpathlineto{\pgfqpoint{2.626076in}{2.268156in}}%
\pgfpathlineto{\pgfqpoint{2.626076in}{2.249544in}}%
\pgfpathclose%
\pgfusepath{fill}%
\end{pgfscope}%
\begin{pgfscope}%
\pgfpathrectangle{\pgfqpoint{0.697024in}{0.857143in}}{\pgfqpoint{2.627103in}{1.813434in}}%
\pgfusepath{clip}%
\pgfsetbuttcap%
\pgfsetmiterjoin%
\definecolor{currentfill}{rgb}{0.754268,0.565033,0.211761}%
\pgfsetfillcolor{currentfill}%
\pgfsetlinewidth{0.000000pt}%
\definecolor{currentstroke}{rgb}{0.000000,0.000000,0.000000}%
\pgfsetstrokecolor{currentstroke}%
\pgfsetstrokeopacity{0.000000}%
\pgfsetdash{}{0pt}%
\pgfpathmoveto{\pgfqpoint{2.637247in}{1.821048in}}%
\pgfpathlineto{\pgfqpoint{2.646183in}{1.821048in}}%
\pgfpathlineto{\pgfqpoint{2.646183in}{1.800007in}}%
\pgfpathlineto{\pgfqpoint{2.637247in}{1.800007in}}%
\pgfpathlineto{\pgfqpoint{2.637247in}{1.821048in}}%
\pgfpathclose%
\pgfusepath{fill}%
\end{pgfscope}%
\begin{pgfscope}%
\pgfpathrectangle{\pgfqpoint{0.697024in}{0.857143in}}{\pgfqpoint{2.627103in}{1.813434in}}%
\pgfusepath{clip}%
\pgfsetbuttcap%
\pgfsetmiterjoin%
\definecolor{currentfill}{rgb}{0.754268,0.565033,0.211761}%
\pgfsetfillcolor{currentfill}%
\pgfsetlinewidth{0.000000pt}%
\definecolor{currentstroke}{rgb}{0.000000,0.000000,0.000000}%
\pgfsetstrokecolor{currentstroke}%
\pgfsetstrokeopacity{0.000000}%
\pgfsetdash{}{0pt}%
\pgfpathmoveto{\pgfqpoint{2.648417in}{1.820868in}}%
\pgfpathlineto{\pgfqpoint{2.657354in}{1.820868in}}%
\pgfpathlineto{\pgfqpoint{2.657354in}{1.782330in}}%
\pgfpathlineto{\pgfqpoint{2.648417in}{1.782330in}}%
\pgfpathlineto{\pgfqpoint{2.648417in}{1.820868in}}%
\pgfpathclose%
\pgfusepath{fill}%
\end{pgfscope}%
\begin{pgfscope}%
\pgfpathrectangle{\pgfqpoint{0.697024in}{0.857143in}}{\pgfqpoint{2.627103in}{1.813434in}}%
\pgfusepath{clip}%
\pgfsetbuttcap%
\pgfsetmiterjoin%
\definecolor{currentfill}{rgb}{0.754268,0.565033,0.211761}%
\pgfsetfillcolor{currentfill}%
\pgfsetlinewidth{0.000000pt}%
\definecolor{currentstroke}{rgb}{0.000000,0.000000,0.000000}%
\pgfsetstrokecolor{currentstroke}%
\pgfsetstrokeopacity{0.000000}%
\pgfsetdash{}{0pt}%
\pgfpathmoveto{\pgfqpoint{2.659588in}{1.820736in}}%
\pgfpathlineto{\pgfqpoint{2.668525in}{1.820736in}}%
\pgfpathlineto{\pgfqpoint{2.668525in}{1.778636in}}%
\pgfpathlineto{\pgfqpoint{2.659588in}{1.778636in}}%
\pgfpathlineto{\pgfqpoint{2.659588in}{1.820736in}}%
\pgfpathclose%
\pgfusepath{fill}%
\end{pgfscope}%
\begin{pgfscope}%
\pgfpathrectangle{\pgfqpoint{0.697024in}{0.857143in}}{\pgfqpoint{2.627103in}{1.813434in}}%
\pgfusepath{clip}%
\pgfsetbuttcap%
\pgfsetmiterjoin%
\definecolor{currentfill}{rgb}{0.754268,0.565033,0.211761}%
\pgfsetfillcolor{currentfill}%
\pgfsetlinewidth{0.000000pt}%
\definecolor{currentstroke}{rgb}{0.000000,0.000000,0.000000}%
\pgfsetstrokecolor{currentstroke}%
\pgfsetstrokeopacity{0.000000}%
\pgfsetdash{}{0pt}%
\pgfpathmoveto{\pgfqpoint{2.670759in}{1.820584in}}%
\pgfpathlineto{\pgfqpoint{2.679695in}{1.820584in}}%
\pgfpathlineto{\pgfqpoint{2.679695in}{1.795088in}}%
\pgfpathlineto{\pgfqpoint{2.670759in}{1.795088in}}%
\pgfpathlineto{\pgfqpoint{2.670759in}{1.820584in}}%
\pgfpathclose%
\pgfusepath{fill}%
\end{pgfscope}%
\begin{pgfscope}%
\pgfpathrectangle{\pgfqpoint{0.697024in}{0.857143in}}{\pgfqpoint{2.627103in}{1.813434in}}%
\pgfusepath{clip}%
\pgfsetbuttcap%
\pgfsetmiterjoin%
\definecolor{currentfill}{rgb}{0.754268,0.565033,0.211761}%
\pgfsetfillcolor{currentfill}%
\pgfsetlinewidth{0.000000pt}%
\definecolor{currentstroke}{rgb}{0.000000,0.000000,0.000000}%
\pgfsetstrokecolor{currentstroke}%
\pgfsetstrokeopacity{0.000000}%
\pgfsetdash{}{0pt}%
\pgfpathmoveto{\pgfqpoint{2.681929in}{2.158131in}}%
\pgfpathlineto{\pgfqpoint{2.690866in}{2.158131in}}%
\pgfpathlineto{\pgfqpoint{2.690866in}{2.183907in}}%
\pgfpathlineto{\pgfqpoint{2.681929in}{2.183907in}}%
\pgfpathlineto{\pgfqpoint{2.681929in}{2.158131in}}%
\pgfpathclose%
\pgfusepath{fill}%
\end{pgfscope}%
\begin{pgfscope}%
\pgfpathrectangle{\pgfqpoint{0.697024in}{0.857143in}}{\pgfqpoint{2.627103in}{1.813434in}}%
\pgfusepath{clip}%
\pgfsetbuttcap%
\pgfsetmiterjoin%
\definecolor{currentfill}{rgb}{0.754268,0.565033,0.211761}%
\pgfsetfillcolor{currentfill}%
\pgfsetlinewidth{0.000000pt}%
\definecolor{currentstroke}{rgb}{0.000000,0.000000,0.000000}%
\pgfsetstrokecolor{currentstroke}%
\pgfsetstrokeopacity{0.000000}%
\pgfsetdash{}{0pt}%
\pgfpathmoveto{\pgfqpoint{2.693100in}{2.010036in}}%
\pgfpathlineto{\pgfqpoint{2.702036in}{2.010036in}}%
\pgfpathlineto{\pgfqpoint{2.702036in}{2.070007in}}%
\pgfpathlineto{\pgfqpoint{2.693100in}{2.070007in}}%
\pgfpathlineto{\pgfqpoint{2.693100in}{2.010036in}}%
\pgfpathclose%
\pgfusepath{fill}%
\end{pgfscope}%
\begin{pgfscope}%
\pgfpathrectangle{\pgfqpoint{0.697024in}{0.857143in}}{\pgfqpoint{2.627103in}{1.813434in}}%
\pgfusepath{clip}%
\pgfsetbuttcap%
\pgfsetmiterjoin%
\definecolor{currentfill}{rgb}{0.754268,0.565033,0.211761}%
\pgfsetfillcolor{currentfill}%
\pgfsetlinewidth{0.000000pt}%
\definecolor{currentstroke}{rgb}{0.000000,0.000000,0.000000}%
\pgfsetstrokecolor{currentstroke}%
\pgfsetstrokeopacity{0.000000}%
\pgfsetdash{}{0pt}%
\pgfpathmoveto{\pgfqpoint{2.704270in}{2.033013in}}%
\pgfpathlineto{\pgfqpoint{2.713207in}{2.033013in}}%
\pgfpathlineto{\pgfqpoint{2.713207in}{2.040296in}}%
\pgfpathlineto{\pgfqpoint{2.704270in}{2.040296in}}%
\pgfpathlineto{\pgfqpoint{2.704270in}{2.033013in}}%
\pgfpathclose%
\pgfusepath{fill}%
\end{pgfscope}%
\begin{pgfscope}%
\pgfpathrectangle{\pgfqpoint{0.697024in}{0.857143in}}{\pgfqpoint{2.627103in}{1.813434in}}%
\pgfusepath{clip}%
\pgfsetbuttcap%
\pgfsetmiterjoin%
\definecolor{currentfill}{rgb}{0.754268,0.565033,0.211761}%
\pgfsetfillcolor{currentfill}%
\pgfsetlinewidth{0.000000pt}%
\definecolor{currentstroke}{rgb}{0.000000,0.000000,0.000000}%
\pgfsetstrokecolor{currentstroke}%
\pgfsetstrokeopacity{0.000000}%
\pgfsetdash{}{0pt}%
\pgfpathmoveto{\pgfqpoint{2.715441in}{1.505197in}}%
\pgfpathlineto{\pgfqpoint{2.724378in}{1.505197in}}%
\pgfpathlineto{\pgfqpoint{2.724378in}{1.490273in}}%
\pgfpathlineto{\pgfqpoint{2.715441in}{1.490273in}}%
\pgfpathlineto{\pgfqpoint{2.715441in}{1.505197in}}%
\pgfpathclose%
\pgfusepath{fill}%
\end{pgfscope}%
\begin{pgfscope}%
\pgfpathrectangle{\pgfqpoint{0.697024in}{0.857143in}}{\pgfqpoint{2.627103in}{1.813434in}}%
\pgfusepath{clip}%
\pgfsetbuttcap%
\pgfsetmiterjoin%
\definecolor{currentfill}{rgb}{0.754268,0.565033,0.211761}%
\pgfsetfillcolor{currentfill}%
\pgfsetlinewidth{0.000000pt}%
\definecolor{currentstroke}{rgb}{0.000000,0.000000,0.000000}%
\pgfsetstrokecolor{currentstroke}%
\pgfsetstrokeopacity{0.000000}%
\pgfsetdash{}{0pt}%
\pgfpathmoveto{\pgfqpoint{2.726612in}{1.372710in}}%
\pgfpathlineto{\pgfqpoint{2.735548in}{1.372710in}}%
\pgfpathlineto{\pgfqpoint{2.735548in}{1.350342in}}%
\pgfpathlineto{\pgfqpoint{2.726612in}{1.350342in}}%
\pgfpathlineto{\pgfqpoint{2.726612in}{1.372710in}}%
\pgfpathclose%
\pgfusepath{fill}%
\end{pgfscope}%
\begin{pgfscope}%
\pgfpathrectangle{\pgfqpoint{0.697024in}{0.857143in}}{\pgfqpoint{2.627103in}{1.813434in}}%
\pgfusepath{clip}%
\pgfsetbuttcap%
\pgfsetmiterjoin%
\definecolor{currentfill}{rgb}{0.754268,0.565033,0.211761}%
\pgfsetfillcolor{currentfill}%
\pgfsetlinewidth{0.000000pt}%
\definecolor{currentstroke}{rgb}{0.000000,0.000000,0.000000}%
\pgfsetstrokecolor{currentstroke}%
\pgfsetstrokeopacity{0.000000}%
\pgfsetdash{}{0pt}%
\pgfpathmoveto{\pgfqpoint{2.737782in}{2.137425in}}%
\pgfpathlineto{\pgfqpoint{2.746719in}{2.137425in}}%
\pgfpathlineto{\pgfqpoint{2.746719in}{2.155518in}}%
\pgfpathlineto{\pgfqpoint{2.737782in}{2.155518in}}%
\pgfpathlineto{\pgfqpoint{2.737782in}{2.137425in}}%
\pgfpathclose%
\pgfusepath{fill}%
\end{pgfscope}%
\begin{pgfscope}%
\pgfpathrectangle{\pgfqpoint{0.697024in}{0.857143in}}{\pgfqpoint{2.627103in}{1.813434in}}%
\pgfusepath{clip}%
\pgfsetbuttcap%
\pgfsetmiterjoin%
\definecolor{currentfill}{rgb}{0.754268,0.565033,0.211761}%
\pgfsetfillcolor{currentfill}%
\pgfsetlinewidth{0.000000pt}%
\definecolor{currentstroke}{rgb}{0.000000,0.000000,0.000000}%
\pgfsetstrokecolor{currentstroke}%
\pgfsetstrokeopacity{0.000000}%
\pgfsetdash{}{0pt}%
\pgfpathmoveto{\pgfqpoint{2.748953in}{2.150328in}}%
\pgfpathlineto{\pgfqpoint{2.757889in}{2.150328in}}%
\pgfpathlineto{\pgfqpoint{2.757889in}{2.191518in}}%
\pgfpathlineto{\pgfqpoint{2.748953in}{2.191518in}}%
\pgfpathlineto{\pgfqpoint{2.748953in}{2.150328in}}%
\pgfpathclose%
\pgfusepath{fill}%
\end{pgfscope}%
\begin{pgfscope}%
\pgfpathrectangle{\pgfqpoint{0.697024in}{0.857143in}}{\pgfqpoint{2.627103in}{1.813434in}}%
\pgfusepath{clip}%
\pgfsetbuttcap%
\pgfsetmiterjoin%
\definecolor{currentfill}{rgb}{0.754268,0.565033,0.211761}%
\pgfsetfillcolor{currentfill}%
\pgfsetlinewidth{0.000000pt}%
\definecolor{currentstroke}{rgb}{0.000000,0.000000,0.000000}%
\pgfsetstrokecolor{currentstroke}%
\pgfsetstrokeopacity{0.000000}%
\pgfsetdash{}{0pt}%
\pgfpathmoveto{\pgfqpoint{2.760124in}{2.172579in}}%
\pgfpathlineto{\pgfqpoint{2.769060in}{2.172579in}}%
\pgfpathlineto{\pgfqpoint{2.769060in}{2.204755in}}%
\pgfpathlineto{\pgfqpoint{2.760124in}{2.204755in}}%
\pgfpathlineto{\pgfqpoint{2.760124in}{2.172579in}}%
\pgfpathclose%
\pgfusepath{fill}%
\end{pgfscope}%
\begin{pgfscope}%
\pgfpathrectangle{\pgfqpoint{0.697024in}{0.857143in}}{\pgfqpoint{2.627103in}{1.813434in}}%
\pgfusepath{clip}%
\pgfsetbuttcap%
\pgfsetmiterjoin%
\definecolor{currentfill}{rgb}{0.754268,0.565033,0.211761}%
\pgfsetfillcolor{currentfill}%
\pgfsetlinewidth{0.000000pt}%
\definecolor{currentstroke}{rgb}{0.000000,0.000000,0.000000}%
\pgfsetstrokecolor{currentstroke}%
\pgfsetstrokeopacity{0.000000}%
\pgfsetdash{}{0pt}%
\pgfpathmoveto{\pgfqpoint{2.771294in}{2.095956in}}%
\pgfpathlineto{\pgfqpoint{2.780231in}{2.095956in}}%
\pgfpathlineto{\pgfqpoint{2.780231in}{2.150246in}}%
\pgfpathlineto{\pgfqpoint{2.771294in}{2.150246in}}%
\pgfpathlineto{\pgfqpoint{2.771294in}{2.095956in}}%
\pgfpathclose%
\pgfusepath{fill}%
\end{pgfscope}%
\begin{pgfscope}%
\pgfpathrectangle{\pgfqpoint{0.697024in}{0.857143in}}{\pgfqpoint{2.627103in}{1.813434in}}%
\pgfusepath{clip}%
\pgfsetbuttcap%
\pgfsetmiterjoin%
\definecolor{currentfill}{rgb}{0.754268,0.565033,0.211761}%
\pgfsetfillcolor{currentfill}%
\pgfsetlinewidth{0.000000pt}%
\definecolor{currentstroke}{rgb}{0.000000,0.000000,0.000000}%
\pgfsetstrokecolor{currentstroke}%
\pgfsetstrokeopacity{0.000000}%
\pgfsetdash{}{0pt}%
\pgfpathmoveto{\pgfqpoint{2.782465in}{2.143905in}}%
\pgfpathlineto{\pgfqpoint{2.791401in}{2.143905in}}%
\pgfpathlineto{\pgfqpoint{2.791401in}{2.169728in}}%
\pgfpathlineto{\pgfqpoint{2.782465in}{2.169728in}}%
\pgfpathlineto{\pgfqpoint{2.782465in}{2.143905in}}%
\pgfpathclose%
\pgfusepath{fill}%
\end{pgfscope}%
\begin{pgfscope}%
\pgfpathrectangle{\pgfqpoint{0.697024in}{0.857143in}}{\pgfqpoint{2.627103in}{1.813434in}}%
\pgfusepath{clip}%
\pgfsetbuttcap%
\pgfsetmiterjoin%
\definecolor{currentfill}{rgb}{0.754268,0.565033,0.211761}%
\pgfsetfillcolor{currentfill}%
\pgfsetlinewidth{0.000000pt}%
\definecolor{currentstroke}{rgb}{0.000000,0.000000,0.000000}%
\pgfsetstrokecolor{currentstroke}%
\pgfsetstrokeopacity{0.000000}%
\pgfsetdash{}{0pt}%
\pgfpathmoveto{\pgfqpoint{2.793635in}{2.105041in}}%
\pgfpathlineto{\pgfqpoint{2.802572in}{2.105041in}}%
\pgfpathlineto{\pgfqpoint{2.802572in}{2.167376in}}%
\pgfpathlineto{\pgfqpoint{2.793635in}{2.167376in}}%
\pgfpathlineto{\pgfqpoint{2.793635in}{2.105041in}}%
\pgfpathclose%
\pgfusepath{fill}%
\end{pgfscope}%
\begin{pgfscope}%
\pgfpathrectangle{\pgfqpoint{0.697024in}{0.857143in}}{\pgfqpoint{2.627103in}{1.813434in}}%
\pgfusepath{clip}%
\pgfsetbuttcap%
\pgfsetmiterjoin%
\definecolor{currentfill}{rgb}{0.754268,0.565033,0.211761}%
\pgfsetfillcolor{currentfill}%
\pgfsetlinewidth{0.000000pt}%
\definecolor{currentstroke}{rgb}{0.000000,0.000000,0.000000}%
\pgfsetstrokecolor{currentstroke}%
\pgfsetstrokeopacity{0.000000}%
\pgfsetdash{}{0pt}%
\pgfpathmoveto{\pgfqpoint{2.804806in}{2.119431in}}%
\pgfpathlineto{\pgfqpoint{2.813742in}{2.119431in}}%
\pgfpathlineto{\pgfqpoint{2.813742in}{2.167998in}}%
\pgfpathlineto{\pgfqpoint{2.804806in}{2.167998in}}%
\pgfpathlineto{\pgfqpoint{2.804806in}{2.119431in}}%
\pgfpathclose%
\pgfusepath{fill}%
\end{pgfscope}%
\begin{pgfscope}%
\pgfpathrectangle{\pgfqpoint{0.697024in}{0.857143in}}{\pgfqpoint{2.627103in}{1.813434in}}%
\pgfusepath{clip}%
\pgfsetbuttcap%
\pgfsetmiterjoin%
\definecolor{currentfill}{rgb}{0.754268,0.565033,0.211761}%
\pgfsetfillcolor{currentfill}%
\pgfsetlinewidth{0.000000pt}%
\definecolor{currentstroke}{rgb}{0.000000,0.000000,0.000000}%
\pgfsetstrokecolor{currentstroke}%
\pgfsetstrokeopacity{0.000000}%
\pgfsetdash{}{0pt}%
\pgfpathmoveto{\pgfqpoint{2.815977in}{2.009777in}}%
\pgfpathlineto{\pgfqpoint{2.824913in}{2.009777in}}%
\pgfpathlineto{\pgfqpoint{2.824913in}{2.077594in}}%
\pgfpathlineto{\pgfqpoint{2.815977in}{2.077594in}}%
\pgfpathlineto{\pgfqpoint{2.815977in}{2.009777in}}%
\pgfpathclose%
\pgfusepath{fill}%
\end{pgfscope}%
\begin{pgfscope}%
\pgfpathrectangle{\pgfqpoint{0.697024in}{0.857143in}}{\pgfqpoint{2.627103in}{1.813434in}}%
\pgfusepath{clip}%
\pgfsetbuttcap%
\pgfsetmiterjoin%
\definecolor{currentfill}{rgb}{0.754268,0.565033,0.211761}%
\pgfsetfillcolor{currentfill}%
\pgfsetlinewidth{0.000000pt}%
\definecolor{currentstroke}{rgb}{0.000000,0.000000,0.000000}%
\pgfsetstrokecolor{currentstroke}%
\pgfsetstrokeopacity{0.000000}%
\pgfsetdash{}{0pt}%
\pgfpathmoveto{\pgfqpoint{2.827147in}{2.013588in}}%
\pgfpathlineto{\pgfqpoint{2.836084in}{2.013588in}}%
\pgfpathlineto{\pgfqpoint{2.836084in}{2.069975in}}%
\pgfpathlineto{\pgfqpoint{2.827147in}{2.069975in}}%
\pgfpathlineto{\pgfqpoint{2.827147in}{2.013588in}}%
\pgfpathclose%
\pgfusepath{fill}%
\end{pgfscope}%
\begin{pgfscope}%
\pgfpathrectangle{\pgfqpoint{0.697024in}{0.857143in}}{\pgfqpoint{2.627103in}{1.813434in}}%
\pgfusepath{clip}%
\pgfsetbuttcap%
\pgfsetmiterjoin%
\definecolor{currentfill}{rgb}{0.754268,0.565033,0.211761}%
\pgfsetfillcolor{currentfill}%
\pgfsetlinewidth{0.000000pt}%
\definecolor{currentstroke}{rgb}{0.000000,0.000000,0.000000}%
\pgfsetstrokecolor{currentstroke}%
\pgfsetstrokeopacity{0.000000}%
\pgfsetdash{}{0pt}%
\pgfpathmoveto{\pgfqpoint{2.838318in}{1.306492in}}%
\pgfpathlineto{\pgfqpoint{2.847254in}{1.306492in}}%
\pgfpathlineto{\pgfqpoint{2.847254in}{1.298013in}}%
\pgfpathlineto{\pgfqpoint{2.838318in}{1.298013in}}%
\pgfpathlineto{\pgfqpoint{2.838318in}{1.306492in}}%
\pgfpathclose%
\pgfusepath{fill}%
\end{pgfscope}%
\begin{pgfscope}%
\pgfpathrectangle{\pgfqpoint{0.697024in}{0.857143in}}{\pgfqpoint{2.627103in}{1.813434in}}%
\pgfusepath{clip}%
\pgfsetbuttcap%
\pgfsetmiterjoin%
\definecolor{currentfill}{rgb}{0.754268,0.565033,0.211761}%
\pgfsetfillcolor{currentfill}%
\pgfsetlinewidth{0.000000pt}%
\definecolor{currentstroke}{rgb}{0.000000,0.000000,0.000000}%
\pgfsetstrokecolor{currentstroke}%
\pgfsetstrokeopacity{0.000000}%
\pgfsetdash{}{0pt}%
\pgfpathmoveto{\pgfqpoint{2.849488in}{1.957877in}}%
\pgfpathlineto{\pgfqpoint{2.858425in}{1.957877in}}%
\pgfpathlineto{\pgfqpoint{2.858425in}{2.017802in}}%
\pgfpathlineto{\pgfqpoint{2.849488in}{2.017802in}}%
\pgfpathlineto{\pgfqpoint{2.849488in}{1.957877in}}%
\pgfpathclose%
\pgfusepath{fill}%
\end{pgfscope}%
\begin{pgfscope}%
\pgfpathrectangle{\pgfqpoint{0.697024in}{0.857143in}}{\pgfqpoint{2.627103in}{1.813434in}}%
\pgfusepath{clip}%
\pgfsetbuttcap%
\pgfsetmiterjoin%
\definecolor{currentfill}{rgb}{0.754268,0.565033,0.211761}%
\pgfsetfillcolor{currentfill}%
\pgfsetlinewidth{0.000000pt}%
\definecolor{currentstroke}{rgb}{0.000000,0.000000,0.000000}%
\pgfsetstrokecolor{currentstroke}%
\pgfsetstrokeopacity{0.000000}%
\pgfsetdash{}{0pt}%
\pgfpathmoveto{\pgfqpoint{2.860659in}{2.022127in}}%
\pgfpathlineto{\pgfqpoint{2.869595in}{2.022127in}}%
\pgfpathlineto{\pgfqpoint{2.869595in}{2.052455in}}%
\pgfpathlineto{\pgfqpoint{2.860659in}{2.052455in}}%
\pgfpathlineto{\pgfqpoint{2.860659in}{2.022127in}}%
\pgfpathclose%
\pgfusepath{fill}%
\end{pgfscope}%
\begin{pgfscope}%
\pgfpathrectangle{\pgfqpoint{0.697024in}{0.857143in}}{\pgfqpoint{2.627103in}{1.813434in}}%
\pgfusepath{clip}%
\pgfsetbuttcap%
\pgfsetmiterjoin%
\definecolor{currentfill}{rgb}{0.754268,0.565033,0.211761}%
\pgfsetfillcolor{currentfill}%
\pgfsetlinewidth{0.000000pt}%
\definecolor{currentstroke}{rgb}{0.000000,0.000000,0.000000}%
\pgfsetstrokecolor{currentstroke}%
\pgfsetstrokeopacity{0.000000}%
\pgfsetdash{}{0pt}%
\pgfpathmoveto{\pgfqpoint{2.871830in}{1.945439in}}%
\pgfpathlineto{\pgfqpoint{2.880766in}{1.945439in}}%
\pgfpathlineto{\pgfqpoint{2.880766in}{1.993558in}}%
\pgfpathlineto{\pgfqpoint{2.871830in}{1.993558in}}%
\pgfpathlineto{\pgfqpoint{2.871830in}{1.945439in}}%
\pgfpathclose%
\pgfusepath{fill}%
\end{pgfscope}%
\begin{pgfscope}%
\pgfpathrectangle{\pgfqpoint{0.697024in}{0.857143in}}{\pgfqpoint{2.627103in}{1.813434in}}%
\pgfusepath{clip}%
\pgfsetbuttcap%
\pgfsetmiterjoin%
\definecolor{currentfill}{rgb}{0.754268,0.565033,0.211761}%
\pgfsetfillcolor{currentfill}%
\pgfsetlinewidth{0.000000pt}%
\definecolor{currentstroke}{rgb}{0.000000,0.000000,0.000000}%
\pgfsetstrokecolor{currentstroke}%
\pgfsetstrokeopacity{0.000000}%
\pgfsetdash{}{0pt}%
\pgfpathmoveto{\pgfqpoint{2.883000in}{1.914174in}}%
\pgfpathlineto{\pgfqpoint{2.891937in}{1.914174in}}%
\pgfpathlineto{\pgfqpoint{2.891937in}{1.953427in}}%
\pgfpathlineto{\pgfqpoint{2.883000in}{1.953427in}}%
\pgfpathlineto{\pgfqpoint{2.883000in}{1.914174in}}%
\pgfpathclose%
\pgfusepath{fill}%
\end{pgfscope}%
\begin{pgfscope}%
\pgfpathrectangle{\pgfqpoint{0.697024in}{0.857143in}}{\pgfqpoint{2.627103in}{1.813434in}}%
\pgfusepath{clip}%
\pgfsetbuttcap%
\pgfsetmiterjoin%
\definecolor{currentfill}{rgb}{0.754268,0.565033,0.211761}%
\pgfsetfillcolor{currentfill}%
\pgfsetlinewidth{0.000000pt}%
\definecolor{currentstroke}{rgb}{0.000000,0.000000,0.000000}%
\pgfsetstrokecolor{currentstroke}%
\pgfsetstrokeopacity{0.000000}%
\pgfsetdash{}{0pt}%
\pgfpathmoveto{\pgfqpoint{2.894171in}{1.988232in}}%
\pgfpathlineto{\pgfqpoint{2.903107in}{1.988232in}}%
\pgfpathlineto{\pgfqpoint{2.903107in}{2.013678in}}%
\pgfpathlineto{\pgfqpoint{2.894171in}{2.013678in}}%
\pgfpathlineto{\pgfqpoint{2.894171in}{1.988232in}}%
\pgfpathclose%
\pgfusepath{fill}%
\end{pgfscope}%
\begin{pgfscope}%
\pgfpathrectangle{\pgfqpoint{0.697024in}{0.857143in}}{\pgfqpoint{2.627103in}{1.813434in}}%
\pgfusepath{clip}%
\pgfsetbuttcap%
\pgfsetmiterjoin%
\definecolor{currentfill}{rgb}{0.754268,0.565033,0.211761}%
\pgfsetfillcolor{currentfill}%
\pgfsetlinewidth{0.000000pt}%
\definecolor{currentstroke}{rgb}{0.000000,0.000000,0.000000}%
\pgfsetstrokecolor{currentstroke}%
\pgfsetstrokeopacity{0.000000}%
\pgfsetdash{}{0pt}%
\pgfpathmoveto{\pgfqpoint{2.905341in}{2.030786in}}%
\pgfpathlineto{\pgfqpoint{2.914278in}{2.030786in}}%
\pgfpathlineto{\pgfqpoint{2.914278in}{2.033359in}}%
\pgfpathlineto{\pgfqpoint{2.905341in}{2.033359in}}%
\pgfpathlineto{\pgfqpoint{2.905341in}{2.030786in}}%
\pgfpathclose%
\pgfusepath{fill}%
\end{pgfscope}%
\begin{pgfscope}%
\pgfpathrectangle{\pgfqpoint{0.697024in}{0.857143in}}{\pgfqpoint{2.627103in}{1.813434in}}%
\pgfusepath{clip}%
\pgfsetbuttcap%
\pgfsetmiterjoin%
\definecolor{currentfill}{rgb}{0.754268,0.565033,0.211761}%
\pgfsetfillcolor{currentfill}%
\pgfsetlinewidth{0.000000pt}%
\definecolor{currentstroke}{rgb}{0.000000,0.000000,0.000000}%
\pgfsetstrokecolor{currentstroke}%
\pgfsetstrokeopacity{0.000000}%
\pgfsetdash{}{0pt}%
\pgfpathmoveto{\pgfqpoint{2.916512in}{1.951917in}}%
\pgfpathlineto{\pgfqpoint{2.925448in}{1.951917in}}%
\pgfpathlineto{\pgfqpoint{2.925448in}{1.995282in}}%
\pgfpathlineto{\pgfqpoint{2.916512in}{1.995282in}}%
\pgfpathlineto{\pgfqpoint{2.916512in}{1.951917in}}%
\pgfpathclose%
\pgfusepath{fill}%
\end{pgfscope}%
\begin{pgfscope}%
\pgfpathrectangle{\pgfqpoint{0.697024in}{0.857143in}}{\pgfqpoint{2.627103in}{1.813434in}}%
\pgfusepath{clip}%
\pgfsetbuttcap%
\pgfsetmiterjoin%
\definecolor{currentfill}{rgb}{0.754268,0.565033,0.211761}%
\pgfsetfillcolor{currentfill}%
\pgfsetlinewidth{0.000000pt}%
\definecolor{currentstroke}{rgb}{0.000000,0.000000,0.000000}%
\pgfsetstrokecolor{currentstroke}%
\pgfsetstrokeopacity{0.000000}%
\pgfsetdash{}{0pt}%
\pgfpathmoveto{\pgfqpoint{2.927683in}{1.925300in}}%
\pgfpathlineto{\pgfqpoint{2.936619in}{1.925300in}}%
\pgfpathlineto{\pgfqpoint{2.936619in}{1.969980in}}%
\pgfpathlineto{\pgfqpoint{2.927683in}{1.969980in}}%
\pgfpathlineto{\pgfqpoint{2.927683in}{1.925300in}}%
\pgfpathclose%
\pgfusepath{fill}%
\end{pgfscope}%
\begin{pgfscope}%
\pgfpathrectangle{\pgfqpoint{0.697024in}{0.857143in}}{\pgfqpoint{2.627103in}{1.813434in}}%
\pgfusepath{clip}%
\pgfsetbuttcap%
\pgfsetmiterjoin%
\definecolor{currentfill}{rgb}{0.754268,0.565033,0.211761}%
\pgfsetfillcolor{currentfill}%
\pgfsetlinewidth{0.000000pt}%
\definecolor{currentstroke}{rgb}{0.000000,0.000000,0.000000}%
\pgfsetstrokecolor{currentstroke}%
\pgfsetstrokeopacity{0.000000}%
\pgfsetdash{}{0pt}%
\pgfpathmoveto{\pgfqpoint{2.938853in}{1.970502in}}%
\pgfpathlineto{\pgfqpoint{2.947790in}{1.970502in}}%
\pgfpathlineto{\pgfqpoint{2.947790in}{1.993519in}}%
\pgfpathlineto{\pgfqpoint{2.938853in}{1.993519in}}%
\pgfpathlineto{\pgfqpoint{2.938853in}{1.970502in}}%
\pgfpathclose%
\pgfusepath{fill}%
\end{pgfscope}%
\begin{pgfscope}%
\pgfpathrectangle{\pgfqpoint{0.697024in}{0.857143in}}{\pgfqpoint{2.627103in}{1.813434in}}%
\pgfusepath{clip}%
\pgfsetbuttcap%
\pgfsetmiterjoin%
\definecolor{currentfill}{rgb}{0.754268,0.565033,0.211761}%
\pgfsetfillcolor{currentfill}%
\pgfsetlinewidth{0.000000pt}%
\definecolor{currentstroke}{rgb}{0.000000,0.000000,0.000000}%
\pgfsetstrokecolor{currentstroke}%
\pgfsetstrokeopacity{0.000000}%
\pgfsetdash{}{0pt}%
\pgfpathmoveto{\pgfqpoint{2.950024in}{1.931238in}}%
\pgfpathlineto{\pgfqpoint{2.958960in}{1.931238in}}%
\pgfpathlineto{\pgfqpoint{2.958960in}{1.979359in}}%
\pgfpathlineto{\pgfqpoint{2.950024in}{1.979359in}}%
\pgfpathlineto{\pgfqpoint{2.950024in}{1.931238in}}%
\pgfpathclose%
\pgfusepath{fill}%
\end{pgfscope}%
\begin{pgfscope}%
\pgfpathrectangle{\pgfqpoint{0.697024in}{0.857143in}}{\pgfqpoint{2.627103in}{1.813434in}}%
\pgfusepath{clip}%
\pgfsetbuttcap%
\pgfsetmiterjoin%
\definecolor{currentfill}{rgb}{0.754268,0.565033,0.211761}%
\pgfsetfillcolor{currentfill}%
\pgfsetlinewidth{0.000000pt}%
\definecolor{currentstroke}{rgb}{0.000000,0.000000,0.000000}%
\pgfsetstrokecolor{currentstroke}%
\pgfsetstrokeopacity{0.000000}%
\pgfsetdash{}{0pt}%
\pgfpathmoveto{\pgfqpoint{2.961194in}{1.952906in}}%
\pgfpathlineto{\pgfqpoint{2.970131in}{1.952906in}}%
\pgfpathlineto{\pgfqpoint{2.970131in}{1.977774in}}%
\pgfpathlineto{\pgfqpoint{2.961194in}{1.977774in}}%
\pgfpathlineto{\pgfqpoint{2.961194in}{1.952906in}}%
\pgfpathclose%
\pgfusepath{fill}%
\end{pgfscope}%
\begin{pgfscope}%
\pgfpathrectangle{\pgfqpoint{0.697024in}{0.857143in}}{\pgfqpoint{2.627103in}{1.813434in}}%
\pgfusepath{clip}%
\pgfsetbuttcap%
\pgfsetmiterjoin%
\definecolor{currentfill}{rgb}{0.754268,0.565033,0.211761}%
\pgfsetfillcolor{currentfill}%
\pgfsetlinewidth{0.000000pt}%
\definecolor{currentstroke}{rgb}{0.000000,0.000000,0.000000}%
\pgfsetstrokecolor{currentstroke}%
\pgfsetstrokeopacity{0.000000}%
\pgfsetdash{}{0pt}%
\pgfpathmoveto{\pgfqpoint{2.972365in}{1.285688in}}%
\pgfpathlineto{\pgfqpoint{2.981301in}{1.285688in}}%
\pgfpathlineto{\pgfqpoint{2.981301in}{1.270939in}}%
\pgfpathlineto{\pgfqpoint{2.972365in}{1.270939in}}%
\pgfpathlineto{\pgfqpoint{2.972365in}{1.285688in}}%
\pgfpathclose%
\pgfusepath{fill}%
\end{pgfscope}%
\begin{pgfscope}%
\pgfpathrectangle{\pgfqpoint{0.697024in}{0.857143in}}{\pgfqpoint{2.627103in}{1.813434in}}%
\pgfusepath{clip}%
\pgfsetbuttcap%
\pgfsetmiterjoin%
\definecolor{currentfill}{rgb}{0.754268,0.565033,0.211761}%
\pgfsetfillcolor{currentfill}%
\pgfsetlinewidth{0.000000pt}%
\definecolor{currentstroke}{rgb}{0.000000,0.000000,0.000000}%
\pgfsetstrokecolor{currentstroke}%
\pgfsetstrokeopacity{0.000000}%
\pgfsetdash{}{0pt}%
\pgfpathmoveto{\pgfqpoint{2.983536in}{1.240144in}}%
\pgfpathlineto{\pgfqpoint{2.992472in}{1.240144in}}%
\pgfpathlineto{\pgfqpoint{2.992472in}{1.210807in}}%
\pgfpathlineto{\pgfqpoint{2.983536in}{1.210807in}}%
\pgfpathlineto{\pgfqpoint{2.983536in}{1.240144in}}%
\pgfpathclose%
\pgfusepath{fill}%
\end{pgfscope}%
\begin{pgfscope}%
\pgfpathrectangle{\pgfqpoint{0.697024in}{0.857143in}}{\pgfqpoint{2.627103in}{1.813434in}}%
\pgfusepath{clip}%
\pgfsetbuttcap%
\pgfsetmiterjoin%
\definecolor{currentfill}{rgb}{0.754268,0.565033,0.211761}%
\pgfsetfillcolor{currentfill}%
\pgfsetlinewidth{0.000000pt}%
\definecolor{currentstroke}{rgb}{0.000000,0.000000,0.000000}%
\pgfsetstrokecolor{currentstroke}%
\pgfsetstrokeopacity{0.000000}%
\pgfsetdash{}{0pt}%
\pgfpathmoveto{\pgfqpoint{2.994706in}{1.936448in}}%
\pgfpathlineto{\pgfqpoint{3.003643in}{1.936448in}}%
\pgfpathlineto{\pgfqpoint{3.003643in}{1.994849in}}%
\pgfpathlineto{\pgfqpoint{2.994706in}{1.994849in}}%
\pgfpathlineto{\pgfqpoint{2.994706in}{1.936448in}}%
\pgfpathclose%
\pgfusepath{fill}%
\end{pgfscope}%
\begin{pgfscope}%
\pgfpathrectangle{\pgfqpoint{0.697024in}{0.857143in}}{\pgfqpoint{2.627103in}{1.813434in}}%
\pgfusepath{clip}%
\pgfsetbuttcap%
\pgfsetmiterjoin%
\definecolor{currentfill}{rgb}{0.754268,0.565033,0.211761}%
\pgfsetfillcolor{currentfill}%
\pgfsetlinewidth{0.000000pt}%
\definecolor{currentstroke}{rgb}{0.000000,0.000000,0.000000}%
\pgfsetstrokecolor{currentstroke}%
\pgfsetstrokeopacity{0.000000}%
\pgfsetdash{}{0pt}%
\pgfpathmoveto{\pgfqpoint{3.005877in}{2.021404in}}%
\pgfpathlineto{\pgfqpoint{3.014813in}{2.021404in}}%
\pgfpathlineto{\pgfqpoint{3.014813in}{2.030494in}}%
\pgfpathlineto{\pgfqpoint{3.005877in}{2.030494in}}%
\pgfpathlineto{\pgfqpoint{3.005877in}{2.021404in}}%
\pgfpathclose%
\pgfusepath{fill}%
\end{pgfscope}%
\begin{pgfscope}%
\pgfpathrectangle{\pgfqpoint{0.697024in}{0.857143in}}{\pgfqpoint{2.627103in}{1.813434in}}%
\pgfusepath{clip}%
\pgfsetbuttcap%
\pgfsetmiterjoin%
\definecolor{currentfill}{rgb}{0.754268,0.565033,0.211761}%
\pgfsetfillcolor{currentfill}%
\pgfsetlinewidth{0.000000pt}%
\definecolor{currentstroke}{rgb}{0.000000,0.000000,0.000000}%
\pgfsetstrokecolor{currentstroke}%
\pgfsetstrokeopacity{0.000000}%
\pgfsetdash{}{0pt}%
\pgfpathmoveto{\pgfqpoint{3.017047in}{1.140376in}}%
\pgfpathlineto{\pgfqpoint{3.025984in}{1.140376in}}%
\pgfpathlineto{\pgfqpoint{3.025984in}{1.116512in}}%
\pgfpathlineto{\pgfqpoint{3.017047in}{1.116512in}}%
\pgfpathlineto{\pgfqpoint{3.017047in}{1.140376in}}%
\pgfpathclose%
\pgfusepath{fill}%
\end{pgfscope}%
\begin{pgfscope}%
\pgfpathrectangle{\pgfqpoint{0.697024in}{0.857143in}}{\pgfqpoint{2.627103in}{1.813434in}}%
\pgfusepath{clip}%
\pgfsetbuttcap%
\pgfsetmiterjoin%
\definecolor{currentfill}{rgb}{0.754268,0.565033,0.211761}%
\pgfsetfillcolor{currentfill}%
\pgfsetlinewidth{0.000000pt}%
\definecolor{currentstroke}{rgb}{0.000000,0.000000,0.000000}%
\pgfsetstrokecolor{currentstroke}%
\pgfsetstrokeopacity{0.000000}%
\pgfsetdash{}{0pt}%
\pgfpathmoveto{\pgfqpoint{3.028218in}{1.135721in}}%
\pgfpathlineto{\pgfqpoint{3.037155in}{1.135721in}}%
\pgfpathlineto{\pgfqpoint{3.037155in}{1.101577in}}%
\pgfpathlineto{\pgfqpoint{3.028218in}{1.101577in}}%
\pgfpathlineto{\pgfqpoint{3.028218in}{1.135721in}}%
\pgfpathclose%
\pgfusepath{fill}%
\end{pgfscope}%
\begin{pgfscope}%
\pgfpathrectangle{\pgfqpoint{0.697024in}{0.857143in}}{\pgfqpoint{2.627103in}{1.813434in}}%
\pgfusepath{clip}%
\pgfsetbuttcap%
\pgfsetmiterjoin%
\definecolor{currentfill}{rgb}{0.754268,0.565033,0.211761}%
\pgfsetfillcolor{currentfill}%
\pgfsetlinewidth{0.000000pt}%
\definecolor{currentstroke}{rgb}{0.000000,0.000000,0.000000}%
\pgfsetstrokecolor{currentstroke}%
\pgfsetstrokeopacity{0.000000}%
\pgfsetdash{}{0pt}%
\pgfpathmoveto{\pgfqpoint{3.039389in}{2.016049in}}%
\pgfpathlineto{\pgfqpoint{3.048325in}{2.016049in}}%
\pgfpathlineto{\pgfqpoint{3.048325in}{2.086545in}}%
\pgfpathlineto{\pgfqpoint{3.039389in}{2.086545in}}%
\pgfpathlineto{\pgfqpoint{3.039389in}{2.016049in}}%
\pgfpathclose%
\pgfusepath{fill}%
\end{pgfscope}%
\begin{pgfscope}%
\pgfpathrectangle{\pgfqpoint{0.697024in}{0.857143in}}{\pgfqpoint{2.627103in}{1.813434in}}%
\pgfusepath{clip}%
\pgfsetbuttcap%
\pgfsetmiterjoin%
\definecolor{currentfill}{rgb}{0.754268,0.565033,0.211761}%
\pgfsetfillcolor{currentfill}%
\pgfsetlinewidth{0.000000pt}%
\definecolor{currentstroke}{rgb}{0.000000,0.000000,0.000000}%
\pgfsetstrokecolor{currentstroke}%
\pgfsetstrokeopacity{0.000000}%
\pgfsetdash{}{0pt}%
\pgfpathmoveto{\pgfqpoint{3.050559in}{2.129440in}}%
\pgfpathlineto{\pgfqpoint{3.059496in}{2.129440in}}%
\pgfpathlineto{\pgfqpoint{3.059496in}{2.132786in}}%
\pgfpathlineto{\pgfqpoint{3.050559in}{2.132786in}}%
\pgfpathlineto{\pgfqpoint{3.050559in}{2.129440in}}%
\pgfpathclose%
\pgfusepath{fill}%
\end{pgfscope}%
\begin{pgfscope}%
\pgfpathrectangle{\pgfqpoint{0.697024in}{0.857143in}}{\pgfqpoint{2.627103in}{1.813434in}}%
\pgfusepath{clip}%
\pgfsetbuttcap%
\pgfsetmiterjoin%
\definecolor{currentfill}{rgb}{0.754268,0.565033,0.211761}%
\pgfsetfillcolor{currentfill}%
\pgfsetlinewidth{0.000000pt}%
\definecolor{currentstroke}{rgb}{0.000000,0.000000,0.000000}%
\pgfsetstrokecolor{currentstroke}%
\pgfsetstrokeopacity{0.000000}%
\pgfsetdash{}{0pt}%
\pgfpathmoveto{\pgfqpoint{3.061730in}{2.066726in}}%
\pgfpathlineto{\pgfqpoint{3.070666in}{2.066726in}}%
\pgfpathlineto{\pgfqpoint{3.070666in}{2.102346in}}%
\pgfpathlineto{\pgfqpoint{3.061730in}{2.102346in}}%
\pgfpathlineto{\pgfqpoint{3.061730in}{2.066726in}}%
\pgfpathclose%
\pgfusepath{fill}%
\end{pgfscope}%
\begin{pgfscope}%
\pgfpathrectangle{\pgfqpoint{0.697024in}{0.857143in}}{\pgfqpoint{2.627103in}{1.813434in}}%
\pgfusepath{clip}%
\pgfsetbuttcap%
\pgfsetmiterjoin%
\definecolor{currentfill}{rgb}{0.754268,0.565033,0.211761}%
\pgfsetfillcolor{currentfill}%
\pgfsetlinewidth{0.000000pt}%
\definecolor{currentstroke}{rgb}{0.000000,0.000000,0.000000}%
\pgfsetstrokecolor{currentstroke}%
\pgfsetstrokeopacity{0.000000}%
\pgfsetdash{}{0pt}%
\pgfpathmoveto{\pgfqpoint{3.072900in}{2.070502in}}%
\pgfpathlineto{\pgfqpoint{3.081837in}{2.070502in}}%
\pgfpathlineto{\pgfqpoint{3.081837in}{2.101053in}}%
\pgfpathlineto{\pgfqpoint{3.072900in}{2.101053in}}%
\pgfpathlineto{\pgfqpoint{3.072900in}{2.070502in}}%
\pgfpathclose%
\pgfusepath{fill}%
\end{pgfscope}%
\begin{pgfscope}%
\pgfpathrectangle{\pgfqpoint{0.697024in}{0.857143in}}{\pgfqpoint{2.627103in}{1.813434in}}%
\pgfusepath{clip}%
\pgfsetbuttcap%
\pgfsetmiterjoin%
\definecolor{currentfill}{rgb}{0.754268,0.565033,0.211761}%
\pgfsetfillcolor{currentfill}%
\pgfsetlinewidth{0.000000pt}%
\definecolor{currentstroke}{rgb}{0.000000,0.000000,0.000000}%
\pgfsetstrokecolor{currentstroke}%
\pgfsetstrokeopacity{0.000000}%
\pgfsetdash{}{0pt}%
\pgfpathmoveto{\pgfqpoint{3.084071in}{1.137192in}}%
\pgfpathlineto{\pgfqpoint{3.093008in}{1.137192in}}%
\pgfpathlineto{\pgfqpoint{3.093008in}{1.128698in}}%
\pgfpathlineto{\pgfqpoint{3.084071in}{1.128698in}}%
\pgfpathlineto{\pgfqpoint{3.084071in}{1.137192in}}%
\pgfpathclose%
\pgfusepath{fill}%
\end{pgfscope}%
\begin{pgfscope}%
\pgfpathrectangle{\pgfqpoint{0.697024in}{0.857143in}}{\pgfqpoint{2.627103in}{1.813434in}}%
\pgfusepath{clip}%
\pgfsetbuttcap%
\pgfsetmiterjoin%
\definecolor{currentfill}{rgb}{0.754268,0.565033,0.211761}%
\pgfsetfillcolor{currentfill}%
\pgfsetlinewidth{0.000000pt}%
\definecolor{currentstroke}{rgb}{0.000000,0.000000,0.000000}%
\pgfsetstrokecolor{currentstroke}%
\pgfsetstrokeopacity{0.000000}%
\pgfsetdash{}{0pt}%
\pgfpathmoveto{\pgfqpoint{3.095242in}{2.087776in}}%
\pgfpathlineto{\pgfqpoint{3.104178in}{2.087776in}}%
\pgfpathlineto{\pgfqpoint{3.104178in}{2.111369in}}%
\pgfpathlineto{\pgfqpoint{3.095242in}{2.111369in}}%
\pgfpathlineto{\pgfqpoint{3.095242in}{2.087776in}}%
\pgfpathclose%
\pgfusepath{fill}%
\end{pgfscope}%
\begin{pgfscope}%
\pgfpathrectangle{\pgfqpoint{0.697024in}{0.857143in}}{\pgfqpoint{2.627103in}{1.813434in}}%
\pgfusepath{clip}%
\pgfsetbuttcap%
\pgfsetmiterjoin%
\definecolor{currentfill}{rgb}{0.754268,0.565033,0.211761}%
\pgfsetfillcolor{currentfill}%
\pgfsetlinewidth{0.000000pt}%
\definecolor{currentstroke}{rgb}{0.000000,0.000000,0.000000}%
\pgfsetstrokecolor{currentstroke}%
\pgfsetstrokeopacity{0.000000}%
\pgfsetdash{}{0pt}%
\pgfpathmoveto{\pgfqpoint{3.106412in}{2.042483in}}%
\pgfpathlineto{\pgfqpoint{3.115349in}{2.042483in}}%
\pgfpathlineto{\pgfqpoint{3.115349in}{2.077483in}}%
\pgfpathlineto{\pgfqpoint{3.106412in}{2.077483in}}%
\pgfpathlineto{\pgfqpoint{3.106412in}{2.042483in}}%
\pgfpathclose%
\pgfusepath{fill}%
\end{pgfscope}%
\begin{pgfscope}%
\pgfpathrectangle{\pgfqpoint{0.697024in}{0.857143in}}{\pgfqpoint{2.627103in}{1.813434in}}%
\pgfusepath{clip}%
\pgfsetbuttcap%
\pgfsetmiterjoin%
\definecolor{currentfill}{rgb}{0.754268,0.565033,0.211761}%
\pgfsetfillcolor{currentfill}%
\pgfsetlinewidth{0.000000pt}%
\definecolor{currentstroke}{rgb}{0.000000,0.000000,0.000000}%
\pgfsetstrokecolor{currentstroke}%
\pgfsetstrokeopacity{0.000000}%
\pgfsetdash{}{0pt}%
\pgfpathmoveto{\pgfqpoint{3.117583in}{2.072700in}}%
\pgfpathlineto{\pgfqpoint{3.126519in}{2.072700in}}%
\pgfpathlineto{\pgfqpoint{3.126519in}{2.102081in}}%
\pgfpathlineto{\pgfqpoint{3.117583in}{2.102081in}}%
\pgfpathlineto{\pgfqpoint{3.117583in}{2.072700in}}%
\pgfpathclose%
\pgfusepath{fill}%
\end{pgfscope}%
\begin{pgfscope}%
\pgfpathrectangle{\pgfqpoint{0.697024in}{0.857143in}}{\pgfqpoint{2.627103in}{1.813434in}}%
\pgfusepath{clip}%
\pgfsetbuttcap%
\pgfsetmiterjoin%
\definecolor{currentfill}{rgb}{0.754268,0.565033,0.211761}%
\pgfsetfillcolor{currentfill}%
\pgfsetlinewidth{0.000000pt}%
\definecolor{currentstroke}{rgb}{0.000000,0.000000,0.000000}%
\pgfsetstrokecolor{currentstroke}%
\pgfsetstrokeopacity{0.000000}%
\pgfsetdash{}{0pt}%
\pgfpathmoveto{\pgfqpoint{3.128753in}{2.069804in}}%
\pgfpathlineto{\pgfqpoint{3.137690in}{2.069804in}}%
\pgfpathlineto{\pgfqpoint{3.137690in}{2.101368in}}%
\pgfpathlineto{\pgfqpoint{3.128753in}{2.101368in}}%
\pgfpathlineto{\pgfqpoint{3.128753in}{2.069804in}}%
\pgfpathclose%
\pgfusepath{fill}%
\end{pgfscope}%
\begin{pgfscope}%
\pgfpathrectangle{\pgfqpoint{0.697024in}{0.857143in}}{\pgfqpoint{2.627103in}{1.813434in}}%
\pgfusepath{clip}%
\pgfsetbuttcap%
\pgfsetmiterjoin%
\definecolor{currentfill}{rgb}{0.754268,0.565033,0.211761}%
\pgfsetfillcolor{currentfill}%
\pgfsetlinewidth{0.000000pt}%
\definecolor{currentstroke}{rgb}{0.000000,0.000000,0.000000}%
\pgfsetstrokecolor{currentstroke}%
\pgfsetstrokeopacity{0.000000}%
\pgfsetdash{}{0pt}%
\pgfpathmoveto{\pgfqpoint{3.139924in}{1.142039in}}%
\pgfpathlineto{\pgfqpoint{3.148861in}{1.142039in}}%
\pgfpathlineto{\pgfqpoint{3.148861in}{1.131882in}}%
\pgfpathlineto{\pgfqpoint{3.139924in}{1.131882in}}%
\pgfpathlineto{\pgfqpoint{3.139924in}{1.142039in}}%
\pgfpathclose%
\pgfusepath{fill}%
\end{pgfscope}%
\begin{pgfscope}%
\pgfpathrectangle{\pgfqpoint{0.697024in}{0.857143in}}{\pgfqpoint{2.627103in}{1.813434in}}%
\pgfusepath{clip}%
\pgfsetbuttcap%
\pgfsetmiterjoin%
\definecolor{currentfill}{rgb}{0.754268,0.565033,0.211761}%
\pgfsetfillcolor{currentfill}%
\pgfsetlinewidth{0.000000pt}%
\definecolor{currentstroke}{rgb}{0.000000,0.000000,0.000000}%
\pgfsetstrokecolor{currentstroke}%
\pgfsetstrokeopacity{0.000000}%
\pgfsetdash{}{0pt}%
\pgfpathmoveto{\pgfqpoint{3.151095in}{1.084308in}}%
\pgfpathlineto{\pgfqpoint{3.160031in}{1.084308in}}%
\pgfpathlineto{\pgfqpoint{3.160031in}{1.072988in}}%
\pgfpathlineto{\pgfqpoint{3.151095in}{1.072988in}}%
\pgfpathlineto{\pgfqpoint{3.151095in}{1.084308in}}%
\pgfpathclose%
\pgfusepath{fill}%
\end{pgfscope}%
\begin{pgfscope}%
\pgfpathrectangle{\pgfqpoint{0.697024in}{0.857143in}}{\pgfqpoint{2.627103in}{1.813434in}}%
\pgfusepath{clip}%
\pgfsetbuttcap%
\pgfsetmiterjoin%
\definecolor{currentfill}{rgb}{0.754268,0.565033,0.211761}%
\pgfsetfillcolor{currentfill}%
\pgfsetlinewidth{0.000000pt}%
\definecolor{currentstroke}{rgb}{0.000000,0.000000,0.000000}%
\pgfsetstrokecolor{currentstroke}%
\pgfsetstrokeopacity{0.000000}%
\pgfsetdash{}{0pt}%
\pgfpathmoveto{\pgfqpoint{3.162265in}{1.011978in}}%
\pgfpathlineto{\pgfqpoint{3.171202in}{1.011978in}}%
\pgfpathlineto{\pgfqpoint{3.171202in}{0.982157in}}%
\pgfpathlineto{\pgfqpoint{3.162265in}{0.982157in}}%
\pgfpathlineto{\pgfqpoint{3.162265in}{1.011978in}}%
\pgfpathclose%
\pgfusepath{fill}%
\end{pgfscope}%
\begin{pgfscope}%
\pgfpathrectangle{\pgfqpoint{0.697024in}{0.857143in}}{\pgfqpoint{2.627103in}{1.813434in}}%
\pgfusepath{clip}%
\pgfsetbuttcap%
\pgfsetmiterjoin%
\definecolor{currentfill}{rgb}{0.754268,0.565033,0.211761}%
\pgfsetfillcolor{currentfill}%
\pgfsetlinewidth{0.000000pt}%
\definecolor{currentstroke}{rgb}{0.000000,0.000000,0.000000}%
\pgfsetstrokecolor{currentstroke}%
\pgfsetstrokeopacity{0.000000}%
\pgfsetdash{}{0pt}%
\pgfpathmoveto{\pgfqpoint{3.173436in}{2.217730in}}%
\pgfpathlineto{\pgfqpoint{3.182372in}{2.217730in}}%
\pgfpathlineto{\pgfqpoint{3.182372in}{2.220487in}}%
\pgfpathlineto{\pgfqpoint{3.173436in}{2.220487in}}%
\pgfpathlineto{\pgfqpoint{3.173436in}{2.217730in}}%
\pgfpathclose%
\pgfusepath{fill}%
\end{pgfscope}%
\begin{pgfscope}%
\pgfpathrectangle{\pgfqpoint{0.697024in}{0.857143in}}{\pgfqpoint{2.627103in}{1.813434in}}%
\pgfusepath{clip}%
\pgfsetbuttcap%
\pgfsetmiterjoin%
\definecolor{currentfill}{rgb}{0.754268,0.565033,0.211761}%
\pgfsetfillcolor{currentfill}%
\pgfsetlinewidth{0.000000pt}%
\definecolor{currentstroke}{rgb}{0.000000,0.000000,0.000000}%
\pgfsetstrokecolor{currentstroke}%
\pgfsetstrokeopacity{0.000000}%
\pgfsetdash{}{0pt}%
\pgfpathmoveto{\pgfqpoint{3.184607in}{1.021959in}}%
\pgfpathlineto{\pgfqpoint{3.193543in}{1.021959in}}%
\pgfpathlineto{\pgfqpoint{3.193543in}{0.997746in}}%
\pgfpathlineto{\pgfqpoint{3.184607in}{0.997746in}}%
\pgfpathlineto{\pgfqpoint{3.184607in}{1.021959in}}%
\pgfpathclose%
\pgfusepath{fill}%
\end{pgfscope}%
\begin{pgfscope}%
\pgfpathrectangle{\pgfqpoint{0.697024in}{0.857143in}}{\pgfqpoint{2.627103in}{1.813434in}}%
\pgfusepath{clip}%
\pgfsetbuttcap%
\pgfsetmiterjoin%
\definecolor{currentfill}{rgb}{0.754268,0.565033,0.211761}%
\pgfsetfillcolor{currentfill}%
\pgfsetlinewidth{0.000000pt}%
\definecolor{currentstroke}{rgb}{0.000000,0.000000,0.000000}%
\pgfsetstrokecolor{currentstroke}%
\pgfsetstrokeopacity{0.000000}%
\pgfsetdash{}{0pt}%
\pgfpathmoveto{\pgfqpoint{3.195777in}{1.004448in}}%
\pgfpathlineto{\pgfqpoint{3.204714in}{1.004448in}}%
\pgfpathlineto{\pgfqpoint{3.204714in}{0.995091in}}%
\pgfpathlineto{\pgfqpoint{3.195777in}{0.995091in}}%
\pgfpathlineto{\pgfqpoint{3.195777in}{1.004448in}}%
\pgfpathclose%
\pgfusepath{fill}%
\end{pgfscope}%
\begin{pgfscope}%
\pgfpathrectangle{\pgfqpoint{0.697024in}{0.857143in}}{\pgfqpoint{2.627103in}{1.813434in}}%
\pgfusepath{clip}%
\pgfsetbuttcap%
\pgfsetmiterjoin%
\definecolor{currentfill}{rgb}{0.950697,0.616649,0.428624}%
\pgfsetfillcolor{currentfill}%
\pgfsetlinewidth{0.000000pt}%
\definecolor{currentstroke}{rgb}{0.000000,0.000000,0.000000}%
\pgfsetstrokecolor{currentstroke}%
\pgfsetstrokeopacity{0.000000}%
\pgfsetdash{}{0pt}%
\pgfpathmoveto{\pgfqpoint{0.816438in}{1.894532in}}%
\pgfpathlineto{\pgfqpoint{0.825375in}{1.894532in}}%
\pgfpathlineto{\pgfqpoint{0.825375in}{1.928362in}}%
\pgfpathlineto{\pgfqpoint{0.816438in}{1.928362in}}%
\pgfpathlineto{\pgfqpoint{0.816438in}{1.894532in}}%
\pgfpathclose%
\pgfusepath{fill}%
\end{pgfscope}%
\begin{pgfscope}%
\pgfpathrectangle{\pgfqpoint{0.697024in}{0.857143in}}{\pgfqpoint{2.627103in}{1.813434in}}%
\pgfusepath{clip}%
\pgfsetbuttcap%
\pgfsetmiterjoin%
\definecolor{currentfill}{rgb}{0.950697,0.616649,0.428624}%
\pgfsetfillcolor{currentfill}%
\pgfsetlinewidth{0.000000pt}%
\definecolor{currentstroke}{rgb}{0.000000,0.000000,0.000000}%
\pgfsetstrokecolor{currentstroke}%
\pgfsetstrokeopacity{0.000000}%
\pgfsetdash{}{0pt}%
\pgfpathmoveto{\pgfqpoint{0.827609in}{1.923483in}}%
\pgfpathlineto{\pgfqpoint{0.836545in}{1.923483in}}%
\pgfpathlineto{\pgfqpoint{0.836545in}{1.934328in}}%
\pgfpathlineto{\pgfqpoint{0.827609in}{1.934328in}}%
\pgfpathlineto{\pgfqpoint{0.827609in}{1.923483in}}%
\pgfpathclose%
\pgfusepath{fill}%
\end{pgfscope}%
\begin{pgfscope}%
\pgfpathrectangle{\pgfqpoint{0.697024in}{0.857143in}}{\pgfqpoint{2.627103in}{1.813434in}}%
\pgfusepath{clip}%
\pgfsetbuttcap%
\pgfsetmiterjoin%
\definecolor{currentfill}{rgb}{0.950697,0.616649,0.428624}%
\pgfsetfillcolor{currentfill}%
\pgfsetlinewidth{0.000000pt}%
\definecolor{currentstroke}{rgb}{0.000000,0.000000,0.000000}%
\pgfsetstrokecolor{currentstroke}%
\pgfsetstrokeopacity{0.000000}%
\pgfsetdash{}{0pt}%
\pgfpathmoveto{\pgfqpoint{0.838779in}{1.488412in}}%
\pgfpathlineto{\pgfqpoint{0.847716in}{1.488412in}}%
\pgfpathlineto{\pgfqpoint{0.847716in}{1.476023in}}%
\pgfpathlineto{\pgfqpoint{0.838779in}{1.476023in}}%
\pgfpathlineto{\pgfqpoint{0.838779in}{1.488412in}}%
\pgfpathclose%
\pgfusepath{fill}%
\end{pgfscope}%
\begin{pgfscope}%
\pgfpathrectangle{\pgfqpoint{0.697024in}{0.857143in}}{\pgfqpoint{2.627103in}{1.813434in}}%
\pgfusepath{clip}%
\pgfsetbuttcap%
\pgfsetmiterjoin%
\definecolor{currentfill}{rgb}{0.950697,0.616649,0.428624}%
\pgfsetfillcolor{currentfill}%
\pgfsetlinewidth{0.000000pt}%
\definecolor{currentstroke}{rgb}{0.000000,0.000000,0.000000}%
\pgfsetstrokecolor{currentstroke}%
\pgfsetstrokeopacity{0.000000}%
\pgfsetdash{}{0pt}%
\pgfpathmoveto{\pgfqpoint{0.849950in}{1.553819in}}%
\pgfpathlineto{\pgfqpoint{0.858886in}{1.553819in}}%
\pgfpathlineto{\pgfqpoint{0.858886in}{1.524486in}}%
\pgfpathlineto{\pgfqpoint{0.849950in}{1.524486in}}%
\pgfpathlineto{\pgfqpoint{0.849950in}{1.553819in}}%
\pgfpathclose%
\pgfusepath{fill}%
\end{pgfscope}%
\begin{pgfscope}%
\pgfpathrectangle{\pgfqpoint{0.697024in}{0.857143in}}{\pgfqpoint{2.627103in}{1.813434in}}%
\pgfusepath{clip}%
\pgfsetbuttcap%
\pgfsetmiterjoin%
\definecolor{currentfill}{rgb}{0.950697,0.616649,0.428624}%
\pgfsetfillcolor{currentfill}%
\pgfsetlinewidth{0.000000pt}%
\definecolor{currentstroke}{rgb}{0.000000,0.000000,0.000000}%
\pgfsetstrokecolor{currentstroke}%
\pgfsetstrokeopacity{0.000000}%
\pgfsetdash{}{0pt}%
\pgfpathmoveto{\pgfqpoint{0.861121in}{1.470702in}}%
\pgfpathlineto{\pgfqpoint{0.870057in}{1.470702in}}%
\pgfpathlineto{\pgfqpoint{0.870057in}{1.466491in}}%
\pgfpathlineto{\pgfqpoint{0.861121in}{1.466491in}}%
\pgfpathlineto{\pgfqpoint{0.861121in}{1.470702in}}%
\pgfpathclose%
\pgfusepath{fill}%
\end{pgfscope}%
\begin{pgfscope}%
\pgfpathrectangle{\pgfqpoint{0.697024in}{0.857143in}}{\pgfqpoint{2.627103in}{1.813434in}}%
\pgfusepath{clip}%
\pgfsetbuttcap%
\pgfsetmiterjoin%
\definecolor{currentfill}{rgb}{0.950697,0.616649,0.428624}%
\pgfsetfillcolor{currentfill}%
\pgfsetlinewidth{0.000000pt}%
\definecolor{currentstroke}{rgb}{0.000000,0.000000,0.000000}%
\pgfsetstrokecolor{currentstroke}%
\pgfsetstrokeopacity{0.000000}%
\pgfsetdash{}{0pt}%
\pgfpathmoveto{\pgfqpoint{0.872291in}{1.935839in}}%
\pgfpathlineto{\pgfqpoint{0.881228in}{1.935839in}}%
\pgfpathlineto{\pgfqpoint{0.881228in}{1.947040in}}%
\pgfpathlineto{\pgfqpoint{0.872291in}{1.947040in}}%
\pgfpathlineto{\pgfqpoint{0.872291in}{1.935839in}}%
\pgfpathclose%
\pgfusepath{fill}%
\end{pgfscope}%
\begin{pgfscope}%
\pgfpathrectangle{\pgfqpoint{0.697024in}{0.857143in}}{\pgfqpoint{2.627103in}{1.813434in}}%
\pgfusepath{clip}%
\pgfsetbuttcap%
\pgfsetmiterjoin%
\definecolor{currentfill}{rgb}{0.950697,0.616649,0.428624}%
\pgfsetfillcolor{currentfill}%
\pgfsetlinewidth{0.000000pt}%
\definecolor{currentstroke}{rgb}{0.000000,0.000000,0.000000}%
\pgfsetstrokecolor{currentstroke}%
\pgfsetstrokeopacity{0.000000}%
\pgfsetdash{}{0pt}%
\pgfpathmoveto{\pgfqpoint{0.883462in}{2.023885in}}%
\pgfpathlineto{\pgfqpoint{0.892398in}{2.023885in}}%
\pgfpathlineto{\pgfqpoint{0.892398in}{2.026045in}}%
\pgfpathlineto{\pgfqpoint{0.883462in}{2.026045in}}%
\pgfpathlineto{\pgfqpoint{0.883462in}{2.023885in}}%
\pgfpathclose%
\pgfusepath{fill}%
\end{pgfscope}%
\begin{pgfscope}%
\pgfpathrectangle{\pgfqpoint{0.697024in}{0.857143in}}{\pgfqpoint{2.627103in}{1.813434in}}%
\pgfusepath{clip}%
\pgfsetbuttcap%
\pgfsetmiterjoin%
\definecolor{currentfill}{rgb}{0.950697,0.616649,0.428624}%
\pgfsetfillcolor{currentfill}%
\pgfsetlinewidth{0.000000pt}%
\definecolor{currentstroke}{rgb}{0.000000,0.000000,0.000000}%
\pgfsetstrokecolor{currentstroke}%
\pgfsetstrokeopacity{0.000000}%
\pgfsetdash{}{0pt}%
\pgfpathmoveto{\pgfqpoint{0.894632in}{1.535240in}}%
\pgfpathlineto{\pgfqpoint{0.903569in}{1.535240in}}%
\pgfpathlineto{\pgfqpoint{0.903569in}{1.509115in}}%
\pgfpathlineto{\pgfqpoint{0.894632in}{1.509115in}}%
\pgfpathlineto{\pgfqpoint{0.894632in}{1.535240in}}%
\pgfpathclose%
\pgfusepath{fill}%
\end{pgfscope}%
\begin{pgfscope}%
\pgfpathrectangle{\pgfqpoint{0.697024in}{0.857143in}}{\pgfqpoint{2.627103in}{1.813434in}}%
\pgfusepath{clip}%
\pgfsetbuttcap%
\pgfsetmiterjoin%
\definecolor{currentfill}{rgb}{0.950697,0.616649,0.428624}%
\pgfsetfillcolor{currentfill}%
\pgfsetlinewidth{0.000000pt}%
\definecolor{currentstroke}{rgb}{0.000000,0.000000,0.000000}%
\pgfsetstrokecolor{currentstroke}%
\pgfsetstrokeopacity{0.000000}%
\pgfsetdash{}{0pt}%
\pgfpathmoveto{\pgfqpoint{0.905803in}{1.595804in}}%
\pgfpathlineto{\pgfqpoint{0.914739in}{1.595804in}}%
\pgfpathlineto{\pgfqpoint{0.914739in}{1.567093in}}%
\pgfpathlineto{\pgfqpoint{0.905803in}{1.567093in}}%
\pgfpathlineto{\pgfqpoint{0.905803in}{1.595804in}}%
\pgfpathclose%
\pgfusepath{fill}%
\end{pgfscope}%
\begin{pgfscope}%
\pgfpathrectangle{\pgfqpoint{0.697024in}{0.857143in}}{\pgfqpoint{2.627103in}{1.813434in}}%
\pgfusepath{clip}%
\pgfsetbuttcap%
\pgfsetmiterjoin%
\definecolor{currentfill}{rgb}{0.950697,0.616649,0.428624}%
\pgfsetfillcolor{currentfill}%
\pgfsetlinewidth{0.000000pt}%
\definecolor{currentstroke}{rgb}{0.000000,0.000000,0.000000}%
\pgfsetstrokecolor{currentstroke}%
\pgfsetstrokeopacity{0.000000}%
\pgfsetdash{}{0pt}%
\pgfpathmoveto{\pgfqpoint{0.916974in}{1.593487in}}%
\pgfpathlineto{\pgfqpoint{0.925910in}{1.593487in}}%
\pgfpathlineto{\pgfqpoint{0.925910in}{1.558981in}}%
\pgfpathlineto{\pgfqpoint{0.916974in}{1.558981in}}%
\pgfpathlineto{\pgfqpoint{0.916974in}{1.593487in}}%
\pgfpathclose%
\pgfusepath{fill}%
\end{pgfscope}%
\begin{pgfscope}%
\pgfpathrectangle{\pgfqpoint{0.697024in}{0.857143in}}{\pgfqpoint{2.627103in}{1.813434in}}%
\pgfusepath{clip}%
\pgfsetbuttcap%
\pgfsetmiterjoin%
\definecolor{currentfill}{rgb}{0.950697,0.616649,0.428624}%
\pgfsetfillcolor{currentfill}%
\pgfsetlinewidth{0.000000pt}%
\definecolor{currentstroke}{rgb}{0.000000,0.000000,0.000000}%
\pgfsetstrokecolor{currentstroke}%
\pgfsetstrokeopacity{0.000000}%
\pgfsetdash{}{0pt}%
\pgfpathmoveto{\pgfqpoint{0.928144in}{1.598060in}}%
\pgfpathlineto{\pgfqpoint{0.937081in}{1.598060in}}%
\pgfpathlineto{\pgfqpoint{0.937081in}{1.554378in}}%
\pgfpathlineto{\pgfqpoint{0.928144in}{1.554378in}}%
\pgfpathlineto{\pgfqpoint{0.928144in}{1.598060in}}%
\pgfpathclose%
\pgfusepath{fill}%
\end{pgfscope}%
\begin{pgfscope}%
\pgfpathrectangle{\pgfqpoint{0.697024in}{0.857143in}}{\pgfqpoint{2.627103in}{1.813434in}}%
\pgfusepath{clip}%
\pgfsetbuttcap%
\pgfsetmiterjoin%
\definecolor{currentfill}{rgb}{0.950697,0.616649,0.428624}%
\pgfsetfillcolor{currentfill}%
\pgfsetlinewidth{0.000000pt}%
\definecolor{currentstroke}{rgb}{0.000000,0.000000,0.000000}%
\pgfsetstrokecolor{currentstroke}%
\pgfsetstrokeopacity{0.000000}%
\pgfsetdash{}{0pt}%
\pgfpathmoveto{\pgfqpoint{0.939315in}{1.583232in}}%
\pgfpathlineto{\pgfqpoint{0.948251in}{1.583232in}}%
\pgfpathlineto{\pgfqpoint{0.948251in}{1.515205in}}%
\pgfpathlineto{\pgfqpoint{0.939315in}{1.515205in}}%
\pgfpathlineto{\pgfqpoint{0.939315in}{1.583232in}}%
\pgfpathclose%
\pgfusepath{fill}%
\end{pgfscope}%
\begin{pgfscope}%
\pgfpathrectangle{\pgfqpoint{0.697024in}{0.857143in}}{\pgfqpoint{2.627103in}{1.813434in}}%
\pgfusepath{clip}%
\pgfsetbuttcap%
\pgfsetmiterjoin%
\definecolor{currentfill}{rgb}{0.950697,0.616649,0.428624}%
\pgfsetfillcolor{currentfill}%
\pgfsetlinewidth{0.000000pt}%
\definecolor{currentstroke}{rgb}{0.000000,0.000000,0.000000}%
\pgfsetstrokecolor{currentstroke}%
\pgfsetstrokeopacity{0.000000}%
\pgfsetdash{}{0pt}%
\pgfpathmoveto{\pgfqpoint{0.950485in}{1.617428in}}%
\pgfpathlineto{\pgfqpoint{0.959422in}{1.617428in}}%
\pgfpathlineto{\pgfqpoint{0.959422in}{1.557181in}}%
\pgfpathlineto{\pgfqpoint{0.950485in}{1.557181in}}%
\pgfpathlineto{\pgfqpoint{0.950485in}{1.617428in}}%
\pgfpathclose%
\pgfusepath{fill}%
\end{pgfscope}%
\begin{pgfscope}%
\pgfpathrectangle{\pgfqpoint{0.697024in}{0.857143in}}{\pgfqpoint{2.627103in}{1.813434in}}%
\pgfusepath{clip}%
\pgfsetbuttcap%
\pgfsetmiterjoin%
\definecolor{currentfill}{rgb}{0.950697,0.616649,0.428624}%
\pgfsetfillcolor{currentfill}%
\pgfsetlinewidth{0.000000pt}%
\definecolor{currentstroke}{rgb}{0.000000,0.000000,0.000000}%
\pgfsetstrokecolor{currentstroke}%
\pgfsetstrokeopacity{0.000000}%
\pgfsetdash{}{0pt}%
\pgfpathmoveto{\pgfqpoint{0.961656in}{1.547054in}}%
\pgfpathlineto{\pgfqpoint{0.970593in}{1.547054in}}%
\pgfpathlineto{\pgfqpoint{0.970593in}{1.479671in}}%
\pgfpathlineto{\pgfqpoint{0.961656in}{1.479671in}}%
\pgfpathlineto{\pgfqpoint{0.961656in}{1.547054in}}%
\pgfpathclose%
\pgfusepath{fill}%
\end{pgfscope}%
\begin{pgfscope}%
\pgfpathrectangle{\pgfqpoint{0.697024in}{0.857143in}}{\pgfqpoint{2.627103in}{1.813434in}}%
\pgfusepath{clip}%
\pgfsetbuttcap%
\pgfsetmiterjoin%
\definecolor{currentfill}{rgb}{0.950697,0.616649,0.428624}%
\pgfsetfillcolor{currentfill}%
\pgfsetlinewidth{0.000000pt}%
\definecolor{currentstroke}{rgb}{0.000000,0.000000,0.000000}%
\pgfsetstrokecolor{currentstroke}%
\pgfsetstrokeopacity{0.000000}%
\pgfsetdash{}{0pt}%
\pgfpathmoveto{\pgfqpoint{0.972827in}{1.649970in}}%
\pgfpathlineto{\pgfqpoint{0.981763in}{1.649970in}}%
\pgfpathlineto{\pgfqpoint{0.981763in}{1.585661in}}%
\pgfpathlineto{\pgfqpoint{0.972827in}{1.585661in}}%
\pgfpathlineto{\pgfqpoint{0.972827in}{1.649970in}}%
\pgfpathclose%
\pgfusepath{fill}%
\end{pgfscope}%
\begin{pgfscope}%
\pgfpathrectangle{\pgfqpoint{0.697024in}{0.857143in}}{\pgfqpoint{2.627103in}{1.813434in}}%
\pgfusepath{clip}%
\pgfsetbuttcap%
\pgfsetmiterjoin%
\definecolor{currentfill}{rgb}{0.950697,0.616649,0.428624}%
\pgfsetfillcolor{currentfill}%
\pgfsetlinewidth{0.000000pt}%
\definecolor{currentstroke}{rgb}{0.000000,0.000000,0.000000}%
\pgfsetstrokecolor{currentstroke}%
\pgfsetstrokeopacity{0.000000}%
\pgfsetdash{}{0pt}%
\pgfpathmoveto{\pgfqpoint{0.983997in}{1.562055in}}%
\pgfpathlineto{\pgfqpoint{0.992934in}{1.562055in}}%
\pgfpathlineto{\pgfqpoint{0.992934in}{1.492763in}}%
\pgfpathlineto{\pgfqpoint{0.983997in}{1.492763in}}%
\pgfpathlineto{\pgfqpoint{0.983997in}{1.562055in}}%
\pgfpathclose%
\pgfusepath{fill}%
\end{pgfscope}%
\begin{pgfscope}%
\pgfpathrectangle{\pgfqpoint{0.697024in}{0.857143in}}{\pgfqpoint{2.627103in}{1.813434in}}%
\pgfusepath{clip}%
\pgfsetbuttcap%
\pgfsetmiterjoin%
\definecolor{currentfill}{rgb}{0.950697,0.616649,0.428624}%
\pgfsetfillcolor{currentfill}%
\pgfsetlinewidth{0.000000pt}%
\definecolor{currentstroke}{rgb}{0.000000,0.000000,0.000000}%
\pgfsetstrokecolor{currentstroke}%
\pgfsetstrokeopacity{0.000000}%
\pgfsetdash{}{0pt}%
\pgfpathmoveto{\pgfqpoint{0.995168in}{1.422896in}}%
\pgfpathlineto{\pgfqpoint{1.004104in}{1.422896in}}%
\pgfpathlineto{\pgfqpoint{1.004104in}{1.359766in}}%
\pgfpathlineto{\pgfqpoint{0.995168in}{1.359766in}}%
\pgfpathlineto{\pgfqpoint{0.995168in}{1.422896in}}%
\pgfpathclose%
\pgfusepath{fill}%
\end{pgfscope}%
\begin{pgfscope}%
\pgfpathrectangle{\pgfqpoint{0.697024in}{0.857143in}}{\pgfqpoint{2.627103in}{1.813434in}}%
\pgfusepath{clip}%
\pgfsetbuttcap%
\pgfsetmiterjoin%
\definecolor{currentfill}{rgb}{0.950697,0.616649,0.428624}%
\pgfsetfillcolor{currentfill}%
\pgfsetlinewidth{0.000000pt}%
\definecolor{currentstroke}{rgb}{0.000000,0.000000,0.000000}%
\pgfsetstrokecolor{currentstroke}%
\pgfsetstrokeopacity{0.000000}%
\pgfsetdash{}{0pt}%
\pgfpathmoveto{\pgfqpoint{1.006338in}{1.482814in}}%
\pgfpathlineto{\pgfqpoint{1.015275in}{1.482814in}}%
\pgfpathlineto{\pgfqpoint{1.015275in}{1.443721in}}%
\pgfpathlineto{\pgfqpoint{1.006338in}{1.443721in}}%
\pgfpathlineto{\pgfqpoint{1.006338in}{1.482814in}}%
\pgfpathclose%
\pgfusepath{fill}%
\end{pgfscope}%
\begin{pgfscope}%
\pgfpathrectangle{\pgfqpoint{0.697024in}{0.857143in}}{\pgfqpoint{2.627103in}{1.813434in}}%
\pgfusepath{clip}%
\pgfsetbuttcap%
\pgfsetmiterjoin%
\definecolor{currentfill}{rgb}{0.950697,0.616649,0.428624}%
\pgfsetfillcolor{currentfill}%
\pgfsetlinewidth{0.000000pt}%
\definecolor{currentstroke}{rgb}{0.000000,0.000000,0.000000}%
\pgfsetstrokecolor{currentstroke}%
\pgfsetstrokeopacity{0.000000}%
\pgfsetdash{}{0pt}%
\pgfpathmoveto{\pgfqpoint{1.017509in}{1.382679in}}%
\pgfpathlineto{\pgfqpoint{1.026446in}{1.382679in}}%
\pgfpathlineto{\pgfqpoint{1.026446in}{1.344273in}}%
\pgfpathlineto{\pgfqpoint{1.017509in}{1.344273in}}%
\pgfpathlineto{\pgfqpoint{1.017509in}{1.382679in}}%
\pgfpathclose%
\pgfusepath{fill}%
\end{pgfscope}%
\begin{pgfscope}%
\pgfpathrectangle{\pgfqpoint{0.697024in}{0.857143in}}{\pgfqpoint{2.627103in}{1.813434in}}%
\pgfusepath{clip}%
\pgfsetbuttcap%
\pgfsetmiterjoin%
\definecolor{currentfill}{rgb}{0.950697,0.616649,0.428624}%
\pgfsetfillcolor{currentfill}%
\pgfsetlinewidth{0.000000pt}%
\definecolor{currentstroke}{rgb}{0.000000,0.000000,0.000000}%
\pgfsetstrokecolor{currentstroke}%
\pgfsetstrokeopacity{0.000000}%
\pgfsetdash{}{0pt}%
\pgfpathmoveto{\pgfqpoint{1.028680in}{1.405412in}}%
\pgfpathlineto{\pgfqpoint{1.037616in}{1.405412in}}%
\pgfpathlineto{\pgfqpoint{1.037616in}{1.382050in}}%
\pgfpathlineto{\pgfqpoint{1.028680in}{1.382050in}}%
\pgfpathlineto{\pgfqpoint{1.028680in}{1.405412in}}%
\pgfpathclose%
\pgfusepath{fill}%
\end{pgfscope}%
\begin{pgfscope}%
\pgfpathrectangle{\pgfqpoint{0.697024in}{0.857143in}}{\pgfqpoint{2.627103in}{1.813434in}}%
\pgfusepath{clip}%
\pgfsetbuttcap%
\pgfsetmiterjoin%
\definecolor{currentfill}{rgb}{0.950697,0.616649,0.428624}%
\pgfsetfillcolor{currentfill}%
\pgfsetlinewidth{0.000000pt}%
\definecolor{currentstroke}{rgb}{0.000000,0.000000,0.000000}%
\pgfsetstrokecolor{currentstroke}%
\pgfsetstrokeopacity{0.000000}%
\pgfsetdash{}{0pt}%
\pgfpathmoveto{\pgfqpoint{1.039850in}{1.354230in}}%
\pgfpathlineto{\pgfqpoint{1.048787in}{1.354230in}}%
\pgfpathlineto{\pgfqpoint{1.048787in}{1.343354in}}%
\pgfpathlineto{\pgfqpoint{1.039850in}{1.343354in}}%
\pgfpathlineto{\pgfqpoint{1.039850in}{1.354230in}}%
\pgfpathclose%
\pgfusepath{fill}%
\end{pgfscope}%
\begin{pgfscope}%
\pgfpathrectangle{\pgfqpoint{0.697024in}{0.857143in}}{\pgfqpoint{2.627103in}{1.813434in}}%
\pgfusepath{clip}%
\pgfsetbuttcap%
\pgfsetmiterjoin%
\definecolor{currentfill}{rgb}{0.950697,0.616649,0.428624}%
\pgfsetfillcolor{currentfill}%
\pgfsetlinewidth{0.000000pt}%
\definecolor{currentstroke}{rgb}{0.000000,0.000000,0.000000}%
\pgfsetstrokecolor{currentstroke}%
\pgfsetstrokeopacity{0.000000}%
\pgfsetdash{}{0pt}%
\pgfpathmoveto{\pgfqpoint{1.051021in}{1.342546in}}%
\pgfpathlineto{\pgfqpoint{1.059957in}{1.342546in}}%
\pgfpathlineto{\pgfqpoint{1.059957in}{1.322649in}}%
\pgfpathlineto{\pgfqpoint{1.051021in}{1.322649in}}%
\pgfpathlineto{\pgfqpoint{1.051021in}{1.342546in}}%
\pgfpathclose%
\pgfusepath{fill}%
\end{pgfscope}%
\begin{pgfscope}%
\pgfpathrectangle{\pgfqpoint{0.697024in}{0.857143in}}{\pgfqpoint{2.627103in}{1.813434in}}%
\pgfusepath{clip}%
\pgfsetbuttcap%
\pgfsetmiterjoin%
\definecolor{currentfill}{rgb}{0.950697,0.616649,0.428624}%
\pgfsetfillcolor{currentfill}%
\pgfsetlinewidth{0.000000pt}%
\definecolor{currentstroke}{rgb}{0.000000,0.000000,0.000000}%
\pgfsetstrokecolor{currentstroke}%
\pgfsetstrokeopacity{0.000000}%
\pgfsetdash{}{0pt}%
\pgfpathmoveto{\pgfqpoint{1.062191in}{1.194426in}}%
\pgfpathlineto{\pgfqpoint{1.071128in}{1.194426in}}%
\pgfpathlineto{\pgfqpoint{1.071128in}{1.146164in}}%
\pgfpathlineto{\pgfqpoint{1.062191in}{1.146164in}}%
\pgfpathlineto{\pgfqpoint{1.062191in}{1.194426in}}%
\pgfpathclose%
\pgfusepath{fill}%
\end{pgfscope}%
\begin{pgfscope}%
\pgfpathrectangle{\pgfqpoint{0.697024in}{0.857143in}}{\pgfqpoint{2.627103in}{1.813434in}}%
\pgfusepath{clip}%
\pgfsetbuttcap%
\pgfsetmiterjoin%
\definecolor{currentfill}{rgb}{0.950697,0.616649,0.428624}%
\pgfsetfillcolor{currentfill}%
\pgfsetlinewidth{0.000000pt}%
\definecolor{currentstroke}{rgb}{0.000000,0.000000,0.000000}%
\pgfsetstrokecolor{currentstroke}%
\pgfsetstrokeopacity{0.000000}%
\pgfsetdash{}{0pt}%
\pgfpathmoveto{\pgfqpoint{1.073362in}{1.235633in}}%
\pgfpathlineto{\pgfqpoint{1.082299in}{1.235633in}}%
\pgfpathlineto{\pgfqpoint{1.082299in}{1.168832in}}%
\pgfpathlineto{\pgfqpoint{1.073362in}{1.168832in}}%
\pgfpathlineto{\pgfqpoint{1.073362in}{1.235633in}}%
\pgfpathclose%
\pgfusepath{fill}%
\end{pgfscope}%
\begin{pgfscope}%
\pgfpathrectangle{\pgfqpoint{0.697024in}{0.857143in}}{\pgfqpoint{2.627103in}{1.813434in}}%
\pgfusepath{clip}%
\pgfsetbuttcap%
\pgfsetmiterjoin%
\definecolor{currentfill}{rgb}{0.950697,0.616649,0.428624}%
\pgfsetfillcolor{currentfill}%
\pgfsetlinewidth{0.000000pt}%
\definecolor{currentstroke}{rgb}{0.000000,0.000000,0.000000}%
\pgfsetstrokecolor{currentstroke}%
\pgfsetstrokeopacity{0.000000}%
\pgfsetdash{}{0pt}%
\pgfpathmoveto{\pgfqpoint{1.084533in}{1.350686in}}%
\pgfpathlineto{\pgfqpoint{1.093469in}{1.350686in}}%
\pgfpathlineto{\pgfqpoint{1.093469in}{1.290080in}}%
\pgfpathlineto{\pgfqpoint{1.084533in}{1.290080in}}%
\pgfpathlineto{\pgfqpoint{1.084533in}{1.350686in}}%
\pgfpathclose%
\pgfusepath{fill}%
\end{pgfscope}%
\begin{pgfscope}%
\pgfpathrectangle{\pgfqpoint{0.697024in}{0.857143in}}{\pgfqpoint{2.627103in}{1.813434in}}%
\pgfusepath{clip}%
\pgfsetbuttcap%
\pgfsetmiterjoin%
\definecolor{currentfill}{rgb}{0.950697,0.616649,0.428624}%
\pgfsetfillcolor{currentfill}%
\pgfsetlinewidth{0.000000pt}%
\definecolor{currentstroke}{rgb}{0.000000,0.000000,0.000000}%
\pgfsetstrokecolor{currentstroke}%
\pgfsetstrokeopacity{0.000000}%
\pgfsetdash{}{0pt}%
\pgfpathmoveto{\pgfqpoint{1.095703in}{1.355907in}}%
\pgfpathlineto{\pgfqpoint{1.104640in}{1.355907in}}%
\pgfpathlineto{\pgfqpoint{1.104640in}{1.300766in}}%
\pgfpathlineto{\pgfqpoint{1.095703in}{1.300766in}}%
\pgfpathlineto{\pgfqpoint{1.095703in}{1.355907in}}%
\pgfpathclose%
\pgfusepath{fill}%
\end{pgfscope}%
\begin{pgfscope}%
\pgfpathrectangle{\pgfqpoint{0.697024in}{0.857143in}}{\pgfqpoint{2.627103in}{1.813434in}}%
\pgfusepath{clip}%
\pgfsetbuttcap%
\pgfsetmiterjoin%
\definecolor{currentfill}{rgb}{0.950697,0.616649,0.428624}%
\pgfsetfillcolor{currentfill}%
\pgfsetlinewidth{0.000000pt}%
\definecolor{currentstroke}{rgb}{0.000000,0.000000,0.000000}%
\pgfsetstrokecolor{currentstroke}%
\pgfsetstrokeopacity{0.000000}%
\pgfsetdash{}{0pt}%
\pgfpathmoveto{\pgfqpoint{1.106874in}{1.416930in}}%
\pgfpathlineto{\pgfqpoint{1.115810in}{1.416930in}}%
\pgfpathlineto{\pgfqpoint{1.115810in}{1.354081in}}%
\pgfpathlineto{\pgfqpoint{1.106874in}{1.354081in}}%
\pgfpathlineto{\pgfqpoint{1.106874in}{1.416930in}}%
\pgfpathclose%
\pgfusepath{fill}%
\end{pgfscope}%
\begin{pgfscope}%
\pgfpathrectangle{\pgfqpoint{0.697024in}{0.857143in}}{\pgfqpoint{2.627103in}{1.813434in}}%
\pgfusepath{clip}%
\pgfsetbuttcap%
\pgfsetmiterjoin%
\definecolor{currentfill}{rgb}{0.950697,0.616649,0.428624}%
\pgfsetfillcolor{currentfill}%
\pgfsetlinewidth{0.000000pt}%
\definecolor{currentstroke}{rgb}{0.000000,0.000000,0.000000}%
\pgfsetstrokecolor{currentstroke}%
\pgfsetstrokeopacity{0.000000}%
\pgfsetdash{}{0pt}%
\pgfpathmoveto{\pgfqpoint{1.118045in}{1.494598in}}%
\pgfpathlineto{\pgfqpoint{1.126981in}{1.494598in}}%
\pgfpathlineto{\pgfqpoint{1.126981in}{1.415714in}}%
\pgfpathlineto{\pgfqpoint{1.118045in}{1.415714in}}%
\pgfpathlineto{\pgfqpoint{1.118045in}{1.494598in}}%
\pgfpathclose%
\pgfusepath{fill}%
\end{pgfscope}%
\begin{pgfscope}%
\pgfpathrectangle{\pgfqpoint{0.697024in}{0.857143in}}{\pgfqpoint{2.627103in}{1.813434in}}%
\pgfusepath{clip}%
\pgfsetbuttcap%
\pgfsetmiterjoin%
\definecolor{currentfill}{rgb}{0.950697,0.616649,0.428624}%
\pgfsetfillcolor{currentfill}%
\pgfsetlinewidth{0.000000pt}%
\definecolor{currentstroke}{rgb}{0.000000,0.000000,0.000000}%
\pgfsetstrokecolor{currentstroke}%
\pgfsetstrokeopacity{0.000000}%
\pgfsetdash{}{0pt}%
\pgfpathmoveto{\pgfqpoint{1.129215in}{1.578386in}}%
\pgfpathlineto{\pgfqpoint{1.138152in}{1.578386in}}%
\pgfpathlineto{\pgfqpoint{1.138152in}{1.495317in}}%
\pgfpathlineto{\pgfqpoint{1.129215in}{1.495317in}}%
\pgfpathlineto{\pgfqpoint{1.129215in}{1.578386in}}%
\pgfpathclose%
\pgfusepath{fill}%
\end{pgfscope}%
\begin{pgfscope}%
\pgfpathrectangle{\pgfqpoint{0.697024in}{0.857143in}}{\pgfqpoint{2.627103in}{1.813434in}}%
\pgfusepath{clip}%
\pgfsetbuttcap%
\pgfsetmiterjoin%
\definecolor{currentfill}{rgb}{0.950697,0.616649,0.428624}%
\pgfsetfillcolor{currentfill}%
\pgfsetlinewidth{0.000000pt}%
\definecolor{currentstroke}{rgb}{0.000000,0.000000,0.000000}%
\pgfsetstrokecolor{currentstroke}%
\pgfsetstrokeopacity{0.000000}%
\pgfsetdash{}{0pt}%
\pgfpathmoveto{\pgfqpoint{1.140386in}{1.585019in}}%
\pgfpathlineto{\pgfqpoint{1.149322in}{1.585019in}}%
\pgfpathlineto{\pgfqpoint{1.149322in}{1.498388in}}%
\pgfpathlineto{\pgfqpoint{1.140386in}{1.498388in}}%
\pgfpathlineto{\pgfqpoint{1.140386in}{1.585019in}}%
\pgfpathclose%
\pgfusepath{fill}%
\end{pgfscope}%
\begin{pgfscope}%
\pgfpathrectangle{\pgfqpoint{0.697024in}{0.857143in}}{\pgfqpoint{2.627103in}{1.813434in}}%
\pgfusepath{clip}%
\pgfsetbuttcap%
\pgfsetmiterjoin%
\definecolor{currentfill}{rgb}{0.950697,0.616649,0.428624}%
\pgfsetfillcolor{currentfill}%
\pgfsetlinewidth{0.000000pt}%
\definecolor{currentstroke}{rgb}{0.000000,0.000000,0.000000}%
\pgfsetstrokecolor{currentstroke}%
\pgfsetstrokeopacity{0.000000}%
\pgfsetdash{}{0pt}%
\pgfpathmoveto{\pgfqpoint{1.151556in}{1.569715in}}%
\pgfpathlineto{\pgfqpoint{1.160493in}{1.569715in}}%
\pgfpathlineto{\pgfqpoint{1.160493in}{1.485012in}}%
\pgfpathlineto{\pgfqpoint{1.151556in}{1.485012in}}%
\pgfpathlineto{\pgfqpoint{1.151556in}{1.569715in}}%
\pgfpathclose%
\pgfusepath{fill}%
\end{pgfscope}%
\begin{pgfscope}%
\pgfpathrectangle{\pgfqpoint{0.697024in}{0.857143in}}{\pgfqpoint{2.627103in}{1.813434in}}%
\pgfusepath{clip}%
\pgfsetbuttcap%
\pgfsetmiterjoin%
\definecolor{currentfill}{rgb}{0.950697,0.616649,0.428624}%
\pgfsetfillcolor{currentfill}%
\pgfsetlinewidth{0.000000pt}%
\definecolor{currentstroke}{rgb}{0.000000,0.000000,0.000000}%
\pgfsetstrokecolor{currentstroke}%
\pgfsetstrokeopacity{0.000000}%
\pgfsetdash{}{0pt}%
\pgfpathmoveto{\pgfqpoint{1.162727in}{1.674676in}}%
\pgfpathlineto{\pgfqpoint{1.171663in}{1.674676in}}%
\pgfpathlineto{\pgfqpoint{1.171663in}{1.601889in}}%
\pgfpathlineto{\pgfqpoint{1.162727in}{1.601889in}}%
\pgfpathlineto{\pgfqpoint{1.162727in}{1.674676in}}%
\pgfpathclose%
\pgfusepath{fill}%
\end{pgfscope}%
\begin{pgfscope}%
\pgfpathrectangle{\pgfqpoint{0.697024in}{0.857143in}}{\pgfqpoint{2.627103in}{1.813434in}}%
\pgfusepath{clip}%
\pgfsetbuttcap%
\pgfsetmiterjoin%
\definecolor{currentfill}{rgb}{0.950697,0.616649,0.428624}%
\pgfsetfillcolor{currentfill}%
\pgfsetlinewidth{0.000000pt}%
\definecolor{currentstroke}{rgb}{0.000000,0.000000,0.000000}%
\pgfsetstrokecolor{currentstroke}%
\pgfsetstrokeopacity{0.000000}%
\pgfsetdash{}{0pt}%
\pgfpathmoveto{\pgfqpoint{1.173898in}{1.633167in}}%
\pgfpathlineto{\pgfqpoint{1.182834in}{1.633167in}}%
\pgfpathlineto{\pgfqpoint{1.182834in}{1.566227in}}%
\pgfpathlineto{\pgfqpoint{1.173898in}{1.566227in}}%
\pgfpathlineto{\pgfqpoint{1.173898in}{1.633167in}}%
\pgfpathclose%
\pgfusepath{fill}%
\end{pgfscope}%
\begin{pgfscope}%
\pgfpathrectangle{\pgfqpoint{0.697024in}{0.857143in}}{\pgfqpoint{2.627103in}{1.813434in}}%
\pgfusepath{clip}%
\pgfsetbuttcap%
\pgfsetmiterjoin%
\definecolor{currentfill}{rgb}{0.950697,0.616649,0.428624}%
\pgfsetfillcolor{currentfill}%
\pgfsetlinewidth{0.000000pt}%
\definecolor{currentstroke}{rgb}{0.000000,0.000000,0.000000}%
\pgfsetstrokecolor{currentstroke}%
\pgfsetstrokeopacity{0.000000}%
\pgfsetdash{}{0pt}%
\pgfpathmoveto{\pgfqpoint{1.185068in}{1.540863in}}%
\pgfpathlineto{\pgfqpoint{1.194005in}{1.540863in}}%
\pgfpathlineto{\pgfqpoint{1.194005in}{1.482547in}}%
\pgfpathlineto{\pgfqpoint{1.185068in}{1.482547in}}%
\pgfpathlineto{\pgfqpoint{1.185068in}{1.540863in}}%
\pgfpathclose%
\pgfusepath{fill}%
\end{pgfscope}%
\begin{pgfscope}%
\pgfpathrectangle{\pgfqpoint{0.697024in}{0.857143in}}{\pgfqpoint{2.627103in}{1.813434in}}%
\pgfusepath{clip}%
\pgfsetbuttcap%
\pgfsetmiterjoin%
\definecolor{currentfill}{rgb}{0.950697,0.616649,0.428624}%
\pgfsetfillcolor{currentfill}%
\pgfsetlinewidth{0.000000pt}%
\definecolor{currentstroke}{rgb}{0.000000,0.000000,0.000000}%
\pgfsetstrokecolor{currentstroke}%
\pgfsetstrokeopacity{0.000000}%
\pgfsetdash{}{0pt}%
\pgfpathmoveto{\pgfqpoint{1.196239in}{1.394849in}}%
\pgfpathlineto{\pgfqpoint{1.205175in}{1.394849in}}%
\pgfpathlineto{\pgfqpoint{1.205175in}{1.378783in}}%
\pgfpathlineto{\pgfqpoint{1.196239in}{1.378783in}}%
\pgfpathlineto{\pgfqpoint{1.196239in}{1.394849in}}%
\pgfpathclose%
\pgfusepath{fill}%
\end{pgfscope}%
\begin{pgfscope}%
\pgfpathrectangle{\pgfqpoint{0.697024in}{0.857143in}}{\pgfqpoint{2.627103in}{1.813434in}}%
\pgfusepath{clip}%
\pgfsetbuttcap%
\pgfsetmiterjoin%
\definecolor{currentfill}{rgb}{0.950697,0.616649,0.428624}%
\pgfsetfillcolor{currentfill}%
\pgfsetlinewidth{0.000000pt}%
\definecolor{currentstroke}{rgb}{0.000000,0.000000,0.000000}%
\pgfsetstrokecolor{currentstroke}%
\pgfsetstrokeopacity{0.000000}%
\pgfsetdash{}{0pt}%
\pgfpathmoveto{\pgfqpoint{1.207409in}{1.990461in}}%
\pgfpathlineto{\pgfqpoint{1.216346in}{1.990461in}}%
\pgfpathlineto{\pgfqpoint{1.216346in}{2.010415in}}%
\pgfpathlineto{\pgfqpoint{1.207409in}{2.010415in}}%
\pgfpathlineto{\pgfqpoint{1.207409in}{1.990461in}}%
\pgfpathclose%
\pgfusepath{fill}%
\end{pgfscope}%
\begin{pgfscope}%
\pgfpathrectangle{\pgfqpoint{0.697024in}{0.857143in}}{\pgfqpoint{2.627103in}{1.813434in}}%
\pgfusepath{clip}%
\pgfsetbuttcap%
\pgfsetmiterjoin%
\definecolor{currentfill}{rgb}{0.950697,0.616649,0.428624}%
\pgfsetfillcolor{currentfill}%
\pgfsetlinewidth{0.000000pt}%
\definecolor{currentstroke}{rgb}{0.000000,0.000000,0.000000}%
\pgfsetstrokecolor{currentstroke}%
\pgfsetstrokeopacity{0.000000}%
\pgfsetdash{}{0pt}%
\pgfpathmoveto{\pgfqpoint{1.218580in}{2.022538in}}%
\pgfpathlineto{\pgfqpoint{1.227516in}{2.022538in}}%
\pgfpathlineto{\pgfqpoint{1.227516in}{2.064688in}}%
\pgfpathlineto{\pgfqpoint{1.218580in}{2.064688in}}%
\pgfpathlineto{\pgfqpoint{1.218580in}{2.022538in}}%
\pgfpathclose%
\pgfusepath{fill}%
\end{pgfscope}%
\begin{pgfscope}%
\pgfpathrectangle{\pgfqpoint{0.697024in}{0.857143in}}{\pgfqpoint{2.627103in}{1.813434in}}%
\pgfusepath{clip}%
\pgfsetbuttcap%
\pgfsetmiterjoin%
\definecolor{currentfill}{rgb}{0.950697,0.616649,0.428624}%
\pgfsetfillcolor{currentfill}%
\pgfsetlinewidth{0.000000pt}%
\definecolor{currentstroke}{rgb}{0.000000,0.000000,0.000000}%
\pgfsetstrokecolor{currentstroke}%
\pgfsetstrokeopacity{0.000000}%
\pgfsetdash{}{0pt}%
\pgfpathmoveto{\pgfqpoint{1.229751in}{1.992655in}}%
\pgfpathlineto{\pgfqpoint{1.238687in}{1.992655in}}%
\pgfpathlineto{\pgfqpoint{1.238687in}{2.039537in}}%
\pgfpathlineto{\pgfqpoint{1.229751in}{2.039537in}}%
\pgfpathlineto{\pgfqpoint{1.229751in}{1.992655in}}%
\pgfpathclose%
\pgfusepath{fill}%
\end{pgfscope}%
\begin{pgfscope}%
\pgfpathrectangle{\pgfqpoint{0.697024in}{0.857143in}}{\pgfqpoint{2.627103in}{1.813434in}}%
\pgfusepath{clip}%
\pgfsetbuttcap%
\pgfsetmiterjoin%
\definecolor{currentfill}{rgb}{0.950697,0.616649,0.428624}%
\pgfsetfillcolor{currentfill}%
\pgfsetlinewidth{0.000000pt}%
\definecolor{currentstroke}{rgb}{0.000000,0.000000,0.000000}%
\pgfsetstrokecolor{currentstroke}%
\pgfsetstrokeopacity{0.000000}%
\pgfsetdash{}{0pt}%
\pgfpathmoveto{\pgfqpoint{1.240921in}{1.993269in}}%
\pgfpathlineto{\pgfqpoint{1.249858in}{1.993269in}}%
\pgfpathlineto{\pgfqpoint{1.249858in}{2.031101in}}%
\pgfpathlineto{\pgfqpoint{1.240921in}{2.031101in}}%
\pgfpathlineto{\pgfqpoint{1.240921in}{1.993269in}}%
\pgfpathclose%
\pgfusepath{fill}%
\end{pgfscope}%
\begin{pgfscope}%
\pgfpathrectangle{\pgfqpoint{0.697024in}{0.857143in}}{\pgfqpoint{2.627103in}{1.813434in}}%
\pgfusepath{clip}%
\pgfsetbuttcap%
\pgfsetmiterjoin%
\definecolor{currentfill}{rgb}{0.950697,0.616649,0.428624}%
\pgfsetfillcolor{currentfill}%
\pgfsetlinewidth{0.000000pt}%
\definecolor{currentstroke}{rgb}{0.000000,0.000000,0.000000}%
\pgfsetstrokecolor{currentstroke}%
\pgfsetstrokeopacity{0.000000}%
\pgfsetdash{}{0pt}%
\pgfpathmoveto{\pgfqpoint{1.252092in}{2.023881in}}%
\pgfpathlineto{\pgfqpoint{1.261028in}{2.023881in}}%
\pgfpathlineto{\pgfqpoint{1.261028in}{2.053093in}}%
\pgfpathlineto{\pgfqpoint{1.252092in}{2.053093in}}%
\pgfpathlineto{\pgfqpoint{1.252092in}{2.023881in}}%
\pgfpathclose%
\pgfusepath{fill}%
\end{pgfscope}%
\begin{pgfscope}%
\pgfpathrectangle{\pgfqpoint{0.697024in}{0.857143in}}{\pgfqpoint{2.627103in}{1.813434in}}%
\pgfusepath{clip}%
\pgfsetbuttcap%
\pgfsetmiterjoin%
\definecolor{currentfill}{rgb}{0.950697,0.616649,0.428624}%
\pgfsetfillcolor{currentfill}%
\pgfsetlinewidth{0.000000pt}%
\definecolor{currentstroke}{rgb}{0.000000,0.000000,0.000000}%
\pgfsetstrokecolor{currentstroke}%
\pgfsetstrokeopacity{0.000000}%
\pgfsetdash{}{0pt}%
\pgfpathmoveto{\pgfqpoint{1.263262in}{1.998006in}}%
\pgfpathlineto{\pgfqpoint{1.272199in}{1.998006in}}%
\pgfpathlineto{\pgfqpoint{1.272199in}{2.019505in}}%
\pgfpathlineto{\pgfqpoint{1.263262in}{2.019505in}}%
\pgfpathlineto{\pgfqpoint{1.263262in}{1.998006in}}%
\pgfpathclose%
\pgfusepath{fill}%
\end{pgfscope}%
\begin{pgfscope}%
\pgfpathrectangle{\pgfqpoint{0.697024in}{0.857143in}}{\pgfqpoint{2.627103in}{1.813434in}}%
\pgfusepath{clip}%
\pgfsetbuttcap%
\pgfsetmiterjoin%
\definecolor{currentfill}{rgb}{0.950697,0.616649,0.428624}%
\pgfsetfillcolor{currentfill}%
\pgfsetlinewidth{0.000000pt}%
\definecolor{currentstroke}{rgb}{0.000000,0.000000,0.000000}%
\pgfsetstrokecolor{currentstroke}%
\pgfsetstrokeopacity{0.000000}%
\pgfsetdash{}{0pt}%
\pgfpathmoveto{\pgfqpoint{1.274433in}{2.031703in}}%
\pgfpathlineto{\pgfqpoint{1.283369in}{2.031703in}}%
\pgfpathlineto{\pgfqpoint{1.283369in}{2.053058in}}%
\pgfpathlineto{\pgfqpoint{1.274433in}{2.053058in}}%
\pgfpathlineto{\pgfqpoint{1.274433in}{2.031703in}}%
\pgfpathclose%
\pgfusepath{fill}%
\end{pgfscope}%
\begin{pgfscope}%
\pgfpathrectangle{\pgfqpoint{0.697024in}{0.857143in}}{\pgfqpoint{2.627103in}{1.813434in}}%
\pgfusepath{clip}%
\pgfsetbuttcap%
\pgfsetmiterjoin%
\definecolor{currentfill}{rgb}{0.950697,0.616649,0.428624}%
\pgfsetfillcolor{currentfill}%
\pgfsetlinewidth{0.000000pt}%
\definecolor{currentstroke}{rgb}{0.000000,0.000000,0.000000}%
\pgfsetstrokecolor{currentstroke}%
\pgfsetstrokeopacity{0.000000}%
\pgfsetdash{}{0pt}%
\pgfpathmoveto{\pgfqpoint{1.285604in}{2.012008in}}%
\pgfpathlineto{\pgfqpoint{1.294540in}{2.012008in}}%
\pgfpathlineto{\pgfqpoint{1.294540in}{2.028414in}}%
\pgfpathlineto{\pgfqpoint{1.285604in}{2.028414in}}%
\pgfpathlineto{\pgfqpoint{1.285604in}{2.012008in}}%
\pgfpathclose%
\pgfusepath{fill}%
\end{pgfscope}%
\begin{pgfscope}%
\pgfpathrectangle{\pgfqpoint{0.697024in}{0.857143in}}{\pgfqpoint{2.627103in}{1.813434in}}%
\pgfusepath{clip}%
\pgfsetbuttcap%
\pgfsetmiterjoin%
\definecolor{currentfill}{rgb}{0.950697,0.616649,0.428624}%
\pgfsetfillcolor{currentfill}%
\pgfsetlinewidth{0.000000pt}%
\definecolor{currentstroke}{rgb}{0.000000,0.000000,0.000000}%
\pgfsetstrokecolor{currentstroke}%
\pgfsetstrokeopacity{0.000000}%
\pgfsetdash{}{0pt}%
\pgfpathmoveto{\pgfqpoint{1.296774in}{2.060957in}}%
\pgfpathlineto{\pgfqpoint{1.305711in}{2.060957in}}%
\pgfpathlineto{\pgfqpoint{1.305711in}{2.061363in}}%
\pgfpathlineto{\pgfqpoint{1.296774in}{2.061363in}}%
\pgfpathlineto{\pgfqpoint{1.296774in}{2.060957in}}%
\pgfpathclose%
\pgfusepath{fill}%
\end{pgfscope}%
\begin{pgfscope}%
\pgfpathrectangle{\pgfqpoint{0.697024in}{0.857143in}}{\pgfqpoint{2.627103in}{1.813434in}}%
\pgfusepath{clip}%
\pgfsetbuttcap%
\pgfsetmiterjoin%
\definecolor{currentfill}{rgb}{0.950697,0.616649,0.428624}%
\pgfsetfillcolor{currentfill}%
\pgfsetlinewidth{0.000000pt}%
\definecolor{currentstroke}{rgb}{0.000000,0.000000,0.000000}%
\pgfsetstrokecolor{currentstroke}%
\pgfsetstrokeopacity{0.000000}%
\pgfsetdash{}{0pt}%
\pgfpathmoveto{\pgfqpoint{1.307945in}{1.586350in}}%
\pgfpathlineto{\pgfqpoint{1.316881in}{1.586350in}}%
\pgfpathlineto{\pgfqpoint{1.316881in}{1.582552in}}%
\pgfpathlineto{\pgfqpoint{1.307945in}{1.582552in}}%
\pgfpathlineto{\pgfqpoint{1.307945in}{1.586350in}}%
\pgfpathclose%
\pgfusepath{fill}%
\end{pgfscope}%
\begin{pgfscope}%
\pgfpathrectangle{\pgfqpoint{0.697024in}{0.857143in}}{\pgfqpoint{2.627103in}{1.813434in}}%
\pgfusepath{clip}%
\pgfsetbuttcap%
\pgfsetmiterjoin%
\definecolor{currentfill}{rgb}{0.950697,0.616649,0.428624}%
\pgfsetfillcolor{currentfill}%
\pgfsetlinewidth{0.000000pt}%
\definecolor{currentstroke}{rgb}{0.000000,0.000000,0.000000}%
\pgfsetstrokecolor{currentstroke}%
\pgfsetstrokeopacity{0.000000}%
\pgfsetdash{}{0pt}%
\pgfpathmoveto{\pgfqpoint{1.319115in}{1.989111in}}%
\pgfpathlineto{\pgfqpoint{1.328052in}{1.989111in}}%
\pgfpathlineto{\pgfqpoint{1.328052in}{1.992896in}}%
\pgfpathlineto{\pgfqpoint{1.319115in}{1.992896in}}%
\pgfpathlineto{\pgfqpoint{1.319115in}{1.989111in}}%
\pgfpathclose%
\pgfusepath{fill}%
\end{pgfscope}%
\begin{pgfscope}%
\pgfpathrectangle{\pgfqpoint{0.697024in}{0.857143in}}{\pgfqpoint{2.627103in}{1.813434in}}%
\pgfusepath{clip}%
\pgfsetbuttcap%
\pgfsetmiterjoin%
\definecolor{currentfill}{rgb}{0.950697,0.616649,0.428624}%
\pgfsetfillcolor{currentfill}%
\pgfsetlinewidth{0.000000pt}%
\definecolor{currentstroke}{rgb}{0.000000,0.000000,0.000000}%
\pgfsetstrokecolor{currentstroke}%
\pgfsetstrokeopacity{0.000000}%
\pgfsetdash{}{0pt}%
\pgfpathmoveto{\pgfqpoint{1.330286in}{1.581666in}}%
\pgfpathlineto{\pgfqpoint{1.339222in}{1.581666in}}%
\pgfpathlineto{\pgfqpoint{1.339222in}{1.569815in}}%
\pgfpathlineto{\pgfqpoint{1.330286in}{1.569815in}}%
\pgfpathlineto{\pgfqpoint{1.330286in}{1.581666in}}%
\pgfpathclose%
\pgfusepath{fill}%
\end{pgfscope}%
\begin{pgfscope}%
\pgfpathrectangle{\pgfqpoint{0.697024in}{0.857143in}}{\pgfqpoint{2.627103in}{1.813434in}}%
\pgfusepath{clip}%
\pgfsetbuttcap%
\pgfsetmiterjoin%
\definecolor{currentfill}{rgb}{0.950697,0.616649,0.428624}%
\pgfsetfillcolor{currentfill}%
\pgfsetlinewidth{0.000000pt}%
\definecolor{currentstroke}{rgb}{0.000000,0.000000,0.000000}%
\pgfsetstrokecolor{currentstroke}%
\pgfsetstrokeopacity{0.000000}%
\pgfsetdash{}{0pt}%
\pgfpathmoveto{\pgfqpoint{1.341457in}{1.710410in}}%
\pgfpathlineto{\pgfqpoint{1.350393in}{1.710410in}}%
\pgfpathlineto{\pgfqpoint{1.350393in}{1.681325in}}%
\pgfpathlineto{\pgfqpoint{1.341457in}{1.681325in}}%
\pgfpathlineto{\pgfqpoint{1.341457in}{1.710410in}}%
\pgfpathclose%
\pgfusepath{fill}%
\end{pgfscope}%
\begin{pgfscope}%
\pgfpathrectangle{\pgfqpoint{0.697024in}{0.857143in}}{\pgfqpoint{2.627103in}{1.813434in}}%
\pgfusepath{clip}%
\pgfsetbuttcap%
\pgfsetmiterjoin%
\definecolor{currentfill}{rgb}{0.950697,0.616649,0.428624}%
\pgfsetfillcolor{currentfill}%
\pgfsetlinewidth{0.000000pt}%
\definecolor{currentstroke}{rgb}{0.000000,0.000000,0.000000}%
\pgfsetstrokecolor{currentstroke}%
\pgfsetstrokeopacity{0.000000}%
\pgfsetdash{}{0pt}%
\pgfpathmoveto{\pgfqpoint{1.352627in}{1.714419in}}%
\pgfpathlineto{\pgfqpoint{1.361564in}{1.714419in}}%
\pgfpathlineto{\pgfqpoint{1.361564in}{1.680935in}}%
\pgfpathlineto{\pgfqpoint{1.352627in}{1.680935in}}%
\pgfpathlineto{\pgfqpoint{1.352627in}{1.714419in}}%
\pgfpathclose%
\pgfusepath{fill}%
\end{pgfscope}%
\begin{pgfscope}%
\pgfpathrectangle{\pgfqpoint{0.697024in}{0.857143in}}{\pgfqpoint{2.627103in}{1.813434in}}%
\pgfusepath{clip}%
\pgfsetbuttcap%
\pgfsetmiterjoin%
\definecolor{currentfill}{rgb}{0.950697,0.616649,0.428624}%
\pgfsetfillcolor{currentfill}%
\pgfsetlinewidth{0.000000pt}%
\definecolor{currentstroke}{rgb}{0.000000,0.000000,0.000000}%
\pgfsetstrokecolor{currentstroke}%
\pgfsetstrokeopacity{0.000000}%
\pgfsetdash{}{0pt}%
\pgfpathmoveto{\pgfqpoint{1.363798in}{1.764183in}}%
\pgfpathlineto{\pgfqpoint{1.372734in}{1.764183in}}%
\pgfpathlineto{\pgfqpoint{1.372734in}{1.723929in}}%
\pgfpathlineto{\pgfqpoint{1.363798in}{1.723929in}}%
\pgfpathlineto{\pgfqpoint{1.363798in}{1.764183in}}%
\pgfpathclose%
\pgfusepath{fill}%
\end{pgfscope}%
\begin{pgfscope}%
\pgfpathrectangle{\pgfqpoint{0.697024in}{0.857143in}}{\pgfqpoint{2.627103in}{1.813434in}}%
\pgfusepath{clip}%
\pgfsetbuttcap%
\pgfsetmiterjoin%
\definecolor{currentfill}{rgb}{0.950697,0.616649,0.428624}%
\pgfsetfillcolor{currentfill}%
\pgfsetlinewidth{0.000000pt}%
\definecolor{currentstroke}{rgb}{0.000000,0.000000,0.000000}%
\pgfsetstrokecolor{currentstroke}%
\pgfsetstrokeopacity{0.000000}%
\pgfsetdash{}{0pt}%
\pgfpathmoveto{\pgfqpoint{1.374968in}{1.769604in}}%
\pgfpathlineto{\pgfqpoint{1.383905in}{1.769604in}}%
\pgfpathlineto{\pgfqpoint{1.383905in}{1.712452in}}%
\pgfpathlineto{\pgfqpoint{1.374968in}{1.712452in}}%
\pgfpathlineto{\pgfqpoint{1.374968in}{1.769604in}}%
\pgfpathclose%
\pgfusepath{fill}%
\end{pgfscope}%
\begin{pgfscope}%
\pgfpathrectangle{\pgfqpoint{0.697024in}{0.857143in}}{\pgfqpoint{2.627103in}{1.813434in}}%
\pgfusepath{clip}%
\pgfsetbuttcap%
\pgfsetmiterjoin%
\definecolor{currentfill}{rgb}{0.950697,0.616649,0.428624}%
\pgfsetfillcolor{currentfill}%
\pgfsetlinewidth{0.000000pt}%
\definecolor{currentstroke}{rgb}{0.000000,0.000000,0.000000}%
\pgfsetstrokecolor{currentstroke}%
\pgfsetstrokeopacity{0.000000}%
\pgfsetdash{}{0pt}%
\pgfpathmoveto{\pgfqpoint{1.386139in}{1.748359in}}%
\pgfpathlineto{\pgfqpoint{1.395076in}{1.748359in}}%
\pgfpathlineto{\pgfqpoint{1.395076in}{1.698499in}}%
\pgfpathlineto{\pgfqpoint{1.386139in}{1.698499in}}%
\pgfpathlineto{\pgfqpoint{1.386139in}{1.748359in}}%
\pgfpathclose%
\pgfusepath{fill}%
\end{pgfscope}%
\begin{pgfscope}%
\pgfpathrectangle{\pgfqpoint{0.697024in}{0.857143in}}{\pgfqpoint{2.627103in}{1.813434in}}%
\pgfusepath{clip}%
\pgfsetbuttcap%
\pgfsetmiterjoin%
\definecolor{currentfill}{rgb}{0.950697,0.616649,0.428624}%
\pgfsetfillcolor{currentfill}%
\pgfsetlinewidth{0.000000pt}%
\definecolor{currentstroke}{rgb}{0.000000,0.000000,0.000000}%
\pgfsetstrokecolor{currentstroke}%
\pgfsetstrokeopacity{0.000000}%
\pgfsetdash{}{0pt}%
\pgfpathmoveto{\pgfqpoint{1.397310in}{1.800678in}}%
\pgfpathlineto{\pgfqpoint{1.406246in}{1.800678in}}%
\pgfpathlineto{\pgfqpoint{1.406246in}{1.751475in}}%
\pgfpathlineto{\pgfqpoint{1.397310in}{1.751475in}}%
\pgfpathlineto{\pgfqpoint{1.397310in}{1.800678in}}%
\pgfpathclose%
\pgfusepath{fill}%
\end{pgfscope}%
\begin{pgfscope}%
\pgfpathrectangle{\pgfqpoint{0.697024in}{0.857143in}}{\pgfqpoint{2.627103in}{1.813434in}}%
\pgfusepath{clip}%
\pgfsetbuttcap%
\pgfsetmiterjoin%
\definecolor{currentfill}{rgb}{0.950697,0.616649,0.428624}%
\pgfsetfillcolor{currentfill}%
\pgfsetlinewidth{0.000000pt}%
\definecolor{currentstroke}{rgb}{0.000000,0.000000,0.000000}%
\pgfsetstrokecolor{currentstroke}%
\pgfsetstrokeopacity{0.000000}%
\pgfsetdash{}{0pt}%
\pgfpathmoveto{\pgfqpoint{1.408480in}{1.717782in}}%
\pgfpathlineto{\pgfqpoint{1.417417in}{1.717782in}}%
\pgfpathlineto{\pgfqpoint{1.417417in}{1.663964in}}%
\pgfpathlineto{\pgfqpoint{1.408480in}{1.663964in}}%
\pgfpathlineto{\pgfqpoint{1.408480in}{1.717782in}}%
\pgfpathclose%
\pgfusepath{fill}%
\end{pgfscope}%
\begin{pgfscope}%
\pgfpathrectangle{\pgfqpoint{0.697024in}{0.857143in}}{\pgfqpoint{2.627103in}{1.813434in}}%
\pgfusepath{clip}%
\pgfsetbuttcap%
\pgfsetmiterjoin%
\definecolor{currentfill}{rgb}{0.950697,0.616649,0.428624}%
\pgfsetfillcolor{currentfill}%
\pgfsetlinewidth{0.000000pt}%
\definecolor{currentstroke}{rgb}{0.000000,0.000000,0.000000}%
\pgfsetstrokecolor{currentstroke}%
\pgfsetstrokeopacity{0.000000}%
\pgfsetdash{}{0pt}%
\pgfpathmoveto{\pgfqpoint{1.419651in}{1.772776in}}%
\pgfpathlineto{\pgfqpoint{1.428587in}{1.772776in}}%
\pgfpathlineto{\pgfqpoint{1.428587in}{1.708717in}}%
\pgfpathlineto{\pgfqpoint{1.419651in}{1.708717in}}%
\pgfpathlineto{\pgfqpoint{1.419651in}{1.772776in}}%
\pgfpathclose%
\pgfusepath{fill}%
\end{pgfscope}%
\begin{pgfscope}%
\pgfpathrectangle{\pgfqpoint{0.697024in}{0.857143in}}{\pgfqpoint{2.627103in}{1.813434in}}%
\pgfusepath{clip}%
\pgfsetbuttcap%
\pgfsetmiterjoin%
\definecolor{currentfill}{rgb}{0.950697,0.616649,0.428624}%
\pgfsetfillcolor{currentfill}%
\pgfsetlinewidth{0.000000pt}%
\definecolor{currentstroke}{rgb}{0.000000,0.000000,0.000000}%
\pgfsetstrokecolor{currentstroke}%
\pgfsetstrokeopacity{0.000000}%
\pgfsetdash{}{0pt}%
\pgfpathmoveto{\pgfqpoint{1.430821in}{1.710228in}}%
\pgfpathlineto{\pgfqpoint{1.439758in}{1.710228in}}%
\pgfpathlineto{\pgfqpoint{1.439758in}{1.636759in}}%
\pgfpathlineto{\pgfqpoint{1.430821in}{1.636759in}}%
\pgfpathlineto{\pgfqpoint{1.430821in}{1.710228in}}%
\pgfpathclose%
\pgfusepath{fill}%
\end{pgfscope}%
\begin{pgfscope}%
\pgfpathrectangle{\pgfqpoint{0.697024in}{0.857143in}}{\pgfqpoint{2.627103in}{1.813434in}}%
\pgfusepath{clip}%
\pgfsetbuttcap%
\pgfsetmiterjoin%
\definecolor{currentfill}{rgb}{0.950697,0.616649,0.428624}%
\pgfsetfillcolor{currentfill}%
\pgfsetlinewidth{0.000000pt}%
\definecolor{currentstroke}{rgb}{0.000000,0.000000,0.000000}%
\pgfsetstrokecolor{currentstroke}%
\pgfsetstrokeopacity{0.000000}%
\pgfsetdash{}{0pt}%
\pgfpathmoveto{\pgfqpoint{1.441992in}{1.726231in}}%
\pgfpathlineto{\pgfqpoint{1.450929in}{1.726231in}}%
\pgfpathlineto{\pgfqpoint{1.450929in}{1.656603in}}%
\pgfpathlineto{\pgfqpoint{1.441992in}{1.656603in}}%
\pgfpathlineto{\pgfqpoint{1.441992in}{1.726231in}}%
\pgfpathclose%
\pgfusepath{fill}%
\end{pgfscope}%
\begin{pgfscope}%
\pgfpathrectangle{\pgfqpoint{0.697024in}{0.857143in}}{\pgfqpoint{2.627103in}{1.813434in}}%
\pgfusepath{clip}%
\pgfsetbuttcap%
\pgfsetmiterjoin%
\definecolor{currentfill}{rgb}{0.950697,0.616649,0.428624}%
\pgfsetfillcolor{currentfill}%
\pgfsetlinewidth{0.000000pt}%
\definecolor{currentstroke}{rgb}{0.000000,0.000000,0.000000}%
\pgfsetstrokecolor{currentstroke}%
\pgfsetstrokeopacity{0.000000}%
\pgfsetdash{}{0pt}%
\pgfpathmoveto{\pgfqpoint{1.453163in}{1.653772in}}%
\pgfpathlineto{\pgfqpoint{1.462099in}{1.653772in}}%
\pgfpathlineto{\pgfqpoint{1.462099in}{1.620018in}}%
\pgfpathlineto{\pgfqpoint{1.453163in}{1.620018in}}%
\pgfpathlineto{\pgfqpoint{1.453163in}{1.653772in}}%
\pgfpathclose%
\pgfusepath{fill}%
\end{pgfscope}%
\begin{pgfscope}%
\pgfpathrectangle{\pgfqpoint{0.697024in}{0.857143in}}{\pgfqpoint{2.627103in}{1.813434in}}%
\pgfusepath{clip}%
\pgfsetbuttcap%
\pgfsetmiterjoin%
\definecolor{currentfill}{rgb}{0.950697,0.616649,0.428624}%
\pgfsetfillcolor{currentfill}%
\pgfsetlinewidth{0.000000pt}%
\definecolor{currentstroke}{rgb}{0.000000,0.000000,0.000000}%
\pgfsetstrokecolor{currentstroke}%
\pgfsetstrokeopacity{0.000000}%
\pgfsetdash{}{0pt}%
\pgfpathmoveto{\pgfqpoint{1.464333in}{1.617038in}}%
\pgfpathlineto{\pgfqpoint{1.473270in}{1.617038in}}%
\pgfpathlineto{\pgfqpoint{1.473270in}{1.585146in}}%
\pgfpathlineto{\pgfqpoint{1.464333in}{1.585146in}}%
\pgfpathlineto{\pgfqpoint{1.464333in}{1.617038in}}%
\pgfpathclose%
\pgfusepath{fill}%
\end{pgfscope}%
\begin{pgfscope}%
\pgfpathrectangle{\pgfqpoint{0.697024in}{0.857143in}}{\pgfqpoint{2.627103in}{1.813434in}}%
\pgfusepath{clip}%
\pgfsetbuttcap%
\pgfsetmiterjoin%
\definecolor{currentfill}{rgb}{0.950697,0.616649,0.428624}%
\pgfsetfillcolor{currentfill}%
\pgfsetlinewidth{0.000000pt}%
\definecolor{currentstroke}{rgb}{0.000000,0.000000,0.000000}%
\pgfsetstrokecolor{currentstroke}%
\pgfsetstrokeopacity{0.000000}%
\pgfsetdash{}{0pt}%
\pgfpathmoveto{\pgfqpoint{1.475504in}{1.699439in}}%
\pgfpathlineto{\pgfqpoint{1.484440in}{1.699439in}}%
\pgfpathlineto{\pgfqpoint{1.484440in}{1.623228in}}%
\pgfpathlineto{\pgfqpoint{1.475504in}{1.623228in}}%
\pgfpathlineto{\pgfqpoint{1.475504in}{1.699439in}}%
\pgfpathclose%
\pgfusepath{fill}%
\end{pgfscope}%
\begin{pgfscope}%
\pgfpathrectangle{\pgfqpoint{0.697024in}{0.857143in}}{\pgfqpoint{2.627103in}{1.813434in}}%
\pgfusepath{clip}%
\pgfsetbuttcap%
\pgfsetmiterjoin%
\definecolor{currentfill}{rgb}{0.950697,0.616649,0.428624}%
\pgfsetfillcolor{currentfill}%
\pgfsetlinewidth{0.000000pt}%
\definecolor{currentstroke}{rgb}{0.000000,0.000000,0.000000}%
\pgfsetstrokecolor{currentstroke}%
\pgfsetstrokeopacity{0.000000}%
\pgfsetdash{}{0pt}%
\pgfpathmoveto{\pgfqpoint{1.486674in}{1.639420in}}%
\pgfpathlineto{\pgfqpoint{1.495611in}{1.639420in}}%
\pgfpathlineto{\pgfqpoint{1.495611in}{1.564069in}}%
\pgfpathlineto{\pgfqpoint{1.486674in}{1.564069in}}%
\pgfpathlineto{\pgfqpoint{1.486674in}{1.639420in}}%
\pgfpathclose%
\pgfusepath{fill}%
\end{pgfscope}%
\begin{pgfscope}%
\pgfpathrectangle{\pgfqpoint{0.697024in}{0.857143in}}{\pgfqpoint{2.627103in}{1.813434in}}%
\pgfusepath{clip}%
\pgfsetbuttcap%
\pgfsetmiterjoin%
\definecolor{currentfill}{rgb}{0.950697,0.616649,0.428624}%
\pgfsetfillcolor{currentfill}%
\pgfsetlinewidth{0.000000pt}%
\definecolor{currentstroke}{rgb}{0.000000,0.000000,0.000000}%
\pgfsetstrokecolor{currentstroke}%
\pgfsetstrokeopacity{0.000000}%
\pgfsetdash{}{0pt}%
\pgfpathmoveto{\pgfqpoint{1.497845in}{1.702827in}}%
\pgfpathlineto{\pgfqpoint{1.506782in}{1.702827in}}%
\pgfpathlineto{\pgfqpoint{1.506782in}{1.623005in}}%
\pgfpathlineto{\pgfqpoint{1.497845in}{1.623005in}}%
\pgfpathlineto{\pgfqpoint{1.497845in}{1.702827in}}%
\pgfpathclose%
\pgfusepath{fill}%
\end{pgfscope}%
\begin{pgfscope}%
\pgfpathrectangle{\pgfqpoint{0.697024in}{0.857143in}}{\pgfqpoint{2.627103in}{1.813434in}}%
\pgfusepath{clip}%
\pgfsetbuttcap%
\pgfsetmiterjoin%
\definecolor{currentfill}{rgb}{0.950697,0.616649,0.428624}%
\pgfsetfillcolor{currentfill}%
\pgfsetlinewidth{0.000000pt}%
\definecolor{currentstroke}{rgb}{0.000000,0.000000,0.000000}%
\pgfsetstrokecolor{currentstroke}%
\pgfsetstrokeopacity{0.000000}%
\pgfsetdash{}{0pt}%
\pgfpathmoveto{\pgfqpoint{1.509016in}{1.595905in}}%
\pgfpathlineto{\pgfqpoint{1.517952in}{1.595905in}}%
\pgfpathlineto{\pgfqpoint{1.517952in}{1.533002in}}%
\pgfpathlineto{\pgfqpoint{1.509016in}{1.533002in}}%
\pgfpathlineto{\pgfqpoint{1.509016in}{1.595905in}}%
\pgfpathclose%
\pgfusepath{fill}%
\end{pgfscope}%
\begin{pgfscope}%
\pgfpathrectangle{\pgfqpoint{0.697024in}{0.857143in}}{\pgfqpoint{2.627103in}{1.813434in}}%
\pgfusepath{clip}%
\pgfsetbuttcap%
\pgfsetmiterjoin%
\definecolor{currentfill}{rgb}{0.950697,0.616649,0.428624}%
\pgfsetfillcolor{currentfill}%
\pgfsetlinewidth{0.000000pt}%
\definecolor{currentstroke}{rgb}{0.000000,0.000000,0.000000}%
\pgfsetstrokecolor{currentstroke}%
\pgfsetstrokeopacity{0.000000}%
\pgfsetdash{}{0pt}%
\pgfpathmoveto{\pgfqpoint{1.520186in}{1.599730in}}%
\pgfpathlineto{\pgfqpoint{1.529123in}{1.599730in}}%
\pgfpathlineto{\pgfqpoint{1.529123in}{1.557686in}}%
\pgfpathlineto{\pgfqpoint{1.520186in}{1.557686in}}%
\pgfpathlineto{\pgfqpoint{1.520186in}{1.599730in}}%
\pgfpathclose%
\pgfusepath{fill}%
\end{pgfscope}%
\begin{pgfscope}%
\pgfpathrectangle{\pgfqpoint{0.697024in}{0.857143in}}{\pgfqpoint{2.627103in}{1.813434in}}%
\pgfusepath{clip}%
\pgfsetbuttcap%
\pgfsetmiterjoin%
\definecolor{currentfill}{rgb}{0.950697,0.616649,0.428624}%
\pgfsetfillcolor{currentfill}%
\pgfsetlinewidth{0.000000pt}%
\definecolor{currentstroke}{rgb}{0.000000,0.000000,0.000000}%
\pgfsetstrokecolor{currentstroke}%
\pgfsetstrokeopacity{0.000000}%
\pgfsetdash{}{0pt}%
\pgfpathmoveto{\pgfqpoint{1.531357in}{1.565269in}}%
\pgfpathlineto{\pgfqpoint{1.540293in}{1.565269in}}%
\pgfpathlineto{\pgfqpoint{1.540293in}{1.542582in}}%
\pgfpathlineto{\pgfqpoint{1.531357in}{1.542582in}}%
\pgfpathlineto{\pgfqpoint{1.531357in}{1.565269in}}%
\pgfpathclose%
\pgfusepath{fill}%
\end{pgfscope}%
\begin{pgfscope}%
\pgfpathrectangle{\pgfqpoint{0.697024in}{0.857143in}}{\pgfqpoint{2.627103in}{1.813434in}}%
\pgfusepath{clip}%
\pgfsetbuttcap%
\pgfsetmiterjoin%
\definecolor{currentfill}{rgb}{0.950697,0.616649,0.428624}%
\pgfsetfillcolor{currentfill}%
\pgfsetlinewidth{0.000000pt}%
\definecolor{currentstroke}{rgb}{0.000000,0.000000,0.000000}%
\pgfsetstrokecolor{currentstroke}%
\pgfsetstrokeopacity{0.000000}%
\pgfsetdash{}{0pt}%
\pgfpathmoveto{\pgfqpoint{1.542528in}{1.870810in}}%
\pgfpathlineto{\pgfqpoint{1.551464in}{1.870810in}}%
\pgfpathlineto{\pgfqpoint{1.551464in}{1.874449in}}%
\pgfpathlineto{\pgfqpoint{1.542528in}{1.874449in}}%
\pgfpathlineto{\pgfqpoint{1.542528in}{1.870810in}}%
\pgfpathclose%
\pgfusepath{fill}%
\end{pgfscope}%
\begin{pgfscope}%
\pgfpathrectangle{\pgfqpoint{0.697024in}{0.857143in}}{\pgfqpoint{2.627103in}{1.813434in}}%
\pgfusepath{clip}%
\pgfsetbuttcap%
\pgfsetmiterjoin%
\definecolor{currentfill}{rgb}{0.950697,0.616649,0.428624}%
\pgfsetfillcolor{currentfill}%
\pgfsetlinewidth{0.000000pt}%
\definecolor{currentstroke}{rgb}{0.000000,0.000000,0.000000}%
\pgfsetstrokecolor{currentstroke}%
\pgfsetstrokeopacity{0.000000}%
\pgfsetdash{}{0pt}%
\pgfpathmoveto{\pgfqpoint{1.553698in}{1.875473in}}%
\pgfpathlineto{\pgfqpoint{1.562635in}{1.875473in}}%
\pgfpathlineto{\pgfqpoint{1.562635in}{1.892999in}}%
\pgfpathlineto{\pgfqpoint{1.553698in}{1.892999in}}%
\pgfpathlineto{\pgfqpoint{1.553698in}{1.875473in}}%
\pgfpathclose%
\pgfusepath{fill}%
\end{pgfscope}%
\begin{pgfscope}%
\pgfpathrectangle{\pgfqpoint{0.697024in}{0.857143in}}{\pgfqpoint{2.627103in}{1.813434in}}%
\pgfusepath{clip}%
\pgfsetbuttcap%
\pgfsetmiterjoin%
\definecolor{currentfill}{rgb}{0.950697,0.616649,0.428624}%
\pgfsetfillcolor{currentfill}%
\pgfsetlinewidth{0.000000pt}%
\definecolor{currentstroke}{rgb}{0.000000,0.000000,0.000000}%
\pgfsetstrokecolor{currentstroke}%
\pgfsetstrokeopacity{0.000000}%
\pgfsetdash{}{0pt}%
\pgfpathmoveto{\pgfqpoint{1.564869in}{1.876334in}}%
\pgfpathlineto{\pgfqpoint{1.573805in}{1.876334in}}%
\pgfpathlineto{\pgfqpoint{1.573805in}{1.914315in}}%
\pgfpathlineto{\pgfqpoint{1.564869in}{1.914315in}}%
\pgfpathlineto{\pgfqpoint{1.564869in}{1.876334in}}%
\pgfpathclose%
\pgfusepath{fill}%
\end{pgfscope}%
\begin{pgfscope}%
\pgfpathrectangle{\pgfqpoint{0.697024in}{0.857143in}}{\pgfqpoint{2.627103in}{1.813434in}}%
\pgfusepath{clip}%
\pgfsetbuttcap%
\pgfsetmiterjoin%
\definecolor{currentfill}{rgb}{0.950697,0.616649,0.428624}%
\pgfsetfillcolor{currentfill}%
\pgfsetlinewidth{0.000000pt}%
\definecolor{currentstroke}{rgb}{0.000000,0.000000,0.000000}%
\pgfsetstrokecolor{currentstroke}%
\pgfsetstrokeopacity{0.000000}%
\pgfsetdash{}{0pt}%
\pgfpathmoveto{\pgfqpoint{1.576039in}{1.899399in}}%
\pgfpathlineto{\pgfqpoint{1.584976in}{1.899399in}}%
\pgfpathlineto{\pgfqpoint{1.584976in}{1.939373in}}%
\pgfpathlineto{\pgfqpoint{1.576039in}{1.939373in}}%
\pgfpathlineto{\pgfqpoint{1.576039in}{1.899399in}}%
\pgfpathclose%
\pgfusepath{fill}%
\end{pgfscope}%
\begin{pgfscope}%
\pgfpathrectangle{\pgfqpoint{0.697024in}{0.857143in}}{\pgfqpoint{2.627103in}{1.813434in}}%
\pgfusepath{clip}%
\pgfsetbuttcap%
\pgfsetmiterjoin%
\definecolor{currentfill}{rgb}{0.950697,0.616649,0.428624}%
\pgfsetfillcolor{currentfill}%
\pgfsetlinewidth{0.000000pt}%
\definecolor{currentstroke}{rgb}{0.000000,0.000000,0.000000}%
\pgfsetstrokecolor{currentstroke}%
\pgfsetstrokeopacity{0.000000}%
\pgfsetdash{}{0pt}%
\pgfpathmoveto{\pgfqpoint{1.587210in}{1.954975in}}%
\pgfpathlineto{\pgfqpoint{1.596146in}{1.954975in}}%
\pgfpathlineto{\pgfqpoint{1.596146in}{1.975220in}}%
\pgfpathlineto{\pgfqpoint{1.587210in}{1.975220in}}%
\pgfpathlineto{\pgfqpoint{1.587210in}{1.954975in}}%
\pgfpathclose%
\pgfusepath{fill}%
\end{pgfscope}%
\begin{pgfscope}%
\pgfpathrectangle{\pgfqpoint{0.697024in}{0.857143in}}{\pgfqpoint{2.627103in}{1.813434in}}%
\pgfusepath{clip}%
\pgfsetbuttcap%
\pgfsetmiterjoin%
\definecolor{currentfill}{rgb}{0.950697,0.616649,0.428624}%
\pgfsetfillcolor{currentfill}%
\pgfsetlinewidth{0.000000pt}%
\definecolor{currentstroke}{rgb}{0.000000,0.000000,0.000000}%
\pgfsetstrokecolor{currentstroke}%
\pgfsetstrokeopacity{0.000000}%
\pgfsetdash{}{0pt}%
\pgfpathmoveto{\pgfqpoint{1.598381in}{2.019231in}}%
\pgfpathlineto{\pgfqpoint{1.607317in}{2.019231in}}%
\pgfpathlineto{\pgfqpoint{1.607317in}{2.039154in}}%
\pgfpathlineto{\pgfqpoint{1.598381in}{2.039154in}}%
\pgfpathlineto{\pgfqpoint{1.598381in}{2.019231in}}%
\pgfpathclose%
\pgfusepath{fill}%
\end{pgfscope}%
\begin{pgfscope}%
\pgfpathrectangle{\pgfqpoint{0.697024in}{0.857143in}}{\pgfqpoint{2.627103in}{1.813434in}}%
\pgfusepath{clip}%
\pgfsetbuttcap%
\pgfsetmiterjoin%
\definecolor{currentfill}{rgb}{0.950697,0.616649,0.428624}%
\pgfsetfillcolor{currentfill}%
\pgfsetlinewidth{0.000000pt}%
\definecolor{currentstroke}{rgb}{0.000000,0.000000,0.000000}%
\pgfsetstrokecolor{currentstroke}%
\pgfsetstrokeopacity{0.000000}%
\pgfsetdash{}{0pt}%
\pgfpathmoveto{\pgfqpoint{1.609551in}{2.035773in}}%
\pgfpathlineto{\pgfqpoint{1.618488in}{2.035773in}}%
\pgfpathlineto{\pgfqpoint{1.618488in}{2.051595in}}%
\pgfpathlineto{\pgfqpoint{1.609551in}{2.051595in}}%
\pgfpathlineto{\pgfqpoint{1.609551in}{2.035773in}}%
\pgfpathclose%
\pgfusepath{fill}%
\end{pgfscope}%
\begin{pgfscope}%
\pgfpathrectangle{\pgfqpoint{0.697024in}{0.857143in}}{\pgfqpoint{2.627103in}{1.813434in}}%
\pgfusepath{clip}%
\pgfsetbuttcap%
\pgfsetmiterjoin%
\definecolor{currentfill}{rgb}{0.950697,0.616649,0.428624}%
\pgfsetfillcolor{currentfill}%
\pgfsetlinewidth{0.000000pt}%
\definecolor{currentstroke}{rgb}{0.000000,0.000000,0.000000}%
\pgfsetstrokecolor{currentstroke}%
\pgfsetstrokeopacity{0.000000}%
\pgfsetdash{}{0pt}%
\pgfpathmoveto{\pgfqpoint{1.620722in}{2.066723in}}%
\pgfpathlineto{\pgfqpoint{1.629658in}{2.066723in}}%
\pgfpathlineto{\pgfqpoint{1.629658in}{2.081081in}}%
\pgfpathlineto{\pgfqpoint{1.620722in}{2.081081in}}%
\pgfpathlineto{\pgfqpoint{1.620722in}{2.066723in}}%
\pgfpathclose%
\pgfusepath{fill}%
\end{pgfscope}%
\begin{pgfscope}%
\pgfpathrectangle{\pgfqpoint{0.697024in}{0.857143in}}{\pgfqpoint{2.627103in}{1.813434in}}%
\pgfusepath{clip}%
\pgfsetbuttcap%
\pgfsetmiterjoin%
\definecolor{currentfill}{rgb}{0.950697,0.616649,0.428624}%
\pgfsetfillcolor{currentfill}%
\pgfsetlinewidth{0.000000pt}%
\definecolor{currentstroke}{rgb}{0.000000,0.000000,0.000000}%
\pgfsetstrokecolor{currentstroke}%
\pgfsetstrokeopacity{0.000000}%
\pgfsetdash{}{0pt}%
\pgfpathmoveto{\pgfqpoint{1.631892in}{2.071831in}}%
\pgfpathlineto{\pgfqpoint{1.640829in}{2.071831in}}%
\pgfpathlineto{\pgfqpoint{1.640829in}{2.094352in}}%
\pgfpathlineto{\pgfqpoint{1.631892in}{2.094352in}}%
\pgfpathlineto{\pgfqpoint{1.631892in}{2.071831in}}%
\pgfpathclose%
\pgfusepath{fill}%
\end{pgfscope}%
\begin{pgfscope}%
\pgfpathrectangle{\pgfqpoint{0.697024in}{0.857143in}}{\pgfqpoint{2.627103in}{1.813434in}}%
\pgfusepath{clip}%
\pgfsetbuttcap%
\pgfsetmiterjoin%
\definecolor{currentfill}{rgb}{0.950697,0.616649,0.428624}%
\pgfsetfillcolor{currentfill}%
\pgfsetlinewidth{0.000000pt}%
\definecolor{currentstroke}{rgb}{0.000000,0.000000,0.000000}%
\pgfsetstrokecolor{currentstroke}%
\pgfsetstrokeopacity{0.000000}%
\pgfsetdash{}{0pt}%
\pgfpathmoveto{\pgfqpoint{1.643063in}{2.156191in}}%
\pgfpathlineto{\pgfqpoint{1.651999in}{2.156191in}}%
\pgfpathlineto{\pgfqpoint{1.651999in}{2.182543in}}%
\pgfpathlineto{\pgfqpoint{1.643063in}{2.182543in}}%
\pgfpathlineto{\pgfqpoint{1.643063in}{2.156191in}}%
\pgfpathclose%
\pgfusepath{fill}%
\end{pgfscope}%
\begin{pgfscope}%
\pgfpathrectangle{\pgfqpoint{0.697024in}{0.857143in}}{\pgfqpoint{2.627103in}{1.813434in}}%
\pgfusepath{clip}%
\pgfsetbuttcap%
\pgfsetmiterjoin%
\definecolor{currentfill}{rgb}{0.950697,0.616649,0.428624}%
\pgfsetfillcolor{currentfill}%
\pgfsetlinewidth{0.000000pt}%
\definecolor{currentstroke}{rgb}{0.000000,0.000000,0.000000}%
\pgfsetstrokecolor{currentstroke}%
\pgfsetstrokeopacity{0.000000}%
\pgfsetdash{}{0pt}%
\pgfpathmoveto{\pgfqpoint{1.654234in}{2.114525in}}%
\pgfpathlineto{\pgfqpoint{1.663170in}{2.114525in}}%
\pgfpathlineto{\pgfqpoint{1.663170in}{2.145777in}}%
\pgfpathlineto{\pgfqpoint{1.654234in}{2.145777in}}%
\pgfpathlineto{\pgfqpoint{1.654234in}{2.114525in}}%
\pgfpathclose%
\pgfusepath{fill}%
\end{pgfscope}%
\begin{pgfscope}%
\pgfpathrectangle{\pgfqpoint{0.697024in}{0.857143in}}{\pgfqpoint{2.627103in}{1.813434in}}%
\pgfusepath{clip}%
\pgfsetbuttcap%
\pgfsetmiterjoin%
\definecolor{currentfill}{rgb}{0.950697,0.616649,0.428624}%
\pgfsetfillcolor{currentfill}%
\pgfsetlinewidth{0.000000pt}%
\definecolor{currentstroke}{rgb}{0.000000,0.000000,0.000000}%
\pgfsetstrokecolor{currentstroke}%
\pgfsetstrokeopacity{0.000000}%
\pgfsetdash{}{0pt}%
\pgfpathmoveto{\pgfqpoint{1.665404in}{2.176579in}}%
\pgfpathlineto{\pgfqpoint{1.674341in}{2.176579in}}%
\pgfpathlineto{\pgfqpoint{1.674341in}{2.209012in}}%
\pgfpathlineto{\pgfqpoint{1.665404in}{2.209012in}}%
\pgfpathlineto{\pgfqpoint{1.665404in}{2.176579in}}%
\pgfpathclose%
\pgfusepath{fill}%
\end{pgfscope}%
\begin{pgfscope}%
\pgfpathrectangle{\pgfqpoint{0.697024in}{0.857143in}}{\pgfqpoint{2.627103in}{1.813434in}}%
\pgfusepath{clip}%
\pgfsetbuttcap%
\pgfsetmiterjoin%
\definecolor{currentfill}{rgb}{0.950697,0.616649,0.428624}%
\pgfsetfillcolor{currentfill}%
\pgfsetlinewidth{0.000000pt}%
\definecolor{currentstroke}{rgb}{0.000000,0.000000,0.000000}%
\pgfsetstrokecolor{currentstroke}%
\pgfsetstrokeopacity{0.000000}%
\pgfsetdash{}{0pt}%
\pgfpathmoveto{\pgfqpoint{1.676575in}{2.166234in}}%
\pgfpathlineto{\pgfqpoint{1.685511in}{2.166234in}}%
\pgfpathlineto{\pgfqpoint{1.685511in}{2.218261in}}%
\pgfpathlineto{\pgfqpoint{1.676575in}{2.218261in}}%
\pgfpathlineto{\pgfqpoint{1.676575in}{2.166234in}}%
\pgfpathclose%
\pgfusepath{fill}%
\end{pgfscope}%
\begin{pgfscope}%
\pgfpathrectangle{\pgfqpoint{0.697024in}{0.857143in}}{\pgfqpoint{2.627103in}{1.813434in}}%
\pgfusepath{clip}%
\pgfsetbuttcap%
\pgfsetmiterjoin%
\definecolor{currentfill}{rgb}{0.950697,0.616649,0.428624}%
\pgfsetfillcolor{currentfill}%
\pgfsetlinewidth{0.000000pt}%
\definecolor{currentstroke}{rgb}{0.000000,0.000000,0.000000}%
\pgfsetstrokecolor{currentstroke}%
\pgfsetstrokeopacity{0.000000}%
\pgfsetdash{}{0pt}%
\pgfpathmoveto{\pgfqpoint{1.687745in}{2.225778in}}%
\pgfpathlineto{\pgfqpoint{1.696682in}{2.225778in}}%
\pgfpathlineto{\pgfqpoint{1.696682in}{2.265517in}}%
\pgfpathlineto{\pgfqpoint{1.687745in}{2.265517in}}%
\pgfpathlineto{\pgfqpoint{1.687745in}{2.225778in}}%
\pgfpathclose%
\pgfusepath{fill}%
\end{pgfscope}%
\begin{pgfscope}%
\pgfpathrectangle{\pgfqpoint{0.697024in}{0.857143in}}{\pgfqpoint{2.627103in}{1.813434in}}%
\pgfusepath{clip}%
\pgfsetbuttcap%
\pgfsetmiterjoin%
\definecolor{currentfill}{rgb}{0.950697,0.616649,0.428624}%
\pgfsetfillcolor{currentfill}%
\pgfsetlinewidth{0.000000pt}%
\definecolor{currentstroke}{rgb}{0.000000,0.000000,0.000000}%
\pgfsetstrokecolor{currentstroke}%
\pgfsetstrokeopacity{0.000000}%
\pgfsetdash{}{0pt}%
\pgfpathmoveto{\pgfqpoint{1.698916in}{2.195750in}}%
\pgfpathlineto{\pgfqpoint{1.707852in}{2.195750in}}%
\pgfpathlineto{\pgfqpoint{1.707852in}{2.237459in}}%
\pgfpathlineto{\pgfqpoint{1.698916in}{2.237459in}}%
\pgfpathlineto{\pgfqpoint{1.698916in}{2.195750in}}%
\pgfpathclose%
\pgfusepath{fill}%
\end{pgfscope}%
\begin{pgfscope}%
\pgfpathrectangle{\pgfqpoint{0.697024in}{0.857143in}}{\pgfqpoint{2.627103in}{1.813434in}}%
\pgfusepath{clip}%
\pgfsetbuttcap%
\pgfsetmiterjoin%
\definecolor{currentfill}{rgb}{0.950697,0.616649,0.428624}%
\pgfsetfillcolor{currentfill}%
\pgfsetlinewidth{0.000000pt}%
\definecolor{currentstroke}{rgb}{0.000000,0.000000,0.000000}%
\pgfsetstrokecolor{currentstroke}%
\pgfsetstrokeopacity{0.000000}%
\pgfsetdash{}{0pt}%
\pgfpathmoveto{\pgfqpoint{1.710087in}{2.219420in}}%
\pgfpathlineto{\pgfqpoint{1.719023in}{2.219420in}}%
\pgfpathlineto{\pgfqpoint{1.719023in}{2.268306in}}%
\pgfpathlineto{\pgfqpoint{1.710087in}{2.268306in}}%
\pgfpathlineto{\pgfqpoint{1.710087in}{2.219420in}}%
\pgfpathclose%
\pgfusepath{fill}%
\end{pgfscope}%
\begin{pgfscope}%
\pgfpathrectangle{\pgfqpoint{0.697024in}{0.857143in}}{\pgfqpoint{2.627103in}{1.813434in}}%
\pgfusepath{clip}%
\pgfsetbuttcap%
\pgfsetmiterjoin%
\definecolor{currentfill}{rgb}{0.950697,0.616649,0.428624}%
\pgfsetfillcolor{currentfill}%
\pgfsetlinewidth{0.000000pt}%
\definecolor{currentstroke}{rgb}{0.000000,0.000000,0.000000}%
\pgfsetstrokecolor{currentstroke}%
\pgfsetstrokeopacity{0.000000}%
\pgfsetdash{}{0pt}%
\pgfpathmoveto{\pgfqpoint{1.721257in}{2.225791in}}%
\pgfpathlineto{\pgfqpoint{1.730194in}{2.225791in}}%
\pgfpathlineto{\pgfqpoint{1.730194in}{2.285604in}}%
\pgfpathlineto{\pgfqpoint{1.721257in}{2.285604in}}%
\pgfpathlineto{\pgfqpoint{1.721257in}{2.225791in}}%
\pgfpathclose%
\pgfusepath{fill}%
\end{pgfscope}%
\begin{pgfscope}%
\pgfpathrectangle{\pgfqpoint{0.697024in}{0.857143in}}{\pgfqpoint{2.627103in}{1.813434in}}%
\pgfusepath{clip}%
\pgfsetbuttcap%
\pgfsetmiterjoin%
\definecolor{currentfill}{rgb}{0.950697,0.616649,0.428624}%
\pgfsetfillcolor{currentfill}%
\pgfsetlinewidth{0.000000pt}%
\definecolor{currentstroke}{rgb}{0.000000,0.000000,0.000000}%
\pgfsetstrokecolor{currentstroke}%
\pgfsetstrokeopacity{0.000000}%
\pgfsetdash{}{0pt}%
\pgfpathmoveto{\pgfqpoint{1.732428in}{2.206952in}}%
\pgfpathlineto{\pgfqpoint{1.741364in}{2.206952in}}%
\pgfpathlineto{\pgfqpoint{1.741364in}{2.261070in}}%
\pgfpathlineto{\pgfqpoint{1.732428in}{2.261070in}}%
\pgfpathlineto{\pgfqpoint{1.732428in}{2.206952in}}%
\pgfpathclose%
\pgfusepath{fill}%
\end{pgfscope}%
\begin{pgfscope}%
\pgfpathrectangle{\pgfqpoint{0.697024in}{0.857143in}}{\pgfqpoint{2.627103in}{1.813434in}}%
\pgfusepath{clip}%
\pgfsetbuttcap%
\pgfsetmiterjoin%
\definecolor{currentfill}{rgb}{0.950697,0.616649,0.428624}%
\pgfsetfillcolor{currentfill}%
\pgfsetlinewidth{0.000000pt}%
\definecolor{currentstroke}{rgb}{0.000000,0.000000,0.000000}%
\pgfsetstrokecolor{currentstroke}%
\pgfsetstrokeopacity{0.000000}%
\pgfsetdash{}{0pt}%
\pgfpathmoveto{\pgfqpoint{1.743598in}{2.297173in}}%
\pgfpathlineto{\pgfqpoint{1.752535in}{2.297173in}}%
\pgfpathlineto{\pgfqpoint{1.752535in}{2.351429in}}%
\pgfpathlineto{\pgfqpoint{1.743598in}{2.351429in}}%
\pgfpathlineto{\pgfqpoint{1.743598in}{2.297173in}}%
\pgfpathclose%
\pgfusepath{fill}%
\end{pgfscope}%
\begin{pgfscope}%
\pgfpathrectangle{\pgfqpoint{0.697024in}{0.857143in}}{\pgfqpoint{2.627103in}{1.813434in}}%
\pgfusepath{clip}%
\pgfsetbuttcap%
\pgfsetmiterjoin%
\definecolor{currentfill}{rgb}{0.950697,0.616649,0.428624}%
\pgfsetfillcolor{currentfill}%
\pgfsetlinewidth{0.000000pt}%
\definecolor{currentstroke}{rgb}{0.000000,0.000000,0.000000}%
\pgfsetstrokecolor{currentstroke}%
\pgfsetstrokeopacity{0.000000}%
\pgfsetdash{}{0pt}%
\pgfpathmoveto{\pgfqpoint{1.754769in}{2.369671in}}%
\pgfpathlineto{\pgfqpoint{1.763705in}{2.369671in}}%
\pgfpathlineto{\pgfqpoint{1.763705in}{2.423932in}}%
\pgfpathlineto{\pgfqpoint{1.754769in}{2.423932in}}%
\pgfpathlineto{\pgfqpoint{1.754769in}{2.369671in}}%
\pgfpathclose%
\pgfusepath{fill}%
\end{pgfscope}%
\begin{pgfscope}%
\pgfpathrectangle{\pgfqpoint{0.697024in}{0.857143in}}{\pgfqpoint{2.627103in}{1.813434in}}%
\pgfusepath{clip}%
\pgfsetbuttcap%
\pgfsetmiterjoin%
\definecolor{currentfill}{rgb}{0.950697,0.616649,0.428624}%
\pgfsetfillcolor{currentfill}%
\pgfsetlinewidth{0.000000pt}%
\definecolor{currentstroke}{rgb}{0.000000,0.000000,0.000000}%
\pgfsetstrokecolor{currentstroke}%
\pgfsetstrokeopacity{0.000000}%
\pgfsetdash{}{0pt}%
\pgfpathmoveto{\pgfqpoint{1.765940in}{2.320194in}}%
\pgfpathlineto{\pgfqpoint{1.774876in}{2.320194in}}%
\pgfpathlineto{\pgfqpoint{1.774876in}{2.372893in}}%
\pgfpathlineto{\pgfqpoint{1.765940in}{2.372893in}}%
\pgfpathlineto{\pgfqpoint{1.765940in}{2.320194in}}%
\pgfpathclose%
\pgfusepath{fill}%
\end{pgfscope}%
\begin{pgfscope}%
\pgfpathrectangle{\pgfqpoint{0.697024in}{0.857143in}}{\pgfqpoint{2.627103in}{1.813434in}}%
\pgfusepath{clip}%
\pgfsetbuttcap%
\pgfsetmiterjoin%
\definecolor{currentfill}{rgb}{0.950697,0.616649,0.428624}%
\pgfsetfillcolor{currentfill}%
\pgfsetlinewidth{0.000000pt}%
\definecolor{currentstroke}{rgb}{0.000000,0.000000,0.000000}%
\pgfsetstrokecolor{currentstroke}%
\pgfsetstrokeopacity{0.000000}%
\pgfsetdash{}{0pt}%
\pgfpathmoveto{\pgfqpoint{1.777110in}{2.386386in}}%
\pgfpathlineto{\pgfqpoint{1.786047in}{2.386386in}}%
\pgfpathlineto{\pgfqpoint{1.786047in}{2.433506in}}%
\pgfpathlineto{\pgfqpoint{1.777110in}{2.433506in}}%
\pgfpathlineto{\pgfqpoint{1.777110in}{2.386386in}}%
\pgfpathclose%
\pgfusepath{fill}%
\end{pgfscope}%
\begin{pgfscope}%
\pgfpathrectangle{\pgfqpoint{0.697024in}{0.857143in}}{\pgfqpoint{2.627103in}{1.813434in}}%
\pgfusepath{clip}%
\pgfsetbuttcap%
\pgfsetmiterjoin%
\definecolor{currentfill}{rgb}{0.950697,0.616649,0.428624}%
\pgfsetfillcolor{currentfill}%
\pgfsetlinewidth{0.000000pt}%
\definecolor{currentstroke}{rgb}{0.000000,0.000000,0.000000}%
\pgfsetstrokecolor{currentstroke}%
\pgfsetstrokeopacity{0.000000}%
\pgfsetdash{}{0pt}%
\pgfpathmoveto{\pgfqpoint{1.788281in}{2.436448in}}%
\pgfpathlineto{\pgfqpoint{1.797217in}{2.436448in}}%
\pgfpathlineto{\pgfqpoint{1.797217in}{2.486008in}}%
\pgfpathlineto{\pgfqpoint{1.788281in}{2.486008in}}%
\pgfpathlineto{\pgfqpoint{1.788281in}{2.436448in}}%
\pgfpathclose%
\pgfusepath{fill}%
\end{pgfscope}%
\begin{pgfscope}%
\pgfpathrectangle{\pgfqpoint{0.697024in}{0.857143in}}{\pgfqpoint{2.627103in}{1.813434in}}%
\pgfusepath{clip}%
\pgfsetbuttcap%
\pgfsetmiterjoin%
\definecolor{currentfill}{rgb}{0.950697,0.616649,0.428624}%
\pgfsetfillcolor{currentfill}%
\pgfsetlinewidth{0.000000pt}%
\definecolor{currentstroke}{rgb}{0.000000,0.000000,0.000000}%
\pgfsetstrokecolor{currentstroke}%
\pgfsetstrokeopacity{0.000000}%
\pgfsetdash{}{0pt}%
\pgfpathmoveto{\pgfqpoint{1.799451in}{2.493124in}}%
\pgfpathlineto{\pgfqpoint{1.808388in}{2.493124in}}%
\pgfpathlineto{\pgfqpoint{1.808388in}{2.544515in}}%
\pgfpathlineto{\pgfqpoint{1.799451in}{2.544515in}}%
\pgfpathlineto{\pgfqpoint{1.799451in}{2.493124in}}%
\pgfpathclose%
\pgfusepath{fill}%
\end{pgfscope}%
\begin{pgfscope}%
\pgfpathrectangle{\pgfqpoint{0.697024in}{0.857143in}}{\pgfqpoint{2.627103in}{1.813434in}}%
\pgfusepath{clip}%
\pgfsetbuttcap%
\pgfsetmiterjoin%
\definecolor{currentfill}{rgb}{0.950697,0.616649,0.428624}%
\pgfsetfillcolor{currentfill}%
\pgfsetlinewidth{0.000000pt}%
\definecolor{currentstroke}{rgb}{0.000000,0.000000,0.000000}%
\pgfsetstrokecolor{currentstroke}%
\pgfsetstrokeopacity{0.000000}%
\pgfsetdash{}{0pt}%
\pgfpathmoveto{\pgfqpoint{1.810622in}{2.464668in}}%
\pgfpathlineto{\pgfqpoint{1.819559in}{2.464668in}}%
\pgfpathlineto{\pgfqpoint{1.819559in}{2.508139in}}%
\pgfpathlineto{\pgfqpoint{1.810622in}{2.508139in}}%
\pgfpathlineto{\pgfqpoint{1.810622in}{2.464668in}}%
\pgfpathclose%
\pgfusepath{fill}%
\end{pgfscope}%
\begin{pgfscope}%
\pgfpathrectangle{\pgfqpoint{0.697024in}{0.857143in}}{\pgfqpoint{2.627103in}{1.813434in}}%
\pgfusepath{clip}%
\pgfsetbuttcap%
\pgfsetmiterjoin%
\definecolor{currentfill}{rgb}{0.950697,0.616649,0.428624}%
\pgfsetfillcolor{currentfill}%
\pgfsetlinewidth{0.000000pt}%
\definecolor{currentstroke}{rgb}{0.000000,0.000000,0.000000}%
\pgfsetstrokecolor{currentstroke}%
\pgfsetstrokeopacity{0.000000}%
\pgfsetdash{}{0pt}%
\pgfpathmoveto{\pgfqpoint{1.821793in}{2.539751in}}%
\pgfpathlineto{\pgfqpoint{1.830729in}{2.539751in}}%
\pgfpathlineto{\pgfqpoint{1.830729in}{2.574485in}}%
\pgfpathlineto{\pgfqpoint{1.821793in}{2.574485in}}%
\pgfpathlineto{\pgfqpoint{1.821793in}{2.539751in}}%
\pgfpathclose%
\pgfusepath{fill}%
\end{pgfscope}%
\begin{pgfscope}%
\pgfpathrectangle{\pgfqpoint{0.697024in}{0.857143in}}{\pgfqpoint{2.627103in}{1.813434in}}%
\pgfusepath{clip}%
\pgfsetbuttcap%
\pgfsetmiterjoin%
\definecolor{currentfill}{rgb}{0.950697,0.616649,0.428624}%
\pgfsetfillcolor{currentfill}%
\pgfsetlinewidth{0.000000pt}%
\definecolor{currentstroke}{rgb}{0.000000,0.000000,0.000000}%
\pgfsetstrokecolor{currentstroke}%
\pgfsetstrokeopacity{0.000000}%
\pgfsetdash{}{0pt}%
\pgfpathmoveto{\pgfqpoint{1.832963in}{2.554731in}}%
\pgfpathlineto{\pgfqpoint{1.841900in}{2.554731in}}%
\pgfpathlineto{\pgfqpoint{1.841900in}{2.580974in}}%
\pgfpathlineto{\pgfqpoint{1.832963in}{2.580974in}}%
\pgfpathlineto{\pgfqpoint{1.832963in}{2.554731in}}%
\pgfpathclose%
\pgfusepath{fill}%
\end{pgfscope}%
\begin{pgfscope}%
\pgfpathrectangle{\pgfqpoint{0.697024in}{0.857143in}}{\pgfqpoint{2.627103in}{1.813434in}}%
\pgfusepath{clip}%
\pgfsetbuttcap%
\pgfsetmiterjoin%
\definecolor{currentfill}{rgb}{0.950697,0.616649,0.428624}%
\pgfsetfillcolor{currentfill}%
\pgfsetlinewidth{0.000000pt}%
\definecolor{currentstroke}{rgb}{0.000000,0.000000,0.000000}%
\pgfsetstrokecolor{currentstroke}%
\pgfsetstrokeopacity{0.000000}%
\pgfsetdash{}{0pt}%
\pgfpathmoveto{\pgfqpoint{1.844134in}{2.532507in}}%
\pgfpathlineto{\pgfqpoint{1.853070in}{2.532507in}}%
\pgfpathlineto{\pgfqpoint{1.853070in}{2.549198in}}%
\pgfpathlineto{\pgfqpoint{1.844134in}{2.549198in}}%
\pgfpathlineto{\pgfqpoint{1.844134in}{2.532507in}}%
\pgfpathclose%
\pgfusepath{fill}%
\end{pgfscope}%
\begin{pgfscope}%
\pgfpathrectangle{\pgfqpoint{0.697024in}{0.857143in}}{\pgfqpoint{2.627103in}{1.813434in}}%
\pgfusepath{clip}%
\pgfsetbuttcap%
\pgfsetmiterjoin%
\definecolor{currentfill}{rgb}{0.950697,0.616649,0.428624}%
\pgfsetfillcolor{currentfill}%
\pgfsetlinewidth{0.000000pt}%
\definecolor{currentstroke}{rgb}{0.000000,0.000000,0.000000}%
\pgfsetstrokecolor{currentstroke}%
\pgfsetstrokeopacity{0.000000}%
\pgfsetdash{}{0pt}%
\pgfpathmoveto{\pgfqpoint{1.855304in}{2.495537in}}%
\pgfpathlineto{\pgfqpoint{1.864241in}{2.495537in}}%
\pgfpathlineto{\pgfqpoint{1.864241in}{2.515681in}}%
\pgfpathlineto{\pgfqpoint{1.855304in}{2.515681in}}%
\pgfpathlineto{\pgfqpoint{1.855304in}{2.495537in}}%
\pgfpathclose%
\pgfusepath{fill}%
\end{pgfscope}%
\begin{pgfscope}%
\pgfpathrectangle{\pgfqpoint{0.697024in}{0.857143in}}{\pgfqpoint{2.627103in}{1.813434in}}%
\pgfusepath{clip}%
\pgfsetbuttcap%
\pgfsetmiterjoin%
\definecolor{currentfill}{rgb}{0.950697,0.616649,0.428624}%
\pgfsetfillcolor{currentfill}%
\pgfsetlinewidth{0.000000pt}%
\definecolor{currentstroke}{rgb}{0.000000,0.000000,0.000000}%
\pgfsetstrokecolor{currentstroke}%
\pgfsetstrokeopacity{0.000000}%
\pgfsetdash{}{0pt}%
\pgfpathmoveto{\pgfqpoint{1.866475in}{2.557023in}}%
\pgfpathlineto{\pgfqpoint{1.875412in}{2.557023in}}%
\pgfpathlineto{\pgfqpoint{1.875412in}{2.588148in}}%
\pgfpathlineto{\pgfqpoint{1.866475in}{2.588148in}}%
\pgfpathlineto{\pgfqpoint{1.866475in}{2.557023in}}%
\pgfpathclose%
\pgfusepath{fill}%
\end{pgfscope}%
\begin{pgfscope}%
\pgfpathrectangle{\pgfqpoint{0.697024in}{0.857143in}}{\pgfqpoint{2.627103in}{1.813434in}}%
\pgfusepath{clip}%
\pgfsetbuttcap%
\pgfsetmiterjoin%
\definecolor{currentfill}{rgb}{0.950697,0.616649,0.428624}%
\pgfsetfillcolor{currentfill}%
\pgfsetlinewidth{0.000000pt}%
\definecolor{currentstroke}{rgb}{0.000000,0.000000,0.000000}%
\pgfsetstrokecolor{currentstroke}%
\pgfsetstrokeopacity{0.000000}%
\pgfsetdash{}{0pt}%
\pgfpathmoveto{\pgfqpoint{1.877646in}{2.531505in}}%
\pgfpathlineto{\pgfqpoint{1.886582in}{2.531505in}}%
\pgfpathlineto{\pgfqpoint{1.886582in}{2.572084in}}%
\pgfpathlineto{\pgfqpoint{1.877646in}{2.572084in}}%
\pgfpathlineto{\pgfqpoint{1.877646in}{2.531505in}}%
\pgfpathclose%
\pgfusepath{fill}%
\end{pgfscope}%
\begin{pgfscope}%
\pgfpathrectangle{\pgfqpoint{0.697024in}{0.857143in}}{\pgfqpoint{2.627103in}{1.813434in}}%
\pgfusepath{clip}%
\pgfsetbuttcap%
\pgfsetmiterjoin%
\definecolor{currentfill}{rgb}{0.950697,0.616649,0.428624}%
\pgfsetfillcolor{currentfill}%
\pgfsetlinewidth{0.000000pt}%
\definecolor{currentstroke}{rgb}{0.000000,0.000000,0.000000}%
\pgfsetstrokecolor{currentstroke}%
\pgfsetstrokeopacity{0.000000}%
\pgfsetdash{}{0pt}%
\pgfpathmoveto{\pgfqpoint{1.888816in}{2.486022in}}%
\pgfpathlineto{\pgfqpoint{1.897753in}{2.486022in}}%
\pgfpathlineto{\pgfqpoint{1.897753in}{2.527301in}}%
\pgfpathlineto{\pgfqpoint{1.888816in}{2.527301in}}%
\pgfpathlineto{\pgfqpoint{1.888816in}{2.486022in}}%
\pgfpathclose%
\pgfusepath{fill}%
\end{pgfscope}%
\begin{pgfscope}%
\pgfpathrectangle{\pgfqpoint{0.697024in}{0.857143in}}{\pgfqpoint{2.627103in}{1.813434in}}%
\pgfusepath{clip}%
\pgfsetbuttcap%
\pgfsetmiterjoin%
\definecolor{currentfill}{rgb}{0.950697,0.616649,0.428624}%
\pgfsetfillcolor{currentfill}%
\pgfsetlinewidth{0.000000pt}%
\definecolor{currentstroke}{rgb}{0.000000,0.000000,0.000000}%
\pgfsetstrokecolor{currentstroke}%
\pgfsetstrokeopacity{0.000000}%
\pgfsetdash{}{0pt}%
\pgfpathmoveto{\pgfqpoint{1.899987in}{2.466340in}}%
\pgfpathlineto{\pgfqpoint{1.908923in}{2.466340in}}%
\pgfpathlineto{\pgfqpoint{1.908923in}{2.514709in}}%
\pgfpathlineto{\pgfqpoint{1.899987in}{2.514709in}}%
\pgfpathlineto{\pgfqpoint{1.899987in}{2.466340in}}%
\pgfpathclose%
\pgfusepath{fill}%
\end{pgfscope}%
\begin{pgfscope}%
\pgfpathrectangle{\pgfqpoint{0.697024in}{0.857143in}}{\pgfqpoint{2.627103in}{1.813434in}}%
\pgfusepath{clip}%
\pgfsetbuttcap%
\pgfsetmiterjoin%
\definecolor{currentfill}{rgb}{0.950697,0.616649,0.428624}%
\pgfsetfillcolor{currentfill}%
\pgfsetlinewidth{0.000000pt}%
\definecolor{currentstroke}{rgb}{0.000000,0.000000,0.000000}%
\pgfsetstrokecolor{currentstroke}%
\pgfsetstrokeopacity{0.000000}%
\pgfsetdash{}{0pt}%
\pgfpathmoveto{\pgfqpoint{1.911157in}{2.403472in}}%
\pgfpathlineto{\pgfqpoint{1.920094in}{2.403472in}}%
\pgfpathlineto{\pgfqpoint{1.920094in}{2.461025in}}%
\pgfpathlineto{\pgfqpoint{1.911157in}{2.461025in}}%
\pgfpathlineto{\pgfqpoint{1.911157in}{2.403472in}}%
\pgfpathclose%
\pgfusepath{fill}%
\end{pgfscope}%
\begin{pgfscope}%
\pgfpathrectangle{\pgfqpoint{0.697024in}{0.857143in}}{\pgfqpoint{2.627103in}{1.813434in}}%
\pgfusepath{clip}%
\pgfsetbuttcap%
\pgfsetmiterjoin%
\definecolor{currentfill}{rgb}{0.950697,0.616649,0.428624}%
\pgfsetfillcolor{currentfill}%
\pgfsetlinewidth{0.000000pt}%
\definecolor{currentstroke}{rgb}{0.000000,0.000000,0.000000}%
\pgfsetstrokecolor{currentstroke}%
\pgfsetstrokeopacity{0.000000}%
\pgfsetdash{}{0pt}%
\pgfpathmoveto{\pgfqpoint{1.922328in}{2.337884in}}%
\pgfpathlineto{\pgfqpoint{1.931265in}{2.337884in}}%
\pgfpathlineto{\pgfqpoint{1.931265in}{2.398822in}}%
\pgfpathlineto{\pgfqpoint{1.922328in}{2.398822in}}%
\pgfpathlineto{\pgfqpoint{1.922328in}{2.337884in}}%
\pgfpathclose%
\pgfusepath{fill}%
\end{pgfscope}%
\begin{pgfscope}%
\pgfpathrectangle{\pgfqpoint{0.697024in}{0.857143in}}{\pgfqpoint{2.627103in}{1.813434in}}%
\pgfusepath{clip}%
\pgfsetbuttcap%
\pgfsetmiterjoin%
\definecolor{currentfill}{rgb}{0.950697,0.616649,0.428624}%
\pgfsetfillcolor{currentfill}%
\pgfsetlinewidth{0.000000pt}%
\definecolor{currentstroke}{rgb}{0.000000,0.000000,0.000000}%
\pgfsetstrokecolor{currentstroke}%
\pgfsetstrokeopacity{0.000000}%
\pgfsetdash{}{0pt}%
\pgfpathmoveto{\pgfqpoint{1.933499in}{2.338974in}}%
\pgfpathlineto{\pgfqpoint{1.942435in}{2.338974in}}%
\pgfpathlineto{\pgfqpoint{1.942435in}{2.414335in}}%
\pgfpathlineto{\pgfqpoint{1.933499in}{2.414335in}}%
\pgfpathlineto{\pgfqpoint{1.933499in}{2.338974in}}%
\pgfpathclose%
\pgfusepath{fill}%
\end{pgfscope}%
\begin{pgfscope}%
\pgfpathrectangle{\pgfqpoint{0.697024in}{0.857143in}}{\pgfqpoint{2.627103in}{1.813434in}}%
\pgfusepath{clip}%
\pgfsetbuttcap%
\pgfsetmiterjoin%
\definecolor{currentfill}{rgb}{0.950697,0.616649,0.428624}%
\pgfsetfillcolor{currentfill}%
\pgfsetlinewidth{0.000000pt}%
\definecolor{currentstroke}{rgb}{0.000000,0.000000,0.000000}%
\pgfsetstrokecolor{currentstroke}%
\pgfsetstrokeopacity{0.000000}%
\pgfsetdash{}{0pt}%
\pgfpathmoveto{\pgfqpoint{1.944669in}{2.285655in}}%
\pgfpathlineto{\pgfqpoint{1.953606in}{2.285655in}}%
\pgfpathlineto{\pgfqpoint{1.953606in}{2.371801in}}%
\pgfpathlineto{\pgfqpoint{1.944669in}{2.371801in}}%
\pgfpathlineto{\pgfqpoint{1.944669in}{2.285655in}}%
\pgfpathclose%
\pgfusepath{fill}%
\end{pgfscope}%
\begin{pgfscope}%
\pgfpathrectangle{\pgfqpoint{0.697024in}{0.857143in}}{\pgfqpoint{2.627103in}{1.813434in}}%
\pgfusepath{clip}%
\pgfsetbuttcap%
\pgfsetmiterjoin%
\definecolor{currentfill}{rgb}{0.950697,0.616649,0.428624}%
\pgfsetfillcolor{currentfill}%
\pgfsetlinewidth{0.000000pt}%
\definecolor{currentstroke}{rgb}{0.000000,0.000000,0.000000}%
\pgfsetstrokecolor{currentstroke}%
\pgfsetstrokeopacity{0.000000}%
\pgfsetdash{}{0pt}%
\pgfpathmoveto{\pgfqpoint{1.955840in}{2.305523in}}%
\pgfpathlineto{\pgfqpoint{1.964776in}{2.305523in}}%
\pgfpathlineto{\pgfqpoint{1.964776in}{2.390021in}}%
\pgfpathlineto{\pgfqpoint{1.955840in}{2.390021in}}%
\pgfpathlineto{\pgfqpoint{1.955840in}{2.305523in}}%
\pgfpathclose%
\pgfusepath{fill}%
\end{pgfscope}%
\begin{pgfscope}%
\pgfpathrectangle{\pgfqpoint{0.697024in}{0.857143in}}{\pgfqpoint{2.627103in}{1.813434in}}%
\pgfusepath{clip}%
\pgfsetbuttcap%
\pgfsetmiterjoin%
\definecolor{currentfill}{rgb}{0.950697,0.616649,0.428624}%
\pgfsetfillcolor{currentfill}%
\pgfsetlinewidth{0.000000pt}%
\definecolor{currentstroke}{rgb}{0.000000,0.000000,0.000000}%
\pgfsetstrokecolor{currentstroke}%
\pgfsetstrokeopacity{0.000000}%
\pgfsetdash{}{0pt}%
\pgfpathmoveto{\pgfqpoint{1.967011in}{2.323811in}}%
\pgfpathlineto{\pgfqpoint{1.975947in}{2.323811in}}%
\pgfpathlineto{\pgfqpoint{1.975947in}{2.416297in}}%
\pgfpathlineto{\pgfqpoint{1.967011in}{2.416297in}}%
\pgfpathlineto{\pgfqpoint{1.967011in}{2.323811in}}%
\pgfpathclose%
\pgfusepath{fill}%
\end{pgfscope}%
\begin{pgfscope}%
\pgfpathrectangle{\pgfqpoint{0.697024in}{0.857143in}}{\pgfqpoint{2.627103in}{1.813434in}}%
\pgfusepath{clip}%
\pgfsetbuttcap%
\pgfsetmiterjoin%
\definecolor{currentfill}{rgb}{0.950697,0.616649,0.428624}%
\pgfsetfillcolor{currentfill}%
\pgfsetlinewidth{0.000000pt}%
\definecolor{currentstroke}{rgb}{0.000000,0.000000,0.000000}%
\pgfsetstrokecolor{currentstroke}%
\pgfsetstrokeopacity{0.000000}%
\pgfsetdash{}{0pt}%
\pgfpathmoveto{\pgfqpoint{1.978181in}{2.326929in}}%
\pgfpathlineto{\pgfqpoint{1.987118in}{2.326929in}}%
\pgfpathlineto{\pgfqpoint{1.987118in}{2.421483in}}%
\pgfpathlineto{\pgfqpoint{1.978181in}{2.421483in}}%
\pgfpathlineto{\pgfqpoint{1.978181in}{2.326929in}}%
\pgfpathclose%
\pgfusepath{fill}%
\end{pgfscope}%
\begin{pgfscope}%
\pgfpathrectangle{\pgfqpoint{0.697024in}{0.857143in}}{\pgfqpoint{2.627103in}{1.813434in}}%
\pgfusepath{clip}%
\pgfsetbuttcap%
\pgfsetmiterjoin%
\definecolor{currentfill}{rgb}{0.950697,0.616649,0.428624}%
\pgfsetfillcolor{currentfill}%
\pgfsetlinewidth{0.000000pt}%
\definecolor{currentstroke}{rgb}{0.000000,0.000000,0.000000}%
\pgfsetstrokecolor{currentstroke}%
\pgfsetstrokeopacity{0.000000}%
\pgfsetdash{}{0pt}%
\pgfpathmoveto{\pgfqpoint{1.989352in}{2.378473in}}%
\pgfpathlineto{\pgfqpoint{1.998288in}{2.378473in}}%
\pgfpathlineto{\pgfqpoint{1.998288in}{2.480530in}}%
\pgfpathlineto{\pgfqpoint{1.989352in}{2.480530in}}%
\pgfpathlineto{\pgfqpoint{1.989352in}{2.378473in}}%
\pgfpathclose%
\pgfusepath{fill}%
\end{pgfscope}%
\begin{pgfscope}%
\pgfpathrectangle{\pgfqpoint{0.697024in}{0.857143in}}{\pgfqpoint{2.627103in}{1.813434in}}%
\pgfusepath{clip}%
\pgfsetbuttcap%
\pgfsetmiterjoin%
\definecolor{currentfill}{rgb}{0.950697,0.616649,0.428624}%
\pgfsetfillcolor{currentfill}%
\pgfsetlinewidth{0.000000pt}%
\definecolor{currentstroke}{rgb}{0.000000,0.000000,0.000000}%
\pgfsetstrokecolor{currentstroke}%
\pgfsetstrokeopacity{0.000000}%
\pgfsetdash{}{0pt}%
\pgfpathmoveto{\pgfqpoint{2.000522in}{2.347268in}}%
\pgfpathlineto{\pgfqpoint{2.009459in}{2.347268in}}%
\pgfpathlineto{\pgfqpoint{2.009459in}{2.442951in}}%
\pgfpathlineto{\pgfqpoint{2.000522in}{2.442951in}}%
\pgfpathlineto{\pgfqpoint{2.000522in}{2.347268in}}%
\pgfpathclose%
\pgfusepath{fill}%
\end{pgfscope}%
\begin{pgfscope}%
\pgfpathrectangle{\pgfqpoint{0.697024in}{0.857143in}}{\pgfqpoint{2.627103in}{1.813434in}}%
\pgfusepath{clip}%
\pgfsetbuttcap%
\pgfsetmiterjoin%
\definecolor{currentfill}{rgb}{0.950697,0.616649,0.428624}%
\pgfsetfillcolor{currentfill}%
\pgfsetlinewidth{0.000000pt}%
\definecolor{currentstroke}{rgb}{0.000000,0.000000,0.000000}%
\pgfsetstrokecolor{currentstroke}%
\pgfsetstrokeopacity{0.000000}%
\pgfsetdash{}{0pt}%
\pgfpathmoveto{\pgfqpoint{2.011693in}{2.366153in}}%
\pgfpathlineto{\pgfqpoint{2.020629in}{2.366153in}}%
\pgfpathlineto{\pgfqpoint{2.020629in}{2.466140in}}%
\pgfpathlineto{\pgfqpoint{2.011693in}{2.466140in}}%
\pgfpathlineto{\pgfqpoint{2.011693in}{2.366153in}}%
\pgfpathclose%
\pgfusepath{fill}%
\end{pgfscope}%
\begin{pgfscope}%
\pgfpathrectangle{\pgfqpoint{0.697024in}{0.857143in}}{\pgfqpoint{2.627103in}{1.813434in}}%
\pgfusepath{clip}%
\pgfsetbuttcap%
\pgfsetmiterjoin%
\definecolor{currentfill}{rgb}{0.950697,0.616649,0.428624}%
\pgfsetfillcolor{currentfill}%
\pgfsetlinewidth{0.000000pt}%
\definecolor{currentstroke}{rgb}{0.000000,0.000000,0.000000}%
\pgfsetstrokecolor{currentstroke}%
\pgfsetstrokeopacity{0.000000}%
\pgfsetdash{}{0pt}%
\pgfpathmoveto{\pgfqpoint{2.022864in}{2.392575in}}%
\pgfpathlineto{\pgfqpoint{2.031800in}{2.392575in}}%
\pgfpathlineto{\pgfqpoint{2.031800in}{2.488272in}}%
\pgfpathlineto{\pgfqpoint{2.022864in}{2.488272in}}%
\pgfpathlineto{\pgfqpoint{2.022864in}{2.392575in}}%
\pgfpathclose%
\pgfusepath{fill}%
\end{pgfscope}%
\begin{pgfscope}%
\pgfpathrectangle{\pgfqpoint{0.697024in}{0.857143in}}{\pgfqpoint{2.627103in}{1.813434in}}%
\pgfusepath{clip}%
\pgfsetbuttcap%
\pgfsetmiterjoin%
\definecolor{currentfill}{rgb}{0.950697,0.616649,0.428624}%
\pgfsetfillcolor{currentfill}%
\pgfsetlinewidth{0.000000pt}%
\definecolor{currentstroke}{rgb}{0.000000,0.000000,0.000000}%
\pgfsetstrokecolor{currentstroke}%
\pgfsetstrokeopacity{0.000000}%
\pgfsetdash{}{0pt}%
\pgfpathmoveto{\pgfqpoint{2.034034in}{2.368110in}}%
\pgfpathlineto{\pgfqpoint{2.042971in}{2.368110in}}%
\pgfpathlineto{\pgfqpoint{2.042971in}{2.463065in}}%
\pgfpathlineto{\pgfqpoint{2.034034in}{2.463065in}}%
\pgfpathlineto{\pgfqpoint{2.034034in}{2.368110in}}%
\pgfpathclose%
\pgfusepath{fill}%
\end{pgfscope}%
\begin{pgfscope}%
\pgfpathrectangle{\pgfqpoint{0.697024in}{0.857143in}}{\pgfqpoint{2.627103in}{1.813434in}}%
\pgfusepath{clip}%
\pgfsetbuttcap%
\pgfsetmiterjoin%
\definecolor{currentfill}{rgb}{0.950697,0.616649,0.428624}%
\pgfsetfillcolor{currentfill}%
\pgfsetlinewidth{0.000000pt}%
\definecolor{currentstroke}{rgb}{0.000000,0.000000,0.000000}%
\pgfsetstrokecolor{currentstroke}%
\pgfsetstrokeopacity{0.000000}%
\pgfsetdash{}{0pt}%
\pgfpathmoveto{\pgfqpoint{2.045205in}{2.355274in}}%
\pgfpathlineto{\pgfqpoint{2.054141in}{2.355274in}}%
\pgfpathlineto{\pgfqpoint{2.054141in}{2.438686in}}%
\pgfpathlineto{\pgfqpoint{2.045205in}{2.438686in}}%
\pgfpathlineto{\pgfqpoint{2.045205in}{2.355274in}}%
\pgfpathclose%
\pgfusepath{fill}%
\end{pgfscope}%
\begin{pgfscope}%
\pgfpathrectangle{\pgfqpoint{0.697024in}{0.857143in}}{\pgfqpoint{2.627103in}{1.813434in}}%
\pgfusepath{clip}%
\pgfsetbuttcap%
\pgfsetmiterjoin%
\definecolor{currentfill}{rgb}{0.950697,0.616649,0.428624}%
\pgfsetfillcolor{currentfill}%
\pgfsetlinewidth{0.000000pt}%
\definecolor{currentstroke}{rgb}{0.000000,0.000000,0.000000}%
\pgfsetstrokecolor{currentstroke}%
\pgfsetstrokeopacity{0.000000}%
\pgfsetdash{}{0pt}%
\pgfpathmoveto{\pgfqpoint{2.056375in}{2.403056in}}%
\pgfpathlineto{\pgfqpoint{2.065312in}{2.403056in}}%
\pgfpathlineto{\pgfqpoint{2.065312in}{2.481097in}}%
\pgfpathlineto{\pgfqpoint{2.056375in}{2.481097in}}%
\pgfpathlineto{\pgfqpoint{2.056375in}{2.403056in}}%
\pgfpathclose%
\pgfusepath{fill}%
\end{pgfscope}%
\begin{pgfscope}%
\pgfpathrectangle{\pgfqpoint{0.697024in}{0.857143in}}{\pgfqpoint{2.627103in}{1.813434in}}%
\pgfusepath{clip}%
\pgfsetbuttcap%
\pgfsetmiterjoin%
\definecolor{currentfill}{rgb}{0.950697,0.616649,0.428624}%
\pgfsetfillcolor{currentfill}%
\pgfsetlinewidth{0.000000pt}%
\definecolor{currentstroke}{rgb}{0.000000,0.000000,0.000000}%
\pgfsetstrokecolor{currentstroke}%
\pgfsetstrokeopacity{0.000000}%
\pgfsetdash{}{0pt}%
\pgfpathmoveto{\pgfqpoint{2.067546in}{2.415344in}}%
\pgfpathlineto{\pgfqpoint{2.076482in}{2.415344in}}%
\pgfpathlineto{\pgfqpoint{2.076482in}{2.484598in}}%
\pgfpathlineto{\pgfqpoint{2.067546in}{2.484598in}}%
\pgfpathlineto{\pgfqpoint{2.067546in}{2.415344in}}%
\pgfpathclose%
\pgfusepath{fill}%
\end{pgfscope}%
\begin{pgfscope}%
\pgfpathrectangle{\pgfqpoint{0.697024in}{0.857143in}}{\pgfqpoint{2.627103in}{1.813434in}}%
\pgfusepath{clip}%
\pgfsetbuttcap%
\pgfsetmiterjoin%
\definecolor{currentfill}{rgb}{0.950697,0.616649,0.428624}%
\pgfsetfillcolor{currentfill}%
\pgfsetlinewidth{0.000000pt}%
\definecolor{currentstroke}{rgb}{0.000000,0.000000,0.000000}%
\pgfsetstrokecolor{currentstroke}%
\pgfsetstrokeopacity{0.000000}%
\pgfsetdash{}{0pt}%
\pgfpathmoveto{\pgfqpoint{2.078717in}{2.433772in}}%
\pgfpathlineto{\pgfqpoint{2.087653in}{2.433772in}}%
\pgfpathlineto{\pgfqpoint{2.087653in}{2.498342in}}%
\pgfpathlineto{\pgfqpoint{2.078717in}{2.498342in}}%
\pgfpathlineto{\pgfqpoint{2.078717in}{2.433772in}}%
\pgfpathclose%
\pgfusepath{fill}%
\end{pgfscope}%
\begin{pgfscope}%
\pgfpathrectangle{\pgfqpoint{0.697024in}{0.857143in}}{\pgfqpoint{2.627103in}{1.813434in}}%
\pgfusepath{clip}%
\pgfsetbuttcap%
\pgfsetmiterjoin%
\definecolor{currentfill}{rgb}{0.950697,0.616649,0.428624}%
\pgfsetfillcolor{currentfill}%
\pgfsetlinewidth{0.000000pt}%
\definecolor{currentstroke}{rgb}{0.000000,0.000000,0.000000}%
\pgfsetstrokecolor{currentstroke}%
\pgfsetstrokeopacity{0.000000}%
\pgfsetdash{}{0pt}%
\pgfpathmoveto{\pgfqpoint{2.089887in}{2.449224in}}%
\pgfpathlineto{\pgfqpoint{2.098824in}{2.449224in}}%
\pgfpathlineto{\pgfqpoint{2.098824in}{2.505832in}}%
\pgfpathlineto{\pgfqpoint{2.089887in}{2.505832in}}%
\pgfpathlineto{\pgfqpoint{2.089887in}{2.449224in}}%
\pgfpathclose%
\pgfusepath{fill}%
\end{pgfscope}%
\begin{pgfscope}%
\pgfpathrectangle{\pgfqpoint{0.697024in}{0.857143in}}{\pgfqpoint{2.627103in}{1.813434in}}%
\pgfusepath{clip}%
\pgfsetbuttcap%
\pgfsetmiterjoin%
\definecolor{currentfill}{rgb}{0.950697,0.616649,0.428624}%
\pgfsetfillcolor{currentfill}%
\pgfsetlinewidth{0.000000pt}%
\definecolor{currentstroke}{rgb}{0.000000,0.000000,0.000000}%
\pgfsetstrokecolor{currentstroke}%
\pgfsetstrokeopacity{0.000000}%
\pgfsetdash{}{0pt}%
\pgfpathmoveto{\pgfqpoint{2.101058in}{2.413895in}}%
\pgfpathlineto{\pgfqpoint{2.109994in}{2.413895in}}%
\pgfpathlineto{\pgfqpoint{2.109994in}{2.461772in}}%
\pgfpathlineto{\pgfqpoint{2.101058in}{2.461772in}}%
\pgfpathlineto{\pgfqpoint{2.101058in}{2.413895in}}%
\pgfpathclose%
\pgfusepath{fill}%
\end{pgfscope}%
\begin{pgfscope}%
\pgfpathrectangle{\pgfqpoint{0.697024in}{0.857143in}}{\pgfqpoint{2.627103in}{1.813434in}}%
\pgfusepath{clip}%
\pgfsetbuttcap%
\pgfsetmiterjoin%
\definecolor{currentfill}{rgb}{0.950697,0.616649,0.428624}%
\pgfsetfillcolor{currentfill}%
\pgfsetlinewidth{0.000000pt}%
\definecolor{currentstroke}{rgb}{0.000000,0.000000,0.000000}%
\pgfsetstrokecolor{currentstroke}%
\pgfsetstrokeopacity{0.000000}%
\pgfsetdash{}{0pt}%
\pgfpathmoveto{\pgfqpoint{2.112228in}{2.407727in}}%
\pgfpathlineto{\pgfqpoint{2.121165in}{2.407727in}}%
\pgfpathlineto{\pgfqpoint{2.121165in}{2.437777in}}%
\pgfpathlineto{\pgfqpoint{2.112228in}{2.437777in}}%
\pgfpathlineto{\pgfqpoint{2.112228in}{2.407727in}}%
\pgfpathclose%
\pgfusepath{fill}%
\end{pgfscope}%
\begin{pgfscope}%
\pgfpathrectangle{\pgfqpoint{0.697024in}{0.857143in}}{\pgfqpoint{2.627103in}{1.813434in}}%
\pgfusepath{clip}%
\pgfsetbuttcap%
\pgfsetmiterjoin%
\definecolor{currentfill}{rgb}{0.950697,0.616649,0.428624}%
\pgfsetfillcolor{currentfill}%
\pgfsetlinewidth{0.000000pt}%
\definecolor{currentstroke}{rgb}{0.000000,0.000000,0.000000}%
\pgfsetstrokecolor{currentstroke}%
\pgfsetstrokeopacity{0.000000}%
\pgfsetdash{}{0pt}%
\pgfpathmoveto{\pgfqpoint{2.123399in}{2.401973in}}%
\pgfpathlineto{\pgfqpoint{2.132335in}{2.401973in}}%
\pgfpathlineto{\pgfqpoint{2.132335in}{2.423923in}}%
\pgfpathlineto{\pgfqpoint{2.123399in}{2.423923in}}%
\pgfpathlineto{\pgfqpoint{2.123399in}{2.401973in}}%
\pgfpathclose%
\pgfusepath{fill}%
\end{pgfscope}%
\begin{pgfscope}%
\pgfpathrectangle{\pgfqpoint{0.697024in}{0.857143in}}{\pgfqpoint{2.627103in}{1.813434in}}%
\pgfusepath{clip}%
\pgfsetbuttcap%
\pgfsetmiterjoin%
\definecolor{currentfill}{rgb}{0.950697,0.616649,0.428624}%
\pgfsetfillcolor{currentfill}%
\pgfsetlinewidth{0.000000pt}%
\definecolor{currentstroke}{rgb}{0.000000,0.000000,0.000000}%
\pgfsetstrokecolor{currentstroke}%
\pgfsetstrokeopacity{0.000000}%
\pgfsetdash{}{0pt}%
\pgfpathmoveto{\pgfqpoint{2.134570in}{2.360711in}}%
\pgfpathlineto{\pgfqpoint{2.143506in}{2.360711in}}%
\pgfpathlineto{\pgfqpoint{2.143506in}{2.387310in}}%
\pgfpathlineto{\pgfqpoint{2.134570in}{2.387310in}}%
\pgfpathlineto{\pgfqpoint{2.134570in}{2.360711in}}%
\pgfpathclose%
\pgfusepath{fill}%
\end{pgfscope}%
\begin{pgfscope}%
\pgfpathrectangle{\pgfqpoint{0.697024in}{0.857143in}}{\pgfqpoint{2.627103in}{1.813434in}}%
\pgfusepath{clip}%
\pgfsetbuttcap%
\pgfsetmiterjoin%
\definecolor{currentfill}{rgb}{0.950697,0.616649,0.428624}%
\pgfsetfillcolor{currentfill}%
\pgfsetlinewidth{0.000000pt}%
\definecolor{currentstroke}{rgb}{0.000000,0.000000,0.000000}%
\pgfsetstrokecolor{currentstroke}%
\pgfsetstrokeopacity{0.000000}%
\pgfsetdash{}{0pt}%
\pgfpathmoveto{\pgfqpoint{2.145740in}{2.414896in}}%
\pgfpathlineto{\pgfqpoint{2.154677in}{2.414896in}}%
\pgfpathlineto{\pgfqpoint{2.154677in}{2.435929in}}%
\pgfpathlineto{\pgfqpoint{2.145740in}{2.435929in}}%
\pgfpathlineto{\pgfqpoint{2.145740in}{2.414896in}}%
\pgfpathclose%
\pgfusepath{fill}%
\end{pgfscope}%
\begin{pgfscope}%
\pgfpathrectangle{\pgfqpoint{0.697024in}{0.857143in}}{\pgfqpoint{2.627103in}{1.813434in}}%
\pgfusepath{clip}%
\pgfsetbuttcap%
\pgfsetmiterjoin%
\definecolor{currentfill}{rgb}{0.950697,0.616649,0.428624}%
\pgfsetfillcolor{currentfill}%
\pgfsetlinewidth{0.000000pt}%
\definecolor{currentstroke}{rgb}{0.000000,0.000000,0.000000}%
\pgfsetstrokecolor{currentstroke}%
\pgfsetstrokeopacity{0.000000}%
\pgfsetdash{}{0pt}%
\pgfpathmoveto{\pgfqpoint{2.156911in}{2.374657in}}%
\pgfpathlineto{\pgfqpoint{2.165847in}{2.374657in}}%
\pgfpathlineto{\pgfqpoint{2.165847in}{2.389940in}}%
\pgfpathlineto{\pgfqpoint{2.156911in}{2.389940in}}%
\pgfpathlineto{\pgfqpoint{2.156911in}{2.374657in}}%
\pgfpathclose%
\pgfusepath{fill}%
\end{pgfscope}%
\begin{pgfscope}%
\pgfpathrectangle{\pgfqpoint{0.697024in}{0.857143in}}{\pgfqpoint{2.627103in}{1.813434in}}%
\pgfusepath{clip}%
\pgfsetbuttcap%
\pgfsetmiterjoin%
\definecolor{currentfill}{rgb}{0.950697,0.616649,0.428624}%
\pgfsetfillcolor{currentfill}%
\pgfsetlinewidth{0.000000pt}%
\definecolor{currentstroke}{rgb}{0.000000,0.000000,0.000000}%
\pgfsetstrokecolor{currentstroke}%
\pgfsetstrokeopacity{0.000000}%
\pgfsetdash{}{0pt}%
\pgfpathmoveto{\pgfqpoint{2.168081in}{2.385756in}}%
\pgfpathlineto{\pgfqpoint{2.177018in}{2.385756in}}%
\pgfpathlineto{\pgfqpoint{2.177018in}{2.401366in}}%
\pgfpathlineto{\pgfqpoint{2.168081in}{2.401366in}}%
\pgfpathlineto{\pgfqpoint{2.168081in}{2.385756in}}%
\pgfpathclose%
\pgfusepath{fill}%
\end{pgfscope}%
\begin{pgfscope}%
\pgfpathrectangle{\pgfqpoint{0.697024in}{0.857143in}}{\pgfqpoint{2.627103in}{1.813434in}}%
\pgfusepath{clip}%
\pgfsetbuttcap%
\pgfsetmiterjoin%
\definecolor{currentfill}{rgb}{0.950697,0.616649,0.428624}%
\pgfsetfillcolor{currentfill}%
\pgfsetlinewidth{0.000000pt}%
\definecolor{currentstroke}{rgb}{0.000000,0.000000,0.000000}%
\pgfsetstrokecolor{currentstroke}%
\pgfsetstrokeopacity{0.000000}%
\pgfsetdash{}{0pt}%
\pgfpathmoveto{\pgfqpoint{2.179252in}{2.360398in}}%
\pgfpathlineto{\pgfqpoint{2.188189in}{2.360398in}}%
\pgfpathlineto{\pgfqpoint{2.188189in}{2.368944in}}%
\pgfpathlineto{\pgfqpoint{2.179252in}{2.368944in}}%
\pgfpathlineto{\pgfqpoint{2.179252in}{2.360398in}}%
\pgfpathclose%
\pgfusepath{fill}%
\end{pgfscope}%
\begin{pgfscope}%
\pgfpathrectangle{\pgfqpoint{0.697024in}{0.857143in}}{\pgfqpoint{2.627103in}{1.813434in}}%
\pgfusepath{clip}%
\pgfsetbuttcap%
\pgfsetmiterjoin%
\definecolor{currentfill}{rgb}{0.950697,0.616649,0.428624}%
\pgfsetfillcolor{currentfill}%
\pgfsetlinewidth{0.000000pt}%
\definecolor{currentstroke}{rgb}{0.000000,0.000000,0.000000}%
\pgfsetstrokecolor{currentstroke}%
\pgfsetstrokeopacity{0.000000}%
\pgfsetdash{}{0pt}%
\pgfpathmoveto{\pgfqpoint{2.190423in}{1.510320in}}%
\pgfpathlineto{\pgfqpoint{2.199359in}{1.510320in}}%
\pgfpathlineto{\pgfqpoint{2.199359in}{1.503768in}}%
\pgfpathlineto{\pgfqpoint{2.190423in}{1.503768in}}%
\pgfpathlineto{\pgfqpoint{2.190423in}{1.510320in}}%
\pgfpathclose%
\pgfusepath{fill}%
\end{pgfscope}%
\begin{pgfscope}%
\pgfpathrectangle{\pgfqpoint{0.697024in}{0.857143in}}{\pgfqpoint{2.627103in}{1.813434in}}%
\pgfusepath{clip}%
\pgfsetbuttcap%
\pgfsetmiterjoin%
\definecolor{currentfill}{rgb}{0.950697,0.616649,0.428624}%
\pgfsetfillcolor{currentfill}%
\pgfsetlinewidth{0.000000pt}%
\definecolor{currentstroke}{rgb}{0.000000,0.000000,0.000000}%
\pgfsetstrokecolor{currentstroke}%
\pgfsetstrokeopacity{0.000000}%
\pgfsetdash{}{0pt}%
\pgfpathmoveto{\pgfqpoint{2.201593in}{1.524901in}}%
\pgfpathlineto{\pgfqpoint{2.210530in}{1.524901in}}%
\pgfpathlineto{\pgfqpoint{2.210530in}{1.511276in}}%
\pgfpathlineto{\pgfqpoint{2.201593in}{1.511276in}}%
\pgfpathlineto{\pgfqpoint{2.201593in}{1.524901in}}%
\pgfpathclose%
\pgfusepath{fill}%
\end{pgfscope}%
\begin{pgfscope}%
\pgfpathrectangle{\pgfqpoint{0.697024in}{0.857143in}}{\pgfqpoint{2.627103in}{1.813434in}}%
\pgfusepath{clip}%
\pgfsetbuttcap%
\pgfsetmiterjoin%
\definecolor{currentfill}{rgb}{0.950697,0.616649,0.428624}%
\pgfsetfillcolor{currentfill}%
\pgfsetlinewidth{0.000000pt}%
\definecolor{currentstroke}{rgb}{0.000000,0.000000,0.000000}%
\pgfsetstrokecolor{currentstroke}%
\pgfsetstrokeopacity{0.000000}%
\pgfsetdash{}{0pt}%
\pgfpathmoveto{\pgfqpoint{2.212764in}{1.604314in}}%
\pgfpathlineto{\pgfqpoint{2.221700in}{1.604314in}}%
\pgfpathlineto{\pgfqpoint{2.221700in}{1.583507in}}%
\pgfpathlineto{\pgfqpoint{2.212764in}{1.583507in}}%
\pgfpathlineto{\pgfqpoint{2.212764in}{1.604314in}}%
\pgfpathclose%
\pgfusepath{fill}%
\end{pgfscope}%
\begin{pgfscope}%
\pgfpathrectangle{\pgfqpoint{0.697024in}{0.857143in}}{\pgfqpoint{2.627103in}{1.813434in}}%
\pgfusepath{clip}%
\pgfsetbuttcap%
\pgfsetmiterjoin%
\definecolor{currentfill}{rgb}{0.950697,0.616649,0.428624}%
\pgfsetfillcolor{currentfill}%
\pgfsetlinewidth{0.000000pt}%
\definecolor{currentstroke}{rgb}{0.000000,0.000000,0.000000}%
\pgfsetstrokecolor{currentstroke}%
\pgfsetstrokeopacity{0.000000}%
\pgfsetdash{}{0pt}%
\pgfpathmoveto{\pgfqpoint{2.223934in}{1.633924in}}%
\pgfpathlineto{\pgfqpoint{2.232871in}{1.633924in}}%
\pgfpathlineto{\pgfqpoint{2.232871in}{1.599898in}}%
\pgfpathlineto{\pgfqpoint{2.223934in}{1.599898in}}%
\pgfpathlineto{\pgfqpoint{2.223934in}{1.633924in}}%
\pgfpathclose%
\pgfusepath{fill}%
\end{pgfscope}%
\begin{pgfscope}%
\pgfpathrectangle{\pgfqpoint{0.697024in}{0.857143in}}{\pgfqpoint{2.627103in}{1.813434in}}%
\pgfusepath{clip}%
\pgfsetbuttcap%
\pgfsetmiterjoin%
\definecolor{currentfill}{rgb}{0.950697,0.616649,0.428624}%
\pgfsetfillcolor{currentfill}%
\pgfsetlinewidth{0.000000pt}%
\definecolor{currentstroke}{rgb}{0.000000,0.000000,0.000000}%
\pgfsetstrokecolor{currentstroke}%
\pgfsetstrokeopacity{0.000000}%
\pgfsetdash{}{0pt}%
\pgfpathmoveto{\pgfqpoint{2.235105in}{1.653666in}}%
\pgfpathlineto{\pgfqpoint{2.244042in}{1.653666in}}%
\pgfpathlineto{\pgfqpoint{2.244042in}{1.607094in}}%
\pgfpathlineto{\pgfqpoint{2.235105in}{1.607094in}}%
\pgfpathlineto{\pgfqpoint{2.235105in}{1.653666in}}%
\pgfpathclose%
\pgfusepath{fill}%
\end{pgfscope}%
\begin{pgfscope}%
\pgfpathrectangle{\pgfqpoint{0.697024in}{0.857143in}}{\pgfqpoint{2.627103in}{1.813434in}}%
\pgfusepath{clip}%
\pgfsetbuttcap%
\pgfsetmiterjoin%
\definecolor{currentfill}{rgb}{0.950697,0.616649,0.428624}%
\pgfsetfillcolor{currentfill}%
\pgfsetlinewidth{0.000000pt}%
\definecolor{currentstroke}{rgb}{0.000000,0.000000,0.000000}%
\pgfsetstrokecolor{currentstroke}%
\pgfsetstrokeopacity{0.000000}%
\pgfsetdash{}{0pt}%
\pgfpathmoveto{\pgfqpoint{2.246276in}{1.678920in}}%
\pgfpathlineto{\pgfqpoint{2.255212in}{1.678920in}}%
\pgfpathlineto{\pgfqpoint{2.255212in}{1.629889in}}%
\pgfpathlineto{\pgfqpoint{2.246276in}{1.629889in}}%
\pgfpathlineto{\pgfqpoint{2.246276in}{1.678920in}}%
\pgfpathclose%
\pgfusepath{fill}%
\end{pgfscope}%
\begin{pgfscope}%
\pgfpathrectangle{\pgfqpoint{0.697024in}{0.857143in}}{\pgfqpoint{2.627103in}{1.813434in}}%
\pgfusepath{clip}%
\pgfsetbuttcap%
\pgfsetmiterjoin%
\definecolor{currentfill}{rgb}{0.950697,0.616649,0.428624}%
\pgfsetfillcolor{currentfill}%
\pgfsetlinewidth{0.000000pt}%
\definecolor{currentstroke}{rgb}{0.000000,0.000000,0.000000}%
\pgfsetstrokecolor{currentstroke}%
\pgfsetstrokeopacity{0.000000}%
\pgfsetdash{}{0pt}%
\pgfpathmoveto{\pgfqpoint{2.257446in}{1.688186in}}%
\pgfpathlineto{\pgfqpoint{2.266383in}{1.688186in}}%
\pgfpathlineto{\pgfqpoint{2.266383in}{1.638487in}}%
\pgfpathlineto{\pgfqpoint{2.257446in}{1.638487in}}%
\pgfpathlineto{\pgfqpoint{2.257446in}{1.688186in}}%
\pgfpathclose%
\pgfusepath{fill}%
\end{pgfscope}%
\begin{pgfscope}%
\pgfpathrectangle{\pgfqpoint{0.697024in}{0.857143in}}{\pgfqpoint{2.627103in}{1.813434in}}%
\pgfusepath{clip}%
\pgfsetbuttcap%
\pgfsetmiterjoin%
\definecolor{currentfill}{rgb}{0.950697,0.616649,0.428624}%
\pgfsetfillcolor{currentfill}%
\pgfsetlinewidth{0.000000pt}%
\definecolor{currentstroke}{rgb}{0.000000,0.000000,0.000000}%
\pgfsetstrokecolor{currentstroke}%
\pgfsetstrokeopacity{0.000000}%
\pgfsetdash{}{0pt}%
\pgfpathmoveto{\pgfqpoint{2.268617in}{1.723848in}}%
\pgfpathlineto{\pgfqpoint{2.277553in}{1.723848in}}%
\pgfpathlineto{\pgfqpoint{2.277553in}{1.671483in}}%
\pgfpathlineto{\pgfqpoint{2.268617in}{1.671483in}}%
\pgfpathlineto{\pgfqpoint{2.268617in}{1.723848in}}%
\pgfpathclose%
\pgfusepath{fill}%
\end{pgfscope}%
\begin{pgfscope}%
\pgfpathrectangle{\pgfqpoint{0.697024in}{0.857143in}}{\pgfqpoint{2.627103in}{1.813434in}}%
\pgfusepath{clip}%
\pgfsetbuttcap%
\pgfsetmiterjoin%
\definecolor{currentfill}{rgb}{0.950697,0.616649,0.428624}%
\pgfsetfillcolor{currentfill}%
\pgfsetlinewidth{0.000000pt}%
\definecolor{currentstroke}{rgb}{0.000000,0.000000,0.000000}%
\pgfsetstrokecolor{currentstroke}%
\pgfsetstrokeopacity{0.000000}%
\pgfsetdash{}{0pt}%
\pgfpathmoveto{\pgfqpoint{2.279787in}{1.719717in}}%
\pgfpathlineto{\pgfqpoint{2.288724in}{1.719717in}}%
\pgfpathlineto{\pgfqpoint{2.288724in}{1.663684in}}%
\pgfpathlineto{\pgfqpoint{2.279787in}{1.663684in}}%
\pgfpathlineto{\pgfqpoint{2.279787in}{1.719717in}}%
\pgfpathclose%
\pgfusepath{fill}%
\end{pgfscope}%
\begin{pgfscope}%
\pgfpathrectangle{\pgfqpoint{0.697024in}{0.857143in}}{\pgfqpoint{2.627103in}{1.813434in}}%
\pgfusepath{clip}%
\pgfsetbuttcap%
\pgfsetmiterjoin%
\definecolor{currentfill}{rgb}{0.950697,0.616649,0.428624}%
\pgfsetfillcolor{currentfill}%
\pgfsetlinewidth{0.000000pt}%
\definecolor{currentstroke}{rgb}{0.000000,0.000000,0.000000}%
\pgfsetstrokecolor{currentstroke}%
\pgfsetstrokeopacity{0.000000}%
\pgfsetdash{}{0pt}%
\pgfpathmoveto{\pgfqpoint{2.290958in}{1.721015in}}%
\pgfpathlineto{\pgfqpoint{2.299895in}{1.721015in}}%
\pgfpathlineto{\pgfqpoint{2.299895in}{1.660269in}}%
\pgfpathlineto{\pgfqpoint{2.290958in}{1.660269in}}%
\pgfpathlineto{\pgfqpoint{2.290958in}{1.721015in}}%
\pgfpathclose%
\pgfusepath{fill}%
\end{pgfscope}%
\begin{pgfscope}%
\pgfpathrectangle{\pgfqpoint{0.697024in}{0.857143in}}{\pgfqpoint{2.627103in}{1.813434in}}%
\pgfusepath{clip}%
\pgfsetbuttcap%
\pgfsetmiterjoin%
\definecolor{currentfill}{rgb}{0.950697,0.616649,0.428624}%
\pgfsetfillcolor{currentfill}%
\pgfsetlinewidth{0.000000pt}%
\definecolor{currentstroke}{rgb}{0.000000,0.000000,0.000000}%
\pgfsetstrokecolor{currentstroke}%
\pgfsetstrokeopacity{0.000000}%
\pgfsetdash{}{0pt}%
\pgfpathmoveto{\pgfqpoint{2.302129in}{1.766714in}}%
\pgfpathlineto{\pgfqpoint{2.311065in}{1.766714in}}%
\pgfpathlineto{\pgfqpoint{2.311065in}{1.712229in}}%
\pgfpathlineto{\pgfqpoint{2.302129in}{1.712229in}}%
\pgfpathlineto{\pgfqpoint{2.302129in}{1.766714in}}%
\pgfpathclose%
\pgfusepath{fill}%
\end{pgfscope}%
\begin{pgfscope}%
\pgfpathrectangle{\pgfqpoint{0.697024in}{0.857143in}}{\pgfqpoint{2.627103in}{1.813434in}}%
\pgfusepath{clip}%
\pgfsetbuttcap%
\pgfsetmiterjoin%
\definecolor{currentfill}{rgb}{0.950697,0.616649,0.428624}%
\pgfsetfillcolor{currentfill}%
\pgfsetlinewidth{0.000000pt}%
\definecolor{currentstroke}{rgb}{0.000000,0.000000,0.000000}%
\pgfsetstrokecolor{currentstroke}%
\pgfsetstrokeopacity{0.000000}%
\pgfsetdash{}{0pt}%
\pgfpathmoveto{\pgfqpoint{2.313299in}{1.751222in}}%
\pgfpathlineto{\pgfqpoint{2.322236in}{1.751222in}}%
\pgfpathlineto{\pgfqpoint{2.322236in}{1.692748in}}%
\pgfpathlineto{\pgfqpoint{2.313299in}{1.692748in}}%
\pgfpathlineto{\pgfqpoint{2.313299in}{1.751222in}}%
\pgfpathclose%
\pgfusepath{fill}%
\end{pgfscope}%
\begin{pgfscope}%
\pgfpathrectangle{\pgfqpoint{0.697024in}{0.857143in}}{\pgfqpoint{2.627103in}{1.813434in}}%
\pgfusepath{clip}%
\pgfsetbuttcap%
\pgfsetmiterjoin%
\definecolor{currentfill}{rgb}{0.950697,0.616649,0.428624}%
\pgfsetfillcolor{currentfill}%
\pgfsetlinewidth{0.000000pt}%
\definecolor{currentstroke}{rgb}{0.000000,0.000000,0.000000}%
\pgfsetstrokecolor{currentstroke}%
\pgfsetstrokeopacity{0.000000}%
\pgfsetdash{}{0pt}%
\pgfpathmoveto{\pgfqpoint{2.324470in}{1.770001in}}%
\pgfpathlineto{\pgfqpoint{2.333406in}{1.770001in}}%
\pgfpathlineto{\pgfqpoint{2.333406in}{1.696703in}}%
\pgfpathlineto{\pgfqpoint{2.324470in}{1.696703in}}%
\pgfpathlineto{\pgfqpoint{2.324470in}{1.770001in}}%
\pgfpathclose%
\pgfusepath{fill}%
\end{pgfscope}%
\begin{pgfscope}%
\pgfpathrectangle{\pgfqpoint{0.697024in}{0.857143in}}{\pgfqpoint{2.627103in}{1.813434in}}%
\pgfusepath{clip}%
\pgfsetbuttcap%
\pgfsetmiterjoin%
\definecolor{currentfill}{rgb}{0.950697,0.616649,0.428624}%
\pgfsetfillcolor{currentfill}%
\pgfsetlinewidth{0.000000pt}%
\definecolor{currentstroke}{rgb}{0.000000,0.000000,0.000000}%
\pgfsetstrokecolor{currentstroke}%
\pgfsetstrokeopacity{0.000000}%
\pgfsetdash{}{0pt}%
\pgfpathmoveto{\pgfqpoint{2.335640in}{1.741229in}}%
\pgfpathlineto{\pgfqpoint{2.344577in}{1.741229in}}%
\pgfpathlineto{\pgfqpoint{2.344577in}{1.662706in}}%
\pgfpathlineto{\pgfqpoint{2.335640in}{1.662706in}}%
\pgfpathlineto{\pgfqpoint{2.335640in}{1.741229in}}%
\pgfpathclose%
\pgfusepath{fill}%
\end{pgfscope}%
\begin{pgfscope}%
\pgfpathrectangle{\pgfqpoint{0.697024in}{0.857143in}}{\pgfqpoint{2.627103in}{1.813434in}}%
\pgfusepath{clip}%
\pgfsetbuttcap%
\pgfsetmiterjoin%
\definecolor{currentfill}{rgb}{0.950697,0.616649,0.428624}%
\pgfsetfillcolor{currentfill}%
\pgfsetlinewidth{0.000000pt}%
\definecolor{currentstroke}{rgb}{0.000000,0.000000,0.000000}%
\pgfsetstrokecolor{currentstroke}%
\pgfsetstrokeopacity{0.000000}%
\pgfsetdash{}{0pt}%
\pgfpathmoveto{\pgfqpoint{2.346811in}{1.755711in}}%
\pgfpathlineto{\pgfqpoint{2.355748in}{1.755711in}}%
\pgfpathlineto{\pgfqpoint{2.355748in}{1.670774in}}%
\pgfpathlineto{\pgfqpoint{2.346811in}{1.670774in}}%
\pgfpathlineto{\pgfqpoint{2.346811in}{1.755711in}}%
\pgfpathclose%
\pgfusepath{fill}%
\end{pgfscope}%
\begin{pgfscope}%
\pgfpathrectangle{\pgfqpoint{0.697024in}{0.857143in}}{\pgfqpoint{2.627103in}{1.813434in}}%
\pgfusepath{clip}%
\pgfsetbuttcap%
\pgfsetmiterjoin%
\definecolor{currentfill}{rgb}{0.950697,0.616649,0.428624}%
\pgfsetfillcolor{currentfill}%
\pgfsetlinewidth{0.000000pt}%
\definecolor{currentstroke}{rgb}{0.000000,0.000000,0.000000}%
\pgfsetstrokecolor{currentstroke}%
\pgfsetstrokeopacity{0.000000}%
\pgfsetdash{}{0pt}%
\pgfpathmoveto{\pgfqpoint{2.357982in}{1.763193in}}%
\pgfpathlineto{\pgfqpoint{2.366918in}{1.763193in}}%
\pgfpathlineto{\pgfqpoint{2.366918in}{1.690765in}}%
\pgfpathlineto{\pgfqpoint{2.357982in}{1.690765in}}%
\pgfpathlineto{\pgfqpoint{2.357982in}{1.763193in}}%
\pgfpathclose%
\pgfusepath{fill}%
\end{pgfscope}%
\begin{pgfscope}%
\pgfpathrectangle{\pgfqpoint{0.697024in}{0.857143in}}{\pgfqpoint{2.627103in}{1.813434in}}%
\pgfusepath{clip}%
\pgfsetbuttcap%
\pgfsetmiterjoin%
\definecolor{currentfill}{rgb}{0.950697,0.616649,0.428624}%
\pgfsetfillcolor{currentfill}%
\pgfsetlinewidth{0.000000pt}%
\definecolor{currentstroke}{rgb}{0.000000,0.000000,0.000000}%
\pgfsetstrokecolor{currentstroke}%
\pgfsetstrokeopacity{0.000000}%
\pgfsetdash{}{0pt}%
\pgfpathmoveto{\pgfqpoint{2.369152in}{1.809295in}}%
\pgfpathlineto{\pgfqpoint{2.378089in}{1.809295in}}%
\pgfpathlineto{\pgfqpoint{2.378089in}{1.732798in}}%
\pgfpathlineto{\pgfqpoint{2.369152in}{1.732798in}}%
\pgfpathlineto{\pgfqpoint{2.369152in}{1.809295in}}%
\pgfpathclose%
\pgfusepath{fill}%
\end{pgfscope}%
\begin{pgfscope}%
\pgfpathrectangle{\pgfqpoint{0.697024in}{0.857143in}}{\pgfqpoint{2.627103in}{1.813434in}}%
\pgfusepath{clip}%
\pgfsetbuttcap%
\pgfsetmiterjoin%
\definecolor{currentfill}{rgb}{0.950697,0.616649,0.428624}%
\pgfsetfillcolor{currentfill}%
\pgfsetlinewidth{0.000000pt}%
\definecolor{currentstroke}{rgb}{0.000000,0.000000,0.000000}%
\pgfsetstrokecolor{currentstroke}%
\pgfsetstrokeopacity{0.000000}%
\pgfsetdash{}{0pt}%
\pgfpathmoveto{\pgfqpoint{2.380323in}{1.814409in}}%
\pgfpathlineto{\pgfqpoint{2.389259in}{1.814409in}}%
\pgfpathlineto{\pgfqpoint{2.389259in}{1.753544in}}%
\pgfpathlineto{\pgfqpoint{2.380323in}{1.753544in}}%
\pgfpathlineto{\pgfqpoint{2.380323in}{1.814409in}}%
\pgfpathclose%
\pgfusepath{fill}%
\end{pgfscope}%
\begin{pgfscope}%
\pgfpathrectangle{\pgfqpoint{0.697024in}{0.857143in}}{\pgfqpoint{2.627103in}{1.813434in}}%
\pgfusepath{clip}%
\pgfsetbuttcap%
\pgfsetmiterjoin%
\definecolor{currentfill}{rgb}{0.950697,0.616649,0.428624}%
\pgfsetfillcolor{currentfill}%
\pgfsetlinewidth{0.000000pt}%
\definecolor{currentstroke}{rgb}{0.000000,0.000000,0.000000}%
\pgfsetstrokecolor{currentstroke}%
\pgfsetstrokeopacity{0.000000}%
\pgfsetdash{}{0pt}%
\pgfpathmoveto{\pgfqpoint{2.391494in}{1.841825in}}%
\pgfpathlineto{\pgfqpoint{2.400430in}{1.841825in}}%
\pgfpathlineto{\pgfqpoint{2.400430in}{1.800125in}}%
\pgfpathlineto{\pgfqpoint{2.391494in}{1.800125in}}%
\pgfpathlineto{\pgfqpoint{2.391494in}{1.841825in}}%
\pgfpathclose%
\pgfusepath{fill}%
\end{pgfscope}%
\begin{pgfscope}%
\pgfpathrectangle{\pgfqpoint{0.697024in}{0.857143in}}{\pgfqpoint{2.627103in}{1.813434in}}%
\pgfusepath{clip}%
\pgfsetbuttcap%
\pgfsetmiterjoin%
\definecolor{currentfill}{rgb}{0.950697,0.616649,0.428624}%
\pgfsetfillcolor{currentfill}%
\pgfsetlinewidth{0.000000pt}%
\definecolor{currentstroke}{rgb}{0.000000,0.000000,0.000000}%
\pgfsetstrokecolor{currentstroke}%
\pgfsetstrokeopacity{0.000000}%
\pgfsetdash{}{0pt}%
\pgfpathmoveto{\pgfqpoint{2.402664in}{1.847071in}}%
\pgfpathlineto{\pgfqpoint{2.411601in}{1.847071in}}%
\pgfpathlineto{\pgfqpoint{2.411601in}{1.816364in}}%
\pgfpathlineto{\pgfqpoint{2.402664in}{1.816364in}}%
\pgfpathlineto{\pgfqpoint{2.402664in}{1.847071in}}%
\pgfpathclose%
\pgfusepath{fill}%
\end{pgfscope}%
\begin{pgfscope}%
\pgfpathrectangle{\pgfqpoint{0.697024in}{0.857143in}}{\pgfqpoint{2.627103in}{1.813434in}}%
\pgfusepath{clip}%
\pgfsetbuttcap%
\pgfsetmiterjoin%
\definecolor{currentfill}{rgb}{0.950697,0.616649,0.428624}%
\pgfsetfillcolor{currentfill}%
\pgfsetlinewidth{0.000000pt}%
\definecolor{currentstroke}{rgb}{0.000000,0.000000,0.000000}%
\pgfsetstrokecolor{currentstroke}%
\pgfsetstrokeopacity{0.000000}%
\pgfsetdash{}{0pt}%
\pgfpathmoveto{\pgfqpoint{2.413835in}{1.840384in}}%
\pgfpathlineto{\pgfqpoint{2.422771in}{1.840384in}}%
\pgfpathlineto{\pgfqpoint{2.422771in}{1.823576in}}%
\pgfpathlineto{\pgfqpoint{2.413835in}{1.823576in}}%
\pgfpathlineto{\pgfqpoint{2.413835in}{1.840384in}}%
\pgfpathclose%
\pgfusepath{fill}%
\end{pgfscope}%
\begin{pgfscope}%
\pgfpathrectangle{\pgfqpoint{0.697024in}{0.857143in}}{\pgfqpoint{2.627103in}{1.813434in}}%
\pgfusepath{clip}%
\pgfsetbuttcap%
\pgfsetmiterjoin%
\definecolor{currentfill}{rgb}{0.950697,0.616649,0.428624}%
\pgfsetfillcolor{currentfill}%
\pgfsetlinewidth{0.000000pt}%
\definecolor{currentstroke}{rgb}{0.000000,0.000000,0.000000}%
\pgfsetstrokecolor{currentstroke}%
\pgfsetstrokeopacity{0.000000}%
\pgfsetdash{}{0pt}%
\pgfpathmoveto{\pgfqpoint{2.425005in}{1.838978in}}%
\pgfpathlineto{\pgfqpoint{2.433942in}{1.838978in}}%
\pgfpathlineto{\pgfqpoint{2.433942in}{1.825863in}}%
\pgfpathlineto{\pgfqpoint{2.425005in}{1.825863in}}%
\pgfpathlineto{\pgfqpoint{2.425005in}{1.838978in}}%
\pgfpathclose%
\pgfusepath{fill}%
\end{pgfscope}%
\begin{pgfscope}%
\pgfpathrectangle{\pgfqpoint{0.697024in}{0.857143in}}{\pgfqpoint{2.627103in}{1.813434in}}%
\pgfusepath{clip}%
\pgfsetbuttcap%
\pgfsetmiterjoin%
\definecolor{currentfill}{rgb}{0.950697,0.616649,0.428624}%
\pgfsetfillcolor{currentfill}%
\pgfsetlinewidth{0.000000pt}%
\definecolor{currentstroke}{rgb}{0.000000,0.000000,0.000000}%
\pgfsetstrokecolor{currentstroke}%
\pgfsetstrokeopacity{0.000000}%
\pgfsetdash{}{0pt}%
\pgfpathmoveto{\pgfqpoint{2.436176in}{1.837258in}}%
\pgfpathlineto{\pgfqpoint{2.445112in}{1.837258in}}%
\pgfpathlineto{\pgfqpoint{2.445112in}{1.830374in}}%
\pgfpathlineto{\pgfqpoint{2.436176in}{1.830374in}}%
\pgfpathlineto{\pgfqpoint{2.436176in}{1.837258in}}%
\pgfpathclose%
\pgfusepath{fill}%
\end{pgfscope}%
\begin{pgfscope}%
\pgfpathrectangle{\pgfqpoint{0.697024in}{0.857143in}}{\pgfqpoint{2.627103in}{1.813434in}}%
\pgfusepath{clip}%
\pgfsetbuttcap%
\pgfsetmiterjoin%
\definecolor{currentfill}{rgb}{0.950697,0.616649,0.428624}%
\pgfsetfillcolor{currentfill}%
\pgfsetlinewidth{0.000000pt}%
\definecolor{currentstroke}{rgb}{0.000000,0.000000,0.000000}%
\pgfsetstrokecolor{currentstroke}%
\pgfsetstrokeopacity{0.000000}%
\pgfsetdash{}{0pt}%
\pgfpathmoveto{\pgfqpoint{2.447347in}{1.834166in}}%
\pgfpathlineto{\pgfqpoint{2.456283in}{1.834166in}}%
\pgfpathlineto{\pgfqpoint{2.456283in}{1.823124in}}%
\pgfpathlineto{\pgfqpoint{2.447347in}{1.823124in}}%
\pgfpathlineto{\pgfqpoint{2.447347in}{1.834166in}}%
\pgfpathclose%
\pgfusepath{fill}%
\end{pgfscope}%
\begin{pgfscope}%
\pgfpathrectangle{\pgfqpoint{0.697024in}{0.857143in}}{\pgfqpoint{2.627103in}{1.813434in}}%
\pgfusepath{clip}%
\pgfsetbuttcap%
\pgfsetmiterjoin%
\definecolor{currentfill}{rgb}{0.950697,0.616649,0.428624}%
\pgfsetfillcolor{currentfill}%
\pgfsetlinewidth{0.000000pt}%
\definecolor{currentstroke}{rgb}{0.000000,0.000000,0.000000}%
\pgfsetstrokecolor{currentstroke}%
\pgfsetstrokeopacity{0.000000}%
\pgfsetdash{}{0pt}%
\pgfpathmoveto{\pgfqpoint{2.458517in}{2.187571in}}%
\pgfpathlineto{\pgfqpoint{2.467454in}{2.187571in}}%
\pgfpathlineto{\pgfqpoint{2.467454in}{2.189230in}}%
\pgfpathlineto{\pgfqpoint{2.458517in}{2.189230in}}%
\pgfpathlineto{\pgfqpoint{2.458517in}{2.187571in}}%
\pgfpathclose%
\pgfusepath{fill}%
\end{pgfscope}%
\begin{pgfscope}%
\pgfpathrectangle{\pgfqpoint{0.697024in}{0.857143in}}{\pgfqpoint{2.627103in}{1.813434in}}%
\pgfusepath{clip}%
\pgfsetbuttcap%
\pgfsetmiterjoin%
\definecolor{currentfill}{rgb}{0.950697,0.616649,0.428624}%
\pgfsetfillcolor{currentfill}%
\pgfsetlinewidth{0.000000pt}%
\definecolor{currentstroke}{rgb}{0.000000,0.000000,0.000000}%
\pgfsetstrokecolor{currentstroke}%
\pgfsetstrokeopacity{0.000000}%
\pgfsetdash{}{0pt}%
\pgfpathmoveto{\pgfqpoint{2.469688in}{2.217593in}}%
\pgfpathlineto{\pgfqpoint{2.478624in}{2.217593in}}%
\pgfpathlineto{\pgfqpoint{2.478624in}{2.219747in}}%
\pgfpathlineto{\pgfqpoint{2.469688in}{2.219747in}}%
\pgfpathlineto{\pgfqpoint{2.469688in}{2.217593in}}%
\pgfpathclose%
\pgfusepath{fill}%
\end{pgfscope}%
\begin{pgfscope}%
\pgfpathrectangle{\pgfqpoint{0.697024in}{0.857143in}}{\pgfqpoint{2.627103in}{1.813434in}}%
\pgfusepath{clip}%
\pgfsetbuttcap%
\pgfsetmiterjoin%
\definecolor{currentfill}{rgb}{0.950697,0.616649,0.428624}%
\pgfsetfillcolor{currentfill}%
\pgfsetlinewidth{0.000000pt}%
\definecolor{currentstroke}{rgb}{0.000000,0.000000,0.000000}%
\pgfsetstrokecolor{currentstroke}%
\pgfsetstrokeopacity{0.000000}%
\pgfsetdash{}{0pt}%
\pgfpathmoveto{\pgfqpoint{2.480858in}{2.234952in}}%
\pgfpathlineto{\pgfqpoint{2.489795in}{2.234952in}}%
\pgfpathlineto{\pgfqpoint{2.489795in}{2.253313in}}%
\pgfpathlineto{\pgfqpoint{2.480858in}{2.253313in}}%
\pgfpathlineto{\pgfqpoint{2.480858in}{2.234952in}}%
\pgfpathclose%
\pgfusepath{fill}%
\end{pgfscope}%
\begin{pgfscope}%
\pgfpathrectangle{\pgfqpoint{0.697024in}{0.857143in}}{\pgfqpoint{2.627103in}{1.813434in}}%
\pgfusepath{clip}%
\pgfsetbuttcap%
\pgfsetmiterjoin%
\definecolor{currentfill}{rgb}{0.950697,0.616649,0.428624}%
\pgfsetfillcolor{currentfill}%
\pgfsetlinewidth{0.000000pt}%
\definecolor{currentstroke}{rgb}{0.000000,0.000000,0.000000}%
\pgfsetstrokecolor{currentstroke}%
\pgfsetstrokeopacity{0.000000}%
\pgfsetdash{}{0pt}%
\pgfpathmoveto{\pgfqpoint{2.492029in}{2.244252in}}%
\pgfpathlineto{\pgfqpoint{2.500965in}{2.244252in}}%
\pgfpathlineto{\pgfqpoint{2.500965in}{2.266055in}}%
\pgfpathlineto{\pgfqpoint{2.492029in}{2.266055in}}%
\pgfpathlineto{\pgfqpoint{2.492029in}{2.244252in}}%
\pgfpathclose%
\pgfusepath{fill}%
\end{pgfscope}%
\begin{pgfscope}%
\pgfpathrectangle{\pgfqpoint{0.697024in}{0.857143in}}{\pgfqpoint{2.627103in}{1.813434in}}%
\pgfusepath{clip}%
\pgfsetbuttcap%
\pgfsetmiterjoin%
\definecolor{currentfill}{rgb}{0.950697,0.616649,0.428624}%
\pgfsetfillcolor{currentfill}%
\pgfsetlinewidth{0.000000pt}%
\definecolor{currentstroke}{rgb}{0.000000,0.000000,0.000000}%
\pgfsetstrokecolor{currentstroke}%
\pgfsetstrokeopacity{0.000000}%
\pgfsetdash{}{0pt}%
\pgfpathmoveto{\pgfqpoint{2.503200in}{2.279294in}}%
\pgfpathlineto{\pgfqpoint{2.512136in}{2.279294in}}%
\pgfpathlineto{\pgfqpoint{2.512136in}{2.306646in}}%
\pgfpathlineto{\pgfqpoint{2.503200in}{2.306646in}}%
\pgfpathlineto{\pgfqpoint{2.503200in}{2.279294in}}%
\pgfpathclose%
\pgfusepath{fill}%
\end{pgfscope}%
\begin{pgfscope}%
\pgfpathrectangle{\pgfqpoint{0.697024in}{0.857143in}}{\pgfqpoint{2.627103in}{1.813434in}}%
\pgfusepath{clip}%
\pgfsetbuttcap%
\pgfsetmiterjoin%
\definecolor{currentfill}{rgb}{0.950697,0.616649,0.428624}%
\pgfsetfillcolor{currentfill}%
\pgfsetlinewidth{0.000000pt}%
\definecolor{currentstroke}{rgb}{0.000000,0.000000,0.000000}%
\pgfsetstrokecolor{currentstroke}%
\pgfsetstrokeopacity{0.000000}%
\pgfsetdash{}{0pt}%
\pgfpathmoveto{\pgfqpoint{2.514370in}{2.248110in}}%
\pgfpathlineto{\pgfqpoint{2.523307in}{2.248110in}}%
\pgfpathlineto{\pgfqpoint{2.523307in}{2.272483in}}%
\pgfpathlineto{\pgfqpoint{2.514370in}{2.272483in}}%
\pgfpathlineto{\pgfqpoint{2.514370in}{2.248110in}}%
\pgfpathclose%
\pgfusepath{fill}%
\end{pgfscope}%
\begin{pgfscope}%
\pgfpathrectangle{\pgfqpoint{0.697024in}{0.857143in}}{\pgfqpoint{2.627103in}{1.813434in}}%
\pgfusepath{clip}%
\pgfsetbuttcap%
\pgfsetmiterjoin%
\definecolor{currentfill}{rgb}{0.950697,0.616649,0.428624}%
\pgfsetfillcolor{currentfill}%
\pgfsetlinewidth{0.000000pt}%
\definecolor{currentstroke}{rgb}{0.000000,0.000000,0.000000}%
\pgfsetstrokecolor{currentstroke}%
\pgfsetstrokeopacity{0.000000}%
\pgfsetdash{}{0pt}%
\pgfpathmoveto{\pgfqpoint{2.525541in}{2.284391in}}%
\pgfpathlineto{\pgfqpoint{2.534477in}{2.284391in}}%
\pgfpathlineto{\pgfqpoint{2.534477in}{2.312880in}}%
\pgfpathlineto{\pgfqpoint{2.525541in}{2.312880in}}%
\pgfpathlineto{\pgfqpoint{2.525541in}{2.284391in}}%
\pgfpathclose%
\pgfusepath{fill}%
\end{pgfscope}%
\begin{pgfscope}%
\pgfpathrectangle{\pgfqpoint{0.697024in}{0.857143in}}{\pgfqpoint{2.627103in}{1.813434in}}%
\pgfusepath{clip}%
\pgfsetbuttcap%
\pgfsetmiterjoin%
\definecolor{currentfill}{rgb}{0.950697,0.616649,0.428624}%
\pgfsetfillcolor{currentfill}%
\pgfsetlinewidth{0.000000pt}%
\definecolor{currentstroke}{rgb}{0.000000,0.000000,0.000000}%
\pgfsetstrokecolor{currentstroke}%
\pgfsetstrokeopacity{0.000000}%
\pgfsetdash{}{0pt}%
\pgfpathmoveto{\pgfqpoint{2.536711in}{2.317801in}}%
\pgfpathlineto{\pgfqpoint{2.545648in}{2.317801in}}%
\pgfpathlineto{\pgfqpoint{2.545648in}{2.347521in}}%
\pgfpathlineto{\pgfqpoint{2.536711in}{2.347521in}}%
\pgfpathlineto{\pgfqpoint{2.536711in}{2.317801in}}%
\pgfpathclose%
\pgfusepath{fill}%
\end{pgfscope}%
\begin{pgfscope}%
\pgfpathrectangle{\pgfqpoint{0.697024in}{0.857143in}}{\pgfqpoint{2.627103in}{1.813434in}}%
\pgfusepath{clip}%
\pgfsetbuttcap%
\pgfsetmiterjoin%
\definecolor{currentfill}{rgb}{0.950697,0.616649,0.428624}%
\pgfsetfillcolor{currentfill}%
\pgfsetlinewidth{0.000000pt}%
\definecolor{currentstroke}{rgb}{0.000000,0.000000,0.000000}%
\pgfsetstrokecolor{currentstroke}%
\pgfsetstrokeopacity{0.000000}%
\pgfsetdash{}{0pt}%
\pgfpathmoveto{\pgfqpoint{2.547882in}{2.323016in}}%
\pgfpathlineto{\pgfqpoint{2.556818in}{2.323016in}}%
\pgfpathlineto{\pgfqpoint{2.556818in}{2.341213in}}%
\pgfpathlineto{\pgfqpoint{2.547882in}{2.341213in}}%
\pgfpathlineto{\pgfqpoint{2.547882in}{2.323016in}}%
\pgfpathclose%
\pgfusepath{fill}%
\end{pgfscope}%
\begin{pgfscope}%
\pgfpathrectangle{\pgfqpoint{0.697024in}{0.857143in}}{\pgfqpoint{2.627103in}{1.813434in}}%
\pgfusepath{clip}%
\pgfsetbuttcap%
\pgfsetmiterjoin%
\definecolor{currentfill}{rgb}{0.950697,0.616649,0.428624}%
\pgfsetfillcolor{currentfill}%
\pgfsetlinewidth{0.000000pt}%
\definecolor{currentstroke}{rgb}{0.000000,0.000000,0.000000}%
\pgfsetstrokecolor{currentstroke}%
\pgfsetstrokeopacity{0.000000}%
\pgfsetdash{}{0pt}%
\pgfpathmoveto{\pgfqpoint{2.559053in}{2.376971in}}%
\pgfpathlineto{\pgfqpoint{2.567989in}{2.376971in}}%
\pgfpathlineto{\pgfqpoint{2.567989in}{2.389750in}}%
\pgfpathlineto{\pgfqpoint{2.559053in}{2.389750in}}%
\pgfpathlineto{\pgfqpoint{2.559053in}{2.376971in}}%
\pgfpathclose%
\pgfusepath{fill}%
\end{pgfscope}%
\begin{pgfscope}%
\pgfpathrectangle{\pgfqpoint{0.697024in}{0.857143in}}{\pgfqpoint{2.627103in}{1.813434in}}%
\pgfusepath{clip}%
\pgfsetbuttcap%
\pgfsetmiterjoin%
\definecolor{currentfill}{rgb}{0.950697,0.616649,0.428624}%
\pgfsetfillcolor{currentfill}%
\pgfsetlinewidth{0.000000pt}%
\definecolor{currentstroke}{rgb}{0.000000,0.000000,0.000000}%
\pgfsetstrokecolor{currentstroke}%
\pgfsetstrokeopacity{0.000000}%
\pgfsetdash{}{0pt}%
\pgfpathmoveto{\pgfqpoint{2.570223in}{2.319717in}}%
\pgfpathlineto{\pgfqpoint{2.579160in}{2.319717in}}%
\pgfpathlineto{\pgfqpoint{2.579160in}{2.335600in}}%
\pgfpathlineto{\pgfqpoint{2.570223in}{2.335600in}}%
\pgfpathlineto{\pgfqpoint{2.570223in}{2.319717in}}%
\pgfpathclose%
\pgfusepath{fill}%
\end{pgfscope}%
\begin{pgfscope}%
\pgfpathrectangle{\pgfqpoint{0.697024in}{0.857143in}}{\pgfqpoint{2.627103in}{1.813434in}}%
\pgfusepath{clip}%
\pgfsetbuttcap%
\pgfsetmiterjoin%
\definecolor{currentfill}{rgb}{0.950697,0.616649,0.428624}%
\pgfsetfillcolor{currentfill}%
\pgfsetlinewidth{0.000000pt}%
\definecolor{currentstroke}{rgb}{0.000000,0.000000,0.000000}%
\pgfsetstrokecolor{currentstroke}%
\pgfsetstrokeopacity{0.000000}%
\pgfsetdash{}{0pt}%
\pgfpathmoveto{\pgfqpoint{2.581394in}{2.317750in}}%
\pgfpathlineto{\pgfqpoint{2.590330in}{2.317750in}}%
\pgfpathlineto{\pgfqpoint{2.590330in}{2.328392in}}%
\pgfpathlineto{\pgfqpoint{2.581394in}{2.328392in}}%
\pgfpathlineto{\pgfqpoint{2.581394in}{2.317750in}}%
\pgfpathclose%
\pgfusepath{fill}%
\end{pgfscope}%
\begin{pgfscope}%
\pgfpathrectangle{\pgfqpoint{0.697024in}{0.857143in}}{\pgfqpoint{2.627103in}{1.813434in}}%
\pgfusepath{clip}%
\pgfsetbuttcap%
\pgfsetmiterjoin%
\definecolor{currentfill}{rgb}{0.950697,0.616649,0.428624}%
\pgfsetfillcolor{currentfill}%
\pgfsetlinewidth{0.000000pt}%
\definecolor{currentstroke}{rgb}{0.000000,0.000000,0.000000}%
\pgfsetstrokecolor{currentstroke}%
\pgfsetstrokeopacity{0.000000}%
\pgfsetdash{}{0pt}%
\pgfpathmoveto{\pgfqpoint{2.592564in}{1.786277in}}%
\pgfpathlineto{\pgfqpoint{2.601501in}{1.786277in}}%
\pgfpathlineto{\pgfqpoint{2.601501in}{1.778011in}}%
\pgfpathlineto{\pgfqpoint{2.592564in}{1.778011in}}%
\pgfpathlineto{\pgfqpoint{2.592564in}{1.786277in}}%
\pgfpathclose%
\pgfusepath{fill}%
\end{pgfscope}%
\begin{pgfscope}%
\pgfpathrectangle{\pgfqpoint{0.697024in}{0.857143in}}{\pgfqpoint{2.627103in}{1.813434in}}%
\pgfusepath{clip}%
\pgfsetbuttcap%
\pgfsetmiterjoin%
\definecolor{currentfill}{rgb}{0.950697,0.616649,0.428624}%
\pgfsetfillcolor{currentfill}%
\pgfsetlinewidth{0.000000pt}%
\definecolor{currentstroke}{rgb}{0.000000,0.000000,0.000000}%
\pgfsetstrokecolor{currentstroke}%
\pgfsetstrokeopacity{0.000000}%
\pgfsetdash{}{0pt}%
\pgfpathmoveto{\pgfqpoint{2.603735in}{1.782268in}}%
\pgfpathlineto{\pgfqpoint{2.612672in}{1.782268in}}%
\pgfpathlineto{\pgfqpoint{2.612672in}{1.764251in}}%
\pgfpathlineto{\pgfqpoint{2.603735in}{1.764251in}}%
\pgfpathlineto{\pgfqpoint{2.603735in}{1.782268in}}%
\pgfpathclose%
\pgfusepath{fill}%
\end{pgfscope}%
\begin{pgfscope}%
\pgfpathrectangle{\pgfqpoint{0.697024in}{0.857143in}}{\pgfqpoint{2.627103in}{1.813434in}}%
\pgfusepath{clip}%
\pgfsetbuttcap%
\pgfsetmiterjoin%
\definecolor{currentfill}{rgb}{0.950697,0.616649,0.428624}%
\pgfsetfillcolor{currentfill}%
\pgfsetlinewidth{0.000000pt}%
\definecolor{currentstroke}{rgb}{0.000000,0.000000,0.000000}%
\pgfsetstrokecolor{currentstroke}%
\pgfsetstrokeopacity{0.000000}%
\pgfsetdash{}{0pt}%
\pgfpathmoveto{\pgfqpoint{2.614906in}{1.754386in}}%
\pgfpathlineto{\pgfqpoint{2.623842in}{1.754386in}}%
\pgfpathlineto{\pgfqpoint{2.623842in}{1.727025in}}%
\pgfpathlineto{\pgfqpoint{2.614906in}{1.727025in}}%
\pgfpathlineto{\pgfqpoint{2.614906in}{1.754386in}}%
\pgfpathclose%
\pgfusepath{fill}%
\end{pgfscope}%
\begin{pgfscope}%
\pgfpathrectangle{\pgfqpoint{0.697024in}{0.857143in}}{\pgfqpoint{2.627103in}{1.813434in}}%
\pgfusepath{clip}%
\pgfsetbuttcap%
\pgfsetmiterjoin%
\definecolor{currentfill}{rgb}{0.950697,0.616649,0.428624}%
\pgfsetfillcolor{currentfill}%
\pgfsetlinewidth{0.000000pt}%
\definecolor{currentstroke}{rgb}{0.000000,0.000000,0.000000}%
\pgfsetstrokecolor{currentstroke}%
\pgfsetstrokeopacity{0.000000}%
\pgfsetdash{}{0pt}%
\pgfpathmoveto{\pgfqpoint{2.626076in}{1.760957in}}%
\pgfpathlineto{\pgfqpoint{2.635013in}{1.760957in}}%
\pgfpathlineto{\pgfqpoint{2.635013in}{1.738216in}}%
\pgfpathlineto{\pgfqpoint{2.626076in}{1.738216in}}%
\pgfpathlineto{\pgfqpoint{2.626076in}{1.760957in}}%
\pgfpathclose%
\pgfusepath{fill}%
\end{pgfscope}%
\begin{pgfscope}%
\pgfpathrectangle{\pgfqpoint{0.697024in}{0.857143in}}{\pgfqpoint{2.627103in}{1.813434in}}%
\pgfusepath{clip}%
\pgfsetbuttcap%
\pgfsetmiterjoin%
\definecolor{currentfill}{rgb}{0.950697,0.616649,0.428624}%
\pgfsetfillcolor{currentfill}%
\pgfsetlinewidth{0.000000pt}%
\definecolor{currentstroke}{rgb}{0.000000,0.000000,0.000000}%
\pgfsetstrokecolor{currentstroke}%
\pgfsetstrokeopacity{0.000000}%
\pgfsetdash{}{0pt}%
\pgfpathmoveto{\pgfqpoint{2.637247in}{1.800007in}}%
\pgfpathlineto{\pgfqpoint{2.646183in}{1.800007in}}%
\pgfpathlineto{\pgfqpoint{2.646183in}{1.767072in}}%
\pgfpathlineto{\pgfqpoint{2.637247in}{1.767072in}}%
\pgfpathlineto{\pgfqpoint{2.637247in}{1.800007in}}%
\pgfpathclose%
\pgfusepath{fill}%
\end{pgfscope}%
\begin{pgfscope}%
\pgfpathrectangle{\pgfqpoint{0.697024in}{0.857143in}}{\pgfqpoint{2.627103in}{1.813434in}}%
\pgfusepath{clip}%
\pgfsetbuttcap%
\pgfsetmiterjoin%
\definecolor{currentfill}{rgb}{0.950697,0.616649,0.428624}%
\pgfsetfillcolor{currentfill}%
\pgfsetlinewidth{0.000000pt}%
\definecolor{currentstroke}{rgb}{0.000000,0.000000,0.000000}%
\pgfsetstrokecolor{currentstroke}%
\pgfsetstrokeopacity{0.000000}%
\pgfsetdash{}{0pt}%
\pgfpathmoveto{\pgfqpoint{2.648417in}{1.782330in}}%
\pgfpathlineto{\pgfqpoint{2.657354in}{1.782330in}}%
\pgfpathlineto{\pgfqpoint{2.657354in}{1.745337in}}%
\pgfpathlineto{\pgfqpoint{2.648417in}{1.745337in}}%
\pgfpathlineto{\pgfqpoint{2.648417in}{1.782330in}}%
\pgfpathclose%
\pgfusepath{fill}%
\end{pgfscope}%
\begin{pgfscope}%
\pgfpathrectangle{\pgfqpoint{0.697024in}{0.857143in}}{\pgfqpoint{2.627103in}{1.813434in}}%
\pgfusepath{clip}%
\pgfsetbuttcap%
\pgfsetmiterjoin%
\definecolor{currentfill}{rgb}{0.950697,0.616649,0.428624}%
\pgfsetfillcolor{currentfill}%
\pgfsetlinewidth{0.000000pt}%
\definecolor{currentstroke}{rgb}{0.000000,0.000000,0.000000}%
\pgfsetstrokecolor{currentstroke}%
\pgfsetstrokeopacity{0.000000}%
\pgfsetdash{}{0pt}%
\pgfpathmoveto{\pgfqpoint{2.659588in}{1.778636in}}%
\pgfpathlineto{\pgfqpoint{2.668525in}{1.778636in}}%
\pgfpathlineto{\pgfqpoint{2.668525in}{1.746514in}}%
\pgfpathlineto{\pgfqpoint{2.659588in}{1.746514in}}%
\pgfpathlineto{\pgfqpoint{2.659588in}{1.778636in}}%
\pgfpathclose%
\pgfusepath{fill}%
\end{pgfscope}%
\begin{pgfscope}%
\pgfpathrectangle{\pgfqpoint{0.697024in}{0.857143in}}{\pgfqpoint{2.627103in}{1.813434in}}%
\pgfusepath{clip}%
\pgfsetbuttcap%
\pgfsetmiterjoin%
\definecolor{currentfill}{rgb}{0.950697,0.616649,0.428624}%
\pgfsetfillcolor{currentfill}%
\pgfsetlinewidth{0.000000pt}%
\definecolor{currentstroke}{rgb}{0.000000,0.000000,0.000000}%
\pgfsetstrokecolor{currentstroke}%
\pgfsetstrokeopacity{0.000000}%
\pgfsetdash{}{0pt}%
\pgfpathmoveto{\pgfqpoint{2.670759in}{1.795088in}}%
\pgfpathlineto{\pgfqpoint{2.679695in}{1.795088in}}%
\pgfpathlineto{\pgfqpoint{2.679695in}{1.766105in}}%
\pgfpathlineto{\pgfqpoint{2.670759in}{1.766105in}}%
\pgfpathlineto{\pgfqpoint{2.670759in}{1.795088in}}%
\pgfpathclose%
\pgfusepath{fill}%
\end{pgfscope}%
\begin{pgfscope}%
\pgfpathrectangle{\pgfqpoint{0.697024in}{0.857143in}}{\pgfqpoint{2.627103in}{1.813434in}}%
\pgfusepath{clip}%
\pgfsetbuttcap%
\pgfsetmiterjoin%
\definecolor{currentfill}{rgb}{0.950697,0.616649,0.428624}%
\pgfsetfillcolor{currentfill}%
\pgfsetlinewidth{0.000000pt}%
\definecolor{currentstroke}{rgb}{0.000000,0.000000,0.000000}%
\pgfsetstrokecolor{currentstroke}%
\pgfsetstrokeopacity{0.000000}%
\pgfsetdash{}{0pt}%
\pgfpathmoveto{\pgfqpoint{2.681929in}{1.825396in}}%
\pgfpathlineto{\pgfqpoint{2.690866in}{1.825396in}}%
\pgfpathlineto{\pgfqpoint{2.690866in}{1.795130in}}%
\pgfpathlineto{\pgfqpoint{2.681929in}{1.795130in}}%
\pgfpathlineto{\pgfqpoint{2.681929in}{1.825396in}}%
\pgfpathclose%
\pgfusepath{fill}%
\end{pgfscope}%
\begin{pgfscope}%
\pgfpathrectangle{\pgfqpoint{0.697024in}{0.857143in}}{\pgfqpoint{2.627103in}{1.813434in}}%
\pgfusepath{clip}%
\pgfsetbuttcap%
\pgfsetmiterjoin%
\definecolor{currentfill}{rgb}{0.950697,0.616649,0.428624}%
\pgfsetfillcolor{currentfill}%
\pgfsetlinewidth{0.000000pt}%
\definecolor{currentstroke}{rgb}{0.000000,0.000000,0.000000}%
\pgfsetstrokecolor{currentstroke}%
\pgfsetstrokeopacity{0.000000}%
\pgfsetdash{}{0pt}%
\pgfpathmoveto{\pgfqpoint{2.693100in}{2.070007in}}%
\pgfpathlineto{\pgfqpoint{2.702036in}{2.070007in}}%
\pgfpathlineto{\pgfqpoint{2.702036in}{2.073518in}}%
\pgfpathlineto{\pgfqpoint{2.693100in}{2.073518in}}%
\pgfpathlineto{\pgfqpoint{2.693100in}{2.070007in}}%
\pgfpathclose%
\pgfusepath{fill}%
\end{pgfscope}%
\begin{pgfscope}%
\pgfpathrectangle{\pgfqpoint{0.697024in}{0.857143in}}{\pgfqpoint{2.627103in}{1.813434in}}%
\pgfusepath{clip}%
\pgfsetbuttcap%
\pgfsetmiterjoin%
\definecolor{currentfill}{rgb}{0.950697,0.616649,0.428624}%
\pgfsetfillcolor{currentfill}%
\pgfsetlinewidth{0.000000pt}%
\definecolor{currentstroke}{rgb}{0.000000,0.000000,0.000000}%
\pgfsetstrokecolor{currentstroke}%
\pgfsetstrokeopacity{0.000000}%
\pgfsetdash{}{0pt}%
\pgfpathmoveto{\pgfqpoint{2.704270in}{2.040296in}}%
\pgfpathlineto{\pgfqpoint{2.713207in}{2.040296in}}%
\pgfpathlineto{\pgfqpoint{2.713207in}{2.066740in}}%
\pgfpathlineto{\pgfqpoint{2.704270in}{2.066740in}}%
\pgfpathlineto{\pgfqpoint{2.704270in}{2.040296in}}%
\pgfpathclose%
\pgfusepath{fill}%
\end{pgfscope}%
\begin{pgfscope}%
\pgfpathrectangle{\pgfqpoint{0.697024in}{0.857143in}}{\pgfqpoint{2.627103in}{1.813434in}}%
\pgfusepath{clip}%
\pgfsetbuttcap%
\pgfsetmiterjoin%
\definecolor{currentfill}{rgb}{0.950697,0.616649,0.428624}%
\pgfsetfillcolor{currentfill}%
\pgfsetlinewidth{0.000000pt}%
\definecolor{currentstroke}{rgb}{0.000000,0.000000,0.000000}%
\pgfsetstrokecolor{currentstroke}%
\pgfsetstrokeopacity{0.000000}%
\pgfsetdash{}{0pt}%
\pgfpathmoveto{\pgfqpoint{2.715441in}{2.147082in}}%
\pgfpathlineto{\pgfqpoint{2.724378in}{2.147082in}}%
\pgfpathlineto{\pgfqpoint{2.724378in}{2.172620in}}%
\pgfpathlineto{\pgfqpoint{2.715441in}{2.172620in}}%
\pgfpathlineto{\pgfqpoint{2.715441in}{2.147082in}}%
\pgfpathclose%
\pgfusepath{fill}%
\end{pgfscope}%
\begin{pgfscope}%
\pgfpathrectangle{\pgfqpoint{0.697024in}{0.857143in}}{\pgfqpoint{2.627103in}{1.813434in}}%
\pgfusepath{clip}%
\pgfsetbuttcap%
\pgfsetmiterjoin%
\definecolor{currentfill}{rgb}{0.950697,0.616649,0.428624}%
\pgfsetfillcolor{currentfill}%
\pgfsetlinewidth{0.000000pt}%
\definecolor{currentstroke}{rgb}{0.000000,0.000000,0.000000}%
\pgfsetstrokecolor{currentstroke}%
\pgfsetstrokeopacity{0.000000}%
\pgfsetdash{}{0pt}%
\pgfpathmoveto{\pgfqpoint{2.726612in}{2.209477in}}%
\pgfpathlineto{\pgfqpoint{2.735548in}{2.209477in}}%
\pgfpathlineto{\pgfqpoint{2.735548in}{2.233234in}}%
\pgfpathlineto{\pgfqpoint{2.726612in}{2.233234in}}%
\pgfpathlineto{\pgfqpoint{2.726612in}{2.209477in}}%
\pgfpathclose%
\pgfusepath{fill}%
\end{pgfscope}%
\begin{pgfscope}%
\pgfpathrectangle{\pgfqpoint{0.697024in}{0.857143in}}{\pgfqpoint{2.627103in}{1.813434in}}%
\pgfusepath{clip}%
\pgfsetbuttcap%
\pgfsetmiterjoin%
\definecolor{currentfill}{rgb}{0.950697,0.616649,0.428624}%
\pgfsetfillcolor{currentfill}%
\pgfsetlinewidth{0.000000pt}%
\definecolor{currentstroke}{rgb}{0.000000,0.000000,0.000000}%
\pgfsetstrokecolor{currentstroke}%
\pgfsetstrokeopacity{0.000000}%
\pgfsetdash{}{0pt}%
\pgfpathmoveto{\pgfqpoint{2.737782in}{2.155518in}}%
\pgfpathlineto{\pgfqpoint{2.746719in}{2.155518in}}%
\pgfpathlineto{\pgfqpoint{2.746719in}{2.182782in}}%
\pgfpathlineto{\pgfqpoint{2.737782in}{2.182782in}}%
\pgfpathlineto{\pgfqpoint{2.737782in}{2.155518in}}%
\pgfpathclose%
\pgfusepath{fill}%
\end{pgfscope}%
\begin{pgfscope}%
\pgfpathrectangle{\pgfqpoint{0.697024in}{0.857143in}}{\pgfqpoint{2.627103in}{1.813434in}}%
\pgfusepath{clip}%
\pgfsetbuttcap%
\pgfsetmiterjoin%
\definecolor{currentfill}{rgb}{0.950697,0.616649,0.428624}%
\pgfsetfillcolor{currentfill}%
\pgfsetlinewidth{0.000000pt}%
\definecolor{currentstroke}{rgb}{0.000000,0.000000,0.000000}%
\pgfsetstrokecolor{currentstroke}%
\pgfsetstrokeopacity{0.000000}%
\pgfsetdash{}{0pt}%
\pgfpathmoveto{\pgfqpoint{2.748953in}{2.191518in}}%
\pgfpathlineto{\pgfqpoint{2.757889in}{2.191518in}}%
\pgfpathlineto{\pgfqpoint{2.757889in}{2.219148in}}%
\pgfpathlineto{\pgfqpoint{2.748953in}{2.219148in}}%
\pgfpathlineto{\pgfqpoint{2.748953in}{2.191518in}}%
\pgfpathclose%
\pgfusepath{fill}%
\end{pgfscope}%
\begin{pgfscope}%
\pgfpathrectangle{\pgfqpoint{0.697024in}{0.857143in}}{\pgfqpoint{2.627103in}{1.813434in}}%
\pgfusepath{clip}%
\pgfsetbuttcap%
\pgfsetmiterjoin%
\definecolor{currentfill}{rgb}{0.950697,0.616649,0.428624}%
\pgfsetfillcolor{currentfill}%
\pgfsetlinewidth{0.000000pt}%
\definecolor{currentstroke}{rgb}{0.000000,0.000000,0.000000}%
\pgfsetstrokecolor{currentstroke}%
\pgfsetstrokeopacity{0.000000}%
\pgfsetdash{}{0pt}%
\pgfpathmoveto{\pgfqpoint{2.760124in}{2.204755in}}%
\pgfpathlineto{\pgfqpoint{2.769060in}{2.204755in}}%
\pgfpathlineto{\pgfqpoint{2.769060in}{2.232120in}}%
\pgfpathlineto{\pgfqpoint{2.760124in}{2.232120in}}%
\pgfpathlineto{\pgfqpoint{2.760124in}{2.204755in}}%
\pgfpathclose%
\pgfusepath{fill}%
\end{pgfscope}%
\begin{pgfscope}%
\pgfpathrectangle{\pgfqpoint{0.697024in}{0.857143in}}{\pgfqpoint{2.627103in}{1.813434in}}%
\pgfusepath{clip}%
\pgfsetbuttcap%
\pgfsetmiterjoin%
\definecolor{currentfill}{rgb}{0.950697,0.616649,0.428624}%
\pgfsetfillcolor{currentfill}%
\pgfsetlinewidth{0.000000pt}%
\definecolor{currentstroke}{rgb}{0.000000,0.000000,0.000000}%
\pgfsetstrokecolor{currentstroke}%
\pgfsetstrokeopacity{0.000000}%
\pgfsetdash{}{0pt}%
\pgfpathmoveto{\pgfqpoint{2.771294in}{2.150246in}}%
\pgfpathlineto{\pgfqpoint{2.780231in}{2.150246in}}%
\pgfpathlineto{\pgfqpoint{2.780231in}{2.181529in}}%
\pgfpathlineto{\pgfqpoint{2.771294in}{2.181529in}}%
\pgfpathlineto{\pgfqpoint{2.771294in}{2.150246in}}%
\pgfpathclose%
\pgfusepath{fill}%
\end{pgfscope}%
\begin{pgfscope}%
\pgfpathrectangle{\pgfqpoint{0.697024in}{0.857143in}}{\pgfqpoint{2.627103in}{1.813434in}}%
\pgfusepath{clip}%
\pgfsetbuttcap%
\pgfsetmiterjoin%
\definecolor{currentfill}{rgb}{0.950697,0.616649,0.428624}%
\pgfsetfillcolor{currentfill}%
\pgfsetlinewidth{0.000000pt}%
\definecolor{currentstroke}{rgb}{0.000000,0.000000,0.000000}%
\pgfsetstrokecolor{currentstroke}%
\pgfsetstrokeopacity{0.000000}%
\pgfsetdash{}{0pt}%
\pgfpathmoveto{\pgfqpoint{2.782465in}{2.169728in}}%
\pgfpathlineto{\pgfqpoint{2.791401in}{2.169728in}}%
\pgfpathlineto{\pgfqpoint{2.791401in}{2.197647in}}%
\pgfpathlineto{\pgfqpoint{2.782465in}{2.197647in}}%
\pgfpathlineto{\pgfqpoint{2.782465in}{2.169728in}}%
\pgfpathclose%
\pgfusepath{fill}%
\end{pgfscope}%
\begin{pgfscope}%
\pgfpathrectangle{\pgfqpoint{0.697024in}{0.857143in}}{\pgfqpoint{2.627103in}{1.813434in}}%
\pgfusepath{clip}%
\pgfsetbuttcap%
\pgfsetmiterjoin%
\definecolor{currentfill}{rgb}{0.950697,0.616649,0.428624}%
\pgfsetfillcolor{currentfill}%
\pgfsetlinewidth{0.000000pt}%
\definecolor{currentstroke}{rgb}{0.000000,0.000000,0.000000}%
\pgfsetstrokecolor{currentstroke}%
\pgfsetstrokeopacity{0.000000}%
\pgfsetdash{}{0pt}%
\pgfpathmoveto{\pgfqpoint{2.793635in}{2.167376in}}%
\pgfpathlineto{\pgfqpoint{2.802572in}{2.167376in}}%
\pgfpathlineto{\pgfqpoint{2.802572in}{2.195823in}}%
\pgfpathlineto{\pgfqpoint{2.793635in}{2.195823in}}%
\pgfpathlineto{\pgfqpoint{2.793635in}{2.167376in}}%
\pgfpathclose%
\pgfusepath{fill}%
\end{pgfscope}%
\begin{pgfscope}%
\pgfpathrectangle{\pgfqpoint{0.697024in}{0.857143in}}{\pgfqpoint{2.627103in}{1.813434in}}%
\pgfusepath{clip}%
\pgfsetbuttcap%
\pgfsetmiterjoin%
\definecolor{currentfill}{rgb}{0.950697,0.616649,0.428624}%
\pgfsetfillcolor{currentfill}%
\pgfsetlinewidth{0.000000pt}%
\definecolor{currentstroke}{rgb}{0.000000,0.000000,0.000000}%
\pgfsetstrokecolor{currentstroke}%
\pgfsetstrokeopacity{0.000000}%
\pgfsetdash{}{0pt}%
\pgfpathmoveto{\pgfqpoint{2.804806in}{2.167998in}}%
\pgfpathlineto{\pgfqpoint{2.813742in}{2.167998in}}%
\pgfpathlineto{\pgfqpoint{2.813742in}{2.189845in}}%
\pgfpathlineto{\pgfqpoint{2.804806in}{2.189845in}}%
\pgfpathlineto{\pgfqpoint{2.804806in}{2.167998in}}%
\pgfpathclose%
\pgfusepath{fill}%
\end{pgfscope}%
\begin{pgfscope}%
\pgfpathrectangle{\pgfqpoint{0.697024in}{0.857143in}}{\pgfqpoint{2.627103in}{1.813434in}}%
\pgfusepath{clip}%
\pgfsetbuttcap%
\pgfsetmiterjoin%
\definecolor{currentfill}{rgb}{0.950697,0.616649,0.428624}%
\pgfsetfillcolor{currentfill}%
\pgfsetlinewidth{0.000000pt}%
\definecolor{currentstroke}{rgb}{0.000000,0.000000,0.000000}%
\pgfsetstrokecolor{currentstroke}%
\pgfsetstrokeopacity{0.000000}%
\pgfsetdash{}{0pt}%
\pgfpathmoveto{\pgfqpoint{2.815977in}{2.077594in}}%
\pgfpathlineto{\pgfqpoint{2.824913in}{2.077594in}}%
\pgfpathlineto{\pgfqpoint{2.824913in}{2.099166in}}%
\pgfpathlineto{\pgfqpoint{2.815977in}{2.099166in}}%
\pgfpathlineto{\pgfqpoint{2.815977in}{2.077594in}}%
\pgfpathclose%
\pgfusepath{fill}%
\end{pgfscope}%
\begin{pgfscope}%
\pgfpathrectangle{\pgfqpoint{0.697024in}{0.857143in}}{\pgfqpoint{2.627103in}{1.813434in}}%
\pgfusepath{clip}%
\pgfsetbuttcap%
\pgfsetmiterjoin%
\definecolor{currentfill}{rgb}{0.950697,0.616649,0.428624}%
\pgfsetfillcolor{currentfill}%
\pgfsetlinewidth{0.000000pt}%
\definecolor{currentstroke}{rgb}{0.000000,0.000000,0.000000}%
\pgfsetstrokecolor{currentstroke}%
\pgfsetstrokeopacity{0.000000}%
\pgfsetdash{}{0pt}%
\pgfpathmoveto{\pgfqpoint{2.827147in}{2.069975in}}%
\pgfpathlineto{\pgfqpoint{2.836084in}{2.069975in}}%
\pgfpathlineto{\pgfqpoint{2.836084in}{2.101684in}}%
\pgfpathlineto{\pgfqpoint{2.827147in}{2.101684in}}%
\pgfpathlineto{\pgfqpoint{2.827147in}{2.069975in}}%
\pgfpathclose%
\pgfusepath{fill}%
\end{pgfscope}%
\begin{pgfscope}%
\pgfpathrectangle{\pgfqpoint{0.697024in}{0.857143in}}{\pgfqpoint{2.627103in}{1.813434in}}%
\pgfusepath{clip}%
\pgfsetbuttcap%
\pgfsetmiterjoin%
\definecolor{currentfill}{rgb}{0.950697,0.616649,0.428624}%
\pgfsetfillcolor{currentfill}%
\pgfsetlinewidth{0.000000pt}%
\definecolor{currentstroke}{rgb}{0.000000,0.000000,0.000000}%
\pgfsetstrokecolor{currentstroke}%
\pgfsetstrokeopacity{0.000000}%
\pgfsetdash{}{0pt}%
\pgfpathmoveto{\pgfqpoint{2.838318in}{2.133892in}}%
\pgfpathlineto{\pgfqpoint{2.847254in}{2.133892in}}%
\pgfpathlineto{\pgfqpoint{2.847254in}{2.162765in}}%
\pgfpathlineto{\pgfqpoint{2.838318in}{2.162765in}}%
\pgfpathlineto{\pgfqpoint{2.838318in}{2.133892in}}%
\pgfpathclose%
\pgfusepath{fill}%
\end{pgfscope}%
\begin{pgfscope}%
\pgfpathrectangle{\pgfqpoint{0.697024in}{0.857143in}}{\pgfqpoint{2.627103in}{1.813434in}}%
\pgfusepath{clip}%
\pgfsetbuttcap%
\pgfsetmiterjoin%
\definecolor{currentfill}{rgb}{0.950697,0.616649,0.428624}%
\pgfsetfillcolor{currentfill}%
\pgfsetlinewidth{0.000000pt}%
\definecolor{currentstroke}{rgb}{0.000000,0.000000,0.000000}%
\pgfsetstrokecolor{currentstroke}%
\pgfsetstrokeopacity{0.000000}%
\pgfsetdash{}{0pt}%
\pgfpathmoveto{\pgfqpoint{2.849488in}{2.017802in}}%
\pgfpathlineto{\pgfqpoint{2.858425in}{2.017802in}}%
\pgfpathlineto{\pgfqpoint{2.858425in}{2.053999in}}%
\pgfpathlineto{\pgfqpoint{2.849488in}{2.053999in}}%
\pgfpathlineto{\pgfqpoint{2.849488in}{2.017802in}}%
\pgfpathclose%
\pgfusepath{fill}%
\end{pgfscope}%
\begin{pgfscope}%
\pgfpathrectangle{\pgfqpoint{0.697024in}{0.857143in}}{\pgfqpoint{2.627103in}{1.813434in}}%
\pgfusepath{clip}%
\pgfsetbuttcap%
\pgfsetmiterjoin%
\definecolor{currentfill}{rgb}{0.950697,0.616649,0.428624}%
\pgfsetfillcolor{currentfill}%
\pgfsetlinewidth{0.000000pt}%
\definecolor{currentstroke}{rgb}{0.000000,0.000000,0.000000}%
\pgfsetstrokecolor{currentstroke}%
\pgfsetstrokeopacity{0.000000}%
\pgfsetdash{}{0pt}%
\pgfpathmoveto{\pgfqpoint{2.860659in}{2.052455in}}%
\pgfpathlineto{\pgfqpoint{2.869595in}{2.052455in}}%
\pgfpathlineto{\pgfqpoint{2.869595in}{2.084961in}}%
\pgfpathlineto{\pgfqpoint{2.860659in}{2.084961in}}%
\pgfpathlineto{\pgfqpoint{2.860659in}{2.052455in}}%
\pgfpathclose%
\pgfusepath{fill}%
\end{pgfscope}%
\begin{pgfscope}%
\pgfpathrectangle{\pgfqpoint{0.697024in}{0.857143in}}{\pgfqpoint{2.627103in}{1.813434in}}%
\pgfusepath{clip}%
\pgfsetbuttcap%
\pgfsetmiterjoin%
\definecolor{currentfill}{rgb}{0.950697,0.616649,0.428624}%
\pgfsetfillcolor{currentfill}%
\pgfsetlinewidth{0.000000pt}%
\definecolor{currentstroke}{rgb}{0.000000,0.000000,0.000000}%
\pgfsetstrokecolor{currentstroke}%
\pgfsetstrokeopacity{0.000000}%
\pgfsetdash{}{0pt}%
\pgfpathmoveto{\pgfqpoint{2.871830in}{1.993558in}}%
\pgfpathlineto{\pgfqpoint{2.880766in}{1.993558in}}%
\pgfpathlineto{\pgfqpoint{2.880766in}{2.026389in}}%
\pgfpathlineto{\pgfqpoint{2.871830in}{2.026389in}}%
\pgfpathlineto{\pgfqpoint{2.871830in}{1.993558in}}%
\pgfpathclose%
\pgfusepath{fill}%
\end{pgfscope}%
\begin{pgfscope}%
\pgfpathrectangle{\pgfqpoint{0.697024in}{0.857143in}}{\pgfqpoint{2.627103in}{1.813434in}}%
\pgfusepath{clip}%
\pgfsetbuttcap%
\pgfsetmiterjoin%
\definecolor{currentfill}{rgb}{0.950697,0.616649,0.428624}%
\pgfsetfillcolor{currentfill}%
\pgfsetlinewidth{0.000000pt}%
\definecolor{currentstroke}{rgb}{0.000000,0.000000,0.000000}%
\pgfsetstrokecolor{currentstroke}%
\pgfsetstrokeopacity{0.000000}%
\pgfsetdash{}{0pt}%
\pgfpathmoveto{\pgfqpoint{2.883000in}{1.953427in}}%
\pgfpathlineto{\pgfqpoint{2.891937in}{1.953427in}}%
\pgfpathlineto{\pgfqpoint{2.891937in}{1.992868in}}%
\pgfpathlineto{\pgfqpoint{2.883000in}{1.992868in}}%
\pgfpathlineto{\pgfqpoint{2.883000in}{1.953427in}}%
\pgfpathclose%
\pgfusepath{fill}%
\end{pgfscope}%
\begin{pgfscope}%
\pgfpathrectangle{\pgfqpoint{0.697024in}{0.857143in}}{\pgfqpoint{2.627103in}{1.813434in}}%
\pgfusepath{clip}%
\pgfsetbuttcap%
\pgfsetmiterjoin%
\definecolor{currentfill}{rgb}{0.950697,0.616649,0.428624}%
\pgfsetfillcolor{currentfill}%
\pgfsetlinewidth{0.000000pt}%
\definecolor{currentstroke}{rgb}{0.000000,0.000000,0.000000}%
\pgfsetstrokecolor{currentstroke}%
\pgfsetstrokeopacity{0.000000}%
\pgfsetdash{}{0pt}%
\pgfpathmoveto{\pgfqpoint{2.894171in}{2.013678in}}%
\pgfpathlineto{\pgfqpoint{2.903107in}{2.013678in}}%
\pgfpathlineto{\pgfqpoint{2.903107in}{2.048622in}}%
\pgfpathlineto{\pgfqpoint{2.894171in}{2.048622in}}%
\pgfpathlineto{\pgfqpoint{2.894171in}{2.013678in}}%
\pgfpathclose%
\pgfusepath{fill}%
\end{pgfscope}%
\begin{pgfscope}%
\pgfpathrectangle{\pgfqpoint{0.697024in}{0.857143in}}{\pgfqpoint{2.627103in}{1.813434in}}%
\pgfusepath{clip}%
\pgfsetbuttcap%
\pgfsetmiterjoin%
\definecolor{currentfill}{rgb}{0.950697,0.616649,0.428624}%
\pgfsetfillcolor{currentfill}%
\pgfsetlinewidth{0.000000pt}%
\definecolor{currentstroke}{rgb}{0.000000,0.000000,0.000000}%
\pgfsetstrokecolor{currentstroke}%
\pgfsetstrokeopacity{0.000000}%
\pgfsetdash{}{0pt}%
\pgfpathmoveto{\pgfqpoint{2.905341in}{2.033359in}}%
\pgfpathlineto{\pgfqpoint{2.914278in}{2.033359in}}%
\pgfpathlineto{\pgfqpoint{2.914278in}{2.052593in}}%
\pgfpathlineto{\pgfqpoint{2.905341in}{2.052593in}}%
\pgfpathlineto{\pgfqpoint{2.905341in}{2.033359in}}%
\pgfpathclose%
\pgfusepath{fill}%
\end{pgfscope}%
\begin{pgfscope}%
\pgfpathrectangle{\pgfqpoint{0.697024in}{0.857143in}}{\pgfqpoint{2.627103in}{1.813434in}}%
\pgfusepath{clip}%
\pgfsetbuttcap%
\pgfsetmiterjoin%
\definecolor{currentfill}{rgb}{0.950697,0.616649,0.428624}%
\pgfsetfillcolor{currentfill}%
\pgfsetlinewidth{0.000000pt}%
\definecolor{currentstroke}{rgb}{0.000000,0.000000,0.000000}%
\pgfsetstrokecolor{currentstroke}%
\pgfsetstrokeopacity{0.000000}%
\pgfsetdash{}{0pt}%
\pgfpathmoveto{\pgfqpoint{2.916512in}{1.995282in}}%
\pgfpathlineto{\pgfqpoint{2.925448in}{1.995282in}}%
\pgfpathlineto{\pgfqpoint{2.925448in}{2.010638in}}%
\pgfpathlineto{\pgfqpoint{2.916512in}{2.010638in}}%
\pgfpathlineto{\pgfqpoint{2.916512in}{1.995282in}}%
\pgfpathclose%
\pgfusepath{fill}%
\end{pgfscope}%
\begin{pgfscope}%
\pgfpathrectangle{\pgfqpoint{0.697024in}{0.857143in}}{\pgfqpoint{2.627103in}{1.813434in}}%
\pgfusepath{clip}%
\pgfsetbuttcap%
\pgfsetmiterjoin%
\definecolor{currentfill}{rgb}{0.950697,0.616649,0.428624}%
\pgfsetfillcolor{currentfill}%
\pgfsetlinewidth{0.000000pt}%
\definecolor{currentstroke}{rgb}{0.000000,0.000000,0.000000}%
\pgfsetstrokecolor{currentstroke}%
\pgfsetstrokeopacity{0.000000}%
\pgfsetdash{}{0pt}%
\pgfpathmoveto{\pgfqpoint{2.927683in}{1.969980in}}%
\pgfpathlineto{\pgfqpoint{2.936619in}{1.969980in}}%
\pgfpathlineto{\pgfqpoint{2.936619in}{1.996817in}}%
\pgfpathlineto{\pgfqpoint{2.927683in}{1.996817in}}%
\pgfpathlineto{\pgfqpoint{2.927683in}{1.969980in}}%
\pgfpathclose%
\pgfusepath{fill}%
\end{pgfscope}%
\begin{pgfscope}%
\pgfpathrectangle{\pgfqpoint{0.697024in}{0.857143in}}{\pgfqpoint{2.627103in}{1.813434in}}%
\pgfusepath{clip}%
\pgfsetbuttcap%
\pgfsetmiterjoin%
\definecolor{currentfill}{rgb}{0.950697,0.616649,0.428624}%
\pgfsetfillcolor{currentfill}%
\pgfsetlinewidth{0.000000pt}%
\definecolor{currentstroke}{rgb}{0.000000,0.000000,0.000000}%
\pgfsetstrokecolor{currentstroke}%
\pgfsetstrokeopacity{0.000000}%
\pgfsetdash{}{0pt}%
\pgfpathmoveto{\pgfqpoint{2.938853in}{1.993519in}}%
\pgfpathlineto{\pgfqpoint{2.947790in}{1.993519in}}%
\pgfpathlineto{\pgfqpoint{2.947790in}{2.023033in}}%
\pgfpathlineto{\pgfqpoint{2.938853in}{2.023033in}}%
\pgfpathlineto{\pgfqpoint{2.938853in}{1.993519in}}%
\pgfpathclose%
\pgfusepath{fill}%
\end{pgfscope}%
\begin{pgfscope}%
\pgfpathrectangle{\pgfqpoint{0.697024in}{0.857143in}}{\pgfqpoint{2.627103in}{1.813434in}}%
\pgfusepath{clip}%
\pgfsetbuttcap%
\pgfsetmiterjoin%
\definecolor{currentfill}{rgb}{0.950697,0.616649,0.428624}%
\pgfsetfillcolor{currentfill}%
\pgfsetlinewidth{0.000000pt}%
\definecolor{currentstroke}{rgb}{0.000000,0.000000,0.000000}%
\pgfsetstrokecolor{currentstroke}%
\pgfsetstrokeopacity{0.000000}%
\pgfsetdash{}{0pt}%
\pgfpathmoveto{\pgfqpoint{2.950024in}{1.979359in}}%
\pgfpathlineto{\pgfqpoint{2.958960in}{1.979359in}}%
\pgfpathlineto{\pgfqpoint{2.958960in}{2.000730in}}%
\pgfpathlineto{\pgfqpoint{2.950024in}{2.000730in}}%
\pgfpathlineto{\pgfqpoint{2.950024in}{1.979359in}}%
\pgfpathclose%
\pgfusepath{fill}%
\end{pgfscope}%
\begin{pgfscope}%
\pgfpathrectangle{\pgfqpoint{0.697024in}{0.857143in}}{\pgfqpoint{2.627103in}{1.813434in}}%
\pgfusepath{clip}%
\pgfsetbuttcap%
\pgfsetmiterjoin%
\definecolor{currentfill}{rgb}{0.950697,0.616649,0.428624}%
\pgfsetfillcolor{currentfill}%
\pgfsetlinewidth{0.000000pt}%
\definecolor{currentstroke}{rgb}{0.000000,0.000000,0.000000}%
\pgfsetstrokecolor{currentstroke}%
\pgfsetstrokeopacity{0.000000}%
\pgfsetdash{}{0pt}%
\pgfpathmoveto{\pgfqpoint{2.961194in}{1.977774in}}%
\pgfpathlineto{\pgfqpoint{2.970131in}{1.977774in}}%
\pgfpathlineto{\pgfqpoint{2.970131in}{1.996641in}}%
\pgfpathlineto{\pgfqpoint{2.961194in}{1.996641in}}%
\pgfpathlineto{\pgfqpoint{2.961194in}{1.977774in}}%
\pgfpathclose%
\pgfusepath{fill}%
\end{pgfscope}%
\begin{pgfscope}%
\pgfpathrectangle{\pgfqpoint{0.697024in}{0.857143in}}{\pgfqpoint{2.627103in}{1.813434in}}%
\pgfusepath{clip}%
\pgfsetbuttcap%
\pgfsetmiterjoin%
\definecolor{currentfill}{rgb}{0.950697,0.616649,0.428624}%
\pgfsetfillcolor{currentfill}%
\pgfsetlinewidth{0.000000pt}%
\definecolor{currentstroke}{rgb}{0.000000,0.000000,0.000000}%
\pgfsetstrokecolor{currentstroke}%
\pgfsetstrokeopacity{0.000000}%
\pgfsetdash{}{0pt}%
\pgfpathmoveto{\pgfqpoint{2.972365in}{2.020457in}}%
\pgfpathlineto{\pgfqpoint{2.981301in}{2.020457in}}%
\pgfpathlineto{\pgfqpoint{2.981301in}{2.035761in}}%
\pgfpathlineto{\pgfqpoint{2.972365in}{2.035761in}}%
\pgfpathlineto{\pgfqpoint{2.972365in}{2.020457in}}%
\pgfpathclose%
\pgfusepath{fill}%
\end{pgfscope}%
\begin{pgfscope}%
\pgfpathrectangle{\pgfqpoint{0.697024in}{0.857143in}}{\pgfqpoint{2.627103in}{1.813434in}}%
\pgfusepath{clip}%
\pgfsetbuttcap%
\pgfsetmiterjoin%
\definecolor{currentfill}{rgb}{0.950697,0.616649,0.428624}%
\pgfsetfillcolor{currentfill}%
\pgfsetlinewidth{0.000000pt}%
\definecolor{currentstroke}{rgb}{0.000000,0.000000,0.000000}%
\pgfsetstrokecolor{currentstroke}%
\pgfsetstrokeopacity{0.000000}%
\pgfsetdash{}{0pt}%
\pgfpathmoveto{\pgfqpoint{2.983536in}{1.210807in}}%
\pgfpathlineto{\pgfqpoint{2.992472in}{1.210807in}}%
\pgfpathlineto{\pgfqpoint{2.992472in}{1.210778in}}%
\pgfpathlineto{\pgfqpoint{2.983536in}{1.210778in}}%
\pgfpathlineto{\pgfqpoint{2.983536in}{1.210807in}}%
\pgfpathclose%
\pgfusepath{fill}%
\end{pgfscope}%
\begin{pgfscope}%
\pgfpathrectangle{\pgfqpoint{0.697024in}{0.857143in}}{\pgfqpoint{2.627103in}{1.813434in}}%
\pgfusepath{clip}%
\pgfsetbuttcap%
\pgfsetmiterjoin%
\definecolor{currentfill}{rgb}{0.950697,0.616649,0.428624}%
\pgfsetfillcolor{currentfill}%
\pgfsetlinewidth{0.000000pt}%
\definecolor{currentstroke}{rgb}{0.000000,0.000000,0.000000}%
\pgfsetstrokecolor{currentstroke}%
\pgfsetstrokeopacity{0.000000}%
\pgfsetdash{}{0pt}%
\pgfpathmoveto{\pgfqpoint{2.994706in}{1.283266in}}%
\pgfpathlineto{\pgfqpoint{3.003643in}{1.283266in}}%
\pgfpathlineto{\pgfqpoint{3.003643in}{1.282576in}}%
\pgfpathlineto{\pgfqpoint{2.994706in}{1.282576in}}%
\pgfpathlineto{\pgfqpoint{2.994706in}{1.283266in}}%
\pgfpathclose%
\pgfusepath{fill}%
\end{pgfscope}%
\begin{pgfscope}%
\pgfpathrectangle{\pgfqpoint{0.697024in}{0.857143in}}{\pgfqpoint{2.627103in}{1.813434in}}%
\pgfusepath{clip}%
\pgfsetbuttcap%
\pgfsetmiterjoin%
\definecolor{currentfill}{rgb}{0.950697,0.616649,0.428624}%
\pgfsetfillcolor{currentfill}%
\pgfsetlinewidth{0.000000pt}%
\definecolor{currentstroke}{rgb}{0.000000,0.000000,0.000000}%
\pgfsetstrokecolor{currentstroke}%
\pgfsetstrokeopacity{0.000000}%
\pgfsetdash{}{0pt}%
\pgfpathmoveto{\pgfqpoint{3.005877in}{1.271191in}}%
\pgfpathlineto{\pgfqpoint{3.014813in}{1.271191in}}%
\pgfpathlineto{\pgfqpoint{3.014813in}{1.266622in}}%
\pgfpathlineto{\pgfqpoint{3.005877in}{1.266622in}}%
\pgfpathlineto{\pgfqpoint{3.005877in}{1.271191in}}%
\pgfpathclose%
\pgfusepath{fill}%
\end{pgfscope}%
\begin{pgfscope}%
\pgfpathrectangle{\pgfqpoint{0.697024in}{0.857143in}}{\pgfqpoint{2.627103in}{1.813434in}}%
\pgfusepath{clip}%
\pgfsetbuttcap%
\pgfsetmiterjoin%
\definecolor{currentfill}{rgb}{0.950697,0.616649,0.428624}%
\pgfsetfillcolor{currentfill}%
\pgfsetlinewidth{0.000000pt}%
\definecolor{currentstroke}{rgb}{0.000000,0.000000,0.000000}%
\pgfsetstrokecolor{currentstroke}%
\pgfsetstrokeopacity{0.000000}%
\pgfsetdash{}{0pt}%
\pgfpathmoveto{\pgfqpoint{3.017047in}{2.125699in}}%
\pgfpathlineto{\pgfqpoint{3.025984in}{2.125699in}}%
\pgfpathlineto{\pgfqpoint{3.025984in}{2.128604in}}%
\pgfpathlineto{\pgfqpoint{3.017047in}{2.128604in}}%
\pgfpathlineto{\pgfqpoint{3.017047in}{2.125699in}}%
\pgfpathclose%
\pgfusepath{fill}%
\end{pgfscope}%
\begin{pgfscope}%
\pgfpathrectangle{\pgfqpoint{0.697024in}{0.857143in}}{\pgfqpoint{2.627103in}{1.813434in}}%
\pgfusepath{clip}%
\pgfsetbuttcap%
\pgfsetmiterjoin%
\definecolor{currentfill}{rgb}{0.950697,0.616649,0.428624}%
\pgfsetfillcolor{currentfill}%
\pgfsetlinewidth{0.000000pt}%
\definecolor{currentstroke}{rgb}{0.000000,0.000000,0.000000}%
\pgfsetstrokecolor{currentstroke}%
\pgfsetstrokeopacity{0.000000}%
\pgfsetdash{}{0pt}%
\pgfpathmoveto{\pgfqpoint{3.028218in}{1.101577in}}%
\pgfpathlineto{\pgfqpoint{3.037155in}{1.101577in}}%
\pgfpathlineto{\pgfqpoint{3.037155in}{1.094897in}}%
\pgfpathlineto{\pgfqpoint{3.028218in}{1.094897in}}%
\pgfpathlineto{\pgfqpoint{3.028218in}{1.101577in}}%
\pgfpathclose%
\pgfusepath{fill}%
\end{pgfscope}%
\begin{pgfscope}%
\pgfpathrectangle{\pgfqpoint{0.697024in}{0.857143in}}{\pgfqpoint{2.627103in}{1.813434in}}%
\pgfusepath{clip}%
\pgfsetbuttcap%
\pgfsetmiterjoin%
\definecolor{currentfill}{rgb}{0.950697,0.616649,0.428624}%
\pgfsetfillcolor{currentfill}%
\pgfsetlinewidth{0.000000pt}%
\definecolor{currentstroke}{rgb}{0.000000,0.000000,0.000000}%
\pgfsetstrokecolor{currentstroke}%
\pgfsetstrokeopacity{0.000000}%
\pgfsetdash{}{0pt}%
\pgfpathmoveto{\pgfqpoint{3.039389in}{1.173527in}}%
\pgfpathlineto{\pgfqpoint{3.048325in}{1.173527in}}%
\pgfpathlineto{\pgfqpoint{3.048325in}{1.163943in}}%
\pgfpathlineto{\pgfqpoint{3.039389in}{1.163943in}}%
\pgfpathlineto{\pgfqpoint{3.039389in}{1.173527in}}%
\pgfpathclose%
\pgfusepath{fill}%
\end{pgfscope}%
\begin{pgfscope}%
\pgfpathrectangle{\pgfqpoint{0.697024in}{0.857143in}}{\pgfqpoint{2.627103in}{1.813434in}}%
\pgfusepath{clip}%
\pgfsetbuttcap%
\pgfsetmiterjoin%
\definecolor{currentfill}{rgb}{0.950697,0.616649,0.428624}%
\pgfsetfillcolor{currentfill}%
\pgfsetlinewidth{0.000000pt}%
\definecolor{currentstroke}{rgb}{0.000000,0.000000,0.000000}%
\pgfsetstrokecolor{currentstroke}%
\pgfsetstrokeopacity{0.000000}%
\pgfsetdash{}{0pt}%
\pgfpathmoveto{\pgfqpoint{3.050559in}{1.148843in}}%
\pgfpathlineto{\pgfqpoint{3.059496in}{1.148843in}}%
\pgfpathlineto{\pgfqpoint{3.059496in}{1.134554in}}%
\pgfpathlineto{\pgfqpoint{3.050559in}{1.134554in}}%
\pgfpathlineto{\pgfqpoint{3.050559in}{1.148843in}}%
\pgfpathclose%
\pgfusepath{fill}%
\end{pgfscope}%
\begin{pgfscope}%
\pgfpathrectangle{\pgfqpoint{0.697024in}{0.857143in}}{\pgfqpoint{2.627103in}{1.813434in}}%
\pgfusepath{clip}%
\pgfsetbuttcap%
\pgfsetmiterjoin%
\definecolor{currentfill}{rgb}{0.950697,0.616649,0.428624}%
\pgfsetfillcolor{currentfill}%
\pgfsetlinewidth{0.000000pt}%
\definecolor{currentstroke}{rgb}{0.000000,0.000000,0.000000}%
\pgfsetstrokecolor{currentstroke}%
\pgfsetstrokeopacity{0.000000}%
\pgfsetdash{}{0pt}%
\pgfpathmoveto{\pgfqpoint{3.061730in}{1.143266in}}%
\pgfpathlineto{\pgfqpoint{3.070666in}{1.143266in}}%
\pgfpathlineto{\pgfqpoint{3.070666in}{1.134515in}}%
\pgfpathlineto{\pgfqpoint{3.061730in}{1.134515in}}%
\pgfpathlineto{\pgfqpoint{3.061730in}{1.143266in}}%
\pgfpathclose%
\pgfusepath{fill}%
\end{pgfscope}%
\begin{pgfscope}%
\pgfpathrectangle{\pgfqpoint{0.697024in}{0.857143in}}{\pgfqpoint{2.627103in}{1.813434in}}%
\pgfusepath{clip}%
\pgfsetbuttcap%
\pgfsetmiterjoin%
\definecolor{currentfill}{rgb}{0.950697,0.616649,0.428624}%
\pgfsetfillcolor{currentfill}%
\pgfsetlinewidth{0.000000pt}%
\definecolor{currentstroke}{rgb}{0.000000,0.000000,0.000000}%
\pgfsetstrokecolor{currentstroke}%
\pgfsetstrokeopacity{0.000000}%
\pgfsetdash{}{0pt}%
\pgfpathmoveto{\pgfqpoint{3.072900in}{1.186931in}}%
\pgfpathlineto{\pgfqpoint{3.081837in}{1.186931in}}%
\pgfpathlineto{\pgfqpoint{3.081837in}{1.171344in}}%
\pgfpathlineto{\pgfqpoint{3.072900in}{1.171344in}}%
\pgfpathlineto{\pgfqpoint{3.072900in}{1.186931in}}%
\pgfpathclose%
\pgfusepath{fill}%
\end{pgfscope}%
\begin{pgfscope}%
\pgfpathrectangle{\pgfqpoint{0.697024in}{0.857143in}}{\pgfqpoint{2.627103in}{1.813434in}}%
\pgfusepath{clip}%
\pgfsetbuttcap%
\pgfsetmiterjoin%
\definecolor{currentfill}{rgb}{0.950697,0.616649,0.428624}%
\pgfsetfillcolor{currentfill}%
\pgfsetlinewidth{0.000000pt}%
\definecolor{currentstroke}{rgb}{0.000000,0.000000,0.000000}%
\pgfsetstrokecolor{currentstroke}%
\pgfsetstrokeopacity{0.000000}%
\pgfsetdash{}{0pt}%
\pgfpathmoveto{\pgfqpoint{3.084071in}{1.128698in}}%
\pgfpathlineto{\pgfqpoint{3.093008in}{1.128698in}}%
\pgfpathlineto{\pgfqpoint{3.093008in}{1.104387in}}%
\pgfpathlineto{\pgfqpoint{3.084071in}{1.104387in}}%
\pgfpathlineto{\pgfqpoint{3.084071in}{1.128698in}}%
\pgfpathclose%
\pgfusepath{fill}%
\end{pgfscope}%
\begin{pgfscope}%
\pgfpathrectangle{\pgfqpoint{0.697024in}{0.857143in}}{\pgfqpoint{2.627103in}{1.813434in}}%
\pgfusepath{clip}%
\pgfsetbuttcap%
\pgfsetmiterjoin%
\definecolor{currentfill}{rgb}{0.950697,0.616649,0.428624}%
\pgfsetfillcolor{currentfill}%
\pgfsetlinewidth{0.000000pt}%
\definecolor{currentstroke}{rgb}{0.000000,0.000000,0.000000}%
\pgfsetstrokecolor{currentstroke}%
\pgfsetstrokeopacity{0.000000}%
\pgfsetdash{}{0pt}%
\pgfpathmoveto{\pgfqpoint{3.095242in}{1.132166in}}%
\pgfpathlineto{\pgfqpoint{3.104178in}{1.132166in}}%
\pgfpathlineto{\pgfqpoint{3.104178in}{1.114742in}}%
\pgfpathlineto{\pgfqpoint{3.095242in}{1.114742in}}%
\pgfpathlineto{\pgfqpoint{3.095242in}{1.132166in}}%
\pgfpathclose%
\pgfusepath{fill}%
\end{pgfscope}%
\begin{pgfscope}%
\pgfpathrectangle{\pgfqpoint{0.697024in}{0.857143in}}{\pgfqpoint{2.627103in}{1.813434in}}%
\pgfusepath{clip}%
\pgfsetbuttcap%
\pgfsetmiterjoin%
\definecolor{currentfill}{rgb}{0.950697,0.616649,0.428624}%
\pgfsetfillcolor{currentfill}%
\pgfsetlinewidth{0.000000pt}%
\definecolor{currentstroke}{rgb}{0.000000,0.000000,0.000000}%
\pgfsetstrokecolor{currentstroke}%
\pgfsetstrokeopacity{0.000000}%
\pgfsetdash{}{0pt}%
\pgfpathmoveto{\pgfqpoint{3.106412in}{1.180329in}}%
\pgfpathlineto{\pgfqpoint{3.115349in}{1.180329in}}%
\pgfpathlineto{\pgfqpoint{3.115349in}{1.167139in}}%
\pgfpathlineto{\pgfqpoint{3.106412in}{1.167139in}}%
\pgfpathlineto{\pgfqpoint{3.106412in}{1.180329in}}%
\pgfpathclose%
\pgfusepath{fill}%
\end{pgfscope}%
\begin{pgfscope}%
\pgfpathrectangle{\pgfqpoint{0.697024in}{0.857143in}}{\pgfqpoint{2.627103in}{1.813434in}}%
\pgfusepath{clip}%
\pgfsetbuttcap%
\pgfsetmiterjoin%
\definecolor{currentfill}{rgb}{0.950697,0.616649,0.428624}%
\pgfsetfillcolor{currentfill}%
\pgfsetlinewidth{0.000000pt}%
\definecolor{currentstroke}{rgb}{0.000000,0.000000,0.000000}%
\pgfsetstrokecolor{currentstroke}%
\pgfsetstrokeopacity{0.000000}%
\pgfsetdash{}{0pt}%
\pgfpathmoveto{\pgfqpoint{3.117583in}{1.162073in}}%
\pgfpathlineto{\pgfqpoint{3.126519in}{1.162073in}}%
\pgfpathlineto{\pgfqpoint{3.126519in}{1.154369in}}%
\pgfpathlineto{\pgfqpoint{3.117583in}{1.154369in}}%
\pgfpathlineto{\pgfqpoint{3.117583in}{1.162073in}}%
\pgfpathclose%
\pgfusepath{fill}%
\end{pgfscope}%
\begin{pgfscope}%
\pgfpathrectangle{\pgfqpoint{0.697024in}{0.857143in}}{\pgfqpoint{2.627103in}{1.813434in}}%
\pgfusepath{clip}%
\pgfsetbuttcap%
\pgfsetmiterjoin%
\definecolor{currentfill}{rgb}{0.950697,0.616649,0.428624}%
\pgfsetfillcolor{currentfill}%
\pgfsetlinewidth{0.000000pt}%
\definecolor{currentstroke}{rgb}{0.000000,0.000000,0.000000}%
\pgfsetstrokecolor{currentstroke}%
\pgfsetstrokeopacity{0.000000}%
\pgfsetdash{}{0pt}%
\pgfpathmoveto{\pgfqpoint{3.128753in}{1.177771in}}%
\pgfpathlineto{\pgfqpoint{3.137690in}{1.177771in}}%
\pgfpathlineto{\pgfqpoint{3.137690in}{1.163181in}}%
\pgfpathlineto{\pgfqpoint{3.128753in}{1.163181in}}%
\pgfpathlineto{\pgfqpoint{3.128753in}{1.177771in}}%
\pgfpathclose%
\pgfusepath{fill}%
\end{pgfscope}%
\begin{pgfscope}%
\pgfpathrectangle{\pgfqpoint{0.697024in}{0.857143in}}{\pgfqpoint{2.627103in}{1.813434in}}%
\pgfusepath{clip}%
\pgfsetbuttcap%
\pgfsetmiterjoin%
\definecolor{currentfill}{rgb}{0.950697,0.616649,0.428624}%
\pgfsetfillcolor{currentfill}%
\pgfsetlinewidth{0.000000pt}%
\definecolor{currentstroke}{rgb}{0.000000,0.000000,0.000000}%
\pgfsetstrokecolor{currentstroke}%
\pgfsetstrokeopacity{0.000000}%
\pgfsetdash{}{0pt}%
\pgfpathmoveto{\pgfqpoint{3.139924in}{1.131882in}}%
\pgfpathlineto{\pgfqpoint{3.148861in}{1.131882in}}%
\pgfpathlineto{\pgfqpoint{3.148861in}{1.111648in}}%
\pgfpathlineto{\pgfqpoint{3.139924in}{1.111648in}}%
\pgfpathlineto{\pgfqpoint{3.139924in}{1.131882in}}%
\pgfpathclose%
\pgfusepath{fill}%
\end{pgfscope}%
\begin{pgfscope}%
\pgfpathrectangle{\pgfqpoint{0.697024in}{0.857143in}}{\pgfqpoint{2.627103in}{1.813434in}}%
\pgfusepath{clip}%
\pgfsetbuttcap%
\pgfsetmiterjoin%
\definecolor{currentfill}{rgb}{0.950697,0.616649,0.428624}%
\pgfsetfillcolor{currentfill}%
\pgfsetlinewidth{0.000000pt}%
\definecolor{currentstroke}{rgb}{0.000000,0.000000,0.000000}%
\pgfsetstrokecolor{currentstroke}%
\pgfsetstrokeopacity{0.000000}%
\pgfsetdash{}{0pt}%
\pgfpathmoveto{\pgfqpoint{3.151095in}{1.072988in}}%
\pgfpathlineto{\pgfqpoint{3.160031in}{1.072988in}}%
\pgfpathlineto{\pgfqpoint{3.160031in}{1.048772in}}%
\pgfpathlineto{\pgfqpoint{3.151095in}{1.048772in}}%
\pgfpathlineto{\pgfqpoint{3.151095in}{1.072988in}}%
\pgfpathclose%
\pgfusepath{fill}%
\end{pgfscope}%
\begin{pgfscope}%
\pgfpathrectangle{\pgfqpoint{0.697024in}{0.857143in}}{\pgfqpoint{2.627103in}{1.813434in}}%
\pgfusepath{clip}%
\pgfsetbuttcap%
\pgfsetmiterjoin%
\definecolor{currentfill}{rgb}{0.950697,0.616649,0.428624}%
\pgfsetfillcolor{currentfill}%
\pgfsetlinewidth{0.000000pt}%
\definecolor{currentstroke}{rgb}{0.000000,0.000000,0.000000}%
\pgfsetstrokecolor{currentstroke}%
\pgfsetstrokeopacity{0.000000}%
\pgfsetdash{}{0pt}%
\pgfpathmoveto{\pgfqpoint{3.162265in}{0.982157in}}%
\pgfpathlineto{\pgfqpoint{3.171202in}{0.982157in}}%
\pgfpathlineto{\pgfqpoint{3.171202in}{0.939572in}}%
\pgfpathlineto{\pgfqpoint{3.162265in}{0.939572in}}%
\pgfpathlineto{\pgfqpoint{3.162265in}{0.982157in}}%
\pgfpathclose%
\pgfusepath{fill}%
\end{pgfscope}%
\begin{pgfscope}%
\pgfpathrectangle{\pgfqpoint{0.697024in}{0.857143in}}{\pgfqpoint{2.627103in}{1.813434in}}%
\pgfusepath{clip}%
\pgfsetbuttcap%
\pgfsetmiterjoin%
\definecolor{currentfill}{rgb}{0.950697,0.616649,0.428624}%
\pgfsetfillcolor{currentfill}%
\pgfsetlinewidth{0.000000pt}%
\definecolor{currentstroke}{rgb}{0.000000,0.000000,0.000000}%
\pgfsetstrokecolor{currentstroke}%
\pgfsetstrokeopacity{0.000000}%
\pgfsetdash{}{0pt}%
\pgfpathmoveto{\pgfqpoint{3.173436in}{1.052972in}}%
\pgfpathlineto{\pgfqpoint{3.182372in}{1.052972in}}%
\pgfpathlineto{\pgfqpoint{3.182372in}{1.001768in}}%
\pgfpathlineto{\pgfqpoint{3.173436in}{1.001768in}}%
\pgfpathlineto{\pgfqpoint{3.173436in}{1.052972in}}%
\pgfpathclose%
\pgfusepath{fill}%
\end{pgfscope}%
\begin{pgfscope}%
\pgfpathrectangle{\pgfqpoint{0.697024in}{0.857143in}}{\pgfqpoint{2.627103in}{1.813434in}}%
\pgfusepath{clip}%
\pgfsetbuttcap%
\pgfsetmiterjoin%
\definecolor{currentfill}{rgb}{0.950697,0.616649,0.428624}%
\pgfsetfillcolor{currentfill}%
\pgfsetlinewidth{0.000000pt}%
\definecolor{currentstroke}{rgb}{0.000000,0.000000,0.000000}%
\pgfsetstrokecolor{currentstroke}%
\pgfsetstrokeopacity{0.000000}%
\pgfsetdash{}{0pt}%
\pgfpathmoveto{\pgfqpoint{3.184607in}{0.997746in}}%
\pgfpathlineto{\pgfqpoint{3.193543in}{0.997746in}}%
\pgfpathlineto{\pgfqpoint{3.193543in}{0.951870in}}%
\pgfpathlineto{\pgfqpoint{3.184607in}{0.951870in}}%
\pgfpathlineto{\pgfqpoint{3.184607in}{0.997746in}}%
\pgfpathclose%
\pgfusepath{fill}%
\end{pgfscope}%
\begin{pgfscope}%
\pgfpathrectangle{\pgfqpoint{0.697024in}{0.857143in}}{\pgfqpoint{2.627103in}{1.813434in}}%
\pgfusepath{clip}%
\pgfsetbuttcap%
\pgfsetmiterjoin%
\definecolor{currentfill}{rgb}{0.950697,0.616649,0.428624}%
\pgfsetfillcolor{currentfill}%
\pgfsetlinewidth{0.000000pt}%
\definecolor{currentstroke}{rgb}{0.000000,0.000000,0.000000}%
\pgfsetstrokecolor{currentstroke}%
\pgfsetstrokeopacity{0.000000}%
\pgfsetdash{}{0pt}%
\pgfpathmoveto{\pgfqpoint{3.195777in}{0.995091in}}%
\pgfpathlineto{\pgfqpoint{3.204714in}{0.995091in}}%
\pgfpathlineto{\pgfqpoint{3.204714in}{0.948221in}}%
\pgfpathlineto{\pgfqpoint{3.195777in}{0.948221in}}%
\pgfpathlineto{\pgfqpoint{3.195777in}{0.995091in}}%
\pgfpathclose%
\pgfusepath{fill}%
\end{pgfscope}%
\begin{pgfscope}%
\pgfpathrectangle{\pgfqpoint{0.697024in}{0.857143in}}{\pgfqpoint{2.627103in}{1.813434in}}%
\pgfusepath{clip}%
\pgfsetbuttcap%
\pgfsetmiterjoin%
\definecolor{currentfill}{rgb}{0.992771,0.707689,0.712380}%
\pgfsetfillcolor{currentfill}%
\pgfsetlinewidth{0.000000pt}%
\definecolor{currentstroke}{rgb}{0.000000,0.000000,0.000000}%
\pgfsetstrokecolor{currentstroke}%
\pgfsetstrokeopacity{0.000000}%
\pgfsetdash{}{0pt}%
\pgfpathmoveto{\pgfqpoint{0.816438in}{1.928362in}}%
\pgfpathlineto{\pgfqpoint{0.825375in}{1.928362in}}%
\pgfpathlineto{\pgfqpoint{0.825375in}{1.947284in}}%
\pgfpathlineto{\pgfqpoint{0.816438in}{1.947284in}}%
\pgfpathlineto{\pgfqpoint{0.816438in}{1.928362in}}%
\pgfpathclose%
\pgfusepath{fill}%
\end{pgfscope}%
\begin{pgfscope}%
\pgfpathrectangle{\pgfqpoint{0.697024in}{0.857143in}}{\pgfqpoint{2.627103in}{1.813434in}}%
\pgfusepath{clip}%
\pgfsetbuttcap%
\pgfsetmiterjoin%
\definecolor{currentfill}{rgb}{0.992771,0.707689,0.712380}%
\pgfsetfillcolor{currentfill}%
\pgfsetlinewidth{0.000000pt}%
\definecolor{currentstroke}{rgb}{0.000000,0.000000,0.000000}%
\pgfsetstrokecolor{currentstroke}%
\pgfsetstrokeopacity{0.000000}%
\pgfsetdash{}{0pt}%
\pgfpathmoveto{\pgfqpoint{0.827609in}{1.934328in}}%
\pgfpathlineto{\pgfqpoint{0.836545in}{1.934328in}}%
\pgfpathlineto{\pgfqpoint{0.836545in}{1.952451in}}%
\pgfpathlineto{\pgfqpoint{0.827609in}{1.952451in}}%
\pgfpathlineto{\pgfqpoint{0.827609in}{1.934328in}}%
\pgfpathclose%
\pgfusepath{fill}%
\end{pgfscope}%
\begin{pgfscope}%
\pgfpathrectangle{\pgfqpoint{0.697024in}{0.857143in}}{\pgfqpoint{2.627103in}{1.813434in}}%
\pgfusepath{clip}%
\pgfsetbuttcap%
\pgfsetmiterjoin%
\definecolor{currentfill}{rgb}{0.992771,0.707689,0.712380}%
\pgfsetfillcolor{currentfill}%
\pgfsetlinewidth{0.000000pt}%
\definecolor{currentstroke}{rgb}{0.000000,0.000000,0.000000}%
\pgfsetstrokecolor{currentstroke}%
\pgfsetstrokeopacity{0.000000}%
\pgfsetdash{}{0pt}%
\pgfpathmoveto{\pgfqpoint{0.838779in}{2.053228in}}%
\pgfpathlineto{\pgfqpoint{0.847716in}{2.053228in}}%
\pgfpathlineto{\pgfqpoint{0.847716in}{2.065675in}}%
\pgfpathlineto{\pgfqpoint{0.838779in}{2.065675in}}%
\pgfpathlineto{\pgfqpoint{0.838779in}{2.053228in}}%
\pgfpathclose%
\pgfusepath{fill}%
\end{pgfscope}%
\begin{pgfscope}%
\pgfpathrectangle{\pgfqpoint{0.697024in}{0.857143in}}{\pgfqpoint{2.627103in}{1.813434in}}%
\pgfusepath{clip}%
\pgfsetbuttcap%
\pgfsetmiterjoin%
\definecolor{currentfill}{rgb}{0.992771,0.707689,0.712380}%
\pgfsetfillcolor{currentfill}%
\pgfsetlinewidth{0.000000pt}%
\definecolor{currentstroke}{rgb}{0.000000,0.000000,0.000000}%
\pgfsetstrokecolor{currentstroke}%
\pgfsetstrokeopacity{0.000000}%
\pgfsetdash{}{0pt}%
\pgfpathmoveto{\pgfqpoint{0.849950in}{2.008949in}}%
\pgfpathlineto{\pgfqpoint{0.858886in}{2.008949in}}%
\pgfpathlineto{\pgfqpoint{0.858886in}{2.024839in}}%
\pgfpathlineto{\pgfqpoint{0.849950in}{2.024839in}}%
\pgfpathlineto{\pgfqpoint{0.849950in}{2.008949in}}%
\pgfpathclose%
\pgfusepath{fill}%
\end{pgfscope}%
\begin{pgfscope}%
\pgfpathrectangle{\pgfqpoint{0.697024in}{0.857143in}}{\pgfqpoint{2.627103in}{1.813434in}}%
\pgfusepath{clip}%
\pgfsetbuttcap%
\pgfsetmiterjoin%
\definecolor{currentfill}{rgb}{0.992771,0.707689,0.712380}%
\pgfsetfillcolor{currentfill}%
\pgfsetlinewidth{0.000000pt}%
\definecolor{currentstroke}{rgb}{0.000000,0.000000,0.000000}%
\pgfsetstrokecolor{currentstroke}%
\pgfsetstrokeopacity{0.000000}%
\pgfsetdash{}{0pt}%
\pgfpathmoveto{\pgfqpoint{0.861121in}{1.970601in}}%
\pgfpathlineto{\pgfqpoint{0.870057in}{1.970601in}}%
\pgfpathlineto{\pgfqpoint{0.870057in}{2.047178in}}%
\pgfpathlineto{\pgfqpoint{0.861121in}{2.047178in}}%
\pgfpathlineto{\pgfqpoint{0.861121in}{1.970601in}}%
\pgfpathclose%
\pgfusepath{fill}%
\end{pgfscope}%
\begin{pgfscope}%
\pgfpathrectangle{\pgfqpoint{0.697024in}{0.857143in}}{\pgfqpoint{2.627103in}{1.813434in}}%
\pgfusepath{clip}%
\pgfsetbuttcap%
\pgfsetmiterjoin%
\definecolor{currentfill}{rgb}{0.992771,0.707689,0.712380}%
\pgfsetfillcolor{currentfill}%
\pgfsetlinewidth{0.000000pt}%
\definecolor{currentstroke}{rgb}{0.000000,0.000000,0.000000}%
\pgfsetstrokecolor{currentstroke}%
\pgfsetstrokeopacity{0.000000}%
\pgfsetdash{}{0pt}%
\pgfpathmoveto{\pgfqpoint{0.872291in}{1.947040in}}%
\pgfpathlineto{\pgfqpoint{0.881228in}{1.947040in}}%
\pgfpathlineto{\pgfqpoint{0.881228in}{1.992174in}}%
\pgfpathlineto{\pgfqpoint{0.872291in}{1.992174in}}%
\pgfpathlineto{\pgfqpoint{0.872291in}{1.947040in}}%
\pgfpathclose%
\pgfusepath{fill}%
\end{pgfscope}%
\begin{pgfscope}%
\pgfpathrectangle{\pgfqpoint{0.697024in}{0.857143in}}{\pgfqpoint{2.627103in}{1.813434in}}%
\pgfusepath{clip}%
\pgfsetbuttcap%
\pgfsetmiterjoin%
\definecolor{currentfill}{rgb}{0.992771,0.707689,0.712380}%
\pgfsetfillcolor{currentfill}%
\pgfsetlinewidth{0.000000pt}%
\definecolor{currentstroke}{rgb}{0.000000,0.000000,0.000000}%
\pgfsetstrokecolor{currentstroke}%
\pgfsetstrokeopacity{0.000000}%
\pgfsetdash{}{0pt}%
\pgfpathmoveto{\pgfqpoint{0.883462in}{2.026045in}}%
\pgfpathlineto{\pgfqpoint{0.892398in}{2.026045in}}%
\pgfpathlineto{\pgfqpoint{0.892398in}{2.061599in}}%
\pgfpathlineto{\pgfqpoint{0.883462in}{2.061599in}}%
\pgfpathlineto{\pgfqpoint{0.883462in}{2.026045in}}%
\pgfpathclose%
\pgfusepath{fill}%
\end{pgfscope}%
\begin{pgfscope}%
\pgfpathrectangle{\pgfqpoint{0.697024in}{0.857143in}}{\pgfqpoint{2.627103in}{1.813434in}}%
\pgfusepath{clip}%
\pgfsetbuttcap%
\pgfsetmiterjoin%
\definecolor{currentfill}{rgb}{0.992771,0.707689,0.712380}%
\pgfsetfillcolor{currentfill}%
\pgfsetlinewidth{0.000000pt}%
\definecolor{currentstroke}{rgb}{0.000000,0.000000,0.000000}%
\pgfsetstrokecolor{currentstroke}%
\pgfsetstrokeopacity{0.000000}%
\pgfsetdash{}{0pt}%
\pgfpathmoveto{\pgfqpoint{0.894632in}{2.077324in}}%
\pgfpathlineto{\pgfqpoint{0.903569in}{2.077324in}}%
\pgfpathlineto{\pgfqpoint{0.903569in}{2.082848in}}%
\pgfpathlineto{\pgfqpoint{0.894632in}{2.082848in}}%
\pgfpathlineto{\pgfqpoint{0.894632in}{2.077324in}}%
\pgfpathclose%
\pgfusepath{fill}%
\end{pgfscope}%
\begin{pgfscope}%
\pgfpathrectangle{\pgfqpoint{0.697024in}{0.857143in}}{\pgfqpoint{2.627103in}{1.813434in}}%
\pgfusepath{clip}%
\pgfsetbuttcap%
\pgfsetmiterjoin%
\definecolor{currentfill}{rgb}{0.992771,0.707689,0.712380}%
\pgfsetfillcolor{currentfill}%
\pgfsetlinewidth{0.000000pt}%
\definecolor{currentstroke}{rgb}{0.000000,0.000000,0.000000}%
\pgfsetstrokecolor{currentstroke}%
\pgfsetstrokeopacity{0.000000}%
\pgfsetdash{}{0pt}%
\pgfpathmoveto{\pgfqpoint{0.905803in}{2.060983in}}%
\pgfpathlineto{\pgfqpoint{0.914739in}{2.060983in}}%
\pgfpathlineto{\pgfqpoint{0.914739in}{2.113751in}}%
\pgfpathlineto{\pgfqpoint{0.905803in}{2.113751in}}%
\pgfpathlineto{\pgfqpoint{0.905803in}{2.060983in}}%
\pgfpathclose%
\pgfusepath{fill}%
\end{pgfscope}%
\begin{pgfscope}%
\pgfpathrectangle{\pgfqpoint{0.697024in}{0.857143in}}{\pgfqpoint{2.627103in}{1.813434in}}%
\pgfusepath{clip}%
\pgfsetbuttcap%
\pgfsetmiterjoin%
\definecolor{currentfill}{rgb}{0.992771,0.707689,0.712380}%
\pgfsetfillcolor{currentfill}%
\pgfsetlinewidth{0.000000pt}%
\definecolor{currentstroke}{rgb}{0.000000,0.000000,0.000000}%
\pgfsetstrokecolor{currentstroke}%
\pgfsetstrokeopacity{0.000000}%
\pgfsetdash{}{0pt}%
\pgfpathmoveto{\pgfqpoint{0.916974in}{2.048880in}}%
\pgfpathlineto{\pgfqpoint{0.925910in}{2.048880in}}%
\pgfpathlineto{\pgfqpoint{0.925910in}{2.083039in}}%
\pgfpathlineto{\pgfqpoint{0.916974in}{2.083039in}}%
\pgfpathlineto{\pgfqpoint{0.916974in}{2.048880in}}%
\pgfpathclose%
\pgfusepath{fill}%
\end{pgfscope}%
\begin{pgfscope}%
\pgfpathrectangle{\pgfqpoint{0.697024in}{0.857143in}}{\pgfqpoint{2.627103in}{1.813434in}}%
\pgfusepath{clip}%
\pgfsetbuttcap%
\pgfsetmiterjoin%
\definecolor{currentfill}{rgb}{0.992771,0.707689,0.712380}%
\pgfsetfillcolor{currentfill}%
\pgfsetlinewidth{0.000000pt}%
\definecolor{currentstroke}{rgb}{0.000000,0.000000,0.000000}%
\pgfsetstrokecolor{currentstroke}%
\pgfsetstrokeopacity{0.000000}%
\pgfsetdash{}{0pt}%
\pgfpathmoveto{\pgfqpoint{0.928144in}{2.088169in}}%
\pgfpathlineto{\pgfqpoint{0.937081in}{2.088169in}}%
\pgfpathlineto{\pgfqpoint{0.937081in}{2.102222in}}%
\pgfpathlineto{\pgfqpoint{0.928144in}{2.102222in}}%
\pgfpathlineto{\pgfqpoint{0.928144in}{2.088169in}}%
\pgfpathclose%
\pgfusepath{fill}%
\end{pgfscope}%
\begin{pgfscope}%
\pgfpathrectangle{\pgfqpoint{0.697024in}{0.857143in}}{\pgfqpoint{2.627103in}{1.813434in}}%
\pgfusepath{clip}%
\pgfsetbuttcap%
\pgfsetmiterjoin%
\definecolor{currentfill}{rgb}{0.992771,0.707689,0.712380}%
\pgfsetfillcolor{currentfill}%
\pgfsetlinewidth{0.000000pt}%
\definecolor{currentstroke}{rgb}{0.000000,0.000000,0.000000}%
\pgfsetstrokecolor{currentstroke}%
\pgfsetstrokeopacity{0.000000}%
\pgfsetdash{}{0pt}%
\pgfpathmoveto{\pgfqpoint{0.939315in}{2.123526in}}%
\pgfpathlineto{\pgfqpoint{0.948251in}{2.123526in}}%
\pgfpathlineto{\pgfqpoint{0.948251in}{2.128737in}}%
\pgfpathlineto{\pgfqpoint{0.939315in}{2.128737in}}%
\pgfpathlineto{\pgfqpoint{0.939315in}{2.123526in}}%
\pgfpathclose%
\pgfusepath{fill}%
\end{pgfscope}%
\begin{pgfscope}%
\pgfpathrectangle{\pgfqpoint{0.697024in}{0.857143in}}{\pgfqpoint{2.627103in}{1.813434in}}%
\pgfusepath{clip}%
\pgfsetbuttcap%
\pgfsetmiterjoin%
\definecolor{currentfill}{rgb}{0.992771,0.707689,0.712380}%
\pgfsetfillcolor{currentfill}%
\pgfsetlinewidth{0.000000pt}%
\definecolor{currentstroke}{rgb}{0.000000,0.000000,0.000000}%
\pgfsetstrokecolor{currentstroke}%
\pgfsetstrokeopacity{0.000000}%
\pgfsetdash{}{0pt}%
\pgfpathmoveto{\pgfqpoint{0.950485in}{2.020992in}}%
\pgfpathlineto{\pgfqpoint{0.959422in}{2.020992in}}%
\pgfpathlineto{\pgfqpoint{0.959422in}{2.048901in}}%
\pgfpathlineto{\pgfqpoint{0.950485in}{2.048901in}}%
\pgfpathlineto{\pgfqpoint{0.950485in}{2.020992in}}%
\pgfpathclose%
\pgfusepath{fill}%
\end{pgfscope}%
\begin{pgfscope}%
\pgfpathrectangle{\pgfqpoint{0.697024in}{0.857143in}}{\pgfqpoint{2.627103in}{1.813434in}}%
\pgfusepath{clip}%
\pgfsetbuttcap%
\pgfsetmiterjoin%
\definecolor{currentfill}{rgb}{0.992771,0.707689,0.712380}%
\pgfsetfillcolor{currentfill}%
\pgfsetlinewidth{0.000000pt}%
\definecolor{currentstroke}{rgb}{0.000000,0.000000,0.000000}%
\pgfsetstrokecolor{currentstroke}%
\pgfsetstrokeopacity{0.000000}%
\pgfsetdash{}{0pt}%
\pgfpathmoveto{\pgfqpoint{0.961656in}{2.121360in}}%
\pgfpathlineto{\pgfqpoint{0.970593in}{2.121360in}}%
\pgfpathlineto{\pgfqpoint{0.970593in}{2.150652in}}%
\pgfpathlineto{\pgfqpoint{0.961656in}{2.150652in}}%
\pgfpathlineto{\pgfqpoint{0.961656in}{2.121360in}}%
\pgfpathclose%
\pgfusepath{fill}%
\end{pgfscope}%
\begin{pgfscope}%
\pgfpathrectangle{\pgfqpoint{0.697024in}{0.857143in}}{\pgfqpoint{2.627103in}{1.813434in}}%
\pgfusepath{clip}%
\pgfsetbuttcap%
\pgfsetmiterjoin%
\definecolor{currentfill}{rgb}{0.992771,0.707689,0.712380}%
\pgfsetfillcolor{currentfill}%
\pgfsetlinewidth{0.000000pt}%
\definecolor{currentstroke}{rgb}{0.000000,0.000000,0.000000}%
\pgfsetstrokecolor{currentstroke}%
\pgfsetstrokeopacity{0.000000}%
\pgfsetdash{}{0pt}%
\pgfpathmoveto{\pgfqpoint{0.972827in}{1.995558in}}%
\pgfpathlineto{\pgfqpoint{0.981763in}{1.995558in}}%
\pgfpathlineto{\pgfqpoint{0.981763in}{2.015382in}}%
\pgfpathlineto{\pgfqpoint{0.972827in}{2.015382in}}%
\pgfpathlineto{\pgfqpoint{0.972827in}{1.995558in}}%
\pgfpathclose%
\pgfusepath{fill}%
\end{pgfscope}%
\begin{pgfscope}%
\pgfpathrectangle{\pgfqpoint{0.697024in}{0.857143in}}{\pgfqpoint{2.627103in}{1.813434in}}%
\pgfusepath{clip}%
\pgfsetbuttcap%
\pgfsetmiterjoin%
\definecolor{currentfill}{rgb}{0.992771,0.707689,0.712380}%
\pgfsetfillcolor{currentfill}%
\pgfsetlinewidth{0.000000pt}%
\definecolor{currentstroke}{rgb}{0.000000,0.000000,0.000000}%
\pgfsetstrokecolor{currentstroke}%
\pgfsetstrokeopacity{0.000000}%
\pgfsetdash{}{0pt}%
\pgfpathmoveto{\pgfqpoint{0.983997in}{2.052750in}}%
\pgfpathlineto{\pgfqpoint{0.992934in}{2.052750in}}%
\pgfpathlineto{\pgfqpoint{0.992934in}{2.053694in}}%
\pgfpathlineto{\pgfqpoint{0.983997in}{2.053694in}}%
\pgfpathlineto{\pgfqpoint{0.983997in}{2.052750in}}%
\pgfpathclose%
\pgfusepath{fill}%
\end{pgfscope}%
\begin{pgfscope}%
\pgfpathrectangle{\pgfqpoint{0.697024in}{0.857143in}}{\pgfqpoint{2.627103in}{1.813434in}}%
\pgfusepath{clip}%
\pgfsetbuttcap%
\pgfsetmiterjoin%
\definecolor{currentfill}{rgb}{0.992771,0.707689,0.712380}%
\pgfsetfillcolor{currentfill}%
\pgfsetlinewidth{0.000000pt}%
\definecolor{currentstroke}{rgb}{0.000000,0.000000,0.000000}%
\pgfsetstrokecolor{currentstroke}%
\pgfsetstrokeopacity{0.000000}%
\pgfsetdash{}{0pt}%
\pgfpathmoveto{\pgfqpoint{0.995168in}{2.180483in}}%
\pgfpathlineto{\pgfqpoint{1.004104in}{2.180483in}}%
\pgfpathlineto{\pgfqpoint{1.004104in}{2.200634in}}%
\pgfpathlineto{\pgfqpoint{0.995168in}{2.200634in}}%
\pgfpathlineto{\pgfqpoint{0.995168in}{2.180483in}}%
\pgfpathclose%
\pgfusepath{fill}%
\end{pgfscope}%
\begin{pgfscope}%
\pgfpathrectangle{\pgfqpoint{0.697024in}{0.857143in}}{\pgfqpoint{2.627103in}{1.813434in}}%
\pgfusepath{clip}%
\pgfsetbuttcap%
\pgfsetmiterjoin%
\definecolor{currentfill}{rgb}{0.992771,0.707689,0.712380}%
\pgfsetfillcolor{currentfill}%
\pgfsetlinewidth{0.000000pt}%
\definecolor{currentstroke}{rgb}{0.000000,0.000000,0.000000}%
\pgfsetstrokecolor{currentstroke}%
\pgfsetstrokeopacity{0.000000}%
\pgfsetdash{}{0pt}%
\pgfpathmoveto{\pgfqpoint{1.006338in}{2.027208in}}%
\pgfpathlineto{\pgfqpoint{1.015275in}{2.027208in}}%
\pgfpathlineto{\pgfqpoint{1.015275in}{2.079846in}}%
\pgfpathlineto{\pgfqpoint{1.006338in}{2.079846in}}%
\pgfpathlineto{\pgfqpoint{1.006338in}{2.027208in}}%
\pgfpathclose%
\pgfusepath{fill}%
\end{pgfscope}%
\begin{pgfscope}%
\pgfpathrectangle{\pgfqpoint{0.697024in}{0.857143in}}{\pgfqpoint{2.627103in}{1.813434in}}%
\pgfusepath{clip}%
\pgfsetbuttcap%
\pgfsetmiterjoin%
\definecolor{currentfill}{rgb}{0.992771,0.707689,0.712380}%
\pgfsetfillcolor{currentfill}%
\pgfsetlinewidth{0.000000pt}%
\definecolor{currentstroke}{rgb}{0.000000,0.000000,0.000000}%
\pgfsetstrokecolor{currentstroke}%
\pgfsetstrokeopacity{0.000000}%
\pgfsetdash{}{0pt}%
\pgfpathmoveto{\pgfqpoint{1.017509in}{2.183309in}}%
\pgfpathlineto{\pgfqpoint{1.026446in}{2.183309in}}%
\pgfpathlineto{\pgfqpoint{1.026446in}{2.195222in}}%
\pgfpathlineto{\pgfqpoint{1.017509in}{2.195222in}}%
\pgfpathlineto{\pgfqpoint{1.017509in}{2.183309in}}%
\pgfpathclose%
\pgfusepath{fill}%
\end{pgfscope}%
\begin{pgfscope}%
\pgfpathrectangle{\pgfqpoint{0.697024in}{0.857143in}}{\pgfqpoint{2.627103in}{1.813434in}}%
\pgfusepath{clip}%
\pgfsetbuttcap%
\pgfsetmiterjoin%
\definecolor{currentfill}{rgb}{0.992771,0.707689,0.712380}%
\pgfsetfillcolor{currentfill}%
\pgfsetlinewidth{0.000000pt}%
\definecolor{currentstroke}{rgb}{0.000000,0.000000,0.000000}%
\pgfsetstrokecolor{currentstroke}%
\pgfsetstrokeopacity{0.000000}%
\pgfsetdash{}{0pt}%
\pgfpathmoveto{\pgfqpoint{1.028680in}{2.110563in}}%
\pgfpathlineto{\pgfqpoint{1.037616in}{2.110563in}}%
\pgfpathlineto{\pgfqpoint{1.037616in}{2.143451in}}%
\pgfpathlineto{\pgfqpoint{1.028680in}{2.143451in}}%
\pgfpathlineto{\pgfqpoint{1.028680in}{2.110563in}}%
\pgfpathclose%
\pgfusepath{fill}%
\end{pgfscope}%
\begin{pgfscope}%
\pgfpathrectangle{\pgfqpoint{0.697024in}{0.857143in}}{\pgfqpoint{2.627103in}{1.813434in}}%
\pgfusepath{clip}%
\pgfsetbuttcap%
\pgfsetmiterjoin%
\definecolor{currentfill}{rgb}{0.992771,0.707689,0.712380}%
\pgfsetfillcolor{currentfill}%
\pgfsetlinewidth{0.000000pt}%
\definecolor{currentstroke}{rgb}{0.000000,0.000000,0.000000}%
\pgfsetstrokecolor{currentstroke}%
\pgfsetstrokeopacity{0.000000}%
\pgfsetdash{}{0pt}%
\pgfpathmoveto{\pgfqpoint{1.039850in}{2.126257in}}%
\pgfpathlineto{\pgfqpoint{1.048787in}{2.126257in}}%
\pgfpathlineto{\pgfqpoint{1.048787in}{2.162019in}}%
\pgfpathlineto{\pgfqpoint{1.039850in}{2.162019in}}%
\pgfpathlineto{\pgfqpoint{1.039850in}{2.126257in}}%
\pgfpathclose%
\pgfusepath{fill}%
\end{pgfscope}%
\begin{pgfscope}%
\pgfpathrectangle{\pgfqpoint{0.697024in}{0.857143in}}{\pgfqpoint{2.627103in}{1.813434in}}%
\pgfusepath{clip}%
\pgfsetbuttcap%
\pgfsetmiterjoin%
\definecolor{currentfill}{rgb}{0.992771,0.707689,0.712380}%
\pgfsetfillcolor{currentfill}%
\pgfsetlinewidth{0.000000pt}%
\definecolor{currentstroke}{rgb}{0.000000,0.000000,0.000000}%
\pgfsetstrokecolor{currentstroke}%
\pgfsetstrokeopacity{0.000000}%
\pgfsetdash{}{0pt}%
\pgfpathmoveto{\pgfqpoint{1.051021in}{2.189946in}}%
\pgfpathlineto{\pgfqpoint{1.059957in}{2.189946in}}%
\pgfpathlineto{\pgfqpoint{1.059957in}{2.219162in}}%
\pgfpathlineto{\pgfqpoint{1.051021in}{2.219162in}}%
\pgfpathlineto{\pgfqpoint{1.051021in}{2.189946in}}%
\pgfpathclose%
\pgfusepath{fill}%
\end{pgfscope}%
\begin{pgfscope}%
\pgfpathrectangle{\pgfqpoint{0.697024in}{0.857143in}}{\pgfqpoint{2.627103in}{1.813434in}}%
\pgfusepath{clip}%
\pgfsetbuttcap%
\pgfsetmiterjoin%
\definecolor{currentfill}{rgb}{0.992771,0.707689,0.712380}%
\pgfsetfillcolor{currentfill}%
\pgfsetlinewidth{0.000000pt}%
\definecolor{currentstroke}{rgb}{0.000000,0.000000,0.000000}%
\pgfsetstrokecolor{currentstroke}%
\pgfsetstrokeopacity{0.000000}%
\pgfsetdash{}{0pt}%
\pgfpathmoveto{\pgfqpoint{1.062191in}{2.372450in}}%
\pgfpathlineto{\pgfqpoint{1.071128in}{2.372450in}}%
\pgfpathlineto{\pgfqpoint{1.071128in}{2.388513in}}%
\pgfpathlineto{\pgfqpoint{1.062191in}{2.388513in}}%
\pgfpathlineto{\pgfqpoint{1.062191in}{2.372450in}}%
\pgfpathclose%
\pgfusepath{fill}%
\end{pgfscope}%
\begin{pgfscope}%
\pgfpathrectangle{\pgfqpoint{0.697024in}{0.857143in}}{\pgfqpoint{2.627103in}{1.813434in}}%
\pgfusepath{clip}%
\pgfsetbuttcap%
\pgfsetmiterjoin%
\definecolor{currentfill}{rgb}{0.992771,0.707689,0.712380}%
\pgfsetfillcolor{currentfill}%
\pgfsetlinewidth{0.000000pt}%
\definecolor{currentstroke}{rgb}{0.000000,0.000000,0.000000}%
\pgfsetstrokecolor{currentstroke}%
\pgfsetstrokeopacity{0.000000}%
\pgfsetdash{}{0pt}%
\pgfpathmoveto{\pgfqpoint{1.073362in}{2.439075in}}%
\pgfpathlineto{\pgfqpoint{1.082299in}{2.439075in}}%
\pgfpathlineto{\pgfqpoint{1.082299in}{2.441446in}}%
\pgfpathlineto{\pgfqpoint{1.073362in}{2.441446in}}%
\pgfpathlineto{\pgfqpoint{1.073362in}{2.439075in}}%
\pgfpathclose%
\pgfusepath{fill}%
\end{pgfscope}%
\begin{pgfscope}%
\pgfpathrectangle{\pgfqpoint{0.697024in}{0.857143in}}{\pgfqpoint{2.627103in}{1.813434in}}%
\pgfusepath{clip}%
\pgfsetbuttcap%
\pgfsetmiterjoin%
\definecolor{currentfill}{rgb}{0.992771,0.707689,0.712380}%
\pgfsetfillcolor{currentfill}%
\pgfsetlinewidth{0.000000pt}%
\definecolor{currentstroke}{rgb}{0.000000,0.000000,0.000000}%
\pgfsetstrokecolor{currentstroke}%
\pgfsetstrokeopacity{0.000000}%
\pgfsetdash{}{0pt}%
\pgfpathmoveto{\pgfqpoint{1.084533in}{2.318242in}}%
\pgfpathlineto{\pgfqpoint{1.093469in}{2.318242in}}%
\pgfpathlineto{\pgfqpoint{1.093469in}{2.349051in}}%
\pgfpathlineto{\pgfqpoint{1.084533in}{2.349051in}}%
\pgfpathlineto{\pgfqpoint{1.084533in}{2.318242in}}%
\pgfpathclose%
\pgfusepath{fill}%
\end{pgfscope}%
\begin{pgfscope}%
\pgfpathrectangle{\pgfqpoint{0.697024in}{0.857143in}}{\pgfqpoint{2.627103in}{1.813434in}}%
\pgfusepath{clip}%
\pgfsetbuttcap%
\pgfsetmiterjoin%
\definecolor{currentfill}{rgb}{0.992771,0.707689,0.712380}%
\pgfsetfillcolor{currentfill}%
\pgfsetlinewidth{0.000000pt}%
\definecolor{currentstroke}{rgb}{0.000000,0.000000,0.000000}%
\pgfsetstrokecolor{currentstroke}%
\pgfsetstrokeopacity{0.000000}%
\pgfsetdash{}{0pt}%
\pgfpathmoveto{\pgfqpoint{1.095703in}{2.354518in}}%
\pgfpathlineto{\pgfqpoint{1.104640in}{2.354518in}}%
\pgfpathlineto{\pgfqpoint{1.104640in}{2.396713in}}%
\pgfpathlineto{\pgfqpoint{1.095703in}{2.396713in}}%
\pgfpathlineto{\pgfqpoint{1.095703in}{2.354518in}}%
\pgfpathclose%
\pgfusepath{fill}%
\end{pgfscope}%
\begin{pgfscope}%
\pgfpathrectangle{\pgfqpoint{0.697024in}{0.857143in}}{\pgfqpoint{2.627103in}{1.813434in}}%
\pgfusepath{clip}%
\pgfsetbuttcap%
\pgfsetmiterjoin%
\definecolor{currentfill}{rgb}{0.992771,0.707689,0.712380}%
\pgfsetfillcolor{currentfill}%
\pgfsetlinewidth{0.000000pt}%
\definecolor{currentstroke}{rgb}{0.000000,0.000000,0.000000}%
\pgfsetstrokecolor{currentstroke}%
\pgfsetstrokeopacity{0.000000}%
\pgfsetdash{}{0pt}%
\pgfpathmoveto{\pgfqpoint{1.106874in}{2.376869in}}%
\pgfpathlineto{\pgfqpoint{1.115810in}{2.376869in}}%
\pgfpathlineto{\pgfqpoint{1.115810in}{2.383642in}}%
\pgfpathlineto{\pgfqpoint{1.106874in}{2.383642in}}%
\pgfpathlineto{\pgfqpoint{1.106874in}{2.376869in}}%
\pgfpathclose%
\pgfusepath{fill}%
\end{pgfscope}%
\begin{pgfscope}%
\pgfpathrectangle{\pgfqpoint{0.697024in}{0.857143in}}{\pgfqpoint{2.627103in}{1.813434in}}%
\pgfusepath{clip}%
\pgfsetbuttcap%
\pgfsetmiterjoin%
\definecolor{currentfill}{rgb}{0.992771,0.707689,0.712380}%
\pgfsetfillcolor{currentfill}%
\pgfsetlinewidth{0.000000pt}%
\definecolor{currentstroke}{rgb}{0.000000,0.000000,0.000000}%
\pgfsetstrokecolor{currentstroke}%
\pgfsetstrokeopacity{0.000000}%
\pgfsetdash{}{0pt}%
\pgfpathmoveto{\pgfqpoint{1.118045in}{1.415714in}}%
\pgfpathlineto{\pgfqpoint{1.126981in}{1.415714in}}%
\pgfpathlineto{\pgfqpoint{1.126981in}{1.406880in}}%
\pgfpathlineto{\pgfqpoint{1.118045in}{1.406880in}}%
\pgfpathlineto{\pgfqpoint{1.118045in}{1.415714in}}%
\pgfpathclose%
\pgfusepath{fill}%
\end{pgfscope}%
\begin{pgfscope}%
\pgfpathrectangle{\pgfqpoint{0.697024in}{0.857143in}}{\pgfqpoint{2.627103in}{1.813434in}}%
\pgfusepath{clip}%
\pgfsetbuttcap%
\pgfsetmiterjoin%
\definecolor{currentfill}{rgb}{0.992771,0.707689,0.712380}%
\pgfsetfillcolor{currentfill}%
\pgfsetlinewidth{0.000000pt}%
\definecolor{currentstroke}{rgb}{0.000000,0.000000,0.000000}%
\pgfsetstrokecolor{currentstroke}%
\pgfsetstrokeopacity{0.000000}%
\pgfsetdash{}{0pt}%
\pgfpathmoveto{\pgfqpoint{1.129215in}{2.201212in}}%
\pgfpathlineto{\pgfqpoint{1.138152in}{2.201212in}}%
\pgfpathlineto{\pgfqpoint{1.138152in}{2.217821in}}%
\pgfpathlineto{\pgfqpoint{1.129215in}{2.217821in}}%
\pgfpathlineto{\pgfqpoint{1.129215in}{2.201212in}}%
\pgfpathclose%
\pgfusepath{fill}%
\end{pgfscope}%
\begin{pgfscope}%
\pgfpathrectangle{\pgfqpoint{0.697024in}{0.857143in}}{\pgfqpoint{2.627103in}{1.813434in}}%
\pgfusepath{clip}%
\pgfsetbuttcap%
\pgfsetmiterjoin%
\definecolor{currentfill}{rgb}{0.992771,0.707689,0.712380}%
\pgfsetfillcolor{currentfill}%
\pgfsetlinewidth{0.000000pt}%
\definecolor{currentstroke}{rgb}{0.000000,0.000000,0.000000}%
\pgfsetstrokecolor{currentstroke}%
\pgfsetstrokeopacity{0.000000}%
\pgfsetdash{}{0pt}%
\pgfpathmoveto{\pgfqpoint{1.140386in}{1.498388in}}%
\pgfpathlineto{\pgfqpoint{1.149322in}{1.498388in}}%
\pgfpathlineto{\pgfqpoint{1.149322in}{1.479603in}}%
\pgfpathlineto{\pgfqpoint{1.140386in}{1.479603in}}%
\pgfpathlineto{\pgfqpoint{1.140386in}{1.498388in}}%
\pgfpathclose%
\pgfusepath{fill}%
\end{pgfscope}%
\begin{pgfscope}%
\pgfpathrectangle{\pgfqpoint{0.697024in}{0.857143in}}{\pgfqpoint{2.627103in}{1.813434in}}%
\pgfusepath{clip}%
\pgfsetbuttcap%
\pgfsetmiterjoin%
\definecolor{currentfill}{rgb}{0.992771,0.707689,0.712380}%
\pgfsetfillcolor{currentfill}%
\pgfsetlinewidth{0.000000pt}%
\definecolor{currentstroke}{rgb}{0.000000,0.000000,0.000000}%
\pgfsetstrokecolor{currentstroke}%
\pgfsetstrokeopacity{0.000000}%
\pgfsetdash{}{0pt}%
\pgfpathmoveto{\pgfqpoint{1.151556in}{1.485012in}}%
\pgfpathlineto{\pgfqpoint{1.160493in}{1.485012in}}%
\pgfpathlineto{\pgfqpoint{1.160493in}{1.478101in}}%
\pgfpathlineto{\pgfqpoint{1.151556in}{1.478101in}}%
\pgfpathlineto{\pgfqpoint{1.151556in}{1.485012in}}%
\pgfpathclose%
\pgfusepath{fill}%
\end{pgfscope}%
\begin{pgfscope}%
\pgfpathrectangle{\pgfqpoint{0.697024in}{0.857143in}}{\pgfqpoint{2.627103in}{1.813434in}}%
\pgfusepath{clip}%
\pgfsetbuttcap%
\pgfsetmiterjoin%
\definecolor{currentfill}{rgb}{0.992771,0.707689,0.712380}%
\pgfsetfillcolor{currentfill}%
\pgfsetlinewidth{0.000000pt}%
\definecolor{currentstroke}{rgb}{0.000000,0.000000,0.000000}%
\pgfsetstrokecolor{currentstroke}%
\pgfsetstrokeopacity{0.000000}%
\pgfsetdash{}{0pt}%
\pgfpathmoveto{\pgfqpoint{1.162727in}{1.601889in}}%
\pgfpathlineto{\pgfqpoint{1.171663in}{1.601889in}}%
\pgfpathlineto{\pgfqpoint{1.171663in}{1.600992in}}%
\pgfpathlineto{\pgfqpoint{1.162727in}{1.600992in}}%
\pgfpathlineto{\pgfqpoint{1.162727in}{1.601889in}}%
\pgfpathclose%
\pgfusepath{fill}%
\end{pgfscope}%
\begin{pgfscope}%
\pgfpathrectangle{\pgfqpoint{0.697024in}{0.857143in}}{\pgfqpoint{2.627103in}{1.813434in}}%
\pgfusepath{clip}%
\pgfsetbuttcap%
\pgfsetmiterjoin%
\definecolor{currentfill}{rgb}{0.992771,0.707689,0.712380}%
\pgfsetfillcolor{currentfill}%
\pgfsetlinewidth{0.000000pt}%
\definecolor{currentstroke}{rgb}{0.000000,0.000000,0.000000}%
\pgfsetstrokecolor{currentstroke}%
\pgfsetstrokeopacity{0.000000}%
\pgfsetdash{}{0pt}%
\pgfpathmoveto{\pgfqpoint{1.173898in}{1.981808in}}%
\pgfpathlineto{\pgfqpoint{1.182834in}{1.981808in}}%
\pgfpathlineto{\pgfqpoint{1.182834in}{2.002459in}}%
\pgfpathlineto{\pgfqpoint{1.173898in}{2.002459in}}%
\pgfpathlineto{\pgfqpoint{1.173898in}{1.981808in}}%
\pgfpathclose%
\pgfusepath{fill}%
\end{pgfscope}%
\begin{pgfscope}%
\pgfpathrectangle{\pgfqpoint{0.697024in}{0.857143in}}{\pgfqpoint{2.627103in}{1.813434in}}%
\pgfusepath{clip}%
\pgfsetbuttcap%
\pgfsetmiterjoin%
\definecolor{currentfill}{rgb}{0.992771,0.707689,0.712380}%
\pgfsetfillcolor{currentfill}%
\pgfsetlinewidth{0.000000pt}%
\definecolor{currentstroke}{rgb}{0.000000,0.000000,0.000000}%
\pgfsetstrokecolor{currentstroke}%
\pgfsetstrokeopacity{0.000000}%
\pgfsetdash{}{0pt}%
\pgfpathmoveto{\pgfqpoint{1.185068in}{1.482547in}}%
\pgfpathlineto{\pgfqpoint{1.194005in}{1.482547in}}%
\pgfpathlineto{\pgfqpoint{1.194005in}{1.475680in}}%
\pgfpathlineto{\pgfqpoint{1.185068in}{1.475680in}}%
\pgfpathlineto{\pgfqpoint{1.185068in}{1.482547in}}%
\pgfpathclose%
\pgfusepath{fill}%
\end{pgfscope}%
\begin{pgfscope}%
\pgfpathrectangle{\pgfqpoint{0.697024in}{0.857143in}}{\pgfqpoint{2.627103in}{1.813434in}}%
\pgfusepath{clip}%
\pgfsetbuttcap%
\pgfsetmiterjoin%
\definecolor{currentfill}{rgb}{0.992771,0.707689,0.712380}%
\pgfsetfillcolor{currentfill}%
\pgfsetlinewidth{0.000000pt}%
\definecolor{currentstroke}{rgb}{0.000000,0.000000,0.000000}%
\pgfsetstrokecolor{currentstroke}%
\pgfsetstrokeopacity{0.000000}%
\pgfsetdash{}{0pt}%
\pgfpathmoveto{\pgfqpoint{1.196239in}{2.026533in}}%
\pgfpathlineto{\pgfqpoint{1.205175in}{2.026533in}}%
\pgfpathlineto{\pgfqpoint{1.205175in}{2.045353in}}%
\pgfpathlineto{\pgfqpoint{1.196239in}{2.045353in}}%
\pgfpathlineto{\pgfqpoint{1.196239in}{2.026533in}}%
\pgfpathclose%
\pgfusepath{fill}%
\end{pgfscope}%
\begin{pgfscope}%
\pgfpathrectangle{\pgfqpoint{0.697024in}{0.857143in}}{\pgfqpoint{2.627103in}{1.813434in}}%
\pgfusepath{clip}%
\pgfsetbuttcap%
\pgfsetmiterjoin%
\definecolor{currentfill}{rgb}{0.992771,0.707689,0.712380}%
\pgfsetfillcolor{currentfill}%
\pgfsetlinewidth{0.000000pt}%
\definecolor{currentstroke}{rgb}{0.000000,0.000000,0.000000}%
\pgfsetstrokecolor{currentstroke}%
\pgfsetstrokeopacity{0.000000}%
\pgfsetdash{}{0pt}%
\pgfpathmoveto{\pgfqpoint{1.207409in}{2.010415in}}%
\pgfpathlineto{\pgfqpoint{1.216346in}{2.010415in}}%
\pgfpathlineto{\pgfqpoint{1.216346in}{2.030686in}}%
\pgfpathlineto{\pgfqpoint{1.207409in}{2.030686in}}%
\pgfpathlineto{\pgfqpoint{1.207409in}{2.010415in}}%
\pgfpathclose%
\pgfusepath{fill}%
\end{pgfscope}%
\begin{pgfscope}%
\pgfpathrectangle{\pgfqpoint{0.697024in}{0.857143in}}{\pgfqpoint{2.627103in}{1.813434in}}%
\pgfusepath{clip}%
\pgfsetbuttcap%
\pgfsetmiterjoin%
\definecolor{currentfill}{rgb}{0.992771,0.707689,0.712380}%
\pgfsetfillcolor{currentfill}%
\pgfsetlinewidth{0.000000pt}%
\definecolor{currentstroke}{rgb}{0.000000,0.000000,0.000000}%
\pgfsetstrokecolor{currentstroke}%
\pgfsetstrokeopacity{0.000000}%
\pgfsetdash{}{0pt}%
\pgfpathmoveto{\pgfqpoint{1.218580in}{2.064688in}}%
\pgfpathlineto{\pgfqpoint{1.227516in}{2.064688in}}%
\pgfpathlineto{\pgfqpoint{1.227516in}{2.098356in}}%
\pgfpathlineto{\pgfqpoint{1.218580in}{2.098356in}}%
\pgfpathlineto{\pgfqpoint{1.218580in}{2.064688in}}%
\pgfpathclose%
\pgfusepath{fill}%
\end{pgfscope}%
\begin{pgfscope}%
\pgfpathrectangle{\pgfqpoint{0.697024in}{0.857143in}}{\pgfqpoint{2.627103in}{1.813434in}}%
\pgfusepath{clip}%
\pgfsetbuttcap%
\pgfsetmiterjoin%
\definecolor{currentfill}{rgb}{0.992771,0.707689,0.712380}%
\pgfsetfillcolor{currentfill}%
\pgfsetlinewidth{0.000000pt}%
\definecolor{currentstroke}{rgb}{0.000000,0.000000,0.000000}%
\pgfsetstrokecolor{currentstroke}%
\pgfsetstrokeopacity{0.000000}%
\pgfsetdash{}{0pt}%
\pgfpathmoveto{\pgfqpoint{1.229751in}{2.039537in}}%
\pgfpathlineto{\pgfqpoint{1.238687in}{2.039537in}}%
\pgfpathlineto{\pgfqpoint{1.238687in}{2.050933in}}%
\pgfpathlineto{\pgfqpoint{1.229751in}{2.050933in}}%
\pgfpathlineto{\pgfqpoint{1.229751in}{2.039537in}}%
\pgfpathclose%
\pgfusepath{fill}%
\end{pgfscope}%
\begin{pgfscope}%
\pgfpathrectangle{\pgfqpoint{0.697024in}{0.857143in}}{\pgfqpoint{2.627103in}{1.813434in}}%
\pgfusepath{clip}%
\pgfsetbuttcap%
\pgfsetmiterjoin%
\definecolor{currentfill}{rgb}{0.992771,0.707689,0.712380}%
\pgfsetfillcolor{currentfill}%
\pgfsetlinewidth{0.000000pt}%
\definecolor{currentstroke}{rgb}{0.000000,0.000000,0.000000}%
\pgfsetstrokecolor{currentstroke}%
\pgfsetstrokeopacity{0.000000}%
\pgfsetdash{}{0pt}%
\pgfpathmoveto{\pgfqpoint{1.240921in}{2.031101in}}%
\pgfpathlineto{\pgfqpoint{1.249858in}{2.031101in}}%
\pgfpathlineto{\pgfqpoint{1.249858in}{2.043342in}}%
\pgfpathlineto{\pgfqpoint{1.240921in}{2.043342in}}%
\pgfpathlineto{\pgfqpoint{1.240921in}{2.031101in}}%
\pgfpathclose%
\pgfusepath{fill}%
\end{pgfscope}%
\begin{pgfscope}%
\pgfpathrectangle{\pgfqpoint{0.697024in}{0.857143in}}{\pgfqpoint{2.627103in}{1.813434in}}%
\pgfusepath{clip}%
\pgfsetbuttcap%
\pgfsetmiterjoin%
\definecolor{currentfill}{rgb}{0.992771,0.707689,0.712380}%
\pgfsetfillcolor{currentfill}%
\pgfsetlinewidth{0.000000pt}%
\definecolor{currentstroke}{rgb}{0.000000,0.000000,0.000000}%
\pgfsetstrokecolor{currentstroke}%
\pgfsetstrokeopacity{0.000000}%
\pgfsetdash{}{0pt}%
\pgfpathmoveto{\pgfqpoint{1.252092in}{2.053093in}}%
\pgfpathlineto{\pgfqpoint{1.261028in}{2.053093in}}%
\pgfpathlineto{\pgfqpoint{1.261028in}{2.069935in}}%
\pgfpathlineto{\pgfqpoint{1.252092in}{2.069935in}}%
\pgfpathlineto{\pgfqpoint{1.252092in}{2.053093in}}%
\pgfpathclose%
\pgfusepath{fill}%
\end{pgfscope}%
\begin{pgfscope}%
\pgfpathrectangle{\pgfqpoint{0.697024in}{0.857143in}}{\pgfqpoint{2.627103in}{1.813434in}}%
\pgfusepath{clip}%
\pgfsetbuttcap%
\pgfsetmiterjoin%
\definecolor{currentfill}{rgb}{0.992771,0.707689,0.712380}%
\pgfsetfillcolor{currentfill}%
\pgfsetlinewidth{0.000000pt}%
\definecolor{currentstroke}{rgb}{0.000000,0.000000,0.000000}%
\pgfsetstrokecolor{currentstroke}%
\pgfsetstrokeopacity{0.000000}%
\pgfsetdash{}{0pt}%
\pgfpathmoveto{\pgfqpoint{1.263262in}{2.019505in}}%
\pgfpathlineto{\pgfqpoint{1.272199in}{2.019505in}}%
\pgfpathlineto{\pgfqpoint{1.272199in}{2.041653in}}%
\pgfpathlineto{\pgfqpoint{1.263262in}{2.041653in}}%
\pgfpathlineto{\pgfqpoint{1.263262in}{2.019505in}}%
\pgfpathclose%
\pgfusepath{fill}%
\end{pgfscope}%
\begin{pgfscope}%
\pgfpathrectangle{\pgfqpoint{0.697024in}{0.857143in}}{\pgfqpoint{2.627103in}{1.813434in}}%
\pgfusepath{clip}%
\pgfsetbuttcap%
\pgfsetmiterjoin%
\definecolor{currentfill}{rgb}{0.992771,0.707689,0.712380}%
\pgfsetfillcolor{currentfill}%
\pgfsetlinewidth{0.000000pt}%
\definecolor{currentstroke}{rgb}{0.000000,0.000000,0.000000}%
\pgfsetstrokecolor{currentstroke}%
\pgfsetstrokeopacity{0.000000}%
\pgfsetdash{}{0pt}%
\pgfpathmoveto{\pgfqpoint{1.274433in}{2.053058in}}%
\pgfpathlineto{\pgfqpoint{1.283369in}{2.053058in}}%
\pgfpathlineto{\pgfqpoint{1.283369in}{2.086879in}}%
\pgfpathlineto{\pgfqpoint{1.274433in}{2.086879in}}%
\pgfpathlineto{\pgfqpoint{1.274433in}{2.053058in}}%
\pgfpathclose%
\pgfusepath{fill}%
\end{pgfscope}%
\begin{pgfscope}%
\pgfpathrectangle{\pgfqpoint{0.697024in}{0.857143in}}{\pgfqpoint{2.627103in}{1.813434in}}%
\pgfusepath{clip}%
\pgfsetbuttcap%
\pgfsetmiterjoin%
\definecolor{currentfill}{rgb}{0.992771,0.707689,0.712380}%
\pgfsetfillcolor{currentfill}%
\pgfsetlinewidth{0.000000pt}%
\definecolor{currentstroke}{rgb}{0.000000,0.000000,0.000000}%
\pgfsetstrokecolor{currentstroke}%
\pgfsetstrokeopacity{0.000000}%
\pgfsetdash{}{0pt}%
\pgfpathmoveto{\pgfqpoint{1.285604in}{2.028414in}}%
\pgfpathlineto{\pgfqpoint{1.294540in}{2.028414in}}%
\pgfpathlineto{\pgfqpoint{1.294540in}{2.057549in}}%
\pgfpathlineto{\pgfqpoint{1.285604in}{2.057549in}}%
\pgfpathlineto{\pgfqpoint{1.285604in}{2.028414in}}%
\pgfpathclose%
\pgfusepath{fill}%
\end{pgfscope}%
\begin{pgfscope}%
\pgfpathrectangle{\pgfqpoint{0.697024in}{0.857143in}}{\pgfqpoint{2.627103in}{1.813434in}}%
\pgfusepath{clip}%
\pgfsetbuttcap%
\pgfsetmiterjoin%
\definecolor{currentfill}{rgb}{0.992771,0.707689,0.712380}%
\pgfsetfillcolor{currentfill}%
\pgfsetlinewidth{0.000000pt}%
\definecolor{currentstroke}{rgb}{0.000000,0.000000,0.000000}%
\pgfsetstrokecolor{currentstroke}%
\pgfsetstrokeopacity{0.000000}%
\pgfsetdash{}{0pt}%
\pgfpathmoveto{\pgfqpoint{1.296774in}{2.061363in}}%
\pgfpathlineto{\pgfqpoint{1.305711in}{2.061363in}}%
\pgfpathlineto{\pgfqpoint{1.305711in}{2.067730in}}%
\pgfpathlineto{\pgfqpoint{1.296774in}{2.067730in}}%
\pgfpathlineto{\pgfqpoint{1.296774in}{2.061363in}}%
\pgfpathclose%
\pgfusepath{fill}%
\end{pgfscope}%
\begin{pgfscope}%
\pgfpathrectangle{\pgfqpoint{0.697024in}{0.857143in}}{\pgfqpoint{2.627103in}{1.813434in}}%
\pgfusepath{clip}%
\pgfsetbuttcap%
\pgfsetmiterjoin%
\definecolor{currentfill}{rgb}{0.992771,0.707689,0.712380}%
\pgfsetfillcolor{currentfill}%
\pgfsetlinewidth{0.000000pt}%
\definecolor{currentstroke}{rgb}{0.000000,0.000000,0.000000}%
\pgfsetstrokecolor{currentstroke}%
\pgfsetstrokeopacity{0.000000}%
\pgfsetdash{}{0pt}%
\pgfpathmoveto{\pgfqpoint{1.307945in}{2.087047in}}%
\pgfpathlineto{\pgfqpoint{1.316881in}{2.087047in}}%
\pgfpathlineto{\pgfqpoint{1.316881in}{2.112266in}}%
\pgfpathlineto{\pgfqpoint{1.307945in}{2.112266in}}%
\pgfpathlineto{\pgfqpoint{1.307945in}{2.087047in}}%
\pgfpathclose%
\pgfusepath{fill}%
\end{pgfscope}%
\begin{pgfscope}%
\pgfpathrectangle{\pgfqpoint{0.697024in}{0.857143in}}{\pgfqpoint{2.627103in}{1.813434in}}%
\pgfusepath{clip}%
\pgfsetbuttcap%
\pgfsetmiterjoin%
\definecolor{currentfill}{rgb}{0.992771,0.707689,0.712380}%
\pgfsetfillcolor{currentfill}%
\pgfsetlinewidth{0.000000pt}%
\definecolor{currentstroke}{rgb}{0.000000,0.000000,0.000000}%
\pgfsetstrokecolor{currentstroke}%
\pgfsetstrokeopacity{0.000000}%
\pgfsetdash{}{0pt}%
\pgfpathmoveto{\pgfqpoint{1.319115in}{1.992896in}}%
\pgfpathlineto{\pgfqpoint{1.328052in}{1.992896in}}%
\pgfpathlineto{\pgfqpoint{1.328052in}{2.032243in}}%
\pgfpathlineto{\pgfqpoint{1.319115in}{2.032243in}}%
\pgfpathlineto{\pgfqpoint{1.319115in}{1.992896in}}%
\pgfpathclose%
\pgfusepath{fill}%
\end{pgfscope}%
\begin{pgfscope}%
\pgfpathrectangle{\pgfqpoint{0.697024in}{0.857143in}}{\pgfqpoint{2.627103in}{1.813434in}}%
\pgfusepath{clip}%
\pgfsetbuttcap%
\pgfsetmiterjoin%
\definecolor{currentfill}{rgb}{0.992771,0.707689,0.712380}%
\pgfsetfillcolor{currentfill}%
\pgfsetlinewidth{0.000000pt}%
\definecolor{currentstroke}{rgb}{0.000000,0.000000,0.000000}%
\pgfsetstrokecolor{currentstroke}%
\pgfsetstrokeopacity{0.000000}%
\pgfsetdash{}{0pt}%
\pgfpathmoveto{\pgfqpoint{1.330286in}{2.042928in}}%
\pgfpathlineto{\pgfqpoint{1.339222in}{2.042928in}}%
\pgfpathlineto{\pgfqpoint{1.339222in}{2.049702in}}%
\pgfpathlineto{\pgfqpoint{1.330286in}{2.049702in}}%
\pgfpathlineto{\pgfqpoint{1.330286in}{2.042928in}}%
\pgfpathclose%
\pgfusepath{fill}%
\end{pgfscope}%
\begin{pgfscope}%
\pgfpathrectangle{\pgfqpoint{0.697024in}{0.857143in}}{\pgfqpoint{2.627103in}{1.813434in}}%
\pgfusepath{clip}%
\pgfsetbuttcap%
\pgfsetmiterjoin%
\definecolor{currentfill}{rgb}{0.992771,0.707689,0.712380}%
\pgfsetfillcolor{currentfill}%
\pgfsetlinewidth{0.000000pt}%
\definecolor{currentstroke}{rgb}{0.000000,0.000000,0.000000}%
\pgfsetstrokecolor{currentstroke}%
\pgfsetstrokeopacity{0.000000}%
\pgfsetdash{}{0pt}%
\pgfpathmoveto{\pgfqpoint{1.341457in}{1.681325in}}%
\pgfpathlineto{\pgfqpoint{1.350393in}{1.681325in}}%
\pgfpathlineto{\pgfqpoint{1.350393in}{1.679023in}}%
\pgfpathlineto{\pgfqpoint{1.341457in}{1.679023in}}%
\pgfpathlineto{\pgfqpoint{1.341457in}{1.681325in}}%
\pgfpathclose%
\pgfusepath{fill}%
\end{pgfscope}%
\begin{pgfscope}%
\pgfpathrectangle{\pgfqpoint{0.697024in}{0.857143in}}{\pgfqpoint{2.627103in}{1.813434in}}%
\pgfusepath{clip}%
\pgfsetbuttcap%
\pgfsetmiterjoin%
\definecolor{currentfill}{rgb}{0.992771,0.707689,0.712380}%
\pgfsetfillcolor{currentfill}%
\pgfsetlinewidth{0.000000pt}%
\definecolor{currentstroke}{rgb}{0.000000,0.000000,0.000000}%
\pgfsetstrokecolor{currentstroke}%
\pgfsetstrokeopacity{0.000000}%
\pgfsetdash{}{0pt}%
\pgfpathmoveto{\pgfqpoint{1.352627in}{2.034936in}}%
\pgfpathlineto{\pgfqpoint{1.361564in}{2.034936in}}%
\pgfpathlineto{\pgfqpoint{1.361564in}{2.037807in}}%
\pgfpathlineto{\pgfqpoint{1.352627in}{2.037807in}}%
\pgfpathlineto{\pgfqpoint{1.352627in}{2.034936in}}%
\pgfpathclose%
\pgfusepath{fill}%
\end{pgfscope}%
\begin{pgfscope}%
\pgfpathrectangle{\pgfqpoint{0.697024in}{0.857143in}}{\pgfqpoint{2.627103in}{1.813434in}}%
\pgfusepath{clip}%
\pgfsetbuttcap%
\pgfsetmiterjoin%
\definecolor{currentfill}{rgb}{0.992771,0.707689,0.712380}%
\pgfsetfillcolor{currentfill}%
\pgfsetlinewidth{0.000000pt}%
\definecolor{currentstroke}{rgb}{0.000000,0.000000,0.000000}%
\pgfsetstrokecolor{currentstroke}%
\pgfsetstrokeopacity{0.000000}%
\pgfsetdash{}{0pt}%
\pgfpathmoveto{\pgfqpoint{1.363798in}{1.723929in}}%
\pgfpathlineto{\pgfqpoint{1.372734in}{1.723929in}}%
\pgfpathlineto{\pgfqpoint{1.372734in}{1.714263in}}%
\pgfpathlineto{\pgfqpoint{1.363798in}{1.714263in}}%
\pgfpathlineto{\pgfqpoint{1.363798in}{1.723929in}}%
\pgfpathclose%
\pgfusepath{fill}%
\end{pgfscope}%
\begin{pgfscope}%
\pgfpathrectangle{\pgfqpoint{0.697024in}{0.857143in}}{\pgfqpoint{2.627103in}{1.813434in}}%
\pgfusepath{clip}%
\pgfsetbuttcap%
\pgfsetmiterjoin%
\definecolor{currentfill}{rgb}{0.992771,0.707689,0.712380}%
\pgfsetfillcolor{currentfill}%
\pgfsetlinewidth{0.000000pt}%
\definecolor{currentstroke}{rgb}{0.000000,0.000000,0.000000}%
\pgfsetstrokecolor{currentstroke}%
\pgfsetstrokeopacity{0.000000}%
\pgfsetdash{}{0pt}%
\pgfpathmoveto{\pgfqpoint{1.374968in}{1.712452in}}%
\pgfpathlineto{\pgfqpoint{1.383905in}{1.712452in}}%
\pgfpathlineto{\pgfqpoint{1.383905in}{1.685645in}}%
\pgfpathlineto{\pgfqpoint{1.374968in}{1.685645in}}%
\pgfpathlineto{\pgfqpoint{1.374968in}{1.712452in}}%
\pgfpathclose%
\pgfusepath{fill}%
\end{pgfscope}%
\begin{pgfscope}%
\pgfpathrectangle{\pgfqpoint{0.697024in}{0.857143in}}{\pgfqpoint{2.627103in}{1.813434in}}%
\pgfusepath{clip}%
\pgfsetbuttcap%
\pgfsetmiterjoin%
\definecolor{currentfill}{rgb}{0.992771,0.707689,0.712380}%
\pgfsetfillcolor{currentfill}%
\pgfsetlinewidth{0.000000pt}%
\definecolor{currentstroke}{rgb}{0.000000,0.000000,0.000000}%
\pgfsetstrokecolor{currentstroke}%
\pgfsetstrokeopacity{0.000000}%
\pgfsetdash{}{0pt}%
\pgfpathmoveto{\pgfqpoint{1.386139in}{1.698499in}}%
\pgfpathlineto{\pgfqpoint{1.395076in}{1.698499in}}%
\pgfpathlineto{\pgfqpoint{1.395076in}{1.678822in}}%
\pgfpathlineto{\pgfqpoint{1.386139in}{1.678822in}}%
\pgfpathlineto{\pgfqpoint{1.386139in}{1.698499in}}%
\pgfpathclose%
\pgfusepath{fill}%
\end{pgfscope}%
\begin{pgfscope}%
\pgfpathrectangle{\pgfqpoint{0.697024in}{0.857143in}}{\pgfqpoint{2.627103in}{1.813434in}}%
\pgfusepath{clip}%
\pgfsetbuttcap%
\pgfsetmiterjoin%
\definecolor{currentfill}{rgb}{0.992771,0.707689,0.712380}%
\pgfsetfillcolor{currentfill}%
\pgfsetlinewidth{0.000000pt}%
\definecolor{currentstroke}{rgb}{0.000000,0.000000,0.000000}%
\pgfsetstrokecolor{currentstroke}%
\pgfsetstrokeopacity{0.000000}%
\pgfsetdash{}{0pt}%
\pgfpathmoveto{\pgfqpoint{1.397310in}{1.751475in}}%
\pgfpathlineto{\pgfqpoint{1.406246in}{1.751475in}}%
\pgfpathlineto{\pgfqpoint{1.406246in}{1.734239in}}%
\pgfpathlineto{\pgfqpoint{1.397310in}{1.734239in}}%
\pgfpathlineto{\pgfqpoint{1.397310in}{1.751475in}}%
\pgfpathclose%
\pgfusepath{fill}%
\end{pgfscope}%
\begin{pgfscope}%
\pgfpathrectangle{\pgfqpoint{0.697024in}{0.857143in}}{\pgfqpoint{2.627103in}{1.813434in}}%
\pgfusepath{clip}%
\pgfsetbuttcap%
\pgfsetmiterjoin%
\definecolor{currentfill}{rgb}{0.992771,0.707689,0.712380}%
\pgfsetfillcolor{currentfill}%
\pgfsetlinewidth{0.000000pt}%
\definecolor{currentstroke}{rgb}{0.000000,0.000000,0.000000}%
\pgfsetstrokecolor{currentstroke}%
\pgfsetstrokeopacity{0.000000}%
\pgfsetdash{}{0pt}%
\pgfpathmoveto{\pgfqpoint{1.408480in}{1.663964in}}%
\pgfpathlineto{\pgfqpoint{1.417417in}{1.663964in}}%
\pgfpathlineto{\pgfqpoint{1.417417in}{1.638570in}}%
\pgfpathlineto{\pgfqpoint{1.408480in}{1.638570in}}%
\pgfpathlineto{\pgfqpoint{1.408480in}{1.663964in}}%
\pgfpathclose%
\pgfusepath{fill}%
\end{pgfscope}%
\begin{pgfscope}%
\pgfpathrectangle{\pgfqpoint{0.697024in}{0.857143in}}{\pgfqpoint{2.627103in}{1.813434in}}%
\pgfusepath{clip}%
\pgfsetbuttcap%
\pgfsetmiterjoin%
\definecolor{currentfill}{rgb}{0.992771,0.707689,0.712380}%
\pgfsetfillcolor{currentfill}%
\pgfsetlinewidth{0.000000pt}%
\definecolor{currentstroke}{rgb}{0.000000,0.000000,0.000000}%
\pgfsetstrokecolor{currentstroke}%
\pgfsetstrokeopacity{0.000000}%
\pgfsetdash{}{0pt}%
\pgfpathmoveto{\pgfqpoint{1.419651in}{1.708717in}}%
\pgfpathlineto{\pgfqpoint{1.428587in}{1.708717in}}%
\pgfpathlineto{\pgfqpoint{1.428587in}{1.666727in}}%
\pgfpathlineto{\pgfqpoint{1.419651in}{1.666727in}}%
\pgfpathlineto{\pgfqpoint{1.419651in}{1.708717in}}%
\pgfpathclose%
\pgfusepath{fill}%
\end{pgfscope}%
\begin{pgfscope}%
\pgfpathrectangle{\pgfqpoint{0.697024in}{0.857143in}}{\pgfqpoint{2.627103in}{1.813434in}}%
\pgfusepath{clip}%
\pgfsetbuttcap%
\pgfsetmiterjoin%
\definecolor{currentfill}{rgb}{0.992771,0.707689,0.712380}%
\pgfsetfillcolor{currentfill}%
\pgfsetlinewidth{0.000000pt}%
\definecolor{currentstroke}{rgb}{0.000000,0.000000,0.000000}%
\pgfsetstrokecolor{currentstroke}%
\pgfsetstrokeopacity{0.000000}%
\pgfsetdash{}{0pt}%
\pgfpathmoveto{\pgfqpoint{1.430821in}{1.636759in}}%
\pgfpathlineto{\pgfqpoint{1.439758in}{1.636759in}}%
\pgfpathlineto{\pgfqpoint{1.439758in}{1.595847in}}%
\pgfpathlineto{\pgfqpoint{1.430821in}{1.595847in}}%
\pgfpathlineto{\pgfqpoint{1.430821in}{1.636759in}}%
\pgfpathclose%
\pgfusepath{fill}%
\end{pgfscope}%
\begin{pgfscope}%
\pgfpathrectangle{\pgfqpoint{0.697024in}{0.857143in}}{\pgfqpoint{2.627103in}{1.813434in}}%
\pgfusepath{clip}%
\pgfsetbuttcap%
\pgfsetmiterjoin%
\definecolor{currentfill}{rgb}{0.992771,0.707689,0.712380}%
\pgfsetfillcolor{currentfill}%
\pgfsetlinewidth{0.000000pt}%
\definecolor{currentstroke}{rgb}{0.000000,0.000000,0.000000}%
\pgfsetstrokecolor{currentstroke}%
\pgfsetstrokeopacity{0.000000}%
\pgfsetdash{}{0pt}%
\pgfpathmoveto{\pgfqpoint{1.441992in}{1.656603in}}%
\pgfpathlineto{\pgfqpoint{1.450929in}{1.656603in}}%
\pgfpathlineto{\pgfqpoint{1.450929in}{1.602167in}}%
\pgfpathlineto{\pgfqpoint{1.441992in}{1.602167in}}%
\pgfpathlineto{\pgfqpoint{1.441992in}{1.656603in}}%
\pgfpathclose%
\pgfusepath{fill}%
\end{pgfscope}%
\begin{pgfscope}%
\pgfpathrectangle{\pgfqpoint{0.697024in}{0.857143in}}{\pgfqpoint{2.627103in}{1.813434in}}%
\pgfusepath{clip}%
\pgfsetbuttcap%
\pgfsetmiterjoin%
\definecolor{currentfill}{rgb}{0.992771,0.707689,0.712380}%
\pgfsetfillcolor{currentfill}%
\pgfsetlinewidth{0.000000pt}%
\definecolor{currentstroke}{rgb}{0.000000,0.000000,0.000000}%
\pgfsetstrokecolor{currentstroke}%
\pgfsetstrokeopacity{0.000000}%
\pgfsetdash{}{0pt}%
\pgfpathmoveto{\pgfqpoint{1.453163in}{1.620018in}}%
\pgfpathlineto{\pgfqpoint{1.462099in}{1.620018in}}%
\pgfpathlineto{\pgfqpoint{1.462099in}{1.618757in}}%
\pgfpathlineto{\pgfqpoint{1.453163in}{1.618757in}}%
\pgfpathlineto{\pgfqpoint{1.453163in}{1.620018in}}%
\pgfpathclose%
\pgfusepath{fill}%
\end{pgfscope}%
\begin{pgfscope}%
\pgfpathrectangle{\pgfqpoint{0.697024in}{0.857143in}}{\pgfqpoint{2.627103in}{1.813434in}}%
\pgfusepath{clip}%
\pgfsetbuttcap%
\pgfsetmiterjoin%
\definecolor{currentfill}{rgb}{0.992771,0.707689,0.712380}%
\pgfsetfillcolor{currentfill}%
\pgfsetlinewidth{0.000000pt}%
\definecolor{currentstroke}{rgb}{0.000000,0.000000,0.000000}%
\pgfsetstrokecolor{currentstroke}%
\pgfsetstrokeopacity{0.000000}%
\pgfsetdash{}{0pt}%
\pgfpathmoveto{\pgfqpoint{1.464333in}{1.585146in}}%
\pgfpathlineto{\pgfqpoint{1.473270in}{1.585146in}}%
\pgfpathlineto{\pgfqpoint{1.473270in}{1.522052in}}%
\pgfpathlineto{\pgfqpoint{1.464333in}{1.522052in}}%
\pgfpathlineto{\pgfqpoint{1.464333in}{1.585146in}}%
\pgfpathclose%
\pgfusepath{fill}%
\end{pgfscope}%
\begin{pgfscope}%
\pgfpathrectangle{\pgfqpoint{0.697024in}{0.857143in}}{\pgfqpoint{2.627103in}{1.813434in}}%
\pgfusepath{clip}%
\pgfsetbuttcap%
\pgfsetmiterjoin%
\definecolor{currentfill}{rgb}{0.992771,0.707689,0.712380}%
\pgfsetfillcolor{currentfill}%
\pgfsetlinewidth{0.000000pt}%
\definecolor{currentstroke}{rgb}{0.000000,0.000000,0.000000}%
\pgfsetstrokecolor{currentstroke}%
\pgfsetstrokeopacity{0.000000}%
\pgfsetdash{}{0pt}%
\pgfpathmoveto{\pgfqpoint{1.475504in}{1.623228in}}%
\pgfpathlineto{\pgfqpoint{1.484440in}{1.623228in}}%
\pgfpathlineto{\pgfqpoint{1.484440in}{1.489285in}}%
\pgfpathlineto{\pgfqpoint{1.475504in}{1.489285in}}%
\pgfpathlineto{\pgfqpoint{1.475504in}{1.623228in}}%
\pgfpathclose%
\pgfusepath{fill}%
\end{pgfscope}%
\begin{pgfscope}%
\pgfpathrectangle{\pgfqpoint{0.697024in}{0.857143in}}{\pgfqpoint{2.627103in}{1.813434in}}%
\pgfusepath{clip}%
\pgfsetbuttcap%
\pgfsetmiterjoin%
\definecolor{currentfill}{rgb}{0.992771,0.707689,0.712380}%
\pgfsetfillcolor{currentfill}%
\pgfsetlinewidth{0.000000pt}%
\definecolor{currentstroke}{rgb}{0.000000,0.000000,0.000000}%
\pgfsetstrokecolor{currentstroke}%
\pgfsetstrokeopacity{0.000000}%
\pgfsetdash{}{0pt}%
\pgfpathmoveto{\pgfqpoint{1.486674in}{1.564069in}}%
\pgfpathlineto{\pgfqpoint{1.495611in}{1.564069in}}%
\pgfpathlineto{\pgfqpoint{1.495611in}{1.477373in}}%
\pgfpathlineto{\pgfqpoint{1.486674in}{1.477373in}}%
\pgfpathlineto{\pgfqpoint{1.486674in}{1.564069in}}%
\pgfpathclose%
\pgfusepath{fill}%
\end{pgfscope}%
\begin{pgfscope}%
\pgfpathrectangle{\pgfqpoint{0.697024in}{0.857143in}}{\pgfqpoint{2.627103in}{1.813434in}}%
\pgfusepath{clip}%
\pgfsetbuttcap%
\pgfsetmiterjoin%
\definecolor{currentfill}{rgb}{0.992771,0.707689,0.712380}%
\pgfsetfillcolor{currentfill}%
\pgfsetlinewidth{0.000000pt}%
\definecolor{currentstroke}{rgb}{0.000000,0.000000,0.000000}%
\pgfsetstrokecolor{currentstroke}%
\pgfsetstrokeopacity{0.000000}%
\pgfsetdash{}{0pt}%
\pgfpathmoveto{\pgfqpoint{1.497845in}{1.623005in}}%
\pgfpathlineto{\pgfqpoint{1.506782in}{1.623005in}}%
\pgfpathlineto{\pgfqpoint{1.506782in}{1.512839in}}%
\pgfpathlineto{\pgfqpoint{1.497845in}{1.512839in}}%
\pgfpathlineto{\pgfqpoint{1.497845in}{1.623005in}}%
\pgfpathclose%
\pgfusepath{fill}%
\end{pgfscope}%
\begin{pgfscope}%
\pgfpathrectangle{\pgfqpoint{0.697024in}{0.857143in}}{\pgfqpoint{2.627103in}{1.813434in}}%
\pgfusepath{clip}%
\pgfsetbuttcap%
\pgfsetmiterjoin%
\definecolor{currentfill}{rgb}{0.992771,0.707689,0.712380}%
\pgfsetfillcolor{currentfill}%
\pgfsetlinewidth{0.000000pt}%
\definecolor{currentstroke}{rgb}{0.000000,0.000000,0.000000}%
\pgfsetstrokecolor{currentstroke}%
\pgfsetstrokeopacity{0.000000}%
\pgfsetdash{}{0pt}%
\pgfpathmoveto{\pgfqpoint{1.509016in}{1.533002in}}%
\pgfpathlineto{\pgfqpoint{1.517952in}{1.533002in}}%
\pgfpathlineto{\pgfqpoint{1.517952in}{1.467478in}}%
\pgfpathlineto{\pgfqpoint{1.509016in}{1.467478in}}%
\pgfpathlineto{\pgfqpoint{1.509016in}{1.533002in}}%
\pgfpathclose%
\pgfusepath{fill}%
\end{pgfscope}%
\begin{pgfscope}%
\pgfpathrectangle{\pgfqpoint{0.697024in}{0.857143in}}{\pgfqpoint{2.627103in}{1.813434in}}%
\pgfusepath{clip}%
\pgfsetbuttcap%
\pgfsetmiterjoin%
\definecolor{currentfill}{rgb}{0.992771,0.707689,0.712380}%
\pgfsetfillcolor{currentfill}%
\pgfsetlinewidth{0.000000pt}%
\definecolor{currentstroke}{rgb}{0.000000,0.000000,0.000000}%
\pgfsetstrokecolor{currentstroke}%
\pgfsetstrokeopacity{0.000000}%
\pgfsetdash{}{0pt}%
\pgfpathmoveto{\pgfqpoint{1.520186in}{1.557686in}}%
\pgfpathlineto{\pgfqpoint{1.529123in}{1.557686in}}%
\pgfpathlineto{\pgfqpoint{1.529123in}{1.467888in}}%
\pgfpathlineto{\pgfqpoint{1.520186in}{1.467888in}}%
\pgfpathlineto{\pgfqpoint{1.520186in}{1.557686in}}%
\pgfpathclose%
\pgfusepath{fill}%
\end{pgfscope}%
\begin{pgfscope}%
\pgfpathrectangle{\pgfqpoint{0.697024in}{0.857143in}}{\pgfqpoint{2.627103in}{1.813434in}}%
\pgfusepath{clip}%
\pgfsetbuttcap%
\pgfsetmiterjoin%
\definecolor{currentfill}{rgb}{0.992771,0.707689,0.712380}%
\pgfsetfillcolor{currentfill}%
\pgfsetlinewidth{0.000000pt}%
\definecolor{currentstroke}{rgb}{0.000000,0.000000,0.000000}%
\pgfsetstrokecolor{currentstroke}%
\pgfsetstrokeopacity{0.000000}%
\pgfsetdash{}{0pt}%
\pgfpathmoveto{\pgfqpoint{1.531357in}{1.542582in}}%
\pgfpathlineto{\pgfqpoint{1.540293in}{1.542582in}}%
\pgfpathlineto{\pgfqpoint{1.540293in}{1.477310in}}%
\pgfpathlineto{\pgfqpoint{1.531357in}{1.477310in}}%
\pgfpathlineto{\pgfqpoint{1.531357in}{1.542582in}}%
\pgfpathclose%
\pgfusepath{fill}%
\end{pgfscope}%
\begin{pgfscope}%
\pgfpathrectangle{\pgfqpoint{0.697024in}{0.857143in}}{\pgfqpoint{2.627103in}{1.813434in}}%
\pgfusepath{clip}%
\pgfsetbuttcap%
\pgfsetmiterjoin%
\definecolor{currentfill}{rgb}{0.992771,0.707689,0.712380}%
\pgfsetfillcolor{currentfill}%
\pgfsetlinewidth{0.000000pt}%
\definecolor{currentstroke}{rgb}{0.000000,0.000000,0.000000}%
\pgfsetstrokecolor{currentstroke}%
\pgfsetstrokeopacity{0.000000}%
\pgfsetdash{}{0pt}%
\pgfpathmoveto{\pgfqpoint{1.542528in}{1.572041in}}%
\pgfpathlineto{\pgfqpoint{1.551464in}{1.572041in}}%
\pgfpathlineto{\pgfqpoint{1.551464in}{1.509254in}}%
\pgfpathlineto{\pgfqpoint{1.542528in}{1.509254in}}%
\pgfpathlineto{\pgfqpoint{1.542528in}{1.572041in}}%
\pgfpathclose%
\pgfusepath{fill}%
\end{pgfscope}%
\begin{pgfscope}%
\pgfpathrectangle{\pgfqpoint{0.697024in}{0.857143in}}{\pgfqpoint{2.627103in}{1.813434in}}%
\pgfusepath{clip}%
\pgfsetbuttcap%
\pgfsetmiterjoin%
\definecolor{currentfill}{rgb}{0.992771,0.707689,0.712380}%
\pgfsetfillcolor{currentfill}%
\pgfsetlinewidth{0.000000pt}%
\definecolor{currentstroke}{rgb}{0.000000,0.000000,0.000000}%
\pgfsetstrokecolor{currentstroke}%
\pgfsetstrokeopacity{0.000000}%
\pgfsetdash{}{0pt}%
\pgfpathmoveto{\pgfqpoint{1.553698in}{1.594586in}}%
\pgfpathlineto{\pgfqpoint{1.562635in}{1.594586in}}%
\pgfpathlineto{\pgfqpoint{1.562635in}{1.529514in}}%
\pgfpathlineto{\pgfqpoint{1.553698in}{1.529514in}}%
\pgfpathlineto{\pgfqpoint{1.553698in}{1.594586in}}%
\pgfpathclose%
\pgfusepath{fill}%
\end{pgfscope}%
\begin{pgfscope}%
\pgfpathrectangle{\pgfqpoint{0.697024in}{0.857143in}}{\pgfqpoint{2.627103in}{1.813434in}}%
\pgfusepath{clip}%
\pgfsetbuttcap%
\pgfsetmiterjoin%
\definecolor{currentfill}{rgb}{0.992771,0.707689,0.712380}%
\pgfsetfillcolor{currentfill}%
\pgfsetlinewidth{0.000000pt}%
\definecolor{currentstroke}{rgb}{0.000000,0.000000,0.000000}%
\pgfsetstrokecolor{currentstroke}%
\pgfsetstrokeopacity{0.000000}%
\pgfsetdash{}{0pt}%
\pgfpathmoveto{\pgfqpoint{1.564869in}{1.613734in}}%
\pgfpathlineto{\pgfqpoint{1.573805in}{1.613734in}}%
\pgfpathlineto{\pgfqpoint{1.573805in}{1.555937in}}%
\pgfpathlineto{\pgfqpoint{1.564869in}{1.555937in}}%
\pgfpathlineto{\pgfqpoint{1.564869in}{1.613734in}}%
\pgfpathclose%
\pgfusepath{fill}%
\end{pgfscope}%
\begin{pgfscope}%
\pgfpathrectangle{\pgfqpoint{0.697024in}{0.857143in}}{\pgfqpoint{2.627103in}{1.813434in}}%
\pgfusepath{clip}%
\pgfsetbuttcap%
\pgfsetmiterjoin%
\definecolor{currentfill}{rgb}{0.992771,0.707689,0.712380}%
\pgfsetfillcolor{currentfill}%
\pgfsetlinewidth{0.000000pt}%
\definecolor{currentstroke}{rgb}{0.000000,0.000000,0.000000}%
\pgfsetstrokecolor{currentstroke}%
\pgfsetstrokeopacity{0.000000}%
\pgfsetdash{}{0pt}%
\pgfpathmoveto{\pgfqpoint{1.576039in}{1.681604in}}%
\pgfpathlineto{\pgfqpoint{1.584976in}{1.681604in}}%
\pgfpathlineto{\pgfqpoint{1.584976in}{1.615245in}}%
\pgfpathlineto{\pgfqpoint{1.576039in}{1.615245in}}%
\pgfpathlineto{\pgfqpoint{1.576039in}{1.681604in}}%
\pgfpathclose%
\pgfusepath{fill}%
\end{pgfscope}%
\begin{pgfscope}%
\pgfpathrectangle{\pgfqpoint{0.697024in}{0.857143in}}{\pgfqpoint{2.627103in}{1.813434in}}%
\pgfusepath{clip}%
\pgfsetbuttcap%
\pgfsetmiterjoin%
\definecolor{currentfill}{rgb}{0.992771,0.707689,0.712380}%
\pgfsetfillcolor{currentfill}%
\pgfsetlinewidth{0.000000pt}%
\definecolor{currentstroke}{rgb}{0.000000,0.000000,0.000000}%
\pgfsetstrokecolor{currentstroke}%
\pgfsetstrokeopacity{0.000000}%
\pgfsetdash{}{0pt}%
\pgfpathmoveto{\pgfqpoint{1.587210in}{1.716025in}}%
\pgfpathlineto{\pgfqpoint{1.596146in}{1.716025in}}%
\pgfpathlineto{\pgfqpoint{1.596146in}{1.618152in}}%
\pgfpathlineto{\pgfqpoint{1.587210in}{1.618152in}}%
\pgfpathlineto{\pgfqpoint{1.587210in}{1.716025in}}%
\pgfpathclose%
\pgfusepath{fill}%
\end{pgfscope}%
\begin{pgfscope}%
\pgfpathrectangle{\pgfqpoint{0.697024in}{0.857143in}}{\pgfqpoint{2.627103in}{1.813434in}}%
\pgfusepath{clip}%
\pgfsetbuttcap%
\pgfsetmiterjoin%
\definecolor{currentfill}{rgb}{0.992771,0.707689,0.712380}%
\pgfsetfillcolor{currentfill}%
\pgfsetlinewidth{0.000000pt}%
\definecolor{currentstroke}{rgb}{0.000000,0.000000,0.000000}%
\pgfsetstrokecolor{currentstroke}%
\pgfsetstrokeopacity{0.000000}%
\pgfsetdash{}{0pt}%
\pgfpathmoveto{\pgfqpoint{1.598381in}{1.673571in}}%
\pgfpathlineto{\pgfqpoint{1.607317in}{1.673571in}}%
\pgfpathlineto{\pgfqpoint{1.607317in}{1.598488in}}%
\pgfpathlineto{\pgfqpoint{1.598381in}{1.598488in}}%
\pgfpathlineto{\pgfqpoint{1.598381in}{1.673571in}}%
\pgfpathclose%
\pgfusepath{fill}%
\end{pgfscope}%
\begin{pgfscope}%
\pgfpathrectangle{\pgfqpoint{0.697024in}{0.857143in}}{\pgfqpoint{2.627103in}{1.813434in}}%
\pgfusepath{clip}%
\pgfsetbuttcap%
\pgfsetmiterjoin%
\definecolor{currentfill}{rgb}{0.992771,0.707689,0.712380}%
\pgfsetfillcolor{currentfill}%
\pgfsetlinewidth{0.000000pt}%
\definecolor{currentstroke}{rgb}{0.000000,0.000000,0.000000}%
\pgfsetstrokecolor{currentstroke}%
\pgfsetstrokeopacity{0.000000}%
\pgfsetdash{}{0pt}%
\pgfpathmoveto{\pgfqpoint{1.609551in}{1.677358in}}%
\pgfpathlineto{\pgfqpoint{1.618488in}{1.677358in}}%
\pgfpathlineto{\pgfqpoint{1.618488in}{1.585025in}}%
\pgfpathlineto{\pgfqpoint{1.609551in}{1.585025in}}%
\pgfpathlineto{\pgfqpoint{1.609551in}{1.677358in}}%
\pgfpathclose%
\pgfusepath{fill}%
\end{pgfscope}%
\begin{pgfscope}%
\pgfpathrectangle{\pgfqpoint{0.697024in}{0.857143in}}{\pgfqpoint{2.627103in}{1.813434in}}%
\pgfusepath{clip}%
\pgfsetbuttcap%
\pgfsetmiterjoin%
\definecolor{currentfill}{rgb}{0.992771,0.707689,0.712380}%
\pgfsetfillcolor{currentfill}%
\pgfsetlinewidth{0.000000pt}%
\definecolor{currentstroke}{rgb}{0.000000,0.000000,0.000000}%
\pgfsetstrokecolor{currentstroke}%
\pgfsetstrokeopacity{0.000000}%
\pgfsetdash{}{0pt}%
\pgfpathmoveto{\pgfqpoint{1.620722in}{1.637049in}}%
\pgfpathlineto{\pgfqpoint{1.629658in}{1.637049in}}%
\pgfpathlineto{\pgfqpoint{1.629658in}{1.547836in}}%
\pgfpathlineto{\pgfqpoint{1.620722in}{1.547836in}}%
\pgfpathlineto{\pgfqpoint{1.620722in}{1.637049in}}%
\pgfpathclose%
\pgfusepath{fill}%
\end{pgfscope}%
\begin{pgfscope}%
\pgfpathrectangle{\pgfqpoint{0.697024in}{0.857143in}}{\pgfqpoint{2.627103in}{1.813434in}}%
\pgfusepath{clip}%
\pgfsetbuttcap%
\pgfsetmiterjoin%
\definecolor{currentfill}{rgb}{0.992771,0.707689,0.712380}%
\pgfsetfillcolor{currentfill}%
\pgfsetlinewidth{0.000000pt}%
\definecolor{currentstroke}{rgb}{0.000000,0.000000,0.000000}%
\pgfsetstrokecolor{currentstroke}%
\pgfsetstrokeopacity{0.000000}%
\pgfsetdash{}{0pt}%
\pgfpathmoveto{\pgfqpoint{1.631892in}{1.613716in}}%
\pgfpathlineto{\pgfqpoint{1.640829in}{1.613716in}}%
\pgfpathlineto{\pgfqpoint{1.640829in}{1.530328in}}%
\pgfpathlineto{\pgfqpoint{1.631892in}{1.530328in}}%
\pgfpathlineto{\pgfqpoint{1.631892in}{1.613716in}}%
\pgfpathclose%
\pgfusepath{fill}%
\end{pgfscope}%
\begin{pgfscope}%
\pgfpathrectangle{\pgfqpoint{0.697024in}{0.857143in}}{\pgfqpoint{2.627103in}{1.813434in}}%
\pgfusepath{clip}%
\pgfsetbuttcap%
\pgfsetmiterjoin%
\definecolor{currentfill}{rgb}{0.992771,0.707689,0.712380}%
\pgfsetfillcolor{currentfill}%
\pgfsetlinewidth{0.000000pt}%
\definecolor{currentstroke}{rgb}{0.000000,0.000000,0.000000}%
\pgfsetstrokecolor{currentstroke}%
\pgfsetstrokeopacity{0.000000}%
\pgfsetdash{}{0pt}%
\pgfpathmoveto{\pgfqpoint{1.643063in}{1.606278in}}%
\pgfpathlineto{\pgfqpoint{1.651999in}{1.606278in}}%
\pgfpathlineto{\pgfqpoint{1.651999in}{1.530061in}}%
\pgfpathlineto{\pgfqpoint{1.643063in}{1.530061in}}%
\pgfpathlineto{\pgfqpoint{1.643063in}{1.606278in}}%
\pgfpathclose%
\pgfusepath{fill}%
\end{pgfscope}%
\begin{pgfscope}%
\pgfpathrectangle{\pgfqpoint{0.697024in}{0.857143in}}{\pgfqpoint{2.627103in}{1.813434in}}%
\pgfusepath{clip}%
\pgfsetbuttcap%
\pgfsetmiterjoin%
\definecolor{currentfill}{rgb}{0.992771,0.707689,0.712380}%
\pgfsetfillcolor{currentfill}%
\pgfsetlinewidth{0.000000pt}%
\definecolor{currentstroke}{rgb}{0.000000,0.000000,0.000000}%
\pgfsetstrokecolor{currentstroke}%
\pgfsetstrokeopacity{0.000000}%
\pgfsetdash{}{0pt}%
\pgfpathmoveto{\pgfqpoint{1.654234in}{1.645488in}}%
\pgfpathlineto{\pgfqpoint{1.663170in}{1.645488in}}%
\pgfpathlineto{\pgfqpoint{1.663170in}{1.575997in}}%
\pgfpathlineto{\pgfqpoint{1.654234in}{1.575997in}}%
\pgfpathlineto{\pgfqpoint{1.654234in}{1.645488in}}%
\pgfpathclose%
\pgfusepath{fill}%
\end{pgfscope}%
\begin{pgfscope}%
\pgfpathrectangle{\pgfqpoint{0.697024in}{0.857143in}}{\pgfqpoint{2.627103in}{1.813434in}}%
\pgfusepath{clip}%
\pgfsetbuttcap%
\pgfsetmiterjoin%
\definecolor{currentfill}{rgb}{0.992771,0.707689,0.712380}%
\pgfsetfillcolor{currentfill}%
\pgfsetlinewidth{0.000000pt}%
\definecolor{currentstroke}{rgb}{0.000000,0.000000,0.000000}%
\pgfsetstrokecolor{currentstroke}%
\pgfsetstrokeopacity{0.000000}%
\pgfsetdash{}{0pt}%
\pgfpathmoveto{\pgfqpoint{1.665404in}{1.635301in}}%
\pgfpathlineto{\pgfqpoint{1.674341in}{1.635301in}}%
\pgfpathlineto{\pgfqpoint{1.674341in}{1.549763in}}%
\pgfpathlineto{\pgfqpoint{1.665404in}{1.549763in}}%
\pgfpathlineto{\pgfqpoint{1.665404in}{1.635301in}}%
\pgfpathclose%
\pgfusepath{fill}%
\end{pgfscope}%
\begin{pgfscope}%
\pgfpathrectangle{\pgfqpoint{0.697024in}{0.857143in}}{\pgfqpoint{2.627103in}{1.813434in}}%
\pgfusepath{clip}%
\pgfsetbuttcap%
\pgfsetmiterjoin%
\definecolor{currentfill}{rgb}{0.992771,0.707689,0.712380}%
\pgfsetfillcolor{currentfill}%
\pgfsetlinewidth{0.000000pt}%
\definecolor{currentstroke}{rgb}{0.000000,0.000000,0.000000}%
\pgfsetstrokecolor{currentstroke}%
\pgfsetstrokeopacity{0.000000}%
\pgfsetdash{}{0pt}%
\pgfpathmoveto{\pgfqpoint{1.676575in}{1.596146in}}%
\pgfpathlineto{\pgfqpoint{1.685511in}{1.596146in}}%
\pgfpathlineto{\pgfqpoint{1.685511in}{1.546078in}}%
\pgfpathlineto{\pgfqpoint{1.676575in}{1.546078in}}%
\pgfpathlineto{\pgfqpoint{1.676575in}{1.596146in}}%
\pgfpathclose%
\pgfusepath{fill}%
\end{pgfscope}%
\begin{pgfscope}%
\pgfpathrectangle{\pgfqpoint{0.697024in}{0.857143in}}{\pgfqpoint{2.627103in}{1.813434in}}%
\pgfusepath{clip}%
\pgfsetbuttcap%
\pgfsetmiterjoin%
\definecolor{currentfill}{rgb}{0.992771,0.707689,0.712380}%
\pgfsetfillcolor{currentfill}%
\pgfsetlinewidth{0.000000pt}%
\definecolor{currentstroke}{rgb}{0.000000,0.000000,0.000000}%
\pgfsetstrokecolor{currentstroke}%
\pgfsetstrokeopacity{0.000000}%
\pgfsetdash{}{0pt}%
\pgfpathmoveto{\pgfqpoint{1.687745in}{1.652480in}}%
\pgfpathlineto{\pgfqpoint{1.696682in}{1.652480in}}%
\pgfpathlineto{\pgfqpoint{1.696682in}{1.567438in}}%
\pgfpathlineto{\pgfqpoint{1.687745in}{1.567438in}}%
\pgfpathlineto{\pgfqpoint{1.687745in}{1.652480in}}%
\pgfpathclose%
\pgfusepath{fill}%
\end{pgfscope}%
\begin{pgfscope}%
\pgfpathrectangle{\pgfqpoint{0.697024in}{0.857143in}}{\pgfqpoint{2.627103in}{1.813434in}}%
\pgfusepath{clip}%
\pgfsetbuttcap%
\pgfsetmiterjoin%
\definecolor{currentfill}{rgb}{0.992771,0.707689,0.712380}%
\pgfsetfillcolor{currentfill}%
\pgfsetlinewidth{0.000000pt}%
\definecolor{currentstroke}{rgb}{0.000000,0.000000,0.000000}%
\pgfsetstrokecolor{currentstroke}%
\pgfsetstrokeopacity{0.000000}%
\pgfsetdash{}{0pt}%
\pgfpathmoveto{\pgfqpoint{1.698916in}{1.628883in}}%
\pgfpathlineto{\pgfqpoint{1.707852in}{1.628883in}}%
\pgfpathlineto{\pgfqpoint{1.707852in}{1.552497in}}%
\pgfpathlineto{\pgfqpoint{1.698916in}{1.552497in}}%
\pgfpathlineto{\pgfqpoint{1.698916in}{1.628883in}}%
\pgfpathclose%
\pgfusepath{fill}%
\end{pgfscope}%
\begin{pgfscope}%
\pgfpathrectangle{\pgfqpoint{0.697024in}{0.857143in}}{\pgfqpoint{2.627103in}{1.813434in}}%
\pgfusepath{clip}%
\pgfsetbuttcap%
\pgfsetmiterjoin%
\definecolor{currentfill}{rgb}{0.992771,0.707689,0.712380}%
\pgfsetfillcolor{currentfill}%
\pgfsetlinewidth{0.000000pt}%
\definecolor{currentstroke}{rgb}{0.000000,0.000000,0.000000}%
\pgfsetstrokecolor{currentstroke}%
\pgfsetstrokeopacity{0.000000}%
\pgfsetdash{}{0pt}%
\pgfpathmoveto{\pgfqpoint{1.710087in}{1.611267in}}%
\pgfpathlineto{\pgfqpoint{1.719023in}{1.611267in}}%
\pgfpathlineto{\pgfqpoint{1.719023in}{1.544022in}}%
\pgfpathlineto{\pgfqpoint{1.710087in}{1.544022in}}%
\pgfpathlineto{\pgfqpoint{1.710087in}{1.611267in}}%
\pgfpathclose%
\pgfusepath{fill}%
\end{pgfscope}%
\begin{pgfscope}%
\pgfpathrectangle{\pgfqpoint{0.697024in}{0.857143in}}{\pgfqpoint{2.627103in}{1.813434in}}%
\pgfusepath{clip}%
\pgfsetbuttcap%
\pgfsetmiterjoin%
\definecolor{currentfill}{rgb}{0.992771,0.707689,0.712380}%
\pgfsetfillcolor{currentfill}%
\pgfsetlinewidth{0.000000pt}%
\definecolor{currentstroke}{rgb}{0.000000,0.000000,0.000000}%
\pgfsetstrokecolor{currentstroke}%
\pgfsetstrokeopacity{0.000000}%
\pgfsetdash{}{0pt}%
\pgfpathmoveto{\pgfqpoint{1.721257in}{1.597114in}}%
\pgfpathlineto{\pgfqpoint{1.730194in}{1.597114in}}%
\pgfpathlineto{\pgfqpoint{1.730194in}{1.549897in}}%
\pgfpathlineto{\pgfqpoint{1.721257in}{1.549897in}}%
\pgfpathlineto{\pgfqpoint{1.721257in}{1.597114in}}%
\pgfpathclose%
\pgfusepath{fill}%
\end{pgfscope}%
\begin{pgfscope}%
\pgfpathrectangle{\pgfqpoint{0.697024in}{0.857143in}}{\pgfqpoint{2.627103in}{1.813434in}}%
\pgfusepath{clip}%
\pgfsetbuttcap%
\pgfsetmiterjoin%
\definecolor{currentfill}{rgb}{0.992771,0.707689,0.712380}%
\pgfsetfillcolor{currentfill}%
\pgfsetlinewidth{0.000000pt}%
\definecolor{currentstroke}{rgb}{0.000000,0.000000,0.000000}%
\pgfsetstrokecolor{currentstroke}%
\pgfsetstrokeopacity{0.000000}%
\pgfsetdash{}{0pt}%
\pgfpathmoveto{\pgfqpoint{1.732428in}{1.587705in}}%
\pgfpathlineto{\pgfqpoint{1.741364in}{1.587705in}}%
\pgfpathlineto{\pgfqpoint{1.741364in}{1.507252in}}%
\pgfpathlineto{\pgfqpoint{1.732428in}{1.507252in}}%
\pgfpathlineto{\pgfqpoint{1.732428in}{1.587705in}}%
\pgfpathclose%
\pgfusepath{fill}%
\end{pgfscope}%
\begin{pgfscope}%
\pgfpathrectangle{\pgfqpoint{0.697024in}{0.857143in}}{\pgfqpoint{2.627103in}{1.813434in}}%
\pgfusepath{clip}%
\pgfsetbuttcap%
\pgfsetmiterjoin%
\definecolor{currentfill}{rgb}{0.992771,0.707689,0.712380}%
\pgfsetfillcolor{currentfill}%
\pgfsetlinewidth{0.000000pt}%
\definecolor{currentstroke}{rgb}{0.000000,0.000000,0.000000}%
\pgfsetstrokecolor{currentstroke}%
\pgfsetstrokeopacity{0.000000}%
\pgfsetdash{}{0pt}%
\pgfpathmoveto{\pgfqpoint{1.743598in}{1.572511in}}%
\pgfpathlineto{\pgfqpoint{1.752535in}{1.572511in}}%
\pgfpathlineto{\pgfqpoint{1.752535in}{1.523207in}}%
\pgfpathlineto{\pgfqpoint{1.743598in}{1.523207in}}%
\pgfpathlineto{\pgfqpoint{1.743598in}{1.572511in}}%
\pgfpathclose%
\pgfusepath{fill}%
\end{pgfscope}%
\begin{pgfscope}%
\pgfpathrectangle{\pgfqpoint{0.697024in}{0.857143in}}{\pgfqpoint{2.627103in}{1.813434in}}%
\pgfusepath{clip}%
\pgfsetbuttcap%
\pgfsetmiterjoin%
\definecolor{currentfill}{rgb}{0.992771,0.707689,0.712380}%
\pgfsetfillcolor{currentfill}%
\pgfsetlinewidth{0.000000pt}%
\definecolor{currentstroke}{rgb}{0.000000,0.000000,0.000000}%
\pgfsetstrokecolor{currentstroke}%
\pgfsetstrokeopacity{0.000000}%
\pgfsetdash{}{0pt}%
\pgfpathmoveto{\pgfqpoint{1.754769in}{1.524193in}}%
\pgfpathlineto{\pgfqpoint{1.763705in}{1.524193in}}%
\pgfpathlineto{\pgfqpoint{1.763705in}{1.466259in}}%
\pgfpathlineto{\pgfqpoint{1.754769in}{1.466259in}}%
\pgfpathlineto{\pgfqpoint{1.754769in}{1.524193in}}%
\pgfpathclose%
\pgfusepath{fill}%
\end{pgfscope}%
\begin{pgfscope}%
\pgfpathrectangle{\pgfqpoint{0.697024in}{0.857143in}}{\pgfqpoint{2.627103in}{1.813434in}}%
\pgfusepath{clip}%
\pgfsetbuttcap%
\pgfsetmiterjoin%
\definecolor{currentfill}{rgb}{0.992771,0.707689,0.712380}%
\pgfsetfillcolor{currentfill}%
\pgfsetlinewidth{0.000000pt}%
\definecolor{currentstroke}{rgb}{0.000000,0.000000,0.000000}%
\pgfsetstrokecolor{currentstroke}%
\pgfsetstrokeopacity{0.000000}%
\pgfsetdash{}{0pt}%
\pgfpathmoveto{\pgfqpoint{1.765940in}{1.560547in}}%
\pgfpathlineto{\pgfqpoint{1.774876in}{1.560547in}}%
\pgfpathlineto{\pgfqpoint{1.774876in}{1.500969in}}%
\pgfpathlineto{\pgfqpoint{1.765940in}{1.500969in}}%
\pgfpathlineto{\pgfqpoint{1.765940in}{1.560547in}}%
\pgfpathclose%
\pgfusepath{fill}%
\end{pgfscope}%
\begin{pgfscope}%
\pgfpathrectangle{\pgfqpoint{0.697024in}{0.857143in}}{\pgfqpoint{2.627103in}{1.813434in}}%
\pgfusepath{clip}%
\pgfsetbuttcap%
\pgfsetmiterjoin%
\definecolor{currentfill}{rgb}{0.992771,0.707689,0.712380}%
\pgfsetfillcolor{currentfill}%
\pgfsetlinewidth{0.000000pt}%
\definecolor{currentstroke}{rgb}{0.000000,0.000000,0.000000}%
\pgfsetstrokecolor{currentstroke}%
\pgfsetstrokeopacity{0.000000}%
\pgfsetdash{}{0pt}%
\pgfpathmoveto{\pgfqpoint{1.777110in}{1.548795in}}%
\pgfpathlineto{\pgfqpoint{1.786047in}{1.548795in}}%
\pgfpathlineto{\pgfqpoint{1.786047in}{1.491012in}}%
\pgfpathlineto{\pgfqpoint{1.777110in}{1.491012in}}%
\pgfpathlineto{\pgfqpoint{1.777110in}{1.548795in}}%
\pgfpathclose%
\pgfusepath{fill}%
\end{pgfscope}%
\begin{pgfscope}%
\pgfpathrectangle{\pgfqpoint{0.697024in}{0.857143in}}{\pgfqpoint{2.627103in}{1.813434in}}%
\pgfusepath{clip}%
\pgfsetbuttcap%
\pgfsetmiterjoin%
\definecolor{currentfill}{rgb}{0.992771,0.707689,0.712380}%
\pgfsetfillcolor{currentfill}%
\pgfsetlinewidth{0.000000pt}%
\definecolor{currentstroke}{rgb}{0.000000,0.000000,0.000000}%
\pgfsetstrokecolor{currentstroke}%
\pgfsetstrokeopacity{0.000000}%
\pgfsetdash{}{0pt}%
\pgfpathmoveto{\pgfqpoint{1.788281in}{1.485741in}}%
\pgfpathlineto{\pgfqpoint{1.797217in}{1.485741in}}%
\pgfpathlineto{\pgfqpoint{1.797217in}{1.429395in}}%
\pgfpathlineto{\pgfqpoint{1.788281in}{1.429395in}}%
\pgfpathlineto{\pgfqpoint{1.788281in}{1.485741in}}%
\pgfpathclose%
\pgfusepath{fill}%
\end{pgfscope}%
\begin{pgfscope}%
\pgfpathrectangle{\pgfqpoint{0.697024in}{0.857143in}}{\pgfqpoint{2.627103in}{1.813434in}}%
\pgfusepath{clip}%
\pgfsetbuttcap%
\pgfsetmiterjoin%
\definecolor{currentfill}{rgb}{0.992771,0.707689,0.712380}%
\pgfsetfillcolor{currentfill}%
\pgfsetlinewidth{0.000000pt}%
\definecolor{currentstroke}{rgb}{0.000000,0.000000,0.000000}%
\pgfsetstrokecolor{currentstroke}%
\pgfsetstrokeopacity{0.000000}%
\pgfsetdash{}{0pt}%
\pgfpathmoveto{\pgfqpoint{1.799451in}{1.439748in}}%
\pgfpathlineto{\pgfqpoint{1.808388in}{1.439748in}}%
\pgfpathlineto{\pgfqpoint{1.808388in}{1.394031in}}%
\pgfpathlineto{\pgfqpoint{1.799451in}{1.394031in}}%
\pgfpathlineto{\pgfqpoint{1.799451in}{1.439748in}}%
\pgfpathclose%
\pgfusepath{fill}%
\end{pgfscope}%
\begin{pgfscope}%
\pgfpathrectangle{\pgfqpoint{0.697024in}{0.857143in}}{\pgfqpoint{2.627103in}{1.813434in}}%
\pgfusepath{clip}%
\pgfsetbuttcap%
\pgfsetmiterjoin%
\definecolor{currentfill}{rgb}{0.992771,0.707689,0.712380}%
\pgfsetfillcolor{currentfill}%
\pgfsetlinewidth{0.000000pt}%
\definecolor{currentstroke}{rgb}{0.000000,0.000000,0.000000}%
\pgfsetstrokecolor{currentstroke}%
\pgfsetstrokeopacity{0.000000}%
\pgfsetdash{}{0pt}%
\pgfpathmoveto{\pgfqpoint{1.810622in}{1.509357in}}%
\pgfpathlineto{\pgfqpoint{1.819559in}{1.509357in}}%
\pgfpathlineto{\pgfqpoint{1.819559in}{1.453320in}}%
\pgfpathlineto{\pgfqpoint{1.810622in}{1.453320in}}%
\pgfpathlineto{\pgfqpoint{1.810622in}{1.509357in}}%
\pgfpathclose%
\pgfusepath{fill}%
\end{pgfscope}%
\begin{pgfscope}%
\pgfpathrectangle{\pgfqpoint{0.697024in}{0.857143in}}{\pgfqpoint{2.627103in}{1.813434in}}%
\pgfusepath{clip}%
\pgfsetbuttcap%
\pgfsetmiterjoin%
\definecolor{currentfill}{rgb}{0.992771,0.707689,0.712380}%
\pgfsetfillcolor{currentfill}%
\pgfsetlinewidth{0.000000pt}%
\definecolor{currentstroke}{rgb}{0.000000,0.000000,0.000000}%
\pgfsetstrokecolor{currentstroke}%
\pgfsetstrokeopacity{0.000000}%
\pgfsetdash{}{0pt}%
\pgfpathmoveto{\pgfqpoint{1.821793in}{1.461562in}}%
\pgfpathlineto{\pgfqpoint{1.830729in}{1.461562in}}%
\pgfpathlineto{\pgfqpoint{1.830729in}{1.399998in}}%
\pgfpathlineto{\pgfqpoint{1.821793in}{1.399998in}}%
\pgfpathlineto{\pgfqpoint{1.821793in}{1.461562in}}%
\pgfpathclose%
\pgfusepath{fill}%
\end{pgfscope}%
\begin{pgfscope}%
\pgfpathrectangle{\pgfqpoint{0.697024in}{0.857143in}}{\pgfqpoint{2.627103in}{1.813434in}}%
\pgfusepath{clip}%
\pgfsetbuttcap%
\pgfsetmiterjoin%
\definecolor{currentfill}{rgb}{0.992771,0.707689,0.712380}%
\pgfsetfillcolor{currentfill}%
\pgfsetlinewidth{0.000000pt}%
\definecolor{currentstroke}{rgb}{0.000000,0.000000,0.000000}%
\pgfsetstrokecolor{currentstroke}%
\pgfsetstrokeopacity{0.000000}%
\pgfsetdash{}{0pt}%
\pgfpathmoveto{\pgfqpoint{1.832963in}{1.451964in}}%
\pgfpathlineto{\pgfqpoint{1.841900in}{1.451964in}}%
\pgfpathlineto{\pgfqpoint{1.841900in}{1.387479in}}%
\pgfpathlineto{\pgfqpoint{1.832963in}{1.387479in}}%
\pgfpathlineto{\pgfqpoint{1.832963in}{1.451964in}}%
\pgfpathclose%
\pgfusepath{fill}%
\end{pgfscope}%
\begin{pgfscope}%
\pgfpathrectangle{\pgfqpoint{0.697024in}{0.857143in}}{\pgfqpoint{2.627103in}{1.813434in}}%
\pgfusepath{clip}%
\pgfsetbuttcap%
\pgfsetmiterjoin%
\definecolor{currentfill}{rgb}{0.992771,0.707689,0.712380}%
\pgfsetfillcolor{currentfill}%
\pgfsetlinewidth{0.000000pt}%
\definecolor{currentstroke}{rgb}{0.000000,0.000000,0.000000}%
\pgfsetstrokecolor{currentstroke}%
\pgfsetstrokeopacity{0.000000}%
\pgfsetdash{}{0pt}%
\pgfpathmoveto{\pgfqpoint{1.844134in}{1.480404in}}%
\pgfpathlineto{\pgfqpoint{1.853070in}{1.480404in}}%
\pgfpathlineto{\pgfqpoint{1.853070in}{1.402270in}}%
\pgfpathlineto{\pgfqpoint{1.844134in}{1.402270in}}%
\pgfpathlineto{\pgfqpoint{1.844134in}{1.480404in}}%
\pgfpathclose%
\pgfusepath{fill}%
\end{pgfscope}%
\begin{pgfscope}%
\pgfpathrectangle{\pgfqpoint{0.697024in}{0.857143in}}{\pgfqpoint{2.627103in}{1.813434in}}%
\pgfusepath{clip}%
\pgfsetbuttcap%
\pgfsetmiterjoin%
\definecolor{currentfill}{rgb}{0.992771,0.707689,0.712380}%
\pgfsetfillcolor{currentfill}%
\pgfsetlinewidth{0.000000pt}%
\definecolor{currentstroke}{rgb}{0.000000,0.000000,0.000000}%
\pgfsetstrokecolor{currentstroke}%
\pgfsetstrokeopacity{0.000000}%
\pgfsetdash{}{0pt}%
\pgfpathmoveto{\pgfqpoint{1.855304in}{1.478054in}}%
\pgfpathlineto{\pgfqpoint{1.864241in}{1.478054in}}%
\pgfpathlineto{\pgfqpoint{1.864241in}{1.424970in}}%
\pgfpathlineto{\pgfqpoint{1.855304in}{1.424970in}}%
\pgfpathlineto{\pgfqpoint{1.855304in}{1.478054in}}%
\pgfpathclose%
\pgfusepath{fill}%
\end{pgfscope}%
\begin{pgfscope}%
\pgfpathrectangle{\pgfqpoint{0.697024in}{0.857143in}}{\pgfqpoint{2.627103in}{1.813434in}}%
\pgfusepath{clip}%
\pgfsetbuttcap%
\pgfsetmiterjoin%
\definecolor{currentfill}{rgb}{0.992771,0.707689,0.712380}%
\pgfsetfillcolor{currentfill}%
\pgfsetlinewidth{0.000000pt}%
\definecolor{currentstroke}{rgb}{0.000000,0.000000,0.000000}%
\pgfsetstrokecolor{currentstroke}%
\pgfsetstrokeopacity{0.000000}%
\pgfsetdash{}{0pt}%
\pgfpathmoveto{\pgfqpoint{1.866475in}{1.437789in}}%
\pgfpathlineto{\pgfqpoint{1.875412in}{1.437789in}}%
\pgfpathlineto{\pgfqpoint{1.875412in}{1.394294in}}%
\pgfpathlineto{\pgfqpoint{1.866475in}{1.394294in}}%
\pgfpathlineto{\pgfqpoint{1.866475in}{1.437789in}}%
\pgfpathclose%
\pgfusepath{fill}%
\end{pgfscope}%
\begin{pgfscope}%
\pgfpathrectangle{\pgfqpoint{0.697024in}{0.857143in}}{\pgfqpoint{2.627103in}{1.813434in}}%
\pgfusepath{clip}%
\pgfsetbuttcap%
\pgfsetmiterjoin%
\definecolor{currentfill}{rgb}{0.992771,0.707689,0.712380}%
\pgfsetfillcolor{currentfill}%
\pgfsetlinewidth{0.000000pt}%
\definecolor{currentstroke}{rgb}{0.000000,0.000000,0.000000}%
\pgfsetstrokecolor{currentstroke}%
\pgfsetstrokeopacity{0.000000}%
\pgfsetdash{}{0pt}%
\pgfpathmoveto{\pgfqpoint{1.877646in}{1.416682in}}%
\pgfpathlineto{\pgfqpoint{1.886582in}{1.416682in}}%
\pgfpathlineto{\pgfqpoint{1.886582in}{1.364371in}}%
\pgfpathlineto{\pgfqpoint{1.877646in}{1.364371in}}%
\pgfpathlineto{\pgfqpoint{1.877646in}{1.416682in}}%
\pgfpathclose%
\pgfusepath{fill}%
\end{pgfscope}%
\begin{pgfscope}%
\pgfpathrectangle{\pgfqpoint{0.697024in}{0.857143in}}{\pgfqpoint{2.627103in}{1.813434in}}%
\pgfusepath{clip}%
\pgfsetbuttcap%
\pgfsetmiterjoin%
\definecolor{currentfill}{rgb}{0.992771,0.707689,0.712380}%
\pgfsetfillcolor{currentfill}%
\pgfsetlinewidth{0.000000pt}%
\definecolor{currentstroke}{rgb}{0.000000,0.000000,0.000000}%
\pgfsetstrokecolor{currentstroke}%
\pgfsetstrokeopacity{0.000000}%
\pgfsetdash{}{0pt}%
\pgfpathmoveto{\pgfqpoint{1.888816in}{1.445935in}}%
\pgfpathlineto{\pgfqpoint{1.897753in}{1.445935in}}%
\pgfpathlineto{\pgfqpoint{1.897753in}{1.390055in}}%
\pgfpathlineto{\pgfqpoint{1.888816in}{1.390055in}}%
\pgfpathlineto{\pgfqpoint{1.888816in}{1.445935in}}%
\pgfpathclose%
\pgfusepath{fill}%
\end{pgfscope}%
\begin{pgfscope}%
\pgfpathrectangle{\pgfqpoint{0.697024in}{0.857143in}}{\pgfqpoint{2.627103in}{1.813434in}}%
\pgfusepath{clip}%
\pgfsetbuttcap%
\pgfsetmiterjoin%
\definecolor{currentfill}{rgb}{0.992771,0.707689,0.712380}%
\pgfsetfillcolor{currentfill}%
\pgfsetlinewidth{0.000000pt}%
\definecolor{currentstroke}{rgb}{0.000000,0.000000,0.000000}%
\pgfsetstrokecolor{currentstroke}%
\pgfsetstrokeopacity{0.000000}%
\pgfsetdash{}{0pt}%
\pgfpathmoveto{\pgfqpoint{1.899987in}{1.412176in}}%
\pgfpathlineto{\pgfqpoint{1.908923in}{1.412176in}}%
\pgfpathlineto{\pgfqpoint{1.908923in}{1.383247in}}%
\pgfpathlineto{\pgfqpoint{1.899987in}{1.383247in}}%
\pgfpathlineto{\pgfqpoint{1.899987in}{1.412176in}}%
\pgfpathclose%
\pgfusepath{fill}%
\end{pgfscope}%
\begin{pgfscope}%
\pgfpathrectangle{\pgfqpoint{0.697024in}{0.857143in}}{\pgfqpoint{2.627103in}{1.813434in}}%
\pgfusepath{clip}%
\pgfsetbuttcap%
\pgfsetmiterjoin%
\definecolor{currentfill}{rgb}{0.992771,0.707689,0.712380}%
\pgfsetfillcolor{currentfill}%
\pgfsetlinewidth{0.000000pt}%
\definecolor{currentstroke}{rgb}{0.000000,0.000000,0.000000}%
\pgfsetstrokecolor{currentstroke}%
\pgfsetstrokeopacity{0.000000}%
\pgfsetdash{}{0pt}%
\pgfpathmoveto{\pgfqpoint{1.911157in}{1.350132in}}%
\pgfpathlineto{\pgfqpoint{1.920094in}{1.350132in}}%
\pgfpathlineto{\pgfqpoint{1.920094in}{1.301072in}}%
\pgfpathlineto{\pgfqpoint{1.911157in}{1.301072in}}%
\pgfpathlineto{\pgfqpoint{1.911157in}{1.350132in}}%
\pgfpathclose%
\pgfusepath{fill}%
\end{pgfscope}%
\begin{pgfscope}%
\pgfpathrectangle{\pgfqpoint{0.697024in}{0.857143in}}{\pgfqpoint{2.627103in}{1.813434in}}%
\pgfusepath{clip}%
\pgfsetbuttcap%
\pgfsetmiterjoin%
\definecolor{currentfill}{rgb}{0.992771,0.707689,0.712380}%
\pgfsetfillcolor{currentfill}%
\pgfsetlinewidth{0.000000pt}%
\definecolor{currentstroke}{rgb}{0.000000,0.000000,0.000000}%
\pgfsetstrokecolor{currentstroke}%
\pgfsetstrokeopacity{0.000000}%
\pgfsetdash{}{0pt}%
\pgfpathmoveto{\pgfqpoint{1.922328in}{1.417799in}}%
\pgfpathlineto{\pgfqpoint{1.931265in}{1.417799in}}%
\pgfpathlineto{\pgfqpoint{1.931265in}{1.370403in}}%
\pgfpathlineto{\pgfqpoint{1.922328in}{1.370403in}}%
\pgfpathlineto{\pgfqpoint{1.922328in}{1.417799in}}%
\pgfpathclose%
\pgfusepath{fill}%
\end{pgfscope}%
\begin{pgfscope}%
\pgfpathrectangle{\pgfqpoint{0.697024in}{0.857143in}}{\pgfqpoint{2.627103in}{1.813434in}}%
\pgfusepath{clip}%
\pgfsetbuttcap%
\pgfsetmiterjoin%
\definecolor{currentfill}{rgb}{0.992771,0.707689,0.712380}%
\pgfsetfillcolor{currentfill}%
\pgfsetlinewidth{0.000000pt}%
\definecolor{currentstroke}{rgb}{0.000000,0.000000,0.000000}%
\pgfsetstrokecolor{currentstroke}%
\pgfsetstrokeopacity{0.000000}%
\pgfsetdash{}{0pt}%
\pgfpathmoveto{\pgfqpoint{1.933499in}{1.368682in}}%
\pgfpathlineto{\pgfqpoint{1.942435in}{1.368682in}}%
\pgfpathlineto{\pgfqpoint{1.942435in}{1.346383in}}%
\pgfpathlineto{\pgfqpoint{1.933499in}{1.346383in}}%
\pgfpathlineto{\pgfqpoint{1.933499in}{1.368682in}}%
\pgfpathclose%
\pgfusepath{fill}%
\end{pgfscope}%
\begin{pgfscope}%
\pgfpathrectangle{\pgfqpoint{0.697024in}{0.857143in}}{\pgfqpoint{2.627103in}{1.813434in}}%
\pgfusepath{clip}%
\pgfsetbuttcap%
\pgfsetmiterjoin%
\definecolor{currentfill}{rgb}{0.992771,0.707689,0.712380}%
\pgfsetfillcolor{currentfill}%
\pgfsetlinewidth{0.000000pt}%
\definecolor{currentstroke}{rgb}{0.000000,0.000000,0.000000}%
\pgfsetstrokecolor{currentstroke}%
\pgfsetstrokeopacity{0.000000}%
\pgfsetdash{}{0pt}%
\pgfpathmoveto{\pgfqpoint{1.944669in}{1.359730in}}%
\pgfpathlineto{\pgfqpoint{1.953606in}{1.359730in}}%
\pgfpathlineto{\pgfqpoint{1.953606in}{1.324169in}}%
\pgfpathlineto{\pgfqpoint{1.944669in}{1.324169in}}%
\pgfpathlineto{\pgfqpoint{1.944669in}{1.359730in}}%
\pgfpathclose%
\pgfusepath{fill}%
\end{pgfscope}%
\begin{pgfscope}%
\pgfpathrectangle{\pgfqpoint{0.697024in}{0.857143in}}{\pgfqpoint{2.627103in}{1.813434in}}%
\pgfusepath{clip}%
\pgfsetbuttcap%
\pgfsetmiterjoin%
\definecolor{currentfill}{rgb}{0.992771,0.707689,0.712380}%
\pgfsetfillcolor{currentfill}%
\pgfsetlinewidth{0.000000pt}%
\definecolor{currentstroke}{rgb}{0.000000,0.000000,0.000000}%
\pgfsetstrokecolor{currentstroke}%
\pgfsetstrokeopacity{0.000000}%
\pgfsetdash{}{0pt}%
\pgfpathmoveto{\pgfqpoint{1.955840in}{1.417806in}}%
\pgfpathlineto{\pgfqpoint{1.964776in}{1.417806in}}%
\pgfpathlineto{\pgfqpoint{1.964776in}{1.384087in}}%
\pgfpathlineto{\pgfqpoint{1.955840in}{1.384087in}}%
\pgfpathlineto{\pgfqpoint{1.955840in}{1.417806in}}%
\pgfpathclose%
\pgfusepath{fill}%
\end{pgfscope}%
\begin{pgfscope}%
\pgfpathrectangle{\pgfqpoint{0.697024in}{0.857143in}}{\pgfqpoint{2.627103in}{1.813434in}}%
\pgfusepath{clip}%
\pgfsetbuttcap%
\pgfsetmiterjoin%
\definecolor{currentfill}{rgb}{0.992771,0.707689,0.712380}%
\pgfsetfillcolor{currentfill}%
\pgfsetlinewidth{0.000000pt}%
\definecolor{currentstroke}{rgb}{0.000000,0.000000,0.000000}%
\pgfsetstrokecolor{currentstroke}%
\pgfsetstrokeopacity{0.000000}%
\pgfsetdash{}{0pt}%
\pgfpathmoveto{\pgfqpoint{1.967011in}{1.360234in}}%
\pgfpathlineto{\pgfqpoint{1.975947in}{1.360234in}}%
\pgfpathlineto{\pgfqpoint{1.975947in}{1.341514in}}%
\pgfpathlineto{\pgfqpoint{1.967011in}{1.341514in}}%
\pgfpathlineto{\pgfqpoint{1.967011in}{1.360234in}}%
\pgfpathclose%
\pgfusepath{fill}%
\end{pgfscope}%
\begin{pgfscope}%
\pgfpathrectangle{\pgfqpoint{0.697024in}{0.857143in}}{\pgfqpoint{2.627103in}{1.813434in}}%
\pgfusepath{clip}%
\pgfsetbuttcap%
\pgfsetmiterjoin%
\definecolor{currentfill}{rgb}{0.992771,0.707689,0.712380}%
\pgfsetfillcolor{currentfill}%
\pgfsetlinewidth{0.000000pt}%
\definecolor{currentstroke}{rgb}{0.000000,0.000000,0.000000}%
\pgfsetstrokecolor{currentstroke}%
\pgfsetstrokeopacity{0.000000}%
\pgfsetdash{}{0pt}%
\pgfpathmoveto{\pgfqpoint{1.978181in}{1.378682in}}%
\pgfpathlineto{\pgfqpoint{1.987118in}{1.378682in}}%
\pgfpathlineto{\pgfqpoint{1.987118in}{1.355933in}}%
\pgfpathlineto{\pgfqpoint{1.978181in}{1.355933in}}%
\pgfpathlineto{\pgfqpoint{1.978181in}{1.378682in}}%
\pgfpathclose%
\pgfusepath{fill}%
\end{pgfscope}%
\begin{pgfscope}%
\pgfpathrectangle{\pgfqpoint{0.697024in}{0.857143in}}{\pgfqpoint{2.627103in}{1.813434in}}%
\pgfusepath{clip}%
\pgfsetbuttcap%
\pgfsetmiterjoin%
\definecolor{currentfill}{rgb}{0.992771,0.707689,0.712380}%
\pgfsetfillcolor{currentfill}%
\pgfsetlinewidth{0.000000pt}%
\definecolor{currentstroke}{rgb}{0.000000,0.000000,0.000000}%
\pgfsetstrokecolor{currentstroke}%
\pgfsetstrokeopacity{0.000000}%
\pgfsetdash{}{0pt}%
\pgfpathmoveto{\pgfqpoint{1.989352in}{1.317660in}}%
\pgfpathlineto{\pgfqpoint{1.998288in}{1.317660in}}%
\pgfpathlineto{\pgfqpoint{1.998288in}{1.312390in}}%
\pgfpathlineto{\pgfqpoint{1.989352in}{1.312390in}}%
\pgfpathlineto{\pgfqpoint{1.989352in}{1.317660in}}%
\pgfpathclose%
\pgfusepath{fill}%
\end{pgfscope}%
\begin{pgfscope}%
\pgfpathrectangle{\pgfqpoint{0.697024in}{0.857143in}}{\pgfqpoint{2.627103in}{1.813434in}}%
\pgfusepath{clip}%
\pgfsetbuttcap%
\pgfsetmiterjoin%
\definecolor{currentfill}{rgb}{0.992771,0.707689,0.712380}%
\pgfsetfillcolor{currentfill}%
\pgfsetlinewidth{0.000000pt}%
\definecolor{currentstroke}{rgb}{0.000000,0.000000,0.000000}%
\pgfsetstrokecolor{currentstroke}%
\pgfsetstrokeopacity{0.000000}%
\pgfsetdash{}{0pt}%
\pgfpathmoveto{\pgfqpoint{2.000522in}{1.356778in}}%
\pgfpathlineto{\pgfqpoint{2.009459in}{1.356778in}}%
\pgfpathlineto{\pgfqpoint{2.009459in}{1.344540in}}%
\pgfpathlineto{\pgfqpoint{2.000522in}{1.344540in}}%
\pgfpathlineto{\pgfqpoint{2.000522in}{1.356778in}}%
\pgfpathclose%
\pgfusepath{fill}%
\end{pgfscope}%
\begin{pgfscope}%
\pgfpathrectangle{\pgfqpoint{0.697024in}{0.857143in}}{\pgfqpoint{2.627103in}{1.813434in}}%
\pgfusepath{clip}%
\pgfsetbuttcap%
\pgfsetmiterjoin%
\definecolor{currentfill}{rgb}{0.992771,0.707689,0.712380}%
\pgfsetfillcolor{currentfill}%
\pgfsetlinewidth{0.000000pt}%
\definecolor{currentstroke}{rgb}{0.000000,0.000000,0.000000}%
\pgfsetstrokecolor{currentstroke}%
\pgfsetstrokeopacity{0.000000}%
\pgfsetdash{}{0pt}%
\pgfpathmoveto{\pgfqpoint{2.011693in}{2.466140in}}%
\pgfpathlineto{\pgfqpoint{2.020629in}{2.466140in}}%
\pgfpathlineto{\pgfqpoint{2.020629in}{2.471784in}}%
\pgfpathlineto{\pgfqpoint{2.011693in}{2.471784in}}%
\pgfpathlineto{\pgfqpoint{2.011693in}{2.466140in}}%
\pgfpathclose%
\pgfusepath{fill}%
\end{pgfscope}%
\begin{pgfscope}%
\pgfpathrectangle{\pgfqpoint{0.697024in}{0.857143in}}{\pgfqpoint{2.627103in}{1.813434in}}%
\pgfusepath{clip}%
\pgfsetbuttcap%
\pgfsetmiterjoin%
\definecolor{currentfill}{rgb}{0.992771,0.707689,0.712380}%
\pgfsetfillcolor{currentfill}%
\pgfsetlinewidth{0.000000pt}%
\definecolor{currentstroke}{rgb}{0.000000,0.000000,0.000000}%
\pgfsetstrokecolor{currentstroke}%
\pgfsetstrokeopacity{0.000000}%
\pgfsetdash{}{0pt}%
\pgfpathmoveto{\pgfqpoint{2.022864in}{2.488272in}}%
\pgfpathlineto{\pgfqpoint{2.031800in}{2.488272in}}%
\pgfpathlineto{\pgfqpoint{2.031800in}{2.488961in}}%
\pgfpathlineto{\pgfqpoint{2.022864in}{2.488961in}}%
\pgfpathlineto{\pgfqpoint{2.022864in}{2.488272in}}%
\pgfpathclose%
\pgfusepath{fill}%
\end{pgfscope}%
\begin{pgfscope}%
\pgfpathrectangle{\pgfqpoint{0.697024in}{0.857143in}}{\pgfqpoint{2.627103in}{1.813434in}}%
\pgfusepath{clip}%
\pgfsetbuttcap%
\pgfsetmiterjoin%
\definecolor{currentfill}{rgb}{0.992771,0.707689,0.712380}%
\pgfsetfillcolor{currentfill}%
\pgfsetlinewidth{0.000000pt}%
\definecolor{currentstroke}{rgb}{0.000000,0.000000,0.000000}%
\pgfsetstrokecolor{currentstroke}%
\pgfsetstrokeopacity{0.000000}%
\pgfsetdash{}{0pt}%
\pgfpathmoveto{\pgfqpoint{2.034034in}{2.463065in}}%
\pgfpathlineto{\pgfqpoint{2.042971in}{2.463065in}}%
\pgfpathlineto{\pgfqpoint{2.042971in}{2.470333in}}%
\pgfpathlineto{\pgfqpoint{2.034034in}{2.470333in}}%
\pgfpathlineto{\pgfqpoint{2.034034in}{2.463065in}}%
\pgfpathclose%
\pgfusepath{fill}%
\end{pgfscope}%
\begin{pgfscope}%
\pgfpathrectangle{\pgfqpoint{0.697024in}{0.857143in}}{\pgfqpoint{2.627103in}{1.813434in}}%
\pgfusepath{clip}%
\pgfsetbuttcap%
\pgfsetmiterjoin%
\definecolor{currentfill}{rgb}{0.992771,0.707689,0.712380}%
\pgfsetfillcolor{currentfill}%
\pgfsetlinewidth{0.000000pt}%
\definecolor{currentstroke}{rgb}{0.000000,0.000000,0.000000}%
\pgfsetstrokecolor{currentstroke}%
\pgfsetstrokeopacity{0.000000}%
\pgfsetdash{}{0pt}%
\pgfpathmoveto{\pgfqpoint{2.045205in}{1.351788in}}%
\pgfpathlineto{\pgfqpoint{2.054141in}{1.351788in}}%
\pgfpathlineto{\pgfqpoint{2.054141in}{1.343170in}}%
\pgfpathlineto{\pgfqpoint{2.045205in}{1.343170in}}%
\pgfpathlineto{\pgfqpoint{2.045205in}{1.351788in}}%
\pgfpathclose%
\pgfusepath{fill}%
\end{pgfscope}%
\begin{pgfscope}%
\pgfpathrectangle{\pgfqpoint{0.697024in}{0.857143in}}{\pgfqpoint{2.627103in}{1.813434in}}%
\pgfusepath{clip}%
\pgfsetbuttcap%
\pgfsetmiterjoin%
\definecolor{currentfill}{rgb}{0.992771,0.707689,0.712380}%
\pgfsetfillcolor{currentfill}%
\pgfsetlinewidth{0.000000pt}%
\definecolor{currentstroke}{rgb}{0.000000,0.000000,0.000000}%
\pgfsetstrokecolor{currentstroke}%
\pgfsetstrokeopacity{0.000000}%
\pgfsetdash{}{0pt}%
\pgfpathmoveto{\pgfqpoint{2.056375in}{2.481097in}}%
\pgfpathlineto{\pgfqpoint{2.065312in}{2.481097in}}%
\pgfpathlineto{\pgfqpoint{2.065312in}{2.484770in}}%
\pgfpathlineto{\pgfqpoint{2.056375in}{2.484770in}}%
\pgfpathlineto{\pgfqpoint{2.056375in}{2.481097in}}%
\pgfpathclose%
\pgfusepath{fill}%
\end{pgfscope}%
\begin{pgfscope}%
\pgfpathrectangle{\pgfqpoint{0.697024in}{0.857143in}}{\pgfqpoint{2.627103in}{1.813434in}}%
\pgfusepath{clip}%
\pgfsetbuttcap%
\pgfsetmiterjoin%
\definecolor{currentfill}{rgb}{0.992771,0.707689,0.712380}%
\pgfsetfillcolor{currentfill}%
\pgfsetlinewidth{0.000000pt}%
\definecolor{currentstroke}{rgb}{0.000000,0.000000,0.000000}%
\pgfsetstrokecolor{currentstroke}%
\pgfsetstrokeopacity{0.000000}%
\pgfsetdash{}{0pt}%
\pgfpathmoveto{\pgfqpoint{2.067546in}{2.484598in}}%
\pgfpathlineto{\pgfqpoint{2.076482in}{2.484598in}}%
\pgfpathlineto{\pgfqpoint{2.076482in}{2.484730in}}%
\pgfpathlineto{\pgfqpoint{2.067546in}{2.484730in}}%
\pgfpathlineto{\pgfqpoint{2.067546in}{2.484598in}}%
\pgfpathclose%
\pgfusepath{fill}%
\end{pgfscope}%
\begin{pgfscope}%
\pgfpathrectangle{\pgfqpoint{0.697024in}{0.857143in}}{\pgfqpoint{2.627103in}{1.813434in}}%
\pgfusepath{clip}%
\pgfsetbuttcap%
\pgfsetmiterjoin%
\definecolor{currentfill}{rgb}{0.992771,0.707689,0.712380}%
\pgfsetfillcolor{currentfill}%
\pgfsetlinewidth{0.000000pt}%
\definecolor{currentstroke}{rgb}{0.000000,0.000000,0.000000}%
\pgfsetstrokecolor{currentstroke}%
\pgfsetstrokeopacity{0.000000}%
\pgfsetdash{}{0pt}%
\pgfpathmoveto{\pgfqpoint{2.078717in}{2.498342in}}%
\pgfpathlineto{\pgfqpoint{2.087653in}{2.498342in}}%
\pgfpathlineto{\pgfqpoint{2.087653in}{2.507998in}}%
\pgfpathlineto{\pgfqpoint{2.078717in}{2.507998in}}%
\pgfpathlineto{\pgfqpoint{2.078717in}{2.498342in}}%
\pgfpathclose%
\pgfusepath{fill}%
\end{pgfscope}%
\begin{pgfscope}%
\pgfpathrectangle{\pgfqpoint{0.697024in}{0.857143in}}{\pgfqpoint{2.627103in}{1.813434in}}%
\pgfusepath{clip}%
\pgfsetbuttcap%
\pgfsetmiterjoin%
\definecolor{currentfill}{rgb}{0.992771,0.707689,0.712380}%
\pgfsetfillcolor{currentfill}%
\pgfsetlinewidth{0.000000pt}%
\definecolor{currentstroke}{rgb}{0.000000,0.000000,0.000000}%
\pgfsetstrokecolor{currentstroke}%
\pgfsetstrokeopacity{0.000000}%
\pgfsetdash{}{0pt}%
\pgfpathmoveto{\pgfqpoint{2.089887in}{1.286134in}}%
\pgfpathlineto{\pgfqpoint{2.098824in}{1.286134in}}%
\pgfpathlineto{\pgfqpoint{2.098824in}{1.285183in}}%
\pgfpathlineto{\pgfqpoint{2.089887in}{1.285183in}}%
\pgfpathlineto{\pgfqpoint{2.089887in}{1.286134in}}%
\pgfpathclose%
\pgfusepath{fill}%
\end{pgfscope}%
\begin{pgfscope}%
\pgfpathrectangle{\pgfqpoint{0.697024in}{0.857143in}}{\pgfqpoint{2.627103in}{1.813434in}}%
\pgfusepath{clip}%
\pgfsetbuttcap%
\pgfsetmiterjoin%
\definecolor{currentfill}{rgb}{0.992771,0.707689,0.712380}%
\pgfsetfillcolor{currentfill}%
\pgfsetlinewidth{0.000000pt}%
\definecolor{currentstroke}{rgb}{0.000000,0.000000,0.000000}%
\pgfsetstrokecolor{currentstroke}%
\pgfsetstrokeopacity{0.000000}%
\pgfsetdash{}{0pt}%
\pgfpathmoveto{\pgfqpoint{2.101058in}{2.461772in}}%
\pgfpathlineto{\pgfqpoint{2.109994in}{2.461772in}}%
\pgfpathlineto{\pgfqpoint{2.109994in}{2.462266in}}%
\pgfpathlineto{\pgfqpoint{2.101058in}{2.462266in}}%
\pgfpathlineto{\pgfqpoint{2.101058in}{2.461772in}}%
\pgfpathclose%
\pgfusepath{fill}%
\end{pgfscope}%
\begin{pgfscope}%
\pgfpathrectangle{\pgfqpoint{0.697024in}{0.857143in}}{\pgfqpoint{2.627103in}{1.813434in}}%
\pgfusepath{clip}%
\pgfsetbuttcap%
\pgfsetmiterjoin%
\definecolor{currentfill}{rgb}{0.992771,0.707689,0.712380}%
\pgfsetfillcolor{currentfill}%
\pgfsetlinewidth{0.000000pt}%
\definecolor{currentstroke}{rgb}{0.000000,0.000000,0.000000}%
\pgfsetstrokecolor{currentstroke}%
\pgfsetstrokeopacity{0.000000}%
\pgfsetdash{}{0pt}%
\pgfpathmoveto{\pgfqpoint{2.112228in}{1.352449in}}%
\pgfpathlineto{\pgfqpoint{2.121165in}{1.352449in}}%
\pgfpathlineto{\pgfqpoint{2.121165in}{1.342904in}}%
\pgfpathlineto{\pgfqpoint{2.112228in}{1.342904in}}%
\pgfpathlineto{\pgfqpoint{2.112228in}{1.352449in}}%
\pgfpathclose%
\pgfusepath{fill}%
\end{pgfscope}%
\begin{pgfscope}%
\pgfpathrectangle{\pgfqpoint{0.697024in}{0.857143in}}{\pgfqpoint{2.627103in}{1.813434in}}%
\pgfusepath{clip}%
\pgfsetbuttcap%
\pgfsetmiterjoin%
\definecolor{currentfill}{rgb}{0.992771,0.707689,0.712380}%
\pgfsetfillcolor{currentfill}%
\pgfsetlinewidth{0.000000pt}%
\definecolor{currentstroke}{rgb}{0.000000,0.000000,0.000000}%
\pgfsetstrokecolor{currentstroke}%
\pgfsetstrokeopacity{0.000000}%
\pgfsetdash{}{0pt}%
\pgfpathmoveto{\pgfqpoint{2.123399in}{1.341601in}}%
\pgfpathlineto{\pgfqpoint{2.132335in}{1.341601in}}%
\pgfpathlineto{\pgfqpoint{2.132335in}{1.332125in}}%
\pgfpathlineto{\pgfqpoint{2.123399in}{1.332125in}}%
\pgfpathlineto{\pgfqpoint{2.123399in}{1.341601in}}%
\pgfpathclose%
\pgfusepath{fill}%
\end{pgfscope}%
\begin{pgfscope}%
\pgfpathrectangle{\pgfqpoint{0.697024in}{0.857143in}}{\pgfqpoint{2.627103in}{1.813434in}}%
\pgfusepath{clip}%
\pgfsetbuttcap%
\pgfsetmiterjoin%
\definecolor{currentfill}{rgb}{0.992771,0.707689,0.712380}%
\pgfsetfillcolor{currentfill}%
\pgfsetlinewidth{0.000000pt}%
\definecolor{currentstroke}{rgb}{0.000000,0.000000,0.000000}%
\pgfsetstrokecolor{currentstroke}%
\pgfsetstrokeopacity{0.000000}%
\pgfsetdash{}{0pt}%
\pgfpathmoveto{\pgfqpoint{2.134570in}{2.387310in}}%
\pgfpathlineto{\pgfqpoint{2.143506in}{2.387310in}}%
\pgfpathlineto{\pgfqpoint{2.143506in}{2.399251in}}%
\pgfpathlineto{\pgfqpoint{2.134570in}{2.399251in}}%
\pgfpathlineto{\pgfqpoint{2.134570in}{2.387310in}}%
\pgfpathclose%
\pgfusepath{fill}%
\end{pgfscope}%
\begin{pgfscope}%
\pgfpathrectangle{\pgfqpoint{0.697024in}{0.857143in}}{\pgfqpoint{2.627103in}{1.813434in}}%
\pgfusepath{clip}%
\pgfsetbuttcap%
\pgfsetmiterjoin%
\definecolor{currentfill}{rgb}{0.992771,0.707689,0.712380}%
\pgfsetfillcolor{currentfill}%
\pgfsetlinewidth{0.000000pt}%
\definecolor{currentstroke}{rgb}{0.000000,0.000000,0.000000}%
\pgfsetstrokecolor{currentstroke}%
\pgfsetstrokeopacity{0.000000}%
\pgfsetdash{}{0pt}%
\pgfpathmoveto{\pgfqpoint{2.145740in}{1.397383in}}%
\pgfpathlineto{\pgfqpoint{2.154677in}{1.397383in}}%
\pgfpathlineto{\pgfqpoint{2.154677in}{1.388812in}}%
\pgfpathlineto{\pgfqpoint{2.145740in}{1.388812in}}%
\pgfpathlineto{\pgfqpoint{2.145740in}{1.397383in}}%
\pgfpathclose%
\pgfusepath{fill}%
\end{pgfscope}%
\begin{pgfscope}%
\pgfpathrectangle{\pgfqpoint{0.697024in}{0.857143in}}{\pgfqpoint{2.627103in}{1.813434in}}%
\pgfusepath{clip}%
\pgfsetbuttcap%
\pgfsetmiterjoin%
\definecolor{currentfill}{rgb}{0.992771,0.707689,0.712380}%
\pgfsetfillcolor{currentfill}%
\pgfsetlinewidth{0.000000pt}%
\definecolor{currentstroke}{rgb}{0.000000,0.000000,0.000000}%
\pgfsetstrokecolor{currentstroke}%
\pgfsetstrokeopacity{0.000000}%
\pgfsetdash{}{0pt}%
\pgfpathmoveto{\pgfqpoint{2.156911in}{1.426518in}}%
\pgfpathlineto{\pgfqpoint{2.165847in}{1.426518in}}%
\pgfpathlineto{\pgfqpoint{2.165847in}{1.417258in}}%
\pgfpathlineto{\pgfqpoint{2.156911in}{1.417258in}}%
\pgfpathlineto{\pgfqpoint{2.156911in}{1.426518in}}%
\pgfpathclose%
\pgfusepath{fill}%
\end{pgfscope}%
\begin{pgfscope}%
\pgfpathrectangle{\pgfqpoint{0.697024in}{0.857143in}}{\pgfqpoint{2.627103in}{1.813434in}}%
\pgfusepath{clip}%
\pgfsetbuttcap%
\pgfsetmiterjoin%
\definecolor{currentfill}{rgb}{0.992771,0.707689,0.712380}%
\pgfsetfillcolor{currentfill}%
\pgfsetlinewidth{0.000000pt}%
\definecolor{currentstroke}{rgb}{0.000000,0.000000,0.000000}%
\pgfsetstrokecolor{currentstroke}%
\pgfsetstrokeopacity{0.000000}%
\pgfsetdash{}{0pt}%
\pgfpathmoveto{\pgfqpoint{2.168081in}{1.412059in}}%
\pgfpathlineto{\pgfqpoint{2.177018in}{1.412059in}}%
\pgfpathlineto{\pgfqpoint{2.177018in}{1.408178in}}%
\pgfpathlineto{\pgfqpoint{2.168081in}{1.408178in}}%
\pgfpathlineto{\pgfqpoint{2.168081in}{1.412059in}}%
\pgfpathclose%
\pgfusepath{fill}%
\end{pgfscope}%
\begin{pgfscope}%
\pgfpathrectangle{\pgfqpoint{0.697024in}{0.857143in}}{\pgfqpoint{2.627103in}{1.813434in}}%
\pgfusepath{clip}%
\pgfsetbuttcap%
\pgfsetmiterjoin%
\definecolor{currentfill}{rgb}{0.992771,0.707689,0.712380}%
\pgfsetfillcolor{currentfill}%
\pgfsetlinewidth{0.000000pt}%
\definecolor{currentstroke}{rgb}{0.000000,0.000000,0.000000}%
\pgfsetstrokecolor{currentstroke}%
\pgfsetstrokeopacity{0.000000}%
\pgfsetdash{}{0pt}%
\pgfpathmoveto{\pgfqpoint{2.179252in}{1.454816in}}%
\pgfpathlineto{\pgfqpoint{2.188189in}{1.454816in}}%
\pgfpathlineto{\pgfqpoint{2.188189in}{1.447580in}}%
\pgfpathlineto{\pgfqpoint{2.179252in}{1.447580in}}%
\pgfpathlineto{\pgfqpoint{2.179252in}{1.454816in}}%
\pgfpathclose%
\pgfusepath{fill}%
\end{pgfscope}%
\begin{pgfscope}%
\pgfpathrectangle{\pgfqpoint{0.697024in}{0.857143in}}{\pgfqpoint{2.627103in}{1.813434in}}%
\pgfusepath{clip}%
\pgfsetbuttcap%
\pgfsetmiterjoin%
\definecolor{currentfill}{rgb}{0.992771,0.707689,0.712380}%
\pgfsetfillcolor{currentfill}%
\pgfsetlinewidth{0.000000pt}%
\definecolor{currentstroke}{rgb}{0.000000,0.000000,0.000000}%
\pgfsetstrokecolor{currentstroke}%
\pgfsetstrokeopacity{0.000000}%
\pgfsetdash{}{0pt}%
\pgfpathmoveto{\pgfqpoint{2.190423in}{1.503768in}}%
\pgfpathlineto{\pgfqpoint{2.199359in}{1.503768in}}%
\pgfpathlineto{\pgfqpoint{2.199359in}{1.463338in}}%
\pgfpathlineto{\pgfqpoint{2.190423in}{1.463338in}}%
\pgfpathlineto{\pgfqpoint{2.190423in}{1.503768in}}%
\pgfpathclose%
\pgfusepath{fill}%
\end{pgfscope}%
\begin{pgfscope}%
\pgfpathrectangle{\pgfqpoint{0.697024in}{0.857143in}}{\pgfqpoint{2.627103in}{1.813434in}}%
\pgfusepath{clip}%
\pgfsetbuttcap%
\pgfsetmiterjoin%
\definecolor{currentfill}{rgb}{0.992771,0.707689,0.712380}%
\pgfsetfillcolor{currentfill}%
\pgfsetlinewidth{0.000000pt}%
\definecolor{currentstroke}{rgb}{0.000000,0.000000,0.000000}%
\pgfsetstrokecolor{currentstroke}%
\pgfsetstrokeopacity{0.000000}%
\pgfsetdash{}{0pt}%
\pgfpathmoveto{\pgfqpoint{2.201593in}{1.511276in}}%
\pgfpathlineto{\pgfqpoint{2.210530in}{1.511276in}}%
\pgfpathlineto{\pgfqpoint{2.210530in}{1.485188in}}%
\pgfpathlineto{\pgfqpoint{2.201593in}{1.485188in}}%
\pgfpathlineto{\pgfqpoint{2.201593in}{1.511276in}}%
\pgfpathclose%
\pgfusepath{fill}%
\end{pgfscope}%
\begin{pgfscope}%
\pgfpathrectangle{\pgfqpoint{0.697024in}{0.857143in}}{\pgfqpoint{2.627103in}{1.813434in}}%
\pgfusepath{clip}%
\pgfsetbuttcap%
\pgfsetmiterjoin%
\definecolor{currentfill}{rgb}{0.992771,0.707689,0.712380}%
\pgfsetfillcolor{currentfill}%
\pgfsetlinewidth{0.000000pt}%
\definecolor{currentstroke}{rgb}{0.000000,0.000000,0.000000}%
\pgfsetstrokecolor{currentstroke}%
\pgfsetstrokeopacity{0.000000}%
\pgfsetdash{}{0pt}%
\pgfpathmoveto{\pgfqpoint{2.212764in}{1.583507in}}%
\pgfpathlineto{\pgfqpoint{2.221700in}{1.583507in}}%
\pgfpathlineto{\pgfqpoint{2.221700in}{1.563640in}}%
\pgfpathlineto{\pgfqpoint{2.212764in}{1.563640in}}%
\pgfpathlineto{\pgfqpoint{2.212764in}{1.583507in}}%
\pgfpathclose%
\pgfusepath{fill}%
\end{pgfscope}%
\begin{pgfscope}%
\pgfpathrectangle{\pgfqpoint{0.697024in}{0.857143in}}{\pgfqpoint{2.627103in}{1.813434in}}%
\pgfusepath{clip}%
\pgfsetbuttcap%
\pgfsetmiterjoin%
\definecolor{currentfill}{rgb}{0.992771,0.707689,0.712380}%
\pgfsetfillcolor{currentfill}%
\pgfsetlinewidth{0.000000pt}%
\definecolor{currentstroke}{rgb}{0.000000,0.000000,0.000000}%
\pgfsetstrokecolor{currentstroke}%
\pgfsetstrokeopacity{0.000000}%
\pgfsetdash{}{0pt}%
\pgfpathmoveto{\pgfqpoint{2.223934in}{1.599898in}}%
\pgfpathlineto{\pgfqpoint{2.232871in}{1.599898in}}%
\pgfpathlineto{\pgfqpoint{2.232871in}{1.559960in}}%
\pgfpathlineto{\pgfqpoint{2.223934in}{1.559960in}}%
\pgfpathlineto{\pgfqpoint{2.223934in}{1.599898in}}%
\pgfpathclose%
\pgfusepath{fill}%
\end{pgfscope}%
\begin{pgfscope}%
\pgfpathrectangle{\pgfqpoint{0.697024in}{0.857143in}}{\pgfqpoint{2.627103in}{1.813434in}}%
\pgfusepath{clip}%
\pgfsetbuttcap%
\pgfsetmiterjoin%
\definecolor{currentfill}{rgb}{0.992771,0.707689,0.712380}%
\pgfsetfillcolor{currentfill}%
\pgfsetlinewidth{0.000000pt}%
\definecolor{currentstroke}{rgb}{0.000000,0.000000,0.000000}%
\pgfsetstrokecolor{currentstroke}%
\pgfsetstrokeopacity{0.000000}%
\pgfsetdash{}{0pt}%
\pgfpathmoveto{\pgfqpoint{2.235105in}{1.607094in}}%
\pgfpathlineto{\pgfqpoint{2.244042in}{1.607094in}}%
\pgfpathlineto{\pgfqpoint{2.244042in}{1.568624in}}%
\pgfpathlineto{\pgfqpoint{2.235105in}{1.568624in}}%
\pgfpathlineto{\pgfqpoint{2.235105in}{1.607094in}}%
\pgfpathclose%
\pgfusepath{fill}%
\end{pgfscope}%
\begin{pgfscope}%
\pgfpathrectangle{\pgfqpoint{0.697024in}{0.857143in}}{\pgfqpoint{2.627103in}{1.813434in}}%
\pgfusepath{clip}%
\pgfsetbuttcap%
\pgfsetmiterjoin%
\definecolor{currentfill}{rgb}{0.992771,0.707689,0.712380}%
\pgfsetfillcolor{currentfill}%
\pgfsetlinewidth{0.000000pt}%
\definecolor{currentstroke}{rgb}{0.000000,0.000000,0.000000}%
\pgfsetstrokecolor{currentstroke}%
\pgfsetstrokeopacity{0.000000}%
\pgfsetdash{}{0pt}%
\pgfpathmoveto{\pgfqpoint{2.246276in}{1.629889in}}%
\pgfpathlineto{\pgfqpoint{2.255212in}{1.629889in}}%
\pgfpathlineto{\pgfqpoint{2.255212in}{1.587282in}}%
\pgfpathlineto{\pgfqpoint{2.246276in}{1.587282in}}%
\pgfpathlineto{\pgfqpoint{2.246276in}{1.629889in}}%
\pgfpathclose%
\pgfusepath{fill}%
\end{pgfscope}%
\begin{pgfscope}%
\pgfpathrectangle{\pgfqpoint{0.697024in}{0.857143in}}{\pgfqpoint{2.627103in}{1.813434in}}%
\pgfusepath{clip}%
\pgfsetbuttcap%
\pgfsetmiterjoin%
\definecolor{currentfill}{rgb}{0.992771,0.707689,0.712380}%
\pgfsetfillcolor{currentfill}%
\pgfsetlinewidth{0.000000pt}%
\definecolor{currentstroke}{rgb}{0.000000,0.000000,0.000000}%
\pgfsetstrokecolor{currentstroke}%
\pgfsetstrokeopacity{0.000000}%
\pgfsetdash{}{0pt}%
\pgfpathmoveto{\pgfqpoint{2.257446in}{1.638487in}}%
\pgfpathlineto{\pgfqpoint{2.266383in}{1.638487in}}%
\pgfpathlineto{\pgfqpoint{2.266383in}{1.600235in}}%
\pgfpathlineto{\pgfqpoint{2.257446in}{1.600235in}}%
\pgfpathlineto{\pgfqpoint{2.257446in}{1.638487in}}%
\pgfpathclose%
\pgfusepath{fill}%
\end{pgfscope}%
\begin{pgfscope}%
\pgfpathrectangle{\pgfqpoint{0.697024in}{0.857143in}}{\pgfqpoint{2.627103in}{1.813434in}}%
\pgfusepath{clip}%
\pgfsetbuttcap%
\pgfsetmiterjoin%
\definecolor{currentfill}{rgb}{0.992771,0.707689,0.712380}%
\pgfsetfillcolor{currentfill}%
\pgfsetlinewidth{0.000000pt}%
\definecolor{currentstroke}{rgb}{0.000000,0.000000,0.000000}%
\pgfsetstrokecolor{currentstroke}%
\pgfsetstrokeopacity{0.000000}%
\pgfsetdash{}{0pt}%
\pgfpathmoveto{\pgfqpoint{2.268617in}{1.671483in}}%
\pgfpathlineto{\pgfqpoint{2.277553in}{1.671483in}}%
\pgfpathlineto{\pgfqpoint{2.277553in}{1.623325in}}%
\pgfpathlineto{\pgfqpoint{2.268617in}{1.623325in}}%
\pgfpathlineto{\pgfqpoint{2.268617in}{1.671483in}}%
\pgfpathclose%
\pgfusepath{fill}%
\end{pgfscope}%
\begin{pgfscope}%
\pgfpathrectangle{\pgfqpoint{0.697024in}{0.857143in}}{\pgfqpoint{2.627103in}{1.813434in}}%
\pgfusepath{clip}%
\pgfsetbuttcap%
\pgfsetmiterjoin%
\definecolor{currentfill}{rgb}{0.992771,0.707689,0.712380}%
\pgfsetfillcolor{currentfill}%
\pgfsetlinewidth{0.000000pt}%
\definecolor{currentstroke}{rgb}{0.000000,0.000000,0.000000}%
\pgfsetstrokecolor{currentstroke}%
\pgfsetstrokeopacity{0.000000}%
\pgfsetdash{}{0pt}%
\pgfpathmoveto{\pgfqpoint{2.279787in}{1.663684in}}%
\pgfpathlineto{\pgfqpoint{2.288724in}{1.663684in}}%
\pgfpathlineto{\pgfqpoint{2.288724in}{1.617290in}}%
\pgfpathlineto{\pgfqpoint{2.279787in}{1.617290in}}%
\pgfpathlineto{\pgfqpoint{2.279787in}{1.663684in}}%
\pgfpathclose%
\pgfusepath{fill}%
\end{pgfscope}%
\begin{pgfscope}%
\pgfpathrectangle{\pgfqpoint{0.697024in}{0.857143in}}{\pgfqpoint{2.627103in}{1.813434in}}%
\pgfusepath{clip}%
\pgfsetbuttcap%
\pgfsetmiterjoin%
\definecolor{currentfill}{rgb}{0.992771,0.707689,0.712380}%
\pgfsetfillcolor{currentfill}%
\pgfsetlinewidth{0.000000pt}%
\definecolor{currentstroke}{rgb}{0.000000,0.000000,0.000000}%
\pgfsetstrokecolor{currentstroke}%
\pgfsetstrokeopacity{0.000000}%
\pgfsetdash{}{0pt}%
\pgfpathmoveto{\pgfqpoint{2.290958in}{1.660269in}}%
\pgfpathlineto{\pgfqpoint{2.299895in}{1.660269in}}%
\pgfpathlineto{\pgfqpoint{2.299895in}{1.606107in}}%
\pgfpathlineto{\pgfqpoint{2.290958in}{1.606107in}}%
\pgfpathlineto{\pgfqpoint{2.290958in}{1.660269in}}%
\pgfpathclose%
\pgfusepath{fill}%
\end{pgfscope}%
\begin{pgfscope}%
\pgfpathrectangle{\pgfqpoint{0.697024in}{0.857143in}}{\pgfqpoint{2.627103in}{1.813434in}}%
\pgfusepath{clip}%
\pgfsetbuttcap%
\pgfsetmiterjoin%
\definecolor{currentfill}{rgb}{0.992771,0.707689,0.712380}%
\pgfsetfillcolor{currentfill}%
\pgfsetlinewidth{0.000000pt}%
\definecolor{currentstroke}{rgb}{0.000000,0.000000,0.000000}%
\pgfsetstrokecolor{currentstroke}%
\pgfsetstrokeopacity{0.000000}%
\pgfsetdash{}{0pt}%
\pgfpathmoveto{\pgfqpoint{2.302129in}{1.712229in}}%
\pgfpathlineto{\pgfqpoint{2.311065in}{1.712229in}}%
\pgfpathlineto{\pgfqpoint{2.311065in}{1.666663in}}%
\pgfpathlineto{\pgfqpoint{2.302129in}{1.666663in}}%
\pgfpathlineto{\pgfqpoint{2.302129in}{1.712229in}}%
\pgfpathclose%
\pgfusepath{fill}%
\end{pgfscope}%
\begin{pgfscope}%
\pgfpathrectangle{\pgfqpoint{0.697024in}{0.857143in}}{\pgfqpoint{2.627103in}{1.813434in}}%
\pgfusepath{clip}%
\pgfsetbuttcap%
\pgfsetmiterjoin%
\definecolor{currentfill}{rgb}{0.992771,0.707689,0.712380}%
\pgfsetfillcolor{currentfill}%
\pgfsetlinewidth{0.000000pt}%
\definecolor{currentstroke}{rgb}{0.000000,0.000000,0.000000}%
\pgfsetstrokecolor{currentstroke}%
\pgfsetstrokeopacity{0.000000}%
\pgfsetdash{}{0pt}%
\pgfpathmoveto{\pgfqpoint{2.313299in}{1.692748in}}%
\pgfpathlineto{\pgfqpoint{2.322236in}{1.692748in}}%
\pgfpathlineto{\pgfqpoint{2.322236in}{1.636144in}}%
\pgfpathlineto{\pgfqpoint{2.313299in}{1.636144in}}%
\pgfpathlineto{\pgfqpoint{2.313299in}{1.692748in}}%
\pgfpathclose%
\pgfusepath{fill}%
\end{pgfscope}%
\begin{pgfscope}%
\pgfpathrectangle{\pgfqpoint{0.697024in}{0.857143in}}{\pgfqpoint{2.627103in}{1.813434in}}%
\pgfusepath{clip}%
\pgfsetbuttcap%
\pgfsetmiterjoin%
\definecolor{currentfill}{rgb}{0.992771,0.707689,0.712380}%
\pgfsetfillcolor{currentfill}%
\pgfsetlinewidth{0.000000pt}%
\definecolor{currentstroke}{rgb}{0.000000,0.000000,0.000000}%
\pgfsetstrokecolor{currentstroke}%
\pgfsetstrokeopacity{0.000000}%
\pgfsetdash{}{0pt}%
\pgfpathmoveto{\pgfqpoint{2.324470in}{1.696703in}}%
\pgfpathlineto{\pgfqpoint{2.333406in}{1.696703in}}%
\pgfpathlineto{\pgfqpoint{2.333406in}{1.615881in}}%
\pgfpathlineto{\pgfqpoint{2.324470in}{1.615881in}}%
\pgfpathlineto{\pgfqpoint{2.324470in}{1.696703in}}%
\pgfpathclose%
\pgfusepath{fill}%
\end{pgfscope}%
\begin{pgfscope}%
\pgfpathrectangle{\pgfqpoint{0.697024in}{0.857143in}}{\pgfqpoint{2.627103in}{1.813434in}}%
\pgfusepath{clip}%
\pgfsetbuttcap%
\pgfsetmiterjoin%
\definecolor{currentfill}{rgb}{0.992771,0.707689,0.712380}%
\pgfsetfillcolor{currentfill}%
\pgfsetlinewidth{0.000000pt}%
\definecolor{currentstroke}{rgb}{0.000000,0.000000,0.000000}%
\pgfsetstrokecolor{currentstroke}%
\pgfsetstrokeopacity{0.000000}%
\pgfsetdash{}{0pt}%
\pgfpathmoveto{\pgfqpoint{2.335640in}{1.662706in}}%
\pgfpathlineto{\pgfqpoint{2.344577in}{1.662706in}}%
\pgfpathlineto{\pgfqpoint{2.344577in}{1.598389in}}%
\pgfpathlineto{\pgfqpoint{2.335640in}{1.598389in}}%
\pgfpathlineto{\pgfqpoint{2.335640in}{1.662706in}}%
\pgfpathclose%
\pgfusepath{fill}%
\end{pgfscope}%
\begin{pgfscope}%
\pgfpathrectangle{\pgfqpoint{0.697024in}{0.857143in}}{\pgfqpoint{2.627103in}{1.813434in}}%
\pgfusepath{clip}%
\pgfsetbuttcap%
\pgfsetmiterjoin%
\definecolor{currentfill}{rgb}{0.992771,0.707689,0.712380}%
\pgfsetfillcolor{currentfill}%
\pgfsetlinewidth{0.000000pt}%
\definecolor{currentstroke}{rgb}{0.000000,0.000000,0.000000}%
\pgfsetstrokecolor{currentstroke}%
\pgfsetstrokeopacity{0.000000}%
\pgfsetdash{}{0pt}%
\pgfpathmoveto{\pgfqpoint{2.346811in}{1.670774in}}%
\pgfpathlineto{\pgfqpoint{2.355748in}{1.670774in}}%
\pgfpathlineto{\pgfqpoint{2.355748in}{1.603062in}}%
\pgfpathlineto{\pgfqpoint{2.346811in}{1.603062in}}%
\pgfpathlineto{\pgfqpoint{2.346811in}{1.670774in}}%
\pgfpathclose%
\pgfusepath{fill}%
\end{pgfscope}%
\begin{pgfscope}%
\pgfpathrectangle{\pgfqpoint{0.697024in}{0.857143in}}{\pgfqpoint{2.627103in}{1.813434in}}%
\pgfusepath{clip}%
\pgfsetbuttcap%
\pgfsetmiterjoin%
\definecolor{currentfill}{rgb}{0.992771,0.707689,0.712380}%
\pgfsetfillcolor{currentfill}%
\pgfsetlinewidth{0.000000pt}%
\definecolor{currentstroke}{rgb}{0.000000,0.000000,0.000000}%
\pgfsetstrokecolor{currentstroke}%
\pgfsetstrokeopacity{0.000000}%
\pgfsetdash{}{0pt}%
\pgfpathmoveto{\pgfqpoint{2.357982in}{1.690765in}}%
\pgfpathlineto{\pgfqpoint{2.366918in}{1.690765in}}%
\pgfpathlineto{\pgfqpoint{2.366918in}{1.651358in}}%
\pgfpathlineto{\pgfqpoint{2.357982in}{1.651358in}}%
\pgfpathlineto{\pgfqpoint{2.357982in}{1.690765in}}%
\pgfpathclose%
\pgfusepath{fill}%
\end{pgfscope}%
\begin{pgfscope}%
\pgfpathrectangle{\pgfqpoint{0.697024in}{0.857143in}}{\pgfqpoint{2.627103in}{1.813434in}}%
\pgfusepath{clip}%
\pgfsetbuttcap%
\pgfsetmiterjoin%
\definecolor{currentfill}{rgb}{0.992771,0.707689,0.712380}%
\pgfsetfillcolor{currentfill}%
\pgfsetlinewidth{0.000000pt}%
\definecolor{currentstroke}{rgb}{0.000000,0.000000,0.000000}%
\pgfsetstrokecolor{currentstroke}%
\pgfsetstrokeopacity{0.000000}%
\pgfsetdash{}{0pt}%
\pgfpathmoveto{\pgfqpoint{2.369152in}{1.732798in}}%
\pgfpathlineto{\pgfqpoint{2.378089in}{1.732798in}}%
\pgfpathlineto{\pgfqpoint{2.378089in}{1.650372in}}%
\pgfpathlineto{\pgfqpoint{2.369152in}{1.650372in}}%
\pgfpathlineto{\pgfqpoint{2.369152in}{1.732798in}}%
\pgfpathclose%
\pgfusepath{fill}%
\end{pgfscope}%
\begin{pgfscope}%
\pgfpathrectangle{\pgfqpoint{0.697024in}{0.857143in}}{\pgfqpoint{2.627103in}{1.813434in}}%
\pgfusepath{clip}%
\pgfsetbuttcap%
\pgfsetmiterjoin%
\definecolor{currentfill}{rgb}{0.992771,0.707689,0.712380}%
\pgfsetfillcolor{currentfill}%
\pgfsetlinewidth{0.000000pt}%
\definecolor{currentstroke}{rgb}{0.000000,0.000000,0.000000}%
\pgfsetstrokecolor{currentstroke}%
\pgfsetstrokeopacity{0.000000}%
\pgfsetdash{}{0pt}%
\pgfpathmoveto{\pgfqpoint{2.380323in}{1.753544in}}%
\pgfpathlineto{\pgfqpoint{2.389259in}{1.753544in}}%
\pgfpathlineto{\pgfqpoint{2.389259in}{1.710555in}}%
\pgfpathlineto{\pgfqpoint{2.380323in}{1.710555in}}%
\pgfpathlineto{\pgfqpoint{2.380323in}{1.753544in}}%
\pgfpathclose%
\pgfusepath{fill}%
\end{pgfscope}%
\begin{pgfscope}%
\pgfpathrectangle{\pgfqpoint{0.697024in}{0.857143in}}{\pgfqpoint{2.627103in}{1.813434in}}%
\pgfusepath{clip}%
\pgfsetbuttcap%
\pgfsetmiterjoin%
\definecolor{currentfill}{rgb}{0.992771,0.707689,0.712380}%
\pgfsetfillcolor{currentfill}%
\pgfsetlinewidth{0.000000pt}%
\definecolor{currentstroke}{rgb}{0.000000,0.000000,0.000000}%
\pgfsetstrokecolor{currentstroke}%
\pgfsetstrokeopacity{0.000000}%
\pgfsetdash{}{0pt}%
\pgfpathmoveto{\pgfqpoint{2.391494in}{1.800125in}}%
\pgfpathlineto{\pgfqpoint{2.400430in}{1.800125in}}%
\pgfpathlineto{\pgfqpoint{2.400430in}{1.773489in}}%
\pgfpathlineto{\pgfqpoint{2.391494in}{1.773489in}}%
\pgfpathlineto{\pgfqpoint{2.391494in}{1.800125in}}%
\pgfpathclose%
\pgfusepath{fill}%
\end{pgfscope}%
\begin{pgfscope}%
\pgfpathrectangle{\pgfqpoint{0.697024in}{0.857143in}}{\pgfqpoint{2.627103in}{1.813434in}}%
\pgfusepath{clip}%
\pgfsetbuttcap%
\pgfsetmiterjoin%
\definecolor{currentfill}{rgb}{0.992771,0.707689,0.712380}%
\pgfsetfillcolor{currentfill}%
\pgfsetlinewidth{0.000000pt}%
\definecolor{currentstroke}{rgb}{0.000000,0.000000,0.000000}%
\pgfsetstrokecolor{currentstroke}%
\pgfsetstrokeopacity{0.000000}%
\pgfsetdash{}{0pt}%
\pgfpathmoveto{\pgfqpoint{2.402664in}{1.816364in}}%
\pgfpathlineto{\pgfqpoint{2.411601in}{1.816364in}}%
\pgfpathlineto{\pgfqpoint{2.411601in}{1.762093in}}%
\pgfpathlineto{\pgfqpoint{2.402664in}{1.762093in}}%
\pgfpathlineto{\pgfqpoint{2.402664in}{1.816364in}}%
\pgfpathclose%
\pgfusepath{fill}%
\end{pgfscope}%
\begin{pgfscope}%
\pgfpathrectangle{\pgfqpoint{0.697024in}{0.857143in}}{\pgfqpoint{2.627103in}{1.813434in}}%
\pgfusepath{clip}%
\pgfsetbuttcap%
\pgfsetmiterjoin%
\definecolor{currentfill}{rgb}{0.992771,0.707689,0.712380}%
\pgfsetfillcolor{currentfill}%
\pgfsetlinewidth{0.000000pt}%
\definecolor{currentstroke}{rgb}{0.000000,0.000000,0.000000}%
\pgfsetstrokecolor{currentstroke}%
\pgfsetstrokeopacity{0.000000}%
\pgfsetdash{}{0pt}%
\pgfpathmoveto{\pgfqpoint{2.413835in}{1.823576in}}%
\pgfpathlineto{\pgfqpoint{2.422771in}{1.823576in}}%
\pgfpathlineto{\pgfqpoint{2.422771in}{1.795207in}}%
\pgfpathlineto{\pgfqpoint{2.413835in}{1.795207in}}%
\pgfpathlineto{\pgfqpoint{2.413835in}{1.823576in}}%
\pgfpathclose%
\pgfusepath{fill}%
\end{pgfscope}%
\begin{pgfscope}%
\pgfpathrectangle{\pgfqpoint{0.697024in}{0.857143in}}{\pgfqpoint{2.627103in}{1.813434in}}%
\pgfusepath{clip}%
\pgfsetbuttcap%
\pgfsetmiterjoin%
\definecolor{currentfill}{rgb}{0.992771,0.707689,0.712380}%
\pgfsetfillcolor{currentfill}%
\pgfsetlinewidth{0.000000pt}%
\definecolor{currentstroke}{rgb}{0.000000,0.000000,0.000000}%
\pgfsetstrokecolor{currentstroke}%
\pgfsetstrokeopacity{0.000000}%
\pgfsetdash{}{0pt}%
\pgfpathmoveto{\pgfqpoint{2.425005in}{1.825863in}}%
\pgfpathlineto{\pgfqpoint{2.433942in}{1.825863in}}%
\pgfpathlineto{\pgfqpoint{2.433942in}{1.795511in}}%
\pgfpathlineto{\pgfqpoint{2.425005in}{1.795511in}}%
\pgfpathlineto{\pgfqpoint{2.425005in}{1.825863in}}%
\pgfpathclose%
\pgfusepath{fill}%
\end{pgfscope}%
\begin{pgfscope}%
\pgfpathrectangle{\pgfqpoint{0.697024in}{0.857143in}}{\pgfqpoint{2.627103in}{1.813434in}}%
\pgfusepath{clip}%
\pgfsetbuttcap%
\pgfsetmiterjoin%
\definecolor{currentfill}{rgb}{0.992771,0.707689,0.712380}%
\pgfsetfillcolor{currentfill}%
\pgfsetlinewidth{0.000000pt}%
\definecolor{currentstroke}{rgb}{0.000000,0.000000,0.000000}%
\pgfsetstrokecolor{currentstroke}%
\pgfsetstrokeopacity{0.000000}%
\pgfsetdash{}{0pt}%
\pgfpathmoveto{\pgfqpoint{2.436176in}{1.830374in}}%
\pgfpathlineto{\pgfqpoint{2.445112in}{1.830374in}}%
\pgfpathlineto{\pgfqpoint{2.445112in}{1.813296in}}%
\pgfpathlineto{\pgfqpoint{2.436176in}{1.813296in}}%
\pgfpathlineto{\pgfqpoint{2.436176in}{1.830374in}}%
\pgfpathclose%
\pgfusepath{fill}%
\end{pgfscope}%
\begin{pgfscope}%
\pgfpathrectangle{\pgfqpoint{0.697024in}{0.857143in}}{\pgfqpoint{2.627103in}{1.813434in}}%
\pgfusepath{clip}%
\pgfsetbuttcap%
\pgfsetmiterjoin%
\definecolor{currentfill}{rgb}{0.992771,0.707689,0.712380}%
\pgfsetfillcolor{currentfill}%
\pgfsetlinewidth{0.000000pt}%
\definecolor{currentstroke}{rgb}{0.000000,0.000000,0.000000}%
\pgfsetstrokecolor{currentstroke}%
\pgfsetstrokeopacity{0.000000}%
\pgfsetdash{}{0pt}%
\pgfpathmoveto{\pgfqpoint{2.447347in}{1.823124in}}%
\pgfpathlineto{\pgfqpoint{2.456283in}{1.823124in}}%
\pgfpathlineto{\pgfqpoint{2.456283in}{1.792952in}}%
\pgfpathlineto{\pgfqpoint{2.447347in}{1.792952in}}%
\pgfpathlineto{\pgfqpoint{2.447347in}{1.823124in}}%
\pgfpathclose%
\pgfusepath{fill}%
\end{pgfscope}%
\begin{pgfscope}%
\pgfpathrectangle{\pgfqpoint{0.697024in}{0.857143in}}{\pgfqpoint{2.627103in}{1.813434in}}%
\pgfusepath{clip}%
\pgfsetbuttcap%
\pgfsetmiterjoin%
\definecolor{currentfill}{rgb}{0.992771,0.707689,0.712380}%
\pgfsetfillcolor{currentfill}%
\pgfsetlinewidth{0.000000pt}%
\definecolor{currentstroke}{rgb}{0.000000,0.000000,0.000000}%
\pgfsetstrokecolor{currentstroke}%
\pgfsetstrokeopacity{0.000000}%
\pgfsetdash{}{0pt}%
\pgfpathmoveto{\pgfqpoint{2.458517in}{1.832937in}}%
\pgfpathlineto{\pgfqpoint{2.467454in}{1.832937in}}%
\pgfpathlineto{\pgfqpoint{2.467454in}{1.828425in}}%
\pgfpathlineto{\pgfqpoint{2.458517in}{1.828425in}}%
\pgfpathlineto{\pgfqpoint{2.458517in}{1.832937in}}%
\pgfpathclose%
\pgfusepath{fill}%
\end{pgfscope}%
\begin{pgfscope}%
\pgfpathrectangle{\pgfqpoint{0.697024in}{0.857143in}}{\pgfqpoint{2.627103in}{1.813434in}}%
\pgfusepath{clip}%
\pgfsetbuttcap%
\pgfsetmiterjoin%
\definecolor{currentfill}{rgb}{0.992771,0.707689,0.712380}%
\pgfsetfillcolor{currentfill}%
\pgfsetlinewidth{0.000000pt}%
\definecolor{currentstroke}{rgb}{0.000000,0.000000,0.000000}%
\pgfsetstrokecolor{currentstroke}%
\pgfsetstrokeopacity{0.000000}%
\pgfsetdash{}{0pt}%
\pgfpathmoveto{\pgfqpoint{2.469688in}{1.831610in}}%
\pgfpathlineto{\pgfqpoint{2.478624in}{1.831610in}}%
\pgfpathlineto{\pgfqpoint{2.478624in}{1.810676in}}%
\pgfpathlineto{\pgfqpoint{2.469688in}{1.810676in}}%
\pgfpathlineto{\pgfqpoint{2.469688in}{1.831610in}}%
\pgfpathclose%
\pgfusepath{fill}%
\end{pgfscope}%
\begin{pgfscope}%
\pgfpathrectangle{\pgfqpoint{0.697024in}{0.857143in}}{\pgfqpoint{2.627103in}{1.813434in}}%
\pgfusepath{clip}%
\pgfsetbuttcap%
\pgfsetmiterjoin%
\definecolor{currentfill}{rgb}{0.992771,0.707689,0.712380}%
\pgfsetfillcolor{currentfill}%
\pgfsetlinewidth{0.000000pt}%
\definecolor{currentstroke}{rgb}{0.000000,0.000000,0.000000}%
\pgfsetstrokecolor{currentstroke}%
\pgfsetstrokeopacity{0.000000}%
\pgfsetdash{}{0pt}%
\pgfpathmoveto{\pgfqpoint{2.480858in}{1.781588in}}%
\pgfpathlineto{\pgfqpoint{2.489795in}{1.781588in}}%
\pgfpathlineto{\pgfqpoint{2.489795in}{1.778029in}}%
\pgfpathlineto{\pgfqpoint{2.480858in}{1.778029in}}%
\pgfpathlineto{\pgfqpoint{2.480858in}{1.781588in}}%
\pgfpathclose%
\pgfusepath{fill}%
\end{pgfscope}%
\begin{pgfscope}%
\pgfpathrectangle{\pgfqpoint{0.697024in}{0.857143in}}{\pgfqpoint{2.627103in}{1.813434in}}%
\pgfusepath{clip}%
\pgfsetbuttcap%
\pgfsetmiterjoin%
\definecolor{currentfill}{rgb}{0.992771,0.707689,0.712380}%
\pgfsetfillcolor{currentfill}%
\pgfsetlinewidth{0.000000pt}%
\definecolor{currentstroke}{rgb}{0.000000,0.000000,0.000000}%
\pgfsetstrokecolor{currentstroke}%
\pgfsetstrokeopacity{0.000000}%
\pgfsetdash{}{0pt}%
\pgfpathmoveto{\pgfqpoint{2.492029in}{1.784277in}}%
\pgfpathlineto{\pgfqpoint{2.500965in}{1.784277in}}%
\pgfpathlineto{\pgfqpoint{2.500965in}{1.781697in}}%
\pgfpathlineto{\pgfqpoint{2.492029in}{1.781697in}}%
\pgfpathlineto{\pgfqpoint{2.492029in}{1.784277in}}%
\pgfpathclose%
\pgfusepath{fill}%
\end{pgfscope}%
\begin{pgfscope}%
\pgfpathrectangle{\pgfqpoint{0.697024in}{0.857143in}}{\pgfqpoint{2.627103in}{1.813434in}}%
\pgfusepath{clip}%
\pgfsetbuttcap%
\pgfsetmiterjoin%
\definecolor{currentfill}{rgb}{0.992771,0.707689,0.712380}%
\pgfsetfillcolor{currentfill}%
\pgfsetlinewidth{0.000000pt}%
\definecolor{currentstroke}{rgb}{0.000000,0.000000,0.000000}%
\pgfsetstrokecolor{currentstroke}%
\pgfsetstrokeopacity{0.000000}%
\pgfsetdash{}{0pt}%
\pgfpathmoveto{\pgfqpoint{2.503200in}{2.306646in}}%
\pgfpathlineto{\pgfqpoint{2.512136in}{2.306646in}}%
\pgfpathlineto{\pgfqpoint{2.512136in}{2.306973in}}%
\pgfpathlineto{\pgfqpoint{2.503200in}{2.306973in}}%
\pgfpathlineto{\pgfqpoint{2.503200in}{2.306646in}}%
\pgfpathclose%
\pgfusepath{fill}%
\end{pgfscope}%
\begin{pgfscope}%
\pgfpathrectangle{\pgfqpoint{0.697024in}{0.857143in}}{\pgfqpoint{2.627103in}{1.813434in}}%
\pgfusepath{clip}%
\pgfsetbuttcap%
\pgfsetmiterjoin%
\definecolor{currentfill}{rgb}{0.992771,0.707689,0.712380}%
\pgfsetfillcolor{currentfill}%
\pgfsetlinewidth{0.000000pt}%
\definecolor{currentstroke}{rgb}{0.000000,0.000000,0.000000}%
\pgfsetstrokecolor{currentstroke}%
\pgfsetstrokeopacity{0.000000}%
\pgfsetdash{}{0pt}%
\pgfpathmoveto{\pgfqpoint{2.514370in}{1.773730in}}%
\pgfpathlineto{\pgfqpoint{2.523307in}{1.773730in}}%
\pgfpathlineto{\pgfqpoint{2.523307in}{1.768866in}}%
\pgfpathlineto{\pgfqpoint{2.514370in}{1.768866in}}%
\pgfpathlineto{\pgfqpoint{2.514370in}{1.773730in}}%
\pgfpathclose%
\pgfusepath{fill}%
\end{pgfscope}%
\begin{pgfscope}%
\pgfpathrectangle{\pgfqpoint{0.697024in}{0.857143in}}{\pgfqpoint{2.627103in}{1.813434in}}%
\pgfusepath{clip}%
\pgfsetbuttcap%
\pgfsetmiterjoin%
\definecolor{currentfill}{rgb}{0.992771,0.707689,0.712380}%
\pgfsetfillcolor{currentfill}%
\pgfsetlinewidth{0.000000pt}%
\definecolor{currentstroke}{rgb}{0.000000,0.000000,0.000000}%
\pgfsetstrokecolor{currentstroke}%
\pgfsetstrokeopacity{0.000000}%
\pgfsetdash{}{0pt}%
\pgfpathmoveto{\pgfqpoint{2.525541in}{2.312880in}}%
\pgfpathlineto{\pgfqpoint{2.534477in}{2.312880in}}%
\pgfpathlineto{\pgfqpoint{2.534477in}{2.327380in}}%
\pgfpathlineto{\pgfqpoint{2.525541in}{2.327380in}}%
\pgfpathlineto{\pgfqpoint{2.525541in}{2.312880in}}%
\pgfpathclose%
\pgfusepath{fill}%
\end{pgfscope}%
\begin{pgfscope}%
\pgfpathrectangle{\pgfqpoint{0.697024in}{0.857143in}}{\pgfqpoint{2.627103in}{1.813434in}}%
\pgfusepath{clip}%
\pgfsetbuttcap%
\pgfsetmiterjoin%
\definecolor{currentfill}{rgb}{0.992771,0.707689,0.712380}%
\pgfsetfillcolor{currentfill}%
\pgfsetlinewidth{0.000000pt}%
\definecolor{currentstroke}{rgb}{0.000000,0.000000,0.000000}%
\pgfsetstrokecolor{currentstroke}%
\pgfsetstrokeopacity{0.000000}%
\pgfsetdash{}{0pt}%
\pgfpathmoveto{\pgfqpoint{2.536711in}{1.695142in}}%
\pgfpathlineto{\pgfqpoint{2.545648in}{1.695142in}}%
\pgfpathlineto{\pgfqpoint{2.545648in}{1.693617in}}%
\pgfpathlineto{\pgfqpoint{2.536711in}{1.693617in}}%
\pgfpathlineto{\pgfqpoint{2.536711in}{1.695142in}}%
\pgfpathclose%
\pgfusepath{fill}%
\end{pgfscope}%
\begin{pgfscope}%
\pgfpathrectangle{\pgfqpoint{0.697024in}{0.857143in}}{\pgfqpoint{2.627103in}{1.813434in}}%
\pgfusepath{clip}%
\pgfsetbuttcap%
\pgfsetmiterjoin%
\definecolor{currentfill}{rgb}{0.992771,0.707689,0.712380}%
\pgfsetfillcolor{currentfill}%
\pgfsetlinewidth{0.000000pt}%
\definecolor{currentstroke}{rgb}{0.000000,0.000000,0.000000}%
\pgfsetstrokecolor{currentstroke}%
\pgfsetstrokeopacity{0.000000}%
\pgfsetdash{}{0pt}%
\pgfpathmoveto{\pgfqpoint{2.547882in}{1.737511in}}%
\pgfpathlineto{\pgfqpoint{2.556818in}{1.737511in}}%
\pgfpathlineto{\pgfqpoint{2.556818in}{1.733967in}}%
\pgfpathlineto{\pgfqpoint{2.547882in}{1.733967in}}%
\pgfpathlineto{\pgfqpoint{2.547882in}{1.737511in}}%
\pgfpathclose%
\pgfusepath{fill}%
\end{pgfscope}%
\begin{pgfscope}%
\pgfpathrectangle{\pgfqpoint{0.697024in}{0.857143in}}{\pgfqpoint{2.627103in}{1.813434in}}%
\pgfusepath{clip}%
\pgfsetbuttcap%
\pgfsetmiterjoin%
\definecolor{currentfill}{rgb}{0.992771,0.707689,0.712380}%
\pgfsetfillcolor{currentfill}%
\pgfsetlinewidth{0.000000pt}%
\definecolor{currentstroke}{rgb}{0.000000,0.000000,0.000000}%
\pgfsetstrokecolor{currentstroke}%
\pgfsetstrokeopacity{0.000000}%
\pgfsetdash{}{0pt}%
\pgfpathmoveto{\pgfqpoint{2.559053in}{2.389750in}}%
\pgfpathlineto{\pgfqpoint{2.567989in}{2.389750in}}%
\pgfpathlineto{\pgfqpoint{2.567989in}{2.397811in}}%
\pgfpathlineto{\pgfqpoint{2.559053in}{2.397811in}}%
\pgfpathlineto{\pgfqpoint{2.559053in}{2.389750in}}%
\pgfpathclose%
\pgfusepath{fill}%
\end{pgfscope}%
\begin{pgfscope}%
\pgfpathrectangle{\pgfqpoint{0.697024in}{0.857143in}}{\pgfqpoint{2.627103in}{1.813434in}}%
\pgfusepath{clip}%
\pgfsetbuttcap%
\pgfsetmiterjoin%
\definecolor{currentfill}{rgb}{0.992771,0.707689,0.712380}%
\pgfsetfillcolor{currentfill}%
\pgfsetlinewidth{0.000000pt}%
\definecolor{currentstroke}{rgb}{0.000000,0.000000,0.000000}%
\pgfsetstrokecolor{currentstroke}%
\pgfsetstrokeopacity{0.000000}%
\pgfsetdash{}{0pt}%
\pgfpathmoveto{\pgfqpoint{2.570223in}{2.335600in}}%
\pgfpathlineto{\pgfqpoint{2.579160in}{2.335600in}}%
\pgfpathlineto{\pgfqpoint{2.579160in}{2.347475in}}%
\pgfpathlineto{\pgfqpoint{2.570223in}{2.347475in}}%
\pgfpathlineto{\pgfqpoint{2.570223in}{2.335600in}}%
\pgfpathclose%
\pgfusepath{fill}%
\end{pgfscope}%
\begin{pgfscope}%
\pgfpathrectangle{\pgfqpoint{0.697024in}{0.857143in}}{\pgfqpoint{2.627103in}{1.813434in}}%
\pgfusepath{clip}%
\pgfsetbuttcap%
\pgfsetmiterjoin%
\definecolor{currentfill}{rgb}{0.992771,0.707689,0.712380}%
\pgfsetfillcolor{currentfill}%
\pgfsetlinewidth{0.000000pt}%
\definecolor{currentstroke}{rgb}{0.000000,0.000000,0.000000}%
\pgfsetstrokecolor{currentstroke}%
\pgfsetstrokeopacity{0.000000}%
\pgfsetdash{}{0pt}%
\pgfpathmoveto{\pgfqpoint{2.581394in}{1.735680in}}%
\pgfpathlineto{\pgfqpoint{2.590330in}{1.735680in}}%
\pgfpathlineto{\pgfqpoint{2.590330in}{1.729144in}}%
\pgfpathlineto{\pgfqpoint{2.581394in}{1.729144in}}%
\pgfpathlineto{\pgfqpoint{2.581394in}{1.735680in}}%
\pgfpathclose%
\pgfusepath{fill}%
\end{pgfscope}%
\begin{pgfscope}%
\pgfpathrectangle{\pgfqpoint{0.697024in}{0.857143in}}{\pgfqpoint{2.627103in}{1.813434in}}%
\pgfusepath{clip}%
\pgfsetbuttcap%
\pgfsetmiterjoin%
\definecolor{currentfill}{rgb}{0.992771,0.707689,0.712380}%
\pgfsetfillcolor{currentfill}%
\pgfsetlinewidth{0.000000pt}%
\definecolor{currentstroke}{rgb}{0.000000,0.000000,0.000000}%
\pgfsetstrokecolor{currentstroke}%
\pgfsetstrokeopacity{0.000000}%
\pgfsetdash{}{0pt}%
\pgfpathmoveto{\pgfqpoint{2.592564in}{1.778011in}}%
\pgfpathlineto{\pgfqpoint{2.601501in}{1.778011in}}%
\pgfpathlineto{\pgfqpoint{2.601501in}{1.765091in}}%
\pgfpathlineto{\pgfqpoint{2.592564in}{1.765091in}}%
\pgfpathlineto{\pgfqpoint{2.592564in}{1.778011in}}%
\pgfpathclose%
\pgfusepath{fill}%
\end{pgfscope}%
\begin{pgfscope}%
\pgfpathrectangle{\pgfqpoint{0.697024in}{0.857143in}}{\pgfqpoint{2.627103in}{1.813434in}}%
\pgfusepath{clip}%
\pgfsetbuttcap%
\pgfsetmiterjoin%
\definecolor{currentfill}{rgb}{0.992771,0.707689,0.712380}%
\pgfsetfillcolor{currentfill}%
\pgfsetlinewidth{0.000000pt}%
\definecolor{currentstroke}{rgb}{0.000000,0.000000,0.000000}%
\pgfsetstrokecolor{currentstroke}%
\pgfsetstrokeopacity{0.000000}%
\pgfsetdash{}{0pt}%
\pgfpathmoveto{\pgfqpoint{2.603735in}{1.764251in}}%
\pgfpathlineto{\pgfqpoint{2.612672in}{1.764251in}}%
\pgfpathlineto{\pgfqpoint{2.612672in}{1.760565in}}%
\pgfpathlineto{\pgfqpoint{2.603735in}{1.760565in}}%
\pgfpathlineto{\pgfqpoint{2.603735in}{1.764251in}}%
\pgfpathclose%
\pgfusepath{fill}%
\end{pgfscope}%
\begin{pgfscope}%
\pgfpathrectangle{\pgfqpoint{0.697024in}{0.857143in}}{\pgfqpoint{2.627103in}{1.813434in}}%
\pgfusepath{clip}%
\pgfsetbuttcap%
\pgfsetmiterjoin%
\definecolor{currentfill}{rgb}{0.992771,0.707689,0.712380}%
\pgfsetfillcolor{currentfill}%
\pgfsetlinewidth{0.000000pt}%
\definecolor{currentstroke}{rgb}{0.000000,0.000000,0.000000}%
\pgfsetstrokecolor{currentstroke}%
\pgfsetstrokeopacity{0.000000}%
\pgfsetdash{}{0pt}%
\pgfpathmoveto{\pgfqpoint{2.614906in}{1.727025in}}%
\pgfpathlineto{\pgfqpoint{2.623842in}{1.727025in}}%
\pgfpathlineto{\pgfqpoint{2.623842in}{1.710743in}}%
\pgfpathlineto{\pgfqpoint{2.614906in}{1.710743in}}%
\pgfpathlineto{\pgfqpoint{2.614906in}{1.727025in}}%
\pgfpathclose%
\pgfusepath{fill}%
\end{pgfscope}%
\begin{pgfscope}%
\pgfpathrectangle{\pgfqpoint{0.697024in}{0.857143in}}{\pgfqpoint{2.627103in}{1.813434in}}%
\pgfusepath{clip}%
\pgfsetbuttcap%
\pgfsetmiterjoin%
\definecolor{currentfill}{rgb}{0.992771,0.707689,0.712380}%
\pgfsetfillcolor{currentfill}%
\pgfsetlinewidth{0.000000pt}%
\definecolor{currentstroke}{rgb}{0.000000,0.000000,0.000000}%
\pgfsetstrokecolor{currentstroke}%
\pgfsetstrokeopacity{0.000000}%
\pgfsetdash{}{0pt}%
\pgfpathmoveto{\pgfqpoint{2.626076in}{2.268156in}}%
\pgfpathlineto{\pgfqpoint{2.635013in}{2.268156in}}%
\pgfpathlineto{\pgfqpoint{2.635013in}{2.269700in}}%
\pgfpathlineto{\pgfqpoint{2.626076in}{2.269700in}}%
\pgfpathlineto{\pgfqpoint{2.626076in}{2.268156in}}%
\pgfpathclose%
\pgfusepath{fill}%
\end{pgfscope}%
\begin{pgfscope}%
\pgfpathrectangle{\pgfqpoint{0.697024in}{0.857143in}}{\pgfqpoint{2.627103in}{1.813434in}}%
\pgfusepath{clip}%
\pgfsetbuttcap%
\pgfsetmiterjoin%
\definecolor{currentfill}{rgb}{0.992771,0.707689,0.712380}%
\pgfsetfillcolor{currentfill}%
\pgfsetlinewidth{0.000000pt}%
\definecolor{currentstroke}{rgb}{0.000000,0.000000,0.000000}%
\pgfsetstrokecolor{currentstroke}%
\pgfsetstrokeopacity{0.000000}%
\pgfsetdash{}{0pt}%
\pgfpathmoveto{\pgfqpoint{2.637247in}{1.767072in}}%
\pgfpathlineto{\pgfqpoint{2.646183in}{1.767072in}}%
\pgfpathlineto{\pgfqpoint{2.646183in}{1.740045in}}%
\pgfpathlineto{\pgfqpoint{2.637247in}{1.740045in}}%
\pgfpathlineto{\pgfqpoint{2.637247in}{1.767072in}}%
\pgfpathclose%
\pgfusepath{fill}%
\end{pgfscope}%
\begin{pgfscope}%
\pgfpathrectangle{\pgfqpoint{0.697024in}{0.857143in}}{\pgfqpoint{2.627103in}{1.813434in}}%
\pgfusepath{clip}%
\pgfsetbuttcap%
\pgfsetmiterjoin%
\definecolor{currentfill}{rgb}{0.992771,0.707689,0.712380}%
\pgfsetfillcolor{currentfill}%
\pgfsetlinewidth{0.000000pt}%
\definecolor{currentstroke}{rgb}{0.000000,0.000000,0.000000}%
\pgfsetstrokecolor{currentstroke}%
\pgfsetstrokeopacity{0.000000}%
\pgfsetdash{}{0pt}%
\pgfpathmoveto{\pgfqpoint{2.648417in}{1.745337in}}%
\pgfpathlineto{\pgfqpoint{2.657354in}{1.745337in}}%
\pgfpathlineto{\pgfqpoint{2.657354in}{1.727933in}}%
\pgfpathlineto{\pgfqpoint{2.648417in}{1.727933in}}%
\pgfpathlineto{\pgfqpoint{2.648417in}{1.745337in}}%
\pgfpathclose%
\pgfusepath{fill}%
\end{pgfscope}%
\begin{pgfscope}%
\pgfpathrectangle{\pgfqpoint{0.697024in}{0.857143in}}{\pgfqpoint{2.627103in}{1.813434in}}%
\pgfusepath{clip}%
\pgfsetbuttcap%
\pgfsetmiterjoin%
\definecolor{currentfill}{rgb}{0.992771,0.707689,0.712380}%
\pgfsetfillcolor{currentfill}%
\pgfsetlinewidth{0.000000pt}%
\definecolor{currentstroke}{rgb}{0.000000,0.000000,0.000000}%
\pgfsetstrokecolor{currentstroke}%
\pgfsetstrokeopacity{0.000000}%
\pgfsetdash{}{0pt}%
\pgfpathmoveto{\pgfqpoint{2.659588in}{1.746514in}}%
\pgfpathlineto{\pgfqpoint{2.668525in}{1.746514in}}%
\pgfpathlineto{\pgfqpoint{2.668525in}{1.732358in}}%
\pgfpathlineto{\pgfqpoint{2.659588in}{1.732358in}}%
\pgfpathlineto{\pgfqpoint{2.659588in}{1.746514in}}%
\pgfpathclose%
\pgfusepath{fill}%
\end{pgfscope}%
\begin{pgfscope}%
\pgfpathrectangle{\pgfqpoint{0.697024in}{0.857143in}}{\pgfqpoint{2.627103in}{1.813434in}}%
\pgfusepath{clip}%
\pgfsetbuttcap%
\pgfsetmiterjoin%
\definecolor{currentfill}{rgb}{0.992771,0.707689,0.712380}%
\pgfsetfillcolor{currentfill}%
\pgfsetlinewidth{0.000000pt}%
\definecolor{currentstroke}{rgb}{0.000000,0.000000,0.000000}%
\pgfsetstrokecolor{currentstroke}%
\pgfsetstrokeopacity{0.000000}%
\pgfsetdash{}{0pt}%
\pgfpathmoveto{\pgfqpoint{2.670759in}{1.766105in}}%
\pgfpathlineto{\pgfqpoint{2.679695in}{1.766105in}}%
\pgfpathlineto{\pgfqpoint{2.679695in}{1.763697in}}%
\pgfpathlineto{\pgfqpoint{2.670759in}{1.763697in}}%
\pgfpathlineto{\pgfqpoint{2.670759in}{1.766105in}}%
\pgfpathclose%
\pgfusepath{fill}%
\end{pgfscope}%
\begin{pgfscope}%
\pgfpathrectangle{\pgfqpoint{0.697024in}{0.857143in}}{\pgfqpoint{2.627103in}{1.813434in}}%
\pgfusepath{clip}%
\pgfsetbuttcap%
\pgfsetmiterjoin%
\definecolor{currentfill}{rgb}{0.992771,0.707689,0.712380}%
\pgfsetfillcolor{currentfill}%
\pgfsetlinewidth{0.000000pt}%
\definecolor{currentstroke}{rgb}{0.000000,0.000000,0.000000}%
\pgfsetstrokecolor{currentstroke}%
\pgfsetstrokeopacity{0.000000}%
\pgfsetdash{}{0pt}%
\pgfpathmoveto{\pgfqpoint{2.681929in}{1.795130in}}%
\pgfpathlineto{\pgfqpoint{2.690866in}{1.795130in}}%
\pgfpathlineto{\pgfqpoint{2.690866in}{1.776163in}}%
\pgfpathlineto{\pgfqpoint{2.681929in}{1.776163in}}%
\pgfpathlineto{\pgfqpoint{2.681929in}{1.795130in}}%
\pgfpathclose%
\pgfusepath{fill}%
\end{pgfscope}%
\begin{pgfscope}%
\pgfpathrectangle{\pgfqpoint{0.697024in}{0.857143in}}{\pgfqpoint{2.627103in}{1.813434in}}%
\pgfusepath{clip}%
\pgfsetbuttcap%
\pgfsetmiterjoin%
\definecolor{currentfill}{rgb}{0.992771,0.707689,0.712380}%
\pgfsetfillcolor{currentfill}%
\pgfsetlinewidth{0.000000pt}%
\definecolor{currentstroke}{rgb}{0.000000,0.000000,0.000000}%
\pgfsetstrokecolor{currentstroke}%
\pgfsetstrokeopacity{0.000000}%
\pgfsetdash{}{0pt}%
\pgfpathmoveto{\pgfqpoint{2.693100in}{2.073518in}}%
\pgfpathlineto{\pgfqpoint{2.702036in}{2.073518in}}%
\pgfpathlineto{\pgfqpoint{2.702036in}{2.122566in}}%
\pgfpathlineto{\pgfqpoint{2.693100in}{2.122566in}}%
\pgfpathlineto{\pgfqpoint{2.693100in}{2.073518in}}%
\pgfpathclose%
\pgfusepath{fill}%
\end{pgfscope}%
\begin{pgfscope}%
\pgfpathrectangle{\pgfqpoint{0.697024in}{0.857143in}}{\pgfqpoint{2.627103in}{1.813434in}}%
\pgfusepath{clip}%
\pgfsetbuttcap%
\pgfsetmiterjoin%
\definecolor{currentfill}{rgb}{0.992771,0.707689,0.712380}%
\pgfsetfillcolor{currentfill}%
\pgfsetlinewidth{0.000000pt}%
\definecolor{currentstroke}{rgb}{0.000000,0.000000,0.000000}%
\pgfsetstrokecolor{currentstroke}%
\pgfsetstrokeopacity{0.000000}%
\pgfsetdash{}{0pt}%
\pgfpathmoveto{\pgfqpoint{2.704270in}{2.066740in}}%
\pgfpathlineto{\pgfqpoint{2.713207in}{2.066740in}}%
\pgfpathlineto{\pgfqpoint{2.713207in}{2.106678in}}%
\pgfpathlineto{\pgfqpoint{2.704270in}{2.106678in}}%
\pgfpathlineto{\pgfqpoint{2.704270in}{2.066740in}}%
\pgfpathclose%
\pgfusepath{fill}%
\end{pgfscope}%
\begin{pgfscope}%
\pgfpathrectangle{\pgfqpoint{0.697024in}{0.857143in}}{\pgfqpoint{2.627103in}{1.813434in}}%
\pgfusepath{clip}%
\pgfsetbuttcap%
\pgfsetmiterjoin%
\definecolor{currentfill}{rgb}{0.992771,0.707689,0.712380}%
\pgfsetfillcolor{currentfill}%
\pgfsetlinewidth{0.000000pt}%
\definecolor{currentstroke}{rgb}{0.000000,0.000000,0.000000}%
\pgfsetstrokecolor{currentstroke}%
\pgfsetstrokeopacity{0.000000}%
\pgfsetdash{}{0pt}%
\pgfpathmoveto{\pgfqpoint{2.715441in}{2.172620in}}%
\pgfpathlineto{\pgfqpoint{2.724378in}{2.172620in}}%
\pgfpathlineto{\pgfqpoint{2.724378in}{2.184924in}}%
\pgfpathlineto{\pgfqpoint{2.715441in}{2.184924in}}%
\pgfpathlineto{\pgfqpoint{2.715441in}{2.172620in}}%
\pgfpathclose%
\pgfusepath{fill}%
\end{pgfscope}%
\begin{pgfscope}%
\pgfpathrectangle{\pgfqpoint{0.697024in}{0.857143in}}{\pgfqpoint{2.627103in}{1.813434in}}%
\pgfusepath{clip}%
\pgfsetbuttcap%
\pgfsetmiterjoin%
\definecolor{currentfill}{rgb}{0.992771,0.707689,0.712380}%
\pgfsetfillcolor{currentfill}%
\pgfsetlinewidth{0.000000pt}%
\definecolor{currentstroke}{rgb}{0.000000,0.000000,0.000000}%
\pgfsetstrokecolor{currentstroke}%
\pgfsetstrokeopacity{0.000000}%
\pgfsetdash{}{0pt}%
\pgfpathmoveto{\pgfqpoint{2.726612in}{2.233234in}}%
\pgfpathlineto{\pgfqpoint{2.735548in}{2.233234in}}%
\pgfpathlineto{\pgfqpoint{2.735548in}{2.255363in}}%
\pgfpathlineto{\pgfqpoint{2.726612in}{2.255363in}}%
\pgfpathlineto{\pgfqpoint{2.726612in}{2.233234in}}%
\pgfpathclose%
\pgfusepath{fill}%
\end{pgfscope}%
\begin{pgfscope}%
\pgfpathrectangle{\pgfqpoint{0.697024in}{0.857143in}}{\pgfqpoint{2.627103in}{1.813434in}}%
\pgfusepath{clip}%
\pgfsetbuttcap%
\pgfsetmiterjoin%
\definecolor{currentfill}{rgb}{0.992771,0.707689,0.712380}%
\pgfsetfillcolor{currentfill}%
\pgfsetlinewidth{0.000000pt}%
\definecolor{currentstroke}{rgb}{0.000000,0.000000,0.000000}%
\pgfsetstrokecolor{currentstroke}%
\pgfsetstrokeopacity{0.000000}%
\pgfsetdash{}{0pt}%
\pgfpathmoveto{\pgfqpoint{2.737782in}{2.182782in}}%
\pgfpathlineto{\pgfqpoint{2.746719in}{2.182782in}}%
\pgfpathlineto{\pgfqpoint{2.746719in}{2.221516in}}%
\pgfpathlineto{\pgfqpoint{2.737782in}{2.221516in}}%
\pgfpathlineto{\pgfqpoint{2.737782in}{2.182782in}}%
\pgfpathclose%
\pgfusepath{fill}%
\end{pgfscope}%
\begin{pgfscope}%
\pgfpathrectangle{\pgfqpoint{0.697024in}{0.857143in}}{\pgfqpoint{2.627103in}{1.813434in}}%
\pgfusepath{clip}%
\pgfsetbuttcap%
\pgfsetmiterjoin%
\definecolor{currentfill}{rgb}{0.992771,0.707689,0.712380}%
\pgfsetfillcolor{currentfill}%
\pgfsetlinewidth{0.000000pt}%
\definecolor{currentstroke}{rgb}{0.000000,0.000000,0.000000}%
\pgfsetstrokecolor{currentstroke}%
\pgfsetstrokeopacity{0.000000}%
\pgfsetdash{}{0pt}%
\pgfpathmoveto{\pgfqpoint{2.748953in}{2.219148in}}%
\pgfpathlineto{\pgfqpoint{2.757889in}{2.219148in}}%
\pgfpathlineto{\pgfqpoint{2.757889in}{2.260671in}}%
\pgfpathlineto{\pgfqpoint{2.748953in}{2.260671in}}%
\pgfpathlineto{\pgfqpoint{2.748953in}{2.219148in}}%
\pgfpathclose%
\pgfusepath{fill}%
\end{pgfscope}%
\begin{pgfscope}%
\pgfpathrectangle{\pgfqpoint{0.697024in}{0.857143in}}{\pgfqpoint{2.627103in}{1.813434in}}%
\pgfusepath{clip}%
\pgfsetbuttcap%
\pgfsetmiterjoin%
\definecolor{currentfill}{rgb}{0.992771,0.707689,0.712380}%
\pgfsetfillcolor{currentfill}%
\pgfsetlinewidth{0.000000pt}%
\definecolor{currentstroke}{rgb}{0.000000,0.000000,0.000000}%
\pgfsetstrokecolor{currentstroke}%
\pgfsetstrokeopacity{0.000000}%
\pgfsetdash{}{0pt}%
\pgfpathmoveto{\pgfqpoint{2.760124in}{2.232120in}}%
\pgfpathlineto{\pgfqpoint{2.769060in}{2.232120in}}%
\pgfpathlineto{\pgfqpoint{2.769060in}{2.282441in}}%
\pgfpathlineto{\pgfqpoint{2.760124in}{2.282441in}}%
\pgfpathlineto{\pgfqpoint{2.760124in}{2.232120in}}%
\pgfpathclose%
\pgfusepath{fill}%
\end{pgfscope}%
\begin{pgfscope}%
\pgfpathrectangle{\pgfqpoint{0.697024in}{0.857143in}}{\pgfqpoint{2.627103in}{1.813434in}}%
\pgfusepath{clip}%
\pgfsetbuttcap%
\pgfsetmiterjoin%
\definecolor{currentfill}{rgb}{0.992771,0.707689,0.712380}%
\pgfsetfillcolor{currentfill}%
\pgfsetlinewidth{0.000000pt}%
\definecolor{currentstroke}{rgb}{0.000000,0.000000,0.000000}%
\pgfsetstrokecolor{currentstroke}%
\pgfsetstrokeopacity{0.000000}%
\pgfsetdash{}{0pt}%
\pgfpathmoveto{\pgfqpoint{2.771294in}{2.181529in}}%
\pgfpathlineto{\pgfqpoint{2.780231in}{2.181529in}}%
\pgfpathlineto{\pgfqpoint{2.780231in}{2.235349in}}%
\pgfpathlineto{\pgfqpoint{2.771294in}{2.235349in}}%
\pgfpathlineto{\pgfqpoint{2.771294in}{2.181529in}}%
\pgfpathclose%
\pgfusepath{fill}%
\end{pgfscope}%
\begin{pgfscope}%
\pgfpathrectangle{\pgfqpoint{0.697024in}{0.857143in}}{\pgfqpoint{2.627103in}{1.813434in}}%
\pgfusepath{clip}%
\pgfsetbuttcap%
\pgfsetmiterjoin%
\definecolor{currentfill}{rgb}{0.992771,0.707689,0.712380}%
\pgfsetfillcolor{currentfill}%
\pgfsetlinewidth{0.000000pt}%
\definecolor{currentstroke}{rgb}{0.000000,0.000000,0.000000}%
\pgfsetstrokecolor{currentstroke}%
\pgfsetstrokeopacity{0.000000}%
\pgfsetdash{}{0pt}%
\pgfpathmoveto{\pgfqpoint{2.782465in}{2.197647in}}%
\pgfpathlineto{\pgfqpoint{2.791401in}{2.197647in}}%
\pgfpathlineto{\pgfqpoint{2.791401in}{2.244897in}}%
\pgfpathlineto{\pgfqpoint{2.782465in}{2.244897in}}%
\pgfpathlineto{\pgfqpoint{2.782465in}{2.197647in}}%
\pgfpathclose%
\pgfusepath{fill}%
\end{pgfscope}%
\begin{pgfscope}%
\pgfpathrectangle{\pgfqpoint{0.697024in}{0.857143in}}{\pgfqpoint{2.627103in}{1.813434in}}%
\pgfusepath{clip}%
\pgfsetbuttcap%
\pgfsetmiterjoin%
\definecolor{currentfill}{rgb}{0.992771,0.707689,0.712380}%
\pgfsetfillcolor{currentfill}%
\pgfsetlinewidth{0.000000pt}%
\definecolor{currentstroke}{rgb}{0.000000,0.000000,0.000000}%
\pgfsetstrokecolor{currentstroke}%
\pgfsetstrokeopacity{0.000000}%
\pgfsetdash{}{0pt}%
\pgfpathmoveto{\pgfqpoint{2.793635in}{2.195823in}}%
\pgfpathlineto{\pgfqpoint{2.802572in}{2.195823in}}%
\pgfpathlineto{\pgfqpoint{2.802572in}{2.264634in}}%
\pgfpathlineto{\pgfqpoint{2.793635in}{2.264634in}}%
\pgfpathlineto{\pgfqpoint{2.793635in}{2.195823in}}%
\pgfpathclose%
\pgfusepath{fill}%
\end{pgfscope}%
\begin{pgfscope}%
\pgfpathrectangle{\pgfqpoint{0.697024in}{0.857143in}}{\pgfqpoint{2.627103in}{1.813434in}}%
\pgfusepath{clip}%
\pgfsetbuttcap%
\pgfsetmiterjoin%
\definecolor{currentfill}{rgb}{0.992771,0.707689,0.712380}%
\pgfsetfillcolor{currentfill}%
\pgfsetlinewidth{0.000000pt}%
\definecolor{currentstroke}{rgb}{0.000000,0.000000,0.000000}%
\pgfsetstrokecolor{currentstroke}%
\pgfsetstrokeopacity{0.000000}%
\pgfsetdash{}{0pt}%
\pgfpathmoveto{\pgfqpoint{2.804806in}{2.189845in}}%
\pgfpathlineto{\pgfqpoint{2.813742in}{2.189845in}}%
\pgfpathlineto{\pgfqpoint{2.813742in}{2.258593in}}%
\pgfpathlineto{\pgfqpoint{2.804806in}{2.258593in}}%
\pgfpathlineto{\pgfqpoint{2.804806in}{2.189845in}}%
\pgfpathclose%
\pgfusepath{fill}%
\end{pgfscope}%
\begin{pgfscope}%
\pgfpathrectangle{\pgfqpoint{0.697024in}{0.857143in}}{\pgfqpoint{2.627103in}{1.813434in}}%
\pgfusepath{clip}%
\pgfsetbuttcap%
\pgfsetmiterjoin%
\definecolor{currentfill}{rgb}{0.992771,0.707689,0.712380}%
\pgfsetfillcolor{currentfill}%
\pgfsetlinewidth{0.000000pt}%
\definecolor{currentstroke}{rgb}{0.000000,0.000000,0.000000}%
\pgfsetstrokecolor{currentstroke}%
\pgfsetstrokeopacity{0.000000}%
\pgfsetdash{}{0pt}%
\pgfpathmoveto{\pgfqpoint{2.815977in}{2.099166in}}%
\pgfpathlineto{\pgfqpoint{2.824913in}{2.099166in}}%
\pgfpathlineto{\pgfqpoint{2.824913in}{2.165047in}}%
\pgfpathlineto{\pgfqpoint{2.815977in}{2.165047in}}%
\pgfpathlineto{\pgfqpoint{2.815977in}{2.099166in}}%
\pgfpathclose%
\pgfusepath{fill}%
\end{pgfscope}%
\begin{pgfscope}%
\pgfpathrectangle{\pgfqpoint{0.697024in}{0.857143in}}{\pgfqpoint{2.627103in}{1.813434in}}%
\pgfusepath{clip}%
\pgfsetbuttcap%
\pgfsetmiterjoin%
\definecolor{currentfill}{rgb}{0.992771,0.707689,0.712380}%
\pgfsetfillcolor{currentfill}%
\pgfsetlinewidth{0.000000pt}%
\definecolor{currentstroke}{rgb}{0.000000,0.000000,0.000000}%
\pgfsetstrokecolor{currentstroke}%
\pgfsetstrokeopacity{0.000000}%
\pgfsetdash{}{0pt}%
\pgfpathmoveto{\pgfqpoint{2.827147in}{2.101684in}}%
\pgfpathlineto{\pgfqpoint{2.836084in}{2.101684in}}%
\pgfpathlineto{\pgfqpoint{2.836084in}{2.195867in}}%
\pgfpathlineto{\pgfqpoint{2.827147in}{2.195867in}}%
\pgfpathlineto{\pgfqpoint{2.827147in}{2.101684in}}%
\pgfpathclose%
\pgfusepath{fill}%
\end{pgfscope}%
\begin{pgfscope}%
\pgfpathrectangle{\pgfqpoint{0.697024in}{0.857143in}}{\pgfqpoint{2.627103in}{1.813434in}}%
\pgfusepath{clip}%
\pgfsetbuttcap%
\pgfsetmiterjoin%
\definecolor{currentfill}{rgb}{0.992771,0.707689,0.712380}%
\pgfsetfillcolor{currentfill}%
\pgfsetlinewidth{0.000000pt}%
\definecolor{currentstroke}{rgb}{0.000000,0.000000,0.000000}%
\pgfsetstrokecolor{currentstroke}%
\pgfsetstrokeopacity{0.000000}%
\pgfsetdash{}{0pt}%
\pgfpathmoveto{\pgfqpoint{2.838318in}{2.162765in}}%
\pgfpathlineto{\pgfqpoint{2.847254in}{2.162765in}}%
\pgfpathlineto{\pgfqpoint{2.847254in}{2.235650in}}%
\pgfpathlineto{\pgfqpoint{2.838318in}{2.235650in}}%
\pgfpathlineto{\pgfqpoint{2.838318in}{2.162765in}}%
\pgfpathclose%
\pgfusepath{fill}%
\end{pgfscope}%
\begin{pgfscope}%
\pgfpathrectangle{\pgfqpoint{0.697024in}{0.857143in}}{\pgfqpoint{2.627103in}{1.813434in}}%
\pgfusepath{clip}%
\pgfsetbuttcap%
\pgfsetmiterjoin%
\definecolor{currentfill}{rgb}{0.992771,0.707689,0.712380}%
\pgfsetfillcolor{currentfill}%
\pgfsetlinewidth{0.000000pt}%
\definecolor{currentstroke}{rgb}{0.000000,0.000000,0.000000}%
\pgfsetstrokecolor{currentstroke}%
\pgfsetstrokeopacity{0.000000}%
\pgfsetdash{}{0pt}%
\pgfpathmoveto{\pgfqpoint{2.849488in}{2.053999in}}%
\pgfpathlineto{\pgfqpoint{2.858425in}{2.053999in}}%
\pgfpathlineto{\pgfqpoint{2.858425in}{2.136681in}}%
\pgfpathlineto{\pgfqpoint{2.849488in}{2.136681in}}%
\pgfpathlineto{\pgfqpoint{2.849488in}{2.053999in}}%
\pgfpathclose%
\pgfusepath{fill}%
\end{pgfscope}%
\begin{pgfscope}%
\pgfpathrectangle{\pgfqpoint{0.697024in}{0.857143in}}{\pgfqpoint{2.627103in}{1.813434in}}%
\pgfusepath{clip}%
\pgfsetbuttcap%
\pgfsetmiterjoin%
\definecolor{currentfill}{rgb}{0.992771,0.707689,0.712380}%
\pgfsetfillcolor{currentfill}%
\pgfsetlinewidth{0.000000pt}%
\definecolor{currentstroke}{rgb}{0.000000,0.000000,0.000000}%
\pgfsetstrokecolor{currentstroke}%
\pgfsetstrokeopacity{0.000000}%
\pgfsetdash{}{0pt}%
\pgfpathmoveto{\pgfqpoint{2.860659in}{2.084961in}}%
\pgfpathlineto{\pgfqpoint{2.869595in}{2.084961in}}%
\pgfpathlineto{\pgfqpoint{2.869595in}{2.164013in}}%
\pgfpathlineto{\pgfqpoint{2.860659in}{2.164013in}}%
\pgfpathlineto{\pgfqpoint{2.860659in}{2.084961in}}%
\pgfpathclose%
\pgfusepath{fill}%
\end{pgfscope}%
\begin{pgfscope}%
\pgfpathrectangle{\pgfqpoint{0.697024in}{0.857143in}}{\pgfqpoint{2.627103in}{1.813434in}}%
\pgfusepath{clip}%
\pgfsetbuttcap%
\pgfsetmiterjoin%
\definecolor{currentfill}{rgb}{0.992771,0.707689,0.712380}%
\pgfsetfillcolor{currentfill}%
\pgfsetlinewidth{0.000000pt}%
\definecolor{currentstroke}{rgb}{0.000000,0.000000,0.000000}%
\pgfsetstrokecolor{currentstroke}%
\pgfsetstrokeopacity{0.000000}%
\pgfsetdash{}{0pt}%
\pgfpathmoveto{\pgfqpoint{2.871830in}{2.026389in}}%
\pgfpathlineto{\pgfqpoint{2.880766in}{2.026389in}}%
\pgfpathlineto{\pgfqpoint{2.880766in}{2.105131in}}%
\pgfpathlineto{\pgfqpoint{2.871830in}{2.105131in}}%
\pgfpathlineto{\pgfqpoint{2.871830in}{2.026389in}}%
\pgfpathclose%
\pgfusepath{fill}%
\end{pgfscope}%
\begin{pgfscope}%
\pgfpathrectangle{\pgfqpoint{0.697024in}{0.857143in}}{\pgfqpoint{2.627103in}{1.813434in}}%
\pgfusepath{clip}%
\pgfsetbuttcap%
\pgfsetmiterjoin%
\definecolor{currentfill}{rgb}{0.992771,0.707689,0.712380}%
\pgfsetfillcolor{currentfill}%
\pgfsetlinewidth{0.000000pt}%
\definecolor{currentstroke}{rgb}{0.000000,0.000000,0.000000}%
\pgfsetstrokecolor{currentstroke}%
\pgfsetstrokeopacity{0.000000}%
\pgfsetdash{}{0pt}%
\pgfpathmoveto{\pgfqpoint{2.883000in}{1.992868in}}%
\pgfpathlineto{\pgfqpoint{2.891937in}{1.992868in}}%
\pgfpathlineto{\pgfqpoint{2.891937in}{2.088448in}}%
\pgfpathlineto{\pgfqpoint{2.883000in}{2.088448in}}%
\pgfpathlineto{\pgfqpoint{2.883000in}{1.992868in}}%
\pgfpathclose%
\pgfusepath{fill}%
\end{pgfscope}%
\begin{pgfscope}%
\pgfpathrectangle{\pgfqpoint{0.697024in}{0.857143in}}{\pgfqpoint{2.627103in}{1.813434in}}%
\pgfusepath{clip}%
\pgfsetbuttcap%
\pgfsetmiterjoin%
\definecolor{currentfill}{rgb}{0.992771,0.707689,0.712380}%
\pgfsetfillcolor{currentfill}%
\pgfsetlinewidth{0.000000pt}%
\definecolor{currentstroke}{rgb}{0.000000,0.000000,0.000000}%
\pgfsetstrokecolor{currentstroke}%
\pgfsetstrokeopacity{0.000000}%
\pgfsetdash{}{0pt}%
\pgfpathmoveto{\pgfqpoint{2.894171in}{2.048622in}}%
\pgfpathlineto{\pgfqpoint{2.903107in}{2.048622in}}%
\pgfpathlineto{\pgfqpoint{2.903107in}{2.134520in}}%
\pgfpathlineto{\pgfqpoint{2.894171in}{2.134520in}}%
\pgfpathlineto{\pgfqpoint{2.894171in}{2.048622in}}%
\pgfpathclose%
\pgfusepath{fill}%
\end{pgfscope}%
\begin{pgfscope}%
\pgfpathrectangle{\pgfqpoint{0.697024in}{0.857143in}}{\pgfqpoint{2.627103in}{1.813434in}}%
\pgfusepath{clip}%
\pgfsetbuttcap%
\pgfsetmiterjoin%
\definecolor{currentfill}{rgb}{0.992771,0.707689,0.712380}%
\pgfsetfillcolor{currentfill}%
\pgfsetlinewidth{0.000000pt}%
\definecolor{currentstroke}{rgb}{0.000000,0.000000,0.000000}%
\pgfsetstrokecolor{currentstroke}%
\pgfsetstrokeopacity{0.000000}%
\pgfsetdash{}{0pt}%
\pgfpathmoveto{\pgfqpoint{2.905341in}{2.052593in}}%
\pgfpathlineto{\pgfqpoint{2.914278in}{2.052593in}}%
\pgfpathlineto{\pgfqpoint{2.914278in}{2.118468in}}%
\pgfpathlineto{\pgfqpoint{2.905341in}{2.118468in}}%
\pgfpathlineto{\pgfqpoint{2.905341in}{2.052593in}}%
\pgfpathclose%
\pgfusepath{fill}%
\end{pgfscope}%
\begin{pgfscope}%
\pgfpathrectangle{\pgfqpoint{0.697024in}{0.857143in}}{\pgfqpoint{2.627103in}{1.813434in}}%
\pgfusepath{clip}%
\pgfsetbuttcap%
\pgfsetmiterjoin%
\definecolor{currentfill}{rgb}{0.992771,0.707689,0.712380}%
\pgfsetfillcolor{currentfill}%
\pgfsetlinewidth{0.000000pt}%
\definecolor{currentstroke}{rgb}{0.000000,0.000000,0.000000}%
\pgfsetstrokecolor{currentstroke}%
\pgfsetstrokeopacity{0.000000}%
\pgfsetdash{}{0pt}%
\pgfpathmoveto{\pgfqpoint{2.916512in}{2.010638in}}%
\pgfpathlineto{\pgfqpoint{2.925448in}{2.010638in}}%
\pgfpathlineto{\pgfqpoint{2.925448in}{2.085388in}}%
\pgfpathlineto{\pgfqpoint{2.916512in}{2.085388in}}%
\pgfpathlineto{\pgfqpoint{2.916512in}{2.010638in}}%
\pgfpathclose%
\pgfusepath{fill}%
\end{pgfscope}%
\begin{pgfscope}%
\pgfpathrectangle{\pgfqpoint{0.697024in}{0.857143in}}{\pgfqpoint{2.627103in}{1.813434in}}%
\pgfusepath{clip}%
\pgfsetbuttcap%
\pgfsetmiterjoin%
\definecolor{currentfill}{rgb}{0.992771,0.707689,0.712380}%
\pgfsetfillcolor{currentfill}%
\pgfsetlinewidth{0.000000pt}%
\definecolor{currentstroke}{rgb}{0.000000,0.000000,0.000000}%
\pgfsetstrokecolor{currentstroke}%
\pgfsetstrokeopacity{0.000000}%
\pgfsetdash{}{0pt}%
\pgfpathmoveto{\pgfqpoint{2.927683in}{1.996817in}}%
\pgfpathlineto{\pgfqpoint{2.936619in}{1.996817in}}%
\pgfpathlineto{\pgfqpoint{2.936619in}{2.089564in}}%
\pgfpathlineto{\pgfqpoint{2.927683in}{2.089564in}}%
\pgfpathlineto{\pgfqpoint{2.927683in}{1.996817in}}%
\pgfpathclose%
\pgfusepath{fill}%
\end{pgfscope}%
\begin{pgfscope}%
\pgfpathrectangle{\pgfqpoint{0.697024in}{0.857143in}}{\pgfqpoint{2.627103in}{1.813434in}}%
\pgfusepath{clip}%
\pgfsetbuttcap%
\pgfsetmiterjoin%
\definecolor{currentfill}{rgb}{0.992771,0.707689,0.712380}%
\pgfsetfillcolor{currentfill}%
\pgfsetlinewidth{0.000000pt}%
\definecolor{currentstroke}{rgb}{0.000000,0.000000,0.000000}%
\pgfsetstrokecolor{currentstroke}%
\pgfsetstrokeopacity{0.000000}%
\pgfsetdash{}{0pt}%
\pgfpathmoveto{\pgfqpoint{2.938853in}{2.023033in}}%
\pgfpathlineto{\pgfqpoint{2.947790in}{2.023033in}}%
\pgfpathlineto{\pgfqpoint{2.947790in}{2.117440in}}%
\pgfpathlineto{\pgfqpoint{2.938853in}{2.117440in}}%
\pgfpathlineto{\pgfqpoint{2.938853in}{2.023033in}}%
\pgfpathclose%
\pgfusepath{fill}%
\end{pgfscope}%
\begin{pgfscope}%
\pgfpathrectangle{\pgfqpoint{0.697024in}{0.857143in}}{\pgfqpoint{2.627103in}{1.813434in}}%
\pgfusepath{clip}%
\pgfsetbuttcap%
\pgfsetmiterjoin%
\definecolor{currentfill}{rgb}{0.992771,0.707689,0.712380}%
\pgfsetfillcolor{currentfill}%
\pgfsetlinewidth{0.000000pt}%
\definecolor{currentstroke}{rgb}{0.000000,0.000000,0.000000}%
\pgfsetstrokecolor{currentstroke}%
\pgfsetstrokeopacity{0.000000}%
\pgfsetdash{}{0pt}%
\pgfpathmoveto{\pgfqpoint{2.950024in}{2.000730in}}%
\pgfpathlineto{\pgfqpoint{2.958960in}{2.000730in}}%
\pgfpathlineto{\pgfqpoint{2.958960in}{2.071512in}}%
\pgfpathlineto{\pgfqpoint{2.950024in}{2.071512in}}%
\pgfpathlineto{\pgfqpoint{2.950024in}{2.000730in}}%
\pgfpathclose%
\pgfusepath{fill}%
\end{pgfscope}%
\begin{pgfscope}%
\pgfpathrectangle{\pgfqpoint{0.697024in}{0.857143in}}{\pgfqpoint{2.627103in}{1.813434in}}%
\pgfusepath{clip}%
\pgfsetbuttcap%
\pgfsetmiterjoin%
\definecolor{currentfill}{rgb}{0.992771,0.707689,0.712380}%
\pgfsetfillcolor{currentfill}%
\pgfsetlinewidth{0.000000pt}%
\definecolor{currentstroke}{rgb}{0.000000,0.000000,0.000000}%
\pgfsetstrokecolor{currentstroke}%
\pgfsetstrokeopacity{0.000000}%
\pgfsetdash{}{0pt}%
\pgfpathmoveto{\pgfqpoint{2.961194in}{1.996641in}}%
\pgfpathlineto{\pgfqpoint{2.970131in}{1.996641in}}%
\pgfpathlineto{\pgfqpoint{2.970131in}{2.068641in}}%
\pgfpathlineto{\pgfqpoint{2.961194in}{2.068641in}}%
\pgfpathlineto{\pgfqpoint{2.961194in}{1.996641in}}%
\pgfpathclose%
\pgfusepath{fill}%
\end{pgfscope}%
\begin{pgfscope}%
\pgfpathrectangle{\pgfqpoint{0.697024in}{0.857143in}}{\pgfqpoint{2.627103in}{1.813434in}}%
\pgfusepath{clip}%
\pgfsetbuttcap%
\pgfsetmiterjoin%
\definecolor{currentfill}{rgb}{0.992771,0.707689,0.712380}%
\pgfsetfillcolor{currentfill}%
\pgfsetlinewidth{0.000000pt}%
\definecolor{currentstroke}{rgb}{0.000000,0.000000,0.000000}%
\pgfsetstrokecolor{currentstroke}%
\pgfsetstrokeopacity{0.000000}%
\pgfsetdash{}{0pt}%
\pgfpathmoveto{\pgfqpoint{2.972365in}{2.035761in}}%
\pgfpathlineto{\pgfqpoint{2.981301in}{2.035761in}}%
\pgfpathlineto{\pgfqpoint{2.981301in}{2.115381in}}%
\pgfpathlineto{\pgfqpoint{2.972365in}{2.115381in}}%
\pgfpathlineto{\pgfqpoint{2.972365in}{2.035761in}}%
\pgfpathclose%
\pgfusepath{fill}%
\end{pgfscope}%
\begin{pgfscope}%
\pgfpathrectangle{\pgfqpoint{0.697024in}{0.857143in}}{\pgfqpoint{2.627103in}{1.813434in}}%
\pgfusepath{clip}%
\pgfsetbuttcap%
\pgfsetmiterjoin%
\definecolor{currentfill}{rgb}{0.992771,0.707689,0.712380}%
\pgfsetfillcolor{currentfill}%
\pgfsetlinewidth{0.000000pt}%
\definecolor{currentstroke}{rgb}{0.000000,0.000000,0.000000}%
\pgfsetstrokecolor{currentstroke}%
\pgfsetstrokeopacity{0.000000}%
\pgfsetdash{}{0pt}%
\pgfpathmoveto{\pgfqpoint{2.983536in}{2.090520in}}%
\pgfpathlineto{\pgfqpoint{2.992472in}{2.090520in}}%
\pgfpathlineto{\pgfqpoint{2.992472in}{2.141617in}}%
\pgfpathlineto{\pgfqpoint{2.983536in}{2.141617in}}%
\pgfpathlineto{\pgfqpoint{2.983536in}{2.090520in}}%
\pgfpathclose%
\pgfusepath{fill}%
\end{pgfscope}%
\begin{pgfscope}%
\pgfpathrectangle{\pgfqpoint{0.697024in}{0.857143in}}{\pgfqpoint{2.627103in}{1.813434in}}%
\pgfusepath{clip}%
\pgfsetbuttcap%
\pgfsetmiterjoin%
\definecolor{currentfill}{rgb}{0.992771,0.707689,0.712380}%
\pgfsetfillcolor{currentfill}%
\pgfsetlinewidth{0.000000pt}%
\definecolor{currentstroke}{rgb}{0.000000,0.000000,0.000000}%
\pgfsetstrokecolor{currentstroke}%
\pgfsetstrokeopacity{0.000000}%
\pgfsetdash{}{0pt}%
\pgfpathmoveto{\pgfqpoint{2.994706in}{1.994849in}}%
\pgfpathlineto{\pgfqpoint{3.003643in}{1.994849in}}%
\pgfpathlineto{\pgfqpoint{3.003643in}{2.064417in}}%
\pgfpathlineto{\pgfqpoint{2.994706in}{2.064417in}}%
\pgfpathlineto{\pgfqpoint{2.994706in}{1.994849in}}%
\pgfpathclose%
\pgfusepath{fill}%
\end{pgfscope}%
\begin{pgfscope}%
\pgfpathrectangle{\pgfqpoint{0.697024in}{0.857143in}}{\pgfqpoint{2.627103in}{1.813434in}}%
\pgfusepath{clip}%
\pgfsetbuttcap%
\pgfsetmiterjoin%
\definecolor{currentfill}{rgb}{0.992771,0.707689,0.712380}%
\pgfsetfillcolor{currentfill}%
\pgfsetlinewidth{0.000000pt}%
\definecolor{currentstroke}{rgb}{0.000000,0.000000,0.000000}%
\pgfsetstrokecolor{currentstroke}%
\pgfsetstrokeopacity{0.000000}%
\pgfsetdash{}{0pt}%
\pgfpathmoveto{\pgfqpoint{3.005877in}{2.030494in}}%
\pgfpathlineto{\pgfqpoint{3.014813in}{2.030494in}}%
\pgfpathlineto{\pgfqpoint{3.014813in}{2.094310in}}%
\pgfpathlineto{\pgfqpoint{3.005877in}{2.094310in}}%
\pgfpathlineto{\pgfqpoint{3.005877in}{2.030494in}}%
\pgfpathclose%
\pgfusepath{fill}%
\end{pgfscope}%
\begin{pgfscope}%
\pgfpathrectangle{\pgfqpoint{0.697024in}{0.857143in}}{\pgfqpoint{2.627103in}{1.813434in}}%
\pgfusepath{clip}%
\pgfsetbuttcap%
\pgfsetmiterjoin%
\definecolor{currentfill}{rgb}{0.992771,0.707689,0.712380}%
\pgfsetfillcolor{currentfill}%
\pgfsetlinewidth{0.000000pt}%
\definecolor{currentstroke}{rgb}{0.000000,0.000000,0.000000}%
\pgfsetstrokecolor{currentstroke}%
\pgfsetstrokeopacity{0.000000}%
\pgfsetdash{}{0pt}%
\pgfpathmoveto{\pgfqpoint{3.017047in}{2.128604in}}%
\pgfpathlineto{\pgfqpoint{3.025984in}{2.128604in}}%
\pgfpathlineto{\pgfqpoint{3.025984in}{2.217634in}}%
\pgfpathlineto{\pgfqpoint{3.017047in}{2.217634in}}%
\pgfpathlineto{\pgfqpoint{3.017047in}{2.128604in}}%
\pgfpathclose%
\pgfusepath{fill}%
\end{pgfscope}%
\begin{pgfscope}%
\pgfpathrectangle{\pgfqpoint{0.697024in}{0.857143in}}{\pgfqpoint{2.627103in}{1.813434in}}%
\pgfusepath{clip}%
\pgfsetbuttcap%
\pgfsetmiterjoin%
\definecolor{currentfill}{rgb}{0.992771,0.707689,0.712380}%
\pgfsetfillcolor{currentfill}%
\pgfsetlinewidth{0.000000pt}%
\definecolor{currentstroke}{rgb}{0.000000,0.000000,0.000000}%
\pgfsetstrokecolor{currentstroke}%
\pgfsetstrokeopacity{0.000000}%
\pgfsetdash{}{0pt}%
\pgfpathmoveto{\pgfqpoint{3.028218in}{2.180838in}}%
\pgfpathlineto{\pgfqpoint{3.037155in}{2.180838in}}%
\pgfpathlineto{\pgfqpoint{3.037155in}{2.240921in}}%
\pgfpathlineto{\pgfqpoint{3.028218in}{2.240921in}}%
\pgfpathlineto{\pgfqpoint{3.028218in}{2.180838in}}%
\pgfpathclose%
\pgfusepath{fill}%
\end{pgfscope}%
\begin{pgfscope}%
\pgfpathrectangle{\pgfqpoint{0.697024in}{0.857143in}}{\pgfqpoint{2.627103in}{1.813434in}}%
\pgfusepath{clip}%
\pgfsetbuttcap%
\pgfsetmiterjoin%
\definecolor{currentfill}{rgb}{0.992771,0.707689,0.712380}%
\pgfsetfillcolor{currentfill}%
\pgfsetlinewidth{0.000000pt}%
\definecolor{currentstroke}{rgb}{0.000000,0.000000,0.000000}%
\pgfsetstrokecolor{currentstroke}%
\pgfsetstrokeopacity{0.000000}%
\pgfsetdash{}{0pt}%
\pgfpathmoveto{\pgfqpoint{3.039389in}{2.086545in}}%
\pgfpathlineto{\pgfqpoint{3.048325in}{2.086545in}}%
\pgfpathlineto{\pgfqpoint{3.048325in}{2.155190in}}%
\pgfpathlineto{\pgfqpoint{3.039389in}{2.155190in}}%
\pgfpathlineto{\pgfqpoint{3.039389in}{2.086545in}}%
\pgfpathclose%
\pgfusepath{fill}%
\end{pgfscope}%
\begin{pgfscope}%
\pgfpathrectangle{\pgfqpoint{0.697024in}{0.857143in}}{\pgfqpoint{2.627103in}{1.813434in}}%
\pgfusepath{clip}%
\pgfsetbuttcap%
\pgfsetmiterjoin%
\definecolor{currentfill}{rgb}{0.992771,0.707689,0.712380}%
\pgfsetfillcolor{currentfill}%
\pgfsetlinewidth{0.000000pt}%
\definecolor{currentstroke}{rgb}{0.000000,0.000000,0.000000}%
\pgfsetstrokecolor{currentstroke}%
\pgfsetstrokeopacity{0.000000}%
\pgfsetdash{}{0pt}%
\pgfpathmoveto{\pgfqpoint{3.050559in}{2.132786in}}%
\pgfpathlineto{\pgfqpoint{3.059496in}{2.132786in}}%
\pgfpathlineto{\pgfqpoint{3.059496in}{2.192839in}}%
\pgfpathlineto{\pgfqpoint{3.050559in}{2.192839in}}%
\pgfpathlineto{\pgfqpoint{3.050559in}{2.132786in}}%
\pgfpathclose%
\pgfusepath{fill}%
\end{pgfscope}%
\begin{pgfscope}%
\pgfpathrectangle{\pgfqpoint{0.697024in}{0.857143in}}{\pgfqpoint{2.627103in}{1.813434in}}%
\pgfusepath{clip}%
\pgfsetbuttcap%
\pgfsetmiterjoin%
\definecolor{currentfill}{rgb}{0.992771,0.707689,0.712380}%
\pgfsetfillcolor{currentfill}%
\pgfsetlinewidth{0.000000pt}%
\definecolor{currentstroke}{rgb}{0.000000,0.000000,0.000000}%
\pgfsetstrokecolor{currentstroke}%
\pgfsetstrokeopacity{0.000000}%
\pgfsetdash{}{0pt}%
\pgfpathmoveto{\pgfqpoint{3.061730in}{2.102346in}}%
\pgfpathlineto{\pgfqpoint{3.070666in}{2.102346in}}%
\pgfpathlineto{\pgfqpoint{3.070666in}{2.173893in}}%
\pgfpathlineto{\pgfqpoint{3.061730in}{2.173893in}}%
\pgfpathlineto{\pgfqpoint{3.061730in}{2.102346in}}%
\pgfpathclose%
\pgfusepath{fill}%
\end{pgfscope}%
\begin{pgfscope}%
\pgfpathrectangle{\pgfqpoint{0.697024in}{0.857143in}}{\pgfqpoint{2.627103in}{1.813434in}}%
\pgfusepath{clip}%
\pgfsetbuttcap%
\pgfsetmiterjoin%
\definecolor{currentfill}{rgb}{0.992771,0.707689,0.712380}%
\pgfsetfillcolor{currentfill}%
\pgfsetlinewidth{0.000000pt}%
\definecolor{currentstroke}{rgb}{0.000000,0.000000,0.000000}%
\pgfsetstrokecolor{currentstroke}%
\pgfsetstrokeopacity{0.000000}%
\pgfsetdash{}{0pt}%
\pgfpathmoveto{\pgfqpoint{3.072900in}{2.101053in}}%
\pgfpathlineto{\pgfqpoint{3.081837in}{2.101053in}}%
\pgfpathlineto{\pgfqpoint{3.081837in}{2.153257in}}%
\pgfpathlineto{\pgfqpoint{3.072900in}{2.153257in}}%
\pgfpathlineto{\pgfqpoint{3.072900in}{2.101053in}}%
\pgfpathclose%
\pgfusepath{fill}%
\end{pgfscope}%
\begin{pgfscope}%
\pgfpathrectangle{\pgfqpoint{0.697024in}{0.857143in}}{\pgfqpoint{2.627103in}{1.813434in}}%
\pgfusepath{clip}%
\pgfsetbuttcap%
\pgfsetmiterjoin%
\definecolor{currentfill}{rgb}{0.992771,0.707689,0.712380}%
\pgfsetfillcolor{currentfill}%
\pgfsetlinewidth{0.000000pt}%
\definecolor{currentstroke}{rgb}{0.000000,0.000000,0.000000}%
\pgfsetstrokecolor{currentstroke}%
\pgfsetstrokeopacity{0.000000}%
\pgfsetdash{}{0pt}%
\pgfpathmoveto{\pgfqpoint{3.084071in}{2.161378in}}%
\pgfpathlineto{\pgfqpoint{3.093008in}{2.161378in}}%
\pgfpathlineto{\pgfqpoint{3.093008in}{2.213092in}}%
\pgfpathlineto{\pgfqpoint{3.084071in}{2.213092in}}%
\pgfpathlineto{\pgfqpoint{3.084071in}{2.161378in}}%
\pgfpathclose%
\pgfusepath{fill}%
\end{pgfscope}%
\begin{pgfscope}%
\pgfpathrectangle{\pgfqpoint{0.697024in}{0.857143in}}{\pgfqpoint{2.627103in}{1.813434in}}%
\pgfusepath{clip}%
\pgfsetbuttcap%
\pgfsetmiterjoin%
\definecolor{currentfill}{rgb}{0.992771,0.707689,0.712380}%
\pgfsetfillcolor{currentfill}%
\pgfsetlinewidth{0.000000pt}%
\definecolor{currentstroke}{rgb}{0.000000,0.000000,0.000000}%
\pgfsetstrokecolor{currentstroke}%
\pgfsetstrokeopacity{0.000000}%
\pgfsetdash{}{0pt}%
\pgfpathmoveto{\pgfqpoint{3.095242in}{2.111369in}}%
\pgfpathlineto{\pgfqpoint{3.104178in}{2.111369in}}%
\pgfpathlineto{\pgfqpoint{3.104178in}{2.176104in}}%
\pgfpathlineto{\pgfqpoint{3.095242in}{2.176104in}}%
\pgfpathlineto{\pgfqpoint{3.095242in}{2.111369in}}%
\pgfpathclose%
\pgfusepath{fill}%
\end{pgfscope}%
\begin{pgfscope}%
\pgfpathrectangle{\pgfqpoint{0.697024in}{0.857143in}}{\pgfqpoint{2.627103in}{1.813434in}}%
\pgfusepath{clip}%
\pgfsetbuttcap%
\pgfsetmiterjoin%
\definecolor{currentfill}{rgb}{0.992771,0.707689,0.712380}%
\pgfsetfillcolor{currentfill}%
\pgfsetlinewidth{0.000000pt}%
\definecolor{currentstroke}{rgb}{0.000000,0.000000,0.000000}%
\pgfsetstrokecolor{currentstroke}%
\pgfsetstrokeopacity{0.000000}%
\pgfsetdash{}{0pt}%
\pgfpathmoveto{\pgfqpoint{3.106412in}{2.077483in}}%
\pgfpathlineto{\pgfqpoint{3.115349in}{2.077483in}}%
\pgfpathlineto{\pgfqpoint{3.115349in}{2.126366in}}%
\pgfpathlineto{\pgfqpoint{3.106412in}{2.126366in}}%
\pgfpathlineto{\pgfqpoint{3.106412in}{2.077483in}}%
\pgfpathclose%
\pgfusepath{fill}%
\end{pgfscope}%
\begin{pgfscope}%
\pgfpathrectangle{\pgfqpoint{0.697024in}{0.857143in}}{\pgfqpoint{2.627103in}{1.813434in}}%
\pgfusepath{clip}%
\pgfsetbuttcap%
\pgfsetmiterjoin%
\definecolor{currentfill}{rgb}{0.992771,0.707689,0.712380}%
\pgfsetfillcolor{currentfill}%
\pgfsetlinewidth{0.000000pt}%
\definecolor{currentstroke}{rgb}{0.000000,0.000000,0.000000}%
\pgfsetstrokecolor{currentstroke}%
\pgfsetstrokeopacity{0.000000}%
\pgfsetdash{}{0pt}%
\pgfpathmoveto{\pgfqpoint{3.117583in}{2.102081in}}%
\pgfpathlineto{\pgfqpoint{3.126519in}{2.102081in}}%
\pgfpathlineto{\pgfqpoint{3.126519in}{2.169475in}}%
\pgfpathlineto{\pgfqpoint{3.117583in}{2.169475in}}%
\pgfpathlineto{\pgfqpoint{3.117583in}{2.102081in}}%
\pgfpathclose%
\pgfusepath{fill}%
\end{pgfscope}%
\begin{pgfscope}%
\pgfpathrectangle{\pgfqpoint{0.697024in}{0.857143in}}{\pgfqpoint{2.627103in}{1.813434in}}%
\pgfusepath{clip}%
\pgfsetbuttcap%
\pgfsetmiterjoin%
\definecolor{currentfill}{rgb}{0.992771,0.707689,0.712380}%
\pgfsetfillcolor{currentfill}%
\pgfsetlinewidth{0.000000pt}%
\definecolor{currentstroke}{rgb}{0.000000,0.000000,0.000000}%
\pgfsetstrokecolor{currentstroke}%
\pgfsetstrokeopacity{0.000000}%
\pgfsetdash{}{0pt}%
\pgfpathmoveto{\pgfqpoint{3.128753in}{2.101368in}}%
\pgfpathlineto{\pgfqpoint{3.137690in}{2.101368in}}%
\pgfpathlineto{\pgfqpoint{3.137690in}{2.140854in}}%
\pgfpathlineto{\pgfqpoint{3.128753in}{2.140854in}}%
\pgfpathlineto{\pgfqpoint{3.128753in}{2.101368in}}%
\pgfpathclose%
\pgfusepath{fill}%
\end{pgfscope}%
\begin{pgfscope}%
\pgfpathrectangle{\pgfqpoint{0.697024in}{0.857143in}}{\pgfqpoint{2.627103in}{1.813434in}}%
\pgfusepath{clip}%
\pgfsetbuttcap%
\pgfsetmiterjoin%
\definecolor{currentfill}{rgb}{0.992771,0.707689,0.712380}%
\pgfsetfillcolor{currentfill}%
\pgfsetlinewidth{0.000000pt}%
\definecolor{currentstroke}{rgb}{0.000000,0.000000,0.000000}%
\pgfsetstrokecolor{currentstroke}%
\pgfsetstrokeopacity{0.000000}%
\pgfsetdash{}{0pt}%
\pgfpathmoveto{\pgfqpoint{3.139924in}{2.143621in}}%
\pgfpathlineto{\pgfqpoint{3.148861in}{2.143621in}}%
\pgfpathlineto{\pgfqpoint{3.148861in}{2.185222in}}%
\pgfpathlineto{\pgfqpoint{3.139924in}{2.185222in}}%
\pgfpathlineto{\pgfqpoint{3.139924in}{2.143621in}}%
\pgfpathclose%
\pgfusepath{fill}%
\end{pgfscope}%
\begin{pgfscope}%
\pgfpathrectangle{\pgfqpoint{0.697024in}{0.857143in}}{\pgfqpoint{2.627103in}{1.813434in}}%
\pgfusepath{clip}%
\pgfsetbuttcap%
\pgfsetmiterjoin%
\definecolor{currentfill}{rgb}{0.992771,0.707689,0.712380}%
\pgfsetfillcolor{currentfill}%
\pgfsetlinewidth{0.000000pt}%
\definecolor{currentstroke}{rgb}{0.000000,0.000000,0.000000}%
\pgfsetstrokecolor{currentstroke}%
\pgfsetstrokeopacity{0.000000}%
\pgfsetdash{}{0pt}%
\pgfpathmoveto{\pgfqpoint{3.151095in}{2.181567in}}%
\pgfpathlineto{\pgfqpoint{3.160031in}{2.181567in}}%
\pgfpathlineto{\pgfqpoint{3.160031in}{2.229805in}}%
\pgfpathlineto{\pgfqpoint{3.151095in}{2.229805in}}%
\pgfpathlineto{\pgfqpoint{3.151095in}{2.181567in}}%
\pgfpathclose%
\pgfusepath{fill}%
\end{pgfscope}%
\begin{pgfscope}%
\pgfpathrectangle{\pgfqpoint{0.697024in}{0.857143in}}{\pgfqpoint{2.627103in}{1.813434in}}%
\pgfusepath{clip}%
\pgfsetbuttcap%
\pgfsetmiterjoin%
\definecolor{currentfill}{rgb}{0.992771,0.707689,0.712380}%
\pgfsetfillcolor{currentfill}%
\pgfsetlinewidth{0.000000pt}%
\definecolor{currentstroke}{rgb}{0.000000,0.000000,0.000000}%
\pgfsetstrokecolor{currentstroke}%
\pgfsetstrokeopacity{0.000000}%
\pgfsetdash{}{0pt}%
\pgfpathmoveto{\pgfqpoint{3.162265in}{2.274219in}}%
\pgfpathlineto{\pgfqpoint{3.171202in}{2.274219in}}%
\pgfpathlineto{\pgfqpoint{3.171202in}{2.289801in}}%
\pgfpathlineto{\pgfqpoint{3.162265in}{2.289801in}}%
\pgfpathlineto{\pgfqpoint{3.162265in}{2.274219in}}%
\pgfpathclose%
\pgfusepath{fill}%
\end{pgfscope}%
\begin{pgfscope}%
\pgfpathrectangle{\pgfqpoint{0.697024in}{0.857143in}}{\pgfqpoint{2.627103in}{1.813434in}}%
\pgfusepath{clip}%
\pgfsetbuttcap%
\pgfsetmiterjoin%
\definecolor{currentfill}{rgb}{0.992771,0.707689,0.712380}%
\pgfsetfillcolor{currentfill}%
\pgfsetlinewidth{0.000000pt}%
\definecolor{currentstroke}{rgb}{0.000000,0.000000,0.000000}%
\pgfsetstrokecolor{currentstroke}%
\pgfsetstrokeopacity{0.000000}%
\pgfsetdash{}{0pt}%
\pgfpathmoveto{\pgfqpoint{3.173436in}{2.220487in}}%
\pgfpathlineto{\pgfqpoint{3.182372in}{2.220487in}}%
\pgfpathlineto{\pgfqpoint{3.182372in}{2.243019in}}%
\pgfpathlineto{\pgfqpoint{3.173436in}{2.243019in}}%
\pgfpathlineto{\pgfqpoint{3.173436in}{2.220487in}}%
\pgfpathclose%
\pgfusepath{fill}%
\end{pgfscope}%
\begin{pgfscope}%
\pgfpathrectangle{\pgfqpoint{0.697024in}{0.857143in}}{\pgfqpoint{2.627103in}{1.813434in}}%
\pgfusepath{clip}%
\pgfsetbuttcap%
\pgfsetmiterjoin%
\definecolor{currentfill}{rgb}{0.992771,0.707689,0.712380}%
\pgfsetfillcolor{currentfill}%
\pgfsetlinewidth{0.000000pt}%
\definecolor{currentstroke}{rgb}{0.000000,0.000000,0.000000}%
\pgfsetstrokecolor{currentstroke}%
\pgfsetstrokeopacity{0.000000}%
\pgfsetdash{}{0pt}%
\pgfpathmoveto{\pgfqpoint{3.184607in}{2.277559in}}%
\pgfpathlineto{\pgfqpoint{3.193543in}{2.277559in}}%
\pgfpathlineto{\pgfqpoint{3.193543in}{2.316047in}}%
\pgfpathlineto{\pgfqpoint{3.184607in}{2.316047in}}%
\pgfpathlineto{\pgfqpoint{3.184607in}{2.277559in}}%
\pgfpathclose%
\pgfusepath{fill}%
\end{pgfscope}%
\begin{pgfscope}%
\pgfpathrectangle{\pgfqpoint{0.697024in}{0.857143in}}{\pgfqpoint{2.627103in}{1.813434in}}%
\pgfusepath{clip}%
\pgfsetbuttcap%
\pgfsetmiterjoin%
\definecolor{currentfill}{rgb}{0.992771,0.707689,0.712380}%
\pgfsetfillcolor{currentfill}%
\pgfsetlinewidth{0.000000pt}%
\definecolor{currentstroke}{rgb}{0.000000,0.000000,0.000000}%
\pgfsetstrokecolor{currentstroke}%
\pgfsetstrokeopacity{0.000000}%
\pgfsetdash{}{0pt}%
\pgfpathmoveto{\pgfqpoint{3.195777in}{2.291081in}}%
\pgfpathlineto{\pgfqpoint{3.204714in}{2.291081in}}%
\pgfpathlineto{\pgfqpoint{3.204714in}{2.310978in}}%
\pgfpathlineto{\pgfqpoint{3.195777in}{2.310978in}}%
\pgfpathlineto{\pgfqpoint{3.195777in}{2.291081in}}%
\pgfpathclose%
\pgfusepath{fill}%
\end{pgfscope}%
\begin{pgfscope}%
\pgfsetbuttcap%
\pgfsetroundjoin%
\definecolor{currentfill}{rgb}{0.000000,0.000000,0.000000}%
\pgfsetfillcolor{currentfill}%
\pgfsetlinewidth{0.803000pt}%
\definecolor{currentstroke}{rgb}{0.000000,0.000000,0.000000}%
\pgfsetstrokecolor{currentstroke}%
\pgfsetdash{}{0pt}%
\pgfsys@defobject{currentmarker}{\pgfqpoint{0.000000in}{-0.048611in}}{\pgfqpoint{0.000000in}{0.000000in}}{%
\pgfpathmoveto{\pgfqpoint{0.000000in}{0.000000in}}%
\pgfpathlineto{\pgfqpoint{0.000000in}{-0.048611in}}%
\pgfusepath{stroke,fill}%
}%
\begin{pgfscope}%
\pgfsys@transformshift{1.357095in}{0.857143in}%
\pgfsys@useobject{currentmarker}{}%
\end{pgfscope}%
\end{pgfscope}%
\begin{pgfscope}%
\definecolor{textcolor}{rgb}{0.000000,0.000000,0.000000}%
\pgfsetstrokecolor{textcolor}%
\pgfsetfillcolor{textcolor}%
\pgftext[x=1.357095in,y=0.759921in,,top]{\color{textcolor}{\rmfamily\fontsize{10.000000}{12.000000}\selectfont\catcode`\^=\active\def^{\ifmmode\sp\else\^{}\fi}\catcode`\%=\active\def%{\%}1978}}%
\end{pgfscope}%
\begin{pgfscope}%
\pgfsetbuttcap%
\pgfsetroundjoin%
\definecolor{currentfill}{rgb}{0.000000,0.000000,0.000000}%
\pgfsetfillcolor{currentfill}%
\pgfsetlinewidth{0.803000pt}%
\definecolor{currentstroke}{rgb}{0.000000,0.000000,0.000000}%
\pgfsetstrokecolor{currentstroke}%
\pgfsetdash{}{0pt}%
\pgfsys@defobject{currentmarker}{\pgfqpoint{0.000000in}{-0.048611in}}{\pgfqpoint{0.000000in}{0.000000in}}{%
\pgfpathmoveto{\pgfqpoint{0.000000in}{0.000000in}}%
\pgfpathlineto{\pgfqpoint{0.000000in}{-0.048611in}}%
\pgfusepath{stroke,fill}%
}%
\begin{pgfscope}%
\pgfsys@transformshift{1.915626in}{0.857143in}%
\pgfsys@useobject{currentmarker}{}%
\end{pgfscope}%
\end{pgfscope}%
\begin{pgfscope}%
\definecolor{textcolor}{rgb}{0.000000,0.000000,0.000000}%
\pgfsetstrokecolor{textcolor}%
\pgfsetfillcolor{textcolor}%
\pgftext[x=1.915626in,y=0.759921in,,top]{\color{textcolor}{\rmfamily\fontsize{10.000000}{12.000000}\selectfont\catcode`\^=\active\def^{\ifmmode\sp\else\^{}\fi}\catcode`\%=\active\def%{\%}1991}}%
\end{pgfscope}%
\begin{pgfscope}%
\pgfsetbuttcap%
\pgfsetroundjoin%
\definecolor{currentfill}{rgb}{0.000000,0.000000,0.000000}%
\pgfsetfillcolor{currentfill}%
\pgfsetlinewidth{0.803000pt}%
\definecolor{currentstroke}{rgb}{0.000000,0.000000,0.000000}%
\pgfsetstrokecolor{currentstroke}%
\pgfsetdash{}{0pt}%
\pgfsys@defobject{currentmarker}{\pgfqpoint{0.000000in}{-0.048611in}}{\pgfqpoint{0.000000in}{0.000000in}}{%
\pgfpathmoveto{\pgfqpoint{0.000000in}{0.000000in}}%
\pgfpathlineto{\pgfqpoint{0.000000in}{-0.048611in}}%
\pgfusepath{stroke,fill}%
}%
\begin{pgfscope}%
\pgfsys@transformshift{2.474156in}{0.857143in}%
\pgfsys@useobject{currentmarker}{}%
\end{pgfscope}%
\end{pgfscope}%
\begin{pgfscope}%
\definecolor{textcolor}{rgb}{0.000000,0.000000,0.000000}%
\pgfsetstrokecolor{textcolor}%
\pgfsetfillcolor{textcolor}%
\pgftext[x=2.474156in,y=0.759921in,,top]{\color{textcolor}{\rmfamily\fontsize{10.000000}{12.000000}\selectfont\catcode`\^=\active\def^{\ifmmode\sp\else\^{}\fi}\catcode`\%=\active\def%{\%}2003}}%
\end{pgfscope}%
\begin{pgfscope}%
\pgfsetbuttcap%
\pgfsetroundjoin%
\definecolor{currentfill}{rgb}{0.000000,0.000000,0.000000}%
\pgfsetfillcolor{currentfill}%
\pgfsetlinewidth{0.803000pt}%
\definecolor{currentstroke}{rgb}{0.000000,0.000000,0.000000}%
\pgfsetstrokecolor{currentstroke}%
\pgfsetdash{}{0pt}%
\pgfsys@defobject{currentmarker}{\pgfqpoint{0.000000in}{-0.048611in}}{\pgfqpoint{0.000000in}{0.000000in}}{%
\pgfpathmoveto{\pgfqpoint{0.000000in}{0.000000in}}%
\pgfpathlineto{\pgfqpoint{0.000000in}{-0.048611in}}%
\pgfusepath{stroke,fill}%
}%
\begin{pgfscope}%
\pgfsys@transformshift{3.032686in}{0.857143in}%
\pgfsys@useobject{currentmarker}{}%
\end{pgfscope}%
\end{pgfscope}%
\begin{pgfscope}%
\definecolor{textcolor}{rgb}{0.000000,0.000000,0.000000}%
\pgfsetstrokecolor{textcolor}%
\pgfsetfillcolor{textcolor}%
\pgftext[x=3.032686in,y=0.759921in,,top]{\color{textcolor}{\rmfamily\fontsize{10.000000}{12.000000}\selectfont\catcode`\^=\active\def^{\ifmmode\sp\else\^{}\fi}\catcode`\%=\active\def%{\%}2016}}%
\end{pgfscope}%
\begin{pgfscope}%
\pgfsetbuttcap%
\pgfsetroundjoin%
\definecolor{currentfill}{rgb}{0.000000,0.000000,0.000000}%
\pgfsetfillcolor{currentfill}%
\pgfsetlinewidth{0.803000pt}%
\definecolor{currentstroke}{rgb}{0.000000,0.000000,0.000000}%
\pgfsetstrokecolor{currentstroke}%
\pgfsetdash{}{0pt}%
\pgfsys@defobject{currentmarker}{\pgfqpoint{-0.048611in}{0.000000in}}{\pgfqpoint{-0.000000in}{0.000000in}}{%
\pgfpathmoveto{\pgfqpoint{-0.000000in}{0.000000in}}%
\pgfpathlineto{\pgfqpoint{-0.048611in}{0.000000in}}%
\pgfusepath{stroke,fill}%
}%
\begin{pgfscope}%
\pgfsys@transformshift{0.697024in}{1.010886in}%
\pgfsys@useobject{currentmarker}{}%
\end{pgfscope}%
\end{pgfscope}%
\begin{pgfscope}%
\definecolor{textcolor}{rgb}{0.000000,0.000000,0.000000}%
\pgfsetstrokecolor{textcolor}%
\pgfsetfillcolor{textcolor}%
\pgftext[x=0.422332in, y=0.958125in, left, base]{\color{textcolor}{\rmfamily\fontsize{10.000000}{12.000000}\selectfont\catcode`\^=\active\def^{\ifmmode\sp\else\^{}\fi}\catcode`\%=\active\def%{\%}$\mathdefault{\ensuremath{-}4}$}}%
\end{pgfscope}%
\begin{pgfscope}%
\pgfsetbuttcap%
\pgfsetroundjoin%
\definecolor{currentfill}{rgb}{0.000000,0.000000,0.000000}%
\pgfsetfillcolor{currentfill}%
\pgfsetlinewidth{0.803000pt}%
\definecolor{currentstroke}{rgb}{0.000000,0.000000,0.000000}%
\pgfsetstrokecolor{currentstroke}%
\pgfsetdash{}{0pt}%
\pgfsys@defobject{currentmarker}{\pgfqpoint{-0.048611in}{0.000000in}}{\pgfqpoint{-0.000000in}{0.000000in}}{%
\pgfpathmoveto{\pgfqpoint{-0.000000in}{0.000000in}}%
\pgfpathlineto{\pgfqpoint{-0.048611in}{0.000000in}}%
\pgfusepath{stroke,fill}%
}%
\begin{pgfscope}%
\pgfsys@transformshift{0.697024in}{1.428978in}%
\pgfsys@useobject{currentmarker}{}%
\end{pgfscope}%
\end{pgfscope}%
\begin{pgfscope}%
\definecolor{textcolor}{rgb}{0.000000,0.000000,0.000000}%
\pgfsetstrokecolor{textcolor}%
\pgfsetfillcolor{textcolor}%
\pgftext[x=0.422332in, y=1.376217in, left, base]{\color{textcolor}{\rmfamily\fontsize{10.000000}{12.000000}\selectfont\catcode`\^=\active\def^{\ifmmode\sp\else\^{}\fi}\catcode`\%=\active\def%{\%}$\mathdefault{\ensuremath{-}2}$}}%
\end{pgfscope}%
\begin{pgfscope}%
\pgfsetbuttcap%
\pgfsetroundjoin%
\definecolor{currentfill}{rgb}{0.000000,0.000000,0.000000}%
\pgfsetfillcolor{currentfill}%
\pgfsetlinewidth{0.803000pt}%
\definecolor{currentstroke}{rgb}{0.000000,0.000000,0.000000}%
\pgfsetstrokecolor{currentstroke}%
\pgfsetdash{}{0pt}%
\pgfsys@defobject{currentmarker}{\pgfqpoint{-0.048611in}{0.000000in}}{\pgfqpoint{-0.000000in}{0.000000in}}{%
\pgfpathmoveto{\pgfqpoint{-0.000000in}{0.000000in}}%
\pgfpathlineto{\pgfqpoint{-0.048611in}{0.000000in}}%
\pgfusepath{stroke,fill}%
}%
\begin{pgfscope}%
\pgfsys@transformshift{0.697024in}{1.847071in}%
\pgfsys@useobject{currentmarker}{}%
\end{pgfscope}%
\end{pgfscope}%
\begin{pgfscope}%
\definecolor{textcolor}{rgb}{0.000000,0.000000,0.000000}%
\pgfsetstrokecolor{textcolor}%
\pgfsetfillcolor{textcolor}%
\pgftext[x=0.530357in, y=1.794309in, left, base]{\color{textcolor}{\rmfamily\fontsize{10.000000}{12.000000}\selectfont\catcode`\^=\active\def^{\ifmmode\sp\else\^{}\fi}\catcode`\%=\active\def%{\%}$\mathdefault{0}$}}%
\end{pgfscope}%
\begin{pgfscope}%
\pgfsetbuttcap%
\pgfsetroundjoin%
\definecolor{currentfill}{rgb}{0.000000,0.000000,0.000000}%
\pgfsetfillcolor{currentfill}%
\pgfsetlinewidth{0.803000pt}%
\definecolor{currentstroke}{rgb}{0.000000,0.000000,0.000000}%
\pgfsetstrokecolor{currentstroke}%
\pgfsetdash{}{0pt}%
\pgfsys@defobject{currentmarker}{\pgfqpoint{-0.048611in}{0.000000in}}{\pgfqpoint{-0.000000in}{0.000000in}}{%
\pgfpathmoveto{\pgfqpoint{-0.000000in}{0.000000in}}%
\pgfpathlineto{\pgfqpoint{-0.048611in}{0.000000in}}%
\pgfusepath{stroke,fill}%
}%
\begin{pgfscope}%
\pgfsys@transformshift{0.697024in}{2.265163in}%
\pgfsys@useobject{currentmarker}{}%
\end{pgfscope}%
\end{pgfscope}%
\begin{pgfscope}%
\definecolor{textcolor}{rgb}{0.000000,0.000000,0.000000}%
\pgfsetstrokecolor{textcolor}%
\pgfsetfillcolor{textcolor}%
\pgftext[x=0.530357in, y=2.212402in, left, base]{\color{textcolor}{\rmfamily\fontsize{10.000000}{12.000000}\selectfont\catcode`\^=\active\def^{\ifmmode\sp\else\^{}\fi}\catcode`\%=\active\def%{\%}$\mathdefault{2}$}}%
\end{pgfscope}%
\begin{pgfscope}%
\pgfpathrectangle{\pgfqpoint{0.697024in}{0.857143in}}{\pgfqpoint{2.627103in}{1.813434in}}%
\pgfusepath{clip}%
\pgfsetrectcap%
\pgfsetroundjoin%
\pgfsetlinewidth{1.003750pt}%
\definecolor{currentstroke}{rgb}{0.000000,0.000000,0.000000}%
\pgfsetstrokecolor{currentstroke}%
\pgfsetdash{}{0pt}%
\pgfpathmoveto{\pgfqpoint{0.697024in}{1.847071in}}%
\pgfpathlineto{\pgfqpoint{3.324127in}{1.847071in}}%
\pgfusepath{stroke}%
\end{pgfscope}%
\begin{pgfscope}%
\pgfpathrectangle{\pgfqpoint{0.697024in}{0.857143in}}{\pgfqpoint{2.627103in}{1.813434in}}%
\pgfusepath{clip}%
\pgfsetrectcap%
\pgfsetroundjoin%
\pgfsetlinewidth{1.505625pt}%
\definecolor{currentstroke}{rgb}{0.000000,0.000000,0.000000}%
\pgfsetstrokecolor{currentstroke}%
\pgfsetdash{}{0pt}%
\pgfpathmoveto{\pgfqpoint{0.820906in}{1.700518in}}%
\pgfpathlineto{\pgfqpoint{0.832077in}{1.695104in}}%
\pgfpathlineto{\pgfqpoint{0.843248in}{1.694627in}}%
\pgfpathlineto{\pgfqpoint{0.854418in}{1.702255in}}%
\pgfpathlineto{\pgfqpoint{0.865589in}{1.666598in}}%
\pgfpathlineto{\pgfqpoint{0.876759in}{1.621564in}}%
\pgfpathlineto{\pgfqpoint{0.887930in}{1.711404in}}%
\pgfpathlineto{\pgfqpoint{0.899101in}{1.744891in}}%
\pgfpathlineto{\pgfqpoint{0.910271in}{1.833772in}}%
\pgfpathlineto{\pgfqpoint{0.921442in}{1.794949in}}%
\pgfpathlineto{\pgfqpoint{0.932612in}{1.809529in}}%
\pgfpathlineto{\pgfqpoint{0.943783in}{1.796871in}}%
\pgfpathlineto{\pgfqpoint{0.954954in}{1.759012in}}%
\pgfpathlineto{\pgfqpoint{0.966124in}{1.783252in}}%
\pgfpathlineto{\pgfqpoint{0.977295in}{1.753972in}}%
\pgfpathlineto{\pgfqpoint{0.988465in}{1.699386in}}%
\pgfpathlineto{\pgfqpoint{0.999636in}{1.713329in}}%
\pgfpathlineto{\pgfqpoint{1.010807in}{1.676496in}}%
\pgfpathlineto{\pgfqpoint{1.021977in}{1.692424in}}%
\pgfpathlineto{\pgfqpoint{1.033148in}{1.678430in}}%
\pgfpathlineto{\pgfqpoint{1.044319in}{1.658302in}}%
\pgfpathlineto{\pgfqpoint{1.055489in}{1.694740in}}%
\pgfpathlineto{\pgfqpoint{1.066660in}{1.687606in}}%
\pgfpathlineto{\pgfqpoint{1.077830in}{1.763208in}}%
\pgfpathlineto{\pgfqpoint{1.089001in}{1.792060in}}%
\pgfpathlineto{\pgfqpoint{1.100172in}{1.850409in}}%
\pgfpathlineto{\pgfqpoint{1.111342in}{1.890653in}}%
\pgfpathlineto{\pgfqpoint{1.122513in}{1.893280in}}%
\pgfpathlineto{\pgfqpoint{1.133683in}{1.866068in}}%
\pgfpathlineto{\pgfqpoint{1.144854in}{1.806906in}}%
\pgfpathlineto{\pgfqpoint{1.156025in}{1.767431in}}%
\pgfpathlineto{\pgfqpoint{1.167195in}{1.759877in}}%
\pgfpathlineto{\pgfqpoint{1.178366in}{1.721615in}}%
\pgfpathlineto{\pgfqpoint{1.189536in}{1.645100in}}%
\pgfpathlineto{\pgfqpoint{1.200707in}{1.577064in}}%
\pgfpathlineto{\pgfqpoint{1.211878in}{1.564250in}}%
\pgfpathlineto{\pgfqpoint{1.223048in}{1.659350in}}%
\pgfpathlineto{\pgfqpoint{1.234219in}{1.635241in}}%
\pgfpathlineto{\pgfqpoint{1.245389in}{1.763234in}}%
\pgfpathlineto{\pgfqpoint{1.256560in}{1.758103in}}%
\pgfpathlineto{\pgfqpoint{1.267731in}{1.761535in}}%
\pgfpathlineto{\pgfqpoint{1.278901in}{1.748448in}}%
\pgfpathlineto{\pgfqpoint{1.290072in}{1.791599in}}%
\pgfpathlineto{\pgfqpoint{1.301242in}{1.815355in}}%
\pgfpathlineto{\pgfqpoint{1.312413in}{1.847747in}}%
\pgfpathlineto{\pgfqpoint{1.323584in}{1.817035in}}%
\pgfpathlineto{\pgfqpoint{1.334754in}{1.772447in}}%
\pgfpathlineto{\pgfqpoint{1.345925in}{1.889510in}}%
\pgfpathlineto{\pgfqpoint{1.357095in}{1.871670in}}%
\pgfpathlineto{\pgfqpoint{1.368266in}{1.866675in}}%
\pgfpathlineto{\pgfqpoint{1.379437in}{1.855458in}}%
\pgfpathlineto{\pgfqpoint{1.390607in}{1.811803in}}%
\pgfpathlineto{\pgfqpoint{1.401778in}{1.815364in}}%
\pgfpathlineto{\pgfqpoint{1.412948in}{1.854394in}}%
\pgfpathlineto{\pgfqpoint{1.424119in}{1.876866in}}%
\pgfpathlineto{\pgfqpoint{1.435290in}{1.705687in}}%
\pgfpathlineto{\pgfqpoint{1.446460in}{1.704154in}}%
\pgfpathlineto{\pgfqpoint{1.457631in}{1.731645in}}%
\pgfpathlineto{\pgfqpoint{1.468802in}{1.706590in}}%
\pgfpathlineto{\pgfqpoint{1.479972in}{1.660277in}}%
\pgfpathlineto{\pgfqpoint{1.491143in}{1.623924in}}%
\pgfpathlineto{\pgfqpoint{1.502313in}{1.549895in}}%
\pgfpathlineto{\pgfqpoint{1.513484in}{1.526750in}}%
\pgfpathlineto{\pgfqpoint{1.524655in}{1.496077in}}%
\pgfpathlineto{\pgfqpoint{1.535825in}{1.498861in}}%
\pgfpathlineto{\pgfqpoint{1.558166in}{1.575442in}}%
\pgfpathlineto{\pgfqpoint{1.569337in}{1.623182in}}%
\pgfpathlineto{\pgfqpoint{1.580508in}{1.707548in}}%
\pgfpathlineto{\pgfqpoint{1.591678in}{1.746302in}}%
\pgfpathlineto{\pgfqpoint{1.602849in}{1.790572in}}%
\pgfpathlineto{\pgfqpoint{1.614019in}{1.789549in}}%
\pgfpathlineto{\pgfqpoint{1.625190in}{1.781846in}}%
\pgfpathlineto{\pgfqpoint{1.636361in}{1.777609in}}%
\pgfpathlineto{\pgfqpoint{1.647531in}{1.865533in}}%
\pgfpathlineto{\pgfqpoint{1.658702in}{1.874703in}}%
\pgfpathlineto{\pgfqpoint{1.669872in}{1.911704in}}%
\pgfpathlineto{\pgfqpoint{1.681043in}{1.917268in}}%
\pgfpathlineto{\pgfqpoint{1.692214in}{1.985884in}}%
\pgfpathlineto{\pgfqpoint{1.703384in}{1.942885in}}%
\pgfpathlineto{\pgfqpoint{1.725725in}{1.988430in}}%
\pgfpathlineto{\pgfqpoint{1.736896in}{1.921252in}}%
\pgfpathlineto{\pgfqpoint{1.748067in}{2.027565in}}%
\pgfpathlineto{\pgfqpoint{1.759237in}{2.043121in}}%
\pgfpathlineto{\pgfqpoint{1.770408in}{2.026791in}}%
\pgfpathlineto{\pgfqpoint{1.781578in}{2.077447in}}%
\pgfpathlineto{\pgfqpoint{1.792749in}{2.068332in}}%
\pgfpathlineto{\pgfqpoint{1.815090in}{2.114388in}}%
\pgfpathlineto{\pgfqpoint{1.826261in}{2.127412in}}%
\pgfpathlineto{\pgfqpoint{1.837431in}{2.121382in}}%
\pgfpathlineto{\pgfqpoint{1.848602in}{2.104397in}}%
\pgfpathlineto{\pgfqpoint{1.859773in}{2.093581in}}%
\pgfpathlineto{\pgfqpoint{1.870943in}{2.135371in}}%
\pgfpathlineto{\pgfqpoint{1.882114in}{2.089384in}}%
\pgfpathlineto{\pgfqpoint{1.904455in}{2.050885in}}%
\pgfpathlineto{\pgfqpoint{1.915626in}{1.915026in}}%
\pgfpathlineto{\pgfqpoint{1.926796in}{1.922155in}}%
\pgfpathlineto{\pgfqpoint{1.937967in}{1.913648in}}%
\pgfpathlineto{\pgfqpoint{1.949138in}{1.848899in}}%
\pgfpathlineto{\pgfqpoint{1.960308in}{1.927037in}}%
\pgfpathlineto{\pgfqpoint{1.971479in}{1.910740in}}%
\pgfpathlineto{\pgfqpoint{1.982649in}{1.930346in}}%
\pgfpathlineto{\pgfqpoint{1.993820in}{1.945849in}}%
\pgfpathlineto{\pgfqpoint{2.004991in}{1.940420in}}%
\pgfpathlineto{\pgfqpoint{2.016161in}{1.929395in}}%
\pgfpathlineto{\pgfqpoint{2.027332in}{1.942167in}}%
\pgfpathlineto{\pgfqpoint{2.038502in}{1.942289in}}%
\pgfpathlineto{\pgfqpoint{2.049673in}{1.934785in}}%
\pgfpathlineto{\pgfqpoint{2.060844in}{1.964162in}}%
\pgfpathlineto{\pgfqpoint{2.072014in}{1.944978in}}%
\pgfpathlineto{\pgfqpoint{2.083185in}{1.973152in}}%
\pgfpathlineto{\pgfqpoint{2.094355in}{1.943944in}}%
\pgfpathlineto{\pgfqpoint{2.105526in}{1.918618in}}%
\pgfpathlineto{\pgfqpoint{2.116697in}{1.933610in}}%
\pgfpathlineto{\pgfqpoint{2.127867in}{1.908977in}}%
\pgfpathlineto{\pgfqpoint{2.139038in}{1.917131in}}%
\pgfpathlineto{\pgfqpoint{2.150208in}{1.977670in}}%
\pgfpathlineto{\pgfqpoint{2.161379in}{1.960128in}}%
\pgfpathlineto{\pgfqpoint{2.172550in}{1.962473in}}%
\pgfpathlineto{\pgfqpoint{2.183720in}{1.969453in}}%
\pgfpathlineto{\pgfqpoint{2.194891in}{1.963219in}}%
\pgfpathlineto{\pgfqpoint{2.206061in}{1.980955in}}%
\pgfpathlineto{\pgfqpoint{2.217232in}{2.001216in}}%
\pgfpathlineto{\pgfqpoint{2.228403in}{1.999839in}}%
\pgfpathlineto{\pgfqpoint{2.239573in}{2.035616in}}%
\pgfpathlineto{\pgfqpoint{2.250744in}{2.061021in}}%
\pgfpathlineto{\pgfqpoint{2.261915in}{2.097490in}}%
\pgfpathlineto{\pgfqpoint{2.273085in}{2.103436in}}%
\pgfpathlineto{\pgfqpoint{2.284256in}{2.149709in}}%
\pgfpathlineto{\pgfqpoint{2.295426in}{2.160842in}}%
\pgfpathlineto{\pgfqpoint{2.317768in}{2.218871in}}%
\pgfpathlineto{\pgfqpoint{2.328938in}{2.253486in}}%
\pgfpathlineto{\pgfqpoint{2.340109in}{2.250809in}}%
\pgfpathlineto{\pgfqpoint{2.351279in}{2.276190in}}%
\pgfpathlineto{\pgfqpoint{2.362450in}{2.271811in}}%
\pgfpathlineto{\pgfqpoint{2.373621in}{2.213248in}}%
\pgfpathlineto{\pgfqpoint{2.384791in}{2.209478in}}%
\pgfpathlineto{\pgfqpoint{2.395962in}{2.246684in}}%
\pgfpathlineto{\pgfqpoint{2.407132in}{2.169856in}}%
\pgfpathlineto{\pgfqpoint{2.418303in}{2.204885in}}%
\pgfpathlineto{\pgfqpoint{2.429474in}{2.192239in}}%
\pgfpathlineto{\pgfqpoint{2.440644in}{2.154341in}}%
\pgfpathlineto{\pgfqpoint{2.451815in}{2.163759in}}%
\pgfpathlineto{\pgfqpoint{2.462985in}{2.170584in}}%
\pgfpathlineto{\pgfqpoint{2.474156in}{2.183353in}}%
\pgfpathlineto{\pgfqpoint{2.485327in}{2.184271in}}%
\pgfpathlineto{\pgfqpoint{2.496497in}{2.200681in}}%
\pgfpathlineto{\pgfqpoint{2.507668in}{2.184818in}}%
\pgfpathlineto{\pgfqpoint{2.518838in}{2.194278in}}%
\pgfpathlineto{\pgfqpoint{2.530009in}{2.213196in}}%
\pgfpathlineto{\pgfqpoint{2.541180in}{2.194067in}}%
\pgfpathlineto{\pgfqpoint{2.552350in}{2.228110in}}%
\pgfpathlineto{\pgfqpoint{2.563521in}{2.244585in}}%
\pgfpathlineto{\pgfqpoint{2.574691in}{2.209850in}}%
\pgfpathlineto{\pgfqpoint{2.585862in}{2.210464in}}%
\pgfpathlineto{\pgfqpoint{2.597033in}{2.194441in}}%
\pgfpathlineto{\pgfqpoint{2.608203in}{2.202961in}}%
\pgfpathlineto{\pgfqpoint{2.619374in}{2.170148in}}%
\pgfpathlineto{\pgfqpoint{2.630544in}{2.160846in}}%
\pgfpathlineto{\pgfqpoint{2.641715in}{2.142578in}}%
\pgfpathlineto{\pgfqpoint{2.652886in}{2.131135in}}%
\pgfpathlineto{\pgfqpoint{2.664056in}{2.138258in}}%
\pgfpathlineto{\pgfqpoint{2.675227in}{2.138856in}}%
\pgfpathlineto{\pgfqpoint{2.686398in}{2.113000in}}%
\pgfpathlineto{\pgfqpoint{2.697568in}{2.092676in}}%
\pgfpathlineto{\pgfqpoint{2.719909in}{1.828127in}}%
\pgfpathlineto{\pgfqpoint{2.731080in}{1.758635in}}%
\pgfpathlineto{\pgfqpoint{2.742251in}{1.773439in}}%
\pgfpathlineto{\pgfqpoint{2.753421in}{1.763807in}}%
\pgfpathlineto{\pgfqpoint{2.764592in}{1.751512in}}%
\pgfpathlineto{\pgfqpoint{2.775762in}{1.753386in}}%
\pgfpathlineto{\pgfqpoint{2.798104in}{1.732737in}}%
\pgfpathlineto{\pgfqpoint{2.809274in}{1.732095in}}%
\pgfpathlineto{\pgfqpoint{2.820445in}{1.728729in}}%
\pgfpathlineto{\pgfqpoint{2.831615in}{1.697652in}}%
\pgfpathlineto{\pgfqpoint{2.842786in}{1.686592in}}%
\pgfpathlineto{\pgfqpoint{2.853957in}{1.663652in}}%
\pgfpathlineto{\pgfqpoint{2.865127in}{1.663201in}}%
\pgfpathlineto{\pgfqpoint{2.876298in}{1.595509in}}%
\pgfpathlineto{\pgfqpoint{2.887468in}{1.586893in}}%
\pgfpathlineto{\pgfqpoint{2.898639in}{1.582732in}}%
\pgfpathlineto{\pgfqpoint{2.909810in}{1.548617in}}%
\pgfpathlineto{\pgfqpoint{2.920980in}{1.527066in}}%
\pgfpathlineto{\pgfqpoint{2.932151in}{1.517185in}}%
\pgfpathlineto{\pgfqpoint{2.943321in}{1.496166in}}%
\pgfpathlineto{\pgfqpoint{2.954492in}{1.511763in}}%
\pgfpathlineto{\pgfqpoint{2.965663in}{1.514472in}}%
\pgfpathlineto{\pgfqpoint{2.976833in}{1.539249in}}%
\pgfpathlineto{\pgfqpoint{2.988004in}{1.505324in}}%
\pgfpathlineto{\pgfqpoint{2.999174in}{1.499923in}}%
\pgfpathlineto{\pgfqpoint{3.010345in}{1.513861in}}%
\pgfpathlineto{\pgfqpoint{3.021516in}{1.487075in}}%
\pgfpathlineto{\pgfqpoint{3.032686in}{1.488747in}}%
\pgfpathlineto{\pgfqpoint{3.043857in}{1.472062in}}%
\pgfpathlineto{\pgfqpoint{3.055027in}{1.480323in}}%
\pgfpathlineto{\pgfqpoint{3.066198in}{1.461337in}}%
\pgfpathlineto{\pgfqpoint{3.077369in}{1.477530in}}%
\pgfpathlineto{\pgfqpoint{3.088539in}{1.470408in}}%
\pgfpathlineto{\pgfqpoint{3.099710in}{1.443775in}}%
\pgfpathlineto{\pgfqpoint{3.110881in}{1.446434in}}%
\pgfpathlineto{\pgfqpoint{3.122051in}{1.476773in}}%
\pgfpathlineto{\pgfqpoint{3.133222in}{1.456964in}}%
\pgfpathlineto{\pgfqpoint{3.144392in}{1.449799in}}%
\pgfpathlineto{\pgfqpoint{3.155563in}{1.431507in}}%
\pgfpathlineto{\pgfqpoint{3.166734in}{1.382302in}}%
\pgfpathlineto{\pgfqpoint{3.177904in}{1.397716in}}%
\pgfpathlineto{\pgfqpoint{3.189075in}{1.420846in}}%
\pgfpathlineto{\pgfqpoint{3.200245in}{1.412128in}}%
\pgfpathlineto{\pgfqpoint{3.200245in}{1.412128in}}%
\pgfusepath{stroke}%
\end{pgfscope}%
\begin{pgfscope}%
\pgfsetrectcap%
\pgfsetmiterjoin%
\pgfsetlinewidth{0.803000pt}%
\definecolor{currentstroke}{rgb}{0.000000,0.000000,0.000000}%
\pgfsetstrokecolor{currentstroke}%
\pgfsetdash{}{0pt}%
\pgfpathmoveto{\pgfqpoint{0.697024in}{0.857143in}}%
\pgfpathlineto{\pgfqpoint{0.697024in}{2.670576in}}%
\pgfusepath{stroke}%
\end{pgfscope}%
\begin{pgfscope}%
\pgfsetrectcap%
\pgfsetmiterjoin%
\pgfsetlinewidth{0.803000pt}%
\definecolor{currentstroke}{rgb}{0.000000,0.000000,0.000000}%
\pgfsetstrokecolor{currentstroke}%
\pgfsetdash{}{0pt}%
\pgfpathmoveto{\pgfqpoint{3.324127in}{0.857143in}}%
\pgfpathlineto{\pgfqpoint{3.324127in}{2.670576in}}%
\pgfusepath{stroke}%
\end{pgfscope}%
\begin{pgfscope}%
\pgfsetrectcap%
\pgfsetmiterjoin%
\pgfsetlinewidth{0.803000pt}%
\definecolor{currentstroke}{rgb}{0.000000,0.000000,0.000000}%
\pgfsetstrokecolor{currentstroke}%
\pgfsetdash{}{0pt}%
\pgfpathmoveto{\pgfqpoint{0.697024in}{0.857143in}}%
\pgfpathlineto{\pgfqpoint{3.324127in}{0.857143in}}%
\pgfusepath{stroke}%
\end{pgfscope}%
\begin{pgfscope}%
\pgfsetrectcap%
\pgfsetmiterjoin%
\pgfsetlinewidth{0.803000pt}%
\definecolor{currentstroke}{rgb}{0.000000,0.000000,0.000000}%
\pgfsetstrokecolor{currentstroke}%
\pgfsetdash{}{0pt}%
\pgfpathmoveto{\pgfqpoint{0.697024in}{2.670576in}}%
\pgfpathlineto{\pgfqpoint{3.324127in}{2.670576in}}%
\pgfusepath{stroke}%
\end{pgfscope}%
\begin{pgfscope}%
\definecolor{textcolor}{rgb}{0.000000,0.000000,0.000000}%
\pgfsetstrokecolor{textcolor}%
\pgfsetfillcolor{textcolor}%
\pgftext[x=2.010576in,y=2.753910in,,base]{\color{textcolor}{\rmfamily\fontsize{10.000000}{12.000000}\selectfont\catcode`\^=\active\def^{\ifmmode\sp\else\^{}\fi}\catcode`\%=\active\def%{\%}Consumption}}%
\end{pgfscope}%
\begin{pgfscope}%
\pgfsetbuttcap%
\pgfsetmiterjoin%
\definecolor{currentfill}{rgb}{1.000000,1.000000,1.000000}%
\pgfsetfillcolor{currentfill}%
\pgfsetlinewidth{0.000000pt}%
\definecolor{currentstroke}{rgb}{0.000000,0.000000,0.000000}%
\pgfsetstrokecolor{currentstroke}%
\pgfsetstrokeopacity{0.000000}%
\pgfsetdash{}{0pt}%
\pgfpathmoveto{\pgfqpoint{3.722897in}{0.857143in}}%
\pgfpathlineto{\pgfqpoint{6.350000in}{0.857143in}}%
\pgfpathlineto{\pgfqpoint{6.350000in}{2.670576in}}%
\pgfpathlineto{\pgfqpoint{3.722897in}{2.670576in}}%
\pgfpathlineto{\pgfqpoint{3.722897in}{0.857143in}}%
\pgfpathclose%
\pgfusepath{fill}%
\end{pgfscope}%
\begin{pgfscope}%
\pgfpathrectangle{\pgfqpoint{3.722897in}{0.857143in}}{\pgfqpoint{2.627103in}{1.813434in}}%
\pgfusepath{clip}%
\pgfsetbuttcap%
\pgfsetroundjoin%
\definecolor{currentfill}{rgb}{0.827451,0.827451,0.827451}%
\pgfsetfillcolor{currentfill}%
\pgfsetfillopacity{0.500000}%
\pgfsetlinewidth{1.003750pt}%
\definecolor{currentstroke}{rgb}{0.827451,0.827451,0.827451}%
\pgfsetstrokecolor{currentstroke}%
\pgfsetstrokeopacity{0.500000}%
\pgfsetdash{}{0pt}%
\pgfpathmoveto{\pgfqpoint{3.991997in}{2.670576in}}%
\pgfpathlineto{\pgfqpoint{3.991997in}{0.857143in}}%
\pgfpathlineto{\pgfqpoint{4.003168in}{0.857143in}}%
\pgfpathlineto{\pgfqpoint{4.014338in}{0.857143in}}%
\pgfpathlineto{\pgfqpoint{4.025509in}{0.857143in}}%
\pgfpathlineto{\pgfqpoint{4.036679in}{0.857143in}}%
\pgfpathlineto{\pgfqpoint{4.036679in}{2.670576in}}%
\pgfpathlineto{\pgfqpoint{4.036679in}{2.670576in}}%
\pgfpathlineto{\pgfqpoint{4.025509in}{2.670576in}}%
\pgfpathlineto{\pgfqpoint{4.014338in}{2.670576in}}%
\pgfpathlineto{\pgfqpoint{4.003168in}{2.670576in}}%
\pgfpathlineto{\pgfqpoint{3.991997in}{2.670576in}}%
\pgfpathlineto{\pgfqpoint{3.991997in}{2.670576in}}%
\pgfpathclose%
\pgfusepath{stroke,fill}%
\end{pgfscope}%
\begin{pgfscope}%
\pgfpathrectangle{\pgfqpoint{3.722897in}{0.857143in}}{\pgfqpoint{2.627103in}{1.813434in}}%
\pgfusepath{clip}%
\pgfsetbuttcap%
\pgfsetroundjoin%
\definecolor{currentfill}{rgb}{0.827451,0.827451,0.827451}%
\pgfsetfillcolor{currentfill}%
\pgfsetfillopacity{0.500000}%
\pgfsetlinewidth{1.003750pt}%
\definecolor{currentstroke}{rgb}{0.827451,0.827451,0.827451}%
\pgfsetstrokecolor{currentstroke}%
\pgfsetstrokeopacity{0.500000}%
\pgfsetdash{}{0pt}%
\pgfpathmoveto{\pgfqpoint{4.170727in}{2.670576in}}%
\pgfpathlineto{\pgfqpoint{4.170727in}{0.857143in}}%
\pgfpathlineto{\pgfqpoint{4.181897in}{0.857143in}}%
\pgfpathlineto{\pgfqpoint{4.193068in}{0.857143in}}%
\pgfpathlineto{\pgfqpoint{4.204238in}{0.857143in}}%
\pgfpathlineto{\pgfqpoint{4.215409in}{0.857143in}}%
\pgfpathlineto{\pgfqpoint{4.226580in}{0.857143in}}%
\pgfpathlineto{\pgfqpoint{4.226580in}{2.670576in}}%
\pgfpathlineto{\pgfqpoint{4.226580in}{2.670576in}}%
\pgfpathlineto{\pgfqpoint{4.215409in}{2.670576in}}%
\pgfpathlineto{\pgfqpoint{4.204238in}{2.670576in}}%
\pgfpathlineto{\pgfqpoint{4.193068in}{2.670576in}}%
\pgfpathlineto{\pgfqpoint{4.181897in}{2.670576in}}%
\pgfpathlineto{\pgfqpoint{4.170727in}{2.670576in}}%
\pgfpathlineto{\pgfqpoint{4.170727in}{2.670576in}}%
\pgfpathclose%
\pgfusepath{stroke,fill}%
\end{pgfscope}%
\begin{pgfscope}%
\pgfpathrectangle{\pgfqpoint{3.722897in}{0.857143in}}{\pgfqpoint{2.627103in}{1.813434in}}%
\pgfusepath{clip}%
\pgfsetbuttcap%
\pgfsetroundjoin%
\definecolor{currentfill}{rgb}{0.827451,0.827451,0.827451}%
\pgfsetfillcolor{currentfill}%
\pgfsetfillopacity{0.500000}%
\pgfsetlinewidth{1.003750pt}%
\definecolor{currentstroke}{rgb}{0.827451,0.827451,0.827451}%
\pgfsetstrokecolor{currentstroke}%
\pgfsetstrokeopacity{0.500000}%
\pgfsetdash{}{0pt}%
\pgfpathmoveto{\pgfqpoint{4.449992in}{2.670576in}}%
\pgfpathlineto{\pgfqpoint{4.449992in}{0.857143in}}%
\pgfpathlineto{\pgfqpoint{4.461162in}{0.857143in}}%
\pgfpathlineto{\pgfqpoint{4.472333in}{0.857143in}}%
\pgfpathlineto{\pgfqpoint{4.472333in}{2.670576in}}%
\pgfpathlineto{\pgfqpoint{4.472333in}{2.670576in}}%
\pgfpathlineto{\pgfqpoint{4.461162in}{2.670576in}}%
\pgfpathlineto{\pgfqpoint{4.449992in}{2.670576in}}%
\pgfpathlineto{\pgfqpoint{4.449992in}{2.670576in}}%
\pgfpathclose%
\pgfusepath{stroke,fill}%
\end{pgfscope}%
\begin{pgfscope}%
\pgfpathrectangle{\pgfqpoint{3.722897in}{0.857143in}}{\pgfqpoint{2.627103in}{1.813434in}}%
\pgfusepath{clip}%
\pgfsetbuttcap%
\pgfsetroundjoin%
\definecolor{currentfill}{rgb}{0.827451,0.827451,0.827451}%
\pgfsetfillcolor{currentfill}%
\pgfsetfillopacity{0.500000}%
\pgfsetlinewidth{1.003750pt}%
\definecolor{currentstroke}{rgb}{0.827451,0.827451,0.827451}%
\pgfsetstrokecolor{currentstroke}%
\pgfsetstrokeopacity{0.500000}%
\pgfsetdash{}{0pt}%
\pgfpathmoveto{\pgfqpoint{4.517015in}{2.670576in}}%
\pgfpathlineto{\pgfqpoint{4.517015in}{0.857143in}}%
\pgfpathlineto{\pgfqpoint{4.528186in}{0.857143in}}%
\pgfpathlineto{\pgfqpoint{4.539357in}{0.857143in}}%
\pgfpathlineto{\pgfqpoint{4.550527in}{0.857143in}}%
\pgfpathlineto{\pgfqpoint{4.561698in}{0.857143in}}%
\pgfpathlineto{\pgfqpoint{4.572868in}{0.857143in}}%
\pgfpathlineto{\pgfqpoint{4.572868in}{2.670576in}}%
\pgfpathlineto{\pgfqpoint{4.572868in}{2.670576in}}%
\pgfpathlineto{\pgfqpoint{4.561698in}{2.670576in}}%
\pgfpathlineto{\pgfqpoint{4.550527in}{2.670576in}}%
\pgfpathlineto{\pgfqpoint{4.539357in}{2.670576in}}%
\pgfpathlineto{\pgfqpoint{4.528186in}{2.670576in}}%
\pgfpathlineto{\pgfqpoint{4.517015in}{2.670576in}}%
\pgfpathlineto{\pgfqpoint{4.517015in}{2.670576in}}%
\pgfpathclose%
\pgfusepath{stroke,fill}%
\end{pgfscope}%
\begin{pgfscope}%
\pgfpathrectangle{\pgfqpoint{3.722897in}{0.857143in}}{\pgfqpoint{2.627103in}{1.813434in}}%
\pgfusepath{clip}%
\pgfsetbuttcap%
\pgfsetroundjoin%
\definecolor{currentfill}{rgb}{0.827451,0.827451,0.827451}%
\pgfsetfillcolor{currentfill}%
\pgfsetfillopacity{0.500000}%
\pgfsetlinewidth{1.003750pt}%
\definecolor{currentstroke}{rgb}{0.827451,0.827451,0.827451}%
\pgfsetstrokecolor{currentstroke}%
\pgfsetstrokeopacity{0.500000}%
\pgfsetdash{}{0pt}%
\pgfpathmoveto{\pgfqpoint{4.919157in}{2.670576in}}%
\pgfpathlineto{\pgfqpoint{4.919157in}{0.857143in}}%
\pgfpathlineto{\pgfqpoint{4.930328in}{0.857143in}}%
\pgfpathlineto{\pgfqpoint{4.941498in}{0.857143in}}%
\pgfpathlineto{\pgfqpoint{4.941498in}{2.670576in}}%
\pgfpathlineto{\pgfqpoint{4.941498in}{2.670576in}}%
\pgfpathlineto{\pgfqpoint{4.930328in}{2.670576in}}%
\pgfpathlineto{\pgfqpoint{4.919157in}{2.670576in}}%
\pgfpathlineto{\pgfqpoint{4.919157in}{2.670576in}}%
\pgfpathclose%
\pgfusepath{stroke,fill}%
\end{pgfscope}%
\begin{pgfscope}%
\pgfpathrectangle{\pgfqpoint{3.722897in}{0.857143in}}{\pgfqpoint{2.627103in}{1.813434in}}%
\pgfusepath{clip}%
\pgfsetbuttcap%
\pgfsetroundjoin%
\definecolor{currentfill}{rgb}{0.827451,0.827451,0.827451}%
\pgfsetfillcolor{currentfill}%
\pgfsetfillopacity{0.500000}%
\pgfsetlinewidth{1.003750pt}%
\definecolor{currentstroke}{rgb}{0.827451,0.827451,0.827451}%
\pgfsetstrokecolor{currentstroke}%
\pgfsetstrokeopacity{0.500000}%
\pgfsetdash{}{0pt}%
\pgfpathmoveto{\pgfqpoint{5.388323in}{2.670576in}}%
\pgfpathlineto{\pgfqpoint{5.388323in}{0.857143in}}%
\pgfpathlineto{\pgfqpoint{5.399493in}{0.857143in}}%
\pgfpathlineto{\pgfqpoint{5.410664in}{0.857143in}}%
\pgfpathlineto{\pgfqpoint{5.421834in}{0.857143in}}%
\pgfpathlineto{\pgfqpoint{5.421834in}{2.670576in}}%
\pgfpathlineto{\pgfqpoint{5.421834in}{2.670576in}}%
\pgfpathlineto{\pgfqpoint{5.410664in}{2.670576in}}%
\pgfpathlineto{\pgfqpoint{5.399493in}{2.670576in}}%
\pgfpathlineto{\pgfqpoint{5.388323in}{2.670576in}}%
\pgfpathlineto{\pgfqpoint{5.388323in}{2.670576in}}%
\pgfpathclose%
\pgfusepath{stroke,fill}%
\end{pgfscope}%
\begin{pgfscope}%
\pgfpathrectangle{\pgfqpoint{3.722897in}{0.857143in}}{\pgfqpoint{2.627103in}{1.813434in}}%
\pgfusepath{clip}%
\pgfsetbuttcap%
\pgfsetroundjoin%
\definecolor{currentfill}{rgb}{0.827451,0.827451,0.827451}%
\pgfsetfillcolor{currentfill}%
\pgfsetfillopacity{0.500000}%
\pgfsetlinewidth{1.003750pt}%
\definecolor{currentstroke}{rgb}{0.827451,0.827451,0.827451}%
\pgfsetstrokecolor{currentstroke}%
\pgfsetstrokeopacity{0.500000}%
\pgfsetdash{}{0pt}%
\pgfpathmoveto{\pgfqpoint{5.689929in}{2.670576in}}%
\pgfpathlineto{\pgfqpoint{5.689929in}{0.857143in}}%
\pgfpathlineto{\pgfqpoint{5.701100in}{0.857143in}}%
\pgfpathlineto{\pgfqpoint{5.712270in}{0.857143in}}%
\pgfpathlineto{\pgfqpoint{5.723441in}{0.857143in}}%
\pgfpathlineto{\pgfqpoint{5.734611in}{0.857143in}}%
\pgfpathlineto{\pgfqpoint{5.745782in}{0.857143in}}%
\pgfpathlineto{\pgfqpoint{5.756953in}{0.857143in}}%
\pgfpathlineto{\pgfqpoint{5.756953in}{2.670576in}}%
\pgfpathlineto{\pgfqpoint{5.756953in}{2.670576in}}%
\pgfpathlineto{\pgfqpoint{5.745782in}{2.670576in}}%
\pgfpathlineto{\pgfqpoint{5.734611in}{2.670576in}}%
\pgfpathlineto{\pgfqpoint{5.723441in}{2.670576in}}%
\pgfpathlineto{\pgfqpoint{5.712270in}{2.670576in}}%
\pgfpathlineto{\pgfqpoint{5.701100in}{2.670576in}}%
\pgfpathlineto{\pgfqpoint{5.689929in}{2.670576in}}%
\pgfpathlineto{\pgfqpoint{5.689929in}{2.670576in}}%
\pgfpathclose%
\pgfusepath{stroke,fill}%
\end{pgfscope}%
\begin{pgfscope}%
\pgfpathrectangle{\pgfqpoint{3.722897in}{0.857143in}}{\pgfqpoint{2.627103in}{1.813434in}}%
\pgfusepath{clip}%
\pgfsetbuttcap%
\pgfsetroundjoin%
\definecolor{currentfill}{rgb}{0.827451,0.827451,0.827451}%
\pgfsetfillcolor{currentfill}%
\pgfsetfillopacity{0.500000}%
\pgfsetlinewidth{1.003750pt}%
\definecolor{currentstroke}{rgb}{0.827451,0.827451,0.827451}%
\pgfsetstrokecolor{currentstroke}%
\pgfsetstrokeopacity{0.500000}%
\pgfsetdash{}{0pt}%
\pgfpathmoveto{\pgfqpoint{6.226118in}{2.670576in}}%
\pgfpathlineto{\pgfqpoint{6.226118in}{0.857143in}}%
\pgfpathlineto{\pgfqpoint{6.226118in}{2.670576in}}%
\pgfpathlineto{\pgfqpoint{6.226118in}{2.670576in}}%
\pgfpathlineto{\pgfqpoint{6.226118in}{2.670576in}}%
\pgfpathclose%
\pgfusepath{stroke,fill}%
\end{pgfscope}%
\begin{pgfscope}%
\pgfpathrectangle{\pgfqpoint{3.722897in}{0.857143in}}{\pgfqpoint{2.627103in}{1.813434in}}%
\pgfusepath{clip}%
\pgfsetbuttcap%
\pgfsetmiterjoin%
\definecolor{currentfill}{rgb}{0.066899,0.263188,0.377594}%
\pgfsetfillcolor{currentfill}%
\pgfsetlinewidth{0.000000pt}%
\definecolor{currentstroke}{rgb}{0.000000,0.000000,0.000000}%
\pgfsetstrokecolor{currentstroke}%
\pgfsetstrokeopacity{0.000000}%
\pgfsetdash{}{0pt}%
\pgfpathmoveto{\pgfqpoint{3.842311in}{1.813884in}}%
\pgfpathlineto{\pgfqpoint{3.851247in}{1.813884in}}%
\pgfpathlineto{\pgfqpoint{3.851247in}{1.813044in}}%
\pgfpathlineto{\pgfqpoint{3.842311in}{1.813044in}}%
\pgfpathlineto{\pgfqpoint{3.842311in}{1.813884in}}%
\pgfpathclose%
\pgfusepath{fill}%
\end{pgfscope}%
\begin{pgfscope}%
\pgfpathrectangle{\pgfqpoint{3.722897in}{0.857143in}}{\pgfqpoint{2.627103in}{1.813434in}}%
\pgfusepath{clip}%
\pgfsetbuttcap%
\pgfsetmiterjoin%
\definecolor{currentfill}{rgb}{0.066899,0.263188,0.377594}%
\pgfsetfillcolor{currentfill}%
\pgfsetlinewidth{0.000000pt}%
\definecolor{currentstroke}{rgb}{0.000000,0.000000,0.000000}%
\pgfsetstrokecolor{currentstroke}%
\pgfsetstrokeopacity{0.000000}%
\pgfsetdash{}{0pt}%
\pgfpathmoveto{\pgfqpoint{3.853481in}{1.813884in}}%
\pgfpathlineto{\pgfqpoint{3.862418in}{1.813884in}}%
\pgfpathlineto{\pgfqpoint{3.862418in}{1.814336in}}%
\pgfpathlineto{\pgfqpoint{3.853481in}{1.814336in}}%
\pgfpathlineto{\pgfqpoint{3.853481in}{1.813884in}}%
\pgfpathclose%
\pgfusepath{fill}%
\end{pgfscope}%
\begin{pgfscope}%
\pgfpathrectangle{\pgfqpoint{3.722897in}{0.857143in}}{\pgfqpoint{2.627103in}{1.813434in}}%
\pgfusepath{clip}%
\pgfsetbuttcap%
\pgfsetmiterjoin%
\definecolor{currentfill}{rgb}{0.066899,0.263188,0.377594}%
\pgfsetfillcolor{currentfill}%
\pgfsetlinewidth{0.000000pt}%
\definecolor{currentstroke}{rgb}{0.000000,0.000000,0.000000}%
\pgfsetstrokecolor{currentstroke}%
\pgfsetstrokeopacity{0.000000}%
\pgfsetdash{}{0pt}%
\pgfpathmoveto{\pgfqpoint{3.864652in}{1.813884in}}%
\pgfpathlineto{\pgfqpoint{3.873588in}{1.813884in}}%
\pgfpathlineto{\pgfqpoint{3.873588in}{1.815652in}}%
\pgfpathlineto{\pgfqpoint{3.864652in}{1.815652in}}%
\pgfpathlineto{\pgfqpoint{3.864652in}{1.813884in}}%
\pgfpathclose%
\pgfusepath{fill}%
\end{pgfscope}%
\begin{pgfscope}%
\pgfpathrectangle{\pgfqpoint{3.722897in}{0.857143in}}{\pgfqpoint{2.627103in}{1.813434in}}%
\pgfusepath{clip}%
\pgfsetbuttcap%
\pgfsetmiterjoin%
\definecolor{currentfill}{rgb}{0.066899,0.263188,0.377594}%
\pgfsetfillcolor{currentfill}%
\pgfsetlinewidth{0.000000pt}%
\definecolor{currentstroke}{rgb}{0.000000,0.000000,0.000000}%
\pgfsetstrokecolor{currentstroke}%
\pgfsetstrokeopacity{0.000000}%
\pgfsetdash{}{0pt}%
\pgfpathmoveto{\pgfqpoint{3.875823in}{1.813884in}}%
\pgfpathlineto{\pgfqpoint{3.884759in}{1.813884in}}%
\pgfpathlineto{\pgfqpoint{3.884759in}{1.816540in}}%
\pgfpathlineto{\pgfqpoint{3.875823in}{1.816540in}}%
\pgfpathlineto{\pgfqpoint{3.875823in}{1.813884in}}%
\pgfpathclose%
\pgfusepath{fill}%
\end{pgfscope}%
\begin{pgfscope}%
\pgfpathrectangle{\pgfqpoint{3.722897in}{0.857143in}}{\pgfqpoint{2.627103in}{1.813434in}}%
\pgfusepath{clip}%
\pgfsetbuttcap%
\pgfsetmiterjoin%
\definecolor{currentfill}{rgb}{0.066899,0.263188,0.377594}%
\pgfsetfillcolor{currentfill}%
\pgfsetlinewidth{0.000000pt}%
\definecolor{currentstroke}{rgb}{0.000000,0.000000,0.000000}%
\pgfsetstrokecolor{currentstroke}%
\pgfsetstrokeopacity{0.000000}%
\pgfsetdash{}{0pt}%
\pgfpathmoveto{\pgfqpoint{3.886993in}{1.813884in}}%
\pgfpathlineto{\pgfqpoint{3.895930in}{1.813884in}}%
\pgfpathlineto{\pgfqpoint{3.895930in}{1.817502in}}%
\pgfpathlineto{\pgfqpoint{3.886993in}{1.817502in}}%
\pgfpathlineto{\pgfqpoint{3.886993in}{1.813884in}}%
\pgfpathclose%
\pgfusepath{fill}%
\end{pgfscope}%
\begin{pgfscope}%
\pgfpathrectangle{\pgfqpoint{3.722897in}{0.857143in}}{\pgfqpoint{2.627103in}{1.813434in}}%
\pgfusepath{clip}%
\pgfsetbuttcap%
\pgfsetmiterjoin%
\definecolor{currentfill}{rgb}{0.066899,0.263188,0.377594}%
\pgfsetfillcolor{currentfill}%
\pgfsetlinewidth{0.000000pt}%
\definecolor{currentstroke}{rgb}{0.000000,0.000000,0.000000}%
\pgfsetstrokecolor{currentstroke}%
\pgfsetstrokeopacity{0.000000}%
\pgfsetdash{}{0pt}%
\pgfpathmoveto{\pgfqpoint{3.898164in}{1.813884in}}%
\pgfpathlineto{\pgfqpoint{3.907100in}{1.813884in}}%
\pgfpathlineto{\pgfqpoint{3.907100in}{1.817902in}}%
\pgfpathlineto{\pgfqpoint{3.898164in}{1.817902in}}%
\pgfpathlineto{\pgfqpoint{3.898164in}{1.813884in}}%
\pgfpathclose%
\pgfusepath{fill}%
\end{pgfscope}%
\begin{pgfscope}%
\pgfpathrectangle{\pgfqpoint{3.722897in}{0.857143in}}{\pgfqpoint{2.627103in}{1.813434in}}%
\pgfusepath{clip}%
\pgfsetbuttcap%
\pgfsetmiterjoin%
\definecolor{currentfill}{rgb}{0.066899,0.263188,0.377594}%
\pgfsetfillcolor{currentfill}%
\pgfsetlinewidth{0.000000pt}%
\definecolor{currentstroke}{rgb}{0.000000,0.000000,0.000000}%
\pgfsetstrokecolor{currentstroke}%
\pgfsetstrokeopacity{0.000000}%
\pgfsetdash{}{0pt}%
\pgfpathmoveto{\pgfqpoint{3.909334in}{1.813884in}}%
\pgfpathlineto{\pgfqpoint{3.918271in}{1.813884in}}%
\pgfpathlineto{\pgfqpoint{3.918271in}{1.821744in}}%
\pgfpathlineto{\pgfqpoint{3.909334in}{1.821744in}}%
\pgfpathlineto{\pgfqpoint{3.909334in}{1.813884in}}%
\pgfpathclose%
\pgfusepath{fill}%
\end{pgfscope}%
\begin{pgfscope}%
\pgfpathrectangle{\pgfqpoint{3.722897in}{0.857143in}}{\pgfqpoint{2.627103in}{1.813434in}}%
\pgfusepath{clip}%
\pgfsetbuttcap%
\pgfsetmiterjoin%
\definecolor{currentfill}{rgb}{0.066899,0.263188,0.377594}%
\pgfsetfillcolor{currentfill}%
\pgfsetlinewidth{0.000000pt}%
\definecolor{currentstroke}{rgb}{0.000000,0.000000,0.000000}%
\pgfsetstrokecolor{currentstroke}%
\pgfsetstrokeopacity{0.000000}%
\pgfsetdash{}{0pt}%
\pgfpathmoveto{\pgfqpoint{3.920505in}{1.813884in}}%
\pgfpathlineto{\pgfqpoint{3.929442in}{1.813884in}}%
\pgfpathlineto{\pgfqpoint{3.929442in}{1.823552in}}%
\pgfpathlineto{\pgfqpoint{3.920505in}{1.823552in}}%
\pgfpathlineto{\pgfqpoint{3.920505in}{1.813884in}}%
\pgfpathclose%
\pgfusepath{fill}%
\end{pgfscope}%
\begin{pgfscope}%
\pgfpathrectangle{\pgfqpoint{3.722897in}{0.857143in}}{\pgfqpoint{2.627103in}{1.813434in}}%
\pgfusepath{clip}%
\pgfsetbuttcap%
\pgfsetmiterjoin%
\definecolor{currentfill}{rgb}{0.066899,0.263188,0.377594}%
\pgfsetfillcolor{currentfill}%
\pgfsetlinewidth{0.000000pt}%
\definecolor{currentstroke}{rgb}{0.000000,0.000000,0.000000}%
\pgfsetstrokecolor{currentstroke}%
\pgfsetstrokeopacity{0.000000}%
\pgfsetdash{}{0pt}%
\pgfpathmoveto{\pgfqpoint{3.931676in}{1.813884in}}%
\pgfpathlineto{\pgfqpoint{3.940612in}{1.813884in}}%
\pgfpathlineto{\pgfqpoint{3.940612in}{1.824213in}}%
\pgfpathlineto{\pgfqpoint{3.931676in}{1.824213in}}%
\pgfpathlineto{\pgfqpoint{3.931676in}{1.813884in}}%
\pgfpathclose%
\pgfusepath{fill}%
\end{pgfscope}%
\begin{pgfscope}%
\pgfpathrectangle{\pgfqpoint{3.722897in}{0.857143in}}{\pgfqpoint{2.627103in}{1.813434in}}%
\pgfusepath{clip}%
\pgfsetbuttcap%
\pgfsetmiterjoin%
\definecolor{currentfill}{rgb}{0.066899,0.263188,0.377594}%
\pgfsetfillcolor{currentfill}%
\pgfsetlinewidth{0.000000pt}%
\definecolor{currentstroke}{rgb}{0.000000,0.000000,0.000000}%
\pgfsetstrokecolor{currentstroke}%
\pgfsetstrokeopacity{0.000000}%
\pgfsetdash{}{0pt}%
\pgfpathmoveto{\pgfqpoint{3.942846in}{1.813884in}}%
\pgfpathlineto{\pgfqpoint{3.951783in}{1.813884in}}%
\pgfpathlineto{\pgfqpoint{3.951783in}{1.824145in}}%
\pgfpathlineto{\pgfqpoint{3.942846in}{1.824145in}}%
\pgfpathlineto{\pgfqpoint{3.942846in}{1.813884in}}%
\pgfpathclose%
\pgfusepath{fill}%
\end{pgfscope}%
\begin{pgfscope}%
\pgfpathrectangle{\pgfqpoint{3.722897in}{0.857143in}}{\pgfqpoint{2.627103in}{1.813434in}}%
\pgfusepath{clip}%
\pgfsetbuttcap%
\pgfsetmiterjoin%
\definecolor{currentfill}{rgb}{0.066899,0.263188,0.377594}%
\pgfsetfillcolor{currentfill}%
\pgfsetlinewidth{0.000000pt}%
\definecolor{currentstroke}{rgb}{0.000000,0.000000,0.000000}%
\pgfsetstrokecolor{currentstroke}%
\pgfsetstrokeopacity{0.000000}%
\pgfsetdash{}{0pt}%
\pgfpathmoveto{\pgfqpoint{3.954017in}{1.813884in}}%
\pgfpathlineto{\pgfqpoint{3.962953in}{1.813884in}}%
\pgfpathlineto{\pgfqpoint{3.962953in}{1.825665in}}%
\pgfpathlineto{\pgfqpoint{3.954017in}{1.825665in}}%
\pgfpathlineto{\pgfqpoint{3.954017in}{1.813884in}}%
\pgfpathclose%
\pgfusepath{fill}%
\end{pgfscope}%
\begin{pgfscope}%
\pgfpathrectangle{\pgfqpoint{3.722897in}{0.857143in}}{\pgfqpoint{2.627103in}{1.813434in}}%
\pgfusepath{clip}%
\pgfsetbuttcap%
\pgfsetmiterjoin%
\definecolor{currentfill}{rgb}{0.066899,0.263188,0.377594}%
\pgfsetfillcolor{currentfill}%
\pgfsetlinewidth{0.000000pt}%
\definecolor{currentstroke}{rgb}{0.000000,0.000000,0.000000}%
\pgfsetstrokecolor{currentstroke}%
\pgfsetstrokeopacity{0.000000}%
\pgfsetdash{}{0pt}%
\pgfpathmoveto{\pgfqpoint{3.965187in}{1.813884in}}%
\pgfpathlineto{\pgfqpoint{3.974124in}{1.813884in}}%
\pgfpathlineto{\pgfqpoint{3.974124in}{1.824343in}}%
\pgfpathlineto{\pgfqpoint{3.965187in}{1.824343in}}%
\pgfpathlineto{\pgfqpoint{3.965187in}{1.813884in}}%
\pgfpathclose%
\pgfusepath{fill}%
\end{pgfscope}%
\begin{pgfscope}%
\pgfpathrectangle{\pgfqpoint{3.722897in}{0.857143in}}{\pgfqpoint{2.627103in}{1.813434in}}%
\pgfusepath{clip}%
\pgfsetbuttcap%
\pgfsetmiterjoin%
\definecolor{currentfill}{rgb}{0.066899,0.263188,0.377594}%
\pgfsetfillcolor{currentfill}%
\pgfsetlinewidth{0.000000pt}%
\definecolor{currentstroke}{rgb}{0.000000,0.000000,0.000000}%
\pgfsetstrokecolor{currentstroke}%
\pgfsetstrokeopacity{0.000000}%
\pgfsetdash{}{0pt}%
\pgfpathmoveto{\pgfqpoint{3.976358in}{1.813884in}}%
\pgfpathlineto{\pgfqpoint{3.985295in}{1.813884in}}%
\pgfpathlineto{\pgfqpoint{3.985295in}{1.824342in}}%
\pgfpathlineto{\pgfqpoint{3.976358in}{1.824342in}}%
\pgfpathlineto{\pgfqpoint{3.976358in}{1.813884in}}%
\pgfpathclose%
\pgfusepath{fill}%
\end{pgfscope}%
\begin{pgfscope}%
\pgfpathrectangle{\pgfqpoint{3.722897in}{0.857143in}}{\pgfqpoint{2.627103in}{1.813434in}}%
\pgfusepath{clip}%
\pgfsetbuttcap%
\pgfsetmiterjoin%
\definecolor{currentfill}{rgb}{0.066899,0.263188,0.377594}%
\pgfsetfillcolor{currentfill}%
\pgfsetlinewidth{0.000000pt}%
\definecolor{currentstroke}{rgb}{0.000000,0.000000,0.000000}%
\pgfsetstrokecolor{currentstroke}%
\pgfsetstrokeopacity{0.000000}%
\pgfsetdash{}{0pt}%
\pgfpathmoveto{\pgfqpoint{3.987529in}{1.813884in}}%
\pgfpathlineto{\pgfqpoint{3.996465in}{1.813884in}}%
\pgfpathlineto{\pgfqpoint{3.996465in}{1.823242in}}%
\pgfpathlineto{\pgfqpoint{3.987529in}{1.823242in}}%
\pgfpathlineto{\pgfqpoint{3.987529in}{1.813884in}}%
\pgfpathclose%
\pgfusepath{fill}%
\end{pgfscope}%
\begin{pgfscope}%
\pgfpathrectangle{\pgfqpoint{3.722897in}{0.857143in}}{\pgfqpoint{2.627103in}{1.813434in}}%
\pgfusepath{clip}%
\pgfsetbuttcap%
\pgfsetmiterjoin%
\definecolor{currentfill}{rgb}{0.066899,0.263188,0.377594}%
\pgfsetfillcolor{currentfill}%
\pgfsetlinewidth{0.000000pt}%
\definecolor{currentstroke}{rgb}{0.000000,0.000000,0.000000}%
\pgfsetstrokecolor{currentstroke}%
\pgfsetstrokeopacity{0.000000}%
\pgfsetdash{}{0pt}%
\pgfpathmoveto{\pgfqpoint{3.998699in}{1.813884in}}%
\pgfpathlineto{\pgfqpoint{4.007636in}{1.813884in}}%
\pgfpathlineto{\pgfqpoint{4.007636in}{1.822803in}}%
\pgfpathlineto{\pgfqpoint{3.998699in}{1.822803in}}%
\pgfpathlineto{\pgfqpoint{3.998699in}{1.813884in}}%
\pgfpathclose%
\pgfusepath{fill}%
\end{pgfscope}%
\begin{pgfscope}%
\pgfpathrectangle{\pgfqpoint{3.722897in}{0.857143in}}{\pgfqpoint{2.627103in}{1.813434in}}%
\pgfusepath{clip}%
\pgfsetbuttcap%
\pgfsetmiterjoin%
\definecolor{currentfill}{rgb}{0.066899,0.263188,0.377594}%
\pgfsetfillcolor{currentfill}%
\pgfsetlinewidth{0.000000pt}%
\definecolor{currentstroke}{rgb}{0.000000,0.000000,0.000000}%
\pgfsetstrokecolor{currentstroke}%
\pgfsetstrokeopacity{0.000000}%
\pgfsetdash{}{0pt}%
\pgfpathmoveto{\pgfqpoint{4.009870in}{1.813884in}}%
\pgfpathlineto{\pgfqpoint{4.018806in}{1.813884in}}%
\pgfpathlineto{\pgfqpoint{4.018806in}{1.824890in}}%
\pgfpathlineto{\pgfqpoint{4.009870in}{1.824890in}}%
\pgfpathlineto{\pgfqpoint{4.009870in}{1.813884in}}%
\pgfpathclose%
\pgfusepath{fill}%
\end{pgfscope}%
\begin{pgfscope}%
\pgfpathrectangle{\pgfqpoint{3.722897in}{0.857143in}}{\pgfqpoint{2.627103in}{1.813434in}}%
\pgfusepath{clip}%
\pgfsetbuttcap%
\pgfsetmiterjoin%
\definecolor{currentfill}{rgb}{0.066899,0.263188,0.377594}%
\pgfsetfillcolor{currentfill}%
\pgfsetlinewidth{0.000000pt}%
\definecolor{currentstroke}{rgb}{0.000000,0.000000,0.000000}%
\pgfsetstrokecolor{currentstroke}%
\pgfsetstrokeopacity{0.000000}%
\pgfsetdash{}{0pt}%
\pgfpathmoveto{\pgfqpoint{4.021040in}{1.813884in}}%
\pgfpathlineto{\pgfqpoint{4.029977in}{1.813884in}}%
\pgfpathlineto{\pgfqpoint{4.029977in}{1.827211in}}%
\pgfpathlineto{\pgfqpoint{4.021040in}{1.827211in}}%
\pgfpathlineto{\pgfqpoint{4.021040in}{1.813884in}}%
\pgfpathclose%
\pgfusepath{fill}%
\end{pgfscope}%
\begin{pgfscope}%
\pgfpathrectangle{\pgfqpoint{3.722897in}{0.857143in}}{\pgfqpoint{2.627103in}{1.813434in}}%
\pgfusepath{clip}%
\pgfsetbuttcap%
\pgfsetmiterjoin%
\definecolor{currentfill}{rgb}{0.066899,0.263188,0.377594}%
\pgfsetfillcolor{currentfill}%
\pgfsetlinewidth{0.000000pt}%
\definecolor{currentstroke}{rgb}{0.000000,0.000000,0.000000}%
\pgfsetstrokecolor{currentstroke}%
\pgfsetstrokeopacity{0.000000}%
\pgfsetdash{}{0pt}%
\pgfpathmoveto{\pgfqpoint{4.032211in}{1.813884in}}%
\pgfpathlineto{\pgfqpoint{4.041148in}{1.813884in}}%
\pgfpathlineto{\pgfqpoint{4.041148in}{1.826169in}}%
\pgfpathlineto{\pgfqpoint{4.032211in}{1.826169in}}%
\pgfpathlineto{\pgfqpoint{4.032211in}{1.813884in}}%
\pgfpathclose%
\pgfusepath{fill}%
\end{pgfscope}%
\begin{pgfscope}%
\pgfpathrectangle{\pgfqpoint{3.722897in}{0.857143in}}{\pgfqpoint{2.627103in}{1.813434in}}%
\pgfusepath{clip}%
\pgfsetbuttcap%
\pgfsetmiterjoin%
\definecolor{currentfill}{rgb}{0.066899,0.263188,0.377594}%
\pgfsetfillcolor{currentfill}%
\pgfsetlinewidth{0.000000pt}%
\definecolor{currentstroke}{rgb}{0.000000,0.000000,0.000000}%
\pgfsetstrokecolor{currentstroke}%
\pgfsetstrokeopacity{0.000000}%
\pgfsetdash{}{0pt}%
\pgfpathmoveto{\pgfqpoint{4.043382in}{1.813884in}}%
\pgfpathlineto{\pgfqpoint{4.052318in}{1.813884in}}%
\pgfpathlineto{\pgfqpoint{4.052318in}{1.830708in}}%
\pgfpathlineto{\pgfqpoint{4.043382in}{1.830708in}}%
\pgfpathlineto{\pgfqpoint{4.043382in}{1.813884in}}%
\pgfpathclose%
\pgfusepath{fill}%
\end{pgfscope}%
\begin{pgfscope}%
\pgfpathrectangle{\pgfqpoint{3.722897in}{0.857143in}}{\pgfqpoint{2.627103in}{1.813434in}}%
\pgfusepath{clip}%
\pgfsetbuttcap%
\pgfsetmiterjoin%
\definecolor{currentfill}{rgb}{0.066899,0.263188,0.377594}%
\pgfsetfillcolor{currentfill}%
\pgfsetlinewidth{0.000000pt}%
\definecolor{currentstroke}{rgb}{0.000000,0.000000,0.000000}%
\pgfsetstrokecolor{currentstroke}%
\pgfsetstrokeopacity{0.000000}%
\pgfsetdash{}{0pt}%
\pgfpathmoveto{\pgfqpoint{4.054552in}{1.813884in}}%
\pgfpathlineto{\pgfqpoint{4.063489in}{1.813884in}}%
\pgfpathlineto{\pgfqpoint{4.063489in}{1.831364in}}%
\pgfpathlineto{\pgfqpoint{4.054552in}{1.831364in}}%
\pgfpathlineto{\pgfqpoint{4.054552in}{1.813884in}}%
\pgfpathclose%
\pgfusepath{fill}%
\end{pgfscope}%
\begin{pgfscope}%
\pgfpathrectangle{\pgfqpoint{3.722897in}{0.857143in}}{\pgfqpoint{2.627103in}{1.813434in}}%
\pgfusepath{clip}%
\pgfsetbuttcap%
\pgfsetmiterjoin%
\definecolor{currentfill}{rgb}{0.066899,0.263188,0.377594}%
\pgfsetfillcolor{currentfill}%
\pgfsetlinewidth{0.000000pt}%
\definecolor{currentstroke}{rgb}{0.000000,0.000000,0.000000}%
\pgfsetstrokecolor{currentstroke}%
\pgfsetstrokeopacity{0.000000}%
\pgfsetdash{}{0pt}%
\pgfpathmoveto{\pgfqpoint{4.065723in}{1.813884in}}%
\pgfpathlineto{\pgfqpoint{4.074659in}{1.813884in}}%
\pgfpathlineto{\pgfqpoint{4.074659in}{1.833515in}}%
\pgfpathlineto{\pgfqpoint{4.065723in}{1.833515in}}%
\pgfpathlineto{\pgfqpoint{4.065723in}{1.813884in}}%
\pgfpathclose%
\pgfusepath{fill}%
\end{pgfscope}%
\begin{pgfscope}%
\pgfpathrectangle{\pgfqpoint{3.722897in}{0.857143in}}{\pgfqpoint{2.627103in}{1.813434in}}%
\pgfusepath{clip}%
\pgfsetbuttcap%
\pgfsetmiterjoin%
\definecolor{currentfill}{rgb}{0.066899,0.263188,0.377594}%
\pgfsetfillcolor{currentfill}%
\pgfsetlinewidth{0.000000pt}%
\definecolor{currentstroke}{rgb}{0.000000,0.000000,0.000000}%
\pgfsetstrokecolor{currentstroke}%
\pgfsetstrokeopacity{0.000000}%
\pgfsetdash{}{0pt}%
\pgfpathmoveto{\pgfqpoint{4.076893in}{1.813884in}}%
\pgfpathlineto{\pgfqpoint{4.085830in}{1.813884in}}%
\pgfpathlineto{\pgfqpoint{4.085830in}{1.832151in}}%
\pgfpathlineto{\pgfqpoint{4.076893in}{1.832151in}}%
\pgfpathlineto{\pgfqpoint{4.076893in}{1.813884in}}%
\pgfpathclose%
\pgfusepath{fill}%
\end{pgfscope}%
\begin{pgfscope}%
\pgfpathrectangle{\pgfqpoint{3.722897in}{0.857143in}}{\pgfqpoint{2.627103in}{1.813434in}}%
\pgfusepath{clip}%
\pgfsetbuttcap%
\pgfsetmiterjoin%
\definecolor{currentfill}{rgb}{0.066899,0.263188,0.377594}%
\pgfsetfillcolor{currentfill}%
\pgfsetlinewidth{0.000000pt}%
\definecolor{currentstroke}{rgb}{0.000000,0.000000,0.000000}%
\pgfsetstrokecolor{currentstroke}%
\pgfsetstrokeopacity{0.000000}%
\pgfsetdash{}{0pt}%
\pgfpathmoveto{\pgfqpoint{4.088064in}{1.813884in}}%
\pgfpathlineto{\pgfqpoint{4.097001in}{1.813884in}}%
\pgfpathlineto{\pgfqpoint{4.097001in}{1.833754in}}%
\pgfpathlineto{\pgfqpoint{4.088064in}{1.833754in}}%
\pgfpathlineto{\pgfqpoint{4.088064in}{1.813884in}}%
\pgfpathclose%
\pgfusepath{fill}%
\end{pgfscope}%
\begin{pgfscope}%
\pgfpathrectangle{\pgfqpoint{3.722897in}{0.857143in}}{\pgfqpoint{2.627103in}{1.813434in}}%
\pgfusepath{clip}%
\pgfsetbuttcap%
\pgfsetmiterjoin%
\definecolor{currentfill}{rgb}{0.066899,0.263188,0.377594}%
\pgfsetfillcolor{currentfill}%
\pgfsetlinewidth{0.000000pt}%
\definecolor{currentstroke}{rgb}{0.000000,0.000000,0.000000}%
\pgfsetstrokecolor{currentstroke}%
\pgfsetstrokeopacity{0.000000}%
\pgfsetdash{}{0pt}%
\pgfpathmoveto{\pgfqpoint{4.099235in}{1.813884in}}%
\pgfpathlineto{\pgfqpoint{4.108171in}{1.813884in}}%
\pgfpathlineto{\pgfqpoint{4.108171in}{1.837100in}}%
\pgfpathlineto{\pgfqpoint{4.099235in}{1.837100in}}%
\pgfpathlineto{\pgfqpoint{4.099235in}{1.813884in}}%
\pgfpathclose%
\pgfusepath{fill}%
\end{pgfscope}%
\begin{pgfscope}%
\pgfpathrectangle{\pgfqpoint{3.722897in}{0.857143in}}{\pgfqpoint{2.627103in}{1.813434in}}%
\pgfusepath{clip}%
\pgfsetbuttcap%
\pgfsetmiterjoin%
\definecolor{currentfill}{rgb}{0.066899,0.263188,0.377594}%
\pgfsetfillcolor{currentfill}%
\pgfsetlinewidth{0.000000pt}%
\definecolor{currentstroke}{rgb}{0.000000,0.000000,0.000000}%
\pgfsetstrokecolor{currentstroke}%
\pgfsetstrokeopacity{0.000000}%
\pgfsetdash{}{0pt}%
\pgfpathmoveto{\pgfqpoint{4.110405in}{1.813884in}}%
\pgfpathlineto{\pgfqpoint{4.119342in}{1.813884in}}%
\pgfpathlineto{\pgfqpoint{4.119342in}{1.838238in}}%
\pgfpathlineto{\pgfqpoint{4.110405in}{1.838238in}}%
\pgfpathlineto{\pgfqpoint{4.110405in}{1.813884in}}%
\pgfpathclose%
\pgfusepath{fill}%
\end{pgfscope}%
\begin{pgfscope}%
\pgfpathrectangle{\pgfqpoint{3.722897in}{0.857143in}}{\pgfqpoint{2.627103in}{1.813434in}}%
\pgfusepath{clip}%
\pgfsetbuttcap%
\pgfsetmiterjoin%
\definecolor{currentfill}{rgb}{0.066899,0.263188,0.377594}%
\pgfsetfillcolor{currentfill}%
\pgfsetlinewidth{0.000000pt}%
\definecolor{currentstroke}{rgb}{0.000000,0.000000,0.000000}%
\pgfsetstrokecolor{currentstroke}%
\pgfsetstrokeopacity{0.000000}%
\pgfsetdash{}{0pt}%
\pgfpathmoveto{\pgfqpoint{4.121576in}{1.813884in}}%
\pgfpathlineto{\pgfqpoint{4.130512in}{1.813884in}}%
\pgfpathlineto{\pgfqpoint{4.130512in}{1.840005in}}%
\pgfpathlineto{\pgfqpoint{4.121576in}{1.840005in}}%
\pgfpathlineto{\pgfqpoint{4.121576in}{1.813884in}}%
\pgfpathclose%
\pgfusepath{fill}%
\end{pgfscope}%
\begin{pgfscope}%
\pgfpathrectangle{\pgfqpoint{3.722897in}{0.857143in}}{\pgfqpoint{2.627103in}{1.813434in}}%
\pgfusepath{clip}%
\pgfsetbuttcap%
\pgfsetmiterjoin%
\definecolor{currentfill}{rgb}{0.066899,0.263188,0.377594}%
\pgfsetfillcolor{currentfill}%
\pgfsetlinewidth{0.000000pt}%
\definecolor{currentstroke}{rgb}{0.000000,0.000000,0.000000}%
\pgfsetstrokecolor{currentstroke}%
\pgfsetstrokeopacity{0.000000}%
\pgfsetdash{}{0pt}%
\pgfpathmoveto{\pgfqpoint{4.132747in}{1.813884in}}%
\pgfpathlineto{\pgfqpoint{4.141683in}{1.813884in}}%
\pgfpathlineto{\pgfqpoint{4.141683in}{1.842605in}}%
\pgfpathlineto{\pgfqpoint{4.132747in}{1.842605in}}%
\pgfpathlineto{\pgfqpoint{4.132747in}{1.813884in}}%
\pgfpathclose%
\pgfusepath{fill}%
\end{pgfscope}%
\begin{pgfscope}%
\pgfpathrectangle{\pgfqpoint{3.722897in}{0.857143in}}{\pgfqpoint{2.627103in}{1.813434in}}%
\pgfusepath{clip}%
\pgfsetbuttcap%
\pgfsetmiterjoin%
\definecolor{currentfill}{rgb}{0.066899,0.263188,0.377594}%
\pgfsetfillcolor{currentfill}%
\pgfsetlinewidth{0.000000pt}%
\definecolor{currentstroke}{rgb}{0.000000,0.000000,0.000000}%
\pgfsetstrokecolor{currentstroke}%
\pgfsetstrokeopacity{0.000000}%
\pgfsetdash{}{0pt}%
\pgfpathmoveto{\pgfqpoint{4.143917in}{1.813884in}}%
\pgfpathlineto{\pgfqpoint{4.152854in}{1.813884in}}%
\pgfpathlineto{\pgfqpoint{4.152854in}{1.843346in}}%
\pgfpathlineto{\pgfqpoint{4.143917in}{1.843346in}}%
\pgfpathlineto{\pgfqpoint{4.143917in}{1.813884in}}%
\pgfpathclose%
\pgfusepath{fill}%
\end{pgfscope}%
\begin{pgfscope}%
\pgfpathrectangle{\pgfqpoint{3.722897in}{0.857143in}}{\pgfqpoint{2.627103in}{1.813434in}}%
\pgfusepath{clip}%
\pgfsetbuttcap%
\pgfsetmiterjoin%
\definecolor{currentfill}{rgb}{0.066899,0.263188,0.377594}%
\pgfsetfillcolor{currentfill}%
\pgfsetlinewidth{0.000000pt}%
\definecolor{currentstroke}{rgb}{0.000000,0.000000,0.000000}%
\pgfsetstrokecolor{currentstroke}%
\pgfsetstrokeopacity{0.000000}%
\pgfsetdash{}{0pt}%
\pgfpathmoveto{\pgfqpoint{4.155088in}{1.813884in}}%
\pgfpathlineto{\pgfqpoint{4.164024in}{1.813884in}}%
\pgfpathlineto{\pgfqpoint{4.164024in}{1.841205in}}%
\pgfpathlineto{\pgfqpoint{4.155088in}{1.841205in}}%
\pgfpathlineto{\pgfqpoint{4.155088in}{1.813884in}}%
\pgfpathclose%
\pgfusepath{fill}%
\end{pgfscope}%
\begin{pgfscope}%
\pgfpathrectangle{\pgfqpoint{3.722897in}{0.857143in}}{\pgfqpoint{2.627103in}{1.813434in}}%
\pgfusepath{clip}%
\pgfsetbuttcap%
\pgfsetmiterjoin%
\definecolor{currentfill}{rgb}{0.066899,0.263188,0.377594}%
\pgfsetfillcolor{currentfill}%
\pgfsetlinewidth{0.000000pt}%
\definecolor{currentstroke}{rgb}{0.000000,0.000000,0.000000}%
\pgfsetstrokecolor{currentstroke}%
\pgfsetstrokeopacity{0.000000}%
\pgfsetdash{}{0pt}%
\pgfpathmoveto{\pgfqpoint{4.166258in}{1.813884in}}%
\pgfpathlineto{\pgfqpoint{4.175195in}{1.813884in}}%
\pgfpathlineto{\pgfqpoint{4.175195in}{1.842051in}}%
\pgfpathlineto{\pgfqpoint{4.166258in}{1.842051in}}%
\pgfpathlineto{\pgfqpoint{4.166258in}{1.813884in}}%
\pgfpathclose%
\pgfusepath{fill}%
\end{pgfscope}%
\begin{pgfscope}%
\pgfpathrectangle{\pgfqpoint{3.722897in}{0.857143in}}{\pgfqpoint{2.627103in}{1.813434in}}%
\pgfusepath{clip}%
\pgfsetbuttcap%
\pgfsetmiterjoin%
\definecolor{currentfill}{rgb}{0.066899,0.263188,0.377594}%
\pgfsetfillcolor{currentfill}%
\pgfsetlinewidth{0.000000pt}%
\definecolor{currentstroke}{rgb}{0.000000,0.000000,0.000000}%
\pgfsetstrokecolor{currentstroke}%
\pgfsetstrokeopacity{0.000000}%
\pgfsetdash{}{0pt}%
\pgfpathmoveto{\pgfqpoint{4.177429in}{1.813884in}}%
\pgfpathlineto{\pgfqpoint{4.186365in}{1.813884in}}%
\pgfpathlineto{\pgfqpoint{4.186365in}{1.840628in}}%
\pgfpathlineto{\pgfqpoint{4.177429in}{1.840628in}}%
\pgfpathlineto{\pgfqpoint{4.177429in}{1.813884in}}%
\pgfpathclose%
\pgfusepath{fill}%
\end{pgfscope}%
\begin{pgfscope}%
\pgfpathrectangle{\pgfqpoint{3.722897in}{0.857143in}}{\pgfqpoint{2.627103in}{1.813434in}}%
\pgfusepath{clip}%
\pgfsetbuttcap%
\pgfsetmiterjoin%
\definecolor{currentfill}{rgb}{0.066899,0.263188,0.377594}%
\pgfsetfillcolor{currentfill}%
\pgfsetlinewidth{0.000000pt}%
\definecolor{currentstroke}{rgb}{0.000000,0.000000,0.000000}%
\pgfsetstrokecolor{currentstroke}%
\pgfsetstrokeopacity{0.000000}%
\pgfsetdash{}{0pt}%
\pgfpathmoveto{\pgfqpoint{4.188600in}{1.813884in}}%
\pgfpathlineto{\pgfqpoint{4.197536in}{1.813884in}}%
\pgfpathlineto{\pgfqpoint{4.197536in}{1.839704in}}%
\pgfpathlineto{\pgfqpoint{4.188600in}{1.839704in}}%
\pgfpathlineto{\pgfqpoint{4.188600in}{1.813884in}}%
\pgfpathclose%
\pgfusepath{fill}%
\end{pgfscope}%
\begin{pgfscope}%
\pgfpathrectangle{\pgfqpoint{3.722897in}{0.857143in}}{\pgfqpoint{2.627103in}{1.813434in}}%
\pgfusepath{clip}%
\pgfsetbuttcap%
\pgfsetmiterjoin%
\definecolor{currentfill}{rgb}{0.066899,0.263188,0.377594}%
\pgfsetfillcolor{currentfill}%
\pgfsetlinewidth{0.000000pt}%
\definecolor{currentstroke}{rgb}{0.000000,0.000000,0.000000}%
\pgfsetstrokecolor{currentstroke}%
\pgfsetstrokeopacity{0.000000}%
\pgfsetdash{}{0pt}%
\pgfpathmoveto{\pgfqpoint{4.199770in}{1.813884in}}%
\pgfpathlineto{\pgfqpoint{4.208707in}{1.813884in}}%
\pgfpathlineto{\pgfqpoint{4.208707in}{1.837166in}}%
\pgfpathlineto{\pgfqpoint{4.199770in}{1.837166in}}%
\pgfpathlineto{\pgfqpoint{4.199770in}{1.813884in}}%
\pgfpathclose%
\pgfusepath{fill}%
\end{pgfscope}%
\begin{pgfscope}%
\pgfpathrectangle{\pgfqpoint{3.722897in}{0.857143in}}{\pgfqpoint{2.627103in}{1.813434in}}%
\pgfusepath{clip}%
\pgfsetbuttcap%
\pgfsetmiterjoin%
\definecolor{currentfill}{rgb}{0.066899,0.263188,0.377594}%
\pgfsetfillcolor{currentfill}%
\pgfsetlinewidth{0.000000pt}%
\definecolor{currentstroke}{rgb}{0.000000,0.000000,0.000000}%
\pgfsetstrokecolor{currentstroke}%
\pgfsetstrokeopacity{0.000000}%
\pgfsetdash{}{0pt}%
\pgfpathmoveto{\pgfqpoint{4.210941in}{1.813884in}}%
\pgfpathlineto{\pgfqpoint{4.219877in}{1.813884in}}%
\pgfpathlineto{\pgfqpoint{4.219877in}{1.838323in}}%
\pgfpathlineto{\pgfqpoint{4.210941in}{1.838323in}}%
\pgfpathlineto{\pgfqpoint{4.210941in}{1.813884in}}%
\pgfpathclose%
\pgfusepath{fill}%
\end{pgfscope}%
\begin{pgfscope}%
\pgfpathrectangle{\pgfqpoint{3.722897in}{0.857143in}}{\pgfqpoint{2.627103in}{1.813434in}}%
\pgfusepath{clip}%
\pgfsetbuttcap%
\pgfsetmiterjoin%
\definecolor{currentfill}{rgb}{0.066899,0.263188,0.377594}%
\pgfsetfillcolor{currentfill}%
\pgfsetlinewidth{0.000000pt}%
\definecolor{currentstroke}{rgb}{0.000000,0.000000,0.000000}%
\pgfsetstrokecolor{currentstroke}%
\pgfsetstrokeopacity{0.000000}%
\pgfsetdash{}{0pt}%
\pgfpathmoveto{\pgfqpoint{4.222111in}{1.813884in}}%
\pgfpathlineto{\pgfqpoint{4.231048in}{1.813884in}}%
\pgfpathlineto{\pgfqpoint{4.231048in}{1.841108in}}%
\pgfpathlineto{\pgfqpoint{4.222111in}{1.841108in}}%
\pgfpathlineto{\pgfqpoint{4.222111in}{1.813884in}}%
\pgfpathclose%
\pgfusepath{fill}%
\end{pgfscope}%
\begin{pgfscope}%
\pgfpathrectangle{\pgfqpoint{3.722897in}{0.857143in}}{\pgfqpoint{2.627103in}{1.813434in}}%
\pgfusepath{clip}%
\pgfsetbuttcap%
\pgfsetmiterjoin%
\definecolor{currentfill}{rgb}{0.066899,0.263188,0.377594}%
\pgfsetfillcolor{currentfill}%
\pgfsetlinewidth{0.000000pt}%
\definecolor{currentstroke}{rgb}{0.000000,0.000000,0.000000}%
\pgfsetstrokecolor{currentstroke}%
\pgfsetstrokeopacity{0.000000}%
\pgfsetdash{}{0pt}%
\pgfpathmoveto{\pgfqpoint{4.233282in}{1.813884in}}%
\pgfpathlineto{\pgfqpoint{4.242218in}{1.813884in}}%
\pgfpathlineto{\pgfqpoint{4.242218in}{1.843898in}}%
\pgfpathlineto{\pgfqpoint{4.233282in}{1.843898in}}%
\pgfpathlineto{\pgfqpoint{4.233282in}{1.813884in}}%
\pgfpathclose%
\pgfusepath{fill}%
\end{pgfscope}%
\begin{pgfscope}%
\pgfpathrectangle{\pgfqpoint{3.722897in}{0.857143in}}{\pgfqpoint{2.627103in}{1.813434in}}%
\pgfusepath{clip}%
\pgfsetbuttcap%
\pgfsetmiterjoin%
\definecolor{currentfill}{rgb}{0.066899,0.263188,0.377594}%
\pgfsetfillcolor{currentfill}%
\pgfsetlinewidth{0.000000pt}%
\definecolor{currentstroke}{rgb}{0.000000,0.000000,0.000000}%
\pgfsetstrokecolor{currentstroke}%
\pgfsetstrokeopacity{0.000000}%
\pgfsetdash{}{0pt}%
\pgfpathmoveto{\pgfqpoint{4.244453in}{1.813884in}}%
\pgfpathlineto{\pgfqpoint{4.253389in}{1.813884in}}%
\pgfpathlineto{\pgfqpoint{4.253389in}{1.845517in}}%
\pgfpathlineto{\pgfqpoint{4.244453in}{1.845517in}}%
\pgfpathlineto{\pgfqpoint{4.244453in}{1.813884in}}%
\pgfpathclose%
\pgfusepath{fill}%
\end{pgfscope}%
\begin{pgfscope}%
\pgfpathrectangle{\pgfqpoint{3.722897in}{0.857143in}}{\pgfqpoint{2.627103in}{1.813434in}}%
\pgfusepath{clip}%
\pgfsetbuttcap%
\pgfsetmiterjoin%
\definecolor{currentfill}{rgb}{0.066899,0.263188,0.377594}%
\pgfsetfillcolor{currentfill}%
\pgfsetlinewidth{0.000000pt}%
\definecolor{currentstroke}{rgb}{0.000000,0.000000,0.000000}%
\pgfsetstrokecolor{currentstroke}%
\pgfsetstrokeopacity{0.000000}%
\pgfsetdash{}{0pt}%
\pgfpathmoveto{\pgfqpoint{4.255623in}{1.813884in}}%
\pgfpathlineto{\pgfqpoint{4.264560in}{1.813884in}}%
\pgfpathlineto{\pgfqpoint{4.264560in}{1.844966in}}%
\pgfpathlineto{\pgfqpoint{4.255623in}{1.844966in}}%
\pgfpathlineto{\pgfqpoint{4.255623in}{1.813884in}}%
\pgfpathclose%
\pgfusepath{fill}%
\end{pgfscope}%
\begin{pgfscope}%
\pgfpathrectangle{\pgfqpoint{3.722897in}{0.857143in}}{\pgfqpoint{2.627103in}{1.813434in}}%
\pgfusepath{clip}%
\pgfsetbuttcap%
\pgfsetmiterjoin%
\definecolor{currentfill}{rgb}{0.066899,0.263188,0.377594}%
\pgfsetfillcolor{currentfill}%
\pgfsetlinewidth{0.000000pt}%
\definecolor{currentstroke}{rgb}{0.000000,0.000000,0.000000}%
\pgfsetstrokecolor{currentstroke}%
\pgfsetstrokeopacity{0.000000}%
\pgfsetdash{}{0pt}%
\pgfpathmoveto{\pgfqpoint{4.266794in}{1.813884in}}%
\pgfpathlineto{\pgfqpoint{4.275730in}{1.813884in}}%
\pgfpathlineto{\pgfqpoint{4.275730in}{1.846359in}}%
\pgfpathlineto{\pgfqpoint{4.266794in}{1.846359in}}%
\pgfpathlineto{\pgfqpoint{4.266794in}{1.813884in}}%
\pgfpathclose%
\pgfusepath{fill}%
\end{pgfscope}%
\begin{pgfscope}%
\pgfpathrectangle{\pgfqpoint{3.722897in}{0.857143in}}{\pgfqpoint{2.627103in}{1.813434in}}%
\pgfusepath{clip}%
\pgfsetbuttcap%
\pgfsetmiterjoin%
\definecolor{currentfill}{rgb}{0.066899,0.263188,0.377594}%
\pgfsetfillcolor{currentfill}%
\pgfsetlinewidth{0.000000pt}%
\definecolor{currentstroke}{rgb}{0.000000,0.000000,0.000000}%
\pgfsetstrokecolor{currentstroke}%
\pgfsetstrokeopacity{0.000000}%
\pgfsetdash{}{0pt}%
\pgfpathmoveto{\pgfqpoint{4.277964in}{1.813884in}}%
\pgfpathlineto{\pgfqpoint{4.286901in}{1.813884in}}%
\pgfpathlineto{\pgfqpoint{4.286901in}{1.847902in}}%
\pgfpathlineto{\pgfqpoint{4.277964in}{1.847902in}}%
\pgfpathlineto{\pgfqpoint{4.277964in}{1.813884in}}%
\pgfpathclose%
\pgfusepath{fill}%
\end{pgfscope}%
\begin{pgfscope}%
\pgfpathrectangle{\pgfqpoint{3.722897in}{0.857143in}}{\pgfqpoint{2.627103in}{1.813434in}}%
\pgfusepath{clip}%
\pgfsetbuttcap%
\pgfsetmiterjoin%
\definecolor{currentfill}{rgb}{0.066899,0.263188,0.377594}%
\pgfsetfillcolor{currentfill}%
\pgfsetlinewidth{0.000000pt}%
\definecolor{currentstroke}{rgb}{0.000000,0.000000,0.000000}%
\pgfsetstrokecolor{currentstroke}%
\pgfsetstrokeopacity{0.000000}%
\pgfsetdash{}{0pt}%
\pgfpathmoveto{\pgfqpoint{4.289135in}{1.813884in}}%
\pgfpathlineto{\pgfqpoint{4.298071in}{1.813884in}}%
\pgfpathlineto{\pgfqpoint{4.298071in}{1.848321in}}%
\pgfpathlineto{\pgfqpoint{4.289135in}{1.848321in}}%
\pgfpathlineto{\pgfqpoint{4.289135in}{1.813884in}}%
\pgfpathclose%
\pgfusepath{fill}%
\end{pgfscope}%
\begin{pgfscope}%
\pgfpathrectangle{\pgfqpoint{3.722897in}{0.857143in}}{\pgfqpoint{2.627103in}{1.813434in}}%
\pgfusepath{clip}%
\pgfsetbuttcap%
\pgfsetmiterjoin%
\definecolor{currentfill}{rgb}{0.066899,0.263188,0.377594}%
\pgfsetfillcolor{currentfill}%
\pgfsetlinewidth{0.000000pt}%
\definecolor{currentstroke}{rgb}{0.000000,0.000000,0.000000}%
\pgfsetstrokecolor{currentstroke}%
\pgfsetstrokeopacity{0.000000}%
\pgfsetdash{}{0pt}%
\pgfpathmoveto{\pgfqpoint{4.300306in}{1.813884in}}%
\pgfpathlineto{\pgfqpoint{4.309242in}{1.813884in}}%
\pgfpathlineto{\pgfqpoint{4.309242in}{1.848725in}}%
\pgfpathlineto{\pgfqpoint{4.300306in}{1.848725in}}%
\pgfpathlineto{\pgfqpoint{4.300306in}{1.813884in}}%
\pgfpathclose%
\pgfusepath{fill}%
\end{pgfscope}%
\begin{pgfscope}%
\pgfpathrectangle{\pgfqpoint{3.722897in}{0.857143in}}{\pgfqpoint{2.627103in}{1.813434in}}%
\pgfusepath{clip}%
\pgfsetbuttcap%
\pgfsetmiterjoin%
\definecolor{currentfill}{rgb}{0.066899,0.263188,0.377594}%
\pgfsetfillcolor{currentfill}%
\pgfsetlinewidth{0.000000pt}%
\definecolor{currentstroke}{rgb}{0.000000,0.000000,0.000000}%
\pgfsetstrokecolor{currentstroke}%
\pgfsetstrokeopacity{0.000000}%
\pgfsetdash{}{0pt}%
\pgfpathmoveto{\pgfqpoint{4.311476in}{1.813884in}}%
\pgfpathlineto{\pgfqpoint{4.320413in}{1.813884in}}%
\pgfpathlineto{\pgfqpoint{4.320413in}{1.848846in}}%
\pgfpathlineto{\pgfqpoint{4.311476in}{1.848846in}}%
\pgfpathlineto{\pgfqpoint{4.311476in}{1.813884in}}%
\pgfpathclose%
\pgfusepath{fill}%
\end{pgfscope}%
\begin{pgfscope}%
\pgfpathrectangle{\pgfqpoint{3.722897in}{0.857143in}}{\pgfqpoint{2.627103in}{1.813434in}}%
\pgfusepath{clip}%
\pgfsetbuttcap%
\pgfsetmiterjoin%
\definecolor{currentfill}{rgb}{0.066899,0.263188,0.377594}%
\pgfsetfillcolor{currentfill}%
\pgfsetlinewidth{0.000000pt}%
\definecolor{currentstroke}{rgb}{0.000000,0.000000,0.000000}%
\pgfsetstrokecolor{currentstroke}%
\pgfsetstrokeopacity{0.000000}%
\pgfsetdash{}{0pt}%
\pgfpathmoveto{\pgfqpoint{4.322647in}{1.813884in}}%
\pgfpathlineto{\pgfqpoint{4.331583in}{1.813884in}}%
\pgfpathlineto{\pgfqpoint{4.331583in}{1.848608in}}%
\pgfpathlineto{\pgfqpoint{4.322647in}{1.848608in}}%
\pgfpathlineto{\pgfqpoint{4.322647in}{1.813884in}}%
\pgfpathclose%
\pgfusepath{fill}%
\end{pgfscope}%
\begin{pgfscope}%
\pgfpathrectangle{\pgfqpoint{3.722897in}{0.857143in}}{\pgfqpoint{2.627103in}{1.813434in}}%
\pgfusepath{clip}%
\pgfsetbuttcap%
\pgfsetmiterjoin%
\definecolor{currentfill}{rgb}{0.066899,0.263188,0.377594}%
\pgfsetfillcolor{currentfill}%
\pgfsetlinewidth{0.000000pt}%
\definecolor{currentstroke}{rgb}{0.000000,0.000000,0.000000}%
\pgfsetstrokecolor{currentstroke}%
\pgfsetstrokeopacity{0.000000}%
\pgfsetdash{}{0pt}%
\pgfpathmoveto{\pgfqpoint{4.333817in}{1.813884in}}%
\pgfpathlineto{\pgfqpoint{4.342754in}{1.813884in}}%
\pgfpathlineto{\pgfqpoint{4.342754in}{1.849655in}}%
\pgfpathlineto{\pgfqpoint{4.333817in}{1.849655in}}%
\pgfpathlineto{\pgfqpoint{4.333817in}{1.813884in}}%
\pgfpathclose%
\pgfusepath{fill}%
\end{pgfscope}%
\begin{pgfscope}%
\pgfpathrectangle{\pgfqpoint{3.722897in}{0.857143in}}{\pgfqpoint{2.627103in}{1.813434in}}%
\pgfusepath{clip}%
\pgfsetbuttcap%
\pgfsetmiterjoin%
\definecolor{currentfill}{rgb}{0.066899,0.263188,0.377594}%
\pgfsetfillcolor{currentfill}%
\pgfsetlinewidth{0.000000pt}%
\definecolor{currentstroke}{rgb}{0.000000,0.000000,0.000000}%
\pgfsetstrokecolor{currentstroke}%
\pgfsetstrokeopacity{0.000000}%
\pgfsetdash{}{0pt}%
\pgfpathmoveto{\pgfqpoint{4.344988in}{1.813884in}}%
\pgfpathlineto{\pgfqpoint{4.353925in}{1.813884in}}%
\pgfpathlineto{\pgfqpoint{4.353925in}{1.846922in}}%
\pgfpathlineto{\pgfqpoint{4.344988in}{1.846922in}}%
\pgfpathlineto{\pgfqpoint{4.344988in}{1.813884in}}%
\pgfpathclose%
\pgfusepath{fill}%
\end{pgfscope}%
\begin{pgfscope}%
\pgfpathrectangle{\pgfqpoint{3.722897in}{0.857143in}}{\pgfqpoint{2.627103in}{1.813434in}}%
\pgfusepath{clip}%
\pgfsetbuttcap%
\pgfsetmiterjoin%
\definecolor{currentfill}{rgb}{0.066899,0.263188,0.377594}%
\pgfsetfillcolor{currentfill}%
\pgfsetlinewidth{0.000000pt}%
\definecolor{currentstroke}{rgb}{0.000000,0.000000,0.000000}%
\pgfsetstrokecolor{currentstroke}%
\pgfsetstrokeopacity{0.000000}%
\pgfsetdash{}{0pt}%
\pgfpathmoveto{\pgfqpoint{4.356159in}{1.813884in}}%
\pgfpathlineto{\pgfqpoint{4.365095in}{1.813884in}}%
\pgfpathlineto{\pgfqpoint{4.365095in}{1.845556in}}%
\pgfpathlineto{\pgfqpoint{4.356159in}{1.845556in}}%
\pgfpathlineto{\pgfqpoint{4.356159in}{1.813884in}}%
\pgfpathclose%
\pgfusepath{fill}%
\end{pgfscope}%
\begin{pgfscope}%
\pgfpathrectangle{\pgfqpoint{3.722897in}{0.857143in}}{\pgfqpoint{2.627103in}{1.813434in}}%
\pgfusepath{clip}%
\pgfsetbuttcap%
\pgfsetmiterjoin%
\definecolor{currentfill}{rgb}{0.066899,0.263188,0.377594}%
\pgfsetfillcolor{currentfill}%
\pgfsetlinewidth{0.000000pt}%
\definecolor{currentstroke}{rgb}{0.000000,0.000000,0.000000}%
\pgfsetstrokecolor{currentstroke}%
\pgfsetstrokeopacity{0.000000}%
\pgfsetdash{}{0pt}%
\pgfpathmoveto{\pgfqpoint{4.367329in}{1.813884in}}%
\pgfpathlineto{\pgfqpoint{4.376266in}{1.813884in}}%
\pgfpathlineto{\pgfqpoint{4.376266in}{1.846172in}}%
\pgfpathlineto{\pgfqpoint{4.367329in}{1.846172in}}%
\pgfpathlineto{\pgfqpoint{4.367329in}{1.813884in}}%
\pgfpathclose%
\pgfusepath{fill}%
\end{pgfscope}%
\begin{pgfscope}%
\pgfpathrectangle{\pgfqpoint{3.722897in}{0.857143in}}{\pgfqpoint{2.627103in}{1.813434in}}%
\pgfusepath{clip}%
\pgfsetbuttcap%
\pgfsetmiterjoin%
\definecolor{currentfill}{rgb}{0.066899,0.263188,0.377594}%
\pgfsetfillcolor{currentfill}%
\pgfsetlinewidth{0.000000pt}%
\definecolor{currentstroke}{rgb}{0.000000,0.000000,0.000000}%
\pgfsetstrokecolor{currentstroke}%
\pgfsetstrokeopacity{0.000000}%
\pgfsetdash{}{0pt}%
\pgfpathmoveto{\pgfqpoint{4.378500in}{1.813884in}}%
\pgfpathlineto{\pgfqpoint{4.387436in}{1.813884in}}%
\pgfpathlineto{\pgfqpoint{4.387436in}{1.845140in}}%
\pgfpathlineto{\pgfqpoint{4.378500in}{1.845140in}}%
\pgfpathlineto{\pgfqpoint{4.378500in}{1.813884in}}%
\pgfpathclose%
\pgfusepath{fill}%
\end{pgfscope}%
\begin{pgfscope}%
\pgfpathrectangle{\pgfqpoint{3.722897in}{0.857143in}}{\pgfqpoint{2.627103in}{1.813434in}}%
\pgfusepath{clip}%
\pgfsetbuttcap%
\pgfsetmiterjoin%
\definecolor{currentfill}{rgb}{0.066899,0.263188,0.377594}%
\pgfsetfillcolor{currentfill}%
\pgfsetlinewidth{0.000000pt}%
\definecolor{currentstroke}{rgb}{0.000000,0.000000,0.000000}%
\pgfsetstrokecolor{currentstroke}%
\pgfsetstrokeopacity{0.000000}%
\pgfsetdash{}{0pt}%
\pgfpathmoveto{\pgfqpoint{4.389670in}{1.813884in}}%
\pgfpathlineto{\pgfqpoint{4.398607in}{1.813884in}}%
\pgfpathlineto{\pgfqpoint{4.398607in}{1.844018in}}%
\pgfpathlineto{\pgfqpoint{4.389670in}{1.844018in}}%
\pgfpathlineto{\pgfqpoint{4.389670in}{1.813884in}}%
\pgfpathclose%
\pgfusepath{fill}%
\end{pgfscope}%
\begin{pgfscope}%
\pgfpathrectangle{\pgfqpoint{3.722897in}{0.857143in}}{\pgfqpoint{2.627103in}{1.813434in}}%
\pgfusepath{clip}%
\pgfsetbuttcap%
\pgfsetmiterjoin%
\definecolor{currentfill}{rgb}{0.066899,0.263188,0.377594}%
\pgfsetfillcolor{currentfill}%
\pgfsetlinewidth{0.000000pt}%
\definecolor{currentstroke}{rgb}{0.000000,0.000000,0.000000}%
\pgfsetstrokecolor{currentstroke}%
\pgfsetstrokeopacity{0.000000}%
\pgfsetdash{}{0pt}%
\pgfpathmoveto{\pgfqpoint{4.400841in}{1.813884in}}%
\pgfpathlineto{\pgfqpoint{4.409778in}{1.813884in}}%
\pgfpathlineto{\pgfqpoint{4.409778in}{1.841287in}}%
\pgfpathlineto{\pgfqpoint{4.400841in}{1.841287in}}%
\pgfpathlineto{\pgfqpoint{4.400841in}{1.813884in}}%
\pgfpathclose%
\pgfusepath{fill}%
\end{pgfscope}%
\begin{pgfscope}%
\pgfpathrectangle{\pgfqpoint{3.722897in}{0.857143in}}{\pgfqpoint{2.627103in}{1.813434in}}%
\pgfusepath{clip}%
\pgfsetbuttcap%
\pgfsetmiterjoin%
\definecolor{currentfill}{rgb}{0.066899,0.263188,0.377594}%
\pgfsetfillcolor{currentfill}%
\pgfsetlinewidth{0.000000pt}%
\definecolor{currentstroke}{rgb}{0.000000,0.000000,0.000000}%
\pgfsetstrokecolor{currentstroke}%
\pgfsetstrokeopacity{0.000000}%
\pgfsetdash{}{0pt}%
\pgfpathmoveto{\pgfqpoint{4.412012in}{1.813884in}}%
\pgfpathlineto{\pgfqpoint{4.420948in}{1.813884in}}%
\pgfpathlineto{\pgfqpoint{4.420948in}{1.838927in}}%
\pgfpathlineto{\pgfqpoint{4.412012in}{1.838927in}}%
\pgfpathlineto{\pgfqpoint{4.412012in}{1.813884in}}%
\pgfpathclose%
\pgfusepath{fill}%
\end{pgfscope}%
\begin{pgfscope}%
\pgfpathrectangle{\pgfqpoint{3.722897in}{0.857143in}}{\pgfqpoint{2.627103in}{1.813434in}}%
\pgfusepath{clip}%
\pgfsetbuttcap%
\pgfsetmiterjoin%
\definecolor{currentfill}{rgb}{0.066899,0.263188,0.377594}%
\pgfsetfillcolor{currentfill}%
\pgfsetlinewidth{0.000000pt}%
\definecolor{currentstroke}{rgb}{0.000000,0.000000,0.000000}%
\pgfsetstrokecolor{currentstroke}%
\pgfsetstrokeopacity{0.000000}%
\pgfsetdash{}{0pt}%
\pgfpathmoveto{\pgfqpoint{4.423182in}{1.813884in}}%
\pgfpathlineto{\pgfqpoint{4.432119in}{1.813884in}}%
\pgfpathlineto{\pgfqpoint{4.432119in}{1.836324in}}%
\pgfpathlineto{\pgfqpoint{4.423182in}{1.836324in}}%
\pgfpathlineto{\pgfqpoint{4.423182in}{1.813884in}}%
\pgfpathclose%
\pgfusepath{fill}%
\end{pgfscope}%
\begin{pgfscope}%
\pgfpathrectangle{\pgfqpoint{3.722897in}{0.857143in}}{\pgfqpoint{2.627103in}{1.813434in}}%
\pgfusepath{clip}%
\pgfsetbuttcap%
\pgfsetmiterjoin%
\definecolor{currentfill}{rgb}{0.066899,0.263188,0.377594}%
\pgfsetfillcolor{currentfill}%
\pgfsetlinewidth{0.000000pt}%
\definecolor{currentstroke}{rgb}{0.000000,0.000000,0.000000}%
\pgfsetstrokecolor{currentstroke}%
\pgfsetstrokeopacity{0.000000}%
\pgfsetdash{}{0pt}%
\pgfpathmoveto{\pgfqpoint{4.434353in}{1.813884in}}%
\pgfpathlineto{\pgfqpoint{4.443289in}{1.813884in}}%
\pgfpathlineto{\pgfqpoint{4.443289in}{1.833856in}}%
\pgfpathlineto{\pgfqpoint{4.434353in}{1.833856in}}%
\pgfpathlineto{\pgfqpoint{4.434353in}{1.813884in}}%
\pgfpathclose%
\pgfusepath{fill}%
\end{pgfscope}%
\begin{pgfscope}%
\pgfpathrectangle{\pgfqpoint{3.722897in}{0.857143in}}{\pgfqpoint{2.627103in}{1.813434in}}%
\pgfusepath{clip}%
\pgfsetbuttcap%
\pgfsetmiterjoin%
\definecolor{currentfill}{rgb}{0.066899,0.263188,0.377594}%
\pgfsetfillcolor{currentfill}%
\pgfsetlinewidth{0.000000pt}%
\definecolor{currentstroke}{rgb}{0.000000,0.000000,0.000000}%
\pgfsetstrokecolor{currentstroke}%
\pgfsetstrokeopacity{0.000000}%
\pgfsetdash{}{0pt}%
\pgfpathmoveto{\pgfqpoint{4.445523in}{1.813884in}}%
\pgfpathlineto{\pgfqpoint{4.454460in}{1.813884in}}%
\pgfpathlineto{\pgfqpoint{4.454460in}{1.832383in}}%
\pgfpathlineto{\pgfqpoint{4.445523in}{1.832383in}}%
\pgfpathlineto{\pgfqpoint{4.445523in}{1.813884in}}%
\pgfpathclose%
\pgfusepath{fill}%
\end{pgfscope}%
\begin{pgfscope}%
\pgfpathrectangle{\pgfqpoint{3.722897in}{0.857143in}}{\pgfqpoint{2.627103in}{1.813434in}}%
\pgfusepath{clip}%
\pgfsetbuttcap%
\pgfsetmiterjoin%
\definecolor{currentfill}{rgb}{0.066899,0.263188,0.377594}%
\pgfsetfillcolor{currentfill}%
\pgfsetlinewidth{0.000000pt}%
\definecolor{currentstroke}{rgb}{0.000000,0.000000,0.000000}%
\pgfsetstrokecolor{currentstroke}%
\pgfsetstrokeopacity{0.000000}%
\pgfsetdash{}{0pt}%
\pgfpathmoveto{\pgfqpoint{4.456694in}{1.813884in}}%
\pgfpathlineto{\pgfqpoint{4.465631in}{1.813884in}}%
\pgfpathlineto{\pgfqpoint{4.465631in}{1.829130in}}%
\pgfpathlineto{\pgfqpoint{4.456694in}{1.829130in}}%
\pgfpathlineto{\pgfqpoint{4.456694in}{1.813884in}}%
\pgfpathclose%
\pgfusepath{fill}%
\end{pgfscope}%
\begin{pgfscope}%
\pgfpathrectangle{\pgfqpoint{3.722897in}{0.857143in}}{\pgfqpoint{2.627103in}{1.813434in}}%
\pgfusepath{clip}%
\pgfsetbuttcap%
\pgfsetmiterjoin%
\definecolor{currentfill}{rgb}{0.066899,0.263188,0.377594}%
\pgfsetfillcolor{currentfill}%
\pgfsetlinewidth{0.000000pt}%
\definecolor{currentstroke}{rgb}{0.000000,0.000000,0.000000}%
\pgfsetstrokecolor{currentstroke}%
\pgfsetstrokeopacity{0.000000}%
\pgfsetdash{}{0pt}%
\pgfpathmoveto{\pgfqpoint{4.467865in}{1.813884in}}%
\pgfpathlineto{\pgfqpoint{4.476801in}{1.813884in}}%
\pgfpathlineto{\pgfqpoint{4.476801in}{1.827047in}}%
\pgfpathlineto{\pgfqpoint{4.467865in}{1.827047in}}%
\pgfpathlineto{\pgfqpoint{4.467865in}{1.813884in}}%
\pgfpathclose%
\pgfusepath{fill}%
\end{pgfscope}%
\begin{pgfscope}%
\pgfpathrectangle{\pgfqpoint{3.722897in}{0.857143in}}{\pgfqpoint{2.627103in}{1.813434in}}%
\pgfusepath{clip}%
\pgfsetbuttcap%
\pgfsetmiterjoin%
\definecolor{currentfill}{rgb}{0.066899,0.263188,0.377594}%
\pgfsetfillcolor{currentfill}%
\pgfsetlinewidth{0.000000pt}%
\definecolor{currentstroke}{rgb}{0.000000,0.000000,0.000000}%
\pgfsetstrokecolor{currentstroke}%
\pgfsetstrokeopacity{0.000000}%
\pgfsetdash{}{0pt}%
\pgfpathmoveto{\pgfqpoint{4.479035in}{1.813884in}}%
\pgfpathlineto{\pgfqpoint{4.487972in}{1.813884in}}%
\pgfpathlineto{\pgfqpoint{4.487972in}{1.825328in}}%
\pgfpathlineto{\pgfqpoint{4.479035in}{1.825328in}}%
\pgfpathlineto{\pgfqpoint{4.479035in}{1.813884in}}%
\pgfpathclose%
\pgfusepath{fill}%
\end{pgfscope}%
\begin{pgfscope}%
\pgfpathrectangle{\pgfqpoint{3.722897in}{0.857143in}}{\pgfqpoint{2.627103in}{1.813434in}}%
\pgfusepath{clip}%
\pgfsetbuttcap%
\pgfsetmiterjoin%
\definecolor{currentfill}{rgb}{0.066899,0.263188,0.377594}%
\pgfsetfillcolor{currentfill}%
\pgfsetlinewidth{0.000000pt}%
\definecolor{currentstroke}{rgb}{0.000000,0.000000,0.000000}%
\pgfsetstrokecolor{currentstroke}%
\pgfsetstrokeopacity{0.000000}%
\pgfsetdash{}{0pt}%
\pgfpathmoveto{\pgfqpoint{4.490206in}{1.813884in}}%
\pgfpathlineto{\pgfqpoint{4.499142in}{1.813884in}}%
\pgfpathlineto{\pgfqpoint{4.499142in}{1.825608in}}%
\pgfpathlineto{\pgfqpoint{4.490206in}{1.825608in}}%
\pgfpathlineto{\pgfqpoint{4.490206in}{1.813884in}}%
\pgfpathclose%
\pgfusepath{fill}%
\end{pgfscope}%
\begin{pgfscope}%
\pgfpathrectangle{\pgfqpoint{3.722897in}{0.857143in}}{\pgfqpoint{2.627103in}{1.813434in}}%
\pgfusepath{clip}%
\pgfsetbuttcap%
\pgfsetmiterjoin%
\definecolor{currentfill}{rgb}{0.066899,0.263188,0.377594}%
\pgfsetfillcolor{currentfill}%
\pgfsetlinewidth{0.000000pt}%
\definecolor{currentstroke}{rgb}{0.000000,0.000000,0.000000}%
\pgfsetstrokecolor{currentstroke}%
\pgfsetstrokeopacity{0.000000}%
\pgfsetdash{}{0pt}%
\pgfpathmoveto{\pgfqpoint{4.501377in}{1.813884in}}%
\pgfpathlineto{\pgfqpoint{4.510313in}{1.813884in}}%
\pgfpathlineto{\pgfqpoint{4.510313in}{1.821458in}}%
\pgfpathlineto{\pgfqpoint{4.501377in}{1.821458in}}%
\pgfpathlineto{\pgfqpoint{4.501377in}{1.813884in}}%
\pgfpathclose%
\pgfusepath{fill}%
\end{pgfscope}%
\begin{pgfscope}%
\pgfpathrectangle{\pgfqpoint{3.722897in}{0.857143in}}{\pgfqpoint{2.627103in}{1.813434in}}%
\pgfusepath{clip}%
\pgfsetbuttcap%
\pgfsetmiterjoin%
\definecolor{currentfill}{rgb}{0.066899,0.263188,0.377594}%
\pgfsetfillcolor{currentfill}%
\pgfsetlinewidth{0.000000pt}%
\definecolor{currentstroke}{rgb}{0.000000,0.000000,0.000000}%
\pgfsetstrokecolor{currentstroke}%
\pgfsetstrokeopacity{0.000000}%
\pgfsetdash{}{0pt}%
\pgfpathmoveto{\pgfqpoint{4.512547in}{1.813884in}}%
\pgfpathlineto{\pgfqpoint{4.521484in}{1.813884in}}%
\pgfpathlineto{\pgfqpoint{4.521484in}{1.821699in}}%
\pgfpathlineto{\pgfqpoint{4.512547in}{1.821699in}}%
\pgfpathlineto{\pgfqpoint{4.512547in}{1.813884in}}%
\pgfpathclose%
\pgfusepath{fill}%
\end{pgfscope}%
\begin{pgfscope}%
\pgfpathrectangle{\pgfqpoint{3.722897in}{0.857143in}}{\pgfqpoint{2.627103in}{1.813434in}}%
\pgfusepath{clip}%
\pgfsetbuttcap%
\pgfsetmiterjoin%
\definecolor{currentfill}{rgb}{0.066899,0.263188,0.377594}%
\pgfsetfillcolor{currentfill}%
\pgfsetlinewidth{0.000000pt}%
\definecolor{currentstroke}{rgb}{0.000000,0.000000,0.000000}%
\pgfsetstrokecolor{currentstroke}%
\pgfsetstrokeopacity{0.000000}%
\pgfsetdash{}{0pt}%
\pgfpathmoveto{\pgfqpoint{4.523718in}{1.813884in}}%
\pgfpathlineto{\pgfqpoint{4.532654in}{1.813884in}}%
\pgfpathlineto{\pgfqpoint{4.532654in}{1.817949in}}%
\pgfpathlineto{\pgfqpoint{4.523718in}{1.817949in}}%
\pgfpathlineto{\pgfqpoint{4.523718in}{1.813884in}}%
\pgfpathclose%
\pgfusepath{fill}%
\end{pgfscope}%
\begin{pgfscope}%
\pgfpathrectangle{\pgfqpoint{3.722897in}{0.857143in}}{\pgfqpoint{2.627103in}{1.813434in}}%
\pgfusepath{clip}%
\pgfsetbuttcap%
\pgfsetmiterjoin%
\definecolor{currentfill}{rgb}{0.066899,0.263188,0.377594}%
\pgfsetfillcolor{currentfill}%
\pgfsetlinewidth{0.000000pt}%
\definecolor{currentstroke}{rgb}{0.000000,0.000000,0.000000}%
\pgfsetstrokecolor{currentstroke}%
\pgfsetstrokeopacity{0.000000}%
\pgfsetdash{}{0pt}%
\pgfpathmoveto{\pgfqpoint{4.534888in}{1.813884in}}%
\pgfpathlineto{\pgfqpoint{4.543825in}{1.813884in}}%
\pgfpathlineto{\pgfqpoint{4.543825in}{1.815731in}}%
\pgfpathlineto{\pgfqpoint{4.534888in}{1.815731in}}%
\pgfpathlineto{\pgfqpoint{4.534888in}{1.813884in}}%
\pgfpathclose%
\pgfusepath{fill}%
\end{pgfscope}%
\begin{pgfscope}%
\pgfpathrectangle{\pgfqpoint{3.722897in}{0.857143in}}{\pgfqpoint{2.627103in}{1.813434in}}%
\pgfusepath{clip}%
\pgfsetbuttcap%
\pgfsetmiterjoin%
\definecolor{currentfill}{rgb}{0.066899,0.263188,0.377594}%
\pgfsetfillcolor{currentfill}%
\pgfsetlinewidth{0.000000pt}%
\definecolor{currentstroke}{rgb}{0.000000,0.000000,0.000000}%
\pgfsetstrokecolor{currentstroke}%
\pgfsetstrokeopacity{0.000000}%
\pgfsetdash{}{0pt}%
\pgfpathmoveto{\pgfqpoint{4.546059in}{1.813884in}}%
\pgfpathlineto{\pgfqpoint{4.554995in}{1.813884in}}%
\pgfpathlineto{\pgfqpoint{4.554995in}{1.813061in}}%
\pgfpathlineto{\pgfqpoint{4.546059in}{1.813061in}}%
\pgfpathlineto{\pgfqpoint{4.546059in}{1.813884in}}%
\pgfpathclose%
\pgfusepath{fill}%
\end{pgfscope}%
\begin{pgfscope}%
\pgfpathrectangle{\pgfqpoint{3.722897in}{0.857143in}}{\pgfqpoint{2.627103in}{1.813434in}}%
\pgfusepath{clip}%
\pgfsetbuttcap%
\pgfsetmiterjoin%
\definecolor{currentfill}{rgb}{0.066899,0.263188,0.377594}%
\pgfsetfillcolor{currentfill}%
\pgfsetlinewidth{0.000000pt}%
\definecolor{currentstroke}{rgb}{0.000000,0.000000,0.000000}%
\pgfsetstrokecolor{currentstroke}%
\pgfsetstrokeopacity{0.000000}%
\pgfsetdash{}{0pt}%
\pgfpathmoveto{\pgfqpoint{4.557230in}{1.813884in}}%
\pgfpathlineto{\pgfqpoint{4.566166in}{1.813884in}}%
\pgfpathlineto{\pgfqpoint{4.566166in}{1.810996in}}%
\pgfpathlineto{\pgfqpoint{4.557230in}{1.810996in}}%
\pgfpathlineto{\pgfqpoint{4.557230in}{1.813884in}}%
\pgfpathclose%
\pgfusepath{fill}%
\end{pgfscope}%
\begin{pgfscope}%
\pgfpathrectangle{\pgfqpoint{3.722897in}{0.857143in}}{\pgfqpoint{2.627103in}{1.813434in}}%
\pgfusepath{clip}%
\pgfsetbuttcap%
\pgfsetmiterjoin%
\definecolor{currentfill}{rgb}{0.066899,0.263188,0.377594}%
\pgfsetfillcolor{currentfill}%
\pgfsetlinewidth{0.000000pt}%
\definecolor{currentstroke}{rgb}{0.000000,0.000000,0.000000}%
\pgfsetstrokecolor{currentstroke}%
\pgfsetstrokeopacity{0.000000}%
\pgfsetdash{}{0pt}%
\pgfpathmoveto{\pgfqpoint{4.568400in}{1.813884in}}%
\pgfpathlineto{\pgfqpoint{4.577337in}{1.813884in}}%
\pgfpathlineto{\pgfqpoint{4.577337in}{1.810353in}}%
\pgfpathlineto{\pgfqpoint{4.568400in}{1.810353in}}%
\pgfpathlineto{\pgfqpoint{4.568400in}{1.813884in}}%
\pgfpathclose%
\pgfusepath{fill}%
\end{pgfscope}%
\begin{pgfscope}%
\pgfpathrectangle{\pgfqpoint{3.722897in}{0.857143in}}{\pgfqpoint{2.627103in}{1.813434in}}%
\pgfusepath{clip}%
\pgfsetbuttcap%
\pgfsetmiterjoin%
\definecolor{currentfill}{rgb}{0.066899,0.263188,0.377594}%
\pgfsetfillcolor{currentfill}%
\pgfsetlinewidth{0.000000pt}%
\definecolor{currentstroke}{rgb}{0.000000,0.000000,0.000000}%
\pgfsetstrokecolor{currentstroke}%
\pgfsetstrokeopacity{0.000000}%
\pgfsetdash{}{0pt}%
\pgfpathmoveto{\pgfqpoint{4.579571in}{1.813884in}}%
\pgfpathlineto{\pgfqpoint{4.588507in}{1.813884in}}%
\pgfpathlineto{\pgfqpoint{4.588507in}{1.809360in}}%
\pgfpathlineto{\pgfqpoint{4.579571in}{1.809360in}}%
\pgfpathlineto{\pgfqpoint{4.579571in}{1.813884in}}%
\pgfpathclose%
\pgfusepath{fill}%
\end{pgfscope}%
\begin{pgfscope}%
\pgfpathrectangle{\pgfqpoint{3.722897in}{0.857143in}}{\pgfqpoint{2.627103in}{1.813434in}}%
\pgfusepath{clip}%
\pgfsetbuttcap%
\pgfsetmiterjoin%
\definecolor{currentfill}{rgb}{0.066899,0.263188,0.377594}%
\pgfsetfillcolor{currentfill}%
\pgfsetlinewidth{0.000000pt}%
\definecolor{currentstroke}{rgb}{0.000000,0.000000,0.000000}%
\pgfsetstrokecolor{currentstroke}%
\pgfsetstrokeopacity{0.000000}%
\pgfsetdash{}{0pt}%
\pgfpathmoveto{\pgfqpoint{4.590741in}{1.813884in}}%
\pgfpathlineto{\pgfqpoint{4.599678in}{1.813884in}}%
\pgfpathlineto{\pgfqpoint{4.599678in}{1.809515in}}%
\pgfpathlineto{\pgfqpoint{4.590741in}{1.809515in}}%
\pgfpathlineto{\pgfqpoint{4.590741in}{1.813884in}}%
\pgfpathclose%
\pgfusepath{fill}%
\end{pgfscope}%
\begin{pgfscope}%
\pgfpathrectangle{\pgfqpoint{3.722897in}{0.857143in}}{\pgfqpoint{2.627103in}{1.813434in}}%
\pgfusepath{clip}%
\pgfsetbuttcap%
\pgfsetmiterjoin%
\definecolor{currentfill}{rgb}{0.066899,0.263188,0.377594}%
\pgfsetfillcolor{currentfill}%
\pgfsetlinewidth{0.000000pt}%
\definecolor{currentstroke}{rgb}{0.000000,0.000000,0.000000}%
\pgfsetstrokecolor{currentstroke}%
\pgfsetstrokeopacity{0.000000}%
\pgfsetdash{}{0pt}%
\pgfpathmoveto{\pgfqpoint{4.601912in}{1.813884in}}%
\pgfpathlineto{\pgfqpoint{4.610848in}{1.813884in}}%
\pgfpathlineto{\pgfqpoint{4.610848in}{1.807622in}}%
\pgfpathlineto{\pgfqpoint{4.601912in}{1.807622in}}%
\pgfpathlineto{\pgfqpoint{4.601912in}{1.813884in}}%
\pgfpathclose%
\pgfusepath{fill}%
\end{pgfscope}%
\begin{pgfscope}%
\pgfpathrectangle{\pgfqpoint{3.722897in}{0.857143in}}{\pgfqpoint{2.627103in}{1.813434in}}%
\pgfusepath{clip}%
\pgfsetbuttcap%
\pgfsetmiterjoin%
\definecolor{currentfill}{rgb}{0.066899,0.263188,0.377594}%
\pgfsetfillcolor{currentfill}%
\pgfsetlinewidth{0.000000pt}%
\definecolor{currentstroke}{rgb}{0.000000,0.000000,0.000000}%
\pgfsetstrokecolor{currentstroke}%
\pgfsetstrokeopacity{0.000000}%
\pgfsetdash{}{0pt}%
\pgfpathmoveto{\pgfqpoint{4.613083in}{1.813884in}}%
\pgfpathlineto{\pgfqpoint{4.622019in}{1.813884in}}%
\pgfpathlineto{\pgfqpoint{4.622019in}{1.806230in}}%
\pgfpathlineto{\pgfqpoint{4.613083in}{1.806230in}}%
\pgfpathlineto{\pgfqpoint{4.613083in}{1.813884in}}%
\pgfpathclose%
\pgfusepath{fill}%
\end{pgfscope}%
\begin{pgfscope}%
\pgfpathrectangle{\pgfqpoint{3.722897in}{0.857143in}}{\pgfqpoint{2.627103in}{1.813434in}}%
\pgfusepath{clip}%
\pgfsetbuttcap%
\pgfsetmiterjoin%
\definecolor{currentfill}{rgb}{0.066899,0.263188,0.377594}%
\pgfsetfillcolor{currentfill}%
\pgfsetlinewidth{0.000000pt}%
\definecolor{currentstroke}{rgb}{0.000000,0.000000,0.000000}%
\pgfsetstrokecolor{currentstroke}%
\pgfsetstrokeopacity{0.000000}%
\pgfsetdash{}{0pt}%
\pgfpathmoveto{\pgfqpoint{4.624253in}{1.813884in}}%
\pgfpathlineto{\pgfqpoint{4.633190in}{1.813884in}}%
\pgfpathlineto{\pgfqpoint{4.633190in}{1.804432in}}%
\pgfpathlineto{\pgfqpoint{4.624253in}{1.804432in}}%
\pgfpathlineto{\pgfqpoint{4.624253in}{1.813884in}}%
\pgfpathclose%
\pgfusepath{fill}%
\end{pgfscope}%
\begin{pgfscope}%
\pgfpathrectangle{\pgfqpoint{3.722897in}{0.857143in}}{\pgfqpoint{2.627103in}{1.813434in}}%
\pgfusepath{clip}%
\pgfsetbuttcap%
\pgfsetmiterjoin%
\definecolor{currentfill}{rgb}{0.066899,0.263188,0.377594}%
\pgfsetfillcolor{currentfill}%
\pgfsetlinewidth{0.000000pt}%
\definecolor{currentstroke}{rgb}{0.000000,0.000000,0.000000}%
\pgfsetstrokecolor{currentstroke}%
\pgfsetstrokeopacity{0.000000}%
\pgfsetdash{}{0pt}%
\pgfpathmoveto{\pgfqpoint{4.635424in}{1.813884in}}%
\pgfpathlineto{\pgfqpoint{4.644360in}{1.813884in}}%
\pgfpathlineto{\pgfqpoint{4.644360in}{1.803371in}}%
\pgfpathlineto{\pgfqpoint{4.635424in}{1.803371in}}%
\pgfpathlineto{\pgfqpoint{4.635424in}{1.813884in}}%
\pgfpathclose%
\pgfusepath{fill}%
\end{pgfscope}%
\begin{pgfscope}%
\pgfpathrectangle{\pgfqpoint{3.722897in}{0.857143in}}{\pgfqpoint{2.627103in}{1.813434in}}%
\pgfusepath{clip}%
\pgfsetbuttcap%
\pgfsetmiterjoin%
\definecolor{currentfill}{rgb}{0.066899,0.263188,0.377594}%
\pgfsetfillcolor{currentfill}%
\pgfsetlinewidth{0.000000pt}%
\definecolor{currentstroke}{rgb}{0.000000,0.000000,0.000000}%
\pgfsetstrokecolor{currentstroke}%
\pgfsetstrokeopacity{0.000000}%
\pgfsetdash{}{0pt}%
\pgfpathmoveto{\pgfqpoint{4.646594in}{1.813884in}}%
\pgfpathlineto{\pgfqpoint{4.655531in}{1.813884in}}%
\pgfpathlineto{\pgfqpoint{4.655531in}{1.802542in}}%
\pgfpathlineto{\pgfqpoint{4.646594in}{1.802542in}}%
\pgfpathlineto{\pgfqpoint{4.646594in}{1.813884in}}%
\pgfpathclose%
\pgfusepath{fill}%
\end{pgfscope}%
\begin{pgfscope}%
\pgfpathrectangle{\pgfqpoint{3.722897in}{0.857143in}}{\pgfqpoint{2.627103in}{1.813434in}}%
\pgfusepath{clip}%
\pgfsetbuttcap%
\pgfsetmiterjoin%
\definecolor{currentfill}{rgb}{0.066899,0.263188,0.377594}%
\pgfsetfillcolor{currentfill}%
\pgfsetlinewidth{0.000000pt}%
\definecolor{currentstroke}{rgb}{0.000000,0.000000,0.000000}%
\pgfsetstrokecolor{currentstroke}%
\pgfsetstrokeopacity{0.000000}%
\pgfsetdash{}{0pt}%
\pgfpathmoveto{\pgfqpoint{4.657765in}{1.813884in}}%
\pgfpathlineto{\pgfqpoint{4.666701in}{1.813884in}}%
\pgfpathlineto{\pgfqpoint{4.666701in}{1.800992in}}%
\pgfpathlineto{\pgfqpoint{4.657765in}{1.800992in}}%
\pgfpathlineto{\pgfqpoint{4.657765in}{1.813884in}}%
\pgfpathclose%
\pgfusepath{fill}%
\end{pgfscope}%
\begin{pgfscope}%
\pgfpathrectangle{\pgfqpoint{3.722897in}{0.857143in}}{\pgfqpoint{2.627103in}{1.813434in}}%
\pgfusepath{clip}%
\pgfsetbuttcap%
\pgfsetmiterjoin%
\definecolor{currentfill}{rgb}{0.066899,0.263188,0.377594}%
\pgfsetfillcolor{currentfill}%
\pgfsetlinewidth{0.000000pt}%
\definecolor{currentstroke}{rgb}{0.000000,0.000000,0.000000}%
\pgfsetstrokecolor{currentstroke}%
\pgfsetstrokeopacity{0.000000}%
\pgfsetdash{}{0pt}%
\pgfpathmoveto{\pgfqpoint{4.668936in}{1.813884in}}%
\pgfpathlineto{\pgfqpoint{4.677872in}{1.813884in}}%
\pgfpathlineto{\pgfqpoint{4.677872in}{1.799422in}}%
\pgfpathlineto{\pgfqpoint{4.668936in}{1.799422in}}%
\pgfpathlineto{\pgfqpoint{4.668936in}{1.813884in}}%
\pgfpathclose%
\pgfusepath{fill}%
\end{pgfscope}%
\begin{pgfscope}%
\pgfpathrectangle{\pgfqpoint{3.722897in}{0.857143in}}{\pgfqpoint{2.627103in}{1.813434in}}%
\pgfusepath{clip}%
\pgfsetbuttcap%
\pgfsetmiterjoin%
\definecolor{currentfill}{rgb}{0.066899,0.263188,0.377594}%
\pgfsetfillcolor{currentfill}%
\pgfsetlinewidth{0.000000pt}%
\definecolor{currentstroke}{rgb}{0.000000,0.000000,0.000000}%
\pgfsetstrokecolor{currentstroke}%
\pgfsetstrokeopacity{0.000000}%
\pgfsetdash{}{0pt}%
\pgfpathmoveto{\pgfqpoint{4.680106in}{1.813884in}}%
\pgfpathlineto{\pgfqpoint{4.689043in}{1.813884in}}%
\pgfpathlineto{\pgfqpoint{4.689043in}{1.797856in}}%
\pgfpathlineto{\pgfqpoint{4.680106in}{1.797856in}}%
\pgfpathlineto{\pgfqpoint{4.680106in}{1.813884in}}%
\pgfpathclose%
\pgfusepath{fill}%
\end{pgfscope}%
\begin{pgfscope}%
\pgfpathrectangle{\pgfqpoint{3.722897in}{0.857143in}}{\pgfqpoint{2.627103in}{1.813434in}}%
\pgfusepath{clip}%
\pgfsetbuttcap%
\pgfsetmiterjoin%
\definecolor{currentfill}{rgb}{0.066899,0.263188,0.377594}%
\pgfsetfillcolor{currentfill}%
\pgfsetlinewidth{0.000000pt}%
\definecolor{currentstroke}{rgb}{0.000000,0.000000,0.000000}%
\pgfsetstrokecolor{currentstroke}%
\pgfsetstrokeopacity{0.000000}%
\pgfsetdash{}{0pt}%
\pgfpathmoveto{\pgfqpoint{4.691277in}{1.813884in}}%
\pgfpathlineto{\pgfqpoint{4.700213in}{1.813884in}}%
\pgfpathlineto{\pgfqpoint{4.700213in}{1.798543in}}%
\pgfpathlineto{\pgfqpoint{4.691277in}{1.798543in}}%
\pgfpathlineto{\pgfqpoint{4.691277in}{1.813884in}}%
\pgfpathclose%
\pgfusepath{fill}%
\end{pgfscope}%
\begin{pgfscope}%
\pgfpathrectangle{\pgfqpoint{3.722897in}{0.857143in}}{\pgfqpoint{2.627103in}{1.813434in}}%
\pgfusepath{clip}%
\pgfsetbuttcap%
\pgfsetmiterjoin%
\definecolor{currentfill}{rgb}{0.066899,0.263188,0.377594}%
\pgfsetfillcolor{currentfill}%
\pgfsetlinewidth{0.000000pt}%
\definecolor{currentstroke}{rgb}{0.000000,0.000000,0.000000}%
\pgfsetstrokecolor{currentstroke}%
\pgfsetstrokeopacity{0.000000}%
\pgfsetdash{}{0pt}%
\pgfpathmoveto{\pgfqpoint{4.702447in}{1.813884in}}%
\pgfpathlineto{\pgfqpoint{4.711384in}{1.813884in}}%
\pgfpathlineto{\pgfqpoint{4.711384in}{1.797328in}}%
\pgfpathlineto{\pgfqpoint{4.702447in}{1.797328in}}%
\pgfpathlineto{\pgfqpoint{4.702447in}{1.813884in}}%
\pgfpathclose%
\pgfusepath{fill}%
\end{pgfscope}%
\begin{pgfscope}%
\pgfpathrectangle{\pgfqpoint{3.722897in}{0.857143in}}{\pgfqpoint{2.627103in}{1.813434in}}%
\pgfusepath{clip}%
\pgfsetbuttcap%
\pgfsetmiterjoin%
\definecolor{currentfill}{rgb}{0.066899,0.263188,0.377594}%
\pgfsetfillcolor{currentfill}%
\pgfsetlinewidth{0.000000pt}%
\definecolor{currentstroke}{rgb}{0.000000,0.000000,0.000000}%
\pgfsetstrokecolor{currentstroke}%
\pgfsetstrokeopacity{0.000000}%
\pgfsetdash{}{0pt}%
\pgfpathmoveto{\pgfqpoint{4.713618in}{1.813884in}}%
\pgfpathlineto{\pgfqpoint{4.722554in}{1.813884in}}%
\pgfpathlineto{\pgfqpoint{4.722554in}{1.798070in}}%
\pgfpathlineto{\pgfqpoint{4.713618in}{1.798070in}}%
\pgfpathlineto{\pgfqpoint{4.713618in}{1.813884in}}%
\pgfpathclose%
\pgfusepath{fill}%
\end{pgfscope}%
\begin{pgfscope}%
\pgfpathrectangle{\pgfqpoint{3.722897in}{0.857143in}}{\pgfqpoint{2.627103in}{1.813434in}}%
\pgfusepath{clip}%
\pgfsetbuttcap%
\pgfsetmiterjoin%
\definecolor{currentfill}{rgb}{0.066899,0.263188,0.377594}%
\pgfsetfillcolor{currentfill}%
\pgfsetlinewidth{0.000000pt}%
\definecolor{currentstroke}{rgb}{0.000000,0.000000,0.000000}%
\pgfsetstrokecolor{currentstroke}%
\pgfsetstrokeopacity{0.000000}%
\pgfsetdash{}{0pt}%
\pgfpathmoveto{\pgfqpoint{4.724789in}{1.813884in}}%
\pgfpathlineto{\pgfqpoint{4.733725in}{1.813884in}}%
\pgfpathlineto{\pgfqpoint{4.733725in}{1.798327in}}%
\pgfpathlineto{\pgfqpoint{4.724789in}{1.798327in}}%
\pgfpathlineto{\pgfqpoint{4.724789in}{1.813884in}}%
\pgfpathclose%
\pgfusepath{fill}%
\end{pgfscope}%
\begin{pgfscope}%
\pgfpathrectangle{\pgfqpoint{3.722897in}{0.857143in}}{\pgfqpoint{2.627103in}{1.813434in}}%
\pgfusepath{clip}%
\pgfsetbuttcap%
\pgfsetmiterjoin%
\definecolor{currentfill}{rgb}{0.066899,0.263188,0.377594}%
\pgfsetfillcolor{currentfill}%
\pgfsetlinewidth{0.000000pt}%
\definecolor{currentstroke}{rgb}{0.000000,0.000000,0.000000}%
\pgfsetstrokecolor{currentstroke}%
\pgfsetstrokeopacity{0.000000}%
\pgfsetdash{}{0pt}%
\pgfpathmoveto{\pgfqpoint{4.735959in}{1.813884in}}%
\pgfpathlineto{\pgfqpoint{4.744896in}{1.813884in}}%
\pgfpathlineto{\pgfqpoint{4.744896in}{1.798227in}}%
\pgfpathlineto{\pgfqpoint{4.735959in}{1.798227in}}%
\pgfpathlineto{\pgfqpoint{4.735959in}{1.813884in}}%
\pgfpathclose%
\pgfusepath{fill}%
\end{pgfscope}%
\begin{pgfscope}%
\pgfpathrectangle{\pgfqpoint{3.722897in}{0.857143in}}{\pgfqpoint{2.627103in}{1.813434in}}%
\pgfusepath{clip}%
\pgfsetbuttcap%
\pgfsetmiterjoin%
\definecolor{currentfill}{rgb}{0.066899,0.263188,0.377594}%
\pgfsetfillcolor{currentfill}%
\pgfsetlinewidth{0.000000pt}%
\definecolor{currentstroke}{rgb}{0.000000,0.000000,0.000000}%
\pgfsetstrokecolor{currentstroke}%
\pgfsetstrokeopacity{0.000000}%
\pgfsetdash{}{0pt}%
\pgfpathmoveto{\pgfqpoint{4.747130in}{1.813884in}}%
\pgfpathlineto{\pgfqpoint{4.756066in}{1.813884in}}%
\pgfpathlineto{\pgfqpoint{4.756066in}{1.796757in}}%
\pgfpathlineto{\pgfqpoint{4.747130in}{1.796757in}}%
\pgfpathlineto{\pgfqpoint{4.747130in}{1.813884in}}%
\pgfpathclose%
\pgfusepath{fill}%
\end{pgfscope}%
\begin{pgfscope}%
\pgfpathrectangle{\pgfqpoint{3.722897in}{0.857143in}}{\pgfqpoint{2.627103in}{1.813434in}}%
\pgfusepath{clip}%
\pgfsetbuttcap%
\pgfsetmiterjoin%
\definecolor{currentfill}{rgb}{0.066899,0.263188,0.377594}%
\pgfsetfillcolor{currentfill}%
\pgfsetlinewidth{0.000000pt}%
\definecolor{currentstroke}{rgb}{0.000000,0.000000,0.000000}%
\pgfsetstrokecolor{currentstroke}%
\pgfsetstrokeopacity{0.000000}%
\pgfsetdash{}{0pt}%
\pgfpathmoveto{\pgfqpoint{4.758300in}{1.813884in}}%
\pgfpathlineto{\pgfqpoint{4.767237in}{1.813884in}}%
\pgfpathlineto{\pgfqpoint{4.767237in}{1.794472in}}%
\pgfpathlineto{\pgfqpoint{4.758300in}{1.794472in}}%
\pgfpathlineto{\pgfqpoint{4.758300in}{1.813884in}}%
\pgfpathclose%
\pgfusepath{fill}%
\end{pgfscope}%
\begin{pgfscope}%
\pgfpathrectangle{\pgfqpoint{3.722897in}{0.857143in}}{\pgfqpoint{2.627103in}{1.813434in}}%
\pgfusepath{clip}%
\pgfsetbuttcap%
\pgfsetmiterjoin%
\definecolor{currentfill}{rgb}{0.066899,0.263188,0.377594}%
\pgfsetfillcolor{currentfill}%
\pgfsetlinewidth{0.000000pt}%
\definecolor{currentstroke}{rgb}{0.000000,0.000000,0.000000}%
\pgfsetstrokecolor{currentstroke}%
\pgfsetstrokeopacity{0.000000}%
\pgfsetdash{}{0pt}%
\pgfpathmoveto{\pgfqpoint{4.769471in}{1.813884in}}%
\pgfpathlineto{\pgfqpoint{4.778408in}{1.813884in}}%
\pgfpathlineto{\pgfqpoint{4.778408in}{1.794028in}}%
\pgfpathlineto{\pgfqpoint{4.769471in}{1.794028in}}%
\pgfpathlineto{\pgfqpoint{4.769471in}{1.813884in}}%
\pgfpathclose%
\pgfusepath{fill}%
\end{pgfscope}%
\begin{pgfscope}%
\pgfpathrectangle{\pgfqpoint{3.722897in}{0.857143in}}{\pgfqpoint{2.627103in}{1.813434in}}%
\pgfusepath{clip}%
\pgfsetbuttcap%
\pgfsetmiterjoin%
\definecolor{currentfill}{rgb}{0.066899,0.263188,0.377594}%
\pgfsetfillcolor{currentfill}%
\pgfsetlinewidth{0.000000pt}%
\definecolor{currentstroke}{rgb}{0.000000,0.000000,0.000000}%
\pgfsetstrokecolor{currentstroke}%
\pgfsetstrokeopacity{0.000000}%
\pgfsetdash{}{0pt}%
\pgfpathmoveto{\pgfqpoint{4.780642in}{1.813884in}}%
\pgfpathlineto{\pgfqpoint{4.789578in}{1.813884in}}%
\pgfpathlineto{\pgfqpoint{4.789578in}{1.792744in}}%
\pgfpathlineto{\pgfqpoint{4.780642in}{1.792744in}}%
\pgfpathlineto{\pgfqpoint{4.780642in}{1.813884in}}%
\pgfpathclose%
\pgfusepath{fill}%
\end{pgfscope}%
\begin{pgfscope}%
\pgfpathrectangle{\pgfqpoint{3.722897in}{0.857143in}}{\pgfqpoint{2.627103in}{1.813434in}}%
\pgfusepath{clip}%
\pgfsetbuttcap%
\pgfsetmiterjoin%
\definecolor{currentfill}{rgb}{0.066899,0.263188,0.377594}%
\pgfsetfillcolor{currentfill}%
\pgfsetlinewidth{0.000000pt}%
\definecolor{currentstroke}{rgb}{0.000000,0.000000,0.000000}%
\pgfsetstrokecolor{currentstroke}%
\pgfsetstrokeopacity{0.000000}%
\pgfsetdash{}{0pt}%
\pgfpathmoveto{\pgfqpoint{4.791812in}{1.813884in}}%
\pgfpathlineto{\pgfqpoint{4.800749in}{1.813884in}}%
\pgfpathlineto{\pgfqpoint{4.800749in}{1.792494in}}%
\pgfpathlineto{\pgfqpoint{4.791812in}{1.792494in}}%
\pgfpathlineto{\pgfqpoint{4.791812in}{1.813884in}}%
\pgfpathclose%
\pgfusepath{fill}%
\end{pgfscope}%
\begin{pgfscope}%
\pgfpathrectangle{\pgfqpoint{3.722897in}{0.857143in}}{\pgfqpoint{2.627103in}{1.813434in}}%
\pgfusepath{clip}%
\pgfsetbuttcap%
\pgfsetmiterjoin%
\definecolor{currentfill}{rgb}{0.066899,0.263188,0.377594}%
\pgfsetfillcolor{currentfill}%
\pgfsetlinewidth{0.000000pt}%
\definecolor{currentstroke}{rgb}{0.000000,0.000000,0.000000}%
\pgfsetstrokecolor{currentstroke}%
\pgfsetstrokeopacity{0.000000}%
\pgfsetdash{}{0pt}%
\pgfpathmoveto{\pgfqpoint{4.802983in}{1.813884in}}%
\pgfpathlineto{\pgfqpoint{4.811919in}{1.813884in}}%
\pgfpathlineto{\pgfqpoint{4.811919in}{1.791778in}}%
\pgfpathlineto{\pgfqpoint{4.802983in}{1.791778in}}%
\pgfpathlineto{\pgfqpoint{4.802983in}{1.813884in}}%
\pgfpathclose%
\pgfusepath{fill}%
\end{pgfscope}%
\begin{pgfscope}%
\pgfpathrectangle{\pgfqpoint{3.722897in}{0.857143in}}{\pgfqpoint{2.627103in}{1.813434in}}%
\pgfusepath{clip}%
\pgfsetbuttcap%
\pgfsetmiterjoin%
\definecolor{currentfill}{rgb}{0.066899,0.263188,0.377594}%
\pgfsetfillcolor{currentfill}%
\pgfsetlinewidth{0.000000pt}%
\definecolor{currentstroke}{rgb}{0.000000,0.000000,0.000000}%
\pgfsetstrokecolor{currentstroke}%
\pgfsetstrokeopacity{0.000000}%
\pgfsetdash{}{0pt}%
\pgfpathmoveto{\pgfqpoint{4.814153in}{1.813884in}}%
\pgfpathlineto{\pgfqpoint{4.823090in}{1.813884in}}%
\pgfpathlineto{\pgfqpoint{4.823090in}{1.789980in}}%
\pgfpathlineto{\pgfqpoint{4.814153in}{1.789980in}}%
\pgfpathlineto{\pgfqpoint{4.814153in}{1.813884in}}%
\pgfpathclose%
\pgfusepath{fill}%
\end{pgfscope}%
\begin{pgfscope}%
\pgfpathrectangle{\pgfqpoint{3.722897in}{0.857143in}}{\pgfqpoint{2.627103in}{1.813434in}}%
\pgfusepath{clip}%
\pgfsetbuttcap%
\pgfsetmiterjoin%
\definecolor{currentfill}{rgb}{0.066899,0.263188,0.377594}%
\pgfsetfillcolor{currentfill}%
\pgfsetlinewidth{0.000000pt}%
\definecolor{currentstroke}{rgb}{0.000000,0.000000,0.000000}%
\pgfsetstrokecolor{currentstroke}%
\pgfsetstrokeopacity{0.000000}%
\pgfsetdash{}{0pt}%
\pgfpathmoveto{\pgfqpoint{4.825324in}{1.813884in}}%
\pgfpathlineto{\pgfqpoint{4.834261in}{1.813884in}}%
\pgfpathlineto{\pgfqpoint{4.834261in}{1.788960in}}%
\pgfpathlineto{\pgfqpoint{4.825324in}{1.788960in}}%
\pgfpathlineto{\pgfqpoint{4.825324in}{1.813884in}}%
\pgfpathclose%
\pgfusepath{fill}%
\end{pgfscope}%
\begin{pgfscope}%
\pgfpathrectangle{\pgfqpoint{3.722897in}{0.857143in}}{\pgfqpoint{2.627103in}{1.813434in}}%
\pgfusepath{clip}%
\pgfsetbuttcap%
\pgfsetmiterjoin%
\definecolor{currentfill}{rgb}{0.066899,0.263188,0.377594}%
\pgfsetfillcolor{currentfill}%
\pgfsetlinewidth{0.000000pt}%
\definecolor{currentstroke}{rgb}{0.000000,0.000000,0.000000}%
\pgfsetstrokecolor{currentstroke}%
\pgfsetstrokeopacity{0.000000}%
\pgfsetdash{}{0pt}%
\pgfpathmoveto{\pgfqpoint{4.836495in}{1.813884in}}%
\pgfpathlineto{\pgfqpoint{4.845431in}{1.813884in}}%
\pgfpathlineto{\pgfqpoint{4.845431in}{1.787638in}}%
\pgfpathlineto{\pgfqpoint{4.836495in}{1.787638in}}%
\pgfpathlineto{\pgfqpoint{4.836495in}{1.813884in}}%
\pgfpathclose%
\pgfusepath{fill}%
\end{pgfscope}%
\begin{pgfscope}%
\pgfpathrectangle{\pgfqpoint{3.722897in}{0.857143in}}{\pgfqpoint{2.627103in}{1.813434in}}%
\pgfusepath{clip}%
\pgfsetbuttcap%
\pgfsetmiterjoin%
\definecolor{currentfill}{rgb}{0.066899,0.263188,0.377594}%
\pgfsetfillcolor{currentfill}%
\pgfsetlinewidth{0.000000pt}%
\definecolor{currentstroke}{rgb}{0.000000,0.000000,0.000000}%
\pgfsetstrokecolor{currentstroke}%
\pgfsetstrokeopacity{0.000000}%
\pgfsetdash{}{0pt}%
\pgfpathmoveto{\pgfqpoint{4.847665in}{1.813884in}}%
\pgfpathlineto{\pgfqpoint{4.856602in}{1.813884in}}%
\pgfpathlineto{\pgfqpoint{4.856602in}{1.785822in}}%
\pgfpathlineto{\pgfqpoint{4.847665in}{1.785822in}}%
\pgfpathlineto{\pgfqpoint{4.847665in}{1.813884in}}%
\pgfpathclose%
\pgfusepath{fill}%
\end{pgfscope}%
\begin{pgfscope}%
\pgfpathrectangle{\pgfqpoint{3.722897in}{0.857143in}}{\pgfqpoint{2.627103in}{1.813434in}}%
\pgfusepath{clip}%
\pgfsetbuttcap%
\pgfsetmiterjoin%
\definecolor{currentfill}{rgb}{0.066899,0.263188,0.377594}%
\pgfsetfillcolor{currentfill}%
\pgfsetlinewidth{0.000000pt}%
\definecolor{currentstroke}{rgb}{0.000000,0.000000,0.000000}%
\pgfsetstrokecolor{currentstroke}%
\pgfsetstrokeopacity{0.000000}%
\pgfsetdash{}{0pt}%
\pgfpathmoveto{\pgfqpoint{4.858836in}{1.813884in}}%
\pgfpathlineto{\pgfqpoint{4.867772in}{1.813884in}}%
\pgfpathlineto{\pgfqpoint{4.867772in}{1.784672in}}%
\pgfpathlineto{\pgfqpoint{4.858836in}{1.784672in}}%
\pgfpathlineto{\pgfqpoint{4.858836in}{1.813884in}}%
\pgfpathclose%
\pgfusepath{fill}%
\end{pgfscope}%
\begin{pgfscope}%
\pgfpathrectangle{\pgfqpoint{3.722897in}{0.857143in}}{\pgfqpoint{2.627103in}{1.813434in}}%
\pgfusepath{clip}%
\pgfsetbuttcap%
\pgfsetmiterjoin%
\definecolor{currentfill}{rgb}{0.066899,0.263188,0.377594}%
\pgfsetfillcolor{currentfill}%
\pgfsetlinewidth{0.000000pt}%
\definecolor{currentstroke}{rgb}{0.000000,0.000000,0.000000}%
\pgfsetstrokecolor{currentstroke}%
\pgfsetstrokeopacity{0.000000}%
\pgfsetdash{}{0pt}%
\pgfpathmoveto{\pgfqpoint{4.870006in}{1.813884in}}%
\pgfpathlineto{\pgfqpoint{4.878943in}{1.813884in}}%
\pgfpathlineto{\pgfqpoint{4.878943in}{1.783801in}}%
\pgfpathlineto{\pgfqpoint{4.870006in}{1.783801in}}%
\pgfpathlineto{\pgfqpoint{4.870006in}{1.813884in}}%
\pgfpathclose%
\pgfusepath{fill}%
\end{pgfscope}%
\begin{pgfscope}%
\pgfpathrectangle{\pgfqpoint{3.722897in}{0.857143in}}{\pgfqpoint{2.627103in}{1.813434in}}%
\pgfusepath{clip}%
\pgfsetbuttcap%
\pgfsetmiterjoin%
\definecolor{currentfill}{rgb}{0.066899,0.263188,0.377594}%
\pgfsetfillcolor{currentfill}%
\pgfsetlinewidth{0.000000pt}%
\definecolor{currentstroke}{rgb}{0.000000,0.000000,0.000000}%
\pgfsetstrokecolor{currentstroke}%
\pgfsetstrokeopacity{0.000000}%
\pgfsetdash{}{0pt}%
\pgfpathmoveto{\pgfqpoint{4.881177in}{1.813884in}}%
\pgfpathlineto{\pgfqpoint{4.890114in}{1.813884in}}%
\pgfpathlineto{\pgfqpoint{4.890114in}{1.782282in}}%
\pgfpathlineto{\pgfqpoint{4.881177in}{1.782282in}}%
\pgfpathlineto{\pgfqpoint{4.881177in}{1.813884in}}%
\pgfpathclose%
\pgfusepath{fill}%
\end{pgfscope}%
\begin{pgfscope}%
\pgfpathrectangle{\pgfqpoint{3.722897in}{0.857143in}}{\pgfqpoint{2.627103in}{1.813434in}}%
\pgfusepath{clip}%
\pgfsetbuttcap%
\pgfsetmiterjoin%
\definecolor{currentfill}{rgb}{0.066899,0.263188,0.377594}%
\pgfsetfillcolor{currentfill}%
\pgfsetlinewidth{0.000000pt}%
\definecolor{currentstroke}{rgb}{0.000000,0.000000,0.000000}%
\pgfsetstrokecolor{currentstroke}%
\pgfsetstrokeopacity{0.000000}%
\pgfsetdash{}{0pt}%
\pgfpathmoveto{\pgfqpoint{4.892348in}{1.813884in}}%
\pgfpathlineto{\pgfqpoint{4.901284in}{1.813884in}}%
\pgfpathlineto{\pgfqpoint{4.901284in}{1.782091in}}%
\pgfpathlineto{\pgfqpoint{4.892348in}{1.782091in}}%
\pgfpathlineto{\pgfqpoint{4.892348in}{1.813884in}}%
\pgfpathclose%
\pgfusepath{fill}%
\end{pgfscope}%
\begin{pgfscope}%
\pgfpathrectangle{\pgfqpoint{3.722897in}{0.857143in}}{\pgfqpoint{2.627103in}{1.813434in}}%
\pgfusepath{clip}%
\pgfsetbuttcap%
\pgfsetmiterjoin%
\definecolor{currentfill}{rgb}{0.066899,0.263188,0.377594}%
\pgfsetfillcolor{currentfill}%
\pgfsetlinewidth{0.000000pt}%
\definecolor{currentstroke}{rgb}{0.000000,0.000000,0.000000}%
\pgfsetstrokecolor{currentstroke}%
\pgfsetstrokeopacity{0.000000}%
\pgfsetdash{}{0pt}%
\pgfpathmoveto{\pgfqpoint{4.903518in}{1.813884in}}%
\pgfpathlineto{\pgfqpoint{4.912455in}{1.813884in}}%
\pgfpathlineto{\pgfqpoint{4.912455in}{1.782081in}}%
\pgfpathlineto{\pgfqpoint{4.903518in}{1.782081in}}%
\pgfpathlineto{\pgfqpoint{4.903518in}{1.813884in}}%
\pgfpathclose%
\pgfusepath{fill}%
\end{pgfscope}%
\begin{pgfscope}%
\pgfpathrectangle{\pgfqpoint{3.722897in}{0.857143in}}{\pgfqpoint{2.627103in}{1.813434in}}%
\pgfusepath{clip}%
\pgfsetbuttcap%
\pgfsetmiterjoin%
\definecolor{currentfill}{rgb}{0.066899,0.263188,0.377594}%
\pgfsetfillcolor{currentfill}%
\pgfsetlinewidth{0.000000pt}%
\definecolor{currentstroke}{rgb}{0.000000,0.000000,0.000000}%
\pgfsetstrokecolor{currentstroke}%
\pgfsetstrokeopacity{0.000000}%
\pgfsetdash{}{0pt}%
\pgfpathmoveto{\pgfqpoint{4.914689in}{1.813884in}}%
\pgfpathlineto{\pgfqpoint{4.923625in}{1.813884in}}%
\pgfpathlineto{\pgfqpoint{4.923625in}{1.781822in}}%
\pgfpathlineto{\pgfqpoint{4.914689in}{1.781822in}}%
\pgfpathlineto{\pgfqpoint{4.914689in}{1.813884in}}%
\pgfpathclose%
\pgfusepath{fill}%
\end{pgfscope}%
\begin{pgfscope}%
\pgfpathrectangle{\pgfqpoint{3.722897in}{0.857143in}}{\pgfqpoint{2.627103in}{1.813434in}}%
\pgfusepath{clip}%
\pgfsetbuttcap%
\pgfsetmiterjoin%
\definecolor{currentfill}{rgb}{0.066899,0.263188,0.377594}%
\pgfsetfillcolor{currentfill}%
\pgfsetlinewidth{0.000000pt}%
\definecolor{currentstroke}{rgb}{0.000000,0.000000,0.000000}%
\pgfsetstrokecolor{currentstroke}%
\pgfsetstrokeopacity{0.000000}%
\pgfsetdash{}{0pt}%
\pgfpathmoveto{\pgfqpoint{4.925860in}{1.813884in}}%
\pgfpathlineto{\pgfqpoint{4.934796in}{1.813884in}}%
\pgfpathlineto{\pgfqpoint{4.934796in}{1.779363in}}%
\pgfpathlineto{\pgfqpoint{4.925860in}{1.779363in}}%
\pgfpathlineto{\pgfqpoint{4.925860in}{1.813884in}}%
\pgfpathclose%
\pgfusepath{fill}%
\end{pgfscope}%
\begin{pgfscope}%
\pgfpathrectangle{\pgfqpoint{3.722897in}{0.857143in}}{\pgfqpoint{2.627103in}{1.813434in}}%
\pgfusepath{clip}%
\pgfsetbuttcap%
\pgfsetmiterjoin%
\definecolor{currentfill}{rgb}{0.066899,0.263188,0.377594}%
\pgfsetfillcolor{currentfill}%
\pgfsetlinewidth{0.000000pt}%
\definecolor{currentstroke}{rgb}{0.000000,0.000000,0.000000}%
\pgfsetstrokecolor{currentstroke}%
\pgfsetstrokeopacity{0.000000}%
\pgfsetdash{}{0pt}%
\pgfpathmoveto{\pgfqpoint{4.937030in}{1.813884in}}%
\pgfpathlineto{\pgfqpoint{4.945967in}{1.813884in}}%
\pgfpathlineto{\pgfqpoint{4.945967in}{1.778897in}}%
\pgfpathlineto{\pgfqpoint{4.937030in}{1.778897in}}%
\pgfpathlineto{\pgfqpoint{4.937030in}{1.813884in}}%
\pgfpathclose%
\pgfusepath{fill}%
\end{pgfscope}%
\begin{pgfscope}%
\pgfpathrectangle{\pgfqpoint{3.722897in}{0.857143in}}{\pgfqpoint{2.627103in}{1.813434in}}%
\pgfusepath{clip}%
\pgfsetbuttcap%
\pgfsetmiterjoin%
\definecolor{currentfill}{rgb}{0.066899,0.263188,0.377594}%
\pgfsetfillcolor{currentfill}%
\pgfsetlinewidth{0.000000pt}%
\definecolor{currentstroke}{rgb}{0.000000,0.000000,0.000000}%
\pgfsetstrokecolor{currentstroke}%
\pgfsetstrokeopacity{0.000000}%
\pgfsetdash{}{0pt}%
\pgfpathmoveto{\pgfqpoint{4.948201in}{1.813884in}}%
\pgfpathlineto{\pgfqpoint{4.957137in}{1.813884in}}%
\pgfpathlineto{\pgfqpoint{4.957137in}{1.780387in}}%
\pgfpathlineto{\pgfqpoint{4.948201in}{1.780387in}}%
\pgfpathlineto{\pgfqpoint{4.948201in}{1.813884in}}%
\pgfpathclose%
\pgfusepath{fill}%
\end{pgfscope}%
\begin{pgfscope}%
\pgfpathrectangle{\pgfqpoint{3.722897in}{0.857143in}}{\pgfqpoint{2.627103in}{1.813434in}}%
\pgfusepath{clip}%
\pgfsetbuttcap%
\pgfsetmiterjoin%
\definecolor{currentfill}{rgb}{0.066899,0.263188,0.377594}%
\pgfsetfillcolor{currentfill}%
\pgfsetlinewidth{0.000000pt}%
\definecolor{currentstroke}{rgb}{0.000000,0.000000,0.000000}%
\pgfsetstrokecolor{currentstroke}%
\pgfsetstrokeopacity{0.000000}%
\pgfsetdash{}{0pt}%
\pgfpathmoveto{\pgfqpoint{4.959371in}{1.813884in}}%
\pgfpathlineto{\pgfqpoint{4.968308in}{1.813884in}}%
\pgfpathlineto{\pgfqpoint{4.968308in}{1.780501in}}%
\pgfpathlineto{\pgfqpoint{4.959371in}{1.780501in}}%
\pgfpathlineto{\pgfqpoint{4.959371in}{1.813884in}}%
\pgfpathclose%
\pgfusepath{fill}%
\end{pgfscope}%
\begin{pgfscope}%
\pgfpathrectangle{\pgfqpoint{3.722897in}{0.857143in}}{\pgfqpoint{2.627103in}{1.813434in}}%
\pgfusepath{clip}%
\pgfsetbuttcap%
\pgfsetmiterjoin%
\definecolor{currentfill}{rgb}{0.066899,0.263188,0.377594}%
\pgfsetfillcolor{currentfill}%
\pgfsetlinewidth{0.000000pt}%
\definecolor{currentstroke}{rgb}{0.000000,0.000000,0.000000}%
\pgfsetstrokecolor{currentstroke}%
\pgfsetstrokeopacity{0.000000}%
\pgfsetdash{}{0pt}%
\pgfpathmoveto{\pgfqpoint{4.970542in}{1.813884in}}%
\pgfpathlineto{\pgfqpoint{4.979478in}{1.813884in}}%
\pgfpathlineto{\pgfqpoint{4.979478in}{1.780631in}}%
\pgfpathlineto{\pgfqpoint{4.970542in}{1.780631in}}%
\pgfpathlineto{\pgfqpoint{4.970542in}{1.813884in}}%
\pgfpathclose%
\pgfusepath{fill}%
\end{pgfscope}%
\begin{pgfscope}%
\pgfpathrectangle{\pgfqpoint{3.722897in}{0.857143in}}{\pgfqpoint{2.627103in}{1.813434in}}%
\pgfusepath{clip}%
\pgfsetbuttcap%
\pgfsetmiterjoin%
\definecolor{currentfill}{rgb}{0.066899,0.263188,0.377594}%
\pgfsetfillcolor{currentfill}%
\pgfsetlinewidth{0.000000pt}%
\definecolor{currentstroke}{rgb}{0.000000,0.000000,0.000000}%
\pgfsetstrokecolor{currentstroke}%
\pgfsetstrokeopacity{0.000000}%
\pgfsetdash{}{0pt}%
\pgfpathmoveto{\pgfqpoint{4.981713in}{1.813884in}}%
\pgfpathlineto{\pgfqpoint{4.990649in}{1.813884in}}%
\pgfpathlineto{\pgfqpoint{4.990649in}{1.783723in}}%
\pgfpathlineto{\pgfqpoint{4.981713in}{1.783723in}}%
\pgfpathlineto{\pgfqpoint{4.981713in}{1.813884in}}%
\pgfpathclose%
\pgfusepath{fill}%
\end{pgfscope}%
\begin{pgfscope}%
\pgfpathrectangle{\pgfqpoint{3.722897in}{0.857143in}}{\pgfqpoint{2.627103in}{1.813434in}}%
\pgfusepath{clip}%
\pgfsetbuttcap%
\pgfsetmiterjoin%
\definecolor{currentfill}{rgb}{0.066899,0.263188,0.377594}%
\pgfsetfillcolor{currentfill}%
\pgfsetlinewidth{0.000000pt}%
\definecolor{currentstroke}{rgb}{0.000000,0.000000,0.000000}%
\pgfsetstrokecolor{currentstroke}%
\pgfsetstrokeopacity{0.000000}%
\pgfsetdash{}{0pt}%
\pgfpathmoveto{\pgfqpoint{4.992883in}{1.813884in}}%
\pgfpathlineto{\pgfqpoint{5.001820in}{1.813884in}}%
\pgfpathlineto{\pgfqpoint{5.001820in}{1.784930in}}%
\pgfpathlineto{\pgfqpoint{4.992883in}{1.784930in}}%
\pgfpathlineto{\pgfqpoint{4.992883in}{1.813884in}}%
\pgfpathclose%
\pgfusepath{fill}%
\end{pgfscope}%
\begin{pgfscope}%
\pgfpathrectangle{\pgfqpoint{3.722897in}{0.857143in}}{\pgfqpoint{2.627103in}{1.813434in}}%
\pgfusepath{clip}%
\pgfsetbuttcap%
\pgfsetmiterjoin%
\definecolor{currentfill}{rgb}{0.066899,0.263188,0.377594}%
\pgfsetfillcolor{currentfill}%
\pgfsetlinewidth{0.000000pt}%
\definecolor{currentstroke}{rgb}{0.000000,0.000000,0.000000}%
\pgfsetstrokecolor{currentstroke}%
\pgfsetstrokeopacity{0.000000}%
\pgfsetdash{}{0pt}%
\pgfpathmoveto{\pgfqpoint{5.004054in}{1.813884in}}%
\pgfpathlineto{\pgfqpoint{5.012990in}{1.813884in}}%
\pgfpathlineto{\pgfqpoint{5.012990in}{1.786721in}}%
\pgfpathlineto{\pgfqpoint{5.004054in}{1.786721in}}%
\pgfpathlineto{\pgfqpoint{5.004054in}{1.813884in}}%
\pgfpathclose%
\pgfusepath{fill}%
\end{pgfscope}%
\begin{pgfscope}%
\pgfpathrectangle{\pgfqpoint{3.722897in}{0.857143in}}{\pgfqpoint{2.627103in}{1.813434in}}%
\pgfusepath{clip}%
\pgfsetbuttcap%
\pgfsetmiterjoin%
\definecolor{currentfill}{rgb}{0.066899,0.263188,0.377594}%
\pgfsetfillcolor{currentfill}%
\pgfsetlinewidth{0.000000pt}%
\definecolor{currentstroke}{rgb}{0.000000,0.000000,0.000000}%
\pgfsetstrokecolor{currentstroke}%
\pgfsetstrokeopacity{0.000000}%
\pgfsetdash{}{0pt}%
\pgfpathmoveto{\pgfqpoint{5.015224in}{1.813884in}}%
\pgfpathlineto{\pgfqpoint{5.024161in}{1.813884in}}%
\pgfpathlineto{\pgfqpoint{5.024161in}{1.787814in}}%
\pgfpathlineto{\pgfqpoint{5.015224in}{1.787814in}}%
\pgfpathlineto{\pgfqpoint{5.015224in}{1.813884in}}%
\pgfpathclose%
\pgfusepath{fill}%
\end{pgfscope}%
\begin{pgfscope}%
\pgfpathrectangle{\pgfqpoint{3.722897in}{0.857143in}}{\pgfqpoint{2.627103in}{1.813434in}}%
\pgfusepath{clip}%
\pgfsetbuttcap%
\pgfsetmiterjoin%
\definecolor{currentfill}{rgb}{0.066899,0.263188,0.377594}%
\pgfsetfillcolor{currentfill}%
\pgfsetlinewidth{0.000000pt}%
\definecolor{currentstroke}{rgb}{0.000000,0.000000,0.000000}%
\pgfsetstrokecolor{currentstroke}%
\pgfsetstrokeopacity{0.000000}%
\pgfsetdash{}{0pt}%
\pgfpathmoveto{\pgfqpoint{5.026395in}{1.813884in}}%
\pgfpathlineto{\pgfqpoint{5.035331in}{1.813884in}}%
\pgfpathlineto{\pgfqpoint{5.035331in}{1.786467in}}%
\pgfpathlineto{\pgfqpoint{5.026395in}{1.786467in}}%
\pgfpathlineto{\pgfqpoint{5.026395in}{1.813884in}}%
\pgfpathclose%
\pgfusepath{fill}%
\end{pgfscope}%
\begin{pgfscope}%
\pgfpathrectangle{\pgfqpoint{3.722897in}{0.857143in}}{\pgfqpoint{2.627103in}{1.813434in}}%
\pgfusepath{clip}%
\pgfsetbuttcap%
\pgfsetmiterjoin%
\definecolor{currentfill}{rgb}{0.066899,0.263188,0.377594}%
\pgfsetfillcolor{currentfill}%
\pgfsetlinewidth{0.000000pt}%
\definecolor{currentstroke}{rgb}{0.000000,0.000000,0.000000}%
\pgfsetstrokecolor{currentstroke}%
\pgfsetstrokeopacity{0.000000}%
\pgfsetdash{}{0pt}%
\pgfpathmoveto{\pgfqpoint{5.037566in}{1.813884in}}%
\pgfpathlineto{\pgfqpoint{5.046502in}{1.813884in}}%
\pgfpathlineto{\pgfqpoint{5.046502in}{1.785074in}}%
\pgfpathlineto{\pgfqpoint{5.037566in}{1.785074in}}%
\pgfpathlineto{\pgfqpoint{5.037566in}{1.813884in}}%
\pgfpathclose%
\pgfusepath{fill}%
\end{pgfscope}%
\begin{pgfscope}%
\pgfpathrectangle{\pgfqpoint{3.722897in}{0.857143in}}{\pgfqpoint{2.627103in}{1.813434in}}%
\pgfusepath{clip}%
\pgfsetbuttcap%
\pgfsetmiterjoin%
\definecolor{currentfill}{rgb}{0.066899,0.263188,0.377594}%
\pgfsetfillcolor{currentfill}%
\pgfsetlinewidth{0.000000pt}%
\definecolor{currentstroke}{rgb}{0.000000,0.000000,0.000000}%
\pgfsetstrokecolor{currentstroke}%
\pgfsetstrokeopacity{0.000000}%
\pgfsetdash{}{0pt}%
\pgfpathmoveto{\pgfqpoint{5.048736in}{1.813884in}}%
\pgfpathlineto{\pgfqpoint{5.057673in}{1.813884in}}%
\pgfpathlineto{\pgfqpoint{5.057673in}{1.784086in}}%
\pgfpathlineto{\pgfqpoint{5.048736in}{1.784086in}}%
\pgfpathlineto{\pgfqpoint{5.048736in}{1.813884in}}%
\pgfpathclose%
\pgfusepath{fill}%
\end{pgfscope}%
\begin{pgfscope}%
\pgfpathrectangle{\pgfqpoint{3.722897in}{0.857143in}}{\pgfqpoint{2.627103in}{1.813434in}}%
\pgfusepath{clip}%
\pgfsetbuttcap%
\pgfsetmiterjoin%
\definecolor{currentfill}{rgb}{0.066899,0.263188,0.377594}%
\pgfsetfillcolor{currentfill}%
\pgfsetlinewidth{0.000000pt}%
\definecolor{currentstroke}{rgb}{0.000000,0.000000,0.000000}%
\pgfsetstrokecolor{currentstroke}%
\pgfsetstrokeopacity{0.000000}%
\pgfsetdash{}{0pt}%
\pgfpathmoveto{\pgfqpoint{5.059907in}{1.813884in}}%
\pgfpathlineto{\pgfqpoint{5.068843in}{1.813884in}}%
\pgfpathlineto{\pgfqpoint{5.068843in}{1.783861in}}%
\pgfpathlineto{\pgfqpoint{5.059907in}{1.783861in}}%
\pgfpathlineto{\pgfqpoint{5.059907in}{1.813884in}}%
\pgfpathclose%
\pgfusepath{fill}%
\end{pgfscope}%
\begin{pgfscope}%
\pgfpathrectangle{\pgfqpoint{3.722897in}{0.857143in}}{\pgfqpoint{2.627103in}{1.813434in}}%
\pgfusepath{clip}%
\pgfsetbuttcap%
\pgfsetmiterjoin%
\definecolor{currentfill}{rgb}{0.066899,0.263188,0.377594}%
\pgfsetfillcolor{currentfill}%
\pgfsetlinewidth{0.000000pt}%
\definecolor{currentstroke}{rgb}{0.000000,0.000000,0.000000}%
\pgfsetstrokecolor{currentstroke}%
\pgfsetstrokeopacity{0.000000}%
\pgfsetdash{}{0pt}%
\pgfpathmoveto{\pgfqpoint{5.071077in}{1.813884in}}%
\pgfpathlineto{\pgfqpoint{5.080014in}{1.813884in}}%
\pgfpathlineto{\pgfqpoint{5.080014in}{1.783352in}}%
\pgfpathlineto{\pgfqpoint{5.071077in}{1.783352in}}%
\pgfpathlineto{\pgfqpoint{5.071077in}{1.813884in}}%
\pgfpathclose%
\pgfusepath{fill}%
\end{pgfscope}%
\begin{pgfscope}%
\pgfpathrectangle{\pgfqpoint{3.722897in}{0.857143in}}{\pgfqpoint{2.627103in}{1.813434in}}%
\pgfusepath{clip}%
\pgfsetbuttcap%
\pgfsetmiterjoin%
\definecolor{currentfill}{rgb}{0.066899,0.263188,0.377594}%
\pgfsetfillcolor{currentfill}%
\pgfsetlinewidth{0.000000pt}%
\definecolor{currentstroke}{rgb}{0.000000,0.000000,0.000000}%
\pgfsetstrokecolor{currentstroke}%
\pgfsetstrokeopacity{0.000000}%
\pgfsetdash{}{0pt}%
\pgfpathmoveto{\pgfqpoint{5.082248in}{1.813884in}}%
\pgfpathlineto{\pgfqpoint{5.091184in}{1.813884in}}%
\pgfpathlineto{\pgfqpoint{5.091184in}{1.781435in}}%
\pgfpathlineto{\pgfqpoint{5.082248in}{1.781435in}}%
\pgfpathlineto{\pgfqpoint{5.082248in}{1.813884in}}%
\pgfpathclose%
\pgfusepath{fill}%
\end{pgfscope}%
\begin{pgfscope}%
\pgfpathrectangle{\pgfqpoint{3.722897in}{0.857143in}}{\pgfqpoint{2.627103in}{1.813434in}}%
\pgfusepath{clip}%
\pgfsetbuttcap%
\pgfsetmiterjoin%
\definecolor{currentfill}{rgb}{0.066899,0.263188,0.377594}%
\pgfsetfillcolor{currentfill}%
\pgfsetlinewidth{0.000000pt}%
\definecolor{currentstroke}{rgb}{0.000000,0.000000,0.000000}%
\pgfsetstrokecolor{currentstroke}%
\pgfsetstrokeopacity{0.000000}%
\pgfsetdash{}{0pt}%
\pgfpathmoveto{\pgfqpoint{5.093419in}{1.813884in}}%
\pgfpathlineto{\pgfqpoint{5.102355in}{1.813884in}}%
\pgfpathlineto{\pgfqpoint{5.102355in}{1.778545in}}%
\pgfpathlineto{\pgfqpoint{5.093419in}{1.778545in}}%
\pgfpathlineto{\pgfqpoint{5.093419in}{1.813884in}}%
\pgfpathclose%
\pgfusepath{fill}%
\end{pgfscope}%
\begin{pgfscope}%
\pgfpathrectangle{\pgfqpoint{3.722897in}{0.857143in}}{\pgfqpoint{2.627103in}{1.813434in}}%
\pgfusepath{clip}%
\pgfsetbuttcap%
\pgfsetmiterjoin%
\definecolor{currentfill}{rgb}{0.066899,0.263188,0.377594}%
\pgfsetfillcolor{currentfill}%
\pgfsetlinewidth{0.000000pt}%
\definecolor{currentstroke}{rgb}{0.000000,0.000000,0.000000}%
\pgfsetstrokecolor{currentstroke}%
\pgfsetstrokeopacity{0.000000}%
\pgfsetdash{}{0pt}%
\pgfpathmoveto{\pgfqpoint{5.104589in}{1.813884in}}%
\pgfpathlineto{\pgfqpoint{5.113526in}{1.813884in}}%
\pgfpathlineto{\pgfqpoint{5.113526in}{1.778128in}}%
\pgfpathlineto{\pgfqpoint{5.104589in}{1.778128in}}%
\pgfpathlineto{\pgfqpoint{5.104589in}{1.813884in}}%
\pgfpathclose%
\pgfusepath{fill}%
\end{pgfscope}%
\begin{pgfscope}%
\pgfpathrectangle{\pgfqpoint{3.722897in}{0.857143in}}{\pgfqpoint{2.627103in}{1.813434in}}%
\pgfusepath{clip}%
\pgfsetbuttcap%
\pgfsetmiterjoin%
\definecolor{currentfill}{rgb}{0.066899,0.263188,0.377594}%
\pgfsetfillcolor{currentfill}%
\pgfsetlinewidth{0.000000pt}%
\definecolor{currentstroke}{rgb}{0.000000,0.000000,0.000000}%
\pgfsetstrokecolor{currentstroke}%
\pgfsetstrokeopacity{0.000000}%
\pgfsetdash{}{0pt}%
\pgfpathmoveto{\pgfqpoint{5.115760in}{1.813884in}}%
\pgfpathlineto{\pgfqpoint{5.124696in}{1.813884in}}%
\pgfpathlineto{\pgfqpoint{5.124696in}{1.775815in}}%
\pgfpathlineto{\pgfqpoint{5.115760in}{1.775815in}}%
\pgfpathlineto{\pgfqpoint{5.115760in}{1.813884in}}%
\pgfpathclose%
\pgfusepath{fill}%
\end{pgfscope}%
\begin{pgfscope}%
\pgfpathrectangle{\pgfqpoint{3.722897in}{0.857143in}}{\pgfqpoint{2.627103in}{1.813434in}}%
\pgfusepath{clip}%
\pgfsetbuttcap%
\pgfsetmiterjoin%
\definecolor{currentfill}{rgb}{0.066899,0.263188,0.377594}%
\pgfsetfillcolor{currentfill}%
\pgfsetlinewidth{0.000000pt}%
\definecolor{currentstroke}{rgb}{0.000000,0.000000,0.000000}%
\pgfsetstrokecolor{currentstroke}%
\pgfsetstrokeopacity{0.000000}%
\pgfsetdash{}{0pt}%
\pgfpathmoveto{\pgfqpoint{5.126930in}{1.813884in}}%
\pgfpathlineto{\pgfqpoint{5.135867in}{1.813884in}}%
\pgfpathlineto{\pgfqpoint{5.135867in}{1.774607in}}%
\pgfpathlineto{\pgfqpoint{5.126930in}{1.774607in}}%
\pgfpathlineto{\pgfqpoint{5.126930in}{1.813884in}}%
\pgfpathclose%
\pgfusepath{fill}%
\end{pgfscope}%
\begin{pgfscope}%
\pgfpathrectangle{\pgfqpoint{3.722897in}{0.857143in}}{\pgfqpoint{2.627103in}{1.813434in}}%
\pgfusepath{clip}%
\pgfsetbuttcap%
\pgfsetmiterjoin%
\definecolor{currentfill}{rgb}{0.066899,0.263188,0.377594}%
\pgfsetfillcolor{currentfill}%
\pgfsetlinewidth{0.000000pt}%
\definecolor{currentstroke}{rgb}{0.000000,0.000000,0.000000}%
\pgfsetstrokecolor{currentstroke}%
\pgfsetstrokeopacity{0.000000}%
\pgfsetdash{}{0pt}%
\pgfpathmoveto{\pgfqpoint{5.138101in}{1.813884in}}%
\pgfpathlineto{\pgfqpoint{5.147038in}{1.813884in}}%
\pgfpathlineto{\pgfqpoint{5.147038in}{1.771751in}}%
\pgfpathlineto{\pgfqpoint{5.138101in}{1.771751in}}%
\pgfpathlineto{\pgfqpoint{5.138101in}{1.813884in}}%
\pgfpathclose%
\pgfusepath{fill}%
\end{pgfscope}%
\begin{pgfscope}%
\pgfpathrectangle{\pgfqpoint{3.722897in}{0.857143in}}{\pgfqpoint{2.627103in}{1.813434in}}%
\pgfusepath{clip}%
\pgfsetbuttcap%
\pgfsetmiterjoin%
\definecolor{currentfill}{rgb}{0.066899,0.263188,0.377594}%
\pgfsetfillcolor{currentfill}%
\pgfsetlinewidth{0.000000pt}%
\definecolor{currentstroke}{rgb}{0.000000,0.000000,0.000000}%
\pgfsetstrokecolor{currentstroke}%
\pgfsetstrokeopacity{0.000000}%
\pgfsetdash{}{0pt}%
\pgfpathmoveto{\pgfqpoint{5.149272in}{1.813884in}}%
\pgfpathlineto{\pgfqpoint{5.158208in}{1.813884in}}%
\pgfpathlineto{\pgfqpoint{5.158208in}{1.770283in}}%
\pgfpathlineto{\pgfqpoint{5.149272in}{1.770283in}}%
\pgfpathlineto{\pgfqpoint{5.149272in}{1.813884in}}%
\pgfpathclose%
\pgfusepath{fill}%
\end{pgfscope}%
\begin{pgfscope}%
\pgfpathrectangle{\pgfqpoint{3.722897in}{0.857143in}}{\pgfqpoint{2.627103in}{1.813434in}}%
\pgfusepath{clip}%
\pgfsetbuttcap%
\pgfsetmiterjoin%
\definecolor{currentfill}{rgb}{0.066899,0.263188,0.377594}%
\pgfsetfillcolor{currentfill}%
\pgfsetlinewidth{0.000000pt}%
\definecolor{currentstroke}{rgb}{0.000000,0.000000,0.000000}%
\pgfsetstrokecolor{currentstroke}%
\pgfsetstrokeopacity{0.000000}%
\pgfsetdash{}{0pt}%
\pgfpathmoveto{\pgfqpoint{5.160442in}{1.813884in}}%
\pgfpathlineto{\pgfqpoint{5.169379in}{1.813884in}}%
\pgfpathlineto{\pgfqpoint{5.169379in}{1.769071in}}%
\pgfpathlineto{\pgfqpoint{5.160442in}{1.769071in}}%
\pgfpathlineto{\pgfqpoint{5.160442in}{1.813884in}}%
\pgfpathclose%
\pgfusepath{fill}%
\end{pgfscope}%
\begin{pgfscope}%
\pgfpathrectangle{\pgfqpoint{3.722897in}{0.857143in}}{\pgfqpoint{2.627103in}{1.813434in}}%
\pgfusepath{clip}%
\pgfsetbuttcap%
\pgfsetmiterjoin%
\definecolor{currentfill}{rgb}{0.066899,0.263188,0.377594}%
\pgfsetfillcolor{currentfill}%
\pgfsetlinewidth{0.000000pt}%
\definecolor{currentstroke}{rgb}{0.000000,0.000000,0.000000}%
\pgfsetstrokecolor{currentstroke}%
\pgfsetstrokeopacity{0.000000}%
\pgfsetdash{}{0pt}%
\pgfpathmoveto{\pgfqpoint{5.171613in}{1.813884in}}%
\pgfpathlineto{\pgfqpoint{5.180549in}{1.813884in}}%
\pgfpathlineto{\pgfqpoint{5.180549in}{1.768342in}}%
\pgfpathlineto{\pgfqpoint{5.171613in}{1.768342in}}%
\pgfpathlineto{\pgfqpoint{5.171613in}{1.813884in}}%
\pgfpathclose%
\pgfusepath{fill}%
\end{pgfscope}%
\begin{pgfscope}%
\pgfpathrectangle{\pgfqpoint{3.722897in}{0.857143in}}{\pgfqpoint{2.627103in}{1.813434in}}%
\pgfusepath{clip}%
\pgfsetbuttcap%
\pgfsetmiterjoin%
\definecolor{currentfill}{rgb}{0.066899,0.263188,0.377594}%
\pgfsetfillcolor{currentfill}%
\pgfsetlinewidth{0.000000pt}%
\definecolor{currentstroke}{rgb}{0.000000,0.000000,0.000000}%
\pgfsetstrokecolor{currentstroke}%
\pgfsetstrokeopacity{0.000000}%
\pgfsetdash{}{0pt}%
\pgfpathmoveto{\pgfqpoint{5.182783in}{1.813884in}}%
\pgfpathlineto{\pgfqpoint{5.191720in}{1.813884in}}%
\pgfpathlineto{\pgfqpoint{5.191720in}{1.766335in}}%
\pgfpathlineto{\pgfqpoint{5.182783in}{1.766335in}}%
\pgfpathlineto{\pgfqpoint{5.182783in}{1.813884in}}%
\pgfpathclose%
\pgfusepath{fill}%
\end{pgfscope}%
\begin{pgfscope}%
\pgfpathrectangle{\pgfqpoint{3.722897in}{0.857143in}}{\pgfqpoint{2.627103in}{1.813434in}}%
\pgfusepath{clip}%
\pgfsetbuttcap%
\pgfsetmiterjoin%
\definecolor{currentfill}{rgb}{0.066899,0.263188,0.377594}%
\pgfsetfillcolor{currentfill}%
\pgfsetlinewidth{0.000000pt}%
\definecolor{currentstroke}{rgb}{0.000000,0.000000,0.000000}%
\pgfsetstrokecolor{currentstroke}%
\pgfsetstrokeopacity{0.000000}%
\pgfsetdash{}{0pt}%
\pgfpathmoveto{\pgfqpoint{5.193954in}{1.813884in}}%
\pgfpathlineto{\pgfqpoint{5.202891in}{1.813884in}}%
\pgfpathlineto{\pgfqpoint{5.202891in}{1.763968in}}%
\pgfpathlineto{\pgfqpoint{5.193954in}{1.763968in}}%
\pgfpathlineto{\pgfqpoint{5.193954in}{1.813884in}}%
\pgfpathclose%
\pgfusepath{fill}%
\end{pgfscope}%
\begin{pgfscope}%
\pgfpathrectangle{\pgfqpoint{3.722897in}{0.857143in}}{\pgfqpoint{2.627103in}{1.813434in}}%
\pgfusepath{clip}%
\pgfsetbuttcap%
\pgfsetmiterjoin%
\definecolor{currentfill}{rgb}{0.066899,0.263188,0.377594}%
\pgfsetfillcolor{currentfill}%
\pgfsetlinewidth{0.000000pt}%
\definecolor{currentstroke}{rgb}{0.000000,0.000000,0.000000}%
\pgfsetstrokecolor{currentstroke}%
\pgfsetstrokeopacity{0.000000}%
\pgfsetdash{}{0pt}%
\pgfpathmoveto{\pgfqpoint{5.205125in}{1.813884in}}%
\pgfpathlineto{\pgfqpoint{5.214061in}{1.813884in}}%
\pgfpathlineto{\pgfqpoint{5.214061in}{1.761216in}}%
\pgfpathlineto{\pgfqpoint{5.205125in}{1.761216in}}%
\pgfpathlineto{\pgfqpoint{5.205125in}{1.813884in}}%
\pgfpathclose%
\pgfusepath{fill}%
\end{pgfscope}%
\begin{pgfscope}%
\pgfpathrectangle{\pgfqpoint{3.722897in}{0.857143in}}{\pgfqpoint{2.627103in}{1.813434in}}%
\pgfusepath{clip}%
\pgfsetbuttcap%
\pgfsetmiterjoin%
\definecolor{currentfill}{rgb}{0.066899,0.263188,0.377594}%
\pgfsetfillcolor{currentfill}%
\pgfsetlinewidth{0.000000pt}%
\definecolor{currentstroke}{rgb}{0.000000,0.000000,0.000000}%
\pgfsetstrokecolor{currentstroke}%
\pgfsetstrokeopacity{0.000000}%
\pgfsetdash{}{0pt}%
\pgfpathmoveto{\pgfqpoint{5.216295in}{1.813884in}}%
\pgfpathlineto{\pgfqpoint{5.225232in}{1.813884in}}%
\pgfpathlineto{\pgfqpoint{5.225232in}{1.761294in}}%
\pgfpathlineto{\pgfqpoint{5.216295in}{1.761294in}}%
\pgfpathlineto{\pgfqpoint{5.216295in}{1.813884in}}%
\pgfpathclose%
\pgfusepath{fill}%
\end{pgfscope}%
\begin{pgfscope}%
\pgfpathrectangle{\pgfqpoint{3.722897in}{0.857143in}}{\pgfqpoint{2.627103in}{1.813434in}}%
\pgfusepath{clip}%
\pgfsetbuttcap%
\pgfsetmiterjoin%
\definecolor{currentfill}{rgb}{0.066899,0.263188,0.377594}%
\pgfsetfillcolor{currentfill}%
\pgfsetlinewidth{0.000000pt}%
\definecolor{currentstroke}{rgb}{0.000000,0.000000,0.000000}%
\pgfsetstrokecolor{currentstroke}%
\pgfsetstrokeopacity{0.000000}%
\pgfsetdash{}{0pt}%
\pgfpathmoveto{\pgfqpoint{5.227466in}{1.813884in}}%
\pgfpathlineto{\pgfqpoint{5.236402in}{1.813884in}}%
\pgfpathlineto{\pgfqpoint{5.236402in}{1.760448in}}%
\pgfpathlineto{\pgfqpoint{5.227466in}{1.760448in}}%
\pgfpathlineto{\pgfqpoint{5.227466in}{1.813884in}}%
\pgfpathclose%
\pgfusepath{fill}%
\end{pgfscope}%
\begin{pgfscope}%
\pgfpathrectangle{\pgfqpoint{3.722897in}{0.857143in}}{\pgfqpoint{2.627103in}{1.813434in}}%
\pgfusepath{clip}%
\pgfsetbuttcap%
\pgfsetmiterjoin%
\definecolor{currentfill}{rgb}{0.066899,0.263188,0.377594}%
\pgfsetfillcolor{currentfill}%
\pgfsetlinewidth{0.000000pt}%
\definecolor{currentstroke}{rgb}{0.000000,0.000000,0.000000}%
\pgfsetstrokecolor{currentstroke}%
\pgfsetstrokeopacity{0.000000}%
\pgfsetdash{}{0pt}%
\pgfpathmoveto{\pgfqpoint{5.238636in}{1.813884in}}%
\pgfpathlineto{\pgfqpoint{5.247573in}{1.813884in}}%
\pgfpathlineto{\pgfqpoint{5.247573in}{1.759231in}}%
\pgfpathlineto{\pgfqpoint{5.238636in}{1.759231in}}%
\pgfpathlineto{\pgfqpoint{5.238636in}{1.813884in}}%
\pgfpathclose%
\pgfusepath{fill}%
\end{pgfscope}%
\begin{pgfscope}%
\pgfpathrectangle{\pgfqpoint{3.722897in}{0.857143in}}{\pgfqpoint{2.627103in}{1.813434in}}%
\pgfusepath{clip}%
\pgfsetbuttcap%
\pgfsetmiterjoin%
\definecolor{currentfill}{rgb}{0.066899,0.263188,0.377594}%
\pgfsetfillcolor{currentfill}%
\pgfsetlinewidth{0.000000pt}%
\definecolor{currentstroke}{rgb}{0.000000,0.000000,0.000000}%
\pgfsetstrokecolor{currentstroke}%
\pgfsetstrokeopacity{0.000000}%
\pgfsetdash{}{0pt}%
\pgfpathmoveto{\pgfqpoint{5.249807in}{1.813884in}}%
\pgfpathlineto{\pgfqpoint{5.258744in}{1.813884in}}%
\pgfpathlineto{\pgfqpoint{5.258744in}{1.758124in}}%
\pgfpathlineto{\pgfqpoint{5.249807in}{1.758124in}}%
\pgfpathlineto{\pgfqpoint{5.249807in}{1.813884in}}%
\pgfpathclose%
\pgfusepath{fill}%
\end{pgfscope}%
\begin{pgfscope}%
\pgfpathrectangle{\pgfqpoint{3.722897in}{0.857143in}}{\pgfqpoint{2.627103in}{1.813434in}}%
\pgfusepath{clip}%
\pgfsetbuttcap%
\pgfsetmiterjoin%
\definecolor{currentfill}{rgb}{0.066899,0.263188,0.377594}%
\pgfsetfillcolor{currentfill}%
\pgfsetlinewidth{0.000000pt}%
\definecolor{currentstroke}{rgb}{0.000000,0.000000,0.000000}%
\pgfsetstrokecolor{currentstroke}%
\pgfsetstrokeopacity{0.000000}%
\pgfsetdash{}{0pt}%
\pgfpathmoveto{\pgfqpoint{5.260978in}{1.813884in}}%
\pgfpathlineto{\pgfqpoint{5.269914in}{1.813884in}}%
\pgfpathlineto{\pgfqpoint{5.269914in}{1.757162in}}%
\pgfpathlineto{\pgfqpoint{5.260978in}{1.757162in}}%
\pgfpathlineto{\pgfqpoint{5.260978in}{1.813884in}}%
\pgfpathclose%
\pgfusepath{fill}%
\end{pgfscope}%
\begin{pgfscope}%
\pgfpathrectangle{\pgfqpoint{3.722897in}{0.857143in}}{\pgfqpoint{2.627103in}{1.813434in}}%
\pgfusepath{clip}%
\pgfsetbuttcap%
\pgfsetmiterjoin%
\definecolor{currentfill}{rgb}{0.066899,0.263188,0.377594}%
\pgfsetfillcolor{currentfill}%
\pgfsetlinewidth{0.000000pt}%
\definecolor{currentstroke}{rgb}{0.000000,0.000000,0.000000}%
\pgfsetstrokecolor{currentstroke}%
\pgfsetstrokeopacity{0.000000}%
\pgfsetdash{}{0pt}%
\pgfpathmoveto{\pgfqpoint{5.272148in}{1.813884in}}%
\pgfpathlineto{\pgfqpoint{5.281085in}{1.813884in}}%
\pgfpathlineto{\pgfqpoint{5.281085in}{1.757829in}}%
\pgfpathlineto{\pgfqpoint{5.272148in}{1.757829in}}%
\pgfpathlineto{\pgfqpoint{5.272148in}{1.813884in}}%
\pgfpathclose%
\pgfusepath{fill}%
\end{pgfscope}%
\begin{pgfscope}%
\pgfpathrectangle{\pgfqpoint{3.722897in}{0.857143in}}{\pgfqpoint{2.627103in}{1.813434in}}%
\pgfusepath{clip}%
\pgfsetbuttcap%
\pgfsetmiterjoin%
\definecolor{currentfill}{rgb}{0.066899,0.263188,0.377594}%
\pgfsetfillcolor{currentfill}%
\pgfsetlinewidth{0.000000pt}%
\definecolor{currentstroke}{rgb}{0.000000,0.000000,0.000000}%
\pgfsetstrokecolor{currentstroke}%
\pgfsetstrokeopacity{0.000000}%
\pgfsetdash{}{0pt}%
\pgfpathmoveto{\pgfqpoint{5.283319in}{1.813884in}}%
\pgfpathlineto{\pgfqpoint{5.292255in}{1.813884in}}%
\pgfpathlineto{\pgfqpoint{5.292255in}{1.757156in}}%
\pgfpathlineto{\pgfqpoint{5.283319in}{1.757156in}}%
\pgfpathlineto{\pgfqpoint{5.283319in}{1.813884in}}%
\pgfpathclose%
\pgfusepath{fill}%
\end{pgfscope}%
\begin{pgfscope}%
\pgfpathrectangle{\pgfqpoint{3.722897in}{0.857143in}}{\pgfqpoint{2.627103in}{1.813434in}}%
\pgfusepath{clip}%
\pgfsetbuttcap%
\pgfsetmiterjoin%
\definecolor{currentfill}{rgb}{0.066899,0.263188,0.377594}%
\pgfsetfillcolor{currentfill}%
\pgfsetlinewidth{0.000000pt}%
\definecolor{currentstroke}{rgb}{0.000000,0.000000,0.000000}%
\pgfsetstrokecolor{currentstroke}%
\pgfsetstrokeopacity{0.000000}%
\pgfsetdash{}{0pt}%
\pgfpathmoveto{\pgfqpoint{5.294489in}{1.813884in}}%
\pgfpathlineto{\pgfqpoint{5.303426in}{1.813884in}}%
\pgfpathlineto{\pgfqpoint{5.303426in}{1.758177in}}%
\pgfpathlineto{\pgfqpoint{5.294489in}{1.758177in}}%
\pgfpathlineto{\pgfqpoint{5.294489in}{1.813884in}}%
\pgfpathclose%
\pgfusepath{fill}%
\end{pgfscope}%
\begin{pgfscope}%
\pgfpathrectangle{\pgfqpoint{3.722897in}{0.857143in}}{\pgfqpoint{2.627103in}{1.813434in}}%
\pgfusepath{clip}%
\pgfsetbuttcap%
\pgfsetmiterjoin%
\definecolor{currentfill}{rgb}{0.066899,0.263188,0.377594}%
\pgfsetfillcolor{currentfill}%
\pgfsetlinewidth{0.000000pt}%
\definecolor{currentstroke}{rgb}{0.000000,0.000000,0.000000}%
\pgfsetstrokecolor{currentstroke}%
\pgfsetstrokeopacity{0.000000}%
\pgfsetdash{}{0pt}%
\pgfpathmoveto{\pgfqpoint{5.305660in}{1.813884in}}%
\pgfpathlineto{\pgfqpoint{5.314597in}{1.813884in}}%
\pgfpathlineto{\pgfqpoint{5.314597in}{1.757742in}}%
\pgfpathlineto{\pgfqpoint{5.305660in}{1.757742in}}%
\pgfpathlineto{\pgfqpoint{5.305660in}{1.813884in}}%
\pgfpathclose%
\pgfusepath{fill}%
\end{pgfscope}%
\begin{pgfscope}%
\pgfpathrectangle{\pgfqpoint{3.722897in}{0.857143in}}{\pgfqpoint{2.627103in}{1.813434in}}%
\pgfusepath{clip}%
\pgfsetbuttcap%
\pgfsetmiterjoin%
\definecolor{currentfill}{rgb}{0.066899,0.263188,0.377594}%
\pgfsetfillcolor{currentfill}%
\pgfsetlinewidth{0.000000pt}%
\definecolor{currentstroke}{rgb}{0.000000,0.000000,0.000000}%
\pgfsetstrokecolor{currentstroke}%
\pgfsetstrokeopacity{0.000000}%
\pgfsetdash{}{0pt}%
\pgfpathmoveto{\pgfqpoint{5.316831in}{1.813884in}}%
\pgfpathlineto{\pgfqpoint{5.325767in}{1.813884in}}%
\pgfpathlineto{\pgfqpoint{5.325767in}{1.758266in}}%
\pgfpathlineto{\pgfqpoint{5.316831in}{1.758266in}}%
\pgfpathlineto{\pgfqpoint{5.316831in}{1.813884in}}%
\pgfpathclose%
\pgfusepath{fill}%
\end{pgfscope}%
\begin{pgfscope}%
\pgfpathrectangle{\pgfqpoint{3.722897in}{0.857143in}}{\pgfqpoint{2.627103in}{1.813434in}}%
\pgfusepath{clip}%
\pgfsetbuttcap%
\pgfsetmiterjoin%
\definecolor{currentfill}{rgb}{0.066899,0.263188,0.377594}%
\pgfsetfillcolor{currentfill}%
\pgfsetlinewidth{0.000000pt}%
\definecolor{currentstroke}{rgb}{0.000000,0.000000,0.000000}%
\pgfsetstrokecolor{currentstroke}%
\pgfsetstrokeopacity{0.000000}%
\pgfsetdash{}{0pt}%
\pgfpathmoveto{\pgfqpoint{5.328001in}{1.813884in}}%
\pgfpathlineto{\pgfqpoint{5.336938in}{1.813884in}}%
\pgfpathlineto{\pgfqpoint{5.336938in}{1.760102in}}%
\pgfpathlineto{\pgfqpoint{5.328001in}{1.760102in}}%
\pgfpathlineto{\pgfqpoint{5.328001in}{1.813884in}}%
\pgfpathclose%
\pgfusepath{fill}%
\end{pgfscope}%
\begin{pgfscope}%
\pgfpathrectangle{\pgfqpoint{3.722897in}{0.857143in}}{\pgfqpoint{2.627103in}{1.813434in}}%
\pgfusepath{clip}%
\pgfsetbuttcap%
\pgfsetmiterjoin%
\definecolor{currentfill}{rgb}{0.066899,0.263188,0.377594}%
\pgfsetfillcolor{currentfill}%
\pgfsetlinewidth{0.000000pt}%
\definecolor{currentstroke}{rgb}{0.000000,0.000000,0.000000}%
\pgfsetstrokecolor{currentstroke}%
\pgfsetstrokeopacity{0.000000}%
\pgfsetdash{}{0pt}%
\pgfpathmoveto{\pgfqpoint{5.339172in}{1.813884in}}%
\pgfpathlineto{\pgfqpoint{5.348108in}{1.813884in}}%
\pgfpathlineto{\pgfqpoint{5.348108in}{1.759382in}}%
\pgfpathlineto{\pgfqpoint{5.339172in}{1.759382in}}%
\pgfpathlineto{\pgfqpoint{5.339172in}{1.813884in}}%
\pgfpathclose%
\pgfusepath{fill}%
\end{pgfscope}%
\begin{pgfscope}%
\pgfpathrectangle{\pgfqpoint{3.722897in}{0.857143in}}{\pgfqpoint{2.627103in}{1.813434in}}%
\pgfusepath{clip}%
\pgfsetbuttcap%
\pgfsetmiterjoin%
\definecolor{currentfill}{rgb}{0.066899,0.263188,0.377594}%
\pgfsetfillcolor{currentfill}%
\pgfsetlinewidth{0.000000pt}%
\definecolor{currentstroke}{rgb}{0.000000,0.000000,0.000000}%
\pgfsetstrokecolor{currentstroke}%
\pgfsetstrokeopacity{0.000000}%
\pgfsetdash{}{0pt}%
\pgfpathmoveto{\pgfqpoint{5.350343in}{1.813884in}}%
\pgfpathlineto{\pgfqpoint{5.359279in}{1.813884in}}%
\pgfpathlineto{\pgfqpoint{5.359279in}{1.762207in}}%
\pgfpathlineto{\pgfqpoint{5.350343in}{1.762207in}}%
\pgfpathlineto{\pgfqpoint{5.350343in}{1.813884in}}%
\pgfpathclose%
\pgfusepath{fill}%
\end{pgfscope}%
\begin{pgfscope}%
\pgfpathrectangle{\pgfqpoint{3.722897in}{0.857143in}}{\pgfqpoint{2.627103in}{1.813434in}}%
\pgfusepath{clip}%
\pgfsetbuttcap%
\pgfsetmiterjoin%
\definecolor{currentfill}{rgb}{0.066899,0.263188,0.377594}%
\pgfsetfillcolor{currentfill}%
\pgfsetlinewidth{0.000000pt}%
\definecolor{currentstroke}{rgb}{0.000000,0.000000,0.000000}%
\pgfsetstrokecolor{currentstroke}%
\pgfsetstrokeopacity{0.000000}%
\pgfsetdash{}{0pt}%
\pgfpathmoveto{\pgfqpoint{5.361513in}{1.813884in}}%
\pgfpathlineto{\pgfqpoint{5.370450in}{1.813884in}}%
\pgfpathlineto{\pgfqpoint{5.370450in}{1.762279in}}%
\pgfpathlineto{\pgfqpoint{5.361513in}{1.762279in}}%
\pgfpathlineto{\pgfqpoint{5.361513in}{1.813884in}}%
\pgfpathclose%
\pgfusepath{fill}%
\end{pgfscope}%
\begin{pgfscope}%
\pgfpathrectangle{\pgfqpoint{3.722897in}{0.857143in}}{\pgfqpoint{2.627103in}{1.813434in}}%
\pgfusepath{clip}%
\pgfsetbuttcap%
\pgfsetmiterjoin%
\definecolor{currentfill}{rgb}{0.066899,0.263188,0.377594}%
\pgfsetfillcolor{currentfill}%
\pgfsetlinewidth{0.000000pt}%
\definecolor{currentstroke}{rgb}{0.000000,0.000000,0.000000}%
\pgfsetstrokecolor{currentstroke}%
\pgfsetstrokeopacity{0.000000}%
\pgfsetdash{}{0pt}%
\pgfpathmoveto{\pgfqpoint{5.372684in}{1.813884in}}%
\pgfpathlineto{\pgfqpoint{5.381620in}{1.813884in}}%
\pgfpathlineto{\pgfqpoint{5.381620in}{1.764248in}}%
\pgfpathlineto{\pgfqpoint{5.372684in}{1.764248in}}%
\pgfpathlineto{\pgfqpoint{5.372684in}{1.813884in}}%
\pgfpathclose%
\pgfusepath{fill}%
\end{pgfscope}%
\begin{pgfscope}%
\pgfpathrectangle{\pgfqpoint{3.722897in}{0.857143in}}{\pgfqpoint{2.627103in}{1.813434in}}%
\pgfusepath{clip}%
\pgfsetbuttcap%
\pgfsetmiterjoin%
\definecolor{currentfill}{rgb}{0.066899,0.263188,0.377594}%
\pgfsetfillcolor{currentfill}%
\pgfsetlinewidth{0.000000pt}%
\definecolor{currentstroke}{rgb}{0.000000,0.000000,0.000000}%
\pgfsetstrokecolor{currentstroke}%
\pgfsetstrokeopacity{0.000000}%
\pgfsetdash{}{0pt}%
\pgfpathmoveto{\pgfqpoint{5.383854in}{1.813884in}}%
\pgfpathlineto{\pgfqpoint{5.392791in}{1.813884in}}%
\pgfpathlineto{\pgfqpoint{5.392791in}{1.763619in}}%
\pgfpathlineto{\pgfqpoint{5.383854in}{1.763619in}}%
\pgfpathlineto{\pgfqpoint{5.383854in}{1.813884in}}%
\pgfpathclose%
\pgfusepath{fill}%
\end{pgfscope}%
\begin{pgfscope}%
\pgfpathrectangle{\pgfqpoint{3.722897in}{0.857143in}}{\pgfqpoint{2.627103in}{1.813434in}}%
\pgfusepath{clip}%
\pgfsetbuttcap%
\pgfsetmiterjoin%
\definecolor{currentfill}{rgb}{0.066899,0.263188,0.377594}%
\pgfsetfillcolor{currentfill}%
\pgfsetlinewidth{0.000000pt}%
\definecolor{currentstroke}{rgb}{0.000000,0.000000,0.000000}%
\pgfsetstrokecolor{currentstroke}%
\pgfsetstrokeopacity{0.000000}%
\pgfsetdash{}{0pt}%
\pgfpathmoveto{\pgfqpoint{5.395025in}{1.813884in}}%
\pgfpathlineto{\pgfqpoint{5.403961in}{1.813884in}}%
\pgfpathlineto{\pgfqpoint{5.403961in}{1.766991in}}%
\pgfpathlineto{\pgfqpoint{5.395025in}{1.766991in}}%
\pgfpathlineto{\pgfqpoint{5.395025in}{1.813884in}}%
\pgfpathclose%
\pgfusepath{fill}%
\end{pgfscope}%
\begin{pgfscope}%
\pgfpathrectangle{\pgfqpoint{3.722897in}{0.857143in}}{\pgfqpoint{2.627103in}{1.813434in}}%
\pgfusepath{clip}%
\pgfsetbuttcap%
\pgfsetmiterjoin%
\definecolor{currentfill}{rgb}{0.066899,0.263188,0.377594}%
\pgfsetfillcolor{currentfill}%
\pgfsetlinewidth{0.000000pt}%
\definecolor{currentstroke}{rgb}{0.000000,0.000000,0.000000}%
\pgfsetstrokecolor{currentstroke}%
\pgfsetstrokeopacity{0.000000}%
\pgfsetdash{}{0pt}%
\pgfpathmoveto{\pgfqpoint{5.406196in}{1.813884in}}%
\pgfpathlineto{\pgfqpoint{5.415132in}{1.813884in}}%
\pgfpathlineto{\pgfqpoint{5.415132in}{1.768912in}}%
\pgfpathlineto{\pgfqpoint{5.406196in}{1.768912in}}%
\pgfpathlineto{\pgfqpoint{5.406196in}{1.813884in}}%
\pgfpathclose%
\pgfusepath{fill}%
\end{pgfscope}%
\begin{pgfscope}%
\pgfpathrectangle{\pgfqpoint{3.722897in}{0.857143in}}{\pgfqpoint{2.627103in}{1.813434in}}%
\pgfusepath{clip}%
\pgfsetbuttcap%
\pgfsetmiterjoin%
\definecolor{currentfill}{rgb}{0.066899,0.263188,0.377594}%
\pgfsetfillcolor{currentfill}%
\pgfsetlinewidth{0.000000pt}%
\definecolor{currentstroke}{rgb}{0.000000,0.000000,0.000000}%
\pgfsetstrokecolor{currentstroke}%
\pgfsetstrokeopacity{0.000000}%
\pgfsetdash{}{0pt}%
\pgfpathmoveto{\pgfqpoint{5.417366in}{1.813884in}}%
\pgfpathlineto{\pgfqpoint{5.426303in}{1.813884in}}%
\pgfpathlineto{\pgfqpoint{5.426303in}{1.772453in}}%
\pgfpathlineto{\pgfqpoint{5.417366in}{1.772453in}}%
\pgfpathlineto{\pgfqpoint{5.417366in}{1.813884in}}%
\pgfpathclose%
\pgfusepath{fill}%
\end{pgfscope}%
\begin{pgfscope}%
\pgfpathrectangle{\pgfqpoint{3.722897in}{0.857143in}}{\pgfqpoint{2.627103in}{1.813434in}}%
\pgfusepath{clip}%
\pgfsetbuttcap%
\pgfsetmiterjoin%
\definecolor{currentfill}{rgb}{0.066899,0.263188,0.377594}%
\pgfsetfillcolor{currentfill}%
\pgfsetlinewidth{0.000000pt}%
\definecolor{currentstroke}{rgb}{0.000000,0.000000,0.000000}%
\pgfsetstrokecolor{currentstroke}%
\pgfsetstrokeopacity{0.000000}%
\pgfsetdash{}{0pt}%
\pgfpathmoveto{\pgfqpoint{5.428537in}{1.813884in}}%
\pgfpathlineto{\pgfqpoint{5.437473in}{1.813884in}}%
\pgfpathlineto{\pgfqpoint{5.437473in}{1.776981in}}%
\pgfpathlineto{\pgfqpoint{5.428537in}{1.776981in}}%
\pgfpathlineto{\pgfqpoint{5.428537in}{1.813884in}}%
\pgfpathclose%
\pgfusepath{fill}%
\end{pgfscope}%
\begin{pgfscope}%
\pgfpathrectangle{\pgfqpoint{3.722897in}{0.857143in}}{\pgfqpoint{2.627103in}{1.813434in}}%
\pgfusepath{clip}%
\pgfsetbuttcap%
\pgfsetmiterjoin%
\definecolor{currentfill}{rgb}{0.066899,0.263188,0.377594}%
\pgfsetfillcolor{currentfill}%
\pgfsetlinewidth{0.000000pt}%
\definecolor{currentstroke}{rgb}{0.000000,0.000000,0.000000}%
\pgfsetstrokecolor{currentstroke}%
\pgfsetstrokeopacity{0.000000}%
\pgfsetdash{}{0pt}%
\pgfpathmoveto{\pgfqpoint{5.439707in}{1.813884in}}%
\pgfpathlineto{\pgfqpoint{5.448644in}{1.813884in}}%
\pgfpathlineto{\pgfqpoint{5.448644in}{1.779233in}}%
\pgfpathlineto{\pgfqpoint{5.439707in}{1.779233in}}%
\pgfpathlineto{\pgfqpoint{5.439707in}{1.813884in}}%
\pgfpathclose%
\pgfusepath{fill}%
\end{pgfscope}%
\begin{pgfscope}%
\pgfpathrectangle{\pgfqpoint{3.722897in}{0.857143in}}{\pgfqpoint{2.627103in}{1.813434in}}%
\pgfusepath{clip}%
\pgfsetbuttcap%
\pgfsetmiterjoin%
\definecolor{currentfill}{rgb}{0.066899,0.263188,0.377594}%
\pgfsetfillcolor{currentfill}%
\pgfsetlinewidth{0.000000pt}%
\definecolor{currentstroke}{rgb}{0.000000,0.000000,0.000000}%
\pgfsetstrokecolor{currentstroke}%
\pgfsetstrokeopacity{0.000000}%
\pgfsetdash{}{0pt}%
\pgfpathmoveto{\pgfqpoint{5.450878in}{1.813884in}}%
\pgfpathlineto{\pgfqpoint{5.459814in}{1.813884in}}%
\pgfpathlineto{\pgfqpoint{5.459814in}{1.782487in}}%
\pgfpathlineto{\pgfqpoint{5.450878in}{1.782487in}}%
\pgfpathlineto{\pgfqpoint{5.450878in}{1.813884in}}%
\pgfpathclose%
\pgfusepath{fill}%
\end{pgfscope}%
\begin{pgfscope}%
\pgfpathrectangle{\pgfqpoint{3.722897in}{0.857143in}}{\pgfqpoint{2.627103in}{1.813434in}}%
\pgfusepath{clip}%
\pgfsetbuttcap%
\pgfsetmiterjoin%
\definecolor{currentfill}{rgb}{0.066899,0.263188,0.377594}%
\pgfsetfillcolor{currentfill}%
\pgfsetlinewidth{0.000000pt}%
\definecolor{currentstroke}{rgb}{0.000000,0.000000,0.000000}%
\pgfsetstrokecolor{currentstroke}%
\pgfsetstrokeopacity{0.000000}%
\pgfsetdash{}{0pt}%
\pgfpathmoveto{\pgfqpoint{5.462049in}{1.813884in}}%
\pgfpathlineto{\pgfqpoint{5.470985in}{1.813884in}}%
\pgfpathlineto{\pgfqpoint{5.470985in}{1.784002in}}%
\pgfpathlineto{\pgfqpoint{5.462049in}{1.784002in}}%
\pgfpathlineto{\pgfqpoint{5.462049in}{1.813884in}}%
\pgfpathclose%
\pgfusepath{fill}%
\end{pgfscope}%
\begin{pgfscope}%
\pgfpathrectangle{\pgfqpoint{3.722897in}{0.857143in}}{\pgfqpoint{2.627103in}{1.813434in}}%
\pgfusepath{clip}%
\pgfsetbuttcap%
\pgfsetmiterjoin%
\definecolor{currentfill}{rgb}{0.066899,0.263188,0.377594}%
\pgfsetfillcolor{currentfill}%
\pgfsetlinewidth{0.000000pt}%
\definecolor{currentstroke}{rgb}{0.000000,0.000000,0.000000}%
\pgfsetstrokecolor{currentstroke}%
\pgfsetstrokeopacity{0.000000}%
\pgfsetdash{}{0pt}%
\pgfpathmoveto{\pgfqpoint{5.473219in}{1.813884in}}%
\pgfpathlineto{\pgfqpoint{5.482156in}{1.813884in}}%
\pgfpathlineto{\pgfqpoint{5.482156in}{1.787752in}}%
\pgfpathlineto{\pgfqpoint{5.473219in}{1.787752in}}%
\pgfpathlineto{\pgfqpoint{5.473219in}{1.813884in}}%
\pgfpathclose%
\pgfusepath{fill}%
\end{pgfscope}%
\begin{pgfscope}%
\pgfpathrectangle{\pgfqpoint{3.722897in}{0.857143in}}{\pgfqpoint{2.627103in}{1.813434in}}%
\pgfusepath{clip}%
\pgfsetbuttcap%
\pgfsetmiterjoin%
\definecolor{currentfill}{rgb}{0.066899,0.263188,0.377594}%
\pgfsetfillcolor{currentfill}%
\pgfsetlinewidth{0.000000pt}%
\definecolor{currentstroke}{rgb}{0.000000,0.000000,0.000000}%
\pgfsetstrokecolor{currentstroke}%
\pgfsetstrokeopacity{0.000000}%
\pgfsetdash{}{0pt}%
\pgfpathmoveto{\pgfqpoint{5.484390in}{1.813884in}}%
\pgfpathlineto{\pgfqpoint{5.493326in}{1.813884in}}%
\pgfpathlineto{\pgfqpoint{5.493326in}{1.791919in}}%
\pgfpathlineto{\pgfqpoint{5.484390in}{1.791919in}}%
\pgfpathlineto{\pgfqpoint{5.484390in}{1.813884in}}%
\pgfpathclose%
\pgfusepath{fill}%
\end{pgfscope}%
\begin{pgfscope}%
\pgfpathrectangle{\pgfqpoint{3.722897in}{0.857143in}}{\pgfqpoint{2.627103in}{1.813434in}}%
\pgfusepath{clip}%
\pgfsetbuttcap%
\pgfsetmiterjoin%
\definecolor{currentfill}{rgb}{0.066899,0.263188,0.377594}%
\pgfsetfillcolor{currentfill}%
\pgfsetlinewidth{0.000000pt}%
\definecolor{currentstroke}{rgb}{0.000000,0.000000,0.000000}%
\pgfsetstrokecolor{currentstroke}%
\pgfsetstrokeopacity{0.000000}%
\pgfsetdash{}{0pt}%
\pgfpathmoveto{\pgfqpoint{5.495560in}{1.813884in}}%
\pgfpathlineto{\pgfqpoint{5.504497in}{1.813884in}}%
\pgfpathlineto{\pgfqpoint{5.504497in}{1.797064in}}%
\pgfpathlineto{\pgfqpoint{5.495560in}{1.797064in}}%
\pgfpathlineto{\pgfqpoint{5.495560in}{1.813884in}}%
\pgfpathclose%
\pgfusepath{fill}%
\end{pgfscope}%
\begin{pgfscope}%
\pgfpathrectangle{\pgfqpoint{3.722897in}{0.857143in}}{\pgfqpoint{2.627103in}{1.813434in}}%
\pgfusepath{clip}%
\pgfsetbuttcap%
\pgfsetmiterjoin%
\definecolor{currentfill}{rgb}{0.066899,0.263188,0.377594}%
\pgfsetfillcolor{currentfill}%
\pgfsetlinewidth{0.000000pt}%
\definecolor{currentstroke}{rgb}{0.000000,0.000000,0.000000}%
\pgfsetstrokecolor{currentstroke}%
\pgfsetstrokeopacity{0.000000}%
\pgfsetdash{}{0pt}%
\pgfpathmoveto{\pgfqpoint{5.506731in}{1.813884in}}%
\pgfpathlineto{\pgfqpoint{5.515667in}{1.813884in}}%
\pgfpathlineto{\pgfqpoint{5.515667in}{1.800732in}}%
\pgfpathlineto{\pgfqpoint{5.506731in}{1.800732in}}%
\pgfpathlineto{\pgfqpoint{5.506731in}{1.813884in}}%
\pgfpathclose%
\pgfusepath{fill}%
\end{pgfscope}%
\begin{pgfscope}%
\pgfpathrectangle{\pgfqpoint{3.722897in}{0.857143in}}{\pgfqpoint{2.627103in}{1.813434in}}%
\pgfusepath{clip}%
\pgfsetbuttcap%
\pgfsetmiterjoin%
\definecolor{currentfill}{rgb}{0.066899,0.263188,0.377594}%
\pgfsetfillcolor{currentfill}%
\pgfsetlinewidth{0.000000pt}%
\definecolor{currentstroke}{rgb}{0.000000,0.000000,0.000000}%
\pgfsetstrokecolor{currentstroke}%
\pgfsetstrokeopacity{0.000000}%
\pgfsetdash{}{0pt}%
\pgfpathmoveto{\pgfqpoint{5.517902in}{1.813884in}}%
\pgfpathlineto{\pgfqpoint{5.526838in}{1.813884in}}%
\pgfpathlineto{\pgfqpoint{5.526838in}{1.803338in}}%
\pgfpathlineto{\pgfqpoint{5.517902in}{1.803338in}}%
\pgfpathlineto{\pgfqpoint{5.517902in}{1.813884in}}%
\pgfpathclose%
\pgfusepath{fill}%
\end{pgfscope}%
\begin{pgfscope}%
\pgfpathrectangle{\pgfqpoint{3.722897in}{0.857143in}}{\pgfqpoint{2.627103in}{1.813434in}}%
\pgfusepath{clip}%
\pgfsetbuttcap%
\pgfsetmiterjoin%
\definecolor{currentfill}{rgb}{0.066899,0.263188,0.377594}%
\pgfsetfillcolor{currentfill}%
\pgfsetlinewidth{0.000000pt}%
\definecolor{currentstroke}{rgb}{0.000000,0.000000,0.000000}%
\pgfsetstrokecolor{currentstroke}%
\pgfsetstrokeopacity{0.000000}%
\pgfsetdash{}{0pt}%
\pgfpathmoveto{\pgfqpoint{5.529072in}{1.813884in}}%
\pgfpathlineto{\pgfqpoint{5.538009in}{1.813884in}}%
\pgfpathlineto{\pgfqpoint{5.538009in}{1.807007in}}%
\pgfpathlineto{\pgfqpoint{5.529072in}{1.807007in}}%
\pgfpathlineto{\pgfqpoint{5.529072in}{1.813884in}}%
\pgfpathclose%
\pgfusepath{fill}%
\end{pgfscope}%
\begin{pgfscope}%
\pgfpathrectangle{\pgfqpoint{3.722897in}{0.857143in}}{\pgfqpoint{2.627103in}{1.813434in}}%
\pgfusepath{clip}%
\pgfsetbuttcap%
\pgfsetmiterjoin%
\definecolor{currentfill}{rgb}{0.066899,0.263188,0.377594}%
\pgfsetfillcolor{currentfill}%
\pgfsetlinewidth{0.000000pt}%
\definecolor{currentstroke}{rgb}{0.000000,0.000000,0.000000}%
\pgfsetstrokecolor{currentstroke}%
\pgfsetstrokeopacity{0.000000}%
\pgfsetdash{}{0pt}%
\pgfpathmoveto{\pgfqpoint{5.540243in}{1.813884in}}%
\pgfpathlineto{\pgfqpoint{5.549179in}{1.813884in}}%
\pgfpathlineto{\pgfqpoint{5.549179in}{1.809989in}}%
\pgfpathlineto{\pgfqpoint{5.540243in}{1.809989in}}%
\pgfpathlineto{\pgfqpoint{5.540243in}{1.813884in}}%
\pgfpathclose%
\pgfusepath{fill}%
\end{pgfscope}%
\begin{pgfscope}%
\pgfpathrectangle{\pgfqpoint{3.722897in}{0.857143in}}{\pgfqpoint{2.627103in}{1.813434in}}%
\pgfusepath{clip}%
\pgfsetbuttcap%
\pgfsetmiterjoin%
\definecolor{currentfill}{rgb}{0.066899,0.263188,0.377594}%
\pgfsetfillcolor{currentfill}%
\pgfsetlinewidth{0.000000pt}%
\definecolor{currentstroke}{rgb}{0.000000,0.000000,0.000000}%
\pgfsetstrokecolor{currentstroke}%
\pgfsetstrokeopacity{0.000000}%
\pgfsetdash{}{0pt}%
\pgfpathmoveto{\pgfqpoint{5.551413in}{1.813884in}}%
\pgfpathlineto{\pgfqpoint{5.560350in}{1.813884in}}%
\pgfpathlineto{\pgfqpoint{5.560350in}{1.812969in}}%
\pgfpathlineto{\pgfqpoint{5.551413in}{1.812969in}}%
\pgfpathlineto{\pgfqpoint{5.551413in}{1.813884in}}%
\pgfpathclose%
\pgfusepath{fill}%
\end{pgfscope}%
\begin{pgfscope}%
\pgfpathrectangle{\pgfqpoint{3.722897in}{0.857143in}}{\pgfqpoint{2.627103in}{1.813434in}}%
\pgfusepath{clip}%
\pgfsetbuttcap%
\pgfsetmiterjoin%
\definecolor{currentfill}{rgb}{0.066899,0.263188,0.377594}%
\pgfsetfillcolor{currentfill}%
\pgfsetlinewidth{0.000000pt}%
\definecolor{currentstroke}{rgb}{0.000000,0.000000,0.000000}%
\pgfsetstrokecolor{currentstroke}%
\pgfsetstrokeopacity{0.000000}%
\pgfsetdash{}{0pt}%
\pgfpathmoveto{\pgfqpoint{5.562584in}{1.813884in}}%
\pgfpathlineto{\pgfqpoint{5.571521in}{1.813884in}}%
\pgfpathlineto{\pgfqpoint{5.571521in}{1.816719in}}%
\pgfpathlineto{\pgfqpoint{5.562584in}{1.816719in}}%
\pgfpathlineto{\pgfqpoint{5.562584in}{1.813884in}}%
\pgfpathclose%
\pgfusepath{fill}%
\end{pgfscope}%
\begin{pgfscope}%
\pgfpathrectangle{\pgfqpoint{3.722897in}{0.857143in}}{\pgfqpoint{2.627103in}{1.813434in}}%
\pgfusepath{clip}%
\pgfsetbuttcap%
\pgfsetmiterjoin%
\definecolor{currentfill}{rgb}{0.066899,0.263188,0.377594}%
\pgfsetfillcolor{currentfill}%
\pgfsetlinewidth{0.000000pt}%
\definecolor{currentstroke}{rgb}{0.000000,0.000000,0.000000}%
\pgfsetstrokecolor{currentstroke}%
\pgfsetstrokeopacity{0.000000}%
\pgfsetdash{}{0pt}%
\pgfpathmoveto{\pgfqpoint{5.573755in}{1.813884in}}%
\pgfpathlineto{\pgfqpoint{5.582691in}{1.813884in}}%
\pgfpathlineto{\pgfqpoint{5.582691in}{1.818524in}}%
\pgfpathlineto{\pgfqpoint{5.573755in}{1.818524in}}%
\pgfpathlineto{\pgfqpoint{5.573755in}{1.813884in}}%
\pgfpathclose%
\pgfusepath{fill}%
\end{pgfscope}%
\begin{pgfscope}%
\pgfpathrectangle{\pgfqpoint{3.722897in}{0.857143in}}{\pgfqpoint{2.627103in}{1.813434in}}%
\pgfusepath{clip}%
\pgfsetbuttcap%
\pgfsetmiterjoin%
\definecolor{currentfill}{rgb}{0.066899,0.263188,0.377594}%
\pgfsetfillcolor{currentfill}%
\pgfsetlinewidth{0.000000pt}%
\definecolor{currentstroke}{rgb}{0.000000,0.000000,0.000000}%
\pgfsetstrokecolor{currentstroke}%
\pgfsetstrokeopacity{0.000000}%
\pgfsetdash{}{0pt}%
\pgfpathmoveto{\pgfqpoint{5.584925in}{1.813884in}}%
\pgfpathlineto{\pgfqpoint{5.593862in}{1.813884in}}%
\pgfpathlineto{\pgfqpoint{5.593862in}{1.821396in}}%
\pgfpathlineto{\pgfqpoint{5.584925in}{1.821396in}}%
\pgfpathlineto{\pgfqpoint{5.584925in}{1.813884in}}%
\pgfpathclose%
\pgfusepath{fill}%
\end{pgfscope}%
\begin{pgfscope}%
\pgfpathrectangle{\pgfqpoint{3.722897in}{0.857143in}}{\pgfqpoint{2.627103in}{1.813434in}}%
\pgfusepath{clip}%
\pgfsetbuttcap%
\pgfsetmiterjoin%
\definecolor{currentfill}{rgb}{0.066899,0.263188,0.377594}%
\pgfsetfillcolor{currentfill}%
\pgfsetlinewidth{0.000000pt}%
\definecolor{currentstroke}{rgb}{0.000000,0.000000,0.000000}%
\pgfsetstrokecolor{currentstroke}%
\pgfsetstrokeopacity{0.000000}%
\pgfsetdash{}{0pt}%
\pgfpathmoveto{\pgfqpoint{5.596096in}{1.813884in}}%
\pgfpathlineto{\pgfqpoint{5.605032in}{1.813884in}}%
\pgfpathlineto{\pgfqpoint{5.605032in}{1.822688in}}%
\pgfpathlineto{\pgfqpoint{5.596096in}{1.822688in}}%
\pgfpathlineto{\pgfqpoint{5.596096in}{1.813884in}}%
\pgfpathclose%
\pgfusepath{fill}%
\end{pgfscope}%
\begin{pgfscope}%
\pgfpathrectangle{\pgfqpoint{3.722897in}{0.857143in}}{\pgfqpoint{2.627103in}{1.813434in}}%
\pgfusepath{clip}%
\pgfsetbuttcap%
\pgfsetmiterjoin%
\definecolor{currentfill}{rgb}{0.066899,0.263188,0.377594}%
\pgfsetfillcolor{currentfill}%
\pgfsetlinewidth{0.000000pt}%
\definecolor{currentstroke}{rgb}{0.000000,0.000000,0.000000}%
\pgfsetstrokecolor{currentstroke}%
\pgfsetstrokeopacity{0.000000}%
\pgfsetdash{}{0pt}%
\pgfpathmoveto{\pgfqpoint{5.607266in}{1.813884in}}%
\pgfpathlineto{\pgfqpoint{5.616203in}{1.813884in}}%
\pgfpathlineto{\pgfqpoint{5.616203in}{1.824522in}}%
\pgfpathlineto{\pgfqpoint{5.607266in}{1.824522in}}%
\pgfpathlineto{\pgfqpoint{5.607266in}{1.813884in}}%
\pgfpathclose%
\pgfusepath{fill}%
\end{pgfscope}%
\begin{pgfscope}%
\pgfpathrectangle{\pgfqpoint{3.722897in}{0.857143in}}{\pgfqpoint{2.627103in}{1.813434in}}%
\pgfusepath{clip}%
\pgfsetbuttcap%
\pgfsetmiterjoin%
\definecolor{currentfill}{rgb}{0.066899,0.263188,0.377594}%
\pgfsetfillcolor{currentfill}%
\pgfsetlinewidth{0.000000pt}%
\definecolor{currentstroke}{rgb}{0.000000,0.000000,0.000000}%
\pgfsetstrokecolor{currentstroke}%
\pgfsetstrokeopacity{0.000000}%
\pgfsetdash{}{0pt}%
\pgfpathmoveto{\pgfqpoint{5.618437in}{1.813884in}}%
\pgfpathlineto{\pgfqpoint{5.627374in}{1.813884in}}%
\pgfpathlineto{\pgfqpoint{5.627374in}{1.825573in}}%
\pgfpathlineto{\pgfqpoint{5.618437in}{1.825573in}}%
\pgfpathlineto{\pgfqpoint{5.618437in}{1.813884in}}%
\pgfpathclose%
\pgfusepath{fill}%
\end{pgfscope}%
\begin{pgfscope}%
\pgfpathrectangle{\pgfqpoint{3.722897in}{0.857143in}}{\pgfqpoint{2.627103in}{1.813434in}}%
\pgfusepath{clip}%
\pgfsetbuttcap%
\pgfsetmiterjoin%
\definecolor{currentfill}{rgb}{0.066899,0.263188,0.377594}%
\pgfsetfillcolor{currentfill}%
\pgfsetlinewidth{0.000000pt}%
\definecolor{currentstroke}{rgb}{0.000000,0.000000,0.000000}%
\pgfsetstrokecolor{currentstroke}%
\pgfsetstrokeopacity{0.000000}%
\pgfsetdash{}{0pt}%
\pgfpathmoveto{\pgfqpoint{5.629608in}{1.813884in}}%
\pgfpathlineto{\pgfqpoint{5.638544in}{1.813884in}}%
\pgfpathlineto{\pgfqpoint{5.638544in}{1.825735in}}%
\pgfpathlineto{\pgfqpoint{5.629608in}{1.825735in}}%
\pgfpathlineto{\pgfqpoint{5.629608in}{1.813884in}}%
\pgfpathclose%
\pgfusepath{fill}%
\end{pgfscope}%
\begin{pgfscope}%
\pgfpathrectangle{\pgfqpoint{3.722897in}{0.857143in}}{\pgfqpoint{2.627103in}{1.813434in}}%
\pgfusepath{clip}%
\pgfsetbuttcap%
\pgfsetmiterjoin%
\definecolor{currentfill}{rgb}{0.066899,0.263188,0.377594}%
\pgfsetfillcolor{currentfill}%
\pgfsetlinewidth{0.000000pt}%
\definecolor{currentstroke}{rgb}{0.000000,0.000000,0.000000}%
\pgfsetstrokecolor{currentstroke}%
\pgfsetstrokeopacity{0.000000}%
\pgfsetdash{}{0pt}%
\pgfpathmoveto{\pgfqpoint{5.640778in}{1.813884in}}%
\pgfpathlineto{\pgfqpoint{5.649715in}{1.813884in}}%
\pgfpathlineto{\pgfqpoint{5.649715in}{1.826902in}}%
\pgfpathlineto{\pgfqpoint{5.640778in}{1.826902in}}%
\pgfpathlineto{\pgfqpoint{5.640778in}{1.813884in}}%
\pgfpathclose%
\pgfusepath{fill}%
\end{pgfscope}%
\begin{pgfscope}%
\pgfpathrectangle{\pgfqpoint{3.722897in}{0.857143in}}{\pgfqpoint{2.627103in}{1.813434in}}%
\pgfusepath{clip}%
\pgfsetbuttcap%
\pgfsetmiterjoin%
\definecolor{currentfill}{rgb}{0.066899,0.263188,0.377594}%
\pgfsetfillcolor{currentfill}%
\pgfsetlinewidth{0.000000pt}%
\definecolor{currentstroke}{rgb}{0.000000,0.000000,0.000000}%
\pgfsetstrokecolor{currentstroke}%
\pgfsetstrokeopacity{0.000000}%
\pgfsetdash{}{0pt}%
\pgfpathmoveto{\pgfqpoint{5.651949in}{1.813884in}}%
\pgfpathlineto{\pgfqpoint{5.660885in}{1.813884in}}%
\pgfpathlineto{\pgfqpoint{5.660885in}{1.827792in}}%
\pgfpathlineto{\pgfqpoint{5.651949in}{1.827792in}}%
\pgfpathlineto{\pgfqpoint{5.651949in}{1.813884in}}%
\pgfpathclose%
\pgfusepath{fill}%
\end{pgfscope}%
\begin{pgfscope}%
\pgfpathrectangle{\pgfqpoint{3.722897in}{0.857143in}}{\pgfqpoint{2.627103in}{1.813434in}}%
\pgfusepath{clip}%
\pgfsetbuttcap%
\pgfsetmiterjoin%
\definecolor{currentfill}{rgb}{0.066899,0.263188,0.377594}%
\pgfsetfillcolor{currentfill}%
\pgfsetlinewidth{0.000000pt}%
\definecolor{currentstroke}{rgb}{0.000000,0.000000,0.000000}%
\pgfsetstrokecolor{currentstroke}%
\pgfsetstrokeopacity{0.000000}%
\pgfsetdash{}{0pt}%
\pgfpathmoveto{\pgfqpoint{5.663119in}{1.813884in}}%
\pgfpathlineto{\pgfqpoint{5.672056in}{1.813884in}}%
\pgfpathlineto{\pgfqpoint{5.672056in}{1.828191in}}%
\pgfpathlineto{\pgfqpoint{5.663119in}{1.828191in}}%
\pgfpathlineto{\pgfqpoint{5.663119in}{1.813884in}}%
\pgfpathclose%
\pgfusepath{fill}%
\end{pgfscope}%
\begin{pgfscope}%
\pgfpathrectangle{\pgfqpoint{3.722897in}{0.857143in}}{\pgfqpoint{2.627103in}{1.813434in}}%
\pgfusepath{clip}%
\pgfsetbuttcap%
\pgfsetmiterjoin%
\definecolor{currentfill}{rgb}{0.066899,0.263188,0.377594}%
\pgfsetfillcolor{currentfill}%
\pgfsetlinewidth{0.000000pt}%
\definecolor{currentstroke}{rgb}{0.000000,0.000000,0.000000}%
\pgfsetstrokecolor{currentstroke}%
\pgfsetstrokeopacity{0.000000}%
\pgfsetdash{}{0pt}%
\pgfpathmoveto{\pgfqpoint{5.674290in}{1.813884in}}%
\pgfpathlineto{\pgfqpoint{5.683227in}{1.813884in}}%
\pgfpathlineto{\pgfqpoint{5.683227in}{1.829897in}}%
\pgfpathlineto{\pgfqpoint{5.674290in}{1.829897in}}%
\pgfpathlineto{\pgfqpoint{5.674290in}{1.813884in}}%
\pgfpathclose%
\pgfusepath{fill}%
\end{pgfscope}%
\begin{pgfscope}%
\pgfpathrectangle{\pgfqpoint{3.722897in}{0.857143in}}{\pgfqpoint{2.627103in}{1.813434in}}%
\pgfusepath{clip}%
\pgfsetbuttcap%
\pgfsetmiterjoin%
\definecolor{currentfill}{rgb}{0.066899,0.263188,0.377594}%
\pgfsetfillcolor{currentfill}%
\pgfsetlinewidth{0.000000pt}%
\definecolor{currentstroke}{rgb}{0.000000,0.000000,0.000000}%
\pgfsetstrokecolor{currentstroke}%
\pgfsetstrokeopacity{0.000000}%
\pgfsetdash{}{0pt}%
\pgfpathmoveto{\pgfqpoint{5.685461in}{1.813884in}}%
\pgfpathlineto{\pgfqpoint{5.694397in}{1.813884in}}%
\pgfpathlineto{\pgfqpoint{5.694397in}{1.831783in}}%
\pgfpathlineto{\pgfqpoint{5.685461in}{1.831783in}}%
\pgfpathlineto{\pgfqpoint{5.685461in}{1.813884in}}%
\pgfpathclose%
\pgfusepath{fill}%
\end{pgfscope}%
\begin{pgfscope}%
\pgfpathrectangle{\pgfqpoint{3.722897in}{0.857143in}}{\pgfqpoint{2.627103in}{1.813434in}}%
\pgfusepath{clip}%
\pgfsetbuttcap%
\pgfsetmiterjoin%
\definecolor{currentfill}{rgb}{0.066899,0.263188,0.377594}%
\pgfsetfillcolor{currentfill}%
\pgfsetlinewidth{0.000000pt}%
\definecolor{currentstroke}{rgb}{0.000000,0.000000,0.000000}%
\pgfsetstrokecolor{currentstroke}%
\pgfsetstrokeopacity{0.000000}%
\pgfsetdash{}{0pt}%
\pgfpathmoveto{\pgfqpoint{5.696631in}{1.813884in}}%
\pgfpathlineto{\pgfqpoint{5.705568in}{1.813884in}}%
\pgfpathlineto{\pgfqpoint{5.705568in}{1.831701in}}%
\pgfpathlineto{\pgfqpoint{5.696631in}{1.831701in}}%
\pgfpathlineto{\pgfqpoint{5.696631in}{1.813884in}}%
\pgfpathclose%
\pgfusepath{fill}%
\end{pgfscope}%
\begin{pgfscope}%
\pgfpathrectangle{\pgfqpoint{3.722897in}{0.857143in}}{\pgfqpoint{2.627103in}{1.813434in}}%
\pgfusepath{clip}%
\pgfsetbuttcap%
\pgfsetmiterjoin%
\definecolor{currentfill}{rgb}{0.066899,0.263188,0.377594}%
\pgfsetfillcolor{currentfill}%
\pgfsetlinewidth{0.000000pt}%
\definecolor{currentstroke}{rgb}{0.000000,0.000000,0.000000}%
\pgfsetstrokecolor{currentstroke}%
\pgfsetstrokeopacity{0.000000}%
\pgfsetdash{}{0pt}%
\pgfpathmoveto{\pgfqpoint{5.707802in}{1.813884in}}%
\pgfpathlineto{\pgfqpoint{5.716738in}{1.813884in}}%
\pgfpathlineto{\pgfqpoint{5.716738in}{1.834304in}}%
\pgfpathlineto{\pgfqpoint{5.707802in}{1.834304in}}%
\pgfpathlineto{\pgfqpoint{5.707802in}{1.813884in}}%
\pgfpathclose%
\pgfusepath{fill}%
\end{pgfscope}%
\begin{pgfscope}%
\pgfpathrectangle{\pgfqpoint{3.722897in}{0.857143in}}{\pgfqpoint{2.627103in}{1.813434in}}%
\pgfusepath{clip}%
\pgfsetbuttcap%
\pgfsetmiterjoin%
\definecolor{currentfill}{rgb}{0.066899,0.263188,0.377594}%
\pgfsetfillcolor{currentfill}%
\pgfsetlinewidth{0.000000pt}%
\definecolor{currentstroke}{rgb}{0.000000,0.000000,0.000000}%
\pgfsetstrokecolor{currentstroke}%
\pgfsetstrokeopacity{0.000000}%
\pgfsetdash{}{0pt}%
\pgfpathmoveto{\pgfqpoint{5.718972in}{1.813884in}}%
\pgfpathlineto{\pgfqpoint{5.727909in}{1.813884in}}%
\pgfpathlineto{\pgfqpoint{5.727909in}{1.835767in}}%
\pgfpathlineto{\pgfqpoint{5.718972in}{1.835767in}}%
\pgfpathlineto{\pgfqpoint{5.718972in}{1.813884in}}%
\pgfpathclose%
\pgfusepath{fill}%
\end{pgfscope}%
\begin{pgfscope}%
\pgfpathrectangle{\pgfqpoint{3.722897in}{0.857143in}}{\pgfqpoint{2.627103in}{1.813434in}}%
\pgfusepath{clip}%
\pgfsetbuttcap%
\pgfsetmiterjoin%
\definecolor{currentfill}{rgb}{0.066899,0.263188,0.377594}%
\pgfsetfillcolor{currentfill}%
\pgfsetlinewidth{0.000000pt}%
\definecolor{currentstroke}{rgb}{0.000000,0.000000,0.000000}%
\pgfsetstrokecolor{currentstroke}%
\pgfsetstrokeopacity{0.000000}%
\pgfsetdash{}{0pt}%
\pgfpathmoveto{\pgfqpoint{5.730143in}{1.813884in}}%
\pgfpathlineto{\pgfqpoint{5.739080in}{1.813884in}}%
\pgfpathlineto{\pgfqpoint{5.739080in}{1.836894in}}%
\pgfpathlineto{\pgfqpoint{5.730143in}{1.836894in}}%
\pgfpathlineto{\pgfqpoint{5.730143in}{1.813884in}}%
\pgfpathclose%
\pgfusepath{fill}%
\end{pgfscope}%
\begin{pgfscope}%
\pgfpathrectangle{\pgfqpoint{3.722897in}{0.857143in}}{\pgfqpoint{2.627103in}{1.813434in}}%
\pgfusepath{clip}%
\pgfsetbuttcap%
\pgfsetmiterjoin%
\definecolor{currentfill}{rgb}{0.066899,0.263188,0.377594}%
\pgfsetfillcolor{currentfill}%
\pgfsetlinewidth{0.000000pt}%
\definecolor{currentstroke}{rgb}{0.000000,0.000000,0.000000}%
\pgfsetstrokecolor{currentstroke}%
\pgfsetstrokeopacity{0.000000}%
\pgfsetdash{}{0pt}%
\pgfpathmoveto{\pgfqpoint{5.741314in}{1.813884in}}%
\pgfpathlineto{\pgfqpoint{5.750250in}{1.813884in}}%
\pgfpathlineto{\pgfqpoint{5.750250in}{1.840713in}}%
\pgfpathlineto{\pgfqpoint{5.741314in}{1.840713in}}%
\pgfpathlineto{\pgfqpoint{5.741314in}{1.813884in}}%
\pgfpathclose%
\pgfusepath{fill}%
\end{pgfscope}%
\begin{pgfscope}%
\pgfpathrectangle{\pgfqpoint{3.722897in}{0.857143in}}{\pgfqpoint{2.627103in}{1.813434in}}%
\pgfusepath{clip}%
\pgfsetbuttcap%
\pgfsetmiterjoin%
\definecolor{currentfill}{rgb}{0.066899,0.263188,0.377594}%
\pgfsetfillcolor{currentfill}%
\pgfsetlinewidth{0.000000pt}%
\definecolor{currentstroke}{rgb}{0.000000,0.000000,0.000000}%
\pgfsetstrokecolor{currentstroke}%
\pgfsetstrokeopacity{0.000000}%
\pgfsetdash{}{0pt}%
\pgfpathmoveto{\pgfqpoint{5.752484in}{1.813884in}}%
\pgfpathlineto{\pgfqpoint{5.761421in}{1.813884in}}%
\pgfpathlineto{\pgfqpoint{5.761421in}{1.846521in}}%
\pgfpathlineto{\pgfqpoint{5.752484in}{1.846521in}}%
\pgfpathlineto{\pgfqpoint{5.752484in}{1.813884in}}%
\pgfpathclose%
\pgfusepath{fill}%
\end{pgfscope}%
\begin{pgfscope}%
\pgfpathrectangle{\pgfqpoint{3.722897in}{0.857143in}}{\pgfqpoint{2.627103in}{1.813434in}}%
\pgfusepath{clip}%
\pgfsetbuttcap%
\pgfsetmiterjoin%
\definecolor{currentfill}{rgb}{0.066899,0.263188,0.377594}%
\pgfsetfillcolor{currentfill}%
\pgfsetlinewidth{0.000000pt}%
\definecolor{currentstroke}{rgb}{0.000000,0.000000,0.000000}%
\pgfsetstrokecolor{currentstroke}%
\pgfsetstrokeopacity{0.000000}%
\pgfsetdash{}{0pt}%
\pgfpathmoveto{\pgfqpoint{5.763655in}{1.813884in}}%
\pgfpathlineto{\pgfqpoint{5.772591in}{1.813884in}}%
\pgfpathlineto{\pgfqpoint{5.772591in}{1.851560in}}%
\pgfpathlineto{\pgfqpoint{5.763655in}{1.851560in}}%
\pgfpathlineto{\pgfqpoint{5.763655in}{1.813884in}}%
\pgfpathclose%
\pgfusepath{fill}%
\end{pgfscope}%
\begin{pgfscope}%
\pgfpathrectangle{\pgfqpoint{3.722897in}{0.857143in}}{\pgfqpoint{2.627103in}{1.813434in}}%
\pgfusepath{clip}%
\pgfsetbuttcap%
\pgfsetmiterjoin%
\definecolor{currentfill}{rgb}{0.066899,0.263188,0.377594}%
\pgfsetfillcolor{currentfill}%
\pgfsetlinewidth{0.000000pt}%
\definecolor{currentstroke}{rgb}{0.000000,0.000000,0.000000}%
\pgfsetstrokecolor{currentstroke}%
\pgfsetstrokeopacity{0.000000}%
\pgfsetdash{}{0pt}%
\pgfpathmoveto{\pgfqpoint{5.774826in}{1.813884in}}%
\pgfpathlineto{\pgfqpoint{5.783762in}{1.813884in}}%
\pgfpathlineto{\pgfqpoint{5.783762in}{1.856119in}}%
\pgfpathlineto{\pgfqpoint{5.774826in}{1.856119in}}%
\pgfpathlineto{\pgfqpoint{5.774826in}{1.813884in}}%
\pgfpathclose%
\pgfusepath{fill}%
\end{pgfscope}%
\begin{pgfscope}%
\pgfpathrectangle{\pgfqpoint{3.722897in}{0.857143in}}{\pgfqpoint{2.627103in}{1.813434in}}%
\pgfusepath{clip}%
\pgfsetbuttcap%
\pgfsetmiterjoin%
\definecolor{currentfill}{rgb}{0.066899,0.263188,0.377594}%
\pgfsetfillcolor{currentfill}%
\pgfsetlinewidth{0.000000pt}%
\definecolor{currentstroke}{rgb}{0.000000,0.000000,0.000000}%
\pgfsetstrokecolor{currentstroke}%
\pgfsetstrokeopacity{0.000000}%
\pgfsetdash{}{0pt}%
\pgfpathmoveto{\pgfqpoint{5.785996in}{1.813884in}}%
\pgfpathlineto{\pgfqpoint{5.794933in}{1.813884in}}%
\pgfpathlineto{\pgfqpoint{5.794933in}{1.858784in}}%
\pgfpathlineto{\pgfqpoint{5.785996in}{1.858784in}}%
\pgfpathlineto{\pgfqpoint{5.785996in}{1.813884in}}%
\pgfpathclose%
\pgfusepath{fill}%
\end{pgfscope}%
\begin{pgfscope}%
\pgfpathrectangle{\pgfqpoint{3.722897in}{0.857143in}}{\pgfqpoint{2.627103in}{1.813434in}}%
\pgfusepath{clip}%
\pgfsetbuttcap%
\pgfsetmiterjoin%
\definecolor{currentfill}{rgb}{0.066899,0.263188,0.377594}%
\pgfsetfillcolor{currentfill}%
\pgfsetlinewidth{0.000000pt}%
\definecolor{currentstroke}{rgb}{0.000000,0.000000,0.000000}%
\pgfsetstrokecolor{currentstroke}%
\pgfsetstrokeopacity{0.000000}%
\pgfsetdash{}{0pt}%
\pgfpathmoveto{\pgfqpoint{5.797167in}{1.813884in}}%
\pgfpathlineto{\pgfqpoint{5.806103in}{1.813884in}}%
\pgfpathlineto{\pgfqpoint{5.806103in}{1.861690in}}%
\pgfpathlineto{\pgfqpoint{5.797167in}{1.861690in}}%
\pgfpathlineto{\pgfqpoint{5.797167in}{1.813884in}}%
\pgfpathclose%
\pgfusepath{fill}%
\end{pgfscope}%
\begin{pgfscope}%
\pgfpathrectangle{\pgfqpoint{3.722897in}{0.857143in}}{\pgfqpoint{2.627103in}{1.813434in}}%
\pgfusepath{clip}%
\pgfsetbuttcap%
\pgfsetmiterjoin%
\definecolor{currentfill}{rgb}{0.066899,0.263188,0.377594}%
\pgfsetfillcolor{currentfill}%
\pgfsetlinewidth{0.000000pt}%
\definecolor{currentstroke}{rgb}{0.000000,0.000000,0.000000}%
\pgfsetstrokecolor{currentstroke}%
\pgfsetstrokeopacity{0.000000}%
\pgfsetdash{}{0pt}%
\pgfpathmoveto{\pgfqpoint{5.808337in}{1.813884in}}%
\pgfpathlineto{\pgfqpoint{5.817274in}{1.813884in}}%
\pgfpathlineto{\pgfqpoint{5.817274in}{1.864576in}}%
\pgfpathlineto{\pgfqpoint{5.808337in}{1.864576in}}%
\pgfpathlineto{\pgfqpoint{5.808337in}{1.813884in}}%
\pgfpathclose%
\pgfusepath{fill}%
\end{pgfscope}%
\begin{pgfscope}%
\pgfpathrectangle{\pgfqpoint{3.722897in}{0.857143in}}{\pgfqpoint{2.627103in}{1.813434in}}%
\pgfusepath{clip}%
\pgfsetbuttcap%
\pgfsetmiterjoin%
\definecolor{currentfill}{rgb}{0.066899,0.263188,0.377594}%
\pgfsetfillcolor{currentfill}%
\pgfsetlinewidth{0.000000pt}%
\definecolor{currentstroke}{rgb}{0.000000,0.000000,0.000000}%
\pgfsetstrokecolor{currentstroke}%
\pgfsetstrokeopacity{0.000000}%
\pgfsetdash{}{0pt}%
\pgfpathmoveto{\pgfqpoint{5.819508in}{1.813884in}}%
\pgfpathlineto{\pgfqpoint{5.828444in}{1.813884in}}%
\pgfpathlineto{\pgfqpoint{5.828444in}{1.866662in}}%
\pgfpathlineto{\pgfqpoint{5.819508in}{1.866662in}}%
\pgfpathlineto{\pgfqpoint{5.819508in}{1.813884in}}%
\pgfpathclose%
\pgfusepath{fill}%
\end{pgfscope}%
\begin{pgfscope}%
\pgfpathrectangle{\pgfqpoint{3.722897in}{0.857143in}}{\pgfqpoint{2.627103in}{1.813434in}}%
\pgfusepath{clip}%
\pgfsetbuttcap%
\pgfsetmiterjoin%
\definecolor{currentfill}{rgb}{0.066899,0.263188,0.377594}%
\pgfsetfillcolor{currentfill}%
\pgfsetlinewidth{0.000000pt}%
\definecolor{currentstroke}{rgb}{0.000000,0.000000,0.000000}%
\pgfsetstrokecolor{currentstroke}%
\pgfsetstrokeopacity{0.000000}%
\pgfsetdash{}{0pt}%
\pgfpathmoveto{\pgfqpoint{5.830679in}{1.813884in}}%
\pgfpathlineto{\pgfqpoint{5.839615in}{1.813884in}}%
\pgfpathlineto{\pgfqpoint{5.839615in}{1.867454in}}%
\pgfpathlineto{\pgfqpoint{5.830679in}{1.867454in}}%
\pgfpathlineto{\pgfqpoint{5.830679in}{1.813884in}}%
\pgfpathclose%
\pgfusepath{fill}%
\end{pgfscope}%
\begin{pgfscope}%
\pgfpathrectangle{\pgfqpoint{3.722897in}{0.857143in}}{\pgfqpoint{2.627103in}{1.813434in}}%
\pgfusepath{clip}%
\pgfsetbuttcap%
\pgfsetmiterjoin%
\definecolor{currentfill}{rgb}{0.066899,0.263188,0.377594}%
\pgfsetfillcolor{currentfill}%
\pgfsetlinewidth{0.000000pt}%
\definecolor{currentstroke}{rgb}{0.000000,0.000000,0.000000}%
\pgfsetstrokecolor{currentstroke}%
\pgfsetstrokeopacity{0.000000}%
\pgfsetdash{}{0pt}%
\pgfpathmoveto{\pgfqpoint{5.841849in}{1.813884in}}%
\pgfpathlineto{\pgfqpoint{5.850786in}{1.813884in}}%
\pgfpathlineto{\pgfqpoint{5.850786in}{1.867941in}}%
\pgfpathlineto{\pgfqpoint{5.841849in}{1.867941in}}%
\pgfpathlineto{\pgfqpoint{5.841849in}{1.813884in}}%
\pgfpathclose%
\pgfusepath{fill}%
\end{pgfscope}%
\begin{pgfscope}%
\pgfpathrectangle{\pgfqpoint{3.722897in}{0.857143in}}{\pgfqpoint{2.627103in}{1.813434in}}%
\pgfusepath{clip}%
\pgfsetbuttcap%
\pgfsetmiterjoin%
\definecolor{currentfill}{rgb}{0.066899,0.263188,0.377594}%
\pgfsetfillcolor{currentfill}%
\pgfsetlinewidth{0.000000pt}%
\definecolor{currentstroke}{rgb}{0.000000,0.000000,0.000000}%
\pgfsetstrokecolor{currentstroke}%
\pgfsetstrokeopacity{0.000000}%
\pgfsetdash{}{0pt}%
\pgfpathmoveto{\pgfqpoint{5.853020in}{1.813884in}}%
\pgfpathlineto{\pgfqpoint{5.861956in}{1.813884in}}%
\pgfpathlineto{\pgfqpoint{5.861956in}{1.867959in}}%
\pgfpathlineto{\pgfqpoint{5.853020in}{1.867959in}}%
\pgfpathlineto{\pgfqpoint{5.853020in}{1.813884in}}%
\pgfpathclose%
\pgfusepath{fill}%
\end{pgfscope}%
\begin{pgfscope}%
\pgfpathrectangle{\pgfqpoint{3.722897in}{0.857143in}}{\pgfqpoint{2.627103in}{1.813434in}}%
\pgfusepath{clip}%
\pgfsetbuttcap%
\pgfsetmiterjoin%
\definecolor{currentfill}{rgb}{0.066899,0.263188,0.377594}%
\pgfsetfillcolor{currentfill}%
\pgfsetlinewidth{0.000000pt}%
\definecolor{currentstroke}{rgb}{0.000000,0.000000,0.000000}%
\pgfsetstrokecolor{currentstroke}%
\pgfsetstrokeopacity{0.000000}%
\pgfsetdash{}{0pt}%
\pgfpathmoveto{\pgfqpoint{5.864190in}{1.813884in}}%
\pgfpathlineto{\pgfqpoint{5.873127in}{1.813884in}}%
\pgfpathlineto{\pgfqpoint{5.873127in}{1.868531in}}%
\pgfpathlineto{\pgfqpoint{5.864190in}{1.868531in}}%
\pgfpathlineto{\pgfqpoint{5.864190in}{1.813884in}}%
\pgfpathclose%
\pgfusepath{fill}%
\end{pgfscope}%
\begin{pgfscope}%
\pgfpathrectangle{\pgfqpoint{3.722897in}{0.857143in}}{\pgfqpoint{2.627103in}{1.813434in}}%
\pgfusepath{clip}%
\pgfsetbuttcap%
\pgfsetmiterjoin%
\definecolor{currentfill}{rgb}{0.066899,0.263188,0.377594}%
\pgfsetfillcolor{currentfill}%
\pgfsetlinewidth{0.000000pt}%
\definecolor{currentstroke}{rgb}{0.000000,0.000000,0.000000}%
\pgfsetstrokecolor{currentstroke}%
\pgfsetstrokeopacity{0.000000}%
\pgfsetdash{}{0pt}%
\pgfpathmoveto{\pgfqpoint{5.875361in}{1.813884in}}%
\pgfpathlineto{\pgfqpoint{5.884297in}{1.813884in}}%
\pgfpathlineto{\pgfqpoint{5.884297in}{1.868604in}}%
\pgfpathlineto{\pgfqpoint{5.875361in}{1.868604in}}%
\pgfpathlineto{\pgfqpoint{5.875361in}{1.813884in}}%
\pgfpathclose%
\pgfusepath{fill}%
\end{pgfscope}%
\begin{pgfscope}%
\pgfpathrectangle{\pgfqpoint{3.722897in}{0.857143in}}{\pgfqpoint{2.627103in}{1.813434in}}%
\pgfusepath{clip}%
\pgfsetbuttcap%
\pgfsetmiterjoin%
\definecolor{currentfill}{rgb}{0.066899,0.263188,0.377594}%
\pgfsetfillcolor{currentfill}%
\pgfsetlinewidth{0.000000pt}%
\definecolor{currentstroke}{rgb}{0.000000,0.000000,0.000000}%
\pgfsetstrokecolor{currentstroke}%
\pgfsetstrokeopacity{0.000000}%
\pgfsetdash{}{0pt}%
\pgfpathmoveto{\pgfqpoint{5.886532in}{1.813884in}}%
\pgfpathlineto{\pgfqpoint{5.895468in}{1.813884in}}%
\pgfpathlineto{\pgfqpoint{5.895468in}{1.868885in}}%
\pgfpathlineto{\pgfqpoint{5.886532in}{1.868885in}}%
\pgfpathlineto{\pgfqpoint{5.886532in}{1.813884in}}%
\pgfpathclose%
\pgfusepath{fill}%
\end{pgfscope}%
\begin{pgfscope}%
\pgfpathrectangle{\pgfqpoint{3.722897in}{0.857143in}}{\pgfqpoint{2.627103in}{1.813434in}}%
\pgfusepath{clip}%
\pgfsetbuttcap%
\pgfsetmiterjoin%
\definecolor{currentfill}{rgb}{0.066899,0.263188,0.377594}%
\pgfsetfillcolor{currentfill}%
\pgfsetlinewidth{0.000000pt}%
\definecolor{currentstroke}{rgb}{0.000000,0.000000,0.000000}%
\pgfsetstrokecolor{currentstroke}%
\pgfsetstrokeopacity{0.000000}%
\pgfsetdash{}{0pt}%
\pgfpathmoveto{\pgfqpoint{5.897702in}{1.813884in}}%
\pgfpathlineto{\pgfqpoint{5.906639in}{1.813884in}}%
\pgfpathlineto{\pgfqpoint{5.906639in}{1.867412in}}%
\pgfpathlineto{\pgfqpoint{5.897702in}{1.867412in}}%
\pgfpathlineto{\pgfqpoint{5.897702in}{1.813884in}}%
\pgfpathclose%
\pgfusepath{fill}%
\end{pgfscope}%
\begin{pgfscope}%
\pgfpathrectangle{\pgfqpoint{3.722897in}{0.857143in}}{\pgfqpoint{2.627103in}{1.813434in}}%
\pgfusepath{clip}%
\pgfsetbuttcap%
\pgfsetmiterjoin%
\definecolor{currentfill}{rgb}{0.066899,0.263188,0.377594}%
\pgfsetfillcolor{currentfill}%
\pgfsetlinewidth{0.000000pt}%
\definecolor{currentstroke}{rgb}{0.000000,0.000000,0.000000}%
\pgfsetstrokecolor{currentstroke}%
\pgfsetstrokeopacity{0.000000}%
\pgfsetdash{}{0pt}%
\pgfpathmoveto{\pgfqpoint{5.908873in}{1.813884in}}%
\pgfpathlineto{\pgfqpoint{5.917809in}{1.813884in}}%
\pgfpathlineto{\pgfqpoint{5.917809in}{1.865495in}}%
\pgfpathlineto{\pgfqpoint{5.908873in}{1.865495in}}%
\pgfpathlineto{\pgfqpoint{5.908873in}{1.813884in}}%
\pgfpathclose%
\pgfusepath{fill}%
\end{pgfscope}%
\begin{pgfscope}%
\pgfpathrectangle{\pgfqpoint{3.722897in}{0.857143in}}{\pgfqpoint{2.627103in}{1.813434in}}%
\pgfusepath{clip}%
\pgfsetbuttcap%
\pgfsetmiterjoin%
\definecolor{currentfill}{rgb}{0.066899,0.263188,0.377594}%
\pgfsetfillcolor{currentfill}%
\pgfsetlinewidth{0.000000pt}%
\definecolor{currentstroke}{rgb}{0.000000,0.000000,0.000000}%
\pgfsetstrokecolor{currentstroke}%
\pgfsetstrokeopacity{0.000000}%
\pgfsetdash{}{0pt}%
\pgfpathmoveto{\pgfqpoint{5.920043in}{1.813884in}}%
\pgfpathlineto{\pgfqpoint{5.928980in}{1.813884in}}%
\pgfpathlineto{\pgfqpoint{5.928980in}{1.864821in}}%
\pgfpathlineto{\pgfqpoint{5.920043in}{1.864821in}}%
\pgfpathlineto{\pgfqpoint{5.920043in}{1.813884in}}%
\pgfpathclose%
\pgfusepath{fill}%
\end{pgfscope}%
\begin{pgfscope}%
\pgfpathrectangle{\pgfqpoint{3.722897in}{0.857143in}}{\pgfqpoint{2.627103in}{1.813434in}}%
\pgfusepath{clip}%
\pgfsetbuttcap%
\pgfsetmiterjoin%
\definecolor{currentfill}{rgb}{0.066899,0.263188,0.377594}%
\pgfsetfillcolor{currentfill}%
\pgfsetlinewidth{0.000000pt}%
\definecolor{currentstroke}{rgb}{0.000000,0.000000,0.000000}%
\pgfsetstrokecolor{currentstroke}%
\pgfsetstrokeopacity{0.000000}%
\pgfsetdash{}{0pt}%
\pgfpathmoveto{\pgfqpoint{5.931214in}{1.813884in}}%
\pgfpathlineto{\pgfqpoint{5.940150in}{1.813884in}}%
\pgfpathlineto{\pgfqpoint{5.940150in}{1.862982in}}%
\pgfpathlineto{\pgfqpoint{5.931214in}{1.862982in}}%
\pgfpathlineto{\pgfqpoint{5.931214in}{1.813884in}}%
\pgfpathclose%
\pgfusepath{fill}%
\end{pgfscope}%
\begin{pgfscope}%
\pgfpathrectangle{\pgfqpoint{3.722897in}{0.857143in}}{\pgfqpoint{2.627103in}{1.813434in}}%
\pgfusepath{clip}%
\pgfsetbuttcap%
\pgfsetmiterjoin%
\definecolor{currentfill}{rgb}{0.066899,0.263188,0.377594}%
\pgfsetfillcolor{currentfill}%
\pgfsetlinewidth{0.000000pt}%
\definecolor{currentstroke}{rgb}{0.000000,0.000000,0.000000}%
\pgfsetstrokecolor{currentstroke}%
\pgfsetstrokeopacity{0.000000}%
\pgfsetdash{}{0pt}%
\pgfpathmoveto{\pgfqpoint{5.942385in}{1.813884in}}%
\pgfpathlineto{\pgfqpoint{5.951321in}{1.813884in}}%
\pgfpathlineto{\pgfqpoint{5.951321in}{1.862196in}}%
\pgfpathlineto{\pgfqpoint{5.942385in}{1.862196in}}%
\pgfpathlineto{\pgfqpoint{5.942385in}{1.813884in}}%
\pgfpathclose%
\pgfusepath{fill}%
\end{pgfscope}%
\begin{pgfscope}%
\pgfpathrectangle{\pgfqpoint{3.722897in}{0.857143in}}{\pgfqpoint{2.627103in}{1.813434in}}%
\pgfusepath{clip}%
\pgfsetbuttcap%
\pgfsetmiterjoin%
\definecolor{currentfill}{rgb}{0.066899,0.263188,0.377594}%
\pgfsetfillcolor{currentfill}%
\pgfsetlinewidth{0.000000pt}%
\definecolor{currentstroke}{rgb}{0.000000,0.000000,0.000000}%
\pgfsetstrokecolor{currentstroke}%
\pgfsetstrokeopacity{0.000000}%
\pgfsetdash{}{0pt}%
\pgfpathmoveto{\pgfqpoint{5.953555in}{1.813884in}}%
\pgfpathlineto{\pgfqpoint{5.962492in}{1.813884in}}%
\pgfpathlineto{\pgfqpoint{5.962492in}{1.861665in}}%
\pgfpathlineto{\pgfqpoint{5.953555in}{1.861665in}}%
\pgfpathlineto{\pgfqpoint{5.953555in}{1.813884in}}%
\pgfpathclose%
\pgfusepath{fill}%
\end{pgfscope}%
\begin{pgfscope}%
\pgfpathrectangle{\pgfqpoint{3.722897in}{0.857143in}}{\pgfqpoint{2.627103in}{1.813434in}}%
\pgfusepath{clip}%
\pgfsetbuttcap%
\pgfsetmiterjoin%
\definecolor{currentfill}{rgb}{0.066899,0.263188,0.377594}%
\pgfsetfillcolor{currentfill}%
\pgfsetlinewidth{0.000000pt}%
\definecolor{currentstroke}{rgb}{0.000000,0.000000,0.000000}%
\pgfsetstrokecolor{currentstroke}%
\pgfsetstrokeopacity{0.000000}%
\pgfsetdash{}{0pt}%
\pgfpathmoveto{\pgfqpoint{5.964726in}{1.813884in}}%
\pgfpathlineto{\pgfqpoint{5.973662in}{1.813884in}}%
\pgfpathlineto{\pgfqpoint{5.973662in}{1.858249in}}%
\pgfpathlineto{\pgfqpoint{5.964726in}{1.858249in}}%
\pgfpathlineto{\pgfqpoint{5.964726in}{1.813884in}}%
\pgfpathclose%
\pgfusepath{fill}%
\end{pgfscope}%
\begin{pgfscope}%
\pgfpathrectangle{\pgfqpoint{3.722897in}{0.857143in}}{\pgfqpoint{2.627103in}{1.813434in}}%
\pgfusepath{clip}%
\pgfsetbuttcap%
\pgfsetmiterjoin%
\definecolor{currentfill}{rgb}{0.066899,0.263188,0.377594}%
\pgfsetfillcolor{currentfill}%
\pgfsetlinewidth{0.000000pt}%
\definecolor{currentstroke}{rgb}{0.000000,0.000000,0.000000}%
\pgfsetstrokecolor{currentstroke}%
\pgfsetstrokeopacity{0.000000}%
\pgfsetdash{}{0pt}%
\pgfpathmoveto{\pgfqpoint{5.975896in}{1.813884in}}%
\pgfpathlineto{\pgfqpoint{5.984833in}{1.813884in}}%
\pgfpathlineto{\pgfqpoint{5.984833in}{1.856690in}}%
\pgfpathlineto{\pgfqpoint{5.975896in}{1.856690in}}%
\pgfpathlineto{\pgfqpoint{5.975896in}{1.813884in}}%
\pgfpathclose%
\pgfusepath{fill}%
\end{pgfscope}%
\begin{pgfscope}%
\pgfpathrectangle{\pgfqpoint{3.722897in}{0.857143in}}{\pgfqpoint{2.627103in}{1.813434in}}%
\pgfusepath{clip}%
\pgfsetbuttcap%
\pgfsetmiterjoin%
\definecolor{currentfill}{rgb}{0.066899,0.263188,0.377594}%
\pgfsetfillcolor{currentfill}%
\pgfsetlinewidth{0.000000pt}%
\definecolor{currentstroke}{rgb}{0.000000,0.000000,0.000000}%
\pgfsetstrokecolor{currentstroke}%
\pgfsetstrokeopacity{0.000000}%
\pgfsetdash{}{0pt}%
\pgfpathmoveto{\pgfqpoint{5.987067in}{1.813884in}}%
\pgfpathlineto{\pgfqpoint{5.996004in}{1.813884in}}%
\pgfpathlineto{\pgfqpoint{5.996004in}{1.855321in}}%
\pgfpathlineto{\pgfqpoint{5.987067in}{1.855321in}}%
\pgfpathlineto{\pgfqpoint{5.987067in}{1.813884in}}%
\pgfpathclose%
\pgfusepath{fill}%
\end{pgfscope}%
\begin{pgfscope}%
\pgfpathrectangle{\pgfqpoint{3.722897in}{0.857143in}}{\pgfqpoint{2.627103in}{1.813434in}}%
\pgfusepath{clip}%
\pgfsetbuttcap%
\pgfsetmiterjoin%
\definecolor{currentfill}{rgb}{0.066899,0.263188,0.377594}%
\pgfsetfillcolor{currentfill}%
\pgfsetlinewidth{0.000000pt}%
\definecolor{currentstroke}{rgb}{0.000000,0.000000,0.000000}%
\pgfsetstrokecolor{currentstroke}%
\pgfsetstrokeopacity{0.000000}%
\pgfsetdash{}{0pt}%
\pgfpathmoveto{\pgfqpoint{5.998238in}{1.813884in}}%
\pgfpathlineto{\pgfqpoint{6.007174in}{1.813884in}}%
\pgfpathlineto{\pgfqpoint{6.007174in}{1.851905in}}%
\pgfpathlineto{\pgfqpoint{5.998238in}{1.851905in}}%
\pgfpathlineto{\pgfqpoint{5.998238in}{1.813884in}}%
\pgfpathclose%
\pgfusepath{fill}%
\end{pgfscope}%
\begin{pgfscope}%
\pgfpathrectangle{\pgfqpoint{3.722897in}{0.857143in}}{\pgfqpoint{2.627103in}{1.813434in}}%
\pgfusepath{clip}%
\pgfsetbuttcap%
\pgfsetmiterjoin%
\definecolor{currentfill}{rgb}{0.066899,0.263188,0.377594}%
\pgfsetfillcolor{currentfill}%
\pgfsetlinewidth{0.000000pt}%
\definecolor{currentstroke}{rgb}{0.000000,0.000000,0.000000}%
\pgfsetstrokecolor{currentstroke}%
\pgfsetstrokeopacity{0.000000}%
\pgfsetdash{}{0pt}%
\pgfpathmoveto{\pgfqpoint{6.009408in}{1.813884in}}%
\pgfpathlineto{\pgfqpoint{6.018345in}{1.813884in}}%
\pgfpathlineto{\pgfqpoint{6.018345in}{1.850198in}}%
\pgfpathlineto{\pgfqpoint{6.009408in}{1.850198in}}%
\pgfpathlineto{\pgfqpoint{6.009408in}{1.813884in}}%
\pgfpathclose%
\pgfusepath{fill}%
\end{pgfscope}%
\begin{pgfscope}%
\pgfpathrectangle{\pgfqpoint{3.722897in}{0.857143in}}{\pgfqpoint{2.627103in}{1.813434in}}%
\pgfusepath{clip}%
\pgfsetbuttcap%
\pgfsetmiterjoin%
\definecolor{currentfill}{rgb}{0.066899,0.263188,0.377594}%
\pgfsetfillcolor{currentfill}%
\pgfsetlinewidth{0.000000pt}%
\definecolor{currentstroke}{rgb}{0.000000,0.000000,0.000000}%
\pgfsetstrokecolor{currentstroke}%
\pgfsetstrokeopacity{0.000000}%
\pgfsetdash{}{0pt}%
\pgfpathmoveto{\pgfqpoint{6.020579in}{1.813884in}}%
\pgfpathlineto{\pgfqpoint{6.029515in}{1.813884in}}%
\pgfpathlineto{\pgfqpoint{6.029515in}{1.848339in}}%
\pgfpathlineto{\pgfqpoint{6.020579in}{1.848339in}}%
\pgfpathlineto{\pgfqpoint{6.020579in}{1.813884in}}%
\pgfpathclose%
\pgfusepath{fill}%
\end{pgfscope}%
\begin{pgfscope}%
\pgfpathrectangle{\pgfqpoint{3.722897in}{0.857143in}}{\pgfqpoint{2.627103in}{1.813434in}}%
\pgfusepath{clip}%
\pgfsetbuttcap%
\pgfsetmiterjoin%
\definecolor{currentfill}{rgb}{0.066899,0.263188,0.377594}%
\pgfsetfillcolor{currentfill}%
\pgfsetlinewidth{0.000000pt}%
\definecolor{currentstroke}{rgb}{0.000000,0.000000,0.000000}%
\pgfsetstrokecolor{currentstroke}%
\pgfsetstrokeopacity{0.000000}%
\pgfsetdash{}{0pt}%
\pgfpathmoveto{\pgfqpoint{6.031749in}{1.813884in}}%
\pgfpathlineto{\pgfqpoint{6.040686in}{1.813884in}}%
\pgfpathlineto{\pgfqpoint{6.040686in}{1.846312in}}%
\pgfpathlineto{\pgfqpoint{6.031749in}{1.846312in}}%
\pgfpathlineto{\pgfqpoint{6.031749in}{1.813884in}}%
\pgfpathclose%
\pgfusepath{fill}%
\end{pgfscope}%
\begin{pgfscope}%
\pgfpathrectangle{\pgfqpoint{3.722897in}{0.857143in}}{\pgfqpoint{2.627103in}{1.813434in}}%
\pgfusepath{clip}%
\pgfsetbuttcap%
\pgfsetmiterjoin%
\definecolor{currentfill}{rgb}{0.066899,0.263188,0.377594}%
\pgfsetfillcolor{currentfill}%
\pgfsetlinewidth{0.000000pt}%
\definecolor{currentstroke}{rgb}{0.000000,0.000000,0.000000}%
\pgfsetstrokecolor{currentstroke}%
\pgfsetstrokeopacity{0.000000}%
\pgfsetdash{}{0pt}%
\pgfpathmoveto{\pgfqpoint{6.042920in}{1.813884in}}%
\pgfpathlineto{\pgfqpoint{6.051857in}{1.813884in}}%
\pgfpathlineto{\pgfqpoint{6.051857in}{1.842505in}}%
\pgfpathlineto{\pgfqpoint{6.042920in}{1.842505in}}%
\pgfpathlineto{\pgfqpoint{6.042920in}{1.813884in}}%
\pgfpathclose%
\pgfusepath{fill}%
\end{pgfscope}%
\begin{pgfscope}%
\pgfpathrectangle{\pgfqpoint{3.722897in}{0.857143in}}{\pgfqpoint{2.627103in}{1.813434in}}%
\pgfusepath{clip}%
\pgfsetbuttcap%
\pgfsetmiterjoin%
\definecolor{currentfill}{rgb}{0.066899,0.263188,0.377594}%
\pgfsetfillcolor{currentfill}%
\pgfsetlinewidth{0.000000pt}%
\definecolor{currentstroke}{rgb}{0.000000,0.000000,0.000000}%
\pgfsetstrokecolor{currentstroke}%
\pgfsetstrokeopacity{0.000000}%
\pgfsetdash{}{0pt}%
\pgfpathmoveto{\pgfqpoint{6.054091in}{1.813884in}}%
\pgfpathlineto{\pgfqpoint{6.063027in}{1.813884in}}%
\pgfpathlineto{\pgfqpoint{6.063027in}{1.840337in}}%
\pgfpathlineto{\pgfqpoint{6.054091in}{1.840337in}}%
\pgfpathlineto{\pgfqpoint{6.054091in}{1.813884in}}%
\pgfpathclose%
\pgfusepath{fill}%
\end{pgfscope}%
\begin{pgfscope}%
\pgfpathrectangle{\pgfqpoint{3.722897in}{0.857143in}}{\pgfqpoint{2.627103in}{1.813434in}}%
\pgfusepath{clip}%
\pgfsetbuttcap%
\pgfsetmiterjoin%
\definecolor{currentfill}{rgb}{0.066899,0.263188,0.377594}%
\pgfsetfillcolor{currentfill}%
\pgfsetlinewidth{0.000000pt}%
\definecolor{currentstroke}{rgb}{0.000000,0.000000,0.000000}%
\pgfsetstrokecolor{currentstroke}%
\pgfsetstrokeopacity{0.000000}%
\pgfsetdash{}{0pt}%
\pgfpathmoveto{\pgfqpoint{6.065261in}{1.813884in}}%
\pgfpathlineto{\pgfqpoint{6.074198in}{1.813884in}}%
\pgfpathlineto{\pgfqpoint{6.074198in}{1.837493in}}%
\pgfpathlineto{\pgfqpoint{6.065261in}{1.837493in}}%
\pgfpathlineto{\pgfqpoint{6.065261in}{1.813884in}}%
\pgfpathclose%
\pgfusepath{fill}%
\end{pgfscope}%
\begin{pgfscope}%
\pgfpathrectangle{\pgfqpoint{3.722897in}{0.857143in}}{\pgfqpoint{2.627103in}{1.813434in}}%
\pgfusepath{clip}%
\pgfsetbuttcap%
\pgfsetmiterjoin%
\definecolor{currentfill}{rgb}{0.066899,0.263188,0.377594}%
\pgfsetfillcolor{currentfill}%
\pgfsetlinewidth{0.000000pt}%
\definecolor{currentstroke}{rgb}{0.000000,0.000000,0.000000}%
\pgfsetstrokecolor{currentstroke}%
\pgfsetstrokeopacity{0.000000}%
\pgfsetdash{}{0pt}%
\pgfpathmoveto{\pgfqpoint{6.076432in}{1.813884in}}%
\pgfpathlineto{\pgfqpoint{6.085368in}{1.813884in}}%
\pgfpathlineto{\pgfqpoint{6.085368in}{1.834950in}}%
\pgfpathlineto{\pgfqpoint{6.076432in}{1.834950in}}%
\pgfpathlineto{\pgfqpoint{6.076432in}{1.813884in}}%
\pgfpathclose%
\pgfusepath{fill}%
\end{pgfscope}%
\begin{pgfscope}%
\pgfpathrectangle{\pgfqpoint{3.722897in}{0.857143in}}{\pgfqpoint{2.627103in}{1.813434in}}%
\pgfusepath{clip}%
\pgfsetbuttcap%
\pgfsetmiterjoin%
\definecolor{currentfill}{rgb}{0.066899,0.263188,0.377594}%
\pgfsetfillcolor{currentfill}%
\pgfsetlinewidth{0.000000pt}%
\definecolor{currentstroke}{rgb}{0.000000,0.000000,0.000000}%
\pgfsetstrokecolor{currentstroke}%
\pgfsetstrokeopacity{0.000000}%
\pgfsetdash{}{0pt}%
\pgfpathmoveto{\pgfqpoint{6.087602in}{1.813884in}}%
\pgfpathlineto{\pgfqpoint{6.096539in}{1.813884in}}%
\pgfpathlineto{\pgfqpoint{6.096539in}{1.832889in}}%
\pgfpathlineto{\pgfqpoint{6.087602in}{1.832889in}}%
\pgfpathlineto{\pgfqpoint{6.087602in}{1.813884in}}%
\pgfpathclose%
\pgfusepath{fill}%
\end{pgfscope}%
\begin{pgfscope}%
\pgfpathrectangle{\pgfqpoint{3.722897in}{0.857143in}}{\pgfqpoint{2.627103in}{1.813434in}}%
\pgfusepath{clip}%
\pgfsetbuttcap%
\pgfsetmiterjoin%
\definecolor{currentfill}{rgb}{0.066899,0.263188,0.377594}%
\pgfsetfillcolor{currentfill}%
\pgfsetlinewidth{0.000000pt}%
\definecolor{currentstroke}{rgb}{0.000000,0.000000,0.000000}%
\pgfsetstrokecolor{currentstroke}%
\pgfsetstrokeopacity{0.000000}%
\pgfsetdash{}{0pt}%
\pgfpathmoveto{\pgfqpoint{6.098773in}{1.813884in}}%
\pgfpathlineto{\pgfqpoint{6.107710in}{1.813884in}}%
\pgfpathlineto{\pgfqpoint{6.107710in}{1.830205in}}%
\pgfpathlineto{\pgfqpoint{6.098773in}{1.830205in}}%
\pgfpathlineto{\pgfqpoint{6.098773in}{1.813884in}}%
\pgfpathclose%
\pgfusepath{fill}%
\end{pgfscope}%
\begin{pgfscope}%
\pgfpathrectangle{\pgfqpoint{3.722897in}{0.857143in}}{\pgfqpoint{2.627103in}{1.813434in}}%
\pgfusepath{clip}%
\pgfsetbuttcap%
\pgfsetmiterjoin%
\definecolor{currentfill}{rgb}{0.066899,0.263188,0.377594}%
\pgfsetfillcolor{currentfill}%
\pgfsetlinewidth{0.000000pt}%
\definecolor{currentstroke}{rgb}{0.000000,0.000000,0.000000}%
\pgfsetstrokecolor{currentstroke}%
\pgfsetstrokeopacity{0.000000}%
\pgfsetdash{}{0pt}%
\pgfpathmoveto{\pgfqpoint{6.109944in}{1.813884in}}%
\pgfpathlineto{\pgfqpoint{6.118880in}{1.813884in}}%
\pgfpathlineto{\pgfqpoint{6.118880in}{1.827048in}}%
\pgfpathlineto{\pgfqpoint{6.109944in}{1.827048in}}%
\pgfpathlineto{\pgfqpoint{6.109944in}{1.813884in}}%
\pgfpathclose%
\pgfusepath{fill}%
\end{pgfscope}%
\begin{pgfscope}%
\pgfpathrectangle{\pgfqpoint{3.722897in}{0.857143in}}{\pgfqpoint{2.627103in}{1.813434in}}%
\pgfusepath{clip}%
\pgfsetbuttcap%
\pgfsetmiterjoin%
\definecolor{currentfill}{rgb}{0.066899,0.263188,0.377594}%
\pgfsetfillcolor{currentfill}%
\pgfsetlinewidth{0.000000pt}%
\definecolor{currentstroke}{rgb}{0.000000,0.000000,0.000000}%
\pgfsetstrokecolor{currentstroke}%
\pgfsetstrokeopacity{0.000000}%
\pgfsetdash{}{0pt}%
\pgfpathmoveto{\pgfqpoint{6.121114in}{1.813884in}}%
\pgfpathlineto{\pgfqpoint{6.130051in}{1.813884in}}%
\pgfpathlineto{\pgfqpoint{6.130051in}{1.825155in}}%
\pgfpathlineto{\pgfqpoint{6.121114in}{1.825155in}}%
\pgfpathlineto{\pgfqpoint{6.121114in}{1.813884in}}%
\pgfpathclose%
\pgfusepath{fill}%
\end{pgfscope}%
\begin{pgfscope}%
\pgfpathrectangle{\pgfqpoint{3.722897in}{0.857143in}}{\pgfqpoint{2.627103in}{1.813434in}}%
\pgfusepath{clip}%
\pgfsetbuttcap%
\pgfsetmiterjoin%
\definecolor{currentfill}{rgb}{0.066899,0.263188,0.377594}%
\pgfsetfillcolor{currentfill}%
\pgfsetlinewidth{0.000000pt}%
\definecolor{currentstroke}{rgb}{0.000000,0.000000,0.000000}%
\pgfsetstrokecolor{currentstroke}%
\pgfsetstrokeopacity{0.000000}%
\pgfsetdash{}{0pt}%
\pgfpathmoveto{\pgfqpoint{6.132285in}{1.813884in}}%
\pgfpathlineto{\pgfqpoint{6.141221in}{1.813884in}}%
\pgfpathlineto{\pgfqpoint{6.141221in}{1.823233in}}%
\pgfpathlineto{\pgfqpoint{6.132285in}{1.823233in}}%
\pgfpathlineto{\pgfqpoint{6.132285in}{1.813884in}}%
\pgfpathclose%
\pgfusepath{fill}%
\end{pgfscope}%
\begin{pgfscope}%
\pgfpathrectangle{\pgfqpoint{3.722897in}{0.857143in}}{\pgfqpoint{2.627103in}{1.813434in}}%
\pgfusepath{clip}%
\pgfsetbuttcap%
\pgfsetmiterjoin%
\definecolor{currentfill}{rgb}{0.066899,0.263188,0.377594}%
\pgfsetfillcolor{currentfill}%
\pgfsetlinewidth{0.000000pt}%
\definecolor{currentstroke}{rgb}{0.000000,0.000000,0.000000}%
\pgfsetstrokecolor{currentstroke}%
\pgfsetstrokeopacity{0.000000}%
\pgfsetdash{}{0pt}%
\pgfpathmoveto{\pgfqpoint{6.143456in}{1.813884in}}%
\pgfpathlineto{\pgfqpoint{6.152392in}{1.813884in}}%
\pgfpathlineto{\pgfqpoint{6.152392in}{1.820974in}}%
\pgfpathlineto{\pgfqpoint{6.143456in}{1.820974in}}%
\pgfpathlineto{\pgfqpoint{6.143456in}{1.813884in}}%
\pgfpathclose%
\pgfusepath{fill}%
\end{pgfscope}%
\begin{pgfscope}%
\pgfpathrectangle{\pgfqpoint{3.722897in}{0.857143in}}{\pgfqpoint{2.627103in}{1.813434in}}%
\pgfusepath{clip}%
\pgfsetbuttcap%
\pgfsetmiterjoin%
\definecolor{currentfill}{rgb}{0.066899,0.263188,0.377594}%
\pgfsetfillcolor{currentfill}%
\pgfsetlinewidth{0.000000pt}%
\definecolor{currentstroke}{rgb}{0.000000,0.000000,0.000000}%
\pgfsetstrokecolor{currentstroke}%
\pgfsetstrokeopacity{0.000000}%
\pgfsetdash{}{0pt}%
\pgfpathmoveto{\pgfqpoint{6.154626in}{1.813884in}}%
\pgfpathlineto{\pgfqpoint{6.163563in}{1.813884in}}%
\pgfpathlineto{\pgfqpoint{6.163563in}{1.818208in}}%
\pgfpathlineto{\pgfqpoint{6.154626in}{1.818208in}}%
\pgfpathlineto{\pgfqpoint{6.154626in}{1.813884in}}%
\pgfpathclose%
\pgfusepath{fill}%
\end{pgfscope}%
\begin{pgfscope}%
\pgfpathrectangle{\pgfqpoint{3.722897in}{0.857143in}}{\pgfqpoint{2.627103in}{1.813434in}}%
\pgfusepath{clip}%
\pgfsetbuttcap%
\pgfsetmiterjoin%
\definecolor{currentfill}{rgb}{0.066899,0.263188,0.377594}%
\pgfsetfillcolor{currentfill}%
\pgfsetlinewidth{0.000000pt}%
\definecolor{currentstroke}{rgb}{0.000000,0.000000,0.000000}%
\pgfsetstrokecolor{currentstroke}%
\pgfsetstrokeopacity{0.000000}%
\pgfsetdash{}{0pt}%
\pgfpathmoveto{\pgfqpoint{6.165797in}{1.813884in}}%
\pgfpathlineto{\pgfqpoint{6.174733in}{1.813884in}}%
\pgfpathlineto{\pgfqpoint{6.174733in}{1.815763in}}%
\pgfpathlineto{\pgfqpoint{6.165797in}{1.815763in}}%
\pgfpathlineto{\pgfqpoint{6.165797in}{1.813884in}}%
\pgfpathclose%
\pgfusepath{fill}%
\end{pgfscope}%
\begin{pgfscope}%
\pgfpathrectangle{\pgfqpoint{3.722897in}{0.857143in}}{\pgfqpoint{2.627103in}{1.813434in}}%
\pgfusepath{clip}%
\pgfsetbuttcap%
\pgfsetmiterjoin%
\definecolor{currentfill}{rgb}{0.066899,0.263188,0.377594}%
\pgfsetfillcolor{currentfill}%
\pgfsetlinewidth{0.000000pt}%
\definecolor{currentstroke}{rgb}{0.000000,0.000000,0.000000}%
\pgfsetstrokecolor{currentstroke}%
\pgfsetstrokeopacity{0.000000}%
\pgfsetdash{}{0pt}%
\pgfpathmoveto{\pgfqpoint{6.176967in}{1.813884in}}%
\pgfpathlineto{\pgfqpoint{6.185904in}{1.813884in}}%
\pgfpathlineto{\pgfqpoint{6.185904in}{1.812481in}}%
\pgfpathlineto{\pgfqpoint{6.176967in}{1.812481in}}%
\pgfpathlineto{\pgfqpoint{6.176967in}{1.813884in}}%
\pgfpathclose%
\pgfusepath{fill}%
\end{pgfscope}%
\begin{pgfscope}%
\pgfpathrectangle{\pgfqpoint{3.722897in}{0.857143in}}{\pgfqpoint{2.627103in}{1.813434in}}%
\pgfusepath{clip}%
\pgfsetbuttcap%
\pgfsetmiterjoin%
\definecolor{currentfill}{rgb}{0.066899,0.263188,0.377594}%
\pgfsetfillcolor{currentfill}%
\pgfsetlinewidth{0.000000pt}%
\definecolor{currentstroke}{rgb}{0.000000,0.000000,0.000000}%
\pgfsetstrokecolor{currentstroke}%
\pgfsetstrokeopacity{0.000000}%
\pgfsetdash{}{0pt}%
\pgfpathmoveto{\pgfqpoint{6.188138in}{1.813884in}}%
\pgfpathlineto{\pgfqpoint{6.197074in}{1.813884in}}%
\pgfpathlineto{\pgfqpoint{6.197074in}{1.810758in}}%
\pgfpathlineto{\pgfqpoint{6.188138in}{1.810758in}}%
\pgfpathlineto{\pgfqpoint{6.188138in}{1.813884in}}%
\pgfpathclose%
\pgfusepath{fill}%
\end{pgfscope}%
\begin{pgfscope}%
\pgfpathrectangle{\pgfqpoint{3.722897in}{0.857143in}}{\pgfqpoint{2.627103in}{1.813434in}}%
\pgfusepath{clip}%
\pgfsetbuttcap%
\pgfsetmiterjoin%
\definecolor{currentfill}{rgb}{0.066899,0.263188,0.377594}%
\pgfsetfillcolor{currentfill}%
\pgfsetlinewidth{0.000000pt}%
\definecolor{currentstroke}{rgb}{0.000000,0.000000,0.000000}%
\pgfsetstrokecolor{currentstroke}%
\pgfsetstrokeopacity{0.000000}%
\pgfsetdash{}{0pt}%
\pgfpathmoveto{\pgfqpoint{6.199309in}{1.813884in}}%
\pgfpathlineto{\pgfqpoint{6.208245in}{1.813884in}}%
\pgfpathlineto{\pgfqpoint{6.208245in}{1.808892in}}%
\pgfpathlineto{\pgfqpoint{6.199309in}{1.808892in}}%
\pgfpathlineto{\pgfqpoint{6.199309in}{1.813884in}}%
\pgfpathclose%
\pgfusepath{fill}%
\end{pgfscope}%
\begin{pgfscope}%
\pgfpathrectangle{\pgfqpoint{3.722897in}{0.857143in}}{\pgfqpoint{2.627103in}{1.813434in}}%
\pgfusepath{clip}%
\pgfsetbuttcap%
\pgfsetmiterjoin%
\definecolor{currentfill}{rgb}{0.066899,0.263188,0.377594}%
\pgfsetfillcolor{currentfill}%
\pgfsetlinewidth{0.000000pt}%
\definecolor{currentstroke}{rgb}{0.000000,0.000000,0.000000}%
\pgfsetstrokecolor{currentstroke}%
\pgfsetstrokeopacity{0.000000}%
\pgfsetdash{}{0pt}%
\pgfpathmoveto{\pgfqpoint{6.210479in}{1.813884in}}%
\pgfpathlineto{\pgfqpoint{6.219416in}{1.813884in}}%
\pgfpathlineto{\pgfqpoint{6.219416in}{1.808025in}}%
\pgfpathlineto{\pgfqpoint{6.210479in}{1.808025in}}%
\pgfpathlineto{\pgfqpoint{6.210479in}{1.813884in}}%
\pgfpathclose%
\pgfusepath{fill}%
\end{pgfscope}%
\begin{pgfscope}%
\pgfpathrectangle{\pgfqpoint{3.722897in}{0.857143in}}{\pgfqpoint{2.627103in}{1.813434in}}%
\pgfusepath{clip}%
\pgfsetbuttcap%
\pgfsetmiterjoin%
\definecolor{currentfill}{rgb}{0.066899,0.263188,0.377594}%
\pgfsetfillcolor{currentfill}%
\pgfsetlinewidth{0.000000pt}%
\definecolor{currentstroke}{rgb}{0.000000,0.000000,0.000000}%
\pgfsetstrokecolor{currentstroke}%
\pgfsetstrokeopacity{0.000000}%
\pgfsetdash{}{0pt}%
\pgfpathmoveto{\pgfqpoint{6.221650in}{1.813884in}}%
\pgfpathlineto{\pgfqpoint{6.230586in}{1.813884in}}%
\pgfpathlineto{\pgfqpoint{6.230586in}{1.807525in}}%
\pgfpathlineto{\pgfqpoint{6.221650in}{1.807525in}}%
\pgfpathlineto{\pgfqpoint{6.221650in}{1.813884in}}%
\pgfpathclose%
\pgfusepath{fill}%
\end{pgfscope}%
\begin{pgfscope}%
\pgfpathrectangle{\pgfqpoint{3.722897in}{0.857143in}}{\pgfqpoint{2.627103in}{1.813434in}}%
\pgfusepath{clip}%
\pgfsetbuttcap%
\pgfsetmiterjoin%
\definecolor{currentfill}{rgb}{0.133298,0.375282,0.379395}%
\pgfsetfillcolor{currentfill}%
\pgfsetlinewidth{0.000000pt}%
\definecolor{currentstroke}{rgb}{0.000000,0.000000,0.000000}%
\pgfsetstrokecolor{currentstroke}%
\pgfsetstrokeopacity{0.000000}%
\pgfsetdash{}{0pt}%
\pgfpathmoveto{\pgfqpoint{3.842311in}{1.813884in}}%
\pgfpathlineto{\pgfqpoint{3.851247in}{1.813884in}}%
\pgfpathlineto{\pgfqpoint{3.851247in}{1.819333in}}%
\pgfpathlineto{\pgfqpoint{3.842311in}{1.819333in}}%
\pgfpathlineto{\pgfqpoint{3.842311in}{1.813884in}}%
\pgfpathclose%
\pgfusepath{fill}%
\end{pgfscope}%
\begin{pgfscope}%
\pgfpathrectangle{\pgfqpoint{3.722897in}{0.857143in}}{\pgfqpoint{2.627103in}{1.813434in}}%
\pgfusepath{clip}%
\pgfsetbuttcap%
\pgfsetmiterjoin%
\definecolor{currentfill}{rgb}{0.133298,0.375282,0.379395}%
\pgfsetfillcolor{currentfill}%
\pgfsetlinewidth{0.000000pt}%
\definecolor{currentstroke}{rgb}{0.000000,0.000000,0.000000}%
\pgfsetstrokecolor{currentstroke}%
\pgfsetstrokeopacity{0.000000}%
\pgfsetdash{}{0pt}%
\pgfpathmoveto{\pgfqpoint{3.853481in}{1.814336in}}%
\pgfpathlineto{\pgfqpoint{3.862418in}{1.814336in}}%
\pgfpathlineto{\pgfqpoint{3.862418in}{1.824416in}}%
\pgfpathlineto{\pgfqpoint{3.853481in}{1.824416in}}%
\pgfpathlineto{\pgfqpoint{3.853481in}{1.814336in}}%
\pgfpathclose%
\pgfusepath{fill}%
\end{pgfscope}%
\begin{pgfscope}%
\pgfpathrectangle{\pgfqpoint{3.722897in}{0.857143in}}{\pgfqpoint{2.627103in}{1.813434in}}%
\pgfusepath{clip}%
\pgfsetbuttcap%
\pgfsetmiterjoin%
\definecolor{currentfill}{rgb}{0.133298,0.375282,0.379395}%
\pgfsetfillcolor{currentfill}%
\pgfsetlinewidth{0.000000pt}%
\definecolor{currentstroke}{rgb}{0.000000,0.000000,0.000000}%
\pgfsetstrokecolor{currentstroke}%
\pgfsetstrokeopacity{0.000000}%
\pgfsetdash{}{0pt}%
\pgfpathmoveto{\pgfqpoint{3.864652in}{1.815652in}}%
\pgfpathlineto{\pgfqpoint{3.873588in}{1.815652in}}%
\pgfpathlineto{\pgfqpoint{3.873588in}{1.841228in}}%
\pgfpathlineto{\pgfqpoint{3.864652in}{1.841228in}}%
\pgfpathlineto{\pgfqpoint{3.864652in}{1.815652in}}%
\pgfpathclose%
\pgfusepath{fill}%
\end{pgfscope}%
\begin{pgfscope}%
\pgfpathrectangle{\pgfqpoint{3.722897in}{0.857143in}}{\pgfqpoint{2.627103in}{1.813434in}}%
\pgfusepath{clip}%
\pgfsetbuttcap%
\pgfsetmiterjoin%
\definecolor{currentfill}{rgb}{0.133298,0.375282,0.379395}%
\pgfsetfillcolor{currentfill}%
\pgfsetlinewidth{0.000000pt}%
\definecolor{currentstroke}{rgb}{0.000000,0.000000,0.000000}%
\pgfsetstrokecolor{currentstroke}%
\pgfsetstrokeopacity{0.000000}%
\pgfsetdash{}{0pt}%
\pgfpathmoveto{\pgfqpoint{3.875823in}{1.816540in}}%
\pgfpathlineto{\pgfqpoint{3.884759in}{1.816540in}}%
\pgfpathlineto{\pgfqpoint{3.884759in}{1.833313in}}%
\pgfpathlineto{\pgfqpoint{3.875823in}{1.833313in}}%
\pgfpathlineto{\pgfqpoint{3.875823in}{1.816540in}}%
\pgfpathclose%
\pgfusepath{fill}%
\end{pgfscope}%
\begin{pgfscope}%
\pgfpathrectangle{\pgfqpoint{3.722897in}{0.857143in}}{\pgfqpoint{2.627103in}{1.813434in}}%
\pgfusepath{clip}%
\pgfsetbuttcap%
\pgfsetmiterjoin%
\definecolor{currentfill}{rgb}{0.133298,0.375282,0.379395}%
\pgfsetfillcolor{currentfill}%
\pgfsetlinewidth{0.000000pt}%
\definecolor{currentstroke}{rgb}{0.000000,0.000000,0.000000}%
\pgfsetstrokecolor{currentstroke}%
\pgfsetstrokeopacity{0.000000}%
\pgfsetdash{}{0pt}%
\pgfpathmoveto{\pgfqpoint{3.886993in}{1.817502in}}%
\pgfpathlineto{\pgfqpoint{3.895930in}{1.817502in}}%
\pgfpathlineto{\pgfqpoint{3.895930in}{1.818570in}}%
\pgfpathlineto{\pgfqpoint{3.886993in}{1.818570in}}%
\pgfpathlineto{\pgfqpoint{3.886993in}{1.817502in}}%
\pgfpathclose%
\pgfusepath{fill}%
\end{pgfscope}%
\begin{pgfscope}%
\pgfpathrectangle{\pgfqpoint{3.722897in}{0.857143in}}{\pgfqpoint{2.627103in}{1.813434in}}%
\pgfusepath{clip}%
\pgfsetbuttcap%
\pgfsetmiterjoin%
\definecolor{currentfill}{rgb}{0.133298,0.375282,0.379395}%
\pgfsetfillcolor{currentfill}%
\pgfsetlinewidth{0.000000pt}%
\definecolor{currentstroke}{rgb}{0.000000,0.000000,0.000000}%
\pgfsetstrokecolor{currentstroke}%
\pgfsetstrokeopacity{0.000000}%
\pgfsetdash{}{0pt}%
\pgfpathmoveto{\pgfqpoint{3.898164in}{1.813884in}}%
\pgfpathlineto{\pgfqpoint{3.907100in}{1.813884in}}%
\pgfpathlineto{\pgfqpoint{3.907100in}{1.806566in}}%
\pgfpathlineto{\pgfqpoint{3.898164in}{1.806566in}}%
\pgfpathlineto{\pgfqpoint{3.898164in}{1.813884in}}%
\pgfpathclose%
\pgfusepath{fill}%
\end{pgfscope}%
\begin{pgfscope}%
\pgfpathrectangle{\pgfqpoint{3.722897in}{0.857143in}}{\pgfqpoint{2.627103in}{1.813434in}}%
\pgfusepath{clip}%
\pgfsetbuttcap%
\pgfsetmiterjoin%
\definecolor{currentfill}{rgb}{0.133298,0.375282,0.379395}%
\pgfsetfillcolor{currentfill}%
\pgfsetlinewidth{0.000000pt}%
\definecolor{currentstroke}{rgb}{0.000000,0.000000,0.000000}%
\pgfsetstrokecolor{currentstroke}%
\pgfsetstrokeopacity{0.000000}%
\pgfsetdash{}{0pt}%
\pgfpathmoveto{\pgfqpoint{3.909334in}{1.821744in}}%
\pgfpathlineto{\pgfqpoint{3.918271in}{1.821744in}}%
\pgfpathlineto{\pgfqpoint{3.918271in}{1.824681in}}%
\pgfpathlineto{\pgfqpoint{3.909334in}{1.824681in}}%
\pgfpathlineto{\pgfqpoint{3.909334in}{1.821744in}}%
\pgfpathclose%
\pgfusepath{fill}%
\end{pgfscope}%
\begin{pgfscope}%
\pgfpathrectangle{\pgfqpoint{3.722897in}{0.857143in}}{\pgfqpoint{2.627103in}{1.813434in}}%
\pgfusepath{clip}%
\pgfsetbuttcap%
\pgfsetmiterjoin%
\definecolor{currentfill}{rgb}{0.133298,0.375282,0.379395}%
\pgfsetfillcolor{currentfill}%
\pgfsetlinewidth{0.000000pt}%
\definecolor{currentstroke}{rgb}{0.000000,0.000000,0.000000}%
\pgfsetstrokecolor{currentstroke}%
\pgfsetstrokeopacity{0.000000}%
\pgfsetdash{}{0pt}%
\pgfpathmoveto{\pgfqpoint{3.920505in}{1.823552in}}%
\pgfpathlineto{\pgfqpoint{3.929442in}{1.823552in}}%
\pgfpathlineto{\pgfqpoint{3.929442in}{1.837517in}}%
\pgfpathlineto{\pgfqpoint{3.920505in}{1.837517in}}%
\pgfpathlineto{\pgfqpoint{3.920505in}{1.823552in}}%
\pgfpathclose%
\pgfusepath{fill}%
\end{pgfscope}%
\begin{pgfscope}%
\pgfpathrectangle{\pgfqpoint{3.722897in}{0.857143in}}{\pgfqpoint{2.627103in}{1.813434in}}%
\pgfusepath{clip}%
\pgfsetbuttcap%
\pgfsetmiterjoin%
\definecolor{currentfill}{rgb}{0.133298,0.375282,0.379395}%
\pgfsetfillcolor{currentfill}%
\pgfsetlinewidth{0.000000pt}%
\definecolor{currentstroke}{rgb}{0.000000,0.000000,0.000000}%
\pgfsetstrokecolor{currentstroke}%
\pgfsetstrokeopacity{0.000000}%
\pgfsetdash{}{0pt}%
\pgfpathmoveto{\pgfqpoint{3.931676in}{1.824213in}}%
\pgfpathlineto{\pgfqpoint{3.940612in}{1.824213in}}%
\pgfpathlineto{\pgfqpoint{3.940612in}{1.836801in}}%
\pgfpathlineto{\pgfqpoint{3.931676in}{1.836801in}}%
\pgfpathlineto{\pgfqpoint{3.931676in}{1.824213in}}%
\pgfpathclose%
\pgfusepath{fill}%
\end{pgfscope}%
\begin{pgfscope}%
\pgfpathrectangle{\pgfqpoint{3.722897in}{0.857143in}}{\pgfqpoint{2.627103in}{1.813434in}}%
\pgfusepath{clip}%
\pgfsetbuttcap%
\pgfsetmiterjoin%
\definecolor{currentfill}{rgb}{0.133298,0.375282,0.379395}%
\pgfsetfillcolor{currentfill}%
\pgfsetlinewidth{0.000000pt}%
\definecolor{currentstroke}{rgb}{0.000000,0.000000,0.000000}%
\pgfsetstrokecolor{currentstroke}%
\pgfsetstrokeopacity{0.000000}%
\pgfsetdash{}{0pt}%
\pgfpathmoveto{\pgfqpoint{3.942846in}{1.824145in}}%
\pgfpathlineto{\pgfqpoint{3.951783in}{1.824145in}}%
\pgfpathlineto{\pgfqpoint{3.951783in}{1.827785in}}%
\pgfpathlineto{\pgfqpoint{3.942846in}{1.827785in}}%
\pgfpathlineto{\pgfqpoint{3.942846in}{1.824145in}}%
\pgfpathclose%
\pgfusepath{fill}%
\end{pgfscope}%
\begin{pgfscope}%
\pgfpathrectangle{\pgfqpoint{3.722897in}{0.857143in}}{\pgfqpoint{2.627103in}{1.813434in}}%
\pgfusepath{clip}%
\pgfsetbuttcap%
\pgfsetmiterjoin%
\definecolor{currentfill}{rgb}{0.133298,0.375282,0.379395}%
\pgfsetfillcolor{currentfill}%
\pgfsetlinewidth{0.000000pt}%
\definecolor{currentstroke}{rgb}{0.000000,0.000000,0.000000}%
\pgfsetstrokecolor{currentstroke}%
\pgfsetstrokeopacity{0.000000}%
\pgfsetdash{}{0pt}%
\pgfpathmoveto{\pgfqpoint{3.954017in}{1.825665in}}%
\pgfpathlineto{\pgfqpoint{3.962953in}{1.825665in}}%
\pgfpathlineto{\pgfqpoint{3.962953in}{1.839047in}}%
\pgfpathlineto{\pgfqpoint{3.954017in}{1.839047in}}%
\pgfpathlineto{\pgfqpoint{3.954017in}{1.825665in}}%
\pgfpathclose%
\pgfusepath{fill}%
\end{pgfscope}%
\begin{pgfscope}%
\pgfpathrectangle{\pgfqpoint{3.722897in}{0.857143in}}{\pgfqpoint{2.627103in}{1.813434in}}%
\pgfusepath{clip}%
\pgfsetbuttcap%
\pgfsetmiterjoin%
\definecolor{currentfill}{rgb}{0.133298,0.375282,0.379395}%
\pgfsetfillcolor{currentfill}%
\pgfsetlinewidth{0.000000pt}%
\definecolor{currentstroke}{rgb}{0.000000,0.000000,0.000000}%
\pgfsetstrokecolor{currentstroke}%
\pgfsetstrokeopacity{0.000000}%
\pgfsetdash{}{0pt}%
\pgfpathmoveto{\pgfqpoint{3.965187in}{1.824343in}}%
\pgfpathlineto{\pgfqpoint{3.974124in}{1.824343in}}%
\pgfpathlineto{\pgfqpoint{3.974124in}{1.845072in}}%
\pgfpathlineto{\pgfqpoint{3.965187in}{1.845072in}}%
\pgfpathlineto{\pgfqpoint{3.965187in}{1.824343in}}%
\pgfpathclose%
\pgfusepath{fill}%
\end{pgfscope}%
\begin{pgfscope}%
\pgfpathrectangle{\pgfqpoint{3.722897in}{0.857143in}}{\pgfqpoint{2.627103in}{1.813434in}}%
\pgfusepath{clip}%
\pgfsetbuttcap%
\pgfsetmiterjoin%
\definecolor{currentfill}{rgb}{0.133298,0.375282,0.379395}%
\pgfsetfillcolor{currentfill}%
\pgfsetlinewidth{0.000000pt}%
\definecolor{currentstroke}{rgb}{0.000000,0.000000,0.000000}%
\pgfsetstrokecolor{currentstroke}%
\pgfsetstrokeopacity{0.000000}%
\pgfsetdash{}{0pt}%
\pgfpathmoveto{\pgfqpoint{3.976358in}{1.824342in}}%
\pgfpathlineto{\pgfqpoint{3.985295in}{1.824342in}}%
\pgfpathlineto{\pgfqpoint{3.985295in}{1.827787in}}%
\pgfpathlineto{\pgfqpoint{3.976358in}{1.827787in}}%
\pgfpathlineto{\pgfqpoint{3.976358in}{1.824342in}}%
\pgfpathclose%
\pgfusepath{fill}%
\end{pgfscope}%
\begin{pgfscope}%
\pgfpathrectangle{\pgfqpoint{3.722897in}{0.857143in}}{\pgfqpoint{2.627103in}{1.813434in}}%
\pgfusepath{clip}%
\pgfsetbuttcap%
\pgfsetmiterjoin%
\definecolor{currentfill}{rgb}{0.133298,0.375282,0.379395}%
\pgfsetfillcolor{currentfill}%
\pgfsetlinewidth{0.000000pt}%
\definecolor{currentstroke}{rgb}{0.000000,0.000000,0.000000}%
\pgfsetstrokecolor{currentstroke}%
\pgfsetstrokeopacity{0.000000}%
\pgfsetdash{}{0pt}%
\pgfpathmoveto{\pgfqpoint{3.987529in}{1.823242in}}%
\pgfpathlineto{\pgfqpoint{3.996465in}{1.823242in}}%
\pgfpathlineto{\pgfqpoint{3.996465in}{1.840728in}}%
\pgfpathlineto{\pgfqpoint{3.987529in}{1.840728in}}%
\pgfpathlineto{\pgfqpoint{3.987529in}{1.823242in}}%
\pgfpathclose%
\pgfusepath{fill}%
\end{pgfscope}%
\begin{pgfscope}%
\pgfpathrectangle{\pgfqpoint{3.722897in}{0.857143in}}{\pgfqpoint{2.627103in}{1.813434in}}%
\pgfusepath{clip}%
\pgfsetbuttcap%
\pgfsetmiterjoin%
\definecolor{currentfill}{rgb}{0.133298,0.375282,0.379395}%
\pgfsetfillcolor{currentfill}%
\pgfsetlinewidth{0.000000pt}%
\definecolor{currentstroke}{rgb}{0.000000,0.000000,0.000000}%
\pgfsetstrokecolor{currentstroke}%
\pgfsetstrokeopacity{0.000000}%
\pgfsetdash{}{0pt}%
\pgfpathmoveto{\pgfqpoint{3.998699in}{1.813884in}}%
\pgfpathlineto{\pgfqpoint{4.007636in}{1.813884in}}%
\pgfpathlineto{\pgfqpoint{4.007636in}{1.809848in}}%
\pgfpathlineto{\pgfqpoint{3.998699in}{1.809848in}}%
\pgfpathlineto{\pgfqpoint{3.998699in}{1.813884in}}%
\pgfpathclose%
\pgfusepath{fill}%
\end{pgfscope}%
\begin{pgfscope}%
\pgfpathrectangle{\pgfqpoint{3.722897in}{0.857143in}}{\pgfqpoint{2.627103in}{1.813434in}}%
\pgfusepath{clip}%
\pgfsetbuttcap%
\pgfsetmiterjoin%
\definecolor{currentfill}{rgb}{0.133298,0.375282,0.379395}%
\pgfsetfillcolor{currentfill}%
\pgfsetlinewidth{0.000000pt}%
\definecolor{currentstroke}{rgb}{0.000000,0.000000,0.000000}%
\pgfsetstrokecolor{currentstroke}%
\pgfsetstrokeopacity{0.000000}%
\pgfsetdash{}{0pt}%
\pgfpathmoveto{\pgfqpoint{4.009870in}{1.813884in}}%
\pgfpathlineto{\pgfqpoint{4.018806in}{1.813884in}}%
\pgfpathlineto{\pgfqpoint{4.018806in}{1.808889in}}%
\pgfpathlineto{\pgfqpoint{4.009870in}{1.808889in}}%
\pgfpathlineto{\pgfqpoint{4.009870in}{1.813884in}}%
\pgfpathclose%
\pgfusepath{fill}%
\end{pgfscope}%
\begin{pgfscope}%
\pgfpathrectangle{\pgfqpoint{3.722897in}{0.857143in}}{\pgfqpoint{2.627103in}{1.813434in}}%
\pgfusepath{clip}%
\pgfsetbuttcap%
\pgfsetmiterjoin%
\definecolor{currentfill}{rgb}{0.133298,0.375282,0.379395}%
\pgfsetfillcolor{currentfill}%
\pgfsetlinewidth{0.000000pt}%
\definecolor{currentstroke}{rgb}{0.000000,0.000000,0.000000}%
\pgfsetstrokecolor{currentstroke}%
\pgfsetstrokeopacity{0.000000}%
\pgfsetdash{}{0pt}%
\pgfpathmoveto{\pgfqpoint{4.021040in}{1.827211in}}%
\pgfpathlineto{\pgfqpoint{4.029977in}{1.827211in}}%
\pgfpathlineto{\pgfqpoint{4.029977in}{1.835410in}}%
\pgfpathlineto{\pgfqpoint{4.021040in}{1.835410in}}%
\pgfpathlineto{\pgfqpoint{4.021040in}{1.827211in}}%
\pgfpathclose%
\pgfusepath{fill}%
\end{pgfscope}%
\begin{pgfscope}%
\pgfpathrectangle{\pgfqpoint{3.722897in}{0.857143in}}{\pgfqpoint{2.627103in}{1.813434in}}%
\pgfusepath{clip}%
\pgfsetbuttcap%
\pgfsetmiterjoin%
\definecolor{currentfill}{rgb}{0.133298,0.375282,0.379395}%
\pgfsetfillcolor{currentfill}%
\pgfsetlinewidth{0.000000pt}%
\definecolor{currentstroke}{rgb}{0.000000,0.000000,0.000000}%
\pgfsetstrokecolor{currentstroke}%
\pgfsetstrokeopacity{0.000000}%
\pgfsetdash{}{0pt}%
\pgfpathmoveto{\pgfqpoint{4.032211in}{1.813884in}}%
\pgfpathlineto{\pgfqpoint{4.041148in}{1.813884in}}%
\pgfpathlineto{\pgfqpoint{4.041148in}{1.795260in}}%
\pgfpathlineto{\pgfqpoint{4.032211in}{1.795260in}}%
\pgfpathlineto{\pgfqpoint{4.032211in}{1.813884in}}%
\pgfpathclose%
\pgfusepath{fill}%
\end{pgfscope}%
\begin{pgfscope}%
\pgfpathrectangle{\pgfqpoint{3.722897in}{0.857143in}}{\pgfqpoint{2.627103in}{1.813434in}}%
\pgfusepath{clip}%
\pgfsetbuttcap%
\pgfsetmiterjoin%
\definecolor{currentfill}{rgb}{0.133298,0.375282,0.379395}%
\pgfsetfillcolor{currentfill}%
\pgfsetlinewidth{0.000000pt}%
\definecolor{currentstroke}{rgb}{0.000000,0.000000,0.000000}%
\pgfsetstrokecolor{currentstroke}%
\pgfsetstrokeopacity{0.000000}%
\pgfsetdash{}{0pt}%
\pgfpathmoveto{\pgfqpoint{4.043382in}{1.813884in}}%
\pgfpathlineto{\pgfqpoint{4.052318in}{1.813884in}}%
\pgfpathlineto{\pgfqpoint{4.052318in}{1.797959in}}%
\pgfpathlineto{\pgfqpoint{4.043382in}{1.797959in}}%
\pgfpathlineto{\pgfqpoint{4.043382in}{1.813884in}}%
\pgfpathclose%
\pgfusepath{fill}%
\end{pgfscope}%
\begin{pgfscope}%
\pgfpathrectangle{\pgfqpoint{3.722897in}{0.857143in}}{\pgfqpoint{2.627103in}{1.813434in}}%
\pgfusepath{clip}%
\pgfsetbuttcap%
\pgfsetmiterjoin%
\definecolor{currentfill}{rgb}{0.133298,0.375282,0.379395}%
\pgfsetfillcolor{currentfill}%
\pgfsetlinewidth{0.000000pt}%
\definecolor{currentstroke}{rgb}{0.000000,0.000000,0.000000}%
\pgfsetstrokecolor{currentstroke}%
\pgfsetstrokeopacity{0.000000}%
\pgfsetdash{}{0pt}%
\pgfpathmoveto{\pgfqpoint{4.054552in}{1.813884in}}%
\pgfpathlineto{\pgfqpoint{4.063489in}{1.813884in}}%
\pgfpathlineto{\pgfqpoint{4.063489in}{1.794360in}}%
\pgfpathlineto{\pgfqpoint{4.054552in}{1.794360in}}%
\pgfpathlineto{\pgfqpoint{4.054552in}{1.813884in}}%
\pgfpathclose%
\pgfusepath{fill}%
\end{pgfscope}%
\begin{pgfscope}%
\pgfpathrectangle{\pgfqpoint{3.722897in}{0.857143in}}{\pgfqpoint{2.627103in}{1.813434in}}%
\pgfusepath{clip}%
\pgfsetbuttcap%
\pgfsetmiterjoin%
\definecolor{currentfill}{rgb}{0.133298,0.375282,0.379395}%
\pgfsetfillcolor{currentfill}%
\pgfsetlinewidth{0.000000pt}%
\definecolor{currentstroke}{rgb}{0.000000,0.000000,0.000000}%
\pgfsetstrokecolor{currentstroke}%
\pgfsetstrokeopacity{0.000000}%
\pgfsetdash{}{0pt}%
\pgfpathmoveto{\pgfqpoint{4.065723in}{1.813884in}}%
\pgfpathlineto{\pgfqpoint{4.074659in}{1.813884in}}%
\pgfpathlineto{\pgfqpoint{4.074659in}{1.803325in}}%
\pgfpathlineto{\pgfqpoint{4.065723in}{1.803325in}}%
\pgfpathlineto{\pgfqpoint{4.065723in}{1.813884in}}%
\pgfpathclose%
\pgfusepath{fill}%
\end{pgfscope}%
\begin{pgfscope}%
\pgfpathrectangle{\pgfqpoint{3.722897in}{0.857143in}}{\pgfqpoint{2.627103in}{1.813434in}}%
\pgfusepath{clip}%
\pgfsetbuttcap%
\pgfsetmiterjoin%
\definecolor{currentfill}{rgb}{0.133298,0.375282,0.379395}%
\pgfsetfillcolor{currentfill}%
\pgfsetlinewidth{0.000000pt}%
\definecolor{currentstroke}{rgb}{0.000000,0.000000,0.000000}%
\pgfsetstrokecolor{currentstroke}%
\pgfsetstrokeopacity{0.000000}%
\pgfsetdash{}{0pt}%
\pgfpathmoveto{\pgfqpoint{4.076893in}{1.832151in}}%
\pgfpathlineto{\pgfqpoint{4.085830in}{1.832151in}}%
\pgfpathlineto{\pgfqpoint{4.085830in}{1.834801in}}%
\pgfpathlineto{\pgfqpoint{4.076893in}{1.834801in}}%
\pgfpathlineto{\pgfqpoint{4.076893in}{1.832151in}}%
\pgfpathclose%
\pgfusepath{fill}%
\end{pgfscope}%
\begin{pgfscope}%
\pgfpathrectangle{\pgfqpoint{3.722897in}{0.857143in}}{\pgfqpoint{2.627103in}{1.813434in}}%
\pgfusepath{clip}%
\pgfsetbuttcap%
\pgfsetmiterjoin%
\definecolor{currentfill}{rgb}{0.133298,0.375282,0.379395}%
\pgfsetfillcolor{currentfill}%
\pgfsetlinewidth{0.000000pt}%
\definecolor{currentstroke}{rgb}{0.000000,0.000000,0.000000}%
\pgfsetstrokecolor{currentstroke}%
\pgfsetstrokeopacity{0.000000}%
\pgfsetdash{}{0pt}%
\pgfpathmoveto{\pgfqpoint{4.088064in}{1.833754in}}%
\pgfpathlineto{\pgfqpoint{4.097001in}{1.833754in}}%
\pgfpathlineto{\pgfqpoint{4.097001in}{1.841410in}}%
\pgfpathlineto{\pgfqpoint{4.088064in}{1.841410in}}%
\pgfpathlineto{\pgfqpoint{4.088064in}{1.833754in}}%
\pgfpathclose%
\pgfusepath{fill}%
\end{pgfscope}%
\begin{pgfscope}%
\pgfpathrectangle{\pgfqpoint{3.722897in}{0.857143in}}{\pgfqpoint{2.627103in}{1.813434in}}%
\pgfusepath{clip}%
\pgfsetbuttcap%
\pgfsetmiterjoin%
\definecolor{currentfill}{rgb}{0.133298,0.375282,0.379395}%
\pgfsetfillcolor{currentfill}%
\pgfsetlinewidth{0.000000pt}%
\definecolor{currentstroke}{rgb}{0.000000,0.000000,0.000000}%
\pgfsetstrokecolor{currentstroke}%
\pgfsetstrokeopacity{0.000000}%
\pgfsetdash{}{0pt}%
\pgfpathmoveto{\pgfqpoint{4.099235in}{1.837100in}}%
\pgfpathlineto{\pgfqpoint{4.108171in}{1.837100in}}%
\pgfpathlineto{\pgfqpoint{4.108171in}{1.862486in}}%
\pgfpathlineto{\pgfqpoint{4.099235in}{1.862486in}}%
\pgfpathlineto{\pgfqpoint{4.099235in}{1.837100in}}%
\pgfpathclose%
\pgfusepath{fill}%
\end{pgfscope}%
\begin{pgfscope}%
\pgfpathrectangle{\pgfqpoint{3.722897in}{0.857143in}}{\pgfqpoint{2.627103in}{1.813434in}}%
\pgfusepath{clip}%
\pgfsetbuttcap%
\pgfsetmiterjoin%
\definecolor{currentfill}{rgb}{0.133298,0.375282,0.379395}%
\pgfsetfillcolor{currentfill}%
\pgfsetlinewidth{0.000000pt}%
\definecolor{currentstroke}{rgb}{0.000000,0.000000,0.000000}%
\pgfsetstrokecolor{currentstroke}%
\pgfsetstrokeopacity{0.000000}%
\pgfsetdash{}{0pt}%
\pgfpathmoveto{\pgfqpoint{4.110405in}{1.838238in}}%
\pgfpathlineto{\pgfqpoint{4.119342in}{1.838238in}}%
\pgfpathlineto{\pgfqpoint{4.119342in}{1.839849in}}%
\pgfpathlineto{\pgfqpoint{4.110405in}{1.839849in}}%
\pgfpathlineto{\pgfqpoint{4.110405in}{1.838238in}}%
\pgfpathclose%
\pgfusepath{fill}%
\end{pgfscope}%
\begin{pgfscope}%
\pgfpathrectangle{\pgfqpoint{3.722897in}{0.857143in}}{\pgfqpoint{2.627103in}{1.813434in}}%
\pgfusepath{clip}%
\pgfsetbuttcap%
\pgfsetmiterjoin%
\definecolor{currentfill}{rgb}{0.133298,0.375282,0.379395}%
\pgfsetfillcolor{currentfill}%
\pgfsetlinewidth{0.000000pt}%
\definecolor{currentstroke}{rgb}{0.000000,0.000000,0.000000}%
\pgfsetstrokecolor{currentstroke}%
\pgfsetstrokeopacity{0.000000}%
\pgfsetdash{}{0pt}%
\pgfpathmoveto{\pgfqpoint{4.121576in}{1.813884in}}%
\pgfpathlineto{\pgfqpoint{4.130512in}{1.813884in}}%
\pgfpathlineto{\pgfqpoint{4.130512in}{1.810193in}}%
\pgfpathlineto{\pgfqpoint{4.121576in}{1.810193in}}%
\pgfpathlineto{\pgfqpoint{4.121576in}{1.813884in}}%
\pgfpathclose%
\pgfusepath{fill}%
\end{pgfscope}%
\begin{pgfscope}%
\pgfpathrectangle{\pgfqpoint{3.722897in}{0.857143in}}{\pgfqpoint{2.627103in}{1.813434in}}%
\pgfusepath{clip}%
\pgfsetbuttcap%
\pgfsetmiterjoin%
\definecolor{currentfill}{rgb}{0.133298,0.375282,0.379395}%
\pgfsetfillcolor{currentfill}%
\pgfsetlinewidth{0.000000pt}%
\definecolor{currentstroke}{rgb}{0.000000,0.000000,0.000000}%
\pgfsetstrokecolor{currentstroke}%
\pgfsetstrokeopacity{0.000000}%
\pgfsetdash{}{0pt}%
\pgfpathmoveto{\pgfqpoint{4.132747in}{1.813884in}}%
\pgfpathlineto{\pgfqpoint{4.141683in}{1.813884in}}%
\pgfpathlineto{\pgfqpoint{4.141683in}{1.811703in}}%
\pgfpathlineto{\pgfqpoint{4.132747in}{1.811703in}}%
\pgfpathlineto{\pgfqpoint{4.132747in}{1.813884in}}%
\pgfpathclose%
\pgfusepath{fill}%
\end{pgfscope}%
\begin{pgfscope}%
\pgfpathrectangle{\pgfqpoint{3.722897in}{0.857143in}}{\pgfqpoint{2.627103in}{1.813434in}}%
\pgfusepath{clip}%
\pgfsetbuttcap%
\pgfsetmiterjoin%
\definecolor{currentfill}{rgb}{0.133298,0.375282,0.379395}%
\pgfsetfillcolor{currentfill}%
\pgfsetlinewidth{0.000000pt}%
\definecolor{currentstroke}{rgb}{0.000000,0.000000,0.000000}%
\pgfsetstrokecolor{currentstroke}%
\pgfsetstrokeopacity{0.000000}%
\pgfsetdash{}{0pt}%
\pgfpathmoveto{\pgfqpoint{4.143917in}{1.813884in}}%
\pgfpathlineto{\pgfqpoint{4.152854in}{1.813884in}}%
\pgfpathlineto{\pgfqpoint{4.152854in}{1.798458in}}%
\pgfpathlineto{\pgfqpoint{4.143917in}{1.798458in}}%
\pgfpathlineto{\pgfqpoint{4.143917in}{1.813884in}}%
\pgfpathclose%
\pgfusepath{fill}%
\end{pgfscope}%
\begin{pgfscope}%
\pgfpathrectangle{\pgfqpoint{3.722897in}{0.857143in}}{\pgfqpoint{2.627103in}{1.813434in}}%
\pgfusepath{clip}%
\pgfsetbuttcap%
\pgfsetmiterjoin%
\definecolor{currentfill}{rgb}{0.133298,0.375282,0.379395}%
\pgfsetfillcolor{currentfill}%
\pgfsetlinewidth{0.000000pt}%
\definecolor{currentstroke}{rgb}{0.000000,0.000000,0.000000}%
\pgfsetstrokecolor{currentstroke}%
\pgfsetstrokeopacity{0.000000}%
\pgfsetdash{}{0pt}%
\pgfpathmoveto{\pgfqpoint{4.155088in}{1.813884in}}%
\pgfpathlineto{\pgfqpoint{4.164024in}{1.813884in}}%
\pgfpathlineto{\pgfqpoint{4.164024in}{1.784715in}}%
\pgfpathlineto{\pgfqpoint{4.155088in}{1.784715in}}%
\pgfpathlineto{\pgfqpoint{4.155088in}{1.813884in}}%
\pgfpathclose%
\pgfusepath{fill}%
\end{pgfscope}%
\begin{pgfscope}%
\pgfpathrectangle{\pgfqpoint{3.722897in}{0.857143in}}{\pgfqpoint{2.627103in}{1.813434in}}%
\pgfusepath{clip}%
\pgfsetbuttcap%
\pgfsetmiterjoin%
\definecolor{currentfill}{rgb}{0.133298,0.375282,0.379395}%
\pgfsetfillcolor{currentfill}%
\pgfsetlinewidth{0.000000pt}%
\definecolor{currentstroke}{rgb}{0.000000,0.000000,0.000000}%
\pgfsetstrokecolor{currentstroke}%
\pgfsetstrokeopacity{0.000000}%
\pgfsetdash{}{0pt}%
\pgfpathmoveto{\pgfqpoint{4.166258in}{1.813884in}}%
\pgfpathlineto{\pgfqpoint{4.175195in}{1.813884in}}%
\pgfpathlineto{\pgfqpoint{4.175195in}{1.773497in}}%
\pgfpathlineto{\pgfqpoint{4.166258in}{1.773497in}}%
\pgfpathlineto{\pgfqpoint{4.166258in}{1.813884in}}%
\pgfpathclose%
\pgfusepath{fill}%
\end{pgfscope}%
\begin{pgfscope}%
\pgfpathrectangle{\pgfqpoint{3.722897in}{0.857143in}}{\pgfqpoint{2.627103in}{1.813434in}}%
\pgfusepath{clip}%
\pgfsetbuttcap%
\pgfsetmiterjoin%
\definecolor{currentfill}{rgb}{0.133298,0.375282,0.379395}%
\pgfsetfillcolor{currentfill}%
\pgfsetlinewidth{0.000000pt}%
\definecolor{currentstroke}{rgb}{0.000000,0.000000,0.000000}%
\pgfsetstrokecolor{currentstroke}%
\pgfsetstrokeopacity{0.000000}%
\pgfsetdash{}{0pt}%
\pgfpathmoveto{\pgfqpoint{4.177429in}{1.813884in}}%
\pgfpathlineto{\pgfqpoint{4.186365in}{1.813884in}}%
\pgfpathlineto{\pgfqpoint{4.186365in}{1.769606in}}%
\pgfpathlineto{\pgfqpoint{4.177429in}{1.769606in}}%
\pgfpathlineto{\pgfqpoint{4.177429in}{1.813884in}}%
\pgfpathclose%
\pgfusepath{fill}%
\end{pgfscope}%
\begin{pgfscope}%
\pgfpathrectangle{\pgfqpoint{3.722897in}{0.857143in}}{\pgfqpoint{2.627103in}{1.813434in}}%
\pgfusepath{clip}%
\pgfsetbuttcap%
\pgfsetmiterjoin%
\definecolor{currentfill}{rgb}{0.133298,0.375282,0.379395}%
\pgfsetfillcolor{currentfill}%
\pgfsetlinewidth{0.000000pt}%
\definecolor{currentstroke}{rgb}{0.000000,0.000000,0.000000}%
\pgfsetstrokecolor{currentstroke}%
\pgfsetstrokeopacity{0.000000}%
\pgfsetdash{}{0pt}%
\pgfpathmoveto{\pgfqpoint{4.188600in}{1.813884in}}%
\pgfpathlineto{\pgfqpoint{4.197536in}{1.813884in}}%
\pgfpathlineto{\pgfqpoint{4.197536in}{1.744200in}}%
\pgfpathlineto{\pgfqpoint{4.188600in}{1.744200in}}%
\pgfpathlineto{\pgfqpoint{4.188600in}{1.813884in}}%
\pgfpathclose%
\pgfusepath{fill}%
\end{pgfscope}%
\begin{pgfscope}%
\pgfpathrectangle{\pgfqpoint{3.722897in}{0.857143in}}{\pgfqpoint{2.627103in}{1.813434in}}%
\pgfusepath{clip}%
\pgfsetbuttcap%
\pgfsetmiterjoin%
\definecolor{currentfill}{rgb}{0.133298,0.375282,0.379395}%
\pgfsetfillcolor{currentfill}%
\pgfsetlinewidth{0.000000pt}%
\definecolor{currentstroke}{rgb}{0.000000,0.000000,0.000000}%
\pgfsetstrokecolor{currentstroke}%
\pgfsetstrokeopacity{0.000000}%
\pgfsetdash{}{0pt}%
\pgfpathmoveto{\pgfqpoint{4.199770in}{1.813884in}}%
\pgfpathlineto{\pgfqpoint{4.208707in}{1.813884in}}%
\pgfpathlineto{\pgfqpoint{4.208707in}{1.741816in}}%
\pgfpathlineto{\pgfqpoint{4.199770in}{1.741816in}}%
\pgfpathlineto{\pgfqpoint{4.199770in}{1.813884in}}%
\pgfpathclose%
\pgfusepath{fill}%
\end{pgfscope}%
\begin{pgfscope}%
\pgfpathrectangle{\pgfqpoint{3.722897in}{0.857143in}}{\pgfqpoint{2.627103in}{1.813434in}}%
\pgfusepath{clip}%
\pgfsetbuttcap%
\pgfsetmiterjoin%
\definecolor{currentfill}{rgb}{0.133298,0.375282,0.379395}%
\pgfsetfillcolor{currentfill}%
\pgfsetlinewidth{0.000000pt}%
\definecolor{currentstroke}{rgb}{0.000000,0.000000,0.000000}%
\pgfsetstrokecolor{currentstroke}%
\pgfsetstrokeopacity{0.000000}%
\pgfsetdash{}{0pt}%
\pgfpathmoveto{\pgfqpoint{4.210941in}{1.813884in}}%
\pgfpathlineto{\pgfqpoint{4.219877in}{1.813884in}}%
\pgfpathlineto{\pgfqpoint{4.219877in}{1.728439in}}%
\pgfpathlineto{\pgfqpoint{4.210941in}{1.728439in}}%
\pgfpathlineto{\pgfqpoint{4.210941in}{1.813884in}}%
\pgfpathclose%
\pgfusepath{fill}%
\end{pgfscope}%
\begin{pgfscope}%
\pgfpathrectangle{\pgfqpoint{3.722897in}{0.857143in}}{\pgfqpoint{2.627103in}{1.813434in}}%
\pgfusepath{clip}%
\pgfsetbuttcap%
\pgfsetmiterjoin%
\definecolor{currentfill}{rgb}{0.133298,0.375282,0.379395}%
\pgfsetfillcolor{currentfill}%
\pgfsetlinewidth{0.000000pt}%
\definecolor{currentstroke}{rgb}{0.000000,0.000000,0.000000}%
\pgfsetstrokecolor{currentstroke}%
\pgfsetstrokeopacity{0.000000}%
\pgfsetdash{}{0pt}%
\pgfpathmoveto{\pgfqpoint{4.222111in}{1.813884in}}%
\pgfpathlineto{\pgfqpoint{4.231048in}{1.813884in}}%
\pgfpathlineto{\pgfqpoint{4.231048in}{1.713129in}}%
\pgfpathlineto{\pgfqpoint{4.222111in}{1.713129in}}%
\pgfpathlineto{\pgfqpoint{4.222111in}{1.813884in}}%
\pgfpathclose%
\pgfusepath{fill}%
\end{pgfscope}%
\begin{pgfscope}%
\pgfpathrectangle{\pgfqpoint{3.722897in}{0.857143in}}{\pgfqpoint{2.627103in}{1.813434in}}%
\pgfusepath{clip}%
\pgfsetbuttcap%
\pgfsetmiterjoin%
\definecolor{currentfill}{rgb}{0.133298,0.375282,0.379395}%
\pgfsetfillcolor{currentfill}%
\pgfsetlinewidth{0.000000pt}%
\definecolor{currentstroke}{rgb}{0.000000,0.000000,0.000000}%
\pgfsetstrokecolor{currentstroke}%
\pgfsetstrokeopacity{0.000000}%
\pgfsetdash{}{0pt}%
\pgfpathmoveto{\pgfqpoint{4.233282in}{1.813884in}}%
\pgfpathlineto{\pgfqpoint{4.242218in}{1.813884in}}%
\pgfpathlineto{\pgfqpoint{4.242218in}{1.730636in}}%
\pgfpathlineto{\pgfqpoint{4.233282in}{1.730636in}}%
\pgfpathlineto{\pgfqpoint{4.233282in}{1.813884in}}%
\pgfpathclose%
\pgfusepath{fill}%
\end{pgfscope}%
\begin{pgfscope}%
\pgfpathrectangle{\pgfqpoint{3.722897in}{0.857143in}}{\pgfqpoint{2.627103in}{1.813434in}}%
\pgfusepath{clip}%
\pgfsetbuttcap%
\pgfsetmiterjoin%
\definecolor{currentfill}{rgb}{0.133298,0.375282,0.379395}%
\pgfsetfillcolor{currentfill}%
\pgfsetlinewidth{0.000000pt}%
\definecolor{currentstroke}{rgb}{0.000000,0.000000,0.000000}%
\pgfsetstrokecolor{currentstroke}%
\pgfsetstrokeopacity{0.000000}%
\pgfsetdash{}{0pt}%
\pgfpathmoveto{\pgfqpoint{4.244453in}{1.813884in}}%
\pgfpathlineto{\pgfqpoint{4.253389in}{1.813884in}}%
\pgfpathlineto{\pgfqpoint{4.253389in}{1.723090in}}%
\pgfpathlineto{\pgfqpoint{4.244453in}{1.723090in}}%
\pgfpathlineto{\pgfqpoint{4.244453in}{1.813884in}}%
\pgfpathclose%
\pgfusepath{fill}%
\end{pgfscope}%
\begin{pgfscope}%
\pgfpathrectangle{\pgfqpoint{3.722897in}{0.857143in}}{\pgfqpoint{2.627103in}{1.813434in}}%
\pgfusepath{clip}%
\pgfsetbuttcap%
\pgfsetmiterjoin%
\definecolor{currentfill}{rgb}{0.133298,0.375282,0.379395}%
\pgfsetfillcolor{currentfill}%
\pgfsetlinewidth{0.000000pt}%
\definecolor{currentstroke}{rgb}{0.000000,0.000000,0.000000}%
\pgfsetstrokecolor{currentstroke}%
\pgfsetstrokeopacity{0.000000}%
\pgfsetdash{}{0pt}%
\pgfpathmoveto{\pgfqpoint{4.255623in}{1.813884in}}%
\pgfpathlineto{\pgfqpoint{4.264560in}{1.813884in}}%
\pgfpathlineto{\pgfqpoint{4.264560in}{1.734387in}}%
\pgfpathlineto{\pgfqpoint{4.255623in}{1.734387in}}%
\pgfpathlineto{\pgfqpoint{4.255623in}{1.813884in}}%
\pgfpathclose%
\pgfusepath{fill}%
\end{pgfscope}%
\begin{pgfscope}%
\pgfpathrectangle{\pgfqpoint{3.722897in}{0.857143in}}{\pgfqpoint{2.627103in}{1.813434in}}%
\pgfusepath{clip}%
\pgfsetbuttcap%
\pgfsetmiterjoin%
\definecolor{currentfill}{rgb}{0.133298,0.375282,0.379395}%
\pgfsetfillcolor{currentfill}%
\pgfsetlinewidth{0.000000pt}%
\definecolor{currentstroke}{rgb}{0.000000,0.000000,0.000000}%
\pgfsetstrokecolor{currentstroke}%
\pgfsetstrokeopacity{0.000000}%
\pgfsetdash{}{0pt}%
\pgfpathmoveto{\pgfqpoint{4.266794in}{1.813884in}}%
\pgfpathlineto{\pgfqpoint{4.275730in}{1.813884in}}%
\pgfpathlineto{\pgfqpoint{4.275730in}{1.761793in}}%
\pgfpathlineto{\pgfqpoint{4.266794in}{1.761793in}}%
\pgfpathlineto{\pgfqpoint{4.266794in}{1.813884in}}%
\pgfpathclose%
\pgfusepath{fill}%
\end{pgfscope}%
\begin{pgfscope}%
\pgfpathrectangle{\pgfqpoint{3.722897in}{0.857143in}}{\pgfqpoint{2.627103in}{1.813434in}}%
\pgfusepath{clip}%
\pgfsetbuttcap%
\pgfsetmiterjoin%
\definecolor{currentfill}{rgb}{0.133298,0.375282,0.379395}%
\pgfsetfillcolor{currentfill}%
\pgfsetlinewidth{0.000000pt}%
\definecolor{currentstroke}{rgb}{0.000000,0.000000,0.000000}%
\pgfsetstrokecolor{currentstroke}%
\pgfsetstrokeopacity{0.000000}%
\pgfsetdash{}{0pt}%
\pgfpathmoveto{\pgfqpoint{4.277964in}{1.813884in}}%
\pgfpathlineto{\pgfqpoint{4.286901in}{1.813884in}}%
\pgfpathlineto{\pgfqpoint{4.286901in}{1.768457in}}%
\pgfpathlineto{\pgfqpoint{4.277964in}{1.768457in}}%
\pgfpathlineto{\pgfqpoint{4.277964in}{1.813884in}}%
\pgfpathclose%
\pgfusepath{fill}%
\end{pgfscope}%
\begin{pgfscope}%
\pgfpathrectangle{\pgfqpoint{3.722897in}{0.857143in}}{\pgfqpoint{2.627103in}{1.813434in}}%
\pgfusepath{clip}%
\pgfsetbuttcap%
\pgfsetmiterjoin%
\definecolor{currentfill}{rgb}{0.133298,0.375282,0.379395}%
\pgfsetfillcolor{currentfill}%
\pgfsetlinewidth{0.000000pt}%
\definecolor{currentstroke}{rgb}{0.000000,0.000000,0.000000}%
\pgfsetstrokecolor{currentstroke}%
\pgfsetstrokeopacity{0.000000}%
\pgfsetdash{}{0pt}%
\pgfpathmoveto{\pgfqpoint{4.289135in}{1.813884in}}%
\pgfpathlineto{\pgfqpoint{4.298071in}{1.813884in}}%
\pgfpathlineto{\pgfqpoint{4.298071in}{1.764907in}}%
\pgfpathlineto{\pgfqpoint{4.289135in}{1.764907in}}%
\pgfpathlineto{\pgfqpoint{4.289135in}{1.813884in}}%
\pgfpathclose%
\pgfusepath{fill}%
\end{pgfscope}%
\begin{pgfscope}%
\pgfpathrectangle{\pgfqpoint{3.722897in}{0.857143in}}{\pgfqpoint{2.627103in}{1.813434in}}%
\pgfusepath{clip}%
\pgfsetbuttcap%
\pgfsetmiterjoin%
\definecolor{currentfill}{rgb}{0.133298,0.375282,0.379395}%
\pgfsetfillcolor{currentfill}%
\pgfsetlinewidth{0.000000pt}%
\definecolor{currentstroke}{rgb}{0.000000,0.000000,0.000000}%
\pgfsetstrokecolor{currentstroke}%
\pgfsetstrokeopacity{0.000000}%
\pgfsetdash{}{0pt}%
\pgfpathmoveto{\pgfqpoint{4.300306in}{1.813884in}}%
\pgfpathlineto{\pgfqpoint{4.309242in}{1.813884in}}%
\pgfpathlineto{\pgfqpoint{4.309242in}{1.754939in}}%
\pgfpathlineto{\pgfqpoint{4.300306in}{1.754939in}}%
\pgfpathlineto{\pgfqpoint{4.300306in}{1.813884in}}%
\pgfpathclose%
\pgfusepath{fill}%
\end{pgfscope}%
\begin{pgfscope}%
\pgfpathrectangle{\pgfqpoint{3.722897in}{0.857143in}}{\pgfqpoint{2.627103in}{1.813434in}}%
\pgfusepath{clip}%
\pgfsetbuttcap%
\pgfsetmiterjoin%
\definecolor{currentfill}{rgb}{0.133298,0.375282,0.379395}%
\pgfsetfillcolor{currentfill}%
\pgfsetlinewidth{0.000000pt}%
\definecolor{currentstroke}{rgb}{0.000000,0.000000,0.000000}%
\pgfsetstrokecolor{currentstroke}%
\pgfsetstrokeopacity{0.000000}%
\pgfsetdash{}{0pt}%
\pgfpathmoveto{\pgfqpoint{4.311476in}{1.813884in}}%
\pgfpathlineto{\pgfqpoint{4.320413in}{1.813884in}}%
\pgfpathlineto{\pgfqpoint{4.320413in}{1.767192in}}%
\pgfpathlineto{\pgfqpoint{4.311476in}{1.767192in}}%
\pgfpathlineto{\pgfqpoint{4.311476in}{1.813884in}}%
\pgfpathclose%
\pgfusepath{fill}%
\end{pgfscope}%
\begin{pgfscope}%
\pgfpathrectangle{\pgfqpoint{3.722897in}{0.857143in}}{\pgfqpoint{2.627103in}{1.813434in}}%
\pgfusepath{clip}%
\pgfsetbuttcap%
\pgfsetmiterjoin%
\definecolor{currentfill}{rgb}{0.133298,0.375282,0.379395}%
\pgfsetfillcolor{currentfill}%
\pgfsetlinewidth{0.000000pt}%
\definecolor{currentstroke}{rgb}{0.000000,0.000000,0.000000}%
\pgfsetstrokecolor{currentstroke}%
\pgfsetstrokeopacity{0.000000}%
\pgfsetdash{}{0pt}%
\pgfpathmoveto{\pgfqpoint{4.322647in}{1.813884in}}%
\pgfpathlineto{\pgfqpoint{4.331583in}{1.813884in}}%
\pgfpathlineto{\pgfqpoint{4.331583in}{1.781992in}}%
\pgfpathlineto{\pgfqpoint{4.322647in}{1.781992in}}%
\pgfpathlineto{\pgfqpoint{4.322647in}{1.813884in}}%
\pgfpathclose%
\pgfusepath{fill}%
\end{pgfscope}%
\begin{pgfscope}%
\pgfpathrectangle{\pgfqpoint{3.722897in}{0.857143in}}{\pgfqpoint{2.627103in}{1.813434in}}%
\pgfusepath{clip}%
\pgfsetbuttcap%
\pgfsetmiterjoin%
\definecolor{currentfill}{rgb}{0.133298,0.375282,0.379395}%
\pgfsetfillcolor{currentfill}%
\pgfsetlinewidth{0.000000pt}%
\definecolor{currentstroke}{rgb}{0.000000,0.000000,0.000000}%
\pgfsetstrokecolor{currentstroke}%
\pgfsetstrokeopacity{0.000000}%
\pgfsetdash{}{0pt}%
\pgfpathmoveto{\pgfqpoint{4.333817in}{1.813884in}}%
\pgfpathlineto{\pgfqpoint{4.342754in}{1.813884in}}%
\pgfpathlineto{\pgfqpoint{4.342754in}{1.791256in}}%
\pgfpathlineto{\pgfqpoint{4.333817in}{1.791256in}}%
\pgfpathlineto{\pgfqpoint{4.333817in}{1.813884in}}%
\pgfpathclose%
\pgfusepath{fill}%
\end{pgfscope}%
\begin{pgfscope}%
\pgfpathrectangle{\pgfqpoint{3.722897in}{0.857143in}}{\pgfqpoint{2.627103in}{1.813434in}}%
\pgfusepath{clip}%
\pgfsetbuttcap%
\pgfsetmiterjoin%
\definecolor{currentfill}{rgb}{0.133298,0.375282,0.379395}%
\pgfsetfillcolor{currentfill}%
\pgfsetlinewidth{0.000000pt}%
\definecolor{currentstroke}{rgb}{0.000000,0.000000,0.000000}%
\pgfsetstrokecolor{currentstroke}%
\pgfsetstrokeopacity{0.000000}%
\pgfsetdash{}{0pt}%
\pgfpathmoveto{\pgfqpoint{4.344988in}{1.813884in}}%
\pgfpathlineto{\pgfqpoint{4.353925in}{1.813884in}}%
\pgfpathlineto{\pgfqpoint{4.353925in}{1.759372in}}%
\pgfpathlineto{\pgfqpoint{4.344988in}{1.759372in}}%
\pgfpathlineto{\pgfqpoint{4.344988in}{1.813884in}}%
\pgfpathclose%
\pgfusepath{fill}%
\end{pgfscope}%
\begin{pgfscope}%
\pgfpathrectangle{\pgfqpoint{3.722897in}{0.857143in}}{\pgfqpoint{2.627103in}{1.813434in}}%
\pgfusepath{clip}%
\pgfsetbuttcap%
\pgfsetmiterjoin%
\definecolor{currentfill}{rgb}{0.133298,0.375282,0.379395}%
\pgfsetfillcolor{currentfill}%
\pgfsetlinewidth{0.000000pt}%
\definecolor{currentstroke}{rgb}{0.000000,0.000000,0.000000}%
\pgfsetstrokecolor{currentstroke}%
\pgfsetstrokeopacity{0.000000}%
\pgfsetdash{}{0pt}%
\pgfpathmoveto{\pgfqpoint{4.356159in}{1.813884in}}%
\pgfpathlineto{\pgfqpoint{4.365095in}{1.813884in}}%
\pgfpathlineto{\pgfqpoint{4.365095in}{1.794322in}}%
\pgfpathlineto{\pgfqpoint{4.356159in}{1.794322in}}%
\pgfpathlineto{\pgfqpoint{4.356159in}{1.813884in}}%
\pgfpathclose%
\pgfusepath{fill}%
\end{pgfscope}%
\begin{pgfscope}%
\pgfpathrectangle{\pgfqpoint{3.722897in}{0.857143in}}{\pgfqpoint{2.627103in}{1.813434in}}%
\pgfusepath{clip}%
\pgfsetbuttcap%
\pgfsetmiterjoin%
\definecolor{currentfill}{rgb}{0.133298,0.375282,0.379395}%
\pgfsetfillcolor{currentfill}%
\pgfsetlinewidth{0.000000pt}%
\definecolor{currentstroke}{rgb}{0.000000,0.000000,0.000000}%
\pgfsetstrokecolor{currentstroke}%
\pgfsetstrokeopacity{0.000000}%
\pgfsetdash{}{0pt}%
\pgfpathmoveto{\pgfqpoint{4.367329in}{1.813884in}}%
\pgfpathlineto{\pgfqpoint{4.376266in}{1.813884in}}%
\pgfpathlineto{\pgfqpoint{4.376266in}{1.785520in}}%
\pgfpathlineto{\pgfqpoint{4.367329in}{1.785520in}}%
\pgfpathlineto{\pgfqpoint{4.367329in}{1.813884in}}%
\pgfpathclose%
\pgfusepath{fill}%
\end{pgfscope}%
\begin{pgfscope}%
\pgfpathrectangle{\pgfqpoint{3.722897in}{0.857143in}}{\pgfqpoint{2.627103in}{1.813434in}}%
\pgfusepath{clip}%
\pgfsetbuttcap%
\pgfsetmiterjoin%
\definecolor{currentfill}{rgb}{0.133298,0.375282,0.379395}%
\pgfsetfillcolor{currentfill}%
\pgfsetlinewidth{0.000000pt}%
\definecolor{currentstroke}{rgb}{0.000000,0.000000,0.000000}%
\pgfsetstrokecolor{currentstroke}%
\pgfsetstrokeopacity{0.000000}%
\pgfsetdash{}{0pt}%
\pgfpathmoveto{\pgfqpoint{4.378500in}{1.813884in}}%
\pgfpathlineto{\pgfqpoint{4.387436in}{1.813884in}}%
\pgfpathlineto{\pgfqpoint{4.387436in}{1.790816in}}%
\pgfpathlineto{\pgfqpoint{4.378500in}{1.790816in}}%
\pgfpathlineto{\pgfqpoint{4.378500in}{1.813884in}}%
\pgfpathclose%
\pgfusepath{fill}%
\end{pgfscope}%
\begin{pgfscope}%
\pgfpathrectangle{\pgfqpoint{3.722897in}{0.857143in}}{\pgfqpoint{2.627103in}{1.813434in}}%
\pgfusepath{clip}%
\pgfsetbuttcap%
\pgfsetmiterjoin%
\definecolor{currentfill}{rgb}{0.133298,0.375282,0.379395}%
\pgfsetfillcolor{currentfill}%
\pgfsetlinewidth{0.000000pt}%
\definecolor{currentstroke}{rgb}{0.000000,0.000000,0.000000}%
\pgfsetstrokecolor{currentstroke}%
\pgfsetstrokeopacity{0.000000}%
\pgfsetdash{}{0pt}%
\pgfpathmoveto{\pgfqpoint{4.389670in}{1.813884in}}%
\pgfpathlineto{\pgfqpoint{4.398607in}{1.813884in}}%
\pgfpathlineto{\pgfqpoint{4.398607in}{1.783062in}}%
\pgfpathlineto{\pgfqpoint{4.389670in}{1.783062in}}%
\pgfpathlineto{\pgfqpoint{4.389670in}{1.813884in}}%
\pgfpathclose%
\pgfusepath{fill}%
\end{pgfscope}%
\begin{pgfscope}%
\pgfpathrectangle{\pgfqpoint{3.722897in}{0.857143in}}{\pgfqpoint{2.627103in}{1.813434in}}%
\pgfusepath{clip}%
\pgfsetbuttcap%
\pgfsetmiterjoin%
\definecolor{currentfill}{rgb}{0.133298,0.375282,0.379395}%
\pgfsetfillcolor{currentfill}%
\pgfsetlinewidth{0.000000pt}%
\definecolor{currentstroke}{rgb}{0.000000,0.000000,0.000000}%
\pgfsetstrokecolor{currentstroke}%
\pgfsetstrokeopacity{0.000000}%
\pgfsetdash{}{0pt}%
\pgfpathmoveto{\pgfqpoint{4.400841in}{1.813884in}}%
\pgfpathlineto{\pgfqpoint{4.409778in}{1.813884in}}%
\pgfpathlineto{\pgfqpoint{4.409778in}{1.793758in}}%
\pgfpathlineto{\pgfqpoint{4.400841in}{1.793758in}}%
\pgfpathlineto{\pgfqpoint{4.400841in}{1.813884in}}%
\pgfpathclose%
\pgfusepath{fill}%
\end{pgfscope}%
\begin{pgfscope}%
\pgfpathrectangle{\pgfqpoint{3.722897in}{0.857143in}}{\pgfqpoint{2.627103in}{1.813434in}}%
\pgfusepath{clip}%
\pgfsetbuttcap%
\pgfsetmiterjoin%
\definecolor{currentfill}{rgb}{0.133298,0.375282,0.379395}%
\pgfsetfillcolor{currentfill}%
\pgfsetlinewidth{0.000000pt}%
\definecolor{currentstroke}{rgb}{0.000000,0.000000,0.000000}%
\pgfsetstrokecolor{currentstroke}%
\pgfsetstrokeopacity{0.000000}%
\pgfsetdash{}{0pt}%
\pgfpathmoveto{\pgfqpoint{4.412012in}{1.813884in}}%
\pgfpathlineto{\pgfqpoint{4.420948in}{1.813884in}}%
\pgfpathlineto{\pgfqpoint{4.420948in}{1.757088in}}%
\pgfpathlineto{\pgfqpoint{4.412012in}{1.757088in}}%
\pgfpathlineto{\pgfqpoint{4.412012in}{1.813884in}}%
\pgfpathclose%
\pgfusepath{fill}%
\end{pgfscope}%
\begin{pgfscope}%
\pgfpathrectangle{\pgfqpoint{3.722897in}{0.857143in}}{\pgfqpoint{2.627103in}{1.813434in}}%
\pgfusepath{clip}%
\pgfsetbuttcap%
\pgfsetmiterjoin%
\definecolor{currentfill}{rgb}{0.133298,0.375282,0.379395}%
\pgfsetfillcolor{currentfill}%
\pgfsetlinewidth{0.000000pt}%
\definecolor{currentstroke}{rgb}{0.000000,0.000000,0.000000}%
\pgfsetstrokecolor{currentstroke}%
\pgfsetstrokeopacity{0.000000}%
\pgfsetdash{}{0pt}%
\pgfpathmoveto{\pgfqpoint{4.423182in}{1.813884in}}%
\pgfpathlineto{\pgfqpoint{4.432119in}{1.813884in}}%
\pgfpathlineto{\pgfqpoint{4.432119in}{1.779137in}}%
\pgfpathlineto{\pgfqpoint{4.423182in}{1.779137in}}%
\pgfpathlineto{\pgfqpoint{4.423182in}{1.813884in}}%
\pgfpathclose%
\pgfusepath{fill}%
\end{pgfscope}%
\begin{pgfscope}%
\pgfpathrectangle{\pgfqpoint{3.722897in}{0.857143in}}{\pgfqpoint{2.627103in}{1.813434in}}%
\pgfusepath{clip}%
\pgfsetbuttcap%
\pgfsetmiterjoin%
\definecolor{currentfill}{rgb}{0.133298,0.375282,0.379395}%
\pgfsetfillcolor{currentfill}%
\pgfsetlinewidth{0.000000pt}%
\definecolor{currentstroke}{rgb}{0.000000,0.000000,0.000000}%
\pgfsetstrokecolor{currentstroke}%
\pgfsetstrokeopacity{0.000000}%
\pgfsetdash{}{0pt}%
\pgfpathmoveto{\pgfqpoint{4.434353in}{1.813884in}}%
\pgfpathlineto{\pgfqpoint{4.443289in}{1.813884in}}%
\pgfpathlineto{\pgfqpoint{4.443289in}{1.785918in}}%
\pgfpathlineto{\pgfqpoint{4.434353in}{1.785918in}}%
\pgfpathlineto{\pgfqpoint{4.434353in}{1.813884in}}%
\pgfpathclose%
\pgfusepath{fill}%
\end{pgfscope}%
\begin{pgfscope}%
\pgfpathrectangle{\pgfqpoint{3.722897in}{0.857143in}}{\pgfqpoint{2.627103in}{1.813434in}}%
\pgfusepath{clip}%
\pgfsetbuttcap%
\pgfsetmiterjoin%
\definecolor{currentfill}{rgb}{0.133298,0.375282,0.379395}%
\pgfsetfillcolor{currentfill}%
\pgfsetlinewidth{0.000000pt}%
\definecolor{currentstroke}{rgb}{0.000000,0.000000,0.000000}%
\pgfsetstrokecolor{currentstroke}%
\pgfsetstrokeopacity{0.000000}%
\pgfsetdash{}{0pt}%
\pgfpathmoveto{\pgfqpoint{4.445523in}{1.813884in}}%
\pgfpathlineto{\pgfqpoint{4.454460in}{1.813884in}}%
\pgfpathlineto{\pgfqpoint{4.454460in}{1.775156in}}%
\pgfpathlineto{\pgfqpoint{4.445523in}{1.775156in}}%
\pgfpathlineto{\pgfqpoint{4.445523in}{1.813884in}}%
\pgfpathclose%
\pgfusepath{fill}%
\end{pgfscope}%
\begin{pgfscope}%
\pgfpathrectangle{\pgfqpoint{3.722897in}{0.857143in}}{\pgfqpoint{2.627103in}{1.813434in}}%
\pgfusepath{clip}%
\pgfsetbuttcap%
\pgfsetmiterjoin%
\definecolor{currentfill}{rgb}{0.133298,0.375282,0.379395}%
\pgfsetfillcolor{currentfill}%
\pgfsetlinewidth{0.000000pt}%
\definecolor{currentstroke}{rgb}{0.000000,0.000000,0.000000}%
\pgfsetstrokecolor{currentstroke}%
\pgfsetstrokeopacity{0.000000}%
\pgfsetdash{}{0pt}%
\pgfpathmoveto{\pgfqpoint{4.456694in}{1.813884in}}%
\pgfpathlineto{\pgfqpoint{4.465631in}{1.813884in}}%
\pgfpathlineto{\pgfqpoint{4.465631in}{1.776308in}}%
\pgfpathlineto{\pgfqpoint{4.456694in}{1.776308in}}%
\pgfpathlineto{\pgfqpoint{4.456694in}{1.813884in}}%
\pgfpathclose%
\pgfusepath{fill}%
\end{pgfscope}%
\begin{pgfscope}%
\pgfpathrectangle{\pgfqpoint{3.722897in}{0.857143in}}{\pgfqpoint{2.627103in}{1.813434in}}%
\pgfusepath{clip}%
\pgfsetbuttcap%
\pgfsetmiterjoin%
\definecolor{currentfill}{rgb}{0.133298,0.375282,0.379395}%
\pgfsetfillcolor{currentfill}%
\pgfsetlinewidth{0.000000pt}%
\definecolor{currentstroke}{rgb}{0.000000,0.000000,0.000000}%
\pgfsetstrokecolor{currentstroke}%
\pgfsetstrokeopacity{0.000000}%
\pgfsetdash{}{0pt}%
\pgfpathmoveto{\pgfqpoint{4.467865in}{1.813884in}}%
\pgfpathlineto{\pgfqpoint{4.476801in}{1.813884in}}%
\pgfpathlineto{\pgfqpoint{4.476801in}{1.765107in}}%
\pgfpathlineto{\pgfqpoint{4.467865in}{1.765107in}}%
\pgfpathlineto{\pgfqpoint{4.467865in}{1.813884in}}%
\pgfpathclose%
\pgfusepath{fill}%
\end{pgfscope}%
\begin{pgfscope}%
\pgfpathrectangle{\pgfqpoint{3.722897in}{0.857143in}}{\pgfqpoint{2.627103in}{1.813434in}}%
\pgfusepath{clip}%
\pgfsetbuttcap%
\pgfsetmiterjoin%
\definecolor{currentfill}{rgb}{0.133298,0.375282,0.379395}%
\pgfsetfillcolor{currentfill}%
\pgfsetlinewidth{0.000000pt}%
\definecolor{currentstroke}{rgb}{0.000000,0.000000,0.000000}%
\pgfsetstrokecolor{currentstroke}%
\pgfsetstrokeopacity{0.000000}%
\pgfsetdash{}{0pt}%
\pgfpathmoveto{\pgfqpoint{4.479035in}{1.813884in}}%
\pgfpathlineto{\pgfqpoint{4.487972in}{1.813884in}}%
\pgfpathlineto{\pgfqpoint{4.487972in}{1.753649in}}%
\pgfpathlineto{\pgfqpoint{4.479035in}{1.753649in}}%
\pgfpathlineto{\pgfqpoint{4.479035in}{1.813884in}}%
\pgfpathclose%
\pgfusepath{fill}%
\end{pgfscope}%
\begin{pgfscope}%
\pgfpathrectangle{\pgfqpoint{3.722897in}{0.857143in}}{\pgfqpoint{2.627103in}{1.813434in}}%
\pgfusepath{clip}%
\pgfsetbuttcap%
\pgfsetmiterjoin%
\definecolor{currentfill}{rgb}{0.133298,0.375282,0.379395}%
\pgfsetfillcolor{currentfill}%
\pgfsetlinewidth{0.000000pt}%
\definecolor{currentstroke}{rgb}{0.000000,0.000000,0.000000}%
\pgfsetstrokecolor{currentstroke}%
\pgfsetstrokeopacity{0.000000}%
\pgfsetdash{}{0pt}%
\pgfpathmoveto{\pgfqpoint{4.490206in}{1.813884in}}%
\pgfpathlineto{\pgfqpoint{4.499142in}{1.813884in}}%
\pgfpathlineto{\pgfqpoint{4.499142in}{1.761830in}}%
\pgfpathlineto{\pgfqpoint{4.490206in}{1.761830in}}%
\pgfpathlineto{\pgfqpoint{4.490206in}{1.813884in}}%
\pgfpathclose%
\pgfusepath{fill}%
\end{pgfscope}%
\begin{pgfscope}%
\pgfpathrectangle{\pgfqpoint{3.722897in}{0.857143in}}{\pgfqpoint{2.627103in}{1.813434in}}%
\pgfusepath{clip}%
\pgfsetbuttcap%
\pgfsetmiterjoin%
\definecolor{currentfill}{rgb}{0.133298,0.375282,0.379395}%
\pgfsetfillcolor{currentfill}%
\pgfsetlinewidth{0.000000pt}%
\definecolor{currentstroke}{rgb}{0.000000,0.000000,0.000000}%
\pgfsetstrokecolor{currentstroke}%
\pgfsetstrokeopacity{0.000000}%
\pgfsetdash{}{0pt}%
\pgfpathmoveto{\pgfqpoint{4.501377in}{1.813884in}}%
\pgfpathlineto{\pgfqpoint{4.510313in}{1.813884in}}%
\pgfpathlineto{\pgfqpoint{4.510313in}{1.789923in}}%
\pgfpathlineto{\pgfqpoint{4.501377in}{1.789923in}}%
\pgfpathlineto{\pgfqpoint{4.501377in}{1.813884in}}%
\pgfpathclose%
\pgfusepath{fill}%
\end{pgfscope}%
\begin{pgfscope}%
\pgfpathrectangle{\pgfqpoint{3.722897in}{0.857143in}}{\pgfqpoint{2.627103in}{1.813434in}}%
\pgfusepath{clip}%
\pgfsetbuttcap%
\pgfsetmiterjoin%
\definecolor{currentfill}{rgb}{0.133298,0.375282,0.379395}%
\pgfsetfillcolor{currentfill}%
\pgfsetlinewidth{0.000000pt}%
\definecolor{currentstroke}{rgb}{0.000000,0.000000,0.000000}%
\pgfsetstrokecolor{currentstroke}%
\pgfsetstrokeopacity{0.000000}%
\pgfsetdash{}{0pt}%
\pgfpathmoveto{\pgfqpoint{4.512547in}{1.813884in}}%
\pgfpathlineto{\pgfqpoint{4.521484in}{1.813884in}}%
\pgfpathlineto{\pgfqpoint{4.521484in}{1.782142in}}%
\pgfpathlineto{\pgfqpoint{4.512547in}{1.782142in}}%
\pgfpathlineto{\pgfqpoint{4.512547in}{1.813884in}}%
\pgfpathclose%
\pgfusepath{fill}%
\end{pgfscope}%
\begin{pgfscope}%
\pgfpathrectangle{\pgfqpoint{3.722897in}{0.857143in}}{\pgfqpoint{2.627103in}{1.813434in}}%
\pgfusepath{clip}%
\pgfsetbuttcap%
\pgfsetmiterjoin%
\definecolor{currentfill}{rgb}{0.133298,0.375282,0.379395}%
\pgfsetfillcolor{currentfill}%
\pgfsetlinewidth{0.000000pt}%
\definecolor{currentstroke}{rgb}{0.000000,0.000000,0.000000}%
\pgfsetstrokecolor{currentstroke}%
\pgfsetstrokeopacity{0.000000}%
\pgfsetdash{}{0pt}%
\pgfpathmoveto{\pgfqpoint{4.523718in}{1.813884in}}%
\pgfpathlineto{\pgfqpoint{4.532654in}{1.813884in}}%
\pgfpathlineto{\pgfqpoint{4.532654in}{1.788642in}}%
\pgfpathlineto{\pgfqpoint{4.523718in}{1.788642in}}%
\pgfpathlineto{\pgfqpoint{4.523718in}{1.813884in}}%
\pgfpathclose%
\pgfusepath{fill}%
\end{pgfscope}%
\begin{pgfscope}%
\pgfpathrectangle{\pgfqpoint{3.722897in}{0.857143in}}{\pgfqpoint{2.627103in}{1.813434in}}%
\pgfusepath{clip}%
\pgfsetbuttcap%
\pgfsetmiterjoin%
\definecolor{currentfill}{rgb}{0.133298,0.375282,0.379395}%
\pgfsetfillcolor{currentfill}%
\pgfsetlinewidth{0.000000pt}%
\definecolor{currentstroke}{rgb}{0.000000,0.000000,0.000000}%
\pgfsetstrokecolor{currentstroke}%
\pgfsetstrokeopacity{0.000000}%
\pgfsetdash{}{0pt}%
\pgfpathmoveto{\pgfqpoint{4.534888in}{1.813884in}}%
\pgfpathlineto{\pgfqpoint{4.543825in}{1.813884in}}%
\pgfpathlineto{\pgfqpoint{4.543825in}{1.793703in}}%
\pgfpathlineto{\pgfqpoint{4.534888in}{1.793703in}}%
\pgfpathlineto{\pgfqpoint{4.534888in}{1.813884in}}%
\pgfpathclose%
\pgfusepath{fill}%
\end{pgfscope}%
\begin{pgfscope}%
\pgfpathrectangle{\pgfqpoint{3.722897in}{0.857143in}}{\pgfqpoint{2.627103in}{1.813434in}}%
\pgfusepath{clip}%
\pgfsetbuttcap%
\pgfsetmiterjoin%
\definecolor{currentfill}{rgb}{0.133298,0.375282,0.379395}%
\pgfsetfillcolor{currentfill}%
\pgfsetlinewidth{0.000000pt}%
\definecolor{currentstroke}{rgb}{0.000000,0.000000,0.000000}%
\pgfsetstrokecolor{currentstroke}%
\pgfsetstrokeopacity{0.000000}%
\pgfsetdash{}{0pt}%
\pgfpathmoveto{\pgfqpoint{4.546059in}{1.813061in}}%
\pgfpathlineto{\pgfqpoint{4.554995in}{1.813061in}}%
\pgfpathlineto{\pgfqpoint{4.554995in}{1.781798in}}%
\pgfpathlineto{\pgfqpoint{4.546059in}{1.781798in}}%
\pgfpathlineto{\pgfqpoint{4.546059in}{1.813061in}}%
\pgfpathclose%
\pgfusepath{fill}%
\end{pgfscope}%
\begin{pgfscope}%
\pgfpathrectangle{\pgfqpoint{3.722897in}{0.857143in}}{\pgfqpoint{2.627103in}{1.813434in}}%
\pgfusepath{clip}%
\pgfsetbuttcap%
\pgfsetmiterjoin%
\definecolor{currentfill}{rgb}{0.133298,0.375282,0.379395}%
\pgfsetfillcolor{currentfill}%
\pgfsetlinewidth{0.000000pt}%
\definecolor{currentstroke}{rgb}{0.000000,0.000000,0.000000}%
\pgfsetstrokecolor{currentstroke}%
\pgfsetstrokeopacity{0.000000}%
\pgfsetdash{}{0pt}%
\pgfpathmoveto{\pgfqpoint{4.557230in}{1.810996in}}%
\pgfpathlineto{\pgfqpoint{4.566166in}{1.810996in}}%
\pgfpathlineto{\pgfqpoint{4.566166in}{1.780281in}}%
\pgfpathlineto{\pgfqpoint{4.557230in}{1.780281in}}%
\pgfpathlineto{\pgfqpoint{4.557230in}{1.810996in}}%
\pgfpathclose%
\pgfusepath{fill}%
\end{pgfscope}%
\begin{pgfscope}%
\pgfpathrectangle{\pgfqpoint{3.722897in}{0.857143in}}{\pgfqpoint{2.627103in}{1.813434in}}%
\pgfusepath{clip}%
\pgfsetbuttcap%
\pgfsetmiterjoin%
\definecolor{currentfill}{rgb}{0.133298,0.375282,0.379395}%
\pgfsetfillcolor{currentfill}%
\pgfsetlinewidth{0.000000pt}%
\definecolor{currentstroke}{rgb}{0.000000,0.000000,0.000000}%
\pgfsetstrokecolor{currentstroke}%
\pgfsetstrokeopacity{0.000000}%
\pgfsetdash{}{0pt}%
\pgfpathmoveto{\pgfqpoint{4.568400in}{1.810353in}}%
\pgfpathlineto{\pgfqpoint{4.577337in}{1.810353in}}%
\pgfpathlineto{\pgfqpoint{4.577337in}{1.773705in}}%
\pgfpathlineto{\pgfqpoint{4.568400in}{1.773705in}}%
\pgfpathlineto{\pgfqpoint{4.568400in}{1.810353in}}%
\pgfpathclose%
\pgfusepath{fill}%
\end{pgfscope}%
\begin{pgfscope}%
\pgfpathrectangle{\pgfqpoint{3.722897in}{0.857143in}}{\pgfqpoint{2.627103in}{1.813434in}}%
\pgfusepath{clip}%
\pgfsetbuttcap%
\pgfsetmiterjoin%
\definecolor{currentfill}{rgb}{0.133298,0.375282,0.379395}%
\pgfsetfillcolor{currentfill}%
\pgfsetlinewidth{0.000000pt}%
\definecolor{currentstroke}{rgb}{0.000000,0.000000,0.000000}%
\pgfsetstrokecolor{currentstroke}%
\pgfsetstrokeopacity{0.000000}%
\pgfsetdash{}{0pt}%
\pgfpathmoveto{\pgfqpoint{4.579571in}{1.809360in}}%
\pgfpathlineto{\pgfqpoint{4.588507in}{1.809360in}}%
\pgfpathlineto{\pgfqpoint{4.588507in}{1.787806in}}%
\pgfpathlineto{\pgfqpoint{4.579571in}{1.787806in}}%
\pgfpathlineto{\pgfqpoint{4.579571in}{1.809360in}}%
\pgfpathclose%
\pgfusepath{fill}%
\end{pgfscope}%
\begin{pgfscope}%
\pgfpathrectangle{\pgfqpoint{3.722897in}{0.857143in}}{\pgfqpoint{2.627103in}{1.813434in}}%
\pgfusepath{clip}%
\pgfsetbuttcap%
\pgfsetmiterjoin%
\definecolor{currentfill}{rgb}{0.133298,0.375282,0.379395}%
\pgfsetfillcolor{currentfill}%
\pgfsetlinewidth{0.000000pt}%
\definecolor{currentstroke}{rgb}{0.000000,0.000000,0.000000}%
\pgfsetstrokecolor{currentstroke}%
\pgfsetstrokeopacity{0.000000}%
\pgfsetdash{}{0pt}%
\pgfpathmoveto{\pgfqpoint{4.590741in}{1.809515in}}%
\pgfpathlineto{\pgfqpoint{4.599678in}{1.809515in}}%
\pgfpathlineto{\pgfqpoint{4.599678in}{1.782130in}}%
\pgfpathlineto{\pgfqpoint{4.590741in}{1.782130in}}%
\pgfpathlineto{\pgfqpoint{4.590741in}{1.809515in}}%
\pgfpathclose%
\pgfusepath{fill}%
\end{pgfscope}%
\begin{pgfscope}%
\pgfpathrectangle{\pgfqpoint{3.722897in}{0.857143in}}{\pgfqpoint{2.627103in}{1.813434in}}%
\pgfusepath{clip}%
\pgfsetbuttcap%
\pgfsetmiterjoin%
\definecolor{currentfill}{rgb}{0.133298,0.375282,0.379395}%
\pgfsetfillcolor{currentfill}%
\pgfsetlinewidth{0.000000pt}%
\definecolor{currentstroke}{rgb}{0.000000,0.000000,0.000000}%
\pgfsetstrokecolor{currentstroke}%
\pgfsetstrokeopacity{0.000000}%
\pgfsetdash{}{0pt}%
\pgfpathmoveto{\pgfqpoint{4.601912in}{1.807622in}}%
\pgfpathlineto{\pgfqpoint{4.610848in}{1.807622in}}%
\pgfpathlineto{\pgfqpoint{4.610848in}{1.778007in}}%
\pgfpathlineto{\pgfqpoint{4.601912in}{1.778007in}}%
\pgfpathlineto{\pgfqpoint{4.601912in}{1.807622in}}%
\pgfpathclose%
\pgfusepath{fill}%
\end{pgfscope}%
\begin{pgfscope}%
\pgfpathrectangle{\pgfqpoint{3.722897in}{0.857143in}}{\pgfqpoint{2.627103in}{1.813434in}}%
\pgfusepath{clip}%
\pgfsetbuttcap%
\pgfsetmiterjoin%
\definecolor{currentfill}{rgb}{0.133298,0.375282,0.379395}%
\pgfsetfillcolor{currentfill}%
\pgfsetlinewidth{0.000000pt}%
\definecolor{currentstroke}{rgb}{0.000000,0.000000,0.000000}%
\pgfsetstrokecolor{currentstroke}%
\pgfsetstrokeopacity{0.000000}%
\pgfsetdash{}{0pt}%
\pgfpathmoveto{\pgfqpoint{4.613083in}{1.806230in}}%
\pgfpathlineto{\pgfqpoint{4.622019in}{1.806230in}}%
\pgfpathlineto{\pgfqpoint{4.622019in}{1.794736in}}%
\pgfpathlineto{\pgfqpoint{4.613083in}{1.794736in}}%
\pgfpathlineto{\pgfqpoint{4.613083in}{1.806230in}}%
\pgfpathclose%
\pgfusepath{fill}%
\end{pgfscope}%
\begin{pgfscope}%
\pgfpathrectangle{\pgfqpoint{3.722897in}{0.857143in}}{\pgfqpoint{2.627103in}{1.813434in}}%
\pgfusepath{clip}%
\pgfsetbuttcap%
\pgfsetmiterjoin%
\definecolor{currentfill}{rgb}{0.133298,0.375282,0.379395}%
\pgfsetfillcolor{currentfill}%
\pgfsetlinewidth{0.000000pt}%
\definecolor{currentstroke}{rgb}{0.000000,0.000000,0.000000}%
\pgfsetstrokecolor{currentstroke}%
\pgfsetstrokeopacity{0.000000}%
\pgfsetdash{}{0pt}%
\pgfpathmoveto{\pgfqpoint{4.624253in}{1.804432in}}%
\pgfpathlineto{\pgfqpoint{4.633190in}{1.804432in}}%
\pgfpathlineto{\pgfqpoint{4.633190in}{1.780348in}}%
\pgfpathlineto{\pgfqpoint{4.624253in}{1.780348in}}%
\pgfpathlineto{\pgfqpoint{4.624253in}{1.804432in}}%
\pgfpathclose%
\pgfusepath{fill}%
\end{pgfscope}%
\begin{pgfscope}%
\pgfpathrectangle{\pgfqpoint{3.722897in}{0.857143in}}{\pgfqpoint{2.627103in}{1.813434in}}%
\pgfusepath{clip}%
\pgfsetbuttcap%
\pgfsetmiterjoin%
\definecolor{currentfill}{rgb}{0.133298,0.375282,0.379395}%
\pgfsetfillcolor{currentfill}%
\pgfsetlinewidth{0.000000pt}%
\definecolor{currentstroke}{rgb}{0.000000,0.000000,0.000000}%
\pgfsetstrokecolor{currentstroke}%
\pgfsetstrokeopacity{0.000000}%
\pgfsetdash{}{0pt}%
\pgfpathmoveto{\pgfqpoint{4.635424in}{1.803371in}}%
\pgfpathlineto{\pgfqpoint{4.644360in}{1.803371in}}%
\pgfpathlineto{\pgfqpoint{4.644360in}{1.786291in}}%
\pgfpathlineto{\pgfqpoint{4.635424in}{1.786291in}}%
\pgfpathlineto{\pgfqpoint{4.635424in}{1.803371in}}%
\pgfpathclose%
\pgfusepath{fill}%
\end{pgfscope}%
\begin{pgfscope}%
\pgfpathrectangle{\pgfqpoint{3.722897in}{0.857143in}}{\pgfqpoint{2.627103in}{1.813434in}}%
\pgfusepath{clip}%
\pgfsetbuttcap%
\pgfsetmiterjoin%
\definecolor{currentfill}{rgb}{0.133298,0.375282,0.379395}%
\pgfsetfillcolor{currentfill}%
\pgfsetlinewidth{0.000000pt}%
\definecolor{currentstroke}{rgb}{0.000000,0.000000,0.000000}%
\pgfsetstrokecolor{currentstroke}%
\pgfsetstrokeopacity{0.000000}%
\pgfsetdash{}{0pt}%
\pgfpathmoveto{\pgfqpoint{4.646594in}{1.802542in}}%
\pgfpathlineto{\pgfqpoint{4.655531in}{1.802542in}}%
\pgfpathlineto{\pgfqpoint{4.655531in}{1.782089in}}%
\pgfpathlineto{\pgfqpoint{4.646594in}{1.782089in}}%
\pgfpathlineto{\pgfqpoint{4.646594in}{1.802542in}}%
\pgfpathclose%
\pgfusepath{fill}%
\end{pgfscope}%
\begin{pgfscope}%
\pgfpathrectangle{\pgfqpoint{3.722897in}{0.857143in}}{\pgfqpoint{2.627103in}{1.813434in}}%
\pgfusepath{clip}%
\pgfsetbuttcap%
\pgfsetmiterjoin%
\definecolor{currentfill}{rgb}{0.133298,0.375282,0.379395}%
\pgfsetfillcolor{currentfill}%
\pgfsetlinewidth{0.000000pt}%
\definecolor{currentstroke}{rgb}{0.000000,0.000000,0.000000}%
\pgfsetstrokecolor{currentstroke}%
\pgfsetstrokeopacity{0.000000}%
\pgfsetdash{}{0pt}%
\pgfpathmoveto{\pgfqpoint{4.657765in}{1.800992in}}%
\pgfpathlineto{\pgfqpoint{4.666701in}{1.800992in}}%
\pgfpathlineto{\pgfqpoint{4.666701in}{1.777754in}}%
\pgfpathlineto{\pgfqpoint{4.657765in}{1.777754in}}%
\pgfpathlineto{\pgfqpoint{4.657765in}{1.800992in}}%
\pgfpathclose%
\pgfusepath{fill}%
\end{pgfscope}%
\begin{pgfscope}%
\pgfpathrectangle{\pgfqpoint{3.722897in}{0.857143in}}{\pgfqpoint{2.627103in}{1.813434in}}%
\pgfusepath{clip}%
\pgfsetbuttcap%
\pgfsetmiterjoin%
\definecolor{currentfill}{rgb}{0.133298,0.375282,0.379395}%
\pgfsetfillcolor{currentfill}%
\pgfsetlinewidth{0.000000pt}%
\definecolor{currentstroke}{rgb}{0.000000,0.000000,0.000000}%
\pgfsetstrokecolor{currentstroke}%
\pgfsetstrokeopacity{0.000000}%
\pgfsetdash{}{0pt}%
\pgfpathmoveto{\pgfqpoint{4.668936in}{1.799422in}}%
\pgfpathlineto{\pgfqpoint{4.677872in}{1.799422in}}%
\pgfpathlineto{\pgfqpoint{4.677872in}{1.767984in}}%
\pgfpathlineto{\pgfqpoint{4.668936in}{1.767984in}}%
\pgfpathlineto{\pgfqpoint{4.668936in}{1.799422in}}%
\pgfpathclose%
\pgfusepath{fill}%
\end{pgfscope}%
\begin{pgfscope}%
\pgfpathrectangle{\pgfqpoint{3.722897in}{0.857143in}}{\pgfqpoint{2.627103in}{1.813434in}}%
\pgfusepath{clip}%
\pgfsetbuttcap%
\pgfsetmiterjoin%
\definecolor{currentfill}{rgb}{0.133298,0.375282,0.379395}%
\pgfsetfillcolor{currentfill}%
\pgfsetlinewidth{0.000000pt}%
\definecolor{currentstroke}{rgb}{0.000000,0.000000,0.000000}%
\pgfsetstrokecolor{currentstroke}%
\pgfsetstrokeopacity{0.000000}%
\pgfsetdash{}{0pt}%
\pgfpathmoveto{\pgfqpoint{4.680106in}{1.797856in}}%
\pgfpathlineto{\pgfqpoint{4.689043in}{1.797856in}}%
\pgfpathlineto{\pgfqpoint{4.689043in}{1.782390in}}%
\pgfpathlineto{\pgfqpoint{4.680106in}{1.782390in}}%
\pgfpathlineto{\pgfqpoint{4.680106in}{1.797856in}}%
\pgfpathclose%
\pgfusepath{fill}%
\end{pgfscope}%
\begin{pgfscope}%
\pgfpathrectangle{\pgfqpoint{3.722897in}{0.857143in}}{\pgfqpoint{2.627103in}{1.813434in}}%
\pgfusepath{clip}%
\pgfsetbuttcap%
\pgfsetmiterjoin%
\definecolor{currentfill}{rgb}{0.133298,0.375282,0.379395}%
\pgfsetfillcolor{currentfill}%
\pgfsetlinewidth{0.000000pt}%
\definecolor{currentstroke}{rgb}{0.000000,0.000000,0.000000}%
\pgfsetstrokecolor{currentstroke}%
\pgfsetstrokeopacity{0.000000}%
\pgfsetdash{}{0pt}%
\pgfpathmoveto{\pgfqpoint{4.691277in}{1.798543in}}%
\pgfpathlineto{\pgfqpoint{4.700213in}{1.798543in}}%
\pgfpathlineto{\pgfqpoint{4.700213in}{1.779309in}}%
\pgfpathlineto{\pgfqpoint{4.691277in}{1.779309in}}%
\pgfpathlineto{\pgfqpoint{4.691277in}{1.798543in}}%
\pgfpathclose%
\pgfusepath{fill}%
\end{pgfscope}%
\begin{pgfscope}%
\pgfpathrectangle{\pgfqpoint{3.722897in}{0.857143in}}{\pgfqpoint{2.627103in}{1.813434in}}%
\pgfusepath{clip}%
\pgfsetbuttcap%
\pgfsetmiterjoin%
\definecolor{currentfill}{rgb}{0.133298,0.375282,0.379395}%
\pgfsetfillcolor{currentfill}%
\pgfsetlinewidth{0.000000pt}%
\definecolor{currentstroke}{rgb}{0.000000,0.000000,0.000000}%
\pgfsetstrokecolor{currentstroke}%
\pgfsetstrokeopacity{0.000000}%
\pgfsetdash{}{0pt}%
\pgfpathmoveto{\pgfqpoint{4.702447in}{1.797328in}}%
\pgfpathlineto{\pgfqpoint{4.711384in}{1.797328in}}%
\pgfpathlineto{\pgfqpoint{4.711384in}{1.776801in}}%
\pgfpathlineto{\pgfqpoint{4.702447in}{1.776801in}}%
\pgfpathlineto{\pgfqpoint{4.702447in}{1.797328in}}%
\pgfpathclose%
\pgfusepath{fill}%
\end{pgfscope}%
\begin{pgfscope}%
\pgfpathrectangle{\pgfqpoint{3.722897in}{0.857143in}}{\pgfqpoint{2.627103in}{1.813434in}}%
\pgfusepath{clip}%
\pgfsetbuttcap%
\pgfsetmiterjoin%
\definecolor{currentfill}{rgb}{0.133298,0.375282,0.379395}%
\pgfsetfillcolor{currentfill}%
\pgfsetlinewidth{0.000000pt}%
\definecolor{currentstroke}{rgb}{0.000000,0.000000,0.000000}%
\pgfsetstrokecolor{currentstroke}%
\pgfsetstrokeopacity{0.000000}%
\pgfsetdash{}{0pt}%
\pgfpathmoveto{\pgfqpoint{4.713618in}{1.798070in}}%
\pgfpathlineto{\pgfqpoint{4.722554in}{1.798070in}}%
\pgfpathlineto{\pgfqpoint{4.722554in}{1.786441in}}%
\pgfpathlineto{\pgfqpoint{4.713618in}{1.786441in}}%
\pgfpathlineto{\pgfqpoint{4.713618in}{1.798070in}}%
\pgfpathclose%
\pgfusepath{fill}%
\end{pgfscope}%
\begin{pgfscope}%
\pgfpathrectangle{\pgfqpoint{3.722897in}{0.857143in}}{\pgfqpoint{2.627103in}{1.813434in}}%
\pgfusepath{clip}%
\pgfsetbuttcap%
\pgfsetmiterjoin%
\definecolor{currentfill}{rgb}{0.133298,0.375282,0.379395}%
\pgfsetfillcolor{currentfill}%
\pgfsetlinewidth{0.000000pt}%
\definecolor{currentstroke}{rgb}{0.000000,0.000000,0.000000}%
\pgfsetstrokecolor{currentstroke}%
\pgfsetstrokeopacity{0.000000}%
\pgfsetdash{}{0pt}%
\pgfpathmoveto{\pgfqpoint{4.724789in}{1.798327in}}%
\pgfpathlineto{\pgfqpoint{4.733725in}{1.798327in}}%
\pgfpathlineto{\pgfqpoint{4.733725in}{1.782775in}}%
\pgfpathlineto{\pgfqpoint{4.724789in}{1.782775in}}%
\pgfpathlineto{\pgfqpoint{4.724789in}{1.798327in}}%
\pgfpathclose%
\pgfusepath{fill}%
\end{pgfscope}%
\begin{pgfscope}%
\pgfpathrectangle{\pgfqpoint{3.722897in}{0.857143in}}{\pgfqpoint{2.627103in}{1.813434in}}%
\pgfusepath{clip}%
\pgfsetbuttcap%
\pgfsetmiterjoin%
\definecolor{currentfill}{rgb}{0.133298,0.375282,0.379395}%
\pgfsetfillcolor{currentfill}%
\pgfsetlinewidth{0.000000pt}%
\definecolor{currentstroke}{rgb}{0.000000,0.000000,0.000000}%
\pgfsetstrokecolor{currentstroke}%
\pgfsetstrokeopacity{0.000000}%
\pgfsetdash{}{0pt}%
\pgfpathmoveto{\pgfqpoint{4.735959in}{1.798227in}}%
\pgfpathlineto{\pgfqpoint{4.744896in}{1.798227in}}%
\pgfpathlineto{\pgfqpoint{4.744896in}{1.780521in}}%
\pgfpathlineto{\pgfqpoint{4.735959in}{1.780521in}}%
\pgfpathlineto{\pgfqpoint{4.735959in}{1.798227in}}%
\pgfpathclose%
\pgfusepath{fill}%
\end{pgfscope}%
\begin{pgfscope}%
\pgfpathrectangle{\pgfqpoint{3.722897in}{0.857143in}}{\pgfqpoint{2.627103in}{1.813434in}}%
\pgfusepath{clip}%
\pgfsetbuttcap%
\pgfsetmiterjoin%
\definecolor{currentfill}{rgb}{0.133298,0.375282,0.379395}%
\pgfsetfillcolor{currentfill}%
\pgfsetlinewidth{0.000000pt}%
\definecolor{currentstroke}{rgb}{0.000000,0.000000,0.000000}%
\pgfsetstrokecolor{currentstroke}%
\pgfsetstrokeopacity{0.000000}%
\pgfsetdash{}{0pt}%
\pgfpathmoveto{\pgfqpoint{4.747130in}{1.796757in}}%
\pgfpathlineto{\pgfqpoint{4.756066in}{1.796757in}}%
\pgfpathlineto{\pgfqpoint{4.756066in}{1.777804in}}%
\pgfpathlineto{\pgfqpoint{4.747130in}{1.777804in}}%
\pgfpathlineto{\pgfqpoint{4.747130in}{1.796757in}}%
\pgfpathclose%
\pgfusepath{fill}%
\end{pgfscope}%
\begin{pgfscope}%
\pgfpathrectangle{\pgfqpoint{3.722897in}{0.857143in}}{\pgfqpoint{2.627103in}{1.813434in}}%
\pgfusepath{clip}%
\pgfsetbuttcap%
\pgfsetmiterjoin%
\definecolor{currentfill}{rgb}{0.133298,0.375282,0.379395}%
\pgfsetfillcolor{currentfill}%
\pgfsetlinewidth{0.000000pt}%
\definecolor{currentstroke}{rgb}{0.000000,0.000000,0.000000}%
\pgfsetstrokecolor{currentstroke}%
\pgfsetstrokeopacity{0.000000}%
\pgfsetdash{}{0pt}%
\pgfpathmoveto{\pgfqpoint{4.758300in}{1.794472in}}%
\pgfpathlineto{\pgfqpoint{4.767237in}{1.794472in}}%
\pgfpathlineto{\pgfqpoint{4.767237in}{1.781370in}}%
\pgfpathlineto{\pgfqpoint{4.758300in}{1.781370in}}%
\pgfpathlineto{\pgfqpoint{4.758300in}{1.794472in}}%
\pgfpathclose%
\pgfusepath{fill}%
\end{pgfscope}%
\begin{pgfscope}%
\pgfpathrectangle{\pgfqpoint{3.722897in}{0.857143in}}{\pgfqpoint{2.627103in}{1.813434in}}%
\pgfusepath{clip}%
\pgfsetbuttcap%
\pgfsetmiterjoin%
\definecolor{currentfill}{rgb}{0.133298,0.375282,0.379395}%
\pgfsetfillcolor{currentfill}%
\pgfsetlinewidth{0.000000pt}%
\definecolor{currentstroke}{rgb}{0.000000,0.000000,0.000000}%
\pgfsetstrokecolor{currentstroke}%
\pgfsetstrokeopacity{0.000000}%
\pgfsetdash{}{0pt}%
\pgfpathmoveto{\pgfqpoint{4.769471in}{1.794028in}}%
\pgfpathlineto{\pgfqpoint{4.778408in}{1.794028in}}%
\pgfpathlineto{\pgfqpoint{4.778408in}{1.779817in}}%
\pgfpathlineto{\pgfqpoint{4.769471in}{1.779817in}}%
\pgfpathlineto{\pgfqpoint{4.769471in}{1.794028in}}%
\pgfpathclose%
\pgfusepath{fill}%
\end{pgfscope}%
\begin{pgfscope}%
\pgfpathrectangle{\pgfqpoint{3.722897in}{0.857143in}}{\pgfqpoint{2.627103in}{1.813434in}}%
\pgfusepath{clip}%
\pgfsetbuttcap%
\pgfsetmiterjoin%
\definecolor{currentfill}{rgb}{0.133298,0.375282,0.379395}%
\pgfsetfillcolor{currentfill}%
\pgfsetlinewidth{0.000000pt}%
\definecolor{currentstroke}{rgb}{0.000000,0.000000,0.000000}%
\pgfsetstrokecolor{currentstroke}%
\pgfsetstrokeopacity{0.000000}%
\pgfsetdash{}{0pt}%
\pgfpathmoveto{\pgfqpoint{4.780642in}{1.792744in}}%
\pgfpathlineto{\pgfqpoint{4.789578in}{1.792744in}}%
\pgfpathlineto{\pgfqpoint{4.789578in}{1.775286in}}%
\pgfpathlineto{\pgfqpoint{4.780642in}{1.775286in}}%
\pgfpathlineto{\pgfqpoint{4.780642in}{1.792744in}}%
\pgfpathclose%
\pgfusepath{fill}%
\end{pgfscope}%
\begin{pgfscope}%
\pgfpathrectangle{\pgfqpoint{3.722897in}{0.857143in}}{\pgfqpoint{2.627103in}{1.813434in}}%
\pgfusepath{clip}%
\pgfsetbuttcap%
\pgfsetmiterjoin%
\definecolor{currentfill}{rgb}{0.133298,0.375282,0.379395}%
\pgfsetfillcolor{currentfill}%
\pgfsetlinewidth{0.000000pt}%
\definecolor{currentstroke}{rgb}{0.000000,0.000000,0.000000}%
\pgfsetstrokecolor{currentstroke}%
\pgfsetstrokeopacity{0.000000}%
\pgfsetdash{}{0pt}%
\pgfpathmoveto{\pgfqpoint{4.791812in}{1.792494in}}%
\pgfpathlineto{\pgfqpoint{4.800749in}{1.792494in}}%
\pgfpathlineto{\pgfqpoint{4.800749in}{1.783976in}}%
\pgfpathlineto{\pgfqpoint{4.791812in}{1.783976in}}%
\pgfpathlineto{\pgfqpoint{4.791812in}{1.792494in}}%
\pgfpathclose%
\pgfusepath{fill}%
\end{pgfscope}%
\begin{pgfscope}%
\pgfpathrectangle{\pgfqpoint{3.722897in}{0.857143in}}{\pgfqpoint{2.627103in}{1.813434in}}%
\pgfusepath{clip}%
\pgfsetbuttcap%
\pgfsetmiterjoin%
\definecolor{currentfill}{rgb}{0.133298,0.375282,0.379395}%
\pgfsetfillcolor{currentfill}%
\pgfsetlinewidth{0.000000pt}%
\definecolor{currentstroke}{rgb}{0.000000,0.000000,0.000000}%
\pgfsetstrokecolor{currentstroke}%
\pgfsetstrokeopacity{0.000000}%
\pgfsetdash{}{0pt}%
\pgfpathmoveto{\pgfqpoint{4.802983in}{1.791778in}}%
\pgfpathlineto{\pgfqpoint{4.811919in}{1.791778in}}%
\pgfpathlineto{\pgfqpoint{4.811919in}{1.783707in}}%
\pgfpathlineto{\pgfqpoint{4.802983in}{1.783707in}}%
\pgfpathlineto{\pgfqpoint{4.802983in}{1.791778in}}%
\pgfpathclose%
\pgfusepath{fill}%
\end{pgfscope}%
\begin{pgfscope}%
\pgfpathrectangle{\pgfqpoint{3.722897in}{0.857143in}}{\pgfqpoint{2.627103in}{1.813434in}}%
\pgfusepath{clip}%
\pgfsetbuttcap%
\pgfsetmiterjoin%
\definecolor{currentfill}{rgb}{0.133298,0.375282,0.379395}%
\pgfsetfillcolor{currentfill}%
\pgfsetlinewidth{0.000000pt}%
\definecolor{currentstroke}{rgb}{0.000000,0.000000,0.000000}%
\pgfsetstrokecolor{currentstroke}%
\pgfsetstrokeopacity{0.000000}%
\pgfsetdash{}{0pt}%
\pgfpathmoveto{\pgfqpoint{4.814153in}{1.789980in}}%
\pgfpathlineto{\pgfqpoint{4.823090in}{1.789980in}}%
\pgfpathlineto{\pgfqpoint{4.823090in}{1.776881in}}%
\pgfpathlineto{\pgfqpoint{4.814153in}{1.776881in}}%
\pgfpathlineto{\pgfqpoint{4.814153in}{1.789980in}}%
\pgfpathclose%
\pgfusepath{fill}%
\end{pgfscope}%
\begin{pgfscope}%
\pgfpathrectangle{\pgfqpoint{3.722897in}{0.857143in}}{\pgfqpoint{2.627103in}{1.813434in}}%
\pgfusepath{clip}%
\pgfsetbuttcap%
\pgfsetmiterjoin%
\definecolor{currentfill}{rgb}{0.133298,0.375282,0.379395}%
\pgfsetfillcolor{currentfill}%
\pgfsetlinewidth{0.000000pt}%
\definecolor{currentstroke}{rgb}{0.000000,0.000000,0.000000}%
\pgfsetstrokecolor{currentstroke}%
\pgfsetstrokeopacity{0.000000}%
\pgfsetdash{}{0pt}%
\pgfpathmoveto{\pgfqpoint{4.825324in}{1.788960in}}%
\pgfpathlineto{\pgfqpoint{4.834261in}{1.788960in}}%
\pgfpathlineto{\pgfqpoint{4.834261in}{1.773167in}}%
\pgfpathlineto{\pgfqpoint{4.825324in}{1.773167in}}%
\pgfpathlineto{\pgfqpoint{4.825324in}{1.788960in}}%
\pgfpathclose%
\pgfusepath{fill}%
\end{pgfscope}%
\begin{pgfscope}%
\pgfpathrectangle{\pgfqpoint{3.722897in}{0.857143in}}{\pgfqpoint{2.627103in}{1.813434in}}%
\pgfusepath{clip}%
\pgfsetbuttcap%
\pgfsetmiterjoin%
\definecolor{currentfill}{rgb}{0.133298,0.375282,0.379395}%
\pgfsetfillcolor{currentfill}%
\pgfsetlinewidth{0.000000pt}%
\definecolor{currentstroke}{rgb}{0.000000,0.000000,0.000000}%
\pgfsetstrokecolor{currentstroke}%
\pgfsetstrokeopacity{0.000000}%
\pgfsetdash{}{0pt}%
\pgfpathmoveto{\pgfqpoint{4.836495in}{1.813884in}}%
\pgfpathlineto{\pgfqpoint{4.845431in}{1.813884in}}%
\pgfpathlineto{\pgfqpoint{4.845431in}{1.818805in}}%
\pgfpathlineto{\pgfqpoint{4.836495in}{1.818805in}}%
\pgfpathlineto{\pgfqpoint{4.836495in}{1.813884in}}%
\pgfpathclose%
\pgfusepath{fill}%
\end{pgfscope}%
\begin{pgfscope}%
\pgfpathrectangle{\pgfqpoint{3.722897in}{0.857143in}}{\pgfqpoint{2.627103in}{1.813434in}}%
\pgfusepath{clip}%
\pgfsetbuttcap%
\pgfsetmiterjoin%
\definecolor{currentfill}{rgb}{0.133298,0.375282,0.379395}%
\pgfsetfillcolor{currentfill}%
\pgfsetlinewidth{0.000000pt}%
\definecolor{currentstroke}{rgb}{0.000000,0.000000,0.000000}%
\pgfsetstrokecolor{currentstroke}%
\pgfsetstrokeopacity{0.000000}%
\pgfsetdash{}{0pt}%
\pgfpathmoveto{\pgfqpoint{4.847665in}{1.813884in}}%
\pgfpathlineto{\pgfqpoint{4.856602in}{1.813884in}}%
\pgfpathlineto{\pgfqpoint{4.856602in}{1.815758in}}%
\pgfpathlineto{\pgfqpoint{4.847665in}{1.815758in}}%
\pgfpathlineto{\pgfqpoint{4.847665in}{1.813884in}}%
\pgfpathclose%
\pgfusepath{fill}%
\end{pgfscope}%
\begin{pgfscope}%
\pgfpathrectangle{\pgfqpoint{3.722897in}{0.857143in}}{\pgfqpoint{2.627103in}{1.813434in}}%
\pgfusepath{clip}%
\pgfsetbuttcap%
\pgfsetmiterjoin%
\definecolor{currentfill}{rgb}{0.133298,0.375282,0.379395}%
\pgfsetfillcolor{currentfill}%
\pgfsetlinewidth{0.000000pt}%
\definecolor{currentstroke}{rgb}{0.000000,0.000000,0.000000}%
\pgfsetstrokecolor{currentstroke}%
\pgfsetstrokeopacity{0.000000}%
\pgfsetdash{}{0pt}%
\pgfpathmoveto{\pgfqpoint{4.858836in}{1.813884in}}%
\pgfpathlineto{\pgfqpoint{4.867772in}{1.813884in}}%
\pgfpathlineto{\pgfqpoint{4.867772in}{1.819354in}}%
\pgfpathlineto{\pgfqpoint{4.858836in}{1.819354in}}%
\pgfpathlineto{\pgfqpoint{4.858836in}{1.813884in}}%
\pgfpathclose%
\pgfusepath{fill}%
\end{pgfscope}%
\begin{pgfscope}%
\pgfpathrectangle{\pgfqpoint{3.722897in}{0.857143in}}{\pgfqpoint{2.627103in}{1.813434in}}%
\pgfusepath{clip}%
\pgfsetbuttcap%
\pgfsetmiterjoin%
\definecolor{currentfill}{rgb}{0.133298,0.375282,0.379395}%
\pgfsetfillcolor{currentfill}%
\pgfsetlinewidth{0.000000pt}%
\definecolor{currentstroke}{rgb}{0.000000,0.000000,0.000000}%
\pgfsetstrokecolor{currentstroke}%
\pgfsetstrokeopacity{0.000000}%
\pgfsetdash{}{0pt}%
\pgfpathmoveto{\pgfqpoint{4.870006in}{1.813884in}}%
\pgfpathlineto{\pgfqpoint{4.878943in}{1.813884in}}%
\pgfpathlineto{\pgfqpoint{4.878943in}{1.830216in}}%
\pgfpathlineto{\pgfqpoint{4.870006in}{1.830216in}}%
\pgfpathlineto{\pgfqpoint{4.870006in}{1.813884in}}%
\pgfpathclose%
\pgfusepath{fill}%
\end{pgfscope}%
\begin{pgfscope}%
\pgfpathrectangle{\pgfqpoint{3.722897in}{0.857143in}}{\pgfqpoint{2.627103in}{1.813434in}}%
\pgfusepath{clip}%
\pgfsetbuttcap%
\pgfsetmiterjoin%
\definecolor{currentfill}{rgb}{0.133298,0.375282,0.379395}%
\pgfsetfillcolor{currentfill}%
\pgfsetlinewidth{0.000000pt}%
\definecolor{currentstroke}{rgb}{0.000000,0.000000,0.000000}%
\pgfsetstrokecolor{currentstroke}%
\pgfsetstrokeopacity{0.000000}%
\pgfsetdash{}{0pt}%
\pgfpathmoveto{\pgfqpoint{4.881177in}{1.813884in}}%
\pgfpathlineto{\pgfqpoint{4.890114in}{1.813884in}}%
\pgfpathlineto{\pgfqpoint{4.890114in}{1.832039in}}%
\pgfpathlineto{\pgfqpoint{4.881177in}{1.832039in}}%
\pgfpathlineto{\pgfqpoint{4.881177in}{1.813884in}}%
\pgfpathclose%
\pgfusepath{fill}%
\end{pgfscope}%
\begin{pgfscope}%
\pgfpathrectangle{\pgfqpoint{3.722897in}{0.857143in}}{\pgfqpoint{2.627103in}{1.813434in}}%
\pgfusepath{clip}%
\pgfsetbuttcap%
\pgfsetmiterjoin%
\definecolor{currentfill}{rgb}{0.133298,0.375282,0.379395}%
\pgfsetfillcolor{currentfill}%
\pgfsetlinewidth{0.000000pt}%
\definecolor{currentstroke}{rgb}{0.000000,0.000000,0.000000}%
\pgfsetstrokecolor{currentstroke}%
\pgfsetstrokeopacity{0.000000}%
\pgfsetdash{}{0pt}%
\pgfpathmoveto{\pgfqpoint{4.892348in}{1.813884in}}%
\pgfpathlineto{\pgfqpoint{4.901284in}{1.813884in}}%
\pgfpathlineto{\pgfqpoint{4.901284in}{1.819046in}}%
\pgfpathlineto{\pgfqpoint{4.892348in}{1.819046in}}%
\pgfpathlineto{\pgfqpoint{4.892348in}{1.813884in}}%
\pgfpathclose%
\pgfusepath{fill}%
\end{pgfscope}%
\begin{pgfscope}%
\pgfpathrectangle{\pgfqpoint{3.722897in}{0.857143in}}{\pgfqpoint{2.627103in}{1.813434in}}%
\pgfusepath{clip}%
\pgfsetbuttcap%
\pgfsetmiterjoin%
\definecolor{currentfill}{rgb}{0.133298,0.375282,0.379395}%
\pgfsetfillcolor{currentfill}%
\pgfsetlinewidth{0.000000pt}%
\definecolor{currentstroke}{rgb}{0.000000,0.000000,0.000000}%
\pgfsetstrokecolor{currentstroke}%
\pgfsetstrokeopacity{0.000000}%
\pgfsetdash{}{0pt}%
\pgfpathmoveto{\pgfqpoint{4.903518in}{1.813884in}}%
\pgfpathlineto{\pgfqpoint{4.912455in}{1.813884in}}%
\pgfpathlineto{\pgfqpoint{4.912455in}{1.817615in}}%
\pgfpathlineto{\pgfqpoint{4.903518in}{1.817615in}}%
\pgfpathlineto{\pgfqpoint{4.903518in}{1.813884in}}%
\pgfpathclose%
\pgfusepath{fill}%
\end{pgfscope}%
\begin{pgfscope}%
\pgfpathrectangle{\pgfqpoint{3.722897in}{0.857143in}}{\pgfqpoint{2.627103in}{1.813434in}}%
\pgfusepath{clip}%
\pgfsetbuttcap%
\pgfsetmiterjoin%
\definecolor{currentfill}{rgb}{0.133298,0.375282,0.379395}%
\pgfsetfillcolor{currentfill}%
\pgfsetlinewidth{0.000000pt}%
\definecolor{currentstroke}{rgb}{0.000000,0.000000,0.000000}%
\pgfsetstrokecolor{currentstroke}%
\pgfsetstrokeopacity{0.000000}%
\pgfsetdash{}{0pt}%
\pgfpathmoveto{\pgfqpoint{4.914689in}{1.813884in}}%
\pgfpathlineto{\pgfqpoint{4.923625in}{1.813884in}}%
\pgfpathlineto{\pgfqpoint{4.923625in}{1.829647in}}%
\pgfpathlineto{\pgfqpoint{4.914689in}{1.829647in}}%
\pgfpathlineto{\pgfqpoint{4.914689in}{1.813884in}}%
\pgfpathclose%
\pgfusepath{fill}%
\end{pgfscope}%
\begin{pgfscope}%
\pgfpathrectangle{\pgfqpoint{3.722897in}{0.857143in}}{\pgfqpoint{2.627103in}{1.813434in}}%
\pgfusepath{clip}%
\pgfsetbuttcap%
\pgfsetmiterjoin%
\definecolor{currentfill}{rgb}{0.133298,0.375282,0.379395}%
\pgfsetfillcolor{currentfill}%
\pgfsetlinewidth{0.000000pt}%
\definecolor{currentstroke}{rgb}{0.000000,0.000000,0.000000}%
\pgfsetstrokecolor{currentstroke}%
\pgfsetstrokeopacity{0.000000}%
\pgfsetdash{}{0pt}%
\pgfpathmoveto{\pgfqpoint{4.925860in}{1.813884in}}%
\pgfpathlineto{\pgfqpoint{4.934796in}{1.813884in}}%
\pgfpathlineto{\pgfqpoint{4.934796in}{1.834348in}}%
\pgfpathlineto{\pgfqpoint{4.925860in}{1.834348in}}%
\pgfpathlineto{\pgfqpoint{4.925860in}{1.813884in}}%
\pgfpathclose%
\pgfusepath{fill}%
\end{pgfscope}%
\begin{pgfscope}%
\pgfpathrectangle{\pgfqpoint{3.722897in}{0.857143in}}{\pgfqpoint{2.627103in}{1.813434in}}%
\pgfusepath{clip}%
\pgfsetbuttcap%
\pgfsetmiterjoin%
\definecolor{currentfill}{rgb}{0.133298,0.375282,0.379395}%
\pgfsetfillcolor{currentfill}%
\pgfsetlinewidth{0.000000pt}%
\definecolor{currentstroke}{rgb}{0.000000,0.000000,0.000000}%
\pgfsetstrokecolor{currentstroke}%
\pgfsetstrokeopacity{0.000000}%
\pgfsetdash{}{0pt}%
\pgfpathmoveto{\pgfqpoint{4.937030in}{1.813884in}}%
\pgfpathlineto{\pgfqpoint{4.945967in}{1.813884in}}%
\pgfpathlineto{\pgfqpoint{4.945967in}{1.821808in}}%
\pgfpathlineto{\pgfqpoint{4.937030in}{1.821808in}}%
\pgfpathlineto{\pgfqpoint{4.937030in}{1.813884in}}%
\pgfpathclose%
\pgfusepath{fill}%
\end{pgfscope}%
\begin{pgfscope}%
\pgfpathrectangle{\pgfqpoint{3.722897in}{0.857143in}}{\pgfqpoint{2.627103in}{1.813434in}}%
\pgfusepath{clip}%
\pgfsetbuttcap%
\pgfsetmiterjoin%
\definecolor{currentfill}{rgb}{0.133298,0.375282,0.379395}%
\pgfsetfillcolor{currentfill}%
\pgfsetlinewidth{0.000000pt}%
\definecolor{currentstroke}{rgb}{0.000000,0.000000,0.000000}%
\pgfsetstrokecolor{currentstroke}%
\pgfsetstrokeopacity{0.000000}%
\pgfsetdash{}{0pt}%
\pgfpathmoveto{\pgfqpoint{4.948201in}{1.813884in}}%
\pgfpathlineto{\pgfqpoint{4.957137in}{1.813884in}}%
\pgfpathlineto{\pgfqpoint{4.957137in}{1.831061in}}%
\pgfpathlineto{\pgfqpoint{4.948201in}{1.831061in}}%
\pgfpathlineto{\pgfqpoint{4.948201in}{1.813884in}}%
\pgfpathclose%
\pgfusepath{fill}%
\end{pgfscope}%
\begin{pgfscope}%
\pgfpathrectangle{\pgfqpoint{3.722897in}{0.857143in}}{\pgfqpoint{2.627103in}{1.813434in}}%
\pgfusepath{clip}%
\pgfsetbuttcap%
\pgfsetmiterjoin%
\definecolor{currentfill}{rgb}{0.133298,0.375282,0.379395}%
\pgfsetfillcolor{currentfill}%
\pgfsetlinewidth{0.000000pt}%
\definecolor{currentstroke}{rgb}{0.000000,0.000000,0.000000}%
\pgfsetstrokecolor{currentstroke}%
\pgfsetstrokeopacity{0.000000}%
\pgfsetdash{}{0pt}%
\pgfpathmoveto{\pgfqpoint{4.959371in}{1.813884in}}%
\pgfpathlineto{\pgfqpoint{4.968308in}{1.813884in}}%
\pgfpathlineto{\pgfqpoint{4.968308in}{1.828134in}}%
\pgfpathlineto{\pgfqpoint{4.959371in}{1.828134in}}%
\pgfpathlineto{\pgfqpoint{4.959371in}{1.813884in}}%
\pgfpathclose%
\pgfusepath{fill}%
\end{pgfscope}%
\begin{pgfscope}%
\pgfpathrectangle{\pgfqpoint{3.722897in}{0.857143in}}{\pgfqpoint{2.627103in}{1.813434in}}%
\pgfusepath{clip}%
\pgfsetbuttcap%
\pgfsetmiterjoin%
\definecolor{currentfill}{rgb}{0.133298,0.375282,0.379395}%
\pgfsetfillcolor{currentfill}%
\pgfsetlinewidth{0.000000pt}%
\definecolor{currentstroke}{rgb}{0.000000,0.000000,0.000000}%
\pgfsetstrokecolor{currentstroke}%
\pgfsetstrokeopacity{0.000000}%
\pgfsetdash{}{0pt}%
\pgfpathmoveto{\pgfqpoint{4.970542in}{1.813884in}}%
\pgfpathlineto{\pgfqpoint{4.979478in}{1.813884in}}%
\pgfpathlineto{\pgfqpoint{4.979478in}{1.832080in}}%
\pgfpathlineto{\pgfqpoint{4.970542in}{1.832080in}}%
\pgfpathlineto{\pgfqpoint{4.970542in}{1.813884in}}%
\pgfpathclose%
\pgfusepath{fill}%
\end{pgfscope}%
\begin{pgfscope}%
\pgfpathrectangle{\pgfqpoint{3.722897in}{0.857143in}}{\pgfqpoint{2.627103in}{1.813434in}}%
\pgfusepath{clip}%
\pgfsetbuttcap%
\pgfsetmiterjoin%
\definecolor{currentfill}{rgb}{0.133298,0.375282,0.379395}%
\pgfsetfillcolor{currentfill}%
\pgfsetlinewidth{0.000000pt}%
\definecolor{currentstroke}{rgb}{0.000000,0.000000,0.000000}%
\pgfsetstrokecolor{currentstroke}%
\pgfsetstrokeopacity{0.000000}%
\pgfsetdash{}{0pt}%
\pgfpathmoveto{\pgfqpoint{4.981713in}{1.813884in}}%
\pgfpathlineto{\pgfqpoint{4.990649in}{1.813884in}}%
\pgfpathlineto{\pgfqpoint{4.990649in}{1.840360in}}%
\pgfpathlineto{\pgfqpoint{4.981713in}{1.840360in}}%
\pgfpathlineto{\pgfqpoint{4.981713in}{1.813884in}}%
\pgfpathclose%
\pgfusepath{fill}%
\end{pgfscope}%
\begin{pgfscope}%
\pgfpathrectangle{\pgfqpoint{3.722897in}{0.857143in}}{\pgfqpoint{2.627103in}{1.813434in}}%
\pgfusepath{clip}%
\pgfsetbuttcap%
\pgfsetmiterjoin%
\definecolor{currentfill}{rgb}{0.133298,0.375282,0.379395}%
\pgfsetfillcolor{currentfill}%
\pgfsetlinewidth{0.000000pt}%
\definecolor{currentstroke}{rgb}{0.000000,0.000000,0.000000}%
\pgfsetstrokecolor{currentstroke}%
\pgfsetstrokeopacity{0.000000}%
\pgfsetdash{}{0pt}%
\pgfpathmoveto{\pgfqpoint{4.992883in}{1.813884in}}%
\pgfpathlineto{\pgfqpoint{5.001820in}{1.813884in}}%
\pgfpathlineto{\pgfqpoint{5.001820in}{1.834515in}}%
\pgfpathlineto{\pgfqpoint{4.992883in}{1.834515in}}%
\pgfpathlineto{\pgfqpoint{4.992883in}{1.813884in}}%
\pgfpathclose%
\pgfusepath{fill}%
\end{pgfscope}%
\begin{pgfscope}%
\pgfpathrectangle{\pgfqpoint{3.722897in}{0.857143in}}{\pgfqpoint{2.627103in}{1.813434in}}%
\pgfusepath{clip}%
\pgfsetbuttcap%
\pgfsetmiterjoin%
\definecolor{currentfill}{rgb}{0.133298,0.375282,0.379395}%
\pgfsetfillcolor{currentfill}%
\pgfsetlinewidth{0.000000pt}%
\definecolor{currentstroke}{rgb}{0.000000,0.000000,0.000000}%
\pgfsetstrokecolor{currentstroke}%
\pgfsetstrokeopacity{0.000000}%
\pgfsetdash{}{0pt}%
\pgfpathmoveto{\pgfqpoint{5.004054in}{1.813884in}}%
\pgfpathlineto{\pgfqpoint{5.012990in}{1.813884in}}%
\pgfpathlineto{\pgfqpoint{5.012990in}{1.840937in}}%
\pgfpathlineto{\pgfqpoint{5.004054in}{1.840937in}}%
\pgfpathlineto{\pgfqpoint{5.004054in}{1.813884in}}%
\pgfpathclose%
\pgfusepath{fill}%
\end{pgfscope}%
\begin{pgfscope}%
\pgfpathrectangle{\pgfqpoint{3.722897in}{0.857143in}}{\pgfqpoint{2.627103in}{1.813434in}}%
\pgfusepath{clip}%
\pgfsetbuttcap%
\pgfsetmiterjoin%
\definecolor{currentfill}{rgb}{0.133298,0.375282,0.379395}%
\pgfsetfillcolor{currentfill}%
\pgfsetlinewidth{0.000000pt}%
\definecolor{currentstroke}{rgb}{0.000000,0.000000,0.000000}%
\pgfsetstrokecolor{currentstroke}%
\pgfsetstrokeopacity{0.000000}%
\pgfsetdash{}{0pt}%
\pgfpathmoveto{\pgfqpoint{5.015224in}{1.813884in}}%
\pgfpathlineto{\pgfqpoint{5.024161in}{1.813884in}}%
\pgfpathlineto{\pgfqpoint{5.024161in}{1.837694in}}%
\pgfpathlineto{\pgfqpoint{5.015224in}{1.837694in}}%
\pgfpathlineto{\pgfqpoint{5.015224in}{1.813884in}}%
\pgfpathclose%
\pgfusepath{fill}%
\end{pgfscope}%
\begin{pgfscope}%
\pgfpathrectangle{\pgfqpoint{3.722897in}{0.857143in}}{\pgfqpoint{2.627103in}{1.813434in}}%
\pgfusepath{clip}%
\pgfsetbuttcap%
\pgfsetmiterjoin%
\definecolor{currentfill}{rgb}{0.133298,0.375282,0.379395}%
\pgfsetfillcolor{currentfill}%
\pgfsetlinewidth{0.000000pt}%
\definecolor{currentstroke}{rgb}{0.000000,0.000000,0.000000}%
\pgfsetstrokecolor{currentstroke}%
\pgfsetstrokeopacity{0.000000}%
\pgfsetdash{}{0pt}%
\pgfpathmoveto{\pgfqpoint{5.026395in}{1.813884in}}%
\pgfpathlineto{\pgfqpoint{5.035331in}{1.813884in}}%
\pgfpathlineto{\pgfqpoint{5.035331in}{1.853592in}}%
\pgfpathlineto{\pgfqpoint{5.026395in}{1.853592in}}%
\pgfpathlineto{\pgfqpoint{5.026395in}{1.813884in}}%
\pgfpathclose%
\pgfusepath{fill}%
\end{pgfscope}%
\begin{pgfscope}%
\pgfpathrectangle{\pgfqpoint{3.722897in}{0.857143in}}{\pgfqpoint{2.627103in}{1.813434in}}%
\pgfusepath{clip}%
\pgfsetbuttcap%
\pgfsetmiterjoin%
\definecolor{currentfill}{rgb}{0.133298,0.375282,0.379395}%
\pgfsetfillcolor{currentfill}%
\pgfsetlinewidth{0.000000pt}%
\definecolor{currentstroke}{rgb}{0.000000,0.000000,0.000000}%
\pgfsetstrokecolor{currentstroke}%
\pgfsetstrokeopacity{0.000000}%
\pgfsetdash{}{0pt}%
\pgfpathmoveto{\pgfqpoint{5.037566in}{1.813884in}}%
\pgfpathlineto{\pgfqpoint{5.046502in}{1.813884in}}%
\pgfpathlineto{\pgfqpoint{5.046502in}{1.854554in}}%
\pgfpathlineto{\pgfqpoint{5.037566in}{1.854554in}}%
\pgfpathlineto{\pgfqpoint{5.037566in}{1.813884in}}%
\pgfpathclose%
\pgfusepath{fill}%
\end{pgfscope}%
\begin{pgfscope}%
\pgfpathrectangle{\pgfqpoint{3.722897in}{0.857143in}}{\pgfqpoint{2.627103in}{1.813434in}}%
\pgfusepath{clip}%
\pgfsetbuttcap%
\pgfsetmiterjoin%
\definecolor{currentfill}{rgb}{0.133298,0.375282,0.379395}%
\pgfsetfillcolor{currentfill}%
\pgfsetlinewidth{0.000000pt}%
\definecolor{currentstroke}{rgb}{0.000000,0.000000,0.000000}%
\pgfsetstrokecolor{currentstroke}%
\pgfsetstrokeopacity{0.000000}%
\pgfsetdash{}{0pt}%
\pgfpathmoveto{\pgfqpoint{5.048736in}{1.813884in}}%
\pgfpathlineto{\pgfqpoint{5.057673in}{1.813884in}}%
\pgfpathlineto{\pgfqpoint{5.057673in}{1.860403in}}%
\pgfpathlineto{\pgfqpoint{5.048736in}{1.860403in}}%
\pgfpathlineto{\pgfqpoint{5.048736in}{1.813884in}}%
\pgfpathclose%
\pgfusepath{fill}%
\end{pgfscope}%
\begin{pgfscope}%
\pgfpathrectangle{\pgfqpoint{3.722897in}{0.857143in}}{\pgfqpoint{2.627103in}{1.813434in}}%
\pgfusepath{clip}%
\pgfsetbuttcap%
\pgfsetmiterjoin%
\definecolor{currentfill}{rgb}{0.133298,0.375282,0.379395}%
\pgfsetfillcolor{currentfill}%
\pgfsetlinewidth{0.000000pt}%
\definecolor{currentstroke}{rgb}{0.000000,0.000000,0.000000}%
\pgfsetstrokecolor{currentstroke}%
\pgfsetstrokeopacity{0.000000}%
\pgfsetdash{}{0pt}%
\pgfpathmoveto{\pgfqpoint{5.059907in}{1.813884in}}%
\pgfpathlineto{\pgfqpoint{5.068843in}{1.813884in}}%
\pgfpathlineto{\pgfqpoint{5.068843in}{1.866874in}}%
\pgfpathlineto{\pgfqpoint{5.059907in}{1.866874in}}%
\pgfpathlineto{\pgfqpoint{5.059907in}{1.813884in}}%
\pgfpathclose%
\pgfusepath{fill}%
\end{pgfscope}%
\begin{pgfscope}%
\pgfpathrectangle{\pgfqpoint{3.722897in}{0.857143in}}{\pgfqpoint{2.627103in}{1.813434in}}%
\pgfusepath{clip}%
\pgfsetbuttcap%
\pgfsetmiterjoin%
\definecolor{currentfill}{rgb}{0.133298,0.375282,0.379395}%
\pgfsetfillcolor{currentfill}%
\pgfsetlinewidth{0.000000pt}%
\definecolor{currentstroke}{rgb}{0.000000,0.000000,0.000000}%
\pgfsetstrokecolor{currentstroke}%
\pgfsetstrokeopacity{0.000000}%
\pgfsetdash{}{0pt}%
\pgfpathmoveto{\pgfqpoint{5.071077in}{1.813884in}}%
\pgfpathlineto{\pgfqpoint{5.080014in}{1.813884in}}%
\pgfpathlineto{\pgfqpoint{5.080014in}{1.876139in}}%
\pgfpathlineto{\pgfqpoint{5.071077in}{1.876139in}}%
\pgfpathlineto{\pgfqpoint{5.071077in}{1.813884in}}%
\pgfpathclose%
\pgfusepath{fill}%
\end{pgfscope}%
\begin{pgfscope}%
\pgfpathrectangle{\pgfqpoint{3.722897in}{0.857143in}}{\pgfqpoint{2.627103in}{1.813434in}}%
\pgfusepath{clip}%
\pgfsetbuttcap%
\pgfsetmiterjoin%
\definecolor{currentfill}{rgb}{0.133298,0.375282,0.379395}%
\pgfsetfillcolor{currentfill}%
\pgfsetlinewidth{0.000000pt}%
\definecolor{currentstroke}{rgb}{0.000000,0.000000,0.000000}%
\pgfsetstrokecolor{currentstroke}%
\pgfsetstrokeopacity{0.000000}%
\pgfsetdash{}{0pt}%
\pgfpathmoveto{\pgfqpoint{5.082248in}{1.813884in}}%
\pgfpathlineto{\pgfqpoint{5.091184in}{1.813884in}}%
\pgfpathlineto{\pgfqpoint{5.091184in}{1.879552in}}%
\pgfpathlineto{\pgfqpoint{5.082248in}{1.879552in}}%
\pgfpathlineto{\pgfqpoint{5.082248in}{1.813884in}}%
\pgfpathclose%
\pgfusepath{fill}%
\end{pgfscope}%
\begin{pgfscope}%
\pgfpathrectangle{\pgfqpoint{3.722897in}{0.857143in}}{\pgfqpoint{2.627103in}{1.813434in}}%
\pgfusepath{clip}%
\pgfsetbuttcap%
\pgfsetmiterjoin%
\definecolor{currentfill}{rgb}{0.133298,0.375282,0.379395}%
\pgfsetfillcolor{currentfill}%
\pgfsetlinewidth{0.000000pt}%
\definecolor{currentstroke}{rgb}{0.000000,0.000000,0.000000}%
\pgfsetstrokecolor{currentstroke}%
\pgfsetstrokeopacity{0.000000}%
\pgfsetdash{}{0pt}%
\pgfpathmoveto{\pgfqpoint{5.093419in}{1.813884in}}%
\pgfpathlineto{\pgfqpoint{5.102355in}{1.813884in}}%
\pgfpathlineto{\pgfqpoint{5.102355in}{1.884427in}}%
\pgfpathlineto{\pgfqpoint{5.093419in}{1.884427in}}%
\pgfpathlineto{\pgfqpoint{5.093419in}{1.813884in}}%
\pgfpathclose%
\pgfusepath{fill}%
\end{pgfscope}%
\begin{pgfscope}%
\pgfpathrectangle{\pgfqpoint{3.722897in}{0.857143in}}{\pgfqpoint{2.627103in}{1.813434in}}%
\pgfusepath{clip}%
\pgfsetbuttcap%
\pgfsetmiterjoin%
\definecolor{currentfill}{rgb}{0.133298,0.375282,0.379395}%
\pgfsetfillcolor{currentfill}%
\pgfsetlinewidth{0.000000pt}%
\definecolor{currentstroke}{rgb}{0.000000,0.000000,0.000000}%
\pgfsetstrokecolor{currentstroke}%
\pgfsetstrokeopacity{0.000000}%
\pgfsetdash{}{0pt}%
\pgfpathmoveto{\pgfqpoint{5.104589in}{1.813884in}}%
\pgfpathlineto{\pgfqpoint{5.113526in}{1.813884in}}%
\pgfpathlineto{\pgfqpoint{5.113526in}{1.888318in}}%
\pgfpathlineto{\pgfqpoint{5.104589in}{1.888318in}}%
\pgfpathlineto{\pgfqpoint{5.104589in}{1.813884in}}%
\pgfpathclose%
\pgfusepath{fill}%
\end{pgfscope}%
\begin{pgfscope}%
\pgfpathrectangle{\pgfqpoint{3.722897in}{0.857143in}}{\pgfqpoint{2.627103in}{1.813434in}}%
\pgfusepath{clip}%
\pgfsetbuttcap%
\pgfsetmiterjoin%
\definecolor{currentfill}{rgb}{0.133298,0.375282,0.379395}%
\pgfsetfillcolor{currentfill}%
\pgfsetlinewidth{0.000000pt}%
\definecolor{currentstroke}{rgb}{0.000000,0.000000,0.000000}%
\pgfsetstrokecolor{currentstroke}%
\pgfsetstrokeopacity{0.000000}%
\pgfsetdash{}{0pt}%
\pgfpathmoveto{\pgfqpoint{5.115760in}{1.813884in}}%
\pgfpathlineto{\pgfqpoint{5.124696in}{1.813884in}}%
\pgfpathlineto{\pgfqpoint{5.124696in}{1.891406in}}%
\pgfpathlineto{\pgfqpoint{5.115760in}{1.891406in}}%
\pgfpathlineto{\pgfqpoint{5.115760in}{1.813884in}}%
\pgfpathclose%
\pgfusepath{fill}%
\end{pgfscope}%
\begin{pgfscope}%
\pgfpathrectangle{\pgfqpoint{3.722897in}{0.857143in}}{\pgfqpoint{2.627103in}{1.813434in}}%
\pgfusepath{clip}%
\pgfsetbuttcap%
\pgfsetmiterjoin%
\definecolor{currentfill}{rgb}{0.133298,0.375282,0.379395}%
\pgfsetfillcolor{currentfill}%
\pgfsetlinewidth{0.000000pt}%
\definecolor{currentstroke}{rgb}{0.000000,0.000000,0.000000}%
\pgfsetstrokecolor{currentstroke}%
\pgfsetstrokeopacity{0.000000}%
\pgfsetdash{}{0pt}%
\pgfpathmoveto{\pgfqpoint{5.126930in}{1.813884in}}%
\pgfpathlineto{\pgfqpoint{5.135867in}{1.813884in}}%
\pgfpathlineto{\pgfqpoint{5.135867in}{1.898182in}}%
\pgfpathlineto{\pgfqpoint{5.126930in}{1.898182in}}%
\pgfpathlineto{\pgfqpoint{5.126930in}{1.813884in}}%
\pgfpathclose%
\pgfusepath{fill}%
\end{pgfscope}%
\begin{pgfscope}%
\pgfpathrectangle{\pgfqpoint{3.722897in}{0.857143in}}{\pgfqpoint{2.627103in}{1.813434in}}%
\pgfusepath{clip}%
\pgfsetbuttcap%
\pgfsetmiterjoin%
\definecolor{currentfill}{rgb}{0.133298,0.375282,0.379395}%
\pgfsetfillcolor{currentfill}%
\pgfsetlinewidth{0.000000pt}%
\definecolor{currentstroke}{rgb}{0.000000,0.000000,0.000000}%
\pgfsetstrokecolor{currentstroke}%
\pgfsetstrokeopacity{0.000000}%
\pgfsetdash{}{0pt}%
\pgfpathmoveto{\pgfqpoint{5.138101in}{1.813884in}}%
\pgfpathlineto{\pgfqpoint{5.147038in}{1.813884in}}%
\pgfpathlineto{\pgfqpoint{5.147038in}{1.908644in}}%
\pgfpathlineto{\pgfqpoint{5.138101in}{1.908644in}}%
\pgfpathlineto{\pgfqpoint{5.138101in}{1.813884in}}%
\pgfpathclose%
\pgfusepath{fill}%
\end{pgfscope}%
\begin{pgfscope}%
\pgfpathrectangle{\pgfqpoint{3.722897in}{0.857143in}}{\pgfqpoint{2.627103in}{1.813434in}}%
\pgfusepath{clip}%
\pgfsetbuttcap%
\pgfsetmiterjoin%
\definecolor{currentfill}{rgb}{0.133298,0.375282,0.379395}%
\pgfsetfillcolor{currentfill}%
\pgfsetlinewidth{0.000000pt}%
\definecolor{currentstroke}{rgb}{0.000000,0.000000,0.000000}%
\pgfsetstrokecolor{currentstroke}%
\pgfsetstrokeopacity{0.000000}%
\pgfsetdash{}{0pt}%
\pgfpathmoveto{\pgfqpoint{5.149272in}{1.813884in}}%
\pgfpathlineto{\pgfqpoint{5.158208in}{1.813884in}}%
\pgfpathlineto{\pgfqpoint{5.158208in}{1.907468in}}%
\pgfpathlineto{\pgfqpoint{5.149272in}{1.907468in}}%
\pgfpathlineto{\pgfqpoint{5.149272in}{1.813884in}}%
\pgfpathclose%
\pgfusepath{fill}%
\end{pgfscope}%
\begin{pgfscope}%
\pgfpathrectangle{\pgfqpoint{3.722897in}{0.857143in}}{\pgfqpoint{2.627103in}{1.813434in}}%
\pgfusepath{clip}%
\pgfsetbuttcap%
\pgfsetmiterjoin%
\definecolor{currentfill}{rgb}{0.133298,0.375282,0.379395}%
\pgfsetfillcolor{currentfill}%
\pgfsetlinewidth{0.000000pt}%
\definecolor{currentstroke}{rgb}{0.000000,0.000000,0.000000}%
\pgfsetstrokecolor{currentstroke}%
\pgfsetstrokeopacity{0.000000}%
\pgfsetdash{}{0pt}%
\pgfpathmoveto{\pgfqpoint{5.160442in}{1.813884in}}%
\pgfpathlineto{\pgfqpoint{5.169379in}{1.813884in}}%
\pgfpathlineto{\pgfqpoint{5.169379in}{1.910270in}}%
\pgfpathlineto{\pgfqpoint{5.160442in}{1.910270in}}%
\pgfpathlineto{\pgfqpoint{5.160442in}{1.813884in}}%
\pgfpathclose%
\pgfusepath{fill}%
\end{pgfscope}%
\begin{pgfscope}%
\pgfpathrectangle{\pgfqpoint{3.722897in}{0.857143in}}{\pgfqpoint{2.627103in}{1.813434in}}%
\pgfusepath{clip}%
\pgfsetbuttcap%
\pgfsetmiterjoin%
\definecolor{currentfill}{rgb}{0.133298,0.375282,0.379395}%
\pgfsetfillcolor{currentfill}%
\pgfsetlinewidth{0.000000pt}%
\definecolor{currentstroke}{rgb}{0.000000,0.000000,0.000000}%
\pgfsetstrokecolor{currentstroke}%
\pgfsetstrokeopacity{0.000000}%
\pgfsetdash{}{0pt}%
\pgfpathmoveto{\pgfqpoint{5.171613in}{1.813884in}}%
\pgfpathlineto{\pgfqpoint{5.180549in}{1.813884in}}%
\pgfpathlineto{\pgfqpoint{5.180549in}{1.914225in}}%
\pgfpathlineto{\pgfqpoint{5.171613in}{1.914225in}}%
\pgfpathlineto{\pgfqpoint{5.171613in}{1.813884in}}%
\pgfpathclose%
\pgfusepath{fill}%
\end{pgfscope}%
\begin{pgfscope}%
\pgfpathrectangle{\pgfqpoint{3.722897in}{0.857143in}}{\pgfqpoint{2.627103in}{1.813434in}}%
\pgfusepath{clip}%
\pgfsetbuttcap%
\pgfsetmiterjoin%
\definecolor{currentfill}{rgb}{0.133298,0.375282,0.379395}%
\pgfsetfillcolor{currentfill}%
\pgfsetlinewidth{0.000000pt}%
\definecolor{currentstroke}{rgb}{0.000000,0.000000,0.000000}%
\pgfsetstrokecolor{currentstroke}%
\pgfsetstrokeopacity{0.000000}%
\pgfsetdash{}{0pt}%
\pgfpathmoveto{\pgfqpoint{5.182783in}{1.813884in}}%
\pgfpathlineto{\pgfqpoint{5.191720in}{1.813884in}}%
\pgfpathlineto{\pgfqpoint{5.191720in}{1.922052in}}%
\pgfpathlineto{\pgfqpoint{5.182783in}{1.922052in}}%
\pgfpathlineto{\pgfqpoint{5.182783in}{1.813884in}}%
\pgfpathclose%
\pgfusepath{fill}%
\end{pgfscope}%
\begin{pgfscope}%
\pgfpathrectangle{\pgfqpoint{3.722897in}{0.857143in}}{\pgfqpoint{2.627103in}{1.813434in}}%
\pgfusepath{clip}%
\pgfsetbuttcap%
\pgfsetmiterjoin%
\definecolor{currentfill}{rgb}{0.133298,0.375282,0.379395}%
\pgfsetfillcolor{currentfill}%
\pgfsetlinewidth{0.000000pt}%
\definecolor{currentstroke}{rgb}{0.000000,0.000000,0.000000}%
\pgfsetstrokecolor{currentstroke}%
\pgfsetstrokeopacity{0.000000}%
\pgfsetdash{}{0pt}%
\pgfpathmoveto{\pgfqpoint{5.193954in}{1.813884in}}%
\pgfpathlineto{\pgfqpoint{5.202891in}{1.813884in}}%
\pgfpathlineto{\pgfqpoint{5.202891in}{1.917000in}}%
\pgfpathlineto{\pgfqpoint{5.193954in}{1.917000in}}%
\pgfpathlineto{\pgfqpoint{5.193954in}{1.813884in}}%
\pgfpathclose%
\pgfusepath{fill}%
\end{pgfscope}%
\begin{pgfscope}%
\pgfpathrectangle{\pgfqpoint{3.722897in}{0.857143in}}{\pgfqpoint{2.627103in}{1.813434in}}%
\pgfusepath{clip}%
\pgfsetbuttcap%
\pgfsetmiterjoin%
\definecolor{currentfill}{rgb}{0.133298,0.375282,0.379395}%
\pgfsetfillcolor{currentfill}%
\pgfsetlinewidth{0.000000pt}%
\definecolor{currentstroke}{rgb}{0.000000,0.000000,0.000000}%
\pgfsetstrokecolor{currentstroke}%
\pgfsetstrokeopacity{0.000000}%
\pgfsetdash{}{0pt}%
\pgfpathmoveto{\pgfqpoint{5.205125in}{1.813884in}}%
\pgfpathlineto{\pgfqpoint{5.214061in}{1.813884in}}%
\pgfpathlineto{\pgfqpoint{5.214061in}{1.924029in}}%
\pgfpathlineto{\pgfqpoint{5.205125in}{1.924029in}}%
\pgfpathlineto{\pgfqpoint{5.205125in}{1.813884in}}%
\pgfpathclose%
\pgfusepath{fill}%
\end{pgfscope}%
\begin{pgfscope}%
\pgfpathrectangle{\pgfqpoint{3.722897in}{0.857143in}}{\pgfqpoint{2.627103in}{1.813434in}}%
\pgfusepath{clip}%
\pgfsetbuttcap%
\pgfsetmiterjoin%
\definecolor{currentfill}{rgb}{0.133298,0.375282,0.379395}%
\pgfsetfillcolor{currentfill}%
\pgfsetlinewidth{0.000000pt}%
\definecolor{currentstroke}{rgb}{0.000000,0.000000,0.000000}%
\pgfsetstrokecolor{currentstroke}%
\pgfsetstrokeopacity{0.000000}%
\pgfsetdash{}{0pt}%
\pgfpathmoveto{\pgfqpoint{5.216295in}{1.813884in}}%
\pgfpathlineto{\pgfqpoint{5.225232in}{1.813884in}}%
\pgfpathlineto{\pgfqpoint{5.225232in}{1.935304in}}%
\pgfpathlineto{\pgfqpoint{5.216295in}{1.935304in}}%
\pgfpathlineto{\pgfqpoint{5.216295in}{1.813884in}}%
\pgfpathclose%
\pgfusepath{fill}%
\end{pgfscope}%
\begin{pgfscope}%
\pgfpathrectangle{\pgfqpoint{3.722897in}{0.857143in}}{\pgfqpoint{2.627103in}{1.813434in}}%
\pgfusepath{clip}%
\pgfsetbuttcap%
\pgfsetmiterjoin%
\definecolor{currentfill}{rgb}{0.133298,0.375282,0.379395}%
\pgfsetfillcolor{currentfill}%
\pgfsetlinewidth{0.000000pt}%
\definecolor{currentstroke}{rgb}{0.000000,0.000000,0.000000}%
\pgfsetstrokecolor{currentstroke}%
\pgfsetstrokeopacity{0.000000}%
\pgfsetdash{}{0pt}%
\pgfpathmoveto{\pgfqpoint{5.227466in}{1.813884in}}%
\pgfpathlineto{\pgfqpoint{5.236402in}{1.813884in}}%
\pgfpathlineto{\pgfqpoint{5.236402in}{1.929479in}}%
\pgfpathlineto{\pgfqpoint{5.227466in}{1.929479in}}%
\pgfpathlineto{\pgfqpoint{5.227466in}{1.813884in}}%
\pgfpathclose%
\pgfusepath{fill}%
\end{pgfscope}%
\begin{pgfscope}%
\pgfpathrectangle{\pgfqpoint{3.722897in}{0.857143in}}{\pgfqpoint{2.627103in}{1.813434in}}%
\pgfusepath{clip}%
\pgfsetbuttcap%
\pgfsetmiterjoin%
\definecolor{currentfill}{rgb}{0.133298,0.375282,0.379395}%
\pgfsetfillcolor{currentfill}%
\pgfsetlinewidth{0.000000pt}%
\definecolor{currentstroke}{rgb}{0.000000,0.000000,0.000000}%
\pgfsetstrokecolor{currentstroke}%
\pgfsetstrokeopacity{0.000000}%
\pgfsetdash{}{0pt}%
\pgfpathmoveto{\pgfqpoint{5.238636in}{1.813884in}}%
\pgfpathlineto{\pgfqpoint{5.247573in}{1.813884in}}%
\pgfpathlineto{\pgfqpoint{5.247573in}{1.940663in}}%
\pgfpathlineto{\pgfqpoint{5.238636in}{1.940663in}}%
\pgfpathlineto{\pgfqpoint{5.238636in}{1.813884in}}%
\pgfpathclose%
\pgfusepath{fill}%
\end{pgfscope}%
\begin{pgfscope}%
\pgfpathrectangle{\pgfqpoint{3.722897in}{0.857143in}}{\pgfqpoint{2.627103in}{1.813434in}}%
\pgfusepath{clip}%
\pgfsetbuttcap%
\pgfsetmiterjoin%
\definecolor{currentfill}{rgb}{0.133298,0.375282,0.379395}%
\pgfsetfillcolor{currentfill}%
\pgfsetlinewidth{0.000000pt}%
\definecolor{currentstroke}{rgb}{0.000000,0.000000,0.000000}%
\pgfsetstrokecolor{currentstroke}%
\pgfsetstrokeopacity{0.000000}%
\pgfsetdash{}{0pt}%
\pgfpathmoveto{\pgfqpoint{5.249807in}{1.813884in}}%
\pgfpathlineto{\pgfqpoint{5.258744in}{1.813884in}}%
\pgfpathlineto{\pgfqpoint{5.258744in}{1.949479in}}%
\pgfpathlineto{\pgfqpoint{5.249807in}{1.949479in}}%
\pgfpathlineto{\pgfqpoint{5.249807in}{1.813884in}}%
\pgfpathclose%
\pgfusepath{fill}%
\end{pgfscope}%
\begin{pgfscope}%
\pgfpathrectangle{\pgfqpoint{3.722897in}{0.857143in}}{\pgfqpoint{2.627103in}{1.813434in}}%
\pgfusepath{clip}%
\pgfsetbuttcap%
\pgfsetmiterjoin%
\definecolor{currentfill}{rgb}{0.133298,0.375282,0.379395}%
\pgfsetfillcolor{currentfill}%
\pgfsetlinewidth{0.000000pt}%
\definecolor{currentstroke}{rgb}{0.000000,0.000000,0.000000}%
\pgfsetstrokecolor{currentstroke}%
\pgfsetstrokeopacity{0.000000}%
\pgfsetdash{}{0pt}%
\pgfpathmoveto{\pgfqpoint{5.260978in}{1.813884in}}%
\pgfpathlineto{\pgfqpoint{5.269914in}{1.813884in}}%
\pgfpathlineto{\pgfqpoint{5.269914in}{1.947999in}}%
\pgfpathlineto{\pgfqpoint{5.260978in}{1.947999in}}%
\pgfpathlineto{\pgfqpoint{5.260978in}{1.813884in}}%
\pgfpathclose%
\pgfusepath{fill}%
\end{pgfscope}%
\begin{pgfscope}%
\pgfpathrectangle{\pgfqpoint{3.722897in}{0.857143in}}{\pgfqpoint{2.627103in}{1.813434in}}%
\pgfusepath{clip}%
\pgfsetbuttcap%
\pgfsetmiterjoin%
\definecolor{currentfill}{rgb}{0.133298,0.375282,0.379395}%
\pgfsetfillcolor{currentfill}%
\pgfsetlinewidth{0.000000pt}%
\definecolor{currentstroke}{rgb}{0.000000,0.000000,0.000000}%
\pgfsetstrokecolor{currentstroke}%
\pgfsetstrokeopacity{0.000000}%
\pgfsetdash{}{0pt}%
\pgfpathmoveto{\pgfqpoint{5.272148in}{1.813884in}}%
\pgfpathlineto{\pgfqpoint{5.281085in}{1.813884in}}%
\pgfpathlineto{\pgfqpoint{5.281085in}{1.931979in}}%
\pgfpathlineto{\pgfqpoint{5.272148in}{1.931979in}}%
\pgfpathlineto{\pgfqpoint{5.272148in}{1.813884in}}%
\pgfpathclose%
\pgfusepath{fill}%
\end{pgfscope}%
\begin{pgfscope}%
\pgfpathrectangle{\pgfqpoint{3.722897in}{0.857143in}}{\pgfqpoint{2.627103in}{1.813434in}}%
\pgfusepath{clip}%
\pgfsetbuttcap%
\pgfsetmiterjoin%
\definecolor{currentfill}{rgb}{0.133298,0.375282,0.379395}%
\pgfsetfillcolor{currentfill}%
\pgfsetlinewidth{0.000000pt}%
\definecolor{currentstroke}{rgb}{0.000000,0.000000,0.000000}%
\pgfsetstrokecolor{currentstroke}%
\pgfsetstrokeopacity{0.000000}%
\pgfsetdash{}{0pt}%
\pgfpathmoveto{\pgfqpoint{5.283319in}{1.813884in}}%
\pgfpathlineto{\pgfqpoint{5.292255in}{1.813884in}}%
\pgfpathlineto{\pgfqpoint{5.292255in}{1.940582in}}%
\pgfpathlineto{\pgfqpoint{5.283319in}{1.940582in}}%
\pgfpathlineto{\pgfqpoint{5.283319in}{1.813884in}}%
\pgfpathclose%
\pgfusepath{fill}%
\end{pgfscope}%
\begin{pgfscope}%
\pgfpathrectangle{\pgfqpoint{3.722897in}{0.857143in}}{\pgfqpoint{2.627103in}{1.813434in}}%
\pgfusepath{clip}%
\pgfsetbuttcap%
\pgfsetmiterjoin%
\definecolor{currentfill}{rgb}{0.133298,0.375282,0.379395}%
\pgfsetfillcolor{currentfill}%
\pgfsetlinewidth{0.000000pt}%
\definecolor{currentstroke}{rgb}{0.000000,0.000000,0.000000}%
\pgfsetstrokecolor{currentstroke}%
\pgfsetstrokeopacity{0.000000}%
\pgfsetdash{}{0pt}%
\pgfpathmoveto{\pgfqpoint{5.294489in}{1.813884in}}%
\pgfpathlineto{\pgfqpoint{5.303426in}{1.813884in}}%
\pgfpathlineto{\pgfqpoint{5.303426in}{1.935619in}}%
\pgfpathlineto{\pgfqpoint{5.294489in}{1.935619in}}%
\pgfpathlineto{\pgfqpoint{5.294489in}{1.813884in}}%
\pgfpathclose%
\pgfusepath{fill}%
\end{pgfscope}%
\begin{pgfscope}%
\pgfpathrectangle{\pgfqpoint{3.722897in}{0.857143in}}{\pgfqpoint{2.627103in}{1.813434in}}%
\pgfusepath{clip}%
\pgfsetbuttcap%
\pgfsetmiterjoin%
\definecolor{currentfill}{rgb}{0.133298,0.375282,0.379395}%
\pgfsetfillcolor{currentfill}%
\pgfsetlinewidth{0.000000pt}%
\definecolor{currentstroke}{rgb}{0.000000,0.000000,0.000000}%
\pgfsetstrokecolor{currentstroke}%
\pgfsetstrokeopacity{0.000000}%
\pgfsetdash{}{0pt}%
\pgfpathmoveto{\pgfqpoint{5.305660in}{1.813884in}}%
\pgfpathlineto{\pgfqpoint{5.314597in}{1.813884in}}%
\pgfpathlineto{\pgfqpoint{5.314597in}{1.931317in}}%
\pgfpathlineto{\pgfqpoint{5.305660in}{1.931317in}}%
\pgfpathlineto{\pgfqpoint{5.305660in}{1.813884in}}%
\pgfpathclose%
\pgfusepath{fill}%
\end{pgfscope}%
\begin{pgfscope}%
\pgfpathrectangle{\pgfqpoint{3.722897in}{0.857143in}}{\pgfqpoint{2.627103in}{1.813434in}}%
\pgfusepath{clip}%
\pgfsetbuttcap%
\pgfsetmiterjoin%
\definecolor{currentfill}{rgb}{0.133298,0.375282,0.379395}%
\pgfsetfillcolor{currentfill}%
\pgfsetlinewidth{0.000000pt}%
\definecolor{currentstroke}{rgb}{0.000000,0.000000,0.000000}%
\pgfsetstrokecolor{currentstroke}%
\pgfsetstrokeopacity{0.000000}%
\pgfsetdash{}{0pt}%
\pgfpathmoveto{\pgfqpoint{5.316831in}{1.813884in}}%
\pgfpathlineto{\pgfqpoint{5.325767in}{1.813884in}}%
\pgfpathlineto{\pgfqpoint{5.325767in}{1.931629in}}%
\pgfpathlineto{\pgfqpoint{5.316831in}{1.931629in}}%
\pgfpathlineto{\pgfqpoint{5.316831in}{1.813884in}}%
\pgfpathclose%
\pgfusepath{fill}%
\end{pgfscope}%
\begin{pgfscope}%
\pgfpathrectangle{\pgfqpoint{3.722897in}{0.857143in}}{\pgfqpoint{2.627103in}{1.813434in}}%
\pgfusepath{clip}%
\pgfsetbuttcap%
\pgfsetmiterjoin%
\definecolor{currentfill}{rgb}{0.133298,0.375282,0.379395}%
\pgfsetfillcolor{currentfill}%
\pgfsetlinewidth{0.000000pt}%
\definecolor{currentstroke}{rgb}{0.000000,0.000000,0.000000}%
\pgfsetstrokecolor{currentstroke}%
\pgfsetstrokeopacity{0.000000}%
\pgfsetdash{}{0pt}%
\pgfpathmoveto{\pgfqpoint{5.328001in}{1.813884in}}%
\pgfpathlineto{\pgfqpoint{5.336938in}{1.813884in}}%
\pgfpathlineto{\pgfqpoint{5.336938in}{1.918067in}}%
\pgfpathlineto{\pgfqpoint{5.328001in}{1.918067in}}%
\pgfpathlineto{\pgfqpoint{5.328001in}{1.813884in}}%
\pgfpathclose%
\pgfusepath{fill}%
\end{pgfscope}%
\begin{pgfscope}%
\pgfpathrectangle{\pgfqpoint{3.722897in}{0.857143in}}{\pgfqpoint{2.627103in}{1.813434in}}%
\pgfusepath{clip}%
\pgfsetbuttcap%
\pgfsetmiterjoin%
\definecolor{currentfill}{rgb}{0.133298,0.375282,0.379395}%
\pgfsetfillcolor{currentfill}%
\pgfsetlinewidth{0.000000pt}%
\definecolor{currentstroke}{rgb}{0.000000,0.000000,0.000000}%
\pgfsetstrokecolor{currentstroke}%
\pgfsetstrokeopacity{0.000000}%
\pgfsetdash{}{0pt}%
\pgfpathmoveto{\pgfqpoint{5.339172in}{1.813884in}}%
\pgfpathlineto{\pgfqpoint{5.348108in}{1.813884in}}%
\pgfpathlineto{\pgfqpoint{5.348108in}{1.912218in}}%
\pgfpathlineto{\pgfqpoint{5.339172in}{1.912218in}}%
\pgfpathlineto{\pgfqpoint{5.339172in}{1.813884in}}%
\pgfpathclose%
\pgfusepath{fill}%
\end{pgfscope}%
\begin{pgfscope}%
\pgfpathrectangle{\pgfqpoint{3.722897in}{0.857143in}}{\pgfqpoint{2.627103in}{1.813434in}}%
\pgfusepath{clip}%
\pgfsetbuttcap%
\pgfsetmiterjoin%
\definecolor{currentfill}{rgb}{0.133298,0.375282,0.379395}%
\pgfsetfillcolor{currentfill}%
\pgfsetlinewidth{0.000000pt}%
\definecolor{currentstroke}{rgb}{0.000000,0.000000,0.000000}%
\pgfsetstrokecolor{currentstroke}%
\pgfsetstrokeopacity{0.000000}%
\pgfsetdash{}{0pt}%
\pgfpathmoveto{\pgfqpoint{5.350343in}{1.813884in}}%
\pgfpathlineto{\pgfqpoint{5.359279in}{1.813884in}}%
\pgfpathlineto{\pgfqpoint{5.359279in}{1.912132in}}%
\pgfpathlineto{\pgfqpoint{5.350343in}{1.912132in}}%
\pgfpathlineto{\pgfqpoint{5.350343in}{1.813884in}}%
\pgfpathclose%
\pgfusepath{fill}%
\end{pgfscope}%
\begin{pgfscope}%
\pgfpathrectangle{\pgfqpoint{3.722897in}{0.857143in}}{\pgfqpoint{2.627103in}{1.813434in}}%
\pgfusepath{clip}%
\pgfsetbuttcap%
\pgfsetmiterjoin%
\definecolor{currentfill}{rgb}{0.133298,0.375282,0.379395}%
\pgfsetfillcolor{currentfill}%
\pgfsetlinewidth{0.000000pt}%
\definecolor{currentstroke}{rgb}{0.000000,0.000000,0.000000}%
\pgfsetstrokecolor{currentstroke}%
\pgfsetstrokeopacity{0.000000}%
\pgfsetdash{}{0pt}%
\pgfpathmoveto{\pgfqpoint{5.361513in}{1.813884in}}%
\pgfpathlineto{\pgfqpoint{5.370450in}{1.813884in}}%
\pgfpathlineto{\pgfqpoint{5.370450in}{1.909698in}}%
\pgfpathlineto{\pgfqpoint{5.361513in}{1.909698in}}%
\pgfpathlineto{\pgfqpoint{5.361513in}{1.813884in}}%
\pgfpathclose%
\pgfusepath{fill}%
\end{pgfscope}%
\begin{pgfscope}%
\pgfpathrectangle{\pgfqpoint{3.722897in}{0.857143in}}{\pgfqpoint{2.627103in}{1.813434in}}%
\pgfusepath{clip}%
\pgfsetbuttcap%
\pgfsetmiterjoin%
\definecolor{currentfill}{rgb}{0.133298,0.375282,0.379395}%
\pgfsetfillcolor{currentfill}%
\pgfsetlinewidth{0.000000pt}%
\definecolor{currentstroke}{rgb}{0.000000,0.000000,0.000000}%
\pgfsetstrokecolor{currentstroke}%
\pgfsetstrokeopacity{0.000000}%
\pgfsetdash{}{0pt}%
\pgfpathmoveto{\pgfqpoint{5.372684in}{1.813884in}}%
\pgfpathlineto{\pgfqpoint{5.381620in}{1.813884in}}%
\pgfpathlineto{\pgfqpoint{5.381620in}{1.908817in}}%
\pgfpathlineto{\pgfqpoint{5.372684in}{1.908817in}}%
\pgfpathlineto{\pgfqpoint{5.372684in}{1.813884in}}%
\pgfpathclose%
\pgfusepath{fill}%
\end{pgfscope}%
\begin{pgfscope}%
\pgfpathrectangle{\pgfqpoint{3.722897in}{0.857143in}}{\pgfqpoint{2.627103in}{1.813434in}}%
\pgfusepath{clip}%
\pgfsetbuttcap%
\pgfsetmiterjoin%
\definecolor{currentfill}{rgb}{0.133298,0.375282,0.379395}%
\pgfsetfillcolor{currentfill}%
\pgfsetlinewidth{0.000000pt}%
\definecolor{currentstroke}{rgb}{0.000000,0.000000,0.000000}%
\pgfsetstrokecolor{currentstroke}%
\pgfsetstrokeopacity{0.000000}%
\pgfsetdash{}{0pt}%
\pgfpathmoveto{\pgfqpoint{5.383854in}{1.813884in}}%
\pgfpathlineto{\pgfqpoint{5.392791in}{1.813884in}}%
\pgfpathlineto{\pgfqpoint{5.392791in}{1.895676in}}%
\pgfpathlineto{\pgfqpoint{5.383854in}{1.895676in}}%
\pgfpathlineto{\pgfqpoint{5.383854in}{1.813884in}}%
\pgfpathclose%
\pgfusepath{fill}%
\end{pgfscope}%
\begin{pgfscope}%
\pgfpathrectangle{\pgfqpoint{3.722897in}{0.857143in}}{\pgfqpoint{2.627103in}{1.813434in}}%
\pgfusepath{clip}%
\pgfsetbuttcap%
\pgfsetmiterjoin%
\definecolor{currentfill}{rgb}{0.133298,0.375282,0.379395}%
\pgfsetfillcolor{currentfill}%
\pgfsetlinewidth{0.000000pt}%
\definecolor{currentstroke}{rgb}{0.000000,0.000000,0.000000}%
\pgfsetstrokecolor{currentstroke}%
\pgfsetstrokeopacity{0.000000}%
\pgfsetdash{}{0pt}%
\pgfpathmoveto{\pgfqpoint{5.395025in}{1.813884in}}%
\pgfpathlineto{\pgfqpoint{5.403961in}{1.813884in}}%
\pgfpathlineto{\pgfqpoint{5.403961in}{1.894476in}}%
\pgfpathlineto{\pgfqpoint{5.395025in}{1.894476in}}%
\pgfpathlineto{\pgfqpoint{5.395025in}{1.813884in}}%
\pgfpathclose%
\pgfusepath{fill}%
\end{pgfscope}%
\begin{pgfscope}%
\pgfpathrectangle{\pgfqpoint{3.722897in}{0.857143in}}{\pgfqpoint{2.627103in}{1.813434in}}%
\pgfusepath{clip}%
\pgfsetbuttcap%
\pgfsetmiterjoin%
\definecolor{currentfill}{rgb}{0.133298,0.375282,0.379395}%
\pgfsetfillcolor{currentfill}%
\pgfsetlinewidth{0.000000pt}%
\definecolor{currentstroke}{rgb}{0.000000,0.000000,0.000000}%
\pgfsetstrokecolor{currentstroke}%
\pgfsetstrokeopacity{0.000000}%
\pgfsetdash{}{0pt}%
\pgfpathmoveto{\pgfqpoint{5.406196in}{1.813884in}}%
\pgfpathlineto{\pgfqpoint{5.415132in}{1.813884in}}%
\pgfpathlineto{\pgfqpoint{5.415132in}{1.886813in}}%
\pgfpathlineto{\pgfqpoint{5.406196in}{1.886813in}}%
\pgfpathlineto{\pgfqpoint{5.406196in}{1.813884in}}%
\pgfpathclose%
\pgfusepath{fill}%
\end{pgfscope}%
\begin{pgfscope}%
\pgfpathrectangle{\pgfqpoint{3.722897in}{0.857143in}}{\pgfqpoint{2.627103in}{1.813434in}}%
\pgfusepath{clip}%
\pgfsetbuttcap%
\pgfsetmiterjoin%
\definecolor{currentfill}{rgb}{0.133298,0.375282,0.379395}%
\pgfsetfillcolor{currentfill}%
\pgfsetlinewidth{0.000000pt}%
\definecolor{currentstroke}{rgb}{0.000000,0.000000,0.000000}%
\pgfsetstrokecolor{currentstroke}%
\pgfsetstrokeopacity{0.000000}%
\pgfsetdash{}{0pt}%
\pgfpathmoveto{\pgfqpoint{5.417366in}{1.813884in}}%
\pgfpathlineto{\pgfqpoint{5.426303in}{1.813884in}}%
\pgfpathlineto{\pgfqpoint{5.426303in}{1.880397in}}%
\pgfpathlineto{\pgfqpoint{5.417366in}{1.880397in}}%
\pgfpathlineto{\pgfqpoint{5.417366in}{1.813884in}}%
\pgfpathclose%
\pgfusepath{fill}%
\end{pgfscope}%
\begin{pgfscope}%
\pgfpathrectangle{\pgfqpoint{3.722897in}{0.857143in}}{\pgfqpoint{2.627103in}{1.813434in}}%
\pgfusepath{clip}%
\pgfsetbuttcap%
\pgfsetmiterjoin%
\definecolor{currentfill}{rgb}{0.133298,0.375282,0.379395}%
\pgfsetfillcolor{currentfill}%
\pgfsetlinewidth{0.000000pt}%
\definecolor{currentstroke}{rgb}{0.000000,0.000000,0.000000}%
\pgfsetstrokecolor{currentstroke}%
\pgfsetstrokeopacity{0.000000}%
\pgfsetdash{}{0pt}%
\pgfpathmoveto{\pgfqpoint{5.428537in}{1.813884in}}%
\pgfpathlineto{\pgfqpoint{5.437473in}{1.813884in}}%
\pgfpathlineto{\pgfqpoint{5.437473in}{1.871835in}}%
\pgfpathlineto{\pgfqpoint{5.428537in}{1.871835in}}%
\pgfpathlineto{\pgfqpoint{5.428537in}{1.813884in}}%
\pgfpathclose%
\pgfusepath{fill}%
\end{pgfscope}%
\begin{pgfscope}%
\pgfpathrectangle{\pgfqpoint{3.722897in}{0.857143in}}{\pgfqpoint{2.627103in}{1.813434in}}%
\pgfusepath{clip}%
\pgfsetbuttcap%
\pgfsetmiterjoin%
\definecolor{currentfill}{rgb}{0.133298,0.375282,0.379395}%
\pgfsetfillcolor{currentfill}%
\pgfsetlinewidth{0.000000pt}%
\definecolor{currentstroke}{rgb}{0.000000,0.000000,0.000000}%
\pgfsetstrokecolor{currentstroke}%
\pgfsetstrokeopacity{0.000000}%
\pgfsetdash{}{0pt}%
\pgfpathmoveto{\pgfqpoint{5.439707in}{1.813884in}}%
\pgfpathlineto{\pgfqpoint{5.448644in}{1.813884in}}%
\pgfpathlineto{\pgfqpoint{5.448644in}{1.864648in}}%
\pgfpathlineto{\pgfqpoint{5.439707in}{1.864648in}}%
\pgfpathlineto{\pgfqpoint{5.439707in}{1.813884in}}%
\pgfpathclose%
\pgfusepath{fill}%
\end{pgfscope}%
\begin{pgfscope}%
\pgfpathrectangle{\pgfqpoint{3.722897in}{0.857143in}}{\pgfqpoint{2.627103in}{1.813434in}}%
\pgfusepath{clip}%
\pgfsetbuttcap%
\pgfsetmiterjoin%
\definecolor{currentfill}{rgb}{0.133298,0.375282,0.379395}%
\pgfsetfillcolor{currentfill}%
\pgfsetlinewidth{0.000000pt}%
\definecolor{currentstroke}{rgb}{0.000000,0.000000,0.000000}%
\pgfsetstrokecolor{currentstroke}%
\pgfsetstrokeopacity{0.000000}%
\pgfsetdash{}{0pt}%
\pgfpathmoveto{\pgfqpoint{5.450878in}{1.813884in}}%
\pgfpathlineto{\pgfqpoint{5.459814in}{1.813884in}}%
\pgfpathlineto{\pgfqpoint{5.459814in}{1.861226in}}%
\pgfpathlineto{\pgfqpoint{5.450878in}{1.861226in}}%
\pgfpathlineto{\pgfqpoint{5.450878in}{1.813884in}}%
\pgfpathclose%
\pgfusepath{fill}%
\end{pgfscope}%
\begin{pgfscope}%
\pgfpathrectangle{\pgfqpoint{3.722897in}{0.857143in}}{\pgfqpoint{2.627103in}{1.813434in}}%
\pgfusepath{clip}%
\pgfsetbuttcap%
\pgfsetmiterjoin%
\definecolor{currentfill}{rgb}{0.133298,0.375282,0.379395}%
\pgfsetfillcolor{currentfill}%
\pgfsetlinewidth{0.000000pt}%
\definecolor{currentstroke}{rgb}{0.000000,0.000000,0.000000}%
\pgfsetstrokecolor{currentstroke}%
\pgfsetstrokeopacity{0.000000}%
\pgfsetdash{}{0pt}%
\pgfpathmoveto{\pgfqpoint{5.462049in}{1.813884in}}%
\pgfpathlineto{\pgfqpoint{5.470985in}{1.813884in}}%
\pgfpathlineto{\pgfqpoint{5.470985in}{1.857235in}}%
\pgfpathlineto{\pgfqpoint{5.462049in}{1.857235in}}%
\pgfpathlineto{\pgfqpoint{5.462049in}{1.813884in}}%
\pgfpathclose%
\pgfusepath{fill}%
\end{pgfscope}%
\begin{pgfscope}%
\pgfpathrectangle{\pgfqpoint{3.722897in}{0.857143in}}{\pgfqpoint{2.627103in}{1.813434in}}%
\pgfusepath{clip}%
\pgfsetbuttcap%
\pgfsetmiterjoin%
\definecolor{currentfill}{rgb}{0.133298,0.375282,0.379395}%
\pgfsetfillcolor{currentfill}%
\pgfsetlinewidth{0.000000pt}%
\definecolor{currentstroke}{rgb}{0.000000,0.000000,0.000000}%
\pgfsetstrokecolor{currentstroke}%
\pgfsetstrokeopacity{0.000000}%
\pgfsetdash{}{0pt}%
\pgfpathmoveto{\pgfqpoint{5.473219in}{1.813884in}}%
\pgfpathlineto{\pgfqpoint{5.482156in}{1.813884in}}%
\pgfpathlineto{\pgfqpoint{5.482156in}{1.855019in}}%
\pgfpathlineto{\pgfqpoint{5.473219in}{1.855019in}}%
\pgfpathlineto{\pgfqpoint{5.473219in}{1.813884in}}%
\pgfpathclose%
\pgfusepath{fill}%
\end{pgfscope}%
\begin{pgfscope}%
\pgfpathrectangle{\pgfqpoint{3.722897in}{0.857143in}}{\pgfqpoint{2.627103in}{1.813434in}}%
\pgfusepath{clip}%
\pgfsetbuttcap%
\pgfsetmiterjoin%
\definecolor{currentfill}{rgb}{0.133298,0.375282,0.379395}%
\pgfsetfillcolor{currentfill}%
\pgfsetlinewidth{0.000000pt}%
\definecolor{currentstroke}{rgb}{0.000000,0.000000,0.000000}%
\pgfsetstrokecolor{currentstroke}%
\pgfsetstrokeopacity{0.000000}%
\pgfsetdash{}{0pt}%
\pgfpathmoveto{\pgfqpoint{5.484390in}{1.813884in}}%
\pgfpathlineto{\pgfqpoint{5.493326in}{1.813884in}}%
\pgfpathlineto{\pgfqpoint{5.493326in}{1.846584in}}%
\pgfpathlineto{\pgfqpoint{5.484390in}{1.846584in}}%
\pgfpathlineto{\pgfqpoint{5.484390in}{1.813884in}}%
\pgfpathclose%
\pgfusepath{fill}%
\end{pgfscope}%
\begin{pgfscope}%
\pgfpathrectangle{\pgfqpoint{3.722897in}{0.857143in}}{\pgfqpoint{2.627103in}{1.813434in}}%
\pgfusepath{clip}%
\pgfsetbuttcap%
\pgfsetmiterjoin%
\definecolor{currentfill}{rgb}{0.133298,0.375282,0.379395}%
\pgfsetfillcolor{currentfill}%
\pgfsetlinewidth{0.000000pt}%
\definecolor{currentstroke}{rgb}{0.000000,0.000000,0.000000}%
\pgfsetstrokecolor{currentstroke}%
\pgfsetstrokeopacity{0.000000}%
\pgfsetdash{}{0pt}%
\pgfpathmoveto{\pgfqpoint{5.495560in}{1.813884in}}%
\pgfpathlineto{\pgfqpoint{5.504497in}{1.813884in}}%
\pgfpathlineto{\pgfqpoint{5.504497in}{1.838721in}}%
\pgfpathlineto{\pgfqpoint{5.495560in}{1.838721in}}%
\pgfpathlineto{\pgfqpoint{5.495560in}{1.813884in}}%
\pgfpathclose%
\pgfusepath{fill}%
\end{pgfscope}%
\begin{pgfscope}%
\pgfpathrectangle{\pgfqpoint{3.722897in}{0.857143in}}{\pgfqpoint{2.627103in}{1.813434in}}%
\pgfusepath{clip}%
\pgfsetbuttcap%
\pgfsetmiterjoin%
\definecolor{currentfill}{rgb}{0.133298,0.375282,0.379395}%
\pgfsetfillcolor{currentfill}%
\pgfsetlinewidth{0.000000pt}%
\definecolor{currentstroke}{rgb}{0.000000,0.000000,0.000000}%
\pgfsetstrokecolor{currentstroke}%
\pgfsetstrokeopacity{0.000000}%
\pgfsetdash{}{0pt}%
\pgfpathmoveto{\pgfqpoint{5.506731in}{1.813884in}}%
\pgfpathlineto{\pgfqpoint{5.515667in}{1.813884in}}%
\pgfpathlineto{\pgfqpoint{5.515667in}{1.821421in}}%
\pgfpathlineto{\pgfqpoint{5.506731in}{1.821421in}}%
\pgfpathlineto{\pgfqpoint{5.506731in}{1.813884in}}%
\pgfpathclose%
\pgfusepath{fill}%
\end{pgfscope}%
\begin{pgfscope}%
\pgfpathrectangle{\pgfqpoint{3.722897in}{0.857143in}}{\pgfqpoint{2.627103in}{1.813434in}}%
\pgfusepath{clip}%
\pgfsetbuttcap%
\pgfsetmiterjoin%
\definecolor{currentfill}{rgb}{0.133298,0.375282,0.379395}%
\pgfsetfillcolor{currentfill}%
\pgfsetlinewidth{0.000000pt}%
\definecolor{currentstroke}{rgb}{0.000000,0.000000,0.000000}%
\pgfsetstrokecolor{currentstroke}%
\pgfsetstrokeopacity{0.000000}%
\pgfsetdash{}{0pt}%
\pgfpathmoveto{\pgfqpoint{5.517902in}{1.813884in}}%
\pgfpathlineto{\pgfqpoint{5.526838in}{1.813884in}}%
\pgfpathlineto{\pgfqpoint{5.526838in}{1.820844in}}%
\pgfpathlineto{\pgfqpoint{5.517902in}{1.820844in}}%
\pgfpathlineto{\pgfqpoint{5.517902in}{1.813884in}}%
\pgfpathclose%
\pgfusepath{fill}%
\end{pgfscope}%
\begin{pgfscope}%
\pgfpathrectangle{\pgfqpoint{3.722897in}{0.857143in}}{\pgfqpoint{2.627103in}{1.813434in}}%
\pgfusepath{clip}%
\pgfsetbuttcap%
\pgfsetmiterjoin%
\definecolor{currentfill}{rgb}{0.133298,0.375282,0.379395}%
\pgfsetfillcolor{currentfill}%
\pgfsetlinewidth{0.000000pt}%
\definecolor{currentstroke}{rgb}{0.000000,0.000000,0.000000}%
\pgfsetstrokecolor{currentstroke}%
\pgfsetstrokeopacity{0.000000}%
\pgfsetdash{}{0pt}%
\pgfpathmoveto{\pgfqpoint{5.529072in}{1.807007in}}%
\pgfpathlineto{\pgfqpoint{5.538009in}{1.807007in}}%
\pgfpathlineto{\pgfqpoint{5.538009in}{1.804799in}}%
\pgfpathlineto{\pgfqpoint{5.529072in}{1.804799in}}%
\pgfpathlineto{\pgfqpoint{5.529072in}{1.807007in}}%
\pgfpathclose%
\pgfusepath{fill}%
\end{pgfscope}%
\begin{pgfscope}%
\pgfpathrectangle{\pgfqpoint{3.722897in}{0.857143in}}{\pgfqpoint{2.627103in}{1.813434in}}%
\pgfusepath{clip}%
\pgfsetbuttcap%
\pgfsetmiterjoin%
\definecolor{currentfill}{rgb}{0.133298,0.375282,0.379395}%
\pgfsetfillcolor{currentfill}%
\pgfsetlinewidth{0.000000pt}%
\definecolor{currentstroke}{rgb}{0.000000,0.000000,0.000000}%
\pgfsetstrokecolor{currentstroke}%
\pgfsetstrokeopacity{0.000000}%
\pgfsetdash{}{0pt}%
\pgfpathmoveto{\pgfqpoint{5.540243in}{1.813884in}}%
\pgfpathlineto{\pgfqpoint{5.549179in}{1.813884in}}%
\pgfpathlineto{\pgfqpoint{5.549179in}{1.818994in}}%
\pgfpathlineto{\pgfqpoint{5.540243in}{1.818994in}}%
\pgfpathlineto{\pgfqpoint{5.540243in}{1.813884in}}%
\pgfpathclose%
\pgfusepath{fill}%
\end{pgfscope}%
\begin{pgfscope}%
\pgfpathrectangle{\pgfqpoint{3.722897in}{0.857143in}}{\pgfqpoint{2.627103in}{1.813434in}}%
\pgfusepath{clip}%
\pgfsetbuttcap%
\pgfsetmiterjoin%
\definecolor{currentfill}{rgb}{0.133298,0.375282,0.379395}%
\pgfsetfillcolor{currentfill}%
\pgfsetlinewidth{0.000000pt}%
\definecolor{currentstroke}{rgb}{0.000000,0.000000,0.000000}%
\pgfsetstrokecolor{currentstroke}%
\pgfsetstrokeopacity{0.000000}%
\pgfsetdash{}{0pt}%
\pgfpathmoveto{\pgfqpoint{5.551413in}{1.812969in}}%
\pgfpathlineto{\pgfqpoint{5.560350in}{1.812969in}}%
\pgfpathlineto{\pgfqpoint{5.560350in}{1.812808in}}%
\pgfpathlineto{\pgfqpoint{5.551413in}{1.812808in}}%
\pgfpathlineto{\pgfqpoint{5.551413in}{1.812969in}}%
\pgfpathclose%
\pgfusepath{fill}%
\end{pgfscope}%
\begin{pgfscope}%
\pgfpathrectangle{\pgfqpoint{3.722897in}{0.857143in}}{\pgfqpoint{2.627103in}{1.813434in}}%
\pgfusepath{clip}%
\pgfsetbuttcap%
\pgfsetmiterjoin%
\definecolor{currentfill}{rgb}{0.133298,0.375282,0.379395}%
\pgfsetfillcolor{currentfill}%
\pgfsetlinewidth{0.000000pt}%
\definecolor{currentstroke}{rgb}{0.000000,0.000000,0.000000}%
\pgfsetstrokecolor{currentstroke}%
\pgfsetstrokeopacity{0.000000}%
\pgfsetdash{}{0pt}%
\pgfpathmoveto{\pgfqpoint{5.562584in}{1.813884in}}%
\pgfpathlineto{\pgfqpoint{5.571521in}{1.813884in}}%
\pgfpathlineto{\pgfqpoint{5.571521in}{1.808875in}}%
\pgfpathlineto{\pgfqpoint{5.562584in}{1.808875in}}%
\pgfpathlineto{\pgfqpoint{5.562584in}{1.813884in}}%
\pgfpathclose%
\pgfusepath{fill}%
\end{pgfscope}%
\begin{pgfscope}%
\pgfpathrectangle{\pgfqpoint{3.722897in}{0.857143in}}{\pgfqpoint{2.627103in}{1.813434in}}%
\pgfusepath{clip}%
\pgfsetbuttcap%
\pgfsetmiterjoin%
\definecolor{currentfill}{rgb}{0.133298,0.375282,0.379395}%
\pgfsetfillcolor{currentfill}%
\pgfsetlinewidth{0.000000pt}%
\definecolor{currentstroke}{rgb}{0.000000,0.000000,0.000000}%
\pgfsetstrokecolor{currentstroke}%
\pgfsetstrokeopacity{0.000000}%
\pgfsetdash{}{0pt}%
\pgfpathmoveto{\pgfqpoint{5.573755in}{1.818524in}}%
\pgfpathlineto{\pgfqpoint{5.582691in}{1.818524in}}%
\pgfpathlineto{\pgfqpoint{5.582691in}{1.821168in}}%
\pgfpathlineto{\pgfqpoint{5.573755in}{1.821168in}}%
\pgfpathlineto{\pgfqpoint{5.573755in}{1.818524in}}%
\pgfpathclose%
\pgfusepath{fill}%
\end{pgfscope}%
\begin{pgfscope}%
\pgfpathrectangle{\pgfqpoint{3.722897in}{0.857143in}}{\pgfqpoint{2.627103in}{1.813434in}}%
\pgfusepath{clip}%
\pgfsetbuttcap%
\pgfsetmiterjoin%
\definecolor{currentfill}{rgb}{0.133298,0.375282,0.379395}%
\pgfsetfillcolor{currentfill}%
\pgfsetlinewidth{0.000000pt}%
\definecolor{currentstroke}{rgb}{0.000000,0.000000,0.000000}%
\pgfsetstrokecolor{currentstroke}%
\pgfsetstrokeopacity{0.000000}%
\pgfsetdash{}{0pt}%
\pgfpathmoveto{\pgfqpoint{5.584925in}{1.813884in}}%
\pgfpathlineto{\pgfqpoint{5.593862in}{1.813884in}}%
\pgfpathlineto{\pgfqpoint{5.593862in}{1.811206in}}%
\pgfpathlineto{\pgfqpoint{5.584925in}{1.811206in}}%
\pgfpathlineto{\pgfqpoint{5.584925in}{1.813884in}}%
\pgfpathclose%
\pgfusepath{fill}%
\end{pgfscope}%
\begin{pgfscope}%
\pgfpathrectangle{\pgfqpoint{3.722897in}{0.857143in}}{\pgfqpoint{2.627103in}{1.813434in}}%
\pgfusepath{clip}%
\pgfsetbuttcap%
\pgfsetmiterjoin%
\definecolor{currentfill}{rgb}{0.133298,0.375282,0.379395}%
\pgfsetfillcolor{currentfill}%
\pgfsetlinewidth{0.000000pt}%
\definecolor{currentstroke}{rgb}{0.000000,0.000000,0.000000}%
\pgfsetstrokecolor{currentstroke}%
\pgfsetstrokeopacity{0.000000}%
\pgfsetdash{}{0pt}%
\pgfpathmoveto{\pgfqpoint{5.596096in}{1.822688in}}%
\pgfpathlineto{\pgfqpoint{5.605032in}{1.822688in}}%
\pgfpathlineto{\pgfqpoint{5.605032in}{1.823249in}}%
\pgfpathlineto{\pgfqpoint{5.596096in}{1.823249in}}%
\pgfpathlineto{\pgfqpoint{5.596096in}{1.822688in}}%
\pgfpathclose%
\pgfusepath{fill}%
\end{pgfscope}%
\begin{pgfscope}%
\pgfpathrectangle{\pgfqpoint{3.722897in}{0.857143in}}{\pgfqpoint{2.627103in}{1.813434in}}%
\pgfusepath{clip}%
\pgfsetbuttcap%
\pgfsetmiterjoin%
\definecolor{currentfill}{rgb}{0.133298,0.375282,0.379395}%
\pgfsetfillcolor{currentfill}%
\pgfsetlinewidth{0.000000pt}%
\definecolor{currentstroke}{rgb}{0.000000,0.000000,0.000000}%
\pgfsetstrokecolor{currentstroke}%
\pgfsetstrokeopacity{0.000000}%
\pgfsetdash{}{0pt}%
\pgfpathmoveto{\pgfqpoint{5.607266in}{1.824522in}}%
\pgfpathlineto{\pgfqpoint{5.616203in}{1.824522in}}%
\pgfpathlineto{\pgfqpoint{5.616203in}{1.831711in}}%
\pgfpathlineto{\pgfqpoint{5.607266in}{1.831711in}}%
\pgfpathlineto{\pgfqpoint{5.607266in}{1.824522in}}%
\pgfpathclose%
\pgfusepath{fill}%
\end{pgfscope}%
\begin{pgfscope}%
\pgfpathrectangle{\pgfqpoint{3.722897in}{0.857143in}}{\pgfqpoint{2.627103in}{1.813434in}}%
\pgfusepath{clip}%
\pgfsetbuttcap%
\pgfsetmiterjoin%
\definecolor{currentfill}{rgb}{0.133298,0.375282,0.379395}%
\pgfsetfillcolor{currentfill}%
\pgfsetlinewidth{0.000000pt}%
\definecolor{currentstroke}{rgb}{0.000000,0.000000,0.000000}%
\pgfsetstrokecolor{currentstroke}%
\pgfsetstrokeopacity{0.000000}%
\pgfsetdash{}{0pt}%
\pgfpathmoveto{\pgfqpoint{5.618437in}{1.825573in}}%
\pgfpathlineto{\pgfqpoint{5.627374in}{1.825573in}}%
\pgfpathlineto{\pgfqpoint{5.627374in}{1.840008in}}%
\pgfpathlineto{\pgfqpoint{5.618437in}{1.840008in}}%
\pgfpathlineto{\pgfqpoint{5.618437in}{1.825573in}}%
\pgfpathclose%
\pgfusepath{fill}%
\end{pgfscope}%
\begin{pgfscope}%
\pgfpathrectangle{\pgfqpoint{3.722897in}{0.857143in}}{\pgfqpoint{2.627103in}{1.813434in}}%
\pgfusepath{clip}%
\pgfsetbuttcap%
\pgfsetmiterjoin%
\definecolor{currentfill}{rgb}{0.133298,0.375282,0.379395}%
\pgfsetfillcolor{currentfill}%
\pgfsetlinewidth{0.000000pt}%
\definecolor{currentstroke}{rgb}{0.000000,0.000000,0.000000}%
\pgfsetstrokecolor{currentstroke}%
\pgfsetstrokeopacity{0.000000}%
\pgfsetdash{}{0pt}%
\pgfpathmoveto{\pgfqpoint{5.629608in}{1.825735in}}%
\pgfpathlineto{\pgfqpoint{5.638544in}{1.825735in}}%
\pgfpathlineto{\pgfqpoint{5.638544in}{1.844204in}}%
\pgfpathlineto{\pgfqpoint{5.629608in}{1.844204in}}%
\pgfpathlineto{\pgfqpoint{5.629608in}{1.825735in}}%
\pgfpathclose%
\pgfusepath{fill}%
\end{pgfscope}%
\begin{pgfscope}%
\pgfpathrectangle{\pgfqpoint{3.722897in}{0.857143in}}{\pgfqpoint{2.627103in}{1.813434in}}%
\pgfusepath{clip}%
\pgfsetbuttcap%
\pgfsetmiterjoin%
\definecolor{currentfill}{rgb}{0.133298,0.375282,0.379395}%
\pgfsetfillcolor{currentfill}%
\pgfsetlinewidth{0.000000pt}%
\definecolor{currentstroke}{rgb}{0.000000,0.000000,0.000000}%
\pgfsetstrokecolor{currentstroke}%
\pgfsetstrokeopacity{0.000000}%
\pgfsetdash{}{0pt}%
\pgfpathmoveto{\pgfqpoint{5.640778in}{1.826902in}}%
\pgfpathlineto{\pgfqpoint{5.649715in}{1.826902in}}%
\pgfpathlineto{\pgfqpoint{5.649715in}{1.860224in}}%
\pgfpathlineto{\pgfqpoint{5.640778in}{1.860224in}}%
\pgfpathlineto{\pgfqpoint{5.640778in}{1.826902in}}%
\pgfpathclose%
\pgfusepath{fill}%
\end{pgfscope}%
\begin{pgfscope}%
\pgfpathrectangle{\pgfqpoint{3.722897in}{0.857143in}}{\pgfqpoint{2.627103in}{1.813434in}}%
\pgfusepath{clip}%
\pgfsetbuttcap%
\pgfsetmiterjoin%
\definecolor{currentfill}{rgb}{0.133298,0.375282,0.379395}%
\pgfsetfillcolor{currentfill}%
\pgfsetlinewidth{0.000000pt}%
\definecolor{currentstroke}{rgb}{0.000000,0.000000,0.000000}%
\pgfsetstrokecolor{currentstroke}%
\pgfsetstrokeopacity{0.000000}%
\pgfsetdash{}{0pt}%
\pgfpathmoveto{\pgfqpoint{5.651949in}{1.827792in}}%
\pgfpathlineto{\pgfqpoint{5.660885in}{1.827792in}}%
\pgfpathlineto{\pgfqpoint{5.660885in}{1.838012in}}%
\pgfpathlineto{\pgfqpoint{5.651949in}{1.838012in}}%
\pgfpathlineto{\pgfqpoint{5.651949in}{1.827792in}}%
\pgfpathclose%
\pgfusepath{fill}%
\end{pgfscope}%
\begin{pgfscope}%
\pgfpathrectangle{\pgfqpoint{3.722897in}{0.857143in}}{\pgfqpoint{2.627103in}{1.813434in}}%
\pgfusepath{clip}%
\pgfsetbuttcap%
\pgfsetmiterjoin%
\definecolor{currentfill}{rgb}{0.133298,0.375282,0.379395}%
\pgfsetfillcolor{currentfill}%
\pgfsetlinewidth{0.000000pt}%
\definecolor{currentstroke}{rgb}{0.000000,0.000000,0.000000}%
\pgfsetstrokecolor{currentstroke}%
\pgfsetstrokeopacity{0.000000}%
\pgfsetdash{}{0pt}%
\pgfpathmoveto{\pgfqpoint{5.663119in}{1.828191in}}%
\pgfpathlineto{\pgfqpoint{5.672056in}{1.828191in}}%
\pgfpathlineto{\pgfqpoint{5.672056in}{1.852502in}}%
\pgfpathlineto{\pgfqpoint{5.663119in}{1.852502in}}%
\pgfpathlineto{\pgfqpoint{5.663119in}{1.828191in}}%
\pgfpathclose%
\pgfusepath{fill}%
\end{pgfscope}%
\begin{pgfscope}%
\pgfpathrectangle{\pgfqpoint{3.722897in}{0.857143in}}{\pgfqpoint{2.627103in}{1.813434in}}%
\pgfusepath{clip}%
\pgfsetbuttcap%
\pgfsetmiterjoin%
\definecolor{currentfill}{rgb}{0.133298,0.375282,0.379395}%
\pgfsetfillcolor{currentfill}%
\pgfsetlinewidth{0.000000pt}%
\definecolor{currentstroke}{rgb}{0.000000,0.000000,0.000000}%
\pgfsetstrokecolor{currentstroke}%
\pgfsetstrokeopacity{0.000000}%
\pgfsetdash{}{0pt}%
\pgfpathmoveto{\pgfqpoint{5.674290in}{1.829897in}}%
\pgfpathlineto{\pgfqpoint{5.683227in}{1.829897in}}%
\pgfpathlineto{\pgfqpoint{5.683227in}{1.859109in}}%
\pgfpathlineto{\pgfqpoint{5.674290in}{1.859109in}}%
\pgfpathlineto{\pgfqpoint{5.674290in}{1.829897in}}%
\pgfpathclose%
\pgfusepath{fill}%
\end{pgfscope}%
\begin{pgfscope}%
\pgfpathrectangle{\pgfqpoint{3.722897in}{0.857143in}}{\pgfqpoint{2.627103in}{1.813434in}}%
\pgfusepath{clip}%
\pgfsetbuttcap%
\pgfsetmiterjoin%
\definecolor{currentfill}{rgb}{0.133298,0.375282,0.379395}%
\pgfsetfillcolor{currentfill}%
\pgfsetlinewidth{0.000000pt}%
\definecolor{currentstroke}{rgb}{0.000000,0.000000,0.000000}%
\pgfsetstrokecolor{currentstroke}%
\pgfsetstrokeopacity{0.000000}%
\pgfsetdash{}{0pt}%
\pgfpathmoveto{\pgfqpoint{5.685461in}{1.831783in}}%
\pgfpathlineto{\pgfqpoint{5.694397in}{1.831783in}}%
\pgfpathlineto{\pgfqpoint{5.694397in}{1.857978in}}%
\pgfpathlineto{\pgfqpoint{5.685461in}{1.857978in}}%
\pgfpathlineto{\pgfqpoint{5.685461in}{1.831783in}}%
\pgfpathclose%
\pgfusepath{fill}%
\end{pgfscope}%
\begin{pgfscope}%
\pgfpathrectangle{\pgfqpoint{3.722897in}{0.857143in}}{\pgfqpoint{2.627103in}{1.813434in}}%
\pgfusepath{clip}%
\pgfsetbuttcap%
\pgfsetmiterjoin%
\definecolor{currentfill}{rgb}{0.133298,0.375282,0.379395}%
\pgfsetfillcolor{currentfill}%
\pgfsetlinewidth{0.000000pt}%
\definecolor{currentstroke}{rgb}{0.000000,0.000000,0.000000}%
\pgfsetstrokecolor{currentstroke}%
\pgfsetstrokeopacity{0.000000}%
\pgfsetdash{}{0pt}%
\pgfpathmoveto{\pgfqpoint{5.696631in}{1.831701in}}%
\pgfpathlineto{\pgfqpoint{5.705568in}{1.831701in}}%
\pgfpathlineto{\pgfqpoint{5.705568in}{1.861122in}}%
\pgfpathlineto{\pgfqpoint{5.696631in}{1.861122in}}%
\pgfpathlineto{\pgfqpoint{5.696631in}{1.831701in}}%
\pgfpathclose%
\pgfusepath{fill}%
\end{pgfscope}%
\begin{pgfscope}%
\pgfpathrectangle{\pgfqpoint{3.722897in}{0.857143in}}{\pgfqpoint{2.627103in}{1.813434in}}%
\pgfusepath{clip}%
\pgfsetbuttcap%
\pgfsetmiterjoin%
\definecolor{currentfill}{rgb}{0.133298,0.375282,0.379395}%
\pgfsetfillcolor{currentfill}%
\pgfsetlinewidth{0.000000pt}%
\definecolor{currentstroke}{rgb}{0.000000,0.000000,0.000000}%
\pgfsetstrokecolor{currentstroke}%
\pgfsetstrokeopacity{0.000000}%
\pgfsetdash{}{0pt}%
\pgfpathmoveto{\pgfqpoint{5.707802in}{1.834304in}}%
\pgfpathlineto{\pgfqpoint{5.716738in}{1.834304in}}%
\pgfpathlineto{\pgfqpoint{5.716738in}{1.858448in}}%
\pgfpathlineto{\pgfqpoint{5.707802in}{1.858448in}}%
\pgfpathlineto{\pgfqpoint{5.707802in}{1.834304in}}%
\pgfpathclose%
\pgfusepath{fill}%
\end{pgfscope}%
\begin{pgfscope}%
\pgfpathrectangle{\pgfqpoint{3.722897in}{0.857143in}}{\pgfqpoint{2.627103in}{1.813434in}}%
\pgfusepath{clip}%
\pgfsetbuttcap%
\pgfsetmiterjoin%
\definecolor{currentfill}{rgb}{0.133298,0.375282,0.379395}%
\pgfsetfillcolor{currentfill}%
\pgfsetlinewidth{0.000000pt}%
\definecolor{currentstroke}{rgb}{0.000000,0.000000,0.000000}%
\pgfsetstrokecolor{currentstroke}%
\pgfsetstrokeopacity{0.000000}%
\pgfsetdash{}{0pt}%
\pgfpathmoveto{\pgfqpoint{5.718972in}{1.835767in}}%
\pgfpathlineto{\pgfqpoint{5.727909in}{1.835767in}}%
\pgfpathlineto{\pgfqpoint{5.727909in}{1.843662in}}%
\pgfpathlineto{\pgfqpoint{5.718972in}{1.843662in}}%
\pgfpathlineto{\pgfqpoint{5.718972in}{1.835767in}}%
\pgfpathclose%
\pgfusepath{fill}%
\end{pgfscope}%
\begin{pgfscope}%
\pgfpathrectangle{\pgfqpoint{3.722897in}{0.857143in}}{\pgfqpoint{2.627103in}{1.813434in}}%
\pgfusepath{clip}%
\pgfsetbuttcap%
\pgfsetmiterjoin%
\definecolor{currentfill}{rgb}{0.133298,0.375282,0.379395}%
\pgfsetfillcolor{currentfill}%
\pgfsetlinewidth{0.000000pt}%
\definecolor{currentstroke}{rgb}{0.000000,0.000000,0.000000}%
\pgfsetstrokecolor{currentstroke}%
\pgfsetstrokeopacity{0.000000}%
\pgfsetdash{}{0pt}%
\pgfpathmoveto{\pgfqpoint{5.730143in}{1.836894in}}%
\pgfpathlineto{\pgfqpoint{5.739080in}{1.836894in}}%
\pgfpathlineto{\pgfqpoint{5.739080in}{1.858824in}}%
\pgfpathlineto{\pgfqpoint{5.730143in}{1.858824in}}%
\pgfpathlineto{\pgfqpoint{5.730143in}{1.836894in}}%
\pgfpathclose%
\pgfusepath{fill}%
\end{pgfscope}%
\begin{pgfscope}%
\pgfpathrectangle{\pgfqpoint{3.722897in}{0.857143in}}{\pgfqpoint{2.627103in}{1.813434in}}%
\pgfusepath{clip}%
\pgfsetbuttcap%
\pgfsetmiterjoin%
\definecolor{currentfill}{rgb}{0.133298,0.375282,0.379395}%
\pgfsetfillcolor{currentfill}%
\pgfsetlinewidth{0.000000pt}%
\definecolor{currentstroke}{rgb}{0.000000,0.000000,0.000000}%
\pgfsetstrokecolor{currentstroke}%
\pgfsetstrokeopacity{0.000000}%
\pgfsetdash{}{0pt}%
\pgfpathmoveto{\pgfqpoint{5.741314in}{1.840713in}}%
\pgfpathlineto{\pgfqpoint{5.750250in}{1.840713in}}%
\pgfpathlineto{\pgfqpoint{5.750250in}{1.872182in}}%
\pgfpathlineto{\pgfqpoint{5.741314in}{1.872182in}}%
\pgfpathlineto{\pgfqpoint{5.741314in}{1.840713in}}%
\pgfpathclose%
\pgfusepath{fill}%
\end{pgfscope}%
\begin{pgfscope}%
\pgfpathrectangle{\pgfqpoint{3.722897in}{0.857143in}}{\pgfqpoint{2.627103in}{1.813434in}}%
\pgfusepath{clip}%
\pgfsetbuttcap%
\pgfsetmiterjoin%
\definecolor{currentfill}{rgb}{0.133298,0.375282,0.379395}%
\pgfsetfillcolor{currentfill}%
\pgfsetlinewidth{0.000000pt}%
\definecolor{currentstroke}{rgb}{0.000000,0.000000,0.000000}%
\pgfsetstrokecolor{currentstroke}%
\pgfsetstrokeopacity{0.000000}%
\pgfsetdash{}{0pt}%
\pgfpathmoveto{\pgfqpoint{5.752484in}{1.846521in}}%
\pgfpathlineto{\pgfqpoint{5.761421in}{1.846521in}}%
\pgfpathlineto{\pgfqpoint{5.761421in}{1.876884in}}%
\pgfpathlineto{\pgfqpoint{5.752484in}{1.876884in}}%
\pgfpathlineto{\pgfqpoint{5.752484in}{1.846521in}}%
\pgfpathclose%
\pgfusepath{fill}%
\end{pgfscope}%
\begin{pgfscope}%
\pgfpathrectangle{\pgfqpoint{3.722897in}{0.857143in}}{\pgfqpoint{2.627103in}{1.813434in}}%
\pgfusepath{clip}%
\pgfsetbuttcap%
\pgfsetmiterjoin%
\definecolor{currentfill}{rgb}{0.133298,0.375282,0.379395}%
\pgfsetfillcolor{currentfill}%
\pgfsetlinewidth{0.000000pt}%
\definecolor{currentstroke}{rgb}{0.000000,0.000000,0.000000}%
\pgfsetstrokecolor{currentstroke}%
\pgfsetstrokeopacity{0.000000}%
\pgfsetdash{}{0pt}%
\pgfpathmoveto{\pgfqpoint{5.763655in}{1.851560in}}%
\pgfpathlineto{\pgfqpoint{5.772591in}{1.851560in}}%
\pgfpathlineto{\pgfqpoint{5.772591in}{1.864664in}}%
\pgfpathlineto{\pgfqpoint{5.763655in}{1.864664in}}%
\pgfpathlineto{\pgfqpoint{5.763655in}{1.851560in}}%
\pgfpathclose%
\pgfusepath{fill}%
\end{pgfscope}%
\begin{pgfscope}%
\pgfpathrectangle{\pgfqpoint{3.722897in}{0.857143in}}{\pgfqpoint{2.627103in}{1.813434in}}%
\pgfusepath{clip}%
\pgfsetbuttcap%
\pgfsetmiterjoin%
\definecolor{currentfill}{rgb}{0.133298,0.375282,0.379395}%
\pgfsetfillcolor{currentfill}%
\pgfsetlinewidth{0.000000pt}%
\definecolor{currentstroke}{rgb}{0.000000,0.000000,0.000000}%
\pgfsetstrokecolor{currentstroke}%
\pgfsetstrokeopacity{0.000000}%
\pgfsetdash{}{0pt}%
\pgfpathmoveto{\pgfqpoint{5.774826in}{1.856119in}}%
\pgfpathlineto{\pgfqpoint{5.783762in}{1.856119in}}%
\pgfpathlineto{\pgfqpoint{5.783762in}{1.866464in}}%
\pgfpathlineto{\pgfqpoint{5.774826in}{1.866464in}}%
\pgfpathlineto{\pgfqpoint{5.774826in}{1.856119in}}%
\pgfpathclose%
\pgfusepath{fill}%
\end{pgfscope}%
\begin{pgfscope}%
\pgfpathrectangle{\pgfqpoint{3.722897in}{0.857143in}}{\pgfqpoint{2.627103in}{1.813434in}}%
\pgfusepath{clip}%
\pgfsetbuttcap%
\pgfsetmiterjoin%
\definecolor{currentfill}{rgb}{0.133298,0.375282,0.379395}%
\pgfsetfillcolor{currentfill}%
\pgfsetlinewidth{0.000000pt}%
\definecolor{currentstroke}{rgb}{0.000000,0.000000,0.000000}%
\pgfsetstrokecolor{currentstroke}%
\pgfsetstrokeopacity{0.000000}%
\pgfsetdash{}{0pt}%
\pgfpathmoveto{\pgfqpoint{5.785996in}{1.858784in}}%
\pgfpathlineto{\pgfqpoint{5.794933in}{1.858784in}}%
\pgfpathlineto{\pgfqpoint{5.794933in}{1.872264in}}%
\pgfpathlineto{\pgfqpoint{5.785996in}{1.872264in}}%
\pgfpathlineto{\pgfqpoint{5.785996in}{1.858784in}}%
\pgfpathclose%
\pgfusepath{fill}%
\end{pgfscope}%
\begin{pgfscope}%
\pgfpathrectangle{\pgfqpoint{3.722897in}{0.857143in}}{\pgfqpoint{2.627103in}{1.813434in}}%
\pgfusepath{clip}%
\pgfsetbuttcap%
\pgfsetmiterjoin%
\definecolor{currentfill}{rgb}{0.133298,0.375282,0.379395}%
\pgfsetfillcolor{currentfill}%
\pgfsetlinewidth{0.000000pt}%
\definecolor{currentstroke}{rgb}{0.000000,0.000000,0.000000}%
\pgfsetstrokecolor{currentstroke}%
\pgfsetstrokeopacity{0.000000}%
\pgfsetdash{}{0pt}%
\pgfpathmoveto{\pgfqpoint{5.797167in}{1.861690in}}%
\pgfpathlineto{\pgfqpoint{5.806103in}{1.861690in}}%
\pgfpathlineto{\pgfqpoint{5.806103in}{1.863004in}}%
\pgfpathlineto{\pgfqpoint{5.797167in}{1.863004in}}%
\pgfpathlineto{\pgfqpoint{5.797167in}{1.861690in}}%
\pgfpathclose%
\pgfusepath{fill}%
\end{pgfscope}%
\begin{pgfscope}%
\pgfpathrectangle{\pgfqpoint{3.722897in}{0.857143in}}{\pgfqpoint{2.627103in}{1.813434in}}%
\pgfusepath{clip}%
\pgfsetbuttcap%
\pgfsetmiterjoin%
\definecolor{currentfill}{rgb}{0.133298,0.375282,0.379395}%
\pgfsetfillcolor{currentfill}%
\pgfsetlinewidth{0.000000pt}%
\definecolor{currentstroke}{rgb}{0.000000,0.000000,0.000000}%
\pgfsetstrokecolor{currentstroke}%
\pgfsetstrokeopacity{0.000000}%
\pgfsetdash{}{0pt}%
\pgfpathmoveto{\pgfqpoint{5.808337in}{1.864576in}}%
\pgfpathlineto{\pgfqpoint{5.817274in}{1.864576in}}%
\pgfpathlineto{\pgfqpoint{5.817274in}{1.871294in}}%
\pgfpathlineto{\pgfqpoint{5.808337in}{1.871294in}}%
\pgfpathlineto{\pgfqpoint{5.808337in}{1.864576in}}%
\pgfpathclose%
\pgfusepath{fill}%
\end{pgfscope}%
\begin{pgfscope}%
\pgfpathrectangle{\pgfqpoint{3.722897in}{0.857143in}}{\pgfqpoint{2.627103in}{1.813434in}}%
\pgfusepath{clip}%
\pgfsetbuttcap%
\pgfsetmiterjoin%
\definecolor{currentfill}{rgb}{0.133298,0.375282,0.379395}%
\pgfsetfillcolor{currentfill}%
\pgfsetlinewidth{0.000000pt}%
\definecolor{currentstroke}{rgb}{0.000000,0.000000,0.000000}%
\pgfsetstrokecolor{currentstroke}%
\pgfsetstrokeopacity{0.000000}%
\pgfsetdash{}{0pt}%
\pgfpathmoveto{\pgfqpoint{5.819508in}{1.866662in}}%
\pgfpathlineto{\pgfqpoint{5.828444in}{1.866662in}}%
\pgfpathlineto{\pgfqpoint{5.828444in}{1.867631in}}%
\pgfpathlineto{\pgfqpoint{5.819508in}{1.867631in}}%
\pgfpathlineto{\pgfqpoint{5.819508in}{1.866662in}}%
\pgfpathclose%
\pgfusepath{fill}%
\end{pgfscope}%
\begin{pgfscope}%
\pgfpathrectangle{\pgfqpoint{3.722897in}{0.857143in}}{\pgfqpoint{2.627103in}{1.813434in}}%
\pgfusepath{clip}%
\pgfsetbuttcap%
\pgfsetmiterjoin%
\definecolor{currentfill}{rgb}{0.133298,0.375282,0.379395}%
\pgfsetfillcolor{currentfill}%
\pgfsetlinewidth{0.000000pt}%
\definecolor{currentstroke}{rgb}{0.000000,0.000000,0.000000}%
\pgfsetstrokecolor{currentstroke}%
\pgfsetstrokeopacity{0.000000}%
\pgfsetdash{}{0pt}%
\pgfpathmoveto{\pgfqpoint{5.830679in}{1.867454in}}%
\pgfpathlineto{\pgfqpoint{5.839615in}{1.867454in}}%
\pgfpathlineto{\pgfqpoint{5.839615in}{1.872119in}}%
\pgfpathlineto{\pgfqpoint{5.830679in}{1.872119in}}%
\pgfpathlineto{\pgfqpoint{5.830679in}{1.867454in}}%
\pgfpathclose%
\pgfusepath{fill}%
\end{pgfscope}%
\begin{pgfscope}%
\pgfpathrectangle{\pgfqpoint{3.722897in}{0.857143in}}{\pgfqpoint{2.627103in}{1.813434in}}%
\pgfusepath{clip}%
\pgfsetbuttcap%
\pgfsetmiterjoin%
\definecolor{currentfill}{rgb}{0.133298,0.375282,0.379395}%
\pgfsetfillcolor{currentfill}%
\pgfsetlinewidth{0.000000pt}%
\definecolor{currentstroke}{rgb}{0.000000,0.000000,0.000000}%
\pgfsetstrokecolor{currentstroke}%
\pgfsetstrokeopacity{0.000000}%
\pgfsetdash{}{0pt}%
\pgfpathmoveto{\pgfqpoint{5.841849in}{1.813884in}}%
\pgfpathlineto{\pgfqpoint{5.850786in}{1.813884in}}%
\pgfpathlineto{\pgfqpoint{5.850786in}{1.803911in}}%
\pgfpathlineto{\pgfqpoint{5.841849in}{1.803911in}}%
\pgfpathlineto{\pgfqpoint{5.841849in}{1.813884in}}%
\pgfpathclose%
\pgfusepath{fill}%
\end{pgfscope}%
\begin{pgfscope}%
\pgfpathrectangle{\pgfqpoint{3.722897in}{0.857143in}}{\pgfqpoint{2.627103in}{1.813434in}}%
\pgfusepath{clip}%
\pgfsetbuttcap%
\pgfsetmiterjoin%
\definecolor{currentfill}{rgb}{0.133298,0.375282,0.379395}%
\pgfsetfillcolor{currentfill}%
\pgfsetlinewidth{0.000000pt}%
\definecolor{currentstroke}{rgb}{0.000000,0.000000,0.000000}%
\pgfsetstrokecolor{currentstroke}%
\pgfsetstrokeopacity{0.000000}%
\pgfsetdash{}{0pt}%
\pgfpathmoveto{\pgfqpoint{5.853020in}{1.813884in}}%
\pgfpathlineto{\pgfqpoint{5.861956in}{1.813884in}}%
\pgfpathlineto{\pgfqpoint{5.861956in}{1.805759in}}%
\pgfpathlineto{\pgfqpoint{5.853020in}{1.805759in}}%
\pgfpathlineto{\pgfqpoint{5.853020in}{1.813884in}}%
\pgfpathclose%
\pgfusepath{fill}%
\end{pgfscope}%
\begin{pgfscope}%
\pgfpathrectangle{\pgfqpoint{3.722897in}{0.857143in}}{\pgfqpoint{2.627103in}{1.813434in}}%
\pgfusepath{clip}%
\pgfsetbuttcap%
\pgfsetmiterjoin%
\definecolor{currentfill}{rgb}{0.133298,0.375282,0.379395}%
\pgfsetfillcolor{currentfill}%
\pgfsetlinewidth{0.000000pt}%
\definecolor{currentstroke}{rgb}{0.000000,0.000000,0.000000}%
\pgfsetstrokecolor{currentstroke}%
\pgfsetstrokeopacity{0.000000}%
\pgfsetdash{}{0pt}%
\pgfpathmoveto{\pgfqpoint{5.864190in}{1.868531in}}%
\pgfpathlineto{\pgfqpoint{5.873127in}{1.868531in}}%
\pgfpathlineto{\pgfqpoint{5.873127in}{1.877173in}}%
\pgfpathlineto{\pgfqpoint{5.864190in}{1.877173in}}%
\pgfpathlineto{\pgfqpoint{5.864190in}{1.868531in}}%
\pgfpathclose%
\pgfusepath{fill}%
\end{pgfscope}%
\begin{pgfscope}%
\pgfpathrectangle{\pgfqpoint{3.722897in}{0.857143in}}{\pgfqpoint{2.627103in}{1.813434in}}%
\pgfusepath{clip}%
\pgfsetbuttcap%
\pgfsetmiterjoin%
\definecolor{currentfill}{rgb}{0.133298,0.375282,0.379395}%
\pgfsetfillcolor{currentfill}%
\pgfsetlinewidth{0.000000pt}%
\definecolor{currentstroke}{rgb}{0.000000,0.000000,0.000000}%
\pgfsetstrokecolor{currentstroke}%
\pgfsetstrokeopacity{0.000000}%
\pgfsetdash{}{0pt}%
\pgfpathmoveto{\pgfqpoint{5.875361in}{1.813884in}}%
\pgfpathlineto{\pgfqpoint{5.884297in}{1.813884in}}%
\pgfpathlineto{\pgfqpoint{5.884297in}{1.797163in}}%
\pgfpathlineto{\pgfqpoint{5.875361in}{1.797163in}}%
\pgfpathlineto{\pgfqpoint{5.875361in}{1.813884in}}%
\pgfpathclose%
\pgfusepath{fill}%
\end{pgfscope}%
\begin{pgfscope}%
\pgfpathrectangle{\pgfqpoint{3.722897in}{0.857143in}}{\pgfqpoint{2.627103in}{1.813434in}}%
\pgfusepath{clip}%
\pgfsetbuttcap%
\pgfsetmiterjoin%
\definecolor{currentfill}{rgb}{0.133298,0.375282,0.379395}%
\pgfsetfillcolor{currentfill}%
\pgfsetlinewidth{0.000000pt}%
\definecolor{currentstroke}{rgb}{0.000000,0.000000,0.000000}%
\pgfsetstrokecolor{currentstroke}%
\pgfsetstrokeopacity{0.000000}%
\pgfsetdash{}{0pt}%
\pgfpathmoveto{\pgfqpoint{5.886532in}{1.813884in}}%
\pgfpathlineto{\pgfqpoint{5.895468in}{1.813884in}}%
\pgfpathlineto{\pgfqpoint{5.895468in}{1.805619in}}%
\pgfpathlineto{\pgfqpoint{5.886532in}{1.805619in}}%
\pgfpathlineto{\pgfqpoint{5.886532in}{1.813884in}}%
\pgfpathclose%
\pgfusepath{fill}%
\end{pgfscope}%
\begin{pgfscope}%
\pgfpathrectangle{\pgfqpoint{3.722897in}{0.857143in}}{\pgfqpoint{2.627103in}{1.813434in}}%
\pgfusepath{clip}%
\pgfsetbuttcap%
\pgfsetmiterjoin%
\definecolor{currentfill}{rgb}{0.133298,0.375282,0.379395}%
\pgfsetfillcolor{currentfill}%
\pgfsetlinewidth{0.000000pt}%
\definecolor{currentstroke}{rgb}{0.000000,0.000000,0.000000}%
\pgfsetstrokecolor{currentstroke}%
\pgfsetstrokeopacity{0.000000}%
\pgfsetdash{}{0pt}%
\pgfpathmoveto{\pgfqpoint{5.897702in}{1.813884in}}%
\pgfpathlineto{\pgfqpoint{5.906639in}{1.813884in}}%
\pgfpathlineto{\pgfqpoint{5.906639in}{1.797542in}}%
\pgfpathlineto{\pgfqpoint{5.897702in}{1.797542in}}%
\pgfpathlineto{\pgfqpoint{5.897702in}{1.813884in}}%
\pgfpathclose%
\pgfusepath{fill}%
\end{pgfscope}%
\begin{pgfscope}%
\pgfpathrectangle{\pgfqpoint{3.722897in}{0.857143in}}{\pgfqpoint{2.627103in}{1.813434in}}%
\pgfusepath{clip}%
\pgfsetbuttcap%
\pgfsetmiterjoin%
\definecolor{currentfill}{rgb}{0.133298,0.375282,0.379395}%
\pgfsetfillcolor{currentfill}%
\pgfsetlinewidth{0.000000pt}%
\definecolor{currentstroke}{rgb}{0.000000,0.000000,0.000000}%
\pgfsetstrokecolor{currentstroke}%
\pgfsetstrokeopacity{0.000000}%
\pgfsetdash{}{0pt}%
\pgfpathmoveto{\pgfqpoint{5.908873in}{1.813884in}}%
\pgfpathlineto{\pgfqpoint{5.917809in}{1.813884in}}%
\pgfpathlineto{\pgfqpoint{5.917809in}{1.796087in}}%
\pgfpathlineto{\pgfqpoint{5.908873in}{1.796087in}}%
\pgfpathlineto{\pgfqpoint{5.908873in}{1.813884in}}%
\pgfpathclose%
\pgfusepath{fill}%
\end{pgfscope}%
\begin{pgfscope}%
\pgfpathrectangle{\pgfqpoint{3.722897in}{0.857143in}}{\pgfqpoint{2.627103in}{1.813434in}}%
\pgfusepath{clip}%
\pgfsetbuttcap%
\pgfsetmiterjoin%
\definecolor{currentfill}{rgb}{0.133298,0.375282,0.379395}%
\pgfsetfillcolor{currentfill}%
\pgfsetlinewidth{0.000000pt}%
\definecolor{currentstroke}{rgb}{0.000000,0.000000,0.000000}%
\pgfsetstrokecolor{currentstroke}%
\pgfsetstrokeopacity{0.000000}%
\pgfsetdash{}{0pt}%
\pgfpathmoveto{\pgfqpoint{5.920043in}{1.813884in}}%
\pgfpathlineto{\pgfqpoint{5.928980in}{1.813884in}}%
\pgfpathlineto{\pgfqpoint{5.928980in}{1.806587in}}%
\pgfpathlineto{\pgfqpoint{5.920043in}{1.806587in}}%
\pgfpathlineto{\pgfqpoint{5.920043in}{1.813884in}}%
\pgfpathclose%
\pgfusepath{fill}%
\end{pgfscope}%
\begin{pgfscope}%
\pgfpathrectangle{\pgfqpoint{3.722897in}{0.857143in}}{\pgfqpoint{2.627103in}{1.813434in}}%
\pgfusepath{clip}%
\pgfsetbuttcap%
\pgfsetmiterjoin%
\definecolor{currentfill}{rgb}{0.133298,0.375282,0.379395}%
\pgfsetfillcolor{currentfill}%
\pgfsetlinewidth{0.000000pt}%
\definecolor{currentstroke}{rgb}{0.000000,0.000000,0.000000}%
\pgfsetstrokecolor{currentstroke}%
\pgfsetstrokeopacity{0.000000}%
\pgfsetdash{}{0pt}%
\pgfpathmoveto{\pgfqpoint{5.931214in}{1.862982in}}%
\pgfpathlineto{\pgfqpoint{5.940150in}{1.862982in}}%
\pgfpathlineto{\pgfqpoint{5.940150in}{1.863637in}}%
\pgfpathlineto{\pgfqpoint{5.931214in}{1.863637in}}%
\pgfpathlineto{\pgfqpoint{5.931214in}{1.862982in}}%
\pgfpathclose%
\pgfusepath{fill}%
\end{pgfscope}%
\begin{pgfscope}%
\pgfpathrectangle{\pgfqpoint{3.722897in}{0.857143in}}{\pgfqpoint{2.627103in}{1.813434in}}%
\pgfusepath{clip}%
\pgfsetbuttcap%
\pgfsetmiterjoin%
\definecolor{currentfill}{rgb}{0.133298,0.375282,0.379395}%
\pgfsetfillcolor{currentfill}%
\pgfsetlinewidth{0.000000pt}%
\definecolor{currentstroke}{rgb}{0.000000,0.000000,0.000000}%
\pgfsetstrokecolor{currentstroke}%
\pgfsetstrokeopacity{0.000000}%
\pgfsetdash{}{0pt}%
\pgfpathmoveto{\pgfqpoint{5.942385in}{1.813884in}}%
\pgfpathlineto{\pgfqpoint{5.951321in}{1.813884in}}%
\pgfpathlineto{\pgfqpoint{5.951321in}{1.801970in}}%
\pgfpathlineto{\pgfqpoint{5.942385in}{1.801970in}}%
\pgfpathlineto{\pgfqpoint{5.942385in}{1.813884in}}%
\pgfpathclose%
\pgfusepath{fill}%
\end{pgfscope}%
\begin{pgfscope}%
\pgfpathrectangle{\pgfqpoint{3.722897in}{0.857143in}}{\pgfqpoint{2.627103in}{1.813434in}}%
\pgfusepath{clip}%
\pgfsetbuttcap%
\pgfsetmiterjoin%
\definecolor{currentfill}{rgb}{0.133298,0.375282,0.379395}%
\pgfsetfillcolor{currentfill}%
\pgfsetlinewidth{0.000000pt}%
\definecolor{currentstroke}{rgb}{0.000000,0.000000,0.000000}%
\pgfsetstrokecolor{currentstroke}%
\pgfsetstrokeopacity{0.000000}%
\pgfsetdash{}{0pt}%
\pgfpathmoveto{\pgfqpoint{5.953555in}{1.813884in}}%
\pgfpathlineto{\pgfqpoint{5.962492in}{1.813884in}}%
\pgfpathlineto{\pgfqpoint{5.962492in}{1.796585in}}%
\pgfpathlineto{\pgfqpoint{5.953555in}{1.796585in}}%
\pgfpathlineto{\pgfqpoint{5.953555in}{1.813884in}}%
\pgfpathclose%
\pgfusepath{fill}%
\end{pgfscope}%
\begin{pgfscope}%
\pgfpathrectangle{\pgfqpoint{3.722897in}{0.857143in}}{\pgfqpoint{2.627103in}{1.813434in}}%
\pgfusepath{clip}%
\pgfsetbuttcap%
\pgfsetmiterjoin%
\definecolor{currentfill}{rgb}{0.133298,0.375282,0.379395}%
\pgfsetfillcolor{currentfill}%
\pgfsetlinewidth{0.000000pt}%
\definecolor{currentstroke}{rgb}{0.000000,0.000000,0.000000}%
\pgfsetstrokecolor{currentstroke}%
\pgfsetstrokeopacity{0.000000}%
\pgfsetdash{}{0pt}%
\pgfpathmoveto{\pgfqpoint{5.964726in}{1.813884in}}%
\pgfpathlineto{\pgfqpoint{5.973662in}{1.813884in}}%
\pgfpathlineto{\pgfqpoint{5.973662in}{1.807739in}}%
\pgfpathlineto{\pgfqpoint{5.964726in}{1.807739in}}%
\pgfpathlineto{\pgfqpoint{5.964726in}{1.813884in}}%
\pgfpathclose%
\pgfusepath{fill}%
\end{pgfscope}%
\begin{pgfscope}%
\pgfpathrectangle{\pgfqpoint{3.722897in}{0.857143in}}{\pgfqpoint{2.627103in}{1.813434in}}%
\pgfusepath{clip}%
\pgfsetbuttcap%
\pgfsetmiterjoin%
\definecolor{currentfill}{rgb}{0.133298,0.375282,0.379395}%
\pgfsetfillcolor{currentfill}%
\pgfsetlinewidth{0.000000pt}%
\definecolor{currentstroke}{rgb}{0.000000,0.000000,0.000000}%
\pgfsetstrokecolor{currentstroke}%
\pgfsetstrokeopacity{0.000000}%
\pgfsetdash{}{0pt}%
\pgfpathmoveto{\pgfqpoint{5.975896in}{1.813884in}}%
\pgfpathlineto{\pgfqpoint{5.984833in}{1.813884in}}%
\pgfpathlineto{\pgfqpoint{5.984833in}{1.801832in}}%
\pgfpathlineto{\pgfqpoint{5.975896in}{1.801832in}}%
\pgfpathlineto{\pgfqpoint{5.975896in}{1.813884in}}%
\pgfpathclose%
\pgfusepath{fill}%
\end{pgfscope}%
\begin{pgfscope}%
\pgfpathrectangle{\pgfqpoint{3.722897in}{0.857143in}}{\pgfqpoint{2.627103in}{1.813434in}}%
\pgfusepath{clip}%
\pgfsetbuttcap%
\pgfsetmiterjoin%
\definecolor{currentfill}{rgb}{0.133298,0.375282,0.379395}%
\pgfsetfillcolor{currentfill}%
\pgfsetlinewidth{0.000000pt}%
\definecolor{currentstroke}{rgb}{0.000000,0.000000,0.000000}%
\pgfsetstrokecolor{currentstroke}%
\pgfsetstrokeopacity{0.000000}%
\pgfsetdash{}{0pt}%
\pgfpathmoveto{\pgfqpoint{5.987067in}{1.813884in}}%
\pgfpathlineto{\pgfqpoint{5.996004in}{1.813884in}}%
\pgfpathlineto{\pgfqpoint{5.996004in}{1.804150in}}%
\pgfpathlineto{\pgfqpoint{5.987067in}{1.804150in}}%
\pgfpathlineto{\pgfqpoint{5.987067in}{1.813884in}}%
\pgfpathclose%
\pgfusepath{fill}%
\end{pgfscope}%
\begin{pgfscope}%
\pgfpathrectangle{\pgfqpoint{3.722897in}{0.857143in}}{\pgfqpoint{2.627103in}{1.813434in}}%
\pgfusepath{clip}%
\pgfsetbuttcap%
\pgfsetmiterjoin%
\definecolor{currentfill}{rgb}{0.133298,0.375282,0.379395}%
\pgfsetfillcolor{currentfill}%
\pgfsetlinewidth{0.000000pt}%
\definecolor{currentstroke}{rgb}{0.000000,0.000000,0.000000}%
\pgfsetstrokecolor{currentstroke}%
\pgfsetstrokeopacity{0.000000}%
\pgfsetdash{}{0pt}%
\pgfpathmoveto{\pgfqpoint{5.998238in}{1.851905in}}%
\pgfpathlineto{\pgfqpoint{6.007174in}{1.851905in}}%
\pgfpathlineto{\pgfqpoint{6.007174in}{1.855039in}}%
\pgfpathlineto{\pgfqpoint{5.998238in}{1.855039in}}%
\pgfpathlineto{\pgfqpoint{5.998238in}{1.851905in}}%
\pgfpathclose%
\pgfusepath{fill}%
\end{pgfscope}%
\begin{pgfscope}%
\pgfpathrectangle{\pgfqpoint{3.722897in}{0.857143in}}{\pgfqpoint{2.627103in}{1.813434in}}%
\pgfusepath{clip}%
\pgfsetbuttcap%
\pgfsetmiterjoin%
\definecolor{currentfill}{rgb}{0.133298,0.375282,0.379395}%
\pgfsetfillcolor{currentfill}%
\pgfsetlinewidth{0.000000pt}%
\definecolor{currentstroke}{rgb}{0.000000,0.000000,0.000000}%
\pgfsetstrokecolor{currentstroke}%
\pgfsetstrokeopacity{0.000000}%
\pgfsetdash{}{0pt}%
\pgfpathmoveto{\pgfqpoint{6.009408in}{1.850198in}}%
\pgfpathlineto{\pgfqpoint{6.018345in}{1.850198in}}%
\pgfpathlineto{\pgfqpoint{6.018345in}{1.861627in}}%
\pgfpathlineto{\pgfqpoint{6.009408in}{1.861627in}}%
\pgfpathlineto{\pgfqpoint{6.009408in}{1.850198in}}%
\pgfpathclose%
\pgfusepath{fill}%
\end{pgfscope}%
\begin{pgfscope}%
\pgfpathrectangle{\pgfqpoint{3.722897in}{0.857143in}}{\pgfqpoint{2.627103in}{1.813434in}}%
\pgfusepath{clip}%
\pgfsetbuttcap%
\pgfsetmiterjoin%
\definecolor{currentfill}{rgb}{0.133298,0.375282,0.379395}%
\pgfsetfillcolor{currentfill}%
\pgfsetlinewidth{0.000000pt}%
\definecolor{currentstroke}{rgb}{0.000000,0.000000,0.000000}%
\pgfsetstrokecolor{currentstroke}%
\pgfsetstrokeopacity{0.000000}%
\pgfsetdash{}{0pt}%
\pgfpathmoveto{\pgfqpoint{6.020579in}{1.813884in}}%
\pgfpathlineto{\pgfqpoint{6.029515in}{1.813884in}}%
\pgfpathlineto{\pgfqpoint{6.029515in}{1.801223in}}%
\pgfpathlineto{\pgfqpoint{6.020579in}{1.801223in}}%
\pgfpathlineto{\pgfqpoint{6.020579in}{1.813884in}}%
\pgfpathclose%
\pgfusepath{fill}%
\end{pgfscope}%
\begin{pgfscope}%
\pgfpathrectangle{\pgfqpoint{3.722897in}{0.857143in}}{\pgfqpoint{2.627103in}{1.813434in}}%
\pgfusepath{clip}%
\pgfsetbuttcap%
\pgfsetmiterjoin%
\definecolor{currentfill}{rgb}{0.133298,0.375282,0.379395}%
\pgfsetfillcolor{currentfill}%
\pgfsetlinewidth{0.000000pt}%
\definecolor{currentstroke}{rgb}{0.000000,0.000000,0.000000}%
\pgfsetstrokecolor{currentstroke}%
\pgfsetstrokeopacity{0.000000}%
\pgfsetdash{}{0pt}%
\pgfpathmoveto{\pgfqpoint{6.031749in}{1.813884in}}%
\pgfpathlineto{\pgfqpoint{6.040686in}{1.813884in}}%
\pgfpathlineto{\pgfqpoint{6.040686in}{1.811915in}}%
\pgfpathlineto{\pgfqpoint{6.031749in}{1.811915in}}%
\pgfpathlineto{\pgfqpoint{6.031749in}{1.813884in}}%
\pgfpathclose%
\pgfusepath{fill}%
\end{pgfscope}%
\begin{pgfscope}%
\pgfpathrectangle{\pgfqpoint{3.722897in}{0.857143in}}{\pgfqpoint{2.627103in}{1.813434in}}%
\pgfusepath{clip}%
\pgfsetbuttcap%
\pgfsetmiterjoin%
\definecolor{currentfill}{rgb}{0.133298,0.375282,0.379395}%
\pgfsetfillcolor{currentfill}%
\pgfsetlinewidth{0.000000pt}%
\definecolor{currentstroke}{rgb}{0.000000,0.000000,0.000000}%
\pgfsetstrokecolor{currentstroke}%
\pgfsetstrokeopacity{0.000000}%
\pgfsetdash{}{0pt}%
\pgfpathmoveto{\pgfqpoint{6.042920in}{1.842505in}}%
\pgfpathlineto{\pgfqpoint{6.051857in}{1.842505in}}%
\pgfpathlineto{\pgfqpoint{6.051857in}{1.853596in}}%
\pgfpathlineto{\pgfqpoint{6.042920in}{1.853596in}}%
\pgfpathlineto{\pgfqpoint{6.042920in}{1.842505in}}%
\pgfpathclose%
\pgfusepath{fill}%
\end{pgfscope}%
\begin{pgfscope}%
\pgfpathrectangle{\pgfqpoint{3.722897in}{0.857143in}}{\pgfqpoint{2.627103in}{1.813434in}}%
\pgfusepath{clip}%
\pgfsetbuttcap%
\pgfsetmiterjoin%
\definecolor{currentfill}{rgb}{0.133298,0.375282,0.379395}%
\pgfsetfillcolor{currentfill}%
\pgfsetlinewidth{0.000000pt}%
\definecolor{currentstroke}{rgb}{0.000000,0.000000,0.000000}%
\pgfsetstrokecolor{currentstroke}%
\pgfsetstrokeopacity{0.000000}%
\pgfsetdash{}{0pt}%
\pgfpathmoveto{\pgfqpoint{6.054091in}{1.840337in}}%
\pgfpathlineto{\pgfqpoint{6.063027in}{1.840337in}}%
\pgfpathlineto{\pgfqpoint{6.063027in}{1.856540in}}%
\pgfpathlineto{\pgfqpoint{6.054091in}{1.856540in}}%
\pgfpathlineto{\pgfqpoint{6.054091in}{1.840337in}}%
\pgfpathclose%
\pgfusepath{fill}%
\end{pgfscope}%
\begin{pgfscope}%
\pgfpathrectangle{\pgfqpoint{3.722897in}{0.857143in}}{\pgfqpoint{2.627103in}{1.813434in}}%
\pgfusepath{clip}%
\pgfsetbuttcap%
\pgfsetmiterjoin%
\definecolor{currentfill}{rgb}{0.133298,0.375282,0.379395}%
\pgfsetfillcolor{currentfill}%
\pgfsetlinewidth{0.000000pt}%
\definecolor{currentstroke}{rgb}{0.000000,0.000000,0.000000}%
\pgfsetstrokecolor{currentstroke}%
\pgfsetstrokeopacity{0.000000}%
\pgfsetdash{}{0pt}%
\pgfpathmoveto{\pgfqpoint{6.065261in}{1.813884in}}%
\pgfpathlineto{\pgfqpoint{6.074198in}{1.813884in}}%
\pgfpathlineto{\pgfqpoint{6.074198in}{1.802914in}}%
\pgfpathlineto{\pgfqpoint{6.065261in}{1.802914in}}%
\pgfpathlineto{\pgfqpoint{6.065261in}{1.813884in}}%
\pgfpathclose%
\pgfusepath{fill}%
\end{pgfscope}%
\begin{pgfscope}%
\pgfpathrectangle{\pgfqpoint{3.722897in}{0.857143in}}{\pgfqpoint{2.627103in}{1.813434in}}%
\pgfusepath{clip}%
\pgfsetbuttcap%
\pgfsetmiterjoin%
\definecolor{currentfill}{rgb}{0.133298,0.375282,0.379395}%
\pgfsetfillcolor{currentfill}%
\pgfsetlinewidth{0.000000pt}%
\definecolor{currentstroke}{rgb}{0.000000,0.000000,0.000000}%
\pgfsetstrokecolor{currentstroke}%
\pgfsetstrokeopacity{0.000000}%
\pgfsetdash{}{0pt}%
\pgfpathmoveto{\pgfqpoint{6.076432in}{1.834950in}}%
\pgfpathlineto{\pgfqpoint{6.085368in}{1.834950in}}%
\pgfpathlineto{\pgfqpoint{6.085368in}{1.836625in}}%
\pgfpathlineto{\pgfqpoint{6.076432in}{1.836625in}}%
\pgfpathlineto{\pgfqpoint{6.076432in}{1.834950in}}%
\pgfpathclose%
\pgfusepath{fill}%
\end{pgfscope}%
\begin{pgfscope}%
\pgfpathrectangle{\pgfqpoint{3.722897in}{0.857143in}}{\pgfqpoint{2.627103in}{1.813434in}}%
\pgfusepath{clip}%
\pgfsetbuttcap%
\pgfsetmiterjoin%
\definecolor{currentfill}{rgb}{0.133298,0.375282,0.379395}%
\pgfsetfillcolor{currentfill}%
\pgfsetlinewidth{0.000000pt}%
\definecolor{currentstroke}{rgb}{0.000000,0.000000,0.000000}%
\pgfsetstrokecolor{currentstroke}%
\pgfsetstrokeopacity{0.000000}%
\pgfsetdash{}{0pt}%
\pgfpathmoveto{\pgfqpoint{6.087602in}{1.813884in}}%
\pgfpathlineto{\pgfqpoint{6.096539in}{1.813884in}}%
\pgfpathlineto{\pgfqpoint{6.096539in}{1.803025in}}%
\pgfpathlineto{\pgfqpoint{6.087602in}{1.803025in}}%
\pgfpathlineto{\pgfqpoint{6.087602in}{1.813884in}}%
\pgfpathclose%
\pgfusepath{fill}%
\end{pgfscope}%
\begin{pgfscope}%
\pgfpathrectangle{\pgfqpoint{3.722897in}{0.857143in}}{\pgfqpoint{2.627103in}{1.813434in}}%
\pgfusepath{clip}%
\pgfsetbuttcap%
\pgfsetmiterjoin%
\definecolor{currentfill}{rgb}{0.133298,0.375282,0.379395}%
\pgfsetfillcolor{currentfill}%
\pgfsetlinewidth{0.000000pt}%
\definecolor{currentstroke}{rgb}{0.000000,0.000000,0.000000}%
\pgfsetstrokecolor{currentstroke}%
\pgfsetstrokeopacity{0.000000}%
\pgfsetdash{}{0pt}%
\pgfpathmoveto{\pgfqpoint{6.098773in}{1.813884in}}%
\pgfpathlineto{\pgfqpoint{6.107710in}{1.813884in}}%
\pgfpathlineto{\pgfqpoint{6.107710in}{1.800455in}}%
\pgfpathlineto{\pgfqpoint{6.098773in}{1.800455in}}%
\pgfpathlineto{\pgfqpoint{6.098773in}{1.813884in}}%
\pgfpathclose%
\pgfusepath{fill}%
\end{pgfscope}%
\begin{pgfscope}%
\pgfpathrectangle{\pgfqpoint{3.722897in}{0.857143in}}{\pgfqpoint{2.627103in}{1.813434in}}%
\pgfusepath{clip}%
\pgfsetbuttcap%
\pgfsetmiterjoin%
\definecolor{currentfill}{rgb}{0.133298,0.375282,0.379395}%
\pgfsetfillcolor{currentfill}%
\pgfsetlinewidth{0.000000pt}%
\definecolor{currentstroke}{rgb}{0.000000,0.000000,0.000000}%
\pgfsetstrokecolor{currentstroke}%
\pgfsetstrokeopacity{0.000000}%
\pgfsetdash{}{0pt}%
\pgfpathmoveto{\pgfqpoint{6.109944in}{1.813884in}}%
\pgfpathlineto{\pgfqpoint{6.118880in}{1.813884in}}%
\pgfpathlineto{\pgfqpoint{6.118880in}{1.810216in}}%
\pgfpathlineto{\pgfqpoint{6.109944in}{1.810216in}}%
\pgfpathlineto{\pgfqpoint{6.109944in}{1.813884in}}%
\pgfpathclose%
\pgfusepath{fill}%
\end{pgfscope}%
\begin{pgfscope}%
\pgfpathrectangle{\pgfqpoint{3.722897in}{0.857143in}}{\pgfqpoint{2.627103in}{1.813434in}}%
\pgfusepath{clip}%
\pgfsetbuttcap%
\pgfsetmiterjoin%
\definecolor{currentfill}{rgb}{0.133298,0.375282,0.379395}%
\pgfsetfillcolor{currentfill}%
\pgfsetlinewidth{0.000000pt}%
\definecolor{currentstroke}{rgb}{0.000000,0.000000,0.000000}%
\pgfsetstrokecolor{currentstroke}%
\pgfsetstrokeopacity{0.000000}%
\pgfsetdash{}{0pt}%
\pgfpathmoveto{\pgfqpoint{6.121114in}{1.813884in}}%
\pgfpathlineto{\pgfqpoint{6.130051in}{1.813884in}}%
\pgfpathlineto{\pgfqpoint{6.130051in}{1.796321in}}%
\pgfpathlineto{\pgfqpoint{6.121114in}{1.796321in}}%
\pgfpathlineto{\pgfqpoint{6.121114in}{1.813884in}}%
\pgfpathclose%
\pgfusepath{fill}%
\end{pgfscope}%
\begin{pgfscope}%
\pgfpathrectangle{\pgfqpoint{3.722897in}{0.857143in}}{\pgfqpoint{2.627103in}{1.813434in}}%
\pgfusepath{clip}%
\pgfsetbuttcap%
\pgfsetmiterjoin%
\definecolor{currentfill}{rgb}{0.133298,0.375282,0.379395}%
\pgfsetfillcolor{currentfill}%
\pgfsetlinewidth{0.000000pt}%
\definecolor{currentstroke}{rgb}{0.000000,0.000000,0.000000}%
\pgfsetstrokecolor{currentstroke}%
\pgfsetstrokeopacity{0.000000}%
\pgfsetdash{}{0pt}%
\pgfpathmoveto{\pgfqpoint{6.132285in}{1.813884in}}%
\pgfpathlineto{\pgfqpoint{6.141221in}{1.813884in}}%
\pgfpathlineto{\pgfqpoint{6.141221in}{1.786955in}}%
\pgfpathlineto{\pgfqpoint{6.132285in}{1.786955in}}%
\pgfpathlineto{\pgfqpoint{6.132285in}{1.813884in}}%
\pgfpathclose%
\pgfusepath{fill}%
\end{pgfscope}%
\begin{pgfscope}%
\pgfpathrectangle{\pgfqpoint{3.722897in}{0.857143in}}{\pgfqpoint{2.627103in}{1.813434in}}%
\pgfusepath{clip}%
\pgfsetbuttcap%
\pgfsetmiterjoin%
\definecolor{currentfill}{rgb}{0.133298,0.375282,0.379395}%
\pgfsetfillcolor{currentfill}%
\pgfsetlinewidth{0.000000pt}%
\definecolor{currentstroke}{rgb}{0.000000,0.000000,0.000000}%
\pgfsetstrokecolor{currentstroke}%
\pgfsetstrokeopacity{0.000000}%
\pgfsetdash{}{0pt}%
\pgfpathmoveto{\pgfqpoint{6.143456in}{1.813884in}}%
\pgfpathlineto{\pgfqpoint{6.152392in}{1.813884in}}%
\pgfpathlineto{\pgfqpoint{6.152392in}{1.789051in}}%
\pgfpathlineto{\pgfqpoint{6.143456in}{1.789051in}}%
\pgfpathlineto{\pgfqpoint{6.143456in}{1.813884in}}%
\pgfpathclose%
\pgfusepath{fill}%
\end{pgfscope}%
\begin{pgfscope}%
\pgfpathrectangle{\pgfqpoint{3.722897in}{0.857143in}}{\pgfqpoint{2.627103in}{1.813434in}}%
\pgfusepath{clip}%
\pgfsetbuttcap%
\pgfsetmiterjoin%
\definecolor{currentfill}{rgb}{0.133298,0.375282,0.379395}%
\pgfsetfillcolor{currentfill}%
\pgfsetlinewidth{0.000000pt}%
\definecolor{currentstroke}{rgb}{0.000000,0.000000,0.000000}%
\pgfsetstrokecolor{currentstroke}%
\pgfsetstrokeopacity{0.000000}%
\pgfsetdash{}{0pt}%
\pgfpathmoveto{\pgfqpoint{6.154626in}{1.813884in}}%
\pgfpathlineto{\pgfqpoint{6.163563in}{1.813884in}}%
\pgfpathlineto{\pgfqpoint{6.163563in}{1.786190in}}%
\pgfpathlineto{\pgfqpoint{6.154626in}{1.786190in}}%
\pgfpathlineto{\pgfqpoint{6.154626in}{1.813884in}}%
\pgfpathclose%
\pgfusepath{fill}%
\end{pgfscope}%
\begin{pgfscope}%
\pgfpathrectangle{\pgfqpoint{3.722897in}{0.857143in}}{\pgfqpoint{2.627103in}{1.813434in}}%
\pgfusepath{clip}%
\pgfsetbuttcap%
\pgfsetmiterjoin%
\definecolor{currentfill}{rgb}{0.133298,0.375282,0.379395}%
\pgfsetfillcolor{currentfill}%
\pgfsetlinewidth{0.000000pt}%
\definecolor{currentstroke}{rgb}{0.000000,0.000000,0.000000}%
\pgfsetstrokecolor{currentstroke}%
\pgfsetstrokeopacity{0.000000}%
\pgfsetdash{}{0pt}%
\pgfpathmoveto{\pgfqpoint{6.165797in}{1.813884in}}%
\pgfpathlineto{\pgfqpoint{6.174733in}{1.813884in}}%
\pgfpathlineto{\pgfqpoint{6.174733in}{1.794605in}}%
\pgfpathlineto{\pgfqpoint{6.165797in}{1.794605in}}%
\pgfpathlineto{\pgfqpoint{6.165797in}{1.813884in}}%
\pgfpathclose%
\pgfusepath{fill}%
\end{pgfscope}%
\begin{pgfscope}%
\pgfpathrectangle{\pgfqpoint{3.722897in}{0.857143in}}{\pgfqpoint{2.627103in}{1.813434in}}%
\pgfusepath{clip}%
\pgfsetbuttcap%
\pgfsetmiterjoin%
\definecolor{currentfill}{rgb}{0.133298,0.375282,0.379395}%
\pgfsetfillcolor{currentfill}%
\pgfsetlinewidth{0.000000pt}%
\definecolor{currentstroke}{rgb}{0.000000,0.000000,0.000000}%
\pgfsetstrokecolor{currentstroke}%
\pgfsetstrokeopacity{0.000000}%
\pgfsetdash{}{0pt}%
\pgfpathmoveto{\pgfqpoint{6.176967in}{1.812481in}}%
\pgfpathlineto{\pgfqpoint{6.185904in}{1.812481in}}%
\pgfpathlineto{\pgfqpoint{6.185904in}{1.796179in}}%
\pgfpathlineto{\pgfqpoint{6.176967in}{1.796179in}}%
\pgfpathlineto{\pgfqpoint{6.176967in}{1.812481in}}%
\pgfpathclose%
\pgfusepath{fill}%
\end{pgfscope}%
\begin{pgfscope}%
\pgfpathrectangle{\pgfqpoint{3.722897in}{0.857143in}}{\pgfqpoint{2.627103in}{1.813434in}}%
\pgfusepath{clip}%
\pgfsetbuttcap%
\pgfsetmiterjoin%
\definecolor{currentfill}{rgb}{0.133298,0.375282,0.379395}%
\pgfsetfillcolor{currentfill}%
\pgfsetlinewidth{0.000000pt}%
\definecolor{currentstroke}{rgb}{0.000000,0.000000,0.000000}%
\pgfsetstrokecolor{currentstroke}%
\pgfsetstrokeopacity{0.000000}%
\pgfsetdash{}{0pt}%
\pgfpathmoveto{\pgfqpoint{6.188138in}{1.810758in}}%
\pgfpathlineto{\pgfqpoint{6.197074in}{1.810758in}}%
\pgfpathlineto{\pgfqpoint{6.197074in}{1.804271in}}%
\pgfpathlineto{\pgfqpoint{6.188138in}{1.804271in}}%
\pgfpathlineto{\pgfqpoint{6.188138in}{1.810758in}}%
\pgfpathclose%
\pgfusepath{fill}%
\end{pgfscope}%
\begin{pgfscope}%
\pgfpathrectangle{\pgfqpoint{3.722897in}{0.857143in}}{\pgfqpoint{2.627103in}{1.813434in}}%
\pgfusepath{clip}%
\pgfsetbuttcap%
\pgfsetmiterjoin%
\definecolor{currentfill}{rgb}{0.133298,0.375282,0.379395}%
\pgfsetfillcolor{currentfill}%
\pgfsetlinewidth{0.000000pt}%
\definecolor{currentstroke}{rgb}{0.000000,0.000000,0.000000}%
\pgfsetstrokecolor{currentstroke}%
\pgfsetstrokeopacity{0.000000}%
\pgfsetdash{}{0pt}%
\pgfpathmoveto{\pgfqpoint{6.199309in}{1.808892in}}%
\pgfpathlineto{\pgfqpoint{6.208245in}{1.808892in}}%
\pgfpathlineto{\pgfqpoint{6.208245in}{1.790744in}}%
\pgfpathlineto{\pgfqpoint{6.199309in}{1.790744in}}%
\pgfpathlineto{\pgfqpoint{6.199309in}{1.808892in}}%
\pgfpathclose%
\pgfusepath{fill}%
\end{pgfscope}%
\begin{pgfscope}%
\pgfpathrectangle{\pgfqpoint{3.722897in}{0.857143in}}{\pgfqpoint{2.627103in}{1.813434in}}%
\pgfusepath{clip}%
\pgfsetbuttcap%
\pgfsetmiterjoin%
\definecolor{currentfill}{rgb}{0.133298,0.375282,0.379395}%
\pgfsetfillcolor{currentfill}%
\pgfsetlinewidth{0.000000pt}%
\definecolor{currentstroke}{rgb}{0.000000,0.000000,0.000000}%
\pgfsetstrokecolor{currentstroke}%
\pgfsetstrokeopacity{0.000000}%
\pgfsetdash{}{0pt}%
\pgfpathmoveto{\pgfqpoint{6.210479in}{1.808025in}}%
\pgfpathlineto{\pgfqpoint{6.219416in}{1.808025in}}%
\pgfpathlineto{\pgfqpoint{6.219416in}{1.794068in}}%
\pgfpathlineto{\pgfqpoint{6.210479in}{1.794068in}}%
\pgfpathlineto{\pgfqpoint{6.210479in}{1.808025in}}%
\pgfpathclose%
\pgfusepath{fill}%
\end{pgfscope}%
\begin{pgfscope}%
\pgfpathrectangle{\pgfqpoint{3.722897in}{0.857143in}}{\pgfqpoint{2.627103in}{1.813434in}}%
\pgfusepath{clip}%
\pgfsetbuttcap%
\pgfsetmiterjoin%
\definecolor{currentfill}{rgb}{0.133298,0.375282,0.379395}%
\pgfsetfillcolor{currentfill}%
\pgfsetlinewidth{0.000000pt}%
\definecolor{currentstroke}{rgb}{0.000000,0.000000,0.000000}%
\pgfsetstrokecolor{currentstroke}%
\pgfsetstrokeopacity{0.000000}%
\pgfsetdash{}{0pt}%
\pgfpathmoveto{\pgfqpoint{6.221650in}{1.807525in}}%
\pgfpathlineto{\pgfqpoint{6.230586in}{1.807525in}}%
\pgfpathlineto{\pgfqpoint{6.230586in}{1.790803in}}%
\pgfpathlineto{\pgfqpoint{6.221650in}{1.790803in}}%
\pgfpathlineto{\pgfqpoint{6.221650in}{1.807525in}}%
\pgfpathclose%
\pgfusepath{fill}%
\end{pgfscope}%
\begin{pgfscope}%
\pgfpathrectangle{\pgfqpoint{3.722897in}{0.857143in}}{\pgfqpoint{2.627103in}{1.813434in}}%
\pgfusepath{clip}%
\pgfsetbuttcap%
\pgfsetmiterjoin%
\definecolor{currentfill}{rgb}{0.302379,0.450282,0.300122}%
\pgfsetfillcolor{currentfill}%
\pgfsetlinewidth{0.000000pt}%
\definecolor{currentstroke}{rgb}{0.000000,0.000000,0.000000}%
\pgfsetstrokecolor{currentstroke}%
\pgfsetstrokeopacity{0.000000}%
\pgfsetdash{}{0pt}%
\pgfpathmoveto{\pgfqpoint{3.842311in}{1.813044in}}%
\pgfpathlineto{\pgfqpoint{3.851247in}{1.813044in}}%
\pgfpathlineto{\pgfqpoint{3.851247in}{1.776781in}}%
\pgfpathlineto{\pgfqpoint{3.842311in}{1.776781in}}%
\pgfpathlineto{\pgfqpoint{3.842311in}{1.813044in}}%
\pgfpathclose%
\pgfusepath{fill}%
\end{pgfscope}%
\begin{pgfscope}%
\pgfpathrectangle{\pgfqpoint{3.722897in}{0.857143in}}{\pgfqpoint{2.627103in}{1.813434in}}%
\pgfusepath{clip}%
\pgfsetbuttcap%
\pgfsetmiterjoin%
\definecolor{currentfill}{rgb}{0.302379,0.450282,0.300122}%
\pgfsetfillcolor{currentfill}%
\pgfsetlinewidth{0.000000pt}%
\definecolor{currentstroke}{rgb}{0.000000,0.000000,0.000000}%
\pgfsetstrokecolor{currentstroke}%
\pgfsetstrokeopacity{0.000000}%
\pgfsetdash{}{0pt}%
\pgfpathmoveto{\pgfqpoint{3.853481in}{1.813884in}}%
\pgfpathlineto{\pgfqpoint{3.862418in}{1.813884in}}%
\pgfpathlineto{\pgfqpoint{3.862418in}{1.767639in}}%
\pgfpathlineto{\pgfqpoint{3.853481in}{1.767639in}}%
\pgfpathlineto{\pgfqpoint{3.853481in}{1.813884in}}%
\pgfpathclose%
\pgfusepath{fill}%
\end{pgfscope}%
\begin{pgfscope}%
\pgfpathrectangle{\pgfqpoint{3.722897in}{0.857143in}}{\pgfqpoint{2.627103in}{1.813434in}}%
\pgfusepath{clip}%
\pgfsetbuttcap%
\pgfsetmiterjoin%
\definecolor{currentfill}{rgb}{0.302379,0.450282,0.300122}%
\pgfsetfillcolor{currentfill}%
\pgfsetlinewidth{0.000000pt}%
\definecolor{currentstroke}{rgb}{0.000000,0.000000,0.000000}%
\pgfsetstrokecolor{currentstroke}%
\pgfsetstrokeopacity{0.000000}%
\pgfsetdash{}{0pt}%
\pgfpathmoveto{\pgfqpoint{3.864652in}{1.813884in}}%
\pgfpathlineto{\pgfqpoint{3.873588in}{1.813884in}}%
\pgfpathlineto{\pgfqpoint{3.873588in}{1.756131in}}%
\pgfpathlineto{\pgfqpoint{3.864652in}{1.756131in}}%
\pgfpathlineto{\pgfqpoint{3.864652in}{1.813884in}}%
\pgfpathclose%
\pgfusepath{fill}%
\end{pgfscope}%
\begin{pgfscope}%
\pgfpathrectangle{\pgfqpoint{3.722897in}{0.857143in}}{\pgfqpoint{2.627103in}{1.813434in}}%
\pgfusepath{clip}%
\pgfsetbuttcap%
\pgfsetmiterjoin%
\definecolor{currentfill}{rgb}{0.302379,0.450282,0.300122}%
\pgfsetfillcolor{currentfill}%
\pgfsetlinewidth{0.000000pt}%
\definecolor{currentstroke}{rgb}{0.000000,0.000000,0.000000}%
\pgfsetstrokecolor{currentstroke}%
\pgfsetstrokeopacity{0.000000}%
\pgfsetdash{}{0pt}%
\pgfpathmoveto{\pgfqpoint{3.875823in}{1.813884in}}%
\pgfpathlineto{\pgfqpoint{3.884759in}{1.813884in}}%
\pgfpathlineto{\pgfqpoint{3.884759in}{1.744969in}}%
\pgfpathlineto{\pgfqpoint{3.875823in}{1.744969in}}%
\pgfpathlineto{\pgfqpoint{3.875823in}{1.813884in}}%
\pgfpathclose%
\pgfusepath{fill}%
\end{pgfscope}%
\begin{pgfscope}%
\pgfpathrectangle{\pgfqpoint{3.722897in}{0.857143in}}{\pgfqpoint{2.627103in}{1.813434in}}%
\pgfusepath{clip}%
\pgfsetbuttcap%
\pgfsetmiterjoin%
\definecolor{currentfill}{rgb}{0.302379,0.450282,0.300122}%
\pgfsetfillcolor{currentfill}%
\pgfsetlinewidth{0.000000pt}%
\definecolor{currentstroke}{rgb}{0.000000,0.000000,0.000000}%
\pgfsetstrokecolor{currentstroke}%
\pgfsetstrokeopacity{0.000000}%
\pgfsetdash{}{0pt}%
\pgfpathmoveto{\pgfqpoint{3.886993in}{1.813884in}}%
\pgfpathlineto{\pgfqpoint{3.895930in}{1.813884in}}%
\pgfpathlineto{\pgfqpoint{3.895930in}{1.732088in}}%
\pgfpathlineto{\pgfqpoint{3.886993in}{1.732088in}}%
\pgfpathlineto{\pgfqpoint{3.886993in}{1.813884in}}%
\pgfpathclose%
\pgfusepath{fill}%
\end{pgfscope}%
\begin{pgfscope}%
\pgfpathrectangle{\pgfqpoint{3.722897in}{0.857143in}}{\pgfqpoint{2.627103in}{1.813434in}}%
\pgfusepath{clip}%
\pgfsetbuttcap%
\pgfsetmiterjoin%
\definecolor{currentfill}{rgb}{0.302379,0.450282,0.300122}%
\pgfsetfillcolor{currentfill}%
\pgfsetlinewidth{0.000000pt}%
\definecolor{currentstroke}{rgb}{0.000000,0.000000,0.000000}%
\pgfsetstrokecolor{currentstroke}%
\pgfsetstrokeopacity{0.000000}%
\pgfsetdash{}{0pt}%
\pgfpathmoveto{\pgfqpoint{3.898164in}{1.806566in}}%
\pgfpathlineto{\pgfqpoint{3.907100in}{1.806566in}}%
\pgfpathlineto{\pgfqpoint{3.907100in}{1.711139in}}%
\pgfpathlineto{\pgfqpoint{3.898164in}{1.711139in}}%
\pgfpathlineto{\pgfqpoint{3.898164in}{1.806566in}}%
\pgfpathclose%
\pgfusepath{fill}%
\end{pgfscope}%
\begin{pgfscope}%
\pgfpathrectangle{\pgfqpoint{3.722897in}{0.857143in}}{\pgfqpoint{2.627103in}{1.813434in}}%
\pgfusepath{clip}%
\pgfsetbuttcap%
\pgfsetmiterjoin%
\definecolor{currentfill}{rgb}{0.302379,0.450282,0.300122}%
\pgfsetfillcolor{currentfill}%
\pgfsetlinewidth{0.000000pt}%
\definecolor{currentstroke}{rgb}{0.000000,0.000000,0.000000}%
\pgfsetstrokecolor{currentstroke}%
\pgfsetstrokeopacity{0.000000}%
\pgfsetdash{}{0pt}%
\pgfpathmoveto{\pgfqpoint{3.909334in}{1.813884in}}%
\pgfpathlineto{\pgfqpoint{3.918271in}{1.813884in}}%
\pgfpathlineto{\pgfqpoint{3.918271in}{1.706421in}}%
\pgfpathlineto{\pgfqpoint{3.909334in}{1.706421in}}%
\pgfpathlineto{\pgfqpoint{3.909334in}{1.813884in}}%
\pgfpathclose%
\pgfusepath{fill}%
\end{pgfscope}%
\begin{pgfscope}%
\pgfpathrectangle{\pgfqpoint{3.722897in}{0.857143in}}{\pgfqpoint{2.627103in}{1.813434in}}%
\pgfusepath{clip}%
\pgfsetbuttcap%
\pgfsetmiterjoin%
\definecolor{currentfill}{rgb}{0.302379,0.450282,0.300122}%
\pgfsetfillcolor{currentfill}%
\pgfsetlinewidth{0.000000pt}%
\definecolor{currentstroke}{rgb}{0.000000,0.000000,0.000000}%
\pgfsetstrokecolor{currentstroke}%
\pgfsetstrokeopacity{0.000000}%
\pgfsetdash{}{0pt}%
\pgfpathmoveto{\pgfqpoint{3.920505in}{1.813884in}}%
\pgfpathlineto{\pgfqpoint{3.929442in}{1.813884in}}%
\pgfpathlineto{\pgfqpoint{3.929442in}{1.696524in}}%
\pgfpathlineto{\pgfqpoint{3.920505in}{1.696524in}}%
\pgfpathlineto{\pgfqpoint{3.920505in}{1.813884in}}%
\pgfpathclose%
\pgfusepath{fill}%
\end{pgfscope}%
\begin{pgfscope}%
\pgfpathrectangle{\pgfqpoint{3.722897in}{0.857143in}}{\pgfqpoint{2.627103in}{1.813434in}}%
\pgfusepath{clip}%
\pgfsetbuttcap%
\pgfsetmiterjoin%
\definecolor{currentfill}{rgb}{0.302379,0.450282,0.300122}%
\pgfsetfillcolor{currentfill}%
\pgfsetlinewidth{0.000000pt}%
\definecolor{currentstroke}{rgb}{0.000000,0.000000,0.000000}%
\pgfsetstrokecolor{currentstroke}%
\pgfsetstrokeopacity{0.000000}%
\pgfsetdash{}{0pt}%
\pgfpathmoveto{\pgfqpoint{3.931676in}{1.813884in}}%
\pgfpathlineto{\pgfqpoint{3.940612in}{1.813884in}}%
\pgfpathlineto{\pgfqpoint{3.940612in}{1.690133in}}%
\pgfpathlineto{\pgfqpoint{3.931676in}{1.690133in}}%
\pgfpathlineto{\pgfqpoint{3.931676in}{1.813884in}}%
\pgfpathclose%
\pgfusepath{fill}%
\end{pgfscope}%
\begin{pgfscope}%
\pgfpathrectangle{\pgfqpoint{3.722897in}{0.857143in}}{\pgfqpoint{2.627103in}{1.813434in}}%
\pgfusepath{clip}%
\pgfsetbuttcap%
\pgfsetmiterjoin%
\definecolor{currentfill}{rgb}{0.302379,0.450282,0.300122}%
\pgfsetfillcolor{currentfill}%
\pgfsetlinewidth{0.000000pt}%
\definecolor{currentstroke}{rgb}{0.000000,0.000000,0.000000}%
\pgfsetstrokecolor{currentstroke}%
\pgfsetstrokeopacity{0.000000}%
\pgfsetdash{}{0pt}%
\pgfpathmoveto{\pgfqpoint{3.942846in}{1.813884in}}%
\pgfpathlineto{\pgfqpoint{3.951783in}{1.813884in}}%
\pgfpathlineto{\pgfqpoint{3.951783in}{1.685081in}}%
\pgfpathlineto{\pgfqpoint{3.942846in}{1.685081in}}%
\pgfpathlineto{\pgfqpoint{3.942846in}{1.813884in}}%
\pgfpathclose%
\pgfusepath{fill}%
\end{pgfscope}%
\begin{pgfscope}%
\pgfpathrectangle{\pgfqpoint{3.722897in}{0.857143in}}{\pgfqpoint{2.627103in}{1.813434in}}%
\pgfusepath{clip}%
\pgfsetbuttcap%
\pgfsetmiterjoin%
\definecolor{currentfill}{rgb}{0.302379,0.450282,0.300122}%
\pgfsetfillcolor{currentfill}%
\pgfsetlinewidth{0.000000pt}%
\definecolor{currentstroke}{rgb}{0.000000,0.000000,0.000000}%
\pgfsetstrokecolor{currentstroke}%
\pgfsetstrokeopacity{0.000000}%
\pgfsetdash{}{0pt}%
\pgfpathmoveto{\pgfqpoint{3.954017in}{1.813884in}}%
\pgfpathlineto{\pgfqpoint{3.962953in}{1.813884in}}%
\pgfpathlineto{\pgfqpoint{3.962953in}{1.680482in}}%
\pgfpathlineto{\pgfqpoint{3.954017in}{1.680482in}}%
\pgfpathlineto{\pgfqpoint{3.954017in}{1.813884in}}%
\pgfpathclose%
\pgfusepath{fill}%
\end{pgfscope}%
\begin{pgfscope}%
\pgfpathrectangle{\pgfqpoint{3.722897in}{0.857143in}}{\pgfqpoint{2.627103in}{1.813434in}}%
\pgfusepath{clip}%
\pgfsetbuttcap%
\pgfsetmiterjoin%
\definecolor{currentfill}{rgb}{0.302379,0.450282,0.300122}%
\pgfsetfillcolor{currentfill}%
\pgfsetlinewidth{0.000000pt}%
\definecolor{currentstroke}{rgb}{0.000000,0.000000,0.000000}%
\pgfsetstrokecolor{currentstroke}%
\pgfsetstrokeopacity{0.000000}%
\pgfsetdash{}{0pt}%
\pgfpathmoveto{\pgfqpoint{3.965187in}{1.813884in}}%
\pgfpathlineto{\pgfqpoint{3.974124in}{1.813884in}}%
\pgfpathlineto{\pgfqpoint{3.974124in}{1.675667in}}%
\pgfpathlineto{\pgfqpoint{3.965187in}{1.675667in}}%
\pgfpathlineto{\pgfqpoint{3.965187in}{1.813884in}}%
\pgfpathclose%
\pgfusepath{fill}%
\end{pgfscope}%
\begin{pgfscope}%
\pgfpathrectangle{\pgfqpoint{3.722897in}{0.857143in}}{\pgfqpoint{2.627103in}{1.813434in}}%
\pgfusepath{clip}%
\pgfsetbuttcap%
\pgfsetmiterjoin%
\definecolor{currentfill}{rgb}{0.302379,0.450282,0.300122}%
\pgfsetfillcolor{currentfill}%
\pgfsetlinewidth{0.000000pt}%
\definecolor{currentstroke}{rgb}{0.000000,0.000000,0.000000}%
\pgfsetstrokecolor{currentstroke}%
\pgfsetstrokeopacity{0.000000}%
\pgfsetdash{}{0pt}%
\pgfpathmoveto{\pgfqpoint{3.976358in}{1.813884in}}%
\pgfpathlineto{\pgfqpoint{3.985295in}{1.813884in}}%
\pgfpathlineto{\pgfqpoint{3.985295in}{1.671869in}}%
\pgfpathlineto{\pgfqpoint{3.976358in}{1.671869in}}%
\pgfpathlineto{\pgfqpoint{3.976358in}{1.813884in}}%
\pgfpathclose%
\pgfusepath{fill}%
\end{pgfscope}%
\begin{pgfscope}%
\pgfpathrectangle{\pgfqpoint{3.722897in}{0.857143in}}{\pgfqpoint{2.627103in}{1.813434in}}%
\pgfusepath{clip}%
\pgfsetbuttcap%
\pgfsetmiterjoin%
\definecolor{currentfill}{rgb}{0.302379,0.450282,0.300122}%
\pgfsetfillcolor{currentfill}%
\pgfsetlinewidth{0.000000pt}%
\definecolor{currentstroke}{rgb}{0.000000,0.000000,0.000000}%
\pgfsetstrokecolor{currentstroke}%
\pgfsetstrokeopacity{0.000000}%
\pgfsetdash{}{0pt}%
\pgfpathmoveto{\pgfqpoint{3.987529in}{1.813884in}}%
\pgfpathlineto{\pgfqpoint{3.996465in}{1.813884in}}%
\pgfpathlineto{\pgfqpoint{3.996465in}{1.667688in}}%
\pgfpathlineto{\pgfqpoint{3.987529in}{1.667688in}}%
\pgfpathlineto{\pgfqpoint{3.987529in}{1.813884in}}%
\pgfpathclose%
\pgfusepath{fill}%
\end{pgfscope}%
\begin{pgfscope}%
\pgfpathrectangle{\pgfqpoint{3.722897in}{0.857143in}}{\pgfqpoint{2.627103in}{1.813434in}}%
\pgfusepath{clip}%
\pgfsetbuttcap%
\pgfsetmiterjoin%
\definecolor{currentfill}{rgb}{0.302379,0.450282,0.300122}%
\pgfsetfillcolor{currentfill}%
\pgfsetlinewidth{0.000000pt}%
\definecolor{currentstroke}{rgb}{0.000000,0.000000,0.000000}%
\pgfsetstrokecolor{currentstroke}%
\pgfsetstrokeopacity{0.000000}%
\pgfsetdash{}{0pt}%
\pgfpathmoveto{\pgfqpoint{3.998699in}{1.809848in}}%
\pgfpathlineto{\pgfqpoint{4.007636in}{1.809848in}}%
\pgfpathlineto{\pgfqpoint{4.007636in}{1.661245in}}%
\pgfpathlineto{\pgfqpoint{3.998699in}{1.661245in}}%
\pgfpathlineto{\pgfqpoint{3.998699in}{1.809848in}}%
\pgfpathclose%
\pgfusepath{fill}%
\end{pgfscope}%
\begin{pgfscope}%
\pgfpathrectangle{\pgfqpoint{3.722897in}{0.857143in}}{\pgfqpoint{2.627103in}{1.813434in}}%
\pgfusepath{clip}%
\pgfsetbuttcap%
\pgfsetmiterjoin%
\definecolor{currentfill}{rgb}{0.302379,0.450282,0.300122}%
\pgfsetfillcolor{currentfill}%
\pgfsetlinewidth{0.000000pt}%
\definecolor{currentstroke}{rgb}{0.000000,0.000000,0.000000}%
\pgfsetstrokecolor{currentstroke}%
\pgfsetstrokeopacity{0.000000}%
\pgfsetdash{}{0pt}%
\pgfpathmoveto{\pgfqpoint{4.009870in}{1.808889in}}%
\pgfpathlineto{\pgfqpoint{4.018806in}{1.808889in}}%
\pgfpathlineto{\pgfqpoint{4.018806in}{1.656708in}}%
\pgfpathlineto{\pgfqpoint{4.009870in}{1.656708in}}%
\pgfpathlineto{\pgfqpoint{4.009870in}{1.808889in}}%
\pgfpathclose%
\pgfusepath{fill}%
\end{pgfscope}%
\begin{pgfscope}%
\pgfpathrectangle{\pgfqpoint{3.722897in}{0.857143in}}{\pgfqpoint{2.627103in}{1.813434in}}%
\pgfusepath{clip}%
\pgfsetbuttcap%
\pgfsetmiterjoin%
\definecolor{currentfill}{rgb}{0.302379,0.450282,0.300122}%
\pgfsetfillcolor{currentfill}%
\pgfsetlinewidth{0.000000pt}%
\definecolor{currentstroke}{rgb}{0.000000,0.000000,0.000000}%
\pgfsetstrokecolor{currentstroke}%
\pgfsetstrokeopacity{0.000000}%
\pgfsetdash{}{0pt}%
\pgfpathmoveto{\pgfqpoint{4.021040in}{1.813884in}}%
\pgfpathlineto{\pgfqpoint{4.029977in}{1.813884in}}%
\pgfpathlineto{\pgfqpoint{4.029977in}{1.653617in}}%
\pgfpathlineto{\pgfqpoint{4.021040in}{1.653617in}}%
\pgfpathlineto{\pgfqpoint{4.021040in}{1.813884in}}%
\pgfpathclose%
\pgfusepath{fill}%
\end{pgfscope}%
\begin{pgfscope}%
\pgfpathrectangle{\pgfqpoint{3.722897in}{0.857143in}}{\pgfqpoint{2.627103in}{1.813434in}}%
\pgfusepath{clip}%
\pgfsetbuttcap%
\pgfsetmiterjoin%
\definecolor{currentfill}{rgb}{0.302379,0.450282,0.300122}%
\pgfsetfillcolor{currentfill}%
\pgfsetlinewidth{0.000000pt}%
\definecolor{currentstroke}{rgb}{0.000000,0.000000,0.000000}%
\pgfsetstrokecolor{currentstroke}%
\pgfsetstrokeopacity{0.000000}%
\pgfsetdash{}{0pt}%
\pgfpathmoveto{\pgfqpoint{4.032211in}{1.795260in}}%
\pgfpathlineto{\pgfqpoint{4.041148in}{1.795260in}}%
\pgfpathlineto{\pgfqpoint{4.041148in}{1.625915in}}%
\pgfpathlineto{\pgfqpoint{4.032211in}{1.625915in}}%
\pgfpathlineto{\pgfqpoint{4.032211in}{1.795260in}}%
\pgfpathclose%
\pgfusepath{fill}%
\end{pgfscope}%
\begin{pgfscope}%
\pgfpathrectangle{\pgfqpoint{3.722897in}{0.857143in}}{\pgfqpoint{2.627103in}{1.813434in}}%
\pgfusepath{clip}%
\pgfsetbuttcap%
\pgfsetmiterjoin%
\definecolor{currentfill}{rgb}{0.302379,0.450282,0.300122}%
\pgfsetfillcolor{currentfill}%
\pgfsetlinewidth{0.000000pt}%
\definecolor{currentstroke}{rgb}{0.000000,0.000000,0.000000}%
\pgfsetstrokecolor{currentstroke}%
\pgfsetstrokeopacity{0.000000}%
\pgfsetdash{}{0pt}%
\pgfpathmoveto{\pgfqpoint{4.043382in}{1.797959in}}%
\pgfpathlineto{\pgfqpoint{4.052318in}{1.797959in}}%
\pgfpathlineto{\pgfqpoint{4.052318in}{1.617256in}}%
\pgfpathlineto{\pgfqpoint{4.043382in}{1.617256in}}%
\pgfpathlineto{\pgfqpoint{4.043382in}{1.797959in}}%
\pgfpathclose%
\pgfusepath{fill}%
\end{pgfscope}%
\begin{pgfscope}%
\pgfpathrectangle{\pgfqpoint{3.722897in}{0.857143in}}{\pgfqpoint{2.627103in}{1.813434in}}%
\pgfusepath{clip}%
\pgfsetbuttcap%
\pgfsetmiterjoin%
\definecolor{currentfill}{rgb}{0.302379,0.450282,0.300122}%
\pgfsetfillcolor{currentfill}%
\pgfsetlinewidth{0.000000pt}%
\definecolor{currentstroke}{rgb}{0.000000,0.000000,0.000000}%
\pgfsetstrokecolor{currentstroke}%
\pgfsetstrokeopacity{0.000000}%
\pgfsetdash{}{0pt}%
\pgfpathmoveto{\pgfqpoint{4.054552in}{1.794360in}}%
\pgfpathlineto{\pgfqpoint{4.063489in}{1.794360in}}%
\pgfpathlineto{\pgfqpoint{4.063489in}{1.601657in}}%
\pgfpathlineto{\pgfqpoint{4.054552in}{1.601657in}}%
\pgfpathlineto{\pgfqpoint{4.054552in}{1.794360in}}%
\pgfpathclose%
\pgfusepath{fill}%
\end{pgfscope}%
\begin{pgfscope}%
\pgfpathrectangle{\pgfqpoint{3.722897in}{0.857143in}}{\pgfqpoint{2.627103in}{1.813434in}}%
\pgfusepath{clip}%
\pgfsetbuttcap%
\pgfsetmiterjoin%
\definecolor{currentfill}{rgb}{0.302379,0.450282,0.300122}%
\pgfsetfillcolor{currentfill}%
\pgfsetlinewidth{0.000000pt}%
\definecolor{currentstroke}{rgb}{0.000000,0.000000,0.000000}%
\pgfsetstrokecolor{currentstroke}%
\pgfsetstrokeopacity{0.000000}%
\pgfsetdash{}{0pt}%
\pgfpathmoveto{\pgfqpoint{4.065723in}{1.803325in}}%
\pgfpathlineto{\pgfqpoint{4.074659in}{1.803325in}}%
\pgfpathlineto{\pgfqpoint{4.074659in}{1.598306in}}%
\pgfpathlineto{\pgfqpoint{4.065723in}{1.598306in}}%
\pgfpathlineto{\pgfqpoint{4.065723in}{1.803325in}}%
\pgfpathclose%
\pgfusepath{fill}%
\end{pgfscope}%
\begin{pgfscope}%
\pgfpathrectangle{\pgfqpoint{3.722897in}{0.857143in}}{\pgfqpoint{2.627103in}{1.813434in}}%
\pgfusepath{clip}%
\pgfsetbuttcap%
\pgfsetmiterjoin%
\definecolor{currentfill}{rgb}{0.302379,0.450282,0.300122}%
\pgfsetfillcolor{currentfill}%
\pgfsetlinewidth{0.000000pt}%
\definecolor{currentstroke}{rgb}{0.000000,0.000000,0.000000}%
\pgfsetstrokecolor{currentstroke}%
\pgfsetstrokeopacity{0.000000}%
\pgfsetdash{}{0pt}%
\pgfpathmoveto{\pgfqpoint{4.076893in}{1.813884in}}%
\pgfpathlineto{\pgfqpoint{4.085830in}{1.813884in}}%
\pgfpathlineto{\pgfqpoint{4.085830in}{1.596477in}}%
\pgfpathlineto{\pgfqpoint{4.076893in}{1.596477in}}%
\pgfpathlineto{\pgfqpoint{4.076893in}{1.813884in}}%
\pgfpathclose%
\pgfusepath{fill}%
\end{pgfscope}%
\begin{pgfscope}%
\pgfpathrectangle{\pgfqpoint{3.722897in}{0.857143in}}{\pgfqpoint{2.627103in}{1.813434in}}%
\pgfusepath{clip}%
\pgfsetbuttcap%
\pgfsetmiterjoin%
\definecolor{currentfill}{rgb}{0.302379,0.450282,0.300122}%
\pgfsetfillcolor{currentfill}%
\pgfsetlinewidth{0.000000pt}%
\definecolor{currentstroke}{rgb}{0.000000,0.000000,0.000000}%
\pgfsetstrokecolor{currentstroke}%
\pgfsetstrokeopacity{0.000000}%
\pgfsetdash{}{0pt}%
\pgfpathmoveto{\pgfqpoint{4.088064in}{1.813884in}}%
\pgfpathlineto{\pgfqpoint{4.097001in}{1.813884in}}%
\pgfpathlineto{\pgfqpoint{4.097001in}{1.579841in}}%
\pgfpathlineto{\pgfqpoint{4.088064in}{1.579841in}}%
\pgfpathlineto{\pgfqpoint{4.088064in}{1.813884in}}%
\pgfpathclose%
\pgfusepath{fill}%
\end{pgfscope}%
\begin{pgfscope}%
\pgfpathrectangle{\pgfqpoint{3.722897in}{0.857143in}}{\pgfqpoint{2.627103in}{1.813434in}}%
\pgfusepath{clip}%
\pgfsetbuttcap%
\pgfsetmiterjoin%
\definecolor{currentfill}{rgb}{0.302379,0.450282,0.300122}%
\pgfsetfillcolor{currentfill}%
\pgfsetlinewidth{0.000000pt}%
\definecolor{currentstroke}{rgb}{0.000000,0.000000,0.000000}%
\pgfsetstrokecolor{currentstroke}%
\pgfsetstrokeopacity{0.000000}%
\pgfsetdash{}{0pt}%
\pgfpathmoveto{\pgfqpoint{4.099235in}{1.813884in}}%
\pgfpathlineto{\pgfqpoint{4.108171in}{1.813884in}}%
\pgfpathlineto{\pgfqpoint{4.108171in}{1.563069in}}%
\pgfpathlineto{\pgfqpoint{4.099235in}{1.563069in}}%
\pgfpathlineto{\pgfqpoint{4.099235in}{1.813884in}}%
\pgfpathclose%
\pgfusepath{fill}%
\end{pgfscope}%
\begin{pgfscope}%
\pgfpathrectangle{\pgfqpoint{3.722897in}{0.857143in}}{\pgfqpoint{2.627103in}{1.813434in}}%
\pgfusepath{clip}%
\pgfsetbuttcap%
\pgfsetmiterjoin%
\definecolor{currentfill}{rgb}{0.302379,0.450282,0.300122}%
\pgfsetfillcolor{currentfill}%
\pgfsetlinewidth{0.000000pt}%
\definecolor{currentstroke}{rgb}{0.000000,0.000000,0.000000}%
\pgfsetstrokecolor{currentstroke}%
\pgfsetstrokeopacity{0.000000}%
\pgfsetdash{}{0pt}%
\pgfpathmoveto{\pgfqpoint{4.110405in}{1.813884in}}%
\pgfpathlineto{\pgfqpoint{4.119342in}{1.813884in}}%
\pgfpathlineto{\pgfqpoint{4.119342in}{1.551437in}}%
\pgfpathlineto{\pgfqpoint{4.110405in}{1.551437in}}%
\pgfpathlineto{\pgfqpoint{4.110405in}{1.813884in}}%
\pgfpathclose%
\pgfusepath{fill}%
\end{pgfscope}%
\begin{pgfscope}%
\pgfpathrectangle{\pgfqpoint{3.722897in}{0.857143in}}{\pgfqpoint{2.627103in}{1.813434in}}%
\pgfusepath{clip}%
\pgfsetbuttcap%
\pgfsetmiterjoin%
\definecolor{currentfill}{rgb}{0.302379,0.450282,0.300122}%
\pgfsetfillcolor{currentfill}%
\pgfsetlinewidth{0.000000pt}%
\definecolor{currentstroke}{rgb}{0.000000,0.000000,0.000000}%
\pgfsetstrokecolor{currentstroke}%
\pgfsetstrokeopacity{0.000000}%
\pgfsetdash{}{0pt}%
\pgfpathmoveto{\pgfqpoint{4.121576in}{1.810193in}}%
\pgfpathlineto{\pgfqpoint{4.130512in}{1.810193in}}%
\pgfpathlineto{\pgfqpoint{4.130512in}{1.538656in}}%
\pgfpathlineto{\pgfqpoint{4.121576in}{1.538656in}}%
\pgfpathlineto{\pgfqpoint{4.121576in}{1.810193in}}%
\pgfpathclose%
\pgfusepath{fill}%
\end{pgfscope}%
\begin{pgfscope}%
\pgfpathrectangle{\pgfqpoint{3.722897in}{0.857143in}}{\pgfqpoint{2.627103in}{1.813434in}}%
\pgfusepath{clip}%
\pgfsetbuttcap%
\pgfsetmiterjoin%
\definecolor{currentfill}{rgb}{0.302379,0.450282,0.300122}%
\pgfsetfillcolor{currentfill}%
\pgfsetlinewidth{0.000000pt}%
\definecolor{currentstroke}{rgb}{0.000000,0.000000,0.000000}%
\pgfsetstrokecolor{currentstroke}%
\pgfsetstrokeopacity{0.000000}%
\pgfsetdash{}{0pt}%
\pgfpathmoveto{\pgfqpoint{4.132747in}{1.811703in}}%
\pgfpathlineto{\pgfqpoint{4.141683in}{1.811703in}}%
\pgfpathlineto{\pgfqpoint{4.141683in}{1.533470in}}%
\pgfpathlineto{\pgfqpoint{4.132747in}{1.533470in}}%
\pgfpathlineto{\pgfqpoint{4.132747in}{1.811703in}}%
\pgfpathclose%
\pgfusepath{fill}%
\end{pgfscope}%
\begin{pgfscope}%
\pgfpathrectangle{\pgfqpoint{3.722897in}{0.857143in}}{\pgfqpoint{2.627103in}{1.813434in}}%
\pgfusepath{clip}%
\pgfsetbuttcap%
\pgfsetmiterjoin%
\definecolor{currentfill}{rgb}{0.302379,0.450282,0.300122}%
\pgfsetfillcolor{currentfill}%
\pgfsetlinewidth{0.000000pt}%
\definecolor{currentstroke}{rgb}{0.000000,0.000000,0.000000}%
\pgfsetstrokecolor{currentstroke}%
\pgfsetstrokeopacity{0.000000}%
\pgfsetdash{}{0pt}%
\pgfpathmoveto{\pgfqpoint{4.143917in}{1.798458in}}%
\pgfpathlineto{\pgfqpoint{4.152854in}{1.798458in}}%
\pgfpathlineto{\pgfqpoint{4.152854in}{1.517521in}}%
\pgfpathlineto{\pgfqpoint{4.143917in}{1.517521in}}%
\pgfpathlineto{\pgfqpoint{4.143917in}{1.798458in}}%
\pgfpathclose%
\pgfusepath{fill}%
\end{pgfscope}%
\begin{pgfscope}%
\pgfpathrectangle{\pgfqpoint{3.722897in}{0.857143in}}{\pgfqpoint{2.627103in}{1.813434in}}%
\pgfusepath{clip}%
\pgfsetbuttcap%
\pgfsetmiterjoin%
\definecolor{currentfill}{rgb}{0.302379,0.450282,0.300122}%
\pgfsetfillcolor{currentfill}%
\pgfsetlinewidth{0.000000pt}%
\definecolor{currentstroke}{rgb}{0.000000,0.000000,0.000000}%
\pgfsetstrokecolor{currentstroke}%
\pgfsetstrokeopacity{0.000000}%
\pgfsetdash{}{0pt}%
\pgfpathmoveto{\pgfqpoint{4.155088in}{1.784715in}}%
\pgfpathlineto{\pgfqpoint{4.164024in}{1.784715in}}%
\pgfpathlineto{\pgfqpoint{4.164024in}{1.505434in}}%
\pgfpathlineto{\pgfqpoint{4.155088in}{1.505434in}}%
\pgfpathlineto{\pgfqpoint{4.155088in}{1.784715in}}%
\pgfpathclose%
\pgfusepath{fill}%
\end{pgfscope}%
\begin{pgfscope}%
\pgfpathrectangle{\pgfqpoint{3.722897in}{0.857143in}}{\pgfqpoint{2.627103in}{1.813434in}}%
\pgfusepath{clip}%
\pgfsetbuttcap%
\pgfsetmiterjoin%
\definecolor{currentfill}{rgb}{0.302379,0.450282,0.300122}%
\pgfsetfillcolor{currentfill}%
\pgfsetlinewidth{0.000000pt}%
\definecolor{currentstroke}{rgb}{0.000000,0.000000,0.000000}%
\pgfsetstrokecolor{currentstroke}%
\pgfsetstrokeopacity{0.000000}%
\pgfsetdash{}{0pt}%
\pgfpathmoveto{\pgfqpoint{4.166258in}{1.773497in}}%
\pgfpathlineto{\pgfqpoint{4.175195in}{1.773497in}}%
\pgfpathlineto{\pgfqpoint{4.175195in}{1.497545in}}%
\pgfpathlineto{\pgfqpoint{4.166258in}{1.497545in}}%
\pgfpathlineto{\pgfqpoint{4.166258in}{1.773497in}}%
\pgfpathclose%
\pgfusepath{fill}%
\end{pgfscope}%
\begin{pgfscope}%
\pgfpathrectangle{\pgfqpoint{3.722897in}{0.857143in}}{\pgfqpoint{2.627103in}{1.813434in}}%
\pgfusepath{clip}%
\pgfsetbuttcap%
\pgfsetmiterjoin%
\definecolor{currentfill}{rgb}{0.302379,0.450282,0.300122}%
\pgfsetfillcolor{currentfill}%
\pgfsetlinewidth{0.000000pt}%
\definecolor{currentstroke}{rgb}{0.000000,0.000000,0.000000}%
\pgfsetstrokecolor{currentstroke}%
\pgfsetstrokeopacity{0.000000}%
\pgfsetdash{}{0pt}%
\pgfpathmoveto{\pgfqpoint{4.177429in}{1.769606in}}%
\pgfpathlineto{\pgfqpoint{4.186365in}{1.769606in}}%
\pgfpathlineto{\pgfqpoint{4.186365in}{1.496414in}}%
\pgfpathlineto{\pgfqpoint{4.177429in}{1.496414in}}%
\pgfpathlineto{\pgfqpoint{4.177429in}{1.769606in}}%
\pgfpathclose%
\pgfusepath{fill}%
\end{pgfscope}%
\begin{pgfscope}%
\pgfpathrectangle{\pgfqpoint{3.722897in}{0.857143in}}{\pgfqpoint{2.627103in}{1.813434in}}%
\pgfusepath{clip}%
\pgfsetbuttcap%
\pgfsetmiterjoin%
\definecolor{currentfill}{rgb}{0.302379,0.450282,0.300122}%
\pgfsetfillcolor{currentfill}%
\pgfsetlinewidth{0.000000pt}%
\definecolor{currentstroke}{rgb}{0.000000,0.000000,0.000000}%
\pgfsetstrokecolor{currentstroke}%
\pgfsetstrokeopacity{0.000000}%
\pgfsetdash{}{0pt}%
\pgfpathmoveto{\pgfqpoint{4.188600in}{1.744200in}}%
\pgfpathlineto{\pgfqpoint{4.197536in}{1.744200in}}%
\pgfpathlineto{\pgfqpoint{4.197536in}{1.476458in}}%
\pgfpathlineto{\pgfqpoint{4.188600in}{1.476458in}}%
\pgfpathlineto{\pgfqpoint{4.188600in}{1.744200in}}%
\pgfpathclose%
\pgfusepath{fill}%
\end{pgfscope}%
\begin{pgfscope}%
\pgfpathrectangle{\pgfqpoint{3.722897in}{0.857143in}}{\pgfqpoint{2.627103in}{1.813434in}}%
\pgfusepath{clip}%
\pgfsetbuttcap%
\pgfsetmiterjoin%
\definecolor{currentfill}{rgb}{0.302379,0.450282,0.300122}%
\pgfsetfillcolor{currentfill}%
\pgfsetlinewidth{0.000000pt}%
\definecolor{currentstroke}{rgb}{0.000000,0.000000,0.000000}%
\pgfsetstrokecolor{currentstroke}%
\pgfsetstrokeopacity{0.000000}%
\pgfsetdash{}{0pt}%
\pgfpathmoveto{\pgfqpoint{4.199770in}{1.741816in}}%
\pgfpathlineto{\pgfqpoint{4.208707in}{1.741816in}}%
\pgfpathlineto{\pgfqpoint{4.208707in}{1.479990in}}%
\pgfpathlineto{\pgfqpoint{4.199770in}{1.479990in}}%
\pgfpathlineto{\pgfqpoint{4.199770in}{1.741816in}}%
\pgfpathclose%
\pgfusepath{fill}%
\end{pgfscope}%
\begin{pgfscope}%
\pgfpathrectangle{\pgfqpoint{3.722897in}{0.857143in}}{\pgfqpoint{2.627103in}{1.813434in}}%
\pgfusepath{clip}%
\pgfsetbuttcap%
\pgfsetmiterjoin%
\definecolor{currentfill}{rgb}{0.302379,0.450282,0.300122}%
\pgfsetfillcolor{currentfill}%
\pgfsetlinewidth{0.000000pt}%
\definecolor{currentstroke}{rgb}{0.000000,0.000000,0.000000}%
\pgfsetstrokecolor{currentstroke}%
\pgfsetstrokeopacity{0.000000}%
\pgfsetdash{}{0pt}%
\pgfpathmoveto{\pgfqpoint{4.210941in}{1.728439in}}%
\pgfpathlineto{\pgfqpoint{4.219877in}{1.728439in}}%
\pgfpathlineto{\pgfqpoint{4.219877in}{1.469736in}}%
\pgfpathlineto{\pgfqpoint{4.210941in}{1.469736in}}%
\pgfpathlineto{\pgfqpoint{4.210941in}{1.728439in}}%
\pgfpathclose%
\pgfusepath{fill}%
\end{pgfscope}%
\begin{pgfscope}%
\pgfpathrectangle{\pgfqpoint{3.722897in}{0.857143in}}{\pgfqpoint{2.627103in}{1.813434in}}%
\pgfusepath{clip}%
\pgfsetbuttcap%
\pgfsetmiterjoin%
\definecolor{currentfill}{rgb}{0.302379,0.450282,0.300122}%
\pgfsetfillcolor{currentfill}%
\pgfsetlinewidth{0.000000pt}%
\definecolor{currentstroke}{rgb}{0.000000,0.000000,0.000000}%
\pgfsetstrokecolor{currentstroke}%
\pgfsetstrokeopacity{0.000000}%
\pgfsetdash{}{0pt}%
\pgfpathmoveto{\pgfqpoint{4.222111in}{1.713129in}}%
\pgfpathlineto{\pgfqpoint{4.231048in}{1.713129in}}%
\pgfpathlineto{\pgfqpoint{4.231048in}{1.455610in}}%
\pgfpathlineto{\pgfqpoint{4.222111in}{1.455610in}}%
\pgfpathlineto{\pgfqpoint{4.222111in}{1.713129in}}%
\pgfpathclose%
\pgfusepath{fill}%
\end{pgfscope}%
\begin{pgfscope}%
\pgfpathrectangle{\pgfqpoint{3.722897in}{0.857143in}}{\pgfqpoint{2.627103in}{1.813434in}}%
\pgfusepath{clip}%
\pgfsetbuttcap%
\pgfsetmiterjoin%
\definecolor{currentfill}{rgb}{0.302379,0.450282,0.300122}%
\pgfsetfillcolor{currentfill}%
\pgfsetlinewidth{0.000000pt}%
\definecolor{currentstroke}{rgb}{0.000000,0.000000,0.000000}%
\pgfsetstrokecolor{currentstroke}%
\pgfsetstrokeopacity{0.000000}%
\pgfsetdash{}{0pt}%
\pgfpathmoveto{\pgfqpoint{4.233282in}{1.730636in}}%
\pgfpathlineto{\pgfqpoint{4.242218in}{1.730636in}}%
\pgfpathlineto{\pgfqpoint{4.242218in}{1.470408in}}%
\pgfpathlineto{\pgfqpoint{4.233282in}{1.470408in}}%
\pgfpathlineto{\pgfqpoint{4.233282in}{1.730636in}}%
\pgfpathclose%
\pgfusepath{fill}%
\end{pgfscope}%
\begin{pgfscope}%
\pgfpathrectangle{\pgfqpoint{3.722897in}{0.857143in}}{\pgfqpoint{2.627103in}{1.813434in}}%
\pgfusepath{clip}%
\pgfsetbuttcap%
\pgfsetmiterjoin%
\definecolor{currentfill}{rgb}{0.302379,0.450282,0.300122}%
\pgfsetfillcolor{currentfill}%
\pgfsetlinewidth{0.000000pt}%
\definecolor{currentstroke}{rgb}{0.000000,0.000000,0.000000}%
\pgfsetstrokecolor{currentstroke}%
\pgfsetstrokeopacity{0.000000}%
\pgfsetdash{}{0pt}%
\pgfpathmoveto{\pgfqpoint{4.244453in}{1.723090in}}%
\pgfpathlineto{\pgfqpoint{4.253389in}{1.723090in}}%
\pgfpathlineto{\pgfqpoint{4.253389in}{1.461509in}}%
\pgfpathlineto{\pgfqpoint{4.244453in}{1.461509in}}%
\pgfpathlineto{\pgfqpoint{4.244453in}{1.723090in}}%
\pgfpathclose%
\pgfusepath{fill}%
\end{pgfscope}%
\begin{pgfscope}%
\pgfpathrectangle{\pgfqpoint{3.722897in}{0.857143in}}{\pgfqpoint{2.627103in}{1.813434in}}%
\pgfusepath{clip}%
\pgfsetbuttcap%
\pgfsetmiterjoin%
\definecolor{currentfill}{rgb}{0.302379,0.450282,0.300122}%
\pgfsetfillcolor{currentfill}%
\pgfsetlinewidth{0.000000pt}%
\definecolor{currentstroke}{rgb}{0.000000,0.000000,0.000000}%
\pgfsetstrokecolor{currentstroke}%
\pgfsetstrokeopacity{0.000000}%
\pgfsetdash{}{0pt}%
\pgfpathmoveto{\pgfqpoint{4.255623in}{1.734387in}}%
\pgfpathlineto{\pgfqpoint{4.264560in}{1.734387in}}%
\pgfpathlineto{\pgfqpoint{4.264560in}{1.473144in}}%
\pgfpathlineto{\pgfqpoint{4.255623in}{1.473144in}}%
\pgfpathlineto{\pgfqpoint{4.255623in}{1.734387in}}%
\pgfpathclose%
\pgfusepath{fill}%
\end{pgfscope}%
\begin{pgfscope}%
\pgfpathrectangle{\pgfqpoint{3.722897in}{0.857143in}}{\pgfqpoint{2.627103in}{1.813434in}}%
\pgfusepath{clip}%
\pgfsetbuttcap%
\pgfsetmiterjoin%
\definecolor{currentfill}{rgb}{0.302379,0.450282,0.300122}%
\pgfsetfillcolor{currentfill}%
\pgfsetlinewidth{0.000000pt}%
\definecolor{currentstroke}{rgb}{0.000000,0.000000,0.000000}%
\pgfsetstrokecolor{currentstroke}%
\pgfsetstrokeopacity{0.000000}%
\pgfsetdash{}{0pt}%
\pgfpathmoveto{\pgfqpoint{4.266794in}{1.761793in}}%
\pgfpathlineto{\pgfqpoint{4.275730in}{1.761793in}}%
\pgfpathlineto{\pgfqpoint{4.275730in}{1.501476in}}%
\pgfpathlineto{\pgfqpoint{4.266794in}{1.501476in}}%
\pgfpathlineto{\pgfqpoint{4.266794in}{1.761793in}}%
\pgfpathclose%
\pgfusepath{fill}%
\end{pgfscope}%
\begin{pgfscope}%
\pgfpathrectangle{\pgfqpoint{3.722897in}{0.857143in}}{\pgfqpoint{2.627103in}{1.813434in}}%
\pgfusepath{clip}%
\pgfsetbuttcap%
\pgfsetmiterjoin%
\definecolor{currentfill}{rgb}{0.302379,0.450282,0.300122}%
\pgfsetfillcolor{currentfill}%
\pgfsetlinewidth{0.000000pt}%
\definecolor{currentstroke}{rgb}{0.000000,0.000000,0.000000}%
\pgfsetstrokecolor{currentstroke}%
\pgfsetstrokeopacity{0.000000}%
\pgfsetdash{}{0pt}%
\pgfpathmoveto{\pgfqpoint{4.277964in}{1.768457in}}%
\pgfpathlineto{\pgfqpoint{4.286901in}{1.768457in}}%
\pgfpathlineto{\pgfqpoint{4.286901in}{1.508671in}}%
\pgfpathlineto{\pgfqpoint{4.277964in}{1.508671in}}%
\pgfpathlineto{\pgfqpoint{4.277964in}{1.768457in}}%
\pgfpathclose%
\pgfusepath{fill}%
\end{pgfscope}%
\begin{pgfscope}%
\pgfpathrectangle{\pgfqpoint{3.722897in}{0.857143in}}{\pgfqpoint{2.627103in}{1.813434in}}%
\pgfusepath{clip}%
\pgfsetbuttcap%
\pgfsetmiterjoin%
\definecolor{currentfill}{rgb}{0.302379,0.450282,0.300122}%
\pgfsetfillcolor{currentfill}%
\pgfsetlinewidth{0.000000pt}%
\definecolor{currentstroke}{rgb}{0.000000,0.000000,0.000000}%
\pgfsetstrokecolor{currentstroke}%
\pgfsetstrokeopacity{0.000000}%
\pgfsetdash{}{0pt}%
\pgfpathmoveto{\pgfqpoint{4.289135in}{1.764907in}}%
\pgfpathlineto{\pgfqpoint{4.298071in}{1.764907in}}%
\pgfpathlineto{\pgfqpoint{4.298071in}{1.505754in}}%
\pgfpathlineto{\pgfqpoint{4.289135in}{1.505754in}}%
\pgfpathlineto{\pgfqpoint{4.289135in}{1.764907in}}%
\pgfpathclose%
\pgfusepath{fill}%
\end{pgfscope}%
\begin{pgfscope}%
\pgfpathrectangle{\pgfqpoint{3.722897in}{0.857143in}}{\pgfqpoint{2.627103in}{1.813434in}}%
\pgfusepath{clip}%
\pgfsetbuttcap%
\pgfsetmiterjoin%
\definecolor{currentfill}{rgb}{0.302379,0.450282,0.300122}%
\pgfsetfillcolor{currentfill}%
\pgfsetlinewidth{0.000000pt}%
\definecolor{currentstroke}{rgb}{0.000000,0.000000,0.000000}%
\pgfsetstrokecolor{currentstroke}%
\pgfsetstrokeopacity{0.000000}%
\pgfsetdash{}{0pt}%
\pgfpathmoveto{\pgfqpoint{4.300306in}{1.754939in}}%
\pgfpathlineto{\pgfqpoint{4.309242in}{1.754939in}}%
\pgfpathlineto{\pgfqpoint{4.309242in}{1.496815in}}%
\pgfpathlineto{\pgfqpoint{4.300306in}{1.496815in}}%
\pgfpathlineto{\pgfqpoint{4.300306in}{1.754939in}}%
\pgfpathclose%
\pgfusepath{fill}%
\end{pgfscope}%
\begin{pgfscope}%
\pgfpathrectangle{\pgfqpoint{3.722897in}{0.857143in}}{\pgfqpoint{2.627103in}{1.813434in}}%
\pgfusepath{clip}%
\pgfsetbuttcap%
\pgfsetmiterjoin%
\definecolor{currentfill}{rgb}{0.302379,0.450282,0.300122}%
\pgfsetfillcolor{currentfill}%
\pgfsetlinewidth{0.000000pt}%
\definecolor{currentstroke}{rgb}{0.000000,0.000000,0.000000}%
\pgfsetstrokecolor{currentstroke}%
\pgfsetstrokeopacity{0.000000}%
\pgfsetdash{}{0pt}%
\pgfpathmoveto{\pgfqpoint{4.311476in}{1.767192in}}%
\pgfpathlineto{\pgfqpoint{4.320413in}{1.767192in}}%
\pgfpathlineto{\pgfqpoint{4.320413in}{1.510500in}}%
\pgfpathlineto{\pgfqpoint{4.311476in}{1.510500in}}%
\pgfpathlineto{\pgfqpoint{4.311476in}{1.767192in}}%
\pgfpathclose%
\pgfusepath{fill}%
\end{pgfscope}%
\begin{pgfscope}%
\pgfpathrectangle{\pgfqpoint{3.722897in}{0.857143in}}{\pgfqpoint{2.627103in}{1.813434in}}%
\pgfusepath{clip}%
\pgfsetbuttcap%
\pgfsetmiterjoin%
\definecolor{currentfill}{rgb}{0.302379,0.450282,0.300122}%
\pgfsetfillcolor{currentfill}%
\pgfsetlinewidth{0.000000pt}%
\definecolor{currentstroke}{rgb}{0.000000,0.000000,0.000000}%
\pgfsetstrokecolor{currentstroke}%
\pgfsetstrokeopacity{0.000000}%
\pgfsetdash{}{0pt}%
\pgfpathmoveto{\pgfqpoint{4.322647in}{1.781992in}}%
\pgfpathlineto{\pgfqpoint{4.331583in}{1.781992in}}%
\pgfpathlineto{\pgfqpoint{4.331583in}{1.527418in}}%
\pgfpathlineto{\pgfqpoint{4.322647in}{1.527418in}}%
\pgfpathlineto{\pgfqpoint{4.322647in}{1.781992in}}%
\pgfpathclose%
\pgfusepath{fill}%
\end{pgfscope}%
\begin{pgfscope}%
\pgfpathrectangle{\pgfqpoint{3.722897in}{0.857143in}}{\pgfqpoint{2.627103in}{1.813434in}}%
\pgfusepath{clip}%
\pgfsetbuttcap%
\pgfsetmiterjoin%
\definecolor{currentfill}{rgb}{0.302379,0.450282,0.300122}%
\pgfsetfillcolor{currentfill}%
\pgfsetlinewidth{0.000000pt}%
\definecolor{currentstroke}{rgb}{0.000000,0.000000,0.000000}%
\pgfsetstrokecolor{currentstroke}%
\pgfsetstrokeopacity{0.000000}%
\pgfsetdash{}{0pt}%
\pgfpathmoveto{\pgfqpoint{4.333817in}{1.791256in}}%
\pgfpathlineto{\pgfqpoint{4.342754in}{1.791256in}}%
\pgfpathlineto{\pgfqpoint{4.342754in}{1.538683in}}%
\pgfpathlineto{\pgfqpoint{4.333817in}{1.538683in}}%
\pgfpathlineto{\pgfqpoint{4.333817in}{1.791256in}}%
\pgfpathclose%
\pgfusepath{fill}%
\end{pgfscope}%
\begin{pgfscope}%
\pgfpathrectangle{\pgfqpoint{3.722897in}{0.857143in}}{\pgfqpoint{2.627103in}{1.813434in}}%
\pgfusepath{clip}%
\pgfsetbuttcap%
\pgfsetmiterjoin%
\definecolor{currentfill}{rgb}{0.302379,0.450282,0.300122}%
\pgfsetfillcolor{currentfill}%
\pgfsetlinewidth{0.000000pt}%
\definecolor{currentstroke}{rgb}{0.000000,0.000000,0.000000}%
\pgfsetstrokecolor{currentstroke}%
\pgfsetstrokeopacity{0.000000}%
\pgfsetdash{}{0pt}%
\pgfpathmoveto{\pgfqpoint{4.344988in}{1.759372in}}%
\pgfpathlineto{\pgfqpoint{4.353925in}{1.759372in}}%
\pgfpathlineto{\pgfqpoint{4.353925in}{1.511987in}}%
\pgfpathlineto{\pgfqpoint{4.344988in}{1.511987in}}%
\pgfpathlineto{\pgfqpoint{4.344988in}{1.759372in}}%
\pgfpathclose%
\pgfusepath{fill}%
\end{pgfscope}%
\begin{pgfscope}%
\pgfpathrectangle{\pgfqpoint{3.722897in}{0.857143in}}{\pgfqpoint{2.627103in}{1.813434in}}%
\pgfusepath{clip}%
\pgfsetbuttcap%
\pgfsetmiterjoin%
\definecolor{currentfill}{rgb}{0.302379,0.450282,0.300122}%
\pgfsetfillcolor{currentfill}%
\pgfsetlinewidth{0.000000pt}%
\definecolor{currentstroke}{rgb}{0.000000,0.000000,0.000000}%
\pgfsetstrokecolor{currentstroke}%
\pgfsetstrokeopacity{0.000000}%
\pgfsetdash{}{0pt}%
\pgfpathmoveto{\pgfqpoint{4.356159in}{1.794322in}}%
\pgfpathlineto{\pgfqpoint{4.365095in}{1.794322in}}%
\pgfpathlineto{\pgfqpoint{4.365095in}{1.550566in}}%
\pgfpathlineto{\pgfqpoint{4.356159in}{1.550566in}}%
\pgfpathlineto{\pgfqpoint{4.356159in}{1.794322in}}%
\pgfpathclose%
\pgfusepath{fill}%
\end{pgfscope}%
\begin{pgfscope}%
\pgfpathrectangle{\pgfqpoint{3.722897in}{0.857143in}}{\pgfqpoint{2.627103in}{1.813434in}}%
\pgfusepath{clip}%
\pgfsetbuttcap%
\pgfsetmiterjoin%
\definecolor{currentfill}{rgb}{0.302379,0.450282,0.300122}%
\pgfsetfillcolor{currentfill}%
\pgfsetlinewidth{0.000000pt}%
\definecolor{currentstroke}{rgb}{0.000000,0.000000,0.000000}%
\pgfsetstrokecolor{currentstroke}%
\pgfsetstrokeopacity{0.000000}%
\pgfsetdash{}{0pt}%
\pgfpathmoveto{\pgfqpoint{4.367329in}{1.785520in}}%
\pgfpathlineto{\pgfqpoint{4.376266in}{1.785520in}}%
\pgfpathlineto{\pgfqpoint{4.376266in}{1.546661in}}%
\pgfpathlineto{\pgfqpoint{4.367329in}{1.546661in}}%
\pgfpathlineto{\pgfqpoint{4.367329in}{1.785520in}}%
\pgfpathclose%
\pgfusepath{fill}%
\end{pgfscope}%
\begin{pgfscope}%
\pgfpathrectangle{\pgfqpoint{3.722897in}{0.857143in}}{\pgfqpoint{2.627103in}{1.813434in}}%
\pgfusepath{clip}%
\pgfsetbuttcap%
\pgfsetmiterjoin%
\definecolor{currentfill}{rgb}{0.302379,0.450282,0.300122}%
\pgfsetfillcolor{currentfill}%
\pgfsetlinewidth{0.000000pt}%
\definecolor{currentstroke}{rgb}{0.000000,0.000000,0.000000}%
\pgfsetstrokecolor{currentstroke}%
\pgfsetstrokeopacity{0.000000}%
\pgfsetdash{}{0pt}%
\pgfpathmoveto{\pgfqpoint{4.378500in}{1.790816in}}%
\pgfpathlineto{\pgfqpoint{4.387436in}{1.790816in}}%
\pgfpathlineto{\pgfqpoint{4.387436in}{1.559293in}}%
\pgfpathlineto{\pgfqpoint{4.378500in}{1.559293in}}%
\pgfpathlineto{\pgfqpoint{4.378500in}{1.790816in}}%
\pgfpathclose%
\pgfusepath{fill}%
\end{pgfscope}%
\begin{pgfscope}%
\pgfpathrectangle{\pgfqpoint{3.722897in}{0.857143in}}{\pgfqpoint{2.627103in}{1.813434in}}%
\pgfusepath{clip}%
\pgfsetbuttcap%
\pgfsetmiterjoin%
\definecolor{currentfill}{rgb}{0.302379,0.450282,0.300122}%
\pgfsetfillcolor{currentfill}%
\pgfsetlinewidth{0.000000pt}%
\definecolor{currentstroke}{rgb}{0.000000,0.000000,0.000000}%
\pgfsetstrokecolor{currentstroke}%
\pgfsetstrokeopacity{0.000000}%
\pgfsetdash{}{0pt}%
\pgfpathmoveto{\pgfqpoint{4.389670in}{1.783062in}}%
\pgfpathlineto{\pgfqpoint{4.398607in}{1.783062in}}%
\pgfpathlineto{\pgfqpoint{4.398607in}{1.560053in}}%
\pgfpathlineto{\pgfqpoint{4.389670in}{1.560053in}}%
\pgfpathlineto{\pgfqpoint{4.389670in}{1.783062in}}%
\pgfpathclose%
\pgfusepath{fill}%
\end{pgfscope}%
\begin{pgfscope}%
\pgfpathrectangle{\pgfqpoint{3.722897in}{0.857143in}}{\pgfqpoint{2.627103in}{1.813434in}}%
\pgfusepath{clip}%
\pgfsetbuttcap%
\pgfsetmiterjoin%
\definecolor{currentfill}{rgb}{0.302379,0.450282,0.300122}%
\pgfsetfillcolor{currentfill}%
\pgfsetlinewidth{0.000000pt}%
\definecolor{currentstroke}{rgb}{0.000000,0.000000,0.000000}%
\pgfsetstrokecolor{currentstroke}%
\pgfsetstrokeopacity{0.000000}%
\pgfsetdash{}{0pt}%
\pgfpathmoveto{\pgfqpoint{4.400841in}{1.793758in}}%
\pgfpathlineto{\pgfqpoint{4.409778in}{1.793758in}}%
\pgfpathlineto{\pgfqpoint{4.409778in}{1.580592in}}%
\pgfpathlineto{\pgfqpoint{4.400841in}{1.580592in}}%
\pgfpathlineto{\pgfqpoint{4.400841in}{1.793758in}}%
\pgfpathclose%
\pgfusepath{fill}%
\end{pgfscope}%
\begin{pgfscope}%
\pgfpathrectangle{\pgfqpoint{3.722897in}{0.857143in}}{\pgfqpoint{2.627103in}{1.813434in}}%
\pgfusepath{clip}%
\pgfsetbuttcap%
\pgfsetmiterjoin%
\definecolor{currentfill}{rgb}{0.302379,0.450282,0.300122}%
\pgfsetfillcolor{currentfill}%
\pgfsetlinewidth{0.000000pt}%
\definecolor{currentstroke}{rgb}{0.000000,0.000000,0.000000}%
\pgfsetstrokecolor{currentstroke}%
\pgfsetstrokeopacity{0.000000}%
\pgfsetdash{}{0pt}%
\pgfpathmoveto{\pgfqpoint{4.412012in}{1.757088in}}%
\pgfpathlineto{\pgfqpoint{4.420948in}{1.757088in}}%
\pgfpathlineto{\pgfqpoint{4.420948in}{1.557325in}}%
\pgfpathlineto{\pgfqpoint{4.412012in}{1.557325in}}%
\pgfpathlineto{\pgfqpoint{4.412012in}{1.757088in}}%
\pgfpathclose%
\pgfusepath{fill}%
\end{pgfscope}%
\begin{pgfscope}%
\pgfpathrectangle{\pgfqpoint{3.722897in}{0.857143in}}{\pgfqpoint{2.627103in}{1.813434in}}%
\pgfusepath{clip}%
\pgfsetbuttcap%
\pgfsetmiterjoin%
\definecolor{currentfill}{rgb}{0.302379,0.450282,0.300122}%
\pgfsetfillcolor{currentfill}%
\pgfsetlinewidth{0.000000pt}%
\definecolor{currentstroke}{rgb}{0.000000,0.000000,0.000000}%
\pgfsetstrokecolor{currentstroke}%
\pgfsetstrokeopacity{0.000000}%
\pgfsetdash{}{0pt}%
\pgfpathmoveto{\pgfqpoint{4.423182in}{1.779137in}}%
\pgfpathlineto{\pgfqpoint{4.432119in}{1.779137in}}%
\pgfpathlineto{\pgfqpoint{4.432119in}{1.591706in}}%
\pgfpathlineto{\pgfqpoint{4.423182in}{1.591706in}}%
\pgfpathlineto{\pgfqpoint{4.423182in}{1.779137in}}%
\pgfpathclose%
\pgfusepath{fill}%
\end{pgfscope}%
\begin{pgfscope}%
\pgfpathrectangle{\pgfqpoint{3.722897in}{0.857143in}}{\pgfqpoint{2.627103in}{1.813434in}}%
\pgfusepath{clip}%
\pgfsetbuttcap%
\pgfsetmiterjoin%
\definecolor{currentfill}{rgb}{0.302379,0.450282,0.300122}%
\pgfsetfillcolor{currentfill}%
\pgfsetlinewidth{0.000000pt}%
\definecolor{currentstroke}{rgb}{0.000000,0.000000,0.000000}%
\pgfsetstrokecolor{currentstroke}%
\pgfsetstrokeopacity{0.000000}%
\pgfsetdash{}{0pt}%
\pgfpathmoveto{\pgfqpoint{4.434353in}{1.785918in}}%
\pgfpathlineto{\pgfqpoint{4.443289in}{1.785918in}}%
\pgfpathlineto{\pgfqpoint{4.443289in}{1.612296in}}%
\pgfpathlineto{\pgfqpoint{4.434353in}{1.612296in}}%
\pgfpathlineto{\pgfqpoint{4.434353in}{1.785918in}}%
\pgfpathclose%
\pgfusepath{fill}%
\end{pgfscope}%
\begin{pgfscope}%
\pgfpathrectangle{\pgfqpoint{3.722897in}{0.857143in}}{\pgfqpoint{2.627103in}{1.813434in}}%
\pgfusepath{clip}%
\pgfsetbuttcap%
\pgfsetmiterjoin%
\definecolor{currentfill}{rgb}{0.302379,0.450282,0.300122}%
\pgfsetfillcolor{currentfill}%
\pgfsetlinewidth{0.000000pt}%
\definecolor{currentstroke}{rgb}{0.000000,0.000000,0.000000}%
\pgfsetstrokecolor{currentstroke}%
\pgfsetstrokeopacity{0.000000}%
\pgfsetdash{}{0pt}%
\pgfpathmoveto{\pgfqpoint{4.445523in}{1.775156in}}%
\pgfpathlineto{\pgfqpoint{4.454460in}{1.775156in}}%
\pgfpathlineto{\pgfqpoint{4.454460in}{1.619254in}}%
\pgfpathlineto{\pgfqpoint{4.445523in}{1.619254in}}%
\pgfpathlineto{\pgfqpoint{4.445523in}{1.775156in}}%
\pgfpathclose%
\pgfusepath{fill}%
\end{pgfscope}%
\begin{pgfscope}%
\pgfpathrectangle{\pgfqpoint{3.722897in}{0.857143in}}{\pgfqpoint{2.627103in}{1.813434in}}%
\pgfusepath{clip}%
\pgfsetbuttcap%
\pgfsetmiterjoin%
\definecolor{currentfill}{rgb}{0.302379,0.450282,0.300122}%
\pgfsetfillcolor{currentfill}%
\pgfsetlinewidth{0.000000pt}%
\definecolor{currentstroke}{rgb}{0.000000,0.000000,0.000000}%
\pgfsetstrokecolor{currentstroke}%
\pgfsetstrokeopacity{0.000000}%
\pgfsetdash{}{0pt}%
\pgfpathmoveto{\pgfqpoint{4.456694in}{1.776308in}}%
\pgfpathlineto{\pgfqpoint{4.465631in}{1.776308in}}%
\pgfpathlineto{\pgfqpoint{4.465631in}{1.635074in}}%
\pgfpathlineto{\pgfqpoint{4.456694in}{1.635074in}}%
\pgfpathlineto{\pgfqpoint{4.456694in}{1.776308in}}%
\pgfpathclose%
\pgfusepath{fill}%
\end{pgfscope}%
\begin{pgfscope}%
\pgfpathrectangle{\pgfqpoint{3.722897in}{0.857143in}}{\pgfqpoint{2.627103in}{1.813434in}}%
\pgfusepath{clip}%
\pgfsetbuttcap%
\pgfsetmiterjoin%
\definecolor{currentfill}{rgb}{0.302379,0.450282,0.300122}%
\pgfsetfillcolor{currentfill}%
\pgfsetlinewidth{0.000000pt}%
\definecolor{currentstroke}{rgb}{0.000000,0.000000,0.000000}%
\pgfsetstrokecolor{currentstroke}%
\pgfsetstrokeopacity{0.000000}%
\pgfsetdash{}{0pt}%
\pgfpathmoveto{\pgfqpoint{4.467865in}{1.765107in}}%
\pgfpathlineto{\pgfqpoint{4.476801in}{1.765107in}}%
\pgfpathlineto{\pgfqpoint{4.476801in}{1.632861in}}%
\pgfpathlineto{\pgfqpoint{4.467865in}{1.632861in}}%
\pgfpathlineto{\pgfqpoint{4.467865in}{1.765107in}}%
\pgfpathclose%
\pgfusepath{fill}%
\end{pgfscope}%
\begin{pgfscope}%
\pgfpathrectangle{\pgfqpoint{3.722897in}{0.857143in}}{\pgfqpoint{2.627103in}{1.813434in}}%
\pgfusepath{clip}%
\pgfsetbuttcap%
\pgfsetmiterjoin%
\definecolor{currentfill}{rgb}{0.302379,0.450282,0.300122}%
\pgfsetfillcolor{currentfill}%
\pgfsetlinewidth{0.000000pt}%
\definecolor{currentstroke}{rgb}{0.000000,0.000000,0.000000}%
\pgfsetstrokecolor{currentstroke}%
\pgfsetstrokeopacity{0.000000}%
\pgfsetdash{}{0pt}%
\pgfpathmoveto{\pgfqpoint{4.479035in}{1.753649in}}%
\pgfpathlineto{\pgfqpoint{4.487972in}{1.753649in}}%
\pgfpathlineto{\pgfqpoint{4.487972in}{1.630522in}}%
\pgfpathlineto{\pgfqpoint{4.479035in}{1.630522in}}%
\pgfpathlineto{\pgfqpoint{4.479035in}{1.753649in}}%
\pgfpathclose%
\pgfusepath{fill}%
\end{pgfscope}%
\begin{pgfscope}%
\pgfpathrectangle{\pgfqpoint{3.722897in}{0.857143in}}{\pgfqpoint{2.627103in}{1.813434in}}%
\pgfusepath{clip}%
\pgfsetbuttcap%
\pgfsetmiterjoin%
\definecolor{currentfill}{rgb}{0.302379,0.450282,0.300122}%
\pgfsetfillcolor{currentfill}%
\pgfsetlinewidth{0.000000pt}%
\definecolor{currentstroke}{rgb}{0.000000,0.000000,0.000000}%
\pgfsetstrokecolor{currentstroke}%
\pgfsetstrokeopacity{0.000000}%
\pgfsetdash{}{0pt}%
\pgfpathmoveto{\pgfqpoint{4.490206in}{1.761830in}}%
\pgfpathlineto{\pgfqpoint{4.499142in}{1.761830in}}%
\pgfpathlineto{\pgfqpoint{4.499142in}{1.652027in}}%
\pgfpathlineto{\pgfqpoint{4.490206in}{1.652027in}}%
\pgfpathlineto{\pgfqpoint{4.490206in}{1.761830in}}%
\pgfpathclose%
\pgfusepath{fill}%
\end{pgfscope}%
\begin{pgfscope}%
\pgfpathrectangle{\pgfqpoint{3.722897in}{0.857143in}}{\pgfqpoint{2.627103in}{1.813434in}}%
\pgfusepath{clip}%
\pgfsetbuttcap%
\pgfsetmiterjoin%
\definecolor{currentfill}{rgb}{0.302379,0.450282,0.300122}%
\pgfsetfillcolor{currentfill}%
\pgfsetlinewidth{0.000000pt}%
\definecolor{currentstroke}{rgb}{0.000000,0.000000,0.000000}%
\pgfsetstrokecolor{currentstroke}%
\pgfsetstrokeopacity{0.000000}%
\pgfsetdash{}{0pt}%
\pgfpathmoveto{\pgfqpoint{4.501377in}{1.789923in}}%
\pgfpathlineto{\pgfqpoint{4.510313in}{1.789923in}}%
\pgfpathlineto{\pgfqpoint{4.510313in}{1.692743in}}%
\pgfpathlineto{\pgfqpoint{4.501377in}{1.692743in}}%
\pgfpathlineto{\pgfqpoint{4.501377in}{1.789923in}}%
\pgfpathclose%
\pgfusepath{fill}%
\end{pgfscope}%
\begin{pgfscope}%
\pgfpathrectangle{\pgfqpoint{3.722897in}{0.857143in}}{\pgfqpoint{2.627103in}{1.813434in}}%
\pgfusepath{clip}%
\pgfsetbuttcap%
\pgfsetmiterjoin%
\definecolor{currentfill}{rgb}{0.302379,0.450282,0.300122}%
\pgfsetfillcolor{currentfill}%
\pgfsetlinewidth{0.000000pt}%
\definecolor{currentstroke}{rgb}{0.000000,0.000000,0.000000}%
\pgfsetstrokecolor{currentstroke}%
\pgfsetstrokeopacity{0.000000}%
\pgfsetdash{}{0pt}%
\pgfpathmoveto{\pgfqpoint{4.512547in}{1.782142in}}%
\pgfpathlineto{\pgfqpoint{4.521484in}{1.782142in}}%
\pgfpathlineto{\pgfqpoint{4.521484in}{1.696445in}}%
\pgfpathlineto{\pgfqpoint{4.512547in}{1.696445in}}%
\pgfpathlineto{\pgfqpoint{4.512547in}{1.782142in}}%
\pgfpathclose%
\pgfusepath{fill}%
\end{pgfscope}%
\begin{pgfscope}%
\pgfpathrectangle{\pgfqpoint{3.722897in}{0.857143in}}{\pgfqpoint{2.627103in}{1.813434in}}%
\pgfusepath{clip}%
\pgfsetbuttcap%
\pgfsetmiterjoin%
\definecolor{currentfill}{rgb}{0.302379,0.450282,0.300122}%
\pgfsetfillcolor{currentfill}%
\pgfsetlinewidth{0.000000pt}%
\definecolor{currentstroke}{rgb}{0.000000,0.000000,0.000000}%
\pgfsetstrokecolor{currentstroke}%
\pgfsetstrokeopacity{0.000000}%
\pgfsetdash{}{0pt}%
\pgfpathmoveto{\pgfqpoint{4.523718in}{1.788642in}}%
\pgfpathlineto{\pgfqpoint{4.532654in}{1.788642in}}%
\pgfpathlineto{\pgfqpoint{4.532654in}{1.710502in}}%
\pgfpathlineto{\pgfqpoint{4.523718in}{1.710502in}}%
\pgfpathlineto{\pgfqpoint{4.523718in}{1.788642in}}%
\pgfpathclose%
\pgfusepath{fill}%
\end{pgfscope}%
\begin{pgfscope}%
\pgfpathrectangle{\pgfqpoint{3.722897in}{0.857143in}}{\pgfqpoint{2.627103in}{1.813434in}}%
\pgfusepath{clip}%
\pgfsetbuttcap%
\pgfsetmiterjoin%
\definecolor{currentfill}{rgb}{0.302379,0.450282,0.300122}%
\pgfsetfillcolor{currentfill}%
\pgfsetlinewidth{0.000000pt}%
\definecolor{currentstroke}{rgb}{0.000000,0.000000,0.000000}%
\pgfsetstrokecolor{currentstroke}%
\pgfsetstrokeopacity{0.000000}%
\pgfsetdash{}{0pt}%
\pgfpathmoveto{\pgfqpoint{4.534888in}{1.793703in}}%
\pgfpathlineto{\pgfqpoint{4.543825in}{1.793703in}}%
\pgfpathlineto{\pgfqpoint{4.543825in}{1.717026in}}%
\pgfpathlineto{\pgfqpoint{4.534888in}{1.717026in}}%
\pgfpathlineto{\pgfqpoint{4.534888in}{1.793703in}}%
\pgfpathclose%
\pgfusepath{fill}%
\end{pgfscope}%
\begin{pgfscope}%
\pgfpathrectangle{\pgfqpoint{3.722897in}{0.857143in}}{\pgfqpoint{2.627103in}{1.813434in}}%
\pgfusepath{clip}%
\pgfsetbuttcap%
\pgfsetmiterjoin%
\definecolor{currentfill}{rgb}{0.302379,0.450282,0.300122}%
\pgfsetfillcolor{currentfill}%
\pgfsetlinewidth{0.000000pt}%
\definecolor{currentstroke}{rgb}{0.000000,0.000000,0.000000}%
\pgfsetstrokecolor{currentstroke}%
\pgfsetstrokeopacity{0.000000}%
\pgfsetdash{}{0pt}%
\pgfpathmoveto{\pgfqpoint{4.546059in}{1.781798in}}%
\pgfpathlineto{\pgfqpoint{4.554995in}{1.781798in}}%
\pgfpathlineto{\pgfqpoint{4.554995in}{1.707682in}}%
\pgfpathlineto{\pgfqpoint{4.546059in}{1.707682in}}%
\pgfpathlineto{\pgfqpoint{4.546059in}{1.781798in}}%
\pgfpathclose%
\pgfusepath{fill}%
\end{pgfscope}%
\begin{pgfscope}%
\pgfpathrectangle{\pgfqpoint{3.722897in}{0.857143in}}{\pgfqpoint{2.627103in}{1.813434in}}%
\pgfusepath{clip}%
\pgfsetbuttcap%
\pgfsetmiterjoin%
\definecolor{currentfill}{rgb}{0.302379,0.450282,0.300122}%
\pgfsetfillcolor{currentfill}%
\pgfsetlinewidth{0.000000pt}%
\definecolor{currentstroke}{rgb}{0.000000,0.000000,0.000000}%
\pgfsetstrokecolor{currentstroke}%
\pgfsetstrokeopacity{0.000000}%
\pgfsetdash{}{0pt}%
\pgfpathmoveto{\pgfqpoint{4.557230in}{1.780281in}}%
\pgfpathlineto{\pgfqpoint{4.566166in}{1.780281in}}%
\pgfpathlineto{\pgfqpoint{4.566166in}{1.707262in}}%
\pgfpathlineto{\pgfqpoint{4.557230in}{1.707262in}}%
\pgfpathlineto{\pgfqpoint{4.557230in}{1.780281in}}%
\pgfpathclose%
\pgfusepath{fill}%
\end{pgfscope}%
\begin{pgfscope}%
\pgfpathrectangle{\pgfqpoint{3.722897in}{0.857143in}}{\pgfqpoint{2.627103in}{1.813434in}}%
\pgfusepath{clip}%
\pgfsetbuttcap%
\pgfsetmiterjoin%
\definecolor{currentfill}{rgb}{0.302379,0.450282,0.300122}%
\pgfsetfillcolor{currentfill}%
\pgfsetlinewidth{0.000000pt}%
\definecolor{currentstroke}{rgb}{0.000000,0.000000,0.000000}%
\pgfsetstrokecolor{currentstroke}%
\pgfsetstrokeopacity{0.000000}%
\pgfsetdash{}{0pt}%
\pgfpathmoveto{\pgfqpoint{4.568400in}{1.773705in}}%
\pgfpathlineto{\pgfqpoint{4.577337in}{1.773705in}}%
\pgfpathlineto{\pgfqpoint{4.577337in}{1.700383in}}%
\pgfpathlineto{\pgfqpoint{4.568400in}{1.700383in}}%
\pgfpathlineto{\pgfqpoint{4.568400in}{1.773705in}}%
\pgfpathclose%
\pgfusepath{fill}%
\end{pgfscope}%
\begin{pgfscope}%
\pgfpathrectangle{\pgfqpoint{3.722897in}{0.857143in}}{\pgfqpoint{2.627103in}{1.813434in}}%
\pgfusepath{clip}%
\pgfsetbuttcap%
\pgfsetmiterjoin%
\definecolor{currentfill}{rgb}{0.302379,0.450282,0.300122}%
\pgfsetfillcolor{currentfill}%
\pgfsetlinewidth{0.000000pt}%
\definecolor{currentstroke}{rgb}{0.000000,0.000000,0.000000}%
\pgfsetstrokecolor{currentstroke}%
\pgfsetstrokeopacity{0.000000}%
\pgfsetdash{}{0pt}%
\pgfpathmoveto{\pgfqpoint{4.579571in}{1.787806in}}%
\pgfpathlineto{\pgfqpoint{4.588507in}{1.787806in}}%
\pgfpathlineto{\pgfqpoint{4.588507in}{1.713012in}}%
\pgfpathlineto{\pgfqpoint{4.579571in}{1.713012in}}%
\pgfpathlineto{\pgfqpoint{4.579571in}{1.787806in}}%
\pgfpathclose%
\pgfusepath{fill}%
\end{pgfscope}%
\begin{pgfscope}%
\pgfpathrectangle{\pgfqpoint{3.722897in}{0.857143in}}{\pgfqpoint{2.627103in}{1.813434in}}%
\pgfusepath{clip}%
\pgfsetbuttcap%
\pgfsetmiterjoin%
\definecolor{currentfill}{rgb}{0.302379,0.450282,0.300122}%
\pgfsetfillcolor{currentfill}%
\pgfsetlinewidth{0.000000pt}%
\definecolor{currentstroke}{rgb}{0.000000,0.000000,0.000000}%
\pgfsetstrokecolor{currentstroke}%
\pgfsetstrokeopacity{0.000000}%
\pgfsetdash{}{0pt}%
\pgfpathmoveto{\pgfqpoint{4.590741in}{1.782130in}}%
\pgfpathlineto{\pgfqpoint{4.599678in}{1.782130in}}%
\pgfpathlineto{\pgfqpoint{4.599678in}{1.706465in}}%
\pgfpathlineto{\pgfqpoint{4.590741in}{1.706465in}}%
\pgfpathlineto{\pgfqpoint{4.590741in}{1.782130in}}%
\pgfpathclose%
\pgfusepath{fill}%
\end{pgfscope}%
\begin{pgfscope}%
\pgfpathrectangle{\pgfqpoint{3.722897in}{0.857143in}}{\pgfqpoint{2.627103in}{1.813434in}}%
\pgfusepath{clip}%
\pgfsetbuttcap%
\pgfsetmiterjoin%
\definecolor{currentfill}{rgb}{0.302379,0.450282,0.300122}%
\pgfsetfillcolor{currentfill}%
\pgfsetlinewidth{0.000000pt}%
\definecolor{currentstroke}{rgb}{0.000000,0.000000,0.000000}%
\pgfsetstrokecolor{currentstroke}%
\pgfsetstrokeopacity{0.000000}%
\pgfsetdash{}{0pt}%
\pgfpathmoveto{\pgfqpoint{4.601912in}{1.778007in}}%
\pgfpathlineto{\pgfqpoint{4.610848in}{1.778007in}}%
\pgfpathlineto{\pgfqpoint{4.610848in}{1.705952in}}%
\pgfpathlineto{\pgfqpoint{4.601912in}{1.705952in}}%
\pgfpathlineto{\pgfqpoint{4.601912in}{1.778007in}}%
\pgfpathclose%
\pgfusepath{fill}%
\end{pgfscope}%
\begin{pgfscope}%
\pgfpathrectangle{\pgfqpoint{3.722897in}{0.857143in}}{\pgfqpoint{2.627103in}{1.813434in}}%
\pgfusepath{clip}%
\pgfsetbuttcap%
\pgfsetmiterjoin%
\definecolor{currentfill}{rgb}{0.302379,0.450282,0.300122}%
\pgfsetfillcolor{currentfill}%
\pgfsetlinewidth{0.000000pt}%
\definecolor{currentstroke}{rgb}{0.000000,0.000000,0.000000}%
\pgfsetstrokecolor{currentstroke}%
\pgfsetstrokeopacity{0.000000}%
\pgfsetdash{}{0pt}%
\pgfpathmoveto{\pgfqpoint{4.613083in}{1.794736in}}%
\pgfpathlineto{\pgfqpoint{4.622019in}{1.794736in}}%
\pgfpathlineto{\pgfqpoint{4.622019in}{1.729078in}}%
\pgfpathlineto{\pgfqpoint{4.613083in}{1.729078in}}%
\pgfpathlineto{\pgfqpoint{4.613083in}{1.794736in}}%
\pgfpathclose%
\pgfusepath{fill}%
\end{pgfscope}%
\begin{pgfscope}%
\pgfpathrectangle{\pgfqpoint{3.722897in}{0.857143in}}{\pgfqpoint{2.627103in}{1.813434in}}%
\pgfusepath{clip}%
\pgfsetbuttcap%
\pgfsetmiterjoin%
\definecolor{currentfill}{rgb}{0.302379,0.450282,0.300122}%
\pgfsetfillcolor{currentfill}%
\pgfsetlinewidth{0.000000pt}%
\definecolor{currentstroke}{rgb}{0.000000,0.000000,0.000000}%
\pgfsetstrokecolor{currentstroke}%
\pgfsetstrokeopacity{0.000000}%
\pgfsetdash{}{0pt}%
\pgfpathmoveto{\pgfqpoint{4.624253in}{1.780348in}}%
\pgfpathlineto{\pgfqpoint{4.633190in}{1.780348in}}%
\pgfpathlineto{\pgfqpoint{4.633190in}{1.724197in}}%
\pgfpathlineto{\pgfqpoint{4.624253in}{1.724197in}}%
\pgfpathlineto{\pgfqpoint{4.624253in}{1.780348in}}%
\pgfpathclose%
\pgfusepath{fill}%
\end{pgfscope}%
\begin{pgfscope}%
\pgfpathrectangle{\pgfqpoint{3.722897in}{0.857143in}}{\pgfqpoint{2.627103in}{1.813434in}}%
\pgfusepath{clip}%
\pgfsetbuttcap%
\pgfsetmiterjoin%
\definecolor{currentfill}{rgb}{0.302379,0.450282,0.300122}%
\pgfsetfillcolor{currentfill}%
\pgfsetlinewidth{0.000000pt}%
\definecolor{currentstroke}{rgb}{0.000000,0.000000,0.000000}%
\pgfsetstrokecolor{currentstroke}%
\pgfsetstrokeopacity{0.000000}%
\pgfsetdash{}{0pt}%
\pgfpathmoveto{\pgfqpoint{4.635424in}{1.786291in}}%
\pgfpathlineto{\pgfqpoint{4.644360in}{1.786291in}}%
\pgfpathlineto{\pgfqpoint{4.644360in}{1.741595in}}%
\pgfpathlineto{\pgfqpoint{4.635424in}{1.741595in}}%
\pgfpathlineto{\pgfqpoint{4.635424in}{1.786291in}}%
\pgfpathclose%
\pgfusepath{fill}%
\end{pgfscope}%
\begin{pgfscope}%
\pgfpathrectangle{\pgfqpoint{3.722897in}{0.857143in}}{\pgfqpoint{2.627103in}{1.813434in}}%
\pgfusepath{clip}%
\pgfsetbuttcap%
\pgfsetmiterjoin%
\definecolor{currentfill}{rgb}{0.302379,0.450282,0.300122}%
\pgfsetfillcolor{currentfill}%
\pgfsetlinewidth{0.000000pt}%
\definecolor{currentstroke}{rgb}{0.000000,0.000000,0.000000}%
\pgfsetstrokecolor{currentstroke}%
\pgfsetstrokeopacity{0.000000}%
\pgfsetdash{}{0pt}%
\pgfpathmoveto{\pgfqpoint{4.646594in}{1.782089in}}%
\pgfpathlineto{\pgfqpoint{4.655531in}{1.782089in}}%
\pgfpathlineto{\pgfqpoint{4.655531in}{1.749603in}}%
\pgfpathlineto{\pgfqpoint{4.646594in}{1.749603in}}%
\pgfpathlineto{\pgfqpoint{4.646594in}{1.782089in}}%
\pgfpathclose%
\pgfusepath{fill}%
\end{pgfscope}%
\begin{pgfscope}%
\pgfpathrectangle{\pgfqpoint{3.722897in}{0.857143in}}{\pgfqpoint{2.627103in}{1.813434in}}%
\pgfusepath{clip}%
\pgfsetbuttcap%
\pgfsetmiterjoin%
\definecolor{currentfill}{rgb}{0.302379,0.450282,0.300122}%
\pgfsetfillcolor{currentfill}%
\pgfsetlinewidth{0.000000pt}%
\definecolor{currentstroke}{rgb}{0.000000,0.000000,0.000000}%
\pgfsetstrokecolor{currentstroke}%
\pgfsetstrokeopacity{0.000000}%
\pgfsetdash{}{0pt}%
\pgfpathmoveto{\pgfqpoint{4.657765in}{1.777754in}}%
\pgfpathlineto{\pgfqpoint{4.666701in}{1.777754in}}%
\pgfpathlineto{\pgfqpoint{4.666701in}{1.757684in}}%
\pgfpathlineto{\pgfqpoint{4.657765in}{1.757684in}}%
\pgfpathlineto{\pgfqpoint{4.657765in}{1.777754in}}%
\pgfpathclose%
\pgfusepath{fill}%
\end{pgfscope}%
\begin{pgfscope}%
\pgfpathrectangle{\pgfqpoint{3.722897in}{0.857143in}}{\pgfqpoint{2.627103in}{1.813434in}}%
\pgfusepath{clip}%
\pgfsetbuttcap%
\pgfsetmiterjoin%
\definecolor{currentfill}{rgb}{0.302379,0.450282,0.300122}%
\pgfsetfillcolor{currentfill}%
\pgfsetlinewidth{0.000000pt}%
\definecolor{currentstroke}{rgb}{0.000000,0.000000,0.000000}%
\pgfsetstrokecolor{currentstroke}%
\pgfsetstrokeopacity{0.000000}%
\pgfsetdash{}{0pt}%
\pgfpathmoveto{\pgfqpoint{4.668936in}{1.767984in}}%
\pgfpathlineto{\pgfqpoint{4.677872in}{1.767984in}}%
\pgfpathlineto{\pgfqpoint{4.677872in}{1.762383in}}%
\pgfpathlineto{\pgfqpoint{4.668936in}{1.762383in}}%
\pgfpathlineto{\pgfqpoint{4.668936in}{1.767984in}}%
\pgfpathclose%
\pgfusepath{fill}%
\end{pgfscope}%
\begin{pgfscope}%
\pgfpathrectangle{\pgfqpoint{3.722897in}{0.857143in}}{\pgfqpoint{2.627103in}{1.813434in}}%
\pgfusepath{clip}%
\pgfsetbuttcap%
\pgfsetmiterjoin%
\definecolor{currentfill}{rgb}{0.302379,0.450282,0.300122}%
\pgfsetfillcolor{currentfill}%
\pgfsetlinewidth{0.000000pt}%
\definecolor{currentstroke}{rgb}{0.000000,0.000000,0.000000}%
\pgfsetstrokecolor{currentstroke}%
\pgfsetstrokeopacity{0.000000}%
\pgfsetdash{}{0pt}%
\pgfpathmoveto{\pgfqpoint{4.680106in}{1.813884in}}%
\pgfpathlineto{\pgfqpoint{4.689043in}{1.813884in}}%
\pgfpathlineto{\pgfqpoint{4.689043in}{1.822211in}}%
\pgfpathlineto{\pgfqpoint{4.680106in}{1.822211in}}%
\pgfpathlineto{\pgfqpoint{4.680106in}{1.813884in}}%
\pgfpathclose%
\pgfusepath{fill}%
\end{pgfscope}%
\begin{pgfscope}%
\pgfpathrectangle{\pgfqpoint{3.722897in}{0.857143in}}{\pgfqpoint{2.627103in}{1.813434in}}%
\pgfusepath{clip}%
\pgfsetbuttcap%
\pgfsetmiterjoin%
\definecolor{currentfill}{rgb}{0.302379,0.450282,0.300122}%
\pgfsetfillcolor{currentfill}%
\pgfsetlinewidth{0.000000pt}%
\definecolor{currentstroke}{rgb}{0.000000,0.000000,0.000000}%
\pgfsetstrokecolor{currentstroke}%
\pgfsetstrokeopacity{0.000000}%
\pgfsetdash{}{0pt}%
\pgfpathmoveto{\pgfqpoint{4.691277in}{1.813884in}}%
\pgfpathlineto{\pgfqpoint{4.700213in}{1.813884in}}%
\pgfpathlineto{\pgfqpoint{4.700213in}{1.836197in}}%
\pgfpathlineto{\pgfqpoint{4.691277in}{1.836197in}}%
\pgfpathlineto{\pgfqpoint{4.691277in}{1.813884in}}%
\pgfpathclose%
\pgfusepath{fill}%
\end{pgfscope}%
\begin{pgfscope}%
\pgfpathrectangle{\pgfqpoint{3.722897in}{0.857143in}}{\pgfqpoint{2.627103in}{1.813434in}}%
\pgfusepath{clip}%
\pgfsetbuttcap%
\pgfsetmiterjoin%
\definecolor{currentfill}{rgb}{0.302379,0.450282,0.300122}%
\pgfsetfillcolor{currentfill}%
\pgfsetlinewidth{0.000000pt}%
\definecolor{currentstroke}{rgb}{0.000000,0.000000,0.000000}%
\pgfsetstrokecolor{currentstroke}%
\pgfsetstrokeopacity{0.000000}%
\pgfsetdash{}{0pt}%
\pgfpathmoveto{\pgfqpoint{4.702447in}{1.813884in}}%
\pgfpathlineto{\pgfqpoint{4.711384in}{1.813884in}}%
\pgfpathlineto{\pgfqpoint{4.711384in}{1.850266in}}%
\pgfpathlineto{\pgfqpoint{4.702447in}{1.850266in}}%
\pgfpathlineto{\pgfqpoint{4.702447in}{1.813884in}}%
\pgfpathclose%
\pgfusepath{fill}%
\end{pgfscope}%
\begin{pgfscope}%
\pgfpathrectangle{\pgfqpoint{3.722897in}{0.857143in}}{\pgfqpoint{2.627103in}{1.813434in}}%
\pgfusepath{clip}%
\pgfsetbuttcap%
\pgfsetmiterjoin%
\definecolor{currentfill}{rgb}{0.302379,0.450282,0.300122}%
\pgfsetfillcolor{currentfill}%
\pgfsetlinewidth{0.000000pt}%
\definecolor{currentstroke}{rgb}{0.000000,0.000000,0.000000}%
\pgfsetstrokecolor{currentstroke}%
\pgfsetstrokeopacity{0.000000}%
\pgfsetdash{}{0pt}%
\pgfpathmoveto{\pgfqpoint{4.713618in}{1.813884in}}%
\pgfpathlineto{\pgfqpoint{4.722554in}{1.813884in}}%
\pgfpathlineto{\pgfqpoint{4.722554in}{1.865142in}}%
\pgfpathlineto{\pgfqpoint{4.713618in}{1.865142in}}%
\pgfpathlineto{\pgfqpoint{4.713618in}{1.813884in}}%
\pgfpathclose%
\pgfusepath{fill}%
\end{pgfscope}%
\begin{pgfscope}%
\pgfpathrectangle{\pgfqpoint{3.722897in}{0.857143in}}{\pgfqpoint{2.627103in}{1.813434in}}%
\pgfusepath{clip}%
\pgfsetbuttcap%
\pgfsetmiterjoin%
\definecolor{currentfill}{rgb}{0.302379,0.450282,0.300122}%
\pgfsetfillcolor{currentfill}%
\pgfsetlinewidth{0.000000pt}%
\definecolor{currentstroke}{rgb}{0.000000,0.000000,0.000000}%
\pgfsetstrokecolor{currentstroke}%
\pgfsetstrokeopacity{0.000000}%
\pgfsetdash{}{0pt}%
\pgfpathmoveto{\pgfqpoint{4.724789in}{1.813884in}}%
\pgfpathlineto{\pgfqpoint{4.733725in}{1.813884in}}%
\pgfpathlineto{\pgfqpoint{4.733725in}{1.879326in}}%
\pgfpathlineto{\pgfqpoint{4.724789in}{1.879326in}}%
\pgfpathlineto{\pgfqpoint{4.724789in}{1.813884in}}%
\pgfpathclose%
\pgfusepath{fill}%
\end{pgfscope}%
\begin{pgfscope}%
\pgfpathrectangle{\pgfqpoint{3.722897in}{0.857143in}}{\pgfqpoint{2.627103in}{1.813434in}}%
\pgfusepath{clip}%
\pgfsetbuttcap%
\pgfsetmiterjoin%
\definecolor{currentfill}{rgb}{0.302379,0.450282,0.300122}%
\pgfsetfillcolor{currentfill}%
\pgfsetlinewidth{0.000000pt}%
\definecolor{currentstroke}{rgb}{0.000000,0.000000,0.000000}%
\pgfsetstrokecolor{currentstroke}%
\pgfsetstrokeopacity{0.000000}%
\pgfsetdash{}{0pt}%
\pgfpathmoveto{\pgfqpoint{4.735959in}{1.813884in}}%
\pgfpathlineto{\pgfqpoint{4.744896in}{1.813884in}}%
\pgfpathlineto{\pgfqpoint{4.744896in}{1.892761in}}%
\pgfpathlineto{\pgfqpoint{4.735959in}{1.892761in}}%
\pgfpathlineto{\pgfqpoint{4.735959in}{1.813884in}}%
\pgfpathclose%
\pgfusepath{fill}%
\end{pgfscope}%
\begin{pgfscope}%
\pgfpathrectangle{\pgfqpoint{3.722897in}{0.857143in}}{\pgfqpoint{2.627103in}{1.813434in}}%
\pgfusepath{clip}%
\pgfsetbuttcap%
\pgfsetmiterjoin%
\definecolor{currentfill}{rgb}{0.302379,0.450282,0.300122}%
\pgfsetfillcolor{currentfill}%
\pgfsetlinewidth{0.000000pt}%
\definecolor{currentstroke}{rgb}{0.000000,0.000000,0.000000}%
\pgfsetstrokecolor{currentstroke}%
\pgfsetstrokeopacity{0.000000}%
\pgfsetdash{}{0pt}%
\pgfpathmoveto{\pgfqpoint{4.747130in}{1.813884in}}%
\pgfpathlineto{\pgfqpoint{4.756066in}{1.813884in}}%
\pgfpathlineto{\pgfqpoint{4.756066in}{1.906169in}}%
\pgfpathlineto{\pgfqpoint{4.747130in}{1.906169in}}%
\pgfpathlineto{\pgfqpoint{4.747130in}{1.813884in}}%
\pgfpathclose%
\pgfusepath{fill}%
\end{pgfscope}%
\begin{pgfscope}%
\pgfpathrectangle{\pgfqpoint{3.722897in}{0.857143in}}{\pgfqpoint{2.627103in}{1.813434in}}%
\pgfusepath{clip}%
\pgfsetbuttcap%
\pgfsetmiterjoin%
\definecolor{currentfill}{rgb}{0.302379,0.450282,0.300122}%
\pgfsetfillcolor{currentfill}%
\pgfsetlinewidth{0.000000pt}%
\definecolor{currentstroke}{rgb}{0.000000,0.000000,0.000000}%
\pgfsetstrokecolor{currentstroke}%
\pgfsetstrokeopacity{0.000000}%
\pgfsetdash{}{0pt}%
\pgfpathmoveto{\pgfqpoint{4.758300in}{1.813884in}}%
\pgfpathlineto{\pgfqpoint{4.767237in}{1.813884in}}%
\pgfpathlineto{\pgfqpoint{4.767237in}{1.918280in}}%
\pgfpathlineto{\pgfqpoint{4.758300in}{1.918280in}}%
\pgfpathlineto{\pgfqpoint{4.758300in}{1.813884in}}%
\pgfpathclose%
\pgfusepath{fill}%
\end{pgfscope}%
\begin{pgfscope}%
\pgfpathrectangle{\pgfqpoint{3.722897in}{0.857143in}}{\pgfqpoint{2.627103in}{1.813434in}}%
\pgfusepath{clip}%
\pgfsetbuttcap%
\pgfsetmiterjoin%
\definecolor{currentfill}{rgb}{0.302379,0.450282,0.300122}%
\pgfsetfillcolor{currentfill}%
\pgfsetlinewidth{0.000000pt}%
\definecolor{currentstroke}{rgb}{0.000000,0.000000,0.000000}%
\pgfsetstrokecolor{currentstroke}%
\pgfsetstrokeopacity{0.000000}%
\pgfsetdash{}{0pt}%
\pgfpathmoveto{\pgfqpoint{4.769471in}{1.813884in}}%
\pgfpathlineto{\pgfqpoint{4.778408in}{1.813884in}}%
\pgfpathlineto{\pgfqpoint{4.778408in}{1.932212in}}%
\pgfpathlineto{\pgfqpoint{4.769471in}{1.932212in}}%
\pgfpathlineto{\pgfqpoint{4.769471in}{1.813884in}}%
\pgfpathclose%
\pgfusepath{fill}%
\end{pgfscope}%
\begin{pgfscope}%
\pgfpathrectangle{\pgfqpoint{3.722897in}{0.857143in}}{\pgfqpoint{2.627103in}{1.813434in}}%
\pgfusepath{clip}%
\pgfsetbuttcap%
\pgfsetmiterjoin%
\definecolor{currentfill}{rgb}{0.302379,0.450282,0.300122}%
\pgfsetfillcolor{currentfill}%
\pgfsetlinewidth{0.000000pt}%
\definecolor{currentstroke}{rgb}{0.000000,0.000000,0.000000}%
\pgfsetstrokecolor{currentstroke}%
\pgfsetstrokeopacity{0.000000}%
\pgfsetdash{}{0pt}%
\pgfpathmoveto{\pgfqpoint{4.780642in}{1.813884in}}%
\pgfpathlineto{\pgfqpoint{4.789578in}{1.813884in}}%
\pgfpathlineto{\pgfqpoint{4.789578in}{1.949412in}}%
\pgfpathlineto{\pgfqpoint{4.780642in}{1.949412in}}%
\pgfpathlineto{\pgfqpoint{4.780642in}{1.813884in}}%
\pgfpathclose%
\pgfusepath{fill}%
\end{pgfscope}%
\begin{pgfscope}%
\pgfpathrectangle{\pgfqpoint{3.722897in}{0.857143in}}{\pgfqpoint{2.627103in}{1.813434in}}%
\pgfusepath{clip}%
\pgfsetbuttcap%
\pgfsetmiterjoin%
\definecolor{currentfill}{rgb}{0.302379,0.450282,0.300122}%
\pgfsetfillcolor{currentfill}%
\pgfsetlinewidth{0.000000pt}%
\definecolor{currentstroke}{rgb}{0.000000,0.000000,0.000000}%
\pgfsetstrokecolor{currentstroke}%
\pgfsetstrokeopacity{0.000000}%
\pgfsetdash{}{0pt}%
\pgfpathmoveto{\pgfqpoint{4.791812in}{1.813884in}}%
\pgfpathlineto{\pgfqpoint{4.800749in}{1.813884in}}%
\pgfpathlineto{\pgfqpoint{4.800749in}{1.965400in}}%
\pgfpathlineto{\pgfqpoint{4.791812in}{1.965400in}}%
\pgfpathlineto{\pgfqpoint{4.791812in}{1.813884in}}%
\pgfpathclose%
\pgfusepath{fill}%
\end{pgfscope}%
\begin{pgfscope}%
\pgfpathrectangle{\pgfqpoint{3.722897in}{0.857143in}}{\pgfqpoint{2.627103in}{1.813434in}}%
\pgfusepath{clip}%
\pgfsetbuttcap%
\pgfsetmiterjoin%
\definecolor{currentfill}{rgb}{0.302379,0.450282,0.300122}%
\pgfsetfillcolor{currentfill}%
\pgfsetlinewidth{0.000000pt}%
\definecolor{currentstroke}{rgb}{0.000000,0.000000,0.000000}%
\pgfsetstrokecolor{currentstroke}%
\pgfsetstrokeopacity{0.000000}%
\pgfsetdash{}{0pt}%
\pgfpathmoveto{\pgfqpoint{4.802983in}{1.813884in}}%
\pgfpathlineto{\pgfqpoint{4.811919in}{1.813884in}}%
\pgfpathlineto{\pgfqpoint{4.811919in}{1.981606in}}%
\pgfpathlineto{\pgfqpoint{4.802983in}{1.981606in}}%
\pgfpathlineto{\pgfqpoint{4.802983in}{1.813884in}}%
\pgfpathclose%
\pgfusepath{fill}%
\end{pgfscope}%
\begin{pgfscope}%
\pgfpathrectangle{\pgfqpoint{3.722897in}{0.857143in}}{\pgfqpoint{2.627103in}{1.813434in}}%
\pgfusepath{clip}%
\pgfsetbuttcap%
\pgfsetmiterjoin%
\definecolor{currentfill}{rgb}{0.302379,0.450282,0.300122}%
\pgfsetfillcolor{currentfill}%
\pgfsetlinewidth{0.000000pt}%
\definecolor{currentstroke}{rgb}{0.000000,0.000000,0.000000}%
\pgfsetstrokecolor{currentstroke}%
\pgfsetstrokeopacity{0.000000}%
\pgfsetdash{}{0pt}%
\pgfpathmoveto{\pgfqpoint{4.814153in}{1.813884in}}%
\pgfpathlineto{\pgfqpoint{4.823090in}{1.813884in}}%
\pgfpathlineto{\pgfqpoint{4.823090in}{1.999428in}}%
\pgfpathlineto{\pgfqpoint{4.814153in}{1.999428in}}%
\pgfpathlineto{\pgfqpoint{4.814153in}{1.813884in}}%
\pgfpathclose%
\pgfusepath{fill}%
\end{pgfscope}%
\begin{pgfscope}%
\pgfpathrectangle{\pgfqpoint{3.722897in}{0.857143in}}{\pgfqpoint{2.627103in}{1.813434in}}%
\pgfusepath{clip}%
\pgfsetbuttcap%
\pgfsetmiterjoin%
\definecolor{currentfill}{rgb}{0.302379,0.450282,0.300122}%
\pgfsetfillcolor{currentfill}%
\pgfsetlinewidth{0.000000pt}%
\definecolor{currentstroke}{rgb}{0.000000,0.000000,0.000000}%
\pgfsetstrokecolor{currentstroke}%
\pgfsetstrokeopacity{0.000000}%
\pgfsetdash{}{0pt}%
\pgfpathmoveto{\pgfqpoint{4.825324in}{1.813884in}}%
\pgfpathlineto{\pgfqpoint{4.834261in}{1.813884in}}%
\pgfpathlineto{\pgfqpoint{4.834261in}{2.018580in}}%
\pgfpathlineto{\pgfqpoint{4.825324in}{2.018580in}}%
\pgfpathlineto{\pgfqpoint{4.825324in}{1.813884in}}%
\pgfpathclose%
\pgfusepath{fill}%
\end{pgfscope}%
\begin{pgfscope}%
\pgfpathrectangle{\pgfqpoint{3.722897in}{0.857143in}}{\pgfqpoint{2.627103in}{1.813434in}}%
\pgfusepath{clip}%
\pgfsetbuttcap%
\pgfsetmiterjoin%
\definecolor{currentfill}{rgb}{0.302379,0.450282,0.300122}%
\pgfsetfillcolor{currentfill}%
\pgfsetlinewidth{0.000000pt}%
\definecolor{currentstroke}{rgb}{0.000000,0.000000,0.000000}%
\pgfsetstrokecolor{currentstroke}%
\pgfsetstrokeopacity{0.000000}%
\pgfsetdash{}{0pt}%
\pgfpathmoveto{\pgfqpoint{4.836495in}{1.818805in}}%
\pgfpathlineto{\pgfqpoint{4.845431in}{1.818805in}}%
\pgfpathlineto{\pgfqpoint{4.845431in}{2.042259in}}%
\pgfpathlineto{\pgfqpoint{4.836495in}{2.042259in}}%
\pgfpathlineto{\pgfqpoint{4.836495in}{1.818805in}}%
\pgfpathclose%
\pgfusepath{fill}%
\end{pgfscope}%
\begin{pgfscope}%
\pgfpathrectangle{\pgfqpoint{3.722897in}{0.857143in}}{\pgfqpoint{2.627103in}{1.813434in}}%
\pgfusepath{clip}%
\pgfsetbuttcap%
\pgfsetmiterjoin%
\definecolor{currentfill}{rgb}{0.302379,0.450282,0.300122}%
\pgfsetfillcolor{currentfill}%
\pgfsetlinewidth{0.000000pt}%
\definecolor{currentstroke}{rgb}{0.000000,0.000000,0.000000}%
\pgfsetstrokecolor{currentstroke}%
\pgfsetstrokeopacity{0.000000}%
\pgfsetdash{}{0pt}%
\pgfpathmoveto{\pgfqpoint{4.847665in}{1.815758in}}%
\pgfpathlineto{\pgfqpoint{4.856602in}{1.815758in}}%
\pgfpathlineto{\pgfqpoint{4.856602in}{2.058937in}}%
\pgfpathlineto{\pgfqpoint{4.847665in}{2.058937in}}%
\pgfpathlineto{\pgfqpoint{4.847665in}{1.815758in}}%
\pgfpathclose%
\pgfusepath{fill}%
\end{pgfscope}%
\begin{pgfscope}%
\pgfpathrectangle{\pgfqpoint{3.722897in}{0.857143in}}{\pgfqpoint{2.627103in}{1.813434in}}%
\pgfusepath{clip}%
\pgfsetbuttcap%
\pgfsetmiterjoin%
\definecolor{currentfill}{rgb}{0.302379,0.450282,0.300122}%
\pgfsetfillcolor{currentfill}%
\pgfsetlinewidth{0.000000pt}%
\definecolor{currentstroke}{rgb}{0.000000,0.000000,0.000000}%
\pgfsetstrokecolor{currentstroke}%
\pgfsetstrokeopacity{0.000000}%
\pgfsetdash{}{0pt}%
\pgfpathmoveto{\pgfqpoint{4.858836in}{1.819354in}}%
\pgfpathlineto{\pgfqpoint{4.867772in}{1.819354in}}%
\pgfpathlineto{\pgfqpoint{4.867772in}{2.083113in}}%
\pgfpathlineto{\pgfqpoint{4.858836in}{2.083113in}}%
\pgfpathlineto{\pgfqpoint{4.858836in}{1.819354in}}%
\pgfpathclose%
\pgfusepath{fill}%
\end{pgfscope}%
\begin{pgfscope}%
\pgfpathrectangle{\pgfqpoint{3.722897in}{0.857143in}}{\pgfqpoint{2.627103in}{1.813434in}}%
\pgfusepath{clip}%
\pgfsetbuttcap%
\pgfsetmiterjoin%
\definecolor{currentfill}{rgb}{0.302379,0.450282,0.300122}%
\pgfsetfillcolor{currentfill}%
\pgfsetlinewidth{0.000000pt}%
\definecolor{currentstroke}{rgb}{0.000000,0.000000,0.000000}%
\pgfsetstrokecolor{currentstroke}%
\pgfsetstrokeopacity{0.000000}%
\pgfsetdash{}{0pt}%
\pgfpathmoveto{\pgfqpoint{4.870006in}{1.830216in}}%
\pgfpathlineto{\pgfqpoint{4.878943in}{1.830216in}}%
\pgfpathlineto{\pgfqpoint{4.878943in}{2.112997in}}%
\pgfpathlineto{\pgfqpoint{4.870006in}{2.112997in}}%
\pgfpathlineto{\pgfqpoint{4.870006in}{1.830216in}}%
\pgfpathclose%
\pgfusepath{fill}%
\end{pgfscope}%
\begin{pgfscope}%
\pgfpathrectangle{\pgfqpoint{3.722897in}{0.857143in}}{\pgfqpoint{2.627103in}{1.813434in}}%
\pgfusepath{clip}%
\pgfsetbuttcap%
\pgfsetmiterjoin%
\definecolor{currentfill}{rgb}{0.302379,0.450282,0.300122}%
\pgfsetfillcolor{currentfill}%
\pgfsetlinewidth{0.000000pt}%
\definecolor{currentstroke}{rgb}{0.000000,0.000000,0.000000}%
\pgfsetstrokecolor{currentstroke}%
\pgfsetstrokeopacity{0.000000}%
\pgfsetdash{}{0pt}%
\pgfpathmoveto{\pgfqpoint{4.881177in}{1.832039in}}%
\pgfpathlineto{\pgfqpoint{4.890114in}{1.832039in}}%
\pgfpathlineto{\pgfqpoint{4.890114in}{2.131225in}}%
\pgfpathlineto{\pgfqpoint{4.881177in}{2.131225in}}%
\pgfpathlineto{\pgfqpoint{4.881177in}{1.832039in}}%
\pgfpathclose%
\pgfusepath{fill}%
\end{pgfscope}%
\begin{pgfscope}%
\pgfpathrectangle{\pgfqpoint{3.722897in}{0.857143in}}{\pgfqpoint{2.627103in}{1.813434in}}%
\pgfusepath{clip}%
\pgfsetbuttcap%
\pgfsetmiterjoin%
\definecolor{currentfill}{rgb}{0.302379,0.450282,0.300122}%
\pgfsetfillcolor{currentfill}%
\pgfsetlinewidth{0.000000pt}%
\definecolor{currentstroke}{rgb}{0.000000,0.000000,0.000000}%
\pgfsetstrokecolor{currentstroke}%
\pgfsetstrokeopacity{0.000000}%
\pgfsetdash{}{0pt}%
\pgfpathmoveto{\pgfqpoint{4.892348in}{1.819046in}}%
\pgfpathlineto{\pgfqpoint{4.901284in}{1.819046in}}%
\pgfpathlineto{\pgfqpoint{4.901284in}{2.134674in}}%
\pgfpathlineto{\pgfqpoint{4.892348in}{2.134674in}}%
\pgfpathlineto{\pgfqpoint{4.892348in}{1.819046in}}%
\pgfpathclose%
\pgfusepath{fill}%
\end{pgfscope}%
\begin{pgfscope}%
\pgfpathrectangle{\pgfqpoint{3.722897in}{0.857143in}}{\pgfqpoint{2.627103in}{1.813434in}}%
\pgfusepath{clip}%
\pgfsetbuttcap%
\pgfsetmiterjoin%
\definecolor{currentfill}{rgb}{0.302379,0.450282,0.300122}%
\pgfsetfillcolor{currentfill}%
\pgfsetlinewidth{0.000000pt}%
\definecolor{currentstroke}{rgb}{0.000000,0.000000,0.000000}%
\pgfsetstrokecolor{currentstroke}%
\pgfsetstrokeopacity{0.000000}%
\pgfsetdash{}{0pt}%
\pgfpathmoveto{\pgfqpoint{4.903518in}{1.817615in}}%
\pgfpathlineto{\pgfqpoint{4.912455in}{1.817615in}}%
\pgfpathlineto{\pgfqpoint{4.912455in}{2.149088in}}%
\pgfpathlineto{\pgfqpoint{4.903518in}{2.149088in}}%
\pgfpathlineto{\pgfqpoint{4.903518in}{1.817615in}}%
\pgfpathclose%
\pgfusepath{fill}%
\end{pgfscope}%
\begin{pgfscope}%
\pgfpathrectangle{\pgfqpoint{3.722897in}{0.857143in}}{\pgfqpoint{2.627103in}{1.813434in}}%
\pgfusepath{clip}%
\pgfsetbuttcap%
\pgfsetmiterjoin%
\definecolor{currentfill}{rgb}{0.302379,0.450282,0.300122}%
\pgfsetfillcolor{currentfill}%
\pgfsetlinewidth{0.000000pt}%
\definecolor{currentstroke}{rgb}{0.000000,0.000000,0.000000}%
\pgfsetstrokecolor{currentstroke}%
\pgfsetstrokeopacity{0.000000}%
\pgfsetdash{}{0pt}%
\pgfpathmoveto{\pgfqpoint{4.914689in}{1.829647in}}%
\pgfpathlineto{\pgfqpoint{4.923625in}{1.829647in}}%
\pgfpathlineto{\pgfqpoint{4.923625in}{2.174312in}}%
\pgfpathlineto{\pgfqpoint{4.914689in}{2.174312in}}%
\pgfpathlineto{\pgfqpoint{4.914689in}{1.829647in}}%
\pgfpathclose%
\pgfusepath{fill}%
\end{pgfscope}%
\begin{pgfscope}%
\pgfpathrectangle{\pgfqpoint{3.722897in}{0.857143in}}{\pgfqpoint{2.627103in}{1.813434in}}%
\pgfusepath{clip}%
\pgfsetbuttcap%
\pgfsetmiterjoin%
\definecolor{currentfill}{rgb}{0.302379,0.450282,0.300122}%
\pgfsetfillcolor{currentfill}%
\pgfsetlinewidth{0.000000pt}%
\definecolor{currentstroke}{rgb}{0.000000,0.000000,0.000000}%
\pgfsetstrokecolor{currentstroke}%
\pgfsetstrokeopacity{0.000000}%
\pgfsetdash{}{0pt}%
\pgfpathmoveto{\pgfqpoint{4.925860in}{1.834348in}}%
\pgfpathlineto{\pgfqpoint{4.934796in}{1.834348in}}%
\pgfpathlineto{\pgfqpoint{4.934796in}{2.190037in}}%
\pgfpathlineto{\pgfqpoint{4.925860in}{2.190037in}}%
\pgfpathlineto{\pgfqpoint{4.925860in}{1.834348in}}%
\pgfpathclose%
\pgfusepath{fill}%
\end{pgfscope}%
\begin{pgfscope}%
\pgfpathrectangle{\pgfqpoint{3.722897in}{0.857143in}}{\pgfqpoint{2.627103in}{1.813434in}}%
\pgfusepath{clip}%
\pgfsetbuttcap%
\pgfsetmiterjoin%
\definecolor{currentfill}{rgb}{0.302379,0.450282,0.300122}%
\pgfsetfillcolor{currentfill}%
\pgfsetlinewidth{0.000000pt}%
\definecolor{currentstroke}{rgb}{0.000000,0.000000,0.000000}%
\pgfsetstrokecolor{currentstroke}%
\pgfsetstrokeopacity{0.000000}%
\pgfsetdash{}{0pt}%
\pgfpathmoveto{\pgfqpoint{4.937030in}{1.821808in}}%
\pgfpathlineto{\pgfqpoint{4.945967in}{1.821808in}}%
\pgfpathlineto{\pgfqpoint{4.945967in}{2.184939in}}%
\pgfpathlineto{\pgfqpoint{4.937030in}{2.184939in}}%
\pgfpathlineto{\pgfqpoint{4.937030in}{1.821808in}}%
\pgfpathclose%
\pgfusepath{fill}%
\end{pgfscope}%
\begin{pgfscope}%
\pgfpathrectangle{\pgfqpoint{3.722897in}{0.857143in}}{\pgfqpoint{2.627103in}{1.813434in}}%
\pgfusepath{clip}%
\pgfsetbuttcap%
\pgfsetmiterjoin%
\definecolor{currentfill}{rgb}{0.302379,0.450282,0.300122}%
\pgfsetfillcolor{currentfill}%
\pgfsetlinewidth{0.000000pt}%
\definecolor{currentstroke}{rgb}{0.000000,0.000000,0.000000}%
\pgfsetstrokecolor{currentstroke}%
\pgfsetstrokeopacity{0.000000}%
\pgfsetdash{}{0pt}%
\pgfpathmoveto{\pgfqpoint{4.948201in}{1.831061in}}%
\pgfpathlineto{\pgfqpoint{4.957137in}{1.831061in}}%
\pgfpathlineto{\pgfqpoint{4.957137in}{2.198160in}}%
\pgfpathlineto{\pgfqpoint{4.948201in}{2.198160in}}%
\pgfpathlineto{\pgfqpoint{4.948201in}{1.831061in}}%
\pgfpathclose%
\pgfusepath{fill}%
\end{pgfscope}%
\begin{pgfscope}%
\pgfpathrectangle{\pgfqpoint{3.722897in}{0.857143in}}{\pgfqpoint{2.627103in}{1.813434in}}%
\pgfusepath{clip}%
\pgfsetbuttcap%
\pgfsetmiterjoin%
\definecolor{currentfill}{rgb}{0.302379,0.450282,0.300122}%
\pgfsetfillcolor{currentfill}%
\pgfsetlinewidth{0.000000pt}%
\definecolor{currentstroke}{rgb}{0.000000,0.000000,0.000000}%
\pgfsetstrokecolor{currentstroke}%
\pgfsetstrokeopacity{0.000000}%
\pgfsetdash{}{0pt}%
\pgfpathmoveto{\pgfqpoint{4.959371in}{1.828134in}}%
\pgfpathlineto{\pgfqpoint{4.968308in}{1.828134in}}%
\pgfpathlineto{\pgfqpoint{4.968308in}{2.197934in}}%
\pgfpathlineto{\pgfqpoint{4.959371in}{2.197934in}}%
\pgfpathlineto{\pgfqpoint{4.959371in}{1.828134in}}%
\pgfpathclose%
\pgfusepath{fill}%
\end{pgfscope}%
\begin{pgfscope}%
\pgfpathrectangle{\pgfqpoint{3.722897in}{0.857143in}}{\pgfqpoint{2.627103in}{1.813434in}}%
\pgfusepath{clip}%
\pgfsetbuttcap%
\pgfsetmiterjoin%
\definecolor{currentfill}{rgb}{0.302379,0.450282,0.300122}%
\pgfsetfillcolor{currentfill}%
\pgfsetlinewidth{0.000000pt}%
\definecolor{currentstroke}{rgb}{0.000000,0.000000,0.000000}%
\pgfsetstrokecolor{currentstroke}%
\pgfsetstrokeopacity{0.000000}%
\pgfsetdash{}{0pt}%
\pgfpathmoveto{\pgfqpoint{4.970542in}{1.832080in}}%
\pgfpathlineto{\pgfqpoint{4.979478in}{1.832080in}}%
\pgfpathlineto{\pgfqpoint{4.979478in}{2.202923in}}%
\pgfpathlineto{\pgfqpoint{4.970542in}{2.202923in}}%
\pgfpathlineto{\pgfqpoint{4.970542in}{1.832080in}}%
\pgfpathclose%
\pgfusepath{fill}%
\end{pgfscope}%
\begin{pgfscope}%
\pgfpathrectangle{\pgfqpoint{3.722897in}{0.857143in}}{\pgfqpoint{2.627103in}{1.813434in}}%
\pgfusepath{clip}%
\pgfsetbuttcap%
\pgfsetmiterjoin%
\definecolor{currentfill}{rgb}{0.302379,0.450282,0.300122}%
\pgfsetfillcolor{currentfill}%
\pgfsetlinewidth{0.000000pt}%
\definecolor{currentstroke}{rgb}{0.000000,0.000000,0.000000}%
\pgfsetstrokecolor{currentstroke}%
\pgfsetstrokeopacity{0.000000}%
\pgfsetdash{}{0pt}%
\pgfpathmoveto{\pgfqpoint{4.981713in}{1.840360in}}%
\pgfpathlineto{\pgfqpoint{4.990649in}{1.840360in}}%
\pgfpathlineto{\pgfqpoint{4.990649in}{2.211879in}}%
\pgfpathlineto{\pgfqpoint{4.981713in}{2.211879in}}%
\pgfpathlineto{\pgfqpoint{4.981713in}{1.840360in}}%
\pgfpathclose%
\pgfusepath{fill}%
\end{pgfscope}%
\begin{pgfscope}%
\pgfpathrectangle{\pgfqpoint{3.722897in}{0.857143in}}{\pgfqpoint{2.627103in}{1.813434in}}%
\pgfusepath{clip}%
\pgfsetbuttcap%
\pgfsetmiterjoin%
\definecolor{currentfill}{rgb}{0.302379,0.450282,0.300122}%
\pgfsetfillcolor{currentfill}%
\pgfsetlinewidth{0.000000pt}%
\definecolor{currentstroke}{rgb}{0.000000,0.000000,0.000000}%
\pgfsetstrokecolor{currentstroke}%
\pgfsetstrokeopacity{0.000000}%
\pgfsetdash{}{0pt}%
\pgfpathmoveto{\pgfqpoint{4.992883in}{1.834515in}}%
\pgfpathlineto{\pgfqpoint{5.001820in}{1.834515in}}%
\pgfpathlineto{\pgfqpoint{5.001820in}{2.206824in}}%
\pgfpathlineto{\pgfqpoint{4.992883in}{2.206824in}}%
\pgfpathlineto{\pgfqpoint{4.992883in}{1.834515in}}%
\pgfpathclose%
\pgfusepath{fill}%
\end{pgfscope}%
\begin{pgfscope}%
\pgfpathrectangle{\pgfqpoint{3.722897in}{0.857143in}}{\pgfqpoint{2.627103in}{1.813434in}}%
\pgfusepath{clip}%
\pgfsetbuttcap%
\pgfsetmiterjoin%
\definecolor{currentfill}{rgb}{0.302379,0.450282,0.300122}%
\pgfsetfillcolor{currentfill}%
\pgfsetlinewidth{0.000000pt}%
\definecolor{currentstroke}{rgb}{0.000000,0.000000,0.000000}%
\pgfsetstrokecolor{currentstroke}%
\pgfsetstrokeopacity{0.000000}%
\pgfsetdash{}{0pt}%
\pgfpathmoveto{\pgfqpoint{5.004054in}{1.840937in}}%
\pgfpathlineto{\pgfqpoint{5.012990in}{1.840937in}}%
\pgfpathlineto{\pgfqpoint{5.012990in}{2.214204in}}%
\pgfpathlineto{\pgfqpoint{5.004054in}{2.214204in}}%
\pgfpathlineto{\pgfqpoint{5.004054in}{1.840937in}}%
\pgfpathclose%
\pgfusepath{fill}%
\end{pgfscope}%
\begin{pgfscope}%
\pgfpathrectangle{\pgfqpoint{3.722897in}{0.857143in}}{\pgfqpoint{2.627103in}{1.813434in}}%
\pgfusepath{clip}%
\pgfsetbuttcap%
\pgfsetmiterjoin%
\definecolor{currentfill}{rgb}{0.302379,0.450282,0.300122}%
\pgfsetfillcolor{currentfill}%
\pgfsetlinewidth{0.000000pt}%
\definecolor{currentstroke}{rgb}{0.000000,0.000000,0.000000}%
\pgfsetstrokecolor{currentstroke}%
\pgfsetstrokeopacity{0.000000}%
\pgfsetdash{}{0pt}%
\pgfpathmoveto{\pgfqpoint{5.015224in}{1.837694in}}%
\pgfpathlineto{\pgfqpoint{5.024161in}{1.837694in}}%
\pgfpathlineto{\pgfqpoint{5.024161in}{2.212881in}}%
\pgfpathlineto{\pgfqpoint{5.015224in}{2.212881in}}%
\pgfpathlineto{\pgfqpoint{5.015224in}{1.837694in}}%
\pgfpathclose%
\pgfusepath{fill}%
\end{pgfscope}%
\begin{pgfscope}%
\pgfpathrectangle{\pgfqpoint{3.722897in}{0.857143in}}{\pgfqpoint{2.627103in}{1.813434in}}%
\pgfusepath{clip}%
\pgfsetbuttcap%
\pgfsetmiterjoin%
\definecolor{currentfill}{rgb}{0.302379,0.450282,0.300122}%
\pgfsetfillcolor{currentfill}%
\pgfsetlinewidth{0.000000pt}%
\definecolor{currentstroke}{rgb}{0.000000,0.000000,0.000000}%
\pgfsetstrokecolor{currentstroke}%
\pgfsetstrokeopacity{0.000000}%
\pgfsetdash{}{0pt}%
\pgfpathmoveto{\pgfqpoint{5.026395in}{1.853592in}}%
\pgfpathlineto{\pgfqpoint{5.035331in}{1.853592in}}%
\pgfpathlineto{\pgfqpoint{5.035331in}{2.230757in}}%
\pgfpathlineto{\pgfqpoint{5.026395in}{2.230757in}}%
\pgfpathlineto{\pgfqpoint{5.026395in}{1.853592in}}%
\pgfpathclose%
\pgfusepath{fill}%
\end{pgfscope}%
\begin{pgfscope}%
\pgfpathrectangle{\pgfqpoint{3.722897in}{0.857143in}}{\pgfqpoint{2.627103in}{1.813434in}}%
\pgfusepath{clip}%
\pgfsetbuttcap%
\pgfsetmiterjoin%
\definecolor{currentfill}{rgb}{0.302379,0.450282,0.300122}%
\pgfsetfillcolor{currentfill}%
\pgfsetlinewidth{0.000000pt}%
\definecolor{currentstroke}{rgb}{0.000000,0.000000,0.000000}%
\pgfsetstrokecolor{currentstroke}%
\pgfsetstrokeopacity{0.000000}%
\pgfsetdash{}{0pt}%
\pgfpathmoveto{\pgfqpoint{5.037566in}{1.854554in}}%
\pgfpathlineto{\pgfqpoint{5.046502in}{1.854554in}}%
\pgfpathlineto{\pgfqpoint{5.046502in}{2.233794in}}%
\pgfpathlineto{\pgfqpoint{5.037566in}{2.233794in}}%
\pgfpathlineto{\pgfqpoint{5.037566in}{1.854554in}}%
\pgfpathclose%
\pgfusepath{fill}%
\end{pgfscope}%
\begin{pgfscope}%
\pgfpathrectangle{\pgfqpoint{3.722897in}{0.857143in}}{\pgfqpoint{2.627103in}{1.813434in}}%
\pgfusepath{clip}%
\pgfsetbuttcap%
\pgfsetmiterjoin%
\definecolor{currentfill}{rgb}{0.302379,0.450282,0.300122}%
\pgfsetfillcolor{currentfill}%
\pgfsetlinewidth{0.000000pt}%
\definecolor{currentstroke}{rgb}{0.000000,0.000000,0.000000}%
\pgfsetstrokecolor{currentstroke}%
\pgfsetstrokeopacity{0.000000}%
\pgfsetdash{}{0pt}%
\pgfpathmoveto{\pgfqpoint{5.048736in}{1.860403in}}%
\pgfpathlineto{\pgfqpoint{5.057673in}{1.860403in}}%
\pgfpathlineto{\pgfqpoint{5.057673in}{2.242832in}}%
\pgfpathlineto{\pgfqpoint{5.048736in}{2.242832in}}%
\pgfpathlineto{\pgfqpoint{5.048736in}{1.860403in}}%
\pgfpathclose%
\pgfusepath{fill}%
\end{pgfscope}%
\begin{pgfscope}%
\pgfpathrectangle{\pgfqpoint{3.722897in}{0.857143in}}{\pgfqpoint{2.627103in}{1.813434in}}%
\pgfusepath{clip}%
\pgfsetbuttcap%
\pgfsetmiterjoin%
\definecolor{currentfill}{rgb}{0.302379,0.450282,0.300122}%
\pgfsetfillcolor{currentfill}%
\pgfsetlinewidth{0.000000pt}%
\definecolor{currentstroke}{rgb}{0.000000,0.000000,0.000000}%
\pgfsetstrokecolor{currentstroke}%
\pgfsetstrokeopacity{0.000000}%
\pgfsetdash{}{0pt}%
\pgfpathmoveto{\pgfqpoint{5.059907in}{1.866874in}}%
\pgfpathlineto{\pgfqpoint{5.068843in}{1.866874in}}%
\pgfpathlineto{\pgfqpoint{5.068843in}{2.252363in}}%
\pgfpathlineto{\pgfqpoint{5.059907in}{2.252363in}}%
\pgfpathlineto{\pgfqpoint{5.059907in}{1.866874in}}%
\pgfpathclose%
\pgfusepath{fill}%
\end{pgfscope}%
\begin{pgfscope}%
\pgfpathrectangle{\pgfqpoint{3.722897in}{0.857143in}}{\pgfqpoint{2.627103in}{1.813434in}}%
\pgfusepath{clip}%
\pgfsetbuttcap%
\pgfsetmiterjoin%
\definecolor{currentfill}{rgb}{0.302379,0.450282,0.300122}%
\pgfsetfillcolor{currentfill}%
\pgfsetlinewidth{0.000000pt}%
\definecolor{currentstroke}{rgb}{0.000000,0.000000,0.000000}%
\pgfsetstrokecolor{currentstroke}%
\pgfsetstrokeopacity{0.000000}%
\pgfsetdash{}{0pt}%
\pgfpathmoveto{\pgfqpoint{5.071077in}{1.876139in}}%
\pgfpathlineto{\pgfqpoint{5.080014in}{1.876139in}}%
\pgfpathlineto{\pgfqpoint{5.080014in}{2.264057in}}%
\pgfpathlineto{\pgfqpoint{5.071077in}{2.264057in}}%
\pgfpathlineto{\pgfqpoint{5.071077in}{1.876139in}}%
\pgfpathclose%
\pgfusepath{fill}%
\end{pgfscope}%
\begin{pgfscope}%
\pgfpathrectangle{\pgfqpoint{3.722897in}{0.857143in}}{\pgfqpoint{2.627103in}{1.813434in}}%
\pgfusepath{clip}%
\pgfsetbuttcap%
\pgfsetmiterjoin%
\definecolor{currentfill}{rgb}{0.302379,0.450282,0.300122}%
\pgfsetfillcolor{currentfill}%
\pgfsetlinewidth{0.000000pt}%
\definecolor{currentstroke}{rgb}{0.000000,0.000000,0.000000}%
\pgfsetstrokecolor{currentstroke}%
\pgfsetstrokeopacity{0.000000}%
\pgfsetdash{}{0pt}%
\pgfpathmoveto{\pgfqpoint{5.082248in}{1.879552in}}%
\pgfpathlineto{\pgfqpoint{5.091184in}{1.879552in}}%
\pgfpathlineto{\pgfqpoint{5.091184in}{2.271444in}}%
\pgfpathlineto{\pgfqpoint{5.082248in}{2.271444in}}%
\pgfpathlineto{\pgfqpoint{5.082248in}{1.879552in}}%
\pgfpathclose%
\pgfusepath{fill}%
\end{pgfscope}%
\begin{pgfscope}%
\pgfpathrectangle{\pgfqpoint{3.722897in}{0.857143in}}{\pgfqpoint{2.627103in}{1.813434in}}%
\pgfusepath{clip}%
\pgfsetbuttcap%
\pgfsetmiterjoin%
\definecolor{currentfill}{rgb}{0.302379,0.450282,0.300122}%
\pgfsetfillcolor{currentfill}%
\pgfsetlinewidth{0.000000pt}%
\definecolor{currentstroke}{rgb}{0.000000,0.000000,0.000000}%
\pgfsetstrokecolor{currentstroke}%
\pgfsetstrokeopacity{0.000000}%
\pgfsetdash{}{0pt}%
\pgfpathmoveto{\pgfqpoint{5.093419in}{1.884427in}}%
\pgfpathlineto{\pgfqpoint{5.102355in}{1.884427in}}%
\pgfpathlineto{\pgfqpoint{5.102355in}{2.281583in}}%
\pgfpathlineto{\pgfqpoint{5.093419in}{2.281583in}}%
\pgfpathlineto{\pgfqpoint{5.093419in}{1.884427in}}%
\pgfpathclose%
\pgfusepath{fill}%
\end{pgfscope}%
\begin{pgfscope}%
\pgfpathrectangle{\pgfqpoint{3.722897in}{0.857143in}}{\pgfqpoint{2.627103in}{1.813434in}}%
\pgfusepath{clip}%
\pgfsetbuttcap%
\pgfsetmiterjoin%
\definecolor{currentfill}{rgb}{0.302379,0.450282,0.300122}%
\pgfsetfillcolor{currentfill}%
\pgfsetlinewidth{0.000000pt}%
\definecolor{currentstroke}{rgb}{0.000000,0.000000,0.000000}%
\pgfsetstrokecolor{currentstroke}%
\pgfsetstrokeopacity{0.000000}%
\pgfsetdash{}{0pt}%
\pgfpathmoveto{\pgfqpoint{5.104589in}{1.888318in}}%
\pgfpathlineto{\pgfqpoint{5.113526in}{1.888318in}}%
\pgfpathlineto{\pgfqpoint{5.113526in}{2.291229in}}%
\pgfpathlineto{\pgfqpoint{5.104589in}{2.291229in}}%
\pgfpathlineto{\pgfqpoint{5.104589in}{1.888318in}}%
\pgfpathclose%
\pgfusepath{fill}%
\end{pgfscope}%
\begin{pgfscope}%
\pgfpathrectangle{\pgfqpoint{3.722897in}{0.857143in}}{\pgfqpoint{2.627103in}{1.813434in}}%
\pgfusepath{clip}%
\pgfsetbuttcap%
\pgfsetmiterjoin%
\definecolor{currentfill}{rgb}{0.302379,0.450282,0.300122}%
\pgfsetfillcolor{currentfill}%
\pgfsetlinewidth{0.000000pt}%
\definecolor{currentstroke}{rgb}{0.000000,0.000000,0.000000}%
\pgfsetstrokecolor{currentstroke}%
\pgfsetstrokeopacity{0.000000}%
\pgfsetdash{}{0pt}%
\pgfpathmoveto{\pgfqpoint{5.115760in}{1.891406in}}%
\pgfpathlineto{\pgfqpoint{5.124696in}{1.891406in}}%
\pgfpathlineto{\pgfqpoint{5.124696in}{2.300574in}}%
\pgfpathlineto{\pgfqpoint{5.115760in}{2.300574in}}%
\pgfpathlineto{\pgfqpoint{5.115760in}{1.891406in}}%
\pgfpathclose%
\pgfusepath{fill}%
\end{pgfscope}%
\begin{pgfscope}%
\pgfpathrectangle{\pgfqpoint{3.722897in}{0.857143in}}{\pgfqpoint{2.627103in}{1.813434in}}%
\pgfusepath{clip}%
\pgfsetbuttcap%
\pgfsetmiterjoin%
\definecolor{currentfill}{rgb}{0.302379,0.450282,0.300122}%
\pgfsetfillcolor{currentfill}%
\pgfsetlinewidth{0.000000pt}%
\definecolor{currentstroke}{rgb}{0.000000,0.000000,0.000000}%
\pgfsetstrokecolor{currentstroke}%
\pgfsetstrokeopacity{0.000000}%
\pgfsetdash{}{0pt}%
\pgfpathmoveto{\pgfqpoint{5.126930in}{1.898182in}}%
\pgfpathlineto{\pgfqpoint{5.135867in}{1.898182in}}%
\pgfpathlineto{\pgfqpoint{5.135867in}{2.312468in}}%
\pgfpathlineto{\pgfqpoint{5.126930in}{2.312468in}}%
\pgfpathlineto{\pgfqpoint{5.126930in}{1.898182in}}%
\pgfpathclose%
\pgfusepath{fill}%
\end{pgfscope}%
\begin{pgfscope}%
\pgfpathrectangle{\pgfqpoint{3.722897in}{0.857143in}}{\pgfqpoint{2.627103in}{1.813434in}}%
\pgfusepath{clip}%
\pgfsetbuttcap%
\pgfsetmiterjoin%
\definecolor{currentfill}{rgb}{0.302379,0.450282,0.300122}%
\pgfsetfillcolor{currentfill}%
\pgfsetlinewidth{0.000000pt}%
\definecolor{currentstroke}{rgb}{0.000000,0.000000,0.000000}%
\pgfsetstrokecolor{currentstroke}%
\pgfsetstrokeopacity{0.000000}%
\pgfsetdash{}{0pt}%
\pgfpathmoveto{\pgfqpoint{5.138101in}{1.908644in}}%
\pgfpathlineto{\pgfqpoint{5.147038in}{1.908644in}}%
\pgfpathlineto{\pgfqpoint{5.147038in}{2.326844in}}%
\pgfpathlineto{\pgfqpoint{5.138101in}{2.326844in}}%
\pgfpathlineto{\pgfqpoint{5.138101in}{1.908644in}}%
\pgfpathclose%
\pgfusepath{fill}%
\end{pgfscope}%
\begin{pgfscope}%
\pgfpathrectangle{\pgfqpoint{3.722897in}{0.857143in}}{\pgfqpoint{2.627103in}{1.813434in}}%
\pgfusepath{clip}%
\pgfsetbuttcap%
\pgfsetmiterjoin%
\definecolor{currentfill}{rgb}{0.302379,0.450282,0.300122}%
\pgfsetfillcolor{currentfill}%
\pgfsetlinewidth{0.000000pt}%
\definecolor{currentstroke}{rgb}{0.000000,0.000000,0.000000}%
\pgfsetstrokecolor{currentstroke}%
\pgfsetstrokeopacity{0.000000}%
\pgfsetdash{}{0pt}%
\pgfpathmoveto{\pgfqpoint{5.149272in}{1.907468in}}%
\pgfpathlineto{\pgfqpoint{5.158208in}{1.907468in}}%
\pgfpathlineto{\pgfqpoint{5.158208in}{2.329067in}}%
\pgfpathlineto{\pgfqpoint{5.149272in}{2.329067in}}%
\pgfpathlineto{\pgfqpoint{5.149272in}{1.907468in}}%
\pgfpathclose%
\pgfusepath{fill}%
\end{pgfscope}%
\begin{pgfscope}%
\pgfpathrectangle{\pgfqpoint{3.722897in}{0.857143in}}{\pgfqpoint{2.627103in}{1.813434in}}%
\pgfusepath{clip}%
\pgfsetbuttcap%
\pgfsetmiterjoin%
\definecolor{currentfill}{rgb}{0.302379,0.450282,0.300122}%
\pgfsetfillcolor{currentfill}%
\pgfsetlinewidth{0.000000pt}%
\definecolor{currentstroke}{rgb}{0.000000,0.000000,0.000000}%
\pgfsetstrokecolor{currentstroke}%
\pgfsetstrokeopacity{0.000000}%
\pgfsetdash{}{0pt}%
\pgfpathmoveto{\pgfqpoint{5.160442in}{1.910270in}}%
\pgfpathlineto{\pgfqpoint{5.169379in}{1.910270in}}%
\pgfpathlineto{\pgfqpoint{5.169379in}{2.333702in}}%
\pgfpathlineto{\pgfqpoint{5.160442in}{2.333702in}}%
\pgfpathlineto{\pgfqpoint{5.160442in}{1.910270in}}%
\pgfpathclose%
\pgfusepath{fill}%
\end{pgfscope}%
\begin{pgfscope}%
\pgfpathrectangle{\pgfqpoint{3.722897in}{0.857143in}}{\pgfqpoint{2.627103in}{1.813434in}}%
\pgfusepath{clip}%
\pgfsetbuttcap%
\pgfsetmiterjoin%
\definecolor{currentfill}{rgb}{0.302379,0.450282,0.300122}%
\pgfsetfillcolor{currentfill}%
\pgfsetlinewidth{0.000000pt}%
\definecolor{currentstroke}{rgb}{0.000000,0.000000,0.000000}%
\pgfsetstrokecolor{currentstroke}%
\pgfsetstrokeopacity{0.000000}%
\pgfsetdash{}{0pt}%
\pgfpathmoveto{\pgfqpoint{5.171613in}{1.914225in}}%
\pgfpathlineto{\pgfqpoint{5.180549in}{1.914225in}}%
\pgfpathlineto{\pgfqpoint{5.180549in}{2.340218in}}%
\pgfpathlineto{\pgfqpoint{5.171613in}{2.340218in}}%
\pgfpathlineto{\pgfqpoint{5.171613in}{1.914225in}}%
\pgfpathclose%
\pgfusepath{fill}%
\end{pgfscope}%
\begin{pgfscope}%
\pgfpathrectangle{\pgfqpoint{3.722897in}{0.857143in}}{\pgfqpoint{2.627103in}{1.813434in}}%
\pgfusepath{clip}%
\pgfsetbuttcap%
\pgfsetmiterjoin%
\definecolor{currentfill}{rgb}{0.302379,0.450282,0.300122}%
\pgfsetfillcolor{currentfill}%
\pgfsetlinewidth{0.000000pt}%
\definecolor{currentstroke}{rgb}{0.000000,0.000000,0.000000}%
\pgfsetstrokecolor{currentstroke}%
\pgfsetstrokeopacity{0.000000}%
\pgfsetdash{}{0pt}%
\pgfpathmoveto{\pgfqpoint{5.182783in}{1.922052in}}%
\pgfpathlineto{\pgfqpoint{5.191720in}{1.922052in}}%
\pgfpathlineto{\pgfqpoint{5.191720in}{2.350338in}}%
\pgfpathlineto{\pgfqpoint{5.182783in}{2.350338in}}%
\pgfpathlineto{\pgfqpoint{5.182783in}{1.922052in}}%
\pgfpathclose%
\pgfusepath{fill}%
\end{pgfscope}%
\begin{pgfscope}%
\pgfpathrectangle{\pgfqpoint{3.722897in}{0.857143in}}{\pgfqpoint{2.627103in}{1.813434in}}%
\pgfusepath{clip}%
\pgfsetbuttcap%
\pgfsetmiterjoin%
\definecolor{currentfill}{rgb}{0.302379,0.450282,0.300122}%
\pgfsetfillcolor{currentfill}%
\pgfsetlinewidth{0.000000pt}%
\definecolor{currentstroke}{rgb}{0.000000,0.000000,0.000000}%
\pgfsetstrokecolor{currentstroke}%
\pgfsetstrokeopacity{0.000000}%
\pgfsetdash{}{0pt}%
\pgfpathmoveto{\pgfqpoint{5.193954in}{1.917000in}}%
\pgfpathlineto{\pgfqpoint{5.202891in}{1.917000in}}%
\pgfpathlineto{\pgfqpoint{5.202891in}{2.346969in}}%
\pgfpathlineto{\pgfqpoint{5.193954in}{2.346969in}}%
\pgfpathlineto{\pgfqpoint{5.193954in}{1.917000in}}%
\pgfpathclose%
\pgfusepath{fill}%
\end{pgfscope}%
\begin{pgfscope}%
\pgfpathrectangle{\pgfqpoint{3.722897in}{0.857143in}}{\pgfqpoint{2.627103in}{1.813434in}}%
\pgfusepath{clip}%
\pgfsetbuttcap%
\pgfsetmiterjoin%
\definecolor{currentfill}{rgb}{0.302379,0.450282,0.300122}%
\pgfsetfillcolor{currentfill}%
\pgfsetlinewidth{0.000000pt}%
\definecolor{currentstroke}{rgb}{0.000000,0.000000,0.000000}%
\pgfsetstrokecolor{currentstroke}%
\pgfsetstrokeopacity{0.000000}%
\pgfsetdash{}{0pt}%
\pgfpathmoveto{\pgfqpoint{5.205125in}{1.924029in}}%
\pgfpathlineto{\pgfqpoint{5.214061in}{1.924029in}}%
\pgfpathlineto{\pgfqpoint{5.214061in}{2.355047in}}%
\pgfpathlineto{\pgfqpoint{5.205125in}{2.355047in}}%
\pgfpathlineto{\pgfqpoint{5.205125in}{1.924029in}}%
\pgfpathclose%
\pgfusepath{fill}%
\end{pgfscope}%
\begin{pgfscope}%
\pgfpathrectangle{\pgfqpoint{3.722897in}{0.857143in}}{\pgfqpoint{2.627103in}{1.813434in}}%
\pgfusepath{clip}%
\pgfsetbuttcap%
\pgfsetmiterjoin%
\definecolor{currentfill}{rgb}{0.302379,0.450282,0.300122}%
\pgfsetfillcolor{currentfill}%
\pgfsetlinewidth{0.000000pt}%
\definecolor{currentstroke}{rgb}{0.000000,0.000000,0.000000}%
\pgfsetstrokecolor{currentstroke}%
\pgfsetstrokeopacity{0.000000}%
\pgfsetdash{}{0pt}%
\pgfpathmoveto{\pgfqpoint{5.216295in}{1.935304in}}%
\pgfpathlineto{\pgfqpoint{5.225232in}{1.935304in}}%
\pgfpathlineto{\pgfqpoint{5.225232in}{2.366394in}}%
\pgfpathlineto{\pgfqpoint{5.216295in}{2.366394in}}%
\pgfpathlineto{\pgfqpoint{5.216295in}{1.935304in}}%
\pgfpathclose%
\pgfusepath{fill}%
\end{pgfscope}%
\begin{pgfscope}%
\pgfpathrectangle{\pgfqpoint{3.722897in}{0.857143in}}{\pgfqpoint{2.627103in}{1.813434in}}%
\pgfusepath{clip}%
\pgfsetbuttcap%
\pgfsetmiterjoin%
\definecolor{currentfill}{rgb}{0.302379,0.450282,0.300122}%
\pgfsetfillcolor{currentfill}%
\pgfsetlinewidth{0.000000pt}%
\definecolor{currentstroke}{rgb}{0.000000,0.000000,0.000000}%
\pgfsetstrokecolor{currentstroke}%
\pgfsetstrokeopacity{0.000000}%
\pgfsetdash{}{0pt}%
\pgfpathmoveto{\pgfqpoint{5.227466in}{1.929479in}}%
\pgfpathlineto{\pgfqpoint{5.236402in}{1.929479in}}%
\pgfpathlineto{\pgfqpoint{5.236402in}{2.360255in}}%
\pgfpathlineto{\pgfqpoint{5.227466in}{2.360255in}}%
\pgfpathlineto{\pgfqpoint{5.227466in}{1.929479in}}%
\pgfpathclose%
\pgfusepath{fill}%
\end{pgfscope}%
\begin{pgfscope}%
\pgfpathrectangle{\pgfqpoint{3.722897in}{0.857143in}}{\pgfqpoint{2.627103in}{1.813434in}}%
\pgfusepath{clip}%
\pgfsetbuttcap%
\pgfsetmiterjoin%
\definecolor{currentfill}{rgb}{0.302379,0.450282,0.300122}%
\pgfsetfillcolor{currentfill}%
\pgfsetlinewidth{0.000000pt}%
\definecolor{currentstroke}{rgb}{0.000000,0.000000,0.000000}%
\pgfsetstrokecolor{currentstroke}%
\pgfsetstrokeopacity{0.000000}%
\pgfsetdash{}{0pt}%
\pgfpathmoveto{\pgfqpoint{5.238636in}{1.940663in}}%
\pgfpathlineto{\pgfqpoint{5.247573in}{1.940663in}}%
\pgfpathlineto{\pgfqpoint{5.247573in}{2.369298in}}%
\pgfpathlineto{\pgfqpoint{5.238636in}{2.369298in}}%
\pgfpathlineto{\pgfqpoint{5.238636in}{1.940663in}}%
\pgfpathclose%
\pgfusepath{fill}%
\end{pgfscope}%
\begin{pgfscope}%
\pgfpathrectangle{\pgfqpoint{3.722897in}{0.857143in}}{\pgfqpoint{2.627103in}{1.813434in}}%
\pgfusepath{clip}%
\pgfsetbuttcap%
\pgfsetmiterjoin%
\definecolor{currentfill}{rgb}{0.302379,0.450282,0.300122}%
\pgfsetfillcolor{currentfill}%
\pgfsetlinewidth{0.000000pt}%
\definecolor{currentstroke}{rgb}{0.000000,0.000000,0.000000}%
\pgfsetstrokecolor{currentstroke}%
\pgfsetstrokeopacity{0.000000}%
\pgfsetdash{}{0pt}%
\pgfpathmoveto{\pgfqpoint{5.249807in}{1.949479in}}%
\pgfpathlineto{\pgfqpoint{5.258744in}{1.949479in}}%
\pgfpathlineto{\pgfqpoint{5.258744in}{2.374021in}}%
\pgfpathlineto{\pgfqpoint{5.249807in}{2.374021in}}%
\pgfpathlineto{\pgfqpoint{5.249807in}{1.949479in}}%
\pgfpathclose%
\pgfusepath{fill}%
\end{pgfscope}%
\begin{pgfscope}%
\pgfpathrectangle{\pgfqpoint{3.722897in}{0.857143in}}{\pgfqpoint{2.627103in}{1.813434in}}%
\pgfusepath{clip}%
\pgfsetbuttcap%
\pgfsetmiterjoin%
\definecolor{currentfill}{rgb}{0.302379,0.450282,0.300122}%
\pgfsetfillcolor{currentfill}%
\pgfsetlinewidth{0.000000pt}%
\definecolor{currentstroke}{rgb}{0.000000,0.000000,0.000000}%
\pgfsetstrokecolor{currentstroke}%
\pgfsetstrokeopacity{0.000000}%
\pgfsetdash{}{0pt}%
\pgfpathmoveto{\pgfqpoint{5.260978in}{1.947999in}}%
\pgfpathlineto{\pgfqpoint{5.269914in}{1.947999in}}%
\pgfpathlineto{\pgfqpoint{5.269914in}{2.368295in}}%
\pgfpathlineto{\pgfqpoint{5.260978in}{2.368295in}}%
\pgfpathlineto{\pgfqpoint{5.260978in}{1.947999in}}%
\pgfpathclose%
\pgfusepath{fill}%
\end{pgfscope}%
\begin{pgfscope}%
\pgfpathrectangle{\pgfqpoint{3.722897in}{0.857143in}}{\pgfqpoint{2.627103in}{1.813434in}}%
\pgfusepath{clip}%
\pgfsetbuttcap%
\pgfsetmiterjoin%
\definecolor{currentfill}{rgb}{0.302379,0.450282,0.300122}%
\pgfsetfillcolor{currentfill}%
\pgfsetlinewidth{0.000000pt}%
\definecolor{currentstroke}{rgb}{0.000000,0.000000,0.000000}%
\pgfsetstrokecolor{currentstroke}%
\pgfsetstrokeopacity{0.000000}%
\pgfsetdash{}{0pt}%
\pgfpathmoveto{\pgfqpoint{5.272148in}{1.931979in}}%
\pgfpathlineto{\pgfqpoint{5.281085in}{1.931979in}}%
\pgfpathlineto{\pgfqpoint{5.281085in}{2.350264in}}%
\pgfpathlineto{\pgfqpoint{5.272148in}{2.350264in}}%
\pgfpathlineto{\pgfqpoint{5.272148in}{1.931979in}}%
\pgfpathclose%
\pgfusepath{fill}%
\end{pgfscope}%
\begin{pgfscope}%
\pgfpathrectangle{\pgfqpoint{3.722897in}{0.857143in}}{\pgfqpoint{2.627103in}{1.813434in}}%
\pgfusepath{clip}%
\pgfsetbuttcap%
\pgfsetmiterjoin%
\definecolor{currentfill}{rgb}{0.302379,0.450282,0.300122}%
\pgfsetfillcolor{currentfill}%
\pgfsetlinewidth{0.000000pt}%
\definecolor{currentstroke}{rgb}{0.000000,0.000000,0.000000}%
\pgfsetstrokecolor{currentstroke}%
\pgfsetstrokeopacity{0.000000}%
\pgfsetdash{}{0pt}%
\pgfpathmoveto{\pgfqpoint{5.283319in}{1.940582in}}%
\pgfpathlineto{\pgfqpoint{5.292255in}{1.940582in}}%
\pgfpathlineto{\pgfqpoint{5.292255in}{2.356600in}}%
\pgfpathlineto{\pgfqpoint{5.283319in}{2.356600in}}%
\pgfpathlineto{\pgfqpoint{5.283319in}{1.940582in}}%
\pgfpathclose%
\pgfusepath{fill}%
\end{pgfscope}%
\begin{pgfscope}%
\pgfpathrectangle{\pgfqpoint{3.722897in}{0.857143in}}{\pgfqpoint{2.627103in}{1.813434in}}%
\pgfusepath{clip}%
\pgfsetbuttcap%
\pgfsetmiterjoin%
\definecolor{currentfill}{rgb}{0.302379,0.450282,0.300122}%
\pgfsetfillcolor{currentfill}%
\pgfsetlinewidth{0.000000pt}%
\definecolor{currentstroke}{rgb}{0.000000,0.000000,0.000000}%
\pgfsetstrokecolor{currentstroke}%
\pgfsetstrokeopacity{0.000000}%
\pgfsetdash{}{0pt}%
\pgfpathmoveto{\pgfqpoint{5.294489in}{1.935619in}}%
\pgfpathlineto{\pgfqpoint{5.303426in}{1.935619in}}%
\pgfpathlineto{\pgfqpoint{5.303426in}{2.347792in}}%
\pgfpathlineto{\pgfqpoint{5.294489in}{2.347792in}}%
\pgfpathlineto{\pgfqpoint{5.294489in}{1.935619in}}%
\pgfpathclose%
\pgfusepath{fill}%
\end{pgfscope}%
\begin{pgfscope}%
\pgfpathrectangle{\pgfqpoint{3.722897in}{0.857143in}}{\pgfqpoint{2.627103in}{1.813434in}}%
\pgfusepath{clip}%
\pgfsetbuttcap%
\pgfsetmiterjoin%
\definecolor{currentfill}{rgb}{0.302379,0.450282,0.300122}%
\pgfsetfillcolor{currentfill}%
\pgfsetlinewidth{0.000000pt}%
\definecolor{currentstroke}{rgb}{0.000000,0.000000,0.000000}%
\pgfsetstrokecolor{currentstroke}%
\pgfsetstrokeopacity{0.000000}%
\pgfsetdash{}{0pt}%
\pgfpathmoveto{\pgfqpoint{5.305660in}{1.931317in}}%
\pgfpathlineto{\pgfqpoint{5.314597in}{1.931317in}}%
\pgfpathlineto{\pgfqpoint{5.314597in}{2.340718in}}%
\pgfpathlineto{\pgfqpoint{5.305660in}{2.340718in}}%
\pgfpathlineto{\pgfqpoint{5.305660in}{1.931317in}}%
\pgfpathclose%
\pgfusepath{fill}%
\end{pgfscope}%
\begin{pgfscope}%
\pgfpathrectangle{\pgfqpoint{3.722897in}{0.857143in}}{\pgfqpoint{2.627103in}{1.813434in}}%
\pgfusepath{clip}%
\pgfsetbuttcap%
\pgfsetmiterjoin%
\definecolor{currentfill}{rgb}{0.302379,0.450282,0.300122}%
\pgfsetfillcolor{currentfill}%
\pgfsetlinewidth{0.000000pt}%
\definecolor{currentstroke}{rgb}{0.000000,0.000000,0.000000}%
\pgfsetstrokecolor{currentstroke}%
\pgfsetstrokeopacity{0.000000}%
\pgfsetdash{}{0pt}%
\pgfpathmoveto{\pgfqpoint{5.316831in}{1.931629in}}%
\pgfpathlineto{\pgfqpoint{5.325767in}{1.931629in}}%
\pgfpathlineto{\pgfqpoint{5.325767in}{2.338585in}}%
\pgfpathlineto{\pgfqpoint{5.316831in}{2.338585in}}%
\pgfpathlineto{\pgfqpoint{5.316831in}{1.931629in}}%
\pgfpathclose%
\pgfusepath{fill}%
\end{pgfscope}%
\begin{pgfscope}%
\pgfpathrectangle{\pgfqpoint{3.722897in}{0.857143in}}{\pgfqpoint{2.627103in}{1.813434in}}%
\pgfusepath{clip}%
\pgfsetbuttcap%
\pgfsetmiterjoin%
\definecolor{currentfill}{rgb}{0.302379,0.450282,0.300122}%
\pgfsetfillcolor{currentfill}%
\pgfsetlinewidth{0.000000pt}%
\definecolor{currentstroke}{rgb}{0.000000,0.000000,0.000000}%
\pgfsetstrokecolor{currentstroke}%
\pgfsetstrokeopacity{0.000000}%
\pgfsetdash{}{0pt}%
\pgfpathmoveto{\pgfqpoint{5.328001in}{1.918067in}}%
\pgfpathlineto{\pgfqpoint{5.336938in}{1.918067in}}%
\pgfpathlineto{\pgfqpoint{5.336938in}{2.322850in}}%
\pgfpathlineto{\pgfqpoint{5.328001in}{2.322850in}}%
\pgfpathlineto{\pgfqpoint{5.328001in}{1.918067in}}%
\pgfpathclose%
\pgfusepath{fill}%
\end{pgfscope}%
\begin{pgfscope}%
\pgfpathrectangle{\pgfqpoint{3.722897in}{0.857143in}}{\pgfqpoint{2.627103in}{1.813434in}}%
\pgfusepath{clip}%
\pgfsetbuttcap%
\pgfsetmiterjoin%
\definecolor{currentfill}{rgb}{0.302379,0.450282,0.300122}%
\pgfsetfillcolor{currentfill}%
\pgfsetlinewidth{0.000000pt}%
\definecolor{currentstroke}{rgb}{0.000000,0.000000,0.000000}%
\pgfsetstrokecolor{currentstroke}%
\pgfsetstrokeopacity{0.000000}%
\pgfsetdash{}{0pt}%
\pgfpathmoveto{\pgfqpoint{5.339172in}{1.912218in}}%
\pgfpathlineto{\pgfqpoint{5.348108in}{1.912218in}}%
\pgfpathlineto{\pgfqpoint{5.348108in}{2.317086in}}%
\pgfpathlineto{\pgfqpoint{5.339172in}{2.317086in}}%
\pgfpathlineto{\pgfqpoint{5.339172in}{1.912218in}}%
\pgfpathclose%
\pgfusepath{fill}%
\end{pgfscope}%
\begin{pgfscope}%
\pgfpathrectangle{\pgfqpoint{3.722897in}{0.857143in}}{\pgfqpoint{2.627103in}{1.813434in}}%
\pgfusepath{clip}%
\pgfsetbuttcap%
\pgfsetmiterjoin%
\definecolor{currentfill}{rgb}{0.302379,0.450282,0.300122}%
\pgfsetfillcolor{currentfill}%
\pgfsetlinewidth{0.000000pt}%
\definecolor{currentstroke}{rgb}{0.000000,0.000000,0.000000}%
\pgfsetstrokecolor{currentstroke}%
\pgfsetstrokeopacity{0.000000}%
\pgfsetdash{}{0pt}%
\pgfpathmoveto{\pgfqpoint{5.350343in}{1.912132in}}%
\pgfpathlineto{\pgfqpoint{5.359279in}{1.912132in}}%
\pgfpathlineto{\pgfqpoint{5.359279in}{2.318728in}}%
\pgfpathlineto{\pgfqpoint{5.350343in}{2.318728in}}%
\pgfpathlineto{\pgfqpoint{5.350343in}{1.912132in}}%
\pgfpathclose%
\pgfusepath{fill}%
\end{pgfscope}%
\begin{pgfscope}%
\pgfpathrectangle{\pgfqpoint{3.722897in}{0.857143in}}{\pgfqpoint{2.627103in}{1.813434in}}%
\pgfusepath{clip}%
\pgfsetbuttcap%
\pgfsetmiterjoin%
\definecolor{currentfill}{rgb}{0.302379,0.450282,0.300122}%
\pgfsetfillcolor{currentfill}%
\pgfsetlinewidth{0.000000pt}%
\definecolor{currentstroke}{rgb}{0.000000,0.000000,0.000000}%
\pgfsetstrokecolor{currentstroke}%
\pgfsetstrokeopacity{0.000000}%
\pgfsetdash{}{0pt}%
\pgfpathmoveto{\pgfqpoint{5.361513in}{1.909698in}}%
\pgfpathlineto{\pgfqpoint{5.370450in}{1.909698in}}%
\pgfpathlineto{\pgfqpoint{5.370450in}{2.317982in}}%
\pgfpathlineto{\pgfqpoint{5.361513in}{2.317982in}}%
\pgfpathlineto{\pgfqpoint{5.361513in}{1.909698in}}%
\pgfpathclose%
\pgfusepath{fill}%
\end{pgfscope}%
\begin{pgfscope}%
\pgfpathrectangle{\pgfqpoint{3.722897in}{0.857143in}}{\pgfqpoint{2.627103in}{1.813434in}}%
\pgfusepath{clip}%
\pgfsetbuttcap%
\pgfsetmiterjoin%
\definecolor{currentfill}{rgb}{0.302379,0.450282,0.300122}%
\pgfsetfillcolor{currentfill}%
\pgfsetlinewidth{0.000000pt}%
\definecolor{currentstroke}{rgb}{0.000000,0.000000,0.000000}%
\pgfsetstrokecolor{currentstroke}%
\pgfsetstrokeopacity{0.000000}%
\pgfsetdash{}{0pt}%
\pgfpathmoveto{\pgfqpoint{5.372684in}{1.908817in}}%
\pgfpathlineto{\pgfqpoint{5.381620in}{1.908817in}}%
\pgfpathlineto{\pgfqpoint{5.381620in}{2.318073in}}%
\pgfpathlineto{\pgfqpoint{5.372684in}{2.318073in}}%
\pgfpathlineto{\pgfqpoint{5.372684in}{1.908817in}}%
\pgfpathclose%
\pgfusepath{fill}%
\end{pgfscope}%
\begin{pgfscope}%
\pgfpathrectangle{\pgfqpoint{3.722897in}{0.857143in}}{\pgfqpoint{2.627103in}{1.813434in}}%
\pgfusepath{clip}%
\pgfsetbuttcap%
\pgfsetmiterjoin%
\definecolor{currentfill}{rgb}{0.302379,0.450282,0.300122}%
\pgfsetfillcolor{currentfill}%
\pgfsetlinewidth{0.000000pt}%
\definecolor{currentstroke}{rgb}{0.000000,0.000000,0.000000}%
\pgfsetstrokecolor{currentstroke}%
\pgfsetstrokeopacity{0.000000}%
\pgfsetdash{}{0pt}%
\pgfpathmoveto{\pgfqpoint{5.383854in}{1.895676in}}%
\pgfpathlineto{\pgfqpoint{5.392791in}{1.895676in}}%
\pgfpathlineto{\pgfqpoint{5.392791in}{2.305459in}}%
\pgfpathlineto{\pgfqpoint{5.383854in}{2.305459in}}%
\pgfpathlineto{\pgfqpoint{5.383854in}{1.895676in}}%
\pgfpathclose%
\pgfusepath{fill}%
\end{pgfscope}%
\begin{pgfscope}%
\pgfpathrectangle{\pgfqpoint{3.722897in}{0.857143in}}{\pgfqpoint{2.627103in}{1.813434in}}%
\pgfusepath{clip}%
\pgfsetbuttcap%
\pgfsetmiterjoin%
\definecolor{currentfill}{rgb}{0.302379,0.450282,0.300122}%
\pgfsetfillcolor{currentfill}%
\pgfsetlinewidth{0.000000pt}%
\definecolor{currentstroke}{rgb}{0.000000,0.000000,0.000000}%
\pgfsetstrokecolor{currentstroke}%
\pgfsetstrokeopacity{0.000000}%
\pgfsetdash{}{0pt}%
\pgfpathmoveto{\pgfqpoint{5.395025in}{1.894476in}}%
\pgfpathlineto{\pgfqpoint{5.403961in}{1.894476in}}%
\pgfpathlineto{\pgfqpoint{5.403961in}{2.301603in}}%
\pgfpathlineto{\pgfqpoint{5.395025in}{2.301603in}}%
\pgfpathlineto{\pgfqpoint{5.395025in}{1.894476in}}%
\pgfpathclose%
\pgfusepath{fill}%
\end{pgfscope}%
\begin{pgfscope}%
\pgfpathrectangle{\pgfqpoint{3.722897in}{0.857143in}}{\pgfqpoint{2.627103in}{1.813434in}}%
\pgfusepath{clip}%
\pgfsetbuttcap%
\pgfsetmiterjoin%
\definecolor{currentfill}{rgb}{0.302379,0.450282,0.300122}%
\pgfsetfillcolor{currentfill}%
\pgfsetlinewidth{0.000000pt}%
\definecolor{currentstroke}{rgb}{0.000000,0.000000,0.000000}%
\pgfsetstrokecolor{currentstroke}%
\pgfsetstrokeopacity{0.000000}%
\pgfsetdash{}{0pt}%
\pgfpathmoveto{\pgfqpoint{5.406196in}{1.886813in}}%
\pgfpathlineto{\pgfqpoint{5.415132in}{1.886813in}}%
\pgfpathlineto{\pgfqpoint{5.415132in}{2.288200in}}%
\pgfpathlineto{\pgfqpoint{5.406196in}{2.288200in}}%
\pgfpathlineto{\pgfqpoint{5.406196in}{1.886813in}}%
\pgfpathclose%
\pgfusepath{fill}%
\end{pgfscope}%
\begin{pgfscope}%
\pgfpathrectangle{\pgfqpoint{3.722897in}{0.857143in}}{\pgfqpoint{2.627103in}{1.813434in}}%
\pgfusepath{clip}%
\pgfsetbuttcap%
\pgfsetmiterjoin%
\definecolor{currentfill}{rgb}{0.302379,0.450282,0.300122}%
\pgfsetfillcolor{currentfill}%
\pgfsetlinewidth{0.000000pt}%
\definecolor{currentstroke}{rgb}{0.000000,0.000000,0.000000}%
\pgfsetstrokecolor{currentstroke}%
\pgfsetstrokeopacity{0.000000}%
\pgfsetdash{}{0pt}%
\pgfpathmoveto{\pgfqpoint{5.417366in}{1.880397in}}%
\pgfpathlineto{\pgfqpoint{5.426303in}{1.880397in}}%
\pgfpathlineto{\pgfqpoint{5.426303in}{2.274542in}}%
\pgfpathlineto{\pgfqpoint{5.417366in}{2.274542in}}%
\pgfpathlineto{\pgfqpoint{5.417366in}{1.880397in}}%
\pgfpathclose%
\pgfusepath{fill}%
\end{pgfscope}%
\begin{pgfscope}%
\pgfpathrectangle{\pgfqpoint{3.722897in}{0.857143in}}{\pgfqpoint{2.627103in}{1.813434in}}%
\pgfusepath{clip}%
\pgfsetbuttcap%
\pgfsetmiterjoin%
\definecolor{currentfill}{rgb}{0.302379,0.450282,0.300122}%
\pgfsetfillcolor{currentfill}%
\pgfsetlinewidth{0.000000pt}%
\definecolor{currentstroke}{rgb}{0.000000,0.000000,0.000000}%
\pgfsetstrokecolor{currentstroke}%
\pgfsetstrokeopacity{0.000000}%
\pgfsetdash{}{0pt}%
\pgfpathmoveto{\pgfqpoint{5.428537in}{1.871835in}}%
\pgfpathlineto{\pgfqpoint{5.437473in}{1.871835in}}%
\pgfpathlineto{\pgfqpoint{5.437473in}{2.255804in}}%
\pgfpathlineto{\pgfqpoint{5.428537in}{2.255804in}}%
\pgfpathlineto{\pgfqpoint{5.428537in}{1.871835in}}%
\pgfpathclose%
\pgfusepath{fill}%
\end{pgfscope}%
\begin{pgfscope}%
\pgfpathrectangle{\pgfqpoint{3.722897in}{0.857143in}}{\pgfqpoint{2.627103in}{1.813434in}}%
\pgfusepath{clip}%
\pgfsetbuttcap%
\pgfsetmiterjoin%
\definecolor{currentfill}{rgb}{0.302379,0.450282,0.300122}%
\pgfsetfillcolor{currentfill}%
\pgfsetlinewidth{0.000000pt}%
\definecolor{currentstroke}{rgb}{0.000000,0.000000,0.000000}%
\pgfsetstrokecolor{currentstroke}%
\pgfsetstrokeopacity{0.000000}%
\pgfsetdash{}{0pt}%
\pgfpathmoveto{\pgfqpoint{5.439707in}{1.864648in}}%
\pgfpathlineto{\pgfqpoint{5.448644in}{1.864648in}}%
\pgfpathlineto{\pgfqpoint{5.448644in}{2.238107in}}%
\pgfpathlineto{\pgfqpoint{5.439707in}{2.238107in}}%
\pgfpathlineto{\pgfqpoint{5.439707in}{1.864648in}}%
\pgfpathclose%
\pgfusepath{fill}%
\end{pgfscope}%
\begin{pgfscope}%
\pgfpathrectangle{\pgfqpoint{3.722897in}{0.857143in}}{\pgfqpoint{2.627103in}{1.813434in}}%
\pgfusepath{clip}%
\pgfsetbuttcap%
\pgfsetmiterjoin%
\definecolor{currentfill}{rgb}{0.302379,0.450282,0.300122}%
\pgfsetfillcolor{currentfill}%
\pgfsetlinewidth{0.000000pt}%
\definecolor{currentstroke}{rgb}{0.000000,0.000000,0.000000}%
\pgfsetstrokecolor{currentstroke}%
\pgfsetstrokeopacity{0.000000}%
\pgfsetdash{}{0pt}%
\pgfpathmoveto{\pgfqpoint{5.450878in}{1.861226in}}%
\pgfpathlineto{\pgfqpoint{5.459814in}{1.861226in}}%
\pgfpathlineto{\pgfqpoint{5.459814in}{2.224066in}}%
\pgfpathlineto{\pgfqpoint{5.450878in}{2.224066in}}%
\pgfpathlineto{\pgfqpoint{5.450878in}{1.861226in}}%
\pgfpathclose%
\pgfusepath{fill}%
\end{pgfscope}%
\begin{pgfscope}%
\pgfpathrectangle{\pgfqpoint{3.722897in}{0.857143in}}{\pgfqpoint{2.627103in}{1.813434in}}%
\pgfusepath{clip}%
\pgfsetbuttcap%
\pgfsetmiterjoin%
\definecolor{currentfill}{rgb}{0.302379,0.450282,0.300122}%
\pgfsetfillcolor{currentfill}%
\pgfsetlinewidth{0.000000pt}%
\definecolor{currentstroke}{rgb}{0.000000,0.000000,0.000000}%
\pgfsetstrokecolor{currentstroke}%
\pgfsetstrokeopacity{0.000000}%
\pgfsetdash{}{0pt}%
\pgfpathmoveto{\pgfqpoint{5.462049in}{1.857235in}}%
\pgfpathlineto{\pgfqpoint{5.470985in}{1.857235in}}%
\pgfpathlineto{\pgfqpoint{5.470985in}{2.207765in}}%
\pgfpathlineto{\pgfqpoint{5.462049in}{2.207765in}}%
\pgfpathlineto{\pgfqpoint{5.462049in}{1.857235in}}%
\pgfpathclose%
\pgfusepath{fill}%
\end{pgfscope}%
\begin{pgfscope}%
\pgfpathrectangle{\pgfqpoint{3.722897in}{0.857143in}}{\pgfqpoint{2.627103in}{1.813434in}}%
\pgfusepath{clip}%
\pgfsetbuttcap%
\pgfsetmiterjoin%
\definecolor{currentfill}{rgb}{0.302379,0.450282,0.300122}%
\pgfsetfillcolor{currentfill}%
\pgfsetlinewidth{0.000000pt}%
\definecolor{currentstroke}{rgb}{0.000000,0.000000,0.000000}%
\pgfsetstrokecolor{currentstroke}%
\pgfsetstrokeopacity{0.000000}%
\pgfsetdash{}{0pt}%
\pgfpathmoveto{\pgfqpoint{5.473219in}{1.855019in}}%
\pgfpathlineto{\pgfqpoint{5.482156in}{1.855019in}}%
\pgfpathlineto{\pgfqpoint{5.482156in}{2.193145in}}%
\pgfpathlineto{\pgfqpoint{5.473219in}{2.193145in}}%
\pgfpathlineto{\pgfqpoint{5.473219in}{1.855019in}}%
\pgfpathclose%
\pgfusepath{fill}%
\end{pgfscope}%
\begin{pgfscope}%
\pgfpathrectangle{\pgfqpoint{3.722897in}{0.857143in}}{\pgfqpoint{2.627103in}{1.813434in}}%
\pgfusepath{clip}%
\pgfsetbuttcap%
\pgfsetmiterjoin%
\definecolor{currentfill}{rgb}{0.302379,0.450282,0.300122}%
\pgfsetfillcolor{currentfill}%
\pgfsetlinewidth{0.000000pt}%
\definecolor{currentstroke}{rgb}{0.000000,0.000000,0.000000}%
\pgfsetstrokecolor{currentstroke}%
\pgfsetstrokeopacity{0.000000}%
\pgfsetdash{}{0pt}%
\pgfpathmoveto{\pgfqpoint{5.484390in}{1.846584in}}%
\pgfpathlineto{\pgfqpoint{5.493326in}{1.846584in}}%
\pgfpathlineto{\pgfqpoint{5.493326in}{2.173235in}}%
\pgfpathlineto{\pgfqpoint{5.484390in}{2.173235in}}%
\pgfpathlineto{\pgfqpoint{5.484390in}{1.846584in}}%
\pgfpathclose%
\pgfusepath{fill}%
\end{pgfscope}%
\begin{pgfscope}%
\pgfpathrectangle{\pgfqpoint{3.722897in}{0.857143in}}{\pgfqpoint{2.627103in}{1.813434in}}%
\pgfusepath{clip}%
\pgfsetbuttcap%
\pgfsetmiterjoin%
\definecolor{currentfill}{rgb}{0.302379,0.450282,0.300122}%
\pgfsetfillcolor{currentfill}%
\pgfsetlinewidth{0.000000pt}%
\definecolor{currentstroke}{rgb}{0.000000,0.000000,0.000000}%
\pgfsetstrokecolor{currentstroke}%
\pgfsetstrokeopacity{0.000000}%
\pgfsetdash{}{0pt}%
\pgfpathmoveto{\pgfqpoint{5.495560in}{1.838721in}}%
\pgfpathlineto{\pgfqpoint{5.504497in}{1.838721in}}%
\pgfpathlineto{\pgfqpoint{5.504497in}{2.154574in}}%
\pgfpathlineto{\pgfqpoint{5.495560in}{2.154574in}}%
\pgfpathlineto{\pgfqpoint{5.495560in}{1.838721in}}%
\pgfpathclose%
\pgfusepath{fill}%
\end{pgfscope}%
\begin{pgfscope}%
\pgfpathrectangle{\pgfqpoint{3.722897in}{0.857143in}}{\pgfqpoint{2.627103in}{1.813434in}}%
\pgfusepath{clip}%
\pgfsetbuttcap%
\pgfsetmiterjoin%
\definecolor{currentfill}{rgb}{0.302379,0.450282,0.300122}%
\pgfsetfillcolor{currentfill}%
\pgfsetlinewidth{0.000000pt}%
\definecolor{currentstroke}{rgb}{0.000000,0.000000,0.000000}%
\pgfsetstrokecolor{currentstroke}%
\pgfsetstrokeopacity{0.000000}%
\pgfsetdash{}{0pt}%
\pgfpathmoveto{\pgfqpoint{5.506731in}{1.821421in}}%
\pgfpathlineto{\pgfqpoint{5.515667in}{1.821421in}}%
\pgfpathlineto{\pgfqpoint{5.515667in}{2.129267in}}%
\pgfpathlineto{\pgfqpoint{5.506731in}{2.129267in}}%
\pgfpathlineto{\pgfqpoint{5.506731in}{1.821421in}}%
\pgfpathclose%
\pgfusepath{fill}%
\end{pgfscope}%
\begin{pgfscope}%
\pgfpathrectangle{\pgfqpoint{3.722897in}{0.857143in}}{\pgfqpoint{2.627103in}{1.813434in}}%
\pgfusepath{clip}%
\pgfsetbuttcap%
\pgfsetmiterjoin%
\definecolor{currentfill}{rgb}{0.302379,0.450282,0.300122}%
\pgfsetfillcolor{currentfill}%
\pgfsetlinewidth{0.000000pt}%
\definecolor{currentstroke}{rgb}{0.000000,0.000000,0.000000}%
\pgfsetstrokecolor{currentstroke}%
\pgfsetstrokeopacity{0.000000}%
\pgfsetdash{}{0pt}%
\pgfpathmoveto{\pgfqpoint{5.517902in}{1.820844in}}%
\pgfpathlineto{\pgfqpoint{5.526838in}{1.820844in}}%
\pgfpathlineto{\pgfqpoint{5.526838in}{2.122924in}}%
\pgfpathlineto{\pgfqpoint{5.517902in}{2.122924in}}%
\pgfpathlineto{\pgfqpoint{5.517902in}{1.820844in}}%
\pgfpathclose%
\pgfusepath{fill}%
\end{pgfscope}%
\begin{pgfscope}%
\pgfpathrectangle{\pgfqpoint{3.722897in}{0.857143in}}{\pgfqpoint{2.627103in}{1.813434in}}%
\pgfusepath{clip}%
\pgfsetbuttcap%
\pgfsetmiterjoin%
\definecolor{currentfill}{rgb}{0.302379,0.450282,0.300122}%
\pgfsetfillcolor{currentfill}%
\pgfsetlinewidth{0.000000pt}%
\definecolor{currentstroke}{rgb}{0.000000,0.000000,0.000000}%
\pgfsetstrokecolor{currentstroke}%
\pgfsetstrokeopacity{0.000000}%
\pgfsetdash{}{0pt}%
\pgfpathmoveto{\pgfqpoint{5.529072in}{1.813884in}}%
\pgfpathlineto{\pgfqpoint{5.538009in}{1.813884in}}%
\pgfpathlineto{\pgfqpoint{5.538009in}{2.111323in}}%
\pgfpathlineto{\pgfqpoint{5.529072in}{2.111323in}}%
\pgfpathlineto{\pgfqpoint{5.529072in}{1.813884in}}%
\pgfpathclose%
\pgfusepath{fill}%
\end{pgfscope}%
\begin{pgfscope}%
\pgfpathrectangle{\pgfqpoint{3.722897in}{0.857143in}}{\pgfqpoint{2.627103in}{1.813434in}}%
\pgfusepath{clip}%
\pgfsetbuttcap%
\pgfsetmiterjoin%
\definecolor{currentfill}{rgb}{0.302379,0.450282,0.300122}%
\pgfsetfillcolor{currentfill}%
\pgfsetlinewidth{0.000000pt}%
\definecolor{currentstroke}{rgb}{0.000000,0.000000,0.000000}%
\pgfsetstrokecolor{currentstroke}%
\pgfsetstrokeopacity{0.000000}%
\pgfsetdash{}{0pt}%
\pgfpathmoveto{\pgfqpoint{5.540243in}{1.818994in}}%
\pgfpathlineto{\pgfqpoint{5.549179in}{1.818994in}}%
\pgfpathlineto{\pgfqpoint{5.549179in}{2.112532in}}%
\pgfpathlineto{\pgfqpoint{5.540243in}{2.112532in}}%
\pgfpathlineto{\pgfqpoint{5.540243in}{1.818994in}}%
\pgfpathclose%
\pgfusepath{fill}%
\end{pgfscope}%
\begin{pgfscope}%
\pgfpathrectangle{\pgfqpoint{3.722897in}{0.857143in}}{\pgfqpoint{2.627103in}{1.813434in}}%
\pgfusepath{clip}%
\pgfsetbuttcap%
\pgfsetmiterjoin%
\definecolor{currentfill}{rgb}{0.302379,0.450282,0.300122}%
\pgfsetfillcolor{currentfill}%
\pgfsetlinewidth{0.000000pt}%
\definecolor{currentstroke}{rgb}{0.000000,0.000000,0.000000}%
\pgfsetstrokecolor{currentstroke}%
\pgfsetstrokeopacity{0.000000}%
\pgfsetdash{}{0pt}%
\pgfpathmoveto{\pgfqpoint{5.551413in}{1.813884in}}%
\pgfpathlineto{\pgfqpoint{5.560350in}{1.813884in}}%
\pgfpathlineto{\pgfqpoint{5.560350in}{2.104619in}}%
\pgfpathlineto{\pgfqpoint{5.551413in}{2.104619in}}%
\pgfpathlineto{\pgfqpoint{5.551413in}{1.813884in}}%
\pgfpathclose%
\pgfusepath{fill}%
\end{pgfscope}%
\begin{pgfscope}%
\pgfpathrectangle{\pgfqpoint{3.722897in}{0.857143in}}{\pgfqpoint{2.627103in}{1.813434in}}%
\pgfusepath{clip}%
\pgfsetbuttcap%
\pgfsetmiterjoin%
\definecolor{currentfill}{rgb}{0.302379,0.450282,0.300122}%
\pgfsetfillcolor{currentfill}%
\pgfsetlinewidth{0.000000pt}%
\definecolor{currentstroke}{rgb}{0.000000,0.000000,0.000000}%
\pgfsetstrokecolor{currentstroke}%
\pgfsetstrokeopacity{0.000000}%
\pgfsetdash{}{0pt}%
\pgfpathmoveto{\pgfqpoint{5.562584in}{1.816719in}}%
\pgfpathlineto{\pgfqpoint{5.571521in}{1.816719in}}%
\pgfpathlineto{\pgfqpoint{5.571521in}{2.106266in}}%
\pgfpathlineto{\pgfqpoint{5.562584in}{2.106266in}}%
\pgfpathlineto{\pgfqpoint{5.562584in}{1.816719in}}%
\pgfpathclose%
\pgfusepath{fill}%
\end{pgfscope}%
\begin{pgfscope}%
\pgfpathrectangle{\pgfqpoint{3.722897in}{0.857143in}}{\pgfqpoint{2.627103in}{1.813434in}}%
\pgfusepath{clip}%
\pgfsetbuttcap%
\pgfsetmiterjoin%
\definecolor{currentfill}{rgb}{0.302379,0.450282,0.300122}%
\pgfsetfillcolor{currentfill}%
\pgfsetlinewidth{0.000000pt}%
\definecolor{currentstroke}{rgb}{0.000000,0.000000,0.000000}%
\pgfsetstrokecolor{currentstroke}%
\pgfsetstrokeopacity{0.000000}%
\pgfsetdash{}{0pt}%
\pgfpathmoveto{\pgfqpoint{5.573755in}{1.821168in}}%
\pgfpathlineto{\pgfqpoint{5.582691in}{1.821168in}}%
\pgfpathlineto{\pgfqpoint{5.582691in}{2.111460in}}%
\pgfpathlineto{\pgfqpoint{5.573755in}{2.111460in}}%
\pgfpathlineto{\pgfqpoint{5.573755in}{1.821168in}}%
\pgfpathclose%
\pgfusepath{fill}%
\end{pgfscope}%
\begin{pgfscope}%
\pgfpathrectangle{\pgfqpoint{3.722897in}{0.857143in}}{\pgfqpoint{2.627103in}{1.813434in}}%
\pgfusepath{clip}%
\pgfsetbuttcap%
\pgfsetmiterjoin%
\definecolor{currentfill}{rgb}{0.302379,0.450282,0.300122}%
\pgfsetfillcolor{currentfill}%
\pgfsetlinewidth{0.000000pt}%
\definecolor{currentstroke}{rgb}{0.000000,0.000000,0.000000}%
\pgfsetstrokecolor{currentstroke}%
\pgfsetstrokeopacity{0.000000}%
\pgfsetdash{}{0pt}%
\pgfpathmoveto{\pgfqpoint{5.584925in}{1.821396in}}%
\pgfpathlineto{\pgfqpoint{5.593862in}{1.821396in}}%
\pgfpathlineto{\pgfqpoint{5.593862in}{2.114176in}}%
\pgfpathlineto{\pgfqpoint{5.584925in}{2.114176in}}%
\pgfpathlineto{\pgfqpoint{5.584925in}{1.821396in}}%
\pgfpathclose%
\pgfusepath{fill}%
\end{pgfscope}%
\begin{pgfscope}%
\pgfpathrectangle{\pgfqpoint{3.722897in}{0.857143in}}{\pgfqpoint{2.627103in}{1.813434in}}%
\pgfusepath{clip}%
\pgfsetbuttcap%
\pgfsetmiterjoin%
\definecolor{currentfill}{rgb}{0.302379,0.450282,0.300122}%
\pgfsetfillcolor{currentfill}%
\pgfsetlinewidth{0.000000pt}%
\definecolor{currentstroke}{rgb}{0.000000,0.000000,0.000000}%
\pgfsetstrokecolor{currentstroke}%
\pgfsetstrokeopacity{0.000000}%
\pgfsetdash{}{0pt}%
\pgfpathmoveto{\pgfqpoint{5.596096in}{1.823249in}}%
\pgfpathlineto{\pgfqpoint{5.605032in}{1.823249in}}%
\pgfpathlineto{\pgfqpoint{5.605032in}{2.118451in}}%
\pgfpathlineto{\pgfqpoint{5.596096in}{2.118451in}}%
\pgfpathlineto{\pgfqpoint{5.596096in}{1.823249in}}%
\pgfpathclose%
\pgfusepath{fill}%
\end{pgfscope}%
\begin{pgfscope}%
\pgfpathrectangle{\pgfqpoint{3.722897in}{0.857143in}}{\pgfqpoint{2.627103in}{1.813434in}}%
\pgfusepath{clip}%
\pgfsetbuttcap%
\pgfsetmiterjoin%
\definecolor{currentfill}{rgb}{0.302379,0.450282,0.300122}%
\pgfsetfillcolor{currentfill}%
\pgfsetlinewidth{0.000000pt}%
\definecolor{currentstroke}{rgb}{0.000000,0.000000,0.000000}%
\pgfsetstrokecolor{currentstroke}%
\pgfsetstrokeopacity{0.000000}%
\pgfsetdash{}{0pt}%
\pgfpathmoveto{\pgfqpoint{5.607266in}{1.831711in}}%
\pgfpathlineto{\pgfqpoint{5.616203in}{1.831711in}}%
\pgfpathlineto{\pgfqpoint{5.616203in}{2.129231in}}%
\pgfpathlineto{\pgfqpoint{5.607266in}{2.129231in}}%
\pgfpathlineto{\pgfqpoint{5.607266in}{1.831711in}}%
\pgfpathclose%
\pgfusepath{fill}%
\end{pgfscope}%
\begin{pgfscope}%
\pgfpathrectangle{\pgfqpoint{3.722897in}{0.857143in}}{\pgfqpoint{2.627103in}{1.813434in}}%
\pgfusepath{clip}%
\pgfsetbuttcap%
\pgfsetmiterjoin%
\definecolor{currentfill}{rgb}{0.302379,0.450282,0.300122}%
\pgfsetfillcolor{currentfill}%
\pgfsetlinewidth{0.000000pt}%
\definecolor{currentstroke}{rgb}{0.000000,0.000000,0.000000}%
\pgfsetstrokecolor{currentstroke}%
\pgfsetstrokeopacity{0.000000}%
\pgfsetdash{}{0pt}%
\pgfpathmoveto{\pgfqpoint{5.618437in}{1.840008in}}%
\pgfpathlineto{\pgfqpoint{5.627374in}{1.840008in}}%
\pgfpathlineto{\pgfqpoint{5.627374in}{2.138868in}}%
\pgfpathlineto{\pgfqpoint{5.618437in}{2.138868in}}%
\pgfpathlineto{\pgfqpoint{5.618437in}{1.840008in}}%
\pgfpathclose%
\pgfusepath{fill}%
\end{pgfscope}%
\begin{pgfscope}%
\pgfpathrectangle{\pgfqpoint{3.722897in}{0.857143in}}{\pgfqpoint{2.627103in}{1.813434in}}%
\pgfusepath{clip}%
\pgfsetbuttcap%
\pgfsetmiterjoin%
\definecolor{currentfill}{rgb}{0.302379,0.450282,0.300122}%
\pgfsetfillcolor{currentfill}%
\pgfsetlinewidth{0.000000pt}%
\definecolor{currentstroke}{rgb}{0.000000,0.000000,0.000000}%
\pgfsetstrokecolor{currentstroke}%
\pgfsetstrokeopacity{0.000000}%
\pgfsetdash{}{0pt}%
\pgfpathmoveto{\pgfqpoint{5.629608in}{1.844204in}}%
\pgfpathlineto{\pgfqpoint{5.638544in}{1.844204in}}%
\pgfpathlineto{\pgfqpoint{5.638544in}{2.144659in}}%
\pgfpathlineto{\pgfqpoint{5.629608in}{2.144659in}}%
\pgfpathlineto{\pgfqpoint{5.629608in}{1.844204in}}%
\pgfpathclose%
\pgfusepath{fill}%
\end{pgfscope}%
\begin{pgfscope}%
\pgfpathrectangle{\pgfqpoint{3.722897in}{0.857143in}}{\pgfqpoint{2.627103in}{1.813434in}}%
\pgfusepath{clip}%
\pgfsetbuttcap%
\pgfsetmiterjoin%
\definecolor{currentfill}{rgb}{0.302379,0.450282,0.300122}%
\pgfsetfillcolor{currentfill}%
\pgfsetlinewidth{0.000000pt}%
\definecolor{currentstroke}{rgb}{0.000000,0.000000,0.000000}%
\pgfsetstrokecolor{currentstroke}%
\pgfsetstrokeopacity{0.000000}%
\pgfsetdash{}{0pt}%
\pgfpathmoveto{\pgfqpoint{5.640778in}{1.860224in}}%
\pgfpathlineto{\pgfqpoint{5.649715in}{1.860224in}}%
\pgfpathlineto{\pgfqpoint{5.649715in}{2.160388in}}%
\pgfpathlineto{\pgfqpoint{5.640778in}{2.160388in}}%
\pgfpathlineto{\pgfqpoint{5.640778in}{1.860224in}}%
\pgfpathclose%
\pgfusepath{fill}%
\end{pgfscope}%
\begin{pgfscope}%
\pgfpathrectangle{\pgfqpoint{3.722897in}{0.857143in}}{\pgfqpoint{2.627103in}{1.813434in}}%
\pgfusepath{clip}%
\pgfsetbuttcap%
\pgfsetmiterjoin%
\definecolor{currentfill}{rgb}{0.302379,0.450282,0.300122}%
\pgfsetfillcolor{currentfill}%
\pgfsetlinewidth{0.000000pt}%
\definecolor{currentstroke}{rgb}{0.000000,0.000000,0.000000}%
\pgfsetstrokecolor{currentstroke}%
\pgfsetstrokeopacity{0.000000}%
\pgfsetdash{}{0pt}%
\pgfpathmoveto{\pgfqpoint{5.651949in}{1.838012in}}%
\pgfpathlineto{\pgfqpoint{5.660885in}{1.838012in}}%
\pgfpathlineto{\pgfqpoint{5.660885in}{2.137450in}}%
\pgfpathlineto{\pgfqpoint{5.651949in}{2.137450in}}%
\pgfpathlineto{\pgfqpoint{5.651949in}{1.838012in}}%
\pgfpathclose%
\pgfusepath{fill}%
\end{pgfscope}%
\begin{pgfscope}%
\pgfpathrectangle{\pgfqpoint{3.722897in}{0.857143in}}{\pgfqpoint{2.627103in}{1.813434in}}%
\pgfusepath{clip}%
\pgfsetbuttcap%
\pgfsetmiterjoin%
\definecolor{currentfill}{rgb}{0.302379,0.450282,0.300122}%
\pgfsetfillcolor{currentfill}%
\pgfsetlinewidth{0.000000pt}%
\definecolor{currentstroke}{rgb}{0.000000,0.000000,0.000000}%
\pgfsetstrokecolor{currentstroke}%
\pgfsetstrokeopacity{0.000000}%
\pgfsetdash{}{0pt}%
\pgfpathmoveto{\pgfqpoint{5.663119in}{1.852502in}}%
\pgfpathlineto{\pgfqpoint{5.672056in}{1.852502in}}%
\pgfpathlineto{\pgfqpoint{5.672056in}{2.150853in}}%
\pgfpathlineto{\pgfqpoint{5.663119in}{2.150853in}}%
\pgfpathlineto{\pgfqpoint{5.663119in}{1.852502in}}%
\pgfpathclose%
\pgfusepath{fill}%
\end{pgfscope}%
\begin{pgfscope}%
\pgfpathrectangle{\pgfqpoint{3.722897in}{0.857143in}}{\pgfqpoint{2.627103in}{1.813434in}}%
\pgfusepath{clip}%
\pgfsetbuttcap%
\pgfsetmiterjoin%
\definecolor{currentfill}{rgb}{0.302379,0.450282,0.300122}%
\pgfsetfillcolor{currentfill}%
\pgfsetlinewidth{0.000000pt}%
\definecolor{currentstroke}{rgb}{0.000000,0.000000,0.000000}%
\pgfsetstrokecolor{currentstroke}%
\pgfsetstrokeopacity{0.000000}%
\pgfsetdash{}{0pt}%
\pgfpathmoveto{\pgfqpoint{5.674290in}{1.859109in}}%
\pgfpathlineto{\pgfqpoint{5.683227in}{1.859109in}}%
\pgfpathlineto{\pgfqpoint{5.683227in}{2.153958in}}%
\pgfpathlineto{\pgfqpoint{5.674290in}{2.153958in}}%
\pgfpathlineto{\pgfqpoint{5.674290in}{1.859109in}}%
\pgfpathclose%
\pgfusepath{fill}%
\end{pgfscope}%
\begin{pgfscope}%
\pgfpathrectangle{\pgfqpoint{3.722897in}{0.857143in}}{\pgfqpoint{2.627103in}{1.813434in}}%
\pgfusepath{clip}%
\pgfsetbuttcap%
\pgfsetmiterjoin%
\definecolor{currentfill}{rgb}{0.302379,0.450282,0.300122}%
\pgfsetfillcolor{currentfill}%
\pgfsetlinewidth{0.000000pt}%
\definecolor{currentstroke}{rgb}{0.000000,0.000000,0.000000}%
\pgfsetstrokecolor{currentstroke}%
\pgfsetstrokeopacity{0.000000}%
\pgfsetdash{}{0pt}%
\pgfpathmoveto{\pgfqpoint{5.685461in}{1.857978in}}%
\pgfpathlineto{\pgfqpoint{5.694397in}{1.857978in}}%
\pgfpathlineto{\pgfqpoint{5.694397in}{2.148429in}}%
\pgfpathlineto{\pgfqpoint{5.685461in}{2.148429in}}%
\pgfpathlineto{\pgfqpoint{5.685461in}{1.857978in}}%
\pgfpathclose%
\pgfusepath{fill}%
\end{pgfscope}%
\begin{pgfscope}%
\pgfpathrectangle{\pgfqpoint{3.722897in}{0.857143in}}{\pgfqpoint{2.627103in}{1.813434in}}%
\pgfusepath{clip}%
\pgfsetbuttcap%
\pgfsetmiterjoin%
\definecolor{currentfill}{rgb}{0.302379,0.450282,0.300122}%
\pgfsetfillcolor{currentfill}%
\pgfsetlinewidth{0.000000pt}%
\definecolor{currentstroke}{rgb}{0.000000,0.000000,0.000000}%
\pgfsetstrokecolor{currentstroke}%
\pgfsetstrokeopacity{0.000000}%
\pgfsetdash{}{0pt}%
\pgfpathmoveto{\pgfqpoint{5.696631in}{1.861122in}}%
\pgfpathlineto{\pgfqpoint{5.705568in}{1.861122in}}%
\pgfpathlineto{\pgfqpoint{5.705568in}{2.145919in}}%
\pgfpathlineto{\pgfqpoint{5.696631in}{2.145919in}}%
\pgfpathlineto{\pgfqpoint{5.696631in}{1.861122in}}%
\pgfpathclose%
\pgfusepath{fill}%
\end{pgfscope}%
\begin{pgfscope}%
\pgfpathrectangle{\pgfqpoint{3.722897in}{0.857143in}}{\pgfqpoint{2.627103in}{1.813434in}}%
\pgfusepath{clip}%
\pgfsetbuttcap%
\pgfsetmiterjoin%
\definecolor{currentfill}{rgb}{0.302379,0.450282,0.300122}%
\pgfsetfillcolor{currentfill}%
\pgfsetlinewidth{0.000000pt}%
\definecolor{currentstroke}{rgb}{0.000000,0.000000,0.000000}%
\pgfsetstrokecolor{currentstroke}%
\pgfsetstrokeopacity{0.000000}%
\pgfsetdash{}{0pt}%
\pgfpathmoveto{\pgfqpoint{5.707802in}{1.858448in}}%
\pgfpathlineto{\pgfqpoint{5.716738in}{1.858448in}}%
\pgfpathlineto{\pgfqpoint{5.716738in}{2.134875in}}%
\pgfpathlineto{\pgfqpoint{5.707802in}{2.134875in}}%
\pgfpathlineto{\pgfqpoint{5.707802in}{1.858448in}}%
\pgfpathclose%
\pgfusepath{fill}%
\end{pgfscope}%
\begin{pgfscope}%
\pgfpathrectangle{\pgfqpoint{3.722897in}{0.857143in}}{\pgfqpoint{2.627103in}{1.813434in}}%
\pgfusepath{clip}%
\pgfsetbuttcap%
\pgfsetmiterjoin%
\definecolor{currentfill}{rgb}{0.302379,0.450282,0.300122}%
\pgfsetfillcolor{currentfill}%
\pgfsetlinewidth{0.000000pt}%
\definecolor{currentstroke}{rgb}{0.000000,0.000000,0.000000}%
\pgfsetstrokecolor{currentstroke}%
\pgfsetstrokeopacity{0.000000}%
\pgfsetdash{}{0pt}%
\pgfpathmoveto{\pgfqpoint{5.718972in}{1.843662in}}%
\pgfpathlineto{\pgfqpoint{5.727909in}{1.843662in}}%
\pgfpathlineto{\pgfqpoint{5.727909in}{2.109062in}}%
\pgfpathlineto{\pgfqpoint{5.718972in}{2.109062in}}%
\pgfpathlineto{\pgfqpoint{5.718972in}{1.843662in}}%
\pgfpathclose%
\pgfusepath{fill}%
\end{pgfscope}%
\begin{pgfscope}%
\pgfpathrectangle{\pgfqpoint{3.722897in}{0.857143in}}{\pgfqpoint{2.627103in}{1.813434in}}%
\pgfusepath{clip}%
\pgfsetbuttcap%
\pgfsetmiterjoin%
\definecolor{currentfill}{rgb}{0.302379,0.450282,0.300122}%
\pgfsetfillcolor{currentfill}%
\pgfsetlinewidth{0.000000pt}%
\definecolor{currentstroke}{rgb}{0.000000,0.000000,0.000000}%
\pgfsetstrokecolor{currentstroke}%
\pgfsetstrokeopacity{0.000000}%
\pgfsetdash{}{0pt}%
\pgfpathmoveto{\pgfqpoint{5.730143in}{1.858824in}}%
\pgfpathlineto{\pgfqpoint{5.739080in}{1.858824in}}%
\pgfpathlineto{\pgfqpoint{5.739080in}{2.106504in}}%
\pgfpathlineto{\pgfqpoint{5.730143in}{2.106504in}}%
\pgfpathlineto{\pgfqpoint{5.730143in}{1.858824in}}%
\pgfpathclose%
\pgfusepath{fill}%
\end{pgfscope}%
\begin{pgfscope}%
\pgfpathrectangle{\pgfqpoint{3.722897in}{0.857143in}}{\pgfqpoint{2.627103in}{1.813434in}}%
\pgfusepath{clip}%
\pgfsetbuttcap%
\pgfsetmiterjoin%
\definecolor{currentfill}{rgb}{0.302379,0.450282,0.300122}%
\pgfsetfillcolor{currentfill}%
\pgfsetlinewidth{0.000000pt}%
\definecolor{currentstroke}{rgb}{0.000000,0.000000,0.000000}%
\pgfsetstrokecolor{currentstroke}%
\pgfsetstrokeopacity{0.000000}%
\pgfsetdash{}{0pt}%
\pgfpathmoveto{\pgfqpoint{5.741314in}{1.872182in}}%
\pgfpathlineto{\pgfqpoint{5.750250in}{1.872182in}}%
\pgfpathlineto{\pgfqpoint{5.750250in}{2.093098in}}%
\pgfpathlineto{\pgfqpoint{5.741314in}{2.093098in}}%
\pgfpathlineto{\pgfqpoint{5.741314in}{1.872182in}}%
\pgfpathclose%
\pgfusepath{fill}%
\end{pgfscope}%
\begin{pgfscope}%
\pgfpathrectangle{\pgfqpoint{3.722897in}{0.857143in}}{\pgfqpoint{2.627103in}{1.813434in}}%
\pgfusepath{clip}%
\pgfsetbuttcap%
\pgfsetmiterjoin%
\definecolor{currentfill}{rgb}{0.302379,0.450282,0.300122}%
\pgfsetfillcolor{currentfill}%
\pgfsetlinewidth{0.000000pt}%
\definecolor{currentstroke}{rgb}{0.000000,0.000000,0.000000}%
\pgfsetstrokecolor{currentstroke}%
\pgfsetstrokeopacity{0.000000}%
\pgfsetdash{}{0pt}%
\pgfpathmoveto{\pgfqpoint{5.752484in}{1.876884in}}%
\pgfpathlineto{\pgfqpoint{5.761421in}{1.876884in}}%
\pgfpathlineto{\pgfqpoint{5.761421in}{2.064735in}}%
\pgfpathlineto{\pgfqpoint{5.752484in}{2.064735in}}%
\pgfpathlineto{\pgfqpoint{5.752484in}{1.876884in}}%
\pgfpathclose%
\pgfusepath{fill}%
\end{pgfscope}%
\begin{pgfscope}%
\pgfpathrectangle{\pgfqpoint{3.722897in}{0.857143in}}{\pgfqpoint{2.627103in}{1.813434in}}%
\pgfusepath{clip}%
\pgfsetbuttcap%
\pgfsetmiterjoin%
\definecolor{currentfill}{rgb}{0.302379,0.450282,0.300122}%
\pgfsetfillcolor{currentfill}%
\pgfsetlinewidth{0.000000pt}%
\definecolor{currentstroke}{rgb}{0.000000,0.000000,0.000000}%
\pgfsetstrokecolor{currentstroke}%
\pgfsetstrokeopacity{0.000000}%
\pgfsetdash{}{0pt}%
\pgfpathmoveto{\pgfqpoint{5.763655in}{1.864664in}}%
\pgfpathlineto{\pgfqpoint{5.772591in}{1.864664in}}%
\pgfpathlineto{\pgfqpoint{5.772591in}{2.019813in}}%
\pgfpathlineto{\pgfqpoint{5.763655in}{2.019813in}}%
\pgfpathlineto{\pgfqpoint{5.763655in}{1.864664in}}%
\pgfpathclose%
\pgfusepath{fill}%
\end{pgfscope}%
\begin{pgfscope}%
\pgfpathrectangle{\pgfqpoint{3.722897in}{0.857143in}}{\pgfqpoint{2.627103in}{1.813434in}}%
\pgfusepath{clip}%
\pgfsetbuttcap%
\pgfsetmiterjoin%
\definecolor{currentfill}{rgb}{0.302379,0.450282,0.300122}%
\pgfsetfillcolor{currentfill}%
\pgfsetlinewidth{0.000000pt}%
\definecolor{currentstroke}{rgb}{0.000000,0.000000,0.000000}%
\pgfsetstrokecolor{currentstroke}%
\pgfsetstrokeopacity{0.000000}%
\pgfsetdash{}{0pt}%
\pgfpathmoveto{\pgfqpoint{5.774826in}{1.866464in}}%
\pgfpathlineto{\pgfqpoint{5.783762in}{1.866464in}}%
\pgfpathlineto{\pgfqpoint{5.783762in}{1.990008in}}%
\pgfpathlineto{\pgfqpoint{5.774826in}{1.990008in}}%
\pgfpathlineto{\pgfqpoint{5.774826in}{1.866464in}}%
\pgfpathclose%
\pgfusepath{fill}%
\end{pgfscope}%
\begin{pgfscope}%
\pgfpathrectangle{\pgfqpoint{3.722897in}{0.857143in}}{\pgfqpoint{2.627103in}{1.813434in}}%
\pgfusepath{clip}%
\pgfsetbuttcap%
\pgfsetmiterjoin%
\definecolor{currentfill}{rgb}{0.302379,0.450282,0.300122}%
\pgfsetfillcolor{currentfill}%
\pgfsetlinewidth{0.000000pt}%
\definecolor{currentstroke}{rgb}{0.000000,0.000000,0.000000}%
\pgfsetstrokecolor{currentstroke}%
\pgfsetstrokeopacity{0.000000}%
\pgfsetdash{}{0pt}%
\pgfpathmoveto{\pgfqpoint{5.785996in}{1.872264in}}%
\pgfpathlineto{\pgfqpoint{5.794933in}{1.872264in}}%
\pgfpathlineto{\pgfqpoint{5.794933in}{1.964114in}}%
\pgfpathlineto{\pgfqpoint{5.785996in}{1.964114in}}%
\pgfpathlineto{\pgfqpoint{5.785996in}{1.872264in}}%
\pgfpathclose%
\pgfusepath{fill}%
\end{pgfscope}%
\begin{pgfscope}%
\pgfpathrectangle{\pgfqpoint{3.722897in}{0.857143in}}{\pgfqpoint{2.627103in}{1.813434in}}%
\pgfusepath{clip}%
\pgfsetbuttcap%
\pgfsetmiterjoin%
\definecolor{currentfill}{rgb}{0.302379,0.450282,0.300122}%
\pgfsetfillcolor{currentfill}%
\pgfsetlinewidth{0.000000pt}%
\definecolor{currentstroke}{rgb}{0.000000,0.000000,0.000000}%
\pgfsetstrokecolor{currentstroke}%
\pgfsetstrokeopacity{0.000000}%
\pgfsetdash{}{0pt}%
\pgfpathmoveto{\pgfqpoint{5.797167in}{1.863004in}}%
\pgfpathlineto{\pgfqpoint{5.806103in}{1.863004in}}%
\pgfpathlineto{\pgfqpoint{5.806103in}{1.925940in}}%
\pgfpathlineto{\pgfqpoint{5.797167in}{1.925940in}}%
\pgfpathlineto{\pgfqpoint{5.797167in}{1.863004in}}%
\pgfpathclose%
\pgfusepath{fill}%
\end{pgfscope}%
\begin{pgfscope}%
\pgfpathrectangle{\pgfqpoint{3.722897in}{0.857143in}}{\pgfqpoint{2.627103in}{1.813434in}}%
\pgfusepath{clip}%
\pgfsetbuttcap%
\pgfsetmiterjoin%
\definecolor{currentfill}{rgb}{0.302379,0.450282,0.300122}%
\pgfsetfillcolor{currentfill}%
\pgfsetlinewidth{0.000000pt}%
\definecolor{currentstroke}{rgb}{0.000000,0.000000,0.000000}%
\pgfsetstrokecolor{currentstroke}%
\pgfsetstrokeopacity{0.000000}%
\pgfsetdash{}{0pt}%
\pgfpathmoveto{\pgfqpoint{5.808337in}{1.871294in}}%
\pgfpathlineto{\pgfqpoint{5.817274in}{1.871294in}}%
\pgfpathlineto{\pgfqpoint{5.817274in}{1.907462in}}%
\pgfpathlineto{\pgfqpoint{5.808337in}{1.907462in}}%
\pgfpathlineto{\pgfqpoint{5.808337in}{1.871294in}}%
\pgfpathclose%
\pgfusepath{fill}%
\end{pgfscope}%
\begin{pgfscope}%
\pgfpathrectangle{\pgfqpoint{3.722897in}{0.857143in}}{\pgfqpoint{2.627103in}{1.813434in}}%
\pgfusepath{clip}%
\pgfsetbuttcap%
\pgfsetmiterjoin%
\definecolor{currentfill}{rgb}{0.302379,0.450282,0.300122}%
\pgfsetfillcolor{currentfill}%
\pgfsetlinewidth{0.000000pt}%
\definecolor{currentstroke}{rgb}{0.000000,0.000000,0.000000}%
\pgfsetstrokecolor{currentstroke}%
\pgfsetstrokeopacity{0.000000}%
\pgfsetdash{}{0pt}%
\pgfpathmoveto{\pgfqpoint{5.819508in}{1.867631in}}%
\pgfpathlineto{\pgfqpoint{5.828444in}{1.867631in}}%
\pgfpathlineto{\pgfqpoint{5.828444in}{1.877348in}}%
\pgfpathlineto{\pgfqpoint{5.819508in}{1.877348in}}%
\pgfpathlineto{\pgfqpoint{5.819508in}{1.867631in}}%
\pgfpathclose%
\pgfusepath{fill}%
\end{pgfscope}%
\begin{pgfscope}%
\pgfpathrectangle{\pgfqpoint{3.722897in}{0.857143in}}{\pgfqpoint{2.627103in}{1.813434in}}%
\pgfusepath{clip}%
\pgfsetbuttcap%
\pgfsetmiterjoin%
\definecolor{currentfill}{rgb}{0.302379,0.450282,0.300122}%
\pgfsetfillcolor{currentfill}%
\pgfsetlinewidth{0.000000pt}%
\definecolor{currentstroke}{rgb}{0.000000,0.000000,0.000000}%
\pgfsetstrokecolor{currentstroke}%
\pgfsetstrokeopacity{0.000000}%
\pgfsetdash{}{0pt}%
\pgfpathmoveto{\pgfqpoint{5.830679in}{1.813884in}}%
\pgfpathlineto{\pgfqpoint{5.839615in}{1.813884in}}%
\pgfpathlineto{\pgfqpoint{5.839615in}{1.798345in}}%
\pgfpathlineto{\pgfqpoint{5.830679in}{1.798345in}}%
\pgfpathlineto{\pgfqpoint{5.830679in}{1.813884in}}%
\pgfpathclose%
\pgfusepath{fill}%
\end{pgfscope}%
\begin{pgfscope}%
\pgfpathrectangle{\pgfqpoint{3.722897in}{0.857143in}}{\pgfqpoint{2.627103in}{1.813434in}}%
\pgfusepath{clip}%
\pgfsetbuttcap%
\pgfsetmiterjoin%
\definecolor{currentfill}{rgb}{0.302379,0.450282,0.300122}%
\pgfsetfillcolor{currentfill}%
\pgfsetlinewidth{0.000000pt}%
\definecolor{currentstroke}{rgb}{0.000000,0.000000,0.000000}%
\pgfsetstrokecolor{currentstroke}%
\pgfsetstrokeopacity{0.000000}%
\pgfsetdash{}{0pt}%
\pgfpathmoveto{\pgfqpoint{5.841849in}{1.803911in}}%
\pgfpathlineto{\pgfqpoint{5.850786in}{1.803911in}}%
\pgfpathlineto{\pgfqpoint{5.850786in}{1.767409in}}%
\pgfpathlineto{\pgfqpoint{5.841849in}{1.767409in}}%
\pgfpathlineto{\pgfqpoint{5.841849in}{1.803911in}}%
\pgfpathclose%
\pgfusepath{fill}%
\end{pgfscope}%
\begin{pgfscope}%
\pgfpathrectangle{\pgfqpoint{3.722897in}{0.857143in}}{\pgfqpoint{2.627103in}{1.813434in}}%
\pgfusepath{clip}%
\pgfsetbuttcap%
\pgfsetmiterjoin%
\definecolor{currentfill}{rgb}{0.302379,0.450282,0.300122}%
\pgfsetfillcolor{currentfill}%
\pgfsetlinewidth{0.000000pt}%
\definecolor{currentstroke}{rgb}{0.000000,0.000000,0.000000}%
\pgfsetstrokecolor{currentstroke}%
\pgfsetstrokeopacity{0.000000}%
\pgfsetdash{}{0pt}%
\pgfpathmoveto{\pgfqpoint{5.853020in}{1.805759in}}%
\pgfpathlineto{\pgfqpoint{5.861956in}{1.805759in}}%
\pgfpathlineto{\pgfqpoint{5.861956in}{1.749383in}}%
\pgfpathlineto{\pgfqpoint{5.853020in}{1.749383in}}%
\pgfpathlineto{\pgfqpoint{5.853020in}{1.805759in}}%
\pgfpathclose%
\pgfusepath{fill}%
\end{pgfscope}%
\begin{pgfscope}%
\pgfpathrectangle{\pgfqpoint{3.722897in}{0.857143in}}{\pgfqpoint{2.627103in}{1.813434in}}%
\pgfusepath{clip}%
\pgfsetbuttcap%
\pgfsetmiterjoin%
\definecolor{currentfill}{rgb}{0.302379,0.450282,0.300122}%
\pgfsetfillcolor{currentfill}%
\pgfsetlinewidth{0.000000pt}%
\definecolor{currentstroke}{rgb}{0.000000,0.000000,0.000000}%
\pgfsetstrokecolor{currentstroke}%
\pgfsetstrokeopacity{0.000000}%
\pgfsetdash{}{0pt}%
\pgfpathmoveto{\pgfqpoint{5.864190in}{1.813884in}}%
\pgfpathlineto{\pgfqpoint{5.873127in}{1.813884in}}%
\pgfpathlineto{\pgfqpoint{5.873127in}{1.736198in}}%
\pgfpathlineto{\pgfqpoint{5.864190in}{1.736198in}}%
\pgfpathlineto{\pgfqpoint{5.864190in}{1.813884in}}%
\pgfpathclose%
\pgfusepath{fill}%
\end{pgfscope}%
\begin{pgfscope}%
\pgfpathrectangle{\pgfqpoint{3.722897in}{0.857143in}}{\pgfqpoint{2.627103in}{1.813434in}}%
\pgfusepath{clip}%
\pgfsetbuttcap%
\pgfsetmiterjoin%
\definecolor{currentfill}{rgb}{0.302379,0.450282,0.300122}%
\pgfsetfillcolor{currentfill}%
\pgfsetlinewidth{0.000000pt}%
\definecolor{currentstroke}{rgb}{0.000000,0.000000,0.000000}%
\pgfsetstrokecolor{currentstroke}%
\pgfsetstrokeopacity{0.000000}%
\pgfsetdash{}{0pt}%
\pgfpathmoveto{\pgfqpoint{5.875361in}{1.797163in}}%
\pgfpathlineto{\pgfqpoint{5.884297in}{1.797163in}}%
\pgfpathlineto{\pgfqpoint{5.884297in}{1.700659in}}%
\pgfpathlineto{\pgfqpoint{5.875361in}{1.700659in}}%
\pgfpathlineto{\pgfqpoint{5.875361in}{1.797163in}}%
\pgfpathclose%
\pgfusepath{fill}%
\end{pgfscope}%
\begin{pgfscope}%
\pgfpathrectangle{\pgfqpoint{3.722897in}{0.857143in}}{\pgfqpoint{2.627103in}{1.813434in}}%
\pgfusepath{clip}%
\pgfsetbuttcap%
\pgfsetmiterjoin%
\definecolor{currentfill}{rgb}{0.302379,0.450282,0.300122}%
\pgfsetfillcolor{currentfill}%
\pgfsetlinewidth{0.000000pt}%
\definecolor{currentstroke}{rgb}{0.000000,0.000000,0.000000}%
\pgfsetstrokecolor{currentstroke}%
\pgfsetstrokeopacity{0.000000}%
\pgfsetdash{}{0pt}%
\pgfpathmoveto{\pgfqpoint{5.886532in}{1.805619in}}%
\pgfpathlineto{\pgfqpoint{5.895468in}{1.805619in}}%
\pgfpathlineto{\pgfqpoint{5.895468in}{1.692036in}}%
\pgfpathlineto{\pgfqpoint{5.886532in}{1.692036in}}%
\pgfpathlineto{\pgfqpoint{5.886532in}{1.805619in}}%
\pgfpathclose%
\pgfusepath{fill}%
\end{pgfscope}%
\begin{pgfscope}%
\pgfpathrectangle{\pgfqpoint{3.722897in}{0.857143in}}{\pgfqpoint{2.627103in}{1.813434in}}%
\pgfusepath{clip}%
\pgfsetbuttcap%
\pgfsetmiterjoin%
\definecolor{currentfill}{rgb}{0.302379,0.450282,0.300122}%
\pgfsetfillcolor{currentfill}%
\pgfsetlinewidth{0.000000pt}%
\definecolor{currentstroke}{rgb}{0.000000,0.000000,0.000000}%
\pgfsetstrokecolor{currentstroke}%
\pgfsetstrokeopacity{0.000000}%
\pgfsetdash{}{0pt}%
\pgfpathmoveto{\pgfqpoint{5.897702in}{1.797542in}}%
\pgfpathlineto{\pgfqpoint{5.906639in}{1.797542in}}%
\pgfpathlineto{\pgfqpoint{5.906639in}{1.667045in}}%
\pgfpathlineto{\pgfqpoint{5.897702in}{1.667045in}}%
\pgfpathlineto{\pgfqpoint{5.897702in}{1.797542in}}%
\pgfpathclose%
\pgfusepath{fill}%
\end{pgfscope}%
\begin{pgfscope}%
\pgfpathrectangle{\pgfqpoint{3.722897in}{0.857143in}}{\pgfqpoint{2.627103in}{1.813434in}}%
\pgfusepath{clip}%
\pgfsetbuttcap%
\pgfsetmiterjoin%
\definecolor{currentfill}{rgb}{0.302379,0.450282,0.300122}%
\pgfsetfillcolor{currentfill}%
\pgfsetlinewidth{0.000000pt}%
\definecolor{currentstroke}{rgb}{0.000000,0.000000,0.000000}%
\pgfsetstrokecolor{currentstroke}%
\pgfsetstrokeopacity{0.000000}%
\pgfsetdash{}{0pt}%
\pgfpathmoveto{\pgfqpoint{5.908873in}{1.796087in}}%
\pgfpathlineto{\pgfqpoint{5.917809in}{1.796087in}}%
\pgfpathlineto{\pgfqpoint{5.917809in}{1.649765in}}%
\pgfpathlineto{\pgfqpoint{5.908873in}{1.649765in}}%
\pgfpathlineto{\pgfqpoint{5.908873in}{1.796087in}}%
\pgfpathclose%
\pgfusepath{fill}%
\end{pgfscope}%
\begin{pgfscope}%
\pgfpathrectangle{\pgfqpoint{3.722897in}{0.857143in}}{\pgfqpoint{2.627103in}{1.813434in}}%
\pgfusepath{clip}%
\pgfsetbuttcap%
\pgfsetmiterjoin%
\definecolor{currentfill}{rgb}{0.302379,0.450282,0.300122}%
\pgfsetfillcolor{currentfill}%
\pgfsetlinewidth{0.000000pt}%
\definecolor{currentstroke}{rgb}{0.000000,0.000000,0.000000}%
\pgfsetstrokecolor{currentstroke}%
\pgfsetstrokeopacity{0.000000}%
\pgfsetdash{}{0pt}%
\pgfpathmoveto{\pgfqpoint{5.920043in}{1.806587in}}%
\pgfpathlineto{\pgfqpoint{5.928980in}{1.806587in}}%
\pgfpathlineto{\pgfqpoint{5.928980in}{1.643958in}}%
\pgfpathlineto{\pgfqpoint{5.920043in}{1.643958in}}%
\pgfpathlineto{\pgfqpoint{5.920043in}{1.806587in}}%
\pgfpathclose%
\pgfusepath{fill}%
\end{pgfscope}%
\begin{pgfscope}%
\pgfpathrectangle{\pgfqpoint{3.722897in}{0.857143in}}{\pgfqpoint{2.627103in}{1.813434in}}%
\pgfusepath{clip}%
\pgfsetbuttcap%
\pgfsetmiterjoin%
\definecolor{currentfill}{rgb}{0.302379,0.450282,0.300122}%
\pgfsetfillcolor{currentfill}%
\pgfsetlinewidth{0.000000pt}%
\definecolor{currentstroke}{rgb}{0.000000,0.000000,0.000000}%
\pgfsetstrokecolor{currentstroke}%
\pgfsetstrokeopacity{0.000000}%
\pgfsetdash{}{0pt}%
\pgfpathmoveto{\pgfqpoint{5.931214in}{1.813884in}}%
\pgfpathlineto{\pgfqpoint{5.940150in}{1.813884in}}%
\pgfpathlineto{\pgfqpoint{5.940150in}{1.634258in}}%
\pgfpathlineto{\pgfqpoint{5.931214in}{1.634258in}}%
\pgfpathlineto{\pgfqpoint{5.931214in}{1.813884in}}%
\pgfpathclose%
\pgfusepath{fill}%
\end{pgfscope}%
\begin{pgfscope}%
\pgfpathrectangle{\pgfqpoint{3.722897in}{0.857143in}}{\pgfqpoint{2.627103in}{1.813434in}}%
\pgfusepath{clip}%
\pgfsetbuttcap%
\pgfsetmiterjoin%
\definecolor{currentfill}{rgb}{0.302379,0.450282,0.300122}%
\pgfsetfillcolor{currentfill}%
\pgfsetlinewidth{0.000000pt}%
\definecolor{currentstroke}{rgb}{0.000000,0.000000,0.000000}%
\pgfsetstrokecolor{currentstroke}%
\pgfsetstrokeopacity{0.000000}%
\pgfsetdash{}{0pt}%
\pgfpathmoveto{\pgfqpoint{5.942385in}{1.801970in}}%
\pgfpathlineto{\pgfqpoint{5.951321in}{1.801970in}}%
\pgfpathlineto{\pgfqpoint{5.951321in}{1.606285in}}%
\pgfpathlineto{\pgfqpoint{5.942385in}{1.606285in}}%
\pgfpathlineto{\pgfqpoint{5.942385in}{1.801970in}}%
\pgfpathclose%
\pgfusepath{fill}%
\end{pgfscope}%
\begin{pgfscope}%
\pgfpathrectangle{\pgfqpoint{3.722897in}{0.857143in}}{\pgfqpoint{2.627103in}{1.813434in}}%
\pgfusepath{clip}%
\pgfsetbuttcap%
\pgfsetmiterjoin%
\definecolor{currentfill}{rgb}{0.302379,0.450282,0.300122}%
\pgfsetfillcolor{currentfill}%
\pgfsetlinewidth{0.000000pt}%
\definecolor{currentstroke}{rgb}{0.000000,0.000000,0.000000}%
\pgfsetstrokecolor{currentstroke}%
\pgfsetstrokeopacity{0.000000}%
\pgfsetdash{}{0pt}%
\pgfpathmoveto{\pgfqpoint{5.953555in}{1.796585in}}%
\pgfpathlineto{\pgfqpoint{5.962492in}{1.796585in}}%
\pgfpathlineto{\pgfqpoint{5.962492in}{1.585553in}}%
\pgfpathlineto{\pgfqpoint{5.953555in}{1.585553in}}%
\pgfpathlineto{\pgfqpoint{5.953555in}{1.796585in}}%
\pgfpathclose%
\pgfusepath{fill}%
\end{pgfscope}%
\begin{pgfscope}%
\pgfpathrectangle{\pgfqpoint{3.722897in}{0.857143in}}{\pgfqpoint{2.627103in}{1.813434in}}%
\pgfusepath{clip}%
\pgfsetbuttcap%
\pgfsetmiterjoin%
\definecolor{currentfill}{rgb}{0.302379,0.450282,0.300122}%
\pgfsetfillcolor{currentfill}%
\pgfsetlinewidth{0.000000pt}%
\definecolor{currentstroke}{rgb}{0.000000,0.000000,0.000000}%
\pgfsetstrokecolor{currentstroke}%
\pgfsetstrokeopacity{0.000000}%
\pgfsetdash{}{0pt}%
\pgfpathmoveto{\pgfqpoint{5.964726in}{1.807739in}}%
\pgfpathlineto{\pgfqpoint{5.973662in}{1.807739in}}%
\pgfpathlineto{\pgfqpoint{5.973662in}{1.580469in}}%
\pgfpathlineto{\pgfqpoint{5.964726in}{1.580469in}}%
\pgfpathlineto{\pgfqpoint{5.964726in}{1.807739in}}%
\pgfpathclose%
\pgfusepath{fill}%
\end{pgfscope}%
\begin{pgfscope}%
\pgfpathrectangle{\pgfqpoint{3.722897in}{0.857143in}}{\pgfqpoint{2.627103in}{1.813434in}}%
\pgfusepath{clip}%
\pgfsetbuttcap%
\pgfsetmiterjoin%
\definecolor{currentfill}{rgb}{0.302379,0.450282,0.300122}%
\pgfsetfillcolor{currentfill}%
\pgfsetlinewidth{0.000000pt}%
\definecolor{currentstroke}{rgb}{0.000000,0.000000,0.000000}%
\pgfsetstrokecolor{currentstroke}%
\pgfsetstrokeopacity{0.000000}%
\pgfsetdash{}{0pt}%
\pgfpathmoveto{\pgfqpoint{5.975896in}{1.801832in}}%
\pgfpathlineto{\pgfqpoint{5.984833in}{1.801832in}}%
\pgfpathlineto{\pgfqpoint{5.984833in}{1.560025in}}%
\pgfpathlineto{\pgfqpoint{5.975896in}{1.560025in}}%
\pgfpathlineto{\pgfqpoint{5.975896in}{1.801832in}}%
\pgfpathclose%
\pgfusepath{fill}%
\end{pgfscope}%
\begin{pgfscope}%
\pgfpathrectangle{\pgfqpoint{3.722897in}{0.857143in}}{\pgfqpoint{2.627103in}{1.813434in}}%
\pgfusepath{clip}%
\pgfsetbuttcap%
\pgfsetmiterjoin%
\definecolor{currentfill}{rgb}{0.302379,0.450282,0.300122}%
\pgfsetfillcolor{currentfill}%
\pgfsetlinewidth{0.000000pt}%
\definecolor{currentstroke}{rgb}{0.000000,0.000000,0.000000}%
\pgfsetstrokecolor{currentstroke}%
\pgfsetstrokeopacity{0.000000}%
\pgfsetdash{}{0pt}%
\pgfpathmoveto{\pgfqpoint{5.987067in}{1.804150in}}%
\pgfpathlineto{\pgfqpoint{5.996004in}{1.804150in}}%
\pgfpathlineto{\pgfqpoint{5.996004in}{1.549893in}}%
\pgfpathlineto{\pgfqpoint{5.987067in}{1.549893in}}%
\pgfpathlineto{\pgfqpoint{5.987067in}{1.804150in}}%
\pgfpathclose%
\pgfusepath{fill}%
\end{pgfscope}%
\begin{pgfscope}%
\pgfpathrectangle{\pgfqpoint{3.722897in}{0.857143in}}{\pgfqpoint{2.627103in}{1.813434in}}%
\pgfusepath{clip}%
\pgfsetbuttcap%
\pgfsetmiterjoin%
\definecolor{currentfill}{rgb}{0.302379,0.450282,0.300122}%
\pgfsetfillcolor{currentfill}%
\pgfsetlinewidth{0.000000pt}%
\definecolor{currentstroke}{rgb}{0.000000,0.000000,0.000000}%
\pgfsetstrokecolor{currentstroke}%
\pgfsetstrokeopacity{0.000000}%
\pgfsetdash{}{0pt}%
\pgfpathmoveto{\pgfqpoint{5.998238in}{1.813884in}}%
\pgfpathlineto{\pgfqpoint{6.007174in}{1.813884in}}%
\pgfpathlineto{\pgfqpoint{6.007174in}{1.547717in}}%
\pgfpathlineto{\pgfqpoint{5.998238in}{1.547717in}}%
\pgfpathlineto{\pgfqpoint{5.998238in}{1.813884in}}%
\pgfpathclose%
\pgfusepath{fill}%
\end{pgfscope}%
\begin{pgfscope}%
\pgfpathrectangle{\pgfqpoint{3.722897in}{0.857143in}}{\pgfqpoint{2.627103in}{1.813434in}}%
\pgfusepath{clip}%
\pgfsetbuttcap%
\pgfsetmiterjoin%
\definecolor{currentfill}{rgb}{0.302379,0.450282,0.300122}%
\pgfsetfillcolor{currentfill}%
\pgfsetlinewidth{0.000000pt}%
\definecolor{currentstroke}{rgb}{0.000000,0.000000,0.000000}%
\pgfsetstrokecolor{currentstroke}%
\pgfsetstrokeopacity{0.000000}%
\pgfsetdash{}{0pt}%
\pgfpathmoveto{\pgfqpoint{6.009408in}{1.813884in}}%
\pgfpathlineto{\pgfqpoint{6.018345in}{1.813884in}}%
\pgfpathlineto{\pgfqpoint{6.018345in}{1.534771in}}%
\pgfpathlineto{\pgfqpoint{6.009408in}{1.534771in}}%
\pgfpathlineto{\pgfqpoint{6.009408in}{1.813884in}}%
\pgfpathclose%
\pgfusepath{fill}%
\end{pgfscope}%
\begin{pgfscope}%
\pgfpathrectangle{\pgfqpoint{3.722897in}{0.857143in}}{\pgfqpoint{2.627103in}{1.813434in}}%
\pgfusepath{clip}%
\pgfsetbuttcap%
\pgfsetmiterjoin%
\definecolor{currentfill}{rgb}{0.302379,0.450282,0.300122}%
\pgfsetfillcolor{currentfill}%
\pgfsetlinewidth{0.000000pt}%
\definecolor{currentstroke}{rgb}{0.000000,0.000000,0.000000}%
\pgfsetstrokecolor{currentstroke}%
\pgfsetstrokeopacity{0.000000}%
\pgfsetdash{}{0pt}%
\pgfpathmoveto{\pgfqpoint{6.020579in}{1.801223in}}%
\pgfpathlineto{\pgfqpoint{6.029515in}{1.801223in}}%
\pgfpathlineto{\pgfqpoint{6.029515in}{1.510179in}}%
\pgfpathlineto{\pgfqpoint{6.020579in}{1.510179in}}%
\pgfpathlineto{\pgfqpoint{6.020579in}{1.801223in}}%
\pgfpathclose%
\pgfusepath{fill}%
\end{pgfscope}%
\begin{pgfscope}%
\pgfpathrectangle{\pgfqpoint{3.722897in}{0.857143in}}{\pgfqpoint{2.627103in}{1.813434in}}%
\pgfusepath{clip}%
\pgfsetbuttcap%
\pgfsetmiterjoin%
\definecolor{currentfill}{rgb}{0.302379,0.450282,0.300122}%
\pgfsetfillcolor{currentfill}%
\pgfsetlinewidth{0.000000pt}%
\definecolor{currentstroke}{rgb}{0.000000,0.000000,0.000000}%
\pgfsetstrokecolor{currentstroke}%
\pgfsetstrokeopacity{0.000000}%
\pgfsetdash{}{0pt}%
\pgfpathmoveto{\pgfqpoint{6.031749in}{1.811915in}}%
\pgfpathlineto{\pgfqpoint{6.040686in}{1.811915in}}%
\pgfpathlineto{\pgfqpoint{6.040686in}{1.510090in}}%
\pgfpathlineto{\pgfqpoint{6.031749in}{1.510090in}}%
\pgfpathlineto{\pgfqpoint{6.031749in}{1.811915in}}%
\pgfpathclose%
\pgfusepath{fill}%
\end{pgfscope}%
\begin{pgfscope}%
\pgfpathrectangle{\pgfqpoint{3.722897in}{0.857143in}}{\pgfqpoint{2.627103in}{1.813434in}}%
\pgfusepath{clip}%
\pgfsetbuttcap%
\pgfsetmiterjoin%
\definecolor{currentfill}{rgb}{0.302379,0.450282,0.300122}%
\pgfsetfillcolor{currentfill}%
\pgfsetlinewidth{0.000000pt}%
\definecolor{currentstroke}{rgb}{0.000000,0.000000,0.000000}%
\pgfsetstrokecolor{currentstroke}%
\pgfsetstrokeopacity{0.000000}%
\pgfsetdash{}{0pt}%
\pgfpathmoveto{\pgfqpoint{6.042920in}{1.813884in}}%
\pgfpathlineto{\pgfqpoint{6.051857in}{1.813884in}}%
\pgfpathlineto{\pgfqpoint{6.051857in}{1.498309in}}%
\pgfpathlineto{\pgfqpoint{6.042920in}{1.498309in}}%
\pgfpathlineto{\pgfqpoint{6.042920in}{1.813884in}}%
\pgfpathclose%
\pgfusepath{fill}%
\end{pgfscope}%
\begin{pgfscope}%
\pgfpathrectangle{\pgfqpoint{3.722897in}{0.857143in}}{\pgfqpoint{2.627103in}{1.813434in}}%
\pgfusepath{clip}%
\pgfsetbuttcap%
\pgfsetmiterjoin%
\definecolor{currentfill}{rgb}{0.302379,0.450282,0.300122}%
\pgfsetfillcolor{currentfill}%
\pgfsetlinewidth{0.000000pt}%
\definecolor{currentstroke}{rgb}{0.000000,0.000000,0.000000}%
\pgfsetstrokecolor{currentstroke}%
\pgfsetstrokeopacity{0.000000}%
\pgfsetdash{}{0pt}%
\pgfpathmoveto{\pgfqpoint{6.054091in}{1.813884in}}%
\pgfpathlineto{\pgfqpoint{6.063027in}{1.813884in}}%
\pgfpathlineto{\pgfqpoint{6.063027in}{1.482896in}}%
\pgfpathlineto{\pgfqpoint{6.054091in}{1.482896in}}%
\pgfpathlineto{\pgfqpoint{6.054091in}{1.813884in}}%
\pgfpathclose%
\pgfusepath{fill}%
\end{pgfscope}%
\begin{pgfscope}%
\pgfpathrectangle{\pgfqpoint{3.722897in}{0.857143in}}{\pgfqpoint{2.627103in}{1.813434in}}%
\pgfusepath{clip}%
\pgfsetbuttcap%
\pgfsetmiterjoin%
\definecolor{currentfill}{rgb}{0.302379,0.450282,0.300122}%
\pgfsetfillcolor{currentfill}%
\pgfsetlinewidth{0.000000pt}%
\definecolor{currentstroke}{rgb}{0.000000,0.000000,0.000000}%
\pgfsetstrokecolor{currentstroke}%
\pgfsetstrokeopacity{0.000000}%
\pgfsetdash{}{0pt}%
\pgfpathmoveto{\pgfqpoint{6.065261in}{1.802914in}}%
\pgfpathlineto{\pgfqpoint{6.074198in}{1.802914in}}%
\pgfpathlineto{\pgfqpoint{6.074198in}{1.458590in}}%
\pgfpathlineto{\pgfqpoint{6.065261in}{1.458590in}}%
\pgfpathlineto{\pgfqpoint{6.065261in}{1.802914in}}%
\pgfpathclose%
\pgfusepath{fill}%
\end{pgfscope}%
\begin{pgfscope}%
\pgfpathrectangle{\pgfqpoint{3.722897in}{0.857143in}}{\pgfqpoint{2.627103in}{1.813434in}}%
\pgfusepath{clip}%
\pgfsetbuttcap%
\pgfsetmiterjoin%
\definecolor{currentfill}{rgb}{0.302379,0.450282,0.300122}%
\pgfsetfillcolor{currentfill}%
\pgfsetlinewidth{0.000000pt}%
\definecolor{currentstroke}{rgb}{0.000000,0.000000,0.000000}%
\pgfsetstrokecolor{currentstroke}%
\pgfsetstrokeopacity{0.000000}%
\pgfsetdash{}{0pt}%
\pgfpathmoveto{\pgfqpoint{6.076432in}{1.813884in}}%
\pgfpathlineto{\pgfqpoint{6.085368in}{1.813884in}}%
\pgfpathlineto{\pgfqpoint{6.085368in}{1.457213in}}%
\pgfpathlineto{\pgfqpoint{6.076432in}{1.457213in}}%
\pgfpathlineto{\pgfqpoint{6.076432in}{1.813884in}}%
\pgfpathclose%
\pgfusepath{fill}%
\end{pgfscope}%
\begin{pgfscope}%
\pgfpathrectangle{\pgfqpoint{3.722897in}{0.857143in}}{\pgfqpoint{2.627103in}{1.813434in}}%
\pgfusepath{clip}%
\pgfsetbuttcap%
\pgfsetmiterjoin%
\definecolor{currentfill}{rgb}{0.302379,0.450282,0.300122}%
\pgfsetfillcolor{currentfill}%
\pgfsetlinewidth{0.000000pt}%
\definecolor{currentstroke}{rgb}{0.000000,0.000000,0.000000}%
\pgfsetstrokecolor{currentstroke}%
\pgfsetstrokeopacity{0.000000}%
\pgfsetdash{}{0pt}%
\pgfpathmoveto{\pgfqpoint{6.087602in}{1.803025in}}%
\pgfpathlineto{\pgfqpoint{6.096539in}{1.803025in}}%
\pgfpathlineto{\pgfqpoint{6.096539in}{1.434004in}}%
\pgfpathlineto{\pgfqpoint{6.087602in}{1.434004in}}%
\pgfpathlineto{\pgfqpoint{6.087602in}{1.803025in}}%
\pgfpathclose%
\pgfusepath{fill}%
\end{pgfscope}%
\begin{pgfscope}%
\pgfpathrectangle{\pgfqpoint{3.722897in}{0.857143in}}{\pgfqpoint{2.627103in}{1.813434in}}%
\pgfusepath{clip}%
\pgfsetbuttcap%
\pgfsetmiterjoin%
\definecolor{currentfill}{rgb}{0.302379,0.450282,0.300122}%
\pgfsetfillcolor{currentfill}%
\pgfsetlinewidth{0.000000pt}%
\definecolor{currentstroke}{rgb}{0.000000,0.000000,0.000000}%
\pgfsetstrokecolor{currentstroke}%
\pgfsetstrokeopacity{0.000000}%
\pgfsetdash{}{0pt}%
\pgfpathmoveto{\pgfqpoint{6.098773in}{1.800455in}}%
\pgfpathlineto{\pgfqpoint{6.107710in}{1.800455in}}%
\pgfpathlineto{\pgfqpoint{6.107710in}{1.421190in}}%
\pgfpathlineto{\pgfqpoint{6.098773in}{1.421190in}}%
\pgfpathlineto{\pgfqpoint{6.098773in}{1.800455in}}%
\pgfpathclose%
\pgfusepath{fill}%
\end{pgfscope}%
\begin{pgfscope}%
\pgfpathrectangle{\pgfqpoint{3.722897in}{0.857143in}}{\pgfqpoint{2.627103in}{1.813434in}}%
\pgfusepath{clip}%
\pgfsetbuttcap%
\pgfsetmiterjoin%
\definecolor{currentfill}{rgb}{0.302379,0.450282,0.300122}%
\pgfsetfillcolor{currentfill}%
\pgfsetlinewidth{0.000000pt}%
\definecolor{currentstroke}{rgb}{0.000000,0.000000,0.000000}%
\pgfsetstrokecolor{currentstroke}%
\pgfsetstrokeopacity{0.000000}%
\pgfsetdash{}{0pt}%
\pgfpathmoveto{\pgfqpoint{6.109944in}{1.810216in}}%
\pgfpathlineto{\pgfqpoint{6.118880in}{1.810216in}}%
\pgfpathlineto{\pgfqpoint{6.118880in}{1.421212in}}%
\pgfpathlineto{\pgfqpoint{6.109944in}{1.421212in}}%
\pgfpathlineto{\pgfqpoint{6.109944in}{1.810216in}}%
\pgfpathclose%
\pgfusepath{fill}%
\end{pgfscope}%
\begin{pgfscope}%
\pgfpathrectangle{\pgfqpoint{3.722897in}{0.857143in}}{\pgfqpoint{2.627103in}{1.813434in}}%
\pgfusepath{clip}%
\pgfsetbuttcap%
\pgfsetmiterjoin%
\definecolor{currentfill}{rgb}{0.302379,0.450282,0.300122}%
\pgfsetfillcolor{currentfill}%
\pgfsetlinewidth{0.000000pt}%
\definecolor{currentstroke}{rgb}{0.000000,0.000000,0.000000}%
\pgfsetstrokecolor{currentstroke}%
\pgfsetstrokeopacity{0.000000}%
\pgfsetdash{}{0pt}%
\pgfpathmoveto{\pgfqpoint{6.121114in}{1.796321in}}%
\pgfpathlineto{\pgfqpoint{6.130051in}{1.796321in}}%
\pgfpathlineto{\pgfqpoint{6.130051in}{1.397038in}}%
\pgfpathlineto{\pgfqpoint{6.121114in}{1.397038in}}%
\pgfpathlineto{\pgfqpoint{6.121114in}{1.796321in}}%
\pgfpathclose%
\pgfusepath{fill}%
\end{pgfscope}%
\begin{pgfscope}%
\pgfpathrectangle{\pgfqpoint{3.722897in}{0.857143in}}{\pgfqpoint{2.627103in}{1.813434in}}%
\pgfusepath{clip}%
\pgfsetbuttcap%
\pgfsetmiterjoin%
\definecolor{currentfill}{rgb}{0.302379,0.450282,0.300122}%
\pgfsetfillcolor{currentfill}%
\pgfsetlinewidth{0.000000pt}%
\definecolor{currentstroke}{rgb}{0.000000,0.000000,0.000000}%
\pgfsetstrokecolor{currentstroke}%
\pgfsetstrokeopacity{0.000000}%
\pgfsetdash{}{0pt}%
\pgfpathmoveto{\pgfqpoint{6.132285in}{1.786955in}}%
\pgfpathlineto{\pgfqpoint{6.141221in}{1.786955in}}%
\pgfpathlineto{\pgfqpoint{6.141221in}{1.379141in}}%
\pgfpathlineto{\pgfqpoint{6.132285in}{1.379141in}}%
\pgfpathlineto{\pgfqpoint{6.132285in}{1.786955in}}%
\pgfpathclose%
\pgfusepath{fill}%
\end{pgfscope}%
\begin{pgfscope}%
\pgfpathrectangle{\pgfqpoint{3.722897in}{0.857143in}}{\pgfqpoint{2.627103in}{1.813434in}}%
\pgfusepath{clip}%
\pgfsetbuttcap%
\pgfsetmiterjoin%
\definecolor{currentfill}{rgb}{0.302379,0.450282,0.300122}%
\pgfsetfillcolor{currentfill}%
\pgfsetlinewidth{0.000000pt}%
\definecolor{currentstroke}{rgb}{0.000000,0.000000,0.000000}%
\pgfsetstrokecolor{currentstroke}%
\pgfsetstrokeopacity{0.000000}%
\pgfsetdash{}{0pt}%
\pgfpathmoveto{\pgfqpoint{6.143456in}{1.789051in}}%
\pgfpathlineto{\pgfqpoint{6.152392in}{1.789051in}}%
\pgfpathlineto{\pgfqpoint{6.152392in}{1.373670in}}%
\pgfpathlineto{\pgfqpoint{6.143456in}{1.373670in}}%
\pgfpathlineto{\pgfqpoint{6.143456in}{1.789051in}}%
\pgfpathclose%
\pgfusepath{fill}%
\end{pgfscope}%
\begin{pgfscope}%
\pgfpathrectangle{\pgfqpoint{3.722897in}{0.857143in}}{\pgfqpoint{2.627103in}{1.813434in}}%
\pgfusepath{clip}%
\pgfsetbuttcap%
\pgfsetmiterjoin%
\definecolor{currentfill}{rgb}{0.302379,0.450282,0.300122}%
\pgfsetfillcolor{currentfill}%
\pgfsetlinewidth{0.000000pt}%
\definecolor{currentstroke}{rgb}{0.000000,0.000000,0.000000}%
\pgfsetstrokecolor{currentstroke}%
\pgfsetstrokeopacity{0.000000}%
\pgfsetdash{}{0pt}%
\pgfpathmoveto{\pgfqpoint{6.154626in}{1.786190in}}%
\pgfpathlineto{\pgfqpoint{6.163563in}{1.786190in}}%
\pgfpathlineto{\pgfqpoint{6.163563in}{1.364190in}}%
\pgfpathlineto{\pgfqpoint{6.154626in}{1.364190in}}%
\pgfpathlineto{\pgfqpoint{6.154626in}{1.786190in}}%
\pgfpathclose%
\pgfusepath{fill}%
\end{pgfscope}%
\begin{pgfscope}%
\pgfpathrectangle{\pgfqpoint{3.722897in}{0.857143in}}{\pgfqpoint{2.627103in}{1.813434in}}%
\pgfusepath{clip}%
\pgfsetbuttcap%
\pgfsetmiterjoin%
\definecolor{currentfill}{rgb}{0.302379,0.450282,0.300122}%
\pgfsetfillcolor{currentfill}%
\pgfsetlinewidth{0.000000pt}%
\definecolor{currentstroke}{rgb}{0.000000,0.000000,0.000000}%
\pgfsetstrokecolor{currentstroke}%
\pgfsetstrokeopacity{0.000000}%
\pgfsetdash{}{0pt}%
\pgfpathmoveto{\pgfqpoint{6.165797in}{1.794605in}}%
\pgfpathlineto{\pgfqpoint{6.174733in}{1.794605in}}%
\pgfpathlineto{\pgfqpoint{6.174733in}{1.365961in}}%
\pgfpathlineto{\pgfqpoint{6.165797in}{1.365961in}}%
\pgfpathlineto{\pgfqpoint{6.165797in}{1.794605in}}%
\pgfpathclose%
\pgfusepath{fill}%
\end{pgfscope}%
\begin{pgfscope}%
\pgfpathrectangle{\pgfqpoint{3.722897in}{0.857143in}}{\pgfqpoint{2.627103in}{1.813434in}}%
\pgfusepath{clip}%
\pgfsetbuttcap%
\pgfsetmiterjoin%
\definecolor{currentfill}{rgb}{0.302379,0.450282,0.300122}%
\pgfsetfillcolor{currentfill}%
\pgfsetlinewidth{0.000000pt}%
\definecolor{currentstroke}{rgb}{0.000000,0.000000,0.000000}%
\pgfsetstrokecolor{currentstroke}%
\pgfsetstrokeopacity{0.000000}%
\pgfsetdash{}{0pt}%
\pgfpathmoveto{\pgfqpoint{6.176967in}{1.796179in}}%
\pgfpathlineto{\pgfqpoint{6.185904in}{1.796179in}}%
\pgfpathlineto{\pgfqpoint{6.185904in}{1.359361in}}%
\pgfpathlineto{\pgfqpoint{6.176967in}{1.359361in}}%
\pgfpathlineto{\pgfqpoint{6.176967in}{1.796179in}}%
\pgfpathclose%
\pgfusepath{fill}%
\end{pgfscope}%
\begin{pgfscope}%
\pgfpathrectangle{\pgfqpoint{3.722897in}{0.857143in}}{\pgfqpoint{2.627103in}{1.813434in}}%
\pgfusepath{clip}%
\pgfsetbuttcap%
\pgfsetmiterjoin%
\definecolor{currentfill}{rgb}{0.302379,0.450282,0.300122}%
\pgfsetfillcolor{currentfill}%
\pgfsetlinewidth{0.000000pt}%
\definecolor{currentstroke}{rgb}{0.000000,0.000000,0.000000}%
\pgfsetstrokecolor{currentstroke}%
\pgfsetstrokeopacity{0.000000}%
\pgfsetdash{}{0pt}%
\pgfpathmoveto{\pgfqpoint{6.188138in}{1.804271in}}%
\pgfpathlineto{\pgfqpoint{6.197074in}{1.804271in}}%
\pgfpathlineto{\pgfqpoint{6.197074in}{1.356444in}}%
\pgfpathlineto{\pgfqpoint{6.188138in}{1.356444in}}%
\pgfpathlineto{\pgfqpoint{6.188138in}{1.804271in}}%
\pgfpathclose%
\pgfusepath{fill}%
\end{pgfscope}%
\begin{pgfscope}%
\pgfpathrectangle{\pgfqpoint{3.722897in}{0.857143in}}{\pgfqpoint{2.627103in}{1.813434in}}%
\pgfusepath{clip}%
\pgfsetbuttcap%
\pgfsetmiterjoin%
\definecolor{currentfill}{rgb}{0.302379,0.450282,0.300122}%
\pgfsetfillcolor{currentfill}%
\pgfsetlinewidth{0.000000pt}%
\definecolor{currentstroke}{rgb}{0.000000,0.000000,0.000000}%
\pgfsetstrokecolor{currentstroke}%
\pgfsetstrokeopacity{0.000000}%
\pgfsetdash{}{0pt}%
\pgfpathmoveto{\pgfqpoint{6.199309in}{1.790744in}}%
\pgfpathlineto{\pgfqpoint{6.208245in}{1.790744in}}%
\pgfpathlineto{\pgfqpoint{6.208245in}{1.332093in}}%
\pgfpathlineto{\pgfqpoint{6.199309in}{1.332093in}}%
\pgfpathlineto{\pgfqpoint{6.199309in}{1.790744in}}%
\pgfpathclose%
\pgfusepath{fill}%
\end{pgfscope}%
\begin{pgfscope}%
\pgfpathrectangle{\pgfqpoint{3.722897in}{0.857143in}}{\pgfqpoint{2.627103in}{1.813434in}}%
\pgfusepath{clip}%
\pgfsetbuttcap%
\pgfsetmiterjoin%
\definecolor{currentfill}{rgb}{0.302379,0.450282,0.300122}%
\pgfsetfillcolor{currentfill}%
\pgfsetlinewidth{0.000000pt}%
\definecolor{currentstroke}{rgb}{0.000000,0.000000,0.000000}%
\pgfsetstrokecolor{currentstroke}%
\pgfsetstrokeopacity{0.000000}%
\pgfsetdash{}{0pt}%
\pgfpathmoveto{\pgfqpoint{6.210479in}{1.794068in}}%
\pgfpathlineto{\pgfqpoint{6.219416in}{1.794068in}}%
\pgfpathlineto{\pgfqpoint{6.219416in}{1.324640in}}%
\pgfpathlineto{\pgfqpoint{6.210479in}{1.324640in}}%
\pgfpathlineto{\pgfqpoint{6.210479in}{1.794068in}}%
\pgfpathclose%
\pgfusepath{fill}%
\end{pgfscope}%
\begin{pgfscope}%
\pgfpathrectangle{\pgfqpoint{3.722897in}{0.857143in}}{\pgfqpoint{2.627103in}{1.813434in}}%
\pgfusepath{clip}%
\pgfsetbuttcap%
\pgfsetmiterjoin%
\definecolor{currentfill}{rgb}{0.302379,0.450282,0.300122}%
\pgfsetfillcolor{currentfill}%
\pgfsetlinewidth{0.000000pt}%
\definecolor{currentstroke}{rgb}{0.000000,0.000000,0.000000}%
\pgfsetstrokecolor{currentstroke}%
\pgfsetstrokeopacity{0.000000}%
\pgfsetdash{}{0pt}%
\pgfpathmoveto{\pgfqpoint{6.221650in}{1.790803in}}%
\pgfpathlineto{\pgfqpoint{6.230586in}{1.790803in}}%
\pgfpathlineto{\pgfqpoint{6.230586in}{1.309376in}}%
\pgfpathlineto{\pgfqpoint{6.221650in}{1.309376in}}%
\pgfpathlineto{\pgfqpoint{6.221650in}{1.790803in}}%
\pgfpathclose%
\pgfusepath{fill}%
\end{pgfscope}%
\begin{pgfscope}%
\pgfpathrectangle{\pgfqpoint{3.722897in}{0.857143in}}{\pgfqpoint{2.627103in}{1.813434in}}%
\pgfusepath{clip}%
\pgfsetbuttcap%
\pgfsetmiterjoin%
\definecolor{currentfill}{rgb}{0.511253,0.510898,0.193296}%
\pgfsetfillcolor{currentfill}%
\pgfsetlinewidth{0.000000pt}%
\definecolor{currentstroke}{rgb}{0.000000,0.000000,0.000000}%
\pgfsetstrokecolor{currentstroke}%
\pgfsetstrokeopacity{0.000000}%
\pgfsetdash{}{0pt}%
\pgfpathmoveto{\pgfqpoint{3.842311in}{1.776781in}}%
\pgfpathlineto{\pgfqpoint{3.851247in}{1.776781in}}%
\pgfpathlineto{\pgfqpoint{3.851247in}{1.763200in}}%
\pgfpathlineto{\pgfqpoint{3.842311in}{1.763200in}}%
\pgfpathlineto{\pgfqpoint{3.842311in}{1.776781in}}%
\pgfpathclose%
\pgfusepath{fill}%
\end{pgfscope}%
\begin{pgfscope}%
\pgfpathrectangle{\pgfqpoint{3.722897in}{0.857143in}}{\pgfqpoint{2.627103in}{1.813434in}}%
\pgfusepath{clip}%
\pgfsetbuttcap%
\pgfsetmiterjoin%
\definecolor{currentfill}{rgb}{0.511253,0.510898,0.193296}%
\pgfsetfillcolor{currentfill}%
\pgfsetlinewidth{0.000000pt}%
\definecolor{currentstroke}{rgb}{0.000000,0.000000,0.000000}%
\pgfsetstrokecolor{currentstroke}%
\pgfsetstrokeopacity{0.000000}%
\pgfsetdash{}{0pt}%
\pgfpathmoveto{\pgfqpoint{3.853481in}{1.767639in}}%
\pgfpathlineto{\pgfqpoint{3.862418in}{1.767639in}}%
\pgfpathlineto{\pgfqpoint{3.862418in}{1.755179in}}%
\pgfpathlineto{\pgfqpoint{3.853481in}{1.755179in}}%
\pgfpathlineto{\pgfqpoint{3.853481in}{1.767639in}}%
\pgfpathclose%
\pgfusepath{fill}%
\end{pgfscope}%
\begin{pgfscope}%
\pgfpathrectangle{\pgfqpoint{3.722897in}{0.857143in}}{\pgfqpoint{2.627103in}{1.813434in}}%
\pgfusepath{clip}%
\pgfsetbuttcap%
\pgfsetmiterjoin%
\definecolor{currentfill}{rgb}{0.511253,0.510898,0.193296}%
\pgfsetfillcolor{currentfill}%
\pgfsetlinewidth{0.000000pt}%
\definecolor{currentstroke}{rgb}{0.000000,0.000000,0.000000}%
\pgfsetstrokecolor{currentstroke}%
\pgfsetstrokeopacity{0.000000}%
\pgfsetdash{}{0pt}%
\pgfpathmoveto{\pgfqpoint{3.864652in}{1.756131in}}%
\pgfpathlineto{\pgfqpoint{3.873588in}{1.756131in}}%
\pgfpathlineto{\pgfqpoint{3.873588in}{1.747016in}}%
\pgfpathlineto{\pgfqpoint{3.864652in}{1.747016in}}%
\pgfpathlineto{\pgfqpoint{3.864652in}{1.756131in}}%
\pgfpathclose%
\pgfusepath{fill}%
\end{pgfscope}%
\begin{pgfscope}%
\pgfpathrectangle{\pgfqpoint{3.722897in}{0.857143in}}{\pgfqpoint{2.627103in}{1.813434in}}%
\pgfusepath{clip}%
\pgfsetbuttcap%
\pgfsetmiterjoin%
\definecolor{currentfill}{rgb}{0.511253,0.510898,0.193296}%
\pgfsetfillcolor{currentfill}%
\pgfsetlinewidth{0.000000pt}%
\definecolor{currentstroke}{rgb}{0.000000,0.000000,0.000000}%
\pgfsetstrokecolor{currentstroke}%
\pgfsetstrokeopacity{0.000000}%
\pgfsetdash{}{0pt}%
\pgfpathmoveto{\pgfqpoint{3.875823in}{1.744969in}}%
\pgfpathlineto{\pgfqpoint{3.884759in}{1.744969in}}%
\pgfpathlineto{\pgfqpoint{3.884759in}{1.736268in}}%
\pgfpathlineto{\pgfqpoint{3.875823in}{1.736268in}}%
\pgfpathlineto{\pgfqpoint{3.875823in}{1.744969in}}%
\pgfpathclose%
\pgfusepath{fill}%
\end{pgfscope}%
\begin{pgfscope}%
\pgfpathrectangle{\pgfqpoint{3.722897in}{0.857143in}}{\pgfqpoint{2.627103in}{1.813434in}}%
\pgfusepath{clip}%
\pgfsetbuttcap%
\pgfsetmiterjoin%
\definecolor{currentfill}{rgb}{0.511253,0.510898,0.193296}%
\pgfsetfillcolor{currentfill}%
\pgfsetlinewidth{0.000000pt}%
\definecolor{currentstroke}{rgb}{0.000000,0.000000,0.000000}%
\pgfsetstrokecolor{currentstroke}%
\pgfsetstrokeopacity{0.000000}%
\pgfsetdash{}{0pt}%
\pgfpathmoveto{\pgfqpoint{3.886993in}{1.732088in}}%
\pgfpathlineto{\pgfqpoint{3.895930in}{1.732088in}}%
\pgfpathlineto{\pgfqpoint{3.895930in}{1.722278in}}%
\pgfpathlineto{\pgfqpoint{3.886993in}{1.722278in}}%
\pgfpathlineto{\pgfqpoint{3.886993in}{1.732088in}}%
\pgfpathclose%
\pgfusepath{fill}%
\end{pgfscope}%
\begin{pgfscope}%
\pgfpathrectangle{\pgfqpoint{3.722897in}{0.857143in}}{\pgfqpoint{2.627103in}{1.813434in}}%
\pgfusepath{clip}%
\pgfsetbuttcap%
\pgfsetmiterjoin%
\definecolor{currentfill}{rgb}{0.511253,0.510898,0.193296}%
\pgfsetfillcolor{currentfill}%
\pgfsetlinewidth{0.000000pt}%
\definecolor{currentstroke}{rgb}{0.000000,0.000000,0.000000}%
\pgfsetstrokecolor{currentstroke}%
\pgfsetstrokeopacity{0.000000}%
\pgfsetdash{}{0pt}%
\pgfpathmoveto{\pgfqpoint{3.898164in}{1.817902in}}%
\pgfpathlineto{\pgfqpoint{3.907100in}{1.817902in}}%
\pgfpathlineto{\pgfqpoint{3.907100in}{1.819190in}}%
\pgfpathlineto{\pgfqpoint{3.898164in}{1.819190in}}%
\pgfpathlineto{\pgfqpoint{3.898164in}{1.817902in}}%
\pgfpathclose%
\pgfusepath{fill}%
\end{pgfscope}%
\begin{pgfscope}%
\pgfpathrectangle{\pgfqpoint{3.722897in}{0.857143in}}{\pgfqpoint{2.627103in}{1.813434in}}%
\pgfusepath{clip}%
\pgfsetbuttcap%
\pgfsetmiterjoin%
\definecolor{currentfill}{rgb}{0.511253,0.510898,0.193296}%
\pgfsetfillcolor{currentfill}%
\pgfsetlinewidth{0.000000pt}%
\definecolor{currentstroke}{rgb}{0.000000,0.000000,0.000000}%
\pgfsetstrokecolor{currentstroke}%
\pgfsetstrokeopacity{0.000000}%
\pgfsetdash{}{0pt}%
\pgfpathmoveto{\pgfqpoint{3.909334in}{1.824681in}}%
\pgfpathlineto{\pgfqpoint{3.918271in}{1.824681in}}%
\pgfpathlineto{\pgfqpoint{3.918271in}{1.841680in}}%
\pgfpathlineto{\pgfqpoint{3.909334in}{1.841680in}}%
\pgfpathlineto{\pgfqpoint{3.909334in}{1.824681in}}%
\pgfpathclose%
\pgfusepath{fill}%
\end{pgfscope}%
\begin{pgfscope}%
\pgfpathrectangle{\pgfqpoint{3.722897in}{0.857143in}}{\pgfqpoint{2.627103in}{1.813434in}}%
\pgfusepath{clip}%
\pgfsetbuttcap%
\pgfsetmiterjoin%
\definecolor{currentfill}{rgb}{0.511253,0.510898,0.193296}%
\pgfsetfillcolor{currentfill}%
\pgfsetlinewidth{0.000000pt}%
\definecolor{currentstroke}{rgb}{0.000000,0.000000,0.000000}%
\pgfsetstrokecolor{currentstroke}%
\pgfsetstrokeopacity{0.000000}%
\pgfsetdash{}{0pt}%
\pgfpathmoveto{\pgfqpoint{3.920505in}{1.837517in}}%
\pgfpathlineto{\pgfqpoint{3.929442in}{1.837517in}}%
\pgfpathlineto{\pgfqpoint{3.929442in}{1.868332in}}%
\pgfpathlineto{\pgfqpoint{3.920505in}{1.868332in}}%
\pgfpathlineto{\pgfqpoint{3.920505in}{1.837517in}}%
\pgfpathclose%
\pgfusepath{fill}%
\end{pgfscope}%
\begin{pgfscope}%
\pgfpathrectangle{\pgfqpoint{3.722897in}{0.857143in}}{\pgfqpoint{2.627103in}{1.813434in}}%
\pgfusepath{clip}%
\pgfsetbuttcap%
\pgfsetmiterjoin%
\definecolor{currentfill}{rgb}{0.511253,0.510898,0.193296}%
\pgfsetfillcolor{currentfill}%
\pgfsetlinewidth{0.000000pt}%
\definecolor{currentstroke}{rgb}{0.000000,0.000000,0.000000}%
\pgfsetstrokecolor{currentstroke}%
\pgfsetstrokeopacity{0.000000}%
\pgfsetdash{}{0pt}%
\pgfpathmoveto{\pgfqpoint{3.931676in}{1.836801in}}%
\pgfpathlineto{\pgfqpoint{3.940612in}{1.836801in}}%
\pgfpathlineto{\pgfqpoint{3.940612in}{1.874611in}}%
\pgfpathlineto{\pgfqpoint{3.931676in}{1.874611in}}%
\pgfpathlineto{\pgfqpoint{3.931676in}{1.836801in}}%
\pgfpathclose%
\pgfusepath{fill}%
\end{pgfscope}%
\begin{pgfscope}%
\pgfpathrectangle{\pgfqpoint{3.722897in}{0.857143in}}{\pgfqpoint{2.627103in}{1.813434in}}%
\pgfusepath{clip}%
\pgfsetbuttcap%
\pgfsetmiterjoin%
\definecolor{currentfill}{rgb}{0.511253,0.510898,0.193296}%
\pgfsetfillcolor{currentfill}%
\pgfsetlinewidth{0.000000pt}%
\definecolor{currentstroke}{rgb}{0.000000,0.000000,0.000000}%
\pgfsetstrokecolor{currentstroke}%
\pgfsetstrokeopacity{0.000000}%
\pgfsetdash{}{0pt}%
\pgfpathmoveto{\pgfqpoint{3.942846in}{1.827785in}}%
\pgfpathlineto{\pgfqpoint{3.951783in}{1.827785in}}%
\pgfpathlineto{\pgfqpoint{3.951783in}{1.860678in}}%
\pgfpathlineto{\pgfqpoint{3.942846in}{1.860678in}}%
\pgfpathlineto{\pgfqpoint{3.942846in}{1.827785in}}%
\pgfpathclose%
\pgfusepath{fill}%
\end{pgfscope}%
\begin{pgfscope}%
\pgfpathrectangle{\pgfqpoint{3.722897in}{0.857143in}}{\pgfqpoint{2.627103in}{1.813434in}}%
\pgfusepath{clip}%
\pgfsetbuttcap%
\pgfsetmiterjoin%
\definecolor{currentfill}{rgb}{0.511253,0.510898,0.193296}%
\pgfsetfillcolor{currentfill}%
\pgfsetlinewidth{0.000000pt}%
\definecolor{currentstroke}{rgb}{0.000000,0.000000,0.000000}%
\pgfsetstrokecolor{currentstroke}%
\pgfsetstrokeopacity{0.000000}%
\pgfsetdash{}{0pt}%
\pgfpathmoveto{\pgfqpoint{3.954017in}{1.839047in}}%
\pgfpathlineto{\pgfqpoint{3.962953in}{1.839047in}}%
\pgfpathlineto{\pgfqpoint{3.962953in}{1.873535in}}%
\pgfpathlineto{\pgfqpoint{3.954017in}{1.873535in}}%
\pgfpathlineto{\pgfqpoint{3.954017in}{1.839047in}}%
\pgfpathclose%
\pgfusepath{fill}%
\end{pgfscope}%
\begin{pgfscope}%
\pgfpathrectangle{\pgfqpoint{3.722897in}{0.857143in}}{\pgfqpoint{2.627103in}{1.813434in}}%
\pgfusepath{clip}%
\pgfsetbuttcap%
\pgfsetmiterjoin%
\definecolor{currentfill}{rgb}{0.511253,0.510898,0.193296}%
\pgfsetfillcolor{currentfill}%
\pgfsetlinewidth{0.000000pt}%
\definecolor{currentstroke}{rgb}{0.000000,0.000000,0.000000}%
\pgfsetstrokecolor{currentstroke}%
\pgfsetstrokeopacity{0.000000}%
\pgfsetdash{}{0pt}%
\pgfpathmoveto{\pgfqpoint{3.965187in}{1.845072in}}%
\pgfpathlineto{\pgfqpoint{3.974124in}{1.845072in}}%
\pgfpathlineto{\pgfqpoint{3.974124in}{1.885379in}}%
\pgfpathlineto{\pgfqpoint{3.965187in}{1.885379in}}%
\pgfpathlineto{\pgfqpoint{3.965187in}{1.845072in}}%
\pgfpathclose%
\pgfusepath{fill}%
\end{pgfscope}%
\begin{pgfscope}%
\pgfpathrectangle{\pgfqpoint{3.722897in}{0.857143in}}{\pgfqpoint{2.627103in}{1.813434in}}%
\pgfusepath{clip}%
\pgfsetbuttcap%
\pgfsetmiterjoin%
\definecolor{currentfill}{rgb}{0.511253,0.510898,0.193296}%
\pgfsetfillcolor{currentfill}%
\pgfsetlinewidth{0.000000pt}%
\definecolor{currentstroke}{rgb}{0.000000,0.000000,0.000000}%
\pgfsetstrokecolor{currentstroke}%
\pgfsetstrokeopacity{0.000000}%
\pgfsetdash{}{0pt}%
\pgfpathmoveto{\pgfqpoint{3.976358in}{1.827787in}}%
\pgfpathlineto{\pgfqpoint{3.985295in}{1.827787in}}%
\pgfpathlineto{\pgfqpoint{3.985295in}{1.876340in}}%
\pgfpathlineto{\pgfqpoint{3.976358in}{1.876340in}}%
\pgfpathlineto{\pgfqpoint{3.976358in}{1.827787in}}%
\pgfpathclose%
\pgfusepath{fill}%
\end{pgfscope}%
\begin{pgfscope}%
\pgfpathrectangle{\pgfqpoint{3.722897in}{0.857143in}}{\pgfqpoint{2.627103in}{1.813434in}}%
\pgfusepath{clip}%
\pgfsetbuttcap%
\pgfsetmiterjoin%
\definecolor{currentfill}{rgb}{0.511253,0.510898,0.193296}%
\pgfsetfillcolor{currentfill}%
\pgfsetlinewidth{0.000000pt}%
\definecolor{currentstroke}{rgb}{0.000000,0.000000,0.000000}%
\pgfsetstrokecolor{currentstroke}%
\pgfsetstrokeopacity{0.000000}%
\pgfsetdash{}{0pt}%
\pgfpathmoveto{\pgfqpoint{3.987529in}{1.840728in}}%
\pgfpathlineto{\pgfqpoint{3.996465in}{1.840728in}}%
\pgfpathlineto{\pgfqpoint{3.996465in}{1.893204in}}%
\pgfpathlineto{\pgfqpoint{3.987529in}{1.893204in}}%
\pgfpathlineto{\pgfqpoint{3.987529in}{1.840728in}}%
\pgfpathclose%
\pgfusepath{fill}%
\end{pgfscope}%
\begin{pgfscope}%
\pgfpathrectangle{\pgfqpoint{3.722897in}{0.857143in}}{\pgfqpoint{2.627103in}{1.813434in}}%
\pgfusepath{clip}%
\pgfsetbuttcap%
\pgfsetmiterjoin%
\definecolor{currentfill}{rgb}{0.511253,0.510898,0.193296}%
\pgfsetfillcolor{currentfill}%
\pgfsetlinewidth{0.000000pt}%
\definecolor{currentstroke}{rgb}{0.000000,0.000000,0.000000}%
\pgfsetstrokecolor{currentstroke}%
\pgfsetstrokeopacity{0.000000}%
\pgfsetdash{}{0pt}%
\pgfpathmoveto{\pgfqpoint{3.998699in}{1.822803in}}%
\pgfpathlineto{\pgfqpoint{4.007636in}{1.822803in}}%
\pgfpathlineto{\pgfqpoint{4.007636in}{1.863031in}}%
\pgfpathlineto{\pgfqpoint{3.998699in}{1.863031in}}%
\pgfpathlineto{\pgfqpoint{3.998699in}{1.822803in}}%
\pgfpathclose%
\pgfusepath{fill}%
\end{pgfscope}%
\begin{pgfscope}%
\pgfpathrectangle{\pgfqpoint{3.722897in}{0.857143in}}{\pgfqpoint{2.627103in}{1.813434in}}%
\pgfusepath{clip}%
\pgfsetbuttcap%
\pgfsetmiterjoin%
\definecolor{currentfill}{rgb}{0.511253,0.510898,0.193296}%
\pgfsetfillcolor{currentfill}%
\pgfsetlinewidth{0.000000pt}%
\definecolor{currentstroke}{rgb}{0.000000,0.000000,0.000000}%
\pgfsetstrokecolor{currentstroke}%
\pgfsetstrokeopacity{0.000000}%
\pgfsetdash{}{0pt}%
\pgfpathmoveto{\pgfqpoint{4.009870in}{1.824890in}}%
\pgfpathlineto{\pgfqpoint{4.018806in}{1.824890in}}%
\pgfpathlineto{\pgfqpoint{4.018806in}{1.853088in}}%
\pgfpathlineto{\pgfqpoint{4.009870in}{1.853088in}}%
\pgfpathlineto{\pgfqpoint{4.009870in}{1.824890in}}%
\pgfpathclose%
\pgfusepath{fill}%
\end{pgfscope}%
\begin{pgfscope}%
\pgfpathrectangle{\pgfqpoint{3.722897in}{0.857143in}}{\pgfqpoint{2.627103in}{1.813434in}}%
\pgfusepath{clip}%
\pgfsetbuttcap%
\pgfsetmiterjoin%
\definecolor{currentfill}{rgb}{0.511253,0.510898,0.193296}%
\pgfsetfillcolor{currentfill}%
\pgfsetlinewidth{0.000000pt}%
\definecolor{currentstroke}{rgb}{0.000000,0.000000,0.000000}%
\pgfsetstrokecolor{currentstroke}%
\pgfsetstrokeopacity{0.000000}%
\pgfsetdash{}{0pt}%
\pgfpathmoveto{\pgfqpoint{4.021040in}{1.835410in}}%
\pgfpathlineto{\pgfqpoint{4.029977in}{1.835410in}}%
\pgfpathlineto{\pgfqpoint{4.029977in}{1.855059in}}%
\pgfpathlineto{\pgfqpoint{4.021040in}{1.855059in}}%
\pgfpathlineto{\pgfqpoint{4.021040in}{1.835410in}}%
\pgfpathclose%
\pgfusepath{fill}%
\end{pgfscope}%
\begin{pgfscope}%
\pgfpathrectangle{\pgfqpoint{3.722897in}{0.857143in}}{\pgfqpoint{2.627103in}{1.813434in}}%
\pgfusepath{clip}%
\pgfsetbuttcap%
\pgfsetmiterjoin%
\definecolor{currentfill}{rgb}{0.511253,0.510898,0.193296}%
\pgfsetfillcolor{currentfill}%
\pgfsetlinewidth{0.000000pt}%
\definecolor{currentstroke}{rgb}{0.000000,0.000000,0.000000}%
\pgfsetstrokecolor{currentstroke}%
\pgfsetstrokeopacity{0.000000}%
\pgfsetdash{}{0pt}%
\pgfpathmoveto{\pgfqpoint{4.032211in}{1.826169in}}%
\pgfpathlineto{\pgfqpoint{4.041148in}{1.826169in}}%
\pgfpathlineto{\pgfqpoint{4.041148in}{1.830072in}}%
\pgfpathlineto{\pgfqpoint{4.032211in}{1.830072in}}%
\pgfpathlineto{\pgfqpoint{4.032211in}{1.826169in}}%
\pgfpathclose%
\pgfusepath{fill}%
\end{pgfscope}%
\begin{pgfscope}%
\pgfpathrectangle{\pgfqpoint{3.722897in}{0.857143in}}{\pgfqpoint{2.627103in}{1.813434in}}%
\pgfusepath{clip}%
\pgfsetbuttcap%
\pgfsetmiterjoin%
\definecolor{currentfill}{rgb}{0.511253,0.510898,0.193296}%
\pgfsetfillcolor{currentfill}%
\pgfsetlinewidth{0.000000pt}%
\definecolor{currentstroke}{rgb}{0.000000,0.000000,0.000000}%
\pgfsetstrokecolor{currentstroke}%
\pgfsetstrokeopacity{0.000000}%
\pgfsetdash{}{0pt}%
\pgfpathmoveto{\pgfqpoint{4.043382in}{1.617256in}}%
\pgfpathlineto{\pgfqpoint{4.052318in}{1.617256in}}%
\pgfpathlineto{\pgfqpoint{4.052318in}{1.605033in}}%
\pgfpathlineto{\pgfqpoint{4.043382in}{1.605033in}}%
\pgfpathlineto{\pgfqpoint{4.043382in}{1.617256in}}%
\pgfpathclose%
\pgfusepath{fill}%
\end{pgfscope}%
\begin{pgfscope}%
\pgfpathrectangle{\pgfqpoint{3.722897in}{0.857143in}}{\pgfqpoint{2.627103in}{1.813434in}}%
\pgfusepath{clip}%
\pgfsetbuttcap%
\pgfsetmiterjoin%
\definecolor{currentfill}{rgb}{0.511253,0.510898,0.193296}%
\pgfsetfillcolor{currentfill}%
\pgfsetlinewidth{0.000000pt}%
\definecolor{currentstroke}{rgb}{0.000000,0.000000,0.000000}%
\pgfsetstrokecolor{currentstroke}%
\pgfsetstrokeopacity{0.000000}%
\pgfsetdash{}{0pt}%
\pgfpathmoveto{\pgfqpoint{4.054552in}{1.601657in}}%
\pgfpathlineto{\pgfqpoint{4.063489in}{1.601657in}}%
\pgfpathlineto{\pgfqpoint{4.063489in}{1.587619in}}%
\pgfpathlineto{\pgfqpoint{4.054552in}{1.587619in}}%
\pgfpathlineto{\pgfqpoint{4.054552in}{1.601657in}}%
\pgfpathclose%
\pgfusepath{fill}%
\end{pgfscope}%
\begin{pgfscope}%
\pgfpathrectangle{\pgfqpoint{3.722897in}{0.857143in}}{\pgfqpoint{2.627103in}{1.813434in}}%
\pgfusepath{clip}%
\pgfsetbuttcap%
\pgfsetmiterjoin%
\definecolor{currentfill}{rgb}{0.511253,0.510898,0.193296}%
\pgfsetfillcolor{currentfill}%
\pgfsetlinewidth{0.000000pt}%
\definecolor{currentstroke}{rgb}{0.000000,0.000000,0.000000}%
\pgfsetstrokecolor{currentstroke}%
\pgfsetstrokeopacity{0.000000}%
\pgfsetdash{}{0pt}%
\pgfpathmoveto{\pgfqpoint{4.065723in}{1.598306in}}%
\pgfpathlineto{\pgfqpoint{4.074659in}{1.598306in}}%
\pgfpathlineto{\pgfqpoint{4.074659in}{1.585807in}}%
\pgfpathlineto{\pgfqpoint{4.065723in}{1.585807in}}%
\pgfpathlineto{\pgfqpoint{4.065723in}{1.598306in}}%
\pgfpathclose%
\pgfusepath{fill}%
\end{pgfscope}%
\begin{pgfscope}%
\pgfpathrectangle{\pgfqpoint{3.722897in}{0.857143in}}{\pgfqpoint{2.627103in}{1.813434in}}%
\pgfusepath{clip}%
\pgfsetbuttcap%
\pgfsetmiterjoin%
\definecolor{currentfill}{rgb}{0.511253,0.510898,0.193296}%
\pgfsetfillcolor{currentfill}%
\pgfsetlinewidth{0.000000pt}%
\definecolor{currentstroke}{rgb}{0.000000,0.000000,0.000000}%
\pgfsetstrokecolor{currentstroke}%
\pgfsetstrokeopacity{0.000000}%
\pgfsetdash{}{0pt}%
\pgfpathmoveto{\pgfqpoint{4.076893in}{1.596477in}}%
\pgfpathlineto{\pgfqpoint{4.085830in}{1.596477in}}%
\pgfpathlineto{\pgfqpoint{4.085830in}{1.582523in}}%
\pgfpathlineto{\pgfqpoint{4.076893in}{1.582523in}}%
\pgfpathlineto{\pgfqpoint{4.076893in}{1.596477in}}%
\pgfpathclose%
\pgfusepath{fill}%
\end{pgfscope}%
\begin{pgfscope}%
\pgfpathrectangle{\pgfqpoint{3.722897in}{0.857143in}}{\pgfqpoint{2.627103in}{1.813434in}}%
\pgfusepath{clip}%
\pgfsetbuttcap%
\pgfsetmiterjoin%
\definecolor{currentfill}{rgb}{0.511253,0.510898,0.193296}%
\pgfsetfillcolor{currentfill}%
\pgfsetlinewidth{0.000000pt}%
\definecolor{currentstroke}{rgb}{0.000000,0.000000,0.000000}%
\pgfsetstrokecolor{currentstroke}%
\pgfsetstrokeopacity{0.000000}%
\pgfsetdash{}{0pt}%
\pgfpathmoveto{\pgfqpoint{4.088064in}{1.579841in}}%
\pgfpathlineto{\pgfqpoint{4.097001in}{1.579841in}}%
\pgfpathlineto{\pgfqpoint{4.097001in}{1.561024in}}%
\pgfpathlineto{\pgfqpoint{4.088064in}{1.561024in}}%
\pgfpathlineto{\pgfqpoint{4.088064in}{1.579841in}}%
\pgfpathclose%
\pgfusepath{fill}%
\end{pgfscope}%
\begin{pgfscope}%
\pgfpathrectangle{\pgfqpoint{3.722897in}{0.857143in}}{\pgfqpoint{2.627103in}{1.813434in}}%
\pgfusepath{clip}%
\pgfsetbuttcap%
\pgfsetmiterjoin%
\definecolor{currentfill}{rgb}{0.511253,0.510898,0.193296}%
\pgfsetfillcolor{currentfill}%
\pgfsetlinewidth{0.000000pt}%
\definecolor{currentstroke}{rgb}{0.000000,0.000000,0.000000}%
\pgfsetstrokecolor{currentstroke}%
\pgfsetstrokeopacity{0.000000}%
\pgfsetdash{}{0pt}%
\pgfpathmoveto{\pgfqpoint{4.099235in}{1.563069in}}%
\pgfpathlineto{\pgfqpoint{4.108171in}{1.563069in}}%
\pgfpathlineto{\pgfqpoint{4.108171in}{1.550457in}}%
\pgfpathlineto{\pgfqpoint{4.099235in}{1.550457in}}%
\pgfpathlineto{\pgfqpoint{4.099235in}{1.563069in}}%
\pgfpathclose%
\pgfusepath{fill}%
\end{pgfscope}%
\begin{pgfscope}%
\pgfpathrectangle{\pgfqpoint{3.722897in}{0.857143in}}{\pgfqpoint{2.627103in}{1.813434in}}%
\pgfusepath{clip}%
\pgfsetbuttcap%
\pgfsetmiterjoin%
\definecolor{currentfill}{rgb}{0.511253,0.510898,0.193296}%
\pgfsetfillcolor{currentfill}%
\pgfsetlinewidth{0.000000pt}%
\definecolor{currentstroke}{rgb}{0.000000,0.000000,0.000000}%
\pgfsetstrokecolor{currentstroke}%
\pgfsetstrokeopacity{0.000000}%
\pgfsetdash{}{0pt}%
\pgfpathmoveto{\pgfqpoint{4.110405in}{1.551437in}}%
\pgfpathlineto{\pgfqpoint{4.119342in}{1.551437in}}%
\pgfpathlineto{\pgfqpoint{4.119342in}{1.544102in}}%
\pgfpathlineto{\pgfqpoint{4.110405in}{1.544102in}}%
\pgfpathlineto{\pgfqpoint{4.110405in}{1.551437in}}%
\pgfpathclose%
\pgfusepath{fill}%
\end{pgfscope}%
\begin{pgfscope}%
\pgfpathrectangle{\pgfqpoint{3.722897in}{0.857143in}}{\pgfqpoint{2.627103in}{1.813434in}}%
\pgfusepath{clip}%
\pgfsetbuttcap%
\pgfsetmiterjoin%
\definecolor{currentfill}{rgb}{0.511253,0.510898,0.193296}%
\pgfsetfillcolor{currentfill}%
\pgfsetlinewidth{0.000000pt}%
\definecolor{currentstroke}{rgb}{0.000000,0.000000,0.000000}%
\pgfsetstrokecolor{currentstroke}%
\pgfsetstrokeopacity{0.000000}%
\pgfsetdash{}{0pt}%
\pgfpathmoveto{\pgfqpoint{4.121576in}{1.538656in}}%
\pgfpathlineto{\pgfqpoint{4.130512in}{1.538656in}}%
\pgfpathlineto{\pgfqpoint{4.130512in}{1.532734in}}%
\pgfpathlineto{\pgfqpoint{4.121576in}{1.532734in}}%
\pgfpathlineto{\pgfqpoint{4.121576in}{1.538656in}}%
\pgfpathclose%
\pgfusepath{fill}%
\end{pgfscope}%
\begin{pgfscope}%
\pgfpathrectangle{\pgfqpoint{3.722897in}{0.857143in}}{\pgfqpoint{2.627103in}{1.813434in}}%
\pgfusepath{clip}%
\pgfsetbuttcap%
\pgfsetmiterjoin%
\definecolor{currentfill}{rgb}{0.511253,0.510898,0.193296}%
\pgfsetfillcolor{currentfill}%
\pgfsetlinewidth{0.000000pt}%
\definecolor{currentstroke}{rgb}{0.000000,0.000000,0.000000}%
\pgfsetstrokecolor{currentstroke}%
\pgfsetstrokeopacity{0.000000}%
\pgfsetdash{}{0pt}%
\pgfpathmoveto{\pgfqpoint{4.132747in}{1.533470in}}%
\pgfpathlineto{\pgfqpoint{4.141683in}{1.533470in}}%
\pgfpathlineto{\pgfqpoint{4.141683in}{1.532710in}}%
\pgfpathlineto{\pgfqpoint{4.132747in}{1.532710in}}%
\pgfpathlineto{\pgfqpoint{4.132747in}{1.533470in}}%
\pgfpathclose%
\pgfusepath{fill}%
\end{pgfscope}%
\begin{pgfscope}%
\pgfpathrectangle{\pgfqpoint{3.722897in}{0.857143in}}{\pgfqpoint{2.627103in}{1.813434in}}%
\pgfusepath{clip}%
\pgfsetbuttcap%
\pgfsetmiterjoin%
\definecolor{currentfill}{rgb}{0.511253,0.510898,0.193296}%
\pgfsetfillcolor{currentfill}%
\pgfsetlinewidth{0.000000pt}%
\definecolor{currentstroke}{rgb}{0.000000,0.000000,0.000000}%
\pgfsetstrokecolor{currentstroke}%
\pgfsetstrokeopacity{0.000000}%
\pgfsetdash{}{0pt}%
\pgfpathmoveto{\pgfqpoint{4.143917in}{1.843346in}}%
\pgfpathlineto{\pgfqpoint{4.152854in}{1.843346in}}%
\pgfpathlineto{\pgfqpoint{4.152854in}{1.857722in}}%
\pgfpathlineto{\pgfqpoint{4.143917in}{1.857722in}}%
\pgfpathlineto{\pgfqpoint{4.143917in}{1.843346in}}%
\pgfpathclose%
\pgfusepath{fill}%
\end{pgfscope}%
\begin{pgfscope}%
\pgfpathrectangle{\pgfqpoint{3.722897in}{0.857143in}}{\pgfqpoint{2.627103in}{1.813434in}}%
\pgfusepath{clip}%
\pgfsetbuttcap%
\pgfsetmiterjoin%
\definecolor{currentfill}{rgb}{0.511253,0.510898,0.193296}%
\pgfsetfillcolor{currentfill}%
\pgfsetlinewidth{0.000000pt}%
\definecolor{currentstroke}{rgb}{0.000000,0.000000,0.000000}%
\pgfsetstrokecolor{currentstroke}%
\pgfsetstrokeopacity{0.000000}%
\pgfsetdash{}{0pt}%
\pgfpathmoveto{\pgfqpoint{4.155088in}{1.841205in}}%
\pgfpathlineto{\pgfqpoint{4.164024in}{1.841205in}}%
\pgfpathlineto{\pgfqpoint{4.164024in}{1.869320in}}%
\pgfpathlineto{\pgfqpoint{4.155088in}{1.869320in}}%
\pgfpathlineto{\pgfqpoint{4.155088in}{1.841205in}}%
\pgfpathclose%
\pgfusepath{fill}%
\end{pgfscope}%
\begin{pgfscope}%
\pgfpathrectangle{\pgfqpoint{3.722897in}{0.857143in}}{\pgfqpoint{2.627103in}{1.813434in}}%
\pgfusepath{clip}%
\pgfsetbuttcap%
\pgfsetmiterjoin%
\definecolor{currentfill}{rgb}{0.511253,0.510898,0.193296}%
\pgfsetfillcolor{currentfill}%
\pgfsetlinewidth{0.000000pt}%
\definecolor{currentstroke}{rgb}{0.000000,0.000000,0.000000}%
\pgfsetstrokecolor{currentstroke}%
\pgfsetstrokeopacity{0.000000}%
\pgfsetdash{}{0pt}%
\pgfpathmoveto{\pgfqpoint{4.166258in}{1.842051in}}%
\pgfpathlineto{\pgfqpoint{4.175195in}{1.842051in}}%
\pgfpathlineto{\pgfqpoint{4.175195in}{1.884973in}}%
\pgfpathlineto{\pgfqpoint{4.166258in}{1.884973in}}%
\pgfpathlineto{\pgfqpoint{4.166258in}{1.842051in}}%
\pgfpathclose%
\pgfusepath{fill}%
\end{pgfscope}%
\begin{pgfscope}%
\pgfpathrectangle{\pgfqpoint{3.722897in}{0.857143in}}{\pgfqpoint{2.627103in}{1.813434in}}%
\pgfusepath{clip}%
\pgfsetbuttcap%
\pgfsetmiterjoin%
\definecolor{currentfill}{rgb}{0.511253,0.510898,0.193296}%
\pgfsetfillcolor{currentfill}%
\pgfsetlinewidth{0.000000pt}%
\definecolor{currentstroke}{rgb}{0.000000,0.000000,0.000000}%
\pgfsetstrokecolor{currentstroke}%
\pgfsetstrokeopacity{0.000000}%
\pgfsetdash{}{0pt}%
\pgfpathmoveto{\pgfqpoint{4.177429in}{1.840628in}}%
\pgfpathlineto{\pgfqpoint{4.186365in}{1.840628in}}%
\pgfpathlineto{\pgfqpoint{4.186365in}{1.904637in}}%
\pgfpathlineto{\pgfqpoint{4.177429in}{1.904637in}}%
\pgfpathlineto{\pgfqpoint{4.177429in}{1.840628in}}%
\pgfpathclose%
\pgfusepath{fill}%
\end{pgfscope}%
\begin{pgfscope}%
\pgfpathrectangle{\pgfqpoint{3.722897in}{0.857143in}}{\pgfqpoint{2.627103in}{1.813434in}}%
\pgfusepath{clip}%
\pgfsetbuttcap%
\pgfsetmiterjoin%
\definecolor{currentfill}{rgb}{0.511253,0.510898,0.193296}%
\pgfsetfillcolor{currentfill}%
\pgfsetlinewidth{0.000000pt}%
\definecolor{currentstroke}{rgb}{0.000000,0.000000,0.000000}%
\pgfsetstrokecolor{currentstroke}%
\pgfsetstrokeopacity{0.000000}%
\pgfsetdash{}{0pt}%
\pgfpathmoveto{\pgfqpoint{4.188600in}{1.839704in}}%
\pgfpathlineto{\pgfqpoint{4.197536in}{1.839704in}}%
\pgfpathlineto{\pgfqpoint{4.197536in}{1.915907in}}%
\pgfpathlineto{\pgfqpoint{4.188600in}{1.915907in}}%
\pgfpathlineto{\pgfqpoint{4.188600in}{1.839704in}}%
\pgfpathclose%
\pgfusepath{fill}%
\end{pgfscope}%
\begin{pgfscope}%
\pgfpathrectangle{\pgfqpoint{3.722897in}{0.857143in}}{\pgfqpoint{2.627103in}{1.813434in}}%
\pgfusepath{clip}%
\pgfsetbuttcap%
\pgfsetmiterjoin%
\definecolor{currentfill}{rgb}{0.511253,0.510898,0.193296}%
\pgfsetfillcolor{currentfill}%
\pgfsetlinewidth{0.000000pt}%
\definecolor{currentstroke}{rgb}{0.000000,0.000000,0.000000}%
\pgfsetstrokecolor{currentstroke}%
\pgfsetstrokeopacity{0.000000}%
\pgfsetdash{}{0pt}%
\pgfpathmoveto{\pgfqpoint{4.199770in}{1.837166in}}%
\pgfpathlineto{\pgfqpoint{4.208707in}{1.837166in}}%
\pgfpathlineto{\pgfqpoint{4.208707in}{1.917483in}}%
\pgfpathlineto{\pgfqpoint{4.199770in}{1.917483in}}%
\pgfpathlineto{\pgfqpoint{4.199770in}{1.837166in}}%
\pgfpathclose%
\pgfusepath{fill}%
\end{pgfscope}%
\begin{pgfscope}%
\pgfpathrectangle{\pgfqpoint{3.722897in}{0.857143in}}{\pgfqpoint{2.627103in}{1.813434in}}%
\pgfusepath{clip}%
\pgfsetbuttcap%
\pgfsetmiterjoin%
\definecolor{currentfill}{rgb}{0.511253,0.510898,0.193296}%
\pgfsetfillcolor{currentfill}%
\pgfsetlinewidth{0.000000pt}%
\definecolor{currentstroke}{rgb}{0.000000,0.000000,0.000000}%
\pgfsetstrokecolor{currentstroke}%
\pgfsetstrokeopacity{0.000000}%
\pgfsetdash{}{0pt}%
\pgfpathmoveto{\pgfqpoint{4.210941in}{1.838323in}}%
\pgfpathlineto{\pgfqpoint{4.219877in}{1.838323in}}%
\pgfpathlineto{\pgfqpoint{4.219877in}{1.916667in}}%
\pgfpathlineto{\pgfqpoint{4.210941in}{1.916667in}}%
\pgfpathlineto{\pgfqpoint{4.210941in}{1.838323in}}%
\pgfpathclose%
\pgfusepath{fill}%
\end{pgfscope}%
\begin{pgfscope}%
\pgfpathrectangle{\pgfqpoint{3.722897in}{0.857143in}}{\pgfqpoint{2.627103in}{1.813434in}}%
\pgfusepath{clip}%
\pgfsetbuttcap%
\pgfsetmiterjoin%
\definecolor{currentfill}{rgb}{0.511253,0.510898,0.193296}%
\pgfsetfillcolor{currentfill}%
\pgfsetlinewidth{0.000000pt}%
\definecolor{currentstroke}{rgb}{0.000000,0.000000,0.000000}%
\pgfsetstrokecolor{currentstroke}%
\pgfsetstrokeopacity{0.000000}%
\pgfsetdash{}{0pt}%
\pgfpathmoveto{\pgfqpoint{4.222111in}{1.841108in}}%
\pgfpathlineto{\pgfqpoint{4.231048in}{1.841108in}}%
\pgfpathlineto{\pgfqpoint{4.231048in}{1.916947in}}%
\pgfpathlineto{\pgfqpoint{4.222111in}{1.916947in}}%
\pgfpathlineto{\pgfqpoint{4.222111in}{1.841108in}}%
\pgfpathclose%
\pgfusepath{fill}%
\end{pgfscope}%
\begin{pgfscope}%
\pgfpathrectangle{\pgfqpoint{3.722897in}{0.857143in}}{\pgfqpoint{2.627103in}{1.813434in}}%
\pgfusepath{clip}%
\pgfsetbuttcap%
\pgfsetmiterjoin%
\definecolor{currentfill}{rgb}{0.511253,0.510898,0.193296}%
\pgfsetfillcolor{currentfill}%
\pgfsetlinewidth{0.000000pt}%
\definecolor{currentstroke}{rgb}{0.000000,0.000000,0.000000}%
\pgfsetstrokecolor{currentstroke}%
\pgfsetstrokeopacity{0.000000}%
\pgfsetdash{}{0pt}%
\pgfpathmoveto{\pgfqpoint{4.233282in}{1.843898in}}%
\pgfpathlineto{\pgfqpoint{4.242218in}{1.843898in}}%
\pgfpathlineto{\pgfqpoint{4.242218in}{1.913977in}}%
\pgfpathlineto{\pgfqpoint{4.233282in}{1.913977in}}%
\pgfpathlineto{\pgfqpoint{4.233282in}{1.843898in}}%
\pgfpathclose%
\pgfusepath{fill}%
\end{pgfscope}%
\begin{pgfscope}%
\pgfpathrectangle{\pgfqpoint{3.722897in}{0.857143in}}{\pgfqpoint{2.627103in}{1.813434in}}%
\pgfusepath{clip}%
\pgfsetbuttcap%
\pgfsetmiterjoin%
\definecolor{currentfill}{rgb}{0.511253,0.510898,0.193296}%
\pgfsetfillcolor{currentfill}%
\pgfsetlinewidth{0.000000pt}%
\definecolor{currentstroke}{rgb}{0.000000,0.000000,0.000000}%
\pgfsetstrokecolor{currentstroke}%
\pgfsetstrokeopacity{0.000000}%
\pgfsetdash{}{0pt}%
\pgfpathmoveto{\pgfqpoint{4.244453in}{1.845517in}}%
\pgfpathlineto{\pgfqpoint{4.253389in}{1.845517in}}%
\pgfpathlineto{\pgfqpoint{4.253389in}{1.907650in}}%
\pgfpathlineto{\pgfqpoint{4.244453in}{1.907650in}}%
\pgfpathlineto{\pgfqpoint{4.244453in}{1.845517in}}%
\pgfpathclose%
\pgfusepath{fill}%
\end{pgfscope}%
\begin{pgfscope}%
\pgfpathrectangle{\pgfqpoint{3.722897in}{0.857143in}}{\pgfqpoint{2.627103in}{1.813434in}}%
\pgfusepath{clip}%
\pgfsetbuttcap%
\pgfsetmiterjoin%
\definecolor{currentfill}{rgb}{0.511253,0.510898,0.193296}%
\pgfsetfillcolor{currentfill}%
\pgfsetlinewidth{0.000000pt}%
\definecolor{currentstroke}{rgb}{0.000000,0.000000,0.000000}%
\pgfsetstrokecolor{currentstroke}%
\pgfsetstrokeopacity{0.000000}%
\pgfsetdash{}{0pt}%
\pgfpathmoveto{\pgfqpoint{4.255623in}{1.844966in}}%
\pgfpathlineto{\pgfqpoint{4.264560in}{1.844966in}}%
\pgfpathlineto{\pgfqpoint{4.264560in}{1.898078in}}%
\pgfpathlineto{\pgfqpoint{4.255623in}{1.898078in}}%
\pgfpathlineto{\pgfqpoint{4.255623in}{1.844966in}}%
\pgfpathclose%
\pgfusepath{fill}%
\end{pgfscope}%
\begin{pgfscope}%
\pgfpathrectangle{\pgfqpoint{3.722897in}{0.857143in}}{\pgfqpoint{2.627103in}{1.813434in}}%
\pgfusepath{clip}%
\pgfsetbuttcap%
\pgfsetmiterjoin%
\definecolor{currentfill}{rgb}{0.511253,0.510898,0.193296}%
\pgfsetfillcolor{currentfill}%
\pgfsetlinewidth{0.000000pt}%
\definecolor{currentstroke}{rgb}{0.000000,0.000000,0.000000}%
\pgfsetstrokecolor{currentstroke}%
\pgfsetstrokeopacity{0.000000}%
\pgfsetdash{}{0pt}%
\pgfpathmoveto{\pgfqpoint{4.266794in}{1.846359in}}%
\pgfpathlineto{\pgfqpoint{4.275730in}{1.846359in}}%
\pgfpathlineto{\pgfqpoint{4.275730in}{1.894666in}}%
\pgfpathlineto{\pgfqpoint{4.266794in}{1.894666in}}%
\pgfpathlineto{\pgfqpoint{4.266794in}{1.846359in}}%
\pgfpathclose%
\pgfusepath{fill}%
\end{pgfscope}%
\begin{pgfscope}%
\pgfpathrectangle{\pgfqpoint{3.722897in}{0.857143in}}{\pgfqpoint{2.627103in}{1.813434in}}%
\pgfusepath{clip}%
\pgfsetbuttcap%
\pgfsetmiterjoin%
\definecolor{currentfill}{rgb}{0.511253,0.510898,0.193296}%
\pgfsetfillcolor{currentfill}%
\pgfsetlinewidth{0.000000pt}%
\definecolor{currentstroke}{rgb}{0.000000,0.000000,0.000000}%
\pgfsetstrokecolor{currentstroke}%
\pgfsetstrokeopacity{0.000000}%
\pgfsetdash{}{0pt}%
\pgfpathmoveto{\pgfqpoint{4.277964in}{1.847902in}}%
\pgfpathlineto{\pgfqpoint{4.286901in}{1.847902in}}%
\pgfpathlineto{\pgfqpoint{4.286901in}{1.885121in}}%
\pgfpathlineto{\pgfqpoint{4.277964in}{1.885121in}}%
\pgfpathlineto{\pgfqpoint{4.277964in}{1.847902in}}%
\pgfpathclose%
\pgfusepath{fill}%
\end{pgfscope}%
\begin{pgfscope}%
\pgfpathrectangle{\pgfqpoint{3.722897in}{0.857143in}}{\pgfqpoint{2.627103in}{1.813434in}}%
\pgfusepath{clip}%
\pgfsetbuttcap%
\pgfsetmiterjoin%
\definecolor{currentfill}{rgb}{0.511253,0.510898,0.193296}%
\pgfsetfillcolor{currentfill}%
\pgfsetlinewidth{0.000000pt}%
\definecolor{currentstroke}{rgb}{0.000000,0.000000,0.000000}%
\pgfsetstrokecolor{currentstroke}%
\pgfsetstrokeopacity{0.000000}%
\pgfsetdash{}{0pt}%
\pgfpathmoveto{\pgfqpoint{4.289135in}{1.848321in}}%
\pgfpathlineto{\pgfqpoint{4.298071in}{1.848321in}}%
\pgfpathlineto{\pgfqpoint{4.298071in}{1.875137in}}%
\pgfpathlineto{\pgfqpoint{4.289135in}{1.875137in}}%
\pgfpathlineto{\pgfqpoint{4.289135in}{1.848321in}}%
\pgfpathclose%
\pgfusepath{fill}%
\end{pgfscope}%
\begin{pgfscope}%
\pgfpathrectangle{\pgfqpoint{3.722897in}{0.857143in}}{\pgfqpoint{2.627103in}{1.813434in}}%
\pgfusepath{clip}%
\pgfsetbuttcap%
\pgfsetmiterjoin%
\definecolor{currentfill}{rgb}{0.511253,0.510898,0.193296}%
\pgfsetfillcolor{currentfill}%
\pgfsetlinewidth{0.000000pt}%
\definecolor{currentstroke}{rgb}{0.000000,0.000000,0.000000}%
\pgfsetstrokecolor{currentstroke}%
\pgfsetstrokeopacity{0.000000}%
\pgfsetdash{}{0pt}%
\pgfpathmoveto{\pgfqpoint{4.300306in}{1.848725in}}%
\pgfpathlineto{\pgfqpoint{4.309242in}{1.848725in}}%
\pgfpathlineto{\pgfqpoint{4.309242in}{1.865131in}}%
\pgfpathlineto{\pgfqpoint{4.300306in}{1.865131in}}%
\pgfpathlineto{\pgfqpoint{4.300306in}{1.848725in}}%
\pgfpathclose%
\pgfusepath{fill}%
\end{pgfscope}%
\begin{pgfscope}%
\pgfpathrectangle{\pgfqpoint{3.722897in}{0.857143in}}{\pgfqpoint{2.627103in}{1.813434in}}%
\pgfusepath{clip}%
\pgfsetbuttcap%
\pgfsetmiterjoin%
\definecolor{currentfill}{rgb}{0.511253,0.510898,0.193296}%
\pgfsetfillcolor{currentfill}%
\pgfsetlinewidth{0.000000pt}%
\definecolor{currentstroke}{rgb}{0.000000,0.000000,0.000000}%
\pgfsetstrokecolor{currentstroke}%
\pgfsetstrokeopacity{0.000000}%
\pgfsetdash{}{0pt}%
\pgfpathmoveto{\pgfqpoint{4.311476in}{1.848846in}}%
\pgfpathlineto{\pgfqpoint{4.320413in}{1.848846in}}%
\pgfpathlineto{\pgfqpoint{4.320413in}{1.855424in}}%
\pgfpathlineto{\pgfqpoint{4.311476in}{1.855424in}}%
\pgfpathlineto{\pgfqpoint{4.311476in}{1.848846in}}%
\pgfpathclose%
\pgfusepath{fill}%
\end{pgfscope}%
\begin{pgfscope}%
\pgfpathrectangle{\pgfqpoint{3.722897in}{0.857143in}}{\pgfqpoint{2.627103in}{1.813434in}}%
\pgfusepath{clip}%
\pgfsetbuttcap%
\pgfsetmiterjoin%
\definecolor{currentfill}{rgb}{0.511253,0.510898,0.193296}%
\pgfsetfillcolor{currentfill}%
\pgfsetlinewidth{0.000000pt}%
\definecolor{currentstroke}{rgb}{0.000000,0.000000,0.000000}%
\pgfsetstrokecolor{currentstroke}%
\pgfsetstrokeopacity{0.000000}%
\pgfsetdash{}{0pt}%
\pgfpathmoveto{\pgfqpoint{4.322647in}{1.527418in}}%
\pgfpathlineto{\pgfqpoint{4.331583in}{1.527418in}}%
\pgfpathlineto{\pgfqpoint{4.331583in}{1.526120in}}%
\pgfpathlineto{\pgfqpoint{4.322647in}{1.526120in}}%
\pgfpathlineto{\pgfqpoint{4.322647in}{1.527418in}}%
\pgfpathclose%
\pgfusepath{fill}%
\end{pgfscope}%
\begin{pgfscope}%
\pgfpathrectangle{\pgfqpoint{3.722897in}{0.857143in}}{\pgfqpoint{2.627103in}{1.813434in}}%
\pgfusepath{clip}%
\pgfsetbuttcap%
\pgfsetmiterjoin%
\definecolor{currentfill}{rgb}{0.511253,0.510898,0.193296}%
\pgfsetfillcolor{currentfill}%
\pgfsetlinewidth{0.000000pt}%
\definecolor{currentstroke}{rgb}{0.000000,0.000000,0.000000}%
\pgfsetstrokecolor{currentstroke}%
\pgfsetstrokeopacity{0.000000}%
\pgfsetdash{}{0pt}%
\pgfpathmoveto{\pgfqpoint{4.333817in}{1.849655in}}%
\pgfpathlineto{\pgfqpoint{4.342754in}{1.849655in}}%
\pgfpathlineto{\pgfqpoint{4.342754in}{1.849856in}}%
\pgfpathlineto{\pgfqpoint{4.333817in}{1.849856in}}%
\pgfpathlineto{\pgfqpoint{4.333817in}{1.849655in}}%
\pgfpathclose%
\pgfusepath{fill}%
\end{pgfscope}%
\begin{pgfscope}%
\pgfpathrectangle{\pgfqpoint{3.722897in}{0.857143in}}{\pgfqpoint{2.627103in}{1.813434in}}%
\pgfusepath{clip}%
\pgfsetbuttcap%
\pgfsetmiterjoin%
\definecolor{currentfill}{rgb}{0.511253,0.510898,0.193296}%
\pgfsetfillcolor{currentfill}%
\pgfsetlinewidth{0.000000pt}%
\definecolor{currentstroke}{rgb}{0.000000,0.000000,0.000000}%
\pgfsetstrokecolor{currentstroke}%
\pgfsetstrokeopacity{0.000000}%
\pgfsetdash{}{0pt}%
\pgfpathmoveto{\pgfqpoint{4.344988in}{1.846922in}}%
\pgfpathlineto{\pgfqpoint{4.353925in}{1.846922in}}%
\pgfpathlineto{\pgfqpoint{4.353925in}{1.851537in}}%
\pgfpathlineto{\pgfqpoint{4.344988in}{1.851537in}}%
\pgfpathlineto{\pgfqpoint{4.344988in}{1.846922in}}%
\pgfpathclose%
\pgfusepath{fill}%
\end{pgfscope}%
\begin{pgfscope}%
\pgfpathrectangle{\pgfqpoint{3.722897in}{0.857143in}}{\pgfqpoint{2.627103in}{1.813434in}}%
\pgfusepath{clip}%
\pgfsetbuttcap%
\pgfsetmiterjoin%
\definecolor{currentfill}{rgb}{0.511253,0.510898,0.193296}%
\pgfsetfillcolor{currentfill}%
\pgfsetlinewidth{0.000000pt}%
\definecolor{currentstroke}{rgb}{0.000000,0.000000,0.000000}%
\pgfsetstrokecolor{currentstroke}%
\pgfsetstrokeopacity{0.000000}%
\pgfsetdash{}{0pt}%
\pgfpathmoveto{\pgfqpoint{4.356159in}{1.845556in}}%
\pgfpathlineto{\pgfqpoint{4.365095in}{1.845556in}}%
\pgfpathlineto{\pgfqpoint{4.365095in}{1.855699in}}%
\pgfpathlineto{\pgfqpoint{4.356159in}{1.855699in}}%
\pgfpathlineto{\pgfqpoint{4.356159in}{1.845556in}}%
\pgfpathclose%
\pgfusepath{fill}%
\end{pgfscope}%
\begin{pgfscope}%
\pgfpathrectangle{\pgfqpoint{3.722897in}{0.857143in}}{\pgfqpoint{2.627103in}{1.813434in}}%
\pgfusepath{clip}%
\pgfsetbuttcap%
\pgfsetmiterjoin%
\definecolor{currentfill}{rgb}{0.511253,0.510898,0.193296}%
\pgfsetfillcolor{currentfill}%
\pgfsetlinewidth{0.000000pt}%
\definecolor{currentstroke}{rgb}{0.000000,0.000000,0.000000}%
\pgfsetstrokecolor{currentstroke}%
\pgfsetstrokeopacity{0.000000}%
\pgfsetdash{}{0pt}%
\pgfpathmoveto{\pgfqpoint{4.367329in}{1.846172in}}%
\pgfpathlineto{\pgfqpoint{4.376266in}{1.846172in}}%
\pgfpathlineto{\pgfqpoint{4.376266in}{1.870244in}}%
\pgfpathlineto{\pgfqpoint{4.367329in}{1.870244in}}%
\pgfpathlineto{\pgfqpoint{4.367329in}{1.846172in}}%
\pgfpathclose%
\pgfusepath{fill}%
\end{pgfscope}%
\begin{pgfscope}%
\pgfpathrectangle{\pgfqpoint{3.722897in}{0.857143in}}{\pgfqpoint{2.627103in}{1.813434in}}%
\pgfusepath{clip}%
\pgfsetbuttcap%
\pgfsetmiterjoin%
\definecolor{currentfill}{rgb}{0.511253,0.510898,0.193296}%
\pgfsetfillcolor{currentfill}%
\pgfsetlinewidth{0.000000pt}%
\definecolor{currentstroke}{rgb}{0.000000,0.000000,0.000000}%
\pgfsetstrokecolor{currentstroke}%
\pgfsetstrokeopacity{0.000000}%
\pgfsetdash{}{0pt}%
\pgfpathmoveto{\pgfqpoint{4.378500in}{1.845140in}}%
\pgfpathlineto{\pgfqpoint{4.387436in}{1.845140in}}%
\pgfpathlineto{\pgfqpoint{4.387436in}{1.893063in}}%
\pgfpathlineto{\pgfqpoint{4.378500in}{1.893063in}}%
\pgfpathlineto{\pgfqpoint{4.378500in}{1.845140in}}%
\pgfpathclose%
\pgfusepath{fill}%
\end{pgfscope}%
\begin{pgfscope}%
\pgfpathrectangle{\pgfqpoint{3.722897in}{0.857143in}}{\pgfqpoint{2.627103in}{1.813434in}}%
\pgfusepath{clip}%
\pgfsetbuttcap%
\pgfsetmiterjoin%
\definecolor{currentfill}{rgb}{0.511253,0.510898,0.193296}%
\pgfsetfillcolor{currentfill}%
\pgfsetlinewidth{0.000000pt}%
\definecolor{currentstroke}{rgb}{0.000000,0.000000,0.000000}%
\pgfsetstrokecolor{currentstroke}%
\pgfsetstrokeopacity{0.000000}%
\pgfsetdash{}{0pt}%
\pgfpathmoveto{\pgfqpoint{4.389670in}{1.844018in}}%
\pgfpathlineto{\pgfqpoint{4.398607in}{1.844018in}}%
\pgfpathlineto{\pgfqpoint{4.398607in}{1.923586in}}%
\pgfpathlineto{\pgfqpoint{4.389670in}{1.923586in}}%
\pgfpathlineto{\pgfqpoint{4.389670in}{1.844018in}}%
\pgfpathclose%
\pgfusepath{fill}%
\end{pgfscope}%
\begin{pgfscope}%
\pgfpathrectangle{\pgfqpoint{3.722897in}{0.857143in}}{\pgfqpoint{2.627103in}{1.813434in}}%
\pgfusepath{clip}%
\pgfsetbuttcap%
\pgfsetmiterjoin%
\definecolor{currentfill}{rgb}{0.511253,0.510898,0.193296}%
\pgfsetfillcolor{currentfill}%
\pgfsetlinewidth{0.000000pt}%
\definecolor{currentstroke}{rgb}{0.000000,0.000000,0.000000}%
\pgfsetstrokecolor{currentstroke}%
\pgfsetstrokeopacity{0.000000}%
\pgfsetdash{}{0pt}%
\pgfpathmoveto{\pgfqpoint{4.400841in}{1.841287in}}%
\pgfpathlineto{\pgfqpoint{4.409778in}{1.841287in}}%
\pgfpathlineto{\pgfqpoint{4.409778in}{1.954900in}}%
\pgfpathlineto{\pgfqpoint{4.400841in}{1.954900in}}%
\pgfpathlineto{\pgfqpoint{4.400841in}{1.841287in}}%
\pgfpathclose%
\pgfusepath{fill}%
\end{pgfscope}%
\begin{pgfscope}%
\pgfpathrectangle{\pgfqpoint{3.722897in}{0.857143in}}{\pgfqpoint{2.627103in}{1.813434in}}%
\pgfusepath{clip}%
\pgfsetbuttcap%
\pgfsetmiterjoin%
\definecolor{currentfill}{rgb}{0.511253,0.510898,0.193296}%
\pgfsetfillcolor{currentfill}%
\pgfsetlinewidth{0.000000pt}%
\definecolor{currentstroke}{rgb}{0.000000,0.000000,0.000000}%
\pgfsetstrokecolor{currentstroke}%
\pgfsetstrokeopacity{0.000000}%
\pgfsetdash{}{0pt}%
\pgfpathmoveto{\pgfqpoint{4.412012in}{1.838927in}}%
\pgfpathlineto{\pgfqpoint{4.420948in}{1.838927in}}%
\pgfpathlineto{\pgfqpoint{4.420948in}{1.983614in}}%
\pgfpathlineto{\pgfqpoint{4.412012in}{1.983614in}}%
\pgfpathlineto{\pgfqpoint{4.412012in}{1.838927in}}%
\pgfpathclose%
\pgfusepath{fill}%
\end{pgfscope}%
\begin{pgfscope}%
\pgfpathrectangle{\pgfqpoint{3.722897in}{0.857143in}}{\pgfqpoint{2.627103in}{1.813434in}}%
\pgfusepath{clip}%
\pgfsetbuttcap%
\pgfsetmiterjoin%
\definecolor{currentfill}{rgb}{0.511253,0.510898,0.193296}%
\pgfsetfillcolor{currentfill}%
\pgfsetlinewidth{0.000000pt}%
\definecolor{currentstroke}{rgb}{0.000000,0.000000,0.000000}%
\pgfsetstrokecolor{currentstroke}%
\pgfsetstrokeopacity{0.000000}%
\pgfsetdash{}{0pt}%
\pgfpathmoveto{\pgfqpoint{4.423182in}{1.836324in}}%
\pgfpathlineto{\pgfqpoint{4.432119in}{1.836324in}}%
\pgfpathlineto{\pgfqpoint{4.432119in}{2.012090in}}%
\pgfpathlineto{\pgfqpoint{4.423182in}{2.012090in}}%
\pgfpathlineto{\pgfqpoint{4.423182in}{1.836324in}}%
\pgfpathclose%
\pgfusepath{fill}%
\end{pgfscope}%
\begin{pgfscope}%
\pgfpathrectangle{\pgfqpoint{3.722897in}{0.857143in}}{\pgfqpoint{2.627103in}{1.813434in}}%
\pgfusepath{clip}%
\pgfsetbuttcap%
\pgfsetmiterjoin%
\definecolor{currentfill}{rgb}{0.511253,0.510898,0.193296}%
\pgfsetfillcolor{currentfill}%
\pgfsetlinewidth{0.000000pt}%
\definecolor{currentstroke}{rgb}{0.000000,0.000000,0.000000}%
\pgfsetstrokecolor{currentstroke}%
\pgfsetstrokeopacity{0.000000}%
\pgfsetdash{}{0pt}%
\pgfpathmoveto{\pgfqpoint{4.434353in}{1.833856in}}%
\pgfpathlineto{\pgfqpoint{4.443289in}{1.833856in}}%
\pgfpathlineto{\pgfqpoint{4.443289in}{2.031538in}}%
\pgfpathlineto{\pgfqpoint{4.434353in}{2.031538in}}%
\pgfpathlineto{\pgfqpoint{4.434353in}{1.833856in}}%
\pgfpathclose%
\pgfusepath{fill}%
\end{pgfscope}%
\begin{pgfscope}%
\pgfpathrectangle{\pgfqpoint{3.722897in}{0.857143in}}{\pgfqpoint{2.627103in}{1.813434in}}%
\pgfusepath{clip}%
\pgfsetbuttcap%
\pgfsetmiterjoin%
\definecolor{currentfill}{rgb}{0.511253,0.510898,0.193296}%
\pgfsetfillcolor{currentfill}%
\pgfsetlinewidth{0.000000pt}%
\definecolor{currentstroke}{rgb}{0.000000,0.000000,0.000000}%
\pgfsetstrokecolor{currentstroke}%
\pgfsetstrokeopacity{0.000000}%
\pgfsetdash{}{0pt}%
\pgfpathmoveto{\pgfqpoint{4.445523in}{1.832383in}}%
\pgfpathlineto{\pgfqpoint{4.454460in}{1.832383in}}%
\pgfpathlineto{\pgfqpoint{4.454460in}{2.035405in}}%
\pgfpathlineto{\pgfqpoint{4.445523in}{2.035405in}}%
\pgfpathlineto{\pgfqpoint{4.445523in}{1.832383in}}%
\pgfpathclose%
\pgfusepath{fill}%
\end{pgfscope}%
\begin{pgfscope}%
\pgfpathrectangle{\pgfqpoint{3.722897in}{0.857143in}}{\pgfqpoint{2.627103in}{1.813434in}}%
\pgfusepath{clip}%
\pgfsetbuttcap%
\pgfsetmiterjoin%
\definecolor{currentfill}{rgb}{0.511253,0.510898,0.193296}%
\pgfsetfillcolor{currentfill}%
\pgfsetlinewidth{0.000000pt}%
\definecolor{currentstroke}{rgb}{0.000000,0.000000,0.000000}%
\pgfsetstrokecolor{currentstroke}%
\pgfsetstrokeopacity{0.000000}%
\pgfsetdash{}{0pt}%
\pgfpathmoveto{\pgfqpoint{4.456694in}{1.829130in}}%
\pgfpathlineto{\pgfqpoint{4.465631in}{1.829130in}}%
\pgfpathlineto{\pgfqpoint{4.465631in}{2.032265in}}%
\pgfpathlineto{\pgfqpoint{4.456694in}{2.032265in}}%
\pgfpathlineto{\pgfqpoint{4.456694in}{1.829130in}}%
\pgfpathclose%
\pgfusepath{fill}%
\end{pgfscope}%
\begin{pgfscope}%
\pgfpathrectangle{\pgfqpoint{3.722897in}{0.857143in}}{\pgfqpoint{2.627103in}{1.813434in}}%
\pgfusepath{clip}%
\pgfsetbuttcap%
\pgfsetmiterjoin%
\definecolor{currentfill}{rgb}{0.511253,0.510898,0.193296}%
\pgfsetfillcolor{currentfill}%
\pgfsetlinewidth{0.000000pt}%
\definecolor{currentstroke}{rgb}{0.000000,0.000000,0.000000}%
\pgfsetstrokecolor{currentstroke}%
\pgfsetstrokeopacity{0.000000}%
\pgfsetdash{}{0pt}%
\pgfpathmoveto{\pgfqpoint{4.467865in}{1.827047in}}%
\pgfpathlineto{\pgfqpoint{4.476801in}{1.827047in}}%
\pgfpathlineto{\pgfqpoint{4.476801in}{2.029280in}}%
\pgfpathlineto{\pgfqpoint{4.467865in}{2.029280in}}%
\pgfpathlineto{\pgfqpoint{4.467865in}{1.827047in}}%
\pgfpathclose%
\pgfusepath{fill}%
\end{pgfscope}%
\begin{pgfscope}%
\pgfpathrectangle{\pgfqpoint{3.722897in}{0.857143in}}{\pgfqpoint{2.627103in}{1.813434in}}%
\pgfusepath{clip}%
\pgfsetbuttcap%
\pgfsetmiterjoin%
\definecolor{currentfill}{rgb}{0.511253,0.510898,0.193296}%
\pgfsetfillcolor{currentfill}%
\pgfsetlinewidth{0.000000pt}%
\definecolor{currentstroke}{rgb}{0.000000,0.000000,0.000000}%
\pgfsetstrokecolor{currentstroke}%
\pgfsetstrokeopacity{0.000000}%
\pgfsetdash{}{0pt}%
\pgfpathmoveto{\pgfqpoint{4.479035in}{1.825328in}}%
\pgfpathlineto{\pgfqpoint{4.487972in}{1.825328in}}%
\pgfpathlineto{\pgfqpoint{4.487972in}{2.022715in}}%
\pgfpathlineto{\pgfqpoint{4.479035in}{2.022715in}}%
\pgfpathlineto{\pgfqpoint{4.479035in}{1.825328in}}%
\pgfpathclose%
\pgfusepath{fill}%
\end{pgfscope}%
\begin{pgfscope}%
\pgfpathrectangle{\pgfqpoint{3.722897in}{0.857143in}}{\pgfqpoint{2.627103in}{1.813434in}}%
\pgfusepath{clip}%
\pgfsetbuttcap%
\pgfsetmiterjoin%
\definecolor{currentfill}{rgb}{0.511253,0.510898,0.193296}%
\pgfsetfillcolor{currentfill}%
\pgfsetlinewidth{0.000000pt}%
\definecolor{currentstroke}{rgb}{0.000000,0.000000,0.000000}%
\pgfsetstrokecolor{currentstroke}%
\pgfsetstrokeopacity{0.000000}%
\pgfsetdash{}{0pt}%
\pgfpathmoveto{\pgfqpoint{4.490206in}{1.825608in}}%
\pgfpathlineto{\pgfqpoint{4.499142in}{1.825608in}}%
\pgfpathlineto{\pgfqpoint{4.499142in}{2.032611in}}%
\pgfpathlineto{\pgfqpoint{4.490206in}{2.032611in}}%
\pgfpathlineto{\pgfqpoint{4.490206in}{1.825608in}}%
\pgfpathclose%
\pgfusepath{fill}%
\end{pgfscope}%
\begin{pgfscope}%
\pgfpathrectangle{\pgfqpoint{3.722897in}{0.857143in}}{\pgfqpoint{2.627103in}{1.813434in}}%
\pgfusepath{clip}%
\pgfsetbuttcap%
\pgfsetmiterjoin%
\definecolor{currentfill}{rgb}{0.511253,0.510898,0.193296}%
\pgfsetfillcolor{currentfill}%
\pgfsetlinewidth{0.000000pt}%
\definecolor{currentstroke}{rgb}{0.000000,0.000000,0.000000}%
\pgfsetstrokecolor{currentstroke}%
\pgfsetstrokeopacity{0.000000}%
\pgfsetdash{}{0pt}%
\pgfpathmoveto{\pgfqpoint{4.501377in}{1.821458in}}%
\pgfpathlineto{\pgfqpoint{4.510313in}{1.821458in}}%
\pgfpathlineto{\pgfqpoint{4.510313in}{2.049401in}}%
\pgfpathlineto{\pgfqpoint{4.501377in}{2.049401in}}%
\pgfpathlineto{\pgfqpoint{4.501377in}{1.821458in}}%
\pgfpathclose%
\pgfusepath{fill}%
\end{pgfscope}%
\begin{pgfscope}%
\pgfpathrectangle{\pgfqpoint{3.722897in}{0.857143in}}{\pgfqpoint{2.627103in}{1.813434in}}%
\pgfusepath{clip}%
\pgfsetbuttcap%
\pgfsetmiterjoin%
\definecolor{currentfill}{rgb}{0.511253,0.510898,0.193296}%
\pgfsetfillcolor{currentfill}%
\pgfsetlinewidth{0.000000pt}%
\definecolor{currentstroke}{rgb}{0.000000,0.000000,0.000000}%
\pgfsetstrokecolor{currentstroke}%
\pgfsetstrokeopacity{0.000000}%
\pgfsetdash{}{0pt}%
\pgfpathmoveto{\pgfqpoint{4.512547in}{1.821699in}}%
\pgfpathlineto{\pgfqpoint{4.521484in}{1.821699in}}%
\pgfpathlineto{\pgfqpoint{4.521484in}{2.071933in}}%
\pgfpathlineto{\pgfqpoint{4.512547in}{2.071933in}}%
\pgfpathlineto{\pgfqpoint{4.512547in}{1.821699in}}%
\pgfpathclose%
\pgfusepath{fill}%
\end{pgfscope}%
\begin{pgfscope}%
\pgfpathrectangle{\pgfqpoint{3.722897in}{0.857143in}}{\pgfqpoint{2.627103in}{1.813434in}}%
\pgfusepath{clip}%
\pgfsetbuttcap%
\pgfsetmiterjoin%
\definecolor{currentfill}{rgb}{0.511253,0.510898,0.193296}%
\pgfsetfillcolor{currentfill}%
\pgfsetlinewidth{0.000000pt}%
\definecolor{currentstroke}{rgb}{0.000000,0.000000,0.000000}%
\pgfsetstrokecolor{currentstroke}%
\pgfsetstrokeopacity{0.000000}%
\pgfsetdash{}{0pt}%
\pgfpathmoveto{\pgfqpoint{4.523718in}{1.817949in}}%
\pgfpathlineto{\pgfqpoint{4.532654in}{1.817949in}}%
\pgfpathlineto{\pgfqpoint{4.532654in}{2.096891in}}%
\pgfpathlineto{\pgfqpoint{4.523718in}{2.096891in}}%
\pgfpathlineto{\pgfqpoint{4.523718in}{1.817949in}}%
\pgfpathclose%
\pgfusepath{fill}%
\end{pgfscope}%
\begin{pgfscope}%
\pgfpathrectangle{\pgfqpoint{3.722897in}{0.857143in}}{\pgfqpoint{2.627103in}{1.813434in}}%
\pgfusepath{clip}%
\pgfsetbuttcap%
\pgfsetmiterjoin%
\definecolor{currentfill}{rgb}{0.511253,0.510898,0.193296}%
\pgfsetfillcolor{currentfill}%
\pgfsetlinewidth{0.000000pt}%
\definecolor{currentstroke}{rgb}{0.000000,0.000000,0.000000}%
\pgfsetstrokecolor{currentstroke}%
\pgfsetstrokeopacity{0.000000}%
\pgfsetdash{}{0pt}%
\pgfpathmoveto{\pgfqpoint{4.534888in}{1.815731in}}%
\pgfpathlineto{\pgfqpoint{4.543825in}{1.815731in}}%
\pgfpathlineto{\pgfqpoint{4.543825in}{2.108736in}}%
\pgfpathlineto{\pgfqpoint{4.534888in}{2.108736in}}%
\pgfpathlineto{\pgfqpoint{4.534888in}{1.815731in}}%
\pgfpathclose%
\pgfusepath{fill}%
\end{pgfscope}%
\begin{pgfscope}%
\pgfpathrectangle{\pgfqpoint{3.722897in}{0.857143in}}{\pgfqpoint{2.627103in}{1.813434in}}%
\pgfusepath{clip}%
\pgfsetbuttcap%
\pgfsetmiterjoin%
\definecolor{currentfill}{rgb}{0.511253,0.510898,0.193296}%
\pgfsetfillcolor{currentfill}%
\pgfsetlinewidth{0.000000pt}%
\definecolor{currentstroke}{rgb}{0.000000,0.000000,0.000000}%
\pgfsetstrokecolor{currentstroke}%
\pgfsetstrokeopacity{0.000000}%
\pgfsetdash{}{0pt}%
\pgfpathmoveto{\pgfqpoint{4.546059in}{1.813884in}}%
\pgfpathlineto{\pgfqpoint{4.554995in}{1.813884in}}%
\pgfpathlineto{\pgfqpoint{4.554995in}{2.107204in}}%
\pgfpathlineto{\pgfqpoint{4.546059in}{2.107204in}}%
\pgfpathlineto{\pgfqpoint{4.546059in}{1.813884in}}%
\pgfpathclose%
\pgfusepath{fill}%
\end{pgfscope}%
\begin{pgfscope}%
\pgfpathrectangle{\pgfqpoint{3.722897in}{0.857143in}}{\pgfqpoint{2.627103in}{1.813434in}}%
\pgfusepath{clip}%
\pgfsetbuttcap%
\pgfsetmiterjoin%
\definecolor{currentfill}{rgb}{0.511253,0.510898,0.193296}%
\pgfsetfillcolor{currentfill}%
\pgfsetlinewidth{0.000000pt}%
\definecolor{currentstroke}{rgb}{0.000000,0.000000,0.000000}%
\pgfsetstrokecolor{currentstroke}%
\pgfsetstrokeopacity{0.000000}%
\pgfsetdash{}{0pt}%
\pgfpathmoveto{\pgfqpoint{4.557230in}{1.813884in}}%
\pgfpathlineto{\pgfqpoint{4.566166in}{1.813884in}}%
\pgfpathlineto{\pgfqpoint{4.566166in}{2.103986in}}%
\pgfpathlineto{\pgfqpoint{4.557230in}{2.103986in}}%
\pgfpathlineto{\pgfqpoint{4.557230in}{1.813884in}}%
\pgfpathclose%
\pgfusepath{fill}%
\end{pgfscope}%
\begin{pgfscope}%
\pgfpathrectangle{\pgfqpoint{3.722897in}{0.857143in}}{\pgfqpoint{2.627103in}{1.813434in}}%
\pgfusepath{clip}%
\pgfsetbuttcap%
\pgfsetmiterjoin%
\definecolor{currentfill}{rgb}{0.511253,0.510898,0.193296}%
\pgfsetfillcolor{currentfill}%
\pgfsetlinewidth{0.000000pt}%
\definecolor{currentstroke}{rgb}{0.000000,0.000000,0.000000}%
\pgfsetstrokecolor{currentstroke}%
\pgfsetstrokeopacity{0.000000}%
\pgfsetdash{}{0pt}%
\pgfpathmoveto{\pgfqpoint{4.568400in}{1.813884in}}%
\pgfpathlineto{\pgfqpoint{4.577337in}{1.813884in}}%
\pgfpathlineto{\pgfqpoint{4.577337in}{2.083574in}}%
\pgfpathlineto{\pgfqpoint{4.568400in}{2.083574in}}%
\pgfpathlineto{\pgfqpoint{4.568400in}{1.813884in}}%
\pgfpathclose%
\pgfusepath{fill}%
\end{pgfscope}%
\begin{pgfscope}%
\pgfpathrectangle{\pgfqpoint{3.722897in}{0.857143in}}{\pgfqpoint{2.627103in}{1.813434in}}%
\pgfusepath{clip}%
\pgfsetbuttcap%
\pgfsetmiterjoin%
\definecolor{currentfill}{rgb}{0.511253,0.510898,0.193296}%
\pgfsetfillcolor{currentfill}%
\pgfsetlinewidth{0.000000pt}%
\definecolor{currentstroke}{rgb}{0.000000,0.000000,0.000000}%
\pgfsetstrokecolor{currentstroke}%
\pgfsetstrokeopacity{0.000000}%
\pgfsetdash{}{0pt}%
\pgfpathmoveto{\pgfqpoint{4.579571in}{1.813884in}}%
\pgfpathlineto{\pgfqpoint{4.588507in}{1.813884in}}%
\pgfpathlineto{\pgfqpoint{4.588507in}{2.051571in}}%
\pgfpathlineto{\pgfqpoint{4.579571in}{2.051571in}}%
\pgfpathlineto{\pgfqpoint{4.579571in}{1.813884in}}%
\pgfpathclose%
\pgfusepath{fill}%
\end{pgfscope}%
\begin{pgfscope}%
\pgfpathrectangle{\pgfqpoint{3.722897in}{0.857143in}}{\pgfqpoint{2.627103in}{1.813434in}}%
\pgfusepath{clip}%
\pgfsetbuttcap%
\pgfsetmiterjoin%
\definecolor{currentfill}{rgb}{0.511253,0.510898,0.193296}%
\pgfsetfillcolor{currentfill}%
\pgfsetlinewidth{0.000000pt}%
\definecolor{currentstroke}{rgb}{0.000000,0.000000,0.000000}%
\pgfsetstrokecolor{currentstroke}%
\pgfsetstrokeopacity{0.000000}%
\pgfsetdash{}{0pt}%
\pgfpathmoveto{\pgfqpoint{4.590741in}{1.813884in}}%
\pgfpathlineto{\pgfqpoint{4.599678in}{1.813884in}}%
\pgfpathlineto{\pgfqpoint{4.599678in}{2.023717in}}%
\pgfpathlineto{\pgfqpoint{4.590741in}{2.023717in}}%
\pgfpathlineto{\pgfqpoint{4.590741in}{1.813884in}}%
\pgfpathclose%
\pgfusepath{fill}%
\end{pgfscope}%
\begin{pgfscope}%
\pgfpathrectangle{\pgfqpoint{3.722897in}{0.857143in}}{\pgfqpoint{2.627103in}{1.813434in}}%
\pgfusepath{clip}%
\pgfsetbuttcap%
\pgfsetmiterjoin%
\definecolor{currentfill}{rgb}{0.511253,0.510898,0.193296}%
\pgfsetfillcolor{currentfill}%
\pgfsetlinewidth{0.000000pt}%
\definecolor{currentstroke}{rgb}{0.000000,0.000000,0.000000}%
\pgfsetstrokecolor{currentstroke}%
\pgfsetstrokeopacity{0.000000}%
\pgfsetdash{}{0pt}%
\pgfpathmoveto{\pgfqpoint{4.601912in}{1.813884in}}%
\pgfpathlineto{\pgfqpoint{4.610848in}{1.813884in}}%
\pgfpathlineto{\pgfqpoint{4.610848in}{2.001059in}}%
\pgfpathlineto{\pgfqpoint{4.601912in}{2.001059in}}%
\pgfpathlineto{\pgfqpoint{4.601912in}{1.813884in}}%
\pgfpathclose%
\pgfusepath{fill}%
\end{pgfscope}%
\begin{pgfscope}%
\pgfpathrectangle{\pgfqpoint{3.722897in}{0.857143in}}{\pgfqpoint{2.627103in}{1.813434in}}%
\pgfusepath{clip}%
\pgfsetbuttcap%
\pgfsetmiterjoin%
\definecolor{currentfill}{rgb}{0.511253,0.510898,0.193296}%
\pgfsetfillcolor{currentfill}%
\pgfsetlinewidth{0.000000pt}%
\definecolor{currentstroke}{rgb}{0.000000,0.000000,0.000000}%
\pgfsetstrokecolor{currentstroke}%
\pgfsetstrokeopacity{0.000000}%
\pgfsetdash{}{0pt}%
\pgfpathmoveto{\pgfqpoint{4.613083in}{1.813884in}}%
\pgfpathlineto{\pgfqpoint{4.622019in}{1.813884in}}%
\pgfpathlineto{\pgfqpoint{4.622019in}{1.980269in}}%
\pgfpathlineto{\pgfqpoint{4.613083in}{1.980269in}}%
\pgfpathlineto{\pgfqpoint{4.613083in}{1.813884in}}%
\pgfpathclose%
\pgfusepath{fill}%
\end{pgfscope}%
\begin{pgfscope}%
\pgfpathrectangle{\pgfqpoint{3.722897in}{0.857143in}}{\pgfqpoint{2.627103in}{1.813434in}}%
\pgfusepath{clip}%
\pgfsetbuttcap%
\pgfsetmiterjoin%
\definecolor{currentfill}{rgb}{0.511253,0.510898,0.193296}%
\pgfsetfillcolor{currentfill}%
\pgfsetlinewidth{0.000000pt}%
\definecolor{currentstroke}{rgb}{0.000000,0.000000,0.000000}%
\pgfsetstrokecolor{currentstroke}%
\pgfsetstrokeopacity{0.000000}%
\pgfsetdash{}{0pt}%
\pgfpathmoveto{\pgfqpoint{4.624253in}{1.813884in}}%
\pgfpathlineto{\pgfqpoint{4.633190in}{1.813884in}}%
\pgfpathlineto{\pgfqpoint{4.633190in}{1.968072in}}%
\pgfpathlineto{\pgfqpoint{4.624253in}{1.968072in}}%
\pgfpathlineto{\pgfqpoint{4.624253in}{1.813884in}}%
\pgfpathclose%
\pgfusepath{fill}%
\end{pgfscope}%
\begin{pgfscope}%
\pgfpathrectangle{\pgfqpoint{3.722897in}{0.857143in}}{\pgfqpoint{2.627103in}{1.813434in}}%
\pgfusepath{clip}%
\pgfsetbuttcap%
\pgfsetmiterjoin%
\definecolor{currentfill}{rgb}{0.511253,0.510898,0.193296}%
\pgfsetfillcolor{currentfill}%
\pgfsetlinewidth{0.000000pt}%
\definecolor{currentstroke}{rgb}{0.000000,0.000000,0.000000}%
\pgfsetstrokecolor{currentstroke}%
\pgfsetstrokeopacity{0.000000}%
\pgfsetdash{}{0pt}%
\pgfpathmoveto{\pgfqpoint{4.635424in}{1.813884in}}%
\pgfpathlineto{\pgfqpoint{4.644360in}{1.813884in}}%
\pgfpathlineto{\pgfqpoint{4.644360in}{1.966220in}}%
\pgfpathlineto{\pgfqpoint{4.635424in}{1.966220in}}%
\pgfpathlineto{\pgfqpoint{4.635424in}{1.813884in}}%
\pgfpathclose%
\pgfusepath{fill}%
\end{pgfscope}%
\begin{pgfscope}%
\pgfpathrectangle{\pgfqpoint{3.722897in}{0.857143in}}{\pgfqpoint{2.627103in}{1.813434in}}%
\pgfusepath{clip}%
\pgfsetbuttcap%
\pgfsetmiterjoin%
\definecolor{currentfill}{rgb}{0.511253,0.510898,0.193296}%
\pgfsetfillcolor{currentfill}%
\pgfsetlinewidth{0.000000pt}%
\definecolor{currentstroke}{rgb}{0.000000,0.000000,0.000000}%
\pgfsetstrokecolor{currentstroke}%
\pgfsetstrokeopacity{0.000000}%
\pgfsetdash{}{0pt}%
\pgfpathmoveto{\pgfqpoint{4.646594in}{1.813884in}}%
\pgfpathlineto{\pgfqpoint{4.655531in}{1.813884in}}%
\pgfpathlineto{\pgfqpoint{4.655531in}{1.976022in}}%
\pgfpathlineto{\pgfqpoint{4.646594in}{1.976022in}}%
\pgfpathlineto{\pgfqpoint{4.646594in}{1.813884in}}%
\pgfpathclose%
\pgfusepath{fill}%
\end{pgfscope}%
\begin{pgfscope}%
\pgfpathrectangle{\pgfqpoint{3.722897in}{0.857143in}}{\pgfqpoint{2.627103in}{1.813434in}}%
\pgfusepath{clip}%
\pgfsetbuttcap%
\pgfsetmiterjoin%
\definecolor{currentfill}{rgb}{0.511253,0.510898,0.193296}%
\pgfsetfillcolor{currentfill}%
\pgfsetlinewidth{0.000000pt}%
\definecolor{currentstroke}{rgb}{0.000000,0.000000,0.000000}%
\pgfsetstrokecolor{currentstroke}%
\pgfsetstrokeopacity{0.000000}%
\pgfsetdash{}{0pt}%
\pgfpathmoveto{\pgfqpoint{4.657765in}{1.813884in}}%
\pgfpathlineto{\pgfqpoint{4.666701in}{1.813884in}}%
\pgfpathlineto{\pgfqpoint{4.666701in}{1.990167in}}%
\pgfpathlineto{\pgfqpoint{4.657765in}{1.990167in}}%
\pgfpathlineto{\pgfqpoint{4.657765in}{1.813884in}}%
\pgfpathclose%
\pgfusepath{fill}%
\end{pgfscope}%
\begin{pgfscope}%
\pgfpathrectangle{\pgfqpoint{3.722897in}{0.857143in}}{\pgfqpoint{2.627103in}{1.813434in}}%
\pgfusepath{clip}%
\pgfsetbuttcap%
\pgfsetmiterjoin%
\definecolor{currentfill}{rgb}{0.511253,0.510898,0.193296}%
\pgfsetfillcolor{currentfill}%
\pgfsetlinewidth{0.000000pt}%
\definecolor{currentstroke}{rgb}{0.000000,0.000000,0.000000}%
\pgfsetstrokecolor{currentstroke}%
\pgfsetstrokeopacity{0.000000}%
\pgfsetdash{}{0pt}%
\pgfpathmoveto{\pgfqpoint{4.668936in}{1.813884in}}%
\pgfpathlineto{\pgfqpoint{4.677872in}{1.813884in}}%
\pgfpathlineto{\pgfqpoint{4.677872in}{1.997748in}}%
\pgfpathlineto{\pgfqpoint{4.668936in}{1.997748in}}%
\pgfpathlineto{\pgfqpoint{4.668936in}{1.813884in}}%
\pgfpathclose%
\pgfusepath{fill}%
\end{pgfscope}%
\begin{pgfscope}%
\pgfpathrectangle{\pgfqpoint{3.722897in}{0.857143in}}{\pgfqpoint{2.627103in}{1.813434in}}%
\pgfusepath{clip}%
\pgfsetbuttcap%
\pgfsetmiterjoin%
\definecolor{currentfill}{rgb}{0.511253,0.510898,0.193296}%
\pgfsetfillcolor{currentfill}%
\pgfsetlinewidth{0.000000pt}%
\definecolor{currentstroke}{rgb}{0.000000,0.000000,0.000000}%
\pgfsetstrokecolor{currentstroke}%
\pgfsetstrokeopacity{0.000000}%
\pgfsetdash{}{0pt}%
\pgfpathmoveto{\pgfqpoint{4.680106in}{1.822211in}}%
\pgfpathlineto{\pgfqpoint{4.689043in}{1.822211in}}%
\pgfpathlineto{\pgfqpoint{4.689043in}{2.002176in}}%
\pgfpathlineto{\pgfqpoint{4.680106in}{2.002176in}}%
\pgfpathlineto{\pgfqpoint{4.680106in}{1.822211in}}%
\pgfpathclose%
\pgfusepath{fill}%
\end{pgfscope}%
\begin{pgfscope}%
\pgfpathrectangle{\pgfqpoint{3.722897in}{0.857143in}}{\pgfqpoint{2.627103in}{1.813434in}}%
\pgfusepath{clip}%
\pgfsetbuttcap%
\pgfsetmiterjoin%
\definecolor{currentfill}{rgb}{0.511253,0.510898,0.193296}%
\pgfsetfillcolor{currentfill}%
\pgfsetlinewidth{0.000000pt}%
\definecolor{currentstroke}{rgb}{0.000000,0.000000,0.000000}%
\pgfsetstrokecolor{currentstroke}%
\pgfsetstrokeopacity{0.000000}%
\pgfsetdash{}{0pt}%
\pgfpathmoveto{\pgfqpoint{4.691277in}{1.836197in}}%
\pgfpathlineto{\pgfqpoint{4.700213in}{1.836197in}}%
\pgfpathlineto{\pgfqpoint{4.700213in}{2.013253in}}%
\pgfpathlineto{\pgfqpoint{4.691277in}{2.013253in}}%
\pgfpathlineto{\pgfqpoint{4.691277in}{1.836197in}}%
\pgfpathclose%
\pgfusepath{fill}%
\end{pgfscope}%
\begin{pgfscope}%
\pgfpathrectangle{\pgfqpoint{3.722897in}{0.857143in}}{\pgfqpoint{2.627103in}{1.813434in}}%
\pgfusepath{clip}%
\pgfsetbuttcap%
\pgfsetmiterjoin%
\definecolor{currentfill}{rgb}{0.511253,0.510898,0.193296}%
\pgfsetfillcolor{currentfill}%
\pgfsetlinewidth{0.000000pt}%
\definecolor{currentstroke}{rgb}{0.000000,0.000000,0.000000}%
\pgfsetstrokecolor{currentstroke}%
\pgfsetstrokeopacity{0.000000}%
\pgfsetdash{}{0pt}%
\pgfpathmoveto{\pgfqpoint{4.702447in}{1.850266in}}%
\pgfpathlineto{\pgfqpoint{4.711384in}{1.850266in}}%
\pgfpathlineto{\pgfqpoint{4.711384in}{2.024862in}}%
\pgfpathlineto{\pgfqpoint{4.702447in}{2.024862in}}%
\pgfpathlineto{\pgfqpoint{4.702447in}{1.850266in}}%
\pgfpathclose%
\pgfusepath{fill}%
\end{pgfscope}%
\begin{pgfscope}%
\pgfpathrectangle{\pgfqpoint{3.722897in}{0.857143in}}{\pgfqpoint{2.627103in}{1.813434in}}%
\pgfusepath{clip}%
\pgfsetbuttcap%
\pgfsetmiterjoin%
\definecolor{currentfill}{rgb}{0.511253,0.510898,0.193296}%
\pgfsetfillcolor{currentfill}%
\pgfsetlinewidth{0.000000pt}%
\definecolor{currentstroke}{rgb}{0.000000,0.000000,0.000000}%
\pgfsetstrokecolor{currentstroke}%
\pgfsetstrokeopacity{0.000000}%
\pgfsetdash{}{0pt}%
\pgfpathmoveto{\pgfqpoint{4.713618in}{1.865142in}}%
\pgfpathlineto{\pgfqpoint{4.722554in}{1.865142in}}%
\pgfpathlineto{\pgfqpoint{4.722554in}{2.030854in}}%
\pgfpathlineto{\pgfqpoint{4.713618in}{2.030854in}}%
\pgfpathlineto{\pgfqpoint{4.713618in}{1.865142in}}%
\pgfpathclose%
\pgfusepath{fill}%
\end{pgfscope}%
\begin{pgfscope}%
\pgfpathrectangle{\pgfqpoint{3.722897in}{0.857143in}}{\pgfqpoint{2.627103in}{1.813434in}}%
\pgfusepath{clip}%
\pgfsetbuttcap%
\pgfsetmiterjoin%
\definecolor{currentfill}{rgb}{0.511253,0.510898,0.193296}%
\pgfsetfillcolor{currentfill}%
\pgfsetlinewidth{0.000000pt}%
\definecolor{currentstroke}{rgb}{0.000000,0.000000,0.000000}%
\pgfsetstrokecolor{currentstroke}%
\pgfsetstrokeopacity{0.000000}%
\pgfsetdash{}{0pt}%
\pgfpathmoveto{\pgfqpoint{4.724789in}{1.879326in}}%
\pgfpathlineto{\pgfqpoint{4.733725in}{1.879326in}}%
\pgfpathlineto{\pgfqpoint{4.733725in}{2.031183in}}%
\pgfpathlineto{\pgfqpoint{4.724789in}{2.031183in}}%
\pgfpathlineto{\pgfqpoint{4.724789in}{1.879326in}}%
\pgfpathclose%
\pgfusepath{fill}%
\end{pgfscope}%
\begin{pgfscope}%
\pgfpathrectangle{\pgfqpoint{3.722897in}{0.857143in}}{\pgfqpoint{2.627103in}{1.813434in}}%
\pgfusepath{clip}%
\pgfsetbuttcap%
\pgfsetmiterjoin%
\definecolor{currentfill}{rgb}{0.511253,0.510898,0.193296}%
\pgfsetfillcolor{currentfill}%
\pgfsetlinewidth{0.000000pt}%
\definecolor{currentstroke}{rgb}{0.000000,0.000000,0.000000}%
\pgfsetstrokecolor{currentstroke}%
\pgfsetstrokeopacity{0.000000}%
\pgfsetdash{}{0pt}%
\pgfpathmoveto{\pgfqpoint{4.735959in}{1.892761in}}%
\pgfpathlineto{\pgfqpoint{4.744896in}{1.892761in}}%
\pgfpathlineto{\pgfqpoint{4.744896in}{2.035513in}}%
\pgfpathlineto{\pgfqpoint{4.735959in}{2.035513in}}%
\pgfpathlineto{\pgfqpoint{4.735959in}{1.892761in}}%
\pgfpathclose%
\pgfusepath{fill}%
\end{pgfscope}%
\begin{pgfscope}%
\pgfpathrectangle{\pgfqpoint{3.722897in}{0.857143in}}{\pgfqpoint{2.627103in}{1.813434in}}%
\pgfusepath{clip}%
\pgfsetbuttcap%
\pgfsetmiterjoin%
\definecolor{currentfill}{rgb}{0.511253,0.510898,0.193296}%
\pgfsetfillcolor{currentfill}%
\pgfsetlinewidth{0.000000pt}%
\definecolor{currentstroke}{rgb}{0.000000,0.000000,0.000000}%
\pgfsetstrokecolor{currentstroke}%
\pgfsetstrokeopacity{0.000000}%
\pgfsetdash{}{0pt}%
\pgfpathmoveto{\pgfqpoint{4.747130in}{1.906169in}}%
\pgfpathlineto{\pgfqpoint{4.756066in}{1.906169in}}%
\pgfpathlineto{\pgfqpoint{4.756066in}{2.035480in}}%
\pgfpathlineto{\pgfqpoint{4.747130in}{2.035480in}}%
\pgfpathlineto{\pgfqpoint{4.747130in}{1.906169in}}%
\pgfpathclose%
\pgfusepath{fill}%
\end{pgfscope}%
\begin{pgfscope}%
\pgfpathrectangle{\pgfqpoint{3.722897in}{0.857143in}}{\pgfqpoint{2.627103in}{1.813434in}}%
\pgfusepath{clip}%
\pgfsetbuttcap%
\pgfsetmiterjoin%
\definecolor{currentfill}{rgb}{0.511253,0.510898,0.193296}%
\pgfsetfillcolor{currentfill}%
\pgfsetlinewidth{0.000000pt}%
\definecolor{currentstroke}{rgb}{0.000000,0.000000,0.000000}%
\pgfsetstrokecolor{currentstroke}%
\pgfsetstrokeopacity{0.000000}%
\pgfsetdash{}{0pt}%
\pgfpathmoveto{\pgfqpoint{4.758300in}{1.918280in}}%
\pgfpathlineto{\pgfqpoint{4.767237in}{1.918280in}}%
\pgfpathlineto{\pgfqpoint{4.767237in}{2.037129in}}%
\pgfpathlineto{\pgfqpoint{4.758300in}{2.037129in}}%
\pgfpathlineto{\pgfqpoint{4.758300in}{1.918280in}}%
\pgfpathclose%
\pgfusepath{fill}%
\end{pgfscope}%
\begin{pgfscope}%
\pgfpathrectangle{\pgfqpoint{3.722897in}{0.857143in}}{\pgfqpoint{2.627103in}{1.813434in}}%
\pgfusepath{clip}%
\pgfsetbuttcap%
\pgfsetmiterjoin%
\definecolor{currentfill}{rgb}{0.511253,0.510898,0.193296}%
\pgfsetfillcolor{currentfill}%
\pgfsetlinewidth{0.000000pt}%
\definecolor{currentstroke}{rgb}{0.000000,0.000000,0.000000}%
\pgfsetstrokecolor{currentstroke}%
\pgfsetstrokeopacity{0.000000}%
\pgfsetdash{}{0pt}%
\pgfpathmoveto{\pgfqpoint{4.769471in}{1.932212in}}%
\pgfpathlineto{\pgfqpoint{4.778408in}{1.932212in}}%
\pgfpathlineto{\pgfqpoint{4.778408in}{2.043557in}}%
\pgfpathlineto{\pgfqpoint{4.769471in}{2.043557in}}%
\pgfpathlineto{\pgfqpoint{4.769471in}{1.932212in}}%
\pgfpathclose%
\pgfusepath{fill}%
\end{pgfscope}%
\begin{pgfscope}%
\pgfpathrectangle{\pgfqpoint{3.722897in}{0.857143in}}{\pgfqpoint{2.627103in}{1.813434in}}%
\pgfusepath{clip}%
\pgfsetbuttcap%
\pgfsetmiterjoin%
\definecolor{currentfill}{rgb}{0.511253,0.510898,0.193296}%
\pgfsetfillcolor{currentfill}%
\pgfsetlinewidth{0.000000pt}%
\definecolor{currentstroke}{rgb}{0.000000,0.000000,0.000000}%
\pgfsetstrokecolor{currentstroke}%
\pgfsetstrokeopacity{0.000000}%
\pgfsetdash{}{0pt}%
\pgfpathmoveto{\pgfqpoint{4.780642in}{1.949412in}}%
\pgfpathlineto{\pgfqpoint{4.789578in}{1.949412in}}%
\pgfpathlineto{\pgfqpoint{4.789578in}{2.039609in}}%
\pgfpathlineto{\pgfqpoint{4.780642in}{2.039609in}}%
\pgfpathlineto{\pgfqpoint{4.780642in}{1.949412in}}%
\pgfpathclose%
\pgfusepath{fill}%
\end{pgfscope}%
\begin{pgfscope}%
\pgfpathrectangle{\pgfqpoint{3.722897in}{0.857143in}}{\pgfqpoint{2.627103in}{1.813434in}}%
\pgfusepath{clip}%
\pgfsetbuttcap%
\pgfsetmiterjoin%
\definecolor{currentfill}{rgb}{0.511253,0.510898,0.193296}%
\pgfsetfillcolor{currentfill}%
\pgfsetlinewidth{0.000000pt}%
\definecolor{currentstroke}{rgb}{0.000000,0.000000,0.000000}%
\pgfsetstrokecolor{currentstroke}%
\pgfsetstrokeopacity{0.000000}%
\pgfsetdash{}{0pt}%
\pgfpathmoveto{\pgfqpoint{4.791812in}{1.965400in}}%
\pgfpathlineto{\pgfqpoint{4.800749in}{1.965400in}}%
\pgfpathlineto{\pgfqpoint{4.800749in}{2.041370in}}%
\pgfpathlineto{\pgfqpoint{4.791812in}{2.041370in}}%
\pgfpathlineto{\pgfqpoint{4.791812in}{1.965400in}}%
\pgfpathclose%
\pgfusepath{fill}%
\end{pgfscope}%
\begin{pgfscope}%
\pgfpathrectangle{\pgfqpoint{3.722897in}{0.857143in}}{\pgfqpoint{2.627103in}{1.813434in}}%
\pgfusepath{clip}%
\pgfsetbuttcap%
\pgfsetmiterjoin%
\definecolor{currentfill}{rgb}{0.511253,0.510898,0.193296}%
\pgfsetfillcolor{currentfill}%
\pgfsetlinewidth{0.000000pt}%
\definecolor{currentstroke}{rgb}{0.000000,0.000000,0.000000}%
\pgfsetstrokecolor{currentstroke}%
\pgfsetstrokeopacity{0.000000}%
\pgfsetdash{}{0pt}%
\pgfpathmoveto{\pgfqpoint{4.802983in}{1.981606in}}%
\pgfpathlineto{\pgfqpoint{4.811919in}{1.981606in}}%
\pgfpathlineto{\pgfqpoint{4.811919in}{2.048673in}}%
\pgfpathlineto{\pgfqpoint{4.802983in}{2.048673in}}%
\pgfpathlineto{\pgfqpoint{4.802983in}{1.981606in}}%
\pgfpathclose%
\pgfusepath{fill}%
\end{pgfscope}%
\begin{pgfscope}%
\pgfpathrectangle{\pgfqpoint{3.722897in}{0.857143in}}{\pgfqpoint{2.627103in}{1.813434in}}%
\pgfusepath{clip}%
\pgfsetbuttcap%
\pgfsetmiterjoin%
\definecolor{currentfill}{rgb}{0.511253,0.510898,0.193296}%
\pgfsetfillcolor{currentfill}%
\pgfsetlinewidth{0.000000pt}%
\definecolor{currentstroke}{rgb}{0.000000,0.000000,0.000000}%
\pgfsetstrokecolor{currentstroke}%
\pgfsetstrokeopacity{0.000000}%
\pgfsetdash{}{0pt}%
\pgfpathmoveto{\pgfqpoint{4.814153in}{1.999428in}}%
\pgfpathlineto{\pgfqpoint{4.823090in}{1.999428in}}%
\pgfpathlineto{\pgfqpoint{4.823090in}{2.053448in}}%
\pgfpathlineto{\pgfqpoint{4.814153in}{2.053448in}}%
\pgfpathlineto{\pgfqpoint{4.814153in}{1.999428in}}%
\pgfpathclose%
\pgfusepath{fill}%
\end{pgfscope}%
\begin{pgfscope}%
\pgfpathrectangle{\pgfqpoint{3.722897in}{0.857143in}}{\pgfqpoint{2.627103in}{1.813434in}}%
\pgfusepath{clip}%
\pgfsetbuttcap%
\pgfsetmiterjoin%
\definecolor{currentfill}{rgb}{0.511253,0.510898,0.193296}%
\pgfsetfillcolor{currentfill}%
\pgfsetlinewidth{0.000000pt}%
\definecolor{currentstroke}{rgb}{0.000000,0.000000,0.000000}%
\pgfsetstrokecolor{currentstroke}%
\pgfsetstrokeopacity{0.000000}%
\pgfsetdash{}{0pt}%
\pgfpathmoveto{\pgfqpoint{4.825324in}{2.018580in}}%
\pgfpathlineto{\pgfqpoint{4.834261in}{2.018580in}}%
\pgfpathlineto{\pgfqpoint{4.834261in}{2.063945in}}%
\pgfpathlineto{\pgfqpoint{4.825324in}{2.063945in}}%
\pgfpathlineto{\pgfqpoint{4.825324in}{2.018580in}}%
\pgfpathclose%
\pgfusepath{fill}%
\end{pgfscope}%
\begin{pgfscope}%
\pgfpathrectangle{\pgfqpoint{3.722897in}{0.857143in}}{\pgfqpoint{2.627103in}{1.813434in}}%
\pgfusepath{clip}%
\pgfsetbuttcap%
\pgfsetmiterjoin%
\definecolor{currentfill}{rgb}{0.511253,0.510898,0.193296}%
\pgfsetfillcolor{currentfill}%
\pgfsetlinewidth{0.000000pt}%
\definecolor{currentstroke}{rgb}{0.000000,0.000000,0.000000}%
\pgfsetstrokecolor{currentstroke}%
\pgfsetstrokeopacity{0.000000}%
\pgfsetdash{}{0pt}%
\pgfpathmoveto{\pgfqpoint{4.836495in}{2.042259in}}%
\pgfpathlineto{\pgfqpoint{4.845431in}{2.042259in}}%
\pgfpathlineto{\pgfqpoint{4.845431in}{2.075703in}}%
\pgfpathlineto{\pgfqpoint{4.836495in}{2.075703in}}%
\pgfpathlineto{\pgfqpoint{4.836495in}{2.042259in}}%
\pgfpathclose%
\pgfusepath{fill}%
\end{pgfscope}%
\begin{pgfscope}%
\pgfpathrectangle{\pgfqpoint{3.722897in}{0.857143in}}{\pgfqpoint{2.627103in}{1.813434in}}%
\pgfusepath{clip}%
\pgfsetbuttcap%
\pgfsetmiterjoin%
\definecolor{currentfill}{rgb}{0.511253,0.510898,0.193296}%
\pgfsetfillcolor{currentfill}%
\pgfsetlinewidth{0.000000pt}%
\definecolor{currentstroke}{rgb}{0.000000,0.000000,0.000000}%
\pgfsetstrokecolor{currentstroke}%
\pgfsetstrokeopacity{0.000000}%
\pgfsetdash{}{0pt}%
\pgfpathmoveto{\pgfqpoint{4.847665in}{2.058937in}}%
\pgfpathlineto{\pgfqpoint{4.856602in}{2.058937in}}%
\pgfpathlineto{\pgfqpoint{4.856602in}{2.082894in}}%
\pgfpathlineto{\pgfqpoint{4.847665in}{2.082894in}}%
\pgfpathlineto{\pgfqpoint{4.847665in}{2.058937in}}%
\pgfpathclose%
\pgfusepath{fill}%
\end{pgfscope}%
\begin{pgfscope}%
\pgfpathrectangle{\pgfqpoint{3.722897in}{0.857143in}}{\pgfqpoint{2.627103in}{1.813434in}}%
\pgfusepath{clip}%
\pgfsetbuttcap%
\pgfsetmiterjoin%
\definecolor{currentfill}{rgb}{0.511253,0.510898,0.193296}%
\pgfsetfillcolor{currentfill}%
\pgfsetlinewidth{0.000000pt}%
\definecolor{currentstroke}{rgb}{0.000000,0.000000,0.000000}%
\pgfsetstrokecolor{currentstroke}%
\pgfsetstrokeopacity{0.000000}%
\pgfsetdash{}{0pt}%
\pgfpathmoveto{\pgfqpoint{4.858836in}{2.083113in}}%
\pgfpathlineto{\pgfqpoint{4.867772in}{2.083113in}}%
\pgfpathlineto{\pgfqpoint{4.867772in}{2.100055in}}%
\pgfpathlineto{\pgfqpoint{4.858836in}{2.100055in}}%
\pgfpathlineto{\pgfqpoint{4.858836in}{2.083113in}}%
\pgfpathclose%
\pgfusepath{fill}%
\end{pgfscope}%
\begin{pgfscope}%
\pgfpathrectangle{\pgfqpoint{3.722897in}{0.857143in}}{\pgfqpoint{2.627103in}{1.813434in}}%
\pgfusepath{clip}%
\pgfsetbuttcap%
\pgfsetmiterjoin%
\definecolor{currentfill}{rgb}{0.511253,0.510898,0.193296}%
\pgfsetfillcolor{currentfill}%
\pgfsetlinewidth{0.000000pt}%
\definecolor{currentstroke}{rgb}{0.000000,0.000000,0.000000}%
\pgfsetstrokecolor{currentstroke}%
\pgfsetstrokeopacity{0.000000}%
\pgfsetdash{}{0pt}%
\pgfpathmoveto{\pgfqpoint{4.870006in}{2.112997in}}%
\pgfpathlineto{\pgfqpoint{4.878943in}{2.112997in}}%
\pgfpathlineto{\pgfqpoint{4.878943in}{2.126989in}}%
\pgfpathlineto{\pgfqpoint{4.870006in}{2.126989in}}%
\pgfpathlineto{\pgfqpoint{4.870006in}{2.112997in}}%
\pgfpathclose%
\pgfusepath{fill}%
\end{pgfscope}%
\begin{pgfscope}%
\pgfpathrectangle{\pgfqpoint{3.722897in}{0.857143in}}{\pgfqpoint{2.627103in}{1.813434in}}%
\pgfusepath{clip}%
\pgfsetbuttcap%
\pgfsetmiterjoin%
\definecolor{currentfill}{rgb}{0.511253,0.510898,0.193296}%
\pgfsetfillcolor{currentfill}%
\pgfsetlinewidth{0.000000pt}%
\definecolor{currentstroke}{rgb}{0.000000,0.000000,0.000000}%
\pgfsetstrokecolor{currentstroke}%
\pgfsetstrokeopacity{0.000000}%
\pgfsetdash{}{0pt}%
\pgfpathmoveto{\pgfqpoint{4.881177in}{2.131225in}}%
\pgfpathlineto{\pgfqpoint{4.890114in}{2.131225in}}%
\pgfpathlineto{\pgfqpoint{4.890114in}{2.143506in}}%
\pgfpathlineto{\pgfqpoint{4.881177in}{2.143506in}}%
\pgfpathlineto{\pgfqpoint{4.881177in}{2.131225in}}%
\pgfpathclose%
\pgfusepath{fill}%
\end{pgfscope}%
\begin{pgfscope}%
\pgfpathrectangle{\pgfqpoint{3.722897in}{0.857143in}}{\pgfqpoint{2.627103in}{1.813434in}}%
\pgfusepath{clip}%
\pgfsetbuttcap%
\pgfsetmiterjoin%
\definecolor{currentfill}{rgb}{0.511253,0.510898,0.193296}%
\pgfsetfillcolor{currentfill}%
\pgfsetlinewidth{0.000000pt}%
\definecolor{currentstroke}{rgb}{0.000000,0.000000,0.000000}%
\pgfsetstrokecolor{currentstroke}%
\pgfsetstrokeopacity{0.000000}%
\pgfsetdash{}{0pt}%
\pgfpathmoveto{\pgfqpoint{4.892348in}{2.134674in}}%
\pgfpathlineto{\pgfqpoint{4.901284in}{2.134674in}}%
\pgfpathlineto{\pgfqpoint{4.901284in}{2.140362in}}%
\pgfpathlineto{\pgfqpoint{4.892348in}{2.140362in}}%
\pgfpathlineto{\pgfqpoint{4.892348in}{2.134674in}}%
\pgfpathclose%
\pgfusepath{fill}%
\end{pgfscope}%
\begin{pgfscope}%
\pgfpathrectangle{\pgfqpoint{3.722897in}{0.857143in}}{\pgfqpoint{2.627103in}{1.813434in}}%
\pgfusepath{clip}%
\pgfsetbuttcap%
\pgfsetmiterjoin%
\definecolor{currentfill}{rgb}{0.511253,0.510898,0.193296}%
\pgfsetfillcolor{currentfill}%
\pgfsetlinewidth{0.000000pt}%
\definecolor{currentstroke}{rgb}{0.000000,0.000000,0.000000}%
\pgfsetstrokecolor{currentstroke}%
\pgfsetstrokeopacity{0.000000}%
\pgfsetdash{}{0pt}%
\pgfpathmoveto{\pgfqpoint{4.903518in}{1.782081in}}%
\pgfpathlineto{\pgfqpoint{4.912455in}{1.782081in}}%
\pgfpathlineto{\pgfqpoint{4.912455in}{1.780925in}}%
\pgfpathlineto{\pgfqpoint{4.903518in}{1.780925in}}%
\pgfpathlineto{\pgfqpoint{4.903518in}{1.782081in}}%
\pgfpathclose%
\pgfusepath{fill}%
\end{pgfscope}%
\begin{pgfscope}%
\pgfpathrectangle{\pgfqpoint{3.722897in}{0.857143in}}{\pgfqpoint{2.627103in}{1.813434in}}%
\pgfusepath{clip}%
\pgfsetbuttcap%
\pgfsetmiterjoin%
\definecolor{currentfill}{rgb}{0.511253,0.510898,0.193296}%
\pgfsetfillcolor{currentfill}%
\pgfsetlinewidth{0.000000pt}%
\definecolor{currentstroke}{rgb}{0.000000,0.000000,0.000000}%
\pgfsetstrokecolor{currentstroke}%
\pgfsetstrokeopacity{0.000000}%
\pgfsetdash{}{0pt}%
\pgfpathmoveto{\pgfqpoint{4.914689in}{1.781822in}}%
\pgfpathlineto{\pgfqpoint{4.923625in}{1.781822in}}%
\pgfpathlineto{\pgfqpoint{4.923625in}{1.776329in}}%
\pgfpathlineto{\pgfqpoint{4.914689in}{1.776329in}}%
\pgfpathlineto{\pgfqpoint{4.914689in}{1.781822in}}%
\pgfpathclose%
\pgfusepath{fill}%
\end{pgfscope}%
\begin{pgfscope}%
\pgfpathrectangle{\pgfqpoint{3.722897in}{0.857143in}}{\pgfqpoint{2.627103in}{1.813434in}}%
\pgfusepath{clip}%
\pgfsetbuttcap%
\pgfsetmiterjoin%
\definecolor{currentfill}{rgb}{0.511253,0.510898,0.193296}%
\pgfsetfillcolor{currentfill}%
\pgfsetlinewidth{0.000000pt}%
\definecolor{currentstroke}{rgb}{0.000000,0.000000,0.000000}%
\pgfsetstrokecolor{currentstroke}%
\pgfsetstrokeopacity{0.000000}%
\pgfsetdash{}{0pt}%
\pgfpathmoveto{\pgfqpoint{4.925860in}{1.779363in}}%
\pgfpathlineto{\pgfqpoint{4.934796in}{1.779363in}}%
\pgfpathlineto{\pgfqpoint{4.934796in}{1.759251in}}%
\pgfpathlineto{\pgfqpoint{4.925860in}{1.759251in}}%
\pgfpathlineto{\pgfqpoint{4.925860in}{1.779363in}}%
\pgfpathclose%
\pgfusepath{fill}%
\end{pgfscope}%
\begin{pgfscope}%
\pgfpathrectangle{\pgfqpoint{3.722897in}{0.857143in}}{\pgfqpoint{2.627103in}{1.813434in}}%
\pgfusepath{clip}%
\pgfsetbuttcap%
\pgfsetmiterjoin%
\definecolor{currentfill}{rgb}{0.511253,0.510898,0.193296}%
\pgfsetfillcolor{currentfill}%
\pgfsetlinewidth{0.000000pt}%
\definecolor{currentstroke}{rgb}{0.000000,0.000000,0.000000}%
\pgfsetstrokecolor{currentstroke}%
\pgfsetstrokeopacity{0.000000}%
\pgfsetdash{}{0pt}%
\pgfpathmoveto{\pgfqpoint{4.937030in}{1.778897in}}%
\pgfpathlineto{\pgfqpoint{4.945967in}{1.778897in}}%
\pgfpathlineto{\pgfqpoint{4.945967in}{1.740853in}}%
\pgfpathlineto{\pgfqpoint{4.937030in}{1.740853in}}%
\pgfpathlineto{\pgfqpoint{4.937030in}{1.778897in}}%
\pgfpathclose%
\pgfusepath{fill}%
\end{pgfscope}%
\begin{pgfscope}%
\pgfpathrectangle{\pgfqpoint{3.722897in}{0.857143in}}{\pgfqpoint{2.627103in}{1.813434in}}%
\pgfusepath{clip}%
\pgfsetbuttcap%
\pgfsetmiterjoin%
\definecolor{currentfill}{rgb}{0.511253,0.510898,0.193296}%
\pgfsetfillcolor{currentfill}%
\pgfsetlinewidth{0.000000pt}%
\definecolor{currentstroke}{rgb}{0.000000,0.000000,0.000000}%
\pgfsetstrokecolor{currentstroke}%
\pgfsetstrokeopacity{0.000000}%
\pgfsetdash{}{0pt}%
\pgfpathmoveto{\pgfqpoint{4.948201in}{1.780387in}}%
\pgfpathlineto{\pgfqpoint{4.957137in}{1.780387in}}%
\pgfpathlineto{\pgfqpoint{4.957137in}{1.735713in}}%
\pgfpathlineto{\pgfqpoint{4.948201in}{1.735713in}}%
\pgfpathlineto{\pgfqpoint{4.948201in}{1.780387in}}%
\pgfpathclose%
\pgfusepath{fill}%
\end{pgfscope}%
\begin{pgfscope}%
\pgfpathrectangle{\pgfqpoint{3.722897in}{0.857143in}}{\pgfqpoint{2.627103in}{1.813434in}}%
\pgfusepath{clip}%
\pgfsetbuttcap%
\pgfsetmiterjoin%
\definecolor{currentfill}{rgb}{0.511253,0.510898,0.193296}%
\pgfsetfillcolor{currentfill}%
\pgfsetlinewidth{0.000000pt}%
\definecolor{currentstroke}{rgb}{0.000000,0.000000,0.000000}%
\pgfsetstrokecolor{currentstroke}%
\pgfsetstrokeopacity{0.000000}%
\pgfsetdash{}{0pt}%
\pgfpathmoveto{\pgfqpoint{4.959371in}{1.780501in}}%
\pgfpathlineto{\pgfqpoint{4.968308in}{1.780501in}}%
\pgfpathlineto{\pgfqpoint{4.968308in}{1.727951in}}%
\pgfpathlineto{\pgfqpoint{4.959371in}{1.727951in}}%
\pgfpathlineto{\pgfqpoint{4.959371in}{1.780501in}}%
\pgfpathclose%
\pgfusepath{fill}%
\end{pgfscope}%
\begin{pgfscope}%
\pgfpathrectangle{\pgfqpoint{3.722897in}{0.857143in}}{\pgfqpoint{2.627103in}{1.813434in}}%
\pgfusepath{clip}%
\pgfsetbuttcap%
\pgfsetmiterjoin%
\definecolor{currentfill}{rgb}{0.511253,0.510898,0.193296}%
\pgfsetfillcolor{currentfill}%
\pgfsetlinewidth{0.000000pt}%
\definecolor{currentstroke}{rgb}{0.000000,0.000000,0.000000}%
\pgfsetstrokecolor{currentstroke}%
\pgfsetstrokeopacity{0.000000}%
\pgfsetdash{}{0pt}%
\pgfpathmoveto{\pgfqpoint{4.970542in}{1.780631in}}%
\pgfpathlineto{\pgfqpoint{4.979478in}{1.780631in}}%
\pgfpathlineto{\pgfqpoint{4.979478in}{1.723739in}}%
\pgfpathlineto{\pgfqpoint{4.970542in}{1.723739in}}%
\pgfpathlineto{\pgfqpoint{4.970542in}{1.780631in}}%
\pgfpathclose%
\pgfusepath{fill}%
\end{pgfscope}%
\begin{pgfscope}%
\pgfpathrectangle{\pgfqpoint{3.722897in}{0.857143in}}{\pgfqpoint{2.627103in}{1.813434in}}%
\pgfusepath{clip}%
\pgfsetbuttcap%
\pgfsetmiterjoin%
\definecolor{currentfill}{rgb}{0.511253,0.510898,0.193296}%
\pgfsetfillcolor{currentfill}%
\pgfsetlinewidth{0.000000pt}%
\definecolor{currentstroke}{rgb}{0.000000,0.000000,0.000000}%
\pgfsetstrokecolor{currentstroke}%
\pgfsetstrokeopacity{0.000000}%
\pgfsetdash{}{0pt}%
\pgfpathmoveto{\pgfqpoint{4.981713in}{1.783723in}}%
\pgfpathlineto{\pgfqpoint{4.990649in}{1.783723in}}%
\pgfpathlineto{\pgfqpoint{4.990649in}{1.723310in}}%
\pgfpathlineto{\pgfqpoint{4.981713in}{1.723310in}}%
\pgfpathlineto{\pgfqpoint{4.981713in}{1.783723in}}%
\pgfpathclose%
\pgfusepath{fill}%
\end{pgfscope}%
\begin{pgfscope}%
\pgfpathrectangle{\pgfqpoint{3.722897in}{0.857143in}}{\pgfqpoint{2.627103in}{1.813434in}}%
\pgfusepath{clip}%
\pgfsetbuttcap%
\pgfsetmiterjoin%
\definecolor{currentfill}{rgb}{0.511253,0.510898,0.193296}%
\pgfsetfillcolor{currentfill}%
\pgfsetlinewidth{0.000000pt}%
\definecolor{currentstroke}{rgb}{0.000000,0.000000,0.000000}%
\pgfsetstrokecolor{currentstroke}%
\pgfsetstrokeopacity{0.000000}%
\pgfsetdash{}{0pt}%
\pgfpathmoveto{\pgfqpoint{4.992883in}{1.784930in}}%
\pgfpathlineto{\pgfqpoint{5.001820in}{1.784930in}}%
\pgfpathlineto{\pgfqpoint{5.001820in}{1.715859in}}%
\pgfpathlineto{\pgfqpoint{4.992883in}{1.715859in}}%
\pgfpathlineto{\pgfqpoint{4.992883in}{1.784930in}}%
\pgfpathclose%
\pgfusepath{fill}%
\end{pgfscope}%
\begin{pgfscope}%
\pgfpathrectangle{\pgfqpoint{3.722897in}{0.857143in}}{\pgfqpoint{2.627103in}{1.813434in}}%
\pgfusepath{clip}%
\pgfsetbuttcap%
\pgfsetmiterjoin%
\definecolor{currentfill}{rgb}{0.511253,0.510898,0.193296}%
\pgfsetfillcolor{currentfill}%
\pgfsetlinewidth{0.000000pt}%
\definecolor{currentstroke}{rgb}{0.000000,0.000000,0.000000}%
\pgfsetstrokecolor{currentstroke}%
\pgfsetstrokeopacity{0.000000}%
\pgfsetdash{}{0pt}%
\pgfpathmoveto{\pgfqpoint{5.004054in}{1.786721in}}%
\pgfpathlineto{\pgfqpoint{5.012990in}{1.786721in}}%
\pgfpathlineto{\pgfqpoint{5.012990in}{1.713305in}}%
\pgfpathlineto{\pgfqpoint{5.004054in}{1.713305in}}%
\pgfpathlineto{\pgfqpoint{5.004054in}{1.786721in}}%
\pgfpathclose%
\pgfusepath{fill}%
\end{pgfscope}%
\begin{pgfscope}%
\pgfpathrectangle{\pgfqpoint{3.722897in}{0.857143in}}{\pgfqpoint{2.627103in}{1.813434in}}%
\pgfusepath{clip}%
\pgfsetbuttcap%
\pgfsetmiterjoin%
\definecolor{currentfill}{rgb}{0.511253,0.510898,0.193296}%
\pgfsetfillcolor{currentfill}%
\pgfsetlinewidth{0.000000pt}%
\definecolor{currentstroke}{rgb}{0.000000,0.000000,0.000000}%
\pgfsetstrokecolor{currentstroke}%
\pgfsetstrokeopacity{0.000000}%
\pgfsetdash{}{0pt}%
\pgfpathmoveto{\pgfqpoint{5.015224in}{1.787814in}}%
\pgfpathlineto{\pgfqpoint{5.024161in}{1.787814in}}%
\pgfpathlineto{\pgfqpoint{5.024161in}{1.708146in}}%
\pgfpathlineto{\pgfqpoint{5.015224in}{1.708146in}}%
\pgfpathlineto{\pgfqpoint{5.015224in}{1.787814in}}%
\pgfpathclose%
\pgfusepath{fill}%
\end{pgfscope}%
\begin{pgfscope}%
\pgfpathrectangle{\pgfqpoint{3.722897in}{0.857143in}}{\pgfqpoint{2.627103in}{1.813434in}}%
\pgfusepath{clip}%
\pgfsetbuttcap%
\pgfsetmiterjoin%
\definecolor{currentfill}{rgb}{0.511253,0.510898,0.193296}%
\pgfsetfillcolor{currentfill}%
\pgfsetlinewidth{0.000000pt}%
\definecolor{currentstroke}{rgb}{0.000000,0.000000,0.000000}%
\pgfsetstrokecolor{currentstroke}%
\pgfsetstrokeopacity{0.000000}%
\pgfsetdash{}{0pt}%
\pgfpathmoveto{\pgfqpoint{5.026395in}{1.786467in}}%
\pgfpathlineto{\pgfqpoint{5.035331in}{1.786467in}}%
\pgfpathlineto{\pgfqpoint{5.035331in}{1.697455in}}%
\pgfpathlineto{\pgfqpoint{5.026395in}{1.697455in}}%
\pgfpathlineto{\pgfqpoint{5.026395in}{1.786467in}}%
\pgfpathclose%
\pgfusepath{fill}%
\end{pgfscope}%
\begin{pgfscope}%
\pgfpathrectangle{\pgfqpoint{3.722897in}{0.857143in}}{\pgfqpoint{2.627103in}{1.813434in}}%
\pgfusepath{clip}%
\pgfsetbuttcap%
\pgfsetmiterjoin%
\definecolor{currentfill}{rgb}{0.511253,0.510898,0.193296}%
\pgfsetfillcolor{currentfill}%
\pgfsetlinewidth{0.000000pt}%
\definecolor{currentstroke}{rgb}{0.000000,0.000000,0.000000}%
\pgfsetstrokecolor{currentstroke}%
\pgfsetstrokeopacity{0.000000}%
\pgfsetdash{}{0pt}%
\pgfpathmoveto{\pgfqpoint{5.037566in}{1.785074in}}%
\pgfpathlineto{\pgfqpoint{5.046502in}{1.785074in}}%
\pgfpathlineto{\pgfqpoint{5.046502in}{1.683276in}}%
\pgfpathlineto{\pgfqpoint{5.037566in}{1.683276in}}%
\pgfpathlineto{\pgfqpoint{5.037566in}{1.785074in}}%
\pgfpathclose%
\pgfusepath{fill}%
\end{pgfscope}%
\begin{pgfscope}%
\pgfpathrectangle{\pgfqpoint{3.722897in}{0.857143in}}{\pgfqpoint{2.627103in}{1.813434in}}%
\pgfusepath{clip}%
\pgfsetbuttcap%
\pgfsetmiterjoin%
\definecolor{currentfill}{rgb}{0.511253,0.510898,0.193296}%
\pgfsetfillcolor{currentfill}%
\pgfsetlinewidth{0.000000pt}%
\definecolor{currentstroke}{rgb}{0.000000,0.000000,0.000000}%
\pgfsetstrokecolor{currentstroke}%
\pgfsetstrokeopacity{0.000000}%
\pgfsetdash{}{0pt}%
\pgfpathmoveto{\pgfqpoint{5.048736in}{1.784086in}}%
\pgfpathlineto{\pgfqpoint{5.057673in}{1.784086in}}%
\pgfpathlineto{\pgfqpoint{5.057673in}{1.667364in}}%
\pgfpathlineto{\pgfqpoint{5.048736in}{1.667364in}}%
\pgfpathlineto{\pgfqpoint{5.048736in}{1.784086in}}%
\pgfpathclose%
\pgfusepath{fill}%
\end{pgfscope}%
\begin{pgfscope}%
\pgfpathrectangle{\pgfqpoint{3.722897in}{0.857143in}}{\pgfqpoint{2.627103in}{1.813434in}}%
\pgfusepath{clip}%
\pgfsetbuttcap%
\pgfsetmiterjoin%
\definecolor{currentfill}{rgb}{0.511253,0.510898,0.193296}%
\pgfsetfillcolor{currentfill}%
\pgfsetlinewidth{0.000000pt}%
\definecolor{currentstroke}{rgb}{0.000000,0.000000,0.000000}%
\pgfsetstrokecolor{currentstroke}%
\pgfsetstrokeopacity{0.000000}%
\pgfsetdash{}{0pt}%
\pgfpathmoveto{\pgfqpoint{5.059907in}{1.783861in}}%
\pgfpathlineto{\pgfqpoint{5.068843in}{1.783861in}}%
\pgfpathlineto{\pgfqpoint{5.068843in}{1.651323in}}%
\pgfpathlineto{\pgfqpoint{5.059907in}{1.651323in}}%
\pgfpathlineto{\pgfqpoint{5.059907in}{1.783861in}}%
\pgfpathclose%
\pgfusepath{fill}%
\end{pgfscope}%
\begin{pgfscope}%
\pgfpathrectangle{\pgfqpoint{3.722897in}{0.857143in}}{\pgfqpoint{2.627103in}{1.813434in}}%
\pgfusepath{clip}%
\pgfsetbuttcap%
\pgfsetmiterjoin%
\definecolor{currentfill}{rgb}{0.511253,0.510898,0.193296}%
\pgfsetfillcolor{currentfill}%
\pgfsetlinewidth{0.000000pt}%
\definecolor{currentstroke}{rgb}{0.000000,0.000000,0.000000}%
\pgfsetstrokecolor{currentstroke}%
\pgfsetstrokeopacity{0.000000}%
\pgfsetdash{}{0pt}%
\pgfpathmoveto{\pgfqpoint{5.071077in}{1.783352in}}%
\pgfpathlineto{\pgfqpoint{5.080014in}{1.783352in}}%
\pgfpathlineto{\pgfqpoint{5.080014in}{1.642387in}}%
\pgfpathlineto{\pgfqpoint{5.071077in}{1.642387in}}%
\pgfpathlineto{\pgfqpoint{5.071077in}{1.783352in}}%
\pgfpathclose%
\pgfusepath{fill}%
\end{pgfscope}%
\begin{pgfscope}%
\pgfpathrectangle{\pgfqpoint{3.722897in}{0.857143in}}{\pgfqpoint{2.627103in}{1.813434in}}%
\pgfusepath{clip}%
\pgfsetbuttcap%
\pgfsetmiterjoin%
\definecolor{currentfill}{rgb}{0.511253,0.510898,0.193296}%
\pgfsetfillcolor{currentfill}%
\pgfsetlinewidth{0.000000pt}%
\definecolor{currentstroke}{rgb}{0.000000,0.000000,0.000000}%
\pgfsetstrokecolor{currentstroke}%
\pgfsetstrokeopacity{0.000000}%
\pgfsetdash{}{0pt}%
\pgfpathmoveto{\pgfqpoint{5.082248in}{1.781435in}}%
\pgfpathlineto{\pgfqpoint{5.091184in}{1.781435in}}%
\pgfpathlineto{\pgfqpoint{5.091184in}{1.635946in}}%
\pgfpathlineto{\pgfqpoint{5.082248in}{1.635946in}}%
\pgfpathlineto{\pgfqpoint{5.082248in}{1.781435in}}%
\pgfpathclose%
\pgfusepath{fill}%
\end{pgfscope}%
\begin{pgfscope}%
\pgfpathrectangle{\pgfqpoint{3.722897in}{0.857143in}}{\pgfqpoint{2.627103in}{1.813434in}}%
\pgfusepath{clip}%
\pgfsetbuttcap%
\pgfsetmiterjoin%
\definecolor{currentfill}{rgb}{0.511253,0.510898,0.193296}%
\pgfsetfillcolor{currentfill}%
\pgfsetlinewidth{0.000000pt}%
\definecolor{currentstroke}{rgb}{0.000000,0.000000,0.000000}%
\pgfsetstrokecolor{currentstroke}%
\pgfsetstrokeopacity{0.000000}%
\pgfsetdash{}{0pt}%
\pgfpathmoveto{\pgfqpoint{5.093419in}{1.778545in}}%
\pgfpathlineto{\pgfqpoint{5.102355in}{1.778545in}}%
\pgfpathlineto{\pgfqpoint{5.102355in}{1.630265in}}%
\pgfpathlineto{\pgfqpoint{5.093419in}{1.630265in}}%
\pgfpathlineto{\pgfqpoint{5.093419in}{1.778545in}}%
\pgfpathclose%
\pgfusepath{fill}%
\end{pgfscope}%
\begin{pgfscope}%
\pgfpathrectangle{\pgfqpoint{3.722897in}{0.857143in}}{\pgfqpoint{2.627103in}{1.813434in}}%
\pgfusepath{clip}%
\pgfsetbuttcap%
\pgfsetmiterjoin%
\definecolor{currentfill}{rgb}{0.511253,0.510898,0.193296}%
\pgfsetfillcolor{currentfill}%
\pgfsetlinewidth{0.000000pt}%
\definecolor{currentstroke}{rgb}{0.000000,0.000000,0.000000}%
\pgfsetstrokecolor{currentstroke}%
\pgfsetstrokeopacity{0.000000}%
\pgfsetdash{}{0pt}%
\pgfpathmoveto{\pgfqpoint{5.104589in}{1.778128in}}%
\pgfpathlineto{\pgfqpoint{5.113526in}{1.778128in}}%
\pgfpathlineto{\pgfqpoint{5.113526in}{1.627776in}}%
\pgfpathlineto{\pgfqpoint{5.104589in}{1.627776in}}%
\pgfpathlineto{\pgfqpoint{5.104589in}{1.778128in}}%
\pgfpathclose%
\pgfusepath{fill}%
\end{pgfscope}%
\begin{pgfscope}%
\pgfpathrectangle{\pgfqpoint{3.722897in}{0.857143in}}{\pgfqpoint{2.627103in}{1.813434in}}%
\pgfusepath{clip}%
\pgfsetbuttcap%
\pgfsetmiterjoin%
\definecolor{currentfill}{rgb}{0.511253,0.510898,0.193296}%
\pgfsetfillcolor{currentfill}%
\pgfsetlinewidth{0.000000pt}%
\definecolor{currentstroke}{rgb}{0.000000,0.000000,0.000000}%
\pgfsetstrokecolor{currentstroke}%
\pgfsetstrokeopacity{0.000000}%
\pgfsetdash{}{0pt}%
\pgfpathmoveto{\pgfqpoint{5.115760in}{1.775815in}}%
\pgfpathlineto{\pgfqpoint{5.124696in}{1.775815in}}%
\pgfpathlineto{\pgfqpoint{5.124696in}{1.623061in}}%
\pgfpathlineto{\pgfqpoint{5.115760in}{1.623061in}}%
\pgfpathlineto{\pgfqpoint{5.115760in}{1.775815in}}%
\pgfpathclose%
\pgfusepath{fill}%
\end{pgfscope}%
\begin{pgfscope}%
\pgfpathrectangle{\pgfqpoint{3.722897in}{0.857143in}}{\pgfqpoint{2.627103in}{1.813434in}}%
\pgfusepath{clip}%
\pgfsetbuttcap%
\pgfsetmiterjoin%
\definecolor{currentfill}{rgb}{0.511253,0.510898,0.193296}%
\pgfsetfillcolor{currentfill}%
\pgfsetlinewidth{0.000000pt}%
\definecolor{currentstroke}{rgb}{0.000000,0.000000,0.000000}%
\pgfsetstrokecolor{currentstroke}%
\pgfsetstrokeopacity{0.000000}%
\pgfsetdash{}{0pt}%
\pgfpathmoveto{\pgfqpoint{5.126930in}{1.774607in}}%
\pgfpathlineto{\pgfqpoint{5.135867in}{1.774607in}}%
\pgfpathlineto{\pgfqpoint{5.135867in}{1.621383in}}%
\pgfpathlineto{\pgfqpoint{5.126930in}{1.621383in}}%
\pgfpathlineto{\pgfqpoint{5.126930in}{1.774607in}}%
\pgfpathclose%
\pgfusepath{fill}%
\end{pgfscope}%
\begin{pgfscope}%
\pgfpathrectangle{\pgfqpoint{3.722897in}{0.857143in}}{\pgfqpoint{2.627103in}{1.813434in}}%
\pgfusepath{clip}%
\pgfsetbuttcap%
\pgfsetmiterjoin%
\definecolor{currentfill}{rgb}{0.511253,0.510898,0.193296}%
\pgfsetfillcolor{currentfill}%
\pgfsetlinewidth{0.000000pt}%
\definecolor{currentstroke}{rgb}{0.000000,0.000000,0.000000}%
\pgfsetstrokecolor{currentstroke}%
\pgfsetstrokeopacity{0.000000}%
\pgfsetdash{}{0pt}%
\pgfpathmoveto{\pgfqpoint{5.138101in}{1.771751in}}%
\pgfpathlineto{\pgfqpoint{5.147038in}{1.771751in}}%
\pgfpathlineto{\pgfqpoint{5.147038in}{1.617442in}}%
\pgfpathlineto{\pgfqpoint{5.138101in}{1.617442in}}%
\pgfpathlineto{\pgfqpoint{5.138101in}{1.771751in}}%
\pgfpathclose%
\pgfusepath{fill}%
\end{pgfscope}%
\begin{pgfscope}%
\pgfpathrectangle{\pgfqpoint{3.722897in}{0.857143in}}{\pgfqpoint{2.627103in}{1.813434in}}%
\pgfusepath{clip}%
\pgfsetbuttcap%
\pgfsetmiterjoin%
\definecolor{currentfill}{rgb}{0.511253,0.510898,0.193296}%
\pgfsetfillcolor{currentfill}%
\pgfsetlinewidth{0.000000pt}%
\definecolor{currentstroke}{rgb}{0.000000,0.000000,0.000000}%
\pgfsetstrokecolor{currentstroke}%
\pgfsetstrokeopacity{0.000000}%
\pgfsetdash{}{0pt}%
\pgfpathmoveto{\pgfqpoint{5.149272in}{1.770283in}}%
\pgfpathlineto{\pgfqpoint{5.158208in}{1.770283in}}%
\pgfpathlineto{\pgfqpoint{5.158208in}{1.615228in}}%
\pgfpathlineto{\pgfqpoint{5.149272in}{1.615228in}}%
\pgfpathlineto{\pgfqpoint{5.149272in}{1.770283in}}%
\pgfpathclose%
\pgfusepath{fill}%
\end{pgfscope}%
\begin{pgfscope}%
\pgfpathrectangle{\pgfqpoint{3.722897in}{0.857143in}}{\pgfqpoint{2.627103in}{1.813434in}}%
\pgfusepath{clip}%
\pgfsetbuttcap%
\pgfsetmiterjoin%
\definecolor{currentfill}{rgb}{0.511253,0.510898,0.193296}%
\pgfsetfillcolor{currentfill}%
\pgfsetlinewidth{0.000000pt}%
\definecolor{currentstroke}{rgb}{0.000000,0.000000,0.000000}%
\pgfsetstrokecolor{currentstroke}%
\pgfsetstrokeopacity{0.000000}%
\pgfsetdash{}{0pt}%
\pgfpathmoveto{\pgfqpoint{5.160442in}{1.769071in}}%
\pgfpathlineto{\pgfqpoint{5.169379in}{1.769071in}}%
\pgfpathlineto{\pgfqpoint{5.169379in}{1.615657in}}%
\pgfpathlineto{\pgfqpoint{5.160442in}{1.615657in}}%
\pgfpathlineto{\pgfqpoint{5.160442in}{1.769071in}}%
\pgfpathclose%
\pgfusepath{fill}%
\end{pgfscope}%
\begin{pgfscope}%
\pgfpathrectangle{\pgfqpoint{3.722897in}{0.857143in}}{\pgfqpoint{2.627103in}{1.813434in}}%
\pgfusepath{clip}%
\pgfsetbuttcap%
\pgfsetmiterjoin%
\definecolor{currentfill}{rgb}{0.511253,0.510898,0.193296}%
\pgfsetfillcolor{currentfill}%
\pgfsetlinewidth{0.000000pt}%
\definecolor{currentstroke}{rgb}{0.000000,0.000000,0.000000}%
\pgfsetstrokecolor{currentstroke}%
\pgfsetstrokeopacity{0.000000}%
\pgfsetdash{}{0pt}%
\pgfpathmoveto{\pgfqpoint{5.171613in}{1.768342in}}%
\pgfpathlineto{\pgfqpoint{5.180549in}{1.768342in}}%
\pgfpathlineto{\pgfqpoint{5.180549in}{1.614292in}}%
\pgfpathlineto{\pgfqpoint{5.171613in}{1.614292in}}%
\pgfpathlineto{\pgfqpoint{5.171613in}{1.768342in}}%
\pgfpathclose%
\pgfusepath{fill}%
\end{pgfscope}%
\begin{pgfscope}%
\pgfpathrectangle{\pgfqpoint{3.722897in}{0.857143in}}{\pgfqpoint{2.627103in}{1.813434in}}%
\pgfusepath{clip}%
\pgfsetbuttcap%
\pgfsetmiterjoin%
\definecolor{currentfill}{rgb}{0.511253,0.510898,0.193296}%
\pgfsetfillcolor{currentfill}%
\pgfsetlinewidth{0.000000pt}%
\definecolor{currentstroke}{rgb}{0.000000,0.000000,0.000000}%
\pgfsetstrokecolor{currentstroke}%
\pgfsetstrokeopacity{0.000000}%
\pgfsetdash{}{0pt}%
\pgfpathmoveto{\pgfqpoint{5.182783in}{1.766335in}}%
\pgfpathlineto{\pgfqpoint{5.191720in}{1.766335in}}%
\pgfpathlineto{\pgfqpoint{5.191720in}{1.612273in}}%
\pgfpathlineto{\pgfqpoint{5.182783in}{1.612273in}}%
\pgfpathlineto{\pgfqpoint{5.182783in}{1.766335in}}%
\pgfpathclose%
\pgfusepath{fill}%
\end{pgfscope}%
\begin{pgfscope}%
\pgfpathrectangle{\pgfqpoint{3.722897in}{0.857143in}}{\pgfqpoint{2.627103in}{1.813434in}}%
\pgfusepath{clip}%
\pgfsetbuttcap%
\pgfsetmiterjoin%
\definecolor{currentfill}{rgb}{0.511253,0.510898,0.193296}%
\pgfsetfillcolor{currentfill}%
\pgfsetlinewidth{0.000000pt}%
\definecolor{currentstroke}{rgb}{0.000000,0.000000,0.000000}%
\pgfsetstrokecolor{currentstroke}%
\pgfsetstrokeopacity{0.000000}%
\pgfsetdash{}{0pt}%
\pgfpathmoveto{\pgfqpoint{5.193954in}{1.763968in}}%
\pgfpathlineto{\pgfqpoint{5.202891in}{1.763968in}}%
\pgfpathlineto{\pgfqpoint{5.202891in}{1.614780in}}%
\pgfpathlineto{\pgfqpoint{5.193954in}{1.614780in}}%
\pgfpathlineto{\pgfqpoint{5.193954in}{1.763968in}}%
\pgfpathclose%
\pgfusepath{fill}%
\end{pgfscope}%
\begin{pgfscope}%
\pgfpathrectangle{\pgfqpoint{3.722897in}{0.857143in}}{\pgfqpoint{2.627103in}{1.813434in}}%
\pgfusepath{clip}%
\pgfsetbuttcap%
\pgfsetmiterjoin%
\definecolor{currentfill}{rgb}{0.511253,0.510898,0.193296}%
\pgfsetfillcolor{currentfill}%
\pgfsetlinewidth{0.000000pt}%
\definecolor{currentstroke}{rgb}{0.000000,0.000000,0.000000}%
\pgfsetstrokecolor{currentstroke}%
\pgfsetstrokeopacity{0.000000}%
\pgfsetdash{}{0pt}%
\pgfpathmoveto{\pgfqpoint{5.205125in}{1.761216in}}%
\pgfpathlineto{\pgfqpoint{5.214061in}{1.761216in}}%
\pgfpathlineto{\pgfqpoint{5.214061in}{1.617155in}}%
\pgfpathlineto{\pgfqpoint{5.205125in}{1.617155in}}%
\pgfpathlineto{\pgfqpoint{5.205125in}{1.761216in}}%
\pgfpathclose%
\pgfusepath{fill}%
\end{pgfscope}%
\begin{pgfscope}%
\pgfpathrectangle{\pgfqpoint{3.722897in}{0.857143in}}{\pgfqpoint{2.627103in}{1.813434in}}%
\pgfusepath{clip}%
\pgfsetbuttcap%
\pgfsetmiterjoin%
\definecolor{currentfill}{rgb}{0.511253,0.510898,0.193296}%
\pgfsetfillcolor{currentfill}%
\pgfsetlinewidth{0.000000pt}%
\definecolor{currentstroke}{rgb}{0.000000,0.000000,0.000000}%
\pgfsetstrokecolor{currentstroke}%
\pgfsetstrokeopacity{0.000000}%
\pgfsetdash{}{0pt}%
\pgfpathmoveto{\pgfqpoint{5.216295in}{1.761294in}}%
\pgfpathlineto{\pgfqpoint{5.225232in}{1.761294in}}%
\pgfpathlineto{\pgfqpoint{5.225232in}{1.626045in}}%
\pgfpathlineto{\pgfqpoint{5.216295in}{1.626045in}}%
\pgfpathlineto{\pgfqpoint{5.216295in}{1.761294in}}%
\pgfpathclose%
\pgfusepath{fill}%
\end{pgfscope}%
\begin{pgfscope}%
\pgfpathrectangle{\pgfqpoint{3.722897in}{0.857143in}}{\pgfqpoint{2.627103in}{1.813434in}}%
\pgfusepath{clip}%
\pgfsetbuttcap%
\pgfsetmiterjoin%
\definecolor{currentfill}{rgb}{0.511253,0.510898,0.193296}%
\pgfsetfillcolor{currentfill}%
\pgfsetlinewidth{0.000000pt}%
\definecolor{currentstroke}{rgb}{0.000000,0.000000,0.000000}%
\pgfsetstrokecolor{currentstroke}%
\pgfsetstrokeopacity{0.000000}%
\pgfsetdash{}{0pt}%
\pgfpathmoveto{\pgfqpoint{5.227466in}{1.760448in}}%
\pgfpathlineto{\pgfqpoint{5.236402in}{1.760448in}}%
\pgfpathlineto{\pgfqpoint{5.236402in}{1.642993in}}%
\pgfpathlineto{\pgfqpoint{5.227466in}{1.642993in}}%
\pgfpathlineto{\pgfqpoint{5.227466in}{1.760448in}}%
\pgfpathclose%
\pgfusepath{fill}%
\end{pgfscope}%
\begin{pgfscope}%
\pgfpathrectangle{\pgfqpoint{3.722897in}{0.857143in}}{\pgfqpoint{2.627103in}{1.813434in}}%
\pgfusepath{clip}%
\pgfsetbuttcap%
\pgfsetmiterjoin%
\definecolor{currentfill}{rgb}{0.511253,0.510898,0.193296}%
\pgfsetfillcolor{currentfill}%
\pgfsetlinewidth{0.000000pt}%
\definecolor{currentstroke}{rgb}{0.000000,0.000000,0.000000}%
\pgfsetstrokecolor{currentstroke}%
\pgfsetstrokeopacity{0.000000}%
\pgfsetdash{}{0pt}%
\pgfpathmoveto{\pgfqpoint{5.238636in}{1.759231in}}%
\pgfpathlineto{\pgfqpoint{5.247573in}{1.759231in}}%
\pgfpathlineto{\pgfqpoint{5.247573in}{1.658137in}}%
\pgfpathlineto{\pgfqpoint{5.238636in}{1.658137in}}%
\pgfpathlineto{\pgfqpoint{5.238636in}{1.759231in}}%
\pgfpathclose%
\pgfusepath{fill}%
\end{pgfscope}%
\begin{pgfscope}%
\pgfpathrectangle{\pgfqpoint{3.722897in}{0.857143in}}{\pgfqpoint{2.627103in}{1.813434in}}%
\pgfusepath{clip}%
\pgfsetbuttcap%
\pgfsetmiterjoin%
\definecolor{currentfill}{rgb}{0.511253,0.510898,0.193296}%
\pgfsetfillcolor{currentfill}%
\pgfsetlinewidth{0.000000pt}%
\definecolor{currentstroke}{rgb}{0.000000,0.000000,0.000000}%
\pgfsetstrokecolor{currentstroke}%
\pgfsetstrokeopacity{0.000000}%
\pgfsetdash{}{0pt}%
\pgfpathmoveto{\pgfqpoint{5.249807in}{1.758124in}}%
\pgfpathlineto{\pgfqpoint{5.258744in}{1.758124in}}%
\pgfpathlineto{\pgfqpoint{5.258744in}{1.671543in}}%
\pgfpathlineto{\pgfqpoint{5.249807in}{1.671543in}}%
\pgfpathlineto{\pgfqpoint{5.249807in}{1.758124in}}%
\pgfpathclose%
\pgfusepath{fill}%
\end{pgfscope}%
\begin{pgfscope}%
\pgfpathrectangle{\pgfqpoint{3.722897in}{0.857143in}}{\pgfqpoint{2.627103in}{1.813434in}}%
\pgfusepath{clip}%
\pgfsetbuttcap%
\pgfsetmiterjoin%
\definecolor{currentfill}{rgb}{0.511253,0.510898,0.193296}%
\pgfsetfillcolor{currentfill}%
\pgfsetlinewidth{0.000000pt}%
\definecolor{currentstroke}{rgb}{0.000000,0.000000,0.000000}%
\pgfsetstrokecolor{currentstroke}%
\pgfsetstrokeopacity{0.000000}%
\pgfsetdash{}{0pt}%
\pgfpathmoveto{\pgfqpoint{5.260978in}{1.757162in}}%
\pgfpathlineto{\pgfqpoint{5.269914in}{1.757162in}}%
\pgfpathlineto{\pgfqpoint{5.269914in}{1.684197in}}%
\pgfpathlineto{\pgfqpoint{5.260978in}{1.684197in}}%
\pgfpathlineto{\pgfqpoint{5.260978in}{1.757162in}}%
\pgfpathclose%
\pgfusepath{fill}%
\end{pgfscope}%
\begin{pgfscope}%
\pgfpathrectangle{\pgfqpoint{3.722897in}{0.857143in}}{\pgfqpoint{2.627103in}{1.813434in}}%
\pgfusepath{clip}%
\pgfsetbuttcap%
\pgfsetmiterjoin%
\definecolor{currentfill}{rgb}{0.511253,0.510898,0.193296}%
\pgfsetfillcolor{currentfill}%
\pgfsetlinewidth{0.000000pt}%
\definecolor{currentstroke}{rgb}{0.000000,0.000000,0.000000}%
\pgfsetstrokecolor{currentstroke}%
\pgfsetstrokeopacity{0.000000}%
\pgfsetdash{}{0pt}%
\pgfpathmoveto{\pgfqpoint{5.272148in}{1.757829in}}%
\pgfpathlineto{\pgfqpoint{5.281085in}{1.757829in}}%
\pgfpathlineto{\pgfqpoint{5.281085in}{1.693874in}}%
\pgfpathlineto{\pgfqpoint{5.272148in}{1.693874in}}%
\pgfpathlineto{\pgfqpoint{5.272148in}{1.757829in}}%
\pgfpathclose%
\pgfusepath{fill}%
\end{pgfscope}%
\begin{pgfscope}%
\pgfpathrectangle{\pgfqpoint{3.722897in}{0.857143in}}{\pgfqpoint{2.627103in}{1.813434in}}%
\pgfusepath{clip}%
\pgfsetbuttcap%
\pgfsetmiterjoin%
\definecolor{currentfill}{rgb}{0.511253,0.510898,0.193296}%
\pgfsetfillcolor{currentfill}%
\pgfsetlinewidth{0.000000pt}%
\definecolor{currentstroke}{rgb}{0.000000,0.000000,0.000000}%
\pgfsetstrokecolor{currentstroke}%
\pgfsetstrokeopacity{0.000000}%
\pgfsetdash{}{0pt}%
\pgfpathmoveto{\pgfqpoint{5.283319in}{1.757156in}}%
\pgfpathlineto{\pgfqpoint{5.292255in}{1.757156in}}%
\pgfpathlineto{\pgfqpoint{5.292255in}{1.702412in}}%
\pgfpathlineto{\pgfqpoint{5.283319in}{1.702412in}}%
\pgfpathlineto{\pgfqpoint{5.283319in}{1.757156in}}%
\pgfpathclose%
\pgfusepath{fill}%
\end{pgfscope}%
\begin{pgfscope}%
\pgfpathrectangle{\pgfqpoint{3.722897in}{0.857143in}}{\pgfqpoint{2.627103in}{1.813434in}}%
\pgfusepath{clip}%
\pgfsetbuttcap%
\pgfsetmiterjoin%
\definecolor{currentfill}{rgb}{0.511253,0.510898,0.193296}%
\pgfsetfillcolor{currentfill}%
\pgfsetlinewidth{0.000000pt}%
\definecolor{currentstroke}{rgb}{0.000000,0.000000,0.000000}%
\pgfsetstrokecolor{currentstroke}%
\pgfsetstrokeopacity{0.000000}%
\pgfsetdash{}{0pt}%
\pgfpathmoveto{\pgfqpoint{5.294489in}{1.758177in}}%
\pgfpathlineto{\pgfqpoint{5.303426in}{1.758177in}}%
\pgfpathlineto{\pgfqpoint{5.303426in}{1.713697in}}%
\pgfpathlineto{\pgfqpoint{5.294489in}{1.713697in}}%
\pgfpathlineto{\pgfqpoint{5.294489in}{1.758177in}}%
\pgfpathclose%
\pgfusepath{fill}%
\end{pgfscope}%
\begin{pgfscope}%
\pgfpathrectangle{\pgfqpoint{3.722897in}{0.857143in}}{\pgfqpoint{2.627103in}{1.813434in}}%
\pgfusepath{clip}%
\pgfsetbuttcap%
\pgfsetmiterjoin%
\definecolor{currentfill}{rgb}{0.511253,0.510898,0.193296}%
\pgfsetfillcolor{currentfill}%
\pgfsetlinewidth{0.000000pt}%
\definecolor{currentstroke}{rgb}{0.000000,0.000000,0.000000}%
\pgfsetstrokecolor{currentstroke}%
\pgfsetstrokeopacity{0.000000}%
\pgfsetdash{}{0pt}%
\pgfpathmoveto{\pgfqpoint{5.305660in}{1.757742in}}%
\pgfpathlineto{\pgfqpoint{5.314597in}{1.757742in}}%
\pgfpathlineto{\pgfqpoint{5.314597in}{1.719600in}}%
\pgfpathlineto{\pgfqpoint{5.305660in}{1.719600in}}%
\pgfpathlineto{\pgfqpoint{5.305660in}{1.757742in}}%
\pgfpathclose%
\pgfusepath{fill}%
\end{pgfscope}%
\begin{pgfscope}%
\pgfpathrectangle{\pgfqpoint{3.722897in}{0.857143in}}{\pgfqpoint{2.627103in}{1.813434in}}%
\pgfusepath{clip}%
\pgfsetbuttcap%
\pgfsetmiterjoin%
\definecolor{currentfill}{rgb}{0.511253,0.510898,0.193296}%
\pgfsetfillcolor{currentfill}%
\pgfsetlinewidth{0.000000pt}%
\definecolor{currentstroke}{rgb}{0.000000,0.000000,0.000000}%
\pgfsetstrokecolor{currentstroke}%
\pgfsetstrokeopacity{0.000000}%
\pgfsetdash{}{0pt}%
\pgfpathmoveto{\pgfqpoint{5.316831in}{1.758266in}}%
\pgfpathlineto{\pgfqpoint{5.325767in}{1.758266in}}%
\pgfpathlineto{\pgfqpoint{5.325767in}{1.721638in}}%
\pgfpathlineto{\pgfqpoint{5.316831in}{1.721638in}}%
\pgfpathlineto{\pgfqpoint{5.316831in}{1.758266in}}%
\pgfpathclose%
\pgfusepath{fill}%
\end{pgfscope}%
\begin{pgfscope}%
\pgfpathrectangle{\pgfqpoint{3.722897in}{0.857143in}}{\pgfqpoint{2.627103in}{1.813434in}}%
\pgfusepath{clip}%
\pgfsetbuttcap%
\pgfsetmiterjoin%
\definecolor{currentfill}{rgb}{0.511253,0.510898,0.193296}%
\pgfsetfillcolor{currentfill}%
\pgfsetlinewidth{0.000000pt}%
\definecolor{currentstroke}{rgb}{0.000000,0.000000,0.000000}%
\pgfsetstrokecolor{currentstroke}%
\pgfsetstrokeopacity{0.000000}%
\pgfsetdash{}{0pt}%
\pgfpathmoveto{\pgfqpoint{5.328001in}{1.760102in}}%
\pgfpathlineto{\pgfqpoint{5.336938in}{1.760102in}}%
\pgfpathlineto{\pgfqpoint{5.336938in}{1.729283in}}%
\pgfpathlineto{\pgfqpoint{5.328001in}{1.729283in}}%
\pgfpathlineto{\pgfqpoint{5.328001in}{1.760102in}}%
\pgfpathclose%
\pgfusepath{fill}%
\end{pgfscope}%
\begin{pgfscope}%
\pgfpathrectangle{\pgfqpoint{3.722897in}{0.857143in}}{\pgfqpoint{2.627103in}{1.813434in}}%
\pgfusepath{clip}%
\pgfsetbuttcap%
\pgfsetmiterjoin%
\definecolor{currentfill}{rgb}{0.511253,0.510898,0.193296}%
\pgfsetfillcolor{currentfill}%
\pgfsetlinewidth{0.000000pt}%
\definecolor{currentstroke}{rgb}{0.000000,0.000000,0.000000}%
\pgfsetstrokecolor{currentstroke}%
\pgfsetstrokeopacity{0.000000}%
\pgfsetdash{}{0pt}%
\pgfpathmoveto{\pgfqpoint{5.339172in}{1.759382in}}%
\pgfpathlineto{\pgfqpoint{5.348108in}{1.759382in}}%
\pgfpathlineto{\pgfqpoint{5.348108in}{1.731564in}}%
\pgfpathlineto{\pgfqpoint{5.339172in}{1.731564in}}%
\pgfpathlineto{\pgfqpoint{5.339172in}{1.759382in}}%
\pgfpathclose%
\pgfusepath{fill}%
\end{pgfscope}%
\begin{pgfscope}%
\pgfpathrectangle{\pgfqpoint{3.722897in}{0.857143in}}{\pgfqpoint{2.627103in}{1.813434in}}%
\pgfusepath{clip}%
\pgfsetbuttcap%
\pgfsetmiterjoin%
\definecolor{currentfill}{rgb}{0.511253,0.510898,0.193296}%
\pgfsetfillcolor{currentfill}%
\pgfsetlinewidth{0.000000pt}%
\definecolor{currentstroke}{rgb}{0.000000,0.000000,0.000000}%
\pgfsetstrokecolor{currentstroke}%
\pgfsetstrokeopacity{0.000000}%
\pgfsetdash{}{0pt}%
\pgfpathmoveto{\pgfqpoint{5.350343in}{1.762207in}}%
\pgfpathlineto{\pgfqpoint{5.359279in}{1.762207in}}%
\pgfpathlineto{\pgfqpoint{5.359279in}{1.731957in}}%
\pgfpathlineto{\pgfqpoint{5.350343in}{1.731957in}}%
\pgfpathlineto{\pgfqpoint{5.350343in}{1.762207in}}%
\pgfpathclose%
\pgfusepath{fill}%
\end{pgfscope}%
\begin{pgfscope}%
\pgfpathrectangle{\pgfqpoint{3.722897in}{0.857143in}}{\pgfqpoint{2.627103in}{1.813434in}}%
\pgfusepath{clip}%
\pgfsetbuttcap%
\pgfsetmiterjoin%
\definecolor{currentfill}{rgb}{0.511253,0.510898,0.193296}%
\pgfsetfillcolor{currentfill}%
\pgfsetlinewidth{0.000000pt}%
\definecolor{currentstroke}{rgb}{0.000000,0.000000,0.000000}%
\pgfsetstrokecolor{currentstroke}%
\pgfsetstrokeopacity{0.000000}%
\pgfsetdash{}{0pt}%
\pgfpathmoveto{\pgfqpoint{5.361513in}{1.762279in}}%
\pgfpathlineto{\pgfqpoint{5.370450in}{1.762279in}}%
\pgfpathlineto{\pgfqpoint{5.370450in}{1.730146in}}%
\pgfpathlineto{\pgfqpoint{5.361513in}{1.730146in}}%
\pgfpathlineto{\pgfqpoint{5.361513in}{1.762279in}}%
\pgfpathclose%
\pgfusepath{fill}%
\end{pgfscope}%
\begin{pgfscope}%
\pgfpathrectangle{\pgfqpoint{3.722897in}{0.857143in}}{\pgfqpoint{2.627103in}{1.813434in}}%
\pgfusepath{clip}%
\pgfsetbuttcap%
\pgfsetmiterjoin%
\definecolor{currentfill}{rgb}{0.511253,0.510898,0.193296}%
\pgfsetfillcolor{currentfill}%
\pgfsetlinewidth{0.000000pt}%
\definecolor{currentstroke}{rgb}{0.000000,0.000000,0.000000}%
\pgfsetstrokecolor{currentstroke}%
\pgfsetstrokeopacity{0.000000}%
\pgfsetdash{}{0pt}%
\pgfpathmoveto{\pgfqpoint{5.372684in}{1.764248in}}%
\pgfpathlineto{\pgfqpoint{5.381620in}{1.764248in}}%
\pgfpathlineto{\pgfqpoint{5.381620in}{1.722161in}}%
\pgfpathlineto{\pgfqpoint{5.372684in}{1.722161in}}%
\pgfpathlineto{\pgfqpoint{5.372684in}{1.764248in}}%
\pgfpathclose%
\pgfusepath{fill}%
\end{pgfscope}%
\begin{pgfscope}%
\pgfpathrectangle{\pgfqpoint{3.722897in}{0.857143in}}{\pgfqpoint{2.627103in}{1.813434in}}%
\pgfusepath{clip}%
\pgfsetbuttcap%
\pgfsetmiterjoin%
\definecolor{currentfill}{rgb}{0.511253,0.510898,0.193296}%
\pgfsetfillcolor{currentfill}%
\pgfsetlinewidth{0.000000pt}%
\definecolor{currentstroke}{rgb}{0.000000,0.000000,0.000000}%
\pgfsetstrokecolor{currentstroke}%
\pgfsetstrokeopacity{0.000000}%
\pgfsetdash{}{0pt}%
\pgfpathmoveto{\pgfqpoint{5.383854in}{1.763619in}}%
\pgfpathlineto{\pgfqpoint{5.392791in}{1.763619in}}%
\pgfpathlineto{\pgfqpoint{5.392791in}{1.701564in}}%
\pgfpathlineto{\pgfqpoint{5.383854in}{1.701564in}}%
\pgfpathlineto{\pgfqpoint{5.383854in}{1.763619in}}%
\pgfpathclose%
\pgfusepath{fill}%
\end{pgfscope}%
\begin{pgfscope}%
\pgfpathrectangle{\pgfqpoint{3.722897in}{0.857143in}}{\pgfqpoint{2.627103in}{1.813434in}}%
\pgfusepath{clip}%
\pgfsetbuttcap%
\pgfsetmiterjoin%
\definecolor{currentfill}{rgb}{0.511253,0.510898,0.193296}%
\pgfsetfillcolor{currentfill}%
\pgfsetlinewidth{0.000000pt}%
\definecolor{currentstroke}{rgb}{0.000000,0.000000,0.000000}%
\pgfsetstrokecolor{currentstroke}%
\pgfsetstrokeopacity{0.000000}%
\pgfsetdash{}{0pt}%
\pgfpathmoveto{\pgfqpoint{5.395025in}{1.766991in}}%
\pgfpathlineto{\pgfqpoint{5.403961in}{1.766991in}}%
\pgfpathlineto{\pgfqpoint{5.403961in}{1.683758in}}%
\pgfpathlineto{\pgfqpoint{5.395025in}{1.683758in}}%
\pgfpathlineto{\pgfqpoint{5.395025in}{1.766991in}}%
\pgfpathclose%
\pgfusepath{fill}%
\end{pgfscope}%
\begin{pgfscope}%
\pgfpathrectangle{\pgfqpoint{3.722897in}{0.857143in}}{\pgfqpoint{2.627103in}{1.813434in}}%
\pgfusepath{clip}%
\pgfsetbuttcap%
\pgfsetmiterjoin%
\definecolor{currentfill}{rgb}{0.511253,0.510898,0.193296}%
\pgfsetfillcolor{currentfill}%
\pgfsetlinewidth{0.000000pt}%
\definecolor{currentstroke}{rgb}{0.000000,0.000000,0.000000}%
\pgfsetstrokecolor{currentstroke}%
\pgfsetstrokeopacity{0.000000}%
\pgfsetdash{}{0pt}%
\pgfpathmoveto{\pgfqpoint{5.406196in}{1.768912in}}%
\pgfpathlineto{\pgfqpoint{5.415132in}{1.768912in}}%
\pgfpathlineto{\pgfqpoint{5.415132in}{1.668490in}}%
\pgfpathlineto{\pgfqpoint{5.406196in}{1.668490in}}%
\pgfpathlineto{\pgfqpoint{5.406196in}{1.768912in}}%
\pgfpathclose%
\pgfusepath{fill}%
\end{pgfscope}%
\begin{pgfscope}%
\pgfpathrectangle{\pgfqpoint{3.722897in}{0.857143in}}{\pgfqpoint{2.627103in}{1.813434in}}%
\pgfusepath{clip}%
\pgfsetbuttcap%
\pgfsetmiterjoin%
\definecolor{currentfill}{rgb}{0.511253,0.510898,0.193296}%
\pgfsetfillcolor{currentfill}%
\pgfsetlinewidth{0.000000pt}%
\definecolor{currentstroke}{rgb}{0.000000,0.000000,0.000000}%
\pgfsetstrokecolor{currentstroke}%
\pgfsetstrokeopacity{0.000000}%
\pgfsetdash{}{0pt}%
\pgfpathmoveto{\pgfqpoint{5.417366in}{1.772453in}}%
\pgfpathlineto{\pgfqpoint{5.426303in}{1.772453in}}%
\pgfpathlineto{\pgfqpoint{5.426303in}{1.640711in}}%
\pgfpathlineto{\pgfqpoint{5.417366in}{1.640711in}}%
\pgfpathlineto{\pgfqpoint{5.417366in}{1.772453in}}%
\pgfpathclose%
\pgfusepath{fill}%
\end{pgfscope}%
\begin{pgfscope}%
\pgfpathrectangle{\pgfqpoint{3.722897in}{0.857143in}}{\pgfqpoint{2.627103in}{1.813434in}}%
\pgfusepath{clip}%
\pgfsetbuttcap%
\pgfsetmiterjoin%
\definecolor{currentfill}{rgb}{0.511253,0.510898,0.193296}%
\pgfsetfillcolor{currentfill}%
\pgfsetlinewidth{0.000000pt}%
\definecolor{currentstroke}{rgb}{0.000000,0.000000,0.000000}%
\pgfsetstrokecolor{currentstroke}%
\pgfsetstrokeopacity{0.000000}%
\pgfsetdash{}{0pt}%
\pgfpathmoveto{\pgfqpoint{5.428537in}{1.776981in}}%
\pgfpathlineto{\pgfqpoint{5.437473in}{1.776981in}}%
\pgfpathlineto{\pgfqpoint{5.437473in}{1.614522in}}%
\pgfpathlineto{\pgfqpoint{5.428537in}{1.614522in}}%
\pgfpathlineto{\pgfqpoint{5.428537in}{1.776981in}}%
\pgfpathclose%
\pgfusepath{fill}%
\end{pgfscope}%
\begin{pgfscope}%
\pgfpathrectangle{\pgfqpoint{3.722897in}{0.857143in}}{\pgfqpoint{2.627103in}{1.813434in}}%
\pgfusepath{clip}%
\pgfsetbuttcap%
\pgfsetmiterjoin%
\definecolor{currentfill}{rgb}{0.511253,0.510898,0.193296}%
\pgfsetfillcolor{currentfill}%
\pgfsetlinewidth{0.000000pt}%
\definecolor{currentstroke}{rgb}{0.000000,0.000000,0.000000}%
\pgfsetstrokecolor{currentstroke}%
\pgfsetstrokeopacity{0.000000}%
\pgfsetdash{}{0pt}%
\pgfpathmoveto{\pgfqpoint{5.439707in}{1.779233in}}%
\pgfpathlineto{\pgfqpoint{5.448644in}{1.779233in}}%
\pgfpathlineto{\pgfqpoint{5.448644in}{1.596910in}}%
\pgfpathlineto{\pgfqpoint{5.439707in}{1.596910in}}%
\pgfpathlineto{\pgfqpoint{5.439707in}{1.779233in}}%
\pgfpathclose%
\pgfusepath{fill}%
\end{pgfscope}%
\begin{pgfscope}%
\pgfpathrectangle{\pgfqpoint{3.722897in}{0.857143in}}{\pgfqpoint{2.627103in}{1.813434in}}%
\pgfusepath{clip}%
\pgfsetbuttcap%
\pgfsetmiterjoin%
\definecolor{currentfill}{rgb}{0.511253,0.510898,0.193296}%
\pgfsetfillcolor{currentfill}%
\pgfsetlinewidth{0.000000pt}%
\definecolor{currentstroke}{rgb}{0.000000,0.000000,0.000000}%
\pgfsetstrokecolor{currentstroke}%
\pgfsetstrokeopacity{0.000000}%
\pgfsetdash{}{0pt}%
\pgfpathmoveto{\pgfqpoint{5.450878in}{1.782487in}}%
\pgfpathlineto{\pgfqpoint{5.459814in}{1.782487in}}%
\pgfpathlineto{\pgfqpoint{5.459814in}{1.575131in}}%
\pgfpathlineto{\pgfqpoint{5.450878in}{1.575131in}}%
\pgfpathlineto{\pgfqpoint{5.450878in}{1.782487in}}%
\pgfpathclose%
\pgfusepath{fill}%
\end{pgfscope}%
\begin{pgfscope}%
\pgfpathrectangle{\pgfqpoint{3.722897in}{0.857143in}}{\pgfqpoint{2.627103in}{1.813434in}}%
\pgfusepath{clip}%
\pgfsetbuttcap%
\pgfsetmiterjoin%
\definecolor{currentfill}{rgb}{0.511253,0.510898,0.193296}%
\pgfsetfillcolor{currentfill}%
\pgfsetlinewidth{0.000000pt}%
\definecolor{currentstroke}{rgb}{0.000000,0.000000,0.000000}%
\pgfsetstrokecolor{currentstroke}%
\pgfsetstrokeopacity{0.000000}%
\pgfsetdash{}{0pt}%
\pgfpathmoveto{\pgfqpoint{5.462049in}{1.784002in}}%
\pgfpathlineto{\pgfqpoint{5.470985in}{1.784002in}}%
\pgfpathlineto{\pgfqpoint{5.470985in}{1.553107in}}%
\pgfpathlineto{\pgfqpoint{5.462049in}{1.553107in}}%
\pgfpathlineto{\pgfqpoint{5.462049in}{1.784002in}}%
\pgfpathclose%
\pgfusepath{fill}%
\end{pgfscope}%
\begin{pgfscope}%
\pgfpathrectangle{\pgfqpoint{3.722897in}{0.857143in}}{\pgfqpoint{2.627103in}{1.813434in}}%
\pgfusepath{clip}%
\pgfsetbuttcap%
\pgfsetmiterjoin%
\definecolor{currentfill}{rgb}{0.511253,0.510898,0.193296}%
\pgfsetfillcolor{currentfill}%
\pgfsetlinewidth{0.000000pt}%
\definecolor{currentstroke}{rgb}{0.000000,0.000000,0.000000}%
\pgfsetstrokecolor{currentstroke}%
\pgfsetstrokeopacity{0.000000}%
\pgfsetdash{}{0pt}%
\pgfpathmoveto{\pgfqpoint{5.473219in}{1.787752in}}%
\pgfpathlineto{\pgfqpoint{5.482156in}{1.787752in}}%
\pgfpathlineto{\pgfqpoint{5.482156in}{1.533831in}}%
\pgfpathlineto{\pgfqpoint{5.473219in}{1.533831in}}%
\pgfpathlineto{\pgfqpoint{5.473219in}{1.787752in}}%
\pgfpathclose%
\pgfusepath{fill}%
\end{pgfscope}%
\begin{pgfscope}%
\pgfpathrectangle{\pgfqpoint{3.722897in}{0.857143in}}{\pgfqpoint{2.627103in}{1.813434in}}%
\pgfusepath{clip}%
\pgfsetbuttcap%
\pgfsetmiterjoin%
\definecolor{currentfill}{rgb}{0.511253,0.510898,0.193296}%
\pgfsetfillcolor{currentfill}%
\pgfsetlinewidth{0.000000pt}%
\definecolor{currentstroke}{rgb}{0.000000,0.000000,0.000000}%
\pgfsetstrokecolor{currentstroke}%
\pgfsetstrokeopacity{0.000000}%
\pgfsetdash{}{0pt}%
\pgfpathmoveto{\pgfqpoint{5.484390in}{1.791919in}}%
\pgfpathlineto{\pgfqpoint{5.493326in}{1.791919in}}%
\pgfpathlineto{\pgfqpoint{5.493326in}{1.513909in}}%
\pgfpathlineto{\pgfqpoint{5.484390in}{1.513909in}}%
\pgfpathlineto{\pgfqpoint{5.484390in}{1.791919in}}%
\pgfpathclose%
\pgfusepath{fill}%
\end{pgfscope}%
\begin{pgfscope}%
\pgfpathrectangle{\pgfqpoint{3.722897in}{0.857143in}}{\pgfqpoint{2.627103in}{1.813434in}}%
\pgfusepath{clip}%
\pgfsetbuttcap%
\pgfsetmiterjoin%
\definecolor{currentfill}{rgb}{0.511253,0.510898,0.193296}%
\pgfsetfillcolor{currentfill}%
\pgfsetlinewidth{0.000000pt}%
\definecolor{currentstroke}{rgb}{0.000000,0.000000,0.000000}%
\pgfsetstrokecolor{currentstroke}%
\pgfsetstrokeopacity{0.000000}%
\pgfsetdash{}{0pt}%
\pgfpathmoveto{\pgfqpoint{5.495560in}{1.797064in}}%
\pgfpathlineto{\pgfqpoint{5.504497in}{1.797064in}}%
\pgfpathlineto{\pgfqpoint{5.504497in}{1.500738in}}%
\pgfpathlineto{\pgfqpoint{5.495560in}{1.500738in}}%
\pgfpathlineto{\pgfqpoint{5.495560in}{1.797064in}}%
\pgfpathclose%
\pgfusepath{fill}%
\end{pgfscope}%
\begin{pgfscope}%
\pgfpathrectangle{\pgfqpoint{3.722897in}{0.857143in}}{\pgfqpoint{2.627103in}{1.813434in}}%
\pgfusepath{clip}%
\pgfsetbuttcap%
\pgfsetmiterjoin%
\definecolor{currentfill}{rgb}{0.511253,0.510898,0.193296}%
\pgfsetfillcolor{currentfill}%
\pgfsetlinewidth{0.000000pt}%
\definecolor{currentstroke}{rgb}{0.000000,0.000000,0.000000}%
\pgfsetstrokecolor{currentstroke}%
\pgfsetstrokeopacity{0.000000}%
\pgfsetdash{}{0pt}%
\pgfpathmoveto{\pgfqpoint{5.506731in}{1.800732in}}%
\pgfpathlineto{\pgfqpoint{5.515667in}{1.800732in}}%
\pgfpathlineto{\pgfqpoint{5.515667in}{1.495316in}}%
\pgfpathlineto{\pgfqpoint{5.506731in}{1.495316in}}%
\pgfpathlineto{\pgfqpoint{5.506731in}{1.800732in}}%
\pgfpathclose%
\pgfusepath{fill}%
\end{pgfscope}%
\begin{pgfscope}%
\pgfpathrectangle{\pgfqpoint{3.722897in}{0.857143in}}{\pgfqpoint{2.627103in}{1.813434in}}%
\pgfusepath{clip}%
\pgfsetbuttcap%
\pgfsetmiterjoin%
\definecolor{currentfill}{rgb}{0.511253,0.510898,0.193296}%
\pgfsetfillcolor{currentfill}%
\pgfsetlinewidth{0.000000pt}%
\definecolor{currentstroke}{rgb}{0.000000,0.000000,0.000000}%
\pgfsetstrokecolor{currentstroke}%
\pgfsetstrokeopacity{0.000000}%
\pgfsetdash{}{0pt}%
\pgfpathmoveto{\pgfqpoint{5.517902in}{1.803338in}}%
\pgfpathlineto{\pgfqpoint{5.526838in}{1.803338in}}%
\pgfpathlineto{\pgfqpoint{5.526838in}{1.490808in}}%
\pgfpathlineto{\pgfqpoint{5.517902in}{1.490808in}}%
\pgfpathlineto{\pgfqpoint{5.517902in}{1.803338in}}%
\pgfpathclose%
\pgfusepath{fill}%
\end{pgfscope}%
\begin{pgfscope}%
\pgfpathrectangle{\pgfqpoint{3.722897in}{0.857143in}}{\pgfqpoint{2.627103in}{1.813434in}}%
\pgfusepath{clip}%
\pgfsetbuttcap%
\pgfsetmiterjoin%
\definecolor{currentfill}{rgb}{0.511253,0.510898,0.193296}%
\pgfsetfillcolor{currentfill}%
\pgfsetlinewidth{0.000000pt}%
\definecolor{currentstroke}{rgb}{0.000000,0.000000,0.000000}%
\pgfsetstrokecolor{currentstroke}%
\pgfsetstrokeopacity{0.000000}%
\pgfsetdash{}{0pt}%
\pgfpathmoveto{\pgfqpoint{5.529072in}{1.804799in}}%
\pgfpathlineto{\pgfqpoint{5.538009in}{1.804799in}}%
\pgfpathlineto{\pgfqpoint{5.538009in}{1.484546in}}%
\pgfpathlineto{\pgfqpoint{5.529072in}{1.484546in}}%
\pgfpathlineto{\pgfqpoint{5.529072in}{1.804799in}}%
\pgfpathclose%
\pgfusepath{fill}%
\end{pgfscope}%
\begin{pgfscope}%
\pgfpathrectangle{\pgfqpoint{3.722897in}{0.857143in}}{\pgfqpoint{2.627103in}{1.813434in}}%
\pgfusepath{clip}%
\pgfsetbuttcap%
\pgfsetmiterjoin%
\definecolor{currentfill}{rgb}{0.511253,0.510898,0.193296}%
\pgfsetfillcolor{currentfill}%
\pgfsetlinewidth{0.000000pt}%
\definecolor{currentstroke}{rgb}{0.000000,0.000000,0.000000}%
\pgfsetstrokecolor{currentstroke}%
\pgfsetstrokeopacity{0.000000}%
\pgfsetdash{}{0pt}%
\pgfpathmoveto{\pgfqpoint{5.540243in}{1.809989in}}%
\pgfpathlineto{\pgfqpoint{5.549179in}{1.809989in}}%
\pgfpathlineto{\pgfqpoint{5.549179in}{1.486228in}}%
\pgfpathlineto{\pgfqpoint{5.540243in}{1.486228in}}%
\pgfpathlineto{\pgfqpoint{5.540243in}{1.809989in}}%
\pgfpathclose%
\pgfusepath{fill}%
\end{pgfscope}%
\begin{pgfscope}%
\pgfpathrectangle{\pgfqpoint{3.722897in}{0.857143in}}{\pgfqpoint{2.627103in}{1.813434in}}%
\pgfusepath{clip}%
\pgfsetbuttcap%
\pgfsetmiterjoin%
\definecolor{currentfill}{rgb}{0.511253,0.510898,0.193296}%
\pgfsetfillcolor{currentfill}%
\pgfsetlinewidth{0.000000pt}%
\definecolor{currentstroke}{rgb}{0.000000,0.000000,0.000000}%
\pgfsetstrokecolor{currentstroke}%
\pgfsetstrokeopacity{0.000000}%
\pgfsetdash{}{0pt}%
\pgfpathmoveto{\pgfqpoint{5.551413in}{1.812808in}}%
\pgfpathlineto{\pgfqpoint{5.560350in}{1.812808in}}%
\pgfpathlineto{\pgfqpoint{5.560350in}{1.486043in}}%
\pgfpathlineto{\pgfqpoint{5.551413in}{1.486043in}}%
\pgfpathlineto{\pgfqpoint{5.551413in}{1.812808in}}%
\pgfpathclose%
\pgfusepath{fill}%
\end{pgfscope}%
\begin{pgfscope}%
\pgfpathrectangle{\pgfqpoint{3.722897in}{0.857143in}}{\pgfqpoint{2.627103in}{1.813434in}}%
\pgfusepath{clip}%
\pgfsetbuttcap%
\pgfsetmiterjoin%
\definecolor{currentfill}{rgb}{0.511253,0.510898,0.193296}%
\pgfsetfillcolor{currentfill}%
\pgfsetlinewidth{0.000000pt}%
\definecolor{currentstroke}{rgb}{0.000000,0.000000,0.000000}%
\pgfsetstrokecolor{currentstroke}%
\pgfsetstrokeopacity{0.000000}%
\pgfsetdash{}{0pt}%
\pgfpathmoveto{\pgfqpoint{5.562584in}{1.808875in}}%
\pgfpathlineto{\pgfqpoint{5.571521in}{1.808875in}}%
\pgfpathlineto{\pgfqpoint{5.571521in}{1.483138in}}%
\pgfpathlineto{\pgfqpoint{5.562584in}{1.483138in}}%
\pgfpathlineto{\pgfqpoint{5.562584in}{1.808875in}}%
\pgfpathclose%
\pgfusepath{fill}%
\end{pgfscope}%
\begin{pgfscope}%
\pgfpathrectangle{\pgfqpoint{3.722897in}{0.857143in}}{\pgfqpoint{2.627103in}{1.813434in}}%
\pgfusepath{clip}%
\pgfsetbuttcap%
\pgfsetmiterjoin%
\definecolor{currentfill}{rgb}{0.511253,0.510898,0.193296}%
\pgfsetfillcolor{currentfill}%
\pgfsetlinewidth{0.000000pt}%
\definecolor{currentstroke}{rgb}{0.000000,0.000000,0.000000}%
\pgfsetstrokecolor{currentstroke}%
\pgfsetstrokeopacity{0.000000}%
\pgfsetdash{}{0pt}%
\pgfpathmoveto{\pgfqpoint{5.573755in}{1.813884in}}%
\pgfpathlineto{\pgfqpoint{5.582691in}{1.813884in}}%
\pgfpathlineto{\pgfqpoint{5.582691in}{1.491236in}}%
\pgfpathlineto{\pgfqpoint{5.573755in}{1.491236in}}%
\pgfpathlineto{\pgfqpoint{5.573755in}{1.813884in}}%
\pgfpathclose%
\pgfusepath{fill}%
\end{pgfscope}%
\begin{pgfscope}%
\pgfpathrectangle{\pgfqpoint{3.722897in}{0.857143in}}{\pgfqpoint{2.627103in}{1.813434in}}%
\pgfusepath{clip}%
\pgfsetbuttcap%
\pgfsetmiterjoin%
\definecolor{currentfill}{rgb}{0.511253,0.510898,0.193296}%
\pgfsetfillcolor{currentfill}%
\pgfsetlinewidth{0.000000pt}%
\definecolor{currentstroke}{rgb}{0.000000,0.000000,0.000000}%
\pgfsetstrokecolor{currentstroke}%
\pgfsetstrokeopacity{0.000000}%
\pgfsetdash{}{0pt}%
\pgfpathmoveto{\pgfqpoint{5.584925in}{1.811206in}}%
\pgfpathlineto{\pgfqpoint{5.593862in}{1.811206in}}%
\pgfpathlineto{\pgfqpoint{5.593862in}{1.483521in}}%
\pgfpathlineto{\pgfqpoint{5.584925in}{1.483521in}}%
\pgfpathlineto{\pgfqpoint{5.584925in}{1.811206in}}%
\pgfpathclose%
\pgfusepath{fill}%
\end{pgfscope}%
\begin{pgfscope}%
\pgfpathrectangle{\pgfqpoint{3.722897in}{0.857143in}}{\pgfqpoint{2.627103in}{1.813434in}}%
\pgfusepath{clip}%
\pgfsetbuttcap%
\pgfsetmiterjoin%
\definecolor{currentfill}{rgb}{0.511253,0.510898,0.193296}%
\pgfsetfillcolor{currentfill}%
\pgfsetlinewidth{0.000000pt}%
\definecolor{currentstroke}{rgb}{0.000000,0.000000,0.000000}%
\pgfsetstrokecolor{currentstroke}%
\pgfsetstrokeopacity{0.000000}%
\pgfsetdash{}{0pt}%
\pgfpathmoveto{\pgfqpoint{5.596096in}{1.813884in}}%
\pgfpathlineto{\pgfqpoint{5.605032in}{1.813884in}}%
\pgfpathlineto{\pgfqpoint{5.605032in}{1.482098in}}%
\pgfpathlineto{\pgfqpoint{5.596096in}{1.482098in}}%
\pgfpathlineto{\pgfqpoint{5.596096in}{1.813884in}}%
\pgfpathclose%
\pgfusepath{fill}%
\end{pgfscope}%
\begin{pgfscope}%
\pgfpathrectangle{\pgfqpoint{3.722897in}{0.857143in}}{\pgfqpoint{2.627103in}{1.813434in}}%
\pgfusepath{clip}%
\pgfsetbuttcap%
\pgfsetmiterjoin%
\definecolor{currentfill}{rgb}{0.511253,0.510898,0.193296}%
\pgfsetfillcolor{currentfill}%
\pgfsetlinewidth{0.000000pt}%
\definecolor{currentstroke}{rgb}{0.000000,0.000000,0.000000}%
\pgfsetstrokecolor{currentstroke}%
\pgfsetstrokeopacity{0.000000}%
\pgfsetdash{}{0pt}%
\pgfpathmoveto{\pgfqpoint{5.607266in}{1.813884in}}%
\pgfpathlineto{\pgfqpoint{5.616203in}{1.813884in}}%
\pgfpathlineto{\pgfqpoint{5.616203in}{1.485848in}}%
\pgfpathlineto{\pgfqpoint{5.607266in}{1.485848in}}%
\pgfpathlineto{\pgfqpoint{5.607266in}{1.813884in}}%
\pgfpathclose%
\pgfusepath{fill}%
\end{pgfscope}%
\begin{pgfscope}%
\pgfpathrectangle{\pgfqpoint{3.722897in}{0.857143in}}{\pgfqpoint{2.627103in}{1.813434in}}%
\pgfusepath{clip}%
\pgfsetbuttcap%
\pgfsetmiterjoin%
\definecolor{currentfill}{rgb}{0.511253,0.510898,0.193296}%
\pgfsetfillcolor{currentfill}%
\pgfsetlinewidth{0.000000pt}%
\definecolor{currentstroke}{rgb}{0.000000,0.000000,0.000000}%
\pgfsetstrokecolor{currentstroke}%
\pgfsetstrokeopacity{0.000000}%
\pgfsetdash{}{0pt}%
\pgfpathmoveto{\pgfqpoint{5.618437in}{1.813884in}}%
\pgfpathlineto{\pgfqpoint{5.627374in}{1.813884in}}%
\pgfpathlineto{\pgfqpoint{5.627374in}{1.494282in}}%
\pgfpathlineto{\pgfqpoint{5.618437in}{1.494282in}}%
\pgfpathlineto{\pgfqpoint{5.618437in}{1.813884in}}%
\pgfpathclose%
\pgfusepath{fill}%
\end{pgfscope}%
\begin{pgfscope}%
\pgfpathrectangle{\pgfqpoint{3.722897in}{0.857143in}}{\pgfqpoint{2.627103in}{1.813434in}}%
\pgfusepath{clip}%
\pgfsetbuttcap%
\pgfsetmiterjoin%
\definecolor{currentfill}{rgb}{0.511253,0.510898,0.193296}%
\pgfsetfillcolor{currentfill}%
\pgfsetlinewidth{0.000000pt}%
\definecolor{currentstroke}{rgb}{0.000000,0.000000,0.000000}%
\pgfsetstrokecolor{currentstroke}%
\pgfsetstrokeopacity{0.000000}%
\pgfsetdash{}{0pt}%
\pgfpathmoveto{\pgfqpoint{5.629608in}{1.813884in}}%
\pgfpathlineto{\pgfqpoint{5.638544in}{1.813884in}}%
\pgfpathlineto{\pgfqpoint{5.638544in}{1.497588in}}%
\pgfpathlineto{\pgfqpoint{5.629608in}{1.497588in}}%
\pgfpathlineto{\pgfqpoint{5.629608in}{1.813884in}}%
\pgfpathclose%
\pgfusepath{fill}%
\end{pgfscope}%
\begin{pgfscope}%
\pgfpathrectangle{\pgfqpoint{3.722897in}{0.857143in}}{\pgfqpoint{2.627103in}{1.813434in}}%
\pgfusepath{clip}%
\pgfsetbuttcap%
\pgfsetmiterjoin%
\definecolor{currentfill}{rgb}{0.511253,0.510898,0.193296}%
\pgfsetfillcolor{currentfill}%
\pgfsetlinewidth{0.000000pt}%
\definecolor{currentstroke}{rgb}{0.000000,0.000000,0.000000}%
\pgfsetstrokecolor{currentstroke}%
\pgfsetstrokeopacity{0.000000}%
\pgfsetdash{}{0pt}%
\pgfpathmoveto{\pgfqpoint{5.640778in}{1.813884in}}%
\pgfpathlineto{\pgfqpoint{5.649715in}{1.813884in}}%
\pgfpathlineto{\pgfqpoint{5.649715in}{1.498927in}}%
\pgfpathlineto{\pgfqpoint{5.640778in}{1.498927in}}%
\pgfpathlineto{\pgfqpoint{5.640778in}{1.813884in}}%
\pgfpathclose%
\pgfusepath{fill}%
\end{pgfscope}%
\begin{pgfscope}%
\pgfpathrectangle{\pgfqpoint{3.722897in}{0.857143in}}{\pgfqpoint{2.627103in}{1.813434in}}%
\pgfusepath{clip}%
\pgfsetbuttcap%
\pgfsetmiterjoin%
\definecolor{currentfill}{rgb}{0.511253,0.510898,0.193296}%
\pgfsetfillcolor{currentfill}%
\pgfsetlinewidth{0.000000pt}%
\definecolor{currentstroke}{rgb}{0.000000,0.000000,0.000000}%
\pgfsetstrokecolor{currentstroke}%
\pgfsetstrokeopacity{0.000000}%
\pgfsetdash{}{0pt}%
\pgfpathmoveto{\pgfqpoint{5.651949in}{1.813884in}}%
\pgfpathlineto{\pgfqpoint{5.660885in}{1.813884in}}%
\pgfpathlineto{\pgfqpoint{5.660885in}{1.505302in}}%
\pgfpathlineto{\pgfqpoint{5.651949in}{1.505302in}}%
\pgfpathlineto{\pgfqpoint{5.651949in}{1.813884in}}%
\pgfpathclose%
\pgfusepath{fill}%
\end{pgfscope}%
\begin{pgfscope}%
\pgfpathrectangle{\pgfqpoint{3.722897in}{0.857143in}}{\pgfqpoint{2.627103in}{1.813434in}}%
\pgfusepath{clip}%
\pgfsetbuttcap%
\pgfsetmiterjoin%
\definecolor{currentfill}{rgb}{0.511253,0.510898,0.193296}%
\pgfsetfillcolor{currentfill}%
\pgfsetlinewidth{0.000000pt}%
\definecolor{currentstroke}{rgb}{0.000000,0.000000,0.000000}%
\pgfsetstrokecolor{currentstroke}%
\pgfsetstrokeopacity{0.000000}%
\pgfsetdash{}{0pt}%
\pgfpathmoveto{\pgfqpoint{5.663119in}{1.813884in}}%
\pgfpathlineto{\pgfqpoint{5.672056in}{1.813884in}}%
\pgfpathlineto{\pgfqpoint{5.672056in}{1.511296in}}%
\pgfpathlineto{\pgfqpoint{5.663119in}{1.511296in}}%
\pgfpathlineto{\pgfqpoint{5.663119in}{1.813884in}}%
\pgfpathclose%
\pgfusepath{fill}%
\end{pgfscope}%
\begin{pgfscope}%
\pgfpathrectangle{\pgfqpoint{3.722897in}{0.857143in}}{\pgfqpoint{2.627103in}{1.813434in}}%
\pgfusepath{clip}%
\pgfsetbuttcap%
\pgfsetmiterjoin%
\definecolor{currentfill}{rgb}{0.511253,0.510898,0.193296}%
\pgfsetfillcolor{currentfill}%
\pgfsetlinewidth{0.000000pt}%
\definecolor{currentstroke}{rgb}{0.000000,0.000000,0.000000}%
\pgfsetstrokecolor{currentstroke}%
\pgfsetstrokeopacity{0.000000}%
\pgfsetdash{}{0pt}%
\pgfpathmoveto{\pgfqpoint{5.674290in}{1.813884in}}%
\pgfpathlineto{\pgfqpoint{5.683227in}{1.813884in}}%
\pgfpathlineto{\pgfqpoint{5.683227in}{1.519882in}}%
\pgfpathlineto{\pgfqpoint{5.674290in}{1.519882in}}%
\pgfpathlineto{\pgfqpoint{5.674290in}{1.813884in}}%
\pgfpathclose%
\pgfusepath{fill}%
\end{pgfscope}%
\begin{pgfscope}%
\pgfpathrectangle{\pgfqpoint{3.722897in}{0.857143in}}{\pgfqpoint{2.627103in}{1.813434in}}%
\pgfusepath{clip}%
\pgfsetbuttcap%
\pgfsetmiterjoin%
\definecolor{currentfill}{rgb}{0.511253,0.510898,0.193296}%
\pgfsetfillcolor{currentfill}%
\pgfsetlinewidth{0.000000pt}%
\definecolor{currentstroke}{rgb}{0.000000,0.000000,0.000000}%
\pgfsetstrokecolor{currentstroke}%
\pgfsetstrokeopacity{0.000000}%
\pgfsetdash{}{0pt}%
\pgfpathmoveto{\pgfqpoint{5.685461in}{1.813884in}}%
\pgfpathlineto{\pgfqpoint{5.694397in}{1.813884in}}%
\pgfpathlineto{\pgfqpoint{5.694397in}{1.527555in}}%
\pgfpathlineto{\pgfqpoint{5.685461in}{1.527555in}}%
\pgfpathlineto{\pgfqpoint{5.685461in}{1.813884in}}%
\pgfpathclose%
\pgfusepath{fill}%
\end{pgfscope}%
\begin{pgfscope}%
\pgfpathrectangle{\pgfqpoint{3.722897in}{0.857143in}}{\pgfqpoint{2.627103in}{1.813434in}}%
\pgfusepath{clip}%
\pgfsetbuttcap%
\pgfsetmiterjoin%
\definecolor{currentfill}{rgb}{0.511253,0.510898,0.193296}%
\pgfsetfillcolor{currentfill}%
\pgfsetlinewidth{0.000000pt}%
\definecolor{currentstroke}{rgb}{0.000000,0.000000,0.000000}%
\pgfsetstrokecolor{currentstroke}%
\pgfsetstrokeopacity{0.000000}%
\pgfsetdash{}{0pt}%
\pgfpathmoveto{\pgfqpoint{5.696631in}{1.813884in}}%
\pgfpathlineto{\pgfqpoint{5.705568in}{1.813884in}}%
\pgfpathlineto{\pgfqpoint{5.705568in}{1.526744in}}%
\pgfpathlineto{\pgfqpoint{5.696631in}{1.526744in}}%
\pgfpathlineto{\pgfqpoint{5.696631in}{1.813884in}}%
\pgfpathclose%
\pgfusepath{fill}%
\end{pgfscope}%
\begin{pgfscope}%
\pgfpathrectangle{\pgfqpoint{3.722897in}{0.857143in}}{\pgfqpoint{2.627103in}{1.813434in}}%
\pgfusepath{clip}%
\pgfsetbuttcap%
\pgfsetmiterjoin%
\definecolor{currentfill}{rgb}{0.511253,0.510898,0.193296}%
\pgfsetfillcolor{currentfill}%
\pgfsetlinewidth{0.000000pt}%
\definecolor{currentstroke}{rgb}{0.000000,0.000000,0.000000}%
\pgfsetstrokecolor{currentstroke}%
\pgfsetstrokeopacity{0.000000}%
\pgfsetdash{}{0pt}%
\pgfpathmoveto{\pgfqpoint{5.707802in}{1.813884in}}%
\pgfpathlineto{\pgfqpoint{5.716738in}{1.813884in}}%
\pgfpathlineto{\pgfqpoint{5.716738in}{1.518811in}}%
\pgfpathlineto{\pgfqpoint{5.707802in}{1.518811in}}%
\pgfpathlineto{\pgfqpoint{5.707802in}{1.813884in}}%
\pgfpathclose%
\pgfusepath{fill}%
\end{pgfscope}%
\begin{pgfscope}%
\pgfpathrectangle{\pgfqpoint{3.722897in}{0.857143in}}{\pgfqpoint{2.627103in}{1.813434in}}%
\pgfusepath{clip}%
\pgfsetbuttcap%
\pgfsetmiterjoin%
\definecolor{currentfill}{rgb}{0.511253,0.510898,0.193296}%
\pgfsetfillcolor{currentfill}%
\pgfsetlinewidth{0.000000pt}%
\definecolor{currentstroke}{rgb}{0.000000,0.000000,0.000000}%
\pgfsetstrokecolor{currentstroke}%
\pgfsetstrokeopacity{0.000000}%
\pgfsetdash{}{0pt}%
\pgfpathmoveto{\pgfqpoint{5.718972in}{1.813884in}}%
\pgfpathlineto{\pgfqpoint{5.727909in}{1.813884in}}%
\pgfpathlineto{\pgfqpoint{5.727909in}{1.509960in}}%
\pgfpathlineto{\pgfqpoint{5.718972in}{1.509960in}}%
\pgfpathlineto{\pgfqpoint{5.718972in}{1.813884in}}%
\pgfpathclose%
\pgfusepath{fill}%
\end{pgfscope}%
\begin{pgfscope}%
\pgfpathrectangle{\pgfqpoint{3.722897in}{0.857143in}}{\pgfqpoint{2.627103in}{1.813434in}}%
\pgfusepath{clip}%
\pgfsetbuttcap%
\pgfsetmiterjoin%
\definecolor{currentfill}{rgb}{0.511253,0.510898,0.193296}%
\pgfsetfillcolor{currentfill}%
\pgfsetlinewidth{0.000000pt}%
\definecolor{currentstroke}{rgb}{0.000000,0.000000,0.000000}%
\pgfsetstrokecolor{currentstroke}%
\pgfsetstrokeopacity{0.000000}%
\pgfsetdash{}{0pt}%
\pgfpathmoveto{\pgfqpoint{5.730143in}{1.813884in}}%
\pgfpathlineto{\pgfqpoint{5.739080in}{1.813884in}}%
\pgfpathlineto{\pgfqpoint{5.739080in}{1.492369in}}%
\pgfpathlineto{\pgfqpoint{5.730143in}{1.492369in}}%
\pgfpathlineto{\pgfqpoint{5.730143in}{1.813884in}}%
\pgfpathclose%
\pgfusepath{fill}%
\end{pgfscope}%
\begin{pgfscope}%
\pgfpathrectangle{\pgfqpoint{3.722897in}{0.857143in}}{\pgfqpoint{2.627103in}{1.813434in}}%
\pgfusepath{clip}%
\pgfsetbuttcap%
\pgfsetmiterjoin%
\definecolor{currentfill}{rgb}{0.511253,0.510898,0.193296}%
\pgfsetfillcolor{currentfill}%
\pgfsetlinewidth{0.000000pt}%
\definecolor{currentstroke}{rgb}{0.000000,0.000000,0.000000}%
\pgfsetstrokecolor{currentstroke}%
\pgfsetstrokeopacity{0.000000}%
\pgfsetdash{}{0pt}%
\pgfpathmoveto{\pgfqpoint{5.741314in}{1.813884in}}%
\pgfpathlineto{\pgfqpoint{5.750250in}{1.813884in}}%
\pgfpathlineto{\pgfqpoint{5.750250in}{1.476209in}}%
\pgfpathlineto{\pgfqpoint{5.741314in}{1.476209in}}%
\pgfpathlineto{\pgfqpoint{5.741314in}{1.813884in}}%
\pgfpathclose%
\pgfusepath{fill}%
\end{pgfscope}%
\begin{pgfscope}%
\pgfpathrectangle{\pgfqpoint{3.722897in}{0.857143in}}{\pgfqpoint{2.627103in}{1.813434in}}%
\pgfusepath{clip}%
\pgfsetbuttcap%
\pgfsetmiterjoin%
\definecolor{currentfill}{rgb}{0.511253,0.510898,0.193296}%
\pgfsetfillcolor{currentfill}%
\pgfsetlinewidth{0.000000pt}%
\definecolor{currentstroke}{rgb}{0.000000,0.000000,0.000000}%
\pgfsetstrokecolor{currentstroke}%
\pgfsetstrokeopacity{0.000000}%
\pgfsetdash{}{0pt}%
\pgfpathmoveto{\pgfqpoint{5.752484in}{1.813884in}}%
\pgfpathlineto{\pgfqpoint{5.761421in}{1.813884in}}%
\pgfpathlineto{\pgfqpoint{5.761421in}{1.469637in}}%
\pgfpathlineto{\pgfqpoint{5.752484in}{1.469637in}}%
\pgfpathlineto{\pgfqpoint{5.752484in}{1.813884in}}%
\pgfpathclose%
\pgfusepath{fill}%
\end{pgfscope}%
\begin{pgfscope}%
\pgfpathrectangle{\pgfqpoint{3.722897in}{0.857143in}}{\pgfqpoint{2.627103in}{1.813434in}}%
\pgfusepath{clip}%
\pgfsetbuttcap%
\pgfsetmiterjoin%
\definecolor{currentfill}{rgb}{0.511253,0.510898,0.193296}%
\pgfsetfillcolor{currentfill}%
\pgfsetlinewidth{0.000000pt}%
\definecolor{currentstroke}{rgb}{0.000000,0.000000,0.000000}%
\pgfsetstrokecolor{currentstroke}%
\pgfsetstrokeopacity{0.000000}%
\pgfsetdash{}{0pt}%
\pgfpathmoveto{\pgfqpoint{5.763655in}{1.813884in}}%
\pgfpathlineto{\pgfqpoint{5.772591in}{1.813884in}}%
\pgfpathlineto{\pgfqpoint{5.772591in}{1.464290in}}%
\pgfpathlineto{\pgfqpoint{5.763655in}{1.464290in}}%
\pgfpathlineto{\pgfqpoint{5.763655in}{1.813884in}}%
\pgfpathclose%
\pgfusepath{fill}%
\end{pgfscope}%
\begin{pgfscope}%
\pgfpathrectangle{\pgfqpoint{3.722897in}{0.857143in}}{\pgfqpoint{2.627103in}{1.813434in}}%
\pgfusepath{clip}%
\pgfsetbuttcap%
\pgfsetmiterjoin%
\definecolor{currentfill}{rgb}{0.511253,0.510898,0.193296}%
\pgfsetfillcolor{currentfill}%
\pgfsetlinewidth{0.000000pt}%
\definecolor{currentstroke}{rgb}{0.000000,0.000000,0.000000}%
\pgfsetstrokecolor{currentstroke}%
\pgfsetstrokeopacity{0.000000}%
\pgfsetdash{}{0pt}%
\pgfpathmoveto{\pgfqpoint{5.774826in}{1.813884in}}%
\pgfpathlineto{\pgfqpoint{5.783762in}{1.813884in}}%
\pgfpathlineto{\pgfqpoint{5.783762in}{1.457316in}}%
\pgfpathlineto{\pgfqpoint{5.774826in}{1.457316in}}%
\pgfpathlineto{\pgfqpoint{5.774826in}{1.813884in}}%
\pgfpathclose%
\pgfusepath{fill}%
\end{pgfscope}%
\begin{pgfscope}%
\pgfpathrectangle{\pgfqpoint{3.722897in}{0.857143in}}{\pgfqpoint{2.627103in}{1.813434in}}%
\pgfusepath{clip}%
\pgfsetbuttcap%
\pgfsetmiterjoin%
\definecolor{currentfill}{rgb}{0.511253,0.510898,0.193296}%
\pgfsetfillcolor{currentfill}%
\pgfsetlinewidth{0.000000pt}%
\definecolor{currentstroke}{rgb}{0.000000,0.000000,0.000000}%
\pgfsetstrokecolor{currentstroke}%
\pgfsetstrokeopacity{0.000000}%
\pgfsetdash{}{0pt}%
\pgfpathmoveto{\pgfqpoint{5.785996in}{1.813884in}}%
\pgfpathlineto{\pgfqpoint{5.794933in}{1.813884in}}%
\pgfpathlineto{\pgfqpoint{5.794933in}{1.455631in}}%
\pgfpathlineto{\pgfqpoint{5.785996in}{1.455631in}}%
\pgfpathlineto{\pgfqpoint{5.785996in}{1.813884in}}%
\pgfpathclose%
\pgfusepath{fill}%
\end{pgfscope}%
\begin{pgfscope}%
\pgfpathrectangle{\pgfqpoint{3.722897in}{0.857143in}}{\pgfqpoint{2.627103in}{1.813434in}}%
\pgfusepath{clip}%
\pgfsetbuttcap%
\pgfsetmiterjoin%
\definecolor{currentfill}{rgb}{0.511253,0.510898,0.193296}%
\pgfsetfillcolor{currentfill}%
\pgfsetlinewidth{0.000000pt}%
\definecolor{currentstroke}{rgb}{0.000000,0.000000,0.000000}%
\pgfsetstrokecolor{currentstroke}%
\pgfsetstrokeopacity{0.000000}%
\pgfsetdash{}{0pt}%
\pgfpathmoveto{\pgfqpoint{5.797167in}{1.813884in}}%
\pgfpathlineto{\pgfqpoint{5.806103in}{1.813884in}}%
\pgfpathlineto{\pgfqpoint{5.806103in}{1.456144in}}%
\pgfpathlineto{\pgfqpoint{5.797167in}{1.456144in}}%
\pgfpathlineto{\pgfqpoint{5.797167in}{1.813884in}}%
\pgfpathclose%
\pgfusepath{fill}%
\end{pgfscope}%
\begin{pgfscope}%
\pgfpathrectangle{\pgfqpoint{3.722897in}{0.857143in}}{\pgfqpoint{2.627103in}{1.813434in}}%
\pgfusepath{clip}%
\pgfsetbuttcap%
\pgfsetmiterjoin%
\definecolor{currentfill}{rgb}{0.511253,0.510898,0.193296}%
\pgfsetfillcolor{currentfill}%
\pgfsetlinewidth{0.000000pt}%
\definecolor{currentstroke}{rgb}{0.000000,0.000000,0.000000}%
\pgfsetstrokecolor{currentstroke}%
\pgfsetstrokeopacity{0.000000}%
\pgfsetdash{}{0pt}%
\pgfpathmoveto{\pgfqpoint{5.808337in}{1.813884in}}%
\pgfpathlineto{\pgfqpoint{5.817274in}{1.813884in}}%
\pgfpathlineto{\pgfqpoint{5.817274in}{1.460749in}}%
\pgfpathlineto{\pgfqpoint{5.808337in}{1.460749in}}%
\pgfpathlineto{\pgfqpoint{5.808337in}{1.813884in}}%
\pgfpathclose%
\pgfusepath{fill}%
\end{pgfscope}%
\begin{pgfscope}%
\pgfpathrectangle{\pgfqpoint{3.722897in}{0.857143in}}{\pgfqpoint{2.627103in}{1.813434in}}%
\pgfusepath{clip}%
\pgfsetbuttcap%
\pgfsetmiterjoin%
\definecolor{currentfill}{rgb}{0.511253,0.510898,0.193296}%
\pgfsetfillcolor{currentfill}%
\pgfsetlinewidth{0.000000pt}%
\definecolor{currentstroke}{rgb}{0.000000,0.000000,0.000000}%
\pgfsetstrokecolor{currentstroke}%
\pgfsetstrokeopacity{0.000000}%
\pgfsetdash{}{0pt}%
\pgfpathmoveto{\pgfqpoint{5.819508in}{1.813884in}}%
\pgfpathlineto{\pgfqpoint{5.828444in}{1.813884in}}%
\pgfpathlineto{\pgfqpoint{5.828444in}{1.468002in}}%
\pgfpathlineto{\pgfqpoint{5.819508in}{1.468002in}}%
\pgfpathlineto{\pgfqpoint{5.819508in}{1.813884in}}%
\pgfpathclose%
\pgfusepath{fill}%
\end{pgfscope}%
\begin{pgfscope}%
\pgfpathrectangle{\pgfqpoint{3.722897in}{0.857143in}}{\pgfqpoint{2.627103in}{1.813434in}}%
\pgfusepath{clip}%
\pgfsetbuttcap%
\pgfsetmiterjoin%
\definecolor{currentfill}{rgb}{0.511253,0.510898,0.193296}%
\pgfsetfillcolor{currentfill}%
\pgfsetlinewidth{0.000000pt}%
\definecolor{currentstroke}{rgb}{0.000000,0.000000,0.000000}%
\pgfsetstrokecolor{currentstroke}%
\pgfsetstrokeopacity{0.000000}%
\pgfsetdash{}{0pt}%
\pgfpathmoveto{\pgfqpoint{5.830679in}{1.798345in}}%
\pgfpathlineto{\pgfqpoint{5.839615in}{1.798345in}}%
\pgfpathlineto{\pgfqpoint{5.839615in}{1.454935in}}%
\pgfpathlineto{\pgfqpoint{5.830679in}{1.454935in}}%
\pgfpathlineto{\pgfqpoint{5.830679in}{1.798345in}}%
\pgfpathclose%
\pgfusepath{fill}%
\end{pgfscope}%
\begin{pgfscope}%
\pgfpathrectangle{\pgfqpoint{3.722897in}{0.857143in}}{\pgfqpoint{2.627103in}{1.813434in}}%
\pgfusepath{clip}%
\pgfsetbuttcap%
\pgfsetmiterjoin%
\definecolor{currentfill}{rgb}{0.511253,0.510898,0.193296}%
\pgfsetfillcolor{currentfill}%
\pgfsetlinewidth{0.000000pt}%
\definecolor{currentstroke}{rgb}{0.000000,0.000000,0.000000}%
\pgfsetstrokecolor{currentstroke}%
\pgfsetstrokeopacity{0.000000}%
\pgfsetdash{}{0pt}%
\pgfpathmoveto{\pgfqpoint{5.841849in}{1.767409in}}%
\pgfpathlineto{\pgfqpoint{5.850786in}{1.767409in}}%
\pgfpathlineto{\pgfqpoint{5.850786in}{1.419675in}}%
\pgfpathlineto{\pgfqpoint{5.841849in}{1.419675in}}%
\pgfpathlineto{\pgfqpoint{5.841849in}{1.767409in}}%
\pgfpathclose%
\pgfusepath{fill}%
\end{pgfscope}%
\begin{pgfscope}%
\pgfpathrectangle{\pgfqpoint{3.722897in}{0.857143in}}{\pgfqpoint{2.627103in}{1.813434in}}%
\pgfusepath{clip}%
\pgfsetbuttcap%
\pgfsetmiterjoin%
\definecolor{currentfill}{rgb}{0.511253,0.510898,0.193296}%
\pgfsetfillcolor{currentfill}%
\pgfsetlinewidth{0.000000pt}%
\definecolor{currentstroke}{rgb}{0.000000,0.000000,0.000000}%
\pgfsetstrokecolor{currentstroke}%
\pgfsetstrokeopacity{0.000000}%
\pgfsetdash{}{0pt}%
\pgfpathmoveto{\pgfqpoint{5.853020in}{1.749383in}}%
\pgfpathlineto{\pgfqpoint{5.861956in}{1.749383in}}%
\pgfpathlineto{\pgfqpoint{5.861956in}{1.397819in}}%
\pgfpathlineto{\pgfqpoint{5.853020in}{1.397819in}}%
\pgfpathlineto{\pgfqpoint{5.853020in}{1.749383in}}%
\pgfpathclose%
\pgfusepath{fill}%
\end{pgfscope}%
\begin{pgfscope}%
\pgfpathrectangle{\pgfqpoint{3.722897in}{0.857143in}}{\pgfqpoint{2.627103in}{1.813434in}}%
\pgfusepath{clip}%
\pgfsetbuttcap%
\pgfsetmiterjoin%
\definecolor{currentfill}{rgb}{0.511253,0.510898,0.193296}%
\pgfsetfillcolor{currentfill}%
\pgfsetlinewidth{0.000000pt}%
\definecolor{currentstroke}{rgb}{0.000000,0.000000,0.000000}%
\pgfsetstrokecolor{currentstroke}%
\pgfsetstrokeopacity{0.000000}%
\pgfsetdash{}{0pt}%
\pgfpathmoveto{\pgfqpoint{5.864190in}{1.736198in}}%
\pgfpathlineto{\pgfqpoint{5.873127in}{1.736198in}}%
\pgfpathlineto{\pgfqpoint{5.873127in}{1.383814in}}%
\pgfpathlineto{\pgfqpoint{5.864190in}{1.383814in}}%
\pgfpathlineto{\pgfqpoint{5.864190in}{1.736198in}}%
\pgfpathclose%
\pgfusepath{fill}%
\end{pgfscope}%
\begin{pgfscope}%
\pgfpathrectangle{\pgfqpoint{3.722897in}{0.857143in}}{\pgfqpoint{2.627103in}{1.813434in}}%
\pgfusepath{clip}%
\pgfsetbuttcap%
\pgfsetmiterjoin%
\definecolor{currentfill}{rgb}{0.511253,0.510898,0.193296}%
\pgfsetfillcolor{currentfill}%
\pgfsetlinewidth{0.000000pt}%
\definecolor{currentstroke}{rgb}{0.000000,0.000000,0.000000}%
\pgfsetstrokecolor{currentstroke}%
\pgfsetstrokeopacity{0.000000}%
\pgfsetdash{}{0pt}%
\pgfpathmoveto{\pgfqpoint{5.875361in}{1.700659in}}%
\pgfpathlineto{\pgfqpoint{5.884297in}{1.700659in}}%
\pgfpathlineto{\pgfqpoint{5.884297in}{1.356418in}}%
\pgfpathlineto{\pgfqpoint{5.875361in}{1.356418in}}%
\pgfpathlineto{\pgfqpoint{5.875361in}{1.700659in}}%
\pgfpathclose%
\pgfusepath{fill}%
\end{pgfscope}%
\begin{pgfscope}%
\pgfpathrectangle{\pgfqpoint{3.722897in}{0.857143in}}{\pgfqpoint{2.627103in}{1.813434in}}%
\pgfusepath{clip}%
\pgfsetbuttcap%
\pgfsetmiterjoin%
\definecolor{currentfill}{rgb}{0.511253,0.510898,0.193296}%
\pgfsetfillcolor{currentfill}%
\pgfsetlinewidth{0.000000pt}%
\definecolor{currentstroke}{rgb}{0.000000,0.000000,0.000000}%
\pgfsetstrokecolor{currentstroke}%
\pgfsetstrokeopacity{0.000000}%
\pgfsetdash{}{0pt}%
\pgfpathmoveto{\pgfqpoint{5.886532in}{1.692036in}}%
\pgfpathlineto{\pgfqpoint{5.895468in}{1.692036in}}%
\pgfpathlineto{\pgfqpoint{5.895468in}{1.359990in}}%
\pgfpathlineto{\pgfqpoint{5.886532in}{1.359990in}}%
\pgfpathlineto{\pgfqpoint{5.886532in}{1.692036in}}%
\pgfpathclose%
\pgfusepath{fill}%
\end{pgfscope}%
\begin{pgfscope}%
\pgfpathrectangle{\pgfqpoint{3.722897in}{0.857143in}}{\pgfqpoint{2.627103in}{1.813434in}}%
\pgfusepath{clip}%
\pgfsetbuttcap%
\pgfsetmiterjoin%
\definecolor{currentfill}{rgb}{0.511253,0.510898,0.193296}%
\pgfsetfillcolor{currentfill}%
\pgfsetlinewidth{0.000000pt}%
\definecolor{currentstroke}{rgb}{0.000000,0.000000,0.000000}%
\pgfsetstrokecolor{currentstroke}%
\pgfsetstrokeopacity{0.000000}%
\pgfsetdash{}{0pt}%
\pgfpathmoveto{\pgfqpoint{5.897702in}{1.667045in}}%
\pgfpathlineto{\pgfqpoint{5.906639in}{1.667045in}}%
\pgfpathlineto{\pgfqpoint{5.906639in}{1.349603in}}%
\pgfpathlineto{\pgfqpoint{5.897702in}{1.349603in}}%
\pgfpathlineto{\pgfqpoint{5.897702in}{1.667045in}}%
\pgfpathclose%
\pgfusepath{fill}%
\end{pgfscope}%
\begin{pgfscope}%
\pgfpathrectangle{\pgfqpoint{3.722897in}{0.857143in}}{\pgfqpoint{2.627103in}{1.813434in}}%
\pgfusepath{clip}%
\pgfsetbuttcap%
\pgfsetmiterjoin%
\definecolor{currentfill}{rgb}{0.511253,0.510898,0.193296}%
\pgfsetfillcolor{currentfill}%
\pgfsetlinewidth{0.000000pt}%
\definecolor{currentstroke}{rgb}{0.000000,0.000000,0.000000}%
\pgfsetstrokecolor{currentstroke}%
\pgfsetstrokeopacity{0.000000}%
\pgfsetdash{}{0pt}%
\pgfpathmoveto{\pgfqpoint{5.908873in}{1.649765in}}%
\pgfpathlineto{\pgfqpoint{5.917809in}{1.649765in}}%
\pgfpathlineto{\pgfqpoint{5.917809in}{1.350629in}}%
\pgfpathlineto{\pgfqpoint{5.908873in}{1.350629in}}%
\pgfpathlineto{\pgfqpoint{5.908873in}{1.649765in}}%
\pgfpathclose%
\pgfusepath{fill}%
\end{pgfscope}%
\begin{pgfscope}%
\pgfpathrectangle{\pgfqpoint{3.722897in}{0.857143in}}{\pgfqpoint{2.627103in}{1.813434in}}%
\pgfusepath{clip}%
\pgfsetbuttcap%
\pgfsetmiterjoin%
\definecolor{currentfill}{rgb}{0.511253,0.510898,0.193296}%
\pgfsetfillcolor{currentfill}%
\pgfsetlinewidth{0.000000pt}%
\definecolor{currentstroke}{rgb}{0.000000,0.000000,0.000000}%
\pgfsetstrokecolor{currentstroke}%
\pgfsetstrokeopacity{0.000000}%
\pgfsetdash{}{0pt}%
\pgfpathmoveto{\pgfqpoint{5.920043in}{1.643958in}}%
\pgfpathlineto{\pgfqpoint{5.928980in}{1.643958in}}%
\pgfpathlineto{\pgfqpoint{5.928980in}{1.361419in}}%
\pgfpathlineto{\pgfqpoint{5.920043in}{1.361419in}}%
\pgfpathlineto{\pgfqpoint{5.920043in}{1.643958in}}%
\pgfpathclose%
\pgfusepath{fill}%
\end{pgfscope}%
\begin{pgfscope}%
\pgfpathrectangle{\pgfqpoint{3.722897in}{0.857143in}}{\pgfqpoint{2.627103in}{1.813434in}}%
\pgfusepath{clip}%
\pgfsetbuttcap%
\pgfsetmiterjoin%
\definecolor{currentfill}{rgb}{0.511253,0.510898,0.193296}%
\pgfsetfillcolor{currentfill}%
\pgfsetlinewidth{0.000000pt}%
\definecolor{currentstroke}{rgb}{0.000000,0.000000,0.000000}%
\pgfsetstrokecolor{currentstroke}%
\pgfsetstrokeopacity{0.000000}%
\pgfsetdash{}{0pt}%
\pgfpathmoveto{\pgfqpoint{5.931214in}{1.634258in}}%
\pgfpathlineto{\pgfqpoint{5.940150in}{1.634258in}}%
\pgfpathlineto{\pgfqpoint{5.940150in}{1.372732in}}%
\pgfpathlineto{\pgfqpoint{5.931214in}{1.372732in}}%
\pgfpathlineto{\pgfqpoint{5.931214in}{1.634258in}}%
\pgfpathclose%
\pgfusepath{fill}%
\end{pgfscope}%
\begin{pgfscope}%
\pgfpathrectangle{\pgfqpoint{3.722897in}{0.857143in}}{\pgfqpoint{2.627103in}{1.813434in}}%
\pgfusepath{clip}%
\pgfsetbuttcap%
\pgfsetmiterjoin%
\definecolor{currentfill}{rgb}{0.511253,0.510898,0.193296}%
\pgfsetfillcolor{currentfill}%
\pgfsetlinewidth{0.000000pt}%
\definecolor{currentstroke}{rgb}{0.000000,0.000000,0.000000}%
\pgfsetstrokecolor{currentstroke}%
\pgfsetstrokeopacity{0.000000}%
\pgfsetdash{}{0pt}%
\pgfpathmoveto{\pgfqpoint{5.942385in}{1.606285in}}%
\pgfpathlineto{\pgfqpoint{5.951321in}{1.606285in}}%
\pgfpathlineto{\pgfqpoint{5.951321in}{1.370687in}}%
\pgfpathlineto{\pgfqpoint{5.942385in}{1.370687in}}%
\pgfpathlineto{\pgfqpoint{5.942385in}{1.606285in}}%
\pgfpathclose%
\pgfusepath{fill}%
\end{pgfscope}%
\begin{pgfscope}%
\pgfpathrectangle{\pgfqpoint{3.722897in}{0.857143in}}{\pgfqpoint{2.627103in}{1.813434in}}%
\pgfusepath{clip}%
\pgfsetbuttcap%
\pgfsetmiterjoin%
\definecolor{currentfill}{rgb}{0.511253,0.510898,0.193296}%
\pgfsetfillcolor{currentfill}%
\pgfsetlinewidth{0.000000pt}%
\definecolor{currentstroke}{rgb}{0.000000,0.000000,0.000000}%
\pgfsetstrokecolor{currentstroke}%
\pgfsetstrokeopacity{0.000000}%
\pgfsetdash{}{0pt}%
\pgfpathmoveto{\pgfqpoint{5.953555in}{1.585553in}}%
\pgfpathlineto{\pgfqpoint{5.962492in}{1.585553in}}%
\pgfpathlineto{\pgfqpoint{5.962492in}{1.381819in}}%
\pgfpathlineto{\pgfqpoint{5.953555in}{1.381819in}}%
\pgfpathlineto{\pgfqpoint{5.953555in}{1.585553in}}%
\pgfpathclose%
\pgfusepath{fill}%
\end{pgfscope}%
\begin{pgfscope}%
\pgfpathrectangle{\pgfqpoint{3.722897in}{0.857143in}}{\pgfqpoint{2.627103in}{1.813434in}}%
\pgfusepath{clip}%
\pgfsetbuttcap%
\pgfsetmiterjoin%
\definecolor{currentfill}{rgb}{0.511253,0.510898,0.193296}%
\pgfsetfillcolor{currentfill}%
\pgfsetlinewidth{0.000000pt}%
\definecolor{currentstroke}{rgb}{0.000000,0.000000,0.000000}%
\pgfsetstrokecolor{currentstroke}%
\pgfsetstrokeopacity{0.000000}%
\pgfsetdash{}{0pt}%
\pgfpathmoveto{\pgfqpoint{5.964726in}{1.580469in}}%
\pgfpathlineto{\pgfqpoint{5.973662in}{1.580469in}}%
\pgfpathlineto{\pgfqpoint{5.973662in}{1.407041in}}%
\pgfpathlineto{\pgfqpoint{5.964726in}{1.407041in}}%
\pgfpathlineto{\pgfqpoint{5.964726in}{1.580469in}}%
\pgfpathclose%
\pgfusepath{fill}%
\end{pgfscope}%
\begin{pgfscope}%
\pgfpathrectangle{\pgfqpoint{3.722897in}{0.857143in}}{\pgfqpoint{2.627103in}{1.813434in}}%
\pgfusepath{clip}%
\pgfsetbuttcap%
\pgfsetmiterjoin%
\definecolor{currentfill}{rgb}{0.511253,0.510898,0.193296}%
\pgfsetfillcolor{currentfill}%
\pgfsetlinewidth{0.000000pt}%
\definecolor{currentstroke}{rgb}{0.000000,0.000000,0.000000}%
\pgfsetstrokecolor{currentstroke}%
\pgfsetstrokeopacity{0.000000}%
\pgfsetdash{}{0pt}%
\pgfpathmoveto{\pgfqpoint{5.975896in}{1.560025in}}%
\pgfpathlineto{\pgfqpoint{5.984833in}{1.560025in}}%
\pgfpathlineto{\pgfqpoint{5.984833in}{1.412490in}}%
\pgfpathlineto{\pgfqpoint{5.975896in}{1.412490in}}%
\pgfpathlineto{\pgfqpoint{5.975896in}{1.560025in}}%
\pgfpathclose%
\pgfusepath{fill}%
\end{pgfscope}%
\begin{pgfscope}%
\pgfpathrectangle{\pgfqpoint{3.722897in}{0.857143in}}{\pgfqpoint{2.627103in}{1.813434in}}%
\pgfusepath{clip}%
\pgfsetbuttcap%
\pgfsetmiterjoin%
\definecolor{currentfill}{rgb}{0.511253,0.510898,0.193296}%
\pgfsetfillcolor{currentfill}%
\pgfsetlinewidth{0.000000pt}%
\definecolor{currentstroke}{rgb}{0.000000,0.000000,0.000000}%
\pgfsetstrokecolor{currentstroke}%
\pgfsetstrokeopacity{0.000000}%
\pgfsetdash{}{0pt}%
\pgfpathmoveto{\pgfqpoint{5.987067in}{1.549893in}}%
\pgfpathlineto{\pgfqpoint{5.996004in}{1.549893in}}%
\pgfpathlineto{\pgfqpoint{5.996004in}{1.432543in}}%
\pgfpathlineto{\pgfqpoint{5.987067in}{1.432543in}}%
\pgfpathlineto{\pgfqpoint{5.987067in}{1.549893in}}%
\pgfpathclose%
\pgfusepath{fill}%
\end{pgfscope}%
\begin{pgfscope}%
\pgfpathrectangle{\pgfqpoint{3.722897in}{0.857143in}}{\pgfqpoint{2.627103in}{1.813434in}}%
\pgfusepath{clip}%
\pgfsetbuttcap%
\pgfsetmiterjoin%
\definecolor{currentfill}{rgb}{0.511253,0.510898,0.193296}%
\pgfsetfillcolor{currentfill}%
\pgfsetlinewidth{0.000000pt}%
\definecolor{currentstroke}{rgb}{0.000000,0.000000,0.000000}%
\pgfsetstrokecolor{currentstroke}%
\pgfsetstrokeopacity{0.000000}%
\pgfsetdash{}{0pt}%
\pgfpathmoveto{\pgfqpoint{5.998238in}{1.547717in}}%
\pgfpathlineto{\pgfqpoint{6.007174in}{1.547717in}}%
\pgfpathlineto{\pgfqpoint{6.007174in}{1.459581in}}%
\pgfpathlineto{\pgfqpoint{5.998238in}{1.459581in}}%
\pgfpathlineto{\pgfqpoint{5.998238in}{1.547717in}}%
\pgfpathclose%
\pgfusepath{fill}%
\end{pgfscope}%
\begin{pgfscope}%
\pgfpathrectangle{\pgfqpoint{3.722897in}{0.857143in}}{\pgfqpoint{2.627103in}{1.813434in}}%
\pgfusepath{clip}%
\pgfsetbuttcap%
\pgfsetmiterjoin%
\definecolor{currentfill}{rgb}{0.511253,0.510898,0.193296}%
\pgfsetfillcolor{currentfill}%
\pgfsetlinewidth{0.000000pt}%
\definecolor{currentstroke}{rgb}{0.000000,0.000000,0.000000}%
\pgfsetstrokecolor{currentstroke}%
\pgfsetstrokeopacity{0.000000}%
\pgfsetdash{}{0pt}%
\pgfpathmoveto{\pgfqpoint{6.009408in}{1.534771in}}%
\pgfpathlineto{\pgfqpoint{6.018345in}{1.534771in}}%
\pgfpathlineto{\pgfqpoint{6.018345in}{1.473935in}}%
\pgfpathlineto{\pgfqpoint{6.009408in}{1.473935in}}%
\pgfpathlineto{\pgfqpoint{6.009408in}{1.534771in}}%
\pgfpathclose%
\pgfusepath{fill}%
\end{pgfscope}%
\begin{pgfscope}%
\pgfpathrectangle{\pgfqpoint{3.722897in}{0.857143in}}{\pgfqpoint{2.627103in}{1.813434in}}%
\pgfusepath{clip}%
\pgfsetbuttcap%
\pgfsetmiterjoin%
\definecolor{currentfill}{rgb}{0.511253,0.510898,0.193296}%
\pgfsetfillcolor{currentfill}%
\pgfsetlinewidth{0.000000pt}%
\definecolor{currentstroke}{rgb}{0.000000,0.000000,0.000000}%
\pgfsetstrokecolor{currentstroke}%
\pgfsetstrokeopacity{0.000000}%
\pgfsetdash{}{0pt}%
\pgfpathmoveto{\pgfqpoint{6.020579in}{1.510179in}}%
\pgfpathlineto{\pgfqpoint{6.029515in}{1.510179in}}%
\pgfpathlineto{\pgfqpoint{6.029515in}{1.483244in}}%
\pgfpathlineto{\pgfqpoint{6.020579in}{1.483244in}}%
\pgfpathlineto{\pgfqpoint{6.020579in}{1.510179in}}%
\pgfpathclose%
\pgfusepath{fill}%
\end{pgfscope}%
\begin{pgfscope}%
\pgfpathrectangle{\pgfqpoint{3.722897in}{0.857143in}}{\pgfqpoint{2.627103in}{1.813434in}}%
\pgfusepath{clip}%
\pgfsetbuttcap%
\pgfsetmiterjoin%
\definecolor{currentfill}{rgb}{0.511253,0.510898,0.193296}%
\pgfsetfillcolor{currentfill}%
\pgfsetlinewidth{0.000000pt}%
\definecolor{currentstroke}{rgb}{0.000000,0.000000,0.000000}%
\pgfsetstrokecolor{currentstroke}%
\pgfsetstrokeopacity{0.000000}%
\pgfsetdash{}{0pt}%
\pgfpathmoveto{\pgfqpoint{6.031749in}{1.846312in}}%
\pgfpathlineto{\pgfqpoint{6.040686in}{1.846312in}}%
\pgfpathlineto{\pgfqpoint{6.040686in}{1.849577in}}%
\pgfpathlineto{\pgfqpoint{6.031749in}{1.849577in}}%
\pgfpathlineto{\pgfqpoint{6.031749in}{1.846312in}}%
\pgfpathclose%
\pgfusepath{fill}%
\end{pgfscope}%
\begin{pgfscope}%
\pgfpathrectangle{\pgfqpoint{3.722897in}{0.857143in}}{\pgfqpoint{2.627103in}{1.813434in}}%
\pgfusepath{clip}%
\pgfsetbuttcap%
\pgfsetmiterjoin%
\definecolor{currentfill}{rgb}{0.511253,0.510898,0.193296}%
\pgfsetfillcolor{currentfill}%
\pgfsetlinewidth{0.000000pt}%
\definecolor{currentstroke}{rgb}{0.000000,0.000000,0.000000}%
\pgfsetstrokecolor{currentstroke}%
\pgfsetstrokeopacity{0.000000}%
\pgfsetdash{}{0pt}%
\pgfpathmoveto{\pgfqpoint{6.042920in}{1.853596in}}%
\pgfpathlineto{\pgfqpoint{6.051857in}{1.853596in}}%
\pgfpathlineto{\pgfqpoint{6.051857in}{1.881262in}}%
\pgfpathlineto{\pgfqpoint{6.042920in}{1.881262in}}%
\pgfpathlineto{\pgfqpoint{6.042920in}{1.853596in}}%
\pgfpathclose%
\pgfusepath{fill}%
\end{pgfscope}%
\begin{pgfscope}%
\pgfpathrectangle{\pgfqpoint{3.722897in}{0.857143in}}{\pgfqpoint{2.627103in}{1.813434in}}%
\pgfusepath{clip}%
\pgfsetbuttcap%
\pgfsetmiterjoin%
\definecolor{currentfill}{rgb}{0.511253,0.510898,0.193296}%
\pgfsetfillcolor{currentfill}%
\pgfsetlinewidth{0.000000pt}%
\definecolor{currentstroke}{rgb}{0.000000,0.000000,0.000000}%
\pgfsetstrokecolor{currentstroke}%
\pgfsetstrokeopacity{0.000000}%
\pgfsetdash{}{0pt}%
\pgfpathmoveto{\pgfqpoint{6.054091in}{1.856540in}}%
\pgfpathlineto{\pgfqpoint{6.063027in}{1.856540in}}%
\pgfpathlineto{\pgfqpoint{6.063027in}{1.906859in}}%
\pgfpathlineto{\pgfqpoint{6.054091in}{1.906859in}}%
\pgfpathlineto{\pgfqpoint{6.054091in}{1.856540in}}%
\pgfpathclose%
\pgfusepath{fill}%
\end{pgfscope}%
\begin{pgfscope}%
\pgfpathrectangle{\pgfqpoint{3.722897in}{0.857143in}}{\pgfqpoint{2.627103in}{1.813434in}}%
\pgfusepath{clip}%
\pgfsetbuttcap%
\pgfsetmiterjoin%
\definecolor{currentfill}{rgb}{0.511253,0.510898,0.193296}%
\pgfsetfillcolor{currentfill}%
\pgfsetlinewidth{0.000000pt}%
\definecolor{currentstroke}{rgb}{0.000000,0.000000,0.000000}%
\pgfsetstrokecolor{currentstroke}%
\pgfsetstrokeopacity{0.000000}%
\pgfsetdash{}{0pt}%
\pgfpathmoveto{\pgfqpoint{6.065261in}{1.837493in}}%
\pgfpathlineto{\pgfqpoint{6.074198in}{1.837493in}}%
\pgfpathlineto{\pgfqpoint{6.074198in}{1.908492in}}%
\pgfpathlineto{\pgfqpoint{6.065261in}{1.908492in}}%
\pgfpathlineto{\pgfqpoint{6.065261in}{1.837493in}}%
\pgfpathclose%
\pgfusepath{fill}%
\end{pgfscope}%
\begin{pgfscope}%
\pgfpathrectangle{\pgfqpoint{3.722897in}{0.857143in}}{\pgfqpoint{2.627103in}{1.813434in}}%
\pgfusepath{clip}%
\pgfsetbuttcap%
\pgfsetmiterjoin%
\definecolor{currentfill}{rgb}{0.511253,0.510898,0.193296}%
\pgfsetfillcolor{currentfill}%
\pgfsetlinewidth{0.000000pt}%
\definecolor{currentstroke}{rgb}{0.000000,0.000000,0.000000}%
\pgfsetstrokecolor{currentstroke}%
\pgfsetstrokeopacity{0.000000}%
\pgfsetdash{}{0pt}%
\pgfpathmoveto{\pgfqpoint{6.076432in}{1.836625in}}%
\pgfpathlineto{\pgfqpoint{6.085368in}{1.836625in}}%
\pgfpathlineto{\pgfqpoint{6.085368in}{1.926712in}}%
\pgfpathlineto{\pgfqpoint{6.076432in}{1.926712in}}%
\pgfpathlineto{\pgfqpoint{6.076432in}{1.836625in}}%
\pgfpathclose%
\pgfusepath{fill}%
\end{pgfscope}%
\begin{pgfscope}%
\pgfpathrectangle{\pgfqpoint{3.722897in}{0.857143in}}{\pgfqpoint{2.627103in}{1.813434in}}%
\pgfusepath{clip}%
\pgfsetbuttcap%
\pgfsetmiterjoin%
\definecolor{currentfill}{rgb}{0.511253,0.510898,0.193296}%
\pgfsetfillcolor{currentfill}%
\pgfsetlinewidth{0.000000pt}%
\definecolor{currentstroke}{rgb}{0.000000,0.000000,0.000000}%
\pgfsetstrokecolor{currentstroke}%
\pgfsetstrokeopacity{0.000000}%
\pgfsetdash{}{0pt}%
\pgfpathmoveto{\pgfqpoint{6.087602in}{1.832889in}}%
\pgfpathlineto{\pgfqpoint{6.096539in}{1.832889in}}%
\pgfpathlineto{\pgfqpoint{6.096539in}{1.941574in}}%
\pgfpathlineto{\pgfqpoint{6.087602in}{1.941574in}}%
\pgfpathlineto{\pgfqpoint{6.087602in}{1.832889in}}%
\pgfpathclose%
\pgfusepath{fill}%
\end{pgfscope}%
\begin{pgfscope}%
\pgfpathrectangle{\pgfqpoint{3.722897in}{0.857143in}}{\pgfqpoint{2.627103in}{1.813434in}}%
\pgfusepath{clip}%
\pgfsetbuttcap%
\pgfsetmiterjoin%
\definecolor{currentfill}{rgb}{0.511253,0.510898,0.193296}%
\pgfsetfillcolor{currentfill}%
\pgfsetlinewidth{0.000000pt}%
\definecolor{currentstroke}{rgb}{0.000000,0.000000,0.000000}%
\pgfsetstrokecolor{currentstroke}%
\pgfsetstrokeopacity{0.000000}%
\pgfsetdash{}{0pt}%
\pgfpathmoveto{\pgfqpoint{6.098773in}{1.830205in}}%
\pgfpathlineto{\pgfqpoint{6.107710in}{1.830205in}}%
\pgfpathlineto{\pgfqpoint{6.107710in}{1.956161in}}%
\pgfpathlineto{\pgfqpoint{6.098773in}{1.956161in}}%
\pgfpathlineto{\pgfqpoint{6.098773in}{1.830205in}}%
\pgfpathclose%
\pgfusepath{fill}%
\end{pgfscope}%
\begin{pgfscope}%
\pgfpathrectangle{\pgfqpoint{3.722897in}{0.857143in}}{\pgfqpoint{2.627103in}{1.813434in}}%
\pgfusepath{clip}%
\pgfsetbuttcap%
\pgfsetmiterjoin%
\definecolor{currentfill}{rgb}{0.511253,0.510898,0.193296}%
\pgfsetfillcolor{currentfill}%
\pgfsetlinewidth{0.000000pt}%
\definecolor{currentstroke}{rgb}{0.000000,0.000000,0.000000}%
\pgfsetstrokecolor{currentstroke}%
\pgfsetstrokeopacity{0.000000}%
\pgfsetdash{}{0pt}%
\pgfpathmoveto{\pgfqpoint{6.109944in}{1.827048in}}%
\pgfpathlineto{\pgfqpoint{6.118880in}{1.827048in}}%
\pgfpathlineto{\pgfqpoint{6.118880in}{1.966188in}}%
\pgfpathlineto{\pgfqpoint{6.109944in}{1.966188in}}%
\pgfpathlineto{\pgfqpoint{6.109944in}{1.827048in}}%
\pgfpathclose%
\pgfusepath{fill}%
\end{pgfscope}%
\begin{pgfscope}%
\pgfpathrectangle{\pgfqpoint{3.722897in}{0.857143in}}{\pgfqpoint{2.627103in}{1.813434in}}%
\pgfusepath{clip}%
\pgfsetbuttcap%
\pgfsetmiterjoin%
\definecolor{currentfill}{rgb}{0.511253,0.510898,0.193296}%
\pgfsetfillcolor{currentfill}%
\pgfsetlinewidth{0.000000pt}%
\definecolor{currentstroke}{rgb}{0.000000,0.000000,0.000000}%
\pgfsetstrokecolor{currentstroke}%
\pgfsetstrokeopacity{0.000000}%
\pgfsetdash{}{0pt}%
\pgfpathmoveto{\pgfqpoint{6.121114in}{1.825155in}}%
\pgfpathlineto{\pgfqpoint{6.130051in}{1.825155in}}%
\pgfpathlineto{\pgfqpoint{6.130051in}{1.981146in}}%
\pgfpathlineto{\pgfqpoint{6.121114in}{1.981146in}}%
\pgfpathlineto{\pgfqpoint{6.121114in}{1.825155in}}%
\pgfpathclose%
\pgfusepath{fill}%
\end{pgfscope}%
\begin{pgfscope}%
\pgfpathrectangle{\pgfqpoint{3.722897in}{0.857143in}}{\pgfqpoint{2.627103in}{1.813434in}}%
\pgfusepath{clip}%
\pgfsetbuttcap%
\pgfsetmiterjoin%
\definecolor{currentfill}{rgb}{0.511253,0.510898,0.193296}%
\pgfsetfillcolor{currentfill}%
\pgfsetlinewidth{0.000000pt}%
\definecolor{currentstroke}{rgb}{0.000000,0.000000,0.000000}%
\pgfsetstrokecolor{currentstroke}%
\pgfsetstrokeopacity{0.000000}%
\pgfsetdash{}{0pt}%
\pgfpathmoveto{\pgfqpoint{6.132285in}{1.823233in}}%
\pgfpathlineto{\pgfqpoint{6.141221in}{1.823233in}}%
\pgfpathlineto{\pgfqpoint{6.141221in}{2.003142in}}%
\pgfpathlineto{\pgfqpoint{6.132285in}{2.003142in}}%
\pgfpathlineto{\pgfqpoint{6.132285in}{1.823233in}}%
\pgfpathclose%
\pgfusepath{fill}%
\end{pgfscope}%
\begin{pgfscope}%
\pgfpathrectangle{\pgfqpoint{3.722897in}{0.857143in}}{\pgfqpoint{2.627103in}{1.813434in}}%
\pgfusepath{clip}%
\pgfsetbuttcap%
\pgfsetmiterjoin%
\definecolor{currentfill}{rgb}{0.511253,0.510898,0.193296}%
\pgfsetfillcolor{currentfill}%
\pgfsetlinewidth{0.000000pt}%
\definecolor{currentstroke}{rgb}{0.000000,0.000000,0.000000}%
\pgfsetstrokecolor{currentstroke}%
\pgfsetstrokeopacity{0.000000}%
\pgfsetdash{}{0pt}%
\pgfpathmoveto{\pgfqpoint{6.143456in}{1.820974in}}%
\pgfpathlineto{\pgfqpoint{6.152392in}{1.820974in}}%
\pgfpathlineto{\pgfqpoint{6.152392in}{2.025335in}}%
\pgfpathlineto{\pgfqpoint{6.143456in}{2.025335in}}%
\pgfpathlineto{\pgfqpoint{6.143456in}{1.820974in}}%
\pgfpathclose%
\pgfusepath{fill}%
\end{pgfscope}%
\begin{pgfscope}%
\pgfpathrectangle{\pgfqpoint{3.722897in}{0.857143in}}{\pgfqpoint{2.627103in}{1.813434in}}%
\pgfusepath{clip}%
\pgfsetbuttcap%
\pgfsetmiterjoin%
\definecolor{currentfill}{rgb}{0.511253,0.510898,0.193296}%
\pgfsetfillcolor{currentfill}%
\pgfsetlinewidth{0.000000pt}%
\definecolor{currentstroke}{rgb}{0.000000,0.000000,0.000000}%
\pgfsetstrokecolor{currentstroke}%
\pgfsetstrokeopacity{0.000000}%
\pgfsetdash{}{0pt}%
\pgfpathmoveto{\pgfqpoint{6.154626in}{1.818208in}}%
\pgfpathlineto{\pgfqpoint{6.163563in}{1.818208in}}%
\pgfpathlineto{\pgfqpoint{6.163563in}{2.044678in}}%
\pgfpathlineto{\pgfqpoint{6.154626in}{2.044678in}}%
\pgfpathlineto{\pgfqpoint{6.154626in}{1.818208in}}%
\pgfpathclose%
\pgfusepath{fill}%
\end{pgfscope}%
\begin{pgfscope}%
\pgfpathrectangle{\pgfqpoint{3.722897in}{0.857143in}}{\pgfqpoint{2.627103in}{1.813434in}}%
\pgfusepath{clip}%
\pgfsetbuttcap%
\pgfsetmiterjoin%
\definecolor{currentfill}{rgb}{0.511253,0.510898,0.193296}%
\pgfsetfillcolor{currentfill}%
\pgfsetlinewidth{0.000000pt}%
\definecolor{currentstroke}{rgb}{0.000000,0.000000,0.000000}%
\pgfsetstrokecolor{currentstroke}%
\pgfsetstrokeopacity{0.000000}%
\pgfsetdash{}{0pt}%
\pgfpathmoveto{\pgfqpoint{6.165797in}{1.815763in}}%
\pgfpathlineto{\pgfqpoint{6.174733in}{1.815763in}}%
\pgfpathlineto{\pgfqpoint{6.174733in}{2.066777in}}%
\pgfpathlineto{\pgfqpoint{6.165797in}{2.066777in}}%
\pgfpathlineto{\pgfqpoint{6.165797in}{1.815763in}}%
\pgfpathclose%
\pgfusepath{fill}%
\end{pgfscope}%
\begin{pgfscope}%
\pgfpathrectangle{\pgfqpoint{3.722897in}{0.857143in}}{\pgfqpoint{2.627103in}{1.813434in}}%
\pgfusepath{clip}%
\pgfsetbuttcap%
\pgfsetmiterjoin%
\definecolor{currentfill}{rgb}{0.511253,0.510898,0.193296}%
\pgfsetfillcolor{currentfill}%
\pgfsetlinewidth{0.000000pt}%
\definecolor{currentstroke}{rgb}{0.000000,0.000000,0.000000}%
\pgfsetstrokecolor{currentstroke}%
\pgfsetstrokeopacity{0.000000}%
\pgfsetdash{}{0pt}%
\pgfpathmoveto{\pgfqpoint{6.176967in}{1.813884in}}%
\pgfpathlineto{\pgfqpoint{6.185904in}{1.813884in}}%
\pgfpathlineto{\pgfqpoint{6.185904in}{2.088814in}}%
\pgfpathlineto{\pgfqpoint{6.176967in}{2.088814in}}%
\pgfpathlineto{\pgfqpoint{6.176967in}{1.813884in}}%
\pgfpathclose%
\pgfusepath{fill}%
\end{pgfscope}%
\begin{pgfscope}%
\pgfpathrectangle{\pgfqpoint{3.722897in}{0.857143in}}{\pgfqpoint{2.627103in}{1.813434in}}%
\pgfusepath{clip}%
\pgfsetbuttcap%
\pgfsetmiterjoin%
\definecolor{currentfill}{rgb}{0.511253,0.510898,0.193296}%
\pgfsetfillcolor{currentfill}%
\pgfsetlinewidth{0.000000pt}%
\definecolor{currentstroke}{rgb}{0.000000,0.000000,0.000000}%
\pgfsetstrokecolor{currentstroke}%
\pgfsetstrokeopacity{0.000000}%
\pgfsetdash{}{0pt}%
\pgfpathmoveto{\pgfqpoint{6.188138in}{1.813884in}}%
\pgfpathlineto{\pgfqpoint{6.197074in}{1.813884in}}%
\pgfpathlineto{\pgfqpoint{6.197074in}{2.115964in}}%
\pgfpathlineto{\pgfqpoint{6.188138in}{2.115964in}}%
\pgfpathlineto{\pgfqpoint{6.188138in}{1.813884in}}%
\pgfpathclose%
\pgfusepath{fill}%
\end{pgfscope}%
\begin{pgfscope}%
\pgfpathrectangle{\pgfqpoint{3.722897in}{0.857143in}}{\pgfqpoint{2.627103in}{1.813434in}}%
\pgfusepath{clip}%
\pgfsetbuttcap%
\pgfsetmiterjoin%
\definecolor{currentfill}{rgb}{0.511253,0.510898,0.193296}%
\pgfsetfillcolor{currentfill}%
\pgfsetlinewidth{0.000000pt}%
\definecolor{currentstroke}{rgb}{0.000000,0.000000,0.000000}%
\pgfsetstrokecolor{currentstroke}%
\pgfsetstrokeopacity{0.000000}%
\pgfsetdash{}{0pt}%
\pgfpathmoveto{\pgfqpoint{6.199309in}{1.813884in}}%
\pgfpathlineto{\pgfqpoint{6.208245in}{1.813884in}}%
\pgfpathlineto{\pgfqpoint{6.208245in}{2.149883in}}%
\pgfpathlineto{\pgfqpoint{6.199309in}{2.149883in}}%
\pgfpathlineto{\pgfqpoint{6.199309in}{1.813884in}}%
\pgfpathclose%
\pgfusepath{fill}%
\end{pgfscope}%
\begin{pgfscope}%
\pgfpathrectangle{\pgfqpoint{3.722897in}{0.857143in}}{\pgfqpoint{2.627103in}{1.813434in}}%
\pgfusepath{clip}%
\pgfsetbuttcap%
\pgfsetmiterjoin%
\definecolor{currentfill}{rgb}{0.511253,0.510898,0.193296}%
\pgfsetfillcolor{currentfill}%
\pgfsetlinewidth{0.000000pt}%
\definecolor{currentstroke}{rgb}{0.000000,0.000000,0.000000}%
\pgfsetstrokecolor{currentstroke}%
\pgfsetstrokeopacity{0.000000}%
\pgfsetdash{}{0pt}%
\pgfpathmoveto{\pgfqpoint{6.210479in}{1.813884in}}%
\pgfpathlineto{\pgfqpoint{6.219416in}{1.813884in}}%
\pgfpathlineto{\pgfqpoint{6.219416in}{2.182768in}}%
\pgfpathlineto{\pgfqpoint{6.210479in}{2.182768in}}%
\pgfpathlineto{\pgfqpoint{6.210479in}{1.813884in}}%
\pgfpathclose%
\pgfusepath{fill}%
\end{pgfscope}%
\begin{pgfscope}%
\pgfpathrectangle{\pgfqpoint{3.722897in}{0.857143in}}{\pgfqpoint{2.627103in}{1.813434in}}%
\pgfusepath{clip}%
\pgfsetbuttcap%
\pgfsetmiterjoin%
\definecolor{currentfill}{rgb}{0.511253,0.510898,0.193296}%
\pgfsetfillcolor{currentfill}%
\pgfsetlinewidth{0.000000pt}%
\definecolor{currentstroke}{rgb}{0.000000,0.000000,0.000000}%
\pgfsetstrokecolor{currentstroke}%
\pgfsetstrokeopacity{0.000000}%
\pgfsetdash{}{0pt}%
\pgfpathmoveto{\pgfqpoint{6.221650in}{1.813884in}}%
\pgfpathlineto{\pgfqpoint{6.230586in}{1.813884in}}%
\pgfpathlineto{\pgfqpoint{6.230586in}{2.216388in}}%
\pgfpathlineto{\pgfqpoint{6.221650in}{2.216388in}}%
\pgfpathlineto{\pgfqpoint{6.221650in}{1.813884in}}%
\pgfpathclose%
\pgfusepath{fill}%
\end{pgfscope}%
\begin{pgfscope}%
\pgfpathrectangle{\pgfqpoint{3.722897in}{0.857143in}}{\pgfqpoint{2.627103in}{1.813434in}}%
\pgfusepath{clip}%
\pgfsetbuttcap%
\pgfsetmiterjoin%
\definecolor{currentfill}{rgb}{0.754268,0.565033,0.211761}%
\pgfsetfillcolor{currentfill}%
\pgfsetlinewidth{0.000000pt}%
\definecolor{currentstroke}{rgb}{0.000000,0.000000,0.000000}%
\pgfsetstrokecolor{currentstroke}%
\pgfsetstrokeopacity{0.000000}%
\pgfsetdash{}{0pt}%
\pgfpathmoveto{\pgfqpoint{3.842311in}{1.819333in}}%
\pgfpathlineto{\pgfqpoint{3.851247in}{1.819333in}}%
\pgfpathlineto{\pgfqpoint{3.851247in}{1.819845in}}%
\pgfpathlineto{\pgfqpoint{3.842311in}{1.819845in}}%
\pgfpathlineto{\pgfqpoint{3.842311in}{1.819333in}}%
\pgfpathclose%
\pgfusepath{fill}%
\end{pgfscope}%
\begin{pgfscope}%
\pgfpathrectangle{\pgfqpoint{3.722897in}{0.857143in}}{\pgfqpoint{2.627103in}{1.813434in}}%
\pgfusepath{clip}%
\pgfsetbuttcap%
\pgfsetmiterjoin%
\definecolor{currentfill}{rgb}{0.754268,0.565033,0.211761}%
\pgfsetfillcolor{currentfill}%
\pgfsetlinewidth{0.000000pt}%
\definecolor{currentstroke}{rgb}{0.000000,0.000000,0.000000}%
\pgfsetstrokecolor{currentstroke}%
\pgfsetstrokeopacity{0.000000}%
\pgfsetdash{}{0pt}%
\pgfpathmoveto{\pgfqpoint{3.853481in}{1.755179in}}%
\pgfpathlineto{\pgfqpoint{3.862418in}{1.755179in}}%
\pgfpathlineto{\pgfqpoint{3.862418in}{1.752062in}}%
\pgfpathlineto{\pgfqpoint{3.853481in}{1.752062in}}%
\pgfpathlineto{\pgfqpoint{3.853481in}{1.755179in}}%
\pgfpathclose%
\pgfusepath{fill}%
\end{pgfscope}%
\begin{pgfscope}%
\pgfpathrectangle{\pgfqpoint{3.722897in}{0.857143in}}{\pgfqpoint{2.627103in}{1.813434in}}%
\pgfusepath{clip}%
\pgfsetbuttcap%
\pgfsetmiterjoin%
\definecolor{currentfill}{rgb}{0.754268,0.565033,0.211761}%
\pgfsetfillcolor{currentfill}%
\pgfsetlinewidth{0.000000pt}%
\definecolor{currentstroke}{rgb}{0.000000,0.000000,0.000000}%
\pgfsetstrokecolor{currentstroke}%
\pgfsetstrokeopacity{0.000000}%
\pgfsetdash{}{0pt}%
\pgfpathmoveto{\pgfqpoint{3.864652in}{1.747016in}}%
\pgfpathlineto{\pgfqpoint{3.873588in}{1.747016in}}%
\pgfpathlineto{\pgfqpoint{3.873588in}{1.745934in}}%
\pgfpathlineto{\pgfqpoint{3.864652in}{1.745934in}}%
\pgfpathlineto{\pgfqpoint{3.864652in}{1.747016in}}%
\pgfpathclose%
\pgfusepath{fill}%
\end{pgfscope}%
\begin{pgfscope}%
\pgfpathrectangle{\pgfqpoint{3.722897in}{0.857143in}}{\pgfqpoint{2.627103in}{1.813434in}}%
\pgfusepath{clip}%
\pgfsetbuttcap%
\pgfsetmiterjoin%
\definecolor{currentfill}{rgb}{0.754268,0.565033,0.211761}%
\pgfsetfillcolor{currentfill}%
\pgfsetlinewidth{0.000000pt}%
\definecolor{currentstroke}{rgb}{0.000000,0.000000,0.000000}%
\pgfsetstrokecolor{currentstroke}%
\pgfsetstrokeopacity{0.000000}%
\pgfsetdash{}{0pt}%
\pgfpathmoveto{\pgfqpoint{3.875823in}{1.833313in}}%
\pgfpathlineto{\pgfqpoint{3.884759in}{1.833313in}}%
\pgfpathlineto{\pgfqpoint{3.884759in}{1.845496in}}%
\pgfpathlineto{\pgfqpoint{3.875823in}{1.845496in}}%
\pgfpathlineto{\pgfqpoint{3.875823in}{1.833313in}}%
\pgfpathclose%
\pgfusepath{fill}%
\end{pgfscope}%
\begin{pgfscope}%
\pgfpathrectangle{\pgfqpoint{3.722897in}{0.857143in}}{\pgfqpoint{2.627103in}{1.813434in}}%
\pgfusepath{clip}%
\pgfsetbuttcap%
\pgfsetmiterjoin%
\definecolor{currentfill}{rgb}{0.754268,0.565033,0.211761}%
\pgfsetfillcolor{currentfill}%
\pgfsetlinewidth{0.000000pt}%
\definecolor{currentstroke}{rgb}{0.000000,0.000000,0.000000}%
\pgfsetstrokecolor{currentstroke}%
\pgfsetstrokeopacity{0.000000}%
\pgfsetdash{}{0pt}%
\pgfpathmoveto{\pgfqpoint{3.886993in}{1.818570in}}%
\pgfpathlineto{\pgfqpoint{3.895930in}{1.818570in}}%
\pgfpathlineto{\pgfqpoint{3.895930in}{1.835495in}}%
\pgfpathlineto{\pgfqpoint{3.886993in}{1.835495in}}%
\pgfpathlineto{\pgfqpoint{3.886993in}{1.818570in}}%
\pgfpathclose%
\pgfusepath{fill}%
\end{pgfscope}%
\begin{pgfscope}%
\pgfpathrectangle{\pgfqpoint{3.722897in}{0.857143in}}{\pgfqpoint{2.627103in}{1.813434in}}%
\pgfusepath{clip}%
\pgfsetbuttcap%
\pgfsetmiterjoin%
\definecolor{currentfill}{rgb}{0.754268,0.565033,0.211761}%
\pgfsetfillcolor{currentfill}%
\pgfsetlinewidth{0.000000pt}%
\definecolor{currentstroke}{rgb}{0.000000,0.000000,0.000000}%
\pgfsetstrokecolor{currentstroke}%
\pgfsetstrokeopacity{0.000000}%
\pgfsetdash{}{0pt}%
\pgfpathmoveto{\pgfqpoint{3.898164in}{1.819190in}}%
\pgfpathlineto{\pgfqpoint{3.907100in}{1.819190in}}%
\pgfpathlineto{\pgfqpoint{3.907100in}{1.823745in}}%
\pgfpathlineto{\pgfqpoint{3.898164in}{1.823745in}}%
\pgfpathlineto{\pgfqpoint{3.898164in}{1.819190in}}%
\pgfpathclose%
\pgfusepath{fill}%
\end{pgfscope}%
\begin{pgfscope}%
\pgfpathrectangle{\pgfqpoint{3.722897in}{0.857143in}}{\pgfqpoint{2.627103in}{1.813434in}}%
\pgfusepath{clip}%
\pgfsetbuttcap%
\pgfsetmiterjoin%
\definecolor{currentfill}{rgb}{0.754268,0.565033,0.211761}%
\pgfsetfillcolor{currentfill}%
\pgfsetlinewidth{0.000000pt}%
\definecolor{currentstroke}{rgb}{0.000000,0.000000,0.000000}%
\pgfsetstrokecolor{currentstroke}%
\pgfsetstrokeopacity{0.000000}%
\pgfsetdash{}{0pt}%
\pgfpathmoveto{\pgfqpoint{3.909334in}{1.706421in}}%
\pgfpathlineto{\pgfqpoint{3.918271in}{1.706421in}}%
\pgfpathlineto{\pgfqpoint{3.918271in}{1.694555in}}%
\pgfpathlineto{\pgfqpoint{3.909334in}{1.694555in}}%
\pgfpathlineto{\pgfqpoint{3.909334in}{1.706421in}}%
\pgfpathclose%
\pgfusepath{fill}%
\end{pgfscope}%
\begin{pgfscope}%
\pgfpathrectangle{\pgfqpoint{3.722897in}{0.857143in}}{\pgfqpoint{2.627103in}{1.813434in}}%
\pgfusepath{clip}%
\pgfsetbuttcap%
\pgfsetmiterjoin%
\definecolor{currentfill}{rgb}{0.754268,0.565033,0.211761}%
\pgfsetfillcolor{currentfill}%
\pgfsetlinewidth{0.000000pt}%
\definecolor{currentstroke}{rgb}{0.000000,0.000000,0.000000}%
\pgfsetstrokecolor{currentstroke}%
\pgfsetstrokeopacity{0.000000}%
\pgfsetdash{}{0pt}%
\pgfpathmoveto{\pgfqpoint{3.920505in}{1.696524in}}%
\pgfpathlineto{\pgfqpoint{3.929442in}{1.696524in}}%
\pgfpathlineto{\pgfqpoint{3.929442in}{1.673449in}}%
\pgfpathlineto{\pgfqpoint{3.920505in}{1.673449in}}%
\pgfpathlineto{\pgfqpoint{3.920505in}{1.696524in}}%
\pgfpathclose%
\pgfusepath{fill}%
\end{pgfscope}%
\begin{pgfscope}%
\pgfpathrectangle{\pgfqpoint{3.722897in}{0.857143in}}{\pgfqpoint{2.627103in}{1.813434in}}%
\pgfusepath{clip}%
\pgfsetbuttcap%
\pgfsetmiterjoin%
\definecolor{currentfill}{rgb}{0.754268,0.565033,0.211761}%
\pgfsetfillcolor{currentfill}%
\pgfsetlinewidth{0.000000pt}%
\definecolor{currentstroke}{rgb}{0.000000,0.000000,0.000000}%
\pgfsetstrokecolor{currentstroke}%
\pgfsetstrokeopacity{0.000000}%
\pgfsetdash{}{0pt}%
\pgfpathmoveto{\pgfqpoint{3.931676in}{1.690133in}}%
\pgfpathlineto{\pgfqpoint{3.940612in}{1.690133in}}%
\pgfpathlineto{\pgfqpoint{3.940612in}{1.664728in}}%
\pgfpathlineto{\pgfqpoint{3.931676in}{1.664728in}}%
\pgfpathlineto{\pgfqpoint{3.931676in}{1.690133in}}%
\pgfpathclose%
\pgfusepath{fill}%
\end{pgfscope}%
\begin{pgfscope}%
\pgfpathrectangle{\pgfqpoint{3.722897in}{0.857143in}}{\pgfqpoint{2.627103in}{1.813434in}}%
\pgfusepath{clip}%
\pgfsetbuttcap%
\pgfsetmiterjoin%
\definecolor{currentfill}{rgb}{0.754268,0.565033,0.211761}%
\pgfsetfillcolor{currentfill}%
\pgfsetlinewidth{0.000000pt}%
\definecolor{currentstroke}{rgb}{0.000000,0.000000,0.000000}%
\pgfsetstrokecolor{currentstroke}%
\pgfsetstrokeopacity{0.000000}%
\pgfsetdash{}{0pt}%
\pgfpathmoveto{\pgfqpoint{3.942846in}{1.685081in}}%
\pgfpathlineto{\pgfqpoint{3.951783in}{1.685081in}}%
\pgfpathlineto{\pgfqpoint{3.951783in}{1.662237in}}%
\pgfpathlineto{\pgfqpoint{3.942846in}{1.662237in}}%
\pgfpathlineto{\pgfqpoint{3.942846in}{1.685081in}}%
\pgfpathclose%
\pgfusepath{fill}%
\end{pgfscope}%
\begin{pgfscope}%
\pgfpathrectangle{\pgfqpoint{3.722897in}{0.857143in}}{\pgfqpoint{2.627103in}{1.813434in}}%
\pgfusepath{clip}%
\pgfsetbuttcap%
\pgfsetmiterjoin%
\definecolor{currentfill}{rgb}{0.754268,0.565033,0.211761}%
\pgfsetfillcolor{currentfill}%
\pgfsetlinewidth{0.000000pt}%
\definecolor{currentstroke}{rgb}{0.000000,0.000000,0.000000}%
\pgfsetstrokecolor{currentstroke}%
\pgfsetstrokeopacity{0.000000}%
\pgfsetdash{}{0pt}%
\pgfpathmoveto{\pgfqpoint{3.954017in}{1.680482in}}%
\pgfpathlineto{\pgfqpoint{3.962953in}{1.680482in}}%
\pgfpathlineto{\pgfqpoint{3.962953in}{1.643688in}}%
\pgfpathlineto{\pgfqpoint{3.954017in}{1.643688in}}%
\pgfpathlineto{\pgfqpoint{3.954017in}{1.680482in}}%
\pgfpathclose%
\pgfusepath{fill}%
\end{pgfscope}%
\begin{pgfscope}%
\pgfpathrectangle{\pgfqpoint{3.722897in}{0.857143in}}{\pgfqpoint{2.627103in}{1.813434in}}%
\pgfusepath{clip}%
\pgfsetbuttcap%
\pgfsetmiterjoin%
\definecolor{currentfill}{rgb}{0.754268,0.565033,0.211761}%
\pgfsetfillcolor{currentfill}%
\pgfsetlinewidth{0.000000pt}%
\definecolor{currentstroke}{rgb}{0.000000,0.000000,0.000000}%
\pgfsetstrokecolor{currentstroke}%
\pgfsetstrokeopacity{0.000000}%
\pgfsetdash{}{0pt}%
\pgfpathmoveto{\pgfqpoint{3.965187in}{1.675667in}}%
\pgfpathlineto{\pgfqpoint{3.974124in}{1.675667in}}%
\pgfpathlineto{\pgfqpoint{3.974124in}{1.641335in}}%
\pgfpathlineto{\pgfqpoint{3.965187in}{1.641335in}}%
\pgfpathlineto{\pgfqpoint{3.965187in}{1.675667in}}%
\pgfpathclose%
\pgfusepath{fill}%
\end{pgfscope}%
\begin{pgfscope}%
\pgfpathrectangle{\pgfqpoint{3.722897in}{0.857143in}}{\pgfqpoint{2.627103in}{1.813434in}}%
\pgfusepath{clip}%
\pgfsetbuttcap%
\pgfsetmiterjoin%
\definecolor{currentfill}{rgb}{0.754268,0.565033,0.211761}%
\pgfsetfillcolor{currentfill}%
\pgfsetlinewidth{0.000000pt}%
\definecolor{currentstroke}{rgb}{0.000000,0.000000,0.000000}%
\pgfsetstrokecolor{currentstroke}%
\pgfsetstrokeopacity{0.000000}%
\pgfsetdash{}{0pt}%
\pgfpathmoveto{\pgfqpoint{3.976358in}{1.671869in}}%
\pgfpathlineto{\pgfqpoint{3.985295in}{1.671869in}}%
\pgfpathlineto{\pgfqpoint{3.985295in}{1.637785in}}%
\pgfpathlineto{\pgfqpoint{3.976358in}{1.637785in}}%
\pgfpathlineto{\pgfqpoint{3.976358in}{1.671869in}}%
\pgfpathclose%
\pgfusepath{fill}%
\end{pgfscope}%
\begin{pgfscope}%
\pgfpathrectangle{\pgfqpoint{3.722897in}{0.857143in}}{\pgfqpoint{2.627103in}{1.813434in}}%
\pgfusepath{clip}%
\pgfsetbuttcap%
\pgfsetmiterjoin%
\definecolor{currentfill}{rgb}{0.754268,0.565033,0.211761}%
\pgfsetfillcolor{currentfill}%
\pgfsetlinewidth{0.000000pt}%
\definecolor{currentstroke}{rgb}{0.000000,0.000000,0.000000}%
\pgfsetstrokecolor{currentstroke}%
\pgfsetstrokeopacity{0.000000}%
\pgfsetdash{}{0pt}%
\pgfpathmoveto{\pgfqpoint{3.987529in}{1.667688in}}%
\pgfpathlineto{\pgfqpoint{3.996465in}{1.667688in}}%
\pgfpathlineto{\pgfqpoint{3.996465in}{1.635809in}}%
\pgfpathlineto{\pgfqpoint{3.987529in}{1.635809in}}%
\pgfpathlineto{\pgfqpoint{3.987529in}{1.667688in}}%
\pgfpathclose%
\pgfusepath{fill}%
\end{pgfscope}%
\begin{pgfscope}%
\pgfpathrectangle{\pgfqpoint{3.722897in}{0.857143in}}{\pgfqpoint{2.627103in}{1.813434in}}%
\pgfusepath{clip}%
\pgfsetbuttcap%
\pgfsetmiterjoin%
\definecolor{currentfill}{rgb}{0.754268,0.565033,0.211761}%
\pgfsetfillcolor{currentfill}%
\pgfsetlinewidth{0.000000pt}%
\definecolor{currentstroke}{rgb}{0.000000,0.000000,0.000000}%
\pgfsetstrokecolor{currentstroke}%
\pgfsetstrokeopacity{0.000000}%
\pgfsetdash{}{0pt}%
\pgfpathmoveto{\pgfqpoint{3.998699in}{1.661245in}}%
\pgfpathlineto{\pgfqpoint{4.007636in}{1.661245in}}%
\pgfpathlineto{\pgfqpoint{4.007636in}{1.637910in}}%
\pgfpathlineto{\pgfqpoint{3.998699in}{1.637910in}}%
\pgfpathlineto{\pgfqpoint{3.998699in}{1.661245in}}%
\pgfpathclose%
\pgfusepath{fill}%
\end{pgfscope}%
\begin{pgfscope}%
\pgfpathrectangle{\pgfqpoint{3.722897in}{0.857143in}}{\pgfqpoint{2.627103in}{1.813434in}}%
\pgfusepath{clip}%
\pgfsetbuttcap%
\pgfsetmiterjoin%
\definecolor{currentfill}{rgb}{0.754268,0.565033,0.211761}%
\pgfsetfillcolor{currentfill}%
\pgfsetlinewidth{0.000000pt}%
\definecolor{currentstroke}{rgb}{0.000000,0.000000,0.000000}%
\pgfsetstrokecolor{currentstroke}%
\pgfsetstrokeopacity{0.000000}%
\pgfsetdash{}{0pt}%
\pgfpathmoveto{\pgfqpoint{4.009870in}{1.656708in}}%
\pgfpathlineto{\pgfqpoint{4.018806in}{1.656708in}}%
\pgfpathlineto{\pgfqpoint{4.018806in}{1.637210in}}%
\pgfpathlineto{\pgfqpoint{4.009870in}{1.637210in}}%
\pgfpathlineto{\pgfqpoint{4.009870in}{1.656708in}}%
\pgfpathclose%
\pgfusepath{fill}%
\end{pgfscope}%
\begin{pgfscope}%
\pgfpathrectangle{\pgfqpoint{3.722897in}{0.857143in}}{\pgfqpoint{2.627103in}{1.813434in}}%
\pgfusepath{clip}%
\pgfsetbuttcap%
\pgfsetmiterjoin%
\definecolor{currentfill}{rgb}{0.754268,0.565033,0.211761}%
\pgfsetfillcolor{currentfill}%
\pgfsetlinewidth{0.000000pt}%
\definecolor{currentstroke}{rgb}{0.000000,0.000000,0.000000}%
\pgfsetstrokecolor{currentstroke}%
\pgfsetstrokeopacity{0.000000}%
\pgfsetdash{}{0pt}%
\pgfpathmoveto{\pgfqpoint{4.021040in}{1.653617in}}%
\pgfpathlineto{\pgfqpoint{4.029977in}{1.653617in}}%
\pgfpathlineto{\pgfqpoint{4.029977in}{1.632702in}}%
\pgfpathlineto{\pgfqpoint{4.021040in}{1.632702in}}%
\pgfpathlineto{\pgfqpoint{4.021040in}{1.653617in}}%
\pgfpathclose%
\pgfusepath{fill}%
\end{pgfscope}%
\begin{pgfscope}%
\pgfpathrectangle{\pgfqpoint{3.722897in}{0.857143in}}{\pgfqpoint{2.627103in}{1.813434in}}%
\pgfusepath{clip}%
\pgfsetbuttcap%
\pgfsetmiterjoin%
\definecolor{currentfill}{rgb}{0.754268,0.565033,0.211761}%
\pgfsetfillcolor{currentfill}%
\pgfsetlinewidth{0.000000pt}%
\definecolor{currentstroke}{rgb}{0.000000,0.000000,0.000000}%
\pgfsetstrokecolor{currentstroke}%
\pgfsetstrokeopacity{0.000000}%
\pgfsetdash{}{0pt}%
\pgfpathmoveto{\pgfqpoint{4.032211in}{1.625915in}}%
\pgfpathlineto{\pgfqpoint{4.041148in}{1.625915in}}%
\pgfpathlineto{\pgfqpoint{4.041148in}{1.619274in}}%
\pgfpathlineto{\pgfqpoint{4.032211in}{1.619274in}}%
\pgfpathlineto{\pgfqpoint{4.032211in}{1.625915in}}%
\pgfpathclose%
\pgfusepath{fill}%
\end{pgfscope}%
\begin{pgfscope}%
\pgfpathrectangle{\pgfqpoint{3.722897in}{0.857143in}}{\pgfqpoint{2.627103in}{1.813434in}}%
\pgfusepath{clip}%
\pgfsetbuttcap%
\pgfsetmiterjoin%
\definecolor{currentfill}{rgb}{0.754268,0.565033,0.211761}%
\pgfsetfillcolor{currentfill}%
\pgfsetlinewidth{0.000000pt}%
\definecolor{currentstroke}{rgb}{0.000000,0.000000,0.000000}%
\pgfsetstrokecolor{currentstroke}%
\pgfsetstrokeopacity{0.000000}%
\pgfsetdash{}{0pt}%
\pgfpathmoveto{\pgfqpoint{4.043382in}{1.605033in}}%
\pgfpathlineto{\pgfqpoint{4.052318in}{1.605033in}}%
\pgfpathlineto{\pgfqpoint{4.052318in}{1.589310in}}%
\pgfpathlineto{\pgfqpoint{4.043382in}{1.589310in}}%
\pgfpathlineto{\pgfqpoint{4.043382in}{1.605033in}}%
\pgfpathclose%
\pgfusepath{fill}%
\end{pgfscope}%
\begin{pgfscope}%
\pgfpathrectangle{\pgfqpoint{3.722897in}{0.857143in}}{\pgfqpoint{2.627103in}{1.813434in}}%
\pgfusepath{clip}%
\pgfsetbuttcap%
\pgfsetmiterjoin%
\definecolor{currentfill}{rgb}{0.754268,0.565033,0.211761}%
\pgfsetfillcolor{currentfill}%
\pgfsetlinewidth{0.000000pt}%
\definecolor{currentstroke}{rgb}{0.000000,0.000000,0.000000}%
\pgfsetstrokecolor{currentstroke}%
\pgfsetstrokeopacity{0.000000}%
\pgfsetdash{}{0pt}%
\pgfpathmoveto{\pgfqpoint{4.054552in}{1.587619in}}%
\pgfpathlineto{\pgfqpoint{4.063489in}{1.587619in}}%
\pgfpathlineto{\pgfqpoint{4.063489in}{1.544573in}}%
\pgfpathlineto{\pgfqpoint{4.054552in}{1.544573in}}%
\pgfpathlineto{\pgfqpoint{4.054552in}{1.587619in}}%
\pgfpathclose%
\pgfusepath{fill}%
\end{pgfscope}%
\begin{pgfscope}%
\pgfpathrectangle{\pgfqpoint{3.722897in}{0.857143in}}{\pgfqpoint{2.627103in}{1.813434in}}%
\pgfusepath{clip}%
\pgfsetbuttcap%
\pgfsetmiterjoin%
\definecolor{currentfill}{rgb}{0.754268,0.565033,0.211761}%
\pgfsetfillcolor{currentfill}%
\pgfsetlinewidth{0.000000pt}%
\definecolor{currentstroke}{rgb}{0.000000,0.000000,0.000000}%
\pgfsetstrokecolor{currentstroke}%
\pgfsetstrokeopacity{0.000000}%
\pgfsetdash{}{0pt}%
\pgfpathmoveto{\pgfqpoint{4.065723in}{1.585807in}}%
\pgfpathlineto{\pgfqpoint{4.074659in}{1.585807in}}%
\pgfpathlineto{\pgfqpoint{4.074659in}{1.523812in}}%
\pgfpathlineto{\pgfqpoint{4.065723in}{1.523812in}}%
\pgfpathlineto{\pgfqpoint{4.065723in}{1.585807in}}%
\pgfpathclose%
\pgfusepath{fill}%
\end{pgfscope}%
\begin{pgfscope}%
\pgfpathrectangle{\pgfqpoint{3.722897in}{0.857143in}}{\pgfqpoint{2.627103in}{1.813434in}}%
\pgfusepath{clip}%
\pgfsetbuttcap%
\pgfsetmiterjoin%
\definecolor{currentfill}{rgb}{0.754268,0.565033,0.211761}%
\pgfsetfillcolor{currentfill}%
\pgfsetlinewidth{0.000000pt}%
\definecolor{currentstroke}{rgb}{0.000000,0.000000,0.000000}%
\pgfsetstrokecolor{currentstroke}%
\pgfsetstrokeopacity{0.000000}%
\pgfsetdash{}{0pt}%
\pgfpathmoveto{\pgfqpoint{4.076893in}{1.582523in}}%
\pgfpathlineto{\pgfqpoint{4.085830in}{1.582523in}}%
\pgfpathlineto{\pgfqpoint{4.085830in}{1.519313in}}%
\pgfpathlineto{\pgfqpoint{4.076893in}{1.519313in}}%
\pgfpathlineto{\pgfqpoint{4.076893in}{1.582523in}}%
\pgfpathclose%
\pgfusepath{fill}%
\end{pgfscope}%
\begin{pgfscope}%
\pgfpathrectangle{\pgfqpoint{3.722897in}{0.857143in}}{\pgfqpoint{2.627103in}{1.813434in}}%
\pgfusepath{clip}%
\pgfsetbuttcap%
\pgfsetmiterjoin%
\definecolor{currentfill}{rgb}{0.754268,0.565033,0.211761}%
\pgfsetfillcolor{currentfill}%
\pgfsetlinewidth{0.000000pt}%
\definecolor{currentstroke}{rgb}{0.000000,0.000000,0.000000}%
\pgfsetstrokecolor{currentstroke}%
\pgfsetstrokeopacity{0.000000}%
\pgfsetdash{}{0pt}%
\pgfpathmoveto{\pgfqpoint{4.088064in}{1.561024in}}%
\pgfpathlineto{\pgfqpoint{4.097001in}{1.561024in}}%
\pgfpathlineto{\pgfqpoint{4.097001in}{1.503556in}}%
\pgfpathlineto{\pgfqpoint{4.088064in}{1.503556in}}%
\pgfpathlineto{\pgfqpoint{4.088064in}{1.561024in}}%
\pgfpathclose%
\pgfusepath{fill}%
\end{pgfscope}%
\begin{pgfscope}%
\pgfpathrectangle{\pgfqpoint{3.722897in}{0.857143in}}{\pgfqpoint{2.627103in}{1.813434in}}%
\pgfusepath{clip}%
\pgfsetbuttcap%
\pgfsetmiterjoin%
\definecolor{currentfill}{rgb}{0.754268,0.565033,0.211761}%
\pgfsetfillcolor{currentfill}%
\pgfsetlinewidth{0.000000pt}%
\definecolor{currentstroke}{rgb}{0.000000,0.000000,0.000000}%
\pgfsetstrokecolor{currentstroke}%
\pgfsetstrokeopacity{0.000000}%
\pgfsetdash{}{0pt}%
\pgfpathmoveto{\pgfqpoint{4.099235in}{1.550457in}}%
\pgfpathlineto{\pgfqpoint{4.108171in}{1.550457in}}%
\pgfpathlineto{\pgfqpoint{4.108171in}{1.463098in}}%
\pgfpathlineto{\pgfqpoint{4.099235in}{1.463098in}}%
\pgfpathlineto{\pgfqpoint{4.099235in}{1.550457in}}%
\pgfpathclose%
\pgfusepath{fill}%
\end{pgfscope}%
\begin{pgfscope}%
\pgfpathrectangle{\pgfqpoint{3.722897in}{0.857143in}}{\pgfqpoint{2.627103in}{1.813434in}}%
\pgfusepath{clip}%
\pgfsetbuttcap%
\pgfsetmiterjoin%
\definecolor{currentfill}{rgb}{0.754268,0.565033,0.211761}%
\pgfsetfillcolor{currentfill}%
\pgfsetlinewidth{0.000000pt}%
\definecolor{currentstroke}{rgb}{0.000000,0.000000,0.000000}%
\pgfsetstrokecolor{currentstroke}%
\pgfsetstrokeopacity{0.000000}%
\pgfsetdash{}{0pt}%
\pgfpathmoveto{\pgfqpoint{4.110405in}{1.544102in}}%
\pgfpathlineto{\pgfqpoint{4.119342in}{1.544102in}}%
\pgfpathlineto{\pgfqpoint{4.119342in}{1.464342in}}%
\pgfpathlineto{\pgfqpoint{4.110405in}{1.464342in}}%
\pgfpathlineto{\pgfqpoint{4.110405in}{1.544102in}}%
\pgfpathclose%
\pgfusepath{fill}%
\end{pgfscope}%
\begin{pgfscope}%
\pgfpathrectangle{\pgfqpoint{3.722897in}{0.857143in}}{\pgfqpoint{2.627103in}{1.813434in}}%
\pgfusepath{clip}%
\pgfsetbuttcap%
\pgfsetmiterjoin%
\definecolor{currentfill}{rgb}{0.754268,0.565033,0.211761}%
\pgfsetfillcolor{currentfill}%
\pgfsetlinewidth{0.000000pt}%
\definecolor{currentstroke}{rgb}{0.000000,0.000000,0.000000}%
\pgfsetstrokecolor{currentstroke}%
\pgfsetstrokeopacity{0.000000}%
\pgfsetdash{}{0pt}%
\pgfpathmoveto{\pgfqpoint{4.121576in}{1.532734in}}%
\pgfpathlineto{\pgfqpoint{4.130512in}{1.532734in}}%
\pgfpathlineto{\pgfqpoint{4.130512in}{1.450798in}}%
\pgfpathlineto{\pgfqpoint{4.121576in}{1.450798in}}%
\pgfpathlineto{\pgfqpoint{4.121576in}{1.532734in}}%
\pgfpathclose%
\pgfusepath{fill}%
\end{pgfscope}%
\begin{pgfscope}%
\pgfpathrectangle{\pgfqpoint{3.722897in}{0.857143in}}{\pgfqpoint{2.627103in}{1.813434in}}%
\pgfusepath{clip}%
\pgfsetbuttcap%
\pgfsetmiterjoin%
\definecolor{currentfill}{rgb}{0.754268,0.565033,0.211761}%
\pgfsetfillcolor{currentfill}%
\pgfsetlinewidth{0.000000pt}%
\definecolor{currentstroke}{rgb}{0.000000,0.000000,0.000000}%
\pgfsetstrokecolor{currentstroke}%
\pgfsetstrokeopacity{0.000000}%
\pgfsetdash{}{0pt}%
\pgfpathmoveto{\pgfqpoint{4.132747in}{1.532710in}}%
\pgfpathlineto{\pgfqpoint{4.141683in}{1.532710in}}%
\pgfpathlineto{\pgfqpoint{4.141683in}{1.440191in}}%
\pgfpathlineto{\pgfqpoint{4.132747in}{1.440191in}}%
\pgfpathlineto{\pgfqpoint{4.132747in}{1.532710in}}%
\pgfpathclose%
\pgfusepath{fill}%
\end{pgfscope}%
\begin{pgfscope}%
\pgfpathrectangle{\pgfqpoint{3.722897in}{0.857143in}}{\pgfqpoint{2.627103in}{1.813434in}}%
\pgfusepath{clip}%
\pgfsetbuttcap%
\pgfsetmiterjoin%
\definecolor{currentfill}{rgb}{0.754268,0.565033,0.211761}%
\pgfsetfillcolor{currentfill}%
\pgfsetlinewidth{0.000000pt}%
\definecolor{currentstroke}{rgb}{0.000000,0.000000,0.000000}%
\pgfsetstrokecolor{currentstroke}%
\pgfsetstrokeopacity{0.000000}%
\pgfsetdash{}{0pt}%
\pgfpathmoveto{\pgfqpoint{4.143917in}{1.517521in}}%
\pgfpathlineto{\pgfqpoint{4.152854in}{1.517521in}}%
\pgfpathlineto{\pgfqpoint{4.152854in}{1.424829in}}%
\pgfpathlineto{\pgfqpoint{4.143917in}{1.424829in}}%
\pgfpathlineto{\pgfqpoint{4.143917in}{1.517521in}}%
\pgfpathclose%
\pgfusepath{fill}%
\end{pgfscope}%
\begin{pgfscope}%
\pgfpathrectangle{\pgfqpoint{3.722897in}{0.857143in}}{\pgfqpoint{2.627103in}{1.813434in}}%
\pgfusepath{clip}%
\pgfsetbuttcap%
\pgfsetmiterjoin%
\definecolor{currentfill}{rgb}{0.754268,0.565033,0.211761}%
\pgfsetfillcolor{currentfill}%
\pgfsetlinewidth{0.000000pt}%
\definecolor{currentstroke}{rgb}{0.000000,0.000000,0.000000}%
\pgfsetstrokecolor{currentstroke}%
\pgfsetstrokeopacity{0.000000}%
\pgfsetdash{}{0pt}%
\pgfpathmoveto{\pgfqpoint{4.155088in}{1.505434in}}%
\pgfpathlineto{\pgfqpoint{4.164024in}{1.505434in}}%
\pgfpathlineto{\pgfqpoint{4.164024in}{1.405787in}}%
\pgfpathlineto{\pgfqpoint{4.155088in}{1.405787in}}%
\pgfpathlineto{\pgfqpoint{4.155088in}{1.505434in}}%
\pgfpathclose%
\pgfusepath{fill}%
\end{pgfscope}%
\begin{pgfscope}%
\pgfpathrectangle{\pgfqpoint{3.722897in}{0.857143in}}{\pgfqpoint{2.627103in}{1.813434in}}%
\pgfusepath{clip}%
\pgfsetbuttcap%
\pgfsetmiterjoin%
\definecolor{currentfill}{rgb}{0.754268,0.565033,0.211761}%
\pgfsetfillcolor{currentfill}%
\pgfsetlinewidth{0.000000pt}%
\definecolor{currentstroke}{rgb}{0.000000,0.000000,0.000000}%
\pgfsetstrokecolor{currentstroke}%
\pgfsetstrokeopacity{0.000000}%
\pgfsetdash{}{0pt}%
\pgfpathmoveto{\pgfqpoint{4.166258in}{1.497545in}}%
\pgfpathlineto{\pgfqpoint{4.175195in}{1.497545in}}%
\pgfpathlineto{\pgfqpoint{4.175195in}{1.396731in}}%
\pgfpathlineto{\pgfqpoint{4.166258in}{1.396731in}}%
\pgfpathlineto{\pgfqpoint{4.166258in}{1.497545in}}%
\pgfpathclose%
\pgfusepath{fill}%
\end{pgfscope}%
\begin{pgfscope}%
\pgfpathrectangle{\pgfqpoint{3.722897in}{0.857143in}}{\pgfqpoint{2.627103in}{1.813434in}}%
\pgfusepath{clip}%
\pgfsetbuttcap%
\pgfsetmiterjoin%
\definecolor{currentfill}{rgb}{0.754268,0.565033,0.211761}%
\pgfsetfillcolor{currentfill}%
\pgfsetlinewidth{0.000000pt}%
\definecolor{currentstroke}{rgb}{0.000000,0.000000,0.000000}%
\pgfsetstrokecolor{currentstroke}%
\pgfsetstrokeopacity{0.000000}%
\pgfsetdash{}{0pt}%
\pgfpathmoveto{\pgfqpoint{4.177429in}{1.496414in}}%
\pgfpathlineto{\pgfqpoint{4.186365in}{1.496414in}}%
\pgfpathlineto{\pgfqpoint{4.186365in}{1.390466in}}%
\pgfpathlineto{\pgfqpoint{4.177429in}{1.390466in}}%
\pgfpathlineto{\pgfqpoint{4.177429in}{1.496414in}}%
\pgfpathclose%
\pgfusepath{fill}%
\end{pgfscope}%
\begin{pgfscope}%
\pgfpathrectangle{\pgfqpoint{3.722897in}{0.857143in}}{\pgfqpoint{2.627103in}{1.813434in}}%
\pgfusepath{clip}%
\pgfsetbuttcap%
\pgfsetmiterjoin%
\definecolor{currentfill}{rgb}{0.754268,0.565033,0.211761}%
\pgfsetfillcolor{currentfill}%
\pgfsetlinewidth{0.000000pt}%
\definecolor{currentstroke}{rgb}{0.000000,0.000000,0.000000}%
\pgfsetstrokecolor{currentstroke}%
\pgfsetstrokeopacity{0.000000}%
\pgfsetdash{}{0pt}%
\pgfpathmoveto{\pgfqpoint{4.188600in}{1.476458in}}%
\pgfpathlineto{\pgfqpoint{4.197536in}{1.476458in}}%
\pgfpathlineto{\pgfqpoint{4.197536in}{1.367710in}}%
\pgfpathlineto{\pgfqpoint{4.188600in}{1.367710in}}%
\pgfpathlineto{\pgfqpoint{4.188600in}{1.476458in}}%
\pgfpathclose%
\pgfusepath{fill}%
\end{pgfscope}%
\begin{pgfscope}%
\pgfpathrectangle{\pgfqpoint{3.722897in}{0.857143in}}{\pgfqpoint{2.627103in}{1.813434in}}%
\pgfusepath{clip}%
\pgfsetbuttcap%
\pgfsetmiterjoin%
\definecolor{currentfill}{rgb}{0.754268,0.565033,0.211761}%
\pgfsetfillcolor{currentfill}%
\pgfsetlinewidth{0.000000pt}%
\definecolor{currentstroke}{rgb}{0.000000,0.000000,0.000000}%
\pgfsetstrokecolor{currentstroke}%
\pgfsetstrokeopacity{0.000000}%
\pgfsetdash{}{0pt}%
\pgfpathmoveto{\pgfqpoint{4.199770in}{1.479990in}}%
\pgfpathlineto{\pgfqpoint{4.208707in}{1.479990in}}%
\pgfpathlineto{\pgfqpoint{4.208707in}{1.357559in}}%
\pgfpathlineto{\pgfqpoint{4.199770in}{1.357559in}}%
\pgfpathlineto{\pgfqpoint{4.199770in}{1.479990in}}%
\pgfpathclose%
\pgfusepath{fill}%
\end{pgfscope}%
\begin{pgfscope}%
\pgfpathrectangle{\pgfqpoint{3.722897in}{0.857143in}}{\pgfqpoint{2.627103in}{1.813434in}}%
\pgfusepath{clip}%
\pgfsetbuttcap%
\pgfsetmiterjoin%
\definecolor{currentfill}{rgb}{0.754268,0.565033,0.211761}%
\pgfsetfillcolor{currentfill}%
\pgfsetlinewidth{0.000000pt}%
\definecolor{currentstroke}{rgb}{0.000000,0.000000,0.000000}%
\pgfsetstrokecolor{currentstroke}%
\pgfsetstrokeopacity{0.000000}%
\pgfsetdash{}{0pt}%
\pgfpathmoveto{\pgfqpoint{4.210941in}{1.469736in}}%
\pgfpathlineto{\pgfqpoint{4.219877in}{1.469736in}}%
\pgfpathlineto{\pgfqpoint{4.219877in}{1.329991in}}%
\pgfpathlineto{\pgfqpoint{4.210941in}{1.329991in}}%
\pgfpathlineto{\pgfqpoint{4.210941in}{1.469736in}}%
\pgfpathclose%
\pgfusepath{fill}%
\end{pgfscope}%
\begin{pgfscope}%
\pgfpathrectangle{\pgfqpoint{3.722897in}{0.857143in}}{\pgfqpoint{2.627103in}{1.813434in}}%
\pgfusepath{clip}%
\pgfsetbuttcap%
\pgfsetmiterjoin%
\definecolor{currentfill}{rgb}{0.754268,0.565033,0.211761}%
\pgfsetfillcolor{currentfill}%
\pgfsetlinewidth{0.000000pt}%
\definecolor{currentstroke}{rgb}{0.000000,0.000000,0.000000}%
\pgfsetstrokecolor{currentstroke}%
\pgfsetstrokeopacity{0.000000}%
\pgfsetdash{}{0pt}%
\pgfpathmoveto{\pgfqpoint{4.222111in}{1.455610in}}%
\pgfpathlineto{\pgfqpoint{4.231048in}{1.455610in}}%
\pgfpathlineto{\pgfqpoint{4.231048in}{1.282474in}}%
\pgfpathlineto{\pgfqpoint{4.222111in}{1.282474in}}%
\pgfpathlineto{\pgfqpoint{4.222111in}{1.455610in}}%
\pgfpathclose%
\pgfusepath{fill}%
\end{pgfscope}%
\begin{pgfscope}%
\pgfpathrectangle{\pgfqpoint{3.722897in}{0.857143in}}{\pgfqpoint{2.627103in}{1.813434in}}%
\pgfusepath{clip}%
\pgfsetbuttcap%
\pgfsetmiterjoin%
\definecolor{currentfill}{rgb}{0.754268,0.565033,0.211761}%
\pgfsetfillcolor{currentfill}%
\pgfsetlinewidth{0.000000pt}%
\definecolor{currentstroke}{rgb}{0.000000,0.000000,0.000000}%
\pgfsetstrokecolor{currentstroke}%
\pgfsetstrokeopacity{0.000000}%
\pgfsetdash{}{0pt}%
\pgfpathmoveto{\pgfqpoint{4.233282in}{1.470408in}}%
\pgfpathlineto{\pgfqpoint{4.242218in}{1.470408in}}%
\pgfpathlineto{\pgfqpoint{4.242218in}{1.271002in}}%
\pgfpathlineto{\pgfqpoint{4.233282in}{1.271002in}}%
\pgfpathlineto{\pgfqpoint{4.233282in}{1.470408in}}%
\pgfpathclose%
\pgfusepath{fill}%
\end{pgfscope}%
\begin{pgfscope}%
\pgfpathrectangle{\pgfqpoint{3.722897in}{0.857143in}}{\pgfqpoint{2.627103in}{1.813434in}}%
\pgfusepath{clip}%
\pgfsetbuttcap%
\pgfsetmiterjoin%
\definecolor{currentfill}{rgb}{0.754268,0.565033,0.211761}%
\pgfsetfillcolor{currentfill}%
\pgfsetlinewidth{0.000000pt}%
\definecolor{currentstroke}{rgb}{0.000000,0.000000,0.000000}%
\pgfsetstrokecolor{currentstroke}%
\pgfsetstrokeopacity{0.000000}%
\pgfsetdash{}{0pt}%
\pgfpathmoveto{\pgfqpoint{4.244453in}{1.461509in}}%
\pgfpathlineto{\pgfqpoint{4.253389in}{1.461509in}}%
\pgfpathlineto{\pgfqpoint{4.253389in}{1.255008in}}%
\pgfpathlineto{\pgfqpoint{4.244453in}{1.255008in}}%
\pgfpathlineto{\pgfqpoint{4.244453in}{1.461509in}}%
\pgfpathclose%
\pgfusepath{fill}%
\end{pgfscope}%
\begin{pgfscope}%
\pgfpathrectangle{\pgfqpoint{3.722897in}{0.857143in}}{\pgfqpoint{2.627103in}{1.813434in}}%
\pgfusepath{clip}%
\pgfsetbuttcap%
\pgfsetmiterjoin%
\definecolor{currentfill}{rgb}{0.754268,0.565033,0.211761}%
\pgfsetfillcolor{currentfill}%
\pgfsetlinewidth{0.000000pt}%
\definecolor{currentstroke}{rgb}{0.000000,0.000000,0.000000}%
\pgfsetstrokecolor{currentstroke}%
\pgfsetstrokeopacity{0.000000}%
\pgfsetdash{}{0pt}%
\pgfpathmoveto{\pgfqpoint{4.255623in}{1.473144in}}%
\pgfpathlineto{\pgfqpoint{4.264560in}{1.473144in}}%
\pgfpathlineto{\pgfqpoint{4.264560in}{1.253539in}}%
\pgfpathlineto{\pgfqpoint{4.255623in}{1.253539in}}%
\pgfpathlineto{\pgfqpoint{4.255623in}{1.473144in}}%
\pgfpathclose%
\pgfusepath{fill}%
\end{pgfscope}%
\begin{pgfscope}%
\pgfpathrectangle{\pgfqpoint{3.722897in}{0.857143in}}{\pgfqpoint{2.627103in}{1.813434in}}%
\pgfusepath{clip}%
\pgfsetbuttcap%
\pgfsetmiterjoin%
\definecolor{currentfill}{rgb}{0.754268,0.565033,0.211761}%
\pgfsetfillcolor{currentfill}%
\pgfsetlinewidth{0.000000pt}%
\definecolor{currentstroke}{rgb}{0.000000,0.000000,0.000000}%
\pgfsetstrokecolor{currentstroke}%
\pgfsetstrokeopacity{0.000000}%
\pgfsetdash{}{0pt}%
\pgfpathmoveto{\pgfqpoint{4.266794in}{1.501476in}}%
\pgfpathlineto{\pgfqpoint{4.275730in}{1.501476in}}%
\pgfpathlineto{\pgfqpoint{4.275730in}{1.270309in}}%
\pgfpathlineto{\pgfqpoint{4.266794in}{1.270309in}}%
\pgfpathlineto{\pgfqpoint{4.266794in}{1.501476in}}%
\pgfpathclose%
\pgfusepath{fill}%
\end{pgfscope}%
\begin{pgfscope}%
\pgfpathrectangle{\pgfqpoint{3.722897in}{0.857143in}}{\pgfqpoint{2.627103in}{1.813434in}}%
\pgfusepath{clip}%
\pgfsetbuttcap%
\pgfsetmiterjoin%
\definecolor{currentfill}{rgb}{0.754268,0.565033,0.211761}%
\pgfsetfillcolor{currentfill}%
\pgfsetlinewidth{0.000000pt}%
\definecolor{currentstroke}{rgb}{0.000000,0.000000,0.000000}%
\pgfsetstrokecolor{currentstroke}%
\pgfsetstrokeopacity{0.000000}%
\pgfsetdash{}{0pt}%
\pgfpathmoveto{\pgfqpoint{4.277964in}{1.508671in}}%
\pgfpathlineto{\pgfqpoint{4.286901in}{1.508671in}}%
\pgfpathlineto{\pgfqpoint{4.286901in}{1.282205in}}%
\pgfpathlineto{\pgfqpoint{4.277964in}{1.282205in}}%
\pgfpathlineto{\pgfqpoint{4.277964in}{1.508671in}}%
\pgfpathclose%
\pgfusepath{fill}%
\end{pgfscope}%
\begin{pgfscope}%
\pgfpathrectangle{\pgfqpoint{3.722897in}{0.857143in}}{\pgfqpoint{2.627103in}{1.813434in}}%
\pgfusepath{clip}%
\pgfsetbuttcap%
\pgfsetmiterjoin%
\definecolor{currentfill}{rgb}{0.754268,0.565033,0.211761}%
\pgfsetfillcolor{currentfill}%
\pgfsetlinewidth{0.000000pt}%
\definecolor{currentstroke}{rgb}{0.000000,0.000000,0.000000}%
\pgfsetstrokecolor{currentstroke}%
\pgfsetstrokeopacity{0.000000}%
\pgfsetdash{}{0pt}%
\pgfpathmoveto{\pgfqpoint{4.289135in}{1.505754in}}%
\pgfpathlineto{\pgfqpoint{4.298071in}{1.505754in}}%
\pgfpathlineto{\pgfqpoint{4.298071in}{1.285167in}}%
\pgfpathlineto{\pgfqpoint{4.289135in}{1.285167in}}%
\pgfpathlineto{\pgfqpoint{4.289135in}{1.505754in}}%
\pgfpathclose%
\pgfusepath{fill}%
\end{pgfscope}%
\begin{pgfscope}%
\pgfpathrectangle{\pgfqpoint{3.722897in}{0.857143in}}{\pgfqpoint{2.627103in}{1.813434in}}%
\pgfusepath{clip}%
\pgfsetbuttcap%
\pgfsetmiterjoin%
\definecolor{currentfill}{rgb}{0.754268,0.565033,0.211761}%
\pgfsetfillcolor{currentfill}%
\pgfsetlinewidth{0.000000pt}%
\definecolor{currentstroke}{rgb}{0.000000,0.000000,0.000000}%
\pgfsetstrokecolor{currentstroke}%
\pgfsetstrokeopacity{0.000000}%
\pgfsetdash{}{0pt}%
\pgfpathmoveto{\pgfqpoint{4.300306in}{1.496815in}}%
\pgfpathlineto{\pgfqpoint{4.309242in}{1.496815in}}%
\pgfpathlineto{\pgfqpoint{4.309242in}{1.275471in}}%
\pgfpathlineto{\pgfqpoint{4.300306in}{1.275471in}}%
\pgfpathlineto{\pgfqpoint{4.300306in}{1.496815in}}%
\pgfpathclose%
\pgfusepath{fill}%
\end{pgfscope}%
\begin{pgfscope}%
\pgfpathrectangle{\pgfqpoint{3.722897in}{0.857143in}}{\pgfqpoint{2.627103in}{1.813434in}}%
\pgfusepath{clip}%
\pgfsetbuttcap%
\pgfsetmiterjoin%
\definecolor{currentfill}{rgb}{0.754268,0.565033,0.211761}%
\pgfsetfillcolor{currentfill}%
\pgfsetlinewidth{0.000000pt}%
\definecolor{currentstroke}{rgb}{0.000000,0.000000,0.000000}%
\pgfsetstrokecolor{currentstroke}%
\pgfsetstrokeopacity{0.000000}%
\pgfsetdash{}{0pt}%
\pgfpathmoveto{\pgfqpoint{4.311476in}{1.510500in}}%
\pgfpathlineto{\pgfqpoint{4.320413in}{1.510500in}}%
\pgfpathlineto{\pgfqpoint{4.320413in}{1.269500in}}%
\pgfpathlineto{\pgfqpoint{4.311476in}{1.269500in}}%
\pgfpathlineto{\pgfqpoint{4.311476in}{1.510500in}}%
\pgfpathclose%
\pgfusepath{fill}%
\end{pgfscope}%
\begin{pgfscope}%
\pgfpathrectangle{\pgfqpoint{3.722897in}{0.857143in}}{\pgfqpoint{2.627103in}{1.813434in}}%
\pgfusepath{clip}%
\pgfsetbuttcap%
\pgfsetmiterjoin%
\definecolor{currentfill}{rgb}{0.754268,0.565033,0.211761}%
\pgfsetfillcolor{currentfill}%
\pgfsetlinewidth{0.000000pt}%
\definecolor{currentstroke}{rgb}{0.000000,0.000000,0.000000}%
\pgfsetstrokecolor{currentstroke}%
\pgfsetstrokeopacity{0.000000}%
\pgfsetdash{}{0pt}%
\pgfpathmoveto{\pgfqpoint{4.322647in}{1.526120in}}%
\pgfpathlineto{\pgfqpoint{4.331583in}{1.526120in}}%
\pgfpathlineto{\pgfqpoint{4.331583in}{1.270533in}}%
\pgfpathlineto{\pgfqpoint{4.322647in}{1.270533in}}%
\pgfpathlineto{\pgfqpoint{4.322647in}{1.526120in}}%
\pgfpathclose%
\pgfusepath{fill}%
\end{pgfscope}%
\begin{pgfscope}%
\pgfpathrectangle{\pgfqpoint{3.722897in}{0.857143in}}{\pgfqpoint{2.627103in}{1.813434in}}%
\pgfusepath{clip}%
\pgfsetbuttcap%
\pgfsetmiterjoin%
\definecolor{currentfill}{rgb}{0.754268,0.565033,0.211761}%
\pgfsetfillcolor{currentfill}%
\pgfsetlinewidth{0.000000pt}%
\definecolor{currentstroke}{rgb}{0.000000,0.000000,0.000000}%
\pgfsetstrokecolor{currentstroke}%
\pgfsetstrokeopacity{0.000000}%
\pgfsetdash{}{0pt}%
\pgfpathmoveto{\pgfqpoint{4.333817in}{1.538683in}}%
\pgfpathlineto{\pgfqpoint{4.342754in}{1.538683in}}%
\pgfpathlineto{\pgfqpoint{4.342754in}{1.276343in}}%
\pgfpathlineto{\pgfqpoint{4.333817in}{1.276343in}}%
\pgfpathlineto{\pgfqpoint{4.333817in}{1.538683in}}%
\pgfpathclose%
\pgfusepath{fill}%
\end{pgfscope}%
\begin{pgfscope}%
\pgfpathrectangle{\pgfqpoint{3.722897in}{0.857143in}}{\pgfqpoint{2.627103in}{1.813434in}}%
\pgfusepath{clip}%
\pgfsetbuttcap%
\pgfsetmiterjoin%
\definecolor{currentfill}{rgb}{0.754268,0.565033,0.211761}%
\pgfsetfillcolor{currentfill}%
\pgfsetlinewidth{0.000000pt}%
\definecolor{currentstroke}{rgb}{0.000000,0.000000,0.000000}%
\pgfsetstrokecolor{currentstroke}%
\pgfsetstrokeopacity{0.000000}%
\pgfsetdash{}{0pt}%
\pgfpathmoveto{\pgfqpoint{4.344988in}{1.511987in}}%
\pgfpathlineto{\pgfqpoint{4.353925in}{1.511987in}}%
\pgfpathlineto{\pgfqpoint{4.353925in}{1.255001in}}%
\pgfpathlineto{\pgfqpoint{4.344988in}{1.255001in}}%
\pgfpathlineto{\pgfqpoint{4.344988in}{1.511987in}}%
\pgfpathclose%
\pgfusepath{fill}%
\end{pgfscope}%
\begin{pgfscope}%
\pgfpathrectangle{\pgfqpoint{3.722897in}{0.857143in}}{\pgfqpoint{2.627103in}{1.813434in}}%
\pgfusepath{clip}%
\pgfsetbuttcap%
\pgfsetmiterjoin%
\definecolor{currentfill}{rgb}{0.754268,0.565033,0.211761}%
\pgfsetfillcolor{currentfill}%
\pgfsetlinewidth{0.000000pt}%
\definecolor{currentstroke}{rgb}{0.000000,0.000000,0.000000}%
\pgfsetstrokecolor{currentstroke}%
\pgfsetstrokeopacity{0.000000}%
\pgfsetdash{}{0pt}%
\pgfpathmoveto{\pgfqpoint{4.356159in}{1.550566in}}%
\pgfpathlineto{\pgfqpoint{4.365095in}{1.550566in}}%
\pgfpathlineto{\pgfqpoint{4.365095in}{1.271012in}}%
\pgfpathlineto{\pgfqpoint{4.356159in}{1.271012in}}%
\pgfpathlineto{\pgfqpoint{4.356159in}{1.550566in}}%
\pgfpathclose%
\pgfusepath{fill}%
\end{pgfscope}%
\begin{pgfscope}%
\pgfpathrectangle{\pgfqpoint{3.722897in}{0.857143in}}{\pgfqpoint{2.627103in}{1.813434in}}%
\pgfusepath{clip}%
\pgfsetbuttcap%
\pgfsetmiterjoin%
\definecolor{currentfill}{rgb}{0.754268,0.565033,0.211761}%
\pgfsetfillcolor{currentfill}%
\pgfsetlinewidth{0.000000pt}%
\definecolor{currentstroke}{rgb}{0.000000,0.000000,0.000000}%
\pgfsetstrokecolor{currentstroke}%
\pgfsetstrokeopacity{0.000000}%
\pgfsetdash{}{0pt}%
\pgfpathmoveto{\pgfqpoint{4.367329in}{1.546661in}}%
\pgfpathlineto{\pgfqpoint{4.376266in}{1.546661in}}%
\pgfpathlineto{\pgfqpoint{4.376266in}{1.272410in}}%
\pgfpathlineto{\pgfqpoint{4.367329in}{1.272410in}}%
\pgfpathlineto{\pgfqpoint{4.367329in}{1.546661in}}%
\pgfpathclose%
\pgfusepath{fill}%
\end{pgfscope}%
\begin{pgfscope}%
\pgfpathrectangle{\pgfqpoint{3.722897in}{0.857143in}}{\pgfqpoint{2.627103in}{1.813434in}}%
\pgfusepath{clip}%
\pgfsetbuttcap%
\pgfsetmiterjoin%
\definecolor{currentfill}{rgb}{0.754268,0.565033,0.211761}%
\pgfsetfillcolor{currentfill}%
\pgfsetlinewidth{0.000000pt}%
\definecolor{currentstroke}{rgb}{0.000000,0.000000,0.000000}%
\pgfsetstrokecolor{currentstroke}%
\pgfsetstrokeopacity{0.000000}%
\pgfsetdash{}{0pt}%
\pgfpathmoveto{\pgfqpoint{4.378500in}{1.559293in}}%
\pgfpathlineto{\pgfqpoint{4.387436in}{1.559293in}}%
\pgfpathlineto{\pgfqpoint{4.387436in}{1.274510in}}%
\pgfpathlineto{\pgfqpoint{4.378500in}{1.274510in}}%
\pgfpathlineto{\pgfqpoint{4.378500in}{1.559293in}}%
\pgfpathclose%
\pgfusepath{fill}%
\end{pgfscope}%
\begin{pgfscope}%
\pgfpathrectangle{\pgfqpoint{3.722897in}{0.857143in}}{\pgfqpoint{2.627103in}{1.813434in}}%
\pgfusepath{clip}%
\pgfsetbuttcap%
\pgfsetmiterjoin%
\definecolor{currentfill}{rgb}{0.754268,0.565033,0.211761}%
\pgfsetfillcolor{currentfill}%
\pgfsetlinewidth{0.000000pt}%
\definecolor{currentstroke}{rgb}{0.000000,0.000000,0.000000}%
\pgfsetstrokecolor{currentstroke}%
\pgfsetstrokeopacity{0.000000}%
\pgfsetdash{}{0pt}%
\pgfpathmoveto{\pgfqpoint{4.389670in}{1.560053in}}%
\pgfpathlineto{\pgfqpoint{4.398607in}{1.560053in}}%
\pgfpathlineto{\pgfqpoint{4.398607in}{1.278371in}}%
\pgfpathlineto{\pgfqpoint{4.389670in}{1.278371in}}%
\pgfpathlineto{\pgfqpoint{4.389670in}{1.560053in}}%
\pgfpathclose%
\pgfusepath{fill}%
\end{pgfscope}%
\begin{pgfscope}%
\pgfpathrectangle{\pgfqpoint{3.722897in}{0.857143in}}{\pgfqpoint{2.627103in}{1.813434in}}%
\pgfusepath{clip}%
\pgfsetbuttcap%
\pgfsetmiterjoin%
\definecolor{currentfill}{rgb}{0.754268,0.565033,0.211761}%
\pgfsetfillcolor{currentfill}%
\pgfsetlinewidth{0.000000pt}%
\definecolor{currentstroke}{rgb}{0.000000,0.000000,0.000000}%
\pgfsetstrokecolor{currentstroke}%
\pgfsetstrokeopacity{0.000000}%
\pgfsetdash{}{0pt}%
\pgfpathmoveto{\pgfqpoint{4.400841in}{1.580592in}}%
\pgfpathlineto{\pgfqpoint{4.409778in}{1.580592in}}%
\pgfpathlineto{\pgfqpoint{4.409778in}{1.295875in}}%
\pgfpathlineto{\pgfqpoint{4.400841in}{1.295875in}}%
\pgfpathlineto{\pgfqpoint{4.400841in}{1.580592in}}%
\pgfpathclose%
\pgfusepath{fill}%
\end{pgfscope}%
\begin{pgfscope}%
\pgfpathrectangle{\pgfqpoint{3.722897in}{0.857143in}}{\pgfqpoint{2.627103in}{1.813434in}}%
\pgfusepath{clip}%
\pgfsetbuttcap%
\pgfsetmiterjoin%
\definecolor{currentfill}{rgb}{0.754268,0.565033,0.211761}%
\pgfsetfillcolor{currentfill}%
\pgfsetlinewidth{0.000000pt}%
\definecolor{currentstroke}{rgb}{0.000000,0.000000,0.000000}%
\pgfsetstrokecolor{currentstroke}%
\pgfsetstrokeopacity{0.000000}%
\pgfsetdash{}{0pt}%
\pgfpathmoveto{\pgfqpoint{4.412012in}{1.557325in}}%
\pgfpathlineto{\pgfqpoint{4.420948in}{1.557325in}}%
\pgfpathlineto{\pgfqpoint{4.420948in}{1.284742in}}%
\pgfpathlineto{\pgfqpoint{4.412012in}{1.284742in}}%
\pgfpathlineto{\pgfqpoint{4.412012in}{1.557325in}}%
\pgfpathclose%
\pgfusepath{fill}%
\end{pgfscope}%
\begin{pgfscope}%
\pgfpathrectangle{\pgfqpoint{3.722897in}{0.857143in}}{\pgfqpoint{2.627103in}{1.813434in}}%
\pgfusepath{clip}%
\pgfsetbuttcap%
\pgfsetmiterjoin%
\definecolor{currentfill}{rgb}{0.754268,0.565033,0.211761}%
\pgfsetfillcolor{currentfill}%
\pgfsetlinewidth{0.000000pt}%
\definecolor{currentstroke}{rgb}{0.000000,0.000000,0.000000}%
\pgfsetstrokecolor{currentstroke}%
\pgfsetstrokeopacity{0.000000}%
\pgfsetdash{}{0pt}%
\pgfpathmoveto{\pgfqpoint{4.423182in}{1.591706in}}%
\pgfpathlineto{\pgfqpoint{4.432119in}{1.591706in}}%
\pgfpathlineto{\pgfqpoint{4.432119in}{1.308121in}}%
\pgfpathlineto{\pgfqpoint{4.423182in}{1.308121in}}%
\pgfpathlineto{\pgfqpoint{4.423182in}{1.591706in}}%
\pgfpathclose%
\pgfusepath{fill}%
\end{pgfscope}%
\begin{pgfscope}%
\pgfpathrectangle{\pgfqpoint{3.722897in}{0.857143in}}{\pgfqpoint{2.627103in}{1.813434in}}%
\pgfusepath{clip}%
\pgfsetbuttcap%
\pgfsetmiterjoin%
\definecolor{currentfill}{rgb}{0.754268,0.565033,0.211761}%
\pgfsetfillcolor{currentfill}%
\pgfsetlinewidth{0.000000pt}%
\definecolor{currentstroke}{rgb}{0.000000,0.000000,0.000000}%
\pgfsetstrokecolor{currentstroke}%
\pgfsetstrokeopacity{0.000000}%
\pgfsetdash{}{0pt}%
\pgfpathmoveto{\pgfqpoint{4.434353in}{1.612296in}}%
\pgfpathlineto{\pgfqpoint{4.443289in}{1.612296in}}%
\pgfpathlineto{\pgfqpoint{4.443289in}{1.328892in}}%
\pgfpathlineto{\pgfqpoint{4.434353in}{1.328892in}}%
\pgfpathlineto{\pgfqpoint{4.434353in}{1.612296in}}%
\pgfpathclose%
\pgfusepath{fill}%
\end{pgfscope}%
\begin{pgfscope}%
\pgfpathrectangle{\pgfqpoint{3.722897in}{0.857143in}}{\pgfqpoint{2.627103in}{1.813434in}}%
\pgfusepath{clip}%
\pgfsetbuttcap%
\pgfsetmiterjoin%
\definecolor{currentfill}{rgb}{0.754268,0.565033,0.211761}%
\pgfsetfillcolor{currentfill}%
\pgfsetlinewidth{0.000000pt}%
\definecolor{currentstroke}{rgb}{0.000000,0.000000,0.000000}%
\pgfsetstrokecolor{currentstroke}%
\pgfsetstrokeopacity{0.000000}%
\pgfsetdash{}{0pt}%
\pgfpathmoveto{\pgfqpoint{4.445523in}{1.619254in}}%
\pgfpathlineto{\pgfqpoint{4.454460in}{1.619254in}}%
\pgfpathlineto{\pgfqpoint{4.454460in}{1.366779in}}%
\pgfpathlineto{\pgfqpoint{4.445523in}{1.366779in}}%
\pgfpathlineto{\pgfqpoint{4.445523in}{1.619254in}}%
\pgfpathclose%
\pgfusepath{fill}%
\end{pgfscope}%
\begin{pgfscope}%
\pgfpathrectangle{\pgfqpoint{3.722897in}{0.857143in}}{\pgfqpoint{2.627103in}{1.813434in}}%
\pgfusepath{clip}%
\pgfsetbuttcap%
\pgfsetmiterjoin%
\definecolor{currentfill}{rgb}{0.754268,0.565033,0.211761}%
\pgfsetfillcolor{currentfill}%
\pgfsetlinewidth{0.000000pt}%
\definecolor{currentstroke}{rgb}{0.000000,0.000000,0.000000}%
\pgfsetstrokecolor{currentstroke}%
\pgfsetstrokeopacity{0.000000}%
\pgfsetdash{}{0pt}%
\pgfpathmoveto{\pgfqpoint{4.456694in}{1.635074in}}%
\pgfpathlineto{\pgfqpoint{4.465631in}{1.635074in}}%
\pgfpathlineto{\pgfqpoint{4.465631in}{1.402606in}}%
\pgfpathlineto{\pgfqpoint{4.456694in}{1.402606in}}%
\pgfpathlineto{\pgfqpoint{4.456694in}{1.635074in}}%
\pgfpathclose%
\pgfusepath{fill}%
\end{pgfscope}%
\begin{pgfscope}%
\pgfpathrectangle{\pgfqpoint{3.722897in}{0.857143in}}{\pgfqpoint{2.627103in}{1.813434in}}%
\pgfusepath{clip}%
\pgfsetbuttcap%
\pgfsetmiterjoin%
\definecolor{currentfill}{rgb}{0.754268,0.565033,0.211761}%
\pgfsetfillcolor{currentfill}%
\pgfsetlinewidth{0.000000pt}%
\definecolor{currentstroke}{rgb}{0.000000,0.000000,0.000000}%
\pgfsetstrokecolor{currentstroke}%
\pgfsetstrokeopacity{0.000000}%
\pgfsetdash{}{0pt}%
\pgfpathmoveto{\pgfqpoint{4.467865in}{1.632861in}}%
\pgfpathlineto{\pgfqpoint{4.476801in}{1.632861in}}%
\pgfpathlineto{\pgfqpoint{4.476801in}{1.433714in}}%
\pgfpathlineto{\pgfqpoint{4.467865in}{1.433714in}}%
\pgfpathlineto{\pgfqpoint{4.467865in}{1.632861in}}%
\pgfpathclose%
\pgfusepath{fill}%
\end{pgfscope}%
\begin{pgfscope}%
\pgfpathrectangle{\pgfqpoint{3.722897in}{0.857143in}}{\pgfqpoint{2.627103in}{1.813434in}}%
\pgfusepath{clip}%
\pgfsetbuttcap%
\pgfsetmiterjoin%
\definecolor{currentfill}{rgb}{0.754268,0.565033,0.211761}%
\pgfsetfillcolor{currentfill}%
\pgfsetlinewidth{0.000000pt}%
\definecolor{currentstroke}{rgb}{0.000000,0.000000,0.000000}%
\pgfsetstrokecolor{currentstroke}%
\pgfsetstrokeopacity{0.000000}%
\pgfsetdash{}{0pt}%
\pgfpathmoveto{\pgfqpoint{4.479035in}{1.630522in}}%
\pgfpathlineto{\pgfqpoint{4.487972in}{1.630522in}}%
\pgfpathlineto{\pgfqpoint{4.487972in}{1.417293in}}%
\pgfpathlineto{\pgfqpoint{4.479035in}{1.417293in}}%
\pgfpathlineto{\pgfqpoint{4.479035in}{1.630522in}}%
\pgfpathclose%
\pgfusepath{fill}%
\end{pgfscope}%
\begin{pgfscope}%
\pgfpathrectangle{\pgfqpoint{3.722897in}{0.857143in}}{\pgfqpoint{2.627103in}{1.813434in}}%
\pgfusepath{clip}%
\pgfsetbuttcap%
\pgfsetmiterjoin%
\definecolor{currentfill}{rgb}{0.754268,0.565033,0.211761}%
\pgfsetfillcolor{currentfill}%
\pgfsetlinewidth{0.000000pt}%
\definecolor{currentstroke}{rgb}{0.000000,0.000000,0.000000}%
\pgfsetstrokecolor{currentstroke}%
\pgfsetstrokeopacity{0.000000}%
\pgfsetdash{}{0pt}%
\pgfpathmoveto{\pgfqpoint{4.490206in}{1.652027in}}%
\pgfpathlineto{\pgfqpoint{4.499142in}{1.652027in}}%
\pgfpathlineto{\pgfqpoint{4.499142in}{1.433952in}}%
\pgfpathlineto{\pgfqpoint{4.490206in}{1.433952in}}%
\pgfpathlineto{\pgfqpoint{4.490206in}{1.652027in}}%
\pgfpathclose%
\pgfusepath{fill}%
\end{pgfscope}%
\begin{pgfscope}%
\pgfpathrectangle{\pgfqpoint{3.722897in}{0.857143in}}{\pgfqpoint{2.627103in}{1.813434in}}%
\pgfusepath{clip}%
\pgfsetbuttcap%
\pgfsetmiterjoin%
\definecolor{currentfill}{rgb}{0.754268,0.565033,0.211761}%
\pgfsetfillcolor{currentfill}%
\pgfsetlinewidth{0.000000pt}%
\definecolor{currentstroke}{rgb}{0.000000,0.000000,0.000000}%
\pgfsetstrokecolor{currentstroke}%
\pgfsetstrokeopacity{0.000000}%
\pgfsetdash{}{0pt}%
\pgfpathmoveto{\pgfqpoint{4.501377in}{1.692743in}}%
\pgfpathlineto{\pgfqpoint{4.510313in}{1.692743in}}%
\pgfpathlineto{\pgfqpoint{4.510313in}{1.505491in}}%
\pgfpathlineto{\pgfqpoint{4.501377in}{1.505491in}}%
\pgfpathlineto{\pgfqpoint{4.501377in}{1.692743in}}%
\pgfpathclose%
\pgfusepath{fill}%
\end{pgfscope}%
\begin{pgfscope}%
\pgfpathrectangle{\pgfqpoint{3.722897in}{0.857143in}}{\pgfqpoint{2.627103in}{1.813434in}}%
\pgfusepath{clip}%
\pgfsetbuttcap%
\pgfsetmiterjoin%
\definecolor{currentfill}{rgb}{0.754268,0.565033,0.211761}%
\pgfsetfillcolor{currentfill}%
\pgfsetlinewidth{0.000000pt}%
\definecolor{currentstroke}{rgb}{0.000000,0.000000,0.000000}%
\pgfsetstrokecolor{currentstroke}%
\pgfsetstrokeopacity{0.000000}%
\pgfsetdash{}{0pt}%
\pgfpathmoveto{\pgfqpoint{4.512547in}{1.696445in}}%
\pgfpathlineto{\pgfqpoint{4.521484in}{1.696445in}}%
\pgfpathlineto{\pgfqpoint{4.521484in}{1.539465in}}%
\pgfpathlineto{\pgfqpoint{4.512547in}{1.539465in}}%
\pgfpathlineto{\pgfqpoint{4.512547in}{1.696445in}}%
\pgfpathclose%
\pgfusepath{fill}%
\end{pgfscope}%
\begin{pgfscope}%
\pgfpathrectangle{\pgfqpoint{3.722897in}{0.857143in}}{\pgfqpoint{2.627103in}{1.813434in}}%
\pgfusepath{clip}%
\pgfsetbuttcap%
\pgfsetmiterjoin%
\definecolor{currentfill}{rgb}{0.754268,0.565033,0.211761}%
\pgfsetfillcolor{currentfill}%
\pgfsetlinewidth{0.000000pt}%
\definecolor{currentstroke}{rgb}{0.000000,0.000000,0.000000}%
\pgfsetstrokecolor{currentstroke}%
\pgfsetstrokeopacity{0.000000}%
\pgfsetdash{}{0pt}%
\pgfpathmoveto{\pgfqpoint{4.523718in}{1.710502in}}%
\pgfpathlineto{\pgfqpoint{4.532654in}{1.710502in}}%
\pgfpathlineto{\pgfqpoint{4.532654in}{1.605312in}}%
\pgfpathlineto{\pgfqpoint{4.523718in}{1.605312in}}%
\pgfpathlineto{\pgfqpoint{4.523718in}{1.710502in}}%
\pgfpathclose%
\pgfusepath{fill}%
\end{pgfscope}%
\begin{pgfscope}%
\pgfpathrectangle{\pgfqpoint{3.722897in}{0.857143in}}{\pgfqpoint{2.627103in}{1.813434in}}%
\pgfusepath{clip}%
\pgfsetbuttcap%
\pgfsetmiterjoin%
\definecolor{currentfill}{rgb}{0.754268,0.565033,0.211761}%
\pgfsetfillcolor{currentfill}%
\pgfsetlinewidth{0.000000pt}%
\definecolor{currentstroke}{rgb}{0.000000,0.000000,0.000000}%
\pgfsetstrokecolor{currentstroke}%
\pgfsetstrokeopacity{0.000000}%
\pgfsetdash{}{0pt}%
\pgfpathmoveto{\pgfqpoint{4.534888in}{1.717026in}}%
\pgfpathlineto{\pgfqpoint{4.543825in}{1.717026in}}%
\pgfpathlineto{\pgfqpoint{4.543825in}{1.642486in}}%
\pgfpathlineto{\pgfqpoint{4.534888in}{1.642486in}}%
\pgfpathlineto{\pgfqpoint{4.534888in}{1.717026in}}%
\pgfpathclose%
\pgfusepath{fill}%
\end{pgfscope}%
\begin{pgfscope}%
\pgfpathrectangle{\pgfqpoint{3.722897in}{0.857143in}}{\pgfqpoint{2.627103in}{1.813434in}}%
\pgfusepath{clip}%
\pgfsetbuttcap%
\pgfsetmiterjoin%
\definecolor{currentfill}{rgb}{0.754268,0.565033,0.211761}%
\pgfsetfillcolor{currentfill}%
\pgfsetlinewidth{0.000000pt}%
\definecolor{currentstroke}{rgb}{0.000000,0.000000,0.000000}%
\pgfsetstrokecolor{currentstroke}%
\pgfsetstrokeopacity{0.000000}%
\pgfsetdash{}{0pt}%
\pgfpathmoveto{\pgfqpoint{4.546059in}{1.707682in}}%
\pgfpathlineto{\pgfqpoint{4.554995in}{1.707682in}}%
\pgfpathlineto{\pgfqpoint{4.554995in}{1.662950in}}%
\pgfpathlineto{\pgfqpoint{4.546059in}{1.662950in}}%
\pgfpathlineto{\pgfqpoint{4.546059in}{1.707682in}}%
\pgfpathclose%
\pgfusepath{fill}%
\end{pgfscope}%
\begin{pgfscope}%
\pgfpathrectangle{\pgfqpoint{3.722897in}{0.857143in}}{\pgfqpoint{2.627103in}{1.813434in}}%
\pgfusepath{clip}%
\pgfsetbuttcap%
\pgfsetmiterjoin%
\definecolor{currentfill}{rgb}{0.754268,0.565033,0.211761}%
\pgfsetfillcolor{currentfill}%
\pgfsetlinewidth{0.000000pt}%
\definecolor{currentstroke}{rgb}{0.000000,0.000000,0.000000}%
\pgfsetstrokecolor{currentstroke}%
\pgfsetstrokeopacity{0.000000}%
\pgfsetdash{}{0pt}%
\pgfpathmoveto{\pgfqpoint{4.557230in}{1.707262in}}%
\pgfpathlineto{\pgfqpoint{4.566166in}{1.707262in}}%
\pgfpathlineto{\pgfqpoint{4.566166in}{1.704103in}}%
\pgfpathlineto{\pgfqpoint{4.557230in}{1.704103in}}%
\pgfpathlineto{\pgfqpoint{4.557230in}{1.707262in}}%
\pgfpathclose%
\pgfusepath{fill}%
\end{pgfscope}%
\begin{pgfscope}%
\pgfpathrectangle{\pgfqpoint{3.722897in}{0.857143in}}{\pgfqpoint{2.627103in}{1.813434in}}%
\pgfusepath{clip}%
\pgfsetbuttcap%
\pgfsetmiterjoin%
\definecolor{currentfill}{rgb}{0.754268,0.565033,0.211761}%
\pgfsetfillcolor{currentfill}%
\pgfsetlinewidth{0.000000pt}%
\definecolor{currentstroke}{rgb}{0.000000,0.000000,0.000000}%
\pgfsetstrokecolor{currentstroke}%
\pgfsetstrokeopacity{0.000000}%
\pgfsetdash{}{0pt}%
\pgfpathmoveto{\pgfqpoint{4.568400in}{2.083574in}}%
\pgfpathlineto{\pgfqpoint{4.577337in}{2.083574in}}%
\pgfpathlineto{\pgfqpoint{4.577337in}{2.101856in}}%
\pgfpathlineto{\pgfqpoint{4.568400in}{2.101856in}}%
\pgfpathlineto{\pgfqpoint{4.568400in}{2.083574in}}%
\pgfpathclose%
\pgfusepath{fill}%
\end{pgfscope}%
\begin{pgfscope}%
\pgfpathrectangle{\pgfqpoint{3.722897in}{0.857143in}}{\pgfqpoint{2.627103in}{1.813434in}}%
\pgfusepath{clip}%
\pgfsetbuttcap%
\pgfsetmiterjoin%
\definecolor{currentfill}{rgb}{0.754268,0.565033,0.211761}%
\pgfsetfillcolor{currentfill}%
\pgfsetlinewidth{0.000000pt}%
\definecolor{currentstroke}{rgb}{0.000000,0.000000,0.000000}%
\pgfsetstrokecolor{currentstroke}%
\pgfsetstrokeopacity{0.000000}%
\pgfsetdash{}{0pt}%
\pgfpathmoveto{\pgfqpoint{4.579571in}{2.051571in}}%
\pgfpathlineto{\pgfqpoint{4.588507in}{2.051571in}}%
\pgfpathlineto{\pgfqpoint{4.588507in}{2.086581in}}%
\pgfpathlineto{\pgfqpoint{4.579571in}{2.086581in}}%
\pgfpathlineto{\pgfqpoint{4.579571in}{2.051571in}}%
\pgfpathclose%
\pgfusepath{fill}%
\end{pgfscope}%
\begin{pgfscope}%
\pgfpathrectangle{\pgfqpoint{3.722897in}{0.857143in}}{\pgfqpoint{2.627103in}{1.813434in}}%
\pgfusepath{clip}%
\pgfsetbuttcap%
\pgfsetmiterjoin%
\definecolor{currentfill}{rgb}{0.754268,0.565033,0.211761}%
\pgfsetfillcolor{currentfill}%
\pgfsetlinewidth{0.000000pt}%
\definecolor{currentstroke}{rgb}{0.000000,0.000000,0.000000}%
\pgfsetstrokecolor{currentstroke}%
\pgfsetstrokeopacity{0.000000}%
\pgfsetdash{}{0pt}%
\pgfpathmoveto{\pgfqpoint{4.590741in}{2.023717in}}%
\pgfpathlineto{\pgfqpoint{4.599678in}{2.023717in}}%
\pgfpathlineto{\pgfqpoint{4.599678in}{2.078342in}}%
\pgfpathlineto{\pgfqpoint{4.590741in}{2.078342in}}%
\pgfpathlineto{\pgfqpoint{4.590741in}{2.023717in}}%
\pgfpathclose%
\pgfusepath{fill}%
\end{pgfscope}%
\begin{pgfscope}%
\pgfpathrectangle{\pgfqpoint{3.722897in}{0.857143in}}{\pgfqpoint{2.627103in}{1.813434in}}%
\pgfusepath{clip}%
\pgfsetbuttcap%
\pgfsetmiterjoin%
\definecolor{currentfill}{rgb}{0.754268,0.565033,0.211761}%
\pgfsetfillcolor{currentfill}%
\pgfsetlinewidth{0.000000pt}%
\definecolor{currentstroke}{rgb}{0.000000,0.000000,0.000000}%
\pgfsetstrokecolor{currentstroke}%
\pgfsetstrokeopacity{0.000000}%
\pgfsetdash{}{0pt}%
\pgfpathmoveto{\pgfqpoint{4.601912in}{2.001059in}}%
\pgfpathlineto{\pgfqpoint{4.610848in}{2.001059in}}%
\pgfpathlineto{\pgfqpoint{4.610848in}{2.075723in}}%
\pgfpathlineto{\pgfqpoint{4.601912in}{2.075723in}}%
\pgfpathlineto{\pgfqpoint{4.601912in}{2.001059in}}%
\pgfpathclose%
\pgfusepath{fill}%
\end{pgfscope}%
\begin{pgfscope}%
\pgfpathrectangle{\pgfqpoint{3.722897in}{0.857143in}}{\pgfqpoint{2.627103in}{1.813434in}}%
\pgfusepath{clip}%
\pgfsetbuttcap%
\pgfsetmiterjoin%
\definecolor{currentfill}{rgb}{0.754268,0.565033,0.211761}%
\pgfsetfillcolor{currentfill}%
\pgfsetlinewidth{0.000000pt}%
\definecolor{currentstroke}{rgb}{0.000000,0.000000,0.000000}%
\pgfsetstrokecolor{currentstroke}%
\pgfsetstrokeopacity{0.000000}%
\pgfsetdash{}{0pt}%
\pgfpathmoveto{\pgfqpoint{4.613083in}{1.980269in}}%
\pgfpathlineto{\pgfqpoint{4.622019in}{1.980269in}}%
\pgfpathlineto{\pgfqpoint{4.622019in}{2.065519in}}%
\pgfpathlineto{\pgfqpoint{4.613083in}{2.065519in}}%
\pgfpathlineto{\pgfqpoint{4.613083in}{1.980269in}}%
\pgfpathclose%
\pgfusepath{fill}%
\end{pgfscope}%
\begin{pgfscope}%
\pgfpathrectangle{\pgfqpoint{3.722897in}{0.857143in}}{\pgfqpoint{2.627103in}{1.813434in}}%
\pgfusepath{clip}%
\pgfsetbuttcap%
\pgfsetmiterjoin%
\definecolor{currentfill}{rgb}{0.754268,0.565033,0.211761}%
\pgfsetfillcolor{currentfill}%
\pgfsetlinewidth{0.000000pt}%
\definecolor{currentstroke}{rgb}{0.000000,0.000000,0.000000}%
\pgfsetstrokecolor{currentstroke}%
\pgfsetstrokeopacity{0.000000}%
\pgfsetdash{}{0pt}%
\pgfpathmoveto{\pgfqpoint{4.624253in}{1.968072in}}%
\pgfpathlineto{\pgfqpoint{4.633190in}{1.968072in}}%
\pgfpathlineto{\pgfqpoint{4.633190in}{2.075022in}}%
\pgfpathlineto{\pgfqpoint{4.624253in}{2.075022in}}%
\pgfpathlineto{\pgfqpoint{4.624253in}{1.968072in}}%
\pgfpathclose%
\pgfusepath{fill}%
\end{pgfscope}%
\begin{pgfscope}%
\pgfpathrectangle{\pgfqpoint{3.722897in}{0.857143in}}{\pgfqpoint{2.627103in}{1.813434in}}%
\pgfusepath{clip}%
\pgfsetbuttcap%
\pgfsetmiterjoin%
\definecolor{currentfill}{rgb}{0.754268,0.565033,0.211761}%
\pgfsetfillcolor{currentfill}%
\pgfsetlinewidth{0.000000pt}%
\definecolor{currentstroke}{rgb}{0.000000,0.000000,0.000000}%
\pgfsetstrokecolor{currentstroke}%
\pgfsetstrokeopacity{0.000000}%
\pgfsetdash{}{0pt}%
\pgfpathmoveto{\pgfqpoint{4.635424in}{1.966220in}}%
\pgfpathlineto{\pgfqpoint{4.644360in}{1.966220in}}%
\pgfpathlineto{\pgfqpoint{4.644360in}{2.084719in}}%
\pgfpathlineto{\pgfqpoint{4.635424in}{2.084719in}}%
\pgfpathlineto{\pgfqpoint{4.635424in}{1.966220in}}%
\pgfpathclose%
\pgfusepath{fill}%
\end{pgfscope}%
\begin{pgfscope}%
\pgfpathrectangle{\pgfqpoint{3.722897in}{0.857143in}}{\pgfqpoint{2.627103in}{1.813434in}}%
\pgfusepath{clip}%
\pgfsetbuttcap%
\pgfsetmiterjoin%
\definecolor{currentfill}{rgb}{0.754268,0.565033,0.211761}%
\pgfsetfillcolor{currentfill}%
\pgfsetlinewidth{0.000000pt}%
\definecolor{currentstroke}{rgb}{0.000000,0.000000,0.000000}%
\pgfsetstrokecolor{currentstroke}%
\pgfsetstrokeopacity{0.000000}%
\pgfsetdash{}{0pt}%
\pgfpathmoveto{\pgfqpoint{4.646594in}{1.976022in}}%
\pgfpathlineto{\pgfqpoint{4.655531in}{1.976022in}}%
\pgfpathlineto{\pgfqpoint{4.655531in}{2.115037in}}%
\pgfpathlineto{\pgfqpoint{4.646594in}{2.115037in}}%
\pgfpathlineto{\pgfqpoint{4.646594in}{1.976022in}}%
\pgfpathclose%
\pgfusepath{fill}%
\end{pgfscope}%
\begin{pgfscope}%
\pgfpathrectangle{\pgfqpoint{3.722897in}{0.857143in}}{\pgfqpoint{2.627103in}{1.813434in}}%
\pgfusepath{clip}%
\pgfsetbuttcap%
\pgfsetmiterjoin%
\definecolor{currentfill}{rgb}{0.754268,0.565033,0.211761}%
\pgfsetfillcolor{currentfill}%
\pgfsetlinewidth{0.000000pt}%
\definecolor{currentstroke}{rgb}{0.000000,0.000000,0.000000}%
\pgfsetstrokecolor{currentstroke}%
\pgfsetstrokeopacity{0.000000}%
\pgfsetdash{}{0pt}%
\pgfpathmoveto{\pgfqpoint{4.657765in}{1.990167in}}%
\pgfpathlineto{\pgfqpoint{4.666701in}{1.990167in}}%
\pgfpathlineto{\pgfqpoint{4.666701in}{2.152837in}}%
\pgfpathlineto{\pgfqpoint{4.657765in}{2.152837in}}%
\pgfpathlineto{\pgfqpoint{4.657765in}{1.990167in}}%
\pgfpathclose%
\pgfusepath{fill}%
\end{pgfscope}%
\begin{pgfscope}%
\pgfpathrectangle{\pgfqpoint{3.722897in}{0.857143in}}{\pgfqpoint{2.627103in}{1.813434in}}%
\pgfusepath{clip}%
\pgfsetbuttcap%
\pgfsetmiterjoin%
\definecolor{currentfill}{rgb}{0.754268,0.565033,0.211761}%
\pgfsetfillcolor{currentfill}%
\pgfsetlinewidth{0.000000pt}%
\definecolor{currentstroke}{rgb}{0.000000,0.000000,0.000000}%
\pgfsetstrokecolor{currentstroke}%
\pgfsetstrokeopacity{0.000000}%
\pgfsetdash{}{0pt}%
\pgfpathmoveto{\pgfqpoint{4.668936in}{1.997748in}}%
\pgfpathlineto{\pgfqpoint{4.677872in}{1.997748in}}%
\pgfpathlineto{\pgfqpoint{4.677872in}{2.181097in}}%
\pgfpathlineto{\pgfqpoint{4.668936in}{2.181097in}}%
\pgfpathlineto{\pgfqpoint{4.668936in}{1.997748in}}%
\pgfpathclose%
\pgfusepath{fill}%
\end{pgfscope}%
\begin{pgfscope}%
\pgfpathrectangle{\pgfqpoint{3.722897in}{0.857143in}}{\pgfqpoint{2.627103in}{1.813434in}}%
\pgfusepath{clip}%
\pgfsetbuttcap%
\pgfsetmiterjoin%
\definecolor{currentfill}{rgb}{0.754268,0.565033,0.211761}%
\pgfsetfillcolor{currentfill}%
\pgfsetlinewidth{0.000000pt}%
\definecolor{currentstroke}{rgb}{0.000000,0.000000,0.000000}%
\pgfsetstrokecolor{currentstroke}%
\pgfsetstrokeopacity{0.000000}%
\pgfsetdash{}{0pt}%
\pgfpathmoveto{\pgfqpoint{4.680106in}{2.002176in}}%
\pgfpathlineto{\pgfqpoint{4.689043in}{2.002176in}}%
\pgfpathlineto{\pgfqpoint{4.689043in}{2.186693in}}%
\pgfpathlineto{\pgfqpoint{4.680106in}{2.186693in}}%
\pgfpathlineto{\pgfqpoint{4.680106in}{2.002176in}}%
\pgfpathclose%
\pgfusepath{fill}%
\end{pgfscope}%
\begin{pgfscope}%
\pgfpathrectangle{\pgfqpoint{3.722897in}{0.857143in}}{\pgfqpoint{2.627103in}{1.813434in}}%
\pgfusepath{clip}%
\pgfsetbuttcap%
\pgfsetmiterjoin%
\definecolor{currentfill}{rgb}{0.754268,0.565033,0.211761}%
\pgfsetfillcolor{currentfill}%
\pgfsetlinewidth{0.000000pt}%
\definecolor{currentstroke}{rgb}{0.000000,0.000000,0.000000}%
\pgfsetstrokecolor{currentstroke}%
\pgfsetstrokeopacity{0.000000}%
\pgfsetdash{}{0pt}%
\pgfpathmoveto{\pgfqpoint{4.691277in}{2.013253in}}%
\pgfpathlineto{\pgfqpoint{4.700213in}{2.013253in}}%
\pgfpathlineto{\pgfqpoint{4.700213in}{2.210695in}}%
\pgfpathlineto{\pgfqpoint{4.691277in}{2.210695in}}%
\pgfpathlineto{\pgfqpoint{4.691277in}{2.013253in}}%
\pgfpathclose%
\pgfusepath{fill}%
\end{pgfscope}%
\begin{pgfscope}%
\pgfpathrectangle{\pgfqpoint{3.722897in}{0.857143in}}{\pgfqpoint{2.627103in}{1.813434in}}%
\pgfusepath{clip}%
\pgfsetbuttcap%
\pgfsetmiterjoin%
\definecolor{currentfill}{rgb}{0.754268,0.565033,0.211761}%
\pgfsetfillcolor{currentfill}%
\pgfsetlinewidth{0.000000pt}%
\definecolor{currentstroke}{rgb}{0.000000,0.000000,0.000000}%
\pgfsetstrokecolor{currentstroke}%
\pgfsetstrokeopacity{0.000000}%
\pgfsetdash{}{0pt}%
\pgfpathmoveto{\pgfqpoint{4.702447in}{2.024862in}}%
\pgfpathlineto{\pgfqpoint{4.711384in}{2.024862in}}%
\pgfpathlineto{\pgfqpoint{4.711384in}{2.233450in}}%
\pgfpathlineto{\pgfqpoint{4.702447in}{2.233450in}}%
\pgfpathlineto{\pgfqpoint{4.702447in}{2.024862in}}%
\pgfpathclose%
\pgfusepath{fill}%
\end{pgfscope}%
\begin{pgfscope}%
\pgfpathrectangle{\pgfqpoint{3.722897in}{0.857143in}}{\pgfqpoint{2.627103in}{1.813434in}}%
\pgfusepath{clip}%
\pgfsetbuttcap%
\pgfsetmiterjoin%
\definecolor{currentfill}{rgb}{0.754268,0.565033,0.211761}%
\pgfsetfillcolor{currentfill}%
\pgfsetlinewidth{0.000000pt}%
\definecolor{currentstroke}{rgb}{0.000000,0.000000,0.000000}%
\pgfsetstrokecolor{currentstroke}%
\pgfsetstrokeopacity{0.000000}%
\pgfsetdash{}{0pt}%
\pgfpathmoveto{\pgfqpoint{4.713618in}{2.030854in}}%
\pgfpathlineto{\pgfqpoint{4.722554in}{2.030854in}}%
\pgfpathlineto{\pgfqpoint{4.722554in}{2.252026in}}%
\pgfpathlineto{\pgfqpoint{4.713618in}{2.252026in}}%
\pgfpathlineto{\pgfqpoint{4.713618in}{2.030854in}}%
\pgfpathclose%
\pgfusepath{fill}%
\end{pgfscope}%
\begin{pgfscope}%
\pgfpathrectangle{\pgfqpoint{3.722897in}{0.857143in}}{\pgfqpoint{2.627103in}{1.813434in}}%
\pgfusepath{clip}%
\pgfsetbuttcap%
\pgfsetmiterjoin%
\definecolor{currentfill}{rgb}{0.754268,0.565033,0.211761}%
\pgfsetfillcolor{currentfill}%
\pgfsetlinewidth{0.000000pt}%
\definecolor{currentstroke}{rgb}{0.000000,0.000000,0.000000}%
\pgfsetstrokecolor{currentstroke}%
\pgfsetstrokeopacity{0.000000}%
\pgfsetdash{}{0pt}%
\pgfpathmoveto{\pgfqpoint{4.724789in}{2.031183in}}%
\pgfpathlineto{\pgfqpoint{4.733725in}{2.031183in}}%
\pgfpathlineto{\pgfqpoint{4.733725in}{2.267328in}}%
\pgfpathlineto{\pgfqpoint{4.724789in}{2.267328in}}%
\pgfpathlineto{\pgfqpoint{4.724789in}{2.031183in}}%
\pgfpathclose%
\pgfusepath{fill}%
\end{pgfscope}%
\begin{pgfscope}%
\pgfpathrectangle{\pgfqpoint{3.722897in}{0.857143in}}{\pgfqpoint{2.627103in}{1.813434in}}%
\pgfusepath{clip}%
\pgfsetbuttcap%
\pgfsetmiterjoin%
\definecolor{currentfill}{rgb}{0.754268,0.565033,0.211761}%
\pgfsetfillcolor{currentfill}%
\pgfsetlinewidth{0.000000pt}%
\definecolor{currentstroke}{rgb}{0.000000,0.000000,0.000000}%
\pgfsetstrokecolor{currentstroke}%
\pgfsetstrokeopacity{0.000000}%
\pgfsetdash{}{0pt}%
\pgfpathmoveto{\pgfqpoint{4.735959in}{2.035513in}}%
\pgfpathlineto{\pgfqpoint{4.744896in}{2.035513in}}%
\pgfpathlineto{\pgfqpoint{4.744896in}{2.283348in}}%
\pgfpathlineto{\pgfqpoint{4.735959in}{2.283348in}}%
\pgfpathlineto{\pgfqpoint{4.735959in}{2.035513in}}%
\pgfpathclose%
\pgfusepath{fill}%
\end{pgfscope}%
\begin{pgfscope}%
\pgfpathrectangle{\pgfqpoint{3.722897in}{0.857143in}}{\pgfqpoint{2.627103in}{1.813434in}}%
\pgfusepath{clip}%
\pgfsetbuttcap%
\pgfsetmiterjoin%
\definecolor{currentfill}{rgb}{0.754268,0.565033,0.211761}%
\pgfsetfillcolor{currentfill}%
\pgfsetlinewidth{0.000000pt}%
\definecolor{currentstroke}{rgb}{0.000000,0.000000,0.000000}%
\pgfsetstrokecolor{currentstroke}%
\pgfsetstrokeopacity{0.000000}%
\pgfsetdash{}{0pt}%
\pgfpathmoveto{\pgfqpoint{4.747130in}{2.035480in}}%
\pgfpathlineto{\pgfqpoint{4.756066in}{2.035480in}}%
\pgfpathlineto{\pgfqpoint{4.756066in}{2.291178in}}%
\pgfpathlineto{\pgfqpoint{4.747130in}{2.291178in}}%
\pgfpathlineto{\pgfqpoint{4.747130in}{2.035480in}}%
\pgfpathclose%
\pgfusepath{fill}%
\end{pgfscope}%
\begin{pgfscope}%
\pgfpathrectangle{\pgfqpoint{3.722897in}{0.857143in}}{\pgfqpoint{2.627103in}{1.813434in}}%
\pgfusepath{clip}%
\pgfsetbuttcap%
\pgfsetmiterjoin%
\definecolor{currentfill}{rgb}{0.754268,0.565033,0.211761}%
\pgfsetfillcolor{currentfill}%
\pgfsetlinewidth{0.000000pt}%
\definecolor{currentstroke}{rgb}{0.000000,0.000000,0.000000}%
\pgfsetstrokecolor{currentstroke}%
\pgfsetstrokeopacity{0.000000}%
\pgfsetdash{}{0pt}%
\pgfpathmoveto{\pgfqpoint{4.758300in}{2.037129in}}%
\pgfpathlineto{\pgfqpoint{4.767237in}{2.037129in}}%
\pgfpathlineto{\pgfqpoint{4.767237in}{2.291519in}}%
\pgfpathlineto{\pgfqpoint{4.758300in}{2.291519in}}%
\pgfpathlineto{\pgfqpoint{4.758300in}{2.037129in}}%
\pgfpathclose%
\pgfusepath{fill}%
\end{pgfscope}%
\begin{pgfscope}%
\pgfpathrectangle{\pgfqpoint{3.722897in}{0.857143in}}{\pgfqpoint{2.627103in}{1.813434in}}%
\pgfusepath{clip}%
\pgfsetbuttcap%
\pgfsetmiterjoin%
\definecolor{currentfill}{rgb}{0.754268,0.565033,0.211761}%
\pgfsetfillcolor{currentfill}%
\pgfsetlinewidth{0.000000pt}%
\definecolor{currentstroke}{rgb}{0.000000,0.000000,0.000000}%
\pgfsetstrokecolor{currentstroke}%
\pgfsetstrokeopacity{0.000000}%
\pgfsetdash{}{0pt}%
\pgfpathmoveto{\pgfqpoint{4.769471in}{2.043557in}}%
\pgfpathlineto{\pgfqpoint{4.778408in}{2.043557in}}%
\pgfpathlineto{\pgfqpoint{4.778408in}{2.297081in}}%
\pgfpathlineto{\pgfqpoint{4.769471in}{2.297081in}}%
\pgfpathlineto{\pgfqpoint{4.769471in}{2.043557in}}%
\pgfpathclose%
\pgfusepath{fill}%
\end{pgfscope}%
\begin{pgfscope}%
\pgfpathrectangle{\pgfqpoint{3.722897in}{0.857143in}}{\pgfqpoint{2.627103in}{1.813434in}}%
\pgfusepath{clip}%
\pgfsetbuttcap%
\pgfsetmiterjoin%
\definecolor{currentfill}{rgb}{0.754268,0.565033,0.211761}%
\pgfsetfillcolor{currentfill}%
\pgfsetlinewidth{0.000000pt}%
\definecolor{currentstroke}{rgb}{0.000000,0.000000,0.000000}%
\pgfsetstrokecolor{currentstroke}%
\pgfsetstrokeopacity{0.000000}%
\pgfsetdash{}{0pt}%
\pgfpathmoveto{\pgfqpoint{4.780642in}{2.039609in}}%
\pgfpathlineto{\pgfqpoint{4.789578in}{2.039609in}}%
\pgfpathlineto{\pgfqpoint{4.789578in}{2.290251in}}%
\pgfpathlineto{\pgfqpoint{4.780642in}{2.290251in}}%
\pgfpathlineto{\pgfqpoint{4.780642in}{2.039609in}}%
\pgfpathclose%
\pgfusepath{fill}%
\end{pgfscope}%
\begin{pgfscope}%
\pgfpathrectangle{\pgfqpoint{3.722897in}{0.857143in}}{\pgfqpoint{2.627103in}{1.813434in}}%
\pgfusepath{clip}%
\pgfsetbuttcap%
\pgfsetmiterjoin%
\definecolor{currentfill}{rgb}{0.754268,0.565033,0.211761}%
\pgfsetfillcolor{currentfill}%
\pgfsetlinewidth{0.000000pt}%
\definecolor{currentstroke}{rgb}{0.000000,0.000000,0.000000}%
\pgfsetstrokecolor{currentstroke}%
\pgfsetstrokeopacity{0.000000}%
\pgfsetdash{}{0pt}%
\pgfpathmoveto{\pgfqpoint{4.791812in}{2.041370in}}%
\pgfpathlineto{\pgfqpoint{4.800749in}{2.041370in}}%
\pgfpathlineto{\pgfqpoint{4.800749in}{2.288209in}}%
\pgfpathlineto{\pgfqpoint{4.791812in}{2.288209in}}%
\pgfpathlineto{\pgfqpoint{4.791812in}{2.041370in}}%
\pgfpathclose%
\pgfusepath{fill}%
\end{pgfscope}%
\begin{pgfscope}%
\pgfpathrectangle{\pgfqpoint{3.722897in}{0.857143in}}{\pgfqpoint{2.627103in}{1.813434in}}%
\pgfusepath{clip}%
\pgfsetbuttcap%
\pgfsetmiterjoin%
\definecolor{currentfill}{rgb}{0.754268,0.565033,0.211761}%
\pgfsetfillcolor{currentfill}%
\pgfsetlinewidth{0.000000pt}%
\definecolor{currentstroke}{rgb}{0.000000,0.000000,0.000000}%
\pgfsetstrokecolor{currentstroke}%
\pgfsetstrokeopacity{0.000000}%
\pgfsetdash{}{0pt}%
\pgfpathmoveto{\pgfqpoint{4.802983in}{2.048673in}}%
\pgfpathlineto{\pgfqpoint{4.811919in}{2.048673in}}%
\pgfpathlineto{\pgfqpoint{4.811919in}{2.294835in}}%
\pgfpathlineto{\pgfqpoint{4.802983in}{2.294835in}}%
\pgfpathlineto{\pgfqpoint{4.802983in}{2.048673in}}%
\pgfpathclose%
\pgfusepath{fill}%
\end{pgfscope}%
\begin{pgfscope}%
\pgfpathrectangle{\pgfqpoint{3.722897in}{0.857143in}}{\pgfqpoint{2.627103in}{1.813434in}}%
\pgfusepath{clip}%
\pgfsetbuttcap%
\pgfsetmiterjoin%
\definecolor{currentfill}{rgb}{0.754268,0.565033,0.211761}%
\pgfsetfillcolor{currentfill}%
\pgfsetlinewidth{0.000000pt}%
\definecolor{currentstroke}{rgb}{0.000000,0.000000,0.000000}%
\pgfsetstrokecolor{currentstroke}%
\pgfsetstrokeopacity{0.000000}%
\pgfsetdash{}{0pt}%
\pgfpathmoveto{\pgfqpoint{4.814153in}{2.053448in}}%
\pgfpathlineto{\pgfqpoint{4.823090in}{2.053448in}}%
\pgfpathlineto{\pgfqpoint{4.823090in}{2.297030in}}%
\pgfpathlineto{\pgfqpoint{4.814153in}{2.297030in}}%
\pgfpathlineto{\pgfqpoint{4.814153in}{2.053448in}}%
\pgfpathclose%
\pgfusepath{fill}%
\end{pgfscope}%
\begin{pgfscope}%
\pgfpathrectangle{\pgfqpoint{3.722897in}{0.857143in}}{\pgfqpoint{2.627103in}{1.813434in}}%
\pgfusepath{clip}%
\pgfsetbuttcap%
\pgfsetmiterjoin%
\definecolor{currentfill}{rgb}{0.754268,0.565033,0.211761}%
\pgfsetfillcolor{currentfill}%
\pgfsetlinewidth{0.000000pt}%
\definecolor{currentstroke}{rgb}{0.000000,0.000000,0.000000}%
\pgfsetstrokecolor{currentstroke}%
\pgfsetstrokeopacity{0.000000}%
\pgfsetdash{}{0pt}%
\pgfpathmoveto{\pgfqpoint{4.825324in}{2.063945in}}%
\pgfpathlineto{\pgfqpoint{4.834261in}{2.063945in}}%
\pgfpathlineto{\pgfqpoint{4.834261in}{2.298922in}}%
\pgfpathlineto{\pgfqpoint{4.825324in}{2.298922in}}%
\pgfpathlineto{\pgfqpoint{4.825324in}{2.063945in}}%
\pgfpathclose%
\pgfusepath{fill}%
\end{pgfscope}%
\begin{pgfscope}%
\pgfpathrectangle{\pgfqpoint{3.722897in}{0.857143in}}{\pgfqpoint{2.627103in}{1.813434in}}%
\pgfusepath{clip}%
\pgfsetbuttcap%
\pgfsetmiterjoin%
\definecolor{currentfill}{rgb}{0.754268,0.565033,0.211761}%
\pgfsetfillcolor{currentfill}%
\pgfsetlinewidth{0.000000pt}%
\definecolor{currentstroke}{rgb}{0.000000,0.000000,0.000000}%
\pgfsetstrokecolor{currentstroke}%
\pgfsetstrokeopacity{0.000000}%
\pgfsetdash{}{0pt}%
\pgfpathmoveto{\pgfqpoint{4.836495in}{2.075703in}}%
\pgfpathlineto{\pgfqpoint{4.845431in}{2.075703in}}%
\pgfpathlineto{\pgfqpoint{4.845431in}{2.299475in}}%
\pgfpathlineto{\pgfqpoint{4.836495in}{2.299475in}}%
\pgfpathlineto{\pgfqpoint{4.836495in}{2.075703in}}%
\pgfpathclose%
\pgfusepath{fill}%
\end{pgfscope}%
\begin{pgfscope}%
\pgfpathrectangle{\pgfqpoint{3.722897in}{0.857143in}}{\pgfqpoint{2.627103in}{1.813434in}}%
\pgfusepath{clip}%
\pgfsetbuttcap%
\pgfsetmiterjoin%
\definecolor{currentfill}{rgb}{0.754268,0.565033,0.211761}%
\pgfsetfillcolor{currentfill}%
\pgfsetlinewidth{0.000000pt}%
\definecolor{currentstroke}{rgb}{0.000000,0.000000,0.000000}%
\pgfsetstrokecolor{currentstroke}%
\pgfsetstrokeopacity{0.000000}%
\pgfsetdash{}{0pt}%
\pgfpathmoveto{\pgfqpoint{4.847665in}{2.082894in}}%
\pgfpathlineto{\pgfqpoint{4.856602in}{2.082894in}}%
\pgfpathlineto{\pgfqpoint{4.856602in}{2.310904in}}%
\pgfpathlineto{\pgfqpoint{4.847665in}{2.310904in}}%
\pgfpathlineto{\pgfqpoint{4.847665in}{2.082894in}}%
\pgfpathclose%
\pgfusepath{fill}%
\end{pgfscope}%
\begin{pgfscope}%
\pgfpathrectangle{\pgfqpoint{3.722897in}{0.857143in}}{\pgfqpoint{2.627103in}{1.813434in}}%
\pgfusepath{clip}%
\pgfsetbuttcap%
\pgfsetmiterjoin%
\definecolor{currentfill}{rgb}{0.754268,0.565033,0.211761}%
\pgfsetfillcolor{currentfill}%
\pgfsetlinewidth{0.000000pt}%
\definecolor{currentstroke}{rgb}{0.000000,0.000000,0.000000}%
\pgfsetstrokecolor{currentstroke}%
\pgfsetstrokeopacity{0.000000}%
\pgfsetdash{}{0pt}%
\pgfpathmoveto{\pgfqpoint{4.858836in}{2.100055in}}%
\pgfpathlineto{\pgfqpoint{4.867772in}{2.100055in}}%
\pgfpathlineto{\pgfqpoint{4.867772in}{2.330310in}}%
\pgfpathlineto{\pgfqpoint{4.858836in}{2.330310in}}%
\pgfpathlineto{\pgfqpoint{4.858836in}{2.100055in}}%
\pgfpathclose%
\pgfusepath{fill}%
\end{pgfscope}%
\begin{pgfscope}%
\pgfpathrectangle{\pgfqpoint{3.722897in}{0.857143in}}{\pgfqpoint{2.627103in}{1.813434in}}%
\pgfusepath{clip}%
\pgfsetbuttcap%
\pgfsetmiterjoin%
\definecolor{currentfill}{rgb}{0.754268,0.565033,0.211761}%
\pgfsetfillcolor{currentfill}%
\pgfsetlinewidth{0.000000pt}%
\definecolor{currentstroke}{rgb}{0.000000,0.000000,0.000000}%
\pgfsetstrokecolor{currentstroke}%
\pgfsetstrokeopacity{0.000000}%
\pgfsetdash{}{0pt}%
\pgfpathmoveto{\pgfqpoint{4.870006in}{2.126989in}}%
\pgfpathlineto{\pgfqpoint{4.878943in}{2.126989in}}%
\pgfpathlineto{\pgfqpoint{4.878943in}{2.362365in}}%
\pgfpathlineto{\pgfqpoint{4.870006in}{2.362365in}}%
\pgfpathlineto{\pgfqpoint{4.870006in}{2.126989in}}%
\pgfpathclose%
\pgfusepath{fill}%
\end{pgfscope}%
\begin{pgfscope}%
\pgfpathrectangle{\pgfqpoint{3.722897in}{0.857143in}}{\pgfqpoint{2.627103in}{1.813434in}}%
\pgfusepath{clip}%
\pgfsetbuttcap%
\pgfsetmiterjoin%
\definecolor{currentfill}{rgb}{0.754268,0.565033,0.211761}%
\pgfsetfillcolor{currentfill}%
\pgfsetlinewidth{0.000000pt}%
\definecolor{currentstroke}{rgb}{0.000000,0.000000,0.000000}%
\pgfsetstrokecolor{currentstroke}%
\pgfsetstrokeopacity{0.000000}%
\pgfsetdash{}{0pt}%
\pgfpathmoveto{\pgfqpoint{4.881177in}{2.143506in}}%
\pgfpathlineto{\pgfqpoint{4.890114in}{2.143506in}}%
\pgfpathlineto{\pgfqpoint{4.890114in}{2.391850in}}%
\pgfpathlineto{\pgfqpoint{4.881177in}{2.391850in}}%
\pgfpathlineto{\pgfqpoint{4.881177in}{2.143506in}}%
\pgfpathclose%
\pgfusepath{fill}%
\end{pgfscope}%
\begin{pgfscope}%
\pgfpathrectangle{\pgfqpoint{3.722897in}{0.857143in}}{\pgfqpoint{2.627103in}{1.813434in}}%
\pgfusepath{clip}%
\pgfsetbuttcap%
\pgfsetmiterjoin%
\definecolor{currentfill}{rgb}{0.754268,0.565033,0.211761}%
\pgfsetfillcolor{currentfill}%
\pgfsetlinewidth{0.000000pt}%
\definecolor{currentstroke}{rgb}{0.000000,0.000000,0.000000}%
\pgfsetstrokecolor{currentstroke}%
\pgfsetstrokeopacity{0.000000}%
\pgfsetdash{}{0pt}%
\pgfpathmoveto{\pgfqpoint{4.892348in}{2.140362in}}%
\pgfpathlineto{\pgfqpoint{4.901284in}{2.140362in}}%
\pgfpathlineto{\pgfqpoint{4.901284in}{2.399790in}}%
\pgfpathlineto{\pgfqpoint{4.892348in}{2.399790in}}%
\pgfpathlineto{\pgfqpoint{4.892348in}{2.140362in}}%
\pgfpathclose%
\pgfusepath{fill}%
\end{pgfscope}%
\begin{pgfscope}%
\pgfpathrectangle{\pgfqpoint{3.722897in}{0.857143in}}{\pgfqpoint{2.627103in}{1.813434in}}%
\pgfusepath{clip}%
\pgfsetbuttcap%
\pgfsetmiterjoin%
\definecolor{currentfill}{rgb}{0.754268,0.565033,0.211761}%
\pgfsetfillcolor{currentfill}%
\pgfsetlinewidth{0.000000pt}%
\definecolor{currentstroke}{rgb}{0.000000,0.000000,0.000000}%
\pgfsetstrokecolor{currentstroke}%
\pgfsetstrokeopacity{0.000000}%
\pgfsetdash{}{0pt}%
\pgfpathmoveto{\pgfqpoint{4.903518in}{2.149088in}}%
\pgfpathlineto{\pgfqpoint{4.912455in}{2.149088in}}%
\pgfpathlineto{\pgfqpoint{4.912455in}{2.402805in}}%
\pgfpathlineto{\pgfqpoint{4.903518in}{2.402805in}}%
\pgfpathlineto{\pgfqpoint{4.903518in}{2.149088in}}%
\pgfpathclose%
\pgfusepath{fill}%
\end{pgfscope}%
\begin{pgfscope}%
\pgfpathrectangle{\pgfqpoint{3.722897in}{0.857143in}}{\pgfqpoint{2.627103in}{1.813434in}}%
\pgfusepath{clip}%
\pgfsetbuttcap%
\pgfsetmiterjoin%
\definecolor{currentfill}{rgb}{0.754268,0.565033,0.211761}%
\pgfsetfillcolor{currentfill}%
\pgfsetlinewidth{0.000000pt}%
\definecolor{currentstroke}{rgb}{0.000000,0.000000,0.000000}%
\pgfsetstrokecolor{currentstroke}%
\pgfsetstrokeopacity{0.000000}%
\pgfsetdash{}{0pt}%
\pgfpathmoveto{\pgfqpoint{4.914689in}{2.174312in}}%
\pgfpathlineto{\pgfqpoint{4.923625in}{2.174312in}}%
\pgfpathlineto{\pgfqpoint{4.923625in}{2.421283in}}%
\pgfpathlineto{\pgfqpoint{4.914689in}{2.421283in}}%
\pgfpathlineto{\pgfqpoint{4.914689in}{2.174312in}}%
\pgfpathclose%
\pgfusepath{fill}%
\end{pgfscope}%
\begin{pgfscope}%
\pgfpathrectangle{\pgfqpoint{3.722897in}{0.857143in}}{\pgfqpoint{2.627103in}{1.813434in}}%
\pgfusepath{clip}%
\pgfsetbuttcap%
\pgfsetmiterjoin%
\definecolor{currentfill}{rgb}{0.754268,0.565033,0.211761}%
\pgfsetfillcolor{currentfill}%
\pgfsetlinewidth{0.000000pt}%
\definecolor{currentstroke}{rgb}{0.000000,0.000000,0.000000}%
\pgfsetstrokecolor{currentstroke}%
\pgfsetstrokeopacity{0.000000}%
\pgfsetdash{}{0pt}%
\pgfpathmoveto{\pgfqpoint{4.925860in}{2.190037in}}%
\pgfpathlineto{\pgfqpoint{4.934796in}{2.190037in}}%
\pgfpathlineto{\pgfqpoint{4.934796in}{2.439104in}}%
\pgfpathlineto{\pgfqpoint{4.925860in}{2.439104in}}%
\pgfpathlineto{\pgfqpoint{4.925860in}{2.190037in}}%
\pgfpathclose%
\pgfusepath{fill}%
\end{pgfscope}%
\begin{pgfscope}%
\pgfpathrectangle{\pgfqpoint{3.722897in}{0.857143in}}{\pgfqpoint{2.627103in}{1.813434in}}%
\pgfusepath{clip}%
\pgfsetbuttcap%
\pgfsetmiterjoin%
\definecolor{currentfill}{rgb}{0.754268,0.565033,0.211761}%
\pgfsetfillcolor{currentfill}%
\pgfsetlinewidth{0.000000pt}%
\definecolor{currentstroke}{rgb}{0.000000,0.000000,0.000000}%
\pgfsetstrokecolor{currentstroke}%
\pgfsetstrokeopacity{0.000000}%
\pgfsetdash{}{0pt}%
\pgfpathmoveto{\pgfqpoint{4.937030in}{2.184939in}}%
\pgfpathlineto{\pgfqpoint{4.945967in}{2.184939in}}%
\pgfpathlineto{\pgfqpoint{4.945967in}{2.440168in}}%
\pgfpathlineto{\pgfqpoint{4.937030in}{2.440168in}}%
\pgfpathlineto{\pgfqpoint{4.937030in}{2.184939in}}%
\pgfpathclose%
\pgfusepath{fill}%
\end{pgfscope}%
\begin{pgfscope}%
\pgfpathrectangle{\pgfqpoint{3.722897in}{0.857143in}}{\pgfqpoint{2.627103in}{1.813434in}}%
\pgfusepath{clip}%
\pgfsetbuttcap%
\pgfsetmiterjoin%
\definecolor{currentfill}{rgb}{0.754268,0.565033,0.211761}%
\pgfsetfillcolor{currentfill}%
\pgfsetlinewidth{0.000000pt}%
\definecolor{currentstroke}{rgb}{0.000000,0.000000,0.000000}%
\pgfsetstrokecolor{currentstroke}%
\pgfsetstrokeopacity{0.000000}%
\pgfsetdash{}{0pt}%
\pgfpathmoveto{\pgfqpoint{4.948201in}{2.198160in}}%
\pgfpathlineto{\pgfqpoint{4.957137in}{2.198160in}}%
\pgfpathlineto{\pgfqpoint{4.957137in}{2.443935in}}%
\pgfpathlineto{\pgfqpoint{4.948201in}{2.443935in}}%
\pgfpathlineto{\pgfqpoint{4.948201in}{2.198160in}}%
\pgfpathclose%
\pgfusepath{fill}%
\end{pgfscope}%
\begin{pgfscope}%
\pgfpathrectangle{\pgfqpoint{3.722897in}{0.857143in}}{\pgfqpoint{2.627103in}{1.813434in}}%
\pgfusepath{clip}%
\pgfsetbuttcap%
\pgfsetmiterjoin%
\definecolor{currentfill}{rgb}{0.754268,0.565033,0.211761}%
\pgfsetfillcolor{currentfill}%
\pgfsetlinewidth{0.000000pt}%
\definecolor{currentstroke}{rgb}{0.000000,0.000000,0.000000}%
\pgfsetstrokecolor{currentstroke}%
\pgfsetstrokeopacity{0.000000}%
\pgfsetdash{}{0pt}%
\pgfpathmoveto{\pgfqpoint{4.959371in}{2.197934in}}%
\pgfpathlineto{\pgfqpoint{4.968308in}{2.197934in}}%
\pgfpathlineto{\pgfqpoint{4.968308in}{2.437772in}}%
\pgfpathlineto{\pgfqpoint{4.959371in}{2.437772in}}%
\pgfpathlineto{\pgfqpoint{4.959371in}{2.197934in}}%
\pgfpathclose%
\pgfusepath{fill}%
\end{pgfscope}%
\begin{pgfscope}%
\pgfpathrectangle{\pgfqpoint{3.722897in}{0.857143in}}{\pgfqpoint{2.627103in}{1.813434in}}%
\pgfusepath{clip}%
\pgfsetbuttcap%
\pgfsetmiterjoin%
\definecolor{currentfill}{rgb}{0.754268,0.565033,0.211761}%
\pgfsetfillcolor{currentfill}%
\pgfsetlinewidth{0.000000pt}%
\definecolor{currentstroke}{rgb}{0.000000,0.000000,0.000000}%
\pgfsetstrokecolor{currentstroke}%
\pgfsetstrokeopacity{0.000000}%
\pgfsetdash{}{0pt}%
\pgfpathmoveto{\pgfqpoint{4.970542in}{2.202923in}}%
\pgfpathlineto{\pgfqpoint{4.979478in}{2.202923in}}%
\pgfpathlineto{\pgfqpoint{4.979478in}{2.434859in}}%
\pgfpathlineto{\pgfqpoint{4.970542in}{2.434859in}}%
\pgfpathlineto{\pgfqpoint{4.970542in}{2.202923in}}%
\pgfpathclose%
\pgfusepath{fill}%
\end{pgfscope}%
\begin{pgfscope}%
\pgfpathrectangle{\pgfqpoint{3.722897in}{0.857143in}}{\pgfqpoint{2.627103in}{1.813434in}}%
\pgfusepath{clip}%
\pgfsetbuttcap%
\pgfsetmiterjoin%
\definecolor{currentfill}{rgb}{0.754268,0.565033,0.211761}%
\pgfsetfillcolor{currentfill}%
\pgfsetlinewidth{0.000000pt}%
\definecolor{currentstroke}{rgb}{0.000000,0.000000,0.000000}%
\pgfsetstrokecolor{currentstroke}%
\pgfsetstrokeopacity{0.000000}%
\pgfsetdash{}{0pt}%
\pgfpathmoveto{\pgfqpoint{4.981713in}{2.211879in}}%
\pgfpathlineto{\pgfqpoint{4.990649in}{2.211879in}}%
\pgfpathlineto{\pgfqpoint{4.990649in}{2.439395in}}%
\pgfpathlineto{\pgfqpoint{4.981713in}{2.439395in}}%
\pgfpathlineto{\pgfqpoint{4.981713in}{2.211879in}}%
\pgfpathclose%
\pgfusepath{fill}%
\end{pgfscope}%
\begin{pgfscope}%
\pgfpathrectangle{\pgfqpoint{3.722897in}{0.857143in}}{\pgfqpoint{2.627103in}{1.813434in}}%
\pgfusepath{clip}%
\pgfsetbuttcap%
\pgfsetmiterjoin%
\definecolor{currentfill}{rgb}{0.754268,0.565033,0.211761}%
\pgfsetfillcolor{currentfill}%
\pgfsetlinewidth{0.000000pt}%
\definecolor{currentstroke}{rgb}{0.000000,0.000000,0.000000}%
\pgfsetstrokecolor{currentstroke}%
\pgfsetstrokeopacity{0.000000}%
\pgfsetdash{}{0pt}%
\pgfpathmoveto{\pgfqpoint{4.992883in}{2.206824in}}%
\pgfpathlineto{\pgfqpoint{5.001820in}{2.206824in}}%
\pgfpathlineto{\pgfqpoint{5.001820in}{2.431391in}}%
\pgfpathlineto{\pgfqpoint{4.992883in}{2.431391in}}%
\pgfpathlineto{\pgfqpoint{4.992883in}{2.206824in}}%
\pgfpathclose%
\pgfusepath{fill}%
\end{pgfscope}%
\begin{pgfscope}%
\pgfpathrectangle{\pgfqpoint{3.722897in}{0.857143in}}{\pgfqpoint{2.627103in}{1.813434in}}%
\pgfusepath{clip}%
\pgfsetbuttcap%
\pgfsetmiterjoin%
\definecolor{currentfill}{rgb}{0.754268,0.565033,0.211761}%
\pgfsetfillcolor{currentfill}%
\pgfsetlinewidth{0.000000pt}%
\definecolor{currentstroke}{rgb}{0.000000,0.000000,0.000000}%
\pgfsetstrokecolor{currentstroke}%
\pgfsetstrokeopacity{0.000000}%
\pgfsetdash{}{0pt}%
\pgfpathmoveto{\pgfqpoint{5.004054in}{2.214204in}}%
\pgfpathlineto{\pgfqpoint{5.012990in}{2.214204in}}%
\pgfpathlineto{\pgfqpoint{5.012990in}{2.426187in}}%
\pgfpathlineto{\pgfqpoint{5.004054in}{2.426187in}}%
\pgfpathlineto{\pgfqpoint{5.004054in}{2.214204in}}%
\pgfpathclose%
\pgfusepath{fill}%
\end{pgfscope}%
\begin{pgfscope}%
\pgfpathrectangle{\pgfqpoint{3.722897in}{0.857143in}}{\pgfqpoint{2.627103in}{1.813434in}}%
\pgfusepath{clip}%
\pgfsetbuttcap%
\pgfsetmiterjoin%
\definecolor{currentfill}{rgb}{0.754268,0.565033,0.211761}%
\pgfsetfillcolor{currentfill}%
\pgfsetlinewidth{0.000000pt}%
\definecolor{currentstroke}{rgb}{0.000000,0.000000,0.000000}%
\pgfsetstrokecolor{currentstroke}%
\pgfsetstrokeopacity{0.000000}%
\pgfsetdash{}{0pt}%
\pgfpathmoveto{\pgfqpoint{5.015224in}{2.212881in}}%
\pgfpathlineto{\pgfqpoint{5.024161in}{2.212881in}}%
\pgfpathlineto{\pgfqpoint{5.024161in}{2.414582in}}%
\pgfpathlineto{\pgfqpoint{5.015224in}{2.414582in}}%
\pgfpathlineto{\pgfqpoint{5.015224in}{2.212881in}}%
\pgfpathclose%
\pgfusepath{fill}%
\end{pgfscope}%
\begin{pgfscope}%
\pgfpathrectangle{\pgfqpoint{3.722897in}{0.857143in}}{\pgfqpoint{2.627103in}{1.813434in}}%
\pgfusepath{clip}%
\pgfsetbuttcap%
\pgfsetmiterjoin%
\definecolor{currentfill}{rgb}{0.754268,0.565033,0.211761}%
\pgfsetfillcolor{currentfill}%
\pgfsetlinewidth{0.000000pt}%
\definecolor{currentstroke}{rgb}{0.000000,0.000000,0.000000}%
\pgfsetstrokecolor{currentstroke}%
\pgfsetstrokeopacity{0.000000}%
\pgfsetdash{}{0pt}%
\pgfpathmoveto{\pgfqpoint{5.026395in}{2.230757in}}%
\pgfpathlineto{\pgfqpoint{5.035331in}{2.230757in}}%
\pgfpathlineto{\pgfqpoint{5.035331in}{2.414882in}}%
\pgfpathlineto{\pgfqpoint{5.026395in}{2.414882in}}%
\pgfpathlineto{\pgfqpoint{5.026395in}{2.230757in}}%
\pgfpathclose%
\pgfusepath{fill}%
\end{pgfscope}%
\begin{pgfscope}%
\pgfpathrectangle{\pgfqpoint{3.722897in}{0.857143in}}{\pgfqpoint{2.627103in}{1.813434in}}%
\pgfusepath{clip}%
\pgfsetbuttcap%
\pgfsetmiterjoin%
\definecolor{currentfill}{rgb}{0.754268,0.565033,0.211761}%
\pgfsetfillcolor{currentfill}%
\pgfsetlinewidth{0.000000pt}%
\definecolor{currentstroke}{rgb}{0.000000,0.000000,0.000000}%
\pgfsetstrokecolor{currentstroke}%
\pgfsetstrokeopacity{0.000000}%
\pgfsetdash{}{0pt}%
\pgfpathmoveto{\pgfqpoint{5.037566in}{2.233794in}}%
\pgfpathlineto{\pgfqpoint{5.046502in}{2.233794in}}%
\pgfpathlineto{\pgfqpoint{5.046502in}{2.405450in}}%
\pgfpathlineto{\pgfqpoint{5.037566in}{2.405450in}}%
\pgfpathlineto{\pgfqpoint{5.037566in}{2.233794in}}%
\pgfpathclose%
\pgfusepath{fill}%
\end{pgfscope}%
\begin{pgfscope}%
\pgfpathrectangle{\pgfqpoint{3.722897in}{0.857143in}}{\pgfqpoint{2.627103in}{1.813434in}}%
\pgfusepath{clip}%
\pgfsetbuttcap%
\pgfsetmiterjoin%
\definecolor{currentfill}{rgb}{0.754268,0.565033,0.211761}%
\pgfsetfillcolor{currentfill}%
\pgfsetlinewidth{0.000000pt}%
\definecolor{currentstroke}{rgb}{0.000000,0.000000,0.000000}%
\pgfsetstrokecolor{currentstroke}%
\pgfsetstrokeopacity{0.000000}%
\pgfsetdash{}{0pt}%
\pgfpathmoveto{\pgfqpoint{5.048736in}{2.242832in}}%
\pgfpathlineto{\pgfqpoint{5.057673in}{2.242832in}}%
\pgfpathlineto{\pgfqpoint{5.057673in}{2.401130in}}%
\pgfpathlineto{\pgfqpoint{5.048736in}{2.401130in}}%
\pgfpathlineto{\pgfqpoint{5.048736in}{2.242832in}}%
\pgfpathclose%
\pgfusepath{fill}%
\end{pgfscope}%
\begin{pgfscope}%
\pgfpathrectangle{\pgfqpoint{3.722897in}{0.857143in}}{\pgfqpoint{2.627103in}{1.813434in}}%
\pgfusepath{clip}%
\pgfsetbuttcap%
\pgfsetmiterjoin%
\definecolor{currentfill}{rgb}{0.754268,0.565033,0.211761}%
\pgfsetfillcolor{currentfill}%
\pgfsetlinewidth{0.000000pt}%
\definecolor{currentstroke}{rgb}{0.000000,0.000000,0.000000}%
\pgfsetstrokecolor{currentstroke}%
\pgfsetstrokeopacity{0.000000}%
\pgfsetdash{}{0pt}%
\pgfpathmoveto{\pgfqpoint{5.059907in}{2.252363in}}%
\pgfpathlineto{\pgfqpoint{5.068843in}{2.252363in}}%
\pgfpathlineto{\pgfqpoint{5.068843in}{2.398663in}}%
\pgfpathlineto{\pgfqpoint{5.059907in}{2.398663in}}%
\pgfpathlineto{\pgfqpoint{5.059907in}{2.252363in}}%
\pgfpathclose%
\pgfusepath{fill}%
\end{pgfscope}%
\begin{pgfscope}%
\pgfpathrectangle{\pgfqpoint{3.722897in}{0.857143in}}{\pgfqpoint{2.627103in}{1.813434in}}%
\pgfusepath{clip}%
\pgfsetbuttcap%
\pgfsetmiterjoin%
\definecolor{currentfill}{rgb}{0.754268,0.565033,0.211761}%
\pgfsetfillcolor{currentfill}%
\pgfsetlinewidth{0.000000pt}%
\definecolor{currentstroke}{rgb}{0.000000,0.000000,0.000000}%
\pgfsetstrokecolor{currentstroke}%
\pgfsetstrokeopacity{0.000000}%
\pgfsetdash{}{0pt}%
\pgfpathmoveto{\pgfqpoint{5.071077in}{2.264057in}}%
\pgfpathlineto{\pgfqpoint{5.080014in}{2.264057in}}%
\pgfpathlineto{\pgfqpoint{5.080014in}{2.400189in}}%
\pgfpathlineto{\pgfqpoint{5.071077in}{2.400189in}}%
\pgfpathlineto{\pgfqpoint{5.071077in}{2.264057in}}%
\pgfpathclose%
\pgfusepath{fill}%
\end{pgfscope}%
\begin{pgfscope}%
\pgfpathrectangle{\pgfqpoint{3.722897in}{0.857143in}}{\pgfqpoint{2.627103in}{1.813434in}}%
\pgfusepath{clip}%
\pgfsetbuttcap%
\pgfsetmiterjoin%
\definecolor{currentfill}{rgb}{0.754268,0.565033,0.211761}%
\pgfsetfillcolor{currentfill}%
\pgfsetlinewidth{0.000000pt}%
\definecolor{currentstroke}{rgb}{0.000000,0.000000,0.000000}%
\pgfsetstrokecolor{currentstroke}%
\pgfsetstrokeopacity{0.000000}%
\pgfsetdash{}{0pt}%
\pgfpathmoveto{\pgfqpoint{5.082248in}{2.271444in}}%
\pgfpathlineto{\pgfqpoint{5.091184in}{2.271444in}}%
\pgfpathlineto{\pgfqpoint{5.091184in}{2.400482in}}%
\pgfpathlineto{\pgfqpoint{5.082248in}{2.400482in}}%
\pgfpathlineto{\pgfqpoint{5.082248in}{2.271444in}}%
\pgfpathclose%
\pgfusepath{fill}%
\end{pgfscope}%
\begin{pgfscope}%
\pgfpathrectangle{\pgfqpoint{3.722897in}{0.857143in}}{\pgfqpoint{2.627103in}{1.813434in}}%
\pgfusepath{clip}%
\pgfsetbuttcap%
\pgfsetmiterjoin%
\definecolor{currentfill}{rgb}{0.754268,0.565033,0.211761}%
\pgfsetfillcolor{currentfill}%
\pgfsetlinewidth{0.000000pt}%
\definecolor{currentstroke}{rgb}{0.000000,0.000000,0.000000}%
\pgfsetstrokecolor{currentstroke}%
\pgfsetstrokeopacity{0.000000}%
\pgfsetdash{}{0pt}%
\pgfpathmoveto{\pgfqpoint{5.093419in}{2.281583in}}%
\pgfpathlineto{\pgfqpoint{5.102355in}{2.281583in}}%
\pgfpathlineto{\pgfqpoint{5.102355in}{2.406660in}}%
\pgfpathlineto{\pgfqpoint{5.093419in}{2.406660in}}%
\pgfpathlineto{\pgfqpoint{5.093419in}{2.281583in}}%
\pgfpathclose%
\pgfusepath{fill}%
\end{pgfscope}%
\begin{pgfscope}%
\pgfpathrectangle{\pgfqpoint{3.722897in}{0.857143in}}{\pgfqpoint{2.627103in}{1.813434in}}%
\pgfusepath{clip}%
\pgfsetbuttcap%
\pgfsetmiterjoin%
\definecolor{currentfill}{rgb}{0.754268,0.565033,0.211761}%
\pgfsetfillcolor{currentfill}%
\pgfsetlinewidth{0.000000pt}%
\definecolor{currentstroke}{rgb}{0.000000,0.000000,0.000000}%
\pgfsetstrokecolor{currentstroke}%
\pgfsetstrokeopacity{0.000000}%
\pgfsetdash{}{0pt}%
\pgfpathmoveto{\pgfqpoint{5.104589in}{2.291229in}}%
\pgfpathlineto{\pgfqpoint{5.113526in}{2.291229in}}%
\pgfpathlineto{\pgfqpoint{5.113526in}{2.413241in}}%
\pgfpathlineto{\pgfqpoint{5.104589in}{2.413241in}}%
\pgfpathlineto{\pgfqpoint{5.104589in}{2.291229in}}%
\pgfpathclose%
\pgfusepath{fill}%
\end{pgfscope}%
\begin{pgfscope}%
\pgfpathrectangle{\pgfqpoint{3.722897in}{0.857143in}}{\pgfqpoint{2.627103in}{1.813434in}}%
\pgfusepath{clip}%
\pgfsetbuttcap%
\pgfsetmiterjoin%
\definecolor{currentfill}{rgb}{0.754268,0.565033,0.211761}%
\pgfsetfillcolor{currentfill}%
\pgfsetlinewidth{0.000000pt}%
\definecolor{currentstroke}{rgb}{0.000000,0.000000,0.000000}%
\pgfsetstrokecolor{currentstroke}%
\pgfsetstrokeopacity{0.000000}%
\pgfsetdash{}{0pt}%
\pgfpathmoveto{\pgfqpoint{5.115760in}{2.300574in}}%
\pgfpathlineto{\pgfqpoint{5.124696in}{2.300574in}}%
\pgfpathlineto{\pgfqpoint{5.124696in}{2.423484in}}%
\pgfpathlineto{\pgfqpoint{5.115760in}{2.423484in}}%
\pgfpathlineto{\pgfqpoint{5.115760in}{2.300574in}}%
\pgfpathclose%
\pgfusepath{fill}%
\end{pgfscope}%
\begin{pgfscope}%
\pgfpathrectangle{\pgfqpoint{3.722897in}{0.857143in}}{\pgfqpoint{2.627103in}{1.813434in}}%
\pgfusepath{clip}%
\pgfsetbuttcap%
\pgfsetmiterjoin%
\definecolor{currentfill}{rgb}{0.754268,0.565033,0.211761}%
\pgfsetfillcolor{currentfill}%
\pgfsetlinewidth{0.000000pt}%
\definecolor{currentstroke}{rgb}{0.000000,0.000000,0.000000}%
\pgfsetstrokecolor{currentstroke}%
\pgfsetstrokeopacity{0.000000}%
\pgfsetdash{}{0pt}%
\pgfpathmoveto{\pgfqpoint{5.126930in}{2.312468in}}%
\pgfpathlineto{\pgfqpoint{5.135867in}{2.312468in}}%
\pgfpathlineto{\pgfqpoint{5.135867in}{2.440340in}}%
\pgfpathlineto{\pgfqpoint{5.126930in}{2.440340in}}%
\pgfpathlineto{\pgfqpoint{5.126930in}{2.312468in}}%
\pgfpathclose%
\pgfusepath{fill}%
\end{pgfscope}%
\begin{pgfscope}%
\pgfpathrectangle{\pgfqpoint{3.722897in}{0.857143in}}{\pgfqpoint{2.627103in}{1.813434in}}%
\pgfusepath{clip}%
\pgfsetbuttcap%
\pgfsetmiterjoin%
\definecolor{currentfill}{rgb}{0.754268,0.565033,0.211761}%
\pgfsetfillcolor{currentfill}%
\pgfsetlinewidth{0.000000pt}%
\definecolor{currentstroke}{rgb}{0.000000,0.000000,0.000000}%
\pgfsetstrokecolor{currentstroke}%
\pgfsetstrokeopacity{0.000000}%
\pgfsetdash{}{0pt}%
\pgfpathmoveto{\pgfqpoint{5.138101in}{2.326844in}}%
\pgfpathlineto{\pgfqpoint{5.147038in}{2.326844in}}%
\pgfpathlineto{\pgfqpoint{5.147038in}{2.464582in}}%
\pgfpathlineto{\pgfqpoint{5.138101in}{2.464582in}}%
\pgfpathlineto{\pgfqpoint{5.138101in}{2.326844in}}%
\pgfpathclose%
\pgfusepath{fill}%
\end{pgfscope}%
\begin{pgfscope}%
\pgfpathrectangle{\pgfqpoint{3.722897in}{0.857143in}}{\pgfqpoint{2.627103in}{1.813434in}}%
\pgfusepath{clip}%
\pgfsetbuttcap%
\pgfsetmiterjoin%
\definecolor{currentfill}{rgb}{0.754268,0.565033,0.211761}%
\pgfsetfillcolor{currentfill}%
\pgfsetlinewidth{0.000000pt}%
\definecolor{currentstroke}{rgb}{0.000000,0.000000,0.000000}%
\pgfsetstrokecolor{currentstroke}%
\pgfsetstrokeopacity{0.000000}%
\pgfsetdash{}{0pt}%
\pgfpathmoveto{\pgfqpoint{5.149272in}{2.329067in}}%
\pgfpathlineto{\pgfqpoint{5.158208in}{2.329067in}}%
\pgfpathlineto{\pgfqpoint{5.158208in}{2.474402in}}%
\pgfpathlineto{\pgfqpoint{5.149272in}{2.474402in}}%
\pgfpathlineto{\pgfqpoint{5.149272in}{2.329067in}}%
\pgfpathclose%
\pgfusepath{fill}%
\end{pgfscope}%
\begin{pgfscope}%
\pgfpathrectangle{\pgfqpoint{3.722897in}{0.857143in}}{\pgfqpoint{2.627103in}{1.813434in}}%
\pgfusepath{clip}%
\pgfsetbuttcap%
\pgfsetmiterjoin%
\definecolor{currentfill}{rgb}{0.754268,0.565033,0.211761}%
\pgfsetfillcolor{currentfill}%
\pgfsetlinewidth{0.000000pt}%
\definecolor{currentstroke}{rgb}{0.000000,0.000000,0.000000}%
\pgfsetstrokecolor{currentstroke}%
\pgfsetstrokeopacity{0.000000}%
\pgfsetdash{}{0pt}%
\pgfpathmoveto{\pgfqpoint{5.160442in}{2.333702in}}%
\pgfpathlineto{\pgfqpoint{5.169379in}{2.333702in}}%
\pgfpathlineto{\pgfqpoint{5.169379in}{2.486115in}}%
\pgfpathlineto{\pgfqpoint{5.160442in}{2.486115in}}%
\pgfpathlineto{\pgfqpoint{5.160442in}{2.333702in}}%
\pgfpathclose%
\pgfusepath{fill}%
\end{pgfscope}%
\begin{pgfscope}%
\pgfpathrectangle{\pgfqpoint{3.722897in}{0.857143in}}{\pgfqpoint{2.627103in}{1.813434in}}%
\pgfusepath{clip}%
\pgfsetbuttcap%
\pgfsetmiterjoin%
\definecolor{currentfill}{rgb}{0.754268,0.565033,0.211761}%
\pgfsetfillcolor{currentfill}%
\pgfsetlinewidth{0.000000pt}%
\definecolor{currentstroke}{rgb}{0.000000,0.000000,0.000000}%
\pgfsetstrokecolor{currentstroke}%
\pgfsetstrokeopacity{0.000000}%
\pgfsetdash{}{0pt}%
\pgfpathmoveto{\pgfqpoint{5.171613in}{2.340218in}}%
\pgfpathlineto{\pgfqpoint{5.180549in}{2.340218in}}%
\pgfpathlineto{\pgfqpoint{5.180549in}{2.498164in}}%
\pgfpathlineto{\pgfqpoint{5.171613in}{2.498164in}}%
\pgfpathlineto{\pgfqpoint{5.171613in}{2.340218in}}%
\pgfpathclose%
\pgfusepath{fill}%
\end{pgfscope}%
\begin{pgfscope}%
\pgfpathrectangle{\pgfqpoint{3.722897in}{0.857143in}}{\pgfqpoint{2.627103in}{1.813434in}}%
\pgfusepath{clip}%
\pgfsetbuttcap%
\pgfsetmiterjoin%
\definecolor{currentfill}{rgb}{0.754268,0.565033,0.211761}%
\pgfsetfillcolor{currentfill}%
\pgfsetlinewidth{0.000000pt}%
\definecolor{currentstroke}{rgb}{0.000000,0.000000,0.000000}%
\pgfsetstrokecolor{currentstroke}%
\pgfsetstrokeopacity{0.000000}%
\pgfsetdash{}{0pt}%
\pgfpathmoveto{\pgfqpoint{5.182783in}{2.350338in}}%
\pgfpathlineto{\pgfqpoint{5.191720in}{2.350338in}}%
\pgfpathlineto{\pgfqpoint{5.191720in}{2.513657in}}%
\pgfpathlineto{\pgfqpoint{5.182783in}{2.513657in}}%
\pgfpathlineto{\pgfqpoint{5.182783in}{2.350338in}}%
\pgfpathclose%
\pgfusepath{fill}%
\end{pgfscope}%
\begin{pgfscope}%
\pgfpathrectangle{\pgfqpoint{3.722897in}{0.857143in}}{\pgfqpoint{2.627103in}{1.813434in}}%
\pgfusepath{clip}%
\pgfsetbuttcap%
\pgfsetmiterjoin%
\definecolor{currentfill}{rgb}{0.754268,0.565033,0.211761}%
\pgfsetfillcolor{currentfill}%
\pgfsetlinewidth{0.000000pt}%
\definecolor{currentstroke}{rgb}{0.000000,0.000000,0.000000}%
\pgfsetstrokecolor{currentstroke}%
\pgfsetstrokeopacity{0.000000}%
\pgfsetdash{}{0pt}%
\pgfpathmoveto{\pgfqpoint{5.193954in}{2.346969in}}%
\pgfpathlineto{\pgfqpoint{5.202891in}{2.346969in}}%
\pgfpathlineto{\pgfqpoint{5.202891in}{2.519779in}}%
\pgfpathlineto{\pgfqpoint{5.193954in}{2.519779in}}%
\pgfpathlineto{\pgfqpoint{5.193954in}{2.346969in}}%
\pgfpathclose%
\pgfusepath{fill}%
\end{pgfscope}%
\begin{pgfscope}%
\pgfpathrectangle{\pgfqpoint{3.722897in}{0.857143in}}{\pgfqpoint{2.627103in}{1.813434in}}%
\pgfusepath{clip}%
\pgfsetbuttcap%
\pgfsetmiterjoin%
\definecolor{currentfill}{rgb}{0.754268,0.565033,0.211761}%
\pgfsetfillcolor{currentfill}%
\pgfsetlinewidth{0.000000pt}%
\definecolor{currentstroke}{rgb}{0.000000,0.000000,0.000000}%
\pgfsetstrokecolor{currentstroke}%
\pgfsetstrokeopacity{0.000000}%
\pgfsetdash{}{0pt}%
\pgfpathmoveto{\pgfqpoint{5.205125in}{2.355047in}}%
\pgfpathlineto{\pgfqpoint{5.214061in}{2.355047in}}%
\pgfpathlineto{\pgfqpoint{5.214061in}{2.528779in}}%
\pgfpathlineto{\pgfqpoint{5.205125in}{2.528779in}}%
\pgfpathlineto{\pgfqpoint{5.205125in}{2.355047in}}%
\pgfpathclose%
\pgfusepath{fill}%
\end{pgfscope}%
\begin{pgfscope}%
\pgfpathrectangle{\pgfqpoint{3.722897in}{0.857143in}}{\pgfqpoint{2.627103in}{1.813434in}}%
\pgfusepath{clip}%
\pgfsetbuttcap%
\pgfsetmiterjoin%
\definecolor{currentfill}{rgb}{0.754268,0.565033,0.211761}%
\pgfsetfillcolor{currentfill}%
\pgfsetlinewidth{0.000000pt}%
\definecolor{currentstroke}{rgb}{0.000000,0.000000,0.000000}%
\pgfsetstrokecolor{currentstroke}%
\pgfsetstrokeopacity{0.000000}%
\pgfsetdash{}{0pt}%
\pgfpathmoveto{\pgfqpoint{5.216295in}{2.366394in}}%
\pgfpathlineto{\pgfqpoint{5.225232in}{2.366394in}}%
\pgfpathlineto{\pgfqpoint{5.225232in}{2.538269in}}%
\pgfpathlineto{\pgfqpoint{5.216295in}{2.538269in}}%
\pgfpathlineto{\pgfqpoint{5.216295in}{2.366394in}}%
\pgfpathclose%
\pgfusepath{fill}%
\end{pgfscope}%
\begin{pgfscope}%
\pgfpathrectangle{\pgfqpoint{3.722897in}{0.857143in}}{\pgfqpoint{2.627103in}{1.813434in}}%
\pgfusepath{clip}%
\pgfsetbuttcap%
\pgfsetmiterjoin%
\definecolor{currentfill}{rgb}{0.754268,0.565033,0.211761}%
\pgfsetfillcolor{currentfill}%
\pgfsetlinewidth{0.000000pt}%
\definecolor{currentstroke}{rgb}{0.000000,0.000000,0.000000}%
\pgfsetstrokecolor{currentstroke}%
\pgfsetstrokeopacity{0.000000}%
\pgfsetdash{}{0pt}%
\pgfpathmoveto{\pgfqpoint{5.227466in}{2.360255in}}%
\pgfpathlineto{\pgfqpoint{5.236402in}{2.360255in}}%
\pgfpathlineto{\pgfqpoint{5.236402in}{2.546174in}}%
\pgfpathlineto{\pgfqpoint{5.227466in}{2.546174in}}%
\pgfpathlineto{\pgfqpoint{5.227466in}{2.360255in}}%
\pgfpathclose%
\pgfusepath{fill}%
\end{pgfscope}%
\begin{pgfscope}%
\pgfpathrectangle{\pgfqpoint{3.722897in}{0.857143in}}{\pgfqpoint{2.627103in}{1.813434in}}%
\pgfusepath{clip}%
\pgfsetbuttcap%
\pgfsetmiterjoin%
\definecolor{currentfill}{rgb}{0.754268,0.565033,0.211761}%
\pgfsetfillcolor{currentfill}%
\pgfsetlinewidth{0.000000pt}%
\definecolor{currentstroke}{rgb}{0.000000,0.000000,0.000000}%
\pgfsetstrokecolor{currentstroke}%
\pgfsetstrokeopacity{0.000000}%
\pgfsetdash{}{0pt}%
\pgfpathmoveto{\pgfqpoint{5.238636in}{2.369298in}}%
\pgfpathlineto{\pgfqpoint{5.247573in}{2.369298in}}%
\pgfpathlineto{\pgfqpoint{5.247573in}{2.559876in}}%
\pgfpathlineto{\pgfqpoint{5.238636in}{2.559876in}}%
\pgfpathlineto{\pgfqpoint{5.238636in}{2.369298in}}%
\pgfpathclose%
\pgfusepath{fill}%
\end{pgfscope}%
\begin{pgfscope}%
\pgfpathrectangle{\pgfqpoint{3.722897in}{0.857143in}}{\pgfqpoint{2.627103in}{1.813434in}}%
\pgfusepath{clip}%
\pgfsetbuttcap%
\pgfsetmiterjoin%
\definecolor{currentfill}{rgb}{0.754268,0.565033,0.211761}%
\pgfsetfillcolor{currentfill}%
\pgfsetlinewidth{0.000000pt}%
\definecolor{currentstroke}{rgb}{0.000000,0.000000,0.000000}%
\pgfsetstrokecolor{currentstroke}%
\pgfsetstrokeopacity{0.000000}%
\pgfsetdash{}{0pt}%
\pgfpathmoveto{\pgfqpoint{5.249807in}{2.374021in}}%
\pgfpathlineto{\pgfqpoint{5.258744in}{2.374021in}}%
\pgfpathlineto{\pgfqpoint{5.258744in}{2.572246in}}%
\pgfpathlineto{\pgfqpoint{5.249807in}{2.572246in}}%
\pgfpathlineto{\pgfqpoint{5.249807in}{2.374021in}}%
\pgfpathclose%
\pgfusepath{fill}%
\end{pgfscope}%
\begin{pgfscope}%
\pgfpathrectangle{\pgfqpoint{3.722897in}{0.857143in}}{\pgfqpoint{2.627103in}{1.813434in}}%
\pgfusepath{clip}%
\pgfsetbuttcap%
\pgfsetmiterjoin%
\definecolor{currentfill}{rgb}{0.754268,0.565033,0.211761}%
\pgfsetfillcolor{currentfill}%
\pgfsetlinewidth{0.000000pt}%
\definecolor{currentstroke}{rgb}{0.000000,0.000000,0.000000}%
\pgfsetstrokecolor{currentstroke}%
\pgfsetstrokeopacity{0.000000}%
\pgfsetdash{}{0pt}%
\pgfpathmoveto{\pgfqpoint{5.260978in}{2.368295in}}%
\pgfpathlineto{\pgfqpoint{5.269914in}{2.368295in}}%
\pgfpathlineto{\pgfqpoint{5.269914in}{2.580845in}}%
\pgfpathlineto{\pgfqpoint{5.260978in}{2.580845in}}%
\pgfpathlineto{\pgfqpoint{5.260978in}{2.368295in}}%
\pgfpathclose%
\pgfusepath{fill}%
\end{pgfscope}%
\begin{pgfscope}%
\pgfpathrectangle{\pgfqpoint{3.722897in}{0.857143in}}{\pgfqpoint{2.627103in}{1.813434in}}%
\pgfusepath{clip}%
\pgfsetbuttcap%
\pgfsetmiterjoin%
\definecolor{currentfill}{rgb}{0.754268,0.565033,0.211761}%
\pgfsetfillcolor{currentfill}%
\pgfsetlinewidth{0.000000pt}%
\definecolor{currentstroke}{rgb}{0.000000,0.000000,0.000000}%
\pgfsetstrokecolor{currentstroke}%
\pgfsetstrokeopacity{0.000000}%
\pgfsetdash{}{0pt}%
\pgfpathmoveto{\pgfqpoint{5.272148in}{2.350264in}}%
\pgfpathlineto{\pgfqpoint{5.281085in}{2.350264in}}%
\pgfpathlineto{\pgfqpoint{5.281085in}{2.572544in}}%
\pgfpathlineto{\pgfqpoint{5.272148in}{2.572544in}}%
\pgfpathlineto{\pgfqpoint{5.272148in}{2.350264in}}%
\pgfpathclose%
\pgfusepath{fill}%
\end{pgfscope}%
\begin{pgfscope}%
\pgfpathrectangle{\pgfqpoint{3.722897in}{0.857143in}}{\pgfqpoint{2.627103in}{1.813434in}}%
\pgfusepath{clip}%
\pgfsetbuttcap%
\pgfsetmiterjoin%
\definecolor{currentfill}{rgb}{0.754268,0.565033,0.211761}%
\pgfsetfillcolor{currentfill}%
\pgfsetlinewidth{0.000000pt}%
\definecolor{currentstroke}{rgb}{0.000000,0.000000,0.000000}%
\pgfsetstrokecolor{currentstroke}%
\pgfsetstrokeopacity{0.000000}%
\pgfsetdash{}{0pt}%
\pgfpathmoveto{\pgfqpoint{5.283319in}{2.356600in}}%
\pgfpathlineto{\pgfqpoint{5.292255in}{2.356600in}}%
\pgfpathlineto{\pgfqpoint{5.292255in}{2.582025in}}%
\pgfpathlineto{\pgfqpoint{5.283319in}{2.582025in}}%
\pgfpathlineto{\pgfqpoint{5.283319in}{2.356600in}}%
\pgfpathclose%
\pgfusepath{fill}%
\end{pgfscope}%
\begin{pgfscope}%
\pgfpathrectangle{\pgfqpoint{3.722897in}{0.857143in}}{\pgfqpoint{2.627103in}{1.813434in}}%
\pgfusepath{clip}%
\pgfsetbuttcap%
\pgfsetmiterjoin%
\definecolor{currentfill}{rgb}{0.754268,0.565033,0.211761}%
\pgfsetfillcolor{currentfill}%
\pgfsetlinewidth{0.000000pt}%
\definecolor{currentstroke}{rgb}{0.000000,0.000000,0.000000}%
\pgfsetstrokecolor{currentstroke}%
\pgfsetstrokeopacity{0.000000}%
\pgfsetdash{}{0pt}%
\pgfpathmoveto{\pgfqpoint{5.294489in}{2.347792in}}%
\pgfpathlineto{\pgfqpoint{5.303426in}{2.347792in}}%
\pgfpathlineto{\pgfqpoint{5.303426in}{2.578646in}}%
\pgfpathlineto{\pgfqpoint{5.294489in}{2.578646in}}%
\pgfpathlineto{\pgfqpoint{5.294489in}{2.347792in}}%
\pgfpathclose%
\pgfusepath{fill}%
\end{pgfscope}%
\begin{pgfscope}%
\pgfpathrectangle{\pgfqpoint{3.722897in}{0.857143in}}{\pgfqpoint{2.627103in}{1.813434in}}%
\pgfusepath{clip}%
\pgfsetbuttcap%
\pgfsetmiterjoin%
\definecolor{currentfill}{rgb}{0.754268,0.565033,0.211761}%
\pgfsetfillcolor{currentfill}%
\pgfsetlinewidth{0.000000pt}%
\definecolor{currentstroke}{rgb}{0.000000,0.000000,0.000000}%
\pgfsetstrokecolor{currentstroke}%
\pgfsetstrokeopacity{0.000000}%
\pgfsetdash{}{0pt}%
\pgfpathmoveto{\pgfqpoint{5.305660in}{2.340718in}}%
\pgfpathlineto{\pgfqpoint{5.314597in}{2.340718in}}%
\pgfpathlineto{\pgfqpoint{5.314597in}{2.572691in}}%
\pgfpathlineto{\pgfqpoint{5.305660in}{2.572691in}}%
\pgfpathlineto{\pgfqpoint{5.305660in}{2.340718in}}%
\pgfpathclose%
\pgfusepath{fill}%
\end{pgfscope}%
\begin{pgfscope}%
\pgfpathrectangle{\pgfqpoint{3.722897in}{0.857143in}}{\pgfqpoint{2.627103in}{1.813434in}}%
\pgfusepath{clip}%
\pgfsetbuttcap%
\pgfsetmiterjoin%
\definecolor{currentfill}{rgb}{0.754268,0.565033,0.211761}%
\pgfsetfillcolor{currentfill}%
\pgfsetlinewidth{0.000000pt}%
\definecolor{currentstroke}{rgb}{0.000000,0.000000,0.000000}%
\pgfsetstrokecolor{currentstroke}%
\pgfsetstrokeopacity{0.000000}%
\pgfsetdash{}{0pt}%
\pgfpathmoveto{\pgfqpoint{5.316831in}{2.338585in}}%
\pgfpathlineto{\pgfqpoint{5.325767in}{2.338585in}}%
\pgfpathlineto{\pgfqpoint{5.325767in}{2.570390in}}%
\pgfpathlineto{\pgfqpoint{5.316831in}{2.570390in}}%
\pgfpathlineto{\pgfqpoint{5.316831in}{2.338585in}}%
\pgfpathclose%
\pgfusepath{fill}%
\end{pgfscope}%
\begin{pgfscope}%
\pgfpathrectangle{\pgfqpoint{3.722897in}{0.857143in}}{\pgfqpoint{2.627103in}{1.813434in}}%
\pgfusepath{clip}%
\pgfsetbuttcap%
\pgfsetmiterjoin%
\definecolor{currentfill}{rgb}{0.754268,0.565033,0.211761}%
\pgfsetfillcolor{currentfill}%
\pgfsetlinewidth{0.000000pt}%
\definecolor{currentstroke}{rgb}{0.000000,0.000000,0.000000}%
\pgfsetstrokecolor{currentstroke}%
\pgfsetstrokeopacity{0.000000}%
\pgfsetdash{}{0pt}%
\pgfpathmoveto{\pgfqpoint{5.328001in}{2.322850in}}%
\pgfpathlineto{\pgfqpoint{5.336938in}{2.322850in}}%
\pgfpathlineto{\pgfqpoint{5.336938in}{2.557075in}}%
\pgfpathlineto{\pgfqpoint{5.328001in}{2.557075in}}%
\pgfpathlineto{\pgfqpoint{5.328001in}{2.322850in}}%
\pgfpathclose%
\pgfusepath{fill}%
\end{pgfscope}%
\begin{pgfscope}%
\pgfpathrectangle{\pgfqpoint{3.722897in}{0.857143in}}{\pgfqpoint{2.627103in}{1.813434in}}%
\pgfusepath{clip}%
\pgfsetbuttcap%
\pgfsetmiterjoin%
\definecolor{currentfill}{rgb}{0.754268,0.565033,0.211761}%
\pgfsetfillcolor{currentfill}%
\pgfsetlinewidth{0.000000pt}%
\definecolor{currentstroke}{rgb}{0.000000,0.000000,0.000000}%
\pgfsetstrokecolor{currentstroke}%
\pgfsetstrokeopacity{0.000000}%
\pgfsetdash{}{0pt}%
\pgfpathmoveto{\pgfqpoint{5.339172in}{2.317086in}}%
\pgfpathlineto{\pgfqpoint{5.348108in}{2.317086in}}%
\pgfpathlineto{\pgfqpoint{5.348108in}{2.547919in}}%
\pgfpathlineto{\pgfqpoint{5.339172in}{2.547919in}}%
\pgfpathlineto{\pgfqpoint{5.339172in}{2.317086in}}%
\pgfpathclose%
\pgfusepath{fill}%
\end{pgfscope}%
\begin{pgfscope}%
\pgfpathrectangle{\pgfqpoint{3.722897in}{0.857143in}}{\pgfqpoint{2.627103in}{1.813434in}}%
\pgfusepath{clip}%
\pgfsetbuttcap%
\pgfsetmiterjoin%
\definecolor{currentfill}{rgb}{0.754268,0.565033,0.211761}%
\pgfsetfillcolor{currentfill}%
\pgfsetlinewidth{0.000000pt}%
\definecolor{currentstroke}{rgb}{0.000000,0.000000,0.000000}%
\pgfsetstrokecolor{currentstroke}%
\pgfsetstrokeopacity{0.000000}%
\pgfsetdash{}{0pt}%
\pgfpathmoveto{\pgfqpoint{5.350343in}{2.318728in}}%
\pgfpathlineto{\pgfqpoint{5.359279in}{2.318728in}}%
\pgfpathlineto{\pgfqpoint{5.359279in}{2.543876in}}%
\pgfpathlineto{\pgfqpoint{5.350343in}{2.543876in}}%
\pgfpathlineto{\pgfqpoint{5.350343in}{2.318728in}}%
\pgfpathclose%
\pgfusepath{fill}%
\end{pgfscope}%
\begin{pgfscope}%
\pgfpathrectangle{\pgfqpoint{3.722897in}{0.857143in}}{\pgfqpoint{2.627103in}{1.813434in}}%
\pgfusepath{clip}%
\pgfsetbuttcap%
\pgfsetmiterjoin%
\definecolor{currentfill}{rgb}{0.754268,0.565033,0.211761}%
\pgfsetfillcolor{currentfill}%
\pgfsetlinewidth{0.000000pt}%
\definecolor{currentstroke}{rgb}{0.000000,0.000000,0.000000}%
\pgfsetstrokecolor{currentstroke}%
\pgfsetstrokeopacity{0.000000}%
\pgfsetdash{}{0pt}%
\pgfpathmoveto{\pgfqpoint{5.361513in}{2.317982in}}%
\pgfpathlineto{\pgfqpoint{5.370450in}{2.317982in}}%
\pgfpathlineto{\pgfqpoint{5.370450in}{2.542405in}}%
\pgfpathlineto{\pgfqpoint{5.361513in}{2.542405in}}%
\pgfpathlineto{\pgfqpoint{5.361513in}{2.317982in}}%
\pgfpathclose%
\pgfusepath{fill}%
\end{pgfscope}%
\begin{pgfscope}%
\pgfpathrectangle{\pgfqpoint{3.722897in}{0.857143in}}{\pgfqpoint{2.627103in}{1.813434in}}%
\pgfusepath{clip}%
\pgfsetbuttcap%
\pgfsetmiterjoin%
\definecolor{currentfill}{rgb}{0.754268,0.565033,0.211761}%
\pgfsetfillcolor{currentfill}%
\pgfsetlinewidth{0.000000pt}%
\definecolor{currentstroke}{rgb}{0.000000,0.000000,0.000000}%
\pgfsetstrokecolor{currentstroke}%
\pgfsetstrokeopacity{0.000000}%
\pgfsetdash{}{0pt}%
\pgfpathmoveto{\pgfqpoint{5.372684in}{2.318073in}}%
\pgfpathlineto{\pgfqpoint{5.381620in}{2.318073in}}%
\pgfpathlineto{\pgfqpoint{5.381620in}{2.546046in}}%
\pgfpathlineto{\pgfqpoint{5.372684in}{2.546046in}}%
\pgfpathlineto{\pgfqpoint{5.372684in}{2.318073in}}%
\pgfpathclose%
\pgfusepath{fill}%
\end{pgfscope}%
\begin{pgfscope}%
\pgfpathrectangle{\pgfqpoint{3.722897in}{0.857143in}}{\pgfqpoint{2.627103in}{1.813434in}}%
\pgfusepath{clip}%
\pgfsetbuttcap%
\pgfsetmiterjoin%
\definecolor{currentfill}{rgb}{0.754268,0.565033,0.211761}%
\pgfsetfillcolor{currentfill}%
\pgfsetlinewidth{0.000000pt}%
\definecolor{currentstroke}{rgb}{0.000000,0.000000,0.000000}%
\pgfsetstrokecolor{currentstroke}%
\pgfsetstrokeopacity{0.000000}%
\pgfsetdash{}{0pt}%
\pgfpathmoveto{\pgfqpoint{5.383854in}{2.305459in}}%
\pgfpathlineto{\pgfqpoint{5.392791in}{2.305459in}}%
\pgfpathlineto{\pgfqpoint{5.392791in}{2.538428in}}%
\pgfpathlineto{\pgfqpoint{5.383854in}{2.538428in}}%
\pgfpathlineto{\pgfqpoint{5.383854in}{2.305459in}}%
\pgfpathclose%
\pgfusepath{fill}%
\end{pgfscope}%
\begin{pgfscope}%
\pgfpathrectangle{\pgfqpoint{3.722897in}{0.857143in}}{\pgfqpoint{2.627103in}{1.813434in}}%
\pgfusepath{clip}%
\pgfsetbuttcap%
\pgfsetmiterjoin%
\definecolor{currentfill}{rgb}{0.754268,0.565033,0.211761}%
\pgfsetfillcolor{currentfill}%
\pgfsetlinewidth{0.000000pt}%
\definecolor{currentstroke}{rgb}{0.000000,0.000000,0.000000}%
\pgfsetstrokecolor{currentstroke}%
\pgfsetstrokeopacity{0.000000}%
\pgfsetdash{}{0pt}%
\pgfpathmoveto{\pgfqpoint{5.395025in}{2.301603in}}%
\pgfpathlineto{\pgfqpoint{5.403961in}{2.301603in}}%
\pgfpathlineto{\pgfqpoint{5.403961in}{2.532114in}}%
\pgfpathlineto{\pgfqpoint{5.395025in}{2.532114in}}%
\pgfpathlineto{\pgfqpoint{5.395025in}{2.301603in}}%
\pgfpathclose%
\pgfusepath{fill}%
\end{pgfscope}%
\begin{pgfscope}%
\pgfpathrectangle{\pgfqpoint{3.722897in}{0.857143in}}{\pgfqpoint{2.627103in}{1.813434in}}%
\pgfusepath{clip}%
\pgfsetbuttcap%
\pgfsetmiterjoin%
\definecolor{currentfill}{rgb}{0.754268,0.565033,0.211761}%
\pgfsetfillcolor{currentfill}%
\pgfsetlinewidth{0.000000pt}%
\definecolor{currentstroke}{rgb}{0.000000,0.000000,0.000000}%
\pgfsetstrokecolor{currentstroke}%
\pgfsetstrokeopacity{0.000000}%
\pgfsetdash{}{0pt}%
\pgfpathmoveto{\pgfqpoint{5.406196in}{2.288200in}}%
\pgfpathlineto{\pgfqpoint{5.415132in}{2.288200in}}%
\pgfpathlineto{\pgfqpoint{5.415132in}{2.511656in}}%
\pgfpathlineto{\pgfqpoint{5.406196in}{2.511656in}}%
\pgfpathlineto{\pgfqpoint{5.406196in}{2.288200in}}%
\pgfpathclose%
\pgfusepath{fill}%
\end{pgfscope}%
\begin{pgfscope}%
\pgfpathrectangle{\pgfqpoint{3.722897in}{0.857143in}}{\pgfqpoint{2.627103in}{1.813434in}}%
\pgfusepath{clip}%
\pgfsetbuttcap%
\pgfsetmiterjoin%
\definecolor{currentfill}{rgb}{0.754268,0.565033,0.211761}%
\pgfsetfillcolor{currentfill}%
\pgfsetlinewidth{0.000000pt}%
\definecolor{currentstroke}{rgb}{0.000000,0.000000,0.000000}%
\pgfsetstrokecolor{currentstroke}%
\pgfsetstrokeopacity{0.000000}%
\pgfsetdash{}{0pt}%
\pgfpathmoveto{\pgfqpoint{5.417366in}{2.274542in}}%
\pgfpathlineto{\pgfqpoint{5.426303in}{2.274542in}}%
\pgfpathlineto{\pgfqpoint{5.426303in}{2.492897in}}%
\pgfpathlineto{\pgfqpoint{5.417366in}{2.492897in}}%
\pgfpathlineto{\pgfqpoint{5.417366in}{2.274542in}}%
\pgfpathclose%
\pgfusepath{fill}%
\end{pgfscope}%
\begin{pgfscope}%
\pgfpathrectangle{\pgfqpoint{3.722897in}{0.857143in}}{\pgfqpoint{2.627103in}{1.813434in}}%
\pgfusepath{clip}%
\pgfsetbuttcap%
\pgfsetmiterjoin%
\definecolor{currentfill}{rgb}{0.754268,0.565033,0.211761}%
\pgfsetfillcolor{currentfill}%
\pgfsetlinewidth{0.000000pt}%
\definecolor{currentstroke}{rgb}{0.000000,0.000000,0.000000}%
\pgfsetstrokecolor{currentstroke}%
\pgfsetstrokeopacity{0.000000}%
\pgfsetdash{}{0pt}%
\pgfpathmoveto{\pgfqpoint{5.428537in}{2.255804in}}%
\pgfpathlineto{\pgfqpoint{5.437473in}{2.255804in}}%
\pgfpathlineto{\pgfqpoint{5.437473in}{2.464877in}}%
\pgfpathlineto{\pgfqpoint{5.428537in}{2.464877in}}%
\pgfpathlineto{\pgfqpoint{5.428537in}{2.255804in}}%
\pgfpathclose%
\pgfusepath{fill}%
\end{pgfscope}%
\begin{pgfscope}%
\pgfpathrectangle{\pgfqpoint{3.722897in}{0.857143in}}{\pgfqpoint{2.627103in}{1.813434in}}%
\pgfusepath{clip}%
\pgfsetbuttcap%
\pgfsetmiterjoin%
\definecolor{currentfill}{rgb}{0.754268,0.565033,0.211761}%
\pgfsetfillcolor{currentfill}%
\pgfsetlinewidth{0.000000pt}%
\definecolor{currentstroke}{rgb}{0.000000,0.000000,0.000000}%
\pgfsetstrokecolor{currentstroke}%
\pgfsetstrokeopacity{0.000000}%
\pgfsetdash{}{0pt}%
\pgfpathmoveto{\pgfqpoint{5.439707in}{2.238107in}}%
\pgfpathlineto{\pgfqpoint{5.448644in}{2.238107in}}%
\pgfpathlineto{\pgfqpoint{5.448644in}{2.434121in}}%
\pgfpathlineto{\pgfqpoint{5.439707in}{2.434121in}}%
\pgfpathlineto{\pgfqpoint{5.439707in}{2.238107in}}%
\pgfpathclose%
\pgfusepath{fill}%
\end{pgfscope}%
\begin{pgfscope}%
\pgfpathrectangle{\pgfqpoint{3.722897in}{0.857143in}}{\pgfqpoint{2.627103in}{1.813434in}}%
\pgfusepath{clip}%
\pgfsetbuttcap%
\pgfsetmiterjoin%
\definecolor{currentfill}{rgb}{0.754268,0.565033,0.211761}%
\pgfsetfillcolor{currentfill}%
\pgfsetlinewidth{0.000000pt}%
\definecolor{currentstroke}{rgb}{0.000000,0.000000,0.000000}%
\pgfsetstrokecolor{currentstroke}%
\pgfsetstrokeopacity{0.000000}%
\pgfsetdash{}{0pt}%
\pgfpathmoveto{\pgfqpoint{5.450878in}{2.224066in}}%
\pgfpathlineto{\pgfqpoint{5.459814in}{2.224066in}}%
\pgfpathlineto{\pgfqpoint{5.459814in}{2.406954in}}%
\pgfpathlineto{\pgfqpoint{5.450878in}{2.406954in}}%
\pgfpathlineto{\pgfqpoint{5.450878in}{2.224066in}}%
\pgfpathclose%
\pgfusepath{fill}%
\end{pgfscope}%
\begin{pgfscope}%
\pgfpathrectangle{\pgfqpoint{3.722897in}{0.857143in}}{\pgfqpoint{2.627103in}{1.813434in}}%
\pgfusepath{clip}%
\pgfsetbuttcap%
\pgfsetmiterjoin%
\definecolor{currentfill}{rgb}{0.754268,0.565033,0.211761}%
\pgfsetfillcolor{currentfill}%
\pgfsetlinewidth{0.000000pt}%
\definecolor{currentstroke}{rgb}{0.000000,0.000000,0.000000}%
\pgfsetstrokecolor{currentstroke}%
\pgfsetstrokeopacity{0.000000}%
\pgfsetdash{}{0pt}%
\pgfpathmoveto{\pgfqpoint{5.462049in}{2.207765in}}%
\pgfpathlineto{\pgfqpoint{5.470985in}{2.207765in}}%
\pgfpathlineto{\pgfqpoint{5.470985in}{2.373822in}}%
\pgfpathlineto{\pgfqpoint{5.462049in}{2.373822in}}%
\pgfpathlineto{\pgfqpoint{5.462049in}{2.207765in}}%
\pgfpathclose%
\pgfusepath{fill}%
\end{pgfscope}%
\begin{pgfscope}%
\pgfpathrectangle{\pgfqpoint{3.722897in}{0.857143in}}{\pgfqpoint{2.627103in}{1.813434in}}%
\pgfusepath{clip}%
\pgfsetbuttcap%
\pgfsetmiterjoin%
\definecolor{currentfill}{rgb}{0.754268,0.565033,0.211761}%
\pgfsetfillcolor{currentfill}%
\pgfsetlinewidth{0.000000pt}%
\definecolor{currentstroke}{rgb}{0.000000,0.000000,0.000000}%
\pgfsetstrokecolor{currentstroke}%
\pgfsetstrokeopacity{0.000000}%
\pgfsetdash{}{0pt}%
\pgfpathmoveto{\pgfqpoint{5.473219in}{2.193145in}}%
\pgfpathlineto{\pgfqpoint{5.482156in}{2.193145in}}%
\pgfpathlineto{\pgfqpoint{5.482156in}{2.340017in}}%
\pgfpathlineto{\pgfqpoint{5.473219in}{2.340017in}}%
\pgfpathlineto{\pgfqpoint{5.473219in}{2.193145in}}%
\pgfpathclose%
\pgfusepath{fill}%
\end{pgfscope}%
\begin{pgfscope}%
\pgfpathrectangle{\pgfqpoint{3.722897in}{0.857143in}}{\pgfqpoint{2.627103in}{1.813434in}}%
\pgfusepath{clip}%
\pgfsetbuttcap%
\pgfsetmiterjoin%
\definecolor{currentfill}{rgb}{0.754268,0.565033,0.211761}%
\pgfsetfillcolor{currentfill}%
\pgfsetlinewidth{0.000000pt}%
\definecolor{currentstroke}{rgb}{0.000000,0.000000,0.000000}%
\pgfsetstrokecolor{currentstroke}%
\pgfsetstrokeopacity{0.000000}%
\pgfsetdash{}{0pt}%
\pgfpathmoveto{\pgfqpoint{5.484390in}{2.173235in}}%
\pgfpathlineto{\pgfqpoint{5.493326in}{2.173235in}}%
\pgfpathlineto{\pgfqpoint{5.493326in}{2.302076in}}%
\pgfpathlineto{\pgfqpoint{5.484390in}{2.302076in}}%
\pgfpathlineto{\pgfqpoint{5.484390in}{2.173235in}}%
\pgfpathclose%
\pgfusepath{fill}%
\end{pgfscope}%
\begin{pgfscope}%
\pgfpathrectangle{\pgfqpoint{3.722897in}{0.857143in}}{\pgfqpoint{2.627103in}{1.813434in}}%
\pgfusepath{clip}%
\pgfsetbuttcap%
\pgfsetmiterjoin%
\definecolor{currentfill}{rgb}{0.754268,0.565033,0.211761}%
\pgfsetfillcolor{currentfill}%
\pgfsetlinewidth{0.000000pt}%
\definecolor{currentstroke}{rgb}{0.000000,0.000000,0.000000}%
\pgfsetstrokecolor{currentstroke}%
\pgfsetstrokeopacity{0.000000}%
\pgfsetdash{}{0pt}%
\pgfpathmoveto{\pgfqpoint{5.495560in}{2.154574in}}%
\pgfpathlineto{\pgfqpoint{5.504497in}{2.154574in}}%
\pgfpathlineto{\pgfqpoint{5.504497in}{2.271689in}}%
\pgfpathlineto{\pgfqpoint{5.495560in}{2.271689in}}%
\pgfpathlineto{\pgfqpoint{5.495560in}{2.154574in}}%
\pgfpathclose%
\pgfusepath{fill}%
\end{pgfscope}%
\begin{pgfscope}%
\pgfpathrectangle{\pgfqpoint{3.722897in}{0.857143in}}{\pgfqpoint{2.627103in}{1.813434in}}%
\pgfusepath{clip}%
\pgfsetbuttcap%
\pgfsetmiterjoin%
\definecolor{currentfill}{rgb}{0.754268,0.565033,0.211761}%
\pgfsetfillcolor{currentfill}%
\pgfsetlinewidth{0.000000pt}%
\definecolor{currentstroke}{rgb}{0.000000,0.000000,0.000000}%
\pgfsetstrokecolor{currentstroke}%
\pgfsetstrokeopacity{0.000000}%
\pgfsetdash{}{0pt}%
\pgfpathmoveto{\pgfqpoint{5.506731in}{2.129267in}}%
\pgfpathlineto{\pgfqpoint{5.515667in}{2.129267in}}%
\pgfpathlineto{\pgfqpoint{5.515667in}{2.226194in}}%
\pgfpathlineto{\pgfqpoint{5.506731in}{2.226194in}}%
\pgfpathlineto{\pgfqpoint{5.506731in}{2.129267in}}%
\pgfpathclose%
\pgfusepath{fill}%
\end{pgfscope}%
\begin{pgfscope}%
\pgfpathrectangle{\pgfqpoint{3.722897in}{0.857143in}}{\pgfqpoint{2.627103in}{1.813434in}}%
\pgfusepath{clip}%
\pgfsetbuttcap%
\pgfsetmiterjoin%
\definecolor{currentfill}{rgb}{0.754268,0.565033,0.211761}%
\pgfsetfillcolor{currentfill}%
\pgfsetlinewidth{0.000000pt}%
\definecolor{currentstroke}{rgb}{0.000000,0.000000,0.000000}%
\pgfsetstrokecolor{currentstroke}%
\pgfsetstrokeopacity{0.000000}%
\pgfsetdash{}{0pt}%
\pgfpathmoveto{\pgfqpoint{5.517902in}{2.122924in}}%
\pgfpathlineto{\pgfqpoint{5.526838in}{2.122924in}}%
\pgfpathlineto{\pgfqpoint{5.526838in}{2.199794in}}%
\pgfpathlineto{\pgfqpoint{5.517902in}{2.199794in}}%
\pgfpathlineto{\pgfqpoint{5.517902in}{2.122924in}}%
\pgfpathclose%
\pgfusepath{fill}%
\end{pgfscope}%
\begin{pgfscope}%
\pgfpathrectangle{\pgfqpoint{3.722897in}{0.857143in}}{\pgfqpoint{2.627103in}{1.813434in}}%
\pgfusepath{clip}%
\pgfsetbuttcap%
\pgfsetmiterjoin%
\definecolor{currentfill}{rgb}{0.754268,0.565033,0.211761}%
\pgfsetfillcolor{currentfill}%
\pgfsetlinewidth{0.000000pt}%
\definecolor{currentstroke}{rgb}{0.000000,0.000000,0.000000}%
\pgfsetstrokecolor{currentstroke}%
\pgfsetstrokeopacity{0.000000}%
\pgfsetdash{}{0pt}%
\pgfpathmoveto{\pgfqpoint{5.529072in}{2.111323in}}%
\pgfpathlineto{\pgfqpoint{5.538009in}{2.111323in}}%
\pgfpathlineto{\pgfqpoint{5.538009in}{2.165780in}}%
\pgfpathlineto{\pgfqpoint{5.529072in}{2.165780in}}%
\pgfpathlineto{\pgfqpoint{5.529072in}{2.111323in}}%
\pgfpathclose%
\pgfusepath{fill}%
\end{pgfscope}%
\begin{pgfscope}%
\pgfpathrectangle{\pgfqpoint{3.722897in}{0.857143in}}{\pgfqpoint{2.627103in}{1.813434in}}%
\pgfusepath{clip}%
\pgfsetbuttcap%
\pgfsetmiterjoin%
\definecolor{currentfill}{rgb}{0.754268,0.565033,0.211761}%
\pgfsetfillcolor{currentfill}%
\pgfsetlinewidth{0.000000pt}%
\definecolor{currentstroke}{rgb}{0.000000,0.000000,0.000000}%
\pgfsetstrokecolor{currentstroke}%
\pgfsetstrokeopacity{0.000000}%
\pgfsetdash{}{0pt}%
\pgfpathmoveto{\pgfqpoint{5.540243in}{2.112532in}}%
\pgfpathlineto{\pgfqpoint{5.549179in}{2.112532in}}%
\pgfpathlineto{\pgfqpoint{5.549179in}{2.142353in}}%
\pgfpathlineto{\pgfqpoint{5.540243in}{2.142353in}}%
\pgfpathlineto{\pgfqpoint{5.540243in}{2.112532in}}%
\pgfpathclose%
\pgfusepath{fill}%
\end{pgfscope}%
\begin{pgfscope}%
\pgfpathrectangle{\pgfqpoint{3.722897in}{0.857143in}}{\pgfqpoint{2.627103in}{1.813434in}}%
\pgfusepath{clip}%
\pgfsetbuttcap%
\pgfsetmiterjoin%
\definecolor{currentfill}{rgb}{0.754268,0.565033,0.211761}%
\pgfsetfillcolor{currentfill}%
\pgfsetlinewidth{0.000000pt}%
\definecolor{currentstroke}{rgb}{0.000000,0.000000,0.000000}%
\pgfsetstrokecolor{currentstroke}%
\pgfsetstrokeopacity{0.000000}%
\pgfsetdash{}{0pt}%
\pgfpathmoveto{\pgfqpoint{5.551413in}{2.104619in}}%
\pgfpathlineto{\pgfqpoint{5.560350in}{2.104619in}}%
\pgfpathlineto{\pgfqpoint{5.560350in}{2.118832in}}%
\pgfpathlineto{\pgfqpoint{5.551413in}{2.118832in}}%
\pgfpathlineto{\pgfqpoint{5.551413in}{2.104619in}}%
\pgfpathclose%
\pgfusepath{fill}%
\end{pgfscope}%
\begin{pgfscope}%
\pgfpathrectangle{\pgfqpoint{3.722897in}{0.857143in}}{\pgfqpoint{2.627103in}{1.813434in}}%
\pgfusepath{clip}%
\pgfsetbuttcap%
\pgfsetmiterjoin%
\definecolor{currentfill}{rgb}{0.754268,0.565033,0.211761}%
\pgfsetfillcolor{currentfill}%
\pgfsetlinewidth{0.000000pt}%
\definecolor{currentstroke}{rgb}{0.000000,0.000000,0.000000}%
\pgfsetstrokecolor{currentstroke}%
\pgfsetstrokeopacity{0.000000}%
\pgfsetdash{}{0pt}%
\pgfpathmoveto{\pgfqpoint{5.562584in}{1.483138in}}%
\pgfpathlineto{\pgfqpoint{5.571521in}{1.483138in}}%
\pgfpathlineto{\pgfqpoint{5.571521in}{1.480446in}}%
\pgfpathlineto{\pgfqpoint{5.562584in}{1.480446in}}%
\pgfpathlineto{\pgfqpoint{5.562584in}{1.483138in}}%
\pgfpathclose%
\pgfusepath{fill}%
\end{pgfscope}%
\begin{pgfscope}%
\pgfpathrectangle{\pgfqpoint{3.722897in}{0.857143in}}{\pgfqpoint{2.627103in}{1.813434in}}%
\pgfusepath{clip}%
\pgfsetbuttcap%
\pgfsetmiterjoin%
\definecolor{currentfill}{rgb}{0.754268,0.565033,0.211761}%
\pgfsetfillcolor{currentfill}%
\pgfsetlinewidth{0.000000pt}%
\definecolor{currentstroke}{rgb}{0.000000,0.000000,0.000000}%
\pgfsetstrokecolor{currentstroke}%
\pgfsetstrokeopacity{0.000000}%
\pgfsetdash{}{0pt}%
\pgfpathmoveto{\pgfqpoint{5.573755in}{1.491236in}}%
\pgfpathlineto{\pgfqpoint{5.582691in}{1.491236in}}%
\pgfpathlineto{\pgfqpoint{5.582691in}{1.474518in}}%
\pgfpathlineto{\pgfqpoint{5.573755in}{1.474518in}}%
\pgfpathlineto{\pgfqpoint{5.573755in}{1.491236in}}%
\pgfpathclose%
\pgfusepath{fill}%
\end{pgfscope}%
\begin{pgfscope}%
\pgfpathrectangle{\pgfqpoint{3.722897in}{0.857143in}}{\pgfqpoint{2.627103in}{1.813434in}}%
\pgfusepath{clip}%
\pgfsetbuttcap%
\pgfsetmiterjoin%
\definecolor{currentfill}{rgb}{0.754268,0.565033,0.211761}%
\pgfsetfillcolor{currentfill}%
\pgfsetlinewidth{0.000000pt}%
\definecolor{currentstroke}{rgb}{0.000000,0.000000,0.000000}%
\pgfsetstrokecolor{currentstroke}%
\pgfsetstrokeopacity{0.000000}%
\pgfsetdash{}{0pt}%
\pgfpathmoveto{\pgfqpoint{5.584925in}{1.483521in}}%
\pgfpathlineto{\pgfqpoint{5.593862in}{1.483521in}}%
\pgfpathlineto{\pgfqpoint{5.593862in}{1.459405in}}%
\pgfpathlineto{\pgfqpoint{5.584925in}{1.459405in}}%
\pgfpathlineto{\pgfqpoint{5.584925in}{1.483521in}}%
\pgfpathclose%
\pgfusepath{fill}%
\end{pgfscope}%
\begin{pgfscope}%
\pgfpathrectangle{\pgfqpoint{3.722897in}{0.857143in}}{\pgfqpoint{2.627103in}{1.813434in}}%
\pgfusepath{clip}%
\pgfsetbuttcap%
\pgfsetmiterjoin%
\definecolor{currentfill}{rgb}{0.754268,0.565033,0.211761}%
\pgfsetfillcolor{currentfill}%
\pgfsetlinewidth{0.000000pt}%
\definecolor{currentstroke}{rgb}{0.000000,0.000000,0.000000}%
\pgfsetstrokecolor{currentstroke}%
\pgfsetstrokeopacity{0.000000}%
\pgfsetdash{}{0pt}%
\pgfpathmoveto{\pgfqpoint{5.596096in}{1.482098in}}%
\pgfpathlineto{\pgfqpoint{5.605032in}{1.482098in}}%
\pgfpathlineto{\pgfqpoint{5.605032in}{1.446711in}}%
\pgfpathlineto{\pgfqpoint{5.596096in}{1.446711in}}%
\pgfpathlineto{\pgfqpoint{5.596096in}{1.482098in}}%
\pgfpathclose%
\pgfusepath{fill}%
\end{pgfscope}%
\begin{pgfscope}%
\pgfpathrectangle{\pgfqpoint{3.722897in}{0.857143in}}{\pgfqpoint{2.627103in}{1.813434in}}%
\pgfusepath{clip}%
\pgfsetbuttcap%
\pgfsetmiterjoin%
\definecolor{currentfill}{rgb}{0.754268,0.565033,0.211761}%
\pgfsetfillcolor{currentfill}%
\pgfsetlinewidth{0.000000pt}%
\definecolor{currentstroke}{rgb}{0.000000,0.000000,0.000000}%
\pgfsetstrokecolor{currentstroke}%
\pgfsetstrokeopacity{0.000000}%
\pgfsetdash{}{0pt}%
\pgfpathmoveto{\pgfqpoint{5.607266in}{1.485848in}}%
\pgfpathlineto{\pgfqpoint{5.616203in}{1.485848in}}%
\pgfpathlineto{\pgfqpoint{5.616203in}{1.445675in}}%
\pgfpathlineto{\pgfqpoint{5.607266in}{1.445675in}}%
\pgfpathlineto{\pgfqpoint{5.607266in}{1.485848in}}%
\pgfpathclose%
\pgfusepath{fill}%
\end{pgfscope}%
\begin{pgfscope}%
\pgfpathrectangle{\pgfqpoint{3.722897in}{0.857143in}}{\pgfqpoint{2.627103in}{1.813434in}}%
\pgfusepath{clip}%
\pgfsetbuttcap%
\pgfsetmiterjoin%
\definecolor{currentfill}{rgb}{0.754268,0.565033,0.211761}%
\pgfsetfillcolor{currentfill}%
\pgfsetlinewidth{0.000000pt}%
\definecolor{currentstroke}{rgb}{0.000000,0.000000,0.000000}%
\pgfsetstrokecolor{currentstroke}%
\pgfsetstrokeopacity{0.000000}%
\pgfsetdash{}{0pt}%
\pgfpathmoveto{\pgfqpoint{5.618437in}{1.494282in}}%
\pgfpathlineto{\pgfqpoint{5.627374in}{1.494282in}}%
\pgfpathlineto{\pgfqpoint{5.627374in}{1.456498in}}%
\pgfpathlineto{\pgfqpoint{5.618437in}{1.456498in}}%
\pgfpathlineto{\pgfqpoint{5.618437in}{1.494282in}}%
\pgfpathclose%
\pgfusepath{fill}%
\end{pgfscope}%
\begin{pgfscope}%
\pgfpathrectangle{\pgfqpoint{3.722897in}{0.857143in}}{\pgfqpoint{2.627103in}{1.813434in}}%
\pgfusepath{clip}%
\pgfsetbuttcap%
\pgfsetmiterjoin%
\definecolor{currentfill}{rgb}{0.754268,0.565033,0.211761}%
\pgfsetfillcolor{currentfill}%
\pgfsetlinewidth{0.000000pt}%
\definecolor{currentstroke}{rgb}{0.000000,0.000000,0.000000}%
\pgfsetstrokecolor{currentstroke}%
\pgfsetstrokeopacity{0.000000}%
\pgfsetdash{}{0pt}%
\pgfpathmoveto{\pgfqpoint{5.629608in}{1.497588in}}%
\pgfpathlineto{\pgfqpoint{5.638544in}{1.497588in}}%
\pgfpathlineto{\pgfqpoint{5.638544in}{1.458745in}}%
\pgfpathlineto{\pgfqpoint{5.629608in}{1.458745in}}%
\pgfpathlineto{\pgfqpoint{5.629608in}{1.497588in}}%
\pgfpathclose%
\pgfusepath{fill}%
\end{pgfscope}%
\begin{pgfscope}%
\pgfpathrectangle{\pgfqpoint{3.722897in}{0.857143in}}{\pgfqpoint{2.627103in}{1.813434in}}%
\pgfusepath{clip}%
\pgfsetbuttcap%
\pgfsetmiterjoin%
\definecolor{currentfill}{rgb}{0.754268,0.565033,0.211761}%
\pgfsetfillcolor{currentfill}%
\pgfsetlinewidth{0.000000pt}%
\definecolor{currentstroke}{rgb}{0.000000,0.000000,0.000000}%
\pgfsetstrokecolor{currentstroke}%
\pgfsetstrokeopacity{0.000000}%
\pgfsetdash{}{0pt}%
\pgfpathmoveto{\pgfqpoint{5.640778in}{1.498927in}}%
\pgfpathlineto{\pgfqpoint{5.649715in}{1.498927in}}%
\pgfpathlineto{\pgfqpoint{5.649715in}{1.467207in}}%
\pgfpathlineto{\pgfqpoint{5.640778in}{1.467207in}}%
\pgfpathlineto{\pgfqpoint{5.640778in}{1.498927in}}%
\pgfpathclose%
\pgfusepath{fill}%
\end{pgfscope}%
\begin{pgfscope}%
\pgfpathrectangle{\pgfqpoint{3.722897in}{0.857143in}}{\pgfqpoint{2.627103in}{1.813434in}}%
\pgfusepath{clip}%
\pgfsetbuttcap%
\pgfsetmiterjoin%
\definecolor{currentfill}{rgb}{0.754268,0.565033,0.211761}%
\pgfsetfillcolor{currentfill}%
\pgfsetlinewidth{0.000000pt}%
\definecolor{currentstroke}{rgb}{0.000000,0.000000,0.000000}%
\pgfsetstrokecolor{currentstroke}%
\pgfsetstrokeopacity{0.000000}%
\pgfsetdash{}{0pt}%
\pgfpathmoveto{\pgfqpoint{5.651949in}{1.505302in}}%
\pgfpathlineto{\pgfqpoint{5.660885in}{1.505302in}}%
\pgfpathlineto{\pgfqpoint{5.660885in}{1.492936in}}%
\pgfpathlineto{\pgfqpoint{5.651949in}{1.492936in}}%
\pgfpathlineto{\pgfqpoint{5.651949in}{1.505302in}}%
\pgfpathclose%
\pgfusepath{fill}%
\end{pgfscope}%
\begin{pgfscope}%
\pgfpathrectangle{\pgfqpoint{3.722897in}{0.857143in}}{\pgfqpoint{2.627103in}{1.813434in}}%
\pgfusepath{clip}%
\pgfsetbuttcap%
\pgfsetmiterjoin%
\definecolor{currentfill}{rgb}{0.754268,0.565033,0.211761}%
\pgfsetfillcolor{currentfill}%
\pgfsetlinewidth{0.000000pt}%
\definecolor{currentstroke}{rgb}{0.000000,0.000000,0.000000}%
\pgfsetstrokecolor{currentstroke}%
\pgfsetstrokeopacity{0.000000}%
\pgfsetdash{}{0pt}%
\pgfpathmoveto{\pgfqpoint{5.663119in}{1.511296in}}%
\pgfpathlineto{\pgfqpoint{5.672056in}{1.511296in}}%
\pgfpathlineto{\pgfqpoint{5.672056in}{1.496017in}}%
\pgfpathlineto{\pgfqpoint{5.663119in}{1.496017in}}%
\pgfpathlineto{\pgfqpoint{5.663119in}{1.511296in}}%
\pgfpathclose%
\pgfusepath{fill}%
\end{pgfscope}%
\begin{pgfscope}%
\pgfpathrectangle{\pgfqpoint{3.722897in}{0.857143in}}{\pgfqpoint{2.627103in}{1.813434in}}%
\pgfusepath{clip}%
\pgfsetbuttcap%
\pgfsetmiterjoin%
\definecolor{currentfill}{rgb}{0.754268,0.565033,0.211761}%
\pgfsetfillcolor{currentfill}%
\pgfsetlinewidth{0.000000pt}%
\definecolor{currentstroke}{rgb}{0.000000,0.000000,0.000000}%
\pgfsetstrokecolor{currentstroke}%
\pgfsetstrokeopacity{0.000000}%
\pgfsetdash{}{0pt}%
\pgfpathmoveto{\pgfqpoint{5.674290in}{1.519882in}}%
\pgfpathlineto{\pgfqpoint{5.683227in}{1.519882in}}%
\pgfpathlineto{\pgfqpoint{5.683227in}{1.510705in}}%
\pgfpathlineto{\pgfqpoint{5.674290in}{1.510705in}}%
\pgfpathlineto{\pgfqpoint{5.674290in}{1.519882in}}%
\pgfpathclose%
\pgfusepath{fill}%
\end{pgfscope}%
\begin{pgfscope}%
\pgfpathrectangle{\pgfqpoint{3.722897in}{0.857143in}}{\pgfqpoint{2.627103in}{1.813434in}}%
\pgfusepath{clip}%
\pgfsetbuttcap%
\pgfsetmiterjoin%
\definecolor{currentfill}{rgb}{0.754268,0.565033,0.211761}%
\pgfsetfillcolor{currentfill}%
\pgfsetlinewidth{0.000000pt}%
\definecolor{currentstroke}{rgb}{0.000000,0.000000,0.000000}%
\pgfsetstrokecolor{currentstroke}%
\pgfsetstrokeopacity{0.000000}%
\pgfsetdash{}{0pt}%
\pgfpathmoveto{\pgfqpoint{5.685461in}{2.148429in}}%
\pgfpathlineto{\pgfqpoint{5.694397in}{2.148429in}}%
\pgfpathlineto{\pgfqpoint{5.694397in}{2.150079in}}%
\pgfpathlineto{\pgfqpoint{5.685461in}{2.150079in}}%
\pgfpathlineto{\pgfqpoint{5.685461in}{2.148429in}}%
\pgfpathclose%
\pgfusepath{fill}%
\end{pgfscope}%
\begin{pgfscope}%
\pgfpathrectangle{\pgfqpoint{3.722897in}{0.857143in}}{\pgfqpoint{2.627103in}{1.813434in}}%
\pgfusepath{clip}%
\pgfsetbuttcap%
\pgfsetmiterjoin%
\definecolor{currentfill}{rgb}{0.754268,0.565033,0.211761}%
\pgfsetfillcolor{currentfill}%
\pgfsetlinewidth{0.000000pt}%
\definecolor{currentstroke}{rgb}{0.000000,0.000000,0.000000}%
\pgfsetstrokecolor{currentstroke}%
\pgfsetstrokeopacity{0.000000}%
\pgfsetdash{}{0pt}%
\pgfpathmoveto{\pgfqpoint{5.696631in}{2.145919in}}%
\pgfpathlineto{\pgfqpoint{5.705568in}{2.145919in}}%
\pgfpathlineto{\pgfqpoint{5.705568in}{2.159834in}}%
\pgfpathlineto{\pgfqpoint{5.696631in}{2.159834in}}%
\pgfpathlineto{\pgfqpoint{5.696631in}{2.145919in}}%
\pgfpathclose%
\pgfusepath{fill}%
\end{pgfscope}%
\begin{pgfscope}%
\pgfpathrectangle{\pgfqpoint{3.722897in}{0.857143in}}{\pgfqpoint{2.627103in}{1.813434in}}%
\pgfusepath{clip}%
\pgfsetbuttcap%
\pgfsetmiterjoin%
\definecolor{currentfill}{rgb}{0.754268,0.565033,0.211761}%
\pgfsetfillcolor{currentfill}%
\pgfsetlinewidth{0.000000pt}%
\definecolor{currentstroke}{rgb}{0.000000,0.000000,0.000000}%
\pgfsetstrokecolor{currentstroke}%
\pgfsetstrokeopacity{0.000000}%
\pgfsetdash{}{0pt}%
\pgfpathmoveto{\pgfqpoint{5.707802in}{2.134875in}}%
\pgfpathlineto{\pgfqpoint{5.716738in}{2.134875in}}%
\pgfpathlineto{\pgfqpoint{5.716738in}{2.157333in}}%
\pgfpathlineto{\pgfqpoint{5.707802in}{2.157333in}}%
\pgfpathlineto{\pgfqpoint{5.707802in}{2.134875in}}%
\pgfpathclose%
\pgfusepath{fill}%
\end{pgfscope}%
\begin{pgfscope}%
\pgfpathrectangle{\pgfqpoint{3.722897in}{0.857143in}}{\pgfqpoint{2.627103in}{1.813434in}}%
\pgfusepath{clip}%
\pgfsetbuttcap%
\pgfsetmiterjoin%
\definecolor{currentfill}{rgb}{0.754268,0.565033,0.211761}%
\pgfsetfillcolor{currentfill}%
\pgfsetlinewidth{0.000000pt}%
\definecolor{currentstroke}{rgb}{0.000000,0.000000,0.000000}%
\pgfsetstrokecolor{currentstroke}%
\pgfsetstrokeopacity{0.000000}%
\pgfsetdash{}{0pt}%
\pgfpathmoveto{\pgfqpoint{5.718972in}{2.109062in}}%
\pgfpathlineto{\pgfqpoint{5.727909in}{2.109062in}}%
\pgfpathlineto{\pgfqpoint{5.727909in}{2.126468in}}%
\pgfpathlineto{\pgfqpoint{5.718972in}{2.126468in}}%
\pgfpathlineto{\pgfqpoint{5.718972in}{2.109062in}}%
\pgfpathclose%
\pgfusepath{fill}%
\end{pgfscope}%
\begin{pgfscope}%
\pgfpathrectangle{\pgfqpoint{3.722897in}{0.857143in}}{\pgfqpoint{2.627103in}{1.813434in}}%
\pgfusepath{clip}%
\pgfsetbuttcap%
\pgfsetmiterjoin%
\definecolor{currentfill}{rgb}{0.754268,0.565033,0.211761}%
\pgfsetfillcolor{currentfill}%
\pgfsetlinewidth{0.000000pt}%
\definecolor{currentstroke}{rgb}{0.000000,0.000000,0.000000}%
\pgfsetstrokecolor{currentstroke}%
\pgfsetstrokeopacity{0.000000}%
\pgfsetdash{}{0pt}%
\pgfpathmoveto{\pgfqpoint{5.730143in}{2.106504in}}%
\pgfpathlineto{\pgfqpoint{5.739080in}{2.106504in}}%
\pgfpathlineto{\pgfqpoint{5.739080in}{2.107973in}}%
\pgfpathlineto{\pgfqpoint{5.730143in}{2.107973in}}%
\pgfpathlineto{\pgfqpoint{5.730143in}{2.106504in}}%
\pgfpathclose%
\pgfusepath{fill}%
\end{pgfscope}%
\begin{pgfscope}%
\pgfpathrectangle{\pgfqpoint{3.722897in}{0.857143in}}{\pgfqpoint{2.627103in}{1.813434in}}%
\pgfusepath{clip}%
\pgfsetbuttcap%
\pgfsetmiterjoin%
\definecolor{currentfill}{rgb}{0.754268,0.565033,0.211761}%
\pgfsetfillcolor{currentfill}%
\pgfsetlinewidth{0.000000pt}%
\definecolor{currentstroke}{rgb}{0.000000,0.000000,0.000000}%
\pgfsetstrokecolor{currentstroke}%
\pgfsetstrokeopacity{0.000000}%
\pgfsetdash{}{0pt}%
\pgfpathmoveto{\pgfqpoint{5.741314in}{1.476209in}}%
\pgfpathlineto{\pgfqpoint{5.750250in}{1.476209in}}%
\pgfpathlineto{\pgfqpoint{5.750250in}{1.474303in}}%
\pgfpathlineto{\pgfqpoint{5.741314in}{1.474303in}}%
\pgfpathlineto{\pgfqpoint{5.741314in}{1.476209in}}%
\pgfpathclose%
\pgfusepath{fill}%
\end{pgfscope}%
\begin{pgfscope}%
\pgfpathrectangle{\pgfqpoint{3.722897in}{0.857143in}}{\pgfqpoint{2.627103in}{1.813434in}}%
\pgfusepath{clip}%
\pgfsetbuttcap%
\pgfsetmiterjoin%
\definecolor{currentfill}{rgb}{0.754268,0.565033,0.211761}%
\pgfsetfillcolor{currentfill}%
\pgfsetlinewidth{0.000000pt}%
\definecolor{currentstroke}{rgb}{0.000000,0.000000,0.000000}%
\pgfsetstrokecolor{currentstroke}%
\pgfsetstrokeopacity{0.000000}%
\pgfsetdash{}{0pt}%
\pgfpathmoveto{\pgfqpoint{5.752484in}{2.064735in}}%
\pgfpathlineto{\pgfqpoint{5.761421in}{2.064735in}}%
\pgfpathlineto{\pgfqpoint{5.761421in}{2.065823in}}%
\pgfpathlineto{\pgfqpoint{5.752484in}{2.065823in}}%
\pgfpathlineto{\pgfqpoint{5.752484in}{2.064735in}}%
\pgfpathclose%
\pgfusepath{fill}%
\end{pgfscope}%
\begin{pgfscope}%
\pgfpathrectangle{\pgfqpoint{3.722897in}{0.857143in}}{\pgfqpoint{2.627103in}{1.813434in}}%
\pgfusepath{clip}%
\pgfsetbuttcap%
\pgfsetmiterjoin%
\definecolor{currentfill}{rgb}{0.754268,0.565033,0.211761}%
\pgfsetfillcolor{currentfill}%
\pgfsetlinewidth{0.000000pt}%
\definecolor{currentstroke}{rgb}{0.000000,0.000000,0.000000}%
\pgfsetstrokecolor{currentstroke}%
\pgfsetstrokeopacity{0.000000}%
\pgfsetdash{}{0pt}%
\pgfpathmoveto{\pgfqpoint{5.763655in}{2.019813in}}%
\pgfpathlineto{\pgfqpoint{5.772591in}{2.019813in}}%
\pgfpathlineto{\pgfqpoint{5.772591in}{2.027039in}}%
\pgfpathlineto{\pgfqpoint{5.763655in}{2.027039in}}%
\pgfpathlineto{\pgfqpoint{5.763655in}{2.019813in}}%
\pgfpathclose%
\pgfusepath{fill}%
\end{pgfscope}%
\begin{pgfscope}%
\pgfpathrectangle{\pgfqpoint{3.722897in}{0.857143in}}{\pgfqpoint{2.627103in}{1.813434in}}%
\pgfusepath{clip}%
\pgfsetbuttcap%
\pgfsetmiterjoin%
\definecolor{currentfill}{rgb}{0.754268,0.565033,0.211761}%
\pgfsetfillcolor{currentfill}%
\pgfsetlinewidth{0.000000pt}%
\definecolor{currentstroke}{rgb}{0.000000,0.000000,0.000000}%
\pgfsetstrokecolor{currentstroke}%
\pgfsetstrokeopacity{0.000000}%
\pgfsetdash{}{0pt}%
\pgfpathmoveto{\pgfqpoint{5.774826in}{1.990008in}}%
\pgfpathlineto{\pgfqpoint{5.783762in}{1.990008in}}%
\pgfpathlineto{\pgfqpoint{5.783762in}{1.993353in}}%
\pgfpathlineto{\pgfqpoint{5.774826in}{1.993353in}}%
\pgfpathlineto{\pgfqpoint{5.774826in}{1.990008in}}%
\pgfpathclose%
\pgfusepath{fill}%
\end{pgfscope}%
\begin{pgfscope}%
\pgfpathrectangle{\pgfqpoint{3.722897in}{0.857143in}}{\pgfqpoint{2.627103in}{1.813434in}}%
\pgfusepath{clip}%
\pgfsetbuttcap%
\pgfsetmiterjoin%
\definecolor{currentfill}{rgb}{0.754268,0.565033,0.211761}%
\pgfsetfillcolor{currentfill}%
\pgfsetlinewidth{0.000000pt}%
\definecolor{currentstroke}{rgb}{0.000000,0.000000,0.000000}%
\pgfsetstrokecolor{currentstroke}%
\pgfsetstrokeopacity{0.000000}%
\pgfsetdash{}{0pt}%
\pgfpathmoveto{\pgfqpoint{5.785996in}{1.455631in}}%
\pgfpathlineto{\pgfqpoint{5.794933in}{1.455631in}}%
\pgfpathlineto{\pgfqpoint{5.794933in}{1.448280in}}%
\pgfpathlineto{\pgfqpoint{5.785996in}{1.448280in}}%
\pgfpathlineto{\pgfqpoint{5.785996in}{1.455631in}}%
\pgfpathclose%
\pgfusepath{fill}%
\end{pgfscope}%
\begin{pgfscope}%
\pgfpathrectangle{\pgfqpoint{3.722897in}{0.857143in}}{\pgfqpoint{2.627103in}{1.813434in}}%
\pgfusepath{clip}%
\pgfsetbuttcap%
\pgfsetmiterjoin%
\definecolor{currentfill}{rgb}{0.754268,0.565033,0.211761}%
\pgfsetfillcolor{currentfill}%
\pgfsetlinewidth{0.000000pt}%
\definecolor{currentstroke}{rgb}{0.000000,0.000000,0.000000}%
\pgfsetstrokecolor{currentstroke}%
\pgfsetstrokeopacity{0.000000}%
\pgfsetdash{}{0pt}%
\pgfpathmoveto{\pgfqpoint{5.797167in}{1.456144in}}%
\pgfpathlineto{\pgfqpoint{5.806103in}{1.456144in}}%
\pgfpathlineto{\pgfqpoint{5.806103in}{1.440085in}}%
\pgfpathlineto{\pgfqpoint{5.797167in}{1.440085in}}%
\pgfpathlineto{\pgfqpoint{5.797167in}{1.456144in}}%
\pgfpathclose%
\pgfusepath{fill}%
\end{pgfscope}%
\begin{pgfscope}%
\pgfpathrectangle{\pgfqpoint{3.722897in}{0.857143in}}{\pgfqpoint{2.627103in}{1.813434in}}%
\pgfusepath{clip}%
\pgfsetbuttcap%
\pgfsetmiterjoin%
\definecolor{currentfill}{rgb}{0.754268,0.565033,0.211761}%
\pgfsetfillcolor{currentfill}%
\pgfsetlinewidth{0.000000pt}%
\definecolor{currentstroke}{rgb}{0.000000,0.000000,0.000000}%
\pgfsetstrokecolor{currentstroke}%
\pgfsetstrokeopacity{0.000000}%
\pgfsetdash{}{0pt}%
\pgfpathmoveto{\pgfqpoint{5.808337in}{1.460749in}}%
\pgfpathlineto{\pgfqpoint{5.817274in}{1.460749in}}%
\pgfpathlineto{\pgfqpoint{5.817274in}{1.429762in}}%
\pgfpathlineto{\pgfqpoint{5.808337in}{1.429762in}}%
\pgfpathlineto{\pgfqpoint{5.808337in}{1.460749in}}%
\pgfpathclose%
\pgfusepath{fill}%
\end{pgfscope}%
\begin{pgfscope}%
\pgfpathrectangle{\pgfqpoint{3.722897in}{0.857143in}}{\pgfqpoint{2.627103in}{1.813434in}}%
\pgfusepath{clip}%
\pgfsetbuttcap%
\pgfsetmiterjoin%
\definecolor{currentfill}{rgb}{0.754268,0.565033,0.211761}%
\pgfsetfillcolor{currentfill}%
\pgfsetlinewidth{0.000000pt}%
\definecolor{currentstroke}{rgb}{0.000000,0.000000,0.000000}%
\pgfsetstrokecolor{currentstroke}%
\pgfsetstrokeopacity{0.000000}%
\pgfsetdash{}{0pt}%
\pgfpathmoveto{\pgfqpoint{5.819508in}{1.468002in}}%
\pgfpathlineto{\pgfqpoint{5.828444in}{1.468002in}}%
\pgfpathlineto{\pgfqpoint{5.828444in}{1.429740in}}%
\pgfpathlineto{\pgfqpoint{5.819508in}{1.429740in}}%
\pgfpathlineto{\pgfqpoint{5.819508in}{1.468002in}}%
\pgfpathclose%
\pgfusepath{fill}%
\end{pgfscope}%
\begin{pgfscope}%
\pgfpathrectangle{\pgfqpoint{3.722897in}{0.857143in}}{\pgfqpoint{2.627103in}{1.813434in}}%
\pgfusepath{clip}%
\pgfsetbuttcap%
\pgfsetmiterjoin%
\definecolor{currentfill}{rgb}{0.754268,0.565033,0.211761}%
\pgfsetfillcolor{currentfill}%
\pgfsetlinewidth{0.000000pt}%
\definecolor{currentstroke}{rgb}{0.000000,0.000000,0.000000}%
\pgfsetstrokecolor{currentstroke}%
\pgfsetstrokeopacity{0.000000}%
\pgfsetdash{}{0pt}%
\pgfpathmoveto{\pgfqpoint{5.830679in}{1.454935in}}%
\pgfpathlineto{\pgfqpoint{5.839615in}{1.454935in}}%
\pgfpathlineto{\pgfqpoint{5.839615in}{1.400096in}}%
\pgfpathlineto{\pgfqpoint{5.830679in}{1.400096in}}%
\pgfpathlineto{\pgfqpoint{5.830679in}{1.454935in}}%
\pgfpathclose%
\pgfusepath{fill}%
\end{pgfscope}%
\begin{pgfscope}%
\pgfpathrectangle{\pgfqpoint{3.722897in}{0.857143in}}{\pgfqpoint{2.627103in}{1.813434in}}%
\pgfusepath{clip}%
\pgfsetbuttcap%
\pgfsetmiterjoin%
\definecolor{currentfill}{rgb}{0.754268,0.565033,0.211761}%
\pgfsetfillcolor{currentfill}%
\pgfsetlinewidth{0.000000pt}%
\definecolor{currentstroke}{rgb}{0.000000,0.000000,0.000000}%
\pgfsetstrokecolor{currentstroke}%
\pgfsetstrokeopacity{0.000000}%
\pgfsetdash{}{0pt}%
\pgfpathmoveto{\pgfqpoint{5.841849in}{1.419675in}}%
\pgfpathlineto{\pgfqpoint{5.850786in}{1.419675in}}%
\pgfpathlineto{\pgfqpoint{5.850786in}{1.351820in}}%
\pgfpathlineto{\pgfqpoint{5.841849in}{1.351820in}}%
\pgfpathlineto{\pgfqpoint{5.841849in}{1.419675in}}%
\pgfpathclose%
\pgfusepath{fill}%
\end{pgfscope}%
\begin{pgfscope}%
\pgfpathrectangle{\pgfqpoint{3.722897in}{0.857143in}}{\pgfqpoint{2.627103in}{1.813434in}}%
\pgfusepath{clip}%
\pgfsetbuttcap%
\pgfsetmiterjoin%
\definecolor{currentfill}{rgb}{0.754268,0.565033,0.211761}%
\pgfsetfillcolor{currentfill}%
\pgfsetlinewidth{0.000000pt}%
\definecolor{currentstroke}{rgb}{0.000000,0.000000,0.000000}%
\pgfsetstrokecolor{currentstroke}%
\pgfsetstrokeopacity{0.000000}%
\pgfsetdash{}{0pt}%
\pgfpathmoveto{\pgfqpoint{5.853020in}{1.397819in}}%
\pgfpathlineto{\pgfqpoint{5.861956in}{1.397819in}}%
\pgfpathlineto{\pgfqpoint{5.861956in}{1.312009in}}%
\pgfpathlineto{\pgfqpoint{5.853020in}{1.312009in}}%
\pgfpathlineto{\pgfqpoint{5.853020in}{1.397819in}}%
\pgfpathclose%
\pgfusepath{fill}%
\end{pgfscope}%
\begin{pgfscope}%
\pgfpathrectangle{\pgfqpoint{3.722897in}{0.857143in}}{\pgfqpoint{2.627103in}{1.813434in}}%
\pgfusepath{clip}%
\pgfsetbuttcap%
\pgfsetmiterjoin%
\definecolor{currentfill}{rgb}{0.754268,0.565033,0.211761}%
\pgfsetfillcolor{currentfill}%
\pgfsetlinewidth{0.000000pt}%
\definecolor{currentstroke}{rgb}{0.000000,0.000000,0.000000}%
\pgfsetstrokecolor{currentstroke}%
\pgfsetstrokeopacity{0.000000}%
\pgfsetdash{}{0pt}%
\pgfpathmoveto{\pgfqpoint{5.864190in}{1.383814in}}%
\pgfpathlineto{\pgfqpoint{5.873127in}{1.383814in}}%
\pgfpathlineto{\pgfqpoint{5.873127in}{1.282282in}}%
\pgfpathlineto{\pgfqpoint{5.864190in}{1.282282in}}%
\pgfpathlineto{\pgfqpoint{5.864190in}{1.383814in}}%
\pgfpathclose%
\pgfusepath{fill}%
\end{pgfscope}%
\begin{pgfscope}%
\pgfpathrectangle{\pgfqpoint{3.722897in}{0.857143in}}{\pgfqpoint{2.627103in}{1.813434in}}%
\pgfusepath{clip}%
\pgfsetbuttcap%
\pgfsetmiterjoin%
\definecolor{currentfill}{rgb}{0.754268,0.565033,0.211761}%
\pgfsetfillcolor{currentfill}%
\pgfsetlinewidth{0.000000pt}%
\definecolor{currentstroke}{rgb}{0.000000,0.000000,0.000000}%
\pgfsetstrokecolor{currentstroke}%
\pgfsetstrokeopacity{0.000000}%
\pgfsetdash{}{0pt}%
\pgfpathmoveto{\pgfqpoint{5.875361in}{1.356418in}}%
\pgfpathlineto{\pgfqpoint{5.884297in}{1.356418in}}%
\pgfpathlineto{\pgfqpoint{5.884297in}{1.257543in}}%
\pgfpathlineto{\pgfqpoint{5.875361in}{1.257543in}}%
\pgfpathlineto{\pgfqpoint{5.875361in}{1.356418in}}%
\pgfpathclose%
\pgfusepath{fill}%
\end{pgfscope}%
\begin{pgfscope}%
\pgfpathrectangle{\pgfqpoint{3.722897in}{0.857143in}}{\pgfqpoint{2.627103in}{1.813434in}}%
\pgfusepath{clip}%
\pgfsetbuttcap%
\pgfsetmiterjoin%
\definecolor{currentfill}{rgb}{0.754268,0.565033,0.211761}%
\pgfsetfillcolor{currentfill}%
\pgfsetlinewidth{0.000000pt}%
\definecolor{currentstroke}{rgb}{0.000000,0.000000,0.000000}%
\pgfsetstrokecolor{currentstroke}%
\pgfsetstrokeopacity{0.000000}%
\pgfsetdash{}{0pt}%
\pgfpathmoveto{\pgfqpoint{5.886532in}{1.359990in}}%
\pgfpathlineto{\pgfqpoint{5.895468in}{1.359990in}}%
\pgfpathlineto{\pgfqpoint{5.895468in}{1.246848in}}%
\pgfpathlineto{\pgfqpoint{5.886532in}{1.246848in}}%
\pgfpathlineto{\pgfqpoint{5.886532in}{1.359990in}}%
\pgfpathclose%
\pgfusepath{fill}%
\end{pgfscope}%
\begin{pgfscope}%
\pgfpathrectangle{\pgfqpoint{3.722897in}{0.857143in}}{\pgfqpoint{2.627103in}{1.813434in}}%
\pgfusepath{clip}%
\pgfsetbuttcap%
\pgfsetmiterjoin%
\definecolor{currentfill}{rgb}{0.754268,0.565033,0.211761}%
\pgfsetfillcolor{currentfill}%
\pgfsetlinewidth{0.000000pt}%
\definecolor{currentstroke}{rgb}{0.000000,0.000000,0.000000}%
\pgfsetstrokecolor{currentstroke}%
\pgfsetstrokeopacity{0.000000}%
\pgfsetdash{}{0pt}%
\pgfpathmoveto{\pgfqpoint{5.897702in}{1.349603in}}%
\pgfpathlineto{\pgfqpoint{5.906639in}{1.349603in}}%
\pgfpathlineto{\pgfqpoint{5.906639in}{1.229741in}}%
\pgfpathlineto{\pgfqpoint{5.897702in}{1.229741in}}%
\pgfpathlineto{\pgfqpoint{5.897702in}{1.349603in}}%
\pgfpathclose%
\pgfusepath{fill}%
\end{pgfscope}%
\begin{pgfscope}%
\pgfpathrectangle{\pgfqpoint{3.722897in}{0.857143in}}{\pgfqpoint{2.627103in}{1.813434in}}%
\pgfusepath{clip}%
\pgfsetbuttcap%
\pgfsetmiterjoin%
\definecolor{currentfill}{rgb}{0.754268,0.565033,0.211761}%
\pgfsetfillcolor{currentfill}%
\pgfsetlinewidth{0.000000pt}%
\definecolor{currentstroke}{rgb}{0.000000,0.000000,0.000000}%
\pgfsetstrokecolor{currentstroke}%
\pgfsetstrokeopacity{0.000000}%
\pgfsetdash{}{0pt}%
\pgfpathmoveto{\pgfqpoint{5.908873in}{1.350629in}}%
\pgfpathlineto{\pgfqpoint{5.917809in}{1.350629in}}%
\pgfpathlineto{\pgfqpoint{5.917809in}{1.219698in}}%
\pgfpathlineto{\pgfqpoint{5.908873in}{1.219698in}}%
\pgfpathlineto{\pgfqpoint{5.908873in}{1.350629in}}%
\pgfpathclose%
\pgfusepath{fill}%
\end{pgfscope}%
\begin{pgfscope}%
\pgfpathrectangle{\pgfqpoint{3.722897in}{0.857143in}}{\pgfqpoint{2.627103in}{1.813434in}}%
\pgfusepath{clip}%
\pgfsetbuttcap%
\pgfsetmiterjoin%
\definecolor{currentfill}{rgb}{0.754268,0.565033,0.211761}%
\pgfsetfillcolor{currentfill}%
\pgfsetlinewidth{0.000000pt}%
\definecolor{currentstroke}{rgb}{0.000000,0.000000,0.000000}%
\pgfsetstrokecolor{currentstroke}%
\pgfsetstrokeopacity{0.000000}%
\pgfsetdash{}{0pt}%
\pgfpathmoveto{\pgfqpoint{5.920043in}{1.361419in}}%
\pgfpathlineto{\pgfqpoint{5.928980in}{1.361419in}}%
\pgfpathlineto{\pgfqpoint{5.928980in}{1.221602in}}%
\pgfpathlineto{\pgfqpoint{5.920043in}{1.221602in}}%
\pgfpathlineto{\pgfqpoint{5.920043in}{1.361419in}}%
\pgfpathclose%
\pgfusepath{fill}%
\end{pgfscope}%
\begin{pgfscope}%
\pgfpathrectangle{\pgfqpoint{3.722897in}{0.857143in}}{\pgfqpoint{2.627103in}{1.813434in}}%
\pgfusepath{clip}%
\pgfsetbuttcap%
\pgfsetmiterjoin%
\definecolor{currentfill}{rgb}{0.754268,0.565033,0.211761}%
\pgfsetfillcolor{currentfill}%
\pgfsetlinewidth{0.000000pt}%
\definecolor{currentstroke}{rgb}{0.000000,0.000000,0.000000}%
\pgfsetstrokecolor{currentstroke}%
\pgfsetstrokeopacity{0.000000}%
\pgfsetdash{}{0pt}%
\pgfpathmoveto{\pgfqpoint{5.931214in}{1.372732in}}%
\pgfpathlineto{\pgfqpoint{5.940150in}{1.372732in}}%
\pgfpathlineto{\pgfqpoint{5.940150in}{1.227719in}}%
\pgfpathlineto{\pgfqpoint{5.931214in}{1.227719in}}%
\pgfpathlineto{\pgfqpoint{5.931214in}{1.372732in}}%
\pgfpathclose%
\pgfusepath{fill}%
\end{pgfscope}%
\begin{pgfscope}%
\pgfpathrectangle{\pgfqpoint{3.722897in}{0.857143in}}{\pgfqpoint{2.627103in}{1.813434in}}%
\pgfusepath{clip}%
\pgfsetbuttcap%
\pgfsetmiterjoin%
\definecolor{currentfill}{rgb}{0.754268,0.565033,0.211761}%
\pgfsetfillcolor{currentfill}%
\pgfsetlinewidth{0.000000pt}%
\definecolor{currentstroke}{rgb}{0.000000,0.000000,0.000000}%
\pgfsetstrokecolor{currentstroke}%
\pgfsetstrokeopacity{0.000000}%
\pgfsetdash{}{0pt}%
\pgfpathmoveto{\pgfqpoint{5.942385in}{1.370687in}}%
\pgfpathlineto{\pgfqpoint{5.951321in}{1.370687in}}%
\pgfpathlineto{\pgfqpoint{5.951321in}{1.227515in}}%
\pgfpathlineto{\pgfqpoint{5.942385in}{1.227515in}}%
\pgfpathlineto{\pgfqpoint{5.942385in}{1.370687in}}%
\pgfpathclose%
\pgfusepath{fill}%
\end{pgfscope}%
\begin{pgfscope}%
\pgfpathrectangle{\pgfqpoint{3.722897in}{0.857143in}}{\pgfqpoint{2.627103in}{1.813434in}}%
\pgfusepath{clip}%
\pgfsetbuttcap%
\pgfsetmiterjoin%
\definecolor{currentfill}{rgb}{0.754268,0.565033,0.211761}%
\pgfsetfillcolor{currentfill}%
\pgfsetlinewidth{0.000000pt}%
\definecolor{currentstroke}{rgb}{0.000000,0.000000,0.000000}%
\pgfsetstrokecolor{currentstroke}%
\pgfsetstrokeopacity{0.000000}%
\pgfsetdash{}{0pt}%
\pgfpathmoveto{\pgfqpoint{5.953555in}{1.381819in}}%
\pgfpathlineto{\pgfqpoint{5.962492in}{1.381819in}}%
\pgfpathlineto{\pgfqpoint{5.962492in}{1.230415in}}%
\pgfpathlineto{\pgfqpoint{5.953555in}{1.230415in}}%
\pgfpathlineto{\pgfqpoint{5.953555in}{1.381819in}}%
\pgfpathclose%
\pgfusepath{fill}%
\end{pgfscope}%
\begin{pgfscope}%
\pgfpathrectangle{\pgfqpoint{3.722897in}{0.857143in}}{\pgfqpoint{2.627103in}{1.813434in}}%
\pgfusepath{clip}%
\pgfsetbuttcap%
\pgfsetmiterjoin%
\definecolor{currentfill}{rgb}{0.754268,0.565033,0.211761}%
\pgfsetfillcolor{currentfill}%
\pgfsetlinewidth{0.000000pt}%
\definecolor{currentstroke}{rgb}{0.000000,0.000000,0.000000}%
\pgfsetstrokecolor{currentstroke}%
\pgfsetstrokeopacity{0.000000}%
\pgfsetdash{}{0pt}%
\pgfpathmoveto{\pgfqpoint{5.964726in}{1.407041in}}%
\pgfpathlineto{\pgfqpoint{5.973662in}{1.407041in}}%
\pgfpathlineto{\pgfqpoint{5.973662in}{1.246414in}}%
\pgfpathlineto{\pgfqpoint{5.964726in}{1.246414in}}%
\pgfpathlineto{\pgfqpoint{5.964726in}{1.407041in}}%
\pgfpathclose%
\pgfusepath{fill}%
\end{pgfscope}%
\begin{pgfscope}%
\pgfpathrectangle{\pgfqpoint{3.722897in}{0.857143in}}{\pgfqpoint{2.627103in}{1.813434in}}%
\pgfusepath{clip}%
\pgfsetbuttcap%
\pgfsetmiterjoin%
\definecolor{currentfill}{rgb}{0.754268,0.565033,0.211761}%
\pgfsetfillcolor{currentfill}%
\pgfsetlinewidth{0.000000pt}%
\definecolor{currentstroke}{rgb}{0.000000,0.000000,0.000000}%
\pgfsetstrokecolor{currentstroke}%
\pgfsetstrokeopacity{0.000000}%
\pgfsetdash{}{0pt}%
\pgfpathmoveto{\pgfqpoint{5.975896in}{1.412490in}}%
\pgfpathlineto{\pgfqpoint{5.984833in}{1.412490in}}%
\pgfpathlineto{\pgfqpoint{5.984833in}{1.248562in}}%
\pgfpathlineto{\pgfqpoint{5.975896in}{1.248562in}}%
\pgfpathlineto{\pgfqpoint{5.975896in}{1.412490in}}%
\pgfpathclose%
\pgfusepath{fill}%
\end{pgfscope}%
\begin{pgfscope}%
\pgfpathrectangle{\pgfqpoint{3.722897in}{0.857143in}}{\pgfqpoint{2.627103in}{1.813434in}}%
\pgfusepath{clip}%
\pgfsetbuttcap%
\pgfsetmiterjoin%
\definecolor{currentfill}{rgb}{0.754268,0.565033,0.211761}%
\pgfsetfillcolor{currentfill}%
\pgfsetlinewidth{0.000000pt}%
\definecolor{currentstroke}{rgb}{0.000000,0.000000,0.000000}%
\pgfsetstrokecolor{currentstroke}%
\pgfsetstrokeopacity{0.000000}%
\pgfsetdash{}{0pt}%
\pgfpathmoveto{\pgfqpoint{5.987067in}{1.432543in}}%
\pgfpathlineto{\pgfqpoint{5.996004in}{1.432543in}}%
\pgfpathlineto{\pgfqpoint{5.996004in}{1.258844in}}%
\pgfpathlineto{\pgfqpoint{5.987067in}{1.258844in}}%
\pgfpathlineto{\pgfqpoint{5.987067in}{1.432543in}}%
\pgfpathclose%
\pgfusepath{fill}%
\end{pgfscope}%
\begin{pgfscope}%
\pgfpathrectangle{\pgfqpoint{3.722897in}{0.857143in}}{\pgfqpoint{2.627103in}{1.813434in}}%
\pgfusepath{clip}%
\pgfsetbuttcap%
\pgfsetmiterjoin%
\definecolor{currentfill}{rgb}{0.754268,0.565033,0.211761}%
\pgfsetfillcolor{currentfill}%
\pgfsetlinewidth{0.000000pt}%
\definecolor{currentstroke}{rgb}{0.000000,0.000000,0.000000}%
\pgfsetstrokecolor{currentstroke}%
\pgfsetstrokeopacity{0.000000}%
\pgfsetdash{}{0pt}%
\pgfpathmoveto{\pgfqpoint{5.998238in}{1.459581in}}%
\pgfpathlineto{\pgfqpoint{6.007174in}{1.459581in}}%
\pgfpathlineto{\pgfqpoint{6.007174in}{1.281630in}}%
\pgfpathlineto{\pgfqpoint{5.998238in}{1.281630in}}%
\pgfpathlineto{\pgfqpoint{5.998238in}{1.459581in}}%
\pgfpathclose%
\pgfusepath{fill}%
\end{pgfscope}%
\begin{pgfscope}%
\pgfpathrectangle{\pgfqpoint{3.722897in}{0.857143in}}{\pgfqpoint{2.627103in}{1.813434in}}%
\pgfusepath{clip}%
\pgfsetbuttcap%
\pgfsetmiterjoin%
\definecolor{currentfill}{rgb}{0.754268,0.565033,0.211761}%
\pgfsetfillcolor{currentfill}%
\pgfsetlinewidth{0.000000pt}%
\definecolor{currentstroke}{rgb}{0.000000,0.000000,0.000000}%
\pgfsetstrokecolor{currentstroke}%
\pgfsetstrokeopacity{0.000000}%
\pgfsetdash{}{0pt}%
\pgfpathmoveto{\pgfqpoint{6.009408in}{1.473935in}}%
\pgfpathlineto{\pgfqpoint{6.018345in}{1.473935in}}%
\pgfpathlineto{\pgfqpoint{6.018345in}{1.302510in}}%
\pgfpathlineto{\pgfqpoint{6.009408in}{1.302510in}}%
\pgfpathlineto{\pgfqpoint{6.009408in}{1.473935in}}%
\pgfpathclose%
\pgfusepath{fill}%
\end{pgfscope}%
\begin{pgfscope}%
\pgfpathrectangle{\pgfqpoint{3.722897in}{0.857143in}}{\pgfqpoint{2.627103in}{1.813434in}}%
\pgfusepath{clip}%
\pgfsetbuttcap%
\pgfsetmiterjoin%
\definecolor{currentfill}{rgb}{0.754268,0.565033,0.211761}%
\pgfsetfillcolor{currentfill}%
\pgfsetlinewidth{0.000000pt}%
\definecolor{currentstroke}{rgb}{0.000000,0.000000,0.000000}%
\pgfsetstrokecolor{currentstroke}%
\pgfsetstrokeopacity{0.000000}%
\pgfsetdash{}{0pt}%
\pgfpathmoveto{\pgfqpoint{6.020579in}{1.483244in}}%
\pgfpathlineto{\pgfqpoint{6.029515in}{1.483244in}}%
\pgfpathlineto{\pgfqpoint{6.029515in}{1.324205in}}%
\pgfpathlineto{\pgfqpoint{6.020579in}{1.324205in}}%
\pgfpathlineto{\pgfqpoint{6.020579in}{1.483244in}}%
\pgfpathclose%
\pgfusepath{fill}%
\end{pgfscope}%
\begin{pgfscope}%
\pgfpathrectangle{\pgfqpoint{3.722897in}{0.857143in}}{\pgfqpoint{2.627103in}{1.813434in}}%
\pgfusepath{clip}%
\pgfsetbuttcap%
\pgfsetmiterjoin%
\definecolor{currentfill}{rgb}{0.754268,0.565033,0.211761}%
\pgfsetfillcolor{currentfill}%
\pgfsetlinewidth{0.000000pt}%
\definecolor{currentstroke}{rgb}{0.000000,0.000000,0.000000}%
\pgfsetstrokecolor{currentstroke}%
\pgfsetstrokeopacity{0.000000}%
\pgfsetdash{}{0pt}%
\pgfpathmoveto{\pgfqpoint{6.031749in}{1.510090in}}%
\pgfpathlineto{\pgfqpoint{6.040686in}{1.510090in}}%
\pgfpathlineto{\pgfqpoint{6.040686in}{1.340281in}}%
\pgfpathlineto{\pgfqpoint{6.031749in}{1.340281in}}%
\pgfpathlineto{\pgfqpoint{6.031749in}{1.510090in}}%
\pgfpathclose%
\pgfusepath{fill}%
\end{pgfscope}%
\begin{pgfscope}%
\pgfpathrectangle{\pgfqpoint{3.722897in}{0.857143in}}{\pgfqpoint{2.627103in}{1.813434in}}%
\pgfusepath{clip}%
\pgfsetbuttcap%
\pgfsetmiterjoin%
\definecolor{currentfill}{rgb}{0.754268,0.565033,0.211761}%
\pgfsetfillcolor{currentfill}%
\pgfsetlinewidth{0.000000pt}%
\definecolor{currentstroke}{rgb}{0.000000,0.000000,0.000000}%
\pgfsetstrokecolor{currentstroke}%
\pgfsetstrokeopacity{0.000000}%
\pgfsetdash{}{0pt}%
\pgfpathmoveto{\pgfqpoint{6.042920in}{1.498309in}}%
\pgfpathlineto{\pgfqpoint{6.051857in}{1.498309in}}%
\pgfpathlineto{\pgfqpoint{6.051857in}{1.329028in}}%
\pgfpathlineto{\pgfqpoint{6.042920in}{1.329028in}}%
\pgfpathlineto{\pgfqpoint{6.042920in}{1.498309in}}%
\pgfpathclose%
\pgfusepath{fill}%
\end{pgfscope}%
\begin{pgfscope}%
\pgfpathrectangle{\pgfqpoint{3.722897in}{0.857143in}}{\pgfqpoint{2.627103in}{1.813434in}}%
\pgfusepath{clip}%
\pgfsetbuttcap%
\pgfsetmiterjoin%
\definecolor{currentfill}{rgb}{0.754268,0.565033,0.211761}%
\pgfsetfillcolor{currentfill}%
\pgfsetlinewidth{0.000000pt}%
\definecolor{currentstroke}{rgb}{0.000000,0.000000,0.000000}%
\pgfsetstrokecolor{currentstroke}%
\pgfsetstrokeopacity{0.000000}%
\pgfsetdash{}{0pt}%
\pgfpathmoveto{\pgfqpoint{6.054091in}{1.482896in}}%
\pgfpathlineto{\pgfqpoint{6.063027in}{1.482896in}}%
\pgfpathlineto{\pgfqpoint{6.063027in}{1.323100in}}%
\pgfpathlineto{\pgfqpoint{6.054091in}{1.323100in}}%
\pgfpathlineto{\pgfqpoint{6.054091in}{1.482896in}}%
\pgfpathclose%
\pgfusepath{fill}%
\end{pgfscope}%
\begin{pgfscope}%
\pgfpathrectangle{\pgfqpoint{3.722897in}{0.857143in}}{\pgfqpoint{2.627103in}{1.813434in}}%
\pgfusepath{clip}%
\pgfsetbuttcap%
\pgfsetmiterjoin%
\definecolor{currentfill}{rgb}{0.754268,0.565033,0.211761}%
\pgfsetfillcolor{currentfill}%
\pgfsetlinewidth{0.000000pt}%
\definecolor{currentstroke}{rgb}{0.000000,0.000000,0.000000}%
\pgfsetstrokecolor{currentstroke}%
\pgfsetstrokeopacity{0.000000}%
\pgfsetdash{}{0pt}%
\pgfpathmoveto{\pgfqpoint{6.065261in}{1.458590in}}%
\pgfpathlineto{\pgfqpoint{6.074198in}{1.458590in}}%
\pgfpathlineto{\pgfqpoint{6.074198in}{1.312960in}}%
\pgfpathlineto{\pgfqpoint{6.065261in}{1.312960in}}%
\pgfpathlineto{\pgfqpoint{6.065261in}{1.458590in}}%
\pgfpathclose%
\pgfusepath{fill}%
\end{pgfscope}%
\begin{pgfscope}%
\pgfpathrectangle{\pgfqpoint{3.722897in}{0.857143in}}{\pgfqpoint{2.627103in}{1.813434in}}%
\pgfusepath{clip}%
\pgfsetbuttcap%
\pgfsetmiterjoin%
\definecolor{currentfill}{rgb}{0.754268,0.565033,0.211761}%
\pgfsetfillcolor{currentfill}%
\pgfsetlinewidth{0.000000pt}%
\definecolor{currentstroke}{rgb}{0.000000,0.000000,0.000000}%
\pgfsetstrokecolor{currentstroke}%
\pgfsetstrokeopacity{0.000000}%
\pgfsetdash{}{0pt}%
\pgfpathmoveto{\pgfqpoint{6.076432in}{1.457213in}}%
\pgfpathlineto{\pgfqpoint{6.085368in}{1.457213in}}%
\pgfpathlineto{\pgfqpoint{6.085368in}{1.297680in}}%
\pgfpathlineto{\pgfqpoint{6.076432in}{1.297680in}}%
\pgfpathlineto{\pgfqpoint{6.076432in}{1.457213in}}%
\pgfpathclose%
\pgfusepath{fill}%
\end{pgfscope}%
\begin{pgfscope}%
\pgfpathrectangle{\pgfqpoint{3.722897in}{0.857143in}}{\pgfqpoint{2.627103in}{1.813434in}}%
\pgfusepath{clip}%
\pgfsetbuttcap%
\pgfsetmiterjoin%
\definecolor{currentfill}{rgb}{0.754268,0.565033,0.211761}%
\pgfsetfillcolor{currentfill}%
\pgfsetlinewidth{0.000000pt}%
\definecolor{currentstroke}{rgb}{0.000000,0.000000,0.000000}%
\pgfsetstrokecolor{currentstroke}%
\pgfsetstrokeopacity{0.000000}%
\pgfsetdash{}{0pt}%
\pgfpathmoveto{\pgfqpoint{6.087602in}{1.434004in}}%
\pgfpathlineto{\pgfqpoint{6.096539in}{1.434004in}}%
\pgfpathlineto{\pgfqpoint{6.096539in}{1.276908in}}%
\pgfpathlineto{\pgfqpoint{6.087602in}{1.276908in}}%
\pgfpathlineto{\pgfqpoint{6.087602in}{1.434004in}}%
\pgfpathclose%
\pgfusepath{fill}%
\end{pgfscope}%
\begin{pgfscope}%
\pgfpathrectangle{\pgfqpoint{3.722897in}{0.857143in}}{\pgfqpoint{2.627103in}{1.813434in}}%
\pgfusepath{clip}%
\pgfsetbuttcap%
\pgfsetmiterjoin%
\definecolor{currentfill}{rgb}{0.754268,0.565033,0.211761}%
\pgfsetfillcolor{currentfill}%
\pgfsetlinewidth{0.000000pt}%
\definecolor{currentstroke}{rgb}{0.000000,0.000000,0.000000}%
\pgfsetstrokecolor{currentstroke}%
\pgfsetstrokeopacity{0.000000}%
\pgfsetdash{}{0pt}%
\pgfpathmoveto{\pgfqpoint{6.098773in}{1.421190in}}%
\pgfpathlineto{\pgfqpoint{6.107710in}{1.421190in}}%
\pgfpathlineto{\pgfqpoint{6.107710in}{1.258402in}}%
\pgfpathlineto{\pgfqpoint{6.098773in}{1.258402in}}%
\pgfpathlineto{\pgfqpoint{6.098773in}{1.421190in}}%
\pgfpathclose%
\pgfusepath{fill}%
\end{pgfscope}%
\begin{pgfscope}%
\pgfpathrectangle{\pgfqpoint{3.722897in}{0.857143in}}{\pgfqpoint{2.627103in}{1.813434in}}%
\pgfusepath{clip}%
\pgfsetbuttcap%
\pgfsetmiterjoin%
\definecolor{currentfill}{rgb}{0.754268,0.565033,0.211761}%
\pgfsetfillcolor{currentfill}%
\pgfsetlinewidth{0.000000pt}%
\definecolor{currentstroke}{rgb}{0.000000,0.000000,0.000000}%
\pgfsetstrokecolor{currentstroke}%
\pgfsetstrokeopacity{0.000000}%
\pgfsetdash{}{0pt}%
\pgfpathmoveto{\pgfqpoint{6.109944in}{1.421212in}}%
\pgfpathlineto{\pgfqpoint{6.118880in}{1.421212in}}%
\pgfpathlineto{\pgfqpoint{6.118880in}{1.253246in}}%
\pgfpathlineto{\pgfqpoint{6.109944in}{1.253246in}}%
\pgfpathlineto{\pgfqpoint{6.109944in}{1.421212in}}%
\pgfpathclose%
\pgfusepath{fill}%
\end{pgfscope}%
\begin{pgfscope}%
\pgfpathrectangle{\pgfqpoint{3.722897in}{0.857143in}}{\pgfqpoint{2.627103in}{1.813434in}}%
\pgfusepath{clip}%
\pgfsetbuttcap%
\pgfsetmiterjoin%
\definecolor{currentfill}{rgb}{0.754268,0.565033,0.211761}%
\pgfsetfillcolor{currentfill}%
\pgfsetlinewidth{0.000000pt}%
\definecolor{currentstroke}{rgb}{0.000000,0.000000,0.000000}%
\pgfsetstrokecolor{currentstroke}%
\pgfsetstrokeopacity{0.000000}%
\pgfsetdash{}{0pt}%
\pgfpathmoveto{\pgfqpoint{6.121114in}{1.397038in}}%
\pgfpathlineto{\pgfqpoint{6.130051in}{1.397038in}}%
\pgfpathlineto{\pgfqpoint{6.130051in}{1.234586in}}%
\pgfpathlineto{\pgfqpoint{6.121114in}{1.234586in}}%
\pgfpathlineto{\pgfqpoint{6.121114in}{1.397038in}}%
\pgfpathclose%
\pgfusepath{fill}%
\end{pgfscope}%
\begin{pgfscope}%
\pgfpathrectangle{\pgfqpoint{3.722897in}{0.857143in}}{\pgfqpoint{2.627103in}{1.813434in}}%
\pgfusepath{clip}%
\pgfsetbuttcap%
\pgfsetmiterjoin%
\definecolor{currentfill}{rgb}{0.754268,0.565033,0.211761}%
\pgfsetfillcolor{currentfill}%
\pgfsetlinewidth{0.000000pt}%
\definecolor{currentstroke}{rgb}{0.000000,0.000000,0.000000}%
\pgfsetstrokecolor{currentstroke}%
\pgfsetstrokeopacity{0.000000}%
\pgfsetdash{}{0pt}%
\pgfpathmoveto{\pgfqpoint{6.132285in}{1.379141in}}%
\pgfpathlineto{\pgfqpoint{6.141221in}{1.379141in}}%
\pgfpathlineto{\pgfqpoint{6.141221in}{1.214575in}}%
\pgfpathlineto{\pgfqpoint{6.132285in}{1.214575in}}%
\pgfpathlineto{\pgfqpoint{6.132285in}{1.379141in}}%
\pgfpathclose%
\pgfusepath{fill}%
\end{pgfscope}%
\begin{pgfscope}%
\pgfpathrectangle{\pgfqpoint{3.722897in}{0.857143in}}{\pgfqpoint{2.627103in}{1.813434in}}%
\pgfusepath{clip}%
\pgfsetbuttcap%
\pgfsetmiterjoin%
\definecolor{currentfill}{rgb}{0.754268,0.565033,0.211761}%
\pgfsetfillcolor{currentfill}%
\pgfsetlinewidth{0.000000pt}%
\definecolor{currentstroke}{rgb}{0.000000,0.000000,0.000000}%
\pgfsetstrokecolor{currentstroke}%
\pgfsetstrokeopacity{0.000000}%
\pgfsetdash{}{0pt}%
\pgfpathmoveto{\pgfqpoint{6.143456in}{1.373670in}}%
\pgfpathlineto{\pgfqpoint{6.152392in}{1.373670in}}%
\pgfpathlineto{\pgfqpoint{6.152392in}{1.203622in}}%
\pgfpathlineto{\pgfqpoint{6.143456in}{1.203622in}}%
\pgfpathlineto{\pgfqpoint{6.143456in}{1.373670in}}%
\pgfpathclose%
\pgfusepath{fill}%
\end{pgfscope}%
\begin{pgfscope}%
\pgfpathrectangle{\pgfqpoint{3.722897in}{0.857143in}}{\pgfqpoint{2.627103in}{1.813434in}}%
\pgfusepath{clip}%
\pgfsetbuttcap%
\pgfsetmiterjoin%
\definecolor{currentfill}{rgb}{0.754268,0.565033,0.211761}%
\pgfsetfillcolor{currentfill}%
\pgfsetlinewidth{0.000000pt}%
\definecolor{currentstroke}{rgb}{0.000000,0.000000,0.000000}%
\pgfsetstrokecolor{currentstroke}%
\pgfsetstrokeopacity{0.000000}%
\pgfsetdash{}{0pt}%
\pgfpathmoveto{\pgfqpoint{6.154626in}{1.364190in}}%
\pgfpathlineto{\pgfqpoint{6.163563in}{1.364190in}}%
\pgfpathlineto{\pgfqpoint{6.163563in}{1.189858in}}%
\pgfpathlineto{\pgfqpoint{6.154626in}{1.189858in}}%
\pgfpathlineto{\pgfqpoint{6.154626in}{1.364190in}}%
\pgfpathclose%
\pgfusepath{fill}%
\end{pgfscope}%
\begin{pgfscope}%
\pgfpathrectangle{\pgfqpoint{3.722897in}{0.857143in}}{\pgfqpoint{2.627103in}{1.813434in}}%
\pgfusepath{clip}%
\pgfsetbuttcap%
\pgfsetmiterjoin%
\definecolor{currentfill}{rgb}{0.754268,0.565033,0.211761}%
\pgfsetfillcolor{currentfill}%
\pgfsetlinewidth{0.000000pt}%
\definecolor{currentstroke}{rgb}{0.000000,0.000000,0.000000}%
\pgfsetstrokecolor{currentstroke}%
\pgfsetstrokeopacity{0.000000}%
\pgfsetdash{}{0pt}%
\pgfpathmoveto{\pgfqpoint{6.165797in}{1.365961in}}%
\pgfpathlineto{\pgfqpoint{6.174733in}{1.365961in}}%
\pgfpathlineto{\pgfqpoint{6.174733in}{1.186296in}}%
\pgfpathlineto{\pgfqpoint{6.165797in}{1.186296in}}%
\pgfpathlineto{\pgfqpoint{6.165797in}{1.365961in}}%
\pgfpathclose%
\pgfusepath{fill}%
\end{pgfscope}%
\begin{pgfscope}%
\pgfpathrectangle{\pgfqpoint{3.722897in}{0.857143in}}{\pgfqpoint{2.627103in}{1.813434in}}%
\pgfusepath{clip}%
\pgfsetbuttcap%
\pgfsetmiterjoin%
\definecolor{currentfill}{rgb}{0.754268,0.565033,0.211761}%
\pgfsetfillcolor{currentfill}%
\pgfsetlinewidth{0.000000pt}%
\definecolor{currentstroke}{rgb}{0.000000,0.000000,0.000000}%
\pgfsetstrokecolor{currentstroke}%
\pgfsetstrokeopacity{0.000000}%
\pgfsetdash{}{0pt}%
\pgfpathmoveto{\pgfqpoint{6.176967in}{1.359361in}}%
\pgfpathlineto{\pgfqpoint{6.185904in}{1.359361in}}%
\pgfpathlineto{\pgfqpoint{6.185904in}{1.185421in}}%
\pgfpathlineto{\pgfqpoint{6.176967in}{1.185421in}}%
\pgfpathlineto{\pgfqpoint{6.176967in}{1.359361in}}%
\pgfpathclose%
\pgfusepath{fill}%
\end{pgfscope}%
\begin{pgfscope}%
\pgfpathrectangle{\pgfqpoint{3.722897in}{0.857143in}}{\pgfqpoint{2.627103in}{1.813434in}}%
\pgfusepath{clip}%
\pgfsetbuttcap%
\pgfsetmiterjoin%
\definecolor{currentfill}{rgb}{0.754268,0.565033,0.211761}%
\pgfsetfillcolor{currentfill}%
\pgfsetlinewidth{0.000000pt}%
\definecolor{currentstroke}{rgb}{0.000000,0.000000,0.000000}%
\pgfsetstrokecolor{currentstroke}%
\pgfsetstrokeopacity{0.000000}%
\pgfsetdash{}{0pt}%
\pgfpathmoveto{\pgfqpoint{6.188138in}{1.356444in}}%
\pgfpathlineto{\pgfqpoint{6.197074in}{1.356444in}}%
\pgfpathlineto{\pgfqpoint{6.197074in}{1.189139in}}%
\pgfpathlineto{\pgfqpoint{6.188138in}{1.189139in}}%
\pgfpathlineto{\pgfqpoint{6.188138in}{1.356444in}}%
\pgfpathclose%
\pgfusepath{fill}%
\end{pgfscope}%
\begin{pgfscope}%
\pgfpathrectangle{\pgfqpoint{3.722897in}{0.857143in}}{\pgfqpoint{2.627103in}{1.813434in}}%
\pgfusepath{clip}%
\pgfsetbuttcap%
\pgfsetmiterjoin%
\definecolor{currentfill}{rgb}{0.754268,0.565033,0.211761}%
\pgfsetfillcolor{currentfill}%
\pgfsetlinewidth{0.000000pt}%
\definecolor{currentstroke}{rgb}{0.000000,0.000000,0.000000}%
\pgfsetstrokecolor{currentstroke}%
\pgfsetstrokeopacity{0.000000}%
\pgfsetdash{}{0pt}%
\pgfpathmoveto{\pgfqpoint{6.199309in}{1.332093in}}%
\pgfpathlineto{\pgfqpoint{6.208245in}{1.332093in}}%
\pgfpathlineto{\pgfqpoint{6.208245in}{1.177051in}}%
\pgfpathlineto{\pgfqpoint{6.199309in}{1.177051in}}%
\pgfpathlineto{\pgfqpoint{6.199309in}{1.332093in}}%
\pgfpathclose%
\pgfusepath{fill}%
\end{pgfscope}%
\begin{pgfscope}%
\pgfpathrectangle{\pgfqpoint{3.722897in}{0.857143in}}{\pgfqpoint{2.627103in}{1.813434in}}%
\pgfusepath{clip}%
\pgfsetbuttcap%
\pgfsetmiterjoin%
\definecolor{currentfill}{rgb}{0.754268,0.565033,0.211761}%
\pgfsetfillcolor{currentfill}%
\pgfsetlinewidth{0.000000pt}%
\definecolor{currentstroke}{rgb}{0.000000,0.000000,0.000000}%
\pgfsetstrokecolor{currentstroke}%
\pgfsetstrokeopacity{0.000000}%
\pgfsetdash{}{0pt}%
\pgfpathmoveto{\pgfqpoint{6.210479in}{1.324640in}}%
\pgfpathlineto{\pgfqpoint{6.219416in}{1.324640in}}%
\pgfpathlineto{\pgfqpoint{6.219416in}{1.173336in}}%
\pgfpathlineto{\pgfqpoint{6.210479in}{1.173336in}}%
\pgfpathlineto{\pgfqpoint{6.210479in}{1.324640in}}%
\pgfpathclose%
\pgfusepath{fill}%
\end{pgfscope}%
\begin{pgfscope}%
\pgfpathrectangle{\pgfqpoint{3.722897in}{0.857143in}}{\pgfqpoint{2.627103in}{1.813434in}}%
\pgfusepath{clip}%
\pgfsetbuttcap%
\pgfsetmiterjoin%
\definecolor{currentfill}{rgb}{0.754268,0.565033,0.211761}%
\pgfsetfillcolor{currentfill}%
\pgfsetlinewidth{0.000000pt}%
\definecolor{currentstroke}{rgb}{0.000000,0.000000,0.000000}%
\pgfsetstrokecolor{currentstroke}%
\pgfsetstrokeopacity{0.000000}%
\pgfsetdash{}{0pt}%
\pgfpathmoveto{\pgfqpoint{6.221650in}{1.309376in}}%
\pgfpathlineto{\pgfqpoint{6.230586in}{1.309376in}}%
\pgfpathlineto{\pgfqpoint{6.230586in}{1.168605in}}%
\pgfpathlineto{\pgfqpoint{6.221650in}{1.168605in}}%
\pgfpathlineto{\pgfqpoint{6.221650in}{1.309376in}}%
\pgfpathclose%
\pgfusepath{fill}%
\end{pgfscope}%
\begin{pgfscope}%
\pgfpathrectangle{\pgfqpoint{3.722897in}{0.857143in}}{\pgfqpoint{2.627103in}{1.813434in}}%
\pgfusepath{clip}%
\pgfsetbuttcap%
\pgfsetmiterjoin%
\definecolor{currentfill}{rgb}{0.950697,0.616649,0.428624}%
\pgfsetfillcolor{currentfill}%
\pgfsetlinewidth{0.000000pt}%
\definecolor{currentstroke}{rgb}{0.000000,0.000000,0.000000}%
\pgfsetstrokecolor{currentstroke}%
\pgfsetstrokeopacity{0.000000}%
\pgfsetdash{}{0pt}%
\pgfpathmoveto{\pgfqpoint{3.842311in}{1.763200in}}%
\pgfpathlineto{\pgfqpoint{3.851247in}{1.763200in}}%
\pgfpathlineto{\pgfqpoint{3.851247in}{1.706370in}}%
\pgfpathlineto{\pgfqpoint{3.842311in}{1.706370in}}%
\pgfpathlineto{\pgfqpoint{3.842311in}{1.763200in}}%
\pgfpathclose%
\pgfusepath{fill}%
\end{pgfscope}%
\begin{pgfscope}%
\pgfpathrectangle{\pgfqpoint{3.722897in}{0.857143in}}{\pgfqpoint{2.627103in}{1.813434in}}%
\pgfusepath{clip}%
\pgfsetbuttcap%
\pgfsetmiterjoin%
\definecolor{currentfill}{rgb}{0.950697,0.616649,0.428624}%
\pgfsetfillcolor{currentfill}%
\pgfsetlinewidth{0.000000pt}%
\definecolor{currentstroke}{rgb}{0.000000,0.000000,0.000000}%
\pgfsetstrokecolor{currentstroke}%
\pgfsetstrokeopacity{0.000000}%
\pgfsetdash{}{0pt}%
\pgfpathmoveto{\pgfqpoint{3.853481in}{1.752062in}}%
\pgfpathlineto{\pgfqpoint{3.862418in}{1.752062in}}%
\pgfpathlineto{\pgfqpoint{3.862418in}{1.678637in}}%
\pgfpathlineto{\pgfqpoint{3.853481in}{1.678637in}}%
\pgfpathlineto{\pgfqpoint{3.853481in}{1.752062in}}%
\pgfpathclose%
\pgfusepath{fill}%
\end{pgfscope}%
\begin{pgfscope}%
\pgfpathrectangle{\pgfqpoint{3.722897in}{0.857143in}}{\pgfqpoint{2.627103in}{1.813434in}}%
\pgfusepath{clip}%
\pgfsetbuttcap%
\pgfsetmiterjoin%
\definecolor{currentfill}{rgb}{0.950697,0.616649,0.428624}%
\pgfsetfillcolor{currentfill}%
\pgfsetlinewidth{0.000000pt}%
\definecolor{currentstroke}{rgb}{0.000000,0.000000,0.000000}%
\pgfsetstrokecolor{currentstroke}%
\pgfsetstrokeopacity{0.000000}%
\pgfsetdash{}{0pt}%
\pgfpathmoveto{\pgfqpoint{3.864652in}{1.745934in}}%
\pgfpathlineto{\pgfqpoint{3.873588in}{1.745934in}}%
\pgfpathlineto{\pgfqpoint{3.873588in}{1.660057in}}%
\pgfpathlineto{\pgfqpoint{3.864652in}{1.660057in}}%
\pgfpathlineto{\pgfqpoint{3.864652in}{1.745934in}}%
\pgfpathclose%
\pgfusepath{fill}%
\end{pgfscope}%
\begin{pgfscope}%
\pgfpathrectangle{\pgfqpoint{3.722897in}{0.857143in}}{\pgfqpoint{2.627103in}{1.813434in}}%
\pgfusepath{clip}%
\pgfsetbuttcap%
\pgfsetmiterjoin%
\definecolor{currentfill}{rgb}{0.950697,0.616649,0.428624}%
\pgfsetfillcolor{currentfill}%
\pgfsetlinewidth{0.000000pt}%
\definecolor{currentstroke}{rgb}{0.000000,0.000000,0.000000}%
\pgfsetstrokecolor{currentstroke}%
\pgfsetstrokeopacity{0.000000}%
\pgfsetdash{}{0pt}%
\pgfpathmoveto{\pgfqpoint{3.875823in}{1.736268in}}%
\pgfpathlineto{\pgfqpoint{3.884759in}{1.736268in}}%
\pgfpathlineto{\pgfqpoint{3.884759in}{1.651291in}}%
\pgfpathlineto{\pgfqpoint{3.875823in}{1.651291in}}%
\pgfpathlineto{\pgfqpoint{3.875823in}{1.736268in}}%
\pgfpathclose%
\pgfusepath{fill}%
\end{pgfscope}%
\begin{pgfscope}%
\pgfpathrectangle{\pgfqpoint{3.722897in}{0.857143in}}{\pgfqpoint{2.627103in}{1.813434in}}%
\pgfusepath{clip}%
\pgfsetbuttcap%
\pgfsetmiterjoin%
\definecolor{currentfill}{rgb}{0.950697,0.616649,0.428624}%
\pgfsetfillcolor{currentfill}%
\pgfsetlinewidth{0.000000pt}%
\definecolor{currentstroke}{rgb}{0.000000,0.000000,0.000000}%
\pgfsetstrokecolor{currentstroke}%
\pgfsetstrokeopacity{0.000000}%
\pgfsetdash{}{0pt}%
\pgfpathmoveto{\pgfqpoint{3.886993in}{1.722278in}}%
\pgfpathlineto{\pgfqpoint{3.895930in}{1.722278in}}%
\pgfpathlineto{\pgfqpoint{3.895930in}{1.683493in}}%
\pgfpathlineto{\pgfqpoint{3.886993in}{1.683493in}}%
\pgfpathlineto{\pgfqpoint{3.886993in}{1.722278in}}%
\pgfpathclose%
\pgfusepath{fill}%
\end{pgfscope}%
\begin{pgfscope}%
\pgfpathrectangle{\pgfqpoint{3.722897in}{0.857143in}}{\pgfqpoint{2.627103in}{1.813434in}}%
\pgfusepath{clip}%
\pgfsetbuttcap%
\pgfsetmiterjoin%
\definecolor{currentfill}{rgb}{0.950697,0.616649,0.428624}%
\pgfsetfillcolor{currentfill}%
\pgfsetlinewidth{0.000000pt}%
\definecolor{currentstroke}{rgb}{0.000000,0.000000,0.000000}%
\pgfsetstrokecolor{currentstroke}%
\pgfsetstrokeopacity{0.000000}%
\pgfsetdash{}{0pt}%
\pgfpathmoveto{\pgfqpoint{3.898164in}{1.711139in}}%
\pgfpathlineto{\pgfqpoint{3.907100in}{1.711139in}}%
\pgfpathlineto{\pgfqpoint{3.907100in}{1.669182in}}%
\pgfpathlineto{\pgfqpoint{3.898164in}{1.669182in}}%
\pgfpathlineto{\pgfqpoint{3.898164in}{1.711139in}}%
\pgfpathclose%
\pgfusepath{fill}%
\end{pgfscope}%
\begin{pgfscope}%
\pgfpathrectangle{\pgfqpoint{3.722897in}{0.857143in}}{\pgfqpoint{2.627103in}{1.813434in}}%
\pgfusepath{clip}%
\pgfsetbuttcap%
\pgfsetmiterjoin%
\definecolor{currentfill}{rgb}{0.950697,0.616649,0.428624}%
\pgfsetfillcolor{currentfill}%
\pgfsetlinewidth{0.000000pt}%
\definecolor{currentstroke}{rgb}{0.000000,0.000000,0.000000}%
\pgfsetstrokecolor{currentstroke}%
\pgfsetstrokeopacity{0.000000}%
\pgfsetdash{}{0pt}%
\pgfpathmoveto{\pgfqpoint{3.909334in}{1.694555in}}%
\pgfpathlineto{\pgfqpoint{3.918271in}{1.694555in}}%
\pgfpathlineto{\pgfqpoint{3.918271in}{1.631759in}}%
\pgfpathlineto{\pgfqpoint{3.909334in}{1.631759in}}%
\pgfpathlineto{\pgfqpoint{3.909334in}{1.694555in}}%
\pgfpathclose%
\pgfusepath{fill}%
\end{pgfscope}%
\begin{pgfscope}%
\pgfpathrectangle{\pgfqpoint{3.722897in}{0.857143in}}{\pgfqpoint{2.627103in}{1.813434in}}%
\pgfusepath{clip}%
\pgfsetbuttcap%
\pgfsetmiterjoin%
\definecolor{currentfill}{rgb}{0.950697,0.616649,0.428624}%
\pgfsetfillcolor{currentfill}%
\pgfsetlinewidth{0.000000pt}%
\definecolor{currentstroke}{rgb}{0.000000,0.000000,0.000000}%
\pgfsetstrokecolor{currentstroke}%
\pgfsetstrokeopacity{0.000000}%
\pgfsetdash{}{0pt}%
\pgfpathmoveto{\pgfqpoint{3.920505in}{1.673449in}}%
\pgfpathlineto{\pgfqpoint{3.929442in}{1.673449in}}%
\pgfpathlineto{\pgfqpoint{3.929442in}{1.588294in}}%
\pgfpathlineto{\pgfqpoint{3.920505in}{1.588294in}}%
\pgfpathlineto{\pgfqpoint{3.920505in}{1.673449in}}%
\pgfpathclose%
\pgfusepath{fill}%
\end{pgfscope}%
\begin{pgfscope}%
\pgfpathrectangle{\pgfqpoint{3.722897in}{0.857143in}}{\pgfqpoint{2.627103in}{1.813434in}}%
\pgfusepath{clip}%
\pgfsetbuttcap%
\pgfsetmiterjoin%
\definecolor{currentfill}{rgb}{0.950697,0.616649,0.428624}%
\pgfsetfillcolor{currentfill}%
\pgfsetlinewidth{0.000000pt}%
\definecolor{currentstroke}{rgb}{0.000000,0.000000,0.000000}%
\pgfsetstrokecolor{currentstroke}%
\pgfsetstrokeopacity{0.000000}%
\pgfsetdash{}{0pt}%
\pgfpathmoveto{\pgfqpoint{3.931676in}{1.664728in}}%
\pgfpathlineto{\pgfqpoint{3.940612in}{1.664728in}}%
\pgfpathlineto{\pgfqpoint{3.940612in}{1.604920in}}%
\pgfpathlineto{\pgfqpoint{3.931676in}{1.604920in}}%
\pgfpathlineto{\pgfqpoint{3.931676in}{1.664728in}}%
\pgfpathclose%
\pgfusepath{fill}%
\end{pgfscope}%
\begin{pgfscope}%
\pgfpathrectangle{\pgfqpoint{3.722897in}{0.857143in}}{\pgfqpoint{2.627103in}{1.813434in}}%
\pgfusepath{clip}%
\pgfsetbuttcap%
\pgfsetmiterjoin%
\definecolor{currentfill}{rgb}{0.950697,0.616649,0.428624}%
\pgfsetfillcolor{currentfill}%
\pgfsetlinewidth{0.000000pt}%
\definecolor{currentstroke}{rgb}{0.000000,0.000000,0.000000}%
\pgfsetstrokecolor{currentstroke}%
\pgfsetstrokeopacity{0.000000}%
\pgfsetdash{}{0pt}%
\pgfpathmoveto{\pgfqpoint{3.942846in}{1.662237in}}%
\pgfpathlineto{\pgfqpoint{3.951783in}{1.662237in}}%
\pgfpathlineto{\pgfqpoint{3.951783in}{1.602615in}}%
\pgfpathlineto{\pgfqpoint{3.942846in}{1.602615in}}%
\pgfpathlineto{\pgfqpoint{3.942846in}{1.662237in}}%
\pgfpathclose%
\pgfusepath{fill}%
\end{pgfscope}%
\begin{pgfscope}%
\pgfpathrectangle{\pgfqpoint{3.722897in}{0.857143in}}{\pgfqpoint{2.627103in}{1.813434in}}%
\pgfusepath{clip}%
\pgfsetbuttcap%
\pgfsetmiterjoin%
\definecolor{currentfill}{rgb}{0.950697,0.616649,0.428624}%
\pgfsetfillcolor{currentfill}%
\pgfsetlinewidth{0.000000pt}%
\definecolor{currentstroke}{rgb}{0.000000,0.000000,0.000000}%
\pgfsetstrokecolor{currentstroke}%
\pgfsetstrokeopacity{0.000000}%
\pgfsetdash{}{0pt}%
\pgfpathmoveto{\pgfqpoint{3.954017in}{1.643688in}}%
\pgfpathlineto{\pgfqpoint{3.962953in}{1.643688in}}%
\pgfpathlineto{\pgfqpoint{3.962953in}{1.588155in}}%
\pgfpathlineto{\pgfqpoint{3.954017in}{1.588155in}}%
\pgfpathlineto{\pgfqpoint{3.954017in}{1.643688in}}%
\pgfpathclose%
\pgfusepath{fill}%
\end{pgfscope}%
\begin{pgfscope}%
\pgfpathrectangle{\pgfqpoint{3.722897in}{0.857143in}}{\pgfqpoint{2.627103in}{1.813434in}}%
\pgfusepath{clip}%
\pgfsetbuttcap%
\pgfsetmiterjoin%
\definecolor{currentfill}{rgb}{0.950697,0.616649,0.428624}%
\pgfsetfillcolor{currentfill}%
\pgfsetlinewidth{0.000000pt}%
\definecolor{currentstroke}{rgb}{0.000000,0.000000,0.000000}%
\pgfsetstrokecolor{currentstroke}%
\pgfsetstrokeopacity{0.000000}%
\pgfsetdash{}{0pt}%
\pgfpathmoveto{\pgfqpoint{3.965187in}{1.641335in}}%
\pgfpathlineto{\pgfqpoint{3.974124in}{1.641335in}}%
\pgfpathlineto{\pgfqpoint{3.974124in}{1.577271in}}%
\pgfpathlineto{\pgfqpoint{3.965187in}{1.577271in}}%
\pgfpathlineto{\pgfqpoint{3.965187in}{1.641335in}}%
\pgfpathclose%
\pgfusepath{fill}%
\end{pgfscope}%
\begin{pgfscope}%
\pgfpathrectangle{\pgfqpoint{3.722897in}{0.857143in}}{\pgfqpoint{2.627103in}{1.813434in}}%
\pgfusepath{clip}%
\pgfsetbuttcap%
\pgfsetmiterjoin%
\definecolor{currentfill}{rgb}{0.950697,0.616649,0.428624}%
\pgfsetfillcolor{currentfill}%
\pgfsetlinewidth{0.000000pt}%
\definecolor{currentstroke}{rgb}{0.000000,0.000000,0.000000}%
\pgfsetstrokecolor{currentstroke}%
\pgfsetstrokeopacity{0.000000}%
\pgfsetdash{}{0pt}%
\pgfpathmoveto{\pgfqpoint{3.976358in}{1.637785in}}%
\pgfpathlineto{\pgfqpoint{3.985295in}{1.637785in}}%
\pgfpathlineto{\pgfqpoint{3.985295in}{1.618147in}}%
\pgfpathlineto{\pgfqpoint{3.976358in}{1.618147in}}%
\pgfpathlineto{\pgfqpoint{3.976358in}{1.637785in}}%
\pgfpathclose%
\pgfusepath{fill}%
\end{pgfscope}%
\begin{pgfscope}%
\pgfpathrectangle{\pgfqpoint{3.722897in}{0.857143in}}{\pgfqpoint{2.627103in}{1.813434in}}%
\pgfusepath{clip}%
\pgfsetbuttcap%
\pgfsetmiterjoin%
\definecolor{currentfill}{rgb}{0.950697,0.616649,0.428624}%
\pgfsetfillcolor{currentfill}%
\pgfsetlinewidth{0.000000pt}%
\definecolor{currentstroke}{rgb}{0.000000,0.000000,0.000000}%
\pgfsetstrokecolor{currentstroke}%
\pgfsetstrokeopacity{0.000000}%
\pgfsetdash{}{0pt}%
\pgfpathmoveto{\pgfqpoint{3.987529in}{1.635809in}}%
\pgfpathlineto{\pgfqpoint{3.996465in}{1.635809in}}%
\pgfpathlineto{\pgfqpoint{3.996465in}{1.614938in}}%
\pgfpathlineto{\pgfqpoint{3.987529in}{1.614938in}}%
\pgfpathlineto{\pgfqpoint{3.987529in}{1.635809in}}%
\pgfpathclose%
\pgfusepath{fill}%
\end{pgfscope}%
\begin{pgfscope}%
\pgfpathrectangle{\pgfqpoint{3.722897in}{0.857143in}}{\pgfqpoint{2.627103in}{1.813434in}}%
\pgfusepath{clip}%
\pgfsetbuttcap%
\pgfsetmiterjoin%
\definecolor{currentfill}{rgb}{0.950697,0.616649,0.428624}%
\pgfsetfillcolor{currentfill}%
\pgfsetlinewidth{0.000000pt}%
\definecolor{currentstroke}{rgb}{0.000000,0.000000,0.000000}%
\pgfsetstrokecolor{currentstroke}%
\pgfsetstrokeopacity{0.000000}%
\pgfsetdash{}{0pt}%
\pgfpathmoveto{\pgfqpoint{3.998699in}{1.863031in}}%
\pgfpathlineto{\pgfqpoint{4.007636in}{1.863031in}}%
\pgfpathlineto{\pgfqpoint{4.007636in}{1.870021in}}%
\pgfpathlineto{\pgfqpoint{3.998699in}{1.870021in}}%
\pgfpathlineto{\pgfqpoint{3.998699in}{1.863031in}}%
\pgfpathclose%
\pgfusepath{fill}%
\end{pgfscope}%
\begin{pgfscope}%
\pgfpathrectangle{\pgfqpoint{3.722897in}{0.857143in}}{\pgfqpoint{2.627103in}{1.813434in}}%
\pgfusepath{clip}%
\pgfsetbuttcap%
\pgfsetmiterjoin%
\definecolor{currentfill}{rgb}{0.950697,0.616649,0.428624}%
\pgfsetfillcolor{currentfill}%
\pgfsetlinewidth{0.000000pt}%
\definecolor{currentstroke}{rgb}{0.000000,0.000000,0.000000}%
\pgfsetstrokecolor{currentstroke}%
\pgfsetstrokeopacity{0.000000}%
\pgfsetdash{}{0pt}%
\pgfpathmoveto{\pgfqpoint{4.009870in}{1.853088in}}%
\pgfpathlineto{\pgfqpoint{4.018806in}{1.853088in}}%
\pgfpathlineto{\pgfqpoint{4.018806in}{1.867528in}}%
\pgfpathlineto{\pgfqpoint{4.009870in}{1.867528in}}%
\pgfpathlineto{\pgfqpoint{4.009870in}{1.853088in}}%
\pgfpathclose%
\pgfusepath{fill}%
\end{pgfscope}%
\begin{pgfscope}%
\pgfpathrectangle{\pgfqpoint{3.722897in}{0.857143in}}{\pgfqpoint{2.627103in}{1.813434in}}%
\pgfusepath{clip}%
\pgfsetbuttcap%
\pgfsetmiterjoin%
\definecolor{currentfill}{rgb}{0.950697,0.616649,0.428624}%
\pgfsetfillcolor{currentfill}%
\pgfsetlinewidth{0.000000pt}%
\definecolor{currentstroke}{rgb}{0.000000,0.000000,0.000000}%
\pgfsetstrokecolor{currentstroke}%
\pgfsetstrokeopacity{0.000000}%
\pgfsetdash{}{0pt}%
\pgfpathmoveto{\pgfqpoint{4.021040in}{1.855059in}}%
\pgfpathlineto{\pgfqpoint{4.029977in}{1.855059in}}%
\pgfpathlineto{\pgfqpoint{4.029977in}{1.898361in}}%
\pgfpathlineto{\pgfqpoint{4.021040in}{1.898361in}}%
\pgfpathlineto{\pgfqpoint{4.021040in}{1.855059in}}%
\pgfpathclose%
\pgfusepath{fill}%
\end{pgfscope}%
\begin{pgfscope}%
\pgfpathrectangle{\pgfqpoint{3.722897in}{0.857143in}}{\pgfqpoint{2.627103in}{1.813434in}}%
\pgfusepath{clip}%
\pgfsetbuttcap%
\pgfsetmiterjoin%
\definecolor{currentfill}{rgb}{0.950697,0.616649,0.428624}%
\pgfsetfillcolor{currentfill}%
\pgfsetlinewidth{0.000000pt}%
\definecolor{currentstroke}{rgb}{0.000000,0.000000,0.000000}%
\pgfsetstrokecolor{currentstroke}%
\pgfsetstrokeopacity{0.000000}%
\pgfsetdash{}{0pt}%
\pgfpathmoveto{\pgfqpoint{4.032211in}{1.830072in}}%
\pgfpathlineto{\pgfqpoint{4.041148in}{1.830072in}}%
\pgfpathlineto{\pgfqpoint{4.041148in}{1.909963in}}%
\pgfpathlineto{\pgfqpoint{4.032211in}{1.909963in}}%
\pgfpathlineto{\pgfqpoint{4.032211in}{1.830072in}}%
\pgfpathclose%
\pgfusepath{fill}%
\end{pgfscope}%
\begin{pgfscope}%
\pgfpathrectangle{\pgfqpoint{3.722897in}{0.857143in}}{\pgfqpoint{2.627103in}{1.813434in}}%
\pgfusepath{clip}%
\pgfsetbuttcap%
\pgfsetmiterjoin%
\definecolor{currentfill}{rgb}{0.950697,0.616649,0.428624}%
\pgfsetfillcolor{currentfill}%
\pgfsetlinewidth{0.000000pt}%
\definecolor{currentstroke}{rgb}{0.000000,0.000000,0.000000}%
\pgfsetstrokecolor{currentstroke}%
\pgfsetstrokeopacity{0.000000}%
\pgfsetdash{}{0pt}%
\pgfpathmoveto{\pgfqpoint{4.043382in}{1.830708in}}%
\pgfpathlineto{\pgfqpoint{4.052318in}{1.830708in}}%
\pgfpathlineto{\pgfqpoint{4.052318in}{1.902915in}}%
\pgfpathlineto{\pgfqpoint{4.043382in}{1.902915in}}%
\pgfpathlineto{\pgfqpoint{4.043382in}{1.830708in}}%
\pgfpathclose%
\pgfusepath{fill}%
\end{pgfscope}%
\begin{pgfscope}%
\pgfpathrectangle{\pgfqpoint{3.722897in}{0.857143in}}{\pgfqpoint{2.627103in}{1.813434in}}%
\pgfusepath{clip}%
\pgfsetbuttcap%
\pgfsetmiterjoin%
\definecolor{currentfill}{rgb}{0.950697,0.616649,0.428624}%
\pgfsetfillcolor{currentfill}%
\pgfsetlinewidth{0.000000pt}%
\definecolor{currentstroke}{rgb}{0.000000,0.000000,0.000000}%
\pgfsetstrokecolor{currentstroke}%
\pgfsetstrokeopacity{0.000000}%
\pgfsetdash{}{0pt}%
\pgfpathmoveto{\pgfqpoint{4.054552in}{1.831364in}}%
\pgfpathlineto{\pgfqpoint{4.063489in}{1.831364in}}%
\pgfpathlineto{\pgfqpoint{4.063489in}{1.931745in}}%
\pgfpathlineto{\pgfqpoint{4.054552in}{1.931745in}}%
\pgfpathlineto{\pgfqpoint{4.054552in}{1.831364in}}%
\pgfpathclose%
\pgfusepath{fill}%
\end{pgfscope}%
\begin{pgfscope}%
\pgfpathrectangle{\pgfqpoint{3.722897in}{0.857143in}}{\pgfqpoint{2.627103in}{1.813434in}}%
\pgfusepath{clip}%
\pgfsetbuttcap%
\pgfsetmiterjoin%
\definecolor{currentfill}{rgb}{0.950697,0.616649,0.428624}%
\pgfsetfillcolor{currentfill}%
\pgfsetlinewidth{0.000000pt}%
\definecolor{currentstroke}{rgb}{0.000000,0.000000,0.000000}%
\pgfsetstrokecolor{currentstroke}%
\pgfsetstrokeopacity{0.000000}%
\pgfsetdash{}{0pt}%
\pgfpathmoveto{\pgfqpoint{4.065723in}{1.833515in}}%
\pgfpathlineto{\pgfqpoint{4.074659in}{1.833515in}}%
\pgfpathlineto{\pgfqpoint{4.074659in}{1.940649in}}%
\pgfpathlineto{\pgfqpoint{4.065723in}{1.940649in}}%
\pgfpathlineto{\pgfqpoint{4.065723in}{1.833515in}}%
\pgfpathclose%
\pgfusepath{fill}%
\end{pgfscope}%
\begin{pgfscope}%
\pgfpathrectangle{\pgfqpoint{3.722897in}{0.857143in}}{\pgfqpoint{2.627103in}{1.813434in}}%
\pgfusepath{clip}%
\pgfsetbuttcap%
\pgfsetmiterjoin%
\definecolor{currentfill}{rgb}{0.950697,0.616649,0.428624}%
\pgfsetfillcolor{currentfill}%
\pgfsetlinewidth{0.000000pt}%
\definecolor{currentstroke}{rgb}{0.000000,0.000000,0.000000}%
\pgfsetstrokecolor{currentstroke}%
\pgfsetstrokeopacity{0.000000}%
\pgfsetdash{}{0pt}%
\pgfpathmoveto{\pgfqpoint{4.076893in}{1.834801in}}%
\pgfpathlineto{\pgfqpoint{4.085830in}{1.834801in}}%
\pgfpathlineto{\pgfqpoint{4.085830in}{1.926270in}}%
\pgfpathlineto{\pgfqpoint{4.076893in}{1.926270in}}%
\pgfpathlineto{\pgfqpoint{4.076893in}{1.834801in}}%
\pgfpathclose%
\pgfusepath{fill}%
\end{pgfscope}%
\begin{pgfscope}%
\pgfpathrectangle{\pgfqpoint{3.722897in}{0.857143in}}{\pgfqpoint{2.627103in}{1.813434in}}%
\pgfusepath{clip}%
\pgfsetbuttcap%
\pgfsetmiterjoin%
\definecolor{currentfill}{rgb}{0.950697,0.616649,0.428624}%
\pgfsetfillcolor{currentfill}%
\pgfsetlinewidth{0.000000pt}%
\definecolor{currentstroke}{rgb}{0.000000,0.000000,0.000000}%
\pgfsetstrokecolor{currentstroke}%
\pgfsetstrokeopacity{0.000000}%
\pgfsetdash{}{0pt}%
\pgfpathmoveto{\pgfqpoint{4.088064in}{1.841410in}}%
\pgfpathlineto{\pgfqpoint{4.097001in}{1.841410in}}%
\pgfpathlineto{\pgfqpoint{4.097001in}{1.914510in}}%
\pgfpathlineto{\pgfqpoint{4.088064in}{1.914510in}}%
\pgfpathlineto{\pgfqpoint{4.088064in}{1.841410in}}%
\pgfpathclose%
\pgfusepath{fill}%
\end{pgfscope}%
\begin{pgfscope}%
\pgfpathrectangle{\pgfqpoint{3.722897in}{0.857143in}}{\pgfqpoint{2.627103in}{1.813434in}}%
\pgfusepath{clip}%
\pgfsetbuttcap%
\pgfsetmiterjoin%
\definecolor{currentfill}{rgb}{0.950697,0.616649,0.428624}%
\pgfsetfillcolor{currentfill}%
\pgfsetlinewidth{0.000000pt}%
\definecolor{currentstroke}{rgb}{0.000000,0.000000,0.000000}%
\pgfsetstrokecolor{currentstroke}%
\pgfsetstrokeopacity{0.000000}%
\pgfsetdash{}{0pt}%
\pgfpathmoveto{\pgfqpoint{4.099235in}{1.862486in}}%
\pgfpathlineto{\pgfqpoint{4.108171in}{1.862486in}}%
\pgfpathlineto{\pgfqpoint{4.108171in}{1.947683in}}%
\pgfpathlineto{\pgfqpoint{4.099235in}{1.947683in}}%
\pgfpathlineto{\pgfqpoint{4.099235in}{1.862486in}}%
\pgfpathclose%
\pgfusepath{fill}%
\end{pgfscope}%
\begin{pgfscope}%
\pgfpathrectangle{\pgfqpoint{3.722897in}{0.857143in}}{\pgfqpoint{2.627103in}{1.813434in}}%
\pgfusepath{clip}%
\pgfsetbuttcap%
\pgfsetmiterjoin%
\definecolor{currentfill}{rgb}{0.950697,0.616649,0.428624}%
\pgfsetfillcolor{currentfill}%
\pgfsetlinewidth{0.000000pt}%
\definecolor{currentstroke}{rgb}{0.000000,0.000000,0.000000}%
\pgfsetstrokecolor{currentstroke}%
\pgfsetstrokeopacity{0.000000}%
\pgfsetdash{}{0pt}%
\pgfpathmoveto{\pgfqpoint{4.110405in}{1.839849in}}%
\pgfpathlineto{\pgfqpoint{4.119342in}{1.839849in}}%
\pgfpathlineto{\pgfqpoint{4.119342in}{1.958953in}}%
\pgfpathlineto{\pgfqpoint{4.110405in}{1.958953in}}%
\pgfpathlineto{\pgfqpoint{4.110405in}{1.839849in}}%
\pgfpathclose%
\pgfusepath{fill}%
\end{pgfscope}%
\begin{pgfscope}%
\pgfpathrectangle{\pgfqpoint{3.722897in}{0.857143in}}{\pgfqpoint{2.627103in}{1.813434in}}%
\pgfusepath{clip}%
\pgfsetbuttcap%
\pgfsetmiterjoin%
\definecolor{currentfill}{rgb}{0.950697,0.616649,0.428624}%
\pgfsetfillcolor{currentfill}%
\pgfsetlinewidth{0.000000pt}%
\definecolor{currentstroke}{rgb}{0.000000,0.000000,0.000000}%
\pgfsetstrokecolor{currentstroke}%
\pgfsetstrokeopacity{0.000000}%
\pgfsetdash{}{0pt}%
\pgfpathmoveto{\pgfqpoint{4.121576in}{1.840005in}}%
\pgfpathlineto{\pgfqpoint{4.130512in}{1.840005in}}%
\pgfpathlineto{\pgfqpoint{4.130512in}{1.972578in}}%
\pgfpathlineto{\pgfqpoint{4.121576in}{1.972578in}}%
\pgfpathlineto{\pgfqpoint{4.121576in}{1.840005in}}%
\pgfpathclose%
\pgfusepath{fill}%
\end{pgfscope}%
\begin{pgfscope}%
\pgfpathrectangle{\pgfqpoint{3.722897in}{0.857143in}}{\pgfqpoint{2.627103in}{1.813434in}}%
\pgfusepath{clip}%
\pgfsetbuttcap%
\pgfsetmiterjoin%
\definecolor{currentfill}{rgb}{0.950697,0.616649,0.428624}%
\pgfsetfillcolor{currentfill}%
\pgfsetlinewidth{0.000000pt}%
\definecolor{currentstroke}{rgb}{0.000000,0.000000,0.000000}%
\pgfsetstrokecolor{currentstroke}%
\pgfsetstrokeopacity{0.000000}%
\pgfsetdash{}{0pt}%
\pgfpathmoveto{\pgfqpoint{4.132747in}{1.842605in}}%
\pgfpathlineto{\pgfqpoint{4.141683in}{1.842605in}}%
\pgfpathlineto{\pgfqpoint{4.141683in}{1.976198in}}%
\pgfpathlineto{\pgfqpoint{4.132747in}{1.976198in}}%
\pgfpathlineto{\pgfqpoint{4.132747in}{1.842605in}}%
\pgfpathclose%
\pgfusepath{fill}%
\end{pgfscope}%
\begin{pgfscope}%
\pgfpathrectangle{\pgfqpoint{3.722897in}{0.857143in}}{\pgfqpoint{2.627103in}{1.813434in}}%
\pgfusepath{clip}%
\pgfsetbuttcap%
\pgfsetmiterjoin%
\definecolor{currentfill}{rgb}{0.950697,0.616649,0.428624}%
\pgfsetfillcolor{currentfill}%
\pgfsetlinewidth{0.000000pt}%
\definecolor{currentstroke}{rgb}{0.000000,0.000000,0.000000}%
\pgfsetstrokecolor{currentstroke}%
\pgfsetstrokeopacity{0.000000}%
\pgfsetdash{}{0pt}%
\pgfpathmoveto{\pgfqpoint{4.143917in}{1.857722in}}%
\pgfpathlineto{\pgfqpoint{4.152854in}{1.857722in}}%
\pgfpathlineto{\pgfqpoint{4.152854in}{1.994874in}}%
\pgfpathlineto{\pgfqpoint{4.143917in}{1.994874in}}%
\pgfpathlineto{\pgfqpoint{4.143917in}{1.857722in}}%
\pgfpathclose%
\pgfusepath{fill}%
\end{pgfscope}%
\begin{pgfscope}%
\pgfpathrectangle{\pgfqpoint{3.722897in}{0.857143in}}{\pgfqpoint{2.627103in}{1.813434in}}%
\pgfusepath{clip}%
\pgfsetbuttcap%
\pgfsetmiterjoin%
\definecolor{currentfill}{rgb}{0.950697,0.616649,0.428624}%
\pgfsetfillcolor{currentfill}%
\pgfsetlinewidth{0.000000pt}%
\definecolor{currentstroke}{rgb}{0.000000,0.000000,0.000000}%
\pgfsetstrokecolor{currentstroke}%
\pgfsetstrokeopacity{0.000000}%
\pgfsetdash{}{0pt}%
\pgfpathmoveto{\pgfqpoint{4.155088in}{1.869320in}}%
\pgfpathlineto{\pgfqpoint{4.164024in}{1.869320in}}%
\pgfpathlineto{\pgfqpoint{4.164024in}{2.031879in}}%
\pgfpathlineto{\pgfqpoint{4.155088in}{2.031879in}}%
\pgfpathlineto{\pgfqpoint{4.155088in}{1.869320in}}%
\pgfpathclose%
\pgfusepath{fill}%
\end{pgfscope}%
\begin{pgfscope}%
\pgfpathrectangle{\pgfqpoint{3.722897in}{0.857143in}}{\pgfqpoint{2.627103in}{1.813434in}}%
\pgfusepath{clip}%
\pgfsetbuttcap%
\pgfsetmiterjoin%
\definecolor{currentfill}{rgb}{0.950697,0.616649,0.428624}%
\pgfsetfillcolor{currentfill}%
\pgfsetlinewidth{0.000000pt}%
\definecolor{currentstroke}{rgb}{0.000000,0.000000,0.000000}%
\pgfsetstrokecolor{currentstroke}%
\pgfsetstrokeopacity{0.000000}%
\pgfsetdash{}{0pt}%
\pgfpathmoveto{\pgfqpoint{4.166258in}{1.884973in}}%
\pgfpathlineto{\pgfqpoint{4.175195in}{1.884973in}}%
\pgfpathlineto{\pgfqpoint{4.175195in}{2.064334in}}%
\pgfpathlineto{\pgfqpoint{4.166258in}{2.064334in}}%
\pgfpathlineto{\pgfqpoint{4.166258in}{1.884973in}}%
\pgfpathclose%
\pgfusepath{fill}%
\end{pgfscope}%
\begin{pgfscope}%
\pgfpathrectangle{\pgfqpoint{3.722897in}{0.857143in}}{\pgfqpoint{2.627103in}{1.813434in}}%
\pgfusepath{clip}%
\pgfsetbuttcap%
\pgfsetmiterjoin%
\definecolor{currentfill}{rgb}{0.950697,0.616649,0.428624}%
\pgfsetfillcolor{currentfill}%
\pgfsetlinewidth{0.000000pt}%
\definecolor{currentstroke}{rgb}{0.000000,0.000000,0.000000}%
\pgfsetstrokecolor{currentstroke}%
\pgfsetstrokeopacity{0.000000}%
\pgfsetdash{}{0pt}%
\pgfpathmoveto{\pgfqpoint{4.177429in}{1.904637in}}%
\pgfpathlineto{\pgfqpoint{4.186365in}{1.904637in}}%
\pgfpathlineto{\pgfqpoint{4.186365in}{2.108683in}}%
\pgfpathlineto{\pgfqpoint{4.177429in}{2.108683in}}%
\pgfpathlineto{\pgfqpoint{4.177429in}{1.904637in}}%
\pgfpathclose%
\pgfusepath{fill}%
\end{pgfscope}%
\begin{pgfscope}%
\pgfpathrectangle{\pgfqpoint{3.722897in}{0.857143in}}{\pgfqpoint{2.627103in}{1.813434in}}%
\pgfusepath{clip}%
\pgfsetbuttcap%
\pgfsetmiterjoin%
\definecolor{currentfill}{rgb}{0.950697,0.616649,0.428624}%
\pgfsetfillcolor{currentfill}%
\pgfsetlinewidth{0.000000pt}%
\definecolor{currentstroke}{rgb}{0.000000,0.000000,0.000000}%
\pgfsetstrokecolor{currentstroke}%
\pgfsetstrokeopacity{0.000000}%
\pgfsetdash{}{0pt}%
\pgfpathmoveto{\pgfqpoint{4.188600in}{1.915907in}}%
\pgfpathlineto{\pgfqpoint{4.197536in}{1.915907in}}%
\pgfpathlineto{\pgfqpoint{4.197536in}{2.150350in}}%
\pgfpathlineto{\pgfqpoint{4.188600in}{2.150350in}}%
\pgfpathlineto{\pgfqpoint{4.188600in}{1.915907in}}%
\pgfpathclose%
\pgfusepath{fill}%
\end{pgfscope}%
\begin{pgfscope}%
\pgfpathrectangle{\pgfqpoint{3.722897in}{0.857143in}}{\pgfqpoint{2.627103in}{1.813434in}}%
\pgfusepath{clip}%
\pgfsetbuttcap%
\pgfsetmiterjoin%
\definecolor{currentfill}{rgb}{0.950697,0.616649,0.428624}%
\pgfsetfillcolor{currentfill}%
\pgfsetlinewidth{0.000000pt}%
\definecolor{currentstroke}{rgb}{0.000000,0.000000,0.000000}%
\pgfsetstrokecolor{currentstroke}%
\pgfsetstrokeopacity{0.000000}%
\pgfsetdash{}{0pt}%
\pgfpathmoveto{\pgfqpoint{4.199770in}{1.917483in}}%
\pgfpathlineto{\pgfqpoint{4.208707in}{1.917483in}}%
\pgfpathlineto{\pgfqpoint{4.208707in}{2.165291in}}%
\pgfpathlineto{\pgfqpoint{4.199770in}{2.165291in}}%
\pgfpathlineto{\pgfqpoint{4.199770in}{1.917483in}}%
\pgfpathclose%
\pgfusepath{fill}%
\end{pgfscope}%
\begin{pgfscope}%
\pgfpathrectangle{\pgfqpoint{3.722897in}{0.857143in}}{\pgfqpoint{2.627103in}{1.813434in}}%
\pgfusepath{clip}%
\pgfsetbuttcap%
\pgfsetmiterjoin%
\definecolor{currentfill}{rgb}{0.950697,0.616649,0.428624}%
\pgfsetfillcolor{currentfill}%
\pgfsetlinewidth{0.000000pt}%
\definecolor{currentstroke}{rgb}{0.000000,0.000000,0.000000}%
\pgfsetstrokecolor{currentstroke}%
\pgfsetstrokeopacity{0.000000}%
\pgfsetdash{}{0pt}%
\pgfpathmoveto{\pgfqpoint{4.210941in}{1.916667in}}%
\pgfpathlineto{\pgfqpoint{4.219877in}{1.916667in}}%
\pgfpathlineto{\pgfqpoint{4.219877in}{2.185148in}}%
\pgfpathlineto{\pgfqpoint{4.210941in}{2.185148in}}%
\pgfpathlineto{\pgfqpoint{4.210941in}{1.916667in}}%
\pgfpathclose%
\pgfusepath{fill}%
\end{pgfscope}%
\begin{pgfscope}%
\pgfpathrectangle{\pgfqpoint{3.722897in}{0.857143in}}{\pgfqpoint{2.627103in}{1.813434in}}%
\pgfusepath{clip}%
\pgfsetbuttcap%
\pgfsetmiterjoin%
\definecolor{currentfill}{rgb}{0.950697,0.616649,0.428624}%
\pgfsetfillcolor{currentfill}%
\pgfsetlinewidth{0.000000pt}%
\definecolor{currentstroke}{rgb}{0.000000,0.000000,0.000000}%
\pgfsetstrokecolor{currentstroke}%
\pgfsetstrokeopacity{0.000000}%
\pgfsetdash{}{0pt}%
\pgfpathmoveto{\pgfqpoint{4.222111in}{1.916947in}}%
\pgfpathlineto{\pgfqpoint{4.231048in}{1.916947in}}%
\pgfpathlineto{\pgfqpoint{4.231048in}{2.237327in}}%
\pgfpathlineto{\pgfqpoint{4.222111in}{2.237327in}}%
\pgfpathlineto{\pgfqpoint{4.222111in}{1.916947in}}%
\pgfpathclose%
\pgfusepath{fill}%
\end{pgfscope}%
\begin{pgfscope}%
\pgfpathrectangle{\pgfqpoint{3.722897in}{0.857143in}}{\pgfqpoint{2.627103in}{1.813434in}}%
\pgfusepath{clip}%
\pgfsetbuttcap%
\pgfsetmiterjoin%
\definecolor{currentfill}{rgb}{0.950697,0.616649,0.428624}%
\pgfsetfillcolor{currentfill}%
\pgfsetlinewidth{0.000000pt}%
\definecolor{currentstroke}{rgb}{0.000000,0.000000,0.000000}%
\pgfsetstrokecolor{currentstroke}%
\pgfsetstrokeopacity{0.000000}%
\pgfsetdash{}{0pt}%
\pgfpathmoveto{\pgfqpoint{4.233282in}{1.913977in}}%
\pgfpathlineto{\pgfqpoint{4.242218in}{1.913977in}}%
\pgfpathlineto{\pgfqpoint{4.242218in}{2.242489in}}%
\pgfpathlineto{\pgfqpoint{4.233282in}{2.242489in}}%
\pgfpathlineto{\pgfqpoint{4.233282in}{1.913977in}}%
\pgfpathclose%
\pgfusepath{fill}%
\end{pgfscope}%
\begin{pgfscope}%
\pgfpathrectangle{\pgfqpoint{3.722897in}{0.857143in}}{\pgfqpoint{2.627103in}{1.813434in}}%
\pgfusepath{clip}%
\pgfsetbuttcap%
\pgfsetmiterjoin%
\definecolor{currentfill}{rgb}{0.950697,0.616649,0.428624}%
\pgfsetfillcolor{currentfill}%
\pgfsetlinewidth{0.000000pt}%
\definecolor{currentstroke}{rgb}{0.000000,0.000000,0.000000}%
\pgfsetstrokecolor{currentstroke}%
\pgfsetstrokeopacity{0.000000}%
\pgfsetdash{}{0pt}%
\pgfpathmoveto{\pgfqpoint{4.244453in}{1.907650in}}%
\pgfpathlineto{\pgfqpoint{4.253389in}{1.907650in}}%
\pgfpathlineto{\pgfqpoint{4.253389in}{2.229038in}}%
\pgfpathlineto{\pgfqpoint{4.244453in}{2.229038in}}%
\pgfpathlineto{\pgfqpoint{4.244453in}{1.907650in}}%
\pgfpathclose%
\pgfusepath{fill}%
\end{pgfscope}%
\begin{pgfscope}%
\pgfpathrectangle{\pgfqpoint{3.722897in}{0.857143in}}{\pgfqpoint{2.627103in}{1.813434in}}%
\pgfusepath{clip}%
\pgfsetbuttcap%
\pgfsetmiterjoin%
\definecolor{currentfill}{rgb}{0.950697,0.616649,0.428624}%
\pgfsetfillcolor{currentfill}%
\pgfsetlinewidth{0.000000pt}%
\definecolor{currentstroke}{rgb}{0.000000,0.000000,0.000000}%
\pgfsetstrokecolor{currentstroke}%
\pgfsetstrokeopacity{0.000000}%
\pgfsetdash{}{0pt}%
\pgfpathmoveto{\pgfqpoint{4.255623in}{1.898078in}}%
\pgfpathlineto{\pgfqpoint{4.264560in}{1.898078in}}%
\pgfpathlineto{\pgfqpoint{4.264560in}{2.196939in}}%
\pgfpathlineto{\pgfqpoint{4.255623in}{2.196939in}}%
\pgfpathlineto{\pgfqpoint{4.255623in}{1.898078in}}%
\pgfpathclose%
\pgfusepath{fill}%
\end{pgfscope}%
\begin{pgfscope}%
\pgfpathrectangle{\pgfqpoint{3.722897in}{0.857143in}}{\pgfqpoint{2.627103in}{1.813434in}}%
\pgfusepath{clip}%
\pgfsetbuttcap%
\pgfsetmiterjoin%
\definecolor{currentfill}{rgb}{0.950697,0.616649,0.428624}%
\pgfsetfillcolor{currentfill}%
\pgfsetlinewidth{0.000000pt}%
\definecolor{currentstroke}{rgb}{0.000000,0.000000,0.000000}%
\pgfsetstrokecolor{currentstroke}%
\pgfsetstrokeopacity{0.000000}%
\pgfsetdash{}{0pt}%
\pgfpathmoveto{\pgfqpoint{4.266794in}{1.894666in}}%
\pgfpathlineto{\pgfqpoint{4.275730in}{1.894666in}}%
\pgfpathlineto{\pgfqpoint{4.275730in}{2.166916in}}%
\pgfpathlineto{\pgfqpoint{4.266794in}{2.166916in}}%
\pgfpathlineto{\pgfqpoint{4.266794in}{1.894666in}}%
\pgfpathclose%
\pgfusepath{fill}%
\end{pgfscope}%
\begin{pgfscope}%
\pgfpathrectangle{\pgfqpoint{3.722897in}{0.857143in}}{\pgfqpoint{2.627103in}{1.813434in}}%
\pgfusepath{clip}%
\pgfsetbuttcap%
\pgfsetmiterjoin%
\definecolor{currentfill}{rgb}{0.950697,0.616649,0.428624}%
\pgfsetfillcolor{currentfill}%
\pgfsetlinewidth{0.000000pt}%
\definecolor{currentstroke}{rgb}{0.000000,0.000000,0.000000}%
\pgfsetstrokecolor{currentstroke}%
\pgfsetstrokeopacity{0.000000}%
\pgfsetdash{}{0pt}%
\pgfpathmoveto{\pgfqpoint{4.277964in}{1.885121in}}%
\pgfpathlineto{\pgfqpoint{4.286901in}{1.885121in}}%
\pgfpathlineto{\pgfqpoint{4.286901in}{2.142394in}}%
\pgfpathlineto{\pgfqpoint{4.277964in}{2.142394in}}%
\pgfpathlineto{\pgfqpoint{4.277964in}{1.885121in}}%
\pgfpathclose%
\pgfusepath{fill}%
\end{pgfscope}%
\begin{pgfscope}%
\pgfpathrectangle{\pgfqpoint{3.722897in}{0.857143in}}{\pgfqpoint{2.627103in}{1.813434in}}%
\pgfusepath{clip}%
\pgfsetbuttcap%
\pgfsetmiterjoin%
\definecolor{currentfill}{rgb}{0.950697,0.616649,0.428624}%
\pgfsetfillcolor{currentfill}%
\pgfsetlinewidth{0.000000pt}%
\definecolor{currentstroke}{rgb}{0.000000,0.000000,0.000000}%
\pgfsetstrokecolor{currentstroke}%
\pgfsetstrokeopacity{0.000000}%
\pgfsetdash{}{0pt}%
\pgfpathmoveto{\pgfqpoint{4.289135in}{1.875137in}}%
\pgfpathlineto{\pgfqpoint{4.298071in}{1.875137in}}%
\pgfpathlineto{\pgfqpoint{4.298071in}{2.118820in}}%
\pgfpathlineto{\pgfqpoint{4.289135in}{2.118820in}}%
\pgfpathlineto{\pgfqpoint{4.289135in}{1.875137in}}%
\pgfpathclose%
\pgfusepath{fill}%
\end{pgfscope}%
\begin{pgfscope}%
\pgfpathrectangle{\pgfqpoint{3.722897in}{0.857143in}}{\pgfqpoint{2.627103in}{1.813434in}}%
\pgfusepath{clip}%
\pgfsetbuttcap%
\pgfsetmiterjoin%
\definecolor{currentfill}{rgb}{0.950697,0.616649,0.428624}%
\pgfsetfillcolor{currentfill}%
\pgfsetlinewidth{0.000000pt}%
\definecolor{currentstroke}{rgb}{0.000000,0.000000,0.000000}%
\pgfsetstrokecolor{currentstroke}%
\pgfsetstrokeopacity{0.000000}%
\pgfsetdash{}{0pt}%
\pgfpathmoveto{\pgfqpoint{4.300306in}{1.865131in}}%
\pgfpathlineto{\pgfqpoint{4.309242in}{1.865131in}}%
\pgfpathlineto{\pgfqpoint{4.309242in}{2.105051in}}%
\pgfpathlineto{\pgfqpoint{4.300306in}{2.105051in}}%
\pgfpathlineto{\pgfqpoint{4.300306in}{1.865131in}}%
\pgfpathclose%
\pgfusepath{fill}%
\end{pgfscope}%
\begin{pgfscope}%
\pgfpathrectangle{\pgfqpoint{3.722897in}{0.857143in}}{\pgfqpoint{2.627103in}{1.813434in}}%
\pgfusepath{clip}%
\pgfsetbuttcap%
\pgfsetmiterjoin%
\definecolor{currentfill}{rgb}{0.950697,0.616649,0.428624}%
\pgfsetfillcolor{currentfill}%
\pgfsetlinewidth{0.000000pt}%
\definecolor{currentstroke}{rgb}{0.000000,0.000000,0.000000}%
\pgfsetstrokecolor{currentstroke}%
\pgfsetstrokeopacity{0.000000}%
\pgfsetdash{}{0pt}%
\pgfpathmoveto{\pgfqpoint{4.311476in}{1.855424in}}%
\pgfpathlineto{\pgfqpoint{4.320413in}{1.855424in}}%
\pgfpathlineto{\pgfqpoint{4.320413in}{2.080482in}}%
\pgfpathlineto{\pgfqpoint{4.311476in}{2.080482in}}%
\pgfpathlineto{\pgfqpoint{4.311476in}{1.855424in}}%
\pgfpathclose%
\pgfusepath{fill}%
\end{pgfscope}%
\begin{pgfscope}%
\pgfpathrectangle{\pgfqpoint{3.722897in}{0.857143in}}{\pgfqpoint{2.627103in}{1.813434in}}%
\pgfusepath{clip}%
\pgfsetbuttcap%
\pgfsetmiterjoin%
\definecolor{currentfill}{rgb}{0.950697,0.616649,0.428624}%
\pgfsetfillcolor{currentfill}%
\pgfsetlinewidth{0.000000pt}%
\definecolor{currentstroke}{rgb}{0.000000,0.000000,0.000000}%
\pgfsetstrokecolor{currentstroke}%
\pgfsetstrokeopacity{0.000000}%
\pgfsetdash{}{0pt}%
\pgfpathmoveto{\pgfqpoint{4.322647in}{1.848608in}}%
\pgfpathlineto{\pgfqpoint{4.331583in}{1.848608in}}%
\pgfpathlineto{\pgfqpoint{4.331583in}{2.052765in}}%
\pgfpathlineto{\pgfqpoint{4.322647in}{2.052765in}}%
\pgfpathlineto{\pgfqpoint{4.322647in}{1.848608in}}%
\pgfpathclose%
\pgfusepath{fill}%
\end{pgfscope}%
\begin{pgfscope}%
\pgfpathrectangle{\pgfqpoint{3.722897in}{0.857143in}}{\pgfqpoint{2.627103in}{1.813434in}}%
\pgfusepath{clip}%
\pgfsetbuttcap%
\pgfsetmiterjoin%
\definecolor{currentfill}{rgb}{0.950697,0.616649,0.428624}%
\pgfsetfillcolor{currentfill}%
\pgfsetlinewidth{0.000000pt}%
\definecolor{currentstroke}{rgb}{0.000000,0.000000,0.000000}%
\pgfsetstrokecolor{currentstroke}%
\pgfsetstrokeopacity{0.000000}%
\pgfsetdash{}{0pt}%
\pgfpathmoveto{\pgfqpoint{4.333817in}{1.849856in}}%
\pgfpathlineto{\pgfqpoint{4.342754in}{1.849856in}}%
\pgfpathlineto{\pgfqpoint{4.342754in}{2.058073in}}%
\pgfpathlineto{\pgfqpoint{4.333817in}{2.058073in}}%
\pgfpathlineto{\pgfqpoint{4.333817in}{1.849856in}}%
\pgfpathclose%
\pgfusepath{fill}%
\end{pgfscope}%
\begin{pgfscope}%
\pgfpathrectangle{\pgfqpoint{3.722897in}{0.857143in}}{\pgfqpoint{2.627103in}{1.813434in}}%
\pgfusepath{clip}%
\pgfsetbuttcap%
\pgfsetmiterjoin%
\definecolor{currentfill}{rgb}{0.950697,0.616649,0.428624}%
\pgfsetfillcolor{currentfill}%
\pgfsetlinewidth{0.000000pt}%
\definecolor{currentstroke}{rgb}{0.000000,0.000000,0.000000}%
\pgfsetstrokecolor{currentstroke}%
\pgfsetstrokeopacity{0.000000}%
\pgfsetdash{}{0pt}%
\pgfpathmoveto{\pgfqpoint{4.344988in}{1.851537in}}%
\pgfpathlineto{\pgfqpoint{4.353925in}{1.851537in}}%
\pgfpathlineto{\pgfqpoint{4.353925in}{2.067201in}}%
\pgfpathlineto{\pgfqpoint{4.344988in}{2.067201in}}%
\pgfpathlineto{\pgfqpoint{4.344988in}{1.851537in}}%
\pgfpathclose%
\pgfusepath{fill}%
\end{pgfscope}%
\begin{pgfscope}%
\pgfpathrectangle{\pgfqpoint{3.722897in}{0.857143in}}{\pgfqpoint{2.627103in}{1.813434in}}%
\pgfusepath{clip}%
\pgfsetbuttcap%
\pgfsetmiterjoin%
\definecolor{currentfill}{rgb}{0.950697,0.616649,0.428624}%
\pgfsetfillcolor{currentfill}%
\pgfsetlinewidth{0.000000pt}%
\definecolor{currentstroke}{rgb}{0.000000,0.000000,0.000000}%
\pgfsetstrokecolor{currentstroke}%
\pgfsetstrokeopacity{0.000000}%
\pgfsetdash{}{0pt}%
\pgfpathmoveto{\pgfqpoint{4.356159in}{1.855699in}}%
\pgfpathlineto{\pgfqpoint{4.365095in}{1.855699in}}%
\pgfpathlineto{\pgfqpoint{4.365095in}{2.043939in}}%
\pgfpathlineto{\pgfqpoint{4.356159in}{2.043939in}}%
\pgfpathlineto{\pgfqpoint{4.356159in}{1.855699in}}%
\pgfpathclose%
\pgfusepath{fill}%
\end{pgfscope}%
\begin{pgfscope}%
\pgfpathrectangle{\pgfqpoint{3.722897in}{0.857143in}}{\pgfqpoint{2.627103in}{1.813434in}}%
\pgfusepath{clip}%
\pgfsetbuttcap%
\pgfsetmiterjoin%
\definecolor{currentfill}{rgb}{0.950697,0.616649,0.428624}%
\pgfsetfillcolor{currentfill}%
\pgfsetlinewidth{0.000000pt}%
\definecolor{currentstroke}{rgb}{0.000000,0.000000,0.000000}%
\pgfsetstrokecolor{currentstroke}%
\pgfsetstrokeopacity{0.000000}%
\pgfsetdash{}{0pt}%
\pgfpathmoveto{\pgfqpoint{4.367329in}{1.870244in}}%
\pgfpathlineto{\pgfqpoint{4.376266in}{1.870244in}}%
\pgfpathlineto{\pgfqpoint{4.376266in}{2.052888in}}%
\pgfpathlineto{\pgfqpoint{4.367329in}{2.052888in}}%
\pgfpathlineto{\pgfqpoint{4.367329in}{1.870244in}}%
\pgfpathclose%
\pgfusepath{fill}%
\end{pgfscope}%
\begin{pgfscope}%
\pgfpathrectangle{\pgfqpoint{3.722897in}{0.857143in}}{\pgfqpoint{2.627103in}{1.813434in}}%
\pgfusepath{clip}%
\pgfsetbuttcap%
\pgfsetmiterjoin%
\definecolor{currentfill}{rgb}{0.950697,0.616649,0.428624}%
\pgfsetfillcolor{currentfill}%
\pgfsetlinewidth{0.000000pt}%
\definecolor{currentstroke}{rgb}{0.000000,0.000000,0.000000}%
\pgfsetstrokecolor{currentstroke}%
\pgfsetstrokeopacity{0.000000}%
\pgfsetdash{}{0pt}%
\pgfpathmoveto{\pgfqpoint{4.378500in}{1.893063in}}%
\pgfpathlineto{\pgfqpoint{4.387436in}{1.893063in}}%
\pgfpathlineto{\pgfqpoint{4.387436in}{2.086623in}}%
\pgfpathlineto{\pgfqpoint{4.378500in}{2.086623in}}%
\pgfpathlineto{\pgfqpoint{4.378500in}{1.893063in}}%
\pgfpathclose%
\pgfusepath{fill}%
\end{pgfscope}%
\begin{pgfscope}%
\pgfpathrectangle{\pgfqpoint{3.722897in}{0.857143in}}{\pgfqpoint{2.627103in}{1.813434in}}%
\pgfusepath{clip}%
\pgfsetbuttcap%
\pgfsetmiterjoin%
\definecolor{currentfill}{rgb}{0.950697,0.616649,0.428624}%
\pgfsetfillcolor{currentfill}%
\pgfsetlinewidth{0.000000pt}%
\definecolor{currentstroke}{rgb}{0.000000,0.000000,0.000000}%
\pgfsetstrokecolor{currentstroke}%
\pgfsetstrokeopacity{0.000000}%
\pgfsetdash{}{0pt}%
\pgfpathmoveto{\pgfqpoint{4.389670in}{1.923586in}}%
\pgfpathlineto{\pgfqpoint{4.398607in}{1.923586in}}%
\pgfpathlineto{\pgfqpoint{4.398607in}{2.116757in}}%
\pgfpathlineto{\pgfqpoint{4.389670in}{2.116757in}}%
\pgfpathlineto{\pgfqpoint{4.389670in}{1.923586in}}%
\pgfpathclose%
\pgfusepath{fill}%
\end{pgfscope}%
\begin{pgfscope}%
\pgfpathrectangle{\pgfqpoint{3.722897in}{0.857143in}}{\pgfqpoint{2.627103in}{1.813434in}}%
\pgfusepath{clip}%
\pgfsetbuttcap%
\pgfsetmiterjoin%
\definecolor{currentfill}{rgb}{0.950697,0.616649,0.428624}%
\pgfsetfillcolor{currentfill}%
\pgfsetlinewidth{0.000000pt}%
\definecolor{currentstroke}{rgb}{0.000000,0.000000,0.000000}%
\pgfsetstrokecolor{currentstroke}%
\pgfsetstrokeopacity{0.000000}%
\pgfsetdash{}{0pt}%
\pgfpathmoveto{\pgfqpoint{4.400841in}{1.954900in}}%
\pgfpathlineto{\pgfqpoint{4.409778in}{1.954900in}}%
\pgfpathlineto{\pgfqpoint{4.409778in}{2.141787in}}%
\pgfpathlineto{\pgfqpoint{4.400841in}{2.141787in}}%
\pgfpathlineto{\pgfqpoint{4.400841in}{1.954900in}}%
\pgfpathclose%
\pgfusepath{fill}%
\end{pgfscope}%
\begin{pgfscope}%
\pgfpathrectangle{\pgfqpoint{3.722897in}{0.857143in}}{\pgfqpoint{2.627103in}{1.813434in}}%
\pgfusepath{clip}%
\pgfsetbuttcap%
\pgfsetmiterjoin%
\definecolor{currentfill}{rgb}{0.950697,0.616649,0.428624}%
\pgfsetfillcolor{currentfill}%
\pgfsetlinewidth{0.000000pt}%
\definecolor{currentstroke}{rgb}{0.000000,0.000000,0.000000}%
\pgfsetstrokecolor{currentstroke}%
\pgfsetstrokeopacity{0.000000}%
\pgfsetdash{}{0pt}%
\pgfpathmoveto{\pgfqpoint{4.412012in}{1.983614in}}%
\pgfpathlineto{\pgfqpoint{4.420948in}{1.983614in}}%
\pgfpathlineto{\pgfqpoint{4.420948in}{2.202364in}}%
\pgfpathlineto{\pgfqpoint{4.412012in}{2.202364in}}%
\pgfpathlineto{\pgfqpoint{4.412012in}{1.983614in}}%
\pgfpathclose%
\pgfusepath{fill}%
\end{pgfscope}%
\begin{pgfscope}%
\pgfpathrectangle{\pgfqpoint{3.722897in}{0.857143in}}{\pgfqpoint{2.627103in}{1.813434in}}%
\pgfusepath{clip}%
\pgfsetbuttcap%
\pgfsetmiterjoin%
\definecolor{currentfill}{rgb}{0.950697,0.616649,0.428624}%
\pgfsetfillcolor{currentfill}%
\pgfsetlinewidth{0.000000pt}%
\definecolor{currentstroke}{rgb}{0.000000,0.000000,0.000000}%
\pgfsetstrokecolor{currentstroke}%
\pgfsetstrokeopacity{0.000000}%
\pgfsetdash{}{0pt}%
\pgfpathmoveto{\pgfqpoint{4.423182in}{2.012090in}}%
\pgfpathlineto{\pgfqpoint{4.432119in}{2.012090in}}%
\pgfpathlineto{\pgfqpoint{4.432119in}{2.233230in}}%
\pgfpathlineto{\pgfqpoint{4.423182in}{2.233230in}}%
\pgfpathlineto{\pgfqpoint{4.423182in}{2.012090in}}%
\pgfpathclose%
\pgfusepath{fill}%
\end{pgfscope}%
\begin{pgfscope}%
\pgfpathrectangle{\pgfqpoint{3.722897in}{0.857143in}}{\pgfqpoint{2.627103in}{1.813434in}}%
\pgfusepath{clip}%
\pgfsetbuttcap%
\pgfsetmiterjoin%
\definecolor{currentfill}{rgb}{0.950697,0.616649,0.428624}%
\pgfsetfillcolor{currentfill}%
\pgfsetlinewidth{0.000000pt}%
\definecolor{currentstroke}{rgb}{0.000000,0.000000,0.000000}%
\pgfsetstrokecolor{currentstroke}%
\pgfsetstrokeopacity{0.000000}%
\pgfsetdash{}{0pt}%
\pgfpathmoveto{\pgfqpoint{4.434353in}{2.031538in}}%
\pgfpathlineto{\pgfqpoint{4.443289in}{2.031538in}}%
\pgfpathlineto{\pgfqpoint{4.443289in}{2.257617in}}%
\pgfpathlineto{\pgfqpoint{4.434353in}{2.257617in}}%
\pgfpathlineto{\pgfqpoint{4.434353in}{2.031538in}}%
\pgfpathclose%
\pgfusepath{fill}%
\end{pgfscope}%
\begin{pgfscope}%
\pgfpathrectangle{\pgfqpoint{3.722897in}{0.857143in}}{\pgfqpoint{2.627103in}{1.813434in}}%
\pgfusepath{clip}%
\pgfsetbuttcap%
\pgfsetmiterjoin%
\definecolor{currentfill}{rgb}{0.950697,0.616649,0.428624}%
\pgfsetfillcolor{currentfill}%
\pgfsetlinewidth{0.000000pt}%
\definecolor{currentstroke}{rgb}{0.000000,0.000000,0.000000}%
\pgfsetstrokecolor{currentstroke}%
\pgfsetstrokeopacity{0.000000}%
\pgfsetdash{}{0pt}%
\pgfpathmoveto{\pgfqpoint{4.445523in}{2.035405in}}%
\pgfpathlineto{\pgfqpoint{4.454460in}{2.035405in}}%
\pgfpathlineto{\pgfqpoint{4.454460in}{2.264146in}}%
\pgfpathlineto{\pgfqpoint{4.445523in}{2.264146in}}%
\pgfpathlineto{\pgfqpoint{4.445523in}{2.035405in}}%
\pgfpathclose%
\pgfusepath{fill}%
\end{pgfscope}%
\begin{pgfscope}%
\pgfpathrectangle{\pgfqpoint{3.722897in}{0.857143in}}{\pgfqpoint{2.627103in}{1.813434in}}%
\pgfusepath{clip}%
\pgfsetbuttcap%
\pgfsetmiterjoin%
\definecolor{currentfill}{rgb}{0.950697,0.616649,0.428624}%
\pgfsetfillcolor{currentfill}%
\pgfsetlinewidth{0.000000pt}%
\definecolor{currentstroke}{rgb}{0.000000,0.000000,0.000000}%
\pgfsetstrokecolor{currentstroke}%
\pgfsetstrokeopacity{0.000000}%
\pgfsetdash{}{0pt}%
\pgfpathmoveto{\pgfqpoint{4.456694in}{2.032265in}}%
\pgfpathlineto{\pgfqpoint{4.465631in}{2.032265in}}%
\pgfpathlineto{\pgfqpoint{4.465631in}{2.271072in}}%
\pgfpathlineto{\pgfqpoint{4.456694in}{2.271072in}}%
\pgfpathlineto{\pgfqpoint{4.456694in}{2.032265in}}%
\pgfpathclose%
\pgfusepath{fill}%
\end{pgfscope}%
\begin{pgfscope}%
\pgfpathrectangle{\pgfqpoint{3.722897in}{0.857143in}}{\pgfqpoint{2.627103in}{1.813434in}}%
\pgfusepath{clip}%
\pgfsetbuttcap%
\pgfsetmiterjoin%
\definecolor{currentfill}{rgb}{0.950697,0.616649,0.428624}%
\pgfsetfillcolor{currentfill}%
\pgfsetlinewidth{0.000000pt}%
\definecolor{currentstroke}{rgb}{0.000000,0.000000,0.000000}%
\pgfsetstrokecolor{currentstroke}%
\pgfsetstrokeopacity{0.000000}%
\pgfsetdash{}{0pt}%
\pgfpathmoveto{\pgfqpoint{4.467865in}{2.029280in}}%
\pgfpathlineto{\pgfqpoint{4.476801in}{2.029280in}}%
\pgfpathlineto{\pgfqpoint{4.476801in}{2.293634in}}%
\pgfpathlineto{\pgfqpoint{4.467865in}{2.293634in}}%
\pgfpathlineto{\pgfqpoint{4.467865in}{2.029280in}}%
\pgfpathclose%
\pgfusepath{fill}%
\end{pgfscope}%
\begin{pgfscope}%
\pgfpathrectangle{\pgfqpoint{3.722897in}{0.857143in}}{\pgfqpoint{2.627103in}{1.813434in}}%
\pgfusepath{clip}%
\pgfsetbuttcap%
\pgfsetmiterjoin%
\definecolor{currentfill}{rgb}{0.950697,0.616649,0.428624}%
\pgfsetfillcolor{currentfill}%
\pgfsetlinewidth{0.000000pt}%
\definecolor{currentstroke}{rgb}{0.000000,0.000000,0.000000}%
\pgfsetstrokecolor{currentstroke}%
\pgfsetstrokeopacity{0.000000}%
\pgfsetdash{}{0pt}%
\pgfpathmoveto{\pgfqpoint{4.479035in}{2.022715in}}%
\pgfpathlineto{\pgfqpoint{4.487972in}{2.022715in}}%
\pgfpathlineto{\pgfqpoint{4.487972in}{2.336299in}}%
\pgfpathlineto{\pgfqpoint{4.479035in}{2.336299in}}%
\pgfpathlineto{\pgfqpoint{4.479035in}{2.022715in}}%
\pgfpathclose%
\pgfusepath{fill}%
\end{pgfscope}%
\begin{pgfscope}%
\pgfpathrectangle{\pgfqpoint{3.722897in}{0.857143in}}{\pgfqpoint{2.627103in}{1.813434in}}%
\pgfusepath{clip}%
\pgfsetbuttcap%
\pgfsetmiterjoin%
\definecolor{currentfill}{rgb}{0.950697,0.616649,0.428624}%
\pgfsetfillcolor{currentfill}%
\pgfsetlinewidth{0.000000pt}%
\definecolor{currentstroke}{rgb}{0.000000,0.000000,0.000000}%
\pgfsetstrokecolor{currentstroke}%
\pgfsetstrokeopacity{0.000000}%
\pgfsetdash{}{0pt}%
\pgfpathmoveto{\pgfqpoint{4.490206in}{2.032611in}}%
\pgfpathlineto{\pgfqpoint{4.499142in}{2.032611in}}%
\pgfpathlineto{\pgfqpoint{4.499142in}{2.325371in}}%
\pgfpathlineto{\pgfqpoint{4.490206in}{2.325371in}}%
\pgfpathlineto{\pgfqpoint{4.490206in}{2.032611in}}%
\pgfpathclose%
\pgfusepath{fill}%
\end{pgfscope}%
\begin{pgfscope}%
\pgfpathrectangle{\pgfqpoint{3.722897in}{0.857143in}}{\pgfqpoint{2.627103in}{1.813434in}}%
\pgfusepath{clip}%
\pgfsetbuttcap%
\pgfsetmiterjoin%
\definecolor{currentfill}{rgb}{0.950697,0.616649,0.428624}%
\pgfsetfillcolor{currentfill}%
\pgfsetlinewidth{0.000000pt}%
\definecolor{currentstroke}{rgb}{0.000000,0.000000,0.000000}%
\pgfsetstrokecolor{currentstroke}%
\pgfsetstrokeopacity{0.000000}%
\pgfsetdash{}{0pt}%
\pgfpathmoveto{\pgfqpoint{4.501377in}{2.049401in}}%
\pgfpathlineto{\pgfqpoint{4.510313in}{2.049401in}}%
\pgfpathlineto{\pgfqpoint{4.510313in}{2.300949in}}%
\pgfpathlineto{\pgfqpoint{4.501377in}{2.300949in}}%
\pgfpathlineto{\pgfqpoint{4.501377in}{2.049401in}}%
\pgfpathclose%
\pgfusepath{fill}%
\end{pgfscope}%
\begin{pgfscope}%
\pgfpathrectangle{\pgfqpoint{3.722897in}{0.857143in}}{\pgfqpoint{2.627103in}{1.813434in}}%
\pgfusepath{clip}%
\pgfsetbuttcap%
\pgfsetmiterjoin%
\definecolor{currentfill}{rgb}{0.950697,0.616649,0.428624}%
\pgfsetfillcolor{currentfill}%
\pgfsetlinewidth{0.000000pt}%
\definecolor{currentstroke}{rgb}{0.000000,0.000000,0.000000}%
\pgfsetstrokecolor{currentstroke}%
\pgfsetstrokeopacity{0.000000}%
\pgfsetdash{}{0pt}%
\pgfpathmoveto{\pgfqpoint{4.512547in}{2.071933in}}%
\pgfpathlineto{\pgfqpoint{4.521484in}{2.071933in}}%
\pgfpathlineto{\pgfqpoint{4.521484in}{2.374305in}}%
\pgfpathlineto{\pgfqpoint{4.512547in}{2.374305in}}%
\pgfpathlineto{\pgfqpoint{4.512547in}{2.071933in}}%
\pgfpathclose%
\pgfusepath{fill}%
\end{pgfscope}%
\begin{pgfscope}%
\pgfpathrectangle{\pgfqpoint{3.722897in}{0.857143in}}{\pgfqpoint{2.627103in}{1.813434in}}%
\pgfusepath{clip}%
\pgfsetbuttcap%
\pgfsetmiterjoin%
\definecolor{currentfill}{rgb}{0.950697,0.616649,0.428624}%
\pgfsetfillcolor{currentfill}%
\pgfsetlinewidth{0.000000pt}%
\definecolor{currentstroke}{rgb}{0.000000,0.000000,0.000000}%
\pgfsetstrokecolor{currentstroke}%
\pgfsetstrokeopacity{0.000000}%
\pgfsetdash{}{0pt}%
\pgfpathmoveto{\pgfqpoint{4.523718in}{2.096891in}}%
\pgfpathlineto{\pgfqpoint{4.532654in}{2.096891in}}%
\pgfpathlineto{\pgfqpoint{4.532654in}{2.403098in}}%
\pgfpathlineto{\pgfqpoint{4.523718in}{2.403098in}}%
\pgfpathlineto{\pgfqpoint{4.523718in}{2.096891in}}%
\pgfpathclose%
\pgfusepath{fill}%
\end{pgfscope}%
\begin{pgfscope}%
\pgfpathrectangle{\pgfqpoint{3.722897in}{0.857143in}}{\pgfqpoint{2.627103in}{1.813434in}}%
\pgfusepath{clip}%
\pgfsetbuttcap%
\pgfsetmiterjoin%
\definecolor{currentfill}{rgb}{0.950697,0.616649,0.428624}%
\pgfsetfillcolor{currentfill}%
\pgfsetlinewidth{0.000000pt}%
\definecolor{currentstroke}{rgb}{0.000000,0.000000,0.000000}%
\pgfsetstrokecolor{currentstroke}%
\pgfsetstrokeopacity{0.000000}%
\pgfsetdash{}{0pt}%
\pgfpathmoveto{\pgfqpoint{4.534888in}{2.108736in}}%
\pgfpathlineto{\pgfqpoint{4.543825in}{2.108736in}}%
\pgfpathlineto{\pgfqpoint{4.543825in}{2.454011in}}%
\pgfpathlineto{\pgfqpoint{4.534888in}{2.454011in}}%
\pgfpathlineto{\pgfqpoint{4.534888in}{2.108736in}}%
\pgfpathclose%
\pgfusepath{fill}%
\end{pgfscope}%
\begin{pgfscope}%
\pgfpathrectangle{\pgfqpoint{3.722897in}{0.857143in}}{\pgfqpoint{2.627103in}{1.813434in}}%
\pgfusepath{clip}%
\pgfsetbuttcap%
\pgfsetmiterjoin%
\definecolor{currentfill}{rgb}{0.950697,0.616649,0.428624}%
\pgfsetfillcolor{currentfill}%
\pgfsetlinewidth{0.000000pt}%
\definecolor{currentstroke}{rgb}{0.000000,0.000000,0.000000}%
\pgfsetstrokecolor{currentstroke}%
\pgfsetstrokeopacity{0.000000}%
\pgfsetdash{}{0pt}%
\pgfpathmoveto{\pgfqpoint{4.546059in}{2.107204in}}%
\pgfpathlineto{\pgfqpoint{4.554995in}{2.107204in}}%
\pgfpathlineto{\pgfqpoint{4.554995in}{2.472773in}}%
\pgfpathlineto{\pgfqpoint{4.546059in}{2.472773in}}%
\pgfpathlineto{\pgfqpoint{4.546059in}{2.107204in}}%
\pgfpathclose%
\pgfusepath{fill}%
\end{pgfscope}%
\begin{pgfscope}%
\pgfpathrectangle{\pgfqpoint{3.722897in}{0.857143in}}{\pgfqpoint{2.627103in}{1.813434in}}%
\pgfusepath{clip}%
\pgfsetbuttcap%
\pgfsetmiterjoin%
\definecolor{currentfill}{rgb}{0.950697,0.616649,0.428624}%
\pgfsetfillcolor{currentfill}%
\pgfsetlinewidth{0.000000pt}%
\definecolor{currentstroke}{rgb}{0.000000,0.000000,0.000000}%
\pgfsetstrokecolor{currentstroke}%
\pgfsetstrokeopacity{0.000000}%
\pgfsetdash{}{0pt}%
\pgfpathmoveto{\pgfqpoint{4.557230in}{2.103986in}}%
\pgfpathlineto{\pgfqpoint{4.566166in}{2.103986in}}%
\pgfpathlineto{\pgfqpoint{4.566166in}{2.482677in}}%
\pgfpathlineto{\pgfqpoint{4.557230in}{2.482677in}}%
\pgfpathlineto{\pgfqpoint{4.557230in}{2.103986in}}%
\pgfpathclose%
\pgfusepath{fill}%
\end{pgfscope}%
\begin{pgfscope}%
\pgfpathrectangle{\pgfqpoint{3.722897in}{0.857143in}}{\pgfqpoint{2.627103in}{1.813434in}}%
\pgfusepath{clip}%
\pgfsetbuttcap%
\pgfsetmiterjoin%
\definecolor{currentfill}{rgb}{0.950697,0.616649,0.428624}%
\pgfsetfillcolor{currentfill}%
\pgfsetlinewidth{0.000000pt}%
\definecolor{currentstroke}{rgb}{0.000000,0.000000,0.000000}%
\pgfsetstrokecolor{currentstroke}%
\pgfsetstrokeopacity{0.000000}%
\pgfsetdash{}{0pt}%
\pgfpathmoveto{\pgfqpoint{4.568400in}{2.101856in}}%
\pgfpathlineto{\pgfqpoint{4.577337in}{2.101856in}}%
\pgfpathlineto{\pgfqpoint{4.577337in}{2.497468in}}%
\pgfpathlineto{\pgfqpoint{4.568400in}{2.497468in}}%
\pgfpathlineto{\pgfqpoint{4.568400in}{2.101856in}}%
\pgfpathclose%
\pgfusepath{fill}%
\end{pgfscope}%
\begin{pgfscope}%
\pgfpathrectangle{\pgfqpoint{3.722897in}{0.857143in}}{\pgfqpoint{2.627103in}{1.813434in}}%
\pgfusepath{clip}%
\pgfsetbuttcap%
\pgfsetmiterjoin%
\definecolor{currentfill}{rgb}{0.950697,0.616649,0.428624}%
\pgfsetfillcolor{currentfill}%
\pgfsetlinewidth{0.000000pt}%
\definecolor{currentstroke}{rgb}{0.000000,0.000000,0.000000}%
\pgfsetstrokecolor{currentstroke}%
\pgfsetstrokeopacity{0.000000}%
\pgfsetdash{}{0pt}%
\pgfpathmoveto{\pgfqpoint{4.579571in}{2.086581in}}%
\pgfpathlineto{\pgfqpoint{4.588507in}{2.086581in}}%
\pgfpathlineto{\pgfqpoint{4.588507in}{2.473023in}}%
\pgfpathlineto{\pgfqpoint{4.579571in}{2.473023in}}%
\pgfpathlineto{\pgfqpoint{4.579571in}{2.086581in}}%
\pgfpathclose%
\pgfusepath{fill}%
\end{pgfscope}%
\begin{pgfscope}%
\pgfpathrectangle{\pgfqpoint{3.722897in}{0.857143in}}{\pgfqpoint{2.627103in}{1.813434in}}%
\pgfusepath{clip}%
\pgfsetbuttcap%
\pgfsetmiterjoin%
\definecolor{currentfill}{rgb}{0.950697,0.616649,0.428624}%
\pgfsetfillcolor{currentfill}%
\pgfsetlinewidth{0.000000pt}%
\definecolor{currentstroke}{rgb}{0.000000,0.000000,0.000000}%
\pgfsetstrokecolor{currentstroke}%
\pgfsetstrokeopacity{0.000000}%
\pgfsetdash{}{0pt}%
\pgfpathmoveto{\pgfqpoint{4.590741in}{2.078342in}}%
\pgfpathlineto{\pgfqpoint{4.599678in}{2.078342in}}%
\pgfpathlineto{\pgfqpoint{4.599678in}{2.469956in}}%
\pgfpathlineto{\pgfqpoint{4.590741in}{2.469956in}}%
\pgfpathlineto{\pgfqpoint{4.590741in}{2.078342in}}%
\pgfpathclose%
\pgfusepath{fill}%
\end{pgfscope}%
\begin{pgfscope}%
\pgfpathrectangle{\pgfqpoint{3.722897in}{0.857143in}}{\pgfqpoint{2.627103in}{1.813434in}}%
\pgfusepath{clip}%
\pgfsetbuttcap%
\pgfsetmiterjoin%
\definecolor{currentfill}{rgb}{0.950697,0.616649,0.428624}%
\pgfsetfillcolor{currentfill}%
\pgfsetlinewidth{0.000000pt}%
\definecolor{currentstroke}{rgb}{0.000000,0.000000,0.000000}%
\pgfsetstrokecolor{currentstroke}%
\pgfsetstrokeopacity{0.000000}%
\pgfsetdash{}{0pt}%
\pgfpathmoveto{\pgfqpoint{4.601912in}{2.075723in}}%
\pgfpathlineto{\pgfqpoint{4.610848in}{2.075723in}}%
\pgfpathlineto{\pgfqpoint{4.610848in}{2.440158in}}%
\pgfpathlineto{\pgfqpoint{4.601912in}{2.440158in}}%
\pgfpathlineto{\pgfqpoint{4.601912in}{2.075723in}}%
\pgfpathclose%
\pgfusepath{fill}%
\end{pgfscope}%
\begin{pgfscope}%
\pgfpathrectangle{\pgfqpoint{3.722897in}{0.857143in}}{\pgfqpoint{2.627103in}{1.813434in}}%
\pgfusepath{clip}%
\pgfsetbuttcap%
\pgfsetmiterjoin%
\definecolor{currentfill}{rgb}{0.950697,0.616649,0.428624}%
\pgfsetfillcolor{currentfill}%
\pgfsetlinewidth{0.000000pt}%
\definecolor{currentstroke}{rgb}{0.000000,0.000000,0.000000}%
\pgfsetstrokecolor{currentstroke}%
\pgfsetstrokeopacity{0.000000}%
\pgfsetdash{}{0pt}%
\pgfpathmoveto{\pgfqpoint{4.613083in}{2.065519in}}%
\pgfpathlineto{\pgfqpoint{4.622019in}{2.065519in}}%
\pgfpathlineto{\pgfqpoint{4.622019in}{2.394209in}}%
\pgfpathlineto{\pgfqpoint{4.613083in}{2.394209in}}%
\pgfpathlineto{\pgfqpoint{4.613083in}{2.065519in}}%
\pgfpathclose%
\pgfusepath{fill}%
\end{pgfscope}%
\begin{pgfscope}%
\pgfpathrectangle{\pgfqpoint{3.722897in}{0.857143in}}{\pgfqpoint{2.627103in}{1.813434in}}%
\pgfusepath{clip}%
\pgfsetbuttcap%
\pgfsetmiterjoin%
\definecolor{currentfill}{rgb}{0.950697,0.616649,0.428624}%
\pgfsetfillcolor{currentfill}%
\pgfsetlinewidth{0.000000pt}%
\definecolor{currentstroke}{rgb}{0.000000,0.000000,0.000000}%
\pgfsetstrokecolor{currentstroke}%
\pgfsetstrokeopacity{0.000000}%
\pgfsetdash{}{0pt}%
\pgfpathmoveto{\pgfqpoint{4.624253in}{2.075022in}}%
\pgfpathlineto{\pgfqpoint{4.633190in}{2.075022in}}%
\pgfpathlineto{\pgfqpoint{4.633190in}{2.408831in}}%
\pgfpathlineto{\pgfqpoint{4.624253in}{2.408831in}}%
\pgfpathlineto{\pgfqpoint{4.624253in}{2.075022in}}%
\pgfpathclose%
\pgfusepath{fill}%
\end{pgfscope}%
\begin{pgfscope}%
\pgfpathrectangle{\pgfqpoint{3.722897in}{0.857143in}}{\pgfqpoint{2.627103in}{1.813434in}}%
\pgfusepath{clip}%
\pgfsetbuttcap%
\pgfsetmiterjoin%
\definecolor{currentfill}{rgb}{0.950697,0.616649,0.428624}%
\pgfsetfillcolor{currentfill}%
\pgfsetlinewidth{0.000000pt}%
\definecolor{currentstroke}{rgb}{0.000000,0.000000,0.000000}%
\pgfsetstrokecolor{currentstroke}%
\pgfsetstrokeopacity{0.000000}%
\pgfsetdash{}{0pt}%
\pgfpathmoveto{\pgfqpoint{4.635424in}{2.084719in}}%
\pgfpathlineto{\pgfqpoint{4.644360in}{2.084719in}}%
\pgfpathlineto{\pgfqpoint{4.644360in}{2.401926in}}%
\pgfpathlineto{\pgfqpoint{4.635424in}{2.401926in}}%
\pgfpathlineto{\pgfqpoint{4.635424in}{2.084719in}}%
\pgfpathclose%
\pgfusepath{fill}%
\end{pgfscope}%
\begin{pgfscope}%
\pgfpathrectangle{\pgfqpoint{3.722897in}{0.857143in}}{\pgfqpoint{2.627103in}{1.813434in}}%
\pgfusepath{clip}%
\pgfsetbuttcap%
\pgfsetmiterjoin%
\definecolor{currentfill}{rgb}{0.950697,0.616649,0.428624}%
\pgfsetfillcolor{currentfill}%
\pgfsetlinewidth{0.000000pt}%
\definecolor{currentstroke}{rgb}{0.000000,0.000000,0.000000}%
\pgfsetstrokecolor{currentstroke}%
\pgfsetstrokeopacity{0.000000}%
\pgfsetdash{}{0pt}%
\pgfpathmoveto{\pgfqpoint{4.646594in}{2.115037in}}%
\pgfpathlineto{\pgfqpoint{4.655531in}{2.115037in}}%
\pgfpathlineto{\pgfqpoint{4.655531in}{2.425473in}}%
\pgfpathlineto{\pgfqpoint{4.646594in}{2.425473in}}%
\pgfpathlineto{\pgfqpoint{4.646594in}{2.115037in}}%
\pgfpathclose%
\pgfusepath{fill}%
\end{pgfscope}%
\begin{pgfscope}%
\pgfpathrectangle{\pgfqpoint{3.722897in}{0.857143in}}{\pgfqpoint{2.627103in}{1.813434in}}%
\pgfusepath{clip}%
\pgfsetbuttcap%
\pgfsetmiterjoin%
\definecolor{currentfill}{rgb}{0.950697,0.616649,0.428624}%
\pgfsetfillcolor{currentfill}%
\pgfsetlinewidth{0.000000pt}%
\definecolor{currentstroke}{rgb}{0.000000,0.000000,0.000000}%
\pgfsetstrokecolor{currentstroke}%
\pgfsetstrokeopacity{0.000000}%
\pgfsetdash{}{0pt}%
\pgfpathmoveto{\pgfqpoint{4.657765in}{2.152837in}}%
\pgfpathlineto{\pgfqpoint{4.666701in}{2.152837in}}%
\pgfpathlineto{\pgfqpoint{4.666701in}{2.464434in}}%
\pgfpathlineto{\pgfqpoint{4.657765in}{2.464434in}}%
\pgfpathlineto{\pgfqpoint{4.657765in}{2.152837in}}%
\pgfpathclose%
\pgfusepath{fill}%
\end{pgfscope}%
\begin{pgfscope}%
\pgfpathrectangle{\pgfqpoint{3.722897in}{0.857143in}}{\pgfqpoint{2.627103in}{1.813434in}}%
\pgfusepath{clip}%
\pgfsetbuttcap%
\pgfsetmiterjoin%
\definecolor{currentfill}{rgb}{0.950697,0.616649,0.428624}%
\pgfsetfillcolor{currentfill}%
\pgfsetlinewidth{0.000000pt}%
\definecolor{currentstroke}{rgb}{0.000000,0.000000,0.000000}%
\pgfsetstrokecolor{currentstroke}%
\pgfsetstrokeopacity{0.000000}%
\pgfsetdash{}{0pt}%
\pgfpathmoveto{\pgfqpoint{4.668936in}{2.181097in}}%
\pgfpathlineto{\pgfqpoint{4.677872in}{2.181097in}}%
\pgfpathlineto{\pgfqpoint{4.677872in}{2.479995in}}%
\pgfpathlineto{\pgfqpoint{4.668936in}{2.479995in}}%
\pgfpathlineto{\pgfqpoint{4.668936in}{2.181097in}}%
\pgfpathclose%
\pgfusepath{fill}%
\end{pgfscope}%
\begin{pgfscope}%
\pgfpathrectangle{\pgfqpoint{3.722897in}{0.857143in}}{\pgfqpoint{2.627103in}{1.813434in}}%
\pgfusepath{clip}%
\pgfsetbuttcap%
\pgfsetmiterjoin%
\definecolor{currentfill}{rgb}{0.950697,0.616649,0.428624}%
\pgfsetfillcolor{currentfill}%
\pgfsetlinewidth{0.000000pt}%
\definecolor{currentstroke}{rgb}{0.000000,0.000000,0.000000}%
\pgfsetstrokecolor{currentstroke}%
\pgfsetstrokeopacity{0.000000}%
\pgfsetdash{}{0pt}%
\pgfpathmoveto{\pgfqpoint{4.680106in}{2.186693in}}%
\pgfpathlineto{\pgfqpoint{4.689043in}{2.186693in}}%
\pgfpathlineto{\pgfqpoint{4.689043in}{2.477749in}}%
\pgfpathlineto{\pgfqpoint{4.680106in}{2.477749in}}%
\pgfpathlineto{\pgfqpoint{4.680106in}{2.186693in}}%
\pgfpathclose%
\pgfusepath{fill}%
\end{pgfscope}%
\begin{pgfscope}%
\pgfpathrectangle{\pgfqpoint{3.722897in}{0.857143in}}{\pgfqpoint{2.627103in}{1.813434in}}%
\pgfusepath{clip}%
\pgfsetbuttcap%
\pgfsetmiterjoin%
\definecolor{currentfill}{rgb}{0.950697,0.616649,0.428624}%
\pgfsetfillcolor{currentfill}%
\pgfsetlinewidth{0.000000pt}%
\definecolor{currentstroke}{rgb}{0.000000,0.000000,0.000000}%
\pgfsetstrokecolor{currentstroke}%
\pgfsetstrokeopacity{0.000000}%
\pgfsetdash{}{0pt}%
\pgfpathmoveto{\pgfqpoint{4.691277in}{2.210695in}}%
\pgfpathlineto{\pgfqpoint{4.700213in}{2.210695in}}%
\pgfpathlineto{\pgfqpoint{4.700213in}{2.487101in}}%
\pgfpathlineto{\pgfqpoint{4.691277in}{2.487101in}}%
\pgfpathlineto{\pgfqpoint{4.691277in}{2.210695in}}%
\pgfpathclose%
\pgfusepath{fill}%
\end{pgfscope}%
\begin{pgfscope}%
\pgfpathrectangle{\pgfqpoint{3.722897in}{0.857143in}}{\pgfqpoint{2.627103in}{1.813434in}}%
\pgfusepath{clip}%
\pgfsetbuttcap%
\pgfsetmiterjoin%
\definecolor{currentfill}{rgb}{0.950697,0.616649,0.428624}%
\pgfsetfillcolor{currentfill}%
\pgfsetlinewidth{0.000000pt}%
\definecolor{currentstroke}{rgb}{0.000000,0.000000,0.000000}%
\pgfsetstrokecolor{currentstroke}%
\pgfsetstrokeopacity{0.000000}%
\pgfsetdash{}{0pt}%
\pgfpathmoveto{\pgfqpoint{4.702447in}{2.233450in}}%
\pgfpathlineto{\pgfqpoint{4.711384in}{2.233450in}}%
\pgfpathlineto{\pgfqpoint{4.711384in}{2.518338in}}%
\pgfpathlineto{\pgfqpoint{4.702447in}{2.518338in}}%
\pgfpathlineto{\pgfqpoint{4.702447in}{2.233450in}}%
\pgfpathclose%
\pgfusepath{fill}%
\end{pgfscope}%
\begin{pgfscope}%
\pgfpathrectangle{\pgfqpoint{3.722897in}{0.857143in}}{\pgfqpoint{2.627103in}{1.813434in}}%
\pgfusepath{clip}%
\pgfsetbuttcap%
\pgfsetmiterjoin%
\definecolor{currentfill}{rgb}{0.950697,0.616649,0.428624}%
\pgfsetfillcolor{currentfill}%
\pgfsetlinewidth{0.000000pt}%
\definecolor{currentstroke}{rgb}{0.000000,0.000000,0.000000}%
\pgfsetstrokecolor{currentstroke}%
\pgfsetstrokeopacity{0.000000}%
\pgfsetdash{}{0pt}%
\pgfpathmoveto{\pgfqpoint{4.713618in}{2.252026in}}%
\pgfpathlineto{\pgfqpoint{4.722554in}{2.252026in}}%
\pgfpathlineto{\pgfqpoint{4.722554in}{2.490604in}}%
\pgfpathlineto{\pgfqpoint{4.713618in}{2.490604in}}%
\pgfpathlineto{\pgfqpoint{4.713618in}{2.252026in}}%
\pgfpathclose%
\pgfusepath{fill}%
\end{pgfscope}%
\begin{pgfscope}%
\pgfpathrectangle{\pgfqpoint{3.722897in}{0.857143in}}{\pgfqpoint{2.627103in}{1.813434in}}%
\pgfusepath{clip}%
\pgfsetbuttcap%
\pgfsetmiterjoin%
\definecolor{currentfill}{rgb}{0.950697,0.616649,0.428624}%
\pgfsetfillcolor{currentfill}%
\pgfsetlinewidth{0.000000pt}%
\definecolor{currentstroke}{rgb}{0.000000,0.000000,0.000000}%
\pgfsetstrokecolor{currentstroke}%
\pgfsetstrokeopacity{0.000000}%
\pgfsetdash{}{0pt}%
\pgfpathmoveto{\pgfqpoint{4.724789in}{2.267328in}}%
\pgfpathlineto{\pgfqpoint{4.733725in}{2.267328in}}%
\pgfpathlineto{\pgfqpoint{4.733725in}{2.506067in}}%
\pgfpathlineto{\pgfqpoint{4.724789in}{2.506067in}}%
\pgfpathlineto{\pgfqpoint{4.724789in}{2.267328in}}%
\pgfpathclose%
\pgfusepath{fill}%
\end{pgfscope}%
\begin{pgfscope}%
\pgfpathrectangle{\pgfqpoint{3.722897in}{0.857143in}}{\pgfqpoint{2.627103in}{1.813434in}}%
\pgfusepath{clip}%
\pgfsetbuttcap%
\pgfsetmiterjoin%
\definecolor{currentfill}{rgb}{0.950697,0.616649,0.428624}%
\pgfsetfillcolor{currentfill}%
\pgfsetlinewidth{0.000000pt}%
\definecolor{currentstroke}{rgb}{0.000000,0.000000,0.000000}%
\pgfsetstrokecolor{currentstroke}%
\pgfsetstrokeopacity{0.000000}%
\pgfsetdash{}{0pt}%
\pgfpathmoveto{\pgfqpoint{4.735959in}{2.283348in}}%
\pgfpathlineto{\pgfqpoint{4.744896in}{2.283348in}}%
\pgfpathlineto{\pgfqpoint{4.744896in}{2.511827in}}%
\pgfpathlineto{\pgfqpoint{4.735959in}{2.511827in}}%
\pgfpathlineto{\pgfqpoint{4.735959in}{2.283348in}}%
\pgfpathclose%
\pgfusepath{fill}%
\end{pgfscope}%
\begin{pgfscope}%
\pgfpathrectangle{\pgfqpoint{3.722897in}{0.857143in}}{\pgfqpoint{2.627103in}{1.813434in}}%
\pgfusepath{clip}%
\pgfsetbuttcap%
\pgfsetmiterjoin%
\definecolor{currentfill}{rgb}{0.950697,0.616649,0.428624}%
\pgfsetfillcolor{currentfill}%
\pgfsetlinewidth{0.000000pt}%
\definecolor{currentstroke}{rgb}{0.000000,0.000000,0.000000}%
\pgfsetstrokecolor{currentstroke}%
\pgfsetstrokeopacity{0.000000}%
\pgfsetdash{}{0pt}%
\pgfpathmoveto{\pgfqpoint{4.747130in}{2.291178in}}%
\pgfpathlineto{\pgfqpoint{4.756066in}{2.291178in}}%
\pgfpathlineto{\pgfqpoint{4.756066in}{2.510263in}}%
\pgfpathlineto{\pgfqpoint{4.747130in}{2.510263in}}%
\pgfpathlineto{\pgfqpoint{4.747130in}{2.291178in}}%
\pgfpathclose%
\pgfusepath{fill}%
\end{pgfscope}%
\begin{pgfscope}%
\pgfpathrectangle{\pgfqpoint{3.722897in}{0.857143in}}{\pgfqpoint{2.627103in}{1.813434in}}%
\pgfusepath{clip}%
\pgfsetbuttcap%
\pgfsetmiterjoin%
\definecolor{currentfill}{rgb}{0.950697,0.616649,0.428624}%
\pgfsetfillcolor{currentfill}%
\pgfsetlinewidth{0.000000pt}%
\definecolor{currentstroke}{rgb}{0.000000,0.000000,0.000000}%
\pgfsetstrokecolor{currentstroke}%
\pgfsetstrokeopacity{0.000000}%
\pgfsetdash{}{0pt}%
\pgfpathmoveto{\pgfqpoint{4.758300in}{2.291519in}}%
\pgfpathlineto{\pgfqpoint{4.767237in}{2.291519in}}%
\pgfpathlineto{\pgfqpoint{4.767237in}{2.477714in}}%
\pgfpathlineto{\pgfqpoint{4.758300in}{2.477714in}}%
\pgfpathlineto{\pgfqpoint{4.758300in}{2.291519in}}%
\pgfpathclose%
\pgfusepath{fill}%
\end{pgfscope}%
\begin{pgfscope}%
\pgfpathrectangle{\pgfqpoint{3.722897in}{0.857143in}}{\pgfqpoint{2.627103in}{1.813434in}}%
\pgfusepath{clip}%
\pgfsetbuttcap%
\pgfsetmiterjoin%
\definecolor{currentfill}{rgb}{0.950697,0.616649,0.428624}%
\pgfsetfillcolor{currentfill}%
\pgfsetlinewidth{0.000000pt}%
\definecolor{currentstroke}{rgb}{0.000000,0.000000,0.000000}%
\pgfsetstrokecolor{currentstroke}%
\pgfsetstrokeopacity{0.000000}%
\pgfsetdash{}{0pt}%
\pgfpathmoveto{\pgfqpoint{4.769471in}{2.297081in}}%
\pgfpathlineto{\pgfqpoint{4.778408in}{2.297081in}}%
\pgfpathlineto{\pgfqpoint{4.778408in}{2.471945in}}%
\pgfpathlineto{\pgfqpoint{4.769471in}{2.471945in}}%
\pgfpathlineto{\pgfqpoint{4.769471in}{2.297081in}}%
\pgfpathclose%
\pgfusepath{fill}%
\end{pgfscope}%
\begin{pgfscope}%
\pgfpathrectangle{\pgfqpoint{3.722897in}{0.857143in}}{\pgfqpoint{2.627103in}{1.813434in}}%
\pgfusepath{clip}%
\pgfsetbuttcap%
\pgfsetmiterjoin%
\definecolor{currentfill}{rgb}{0.950697,0.616649,0.428624}%
\pgfsetfillcolor{currentfill}%
\pgfsetlinewidth{0.000000pt}%
\definecolor{currentstroke}{rgb}{0.000000,0.000000,0.000000}%
\pgfsetstrokecolor{currentstroke}%
\pgfsetstrokeopacity{0.000000}%
\pgfsetdash{}{0pt}%
\pgfpathmoveto{\pgfqpoint{4.780642in}{2.290251in}}%
\pgfpathlineto{\pgfqpoint{4.789578in}{2.290251in}}%
\pgfpathlineto{\pgfqpoint{4.789578in}{2.446971in}}%
\pgfpathlineto{\pgfqpoint{4.780642in}{2.446971in}}%
\pgfpathlineto{\pgfqpoint{4.780642in}{2.290251in}}%
\pgfpathclose%
\pgfusepath{fill}%
\end{pgfscope}%
\begin{pgfscope}%
\pgfpathrectangle{\pgfqpoint{3.722897in}{0.857143in}}{\pgfqpoint{2.627103in}{1.813434in}}%
\pgfusepath{clip}%
\pgfsetbuttcap%
\pgfsetmiterjoin%
\definecolor{currentfill}{rgb}{0.950697,0.616649,0.428624}%
\pgfsetfillcolor{currentfill}%
\pgfsetlinewidth{0.000000pt}%
\definecolor{currentstroke}{rgb}{0.000000,0.000000,0.000000}%
\pgfsetstrokecolor{currentstroke}%
\pgfsetstrokeopacity{0.000000}%
\pgfsetdash{}{0pt}%
\pgfpathmoveto{\pgfqpoint{4.791812in}{2.288209in}}%
\pgfpathlineto{\pgfqpoint{4.800749in}{2.288209in}}%
\pgfpathlineto{\pgfqpoint{4.800749in}{2.426466in}}%
\pgfpathlineto{\pgfqpoint{4.791812in}{2.426466in}}%
\pgfpathlineto{\pgfqpoint{4.791812in}{2.288209in}}%
\pgfpathclose%
\pgfusepath{fill}%
\end{pgfscope}%
\begin{pgfscope}%
\pgfpathrectangle{\pgfqpoint{3.722897in}{0.857143in}}{\pgfqpoint{2.627103in}{1.813434in}}%
\pgfusepath{clip}%
\pgfsetbuttcap%
\pgfsetmiterjoin%
\definecolor{currentfill}{rgb}{0.950697,0.616649,0.428624}%
\pgfsetfillcolor{currentfill}%
\pgfsetlinewidth{0.000000pt}%
\definecolor{currentstroke}{rgb}{0.000000,0.000000,0.000000}%
\pgfsetstrokecolor{currentstroke}%
\pgfsetstrokeopacity{0.000000}%
\pgfsetdash{}{0pt}%
\pgfpathmoveto{\pgfqpoint{4.802983in}{2.294835in}}%
\pgfpathlineto{\pgfqpoint{4.811919in}{2.294835in}}%
\pgfpathlineto{\pgfqpoint{4.811919in}{2.411867in}}%
\pgfpathlineto{\pgfqpoint{4.802983in}{2.411867in}}%
\pgfpathlineto{\pgfqpoint{4.802983in}{2.294835in}}%
\pgfpathclose%
\pgfusepath{fill}%
\end{pgfscope}%
\begin{pgfscope}%
\pgfpathrectangle{\pgfqpoint{3.722897in}{0.857143in}}{\pgfqpoint{2.627103in}{1.813434in}}%
\pgfusepath{clip}%
\pgfsetbuttcap%
\pgfsetmiterjoin%
\definecolor{currentfill}{rgb}{0.950697,0.616649,0.428624}%
\pgfsetfillcolor{currentfill}%
\pgfsetlinewidth{0.000000pt}%
\definecolor{currentstroke}{rgb}{0.000000,0.000000,0.000000}%
\pgfsetstrokecolor{currentstroke}%
\pgfsetstrokeopacity{0.000000}%
\pgfsetdash{}{0pt}%
\pgfpathmoveto{\pgfqpoint{4.814153in}{2.297030in}}%
\pgfpathlineto{\pgfqpoint{4.823090in}{2.297030in}}%
\pgfpathlineto{\pgfqpoint{4.823090in}{2.406217in}}%
\pgfpathlineto{\pgfqpoint{4.814153in}{2.406217in}}%
\pgfpathlineto{\pgfqpoint{4.814153in}{2.297030in}}%
\pgfpathclose%
\pgfusepath{fill}%
\end{pgfscope}%
\begin{pgfscope}%
\pgfpathrectangle{\pgfqpoint{3.722897in}{0.857143in}}{\pgfqpoint{2.627103in}{1.813434in}}%
\pgfusepath{clip}%
\pgfsetbuttcap%
\pgfsetmiterjoin%
\definecolor{currentfill}{rgb}{0.950697,0.616649,0.428624}%
\pgfsetfillcolor{currentfill}%
\pgfsetlinewidth{0.000000pt}%
\definecolor{currentstroke}{rgb}{0.000000,0.000000,0.000000}%
\pgfsetstrokecolor{currentstroke}%
\pgfsetstrokeopacity{0.000000}%
\pgfsetdash{}{0pt}%
\pgfpathmoveto{\pgfqpoint{4.825324in}{2.298922in}}%
\pgfpathlineto{\pgfqpoint{4.834261in}{2.298922in}}%
\pgfpathlineto{\pgfqpoint{4.834261in}{2.392352in}}%
\pgfpathlineto{\pgfqpoint{4.825324in}{2.392352in}}%
\pgfpathlineto{\pgfqpoint{4.825324in}{2.298922in}}%
\pgfpathclose%
\pgfusepath{fill}%
\end{pgfscope}%
\begin{pgfscope}%
\pgfpathrectangle{\pgfqpoint{3.722897in}{0.857143in}}{\pgfqpoint{2.627103in}{1.813434in}}%
\pgfusepath{clip}%
\pgfsetbuttcap%
\pgfsetmiterjoin%
\definecolor{currentfill}{rgb}{0.950697,0.616649,0.428624}%
\pgfsetfillcolor{currentfill}%
\pgfsetlinewidth{0.000000pt}%
\definecolor{currentstroke}{rgb}{0.000000,0.000000,0.000000}%
\pgfsetstrokecolor{currentstroke}%
\pgfsetstrokeopacity{0.000000}%
\pgfsetdash{}{0pt}%
\pgfpathmoveto{\pgfqpoint{4.836495in}{2.299475in}}%
\pgfpathlineto{\pgfqpoint{4.845431in}{2.299475in}}%
\pgfpathlineto{\pgfqpoint{4.845431in}{2.367803in}}%
\pgfpathlineto{\pgfqpoint{4.836495in}{2.367803in}}%
\pgfpathlineto{\pgfqpoint{4.836495in}{2.299475in}}%
\pgfpathclose%
\pgfusepath{fill}%
\end{pgfscope}%
\begin{pgfscope}%
\pgfpathrectangle{\pgfqpoint{3.722897in}{0.857143in}}{\pgfqpoint{2.627103in}{1.813434in}}%
\pgfusepath{clip}%
\pgfsetbuttcap%
\pgfsetmiterjoin%
\definecolor{currentfill}{rgb}{0.950697,0.616649,0.428624}%
\pgfsetfillcolor{currentfill}%
\pgfsetlinewidth{0.000000pt}%
\definecolor{currentstroke}{rgb}{0.000000,0.000000,0.000000}%
\pgfsetstrokecolor{currentstroke}%
\pgfsetstrokeopacity{0.000000}%
\pgfsetdash{}{0pt}%
\pgfpathmoveto{\pgfqpoint{4.847665in}{2.310904in}}%
\pgfpathlineto{\pgfqpoint{4.856602in}{2.310904in}}%
\pgfpathlineto{\pgfqpoint{4.856602in}{2.362892in}}%
\pgfpathlineto{\pgfqpoint{4.847665in}{2.362892in}}%
\pgfpathlineto{\pgfqpoint{4.847665in}{2.310904in}}%
\pgfpathclose%
\pgfusepath{fill}%
\end{pgfscope}%
\begin{pgfscope}%
\pgfpathrectangle{\pgfqpoint{3.722897in}{0.857143in}}{\pgfqpoint{2.627103in}{1.813434in}}%
\pgfusepath{clip}%
\pgfsetbuttcap%
\pgfsetmiterjoin%
\definecolor{currentfill}{rgb}{0.950697,0.616649,0.428624}%
\pgfsetfillcolor{currentfill}%
\pgfsetlinewidth{0.000000pt}%
\definecolor{currentstroke}{rgb}{0.000000,0.000000,0.000000}%
\pgfsetstrokecolor{currentstroke}%
\pgfsetstrokeopacity{0.000000}%
\pgfsetdash{}{0pt}%
\pgfpathmoveto{\pgfqpoint{4.858836in}{2.330310in}}%
\pgfpathlineto{\pgfqpoint{4.867772in}{2.330310in}}%
\pgfpathlineto{\pgfqpoint{4.867772in}{2.368063in}}%
\pgfpathlineto{\pgfqpoint{4.858836in}{2.368063in}}%
\pgfpathlineto{\pgfqpoint{4.858836in}{2.330310in}}%
\pgfpathclose%
\pgfusepath{fill}%
\end{pgfscope}%
\begin{pgfscope}%
\pgfpathrectangle{\pgfqpoint{3.722897in}{0.857143in}}{\pgfqpoint{2.627103in}{1.813434in}}%
\pgfusepath{clip}%
\pgfsetbuttcap%
\pgfsetmiterjoin%
\definecolor{currentfill}{rgb}{0.950697,0.616649,0.428624}%
\pgfsetfillcolor{currentfill}%
\pgfsetlinewidth{0.000000pt}%
\definecolor{currentstroke}{rgb}{0.000000,0.000000,0.000000}%
\pgfsetstrokecolor{currentstroke}%
\pgfsetstrokeopacity{0.000000}%
\pgfsetdash{}{0pt}%
\pgfpathmoveto{\pgfqpoint{4.870006in}{2.362365in}}%
\pgfpathlineto{\pgfqpoint{4.878943in}{2.362365in}}%
\pgfpathlineto{\pgfqpoint{4.878943in}{2.386960in}}%
\pgfpathlineto{\pgfqpoint{4.870006in}{2.386960in}}%
\pgfpathlineto{\pgfqpoint{4.870006in}{2.362365in}}%
\pgfpathclose%
\pgfusepath{fill}%
\end{pgfscope}%
\begin{pgfscope}%
\pgfpathrectangle{\pgfqpoint{3.722897in}{0.857143in}}{\pgfqpoint{2.627103in}{1.813434in}}%
\pgfusepath{clip}%
\pgfsetbuttcap%
\pgfsetmiterjoin%
\definecolor{currentfill}{rgb}{0.950697,0.616649,0.428624}%
\pgfsetfillcolor{currentfill}%
\pgfsetlinewidth{0.000000pt}%
\definecolor{currentstroke}{rgb}{0.000000,0.000000,0.000000}%
\pgfsetstrokecolor{currentstroke}%
\pgfsetstrokeopacity{0.000000}%
\pgfsetdash{}{0pt}%
\pgfpathmoveto{\pgfqpoint{4.881177in}{2.391850in}}%
\pgfpathlineto{\pgfqpoint{4.890114in}{2.391850in}}%
\pgfpathlineto{\pgfqpoint{4.890114in}{2.420962in}}%
\pgfpathlineto{\pgfqpoint{4.881177in}{2.420962in}}%
\pgfpathlineto{\pgfqpoint{4.881177in}{2.391850in}}%
\pgfpathclose%
\pgfusepath{fill}%
\end{pgfscope}%
\begin{pgfscope}%
\pgfpathrectangle{\pgfqpoint{3.722897in}{0.857143in}}{\pgfqpoint{2.627103in}{1.813434in}}%
\pgfusepath{clip}%
\pgfsetbuttcap%
\pgfsetmiterjoin%
\definecolor{currentfill}{rgb}{0.950697,0.616649,0.428624}%
\pgfsetfillcolor{currentfill}%
\pgfsetlinewidth{0.000000pt}%
\definecolor{currentstroke}{rgb}{0.000000,0.000000,0.000000}%
\pgfsetstrokecolor{currentstroke}%
\pgfsetstrokeopacity{0.000000}%
\pgfsetdash{}{0pt}%
\pgfpathmoveto{\pgfqpoint{4.892348in}{2.399790in}}%
\pgfpathlineto{\pgfqpoint{4.901284in}{2.399790in}}%
\pgfpathlineto{\pgfqpoint{4.901284in}{2.430522in}}%
\pgfpathlineto{\pgfqpoint{4.892348in}{2.430522in}}%
\pgfpathlineto{\pgfqpoint{4.892348in}{2.399790in}}%
\pgfpathclose%
\pgfusepath{fill}%
\end{pgfscope}%
\begin{pgfscope}%
\pgfpathrectangle{\pgfqpoint{3.722897in}{0.857143in}}{\pgfqpoint{2.627103in}{1.813434in}}%
\pgfusepath{clip}%
\pgfsetbuttcap%
\pgfsetmiterjoin%
\definecolor{currentfill}{rgb}{0.950697,0.616649,0.428624}%
\pgfsetfillcolor{currentfill}%
\pgfsetlinewidth{0.000000pt}%
\definecolor{currentstroke}{rgb}{0.000000,0.000000,0.000000}%
\pgfsetstrokecolor{currentstroke}%
\pgfsetstrokeopacity{0.000000}%
\pgfsetdash{}{0pt}%
\pgfpathmoveto{\pgfqpoint{4.903518in}{2.402805in}}%
\pgfpathlineto{\pgfqpoint{4.912455in}{2.402805in}}%
\pgfpathlineto{\pgfqpoint{4.912455in}{2.426602in}}%
\pgfpathlineto{\pgfqpoint{4.903518in}{2.426602in}}%
\pgfpathlineto{\pgfqpoint{4.903518in}{2.402805in}}%
\pgfpathclose%
\pgfusepath{fill}%
\end{pgfscope}%
\begin{pgfscope}%
\pgfpathrectangle{\pgfqpoint{3.722897in}{0.857143in}}{\pgfqpoint{2.627103in}{1.813434in}}%
\pgfusepath{clip}%
\pgfsetbuttcap%
\pgfsetmiterjoin%
\definecolor{currentfill}{rgb}{0.950697,0.616649,0.428624}%
\pgfsetfillcolor{currentfill}%
\pgfsetlinewidth{0.000000pt}%
\definecolor{currentstroke}{rgb}{0.000000,0.000000,0.000000}%
\pgfsetstrokecolor{currentstroke}%
\pgfsetstrokeopacity{0.000000}%
\pgfsetdash{}{0pt}%
\pgfpathmoveto{\pgfqpoint{4.914689in}{2.421283in}}%
\pgfpathlineto{\pgfqpoint{4.923625in}{2.421283in}}%
\pgfpathlineto{\pgfqpoint{4.923625in}{2.428222in}}%
\pgfpathlineto{\pgfqpoint{4.914689in}{2.428222in}}%
\pgfpathlineto{\pgfqpoint{4.914689in}{2.421283in}}%
\pgfpathclose%
\pgfusepath{fill}%
\end{pgfscope}%
\begin{pgfscope}%
\pgfpathrectangle{\pgfqpoint{3.722897in}{0.857143in}}{\pgfqpoint{2.627103in}{1.813434in}}%
\pgfusepath{clip}%
\pgfsetbuttcap%
\pgfsetmiterjoin%
\definecolor{currentfill}{rgb}{0.950697,0.616649,0.428624}%
\pgfsetfillcolor{currentfill}%
\pgfsetlinewidth{0.000000pt}%
\definecolor{currentstroke}{rgb}{0.000000,0.000000,0.000000}%
\pgfsetstrokecolor{currentstroke}%
\pgfsetstrokeopacity{0.000000}%
\pgfsetdash{}{0pt}%
\pgfpathmoveto{\pgfqpoint{4.925860in}{2.439104in}}%
\pgfpathlineto{\pgfqpoint{4.934796in}{2.439104in}}%
\pgfpathlineto{\pgfqpoint{4.934796in}{2.442759in}}%
\pgfpathlineto{\pgfqpoint{4.925860in}{2.442759in}}%
\pgfpathlineto{\pgfqpoint{4.925860in}{2.439104in}}%
\pgfpathclose%
\pgfusepath{fill}%
\end{pgfscope}%
\begin{pgfscope}%
\pgfpathrectangle{\pgfqpoint{3.722897in}{0.857143in}}{\pgfqpoint{2.627103in}{1.813434in}}%
\pgfusepath{clip}%
\pgfsetbuttcap%
\pgfsetmiterjoin%
\definecolor{currentfill}{rgb}{0.950697,0.616649,0.428624}%
\pgfsetfillcolor{currentfill}%
\pgfsetlinewidth{0.000000pt}%
\definecolor{currentstroke}{rgb}{0.000000,0.000000,0.000000}%
\pgfsetstrokecolor{currentstroke}%
\pgfsetstrokeopacity{0.000000}%
\pgfsetdash{}{0pt}%
\pgfpathmoveto{\pgfqpoint{4.937030in}{1.740853in}}%
\pgfpathlineto{\pgfqpoint{4.945967in}{1.740853in}}%
\pgfpathlineto{\pgfqpoint{4.945967in}{1.735753in}}%
\pgfpathlineto{\pgfqpoint{4.937030in}{1.735753in}}%
\pgfpathlineto{\pgfqpoint{4.937030in}{1.740853in}}%
\pgfpathclose%
\pgfusepath{fill}%
\end{pgfscope}%
\begin{pgfscope}%
\pgfpathrectangle{\pgfqpoint{3.722897in}{0.857143in}}{\pgfqpoint{2.627103in}{1.813434in}}%
\pgfusepath{clip}%
\pgfsetbuttcap%
\pgfsetmiterjoin%
\definecolor{currentfill}{rgb}{0.950697,0.616649,0.428624}%
\pgfsetfillcolor{currentfill}%
\pgfsetlinewidth{0.000000pt}%
\definecolor{currentstroke}{rgb}{0.000000,0.000000,0.000000}%
\pgfsetstrokecolor{currentstroke}%
\pgfsetstrokeopacity{0.000000}%
\pgfsetdash{}{0pt}%
\pgfpathmoveto{\pgfqpoint{4.948201in}{1.735713in}}%
\pgfpathlineto{\pgfqpoint{4.957137in}{1.735713in}}%
\pgfpathlineto{\pgfqpoint{4.957137in}{1.712692in}}%
\pgfpathlineto{\pgfqpoint{4.948201in}{1.712692in}}%
\pgfpathlineto{\pgfqpoint{4.948201in}{1.735713in}}%
\pgfpathclose%
\pgfusepath{fill}%
\end{pgfscope}%
\begin{pgfscope}%
\pgfpathrectangle{\pgfqpoint{3.722897in}{0.857143in}}{\pgfqpoint{2.627103in}{1.813434in}}%
\pgfusepath{clip}%
\pgfsetbuttcap%
\pgfsetmiterjoin%
\definecolor{currentfill}{rgb}{0.950697,0.616649,0.428624}%
\pgfsetfillcolor{currentfill}%
\pgfsetlinewidth{0.000000pt}%
\definecolor{currentstroke}{rgb}{0.000000,0.000000,0.000000}%
\pgfsetstrokecolor{currentstroke}%
\pgfsetstrokeopacity{0.000000}%
\pgfsetdash{}{0pt}%
\pgfpathmoveto{\pgfqpoint{4.959371in}{1.727951in}}%
\pgfpathlineto{\pgfqpoint{4.968308in}{1.727951in}}%
\pgfpathlineto{\pgfqpoint{4.968308in}{1.703087in}}%
\pgfpathlineto{\pgfqpoint{4.959371in}{1.703087in}}%
\pgfpathlineto{\pgfqpoint{4.959371in}{1.727951in}}%
\pgfpathclose%
\pgfusepath{fill}%
\end{pgfscope}%
\begin{pgfscope}%
\pgfpathrectangle{\pgfqpoint{3.722897in}{0.857143in}}{\pgfqpoint{2.627103in}{1.813434in}}%
\pgfusepath{clip}%
\pgfsetbuttcap%
\pgfsetmiterjoin%
\definecolor{currentfill}{rgb}{0.950697,0.616649,0.428624}%
\pgfsetfillcolor{currentfill}%
\pgfsetlinewidth{0.000000pt}%
\definecolor{currentstroke}{rgb}{0.000000,0.000000,0.000000}%
\pgfsetstrokecolor{currentstroke}%
\pgfsetstrokeopacity{0.000000}%
\pgfsetdash{}{0pt}%
\pgfpathmoveto{\pgfqpoint{4.970542in}{1.723739in}}%
\pgfpathlineto{\pgfqpoint{4.979478in}{1.723739in}}%
\pgfpathlineto{\pgfqpoint{4.979478in}{1.680096in}}%
\pgfpathlineto{\pgfqpoint{4.970542in}{1.680096in}}%
\pgfpathlineto{\pgfqpoint{4.970542in}{1.723739in}}%
\pgfpathclose%
\pgfusepath{fill}%
\end{pgfscope}%
\begin{pgfscope}%
\pgfpathrectangle{\pgfqpoint{3.722897in}{0.857143in}}{\pgfqpoint{2.627103in}{1.813434in}}%
\pgfusepath{clip}%
\pgfsetbuttcap%
\pgfsetmiterjoin%
\definecolor{currentfill}{rgb}{0.950697,0.616649,0.428624}%
\pgfsetfillcolor{currentfill}%
\pgfsetlinewidth{0.000000pt}%
\definecolor{currentstroke}{rgb}{0.000000,0.000000,0.000000}%
\pgfsetstrokecolor{currentstroke}%
\pgfsetstrokeopacity{0.000000}%
\pgfsetdash{}{0pt}%
\pgfpathmoveto{\pgfqpoint{4.981713in}{1.723310in}}%
\pgfpathlineto{\pgfqpoint{4.990649in}{1.723310in}}%
\pgfpathlineto{\pgfqpoint{4.990649in}{1.649173in}}%
\pgfpathlineto{\pgfqpoint{4.981713in}{1.649173in}}%
\pgfpathlineto{\pgfqpoint{4.981713in}{1.723310in}}%
\pgfpathclose%
\pgfusepath{fill}%
\end{pgfscope}%
\begin{pgfscope}%
\pgfpathrectangle{\pgfqpoint{3.722897in}{0.857143in}}{\pgfqpoint{2.627103in}{1.813434in}}%
\pgfusepath{clip}%
\pgfsetbuttcap%
\pgfsetmiterjoin%
\definecolor{currentfill}{rgb}{0.950697,0.616649,0.428624}%
\pgfsetfillcolor{currentfill}%
\pgfsetlinewidth{0.000000pt}%
\definecolor{currentstroke}{rgb}{0.000000,0.000000,0.000000}%
\pgfsetstrokecolor{currentstroke}%
\pgfsetstrokeopacity{0.000000}%
\pgfsetdash{}{0pt}%
\pgfpathmoveto{\pgfqpoint{4.992883in}{1.715859in}}%
\pgfpathlineto{\pgfqpoint{5.001820in}{1.715859in}}%
\pgfpathlineto{\pgfqpoint{5.001820in}{1.631510in}}%
\pgfpathlineto{\pgfqpoint{4.992883in}{1.631510in}}%
\pgfpathlineto{\pgfqpoint{4.992883in}{1.715859in}}%
\pgfpathclose%
\pgfusepath{fill}%
\end{pgfscope}%
\begin{pgfscope}%
\pgfpathrectangle{\pgfqpoint{3.722897in}{0.857143in}}{\pgfqpoint{2.627103in}{1.813434in}}%
\pgfusepath{clip}%
\pgfsetbuttcap%
\pgfsetmiterjoin%
\definecolor{currentfill}{rgb}{0.950697,0.616649,0.428624}%
\pgfsetfillcolor{currentfill}%
\pgfsetlinewidth{0.000000pt}%
\definecolor{currentstroke}{rgb}{0.000000,0.000000,0.000000}%
\pgfsetstrokecolor{currentstroke}%
\pgfsetstrokeopacity{0.000000}%
\pgfsetdash{}{0pt}%
\pgfpathmoveto{\pgfqpoint{5.004054in}{1.713305in}}%
\pgfpathlineto{\pgfqpoint{5.012990in}{1.713305in}}%
\pgfpathlineto{\pgfqpoint{5.012990in}{1.601382in}}%
\pgfpathlineto{\pgfqpoint{5.004054in}{1.601382in}}%
\pgfpathlineto{\pgfqpoint{5.004054in}{1.713305in}}%
\pgfpathclose%
\pgfusepath{fill}%
\end{pgfscope}%
\begin{pgfscope}%
\pgfpathrectangle{\pgfqpoint{3.722897in}{0.857143in}}{\pgfqpoint{2.627103in}{1.813434in}}%
\pgfusepath{clip}%
\pgfsetbuttcap%
\pgfsetmiterjoin%
\definecolor{currentfill}{rgb}{0.950697,0.616649,0.428624}%
\pgfsetfillcolor{currentfill}%
\pgfsetlinewidth{0.000000pt}%
\definecolor{currentstroke}{rgb}{0.000000,0.000000,0.000000}%
\pgfsetstrokecolor{currentstroke}%
\pgfsetstrokeopacity{0.000000}%
\pgfsetdash{}{0pt}%
\pgfpathmoveto{\pgfqpoint{5.015224in}{1.708146in}}%
\pgfpathlineto{\pgfqpoint{5.024161in}{1.708146in}}%
\pgfpathlineto{\pgfqpoint{5.024161in}{1.580469in}}%
\pgfpathlineto{\pgfqpoint{5.015224in}{1.580469in}}%
\pgfpathlineto{\pgfqpoint{5.015224in}{1.708146in}}%
\pgfpathclose%
\pgfusepath{fill}%
\end{pgfscope}%
\begin{pgfscope}%
\pgfpathrectangle{\pgfqpoint{3.722897in}{0.857143in}}{\pgfqpoint{2.627103in}{1.813434in}}%
\pgfusepath{clip}%
\pgfsetbuttcap%
\pgfsetmiterjoin%
\definecolor{currentfill}{rgb}{0.950697,0.616649,0.428624}%
\pgfsetfillcolor{currentfill}%
\pgfsetlinewidth{0.000000pt}%
\definecolor{currentstroke}{rgb}{0.000000,0.000000,0.000000}%
\pgfsetstrokecolor{currentstroke}%
\pgfsetstrokeopacity{0.000000}%
\pgfsetdash{}{0pt}%
\pgfpathmoveto{\pgfqpoint{5.026395in}{1.697455in}}%
\pgfpathlineto{\pgfqpoint{5.035331in}{1.697455in}}%
\pgfpathlineto{\pgfqpoint{5.035331in}{1.531917in}}%
\pgfpathlineto{\pgfqpoint{5.026395in}{1.531917in}}%
\pgfpathlineto{\pgfqpoint{5.026395in}{1.697455in}}%
\pgfpathclose%
\pgfusepath{fill}%
\end{pgfscope}%
\begin{pgfscope}%
\pgfpathrectangle{\pgfqpoint{3.722897in}{0.857143in}}{\pgfqpoint{2.627103in}{1.813434in}}%
\pgfusepath{clip}%
\pgfsetbuttcap%
\pgfsetmiterjoin%
\definecolor{currentfill}{rgb}{0.950697,0.616649,0.428624}%
\pgfsetfillcolor{currentfill}%
\pgfsetlinewidth{0.000000pt}%
\definecolor{currentstroke}{rgb}{0.000000,0.000000,0.000000}%
\pgfsetstrokecolor{currentstroke}%
\pgfsetstrokeopacity{0.000000}%
\pgfsetdash{}{0pt}%
\pgfpathmoveto{\pgfqpoint{5.037566in}{1.683276in}}%
\pgfpathlineto{\pgfqpoint{5.046502in}{1.683276in}}%
\pgfpathlineto{\pgfqpoint{5.046502in}{1.505264in}}%
\pgfpathlineto{\pgfqpoint{5.037566in}{1.505264in}}%
\pgfpathlineto{\pgfqpoint{5.037566in}{1.683276in}}%
\pgfpathclose%
\pgfusepath{fill}%
\end{pgfscope}%
\begin{pgfscope}%
\pgfpathrectangle{\pgfqpoint{3.722897in}{0.857143in}}{\pgfqpoint{2.627103in}{1.813434in}}%
\pgfusepath{clip}%
\pgfsetbuttcap%
\pgfsetmiterjoin%
\definecolor{currentfill}{rgb}{0.950697,0.616649,0.428624}%
\pgfsetfillcolor{currentfill}%
\pgfsetlinewidth{0.000000pt}%
\definecolor{currentstroke}{rgb}{0.000000,0.000000,0.000000}%
\pgfsetstrokecolor{currentstroke}%
\pgfsetstrokeopacity{0.000000}%
\pgfsetdash{}{0pt}%
\pgfpathmoveto{\pgfqpoint{5.048736in}{1.667364in}}%
\pgfpathlineto{\pgfqpoint{5.057673in}{1.667364in}}%
\pgfpathlineto{\pgfqpoint{5.057673in}{1.456315in}}%
\pgfpathlineto{\pgfqpoint{5.048736in}{1.456315in}}%
\pgfpathlineto{\pgfqpoint{5.048736in}{1.667364in}}%
\pgfpathclose%
\pgfusepath{fill}%
\end{pgfscope}%
\begin{pgfscope}%
\pgfpathrectangle{\pgfqpoint{3.722897in}{0.857143in}}{\pgfqpoint{2.627103in}{1.813434in}}%
\pgfusepath{clip}%
\pgfsetbuttcap%
\pgfsetmiterjoin%
\definecolor{currentfill}{rgb}{0.950697,0.616649,0.428624}%
\pgfsetfillcolor{currentfill}%
\pgfsetlinewidth{0.000000pt}%
\definecolor{currentstroke}{rgb}{0.000000,0.000000,0.000000}%
\pgfsetstrokecolor{currentstroke}%
\pgfsetstrokeopacity{0.000000}%
\pgfsetdash{}{0pt}%
\pgfpathmoveto{\pgfqpoint{5.059907in}{1.651323in}}%
\pgfpathlineto{\pgfqpoint{5.068843in}{1.651323in}}%
\pgfpathlineto{\pgfqpoint{5.068843in}{1.420964in}}%
\pgfpathlineto{\pgfqpoint{5.059907in}{1.420964in}}%
\pgfpathlineto{\pgfqpoint{5.059907in}{1.651323in}}%
\pgfpathclose%
\pgfusepath{fill}%
\end{pgfscope}%
\begin{pgfscope}%
\pgfpathrectangle{\pgfqpoint{3.722897in}{0.857143in}}{\pgfqpoint{2.627103in}{1.813434in}}%
\pgfusepath{clip}%
\pgfsetbuttcap%
\pgfsetmiterjoin%
\definecolor{currentfill}{rgb}{0.950697,0.616649,0.428624}%
\pgfsetfillcolor{currentfill}%
\pgfsetlinewidth{0.000000pt}%
\definecolor{currentstroke}{rgb}{0.000000,0.000000,0.000000}%
\pgfsetstrokecolor{currentstroke}%
\pgfsetstrokeopacity{0.000000}%
\pgfsetdash{}{0pt}%
\pgfpathmoveto{\pgfqpoint{5.071077in}{1.642387in}}%
\pgfpathlineto{\pgfqpoint{5.080014in}{1.642387in}}%
\pgfpathlineto{\pgfqpoint{5.080014in}{1.377210in}}%
\pgfpathlineto{\pgfqpoint{5.071077in}{1.377210in}}%
\pgfpathlineto{\pgfqpoint{5.071077in}{1.642387in}}%
\pgfpathclose%
\pgfusepath{fill}%
\end{pgfscope}%
\begin{pgfscope}%
\pgfpathrectangle{\pgfqpoint{3.722897in}{0.857143in}}{\pgfqpoint{2.627103in}{1.813434in}}%
\pgfusepath{clip}%
\pgfsetbuttcap%
\pgfsetmiterjoin%
\definecolor{currentfill}{rgb}{0.950697,0.616649,0.428624}%
\pgfsetfillcolor{currentfill}%
\pgfsetlinewidth{0.000000pt}%
\definecolor{currentstroke}{rgb}{0.000000,0.000000,0.000000}%
\pgfsetstrokecolor{currentstroke}%
\pgfsetstrokeopacity{0.000000}%
\pgfsetdash{}{0pt}%
\pgfpathmoveto{\pgfqpoint{5.082248in}{1.635946in}}%
\pgfpathlineto{\pgfqpoint{5.091184in}{1.635946in}}%
\pgfpathlineto{\pgfqpoint{5.091184in}{1.355131in}}%
\pgfpathlineto{\pgfqpoint{5.082248in}{1.355131in}}%
\pgfpathlineto{\pgfqpoint{5.082248in}{1.635946in}}%
\pgfpathclose%
\pgfusepath{fill}%
\end{pgfscope}%
\begin{pgfscope}%
\pgfpathrectangle{\pgfqpoint{3.722897in}{0.857143in}}{\pgfqpoint{2.627103in}{1.813434in}}%
\pgfusepath{clip}%
\pgfsetbuttcap%
\pgfsetmiterjoin%
\definecolor{currentfill}{rgb}{0.950697,0.616649,0.428624}%
\pgfsetfillcolor{currentfill}%
\pgfsetlinewidth{0.000000pt}%
\definecolor{currentstroke}{rgb}{0.000000,0.000000,0.000000}%
\pgfsetstrokecolor{currentstroke}%
\pgfsetstrokeopacity{0.000000}%
\pgfsetdash{}{0pt}%
\pgfpathmoveto{\pgfqpoint{5.093419in}{1.630265in}}%
\pgfpathlineto{\pgfqpoint{5.102355in}{1.630265in}}%
\pgfpathlineto{\pgfqpoint{5.102355in}{1.325353in}}%
\pgfpathlineto{\pgfqpoint{5.093419in}{1.325353in}}%
\pgfpathlineto{\pgfqpoint{5.093419in}{1.630265in}}%
\pgfpathclose%
\pgfusepath{fill}%
\end{pgfscope}%
\begin{pgfscope}%
\pgfpathrectangle{\pgfqpoint{3.722897in}{0.857143in}}{\pgfqpoint{2.627103in}{1.813434in}}%
\pgfusepath{clip}%
\pgfsetbuttcap%
\pgfsetmiterjoin%
\definecolor{currentfill}{rgb}{0.950697,0.616649,0.428624}%
\pgfsetfillcolor{currentfill}%
\pgfsetlinewidth{0.000000pt}%
\definecolor{currentstroke}{rgb}{0.000000,0.000000,0.000000}%
\pgfsetstrokecolor{currentstroke}%
\pgfsetstrokeopacity{0.000000}%
\pgfsetdash{}{0pt}%
\pgfpathmoveto{\pgfqpoint{5.104589in}{1.627776in}}%
\pgfpathlineto{\pgfqpoint{5.113526in}{1.627776in}}%
\pgfpathlineto{\pgfqpoint{5.113526in}{1.309494in}}%
\pgfpathlineto{\pgfqpoint{5.104589in}{1.309494in}}%
\pgfpathlineto{\pgfqpoint{5.104589in}{1.627776in}}%
\pgfpathclose%
\pgfusepath{fill}%
\end{pgfscope}%
\begin{pgfscope}%
\pgfpathrectangle{\pgfqpoint{3.722897in}{0.857143in}}{\pgfqpoint{2.627103in}{1.813434in}}%
\pgfusepath{clip}%
\pgfsetbuttcap%
\pgfsetmiterjoin%
\definecolor{currentfill}{rgb}{0.950697,0.616649,0.428624}%
\pgfsetfillcolor{currentfill}%
\pgfsetlinewidth{0.000000pt}%
\definecolor{currentstroke}{rgb}{0.000000,0.000000,0.000000}%
\pgfsetstrokecolor{currentstroke}%
\pgfsetstrokeopacity{0.000000}%
\pgfsetdash{}{0pt}%
\pgfpathmoveto{\pgfqpoint{5.115760in}{1.623061in}}%
\pgfpathlineto{\pgfqpoint{5.124696in}{1.623061in}}%
\pgfpathlineto{\pgfqpoint{5.124696in}{1.285437in}}%
\pgfpathlineto{\pgfqpoint{5.115760in}{1.285437in}}%
\pgfpathlineto{\pgfqpoint{5.115760in}{1.623061in}}%
\pgfpathclose%
\pgfusepath{fill}%
\end{pgfscope}%
\begin{pgfscope}%
\pgfpathrectangle{\pgfqpoint{3.722897in}{0.857143in}}{\pgfqpoint{2.627103in}{1.813434in}}%
\pgfusepath{clip}%
\pgfsetbuttcap%
\pgfsetmiterjoin%
\definecolor{currentfill}{rgb}{0.950697,0.616649,0.428624}%
\pgfsetfillcolor{currentfill}%
\pgfsetlinewidth{0.000000pt}%
\definecolor{currentstroke}{rgb}{0.000000,0.000000,0.000000}%
\pgfsetstrokecolor{currentstroke}%
\pgfsetstrokeopacity{0.000000}%
\pgfsetdash{}{0pt}%
\pgfpathmoveto{\pgfqpoint{5.126930in}{1.621383in}}%
\pgfpathlineto{\pgfqpoint{5.135867in}{1.621383in}}%
\pgfpathlineto{\pgfqpoint{5.135867in}{1.269057in}}%
\pgfpathlineto{\pgfqpoint{5.126930in}{1.269057in}}%
\pgfpathlineto{\pgfqpoint{5.126930in}{1.621383in}}%
\pgfpathclose%
\pgfusepath{fill}%
\end{pgfscope}%
\begin{pgfscope}%
\pgfpathrectangle{\pgfqpoint{3.722897in}{0.857143in}}{\pgfqpoint{2.627103in}{1.813434in}}%
\pgfusepath{clip}%
\pgfsetbuttcap%
\pgfsetmiterjoin%
\definecolor{currentfill}{rgb}{0.950697,0.616649,0.428624}%
\pgfsetfillcolor{currentfill}%
\pgfsetlinewidth{0.000000pt}%
\definecolor{currentstroke}{rgb}{0.000000,0.000000,0.000000}%
\pgfsetstrokecolor{currentstroke}%
\pgfsetstrokeopacity{0.000000}%
\pgfsetdash{}{0pt}%
\pgfpathmoveto{\pgfqpoint{5.138101in}{1.617442in}}%
\pgfpathlineto{\pgfqpoint{5.147038in}{1.617442in}}%
\pgfpathlineto{\pgfqpoint{5.147038in}{1.243005in}}%
\pgfpathlineto{\pgfqpoint{5.138101in}{1.243005in}}%
\pgfpathlineto{\pgfqpoint{5.138101in}{1.617442in}}%
\pgfpathclose%
\pgfusepath{fill}%
\end{pgfscope}%
\begin{pgfscope}%
\pgfpathrectangle{\pgfqpoint{3.722897in}{0.857143in}}{\pgfqpoint{2.627103in}{1.813434in}}%
\pgfusepath{clip}%
\pgfsetbuttcap%
\pgfsetmiterjoin%
\definecolor{currentfill}{rgb}{0.950697,0.616649,0.428624}%
\pgfsetfillcolor{currentfill}%
\pgfsetlinewidth{0.000000pt}%
\definecolor{currentstroke}{rgb}{0.000000,0.000000,0.000000}%
\pgfsetstrokecolor{currentstroke}%
\pgfsetstrokeopacity{0.000000}%
\pgfsetdash{}{0pt}%
\pgfpathmoveto{\pgfqpoint{5.149272in}{1.615228in}}%
\pgfpathlineto{\pgfqpoint{5.158208in}{1.615228in}}%
\pgfpathlineto{\pgfqpoint{5.158208in}{1.241228in}}%
\pgfpathlineto{\pgfqpoint{5.149272in}{1.241228in}}%
\pgfpathlineto{\pgfqpoint{5.149272in}{1.615228in}}%
\pgfpathclose%
\pgfusepath{fill}%
\end{pgfscope}%
\begin{pgfscope}%
\pgfpathrectangle{\pgfqpoint{3.722897in}{0.857143in}}{\pgfqpoint{2.627103in}{1.813434in}}%
\pgfusepath{clip}%
\pgfsetbuttcap%
\pgfsetmiterjoin%
\definecolor{currentfill}{rgb}{0.950697,0.616649,0.428624}%
\pgfsetfillcolor{currentfill}%
\pgfsetlinewidth{0.000000pt}%
\definecolor{currentstroke}{rgb}{0.000000,0.000000,0.000000}%
\pgfsetstrokecolor{currentstroke}%
\pgfsetstrokeopacity{0.000000}%
\pgfsetdash{}{0pt}%
\pgfpathmoveto{\pgfqpoint{5.160442in}{1.615657in}}%
\pgfpathlineto{\pgfqpoint{5.169379in}{1.615657in}}%
\pgfpathlineto{\pgfqpoint{5.169379in}{1.249436in}}%
\pgfpathlineto{\pgfqpoint{5.160442in}{1.249436in}}%
\pgfpathlineto{\pgfqpoint{5.160442in}{1.615657in}}%
\pgfpathclose%
\pgfusepath{fill}%
\end{pgfscope}%
\begin{pgfscope}%
\pgfpathrectangle{\pgfqpoint{3.722897in}{0.857143in}}{\pgfqpoint{2.627103in}{1.813434in}}%
\pgfusepath{clip}%
\pgfsetbuttcap%
\pgfsetmiterjoin%
\definecolor{currentfill}{rgb}{0.950697,0.616649,0.428624}%
\pgfsetfillcolor{currentfill}%
\pgfsetlinewidth{0.000000pt}%
\definecolor{currentstroke}{rgb}{0.000000,0.000000,0.000000}%
\pgfsetstrokecolor{currentstroke}%
\pgfsetstrokeopacity{0.000000}%
\pgfsetdash{}{0pt}%
\pgfpathmoveto{\pgfqpoint{5.171613in}{1.614292in}}%
\pgfpathlineto{\pgfqpoint{5.180549in}{1.614292in}}%
\pgfpathlineto{\pgfqpoint{5.180549in}{1.235036in}}%
\pgfpathlineto{\pgfqpoint{5.171613in}{1.235036in}}%
\pgfpathlineto{\pgfqpoint{5.171613in}{1.614292in}}%
\pgfpathclose%
\pgfusepath{fill}%
\end{pgfscope}%
\begin{pgfscope}%
\pgfpathrectangle{\pgfqpoint{3.722897in}{0.857143in}}{\pgfqpoint{2.627103in}{1.813434in}}%
\pgfusepath{clip}%
\pgfsetbuttcap%
\pgfsetmiterjoin%
\definecolor{currentfill}{rgb}{0.950697,0.616649,0.428624}%
\pgfsetfillcolor{currentfill}%
\pgfsetlinewidth{0.000000pt}%
\definecolor{currentstroke}{rgb}{0.000000,0.000000,0.000000}%
\pgfsetstrokecolor{currentstroke}%
\pgfsetstrokeopacity{0.000000}%
\pgfsetdash{}{0pt}%
\pgfpathmoveto{\pgfqpoint{5.182783in}{1.612273in}}%
\pgfpathlineto{\pgfqpoint{5.191720in}{1.612273in}}%
\pgfpathlineto{\pgfqpoint{5.191720in}{1.230818in}}%
\pgfpathlineto{\pgfqpoint{5.182783in}{1.230818in}}%
\pgfpathlineto{\pgfqpoint{5.182783in}{1.612273in}}%
\pgfpathclose%
\pgfusepath{fill}%
\end{pgfscope}%
\begin{pgfscope}%
\pgfpathrectangle{\pgfqpoint{3.722897in}{0.857143in}}{\pgfqpoint{2.627103in}{1.813434in}}%
\pgfusepath{clip}%
\pgfsetbuttcap%
\pgfsetmiterjoin%
\definecolor{currentfill}{rgb}{0.950697,0.616649,0.428624}%
\pgfsetfillcolor{currentfill}%
\pgfsetlinewidth{0.000000pt}%
\definecolor{currentstroke}{rgb}{0.000000,0.000000,0.000000}%
\pgfsetstrokecolor{currentstroke}%
\pgfsetstrokeopacity{0.000000}%
\pgfsetdash{}{0pt}%
\pgfpathmoveto{\pgfqpoint{5.193954in}{1.614780in}}%
\pgfpathlineto{\pgfqpoint{5.202891in}{1.614780in}}%
\pgfpathlineto{\pgfqpoint{5.202891in}{1.237968in}}%
\pgfpathlineto{\pgfqpoint{5.193954in}{1.237968in}}%
\pgfpathlineto{\pgfqpoint{5.193954in}{1.614780in}}%
\pgfpathclose%
\pgfusepath{fill}%
\end{pgfscope}%
\begin{pgfscope}%
\pgfpathrectangle{\pgfqpoint{3.722897in}{0.857143in}}{\pgfqpoint{2.627103in}{1.813434in}}%
\pgfusepath{clip}%
\pgfsetbuttcap%
\pgfsetmiterjoin%
\definecolor{currentfill}{rgb}{0.950697,0.616649,0.428624}%
\pgfsetfillcolor{currentfill}%
\pgfsetlinewidth{0.000000pt}%
\definecolor{currentstroke}{rgb}{0.000000,0.000000,0.000000}%
\pgfsetstrokecolor{currentstroke}%
\pgfsetstrokeopacity{0.000000}%
\pgfsetdash{}{0pt}%
\pgfpathmoveto{\pgfqpoint{5.205125in}{1.617155in}}%
\pgfpathlineto{\pgfqpoint{5.214061in}{1.617155in}}%
\pgfpathlineto{\pgfqpoint{5.214061in}{1.232358in}}%
\pgfpathlineto{\pgfqpoint{5.205125in}{1.232358in}}%
\pgfpathlineto{\pgfqpoint{5.205125in}{1.617155in}}%
\pgfpathclose%
\pgfusepath{fill}%
\end{pgfscope}%
\begin{pgfscope}%
\pgfpathrectangle{\pgfqpoint{3.722897in}{0.857143in}}{\pgfqpoint{2.627103in}{1.813434in}}%
\pgfusepath{clip}%
\pgfsetbuttcap%
\pgfsetmiterjoin%
\definecolor{currentfill}{rgb}{0.950697,0.616649,0.428624}%
\pgfsetfillcolor{currentfill}%
\pgfsetlinewidth{0.000000pt}%
\definecolor{currentstroke}{rgb}{0.000000,0.000000,0.000000}%
\pgfsetstrokecolor{currentstroke}%
\pgfsetstrokeopacity{0.000000}%
\pgfsetdash{}{0pt}%
\pgfpathmoveto{\pgfqpoint{5.216295in}{1.626045in}}%
\pgfpathlineto{\pgfqpoint{5.225232in}{1.626045in}}%
\pgfpathlineto{\pgfqpoint{5.225232in}{1.232948in}}%
\pgfpathlineto{\pgfqpoint{5.216295in}{1.232948in}}%
\pgfpathlineto{\pgfqpoint{5.216295in}{1.626045in}}%
\pgfpathclose%
\pgfusepath{fill}%
\end{pgfscope}%
\begin{pgfscope}%
\pgfpathrectangle{\pgfqpoint{3.722897in}{0.857143in}}{\pgfqpoint{2.627103in}{1.813434in}}%
\pgfusepath{clip}%
\pgfsetbuttcap%
\pgfsetmiterjoin%
\definecolor{currentfill}{rgb}{0.950697,0.616649,0.428624}%
\pgfsetfillcolor{currentfill}%
\pgfsetlinewidth{0.000000pt}%
\definecolor{currentstroke}{rgb}{0.000000,0.000000,0.000000}%
\pgfsetstrokecolor{currentstroke}%
\pgfsetstrokeopacity{0.000000}%
\pgfsetdash{}{0pt}%
\pgfpathmoveto{\pgfqpoint{5.227466in}{1.642993in}}%
\pgfpathlineto{\pgfqpoint{5.236402in}{1.642993in}}%
\pgfpathlineto{\pgfqpoint{5.236402in}{1.259556in}}%
\pgfpathlineto{\pgfqpoint{5.227466in}{1.259556in}}%
\pgfpathlineto{\pgfqpoint{5.227466in}{1.642993in}}%
\pgfpathclose%
\pgfusepath{fill}%
\end{pgfscope}%
\begin{pgfscope}%
\pgfpathrectangle{\pgfqpoint{3.722897in}{0.857143in}}{\pgfqpoint{2.627103in}{1.813434in}}%
\pgfusepath{clip}%
\pgfsetbuttcap%
\pgfsetmiterjoin%
\definecolor{currentfill}{rgb}{0.950697,0.616649,0.428624}%
\pgfsetfillcolor{currentfill}%
\pgfsetlinewidth{0.000000pt}%
\definecolor{currentstroke}{rgb}{0.000000,0.000000,0.000000}%
\pgfsetstrokecolor{currentstroke}%
\pgfsetstrokeopacity{0.000000}%
\pgfsetdash{}{0pt}%
\pgfpathmoveto{\pgfqpoint{5.238636in}{1.658137in}}%
\pgfpathlineto{\pgfqpoint{5.247573in}{1.658137in}}%
\pgfpathlineto{\pgfqpoint{5.247573in}{1.278891in}}%
\pgfpathlineto{\pgfqpoint{5.238636in}{1.278891in}}%
\pgfpathlineto{\pgfqpoint{5.238636in}{1.658137in}}%
\pgfpathclose%
\pgfusepath{fill}%
\end{pgfscope}%
\begin{pgfscope}%
\pgfpathrectangle{\pgfqpoint{3.722897in}{0.857143in}}{\pgfqpoint{2.627103in}{1.813434in}}%
\pgfusepath{clip}%
\pgfsetbuttcap%
\pgfsetmiterjoin%
\definecolor{currentfill}{rgb}{0.950697,0.616649,0.428624}%
\pgfsetfillcolor{currentfill}%
\pgfsetlinewidth{0.000000pt}%
\definecolor{currentstroke}{rgb}{0.000000,0.000000,0.000000}%
\pgfsetstrokecolor{currentstroke}%
\pgfsetstrokeopacity{0.000000}%
\pgfsetdash{}{0pt}%
\pgfpathmoveto{\pgfqpoint{5.249807in}{1.671543in}}%
\pgfpathlineto{\pgfqpoint{5.258744in}{1.671543in}}%
\pgfpathlineto{\pgfqpoint{5.258744in}{1.292992in}}%
\pgfpathlineto{\pgfqpoint{5.249807in}{1.292992in}}%
\pgfpathlineto{\pgfqpoint{5.249807in}{1.671543in}}%
\pgfpathclose%
\pgfusepath{fill}%
\end{pgfscope}%
\begin{pgfscope}%
\pgfpathrectangle{\pgfqpoint{3.722897in}{0.857143in}}{\pgfqpoint{2.627103in}{1.813434in}}%
\pgfusepath{clip}%
\pgfsetbuttcap%
\pgfsetmiterjoin%
\definecolor{currentfill}{rgb}{0.950697,0.616649,0.428624}%
\pgfsetfillcolor{currentfill}%
\pgfsetlinewidth{0.000000pt}%
\definecolor{currentstroke}{rgb}{0.000000,0.000000,0.000000}%
\pgfsetstrokecolor{currentstroke}%
\pgfsetstrokeopacity{0.000000}%
\pgfsetdash{}{0pt}%
\pgfpathmoveto{\pgfqpoint{5.260978in}{1.684197in}}%
\pgfpathlineto{\pgfqpoint{5.269914in}{1.684197in}}%
\pgfpathlineto{\pgfqpoint{5.269914in}{1.315004in}}%
\pgfpathlineto{\pgfqpoint{5.260978in}{1.315004in}}%
\pgfpathlineto{\pgfqpoint{5.260978in}{1.684197in}}%
\pgfpathclose%
\pgfusepath{fill}%
\end{pgfscope}%
\begin{pgfscope}%
\pgfpathrectangle{\pgfqpoint{3.722897in}{0.857143in}}{\pgfqpoint{2.627103in}{1.813434in}}%
\pgfusepath{clip}%
\pgfsetbuttcap%
\pgfsetmiterjoin%
\definecolor{currentfill}{rgb}{0.950697,0.616649,0.428624}%
\pgfsetfillcolor{currentfill}%
\pgfsetlinewidth{0.000000pt}%
\definecolor{currentstroke}{rgb}{0.000000,0.000000,0.000000}%
\pgfsetstrokecolor{currentstroke}%
\pgfsetstrokeopacity{0.000000}%
\pgfsetdash{}{0pt}%
\pgfpathmoveto{\pgfqpoint{5.272148in}{1.693874in}}%
\pgfpathlineto{\pgfqpoint{5.281085in}{1.693874in}}%
\pgfpathlineto{\pgfqpoint{5.281085in}{1.346956in}}%
\pgfpathlineto{\pgfqpoint{5.272148in}{1.346956in}}%
\pgfpathlineto{\pgfqpoint{5.272148in}{1.693874in}}%
\pgfpathclose%
\pgfusepath{fill}%
\end{pgfscope}%
\begin{pgfscope}%
\pgfpathrectangle{\pgfqpoint{3.722897in}{0.857143in}}{\pgfqpoint{2.627103in}{1.813434in}}%
\pgfusepath{clip}%
\pgfsetbuttcap%
\pgfsetmiterjoin%
\definecolor{currentfill}{rgb}{0.950697,0.616649,0.428624}%
\pgfsetfillcolor{currentfill}%
\pgfsetlinewidth{0.000000pt}%
\definecolor{currentstroke}{rgb}{0.000000,0.000000,0.000000}%
\pgfsetstrokecolor{currentstroke}%
\pgfsetstrokeopacity{0.000000}%
\pgfsetdash{}{0pt}%
\pgfpathmoveto{\pgfqpoint{5.283319in}{1.702412in}}%
\pgfpathlineto{\pgfqpoint{5.292255in}{1.702412in}}%
\pgfpathlineto{\pgfqpoint{5.292255in}{1.372542in}}%
\pgfpathlineto{\pgfqpoint{5.283319in}{1.372542in}}%
\pgfpathlineto{\pgfqpoint{5.283319in}{1.702412in}}%
\pgfpathclose%
\pgfusepath{fill}%
\end{pgfscope}%
\begin{pgfscope}%
\pgfpathrectangle{\pgfqpoint{3.722897in}{0.857143in}}{\pgfqpoint{2.627103in}{1.813434in}}%
\pgfusepath{clip}%
\pgfsetbuttcap%
\pgfsetmiterjoin%
\definecolor{currentfill}{rgb}{0.950697,0.616649,0.428624}%
\pgfsetfillcolor{currentfill}%
\pgfsetlinewidth{0.000000pt}%
\definecolor{currentstroke}{rgb}{0.000000,0.000000,0.000000}%
\pgfsetstrokecolor{currentstroke}%
\pgfsetstrokeopacity{0.000000}%
\pgfsetdash{}{0pt}%
\pgfpathmoveto{\pgfqpoint{5.294489in}{1.713697in}}%
\pgfpathlineto{\pgfqpoint{5.303426in}{1.713697in}}%
\pgfpathlineto{\pgfqpoint{5.303426in}{1.398634in}}%
\pgfpathlineto{\pgfqpoint{5.294489in}{1.398634in}}%
\pgfpathlineto{\pgfqpoint{5.294489in}{1.713697in}}%
\pgfpathclose%
\pgfusepath{fill}%
\end{pgfscope}%
\begin{pgfscope}%
\pgfpathrectangle{\pgfqpoint{3.722897in}{0.857143in}}{\pgfqpoint{2.627103in}{1.813434in}}%
\pgfusepath{clip}%
\pgfsetbuttcap%
\pgfsetmiterjoin%
\definecolor{currentfill}{rgb}{0.950697,0.616649,0.428624}%
\pgfsetfillcolor{currentfill}%
\pgfsetlinewidth{0.000000pt}%
\definecolor{currentstroke}{rgb}{0.000000,0.000000,0.000000}%
\pgfsetstrokecolor{currentstroke}%
\pgfsetstrokeopacity{0.000000}%
\pgfsetdash{}{0pt}%
\pgfpathmoveto{\pgfqpoint{5.305660in}{1.719600in}}%
\pgfpathlineto{\pgfqpoint{5.314597in}{1.719600in}}%
\pgfpathlineto{\pgfqpoint{5.314597in}{1.420432in}}%
\pgfpathlineto{\pgfqpoint{5.305660in}{1.420432in}}%
\pgfpathlineto{\pgfqpoint{5.305660in}{1.719600in}}%
\pgfpathclose%
\pgfusepath{fill}%
\end{pgfscope}%
\begin{pgfscope}%
\pgfpathrectangle{\pgfqpoint{3.722897in}{0.857143in}}{\pgfqpoint{2.627103in}{1.813434in}}%
\pgfusepath{clip}%
\pgfsetbuttcap%
\pgfsetmiterjoin%
\definecolor{currentfill}{rgb}{0.950697,0.616649,0.428624}%
\pgfsetfillcolor{currentfill}%
\pgfsetlinewidth{0.000000pt}%
\definecolor{currentstroke}{rgb}{0.000000,0.000000,0.000000}%
\pgfsetstrokecolor{currentstroke}%
\pgfsetstrokeopacity{0.000000}%
\pgfsetdash{}{0pt}%
\pgfpathmoveto{\pgfqpoint{5.316831in}{1.721638in}}%
\pgfpathlineto{\pgfqpoint{5.325767in}{1.721638in}}%
\pgfpathlineto{\pgfqpoint{5.325767in}{1.438569in}}%
\pgfpathlineto{\pgfqpoint{5.316831in}{1.438569in}}%
\pgfpathlineto{\pgfqpoint{5.316831in}{1.721638in}}%
\pgfpathclose%
\pgfusepath{fill}%
\end{pgfscope}%
\begin{pgfscope}%
\pgfpathrectangle{\pgfqpoint{3.722897in}{0.857143in}}{\pgfqpoint{2.627103in}{1.813434in}}%
\pgfusepath{clip}%
\pgfsetbuttcap%
\pgfsetmiterjoin%
\definecolor{currentfill}{rgb}{0.950697,0.616649,0.428624}%
\pgfsetfillcolor{currentfill}%
\pgfsetlinewidth{0.000000pt}%
\definecolor{currentstroke}{rgb}{0.000000,0.000000,0.000000}%
\pgfsetstrokecolor{currentstroke}%
\pgfsetstrokeopacity{0.000000}%
\pgfsetdash{}{0pt}%
\pgfpathmoveto{\pgfqpoint{5.328001in}{1.729283in}}%
\pgfpathlineto{\pgfqpoint{5.336938in}{1.729283in}}%
\pgfpathlineto{\pgfqpoint{5.336938in}{1.476120in}}%
\pgfpathlineto{\pgfqpoint{5.328001in}{1.476120in}}%
\pgfpathlineto{\pgfqpoint{5.328001in}{1.729283in}}%
\pgfpathclose%
\pgfusepath{fill}%
\end{pgfscope}%
\begin{pgfscope}%
\pgfpathrectangle{\pgfqpoint{3.722897in}{0.857143in}}{\pgfqpoint{2.627103in}{1.813434in}}%
\pgfusepath{clip}%
\pgfsetbuttcap%
\pgfsetmiterjoin%
\definecolor{currentfill}{rgb}{0.950697,0.616649,0.428624}%
\pgfsetfillcolor{currentfill}%
\pgfsetlinewidth{0.000000pt}%
\definecolor{currentstroke}{rgb}{0.000000,0.000000,0.000000}%
\pgfsetstrokecolor{currentstroke}%
\pgfsetstrokeopacity{0.000000}%
\pgfsetdash{}{0pt}%
\pgfpathmoveto{\pgfqpoint{5.339172in}{1.731564in}}%
\pgfpathlineto{\pgfqpoint{5.348108in}{1.731564in}}%
\pgfpathlineto{\pgfqpoint{5.348108in}{1.486570in}}%
\pgfpathlineto{\pgfqpoint{5.339172in}{1.486570in}}%
\pgfpathlineto{\pgfqpoint{5.339172in}{1.731564in}}%
\pgfpathclose%
\pgfusepath{fill}%
\end{pgfscope}%
\begin{pgfscope}%
\pgfpathrectangle{\pgfqpoint{3.722897in}{0.857143in}}{\pgfqpoint{2.627103in}{1.813434in}}%
\pgfusepath{clip}%
\pgfsetbuttcap%
\pgfsetmiterjoin%
\definecolor{currentfill}{rgb}{0.950697,0.616649,0.428624}%
\pgfsetfillcolor{currentfill}%
\pgfsetlinewidth{0.000000pt}%
\definecolor{currentstroke}{rgb}{0.000000,0.000000,0.000000}%
\pgfsetstrokecolor{currentstroke}%
\pgfsetstrokeopacity{0.000000}%
\pgfsetdash{}{0pt}%
\pgfpathmoveto{\pgfqpoint{5.350343in}{1.731957in}}%
\pgfpathlineto{\pgfqpoint{5.359279in}{1.731957in}}%
\pgfpathlineto{\pgfqpoint{5.359279in}{1.493654in}}%
\pgfpathlineto{\pgfqpoint{5.350343in}{1.493654in}}%
\pgfpathlineto{\pgfqpoint{5.350343in}{1.731957in}}%
\pgfpathclose%
\pgfusepath{fill}%
\end{pgfscope}%
\begin{pgfscope}%
\pgfpathrectangle{\pgfqpoint{3.722897in}{0.857143in}}{\pgfqpoint{2.627103in}{1.813434in}}%
\pgfusepath{clip}%
\pgfsetbuttcap%
\pgfsetmiterjoin%
\definecolor{currentfill}{rgb}{0.950697,0.616649,0.428624}%
\pgfsetfillcolor{currentfill}%
\pgfsetlinewidth{0.000000pt}%
\definecolor{currentstroke}{rgb}{0.000000,0.000000,0.000000}%
\pgfsetstrokecolor{currentstroke}%
\pgfsetstrokeopacity{0.000000}%
\pgfsetdash{}{0pt}%
\pgfpathmoveto{\pgfqpoint{5.361513in}{1.730146in}}%
\pgfpathlineto{\pgfqpoint{5.370450in}{1.730146in}}%
\pgfpathlineto{\pgfqpoint{5.370450in}{1.519631in}}%
\pgfpathlineto{\pgfqpoint{5.361513in}{1.519631in}}%
\pgfpathlineto{\pgfqpoint{5.361513in}{1.730146in}}%
\pgfpathclose%
\pgfusepath{fill}%
\end{pgfscope}%
\begin{pgfscope}%
\pgfpathrectangle{\pgfqpoint{3.722897in}{0.857143in}}{\pgfqpoint{2.627103in}{1.813434in}}%
\pgfusepath{clip}%
\pgfsetbuttcap%
\pgfsetmiterjoin%
\definecolor{currentfill}{rgb}{0.950697,0.616649,0.428624}%
\pgfsetfillcolor{currentfill}%
\pgfsetlinewidth{0.000000pt}%
\definecolor{currentstroke}{rgb}{0.000000,0.000000,0.000000}%
\pgfsetstrokecolor{currentstroke}%
\pgfsetstrokeopacity{0.000000}%
\pgfsetdash{}{0pt}%
\pgfpathmoveto{\pgfqpoint{5.372684in}{1.722161in}}%
\pgfpathlineto{\pgfqpoint{5.381620in}{1.722161in}}%
\pgfpathlineto{\pgfqpoint{5.381620in}{1.530541in}}%
\pgfpathlineto{\pgfqpoint{5.372684in}{1.530541in}}%
\pgfpathlineto{\pgfqpoint{5.372684in}{1.722161in}}%
\pgfpathclose%
\pgfusepath{fill}%
\end{pgfscope}%
\begin{pgfscope}%
\pgfpathrectangle{\pgfqpoint{3.722897in}{0.857143in}}{\pgfqpoint{2.627103in}{1.813434in}}%
\pgfusepath{clip}%
\pgfsetbuttcap%
\pgfsetmiterjoin%
\definecolor{currentfill}{rgb}{0.950697,0.616649,0.428624}%
\pgfsetfillcolor{currentfill}%
\pgfsetlinewidth{0.000000pt}%
\definecolor{currentstroke}{rgb}{0.000000,0.000000,0.000000}%
\pgfsetstrokecolor{currentstroke}%
\pgfsetstrokeopacity{0.000000}%
\pgfsetdash{}{0pt}%
\pgfpathmoveto{\pgfqpoint{5.383854in}{1.701564in}}%
\pgfpathlineto{\pgfqpoint{5.392791in}{1.701564in}}%
\pgfpathlineto{\pgfqpoint{5.392791in}{1.553306in}}%
\pgfpathlineto{\pgfqpoint{5.383854in}{1.553306in}}%
\pgfpathlineto{\pgfqpoint{5.383854in}{1.701564in}}%
\pgfpathclose%
\pgfusepath{fill}%
\end{pgfscope}%
\begin{pgfscope}%
\pgfpathrectangle{\pgfqpoint{3.722897in}{0.857143in}}{\pgfqpoint{2.627103in}{1.813434in}}%
\pgfusepath{clip}%
\pgfsetbuttcap%
\pgfsetmiterjoin%
\definecolor{currentfill}{rgb}{0.950697,0.616649,0.428624}%
\pgfsetfillcolor{currentfill}%
\pgfsetlinewidth{0.000000pt}%
\definecolor{currentstroke}{rgb}{0.000000,0.000000,0.000000}%
\pgfsetstrokecolor{currentstroke}%
\pgfsetstrokeopacity{0.000000}%
\pgfsetdash{}{0pt}%
\pgfpathmoveto{\pgfqpoint{5.395025in}{1.683758in}}%
\pgfpathlineto{\pgfqpoint{5.403961in}{1.683758in}}%
\pgfpathlineto{\pgfqpoint{5.403961in}{1.542006in}}%
\pgfpathlineto{\pgfqpoint{5.395025in}{1.542006in}}%
\pgfpathlineto{\pgfqpoint{5.395025in}{1.683758in}}%
\pgfpathclose%
\pgfusepath{fill}%
\end{pgfscope}%
\begin{pgfscope}%
\pgfpathrectangle{\pgfqpoint{3.722897in}{0.857143in}}{\pgfqpoint{2.627103in}{1.813434in}}%
\pgfusepath{clip}%
\pgfsetbuttcap%
\pgfsetmiterjoin%
\definecolor{currentfill}{rgb}{0.950697,0.616649,0.428624}%
\pgfsetfillcolor{currentfill}%
\pgfsetlinewidth{0.000000pt}%
\definecolor{currentstroke}{rgb}{0.000000,0.000000,0.000000}%
\pgfsetstrokecolor{currentstroke}%
\pgfsetstrokeopacity{0.000000}%
\pgfsetdash{}{0pt}%
\pgfpathmoveto{\pgfqpoint{5.406196in}{1.668490in}}%
\pgfpathlineto{\pgfqpoint{5.415132in}{1.668490in}}%
\pgfpathlineto{\pgfqpoint{5.415132in}{1.571172in}}%
\pgfpathlineto{\pgfqpoint{5.406196in}{1.571172in}}%
\pgfpathlineto{\pgfqpoint{5.406196in}{1.668490in}}%
\pgfpathclose%
\pgfusepath{fill}%
\end{pgfscope}%
\begin{pgfscope}%
\pgfpathrectangle{\pgfqpoint{3.722897in}{0.857143in}}{\pgfqpoint{2.627103in}{1.813434in}}%
\pgfusepath{clip}%
\pgfsetbuttcap%
\pgfsetmiterjoin%
\definecolor{currentfill}{rgb}{0.950697,0.616649,0.428624}%
\pgfsetfillcolor{currentfill}%
\pgfsetlinewidth{0.000000pt}%
\definecolor{currentstroke}{rgb}{0.000000,0.000000,0.000000}%
\pgfsetstrokecolor{currentstroke}%
\pgfsetstrokeopacity{0.000000}%
\pgfsetdash{}{0pt}%
\pgfpathmoveto{\pgfqpoint{5.417366in}{1.640711in}}%
\pgfpathlineto{\pgfqpoint{5.426303in}{1.640711in}}%
\pgfpathlineto{\pgfqpoint{5.426303in}{1.569398in}}%
\pgfpathlineto{\pgfqpoint{5.417366in}{1.569398in}}%
\pgfpathlineto{\pgfqpoint{5.417366in}{1.640711in}}%
\pgfpathclose%
\pgfusepath{fill}%
\end{pgfscope}%
\begin{pgfscope}%
\pgfpathrectangle{\pgfqpoint{3.722897in}{0.857143in}}{\pgfqpoint{2.627103in}{1.813434in}}%
\pgfusepath{clip}%
\pgfsetbuttcap%
\pgfsetmiterjoin%
\definecolor{currentfill}{rgb}{0.950697,0.616649,0.428624}%
\pgfsetfillcolor{currentfill}%
\pgfsetlinewidth{0.000000pt}%
\definecolor{currentstroke}{rgb}{0.000000,0.000000,0.000000}%
\pgfsetstrokecolor{currentstroke}%
\pgfsetstrokeopacity{0.000000}%
\pgfsetdash{}{0pt}%
\pgfpathmoveto{\pgfqpoint{5.428537in}{1.614522in}}%
\pgfpathlineto{\pgfqpoint{5.437473in}{1.614522in}}%
\pgfpathlineto{\pgfqpoint{5.437473in}{1.555325in}}%
\pgfpathlineto{\pgfqpoint{5.428537in}{1.555325in}}%
\pgfpathlineto{\pgfqpoint{5.428537in}{1.614522in}}%
\pgfpathclose%
\pgfusepath{fill}%
\end{pgfscope}%
\begin{pgfscope}%
\pgfpathrectangle{\pgfqpoint{3.722897in}{0.857143in}}{\pgfqpoint{2.627103in}{1.813434in}}%
\pgfusepath{clip}%
\pgfsetbuttcap%
\pgfsetmiterjoin%
\definecolor{currentfill}{rgb}{0.950697,0.616649,0.428624}%
\pgfsetfillcolor{currentfill}%
\pgfsetlinewidth{0.000000pt}%
\definecolor{currentstroke}{rgb}{0.000000,0.000000,0.000000}%
\pgfsetstrokecolor{currentstroke}%
\pgfsetstrokeopacity{0.000000}%
\pgfsetdash{}{0pt}%
\pgfpathmoveto{\pgfqpoint{5.439707in}{1.596910in}}%
\pgfpathlineto{\pgfqpoint{5.448644in}{1.596910in}}%
\pgfpathlineto{\pgfqpoint{5.448644in}{1.556436in}}%
\pgfpathlineto{\pgfqpoint{5.439707in}{1.556436in}}%
\pgfpathlineto{\pgfqpoint{5.439707in}{1.596910in}}%
\pgfpathclose%
\pgfusepath{fill}%
\end{pgfscope}%
\begin{pgfscope}%
\pgfpathrectangle{\pgfqpoint{3.722897in}{0.857143in}}{\pgfqpoint{2.627103in}{1.813434in}}%
\pgfusepath{clip}%
\pgfsetbuttcap%
\pgfsetmiterjoin%
\definecolor{currentfill}{rgb}{0.950697,0.616649,0.428624}%
\pgfsetfillcolor{currentfill}%
\pgfsetlinewidth{0.000000pt}%
\definecolor{currentstroke}{rgb}{0.000000,0.000000,0.000000}%
\pgfsetstrokecolor{currentstroke}%
\pgfsetstrokeopacity{0.000000}%
\pgfsetdash{}{0pt}%
\pgfpathmoveto{\pgfqpoint{5.450878in}{1.575131in}}%
\pgfpathlineto{\pgfqpoint{5.459814in}{1.575131in}}%
\pgfpathlineto{\pgfqpoint{5.459814in}{1.535147in}}%
\pgfpathlineto{\pgfqpoint{5.450878in}{1.535147in}}%
\pgfpathlineto{\pgfqpoint{5.450878in}{1.575131in}}%
\pgfpathclose%
\pgfusepath{fill}%
\end{pgfscope}%
\begin{pgfscope}%
\pgfpathrectangle{\pgfqpoint{3.722897in}{0.857143in}}{\pgfqpoint{2.627103in}{1.813434in}}%
\pgfusepath{clip}%
\pgfsetbuttcap%
\pgfsetmiterjoin%
\definecolor{currentfill}{rgb}{0.950697,0.616649,0.428624}%
\pgfsetfillcolor{currentfill}%
\pgfsetlinewidth{0.000000pt}%
\definecolor{currentstroke}{rgb}{0.000000,0.000000,0.000000}%
\pgfsetstrokecolor{currentstroke}%
\pgfsetstrokeopacity{0.000000}%
\pgfsetdash{}{0pt}%
\pgfpathmoveto{\pgfqpoint{5.462049in}{1.553107in}}%
\pgfpathlineto{\pgfqpoint{5.470985in}{1.553107in}}%
\pgfpathlineto{\pgfqpoint{5.470985in}{1.523865in}}%
\pgfpathlineto{\pgfqpoint{5.462049in}{1.523865in}}%
\pgfpathlineto{\pgfqpoint{5.462049in}{1.553107in}}%
\pgfpathclose%
\pgfusepath{fill}%
\end{pgfscope}%
\begin{pgfscope}%
\pgfpathrectangle{\pgfqpoint{3.722897in}{0.857143in}}{\pgfqpoint{2.627103in}{1.813434in}}%
\pgfusepath{clip}%
\pgfsetbuttcap%
\pgfsetmiterjoin%
\definecolor{currentfill}{rgb}{0.950697,0.616649,0.428624}%
\pgfsetfillcolor{currentfill}%
\pgfsetlinewidth{0.000000pt}%
\definecolor{currentstroke}{rgb}{0.000000,0.000000,0.000000}%
\pgfsetstrokecolor{currentstroke}%
\pgfsetstrokeopacity{0.000000}%
\pgfsetdash{}{0pt}%
\pgfpathmoveto{\pgfqpoint{5.473219in}{1.533831in}}%
\pgfpathlineto{\pgfqpoint{5.482156in}{1.533831in}}%
\pgfpathlineto{\pgfqpoint{5.482156in}{1.498826in}}%
\pgfpathlineto{\pgfqpoint{5.473219in}{1.498826in}}%
\pgfpathlineto{\pgfqpoint{5.473219in}{1.533831in}}%
\pgfpathclose%
\pgfusepath{fill}%
\end{pgfscope}%
\begin{pgfscope}%
\pgfpathrectangle{\pgfqpoint{3.722897in}{0.857143in}}{\pgfqpoint{2.627103in}{1.813434in}}%
\pgfusepath{clip}%
\pgfsetbuttcap%
\pgfsetmiterjoin%
\definecolor{currentfill}{rgb}{0.950697,0.616649,0.428624}%
\pgfsetfillcolor{currentfill}%
\pgfsetlinewidth{0.000000pt}%
\definecolor{currentstroke}{rgb}{0.000000,0.000000,0.000000}%
\pgfsetstrokecolor{currentstroke}%
\pgfsetstrokeopacity{0.000000}%
\pgfsetdash{}{0pt}%
\pgfpathmoveto{\pgfqpoint{5.484390in}{1.513909in}}%
\pgfpathlineto{\pgfqpoint{5.493326in}{1.513909in}}%
\pgfpathlineto{\pgfqpoint{5.493326in}{1.501242in}}%
\pgfpathlineto{\pgfqpoint{5.484390in}{1.501242in}}%
\pgfpathlineto{\pgfqpoint{5.484390in}{1.513909in}}%
\pgfpathclose%
\pgfusepath{fill}%
\end{pgfscope}%
\begin{pgfscope}%
\pgfpathrectangle{\pgfqpoint{3.722897in}{0.857143in}}{\pgfqpoint{2.627103in}{1.813434in}}%
\pgfusepath{clip}%
\pgfsetbuttcap%
\pgfsetmiterjoin%
\definecolor{currentfill}{rgb}{0.950697,0.616649,0.428624}%
\pgfsetfillcolor{currentfill}%
\pgfsetlinewidth{0.000000pt}%
\definecolor{currentstroke}{rgb}{0.000000,0.000000,0.000000}%
\pgfsetstrokecolor{currentstroke}%
\pgfsetstrokeopacity{0.000000}%
\pgfsetdash{}{0pt}%
\pgfpathmoveto{\pgfqpoint{5.495560in}{1.500738in}}%
\pgfpathlineto{\pgfqpoint{5.504497in}{1.500738in}}%
\pgfpathlineto{\pgfqpoint{5.504497in}{1.478963in}}%
\pgfpathlineto{\pgfqpoint{5.495560in}{1.478963in}}%
\pgfpathlineto{\pgfqpoint{5.495560in}{1.500738in}}%
\pgfpathclose%
\pgfusepath{fill}%
\end{pgfscope}%
\begin{pgfscope}%
\pgfpathrectangle{\pgfqpoint{3.722897in}{0.857143in}}{\pgfqpoint{2.627103in}{1.813434in}}%
\pgfusepath{clip}%
\pgfsetbuttcap%
\pgfsetmiterjoin%
\definecolor{currentfill}{rgb}{0.950697,0.616649,0.428624}%
\pgfsetfillcolor{currentfill}%
\pgfsetlinewidth{0.000000pt}%
\definecolor{currentstroke}{rgb}{0.000000,0.000000,0.000000}%
\pgfsetstrokecolor{currentstroke}%
\pgfsetstrokeopacity{0.000000}%
\pgfsetdash{}{0pt}%
\pgfpathmoveto{\pgfqpoint{5.506731in}{1.495316in}}%
\pgfpathlineto{\pgfqpoint{5.515667in}{1.495316in}}%
\pgfpathlineto{\pgfqpoint{5.515667in}{1.493413in}}%
\pgfpathlineto{\pgfqpoint{5.506731in}{1.493413in}}%
\pgfpathlineto{\pgfqpoint{5.506731in}{1.495316in}}%
\pgfpathclose%
\pgfusepath{fill}%
\end{pgfscope}%
\begin{pgfscope}%
\pgfpathrectangle{\pgfqpoint{3.722897in}{0.857143in}}{\pgfqpoint{2.627103in}{1.813434in}}%
\pgfusepath{clip}%
\pgfsetbuttcap%
\pgfsetmiterjoin%
\definecolor{currentfill}{rgb}{0.950697,0.616649,0.428624}%
\pgfsetfillcolor{currentfill}%
\pgfsetlinewidth{0.000000pt}%
\definecolor{currentstroke}{rgb}{0.000000,0.000000,0.000000}%
\pgfsetstrokecolor{currentstroke}%
\pgfsetstrokeopacity{0.000000}%
\pgfsetdash{}{0pt}%
\pgfpathmoveto{\pgfqpoint{5.517902in}{1.490808in}}%
\pgfpathlineto{\pgfqpoint{5.526838in}{1.490808in}}%
\pgfpathlineto{\pgfqpoint{5.526838in}{1.476244in}}%
\pgfpathlineto{\pgfqpoint{5.517902in}{1.476244in}}%
\pgfpathlineto{\pgfqpoint{5.517902in}{1.490808in}}%
\pgfpathclose%
\pgfusepath{fill}%
\end{pgfscope}%
\begin{pgfscope}%
\pgfpathrectangle{\pgfqpoint{3.722897in}{0.857143in}}{\pgfqpoint{2.627103in}{1.813434in}}%
\pgfusepath{clip}%
\pgfsetbuttcap%
\pgfsetmiterjoin%
\definecolor{currentfill}{rgb}{0.950697,0.616649,0.428624}%
\pgfsetfillcolor{currentfill}%
\pgfsetlinewidth{0.000000pt}%
\definecolor{currentstroke}{rgb}{0.000000,0.000000,0.000000}%
\pgfsetstrokecolor{currentstroke}%
\pgfsetstrokeopacity{0.000000}%
\pgfsetdash{}{0pt}%
\pgfpathmoveto{\pgfqpoint{5.529072in}{1.484546in}}%
\pgfpathlineto{\pgfqpoint{5.538009in}{1.484546in}}%
\pgfpathlineto{\pgfqpoint{5.538009in}{1.471089in}}%
\pgfpathlineto{\pgfqpoint{5.529072in}{1.471089in}}%
\pgfpathlineto{\pgfqpoint{5.529072in}{1.484546in}}%
\pgfpathclose%
\pgfusepath{fill}%
\end{pgfscope}%
\begin{pgfscope}%
\pgfpathrectangle{\pgfqpoint{3.722897in}{0.857143in}}{\pgfqpoint{2.627103in}{1.813434in}}%
\pgfusepath{clip}%
\pgfsetbuttcap%
\pgfsetmiterjoin%
\definecolor{currentfill}{rgb}{0.950697,0.616649,0.428624}%
\pgfsetfillcolor{currentfill}%
\pgfsetlinewidth{0.000000pt}%
\definecolor{currentstroke}{rgb}{0.000000,0.000000,0.000000}%
\pgfsetstrokecolor{currentstroke}%
\pgfsetstrokeopacity{0.000000}%
\pgfsetdash{}{0pt}%
\pgfpathmoveto{\pgfqpoint{5.540243in}{1.486228in}}%
\pgfpathlineto{\pgfqpoint{5.549179in}{1.486228in}}%
\pgfpathlineto{\pgfqpoint{5.549179in}{1.459411in}}%
\pgfpathlineto{\pgfqpoint{5.540243in}{1.459411in}}%
\pgfpathlineto{\pgfqpoint{5.540243in}{1.486228in}}%
\pgfpathclose%
\pgfusepath{fill}%
\end{pgfscope}%
\begin{pgfscope}%
\pgfpathrectangle{\pgfqpoint{3.722897in}{0.857143in}}{\pgfqpoint{2.627103in}{1.813434in}}%
\pgfusepath{clip}%
\pgfsetbuttcap%
\pgfsetmiterjoin%
\definecolor{currentfill}{rgb}{0.950697,0.616649,0.428624}%
\pgfsetfillcolor{currentfill}%
\pgfsetlinewidth{0.000000pt}%
\definecolor{currentstroke}{rgb}{0.000000,0.000000,0.000000}%
\pgfsetstrokecolor{currentstroke}%
\pgfsetstrokeopacity{0.000000}%
\pgfsetdash{}{0pt}%
\pgfpathmoveto{\pgfqpoint{5.551413in}{1.486043in}}%
\pgfpathlineto{\pgfqpoint{5.560350in}{1.486043in}}%
\pgfpathlineto{\pgfqpoint{5.560350in}{1.461673in}}%
\pgfpathlineto{\pgfqpoint{5.551413in}{1.461673in}}%
\pgfpathlineto{\pgfqpoint{5.551413in}{1.486043in}}%
\pgfpathclose%
\pgfusepath{fill}%
\end{pgfscope}%
\begin{pgfscope}%
\pgfpathrectangle{\pgfqpoint{3.722897in}{0.857143in}}{\pgfqpoint{2.627103in}{1.813434in}}%
\pgfusepath{clip}%
\pgfsetbuttcap%
\pgfsetmiterjoin%
\definecolor{currentfill}{rgb}{0.950697,0.616649,0.428624}%
\pgfsetfillcolor{currentfill}%
\pgfsetlinewidth{0.000000pt}%
\definecolor{currentstroke}{rgb}{0.000000,0.000000,0.000000}%
\pgfsetstrokecolor{currentstroke}%
\pgfsetstrokeopacity{0.000000}%
\pgfsetdash{}{0pt}%
\pgfpathmoveto{\pgfqpoint{5.562584in}{1.480446in}}%
\pgfpathlineto{\pgfqpoint{5.571521in}{1.480446in}}%
\pgfpathlineto{\pgfqpoint{5.571521in}{1.447506in}}%
\pgfpathlineto{\pgfqpoint{5.562584in}{1.447506in}}%
\pgfpathlineto{\pgfqpoint{5.562584in}{1.480446in}}%
\pgfpathclose%
\pgfusepath{fill}%
\end{pgfscope}%
\begin{pgfscope}%
\pgfpathrectangle{\pgfqpoint{3.722897in}{0.857143in}}{\pgfqpoint{2.627103in}{1.813434in}}%
\pgfusepath{clip}%
\pgfsetbuttcap%
\pgfsetmiterjoin%
\definecolor{currentfill}{rgb}{0.950697,0.616649,0.428624}%
\pgfsetfillcolor{currentfill}%
\pgfsetlinewidth{0.000000pt}%
\definecolor{currentstroke}{rgb}{0.000000,0.000000,0.000000}%
\pgfsetstrokecolor{currentstroke}%
\pgfsetstrokeopacity{0.000000}%
\pgfsetdash{}{0pt}%
\pgfpathmoveto{\pgfqpoint{5.573755in}{1.474518in}}%
\pgfpathlineto{\pgfqpoint{5.582691in}{1.474518in}}%
\pgfpathlineto{\pgfqpoint{5.582691in}{1.422117in}}%
\pgfpathlineto{\pgfqpoint{5.573755in}{1.422117in}}%
\pgfpathlineto{\pgfqpoint{5.573755in}{1.474518in}}%
\pgfpathclose%
\pgfusepath{fill}%
\end{pgfscope}%
\begin{pgfscope}%
\pgfpathrectangle{\pgfqpoint{3.722897in}{0.857143in}}{\pgfqpoint{2.627103in}{1.813434in}}%
\pgfusepath{clip}%
\pgfsetbuttcap%
\pgfsetmiterjoin%
\definecolor{currentfill}{rgb}{0.950697,0.616649,0.428624}%
\pgfsetfillcolor{currentfill}%
\pgfsetlinewidth{0.000000pt}%
\definecolor{currentstroke}{rgb}{0.000000,0.000000,0.000000}%
\pgfsetstrokecolor{currentstroke}%
\pgfsetstrokeopacity{0.000000}%
\pgfsetdash{}{0pt}%
\pgfpathmoveto{\pgfqpoint{5.584925in}{1.459405in}}%
\pgfpathlineto{\pgfqpoint{5.593862in}{1.459405in}}%
\pgfpathlineto{\pgfqpoint{5.593862in}{1.406666in}}%
\pgfpathlineto{\pgfqpoint{5.584925in}{1.406666in}}%
\pgfpathlineto{\pgfqpoint{5.584925in}{1.459405in}}%
\pgfpathclose%
\pgfusepath{fill}%
\end{pgfscope}%
\begin{pgfscope}%
\pgfpathrectangle{\pgfqpoint{3.722897in}{0.857143in}}{\pgfqpoint{2.627103in}{1.813434in}}%
\pgfusepath{clip}%
\pgfsetbuttcap%
\pgfsetmiterjoin%
\definecolor{currentfill}{rgb}{0.950697,0.616649,0.428624}%
\pgfsetfillcolor{currentfill}%
\pgfsetlinewidth{0.000000pt}%
\definecolor{currentstroke}{rgb}{0.000000,0.000000,0.000000}%
\pgfsetstrokecolor{currentstroke}%
\pgfsetstrokeopacity{0.000000}%
\pgfsetdash{}{0pt}%
\pgfpathmoveto{\pgfqpoint{5.596096in}{1.446711in}}%
\pgfpathlineto{\pgfqpoint{5.605032in}{1.446711in}}%
\pgfpathlineto{\pgfqpoint{5.605032in}{1.397961in}}%
\pgfpathlineto{\pgfqpoint{5.596096in}{1.397961in}}%
\pgfpathlineto{\pgfqpoint{5.596096in}{1.446711in}}%
\pgfpathclose%
\pgfusepath{fill}%
\end{pgfscope}%
\begin{pgfscope}%
\pgfpathrectangle{\pgfqpoint{3.722897in}{0.857143in}}{\pgfqpoint{2.627103in}{1.813434in}}%
\pgfusepath{clip}%
\pgfsetbuttcap%
\pgfsetmiterjoin%
\definecolor{currentfill}{rgb}{0.950697,0.616649,0.428624}%
\pgfsetfillcolor{currentfill}%
\pgfsetlinewidth{0.000000pt}%
\definecolor{currentstroke}{rgb}{0.000000,0.000000,0.000000}%
\pgfsetstrokecolor{currentstroke}%
\pgfsetstrokeopacity{0.000000}%
\pgfsetdash{}{0pt}%
\pgfpathmoveto{\pgfqpoint{5.607266in}{1.445675in}}%
\pgfpathlineto{\pgfqpoint{5.616203in}{1.445675in}}%
\pgfpathlineto{\pgfqpoint{5.616203in}{1.385305in}}%
\pgfpathlineto{\pgfqpoint{5.607266in}{1.385305in}}%
\pgfpathlineto{\pgfqpoint{5.607266in}{1.445675in}}%
\pgfpathclose%
\pgfusepath{fill}%
\end{pgfscope}%
\begin{pgfscope}%
\pgfpathrectangle{\pgfqpoint{3.722897in}{0.857143in}}{\pgfqpoint{2.627103in}{1.813434in}}%
\pgfusepath{clip}%
\pgfsetbuttcap%
\pgfsetmiterjoin%
\definecolor{currentfill}{rgb}{0.950697,0.616649,0.428624}%
\pgfsetfillcolor{currentfill}%
\pgfsetlinewidth{0.000000pt}%
\definecolor{currentstroke}{rgb}{0.000000,0.000000,0.000000}%
\pgfsetstrokecolor{currentstroke}%
\pgfsetstrokeopacity{0.000000}%
\pgfsetdash{}{0pt}%
\pgfpathmoveto{\pgfqpoint{5.618437in}{1.456498in}}%
\pgfpathlineto{\pgfqpoint{5.627374in}{1.456498in}}%
\pgfpathlineto{\pgfqpoint{5.627374in}{1.378416in}}%
\pgfpathlineto{\pgfqpoint{5.618437in}{1.378416in}}%
\pgfpathlineto{\pgfqpoint{5.618437in}{1.456498in}}%
\pgfpathclose%
\pgfusepath{fill}%
\end{pgfscope}%
\begin{pgfscope}%
\pgfpathrectangle{\pgfqpoint{3.722897in}{0.857143in}}{\pgfqpoint{2.627103in}{1.813434in}}%
\pgfusepath{clip}%
\pgfsetbuttcap%
\pgfsetmiterjoin%
\definecolor{currentfill}{rgb}{0.950697,0.616649,0.428624}%
\pgfsetfillcolor{currentfill}%
\pgfsetlinewidth{0.000000pt}%
\definecolor{currentstroke}{rgb}{0.000000,0.000000,0.000000}%
\pgfsetstrokecolor{currentstroke}%
\pgfsetstrokeopacity{0.000000}%
\pgfsetdash{}{0pt}%
\pgfpathmoveto{\pgfqpoint{5.629608in}{1.458745in}}%
\pgfpathlineto{\pgfqpoint{5.638544in}{1.458745in}}%
\pgfpathlineto{\pgfqpoint{5.638544in}{1.387334in}}%
\pgfpathlineto{\pgfqpoint{5.629608in}{1.387334in}}%
\pgfpathlineto{\pgfqpoint{5.629608in}{1.458745in}}%
\pgfpathclose%
\pgfusepath{fill}%
\end{pgfscope}%
\begin{pgfscope}%
\pgfpathrectangle{\pgfqpoint{3.722897in}{0.857143in}}{\pgfqpoint{2.627103in}{1.813434in}}%
\pgfusepath{clip}%
\pgfsetbuttcap%
\pgfsetmiterjoin%
\definecolor{currentfill}{rgb}{0.950697,0.616649,0.428624}%
\pgfsetfillcolor{currentfill}%
\pgfsetlinewidth{0.000000pt}%
\definecolor{currentstroke}{rgb}{0.000000,0.000000,0.000000}%
\pgfsetstrokecolor{currentstroke}%
\pgfsetstrokeopacity{0.000000}%
\pgfsetdash{}{0pt}%
\pgfpathmoveto{\pgfqpoint{5.640778in}{1.467207in}}%
\pgfpathlineto{\pgfqpoint{5.649715in}{1.467207in}}%
\pgfpathlineto{\pgfqpoint{5.649715in}{1.396584in}}%
\pgfpathlineto{\pgfqpoint{5.640778in}{1.396584in}}%
\pgfpathlineto{\pgfqpoint{5.640778in}{1.467207in}}%
\pgfpathclose%
\pgfusepath{fill}%
\end{pgfscope}%
\begin{pgfscope}%
\pgfpathrectangle{\pgfqpoint{3.722897in}{0.857143in}}{\pgfqpoint{2.627103in}{1.813434in}}%
\pgfusepath{clip}%
\pgfsetbuttcap%
\pgfsetmiterjoin%
\definecolor{currentfill}{rgb}{0.950697,0.616649,0.428624}%
\pgfsetfillcolor{currentfill}%
\pgfsetlinewidth{0.000000pt}%
\definecolor{currentstroke}{rgb}{0.000000,0.000000,0.000000}%
\pgfsetstrokecolor{currentstroke}%
\pgfsetstrokeopacity{0.000000}%
\pgfsetdash{}{0pt}%
\pgfpathmoveto{\pgfqpoint{5.651949in}{1.492936in}}%
\pgfpathlineto{\pgfqpoint{5.660885in}{1.492936in}}%
\pgfpathlineto{\pgfqpoint{5.660885in}{1.441826in}}%
\pgfpathlineto{\pgfqpoint{5.651949in}{1.441826in}}%
\pgfpathlineto{\pgfqpoint{5.651949in}{1.492936in}}%
\pgfpathclose%
\pgfusepath{fill}%
\end{pgfscope}%
\begin{pgfscope}%
\pgfpathrectangle{\pgfqpoint{3.722897in}{0.857143in}}{\pgfqpoint{2.627103in}{1.813434in}}%
\pgfusepath{clip}%
\pgfsetbuttcap%
\pgfsetmiterjoin%
\definecolor{currentfill}{rgb}{0.950697,0.616649,0.428624}%
\pgfsetfillcolor{currentfill}%
\pgfsetlinewidth{0.000000pt}%
\definecolor{currentstroke}{rgb}{0.000000,0.000000,0.000000}%
\pgfsetstrokecolor{currentstroke}%
\pgfsetstrokeopacity{0.000000}%
\pgfsetdash{}{0pt}%
\pgfpathmoveto{\pgfqpoint{5.663119in}{1.496017in}}%
\pgfpathlineto{\pgfqpoint{5.672056in}{1.496017in}}%
\pgfpathlineto{\pgfqpoint{5.672056in}{1.435796in}}%
\pgfpathlineto{\pgfqpoint{5.663119in}{1.435796in}}%
\pgfpathlineto{\pgfqpoint{5.663119in}{1.496017in}}%
\pgfpathclose%
\pgfusepath{fill}%
\end{pgfscope}%
\begin{pgfscope}%
\pgfpathrectangle{\pgfqpoint{3.722897in}{0.857143in}}{\pgfqpoint{2.627103in}{1.813434in}}%
\pgfusepath{clip}%
\pgfsetbuttcap%
\pgfsetmiterjoin%
\definecolor{currentfill}{rgb}{0.950697,0.616649,0.428624}%
\pgfsetfillcolor{currentfill}%
\pgfsetlinewidth{0.000000pt}%
\definecolor{currentstroke}{rgb}{0.000000,0.000000,0.000000}%
\pgfsetstrokecolor{currentstroke}%
\pgfsetstrokeopacity{0.000000}%
\pgfsetdash{}{0pt}%
\pgfpathmoveto{\pgfqpoint{5.674290in}{1.510705in}}%
\pgfpathlineto{\pgfqpoint{5.683227in}{1.510705in}}%
\pgfpathlineto{\pgfqpoint{5.683227in}{1.464141in}}%
\pgfpathlineto{\pgfqpoint{5.674290in}{1.464141in}}%
\pgfpathlineto{\pgfqpoint{5.674290in}{1.510705in}}%
\pgfpathclose%
\pgfusepath{fill}%
\end{pgfscope}%
\begin{pgfscope}%
\pgfpathrectangle{\pgfqpoint{3.722897in}{0.857143in}}{\pgfqpoint{2.627103in}{1.813434in}}%
\pgfusepath{clip}%
\pgfsetbuttcap%
\pgfsetmiterjoin%
\definecolor{currentfill}{rgb}{0.950697,0.616649,0.428624}%
\pgfsetfillcolor{currentfill}%
\pgfsetlinewidth{0.000000pt}%
\definecolor{currentstroke}{rgb}{0.000000,0.000000,0.000000}%
\pgfsetstrokecolor{currentstroke}%
\pgfsetstrokeopacity{0.000000}%
\pgfsetdash{}{0pt}%
\pgfpathmoveto{\pgfqpoint{5.685461in}{1.527555in}}%
\pgfpathlineto{\pgfqpoint{5.694397in}{1.527555in}}%
\pgfpathlineto{\pgfqpoint{5.694397in}{1.498334in}}%
\pgfpathlineto{\pgfqpoint{5.685461in}{1.498334in}}%
\pgfpathlineto{\pgfqpoint{5.685461in}{1.527555in}}%
\pgfpathclose%
\pgfusepath{fill}%
\end{pgfscope}%
\begin{pgfscope}%
\pgfpathrectangle{\pgfqpoint{3.722897in}{0.857143in}}{\pgfqpoint{2.627103in}{1.813434in}}%
\pgfusepath{clip}%
\pgfsetbuttcap%
\pgfsetmiterjoin%
\definecolor{currentfill}{rgb}{0.950697,0.616649,0.428624}%
\pgfsetfillcolor{currentfill}%
\pgfsetlinewidth{0.000000pt}%
\definecolor{currentstroke}{rgb}{0.000000,0.000000,0.000000}%
\pgfsetstrokecolor{currentstroke}%
\pgfsetstrokeopacity{0.000000}%
\pgfsetdash{}{0pt}%
\pgfpathmoveto{\pgfqpoint{5.696631in}{1.526744in}}%
\pgfpathlineto{\pgfqpoint{5.705568in}{1.526744in}}%
\pgfpathlineto{\pgfqpoint{5.705568in}{1.505748in}}%
\pgfpathlineto{\pgfqpoint{5.696631in}{1.505748in}}%
\pgfpathlineto{\pgfqpoint{5.696631in}{1.526744in}}%
\pgfpathclose%
\pgfusepath{fill}%
\end{pgfscope}%
\begin{pgfscope}%
\pgfpathrectangle{\pgfqpoint{3.722897in}{0.857143in}}{\pgfqpoint{2.627103in}{1.813434in}}%
\pgfusepath{clip}%
\pgfsetbuttcap%
\pgfsetmiterjoin%
\definecolor{currentfill}{rgb}{0.950697,0.616649,0.428624}%
\pgfsetfillcolor{currentfill}%
\pgfsetlinewidth{0.000000pt}%
\definecolor{currentstroke}{rgb}{0.000000,0.000000,0.000000}%
\pgfsetstrokecolor{currentstroke}%
\pgfsetstrokeopacity{0.000000}%
\pgfsetdash{}{0pt}%
\pgfpathmoveto{\pgfqpoint{5.707802in}{1.518811in}}%
\pgfpathlineto{\pgfqpoint{5.716738in}{1.518811in}}%
\pgfpathlineto{\pgfqpoint{5.716738in}{1.502658in}}%
\pgfpathlineto{\pgfqpoint{5.707802in}{1.502658in}}%
\pgfpathlineto{\pgfqpoint{5.707802in}{1.518811in}}%
\pgfpathclose%
\pgfusepath{fill}%
\end{pgfscope}%
\begin{pgfscope}%
\pgfpathrectangle{\pgfqpoint{3.722897in}{0.857143in}}{\pgfqpoint{2.627103in}{1.813434in}}%
\pgfusepath{clip}%
\pgfsetbuttcap%
\pgfsetmiterjoin%
\definecolor{currentfill}{rgb}{0.950697,0.616649,0.428624}%
\pgfsetfillcolor{currentfill}%
\pgfsetlinewidth{0.000000pt}%
\definecolor{currentstroke}{rgb}{0.000000,0.000000,0.000000}%
\pgfsetstrokecolor{currentstroke}%
\pgfsetstrokeopacity{0.000000}%
\pgfsetdash{}{0pt}%
\pgfpathmoveto{\pgfqpoint{5.718972in}{2.126468in}}%
\pgfpathlineto{\pgfqpoint{5.727909in}{2.126468in}}%
\pgfpathlineto{\pgfqpoint{5.727909in}{2.156855in}}%
\pgfpathlineto{\pgfqpoint{5.718972in}{2.156855in}}%
\pgfpathlineto{\pgfqpoint{5.718972in}{2.126468in}}%
\pgfpathclose%
\pgfusepath{fill}%
\end{pgfscope}%
\begin{pgfscope}%
\pgfpathrectangle{\pgfqpoint{3.722897in}{0.857143in}}{\pgfqpoint{2.627103in}{1.813434in}}%
\pgfusepath{clip}%
\pgfsetbuttcap%
\pgfsetmiterjoin%
\definecolor{currentfill}{rgb}{0.950697,0.616649,0.428624}%
\pgfsetfillcolor{currentfill}%
\pgfsetlinewidth{0.000000pt}%
\definecolor{currentstroke}{rgb}{0.000000,0.000000,0.000000}%
\pgfsetstrokecolor{currentstroke}%
\pgfsetstrokeopacity{0.000000}%
\pgfsetdash{}{0pt}%
\pgfpathmoveto{\pgfqpoint{5.730143in}{2.107973in}}%
\pgfpathlineto{\pgfqpoint{5.739080in}{2.107973in}}%
\pgfpathlineto{\pgfqpoint{5.739080in}{2.136187in}}%
\pgfpathlineto{\pgfqpoint{5.730143in}{2.136187in}}%
\pgfpathlineto{\pgfqpoint{5.730143in}{2.107973in}}%
\pgfpathclose%
\pgfusepath{fill}%
\end{pgfscope}%
\begin{pgfscope}%
\pgfpathrectangle{\pgfqpoint{3.722897in}{0.857143in}}{\pgfqpoint{2.627103in}{1.813434in}}%
\pgfusepath{clip}%
\pgfsetbuttcap%
\pgfsetmiterjoin%
\definecolor{currentfill}{rgb}{0.950697,0.616649,0.428624}%
\pgfsetfillcolor{currentfill}%
\pgfsetlinewidth{0.000000pt}%
\definecolor{currentstroke}{rgb}{0.000000,0.000000,0.000000}%
\pgfsetstrokecolor{currentstroke}%
\pgfsetstrokeopacity{0.000000}%
\pgfsetdash{}{0pt}%
\pgfpathmoveto{\pgfqpoint{5.741314in}{2.093098in}}%
\pgfpathlineto{\pgfqpoint{5.750250in}{2.093098in}}%
\pgfpathlineto{\pgfqpoint{5.750250in}{2.100331in}}%
\pgfpathlineto{\pgfqpoint{5.741314in}{2.100331in}}%
\pgfpathlineto{\pgfqpoint{5.741314in}{2.093098in}}%
\pgfpathclose%
\pgfusepath{fill}%
\end{pgfscope}%
\begin{pgfscope}%
\pgfpathrectangle{\pgfqpoint{3.722897in}{0.857143in}}{\pgfqpoint{2.627103in}{1.813434in}}%
\pgfusepath{clip}%
\pgfsetbuttcap%
\pgfsetmiterjoin%
\definecolor{currentfill}{rgb}{0.950697,0.616649,0.428624}%
\pgfsetfillcolor{currentfill}%
\pgfsetlinewidth{0.000000pt}%
\definecolor{currentstroke}{rgb}{0.000000,0.000000,0.000000}%
\pgfsetstrokecolor{currentstroke}%
\pgfsetstrokeopacity{0.000000}%
\pgfsetdash{}{0pt}%
\pgfpathmoveto{\pgfqpoint{5.752484in}{2.065823in}}%
\pgfpathlineto{\pgfqpoint{5.761421in}{2.065823in}}%
\pgfpathlineto{\pgfqpoint{5.761421in}{2.068005in}}%
\pgfpathlineto{\pgfqpoint{5.752484in}{2.068005in}}%
\pgfpathlineto{\pgfqpoint{5.752484in}{2.065823in}}%
\pgfpathclose%
\pgfusepath{fill}%
\end{pgfscope}%
\begin{pgfscope}%
\pgfpathrectangle{\pgfqpoint{3.722897in}{0.857143in}}{\pgfqpoint{2.627103in}{1.813434in}}%
\pgfusepath{clip}%
\pgfsetbuttcap%
\pgfsetmiterjoin%
\definecolor{currentfill}{rgb}{0.950697,0.616649,0.428624}%
\pgfsetfillcolor{currentfill}%
\pgfsetlinewidth{0.000000pt}%
\definecolor{currentstroke}{rgb}{0.000000,0.000000,0.000000}%
\pgfsetstrokecolor{currentstroke}%
\pgfsetstrokeopacity{0.000000}%
\pgfsetdash{}{0pt}%
\pgfpathmoveto{\pgfqpoint{5.763655in}{2.027039in}}%
\pgfpathlineto{\pgfqpoint{5.772591in}{2.027039in}}%
\pgfpathlineto{\pgfqpoint{5.772591in}{2.028351in}}%
\pgfpathlineto{\pgfqpoint{5.763655in}{2.028351in}}%
\pgfpathlineto{\pgfqpoint{5.763655in}{2.027039in}}%
\pgfpathclose%
\pgfusepath{fill}%
\end{pgfscope}%
\begin{pgfscope}%
\pgfpathrectangle{\pgfqpoint{3.722897in}{0.857143in}}{\pgfqpoint{2.627103in}{1.813434in}}%
\pgfusepath{clip}%
\pgfsetbuttcap%
\pgfsetmiterjoin%
\definecolor{currentfill}{rgb}{0.950697,0.616649,0.428624}%
\pgfsetfillcolor{currentfill}%
\pgfsetlinewidth{0.000000pt}%
\definecolor{currentstroke}{rgb}{0.000000,0.000000,0.000000}%
\pgfsetstrokecolor{currentstroke}%
\pgfsetstrokeopacity{0.000000}%
\pgfsetdash{}{0pt}%
\pgfpathmoveto{\pgfqpoint{5.774826in}{1.457316in}}%
\pgfpathlineto{\pgfqpoint{5.783762in}{1.457316in}}%
\pgfpathlineto{\pgfqpoint{5.783762in}{1.449574in}}%
\pgfpathlineto{\pgfqpoint{5.774826in}{1.449574in}}%
\pgfpathlineto{\pgfqpoint{5.774826in}{1.457316in}}%
\pgfpathclose%
\pgfusepath{fill}%
\end{pgfscope}%
\begin{pgfscope}%
\pgfpathrectangle{\pgfqpoint{3.722897in}{0.857143in}}{\pgfqpoint{2.627103in}{1.813434in}}%
\pgfusepath{clip}%
\pgfsetbuttcap%
\pgfsetmiterjoin%
\definecolor{currentfill}{rgb}{0.950697,0.616649,0.428624}%
\pgfsetfillcolor{currentfill}%
\pgfsetlinewidth{0.000000pt}%
\definecolor{currentstroke}{rgb}{0.000000,0.000000,0.000000}%
\pgfsetstrokecolor{currentstroke}%
\pgfsetstrokeopacity{0.000000}%
\pgfsetdash{}{0pt}%
\pgfpathmoveto{\pgfqpoint{5.785996in}{1.448280in}}%
\pgfpathlineto{\pgfqpoint{5.794933in}{1.448280in}}%
\pgfpathlineto{\pgfqpoint{5.794933in}{1.433816in}}%
\pgfpathlineto{\pgfqpoint{5.785996in}{1.433816in}}%
\pgfpathlineto{\pgfqpoint{5.785996in}{1.448280in}}%
\pgfpathclose%
\pgfusepath{fill}%
\end{pgfscope}%
\begin{pgfscope}%
\pgfpathrectangle{\pgfqpoint{3.722897in}{0.857143in}}{\pgfqpoint{2.627103in}{1.813434in}}%
\pgfusepath{clip}%
\pgfsetbuttcap%
\pgfsetmiterjoin%
\definecolor{currentfill}{rgb}{0.950697,0.616649,0.428624}%
\pgfsetfillcolor{currentfill}%
\pgfsetlinewidth{0.000000pt}%
\definecolor{currentstroke}{rgb}{0.000000,0.000000,0.000000}%
\pgfsetstrokecolor{currentstroke}%
\pgfsetstrokeopacity{0.000000}%
\pgfsetdash{}{0pt}%
\pgfpathmoveto{\pgfqpoint{5.797167in}{1.440085in}}%
\pgfpathlineto{\pgfqpoint{5.806103in}{1.440085in}}%
\pgfpathlineto{\pgfqpoint{5.806103in}{1.423549in}}%
\pgfpathlineto{\pgfqpoint{5.797167in}{1.423549in}}%
\pgfpathlineto{\pgfqpoint{5.797167in}{1.440085in}}%
\pgfpathclose%
\pgfusepath{fill}%
\end{pgfscope}%
\begin{pgfscope}%
\pgfpathrectangle{\pgfqpoint{3.722897in}{0.857143in}}{\pgfqpoint{2.627103in}{1.813434in}}%
\pgfusepath{clip}%
\pgfsetbuttcap%
\pgfsetmiterjoin%
\definecolor{currentfill}{rgb}{0.950697,0.616649,0.428624}%
\pgfsetfillcolor{currentfill}%
\pgfsetlinewidth{0.000000pt}%
\definecolor{currentstroke}{rgb}{0.000000,0.000000,0.000000}%
\pgfsetstrokecolor{currentstroke}%
\pgfsetstrokeopacity{0.000000}%
\pgfsetdash{}{0pt}%
\pgfpathmoveto{\pgfqpoint{5.808337in}{1.429762in}}%
\pgfpathlineto{\pgfqpoint{5.817274in}{1.429762in}}%
\pgfpathlineto{\pgfqpoint{5.817274in}{1.398919in}}%
\pgfpathlineto{\pgfqpoint{5.808337in}{1.398919in}}%
\pgfpathlineto{\pgfqpoint{5.808337in}{1.429762in}}%
\pgfpathclose%
\pgfusepath{fill}%
\end{pgfscope}%
\begin{pgfscope}%
\pgfpathrectangle{\pgfqpoint{3.722897in}{0.857143in}}{\pgfqpoint{2.627103in}{1.813434in}}%
\pgfusepath{clip}%
\pgfsetbuttcap%
\pgfsetmiterjoin%
\definecolor{currentfill}{rgb}{0.950697,0.616649,0.428624}%
\pgfsetfillcolor{currentfill}%
\pgfsetlinewidth{0.000000pt}%
\definecolor{currentstroke}{rgb}{0.000000,0.000000,0.000000}%
\pgfsetstrokecolor{currentstroke}%
\pgfsetstrokeopacity{0.000000}%
\pgfsetdash{}{0pt}%
\pgfpathmoveto{\pgfqpoint{5.819508in}{1.429740in}}%
\pgfpathlineto{\pgfqpoint{5.828444in}{1.429740in}}%
\pgfpathlineto{\pgfqpoint{5.828444in}{1.396028in}}%
\pgfpathlineto{\pgfqpoint{5.819508in}{1.396028in}}%
\pgfpathlineto{\pgfqpoint{5.819508in}{1.429740in}}%
\pgfpathclose%
\pgfusepath{fill}%
\end{pgfscope}%
\begin{pgfscope}%
\pgfpathrectangle{\pgfqpoint{3.722897in}{0.857143in}}{\pgfqpoint{2.627103in}{1.813434in}}%
\pgfusepath{clip}%
\pgfsetbuttcap%
\pgfsetmiterjoin%
\definecolor{currentfill}{rgb}{0.950697,0.616649,0.428624}%
\pgfsetfillcolor{currentfill}%
\pgfsetlinewidth{0.000000pt}%
\definecolor{currentstroke}{rgb}{0.000000,0.000000,0.000000}%
\pgfsetstrokecolor{currentstroke}%
\pgfsetstrokeopacity{0.000000}%
\pgfsetdash{}{0pt}%
\pgfpathmoveto{\pgfqpoint{5.830679in}{1.400096in}}%
\pgfpathlineto{\pgfqpoint{5.839615in}{1.400096in}}%
\pgfpathlineto{\pgfqpoint{5.839615in}{1.351532in}}%
\pgfpathlineto{\pgfqpoint{5.830679in}{1.351532in}}%
\pgfpathlineto{\pgfqpoint{5.830679in}{1.400096in}}%
\pgfpathclose%
\pgfusepath{fill}%
\end{pgfscope}%
\begin{pgfscope}%
\pgfpathrectangle{\pgfqpoint{3.722897in}{0.857143in}}{\pgfqpoint{2.627103in}{1.813434in}}%
\pgfusepath{clip}%
\pgfsetbuttcap%
\pgfsetmiterjoin%
\definecolor{currentfill}{rgb}{0.950697,0.616649,0.428624}%
\pgfsetfillcolor{currentfill}%
\pgfsetlinewidth{0.000000pt}%
\definecolor{currentstroke}{rgb}{0.000000,0.000000,0.000000}%
\pgfsetstrokecolor{currentstroke}%
\pgfsetstrokeopacity{0.000000}%
\pgfsetdash{}{0pt}%
\pgfpathmoveto{\pgfqpoint{5.841849in}{1.351820in}}%
\pgfpathlineto{\pgfqpoint{5.850786in}{1.351820in}}%
\pgfpathlineto{\pgfqpoint{5.850786in}{1.303341in}}%
\pgfpathlineto{\pgfqpoint{5.841849in}{1.303341in}}%
\pgfpathlineto{\pgfqpoint{5.841849in}{1.351820in}}%
\pgfpathclose%
\pgfusepath{fill}%
\end{pgfscope}%
\begin{pgfscope}%
\pgfpathrectangle{\pgfqpoint{3.722897in}{0.857143in}}{\pgfqpoint{2.627103in}{1.813434in}}%
\pgfusepath{clip}%
\pgfsetbuttcap%
\pgfsetmiterjoin%
\definecolor{currentfill}{rgb}{0.950697,0.616649,0.428624}%
\pgfsetfillcolor{currentfill}%
\pgfsetlinewidth{0.000000pt}%
\definecolor{currentstroke}{rgb}{0.000000,0.000000,0.000000}%
\pgfsetstrokecolor{currentstroke}%
\pgfsetstrokeopacity{0.000000}%
\pgfsetdash{}{0pt}%
\pgfpathmoveto{\pgfqpoint{5.853020in}{1.312009in}}%
\pgfpathlineto{\pgfqpoint{5.861956in}{1.312009in}}%
\pgfpathlineto{\pgfqpoint{5.861956in}{1.269106in}}%
\pgfpathlineto{\pgfqpoint{5.853020in}{1.269106in}}%
\pgfpathlineto{\pgfqpoint{5.853020in}{1.312009in}}%
\pgfpathclose%
\pgfusepath{fill}%
\end{pgfscope}%
\begin{pgfscope}%
\pgfpathrectangle{\pgfqpoint{3.722897in}{0.857143in}}{\pgfqpoint{2.627103in}{1.813434in}}%
\pgfusepath{clip}%
\pgfsetbuttcap%
\pgfsetmiterjoin%
\definecolor{currentfill}{rgb}{0.950697,0.616649,0.428624}%
\pgfsetfillcolor{currentfill}%
\pgfsetlinewidth{0.000000pt}%
\definecolor{currentstroke}{rgb}{0.000000,0.000000,0.000000}%
\pgfsetstrokecolor{currentstroke}%
\pgfsetstrokeopacity{0.000000}%
\pgfsetdash{}{0pt}%
\pgfpathmoveto{\pgfqpoint{5.864190in}{1.282282in}}%
\pgfpathlineto{\pgfqpoint{5.873127in}{1.282282in}}%
\pgfpathlineto{\pgfqpoint{5.873127in}{1.221002in}}%
\pgfpathlineto{\pgfqpoint{5.864190in}{1.221002in}}%
\pgfpathlineto{\pgfqpoint{5.864190in}{1.282282in}}%
\pgfpathclose%
\pgfusepath{fill}%
\end{pgfscope}%
\begin{pgfscope}%
\pgfpathrectangle{\pgfqpoint{3.722897in}{0.857143in}}{\pgfqpoint{2.627103in}{1.813434in}}%
\pgfusepath{clip}%
\pgfsetbuttcap%
\pgfsetmiterjoin%
\definecolor{currentfill}{rgb}{0.950697,0.616649,0.428624}%
\pgfsetfillcolor{currentfill}%
\pgfsetlinewidth{0.000000pt}%
\definecolor{currentstroke}{rgb}{0.000000,0.000000,0.000000}%
\pgfsetstrokecolor{currentstroke}%
\pgfsetstrokeopacity{0.000000}%
\pgfsetdash{}{0pt}%
\pgfpathmoveto{\pgfqpoint{5.875361in}{1.257543in}}%
\pgfpathlineto{\pgfqpoint{5.884297in}{1.257543in}}%
\pgfpathlineto{\pgfqpoint{5.884297in}{1.201204in}}%
\pgfpathlineto{\pgfqpoint{5.875361in}{1.201204in}}%
\pgfpathlineto{\pgfqpoint{5.875361in}{1.257543in}}%
\pgfpathclose%
\pgfusepath{fill}%
\end{pgfscope}%
\begin{pgfscope}%
\pgfpathrectangle{\pgfqpoint{3.722897in}{0.857143in}}{\pgfqpoint{2.627103in}{1.813434in}}%
\pgfusepath{clip}%
\pgfsetbuttcap%
\pgfsetmiterjoin%
\definecolor{currentfill}{rgb}{0.950697,0.616649,0.428624}%
\pgfsetfillcolor{currentfill}%
\pgfsetlinewidth{0.000000pt}%
\definecolor{currentstroke}{rgb}{0.000000,0.000000,0.000000}%
\pgfsetstrokecolor{currentstroke}%
\pgfsetstrokeopacity{0.000000}%
\pgfsetdash{}{0pt}%
\pgfpathmoveto{\pgfqpoint{5.886532in}{1.246848in}}%
\pgfpathlineto{\pgfqpoint{5.895468in}{1.246848in}}%
\pgfpathlineto{\pgfqpoint{5.895468in}{1.171726in}}%
\pgfpathlineto{\pgfqpoint{5.886532in}{1.171726in}}%
\pgfpathlineto{\pgfqpoint{5.886532in}{1.246848in}}%
\pgfpathclose%
\pgfusepath{fill}%
\end{pgfscope}%
\begin{pgfscope}%
\pgfpathrectangle{\pgfqpoint{3.722897in}{0.857143in}}{\pgfqpoint{2.627103in}{1.813434in}}%
\pgfusepath{clip}%
\pgfsetbuttcap%
\pgfsetmiterjoin%
\definecolor{currentfill}{rgb}{0.950697,0.616649,0.428624}%
\pgfsetfillcolor{currentfill}%
\pgfsetlinewidth{0.000000pt}%
\definecolor{currentstroke}{rgb}{0.000000,0.000000,0.000000}%
\pgfsetstrokecolor{currentstroke}%
\pgfsetstrokeopacity{0.000000}%
\pgfsetdash{}{0pt}%
\pgfpathmoveto{\pgfqpoint{5.897702in}{1.229741in}}%
\pgfpathlineto{\pgfqpoint{5.906639in}{1.229741in}}%
\pgfpathlineto{\pgfqpoint{5.906639in}{1.151539in}}%
\pgfpathlineto{\pgfqpoint{5.897702in}{1.151539in}}%
\pgfpathlineto{\pgfqpoint{5.897702in}{1.229741in}}%
\pgfpathclose%
\pgfusepath{fill}%
\end{pgfscope}%
\begin{pgfscope}%
\pgfpathrectangle{\pgfqpoint{3.722897in}{0.857143in}}{\pgfqpoint{2.627103in}{1.813434in}}%
\pgfusepath{clip}%
\pgfsetbuttcap%
\pgfsetmiterjoin%
\definecolor{currentfill}{rgb}{0.950697,0.616649,0.428624}%
\pgfsetfillcolor{currentfill}%
\pgfsetlinewidth{0.000000pt}%
\definecolor{currentstroke}{rgb}{0.000000,0.000000,0.000000}%
\pgfsetstrokecolor{currentstroke}%
\pgfsetstrokeopacity{0.000000}%
\pgfsetdash{}{0pt}%
\pgfpathmoveto{\pgfqpoint{5.908873in}{1.219698in}}%
\pgfpathlineto{\pgfqpoint{5.917809in}{1.219698in}}%
\pgfpathlineto{\pgfqpoint{5.917809in}{1.140440in}}%
\pgfpathlineto{\pgfqpoint{5.908873in}{1.140440in}}%
\pgfpathlineto{\pgfqpoint{5.908873in}{1.219698in}}%
\pgfpathclose%
\pgfusepath{fill}%
\end{pgfscope}%
\begin{pgfscope}%
\pgfpathrectangle{\pgfqpoint{3.722897in}{0.857143in}}{\pgfqpoint{2.627103in}{1.813434in}}%
\pgfusepath{clip}%
\pgfsetbuttcap%
\pgfsetmiterjoin%
\definecolor{currentfill}{rgb}{0.950697,0.616649,0.428624}%
\pgfsetfillcolor{currentfill}%
\pgfsetlinewidth{0.000000pt}%
\definecolor{currentstroke}{rgb}{0.000000,0.000000,0.000000}%
\pgfsetstrokecolor{currentstroke}%
\pgfsetstrokeopacity{0.000000}%
\pgfsetdash{}{0pt}%
\pgfpathmoveto{\pgfqpoint{5.920043in}{1.221602in}}%
\pgfpathlineto{\pgfqpoint{5.928980in}{1.221602in}}%
\pgfpathlineto{\pgfqpoint{5.928980in}{1.123359in}}%
\pgfpathlineto{\pgfqpoint{5.920043in}{1.123359in}}%
\pgfpathlineto{\pgfqpoint{5.920043in}{1.221602in}}%
\pgfpathclose%
\pgfusepath{fill}%
\end{pgfscope}%
\begin{pgfscope}%
\pgfpathrectangle{\pgfqpoint{3.722897in}{0.857143in}}{\pgfqpoint{2.627103in}{1.813434in}}%
\pgfusepath{clip}%
\pgfsetbuttcap%
\pgfsetmiterjoin%
\definecolor{currentfill}{rgb}{0.950697,0.616649,0.428624}%
\pgfsetfillcolor{currentfill}%
\pgfsetlinewidth{0.000000pt}%
\definecolor{currentstroke}{rgb}{0.000000,0.000000,0.000000}%
\pgfsetstrokecolor{currentstroke}%
\pgfsetstrokeopacity{0.000000}%
\pgfsetdash{}{0pt}%
\pgfpathmoveto{\pgfqpoint{5.931214in}{1.227719in}}%
\pgfpathlineto{\pgfqpoint{5.940150in}{1.227719in}}%
\pgfpathlineto{\pgfqpoint{5.940150in}{1.108905in}}%
\pgfpathlineto{\pgfqpoint{5.931214in}{1.108905in}}%
\pgfpathlineto{\pgfqpoint{5.931214in}{1.227719in}}%
\pgfpathclose%
\pgfusepath{fill}%
\end{pgfscope}%
\begin{pgfscope}%
\pgfpathrectangle{\pgfqpoint{3.722897in}{0.857143in}}{\pgfqpoint{2.627103in}{1.813434in}}%
\pgfusepath{clip}%
\pgfsetbuttcap%
\pgfsetmiterjoin%
\definecolor{currentfill}{rgb}{0.950697,0.616649,0.428624}%
\pgfsetfillcolor{currentfill}%
\pgfsetlinewidth{0.000000pt}%
\definecolor{currentstroke}{rgb}{0.000000,0.000000,0.000000}%
\pgfsetstrokecolor{currentstroke}%
\pgfsetstrokeopacity{0.000000}%
\pgfsetdash{}{0pt}%
\pgfpathmoveto{\pgfqpoint{5.942385in}{1.227515in}}%
\pgfpathlineto{\pgfqpoint{5.951321in}{1.227515in}}%
\pgfpathlineto{\pgfqpoint{5.951321in}{1.112458in}}%
\pgfpathlineto{\pgfqpoint{5.942385in}{1.112458in}}%
\pgfpathlineto{\pgfqpoint{5.942385in}{1.227515in}}%
\pgfpathclose%
\pgfusepath{fill}%
\end{pgfscope}%
\begin{pgfscope}%
\pgfpathrectangle{\pgfqpoint{3.722897in}{0.857143in}}{\pgfqpoint{2.627103in}{1.813434in}}%
\pgfusepath{clip}%
\pgfsetbuttcap%
\pgfsetmiterjoin%
\definecolor{currentfill}{rgb}{0.950697,0.616649,0.428624}%
\pgfsetfillcolor{currentfill}%
\pgfsetlinewidth{0.000000pt}%
\definecolor{currentstroke}{rgb}{0.000000,0.000000,0.000000}%
\pgfsetstrokecolor{currentstroke}%
\pgfsetstrokeopacity{0.000000}%
\pgfsetdash{}{0pt}%
\pgfpathmoveto{\pgfqpoint{5.953555in}{1.230415in}}%
\pgfpathlineto{\pgfqpoint{5.962492in}{1.230415in}}%
\pgfpathlineto{\pgfqpoint{5.962492in}{1.126436in}}%
\pgfpathlineto{\pgfqpoint{5.953555in}{1.126436in}}%
\pgfpathlineto{\pgfqpoint{5.953555in}{1.230415in}}%
\pgfpathclose%
\pgfusepath{fill}%
\end{pgfscope}%
\begin{pgfscope}%
\pgfpathrectangle{\pgfqpoint{3.722897in}{0.857143in}}{\pgfqpoint{2.627103in}{1.813434in}}%
\pgfusepath{clip}%
\pgfsetbuttcap%
\pgfsetmiterjoin%
\definecolor{currentfill}{rgb}{0.950697,0.616649,0.428624}%
\pgfsetfillcolor{currentfill}%
\pgfsetlinewidth{0.000000pt}%
\definecolor{currentstroke}{rgb}{0.000000,0.000000,0.000000}%
\pgfsetstrokecolor{currentstroke}%
\pgfsetstrokeopacity{0.000000}%
\pgfsetdash{}{0pt}%
\pgfpathmoveto{\pgfqpoint{5.964726in}{1.246414in}}%
\pgfpathlineto{\pgfqpoint{5.973662in}{1.246414in}}%
\pgfpathlineto{\pgfqpoint{5.973662in}{1.130625in}}%
\pgfpathlineto{\pgfqpoint{5.964726in}{1.130625in}}%
\pgfpathlineto{\pgfqpoint{5.964726in}{1.246414in}}%
\pgfpathclose%
\pgfusepath{fill}%
\end{pgfscope}%
\begin{pgfscope}%
\pgfpathrectangle{\pgfqpoint{3.722897in}{0.857143in}}{\pgfqpoint{2.627103in}{1.813434in}}%
\pgfusepath{clip}%
\pgfsetbuttcap%
\pgfsetmiterjoin%
\definecolor{currentfill}{rgb}{0.950697,0.616649,0.428624}%
\pgfsetfillcolor{currentfill}%
\pgfsetlinewidth{0.000000pt}%
\definecolor{currentstroke}{rgb}{0.000000,0.000000,0.000000}%
\pgfsetstrokecolor{currentstroke}%
\pgfsetstrokeopacity{0.000000}%
\pgfsetdash{}{0pt}%
\pgfpathmoveto{\pgfqpoint{5.975896in}{1.248562in}}%
\pgfpathlineto{\pgfqpoint{5.984833in}{1.248562in}}%
\pgfpathlineto{\pgfqpoint{5.984833in}{1.117168in}}%
\pgfpathlineto{\pgfqpoint{5.975896in}{1.117168in}}%
\pgfpathlineto{\pgfqpoint{5.975896in}{1.248562in}}%
\pgfpathclose%
\pgfusepath{fill}%
\end{pgfscope}%
\begin{pgfscope}%
\pgfpathrectangle{\pgfqpoint{3.722897in}{0.857143in}}{\pgfqpoint{2.627103in}{1.813434in}}%
\pgfusepath{clip}%
\pgfsetbuttcap%
\pgfsetmiterjoin%
\definecolor{currentfill}{rgb}{0.950697,0.616649,0.428624}%
\pgfsetfillcolor{currentfill}%
\pgfsetlinewidth{0.000000pt}%
\definecolor{currentstroke}{rgb}{0.000000,0.000000,0.000000}%
\pgfsetstrokecolor{currentstroke}%
\pgfsetstrokeopacity{0.000000}%
\pgfsetdash{}{0pt}%
\pgfpathmoveto{\pgfqpoint{5.987067in}{1.258844in}}%
\pgfpathlineto{\pgfqpoint{5.996004in}{1.258844in}}%
\pgfpathlineto{\pgfqpoint{5.996004in}{1.127294in}}%
\pgfpathlineto{\pgfqpoint{5.987067in}{1.127294in}}%
\pgfpathlineto{\pgfqpoint{5.987067in}{1.258844in}}%
\pgfpathclose%
\pgfusepath{fill}%
\end{pgfscope}%
\begin{pgfscope}%
\pgfpathrectangle{\pgfqpoint{3.722897in}{0.857143in}}{\pgfqpoint{2.627103in}{1.813434in}}%
\pgfusepath{clip}%
\pgfsetbuttcap%
\pgfsetmiterjoin%
\definecolor{currentfill}{rgb}{0.950697,0.616649,0.428624}%
\pgfsetfillcolor{currentfill}%
\pgfsetlinewidth{0.000000pt}%
\definecolor{currentstroke}{rgb}{0.000000,0.000000,0.000000}%
\pgfsetstrokecolor{currentstroke}%
\pgfsetstrokeopacity{0.000000}%
\pgfsetdash{}{0pt}%
\pgfpathmoveto{\pgfqpoint{5.998238in}{1.281630in}}%
\pgfpathlineto{\pgfqpoint{6.007174in}{1.281630in}}%
\pgfpathlineto{\pgfqpoint{6.007174in}{1.143854in}}%
\pgfpathlineto{\pgfqpoint{5.998238in}{1.143854in}}%
\pgfpathlineto{\pgfqpoint{5.998238in}{1.281630in}}%
\pgfpathclose%
\pgfusepath{fill}%
\end{pgfscope}%
\begin{pgfscope}%
\pgfpathrectangle{\pgfqpoint{3.722897in}{0.857143in}}{\pgfqpoint{2.627103in}{1.813434in}}%
\pgfusepath{clip}%
\pgfsetbuttcap%
\pgfsetmiterjoin%
\definecolor{currentfill}{rgb}{0.950697,0.616649,0.428624}%
\pgfsetfillcolor{currentfill}%
\pgfsetlinewidth{0.000000pt}%
\definecolor{currentstroke}{rgb}{0.000000,0.000000,0.000000}%
\pgfsetstrokecolor{currentstroke}%
\pgfsetstrokeopacity{0.000000}%
\pgfsetdash{}{0pt}%
\pgfpathmoveto{\pgfqpoint{6.009408in}{1.302510in}}%
\pgfpathlineto{\pgfqpoint{6.018345in}{1.302510in}}%
\pgfpathlineto{\pgfqpoint{6.018345in}{1.148349in}}%
\pgfpathlineto{\pgfqpoint{6.009408in}{1.148349in}}%
\pgfpathlineto{\pgfqpoint{6.009408in}{1.302510in}}%
\pgfpathclose%
\pgfusepath{fill}%
\end{pgfscope}%
\begin{pgfscope}%
\pgfpathrectangle{\pgfqpoint{3.722897in}{0.857143in}}{\pgfqpoint{2.627103in}{1.813434in}}%
\pgfusepath{clip}%
\pgfsetbuttcap%
\pgfsetmiterjoin%
\definecolor{currentfill}{rgb}{0.950697,0.616649,0.428624}%
\pgfsetfillcolor{currentfill}%
\pgfsetlinewidth{0.000000pt}%
\definecolor{currentstroke}{rgb}{0.000000,0.000000,0.000000}%
\pgfsetstrokecolor{currentstroke}%
\pgfsetstrokeopacity{0.000000}%
\pgfsetdash{}{0pt}%
\pgfpathmoveto{\pgfqpoint{6.020579in}{1.324205in}}%
\pgfpathlineto{\pgfqpoint{6.029515in}{1.324205in}}%
\pgfpathlineto{\pgfqpoint{6.029515in}{1.182667in}}%
\pgfpathlineto{\pgfqpoint{6.020579in}{1.182667in}}%
\pgfpathlineto{\pgfqpoint{6.020579in}{1.324205in}}%
\pgfpathclose%
\pgfusepath{fill}%
\end{pgfscope}%
\begin{pgfscope}%
\pgfpathrectangle{\pgfqpoint{3.722897in}{0.857143in}}{\pgfqpoint{2.627103in}{1.813434in}}%
\pgfusepath{clip}%
\pgfsetbuttcap%
\pgfsetmiterjoin%
\definecolor{currentfill}{rgb}{0.950697,0.616649,0.428624}%
\pgfsetfillcolor{currentfill}%
\pgfsetlinewidth{0.000000pt}%
\definecolor{currentstroke}{rgb}{0.000000,0.000000,0.000000}%
\pgfsetstrokecolor{currentstroke}%
\pgfsetstrokeopacity{0.000000}%
\pgfsetdash{}{0pt}%
\pgfpathmoveto{\pgfqpoint{6.031749in}{1.340281in}}%
\pgfpathlineto{\pgfqpoint{6.040686in}{1.340281in}}%
\pgfpathlineto{\pgfqpoint{6.040686in}{1.195002in}}%
\pgfpathlineto{\pgfqpoint{6.031749in}{1.195002in}}%
\pgfpathlineto{\pgfqpoint{6.031749in}{1.340281in}}%
\pgfpathclose%
\pgfusepath{fill}%
\end{pgfscope}%
\begin{pgfscope}%
\pgfpathrectangle{\pgfqpoint{3.722897in}{0.857143in}}{\pgfqpoint{2.627103in}{1.813434in}}%
\pgfusepath{clip}%
\pgfsetbuttcap%
\pgfsetmiterjoin%
\definecolor{currentfill}{rgb}{0.950697,0.616649,0.428624}%
\pgfsetfillcolor{currentfill}%
\pgfsetlinewidth{0.000000pt}%
\definecolor{currentstroke}{rgb}{0.000000,0.000000,0.000000}%
\pgfsetstrokecolor{currentstroke}%
\pgfsetstrokeopacity{0.000000}%
\pgfsetdash{}{0pt}%
\pgfpathmoveto{\pgfqpoint{6.042920in}{1.329028in}}%
\pgfpathlineto{\pgfqpoint{6.051857in}{1.329028in}}%
\pgfpathlineto{\pgfqpoint{6.051857in}{1.198105in}}%
\pgfpathlineto{\pgfqpoint{6.042920in}{1.198105in}}%
\pgfpathlineto{\pgfqpoint{6.042920in}{1.329028in}}%
\pgfpathclose%
\pgfusepath{fill}%
\end{pgfscope}%
\begin{pgfscope}%
\pgfpathrectangle{\pgfqpoint{3.722897in}{0.857143in}}{\pgfqpoint{2.627103in}{1.813434in}}%
\pgfusepath{clip}%
\pgfsetbuttcap%
\pgfsetmiterjoin%
\definecolor{currentfill}{rgb}{0.950697,0.616649,0.428624}%
\pgfsetfillcolor{currentfill}%
\pgfsetlinewidth{0.000000pt}%
\definecolor{currentstroke}{rgb}{0.000000,0.000000,0.000000}%
\pgfsetstrokecolor{currentstroke}%
\pgfsetstrokeopacity{0.000000}%
\pgfsetdash{}{0pt}%
\pgfpathmoveto{\pgfqpoint{6.054091in}{1.323100in}}%
\pgfpathlineto{\pgfqpoint{6.063027in}{1.323100in}}%
\pgfpathlineto{\pgfqpoint{6.063027in}{1.176281in}}%
\pgfpathlineto{\pgfqpoint{6.054091in}{1.176281in}}%
\pgfpathlineto{\pgfqpoint{6.054091in}{1.323100in}}%
\pgfpathclose%
\pgfusepath{fill}%
\end{pgfscope}%
\begin{pgfscope}%
\pgfpathrectangle{\pgfqpoint{3.722897in}{0.857143in}}{\pgfqpoint{2.627103in}{1.813434in}}%
\pgfusepath{clip}%
\pgfsetbuttcap%
\pgfsetmiterjoin%
\definecolor{currentfill}{rgb}{0.950697,0.616649,0.428624}%
\pgfsetfillcolor{currentfill}%
\pgfsetlinewidth{0.000000pt}%
\definecolor{currentstroke}{rgb}{0.000000,0.000000,0.000000}%
\pgfsetstrokecolor{currentstroke}%
\pgfsetstrokeopacity{0.000000}%
\pgfsetdash{}{0pt}%
\pgfpathmoveto{\pgfqpoint{6.065261in}{1.312960in}}%
\pgfpathlineto{\pgfqpoint{6.074198in}{1.312960in}}%
\pgfpathlineto{\pgfqpoint{6.074198in}{1.175053in}}%
\pgfpathlineto{\pgfqpoint{6.065261in}{1.175053in}}%
\pgfpathlineto{\pgfqpoint{6.065261in}{1.312960in}}%
\pgfpathclose%
\pgfusepath{fill}%
\end{pgfscope}%
\begin{pgfscope}%
\pgfpathrectangle{\pgfqpoint{3.722897in}{0.857143in}}{\pgfqpoint{2.627103in}{1.813434in}}%
\pgfusepath{clip}%
\pgfsetbuttcap%
\pgfsetmiterjoin%
\definecolor{currentfill}{rgb}{0.950697,0.616649,0.428624}%
\pgfsetfillcolor{currentfill}%
\pgfsetlinewidth{0.000000pt}%
\definecolor{currentstroke}{rgb}{0.000000,0.000000,0.000000}%
\pgfsetstrokecolor{currentstroke}%
\pgfsetstrokeopacity{0.000000}%
\pgfsetdash{}{0pt}%
\pgfpathmoveto{\pgfqpoint{6.076432in}{1.297680in}}%
\pgfpathlineto{\pgfqpoint{6.085368in}{1.297680in}}%
\pgfpathlineto{\pgfqpoint{6.085368in}{1.159739in}}%
\pgfpathlineto{\pgfqpoint{6.076432in}{1.159739in}}%
\pgfpathlineto{\pgfqpoint{6.076432in}{1.297680in}}%
\pgfpathclose%
\pgfusepath{fill}%
\end{pgfscope}%
\begin{pgfscope}%
\pgfpathrectangle{\pgfqpoint{3.722897in}{0.857143in}}{\pgfqpoint{2.627103in}{1.813434in}}%
\pgfusepath{clip}%
\pgfsetbuttcap%
\pgfsetmiterjoin%
\definecolor{currentfill}{rgb}{0.950697,0.616649,0.428624}%
\pgfsetfillcolor{currentfill}%
\pgfsetlinewidth{0.000000pt}%
\definecolor{currentstroke}{rgb}{0.000000,0.000000,0.000000}%
\pgfsetstrokecolor{currentstroke}%
\pgfsetstrokeopacity{0.000000}%
\pgfsetdash{}{0pt}%
\pgfpathmoveto{\pgfqpoint{6.087602in}{1.276908in}}%
\pgfpathlineto{\pgfqpoint{6.096539in}{1.276908in}}%
\pgfpathlineto{\pgfqpoint{6.096539in}{1.154001in}}%
\pgfpathlineto{\pgfqpoint{6.087602in}{1.154001in}}%
\pgfpathlineto{\pgfqpoint{6.087602in}{1.276908in}}%
\pgfpathclose%
\pgfusepath{fill}%
\end{pgfscope}%
\begin{pgfscope}%
\pgfpathrectangle{\pgfqpoint{3.722897in}{0.857143in}}{\pgfqpoint{2.627103in}{1.813434in}}%
\pgfusepath{clip}%
\pgfsetbuttcap%
\pgfsetmiterjoin%
\definecolor{currentfill}{rgb}{0.950697,0.616649,0.428624}%
\pgfsetfillcolor{currentfill}%
\pgfsetlinewidth{0.000000pt}%
\definecolor{currentstroke}{rgb}{0.000000,0.000000,0.000000}%
\pgfsetstrokecolor{currentstroke}%
\pgfsetstrokeopacity{0.000000}%
\pgfsetdash{}{0pt}%
\pgfpathmoveto{\pgfqpoint{6.098773in}{1.258402in}}%
\pgfpathlineto{\pgfqpoint{6.107710in}{1.258402in}}%
\pgfpathlineto{\pgfqpoint{6.107710in}{1.127056in}}%
\pgfpathlineto{\pgfqpoint{6.098773in}{1.127056in}}%
\pgfpathlineto{\pgfqpoint{6.098773in}{1.258402in}}%
\pgfpathclose%
\pgfusepath{fill}%
\end{pgfscope}%
\begin{pgfscope}%
\pgfpathrectangle{\pgfqpoint{3.722897in}{0.857143in}}{\pgfqpoint{2.627103in}{1.813434in}}%
\pgfusepath{clip}%
\pgfsetbuttcap%
\pgfsetmiterjoin%
\definecolor{currentfill}{rgb}{0.950697,0.616649,0.428624}%
\pgfsetfillcolor{currentfill}%
\pgfsetlinewidth{0.000000pt}%
\definecolor{currentstroke}{rgb}{0.000000,0.000000,0.000000}%
\pgfsetstrokecolor{currentstroke}%
\pgfsetstrokeopacity{0.000000}%
\pgfsetdash{}{0pt}%
\pgfpathmoveto{\pgfqpoint{6.109944in}{1.253246in}}%
\pgfpathlineto{\pgfqpoint{6.118880in}{1.253246in}}%
\pgfpathlineto{\pgfqpoint{6.118880in}{1.124271in}}%
\pgfpathlineto{\pgfqpoint{6.109944in}{1.124271in}}%
\pgfpathlineto{\pgfqpoint{6.109944in}{1.253246in}}%
\pgfpathclose%
\pgfusepath{fill}%
\end{pgfscope}%
\begin{pgfscope}%
\pgfpathrectangle{\pgfqpoint{3.722897in}{0.857143in}}{\pgfqpoint{2.627103in}{1.813434in}}%
\pgfusepath{clip}%
\pgfsetbuttcap%
\pgfsetmiterjoin%
\definecolor{currentfill}{rgb}{0.950697,0.616649,0.428624}%
\pgfsetfillcolor{currentfill}%
\pgfsetlinewidth{0.000000pt}%
\definecolor{currentstroke}{rgb}{0.000000,0.000000,0.000000}%
\pgfsetstrokecolor{currentstroke}%
\pgfsetstrokeopacity{0.000000}%
\pgfsetdash{}{0pt}%
\pgfpathmoveto{\pgfqpoint{6.121114in}{1.234586in}}%
\pgfpathlineto{\pgfqpoint{6.130051in}{1.234586in}}%
\pgfpathlineto{\pgfqpoint{6.130051in}{1.126921in}}%
\pgfpathlineto{\pgfqpoint{6.121114in}{1.126921in}}%
\pgfpathlineto{\pgfqpoint{6.121114in}{1.234586in}}%
\pgfpathclose%
\pgfusepath{fill}%
\end{pgfscope}%
\begin{pgfscope}%
\pgfpathrectangle{\pgfqpoint{3.722897in}{0.857143in}}{\pgfqpoint{2.627103in}{1.813434in}}%
\pgfusepath{clip}%
\pgfsetbuttcap%
\pgfsetmiterjoin%
\definecolor{currentfill}{rgb}{0.950697,0.616649,0.428624}%
\pgfsetfillcolor{currentfill}%
\pgfsetlinewidth{0.000000pt}%
\definecolor{currentstroke}{rgb}{0.000000,0.000000,0.000000}%
\pgfsetstrokecolor{currentstroke}%
\pgfsetstrokeopacity{0.000000}%
\pgfsetdash{}{0pt}%
\pgfpathmoveto{\pgfqpoint{6.132285in}{1.214575in}}%
\pgfpathlineto{\pgfqpoint{6.141221in}{1.214575in}}%
\pgfpathlineto{\pgfqpoint{6.141221in}{1.111252in}}%
\pgfpathlineto{\pgfqpoint{6.132285in}{1.111252in}}%
\pgfpathlineto{\pgfqpoint{6.132285in}{1.214575in}}%
\pgfpathclose%
\pgfusepath{fill}%
\end{pgfscope}%
\begin{pgfscope}%
\pgfpathrectangle{\pgfqpoint{3.722897in}{0.857143in}}{\pgfqpoint{2.627103in}{1.813434in}}%
\pgfusepath{clip}%
\pgfsetbuttcap%
\pgfsetmiterjoin%
\definecolor{currentfill}{rgb}{0.950697,0.616649,0.428624}%
\pgfsetfillcolor{currentfill}%
\pgfsetlinewidth{0.000000pt}%
\definecolor{currentstroke}{rgb}{0.000000,0.000000,0.000000}%
\pgfsetstrokecolor{currentstroke}%
\pgfsetstrokeopacity{0.000000}%
\pgfsetdash{}{0pt}%
\pgfpathmoveto{\pgfqpoint{6.143456in}{1.203622in}}%
\pgfpathlineto{\pgfqpoint{6.152392in}{1.203622in}}%
\pgfpathlineto{\pgfqpoint{6.152392in}{1.108988in}}%
\pgfpathlineto{\pgfqpoint{6.143456in}{1.108988in}}%
\pgfpathlineto{\pgfqpoint{6.143456in}{1.203622in}}%
\pgfpathclose%
\pgfusepath{fill}%
\end{pgfscope}%
\begin{pgfscope}%
\pgfpathrectangle{\pgfqpoint{3.722897in}{0.857143in}}{\pgfqpoint{2.627103in}{1.813434in}}%
\pgfusepath{clip}%
\pgfsetbuttcap%
\pgfsetmiterjoin%
\definecolor{currentfill}{rgb}{0.950697,0.616649,0.428624}%
\pgfsetfillcolor{currentfill}%
\pgfsetlinewidth{0.000000pt}%
\definecolor{currentstroke}{rgb}{0.000000,0.000000,0.000000}%
\pgfsetstrokecolor{currentstroke}%
\pgfsetstrokeopacity{0.000000}%
\pgfsetdash{}{0pt}%
\pgfpathmoveto{\pgfqpoint{6.154626in}{1.189858in}}%
\pgfpathlineto{\pgfqpoint{6.163563in}{1.189858in}}%
\pgfpathlineto{\pgfqpoint{6.163563in}{1.087295in}}%
\pgfpathlineto{\pgfqpoint{6.154626in}{1.087295in}}%
\pgfpathlineto{\pgfqpoint{6.154626in}{1.189858in}}%
\pgfpathclose%
\pgfusepath{fill}%
\end{pgfscope}%
\begin{pgfscope}%
\pgfpathrectangle{\pgfqpoint{3.722897in}{0.857143in}}{\pgfqpoint{2.627103in}{1.813434in}}%
\pgfusepath{clip}%
\pgfsetbuttcap%
\pgfsetmiterjoin%
\definecolor{currentfill}{rgb}{0.950697,0.616649,0.428624}%
\pgfsetfillcolor{currentfill}%
\pgfsetlinewidth{0.000000pt}%
\definecolor{currentstroke}{rgb}{0.000000,0.000000,0.000000}%
\pgfsetstrokecolor{currentstroke}%
\pgfsetstrokeopacity{0.000000}%
\pgfsetdash{}{0pt}%
\pgfpathmoveto{\pgfqpoint{6.165797in}{1.186296in}}%
\pgfpathlineto{\pgfqpoint{6.174733in}{1.186296in}}%
\pgfpathlineto{\pgfqpoint{6.174733in}{1.088899in}}%
\pgfpathlineto{\pgfqpoint{6.165797in}{1.088899in}}%
\pgfpathlineto{\pgfqpoint{6.165797in}{1.186296in}}%
\pgfpathclose%
\pgfusepath{fill}%
\end{pgfscope}%
\begin{pgfscope}%
\pgfpathrectangle{\pgfqpoint{3.722897in}{0.857143in}}{\pgfqpoint{2.627103in}{1.813434in}}%
\pgfusepath{clip}%
\pgfsetbuttcap%
\pgfsetmiterjoin%
\definecolor{currentfill}{rgb}{0.950697,0.616649,0.428624}%
\pgfsetfillcolor{currentfill}%
\pgfsetlinewidth{0.000000pt}%
\definecolor{currentstroke}{rgb}{0.000000,0.000000,0.000000}%
\pgfsetstrokecolor{currentstroke}%
\pgfsetstrokeopacity{0.000000}%
\pgfsetdash{}{0pt}%
\pgfpathmoveto{\pgfqpoint{6.176967in}{1.185421in}}%
\pgfpathlineto{\pgfqpoint{6.185904in}{1.185421in}}%
\pgfpathlineto{\pgfqpoint{6.185904in}{1.093784in}}%
\pgfpathlineto{\pgfqpoint{6.176967in}{1.093784in}}%
\pgfpathlineto{\pgfqpoint{6.176967in}{1.185421in}}%
\pgfpathclose%
\pgfusepath{fill}%
\end{pgfscope}%
\begin{pgfscope}%
\pgfpathrectangle{\pgfqpoint{3.722897in}{0.857143in}}{\pgfqpoint{2.627103in}{1.813434in}}%
\pgfusepath{clip}%
\pgfsetbuttcap%
\pgfsetmiterjoin%
\definecolor{currentfill}{rgb}{0.950697,0.616649,0.428624}%
\pgfsetfillcolor{currentfill}%
\pgfsetlinewidth{0.000000pt}%
\definecolor{currentstroke}{rgb}{0.000000,0.000000,0.000000}%
\pgfsetstrokecolor{currentstroke}%
\pgfsetstrokeopacity{0.000000}%
\pgfsetdash{}{0pt}%
\pgfpathmoveto{\pgfqpoint{6.188138in}{1.189139in}}%
\pgfpathlineto{\pgfqpoint{6.197074in}{1.189139in}}%
\pgfpathlineto{\pgfqpoint{6.197074in}{1.087586in}}%
\pgfpathlineto{\pgfqpoint{6.188138in}{1.087586in}}%
\pgfpathlineto{\pgfqpoint{6.188138in}{1.189139in}}%
\pgfpathclose%
\pgfusepath{fill}%
\end{pgfscope}%
\begin{pgfscope}%
\pgfpathrectangle{\pgfqpoint{3.722897in}{0.857143in}}{\pgfqpoint{2.627103in}{1.813434in}}%
\pgfusepath{clip}%
\pgfsetbuttcap%
\pgfsetmiterjoin%
\definecolor{currentfill}{rgb}{0.950697,0.616649,0.428624}%
\pgfsetfillcolor{currentfill}%
\pgfsetlinewidth{0.000000pt}%
\definecolor{currentstroke}{rgb}{0.000000,0.000000,0.000000}%
\pgfsetstrokecolor{currentstroke}%
\pgfsetstrokeopacity{0.000000}%
\pgfsetdash{}{0pt}%
\pgfpathmoveto{\pgfqpoint{6.199309in}{1.177051in}}%
\pgfpathlineto{\pgfqpoint{6.208245in}{1.177051in}}%
\pgfpathlineto{\pgfqpoint{6.208245in}{1.092350in}}%
\pgfpathlineto{\pgfqpoint{6.199309in}{1.092350in}}%
\pgfpathlineto{\pgfqpoint{6.199309in}{1.177051in}}%
\pgfpathclose%
\pgfusepath{fill}%
\end{pgfscope}%
\begin{pgfscope}%
\pgfpathrectangle{\pgfqpoint{3.722897in}{0.857143in}}{\pgfqpoint{2.627103in}{1.813434in}}%
\pgfusepath{clip}%
\pgfsetbuttcap%
\pgfsetmiterjoin%
\definecolor{currentfill}{rgb}{0.950697,0.616649,0.428624}%
\pgfsetfillcolor{currentfill}%
\pgfsetlinewidth{0.000000pt}%
\definecolor{currentstroke}{rgb}{0.000000,0.000000,0.000000}%
\pgfsetstrokecolor{currentstroke}%
\pgfsetstrokeopacity{0.000000}%
\pgfsetdash{}{0pt}%
\pgfpathmoveto{\pgfqpoint{6.210479in}{1.173336in}}%
\pgfpathlineto{\pgfqpoint{6.219416in}{1.173336in}}%
\pgfpathlineto{\pgfqpoint{6.219416in}{1.113275in}}%
\pgfpathlineto{\pgfqpoint{6.210479in}{1.113275in}}%
\pgfpathlineto{\pgfqpoint{6.210479in}{1.173336in}}%
\pgfpathclose%
\pgfusepath{fill}%
\end{pgfscope}%
\begin{pgfscope}%
\pgfpathrectangle{\pgfqpoint{3.722897in}{0.857143in}}{\pgfqpoint{2.627103in}{1.813434in}}%
\pgfusepath{clip}%
\pgfsetbuttcap%
\pgfsetmiterjoin%
\definecolor{currentfill}{rgb}{0.950697,0.616649,0.428624}%
\pgfsetfillcolor{currentfill}%
\pgfsetlinewidth{0.000000pt}%
\definecolor{currentstroke}{rgb}{0.000000,0.000000,0.000000}%
\pgfsetstrokecolor{currentstroke}%
\pgfsetstrokeopacity{0.000000}%
\pgfsetdash{}{0pt}%
\pgfpathmoveto{\pgfqpoint{6.221650in}{1.168605in}}%
\pgfpathlineto{\pgfqpoint{6.230586in}{1.168605in}}%
\pgfpathlineto{\pgfqpoint{6.230586in}{1.115353in}}%
\pgfpathlineto{\pgfqpoint{6.221650in}{1.115353in}}%
\pgfpathlineto{\pgfqpoint{6.221650in}{1.168605in}}%
\pgfpathclose%
\pgfusepath{fill}%
\end{pgfscope}%
\begin{pgfscope}%
\pgfpathrectangle{\pgfqpoint{3.722897in}{0.857143in}}{\pgfqpoint{2.627103in}{1.813434in}}%
\pgfusepath{clip}%
\pgfsetbuttcap%
\pgfsetmiterjoin%
\definecolor{currentfill}{rgb}{0.992771,0.707689,0.712380}%
\pgfsetfillcolor{currentfill}%
\pgfsetlinewidth{0.000000pt}%
\definecolor{currentstroke}{rgb}{0.000000,0.000000,0.000000}%
\pgfsetstrokecolor{currentstroke}%
\pgfsetstrokeopacity{0.000000}%
\pgfsetdash{}{0pt}%
\pgfpathmoveto{\pgfqpoint{3.842311in}{1.819845in}}%
\pgfpathlineto{\pgfqpoint{3.851247in}{1.819845in}}%
\pgfpathlineto{\pgfqpoint{3.851247in}{2.134221in}}%
\pgfpathlineto{\pgfqpoint{3.842311in}{2.134221in}}%
\pgfpathlineto{\pgfqpoint{3.842311in}{1.819845in}}%
\pgfpathclose%
\pgfusepath{fill}%
\end{pgfscope}%
\begin{pgfscope}%
\pgfpathrectangle{\pgfqpoint{3.722897in}{0.857143in}}{\pgfqpoint{2.627103in}{1.813434in}}%
\pgfusepath{clip}%
\pgfsetbuttcap%
\pgfsetmiterjoin%
\definecolor{currentfill}{rgb}{0.992771,0.707689,0.712380}%
\pgfsetfillcolor{currentfill}%
\pgfsetlinewidth{0.000000pt}%
\definecolor{currentstroke}{rgb}{0.000000,0.000000,0.000000}%
\pgfsetstrokecolor{currentstroke}%
\pgfsetstrokeopacity{0.000000}%
\pgfsetdash{}{0pt}%
\pgfpathmoveto{\pgfqpoint{3.853481in}{1.824416in}}%
\pgfpathlineto{\pgfqpoint{3.862418in}{1.824416in}}%
\pgfpathlineto{\pgfqpoint{3.862418in}{2.156184in}}%
\pgfpathlineto{\pgfqpoint{3.853481in}{2.156184in}}%
\pgfpathlineto{\pgfqpoint{3.853481in}{1.824416in}}%
\pgfpathclose%
\pgfusepath{fill}%
\end{pgfscope}%
\begin{pgfscope}%
\pgfpathrectangle{\pgfqpoint{3.722897in}{0.857143in}}{\pgfqpoint{2.627103in}{1.813434in}}%
\pgfusepath{clip}%
\pgfsetbuttcap%
\pgfsetmiterjoin%
\definecolor{currentfill}{rgb}{0.992771,0.707689,0.712380}%
\pgfsetfillcolor{currentfill}%
\pgfsetlinewidth{0.000000pt}%
\definecolor{currentstroke}{rgb}{0.000000,0.000000,0.000000}%
\pgfsetstrokecolor{currentstroke}%
\pgfsetstrokeopacity{0.000000}%
\pgfsetdash{}{0pt}%
\pgfpathmoveto{\pgfqpoint{3.864652in}{1.841228in}}%
\pgfpathlineto{\pgfqpoint{3.873588in}{1.841228in}}%
\pgfpathlineto{\pgfqpoint{3.873588in}{2.165983in}}%
\pgfpathlineto{\pgfqpoint{3.864652in}{2.165983in}}%
\pgfpathlineto{\pgfqpoint{3.864652in}{1.841228in}}%
\pgfpathclose%
\pgfusepath{fill}%
\end{pgfscope}%
\begin{pgfscope}%
\pgfpathrectangle{\pgfqpoint{3.722897in}{0.857143in}}{\pgfqpoint{2.627103in}{1.813434in}}%
\pgfusepath{clip}%
\pgfsetbuttcap%
\pgfsetmiterjoin%
\definecolor{currentfill}{rgb}{0.992771,0.707689,0.712380}%
\pgfsetfillcolor{currentfill}%
\pgfsetlinewidth{0.000000pt}%
\definecolor{currentstroke}{rgb}{0.000000,0.000000,0.000000}%
\pgfsetstrokecolor{currentstroke}%
\pgfsetstrokeopacity{0.000000}%
\pgfsetdash{}{0pt}%
\pgfpathmoveto{\pgfqpoint{3.875823in}{1.845496in}}%
\pgfpathlineto{\pgfqpoint{3.884759in}{1.845496in}}%
\pgfpathlineto{\pgfqpoint{3.884759in}{2.144348in}}%
\pgfpathlineto{\pgfqpoint{3.875823in}{2.144348in}}%
\pgfpathlineto{\pgfqpoint{3.875823in}{1.845496in}}%
\pgfpathclose%
\pgfusepath{fill}%
\end{pgfscope}%
\begin{pgfscope}%
\pgfpathrectangle{\pgfqpoint{3.722897in}{0.857143in}}{\pgfqpoint{2.627103in}{1.813434in}}%
\pgfusepath{clip}%
\pgfsetbuttcap%
\pgfsetmiterjoin%
\definecolor{currentfill}{rgb}{0.992771,0.707689,0.712380}%
\pgfsetfillcolor{currentfill}%
\pgfsetlinewidth{0.000000pt}%
\definecolor{currentstroke}{rgb}{0.000000,0.000000,0.000000}%
\pgfsetstrokecolor{currentstroke}%
\pgfsetstrokeopacity{0.000000}%
\pgfsetdash{}{0pt}%
\pgfpathmoveto{\pgfqpoint{3.886993in}{1.835495in}}%
\pgfpathlineto{\pgfqpoint{3.895930in}{1.835495in}}%
\pgfpathlineto{\pgfqpoint{3.895930in}{2.124157in}}%
\pgfpathlineto{\pgfqpoint{3.886993in}{2.124157in}}%
\pgfpathlineto{\pgfqpoint{3.886993in}{1.835495in}}%
\pgfpathclose%
\pgfusepath{fill}%
\end{pgfscope}%
\begin{pgfscope}%
\pgfpathrectangle{\pgfqpoint{3.722897in}{0.857143in}}{\pgfqpoint{2.627103in}{1.813434in}}%
\pgfusepath{clip}%
\pgfsetbuttcap%
\pgfsetmiterjoin%
\definecolor{currentfill}{rgb}{0.992771,0.707689,0.712380}%
\pgfsetfillcolor{currentfill}%
\pgfsetlinewidth{0.000000pt}%
\definecolor{currentstroke}{rgb}{0.000000,0.000000,0.000000}%
\pgfsetstrokecolor{currentstroke}%
\pgfsetstrokeopacity{0.000000}%
\pgfsetdash{}{0pt}%
\pgfpathmoveto{\pgfqpoint{3.898164in}{1.823745in}}%
\pgfpathlineto{\pgfqpoint{3.907100in}{1.823745in}}%
\pgfpathlineto{\pgfqpoint{3.907100in}{2.139344in}}%
\pgfpathlineto{\pgfqpoint{3.898164in}{2.139344in}}%
\pgfpathlineto{\pgfqpoint{3.898164in}{1.823745in}}%
\pgfpathclose%
\pgfusepath{fill}%
\end{pgfscope}%
\begin{pgfscope}%
\pgfpathrectangle{\pgfqpoint{3.722897in}{0.857143in}}{\pgfqpoint{2.627103in}{1.813434in}}%
\pgfusepath{clip}%
\pgfsetbuttcap%
\pgfsetmiterjoin%
\definecolor{currentfill}{rgb}{0.992771,0.707689,0.712380}%
\pgfsetfillcolor{currentfill}%
\pgfsetlinewidth{0.000000pt}%
\definecolor{currentstroke}{rgb}{0.000000,0.000000,0.000000}%
\pgfsetstrokecolor{currentstroke}%
\pgfsetstrokeopacity{0.000000}%
\pgfsetdash{}{0pt}%
\pgfpathmoveto{\pgfqpoint{3.909334in}{1.841680in}}%
\pgfpathlineto{\pgfqpoint{3.918271in}{1.841680in}}%
\pgfpathlineto{\pgfqpoint{3.918271in}{2.169672in}}%
\pgfpathlineto{\pgfqpoint{3.909334in}{2.169672in}}%
\pgfpathlineto{\pgfqpoint{3.909334in}{1.841680in}}%
\pgfpathclose%
\pgfusepath{fill}%
\end{pgfscope}%
\begin{pgfscope}%
\pgfpathrectangle{\pgfqpoint{3.722897in}{0.857143in}}{\pgfqpoint{2.627103in}{1.813434in}}%
\pgfusepath{clip}%
\pgfsetbuttcap%
\pgfsetmiterjoin%
\definecolor{currentfill}{rgb}{0.992771,0.707689,0.712380}%
\pgfsetfillcolor{currentfill}%
\pgfsetlinewidth{0.000000pt}%
\definecolor{currentstroke}{rgb}{0.000000,0.000000,0.000000}%
\pgfsetstrokecolor{currentstroke}%
\pgfsetstrokeopacity{0.000000}%
\pgfsetdash{}{0pt}%
\pgfpathmoveto{\pgfqpoint{3.920505in}{1.868332in}}%
\pgfpathlineto{\pgfqpoint{3.929442in}{1.868332in}}%
\pgfpathlineto{\pgfqpoint{3.929442in}{2.188207in}}%
\pgfpathlineto{\pgfqpoint{3.920505in}{2.188207in}}%
\pgfpathlineto{\pgfqpoint{3.920505in}{1.868332in}}%
\pgfpathclose%
\pgfusepath{fill}%
\end{pgfscope}%
\begin{pgfscope}%
\pgfpathrectangle{\pgfqpoint{3.722897in}{0.857143in}}{\pgfqpoint{2.627103in}{1.813434in}}%
\pgfusepath{clip}%
\pgfsetbuttcap%
\pgfsetmiterjoin%
\definecolor{currentfill}{rgb}{0.992771,0.707689,0.712380}%
\pgfsetfillcolor{currentfill}%
\pgfsetlinewidth{0.000000pt}%
\definecolor{currentstroke}{rgb}{0.000000,0.000000,0.000000}%
\pgfsetstrokecolor{currentstroke}%
\pgfsetstrokeopacity{0.000000}%
\pgfsetdash{}{0pt}%
\pgfpathmoveto{\pgfqpoint{3.931676in}{1.874611in}}%
\pgfpathlineto{\pgfqpoint{3.940612in}{1.874611in}}%
\pgfpathlineto{\pgfqpoint{3.940612in}{2.174421in}}%
\pgfpathlineto{\pgfqpoint{3.931676in}{2.174421in}}%
\pgfpathlineto{\pgfqpoint{3.931676in}{1.874611in}}%
\pgfpathclose%
\pgfusepath{fill}%
\end{pgfscope}%
\begin{pgfscope}%
\pgfpathrectangle{\pgfqpoint{3.722897in}{0.857143in}}{\pgfqpoint{2.627103in}{1.813434in}}%
\pgfusepath{clip}%
\pgfsetbuttcap%
\pgfsetmiterjoin%
\definecolor{currentfill}{rgb}{0.992771,0.707689,0.712380}%
\pgfsetfillcolor{currentfill}%
\pgfsetlinewidth{0.000000pt}%
\definecolor{currentstroke}{rgb}{0.000000,0.000000,0.000000}%
\pgfsetstrokecolor{currentstroke}%
\pgfsetstrokeopacity{0.000000}%
\pgfsetdash{}{0pt}%
\pgfpathmoveto{\pgfqpoint{3.942846in}{1.860678in}}%
\pgfpathlineto{\pgfqpoint{3.951783in}{1.860678in}}%
\pgfpathlineto{\pgfqpoint{3.951783in}{2.163772in}}%
\pgfpathlineto{\pgfqpoint{3.942846in}{2.163772in}}%
\pgfpathlineto{\pgfqpoint{3.942846in}{1.860678in}}%
\pgfpathclose%
\pgfusepath{fill}%
\end{pgfscope}%
\begin{pgfscope}%
\pgfpathrectangle{\pgfqpoint{3.722897in}{0.857143in}}{\pgfqpoint{2.627103in}{1.813434in}}%
\pgfusepath{clip}%
\pgfsetbuttcap%
\pgfsetmiterjoin%
\definecolor{currentfill}{rgb}{0.992771,0.707689,0.712380}%
\pgfsetfillcolor{currentfill}%
\pgfsetlinewidth{0.000000pt}%
\definecolor{currentstroke}{rgb}{0.000000,0.000000,0.000000}%
\pgfsetstrokecolor{currentstroke}%
\pgfsetstrokeopacity{0.000000}%
\pgfsetdash{}{0pt}%
\pgfpathmoveto{\pgfqpoint{3.954017in}{1.873535in}}%
\pgfpathlineto{\pgfqpoint{3.962953in}{1.873535in}}%
\pgfpathlineto{\pgfqpoint{3.962953in}{2.164868in}}%
\pgfpathlineto{\pgfqpoint{3.954017in}{2.164868in}}%
\pgfpathlineto{\pgfqpoint{3.954017in}{1.873535in}}%
\pgfpathclose%
\pgfusepath{fill}%
\end{pgfscope}%
\begin{pgfscope}%
\pgfpathrectangle{\pgfqpoint{3.722897in}{0.857143in}}{\pgfqpoint{2.627103in}{1.813434in}}%
\pgfusepath{clip}%
\pgfsetbuttcap%
\pgfsetmiterjoin%
\definecolor{currentfill}{rgb}{0.992771,0.707689,0.712380}%
\pgfsetfillcolor{currentfill}%
\pgfsetlinewidth{0.000000pt}%
\definecolor{currentstroke}{rgb}{0.000000,0.000000,0.000000}%
\pgfsetstrokecolor{currentstroke}%
\pgfsetstrokeopacity{0.000000}%
\pgfsetdash{}{0pt}%
\pgfpathmoveto{\pgfqpoint{3.965187in}{1.885379in}}%
\pgfpathlineto{\pgfqpoint{3.974124in}{1.885379in}}%
\pgfpathlineto{\pgfqpoint{3.974124in}{2.144410in}}%
\pgfpathlineto{\pgfqpoint{3.965187in}{2.144410in}}%
\pgfpathlineto{\pgfqpoint{3.965187in}{1.885379in}}%
\pgfpathclose%
\pgfusepath{fill}%
\end{pgfscope}%
\begin{pgfscope}%
\pgfpathrectangle{\pgfqpoint{3.722897in}{0.857143in}}{\pgfqpoint{2.627103in}{1.813434in}}%
\pgfusepath{clip}%
\pgfsetbuttcap%
\pgfsetmiterjoin%
\definecolor{currentfill}{rgb}{0.992771,0.707689,0.712380}%
\pgfsetfillcolor{currentfill}%
\pgfsetlinewidth{0.000000pt}%
\definecolor{currentstroke}{rgb}{0.000000,0.000000,0.000000}%
\pgfsetstrokecolor{currentstroke}%
\pgfsetstrokeopacity{0.000000}%
\pgfsetdash{}{0pt}%
\pgfpathmoveto{\pgfqpoint{3.976358in}{1.876340in}}%
\pgfpathlineto{\pgfqpoint{3.985295in}{1.876340in}}%
\pgfpathlineto{\pgfqpoint{3.985295in}{2.099788in}}%
\pgfpathlineto{\pgfqpoint{3.976358in}{2.099788in}}%
\pgfpathlineto{\pgfqpoint{3.976358in}{1.876340in}}%
\pgfpathclose%
\pgfusepath{fill}%
\end{pgfscope}%
\begin{pgfscope}%
\pgfpathrectangle{\pgfqpoint{3.722897in}{0.857143in}}{\pgfqpoint{2.627103in}{1.813434in}}%
\pgfusepath{clip}%
\pgfsetbuttcap%
\pgfsetmiterjoin%
\definecolor{currentfill}{rgb}{0.992771,0.707689,0.712380}%
\pgfsetfillcolor{currentfill}%
\pgfsetlinewidth{0.000000pt}%
\definecolor{currentstroke}{rgb}{0.000000,0.000000,0.000000}%
\pgfsetstrokecolor{currentstroke}%
\pgfsetstrokeopacity{0.000000}%
\pgfsetdash{}{0pt}%
\pgfpathmoveto{\pgfqpoint{3.987529in}{1.893204in}}%
\pgfpathlineto{\pgfqpoint{3.996465in}{1.893204in}}%
\pgfpathlineto{\pgfqpoint{3.996465in}{2.098946in}}%
\pgfpathlineto{\pgfqpoint{3.987529in}{2.098946in}}%
\pgfpathlineto{\pgfqpoint{3.987529in}{1.893204in}}%
\pgfpathclose%
\pgfusepath{fill}%
\end{pgfscope}%
\begin{pgfscope}%
\pgfpathrectangle{\pgfqpoint{3.722897in}{0.857143in}}{\pgfqpoint{2.627103in}{1.813434in}}%
\pgfusepath{clip}%
\pgfsetbuttcap%
\pgfsetmiterjoin%
\definecolor{currentfill}{rgb}{0.992771,0.707689,0.712380}%
\pgfsetfillcolor{currentfill}%
\pgfsetlinewidth{0.000000pt}%
\definecolor{currentstroke}{rgb}{0.000000,0.000000,0.000000}%
\pgfsetstrokecolor{currentstroke}%
\pgfsetstrokeopacity{0.000000}%
\pgfsetdash{}{0pt}%
\pgfpathmoveto{\pgfqpoint{3.998699in}{1.870021in}}%
\pgfpathlineto{\pgfqpoint{4.007636in}{1.870021in}}%
\pgfpathlineto{\pgfqpoint{4.007636in}{2.063926in}}%
\pgfpathlineto{\pgfqpoint{3.998699in}{2.063926in}}%
\pgfpathlineto{\pgfqpoint{3.998699in}{1.870021in}}%
\pgfpathclose%
\pgfusepath{fill}%
\end{pgfscope}%
\begin{pgfscope}%
\pgfpathrectangle{\pgfqpoint{3.722897in}{0.857143in}}{\pgfqpoint{2.627103in}{1.813434in}}%
\pgfusepath{clip}%
\pgfsetbuttcap%
\pgfsetmiterjoin%
\definecolor{currentfill}{rgb}{0.992771,0.707689,0.712380}%
\pgfsetfillcolor{currentfill}%
\pgfsetlinewidth{0.000000pt}%
\definecolor{currentstroke}{rgb}{0.000000,0.000000,0.000000}%
\pgfsetstrokecolor{currentstroke}%
\pgfsetstrokeopacity{0.000000}%
\pgfsetdash{}{0pt}%
\pgfpathmoveto{\pgfqpoint{4.009870in}{1.867528in}}%
\pgfpathlineto{\pgfqpoint{4.018806in}{1.867528in}}%
\pgfpathlineto{\pgfqpoint{4.018806in}{2.041580in}}%
\pgfpathlineto{\pgfqpoint{4.009870in}{2.041580in}}%
\pgfpathlineto{\pgfqpoint{4.009870in}{1.867528in}}%
\pgfpathclose%
\pgfusepath{fill}%
\end{pgfscope}%
\begin{pgfscope}%
\pgfpathrectangle{\pgfqpoint{3.722897in}{0.857143in}}{\pgfqpoint{2.627103in}{1.813434in}}%
\pgfusepath{clip}%
\pgfsetbuttcap%
\pgfsetmiterjoin%
\definecolor{currentfill}{rgb}{0.992771,0.707689,0.712380}%
\pgfsetfillcolor{currentfill}%
\pgfsetlinewidth{0.000000pt}%
\definecolor{currentstroke}{rgb}{0.000000,0.000000,0.000000}%
\pgfsetstrokecolor{currentstroke}%
\pgfsetstrokeopacity{0.000000}%
\pgfsetdash{}{0pt}%
\pgfpathmoveto{\pgfqpoint{4.021040in}{1.898361in}}%
\pgfpathlineto{\pgfqpoint{4.029977in}{1.898361in}}%
\pgfpathlineto{\pgfqpoint{4.029977in}{2.046808in}}%
\pgfpathlineto{\pgfqpoint{4.021040in}{2.046808in}}%
\pgfpathlineto{\pgfqpoint{4.021040in}{1.898361in}}%
\pgfpathclose%
\pgfusepath{fill}%
\end{pgfscope}%
\begin{pgfscope}%
\pgfpathrectangle{\pgfqpoint{3.722897in}{0.857143in}}{\pgfqpoint{2.627103in}{1.813434in}}%
\pgfusepath{clip}%
\pgfsetbuttcap%
\pgfsetmiterjoin%
\definecolor{currentfill}{rgb}{0.992771,0.707689,0.712380}%
\pgfsetfillcolor{currentfill}%
\pgfsetlinewidth{0.000000pt}%
\definecolor{currentstroke}{rgb}{0.000000,0.000000,0.000000}%
\pgfsetstrokecolor{currentstroke}%
\pgfsetstrokeopacity{0.000000}%
\pgfsetdash{}{0pt}%
\pgfpathmoveto{\pgfqpoint{4.032211in}{1.909963in}}%
\pgfpathlineto{\pgfqpoint{4.041148in}{1.909963in}}%
\pgfpathlineto{\pgfqpoint{4.041148in}{2.060327in}}%
\pgfpathlineto{\pgfqpoint{4.032211in}{2.060327in}}%
\pgfpathlineto{\pgfqpoint{4.032211in}{1.909963in}}%
\pgfpathclose%
\pgfusepath{fill}%
\end{pgfscope}%
\begin{pgfscope}%
\pgfpathrectangle{\pgfqpoint{3.722897in}{0.857143in}}{\pgfqpoint{2.627103in}{1.813434in}}%
\pgfusepath{clip}%
\pgfsetbuttcap%
\pgfsetmiterjoin%
\definecolor{currentfill}{rgb}{0.992771,0.707689,0.712380}%
\pgfsetfillcolor{currentfill}%
\pgfsetlinewidth{0.000000pt}%
\definecolor{currentstroke}{rgb}{0.000000,0.000000,0.000000}%
\pgfsetstrokecolor{currentstroke}%
\pgfsetstrokeopacity{0.000000}%
\pgfsetdash{}{0pt}%
\pgfpathmoveto{\pgfqpoint{4.043382in}{1.902915in}}%
\pgfpathlineto{\pgfqpoint{4.052318in}{1.902915in}}%
\pgfpathlineto{\pgfqpoint{4.052318in}{2.074344in}}%
\pgfpathlineto{\pgfqpoint{4.043382in}{2.074344in}}%
\pgfpathlineto{\pgfqpoint{4.043382in}{1.902915in}}%
\pgfpathclose%
\pgfusepath{fill}%
\end{pgfscope}%
\begin{pgfscope}%
\pgfpathrectangle{\pgfqpoint{3.722897in}{0.857143in}}{\pgfqpoint{2.627103in}{1.813434in}}%
\pgfusepath{clip}%
\pgfsetbuttcap%
\pgfsetmiterjoin%
\definecolor{currentfill}{rgb}{0.992771,0.707689,0.712380}%
\pgfsetfillcolor{currentfill}%
\pgfsetlinewidth{0.000000pt}%
\definecolor{currentstroke}{rgb}{0.000000,0.000000,0.000000}%
\pgfsetstrokecolor{currentstroke}%
\pgfsetstrokeopacity{0.000000}%
\pgfsetdash{}{0pt}%
\pgfpathmoveto{\pgfqpoint{4.054552in}{1.931745in}}%
\pgfpathlineto{\pgfqpoint{4.063489in}{1.931745in}}%
\pgfpathlineto{\pgfqpoint{4.063489in}{2.110422in}}%
\pgfpathlineto{\pgfqpoint{4.054552in}{2.110422in}}%
\pgfpathlineto{\pgfqpoint{4.054552in}{1.931745in}}%
\pgfpathclose%
\pgfusepath{fill}%
\end{pgfscope}%
\begin{pgfscope}%
\pgfpathrectangle{\pgfqpoint{3.722897in}{0.857143in}}{\pgfqpoint{2.627103in}{1.813434in}}%
\pgfusepath{clip}%
\pgfsetbuttcap%
\pgfsetmiterjoin%
\definecolor{currentfill}{rgb}{0.992771,0.707689,0.712380}%
\pgfsetfillcolor{currentfill}%
\pgfsetlinewidth{0.000000pt}%
\definecolor{currentstroke}{rgb}{0.000000,0.000000,0.000000}%
\pgfsetstrokecolor{currentstroke}%
\pgfsetstrokeopacity{0.000000}%
\pgfsetdash{}{0pt}%
\pgfpathmoveto{\pgfqpoint{4.065723in}{1.940649in}}%
\pgfpathlineto{\pgfqpoint{4.074659in}{1.940649in}}%
\pgfpathlineto{\pgfqpoint{4.074659in}{2.140779in}}%
\pgfpathlineto{\pgfqpoint{4.065723in}{2.140779in}}%
\pgfpathlineto{\pgfqpoint{4.065723in}{1.940649in}}%
\pgfpathclose%
\pgfusepath{fill}%
\end{pgfscope}%
\begin{pgfscope}%
\pgfpathrectangle{\pgfqpoint{3.722897in}{0.857143in}}{\pgfqpoint{2.627103in}{1.813434in}}%
\pgfusepath{clip}%
\pgfsetbuttcap%
\pgfsetmiterjoin%
\definecolor{currentfill}{rgb}{0.992771,0.707689,0.712380}%
\pgfsetfillcolor{currentfill}%
\pgfsetlinewidth{0.000000pt}%
\definecolor{currentstroke}{rgb}{0.000000,0.000000,0.000000}%
\pgfsetstrokecolor{currentstroke}%
\pgfsetstrokeopacity{0.000000}%
\pgfsetdash{}{0pt}%
\pgfpathmoveto{\pgfqpoint{4.076893in}{1.926270in}}%
\pgfpathlineto{\pgfqpoint{4.085830in}{1.926270in}}%
\pgfpathlineto{\pgfqpoint{4.085830in}{2.149793in}}%
\pgfpathlineto{\pgfqpoint{4.076893in}{2.149793in}}%
\pgfpathlineto{\pgfqpoint{4.076893in}{1.926270in}}%
\pgfpathclose%
\pgfusepath{fill}%
\end{pgfscope}%
\begin{pgfscope}%
\pgfpathrectangle{\pgfqpoint{3.722897in}{0.857143in}}{\pgfqpoint{2.627103in}{1.813434in}}%
\pgfusepath{clip}%
\pgfsetbuttcap%
\pgfsetmiterjoin%
\definecolor{currentfill}{rgb}{0.992771,0.707689,0.712380}%
\pgfsetfillcolor{currentfill}%
\pgfsetlinewidth{0.000000pt}%
\definecolor{currentstroke}{rgb}{0.000000,0.000000,0.000000}%
\pgfsetstrokecolor{currentstroke}%
\pgfsetstrokeopacity{0.000000}%
\pgfsetdash{}{0pt}%
\pgfpathmoveto{\pgfqpoint{4.088064in}{1.914510in}}%
\pgfpathlineto{\pgfqpoint{4.097001in}{1.914510in}}%
\pgfpathlineto{\pgfqpoint{4.097001in}{2.150809in}}%
\pgfpathlineto{\pgfqpoint{4.088064in}{2.150809in}}%
\pgfpathlineto{\pgfqpoint{4.088064in}{1.914510in}}%
\pgfpathclose%
\pgfusepath{fill}%
\end{pgfscope}%
\begin{pgfscope}%
\pgfpathrectangle{\pgfqpoint{3.722897in}{0.857143in}}{\pgfqpoint{2.627103in}{1.813434in}}%
\pgfusepath{clip}%
\pgfsetbuttcap%
\pgfsetmiterjoin%
\definecolor{currentfill}{rgb}{0.992771,0.707689,0.712380}%
\pgfsetfillcolor{currentfill}%
\pgfsetlinewidth{0.000000pt}%
\definecolor{currentstroke}{rgb}{0.000000,0.000000,0.000000}%
\pgfsetstrokecolor{currentstroke}%
\pgfsetstrokeopacity{0.000000}%
\pgfsetdash{}{0pt}%
\pgfpathmoveto{\pgfqpoint{4.099235in}{1.947683in}}%
\pgfpathlineto{\pgfqpoint{4.108171in}{1.947683in}}%
\pgfpathlineto{\pgfqpoint{4.108171in}{2.176203in}}%
\pgfpathlineto{\pgfqpoint{4.099235in}{2.176203in}}%
\pgfpathlineto{\pgfqpoint{4.099235in}{1.947683in}}%
\pgfpathclose%
\pgfusepath{fill}%
\end{pgfscope}%
\begin{pgfscope}%
\pgfpathrectangle{\pgfqpoint{3.722897in}{0.857143in}}{\pgfqpoint{2.627103in}{1.813434in}}%
\pgfusepath{clip}%
\pgfsetbuttcap%
\pgfsetmiterjoin%
\definecolor{currentfill}{rgb}{0.992771,0.707689,0.712380}%
\pgfsetfillcolor{currentfill}%
\pgfsetlinewidth{0.000000pt}%
\definecolor{currentstroke}{rgb}{0.000000,0.000000,0.000000}%
\pgfsetstrokecolor{currentstroke}%
\pgfsetstrokeopacity{0.000000}%
\pgfsetdash{}{0pt}%
\pgfpathmoveto{\pgfqpoint{4.110405in}{1.958953in}}%
\pgfpathlineto{\pgfqpoint{4.119342in}{1.958953in}}%
\pgfpathlineto{\pgfqpoint{4.119342in}{2.171385in}}%
\pgfpathlineto{\pgfqpoint{4.110405in}{2.171385in}}%
\pgfpathlineto{\pgfqpoint{4.110405in}{1.958953in}}%
\pgfpathclose%
\pgfusepath{fill}%
\end{pgfscope}%
\begin{pgfscope}%
\pgfpathrectangle{\pgfqpoint{3.722897in}{0.857143in}}{\pgfqpoint{2.627103in}{1.813434in}}%
\pgfusepath{clip}%
\pgfsetbuttcap%
\pgfsetmiterjoin%
\definecolor{currentfill}{rgb}{0.992771,0.707689,0.712380}%
\pgfsetfillcolor{currentfill}%
\pgfsetlinewidth{0.000000pt}%
\definecolor{currentstroke}{rgb}{0.000000,0.000000,0.000000}%
\pgfsetstrokecolor{currentstroke}%
\pgfsetstrokeopacity{0.000000}%
\pgfsetdash{}{0pt}%
\pgfpathmoveto{\pgfqpoint{4.121576in}{1.972578in}}%
\pgfpathlineto{\pgfqpoint{4.130512in}{1.972578in}}%
\pgfpathlineto{\pgfqpoint{4.130512in}{2.189938in}}%
\pgfpathlineto{\pgfqpoint{4.121576in}{2.189938in}}%
\pgfpathlineto{\pgfqpoint{4.121576in}{1.972578in}}%
\pgfpathclose%
\pgfusepath{fill}%
\end{pgfscope}%
\begin{pgfscope}%
\pgfpathrectangle{\pgfqpoint{3.722897in}{0.857143in}}{\pgfqpoint{2.627103in}{1.813434in}}%
\pgfusepath{clip}%
\pgfsetbuttcap%
\pgfsetmiterjoin%
\definecolor{currentfill}{rgb}{0.992771,0.707689,0.712380}%
\pgfsetfillcolor{currentfill}%
\pgfsetlinewidth{0.000000pt}%
\definecolor{currentstroke}{rgb}{0.000000,0.000000,0.000000}%
\pgfsetstrokecolor{currentstroke}%
\pgfsetstrokeopacity{0.000000}%
\pgfsetdash{}{0pt}%
\pgfpathmoveto{\pgfqpoint{4.132747in}{1.976198in}}%
\pgfpathlineto{\pgfqpoint{4.141683in}{1.976198in}}%
\pgfpathlineto{\pgfqpoint{4.141683in}{2.199193in}}%
\pgfpathlineto{\pgfqpoint{4.132747in}{2.199193in}}%
\pgfpathlineto{\pgfqpoint{4.132747in}{1.976198in}}%
\pgfpathclose%
\pgfusepath{fill}%
\end{pgfscope}%
\begin{pgfscope}%
\pgfpathrectangle{\pgfqpoint{3.722897in}{0.857143in}}{\pgfqpoint{2.627103in}{1.813434in}}%
\pgfusepath{clip}%
\pgfsetbuttcap%
\pgfsetmiterjoin%
\definecolor{currentfill}{rgb}{0.992771,0.707689,0.712380}%
\pgfsetfillcolor{currentfill}%
\pgfsetlinewidth{0.000000pt}%
\definecolor{currentstroke}{rgb}{0.000000,0.000000,0.000000}%
\pgfsetstrokecolor{currentstroke}%
\pgfsetstrokeopacity{0.000000}%
\pgfsetdash{}{0pt}%
\pgfpathmoveto{\pgfqpoint{4.143917in}{1.994874in}}%
\pgfpathlineto{\pgfqpoint{4.152854in}{1.994874in}}%
\pgfpathlineto{\pgfqpoint{4.152854in}{2.192643in}}%
\pgfpathlineto{\pgfqpoint{4.143917in}{2.192643in}}%
\pgfpathlineto{\pgfqpoint{4.143917in}{1.994874in}}%
\pgfpathclose%
\pgfusepath{fill}%
\end{pgfscope}%
\begin{pgfscope}%
\pgfpathrectangle{\pgfqpoint{3.722897in}{0.857143in}}{\pgfqpoint{2.627103in}{1.813434in}}%
\pgfusepath{clip}%
\pgfsetbuttcap%
\pgfsetmiterjoin%
\definecolor{currentfill}{rgb}{0.992771,0.707689,0.712380}%
\pgfsetfillcolor{currentfill}%
\pgfsetlinewidth{0.000000pt}%
\definecolor{currentstroke}{rgb}{0.000000,0.000000,0.000000}%
\pgfsetstrokecolor{currentstroke}%
\pgfsetstrokeopacity{0.000000}%
\pgfsetdash{}{0pt}%
\pgfpathmoveto{\pgfqpoint{4.155088in}{2.031879in}}%
\pgfpathlineto{\pgfqpoint{4.164024in}{2.031879in}}%
\pgfpathlineto{\pgfqpoint{4.164024in}{2.193883in}}%
\pgfpathlineto{\pgfqpoint{4.155088in}{2.193883in}}%
\pgfpathlineto{\pgfqpoint{4.155088in}{2.031879in}}%
\pgfpathclose%
\pgfusepath{fill}%
\end{pgfscope}%
\begin{pgfscope}%
\pgfpathrectangle{\pgfqpoint{3.722897in}{0.857143in}}{\pgfqpoint{2.627103in}{1.813434in}}%
\pgfusepath{clip}%
\pgfsetbuttcap%
\pgfsetmiterjoin%
\definecolor{currentfill}{rgb}{0.992771,0.707689,0.712380}%
\pgfsetfillcolor{currentfill}%
\pgfsetlinewidth{0.000000pt}%
\definecolor{currentstroke}{rgb}{0.000000,0.000000,0.000000}%
\pgfsetstrokecolor{currentstroke}%
\pgfsetstrokeopacity{0.000000}%
\pgfsetdash{}{0pt}%
\pgfpathmoveto{\pgfqpoint{4.166258in}{2.064334in}}%
\pgfpathlineto{\pgfqpoint{4.175195in}{2.064334in}}%
\pgfpathlineto{\pgfqpoint{4.175195in}{2.192497in}}%
\pgfpathlineto{\pgfqpoint{4.166258in}{2.192497in}}%
\pgfpathlineto{\pgfqpoint{4.166258in}{2.064334in}}%
\pgfpathclose%
\pgfusepath{fill}%
\end{pgfscope}%
\begin{pgfscope}%
\pgfpathrectangle{\pgfqpoint{3.722897in}{0.857143in}}{\pgfqpoint{2.627103in}{1.813434in}}%
\pgfusepath{clip}%
\pgfsetbuttcap%
\pgfsetmiterjoin%
\definecolor{currentfill}{rgb}{0.992771,0.707689,0.712380}%
\pgfsetfillcolor{currentfill}%
\pgfsetlinewidth{0.000000pt}%
\definecolor{currentstroke}{rgb}{0.000000,0.000000,0.000000}%
\pgfsetstrokecolor{currentstroke}%
\pgfsetstrokeopacity{0.000000}%
\pgfsetdash{}{0pt}%
\pgfpathmoveto{\pgfqpoint{4.177429in}{2.108683in}}%
\pgfpathlineto{\pgfqpoint{4.186365in}{2.108683in}}%
\pgfpathlineto{\pgfqpoint{4.186365in}{2.183394in}}%
\pgfpathlineto{\pgfqpoint{4.177429in}{2.183394in}}%
\pgfpathlineto{\pgfqpoint{4.177429in}{2.108683in}}%
\pgfpathclose%
\pgfusepath{fill}%
\end{pgfscope}%
\begin{pgfscope}%
\pgfpathrectangle{\pgfqpoint{3.722897in}{0.857143in}}{\pgfqpoint{2.627103in}{1.813434in}}%
\pgfusepath{clip}%
\pgfsetbuttcap%
\pgfsetmiterjoin%
\definecolor{currentfill}{rgb}{0.992771,0.707689,0.712380}%
\pgfsetfillcolor{currentfill}%
\pgfsetlinewidth{0.000000pt}%
\definecolor{currentstroke}{rgb}{0.000000,0.000000,0.000000}%
\pgfsetstrokecolor{currentstroke}%
\pgfsetstrokeopacity{0.000000}%
\pgfsetdash{}{0pt}%
\pgfpathmoveto{\pgfqpoint{4.188600in}{2.150350in}}%
\pgfpathlineto{\pgfqpoint{4.197536in}{2.150350in}}%
\pgfpathlineto{\pgfqpoint{4.197536in}{2.181375in}}%
\pgfpathlineto{\pgfqpoint{4.188600in}{2.181375in}}%
\pgfpathlineto{\pgfqpoint{4.188600in}{2.150350in}}%
\pgfpathclose%
\pgfusepath{fill}%
\end{pgfscope}%
\begin{pgfscope}%
\pgfpathrectangle{\pgfqpoint{3.722897in}{0.857143in}}{\pgfqpoint{2.627103in}{1.813434in}}%
\pgfusepath{clip}%
\pgfsetbuttcap%
\pgfsetmiterjoin%
\definecolor{currentfill}{rgb}{0.992771,0.707689,0.712380}%
\pgfsetfillcolor{currentfill}%
\pgfsetlinewidth{0.000000pt}%
\definecolor{currentstroke}{rgb}{0.000000,0.000000,0.000000}%
\pgfsetstrokecolor{currentstroke}%
\pgfsetstrokeopacity{0.000000}%
\pgfsetdash{}{0pt}%
\pgfpathmoveto{\pgfqpoint{4.199770in}{2.165291in}}%
\pgfpathlineto{\pgfqpoint{4.208707in}{2.165291in}}%
\pgfpathlineto{\pgfqpoint{4.208707in}{2.172506in}}%
\pgfpathlineto{\pgfqpoint{4.199770in}{2.172506in}}%
\pgfpathlineto{\pgfqpoint{4.199770in}{2.165291in}}%
\pgfpathclose%
\pgfusepath{fill}%
\end{pgfscope}%
\begin{pgfscope}%
\pgfpathrectangle{\pgfqpoint{3.722897in}{0.857143in}}{\pgfqpoint{2.627103in}{1.813434in}}%
\pgfusepath{clip}%
\pgfsetbuttcap%
\pgfsetmiterjoin%
\definecolor{currentfill}{rgb}{0.992771,0.707689,0.712380}%
\pgfsetfillcolor{currentfill}%
\pgfsetlinewidth{0.000000pt}%
\definecolor{currentstroke}{rgb}{0.000000,0.000000,0.000000}%
\pgfsetstrokecolor{currentstroke}%
\pgfsetstrokeopacity{0.000000}%
\pgfsetdash{}{0pt}%
\pgfpathmoveto{\pgfqpoint{4.210941in}{2.185148in}}%
\pgfpathlineto{\pgfqpoint{4.219877in}{2.185148in}}%
\pgfpathlineto{\pgfqpoint{4.219877in}{2.186339in}}%
\pgfpathlineto{\pgfqpoint{4.210941in}{2.186339in}}%
\pgfpathlineto{\pgfqpoint{4.210941in}{2.185148in}}%
\pgfpathclose%
\pgfusepath{fill}%
\end{pgfscope}%
\begin{pgfscope}%
\pgfpathrectangle{\pgfqpoint{3.722897in}{0.857143in}}{\pgfqpoint{2.627103in}{1.813434in}}%
\pgfusepath{clip}%
\pgfsetbuttcap%
\pgfsetmiterjoin%
\definecolor{currentfill}{rgb}{0.992771,0.707689,0.712380}%
\pgfsetfillcolor{currentfill}%
\pgfsetlinewidth{0.000000pt}%
\definecolor{currentstroke}{rgb}{0.000000,0.000000,0.000000}%
\pgfsetstrokecolor{currentstroke}%
\pgfsetstrokeopacity{0.000000}%
\pgfsetdash{}{0pt}%
\pgfpathmoveto{\pgfqpoint{4.222111in}{1.282474in}}%
\pgfpathlineto{\pgfqpoint{4.231048in}{1.282474in}}%
\pgfpathlineto{\pgfqpoint{4.231048in}{1.278158in}}%
\pgfpathlineto{\pgfqpoint{4.222111in}{1.278158in}}%
\pgfpathlineto{\pgfqpoint{4.222111in}{1.282474in}}%
\pgfpathclose%
\pgfusepath{fill}%
\end{pgfscope}%
\begin{pgfscope}%
\pgfpathrectangle{\pgfqpoint{3.722897in}{0.857143in}}{\pgfqpoint{2.627103in}{1.813434in}}%
\pgfusepath{clip}%
\pgfsetbuttcap%
\pgfsetmiterjoin%
\definecolor{currentfill}{rgb}{0.992771,0.707689,0.712380}%
\pgfsetfillcolor{currentfill}%
\pgfsetlinewidth{0.000000pt}%
\definecolor{currentstroke}{rgb}{0.000000,0.000000,0.000000}%
\pgfsetstrokecolor{currentstroke}%
\pgfsetstrokeopacity{0.000000}%
\pgfsetdash{}{0pt}%
\pgfpathmoveto{\pgfqpoint{4.233282in}{2.242489in}}%
\pgfpathlineto{\pgfqpoint{4.242218in}{2.242489in}}%
\pgfpathlineto{\pgfqpoint{4.242218in}{2.257243in}}%
\pgfpathlineto{\pgfqpoint{4.233282in}{2.257243in}}%
\pgfpathlineto{\pgfqpoint{4.233282in}{2.242489in}}%
\pgfpathclose%
\pgfusepath{fill}%
\end{pgfscope}%
\begin{pgfscope}%
\pgfpathrectangle{\pgfqpoint{3.722897in}{0.857143in}}{\pgfqpoint{2.627103in}{1.813434in}}%
\pgfusepath{clip}%
\pgfsetbuttcap%
\pgfsetmiterjoin%
\definecolor{currentfill}{rgb}{0.992771,0.707689,0.712380}%
\pgfsetfillcolor{currentfill}%
\pgfsetlinewidth{0.000000pt}%
\definecolor{currentstroke}{rgb}{0.000000,0.000000,0.000000}%
\pgfsetstrokecolor{currentstroke}%
\pgfsetstrokeopacity{0.000000}%
\pgfsetdash{}{0pt}%
\pgfpathmoveto{\pgfqpoint{4.244453in}{2.229038in}}%
\pgfpathlineto{\pgfqpoint{4.253389in}{2.229038in}}%
\pgfpathlineto{\pgfqpoint{4.253389in}{2.278010in}}%
\pgfpathlineto{\pgfqpoint{4.244453in}{2.278010in}}%
\pgfpathlineto{\pgfqpoint{4.244453in}{2.229038in}}%
\pgfpathclose%
\pgfusepath{fill}%
\end{pgfscope}%
\begin{pgfscope}%
\pgfpathrectangle{\pgfqpoint{3.722897in}{0.857143in}}{\pgfqpoint{2.627103in}{1.813434in}}%
\pgfusepath{clip}%
\pgfsetbuttcap%
\pgfsetmiterjoin%
\definecolor{currentfill}{rgb}{0.992771,0.707689,0.712380}%
\pgfsetfillcolor{currentfill}%
\pgfsetlinewidth{0.000000pt}%
\definecolor{currentstroke}{rgb}{0.000000,0.000000,0.000000}%
\pgfsetstrokecolor{currentstroke}%
\pgfsetstrokeopacity{0.000000}%
\pgfsetdash{}{0pt}%
\pgfpathmoveto{\pgfqpoint{4.255623in}{2.196939in}}%
\pgfpathlineto{\pgfqpoint{4.264560in}{2.196939in}}%
\pgfpathlineto{\pgfqpoint{4.264560in}{2.287248in}}%
\pgfpathlineto{\pgfqpoint{4.255623in}{2.287248in}}%
\pgfpathlineto{\pgfqpoint{4.255623in}{2.196939in}}%
\pgfpathclose%
\pgfusepath{fill}%
\end{pgfscope}%
\begin{pgfscope}%
\pgfpathrectangle{\pgfqpoint{3.722897in}{0.857143in}}{\pgfqpoint{2.627103in}{1.813434in}}%
\pgfusepath{clip}%
\pgfsetbuttcap%
\pgfsetmiterjoin%
\definecolor{currentfill}{rgb}{0.992771,0.707689,0.712380}%
\pgfsetfillcolor{currentfill}%
\pgfsetlinewidth{0.000000pt}%
\definecolor{currentstroke}{rgb}{0.000000,0.000000,0.000000}%
\pgfsetstrokecolor{currentstroke}%
\pgfsetstrokeopacity{0.000000}%
\pgfsetdash{}{0pt}%
\pgfpathmoveto{\pgfqpoint{4.266794in}{2.166916in}}%
\pgfpathlineto{\pgfqpoint{4.275730in}{2.166916in}}%
\pgfpathlineto{\pgfqpoint{4.275730in}{2.282798in}}%
\pgfpathlineto{\pgfqpoint{4.266794in}{2.282798in}}%
\pgfpathlineto{\pgfqpoint{4.266794in}{2.166916in}}%
\pgfpathclose%
\pgfusepath{fill}%
\end{pgfscope}%
\begin{pgfscope}%
\pgfpathrectangle{\pgfqpoint{3.722897in}{0.857143in}}{\pgfqpoint{2.627103in}{1.813434in}}%
\pgfusepath{clip}%
\pgfsetbuttcap%
\pgfsetmiterjoin%
\definecolor{currentfill}{rgb}{0.992771,0.707689,0.712380}%
\pgfsetfillcolor{currentfill}%
\pgfsetlinewidth{0.000000pt}%
\definecolor{currentstroke}{rgb}{0.000000,0.000000,0.000000}%
\pgfsetstrokecolor{currentstroke}%
\pgfsetstrokeopacity{0.000000}%
\pgfsetdash{}{0pt}%
\pgfpathmoveto{\pgfqpoint{4.277964in}{2.142394in}}%
\pgfpathlineto{\pgfqpoint{4.286901in}{2.142394in}}%
\pgfpathlineto{\pgfqpoint{4.286901in}{2.278163in}}%
\pgfpathlineto{\pgfqpoint{4.277964in}{2.278163in}}%
\pgfpathlineto{\pgfqpoint{4.277964in}{2.142394in}}%
\pgfpathclose%
\pgfusepath{fill}%
\end{pgfscope}%
\begin{pgfscope}%
\pgfpathrectangle{\pgfqpoint{3.722897in}{0.857143in}}{\pgfqpoint{2.627103in}{1.813434in}}%
\pgfusepath{clip}%
\pgfsetbuttcap%
\pgfsetmiterjoin%
\definecolor{currentfill}{rgb}{0.992771,0.707689,0.712380}%
\pgfsetfillcolor{currentfill}%
\pgfsetlinewidth{0.000000pt}%
\definecolor{currentstroke}{rgb}{0.000000,0.000000,0.000000}%
\pgfsetstrokecolor{currentstroke}%
\pgfsetstrokeopacity{0.000000}%
\pgfsetdash{}{0pt}%
\pgfpathmoveto{\pgfqpoint{4.289135in}{2.118820in}}%
\pgfpathlineto{\pgfqpoint{4.298071in}{2.118820in}}%
\pgfpathlineto{\pgfqpoint{4.298071in}{2.273055in}}%
\pgfpathlineto{\pgfqpoint{4.289135in}{2.273055in}}%
\pgfpathlineto{\pgfqpoint{4.289135in}{2.118820in}}%
\pgfpathclose%
\pgfusepath{fill}%
\end{pgfscope}%
\begin{pgfscope}%
\pgfpathrectangle{\pgfqpoint{3.722897in}{0.857143in}}{\pgfqpoint{2.627103in}{1.813434in}}%
\pgfusepath{clip}%
\pgfsetbuttcap%
\pgfsetmiterjoin%
\definecolor{currentfill}{rgb}{0.992771,0.707689,0.712380}%
\pgfsetfillcolor{currentfill}%
\pgfsetlinewidth{0.000000pt}%
\definecolor{currentstroke}{rgb}{0.000000,0.000000,0.000000}%
\pgfsetstrokecolor{currentstroke}%
\pgfsetstrokeopacity{0.000000}%
\pgfsetdash{}{0pt}%
\pgfpathmoveto{\pgfqpoint{4.300306in}{2.105051in}}%
\pgfpathlineto{\pgfqpoint{4.309242in}{2.105051in}}%
\pgfpathlineto{\pgfqpoint{4.309242in}{2.281494in}}%
\pgfpathlineto{\pgfqpoint{4.300306in}{2.281494in}}%
\pgfpathlineto{\pgfqpoint{4.300306in}{2.105051in}}%
\pgfpathclose%
\pgfusepath{fill}%
\end{pgfscope}%
\begin{pgfscope}%
\pgfpathrectangle{\pgfqpoint{3.722897in}{0.857143in}}{\pgfqpoint{2.627103in}{1.813434in}}%
\pgfusepath{clip}%
\pgfsetbuttcap%
\pgfsetmiterjoin%
\definecolor{currentfill}{rgb}{0.992771,0.707689,0.712380}%
\pgfsetfillcolor{currentfill}%
\pgfsetlinewidth{0.000000pt}%
\definecolor{currentstroke}{rgb}{0.000000,0.000000,0.000000}%
\pgfsetstrokecolor{currentstroke}%
\pgfsetstrokeopacity{0.000000}%
\pgfsetdash{}{0pt}%
\pgfpathmoveto{\pgfqpoint{4.311476in}{2.080482in}}%
\pgfpathlineto{\pgfqpoint{4.320413in}{2.080482in}}%
\pgfpathlineto{\pgfqpoint{4.320413in}{2.283801in}}%
\pgfpathlineto{\pgfqpoint{4.311476in}{2.283801in}}%
\pgfpathlineto{\pgfqpoint{4.311476in}{2.080482in}}%
\pgfpathclose%
\pgfusepath{fill}%
\end{pgfscope}%
\begin{pgfscope}%
\pgfpathrectangle{\pgfqpoint{3.722897in}{0.857143in}}{\pgfqpoint{2.627103in}{1.813434in}}%
\pgfusepath{clip}%
\pgfsetbuttcap%
\pgfsetmiterjoin%
\definecolor{currentfill}{rgb}{0.992771,0.707689,0.712380}%
\pgfsetfillcolor{currentfill}%
\pgfsetlinewidth{0.000000pt}%
\definecolor{currentstroke}{rgb}{0.000000,0.000000,0.000000}%
\pgfsetstrokecolor{currentstroke}%
\pgfsetstrokeopacity{0.000000}%
\pgfsetdash{}{0pt}%
\pgfpathmoveto{\pgfqpoint{4.322647in}{2.052765in}}%
\pgfpathlineto{\pgfqpoint{4.331583in}{2.052765in}}%
\pgfpathlineto{\pgfqpoint{4.331583in}{2.269164in}}%
\pgfpathlineto{\pgfqpoint{4.322647in}{2.269164in}}%
\pgfpathlineto{\pgfqpoint{4.322647in}{2.052765in}}%
\pgfpathclose%
\pgfusepath{fill}%
\end{pgfscope}%
\begin{pgfscope}%
\pgfpathrectangle{\pgfqpoint{3.722897in}{0.857143in}}{\pgfqpoint{2.627103in}{1.813434in}}%
\pgfusepath{clip}%
\pgfsetbuttcap%
\pgfsetmiterjoin%
\definecolor{currentfill}{rgb}{0.992771,0.707689,0.712380}%
\pgfsetfillcolor{currentfill}%
\pgfsetlinewidth{0.000000pt}%
\definecolor{currentstroke}{rgb}{0.000000,0.000000,0.000000}%
\pgfsetstrokecolor{currentstroke}%
\pgfsetstrokeopacity{0.000000}%
\pgfsetdash{}{0pt}%
\pgfpathmoveto{\pgfqpoint{4.333817in}{2.058073in}}%
\pgfpathlineto{\pgfqpoint{4.342754in}{2.058073in}}%
\pgfpathlineto{\pgfqpoint{4.342754in}{2.270756in}}%
\pgfpathlineto{\pgfqpoint{4.333817in}{2.270756in}}%
\pgfpathlineto{\pgfqpoint{4.333817in}{2.058073in}}%
\pgfpathclose%
\pgfusepath{fill}%
\end{pgfscope}%
\begin{pgfscope}%
\pgfpathrectangle{\pgfqpoint{3.722897in}{0.857143in}}{\pgfqpoint{2.627103in}{1.813434in}}%
\pgfusepath{clip}%
\pgfsetbuttcap%
\pgfsetmiterjoin%
\definecolor{currentfill}{rgb}{0.992771,0.707689,0.712380}%
\pgfsetfillcolor{currentfill}%
\pgfsetlinewidth{0.000000pt}%
\definecolor{currentstroke}{rgb}{0.000000,0.000000,0.000000}%
\pgfsetstrokecolor{currentstroke}%
\pgfsetstrokeopacity{0.000000}%
\pgfsetdash{}{0pt}%
\pgfpathmoveto{\pgfqpoint{4.344988in}{2.067201in}}%
\pgfpathlineto{\pgfqpoint{4.353925in}{2.067201in}}%
\pgfpathlineto{\pgfqpoint{4.353925in}{2.287465in}}%
\pgfpathlineto{\pgfqpoint{4.344988in}{2.287465in}}%
\pgfpathlineto{\pgfqpoint{4.344988in}{2.067201in}}%
\pgfpathclose%
\pgfusepath{fill}%
\end{pgfscope}%
\begin{pgfscope}%
\pgfpathrectangle{\pgfqpoint{3.722897in}{0.857143in}}{\pgfqpoint{2.627103in}{1.813434in}}%
\pgfusepath{clip}%
\pgfsetbuttcap%
\pgfsetmiterjoin%
\definecolor{currentfill}{rgb}{0.992771,0.707689,0.712380}%
\pgfsetfillcolor{currentfill}%
\pgfsetlinewidth{0.000000pt}%
\definecolor{currentstroke}{rgb}{0.000000,0.000000,0.000000}%
\pgfsetstrokecolor{currentstroke}%
\pgfsetstrokeopacity{0.000000}%
\pgfsetdash{}{0pt}%
\pgfpathmoveto{\pgfqpoint{4.356159in}{2.043939in}}%
\pgfpathlineto{\pgfqpoint{4.365095in}{2.043939in}}%
\pgfpathlineto{\pgfqpoint{4.365095in}{2.270548in}}%
\pgfpathlineto{\pgfqpoint{4.356159in}{2.270548in}}%
\pgfpathlineto{\pgfqpoint{4.356159in}{2.043939in}}%
\pgfpathclose%
\pgfusepath{fill}%
\end{pgfscope}%
\begin{pgfscope}%
\pgfpathrectangle{\pgfqpoint{3.722897in}{0.857143in}}{\pgfqpoint{2.627103in}{1.813434in}}%
\pgfusepath{clip}%
\pgfsetbuttcap%
\pgfsetmiterjoin%
\definecolor{currentfill}{rgb}{0.992771,0.707689,0.712380}%
\pgfsetfillcolor{currentfill}%
\pgfsetlinewidth{0.000000pt}%
\definecolor{currentstroke}{rgb}{0.000000,0.000000,0.000000}%
\pgfsetstrokecolor{currentstroke}%
\pgfsetstrokeopacity{0.000000}%
\pgfsetdash{}{0pt}%
\pgfpathmoveto{\pgfqpoint{4.367329in}{2.052888in}}%
\pgfpathlineto{\pgfqpoint{4.376266in}{2.052888in}}%
\pgfpathlineto{\pgfqpoint{4.376266in}{2.257061in}}%
\pgfpathlineto{\pgfqpoint{4.367329in}{2.257061in}}%
\pgfpathlineto{\pgfqpoint{4.367329in}{2.052888in}}%
\pgfpathclose%
\pgfusepath{fill}%
\end{pgfscope}%
\begin{pgfscope}%
\pgfpathrectangle{\pgfqpoint{3.722897in}{0.857143in}}{\pgfqpoint{2.627103in}{1.813434in}}%
\pgfusepath{clip}%
\pgfsetbuttcap%
\pgfsetmiterjoin%
\definecolor{currentfill}{rgb}{0.992771,0.707689,0.712380}%
\pgfsetfillcolor{currentfill}%
\pgfsetlinewidth{0.000000pt}%
\definecolor{currentstroke}{rgb}{0.000000,0.000000,0.000000}%
\pgfsetstrokecolor{currentstroke}%
\pgfsetstrokeopacity{0.000000}%
\pgfsetdash{}{0pt}%
\pgfpathmoveto{\pgfqpoint{4.378500in}{2.086623in}}%
\pgfpathlineto{\pgfqpoint{4.387436in}{2.086623in}}%
\pgfpathlineto{\pgfqpoint{4.387436in}{2.254637in}}%
\pgfpathlineto{\pgfqpoint{4.378500in}{2.254637in}}%
\pgfpathlineto{\pgfqpoint{4.378500in}{2.086623in}}%
\pgfpathclose%
\pgfusepath{fill}%
\end{pgfscope}%
\begin{pgfscope}%
\pgfpathrectangle{\pgfqpoint{3.722897in}{0.857143in}}{\pgfqpoint{2.627103in}{1.813434in}}%
\pgfusepath{clip}%
\pgfsetbuttcap%
\pgfsetmiterjoin%
\definecolor{currentfill}{rgb}{0.992771,0.707689,0.712380}%
\pgfsetfillcolor{currentfill}%
\pgfsetlinewidth{0.000000pt}%
\definecolor{currentstroke}{rgb}{0.000000,0.000000,0.000000}%
\pgfsetstrokecolor{currentstroke}%
\pgfsetstrokeopacity{0.000000}%
\pgfsetdash{}{0pt}%
\pgfpathmoveto{\pgfqpoint{4.389670in}{2.116757in}}%
\pgfpathlineto{\pgfqpoint{4.398607in}{2.116757in}}%
\pgfpathlineto{\pgfqpoint{4.398607in}{2.242804in}}%
\pgfpathlineto{\pgfqpoint{4.389670in}{2.242804in}}%
\pgfpathlineto{\pgfqpoint{4.389670in}{2.116757in}}%
\pgfpathclose%
\pgfusepath{fill}%
\end{pgfscope}%
\begin{pgfscope}%
\pgfpathrectangle{\pgfqpoint{3.722897in}{0.857143in}}{\pgfqpoint{2.627103in}{1.813434in}}%
\pgfusepath{clip}%
\pgfsetbuttcap%
\pgfsetmiterjoin%
\definecolor{currentfill}{rgb}{0.992771,0.707689,0.712380}%
\pgfsetfillcolor{currentfill}%
\pgfsetlinewidth{0.000000pt}%
\definecolor{currentstroke}{rgb}{0.000000,0.000000,0.000000}%
\pgfsetstrokecolor{currentstroke}%
\pgfsetstrokeopacity{0.000000}%
\pgfsetdash{}{0pt}%
\pgfpathmoveto{\pgfqpoint{4.400841in}{2.141787in}}%
\pgfpathlineto{\pgfqpoint{4.409778in}{2.141787in}}%
\pgfpathlineto{\pgfqpoint{4.409778in}{2.209859in}}%
\pgfpathlineto{\pgfqpoint{4.400841in}{2.209859in}}%
\pgfpathlineto{\pgfqpoint{4.400841in}{2.141787in}}%
\pgfpathclose%
\pgfusepath{fill}%
\end{pgfscope}%
\begin{pgfscope}%
\pgfpathrectangle{\pgfqpoint{3.722897in}{0.857143in}}{\pgfqpoint{2.627103in}{1.813434in}}%
\pgfusepath{clip}%
\pgfsetbuttcap%
\pgfsetmiterjoin%
\definecolor{currentfill}{rgb}{0.992771,0.707689,0.712380}%
\pgfsetfillcolor{currentfill}%
\pgfsetlinewidth{0.000000pt}%
\definecolor{currentstroke}{rgb}{0.000000,0.000000,0.000000}%
\pgfsetstrokecolor{currentstroke}%
\pgfsetstrokeopacity{0.000000}%
\pgfsetdash{}{0pt}%
\pgfpathmoveto{\pgfqpoint{4.412012in}{1.284742in}}%
\pgfpathlineto{\pgfqpoint{4.420948in}{1.284742in}}%
\pgfpathlineto{\pgfqpoint{4.420948in}{1.284075in}}%
\pgfpathlineto{\pgfqpoint{4.412012in}{1.284075in}}%
\pgfpathlineto{\pgfqpoint{4.412012in}{1.284742in}}%
\pgfpathclose%
\pgfusepath{fill}%
\end{pgfscope}%
\begin{pgfscope}%
\pgfpathrectangle{\pgfqpoint{3.722897in}{0.857143in}}{\pgfqpoint{2.627103in}{1.813434in}}%
\pgfusepath{clip}%
\pgfsetbuttcap%
\pgfsetmiterjoin%
\definecolor{currentfill}{rgb}{0.992771,0.707689,0.712380}%
\pgfsetfillcolor{currentfill}%
\pgfsetlinewidth{0.000000pt}%
\definecolor{currentstroke}{rgb}{0.000000,0.000000,0.000000}%
\pgfsetstrokecolor{currentstroke}%
\pgfsetstrokeopacity{0.000000}%
\pgfsetdash{}{0pt}%
\pgfpathmoveto{\pgfqpoint{4.423182in}{1.308121in}}%
\pgfpathlineto{\pgfqpoint{4.432119in}{1.308121in}}%
\pgfpathlineto{\pgfqpoint{4.432119in}{1.247145in}}%
\pgfpathlineto{\pgfqpoint{4.423182in}{1.247145in}}%
\pgfpathlineto{\pgfqpoint{4.423182in}{1.308121in}}%
\pgfpathclose%
\pgfusepath{fill}%
\end{pgfscope}%
\begin{pgfscope}%
\pgfpathrectangle{\pgfqpoint{3.722897in}{0.857143in}}{\pgfqpoint{2.627103in}{1.813434in}}%
\pgfusepath{clip}%
\pgfsetbuttcap%
\pgfsetmiterjoin%
\definecolor{currentfill}{rgb}{0.992771,0.707689,0.712380}%
\pgfsetfillcolor{currentfill}%
\pgfsetlinewidth{0.000000pt}%
\definecolor{currentstroke}{rgb}{0.000000,0.000000,0.000000}%
\pgfsetstrokecolor{currentstroke}%
\pgfsetstrokeopacity{0.000000}%
\pgfsetdash{}{0pt}%
\pgfpathmoveto{\pgfqpoint{4.434353in}{1.328892in}}%
\pgfpathlineto{\pgfqpoint{4.443289in}{1.328892in}}%
\pgfpathlineto{\pgfqpoint{4.443289in}{1.215759in}}%
\pgfpathlineto{\pgfqpoint{4.434353in}{1.215759in}}%
\pgfpathlineto{\pgfqpoint{4.434353in}{1.328892in}}%
\pgfpathclose%
\pgfusepath{fill}%
\end{pgfscope}%
\begin{pgfscope}%
\pgfpathrectangle{\pgfqpoint{3.722897in}{0.857143in}}{\pgfqpoint{2.627103in}{1.813434in}}%
\pgfusepath{clip}%
\pgfsetbuttcap%
\pgfsetmiterjoin%
\definecolor{currentfill}{rgb}{0.992771,0.707689,0.712380}%
\pgfsetfillcolor{currentfill}%
\pgfsetlinewidth{0.000000pt}%
\definecolor{currentstroke}{rgb}{0.000000,0.000000,0.000000}%
\pgfsetstrokecolor{currentstroke}%
\pgfsetstrokeopacity{0.000000}%
\pgfsetdash{}{0pt}%
\pgfpathmoveto{\pgfqpoint{4.445523in}{1.366779in}}%
\pgfpathlineto{\pgfqpoint{4.454460in}{1.366779in}}%
\pgfpathlineto{\pgfqpoint{4.454460in}{1.200139in}}%
\pgfpathlineto{\pgfqpoint{4.445523in}{1.200139in}}%
\pgfpathlineto{\pgfqpoint{4.445523in}{1.366779in}}%
\pgfpathclose%
\pgfusepath{fill}%
\end{pgfscope}%
\begin{pgfscope}%
\pgfpathrectangle{\pgfqpoint{3.722897in}{0.857143in}}{\pgfqpoint{2.627103in}{1.813434in}}%
\pgfusepath{clip}%
\pgfsetbuttcap%
\pgfsetmiterjoin%
\definecolor{currentfill}{rgb}{0.992771,0.707689,0.712380}%
\pgfsetfillcolor{currentfill}%
\pgfsetlinewidth{0.000000pt}%
\definecolor{currentstroke}{rgb}{0.000000,0.000000,0.000000}%
\pgfsetstrokecolor{currentstroke}%
\pgfsetstrokeopacity{0.000000}%
\pgfsetdash{}{0pt}%
\pgfpathmoveto{\pgfqpoint{4.456694in}{1.402606in}}%
\pgfpathlineto{\pgfqpoint{4.465631in}{1.402606in}}%
\pgfpathlineto{\pgfqpoint{4.465631in}{1.178664in}}%
\pgfpathlineto{\pgfqpoint{4.456694in}{1.178664in}}%
\pgfpathlineto{\pgfqpoint{4.456694in}{1.402606in}}%
\pgfpathclose%
\pgfusepath{fill}%
\end{pgfscope}%
\begin{pgfscope}%
\pgfpathrectangle{\pgfqpoint{3.722897in}{0.857143in}}{\pgfqpoint{2.627103in}{1.813434in}}%
\pgfusepath{clip}%
\pgfsetbuttcap%
\pgfsetmiterjoin%
\definecolor{currentfill}{rgb}{0.992771,0.707689,0.712380}%
\pgfsetfillcolor{currentfill}%
\pgfsetlinewidth{0.000000pt}%
\definecolor{currentstroke}{rgb}{0.000000,0.000000,0.000000}%
\pgfsetstrokecolor{currentstroke}%
\pgfsetstrokeopacity{0.000000}%
\pgfsetdash{}{0pt}%
\pgfpathmoveto{\pgfqpoint{4.467865in}{1.433714in}}%
\pgfpathlineto{\pgfqpoint{4.476801in}{1.433714in}}%
\pgfpathlineto{\pgfqpoint{4.476801in}{1.154769in}}%
\pgfpathlineto{\pgfqpoint{4.467865in}{1.154769in}}%
\pgfpathlineto{\pgfqpoint{4.467865in}{1.433714in}}%
\pgfpathclose%
\pgfusepath{fill}%
\end{pgfscope}%
\begin{pgfscope}%
\pgfpathrectangle{\pgfqpoint{3.722897in}{0.857143in}}{\pgfqpoint{2.627103in}{1.813434in}}%
\pgfusepath{clip}%
\pgfsetbuttcap%
\pgfsetmiterjoin%
\definecolor{currentfill}{rgb}{0.992771,0.707689,0.712380}%
\pgfsetfillcolor{currentfill}%
\pgfsetlinewidth{0.000000pt}%
\definecolor{currentstroke}{rgb}{0.000000,0.000000,0.000000}%
\pgfsetstrokecolor{currentstroke}%
\pgfsetstrokeopacity{0.000000}%
\pgfsetdash{}{0pt}%
\pgfpathmoveto{\pgfqpoint{4.479035in}{1.417293in}}%
\pgfpathlineto{\pgfqpoint{4.487972in}{1.417293in}}%
\pgfpathlineto{\pgfqpoint{4.487972in}{1.101637in}}%
\pgfpathlineto{\pgfqpoint{4.479035in}{1.101637in}}%
\pgfpathlineto{\pgfqpoint{4.479035in}{1.417293in}}%
\pgfpathclose%
\pgfusepath{fill}%
\end{pgfscope}%
\begin{pgfscope}%
\pgfpathrectangle{\pgfqpoint{3.722897in}{0.857143in}}{\pgfqpoint{2.627103in}{1.813434in}}%
\pgfusepath{clip}%
\pgfsetbuttcap%
\pgfsetmiterjoin%
\definecolor{currentfill}{rgb}{0.992771,0.707689,0.712380}%
\pgfsetfillcolor{currentfill}%
\pgfsetlinewidth{0.000000pt}%
\definecolor{currentstroke}{rgb}{0.000000,0.000000,0.000000}%
\pgfsetstrokecolor{currentstroke}%
\pgfsetstrokeopacity{0.000000}%
\pgfsetdash{}{0pt}%
\pgfpathmoveto{\pgfqpoint{4.490206in}{1.433952in}}%
\pgfpathlineto{\pgfqpoint{4.499142in}{1.433952in}}%
\pgfpathlineto{\pgfqpoint{4.499142in}{1.110862in}}%
\pgfpathlineto{\pgfqpoint{4.490206in}{1.110862in}}%
\pgfpathlineto{\pgfqpoint{4.490206in}{1.433952in}}%
\pgfpathclose%
\pgfusepath{fill}%
\end{pgfscope}%
\begin{pgfscope}%
\pgfpathrectangle{\pgfqpoint{3.722897in}{0.857143in}}{\pgfqpoint{2.627103in}{1.813434in}}%
\pgfusepath{clip}%
\pgfsetbuttcap%
\pgfsetmiterjoin%
\definecolor{currentfill}{rgb}{0.992771,0.707689,0.712380}%
\pgfsetfillcolor{currentfill}%
\pgfsetlinewidth{0.000000pt}%
\definecolor{currentstroke}{rgb}{0.000000,0.000000,0.000000}%
\pgfsetstrokecolor{currentstroke}%
\pgfsetstrokeopacity{0.000000}%
\pgfsetdash{}{0pt}%
\pgfpathmoveto{\pgfqpoint{4.501377in}{1.505491in}}%
\pgfpathlineto{\pgfqpoint{4.510313in}{1.505491in}}%
\pgfpathlineto{\pgfqpoint{4.510313in}{1.117178in}}%
\pgfpathlineto{\pgfqpoint{4.501377in}{1.117178in}}%
\pgfpathlineto{\pgfqpoint{4.501377in}{1.505491in}}%
\pgfpathclose%
\pgfusepath{fill}%
\end{pgfscope}%
\begin{pgfscope}%
\pgfpathrectangle{\pgfqpoint{3.722897in}{0.857143in}}{\pgfqpoint{2.627103in}{1.813434in}}%
\pgfusepath{clip}%
\pgfsetbuttcap%
\pgfsetmiterjoin%
\definecolor{currentfill}{rgb}{0.992771,0.707689,0.712380}%
\pgfsetfillcolor{currentfill}%
\pgfsetlinewidth{0.000000pt}%
\definecolor{currentstroke}{rgb}{0.000000,0.000000,0.000000}%
\pgfsetstrokecolor{currentstroke}%
\pgfsetstrokeopacity{0.000000}%
\pgfsetdash{}{0pt}%
\pgfpathmoveto{\pgfqpoint{4.512547in}{1.539465in}}%
\pgfpathlineto{\pgfqpoint{4.521484in}{1.539465in}}%
\pgfpathlineto{\pgfqpoint{4.521484in}{1.040550in}}%
\pgfpathlineto{\pgfqpoint{4.512547in}{1.040550in}}%
\pgfpathlineto{\pgfqpoint{4.512547in}{1.539465in}}%
\pgfpathclose%
\pgfusepath{fill}%
\end{pgfscope}%
\begin{pgfscope}%
\pgfpathrectangle{\pgfqpoint{3.722897in}{0.857143in}}{\pgfqpoint{2.627103in}{1.813434in}}%
\pgfusepath{clip}%
\pgfsetbuttcap%
\pgfsetmiterjoin%
\definecolor{currentfill}{rgb}{0.992771,0.707689,0.712380}%
\pgfsetfillcolor{currentfill}%
\pgfsetlinewidth{0.000000pt}%
\definecolor{currentstroke}{rgb}{0.000000,0.000000,0.000000}%
\pgfsetstrokecolor{currentstroke}%
\pgfsetstrokeopacity{0.000000}%
\pgfsetdash{}{0pt}%
\pgfpathmoveto{\pgfqpoint{4.523718in}{1.605312in}}%
\pgfpathlineto{\pgfqpoint{4.532654in}{1.605312in}}%
\pgfpathlineto{\pgfqpoint{4.532654in}{1.011770in}}%
\pgfpathlineto{\pgfqpoint{4.523718in}{1.011770in}}%
\pgfpathlineto{\pgfqpoint{4.523718in}{1.605312in}}%
\pgfpathclose%
\pgfusepath{fill}%
\end{pgfscope}%
\begin{pgfscope}%
\pgfpathrectangle{\pgfqpoint{3.722897in}{0.857143in}}{\pgfqpoint{2.627103in}{1.813434in}}%
\pgfusepath{clip}%
\pgfsetbuttcap%
\pgfsetmiterjoin%
\definecolor{currentfill}{rgb}{0.992771,0.707689,0.712380}%
\pgfsetfillcolor{currentfill}%
\pgfsetlinewidth{0.000000pt}%
\definecolor{currentstroke}{rgb}{0.000000,0.000000,0.000000}%
\pgfsetstrokecolor{currentstroke}%
\pgfsetstrokeopacity{0.000000}%
\pgfsetdash{}{0pt}%
\pgfpathmoveto{\pgfqpoint{4.534888in}{1.642486in}}%
\pgfpathlineto{\pgfqpoint{4.543825in}{1.642486in}}%
\pgfpathlineto{\pgfqpoint{4.543825in}{0.964322in}}%
\pgfpathlineto{\pgfqpoint{4.534888in}{0.964322in}}%
\pgfpathlineto{\pgfqpoint{4.534888in}{1.642486in}}%
\pgfpathclose%
\pgfusepath{fill}%
\end{pgfscope}%
\begin{pgfscope}%
\pgfpathrectangle{\pgfqpoint{3.722897in}{0.857143in}}{\pgfqpoint{2.627103in}{1.813434in}}%
\pgfusepath{clip}%
\pgfsetbuttcap%
\pgfsetmiterjoin%
\definecolor{currentfill}{rgb}{0.992771,0.707689,0.712380}%
\pgfsetfillcolor{currentfill}%
\pgfsetlinewidth{0.000000pt}%
\definecolor{currentstroke}{rgb}{0.000000,0.000000,0.000000}%
\pgfsetstrokecolor{currentstroke}%
\pgfsetstrokeopacity{0.000000}%
\pgfsetdash{}{0pt}%
\pgfpathmoveto{\pgfqpoint{4.546059in}{1.662950in}}%
\pgfpathlineto{\pgfqpoint{4.554995in}{1.662950in}}%
\pgfpathlineto{\pgfqpoint{4.554995in}{0.939572in}}%
\pgfpathlineto{\pgfqpoint{4.546059in}{0.939572in}}%
\pgfpathlineto{\pgfqpoint{4.546059in}{1.662950in}}%
\pgfpathclose%
\pgfusepath{fill}%
\end{pgfscope}%
\begin{pgfscope}%
\pgfpathrectangle{\pgfqpoint{3.722897in}{0.857143in}}{\pgfqpoint{2.627103in}{1.813434in}}%
\pgfusepath{clip}%
\pgfsetbuttcap%
\pgfsetmiterjoin%
\definecolor{currentfill}{rgb}{0.992771,0.707689,0.712380}%
\pgfsetfillcolor{currentfill}%
\pgfsetlinewidth{0.000000pt}%
\definecolor{currentstroke}{rgb}{0.000000,0.000000,0.000000}%
\pgfsetstrokecolor{currentstroke}%
\pgfsetstrokeopacity{0.000000}%
\pgfsetdash{}{0pt}%
\pgfpathmoveto{\pgfqpoint{4.557230in}{1.704103in}}%
\pgfpathlineto{\pgfqpoint{4.566166in}{1.704103in}}%
\pgfpathlineto{\pgfqpoint{4.566166in}{0.950522in}}%
\pgfpathlineto{\pgfqpoint{4.557230in}{0.950522in}}%
\pgfpathlineto{\pgfqpoint{4.557230in}{1.704103in}}%
\pgfpathclose%
\pgfusepath{fill}%
\end{pgfscope}%
\begin{pgfscope}%
\pgfpathrectangle{\pgfqpoint{3.722897in}{0.857143in}}{\pgfqpoint{2.627103in}{1.813434in}}%
\pgfusepath{clip}%
\pgfsetbuttcap%
\pgfsetmiterjoin%
\definecolor{currentfill}{rgb}{0.992771,0.707689,0.712380}%
\pgfsetfillcolor{currentfill}%
\pgfsetlinewidth{0.000000pt}%
\definecolor{currentstroke}{rgb}{0.000000,0.000000,0.000000}%
\pgfsetstrokecolor{currentstroke}%
\pgfsetstrokeopacity{0.000000}%
\pgfsetdash{}{0pt}%
\pgfpathmoveto{\pgfqpoint{4.568400in}{1.700383in}}%
\pgfpathlineto{\pgfqpoint{4.577337in}{1.700383in}}%
\pgfpathlineto{\pgfqpoint{4.577337in}{0.952610in}}%
\pgfpathlineto{\pgfqpoint{4.568400in}{0.952610in}}%
\pgfpathlineto{\pgfqpoint{4.568400in}{1.700383in}}%
\pgfpathclose%
\pgfusepath{fill}%
\end{pgfscope}%
\begin{pgfscope}%
\pgfpathrectangle{\pgfqpoint{3.722897in}{0.857143in}}{\pgfqpoint{2.627103in}{1.813434in}}%
\pgfusepath{clip}%
\pgfsetbuttcap%
\pgfsetmiterjoin%
\definecolor{currentfill}{rgb}{0.992771,0.707689,0.712380}%
\pgfsetfillcolor{currentfill}%
\pgfsetlinewidth{0.000000pt}%
\definecolor{currentstroke}{rgb}{0.000000,0.000000,0.000000}%
\pgfsetstrokecolor{currentstroke}%
\pgfsetstrokeopacity{0.000000}%
\pgfsetdash{}{0pt}%
\pgfpathmoveto{\pgfqpoint{4.579571in}{1.713012in}}%
\pgfpathlineto{\pgfqpoint{4.588507in}{1.713012in}}%
\pgfpathlineto{\pgfqpoint{4.588507in}{0.990429in}}%
\pgfpathlineto{\pgfqpoint{4.579571in}{0.990429in}}%
\pgfpathlineto{\pgfqpoint{4.579571in}{1.713012in}}%
\pgfpathclose%
\pgfusepath{fill}%
\end{pgfscope}%
\begin{pgfscope}%
\pgfpathrectangle{\pgfqpoint{3.722897in}{0.857143in}}{\pgfqpoint{2.627103in}{1.813434in}}%
\pgfusepath{clip}%
\pgfsetbuttcap%
\pgfsetmiterjoin%
\definecolor{currentfill}{rgb}{0.992771,0.707689,0.712380}%
\pgfsetfillcolor{currentfill}%
\pgfsetlinewidth{0.000000pt}%
\definecolor{currentstroke}{rgb}{0.000000,0.000000,0.000000}%
\pgfsetstrokecolor{currentstroke}%
\pgfsetstrokeopacity{0.000000}%
\pgfsetdash{}{0pt}%
\pgfpathmoveto{\pgfqpoint{4.590741in}{1.706465in}}%
\pgfpathlineto{\pgfqpoint{4.599678in}{1.706465in}}%
\pgfpathlineto{\pgfqpoint{4.599678in}{1.021191in}}%
\pgfpathlineto{\pgfqpoint{4.590741in}{1.021191in}}%
\pgfpathlineto{\pgfqpoint{4.590741in}{1.706465in}}%
\pgfpathclose%
\pgfusepath{fill}%
\end{pgfscope}%
\begin{pgfscope}%
\pgfpathrectangle{\pgfqpoint{3.722897in}{0.857143in}}{\pgfqpoint{2.627103in}{1.813434in}}%
\pgfusepath{clip}%
\pgfsetbuttcap%
\pgfsetmiterjoin%
\definecolor{currentfill}{rgb}{0.992771,0.707689,0.712380}%
\pgfsetfillcolor{currentfill}%
\pgfsetlinewidth{0.000000pt}%
\definecolor{currentstroke}{rgb}{0.000000,0.000000,0.000000}%
\pgfsetstrokecolor{currentstroke}%
\pgfsetstrokeopacity{0.000000}%
\pgfsetdash{}{0pt}%
\pgfpathmoveto{\pgfqpoint{4.601912in}{1.705952in}}%
\pgfpathlineto{\pgfqpoint{4.610848in}{1.705952in}}%
\pgfpathlineto{\pgfqpoint{4.610848in}{1.064794in}}%
\pgfpathlineto{\pgfqpoint{4.601912in}{1.064794in}}%
\pgfpathlineto{\pgfqpoint{4.601912in}{1.705952in}}%
\pgfpathclose%
\pgfusepath{fill}%
\end{pgfscope}%
\begin{pgfscope}%
\pgfpathrectangle{\pgfqpoint{3.722897in}{0.857143in}}{\pgfqpoint{2.627103in}{1.813434in}}%
\pgfusepath{clip}%
\pgfsetbuttcap%
\pgfsetmiterjoin%
\definecolor{currentfill}{rgb}{0.992771,0.707689,0.712380}%
\pgfsetfillcolor{currentfill}%
\pgfsetlinewidth{0.000000pt}%
\definecolor{currentstroke}{rgb}{0.000000,0.000000,0.000000}%
\pgfsetstrokecolor{currentstroke}%
\pgfsetstrokeopacity{0.000000}%
\pgfsetdash{}{0pt}%
\pgfpathmoveto{\pgfqpoint{4.613083in}{1.729078in}}%
\pgfpathlineto{\pgfqpoint{4.622019in}{1.729078in}}%
\pgfpathlineto{\pgfqpoint{4.622019in}{1.113237in}}%
\pgfpathlineto{\pgfqpoint{4.613083in}{1.113237in}}%
\pgfpathlineto{\pgfqpoint{4.613083in}{1.729078in}}%
\pgfpathclose%
\pgfusepath{fill}%
\end{pgfscope}%
\begin{pgfscope}%
\pgfpathrectangle{\pgfqpoint{3.722897in}{0.857143in}}{\pgfqpoint{2.627103in}{1.813434in}}%
\pgfusepath{clip}%
\pgfsetbuttcap%
\pgfsetmiterjoin%
\definecolor{currentfill}{rgb}{0.992771,0.707689,0.712380}%
\pgfsetfillcolor{currentfill}%
\pgfsetlinewidth{0.000000pt}%
\definecolor{currentstroke}{rgb}{0.000000,0.000000,0.000000}%
\pgfsetstrokecolor{currentstroke}%
\pgfsetstrokeopacity{0.000000}%
\pgfsetdash{}{0pt}%
\pgfpathmoveto{\pgfqpoint{4.624253in}{1.724197in}}%
\pgfpathlineto{\pgfqpoint{4.633190in}{1.724197in}}%
\pgfpathlineto{\pgfqpoint{4.633190in}{1.108884in}}%
\pgfpathlineto{\pgfqpoint{4.624253in}{1.108884in}}%
\pgfpathlineto{\pgfqpoint{4.624253in}{1.724197in}}%
\pgfpathclose%
\pgfusepath{fill}%
\end{pgfscope}%
\begin{pgfscope}%
\pgfpathrectangle{\pgfqpoint{3.722897in}{0.857143in}}{\pgfqpoint{2.627103in}{1.813434in}}%
\pgfusepath{clip}%
\pgfsetbuttcap%
\pgfsetmiterjoin%
\definecolor{currentfill}{rgb}{0.992771,0.707689,0.712380}%
\pgfsetfillcolor{currentfill}%
\pgfsetlinewidth{0.000000pt}%
\definecolor{currentstroke}{rgb}{0.000000,0.000000,0.000000}%
\pgfsetstrokecolor{currentstroke}%
\pgfsetstrokeopacity{0.000000}%
\pgfsetdash{}{0pt}%
\pgfpathmoveto{\pgfqpoint{4.635424in}{1.741595in}}%
\pgfpathlineto{\pgfqpoint{4.644360in}{1.741595in}}%
\pgfpathlineto{\pgfqpoint{4.644360in}{1.124325in}}%
\pgfpathlineto{\pgfqpoint{4.635424in}{1.124325in}}%
\pgfpathlineto{\pgfqpoint{4.635424in}{1.741595in}}%
\pgfpathclose%
\pgfusepath{fill}%
\end{pgfscope}%
\begin{pgfscope}%
\pgfpathrectangle{\pgfqpoint{3.722897in}{0.857143in}}{\pgfqpoint{2.627103in}{1.813434in}}%
\pgfusepath{clip}%
\pgfsetbuttcap%
\pgfsetmiterjoin%
\definecolor{currentfill}{rgb}{0.992771,0.707689,0.712380}%
\pgfsetfillcolor{currentfill}%
\pgfsetlinewidth{0.000000pt}%
\definecolor{currentstroke}{rgb}{0.000000,0.000000,0.000000}%
\pgfsetstrokecolor{currentstroke}%
\pgfsetstrokeopacity{0.000000}%
\pgfsetdash{}{0pt}%
\pgfpathmoveto{\pgfqpoint{4.646594in}{1.749603in}}%
\pgfpathlineto{\pgfqpoint{4.655531in}{1.749603in}}%
\pgfpathlineto{\pgfqpoint{4.655531in}{1.112948in}}%
\pgfpathlineto{\pgfqpoint{4.646594in}{1.112948in}}%
\pgfpathlineto{\pgfqpoint{4.646594in}{1.749603in}}%
\pgfpathclose%
\pgfusepath{fill}%
\end{pgfscope}%
\begin{pgfscope}%
\pgfpathrectangle{\pgfqpoint{3.722897in}{0.857143in}}{\pgfqpoint{2.627103in}{1.813434in}}%
\pgfusepath{clip}%
\pgfsetbuttcap%
\pgfsetmiterjoin%
\definecolor{currentfill}{rgb}{0.992771,0.707689,0.712380}%
\pgfsetfillcolor{currentfill}%
\pgfsetlinewidth{0.000000pt}%
\definecolor{currentstroke}{rgb}{0.000000,0.000000,0.000000}%
\pgfsetstrokecolor{currentstroke}%
\pgfsetstrokeopacity{0.000000}%
\pgfsetdash{}{0pt}%
\pgfpathmoveto{\pgfqpoint{4.657765in}{1.757684in}}%
\pgfpathlineto{\pgfqpoint{4.666701in}{1.757684in}}%
\pgfpathlineto{\pgfqpoint{4.666701in}{1.099371in}}%
\pgfpathlineto{\pgfqpoint{4.657765in}{1.099371in}}%
\pgfpathlineto{\pgfqpoint{4.657765in}{1.757684in}}%
\pgfpathclose%
\pgfusepath{fill}%
\end{pgfscope}%
\begin{pgfscope}%
\pgfpathrectangle{\pgfqpoint{3.722897in}{0.857143in}}{\pgfqpoint{2.627103in}{1.813434in}}%
\pgfusepath{clip}%
\pgfsetbuttcap%
\pgfsetmiterjoin%
\definecolor{currentfill}{rgb}{0.992771,0.707689,0.712380}%
\pgfsetfillcolor{currentfill}%
\pgfsetlinewidth{0.000000pt}%
\definecolor{currentstroke}{rgb}{0.000000,0.000000,0.000000}%
\pgfsetstrokecolor{currentstroke}%
\pgfsetstrokeopacity{0.000000}%
\pgfsetdash{}{0pt}%
\pgfpathmoveto{\pgfqpoint{4.668936in}{1.762383in}}%
\pgfpathlineto{\pgfqpoint{4.677872in}{1.762383in}}%
\pgfpathlineto{\pgfqpoint{4.677872in}{1.088120in}}%
\pgfpathlineto{\pgfqpoint{4.668936in}{1.088120in}}%
\pgfpathlineto{\pgfqpoint{4.668936in}{1.762383in}}%
\pgfpathclose%
\pgfusepath{fill}%
\end{pgfscope}%
\begin{pgfscope}%
\pgfpathrectangle{\pgfqpoint{3.722897in}{0.857143in}}{\pgfqpoint{2.627103in}{1.813434in}}%
\pgfusepath{clip}%
\pgfsetbuttcap%
\pgfsetmiterjoin%
\definecolor{currentfill}{rgb}{0.992771,0.707689,0.712380}%
\pgfsetfillcolor{currentfill}%
\pgfsetlinewidth{0.000000pt}%
\definecolor{currentstroke}{rgb}{0.000000,0.000000,0.000000}%
\pgfsetstrokecolor{currentstroke}%
\pgfsetstrokeopacity{0.000000}%
\pgfsetdash{}{0pt}%
\pgfpathmoveto{\pgfqpoint{4.680106in}{1.782390in}}%
\pgfpathlineto{\pgfqpoint{4.689043in}{1.782390in}}%
\pgfpathlineto{\pgfqpoint{4.689043in}{1.103137in}}%
\pgfpathlineto{\pgfqpoint{4.680106in}{1.103137in}}%
\pgfpathlineto{\pgfqpoint{4.680106in}{1.782390in}}%
\pgfpathclose%
\pgfusepath{fill}%
\end{pgfscope}%
\begin{pgfscope}%
\pgfpathrectangle{\pgfqpoint{3.722897in}{0.857143in}}{\pgfqpoint{2.627103in}{1.813434in}}%
\pgfusepath{clip}%
\pgfsetbuttcap%
\pgfsetmiterjoin%
\definecolor{currentfill}{rgb}{0.992771,0.707689,0.712380}%
\pgfsetfillcolor{currentfill}%
\pgfsetlinewidth{0.000000pt}%
\definecolor{currentstroke}{rgb}{0.000000,0.000000,0.000000}%
\pgfsetstrokecolor{currentstroke}%
\pgfsetstrokeopacity{0.000000}%
\pgfsetdash{}{0pt}%
\pgfpathmoveto{\pgfqpoint{4.691277in}{1.779309in}}%
\pgfpathlineto{\pgfqpoint{4.700213in}{1.779309in}}%
\pgfpathlineto{\pgfqpoint{4.700213in}{1.099488in}}%
\pgfpathlineto{\pgfqpoint{4.691277in}{1.099488in}}%
\pgfpathlineto{\pgfqpoint{4.691277in}{1.779309in}}%
\pgfpathclose%
\pgfusepath{fill}%
\end{pgfscope}%
\begin{pgfscope}%
\pgfpathrectangle{\pgfqpoint{3.722897in}{0.857143in}}{\pgfqpoint{2.627103in}{1.813434in}}%
\pgfusepath{clip}%
\pgfsetbuttcap%
\pgfsetmiterjoin%
\definecolor{currentfill}{rgb}{0.992771,0.707689,0.712380}%
\pgfsetfillcolor{currentfill}%
\pgfsetlinewidth{0.000000pt}%
\definecolor{currentstroke}{rgb}{0.000000,0.000000,0.000000}%
\pgfsetstrokecolor{currentstroke}%
\pgfsetstrokeopacity{0.000000}%
\pgfsetdash{}{0pt}%
\pgfpathmoveto{\pgfqpoint{4.702447in}{1.776801in}}%
\pgfpathlineto{\pgfqpoint{4.711384in}{1.776801in}}%
\pgfpathlineto{\pgfqpoint{4.711384in}{1.101452in}}%
\pgfpathlineto{\pgfqpoint{4.702447in}{1.101452in}}%
\pgfpathlineto{\pgfqpoint{4.702447in}{1.776801in}}%
\pgfpathclose%
\pgfusepath{fill}%
\end{pgfscope}%
\begin{pgfscope}%
\pgfpathrectangle{\pgfqpoint{3.722897in}{0.857143in}}{\pgfqpoint{2.627103in}{1.813434in}}%
\pgfusepath{clip}%
\pgfsetbuttcap%
\pgfsetmiterjoin%
\definecolor{currentfill}{rgb}{0.992771,0.707689,0.712380}%
\pgfsetfillcolor{currentfill}%
\pgfsetlinewidth{0.000000pt}%
\definecolor{currentstroke}{rgb}{0.000000,0.000000,0.000000}%
\pgfsetstrokecolor{currentstroke}%
\pgfsetstrokeopacity{0.000000}%
\pgfsetdash{}{0pt}%
\pgfpathmoveto{\pgfqpoint{4.713618in}{1.786441in}}%
\pgfpathlineto{\pgfqpoint{4.722554in}{1.786441in}}%
\pgfpathlineto{\pgfqpoint{4.722554in}{1.131505in}}%
\pgfpathlineto{\pgfqpoint{4.713618in}{1.131505in}}%
\pgfpathlineto{\pgfqpoint{4.713618in}{1.786441in}}%
\pgfpathclose%
\pgfusepath{fill}%
\end{pgfscope}%
\begin{pgfscope}%
\pgfpathrectangle{\pgfqpoint{3.722897in}{0.857143in}}{\pgfqpoint{2.627103in}{1.813434in}}%
\pgfusepath{clip}%
\pgfsetbuttcap%
\pgfsetmiterjoin%
\definecolor{currentfill}{rgb}{0.992771,0.707689,0.712380}%
\pgfsetfillcolor{currentfill}%
\pgfsetlinewidth{0.000000pt}%
\definecolor{currentstroke}{rgb}{0.000000,0.000000,0.000000}%
\pgfsetstrokecolor{currentstroke}%
\pgfsetstrokeopacity{0.000000}%
\pgfsetdash{}{0pt}%
\pgfpathmoveto{\pgfqpoint{4.724789in}{1.782775in}}%
\pgfpathlineto{\pgfqpoint{4.733725in}{1.782775in}}%
\pgfpathlineto{\pgfqpoint{4.733725in}{1.130070in}}%
\pgfpathlineto{\pgfqpoint{4.724789in}{1.130070in}}%
\pgfpathlineto{\pgfqpoint{4.724789in}{1.782775in}}%
\pgfpathclose%
\pgfusepath{fill}%
\end{pgfscope}%
\begin{pgfscope}%
\pgfpathrectangle{\pgfqpoint{3.722897in}{0.857143in}}{\pgfqpoint{2.627103in}{1.813434in}}%
\pgfusepath{clip}%
\pgfsetbuttcap%
\pgfsetmiterjoin%
\definecolor{currentfill}{rgb}{0.992771,0.707689,0.712380}%
\pgfsetfillcolor{currentfill}%
\pgfsetlinewidth{0.000000pt}%
\definecolor{currentstroke}{rgb}{0.000000,0.000000,0.000000}%
\pgfsetstrokecolor{currentstroke}%
\pgfsetstrokeopacity{0.000000}%
\pgfsetdash{}{0pt}%
\pgfpathmoveto{\pgfqpoint{4.735959in}{1.780521in}}%
\pgfpathlineto{\pgfqpoint{4.744896in}{1.780521in}}%
\pgfpathlineto{\pgfqpoint{4.744896in}{1.134726in}}%
\pgfpathlineto{\pgfqpoint{4.735959in}{1.134726in}}%
\pgfpathlineto{\pgfqpoint{4.735959in}{1.780521in}}%
\pgfpathclose%
\pgfusepath{fill}%
\end{pgfscope}%
\begin{pgfscope}%
\pgfpathrectangle{\pgfqpoint{3.722897in}{0.857143in}}{\pgfqpoint{2.627103in}{1.813434in}}%
\pgfusepath{clip}%
\pgfsetbuttcap%
\pgfsetmiterjoin%
\definecolor{currentfill}{rgb}{0.992771,0.707689,0.712380}%
\pgfsetfillcolor{currentfill}%
\pgfsetlinewidth{0.000000pt}%
\definecolor{currentstroke}{rgb}{0.000000,0.000000,0.000000}%
\pgfsetstrokecolor{currentstroke}%
\pgfsetstrokeopacity{0.000000}%
\pgfsetdash{}{0pt}%
\pgfpathmoveto{\pgfqpoint{4.747130in}{1.777804in}}%
\pgfpathlineto{\pgfqpoint{4.756066in}{1.777804in}}%
\pgfpathlineto{\pgfqpoint{4.756066in}{1.152565in}}%
\pgfpathlineto{\pgfqpoint{4.747130in}{1.152565in}}%
\pgfpathlineto{\pgfqpoint{4.747130in}{1.777804in}}%
\pgfpathclose%
\pgfusepath{fill}%
\end{pgfscope}%
\begin{pgfscope}%
\pgfpathrectangle{\pgfqpoint{3.722897in}{0.857143in}}{\pgfqpoint{2.627103in}{1.813434in}}%
\pgfusepath{clip}%
\pgfsetbuttcap%
\pgfsetmiterjoin%
\definecolor{currentfill}{rgb}{0.992771,0.707689,0.712380}%
\pgfsetfillcolor{currentfill}%
\pgfsetlinewidth{0.000000pt}%
\definecolor{currentstroke}{rgb}{0.000000,0.000000,0.000000}%
\pgfsetstrokecolor{currentstroke}%
\pgfsetstrokeopacity{0.000000}%
\pgfsetdash{}{0pt}%
\pgfpathmoveto{\pgfqpoint{4.758300in}{1.781370in}}%
\pgfpathlineto{\pgfqpoint{4.767237in}{1.781370in}}%
\pgfpathlineto{\pgfqpoint{4.767237in}{1.181850in}}%
\pgfpathlineto{\pgfqpoint{4.758300in}{1.181850in}}%
\pgfpathlineto{\pgfqpoint{4.758300in}{1.781370in}}%
\pgfpathclose%
\pgfusepath{fill}%
\end{pgfscope}%
\begin{pgfscope}%
\pgfpathrectangle{\pgfqpoint{3.722897in}{0.857143in}}{\pgfqpoint{2.627103in}{1.813434in}}%
\pgfusepath{clip}%
\pgfsetbuttcap%
\pgfsetmiterjoin%
\definecolor{currentfill}{rgb}{0.992771,0.707689,0.712380}%
\pgfsetfillcolor{currentfill}%
\pgfsetlinewidth{0.000000pt}%
\definecolor{currentstroke}{rgb}{0.000000,0.000000,0.000000}%
\pgfsetstrokecolor{currentstroke}%
\pgfsetstrokeopacity{0.000000}%
\pgfsetdash{}{0pt}%
\pgfpathmoveto{\pgfqpoint{4.769471in}{1.779817in}}%
\pgfpathlineto{\pgfqpoint{4.778408in}{1.779817in}}%
\pgfpathlineto{\pgfqpoint{4.778408in}{1.193404in}}%
\pgfpathlineto{\pgfqpoint{4.769471in}{1.193404in}}%
\pgfpathlineto{\pgfqpoint{4.769471in}{1.779817in}}%
\pgfpathclose%
\pgfusepath{fill}%
\end{pgfscope}%
\begin{pgfscope}%
\pgfpathrectangle{\pgfqpoint{3.722897in}{0.857143in}}{\pgfqpoint{2.627103in}{1.813434in}}%
\pgfusepath{clip}%
\pgfsetbuttcap%
\pgfsetmiterjoin%
\definecolor{currentfill}{rgb}{0.992771,0.707689,0.712380}%
\pgfsetfillcolor{currentfill}%
\pgfsetlinewidth{0.000000pt}%
\definecolor{currentstroke}{rgb}{0.000000,0.000000,0.000000}%
\pgfsetstrokecolor{currentstroke}%
\pgfsetstrokeopacity{0.000000}%
\pgfsetdash{}{0pt}%
\pgfpathmoveto{\pgfqpoint{4.780642in}{1.775286in}}%
\pgfpathlineto{\pgfqpoint{4.789578in}{1.775286in}}%
\pgfpathlineto{\pgfqpoint{4.789578in}{1.216388in}}%
\pgfpathlineto{\pgfqpoint{4.780642in}{1.216388in}}%
\pgfpathlineto{\pgfqpoint{4.780642in}{1.775286in}}%
\pgfpathclose%
\pgfusepath{fill}%
\end{pgfscope}%
\begin{pgfscope}%
\pgfpathrectangle{\pgfqpoint{3.722897in}{0.857143in}}{\pgfqpoint{2.627103in}{1.813434in}}%
\pgfusepath{clip}%
\pgfsetbuttcap%
\pgfsetmiterjoin%
\definecolor{currentfill}{rgb}{0.992771,0.707689,0.712380}%
\pgfsetfillcolor{currentfill}%
\pgfsetlinewidth{0.000000pt}%
\definecolor{currentstroke}{rgb}{0.000000,0.000000,0.000000}%
\pgfsetstrokecolor{currentstroke}%
\pgfsetstrokeopacity{0.000000}%
\pgfsetdash{}{0pt}%
\pgfpathmoveto{\pgfqpoint{4.791812in}{1.783976in}}%
\pgfpathlineto{\pgfqpoint{4.800749in}{1.783976in}}%
\pgfpathlineto{\pgfqpoint{4.800749in}{1.246202in}}%
\pgfpathlineto{\pgfqpoint{4.791812in}{1.246202in}}%
\pgfpathlineto{\pgfqpoint{4.791812in}{1.783976in}}%
\pgfpathclose%
\pgfusepath{fill}%
\end{pgfscope}%
\begin{pgfscope}%
\pgfpathrectangle{\pgfqpoint{3.722897in}{0.857143in}}{\pgfqpoint{2.627103in}{1.813434in}}%
\pgfusepath{clip}%
\pgfsetbuttcap%
\pgfsetmiterjoin%
\definecolor{currentfill}{rgb}{0.992771,0.707689,0.712380}%
\pgfsetfillcolor{currentfill}%
\pgfsetlinewidth{0.000000pt}%
\definecolor{currentstroke}{rgb}{0.000000,0.000000,0.000000}%
\pgfsetstrokecolor{currentstroke}%
\pgfsetstrokeopacity{0.000000}%
\pgfsetdash{}{0pt}%
\pgfpathmoveto{\pgfqpoint{4.802983in}{1.783707in}}%
\pgfpathlineto{\pgfqpoint{4.811919in}{1.783707in}}%
\pgfpathlineto{\pgfqpoint{4.811919in}{1.262455in}}%
\pgfpathlineto{\pgfqpoint{4.802983in}{1.262455in}}%
\pgfpathlineto{\pgfqpoint{4.802983in}{1.783707in}}%
\pgfpathclose%
\pgfusepath{fill}%
\end{pgfscope}%
\begin{pgfscope}%
\pgfpathrectangle{\pgfqpoint{3.722897in}{0.857143in}}{\pgfqpoint{2.627103in}{1.813434in}}%
\pgfusepath{clip}%
\pgfsetbuttcap%
\pgfsetmiterjoin%
\definecolor{currentfill}{rgb}{0.992771,0.707689,0.712380}%
\pgfsetfillcolor{currentfill}%
\pgfsetlinewidth{0.000000pt}%
\definecolor{currentstroke}{rgb}{0.000000,0.000000,0.000000}%
\pgfsetstrokecolor{currentstroke}%
\pgfsetstrokeopacity{0.000000}%
\pgfsetdash{}{0pt}%
\pgfpathmoveto{\pgfqpoint{4.814153in}{1.776881in}}%
\pgfpathlineto{\pgfqpoint{4.823090in}{1.776881in}}%
\pgfpathlineto{\pgfqpoint{4.823090in}{1.269378in}}%
\pgfpathlineto{\pgfqpoint{4.814153in}{1.269378in}}%
\pgfpathlineto{\pgfqpoint{4.814153in}{1.776881in}}%
\pgfpathclose%
\pgfusepath{fill}%
\end{pgfscope}%
\begin{pgfscope}%
\pgfpathrectangle{\pgfqpoint{3.722897in}{0.857143in}}{\pgfqpoint{2.627103in}{1.813434in}}%
\pgfusepath{clip}%
\pgfsetbuttcap%
\pgfsetmiterjoin%
\definecolor{currentfill}{rgb}{0.992771,0.707689,0.712380}%
\pgfsetfillcolor{currentfill}%
\pgfsetlinewidth{0.000000pt}%
\definecolor{currentstroke}{rgb}{0.000000,0.000000,0.000000}%
\pgfsetstrokecolor{currentstroke}%
\pgfsetstrokeopacity{0.000000}%
\pgfsetdash{}{0pt}%
\pgfpathmoveto{\pgfqpoint{4.825324in}{1.773167in}}%
\pgfpathlineto{\pgfqpoint{4.834261in}{1.773167in}}%
\pgfpathlineto{\pgfqpoint{4.834261in}{1.280634in}}%
\pgfpathlineto{\pgfqpoint{4.825324in}{1.280634in}}%
\pgfpathlineto{\pgfqpoint{4.825324in}{1.773167in}}%
\pgfpathclose%
\pgfusepath{fill}%
\end{pgfscope}%
\begin{pgfscope}%
\pgfpathrectangle{\pgfqpoint{3.722897in}{0.857143in}}{\pgfqpoint{2.627103in}{1.813434in}}%
\pgfusepath{clip}%
\pgfsetbuttcap%
\pgfsetmiterjoin%
\definecolor{currentfill}{rgb}{0.992771,0.707689,0.712380}%
\pgfsetfillcolor{currentfill}%
\pgfsetlinewidth{0.000000pt}%
\definecolor{currentstroke}{rgb}{0.000000,0.000000,0.000000}%
\pgfsetstrokecolor{currentstroke}%
\pgfsetstrokeopacity{0.000000}%
\pgfsetdash{}{0pt}%
\pgfpathmoveto{\pgfqpoint{4.836495in}{1.787638in}}%
\pgfpathlineto{\pgfqpoint{4.845431in}{1.787638in}}%
\pgfpathlineto{\pgfqpoint{4.845431in}{1.311857in}}%
\pgfpathlineto{\pgfqpoint{4.836495in}{1.311857in}}%
\pgfpathlineto{\pgfqpoint{4.836495in}{1.787638in}}%
\pgfpathclose%
\pgfusepath{fill}%
\end{pgfscope}%
\begin{pgfscope}%
\pgfpathrectangle{\pgfqpoint{3.722897in}{0.857143in}}{\pgfqpoint{2.627103in}{1.813434in}}%
\pgfusepath{clip}%
\pgfsetbuttcap%
\pgfsetmiterjoin%
\definecolor{currentfill}{rgb}{0.992771,0.707689,0.712380}%
\pgfsetfillcolor{currentfill}%
\pgfsetlinewidth{0.000000pt}%
\definecolor{currentstroke}{rgb}{0.000000,0.000000,0.000000}%
\pgfsetstrokecolor{currentstroke}%
\pgfsetstrokeopacity{0.000000}%
\pgfsetdash{}{0pt}%
\pgfpathmoveto{\pgfqpoint{4.847665in}{1.785822in}}%
\pgfpathlineto{\pgfqpoint{4.856602in}{1.785822in}}%
\pgfpathlineto{\pgfqpoint{4.856602in}{1.315268in}}%
\pgfpathlineto{\pgfqpoint{4.847665in}{1.315268in}}%
\pgfpathlineto{\pgfqpoint{4.847665in}{1.785822in}}%
\pgfpathclose%
\pgfusepath{fill}%
\end{pgfscope}%
\begin{pgfscope}%
\pgfpathrectangle{\pgfqpoint{3.722897in}{0.857143in}}{\pgfqpoint{2.627103in}{1.813434in}}%
\pgfusepath{clip}%
\pgfsetbuttcap%
\pgfsetmiterjoin%
\definecolor{currentfill}{rgb}{0.992771,0.707689,0.712380}%
\pgfsetfillcolor{currentfill}%
\pgfsetlinewidth{0.000000pt}%
\definecolor{currentstroke}{rgb}{0.000000,0.000000,0.000000}%
\pgfsetstrokecolor{currentstroke}%
\pgfsetstrokeopacity{0.000000}%
\pgfsetdash{}{0pt}%
\pgfpathmoveto{\pgfqpoint{4.858836in}{1.784672in}}%
\pgfpathlineto{\pgfqpoint{4.867772in}{1.784672in}}%
\pgfpathlineto{\pgfqpoint{4.867772in}{1.308774in}}%
\pgfpathlineto{\pgfqpoint{4.858836in}{1.308774in}}%
\pgfpathlineto{\pgfqpoint{4.858836in}{1.784672in}}%
\pgfpathclose%
\pgfusepath{fill}%
\end{pgfscope}%
\begin{pgfscope}%
\pgfpathrectangle{\pgfqpoint{3.722897in}{0.857143in}}{\pgfqpoint{2.627103in}{1.813434in}}%
\pgfusepath{clip}%
\pgfsetbuttcap%
\pgfsetmiterjoin%
\definecolor{currentfill}{rgb}{0.992771,0.707689,0.712380}%
\pgfsetfillcolor{currentfill}%
\pgfsetlinewidth{0.000000pt}%
\definecolor{currentstroke}{rgb}{0.000000,0.000000,0.000000}%
\pgfsetstrokecolor{currentstroke}%
\pgfsetstrokeopacity{0.000000}%
\pgfsetdash{}{0pt}%
\pgfpathmoveto{\pgfqpoint{4.870006in}{1.783801in}}%
\pgfpathlineto{\pgfqpoint{4.878943in}{1.783801in}}%
\pgfpathlineto{\pgfqpoint{4.878943in}{1.289860in}}%
\pgfpathlineto{\pgfqpoint{4.870006in}{1.289860in}}%
\pgfpathlineto{\pgfqpoint{4.870006in}{1.783801in}}%
\pgfpathclose%
\pgfusepath{fill}%
\end{pgfscope}%
\begin{pgfscope}%
\pgfpathrectangle{\pgfqpoint{3.722897in}{0.857143in}}{\pgfqpoint{2.627103in}{1.813434in}}%
\pgfusepath{clip}%
\pgfsetbuttcap%
\pgfsetmiterjoin%
\definecolor{currentfill}{rgb}{0.992771,0.707689,0.712380}%
\pgfsetfillcolor{currentfill}%
\pgfsetlinewidth{0.000000pt}%
\definecolor{currentstroke}{rgb}{0.000000,0.000000,0.000000}%
\pgfsetstrokecolor{currentstroke}%
\pgfsetstrokeopacity{0.000000}%
\pgfsetdash{}{0pt}%
\pgfpathmoveto{\pgfqpoint{4.881177in}{1.782282in}}%
\pgfpathlineto{\pgfqpoint{4.890114in}{1.782282in}}%
\pgfpathlineto{\pgfqpoint{4.890114in}{1.264736in}}%
\pgfpathlineto{\pgfqpoint{4.881177in}{1.264736in}}%
\pgfpathlineto{\pgfqpoint{4.881177in}{1.782282in}}%
\pgfpathclose%
\pgfusepath{fill}%
\end{pgfscope}%
\begin{pgfscope}%
\pgfpathrectangle{\pgfqpoint{3.722897in}{0.857143in}}{\pgfqpoint{2.627103in}{1.813434in}}%
\pgfusepath{clip}%
\pgfsetbuttcap%
\pgfsetmiterjoin%
\definecolor{currentfill}{rgb}{0.992771,0.707689,0.712380}%
\pgfsetfillcolor{currentfill}%
\pgfsetlinewidth{0.000000pt}%
\definecolor{currentstroke}{rgb}{0.000000,0.000000,0.000000}%
\pgfsetstrokecolor{currentstroke}%
\pgfsetstrokeopacity{0.000000}%
\pgfsetdash{}{0pt}%
\pgfpathmoveto{\pgfqpoint{4.892348in}{1.782091in}}%
\pgfpathlineto{\pgfqpoint{4.901284in}{1.782091in}}%
\pgfpathlineto{\pgfqpoint{4.901284in}{1.261934in}}%
\pgfpathlineto{\pgfqpoint{4.892348in}{1.261934in}}%
\pgfpathlineto{\pgfqpoint{4.892348in}{1.782091in}}%
\pgfpathclose%
\pgfusepath{fill}%
\end{pgfscope}%
\begin{pgfscope}%
\pgfpathrectangle{\pgfqpoint{3.722897in}{0.857143in}}{\pgfqpoint{2.627103in}{1.813434in}}%
\pgfusepath{clip}%
\pgfsetbuttcap%
\pgfsetmiterjoin%
\definecolor{currentfill}{rgb}{0.992771,0.707689,0.712380}%
\pgfsetfillcolor{currentfill}%
\pgfsetlinewidth{0.000000pt}%
\definecolor{currentstroke}{rgb}{0.000000,0.000000,0.000000}%
\pgfsetstrokecolor{currentstroke}%
\pgfsetstrokeopacity{0.000000}%
\pgfsetdash{}{0pt}%
\pgfpathmoveto{\pgfqpoint{4.903518in}{1.780925in}}%
\pgfpathlineto{\pgfqpoint{4.912455in}{1.780925in}}%
\pgfpathlineto{\pgfqpoint{4.912455in}{1.269589in}}%
\pgfpathlineto{\pgfqpoint{4.903518in}{1.269589in}}%
\pgfpathlineto{\pgfqpoint{4.903518in}{1.780925in}}%
\pgfpathclose%
\pgfusepath{fill}%
\end{pgfscope}%
\begin{pgfscope}%
\pgfpathrectangle{\pgfqpoint{3.722897in}{0.857143in}}{\pgfqpoint{2.627103in}{1.813434in}}%
\pgfusepath{clip}%
\pgfsetbuttcap%
\pgfsetmiterjoin%
\definecolor{currentfill}{rgb}{0.992771,0.707689,0.712380}%
\pgfsetfillcolor{currentfill}%
\pgfsetlinewidth{0.000000pt}%
\definecolor{currentstroke}{rgb}{0.000000,0.000000,0.000000}%
\pgfsetstrokecolor{currentstroke}%
\pgfsetstrokeopacity{0.000000}%
\pgfsetdash{}{0pt}%
\pgfpathmoveto{\pgfqpoint{4.914689in}{1.776329in}}%
\pgfpathlineto{\pgfqpoint{4.923625in}{1.776329in}}%
\pgfpathlineto{\pgfqpoint{4.923625in}{1.272032in}}%
\pgfpathlineto{\pgfqpoint{4.914689in}{1.272032in}}%
\pgfpathlineto{\pgfqpoint{4.914689in}{1.776329in}}%
\pgfpathclose%
\pgfusepath{fill}%
\end{pgfscope}%
\begin{pgfscope}%
\pgfpathrectangle{\pgfqpoint{3.722897in}{0.857143in}}{\pgfqpoint{2.627103in}{1.813434in}}%
\pgfusepath{clip}%
\pgfsetbuttcap%
\pgfsetmiterjoin%
\definecolor{currentfill}{rgb}{0.992771,0.707689,0.712380}%
\pgfsetfillcolor{currentfill}%
\pgfsetlinewidth{0.000000pt}%
\definecolor{currentstroke}{rgb}{0.000000,0.000000,0.000000}%
\pgfsetstrokecolor{currentstroke}%
\pgfsetstrokeopacity{0.000000}%
\pgfsetdash{}{0pt}%
\pgfpathmoveto{\pgfqpoint{4.925860in}{1.759251in}}%
\pgfpathlineto{\pgfqpoint{4.934796in}{1.759251in}}%
\pgfpathlineto{\pgfqpoint{4.934796in}{1.270575in}}%
\pgfpathlineto{\pgfqpoint{4.925860in}{1.270575in}}%
\pgfpathlineto{\pgfqpoint{4.925860in}{1.759251in}}%
\pgfpathclose%
\pgfusepath{fill}%
\end{pgfscope}%
\begin{pgfscope}%
\pgfpathrectangle{\pgfqpoint{3.722897in}{0.857143in}}{\pgfqpoint{2.627103in}{1.813434in}}%
\pgfusepath{clip}%
\pgfsetbuttcap%
\pgfsetmiterjoin%
\definecolor{currentfill}{rgb}{0.992771,0.707689,0.712380}%
\pgfsetfillcolor{currentfill}%
\pgfsetlinewidth{0.000000pt}%
\definecolor{currentstroke}{rgb}{0.000000,0.000000,0.000000}%
\pgfsetstrokecolor{currentstroke}%
\pgfsetstrokeopacity{0.000000}%
\pgfsetdash{}{0pt}%
\pgfpathmoveto{\pgfqpoint{4.937030in}{1.735753in}}%
\pgfpathlineto{\pgfqpoint{4.945967in}{1.735753in}}%
\pgfpathlineto{\pgfqpoint{4.945967in}{1.277414in}}%
\pgfpathlineto{\pgfqpoint{4.937030in}{1.277414in}}%
\pgfpathlineto{\pgfqpoint{4.937030in}{1.735753in}}%
\pgfpathclose%
\pgfusepath{fill}%
\end{pgfscope}%
\begin{pgfscope}%
\pgfpathrectangle{\pgfqpoint{3.722897in}{0.857143in}}{\pgfqpoint{2.627103in}{1.813434in}}%
\pgfusepath{clip}%
\pgfsetbuttcap%
\pgfsetmiterjoin%
\definecolor{currentfill}{rgb}{0.992771,0.707689,0.712380}%
\pgfsetfillcolor{currentfill}%
\pgfsetlinewidth{0.000000pt}%
\definecolor{currentstroke}{rgb}{0.000000,0.000000,0.000000}%
\pgfsetstrokecolor{currentstroke}%
\pgfsetstrokeopacity{0.000000}%
\pgfsetdash{}{0pt}%
\pgfpathmoveto{\pgfqpoint{4.948201in}{1.712692in}}%
\pgfpathlineto{\pgfqpoint{4.957137in}{1.712692in}}%
\pgfpathlineto{\pgfqpoint{4.957137in}{1.273623in}}%
\pgfpathlineto{\pgfqpoint{4.948201in}{1.273623in}}%
\pgfpathlineto{\pgfqpoint{4.948201in}{1.712692in}}%
\pgfpathclose%
\pgfusepath{fill}%
\end{pgfscope}%
\begin{pgfscope}%
\pgfpathrectangle{\pgfqpoint{3.722897in}{0.857143in}}{\pgfqpoint{2.627103in}{1.813434in}}%
\pgfusepath{clip}%
\pgfsetbuttcap%
\pgfsetmiterjoin%
\definecolor{currentfill}{rgb}{0.992771,0.707689,0.712380}%
\pgfsetfillcolor{currentfill}%
\pgfsetlinewidth{0.000000pt}%
\definecolor{currentstroke}{rgb}{0.000000,0.000000,0.000000}%
\pgfsetstrokecolor{currentstroke}%
\pgfsetstrokeopacity{0.000000}%
\pgfsetdash{}{0pt}%
\pgfpathmoveto{\pgfqpoint{4.959371in}{1.703087in}}%
\pgfpathlineto{\pgfqpoint{4.968308in}{1.703087in}}%
\pgfpathlineto{\pgfqpoint{4.968308in}{1.289526in}}%
\pgfpathlineto{\pgfqpoint{4.959371in}{1.289526in}}%
\pgfpathlineto{\pgfqpoint{4.959371in}{1.703087in}}%
\pgfpathclose%
\pgfusepath{fill}%
\end{pgfscope}%
\begin{pgfscope}%
\pgfpathrectangle{\pgfqpoint{3.722897in}{0.857143in}}{\pgfqpoint{2.627103in}{1.813434in}}%
\pgfusepath{clip}%
\pgfsetbuttcap%
\pgfsetmiterjoin%
\definecolor{currentfill}{rgb}{0.992771,0.707689,0.712380}%
\pgfsetfillcolor{currentfill}%
\pgfsetlinewidth{0.000000pt}%
\definecolor{currentstroke}{rgb}{0.000000,0.000000,0.000000}%
\pgfsetstrokecolor{currentstroke}%
\pgfsetstrokeopacity{0.000000}%
\pgfsetdash{}{0pt}%
\pgfpathmoveto{\pgfqpoint{4.970542in}{1.680096in}}%
\pgfpathlineto{\pgfqpoint{4.979478in}{1.680096in}}%
\pgfpathlineto{\pgfqpoint{4.979478in}{1.304226in}}%
\pgfpathlineto{\pgfqpoint{4.970542in}{1.304226in}}%
\pgfpathlineto{\pgfqpoint{4.970542in}{1.680096in}}%
\pgfpathclose%
\pgfusepath{fill}%
\end{pgfscope}%
\begin{pgfscope}%
\pgfpathrectangle{\pgfqpoint{3.722897in}{0.857143in}}{\pgfqpoint{2.627103in}{1.813434in}}%
\pgfusepath{clip}%
\pgfsetbuttcap%
\pgfsetmiterjoin%
\definecolor{currentfill}{rgb}{0.992771,0.707689,0.712380}%
\pgfsetfillcolor{currentfill}%
\pgfsetlinewidth{0.000000pt}%
\definecolor{currentstroke}{rgb}{0.000000,0.000000,0.000000}%
\pgfsetstrokecolor{currentstroke}%
\pgfsetstrokeopacity{0.000000}%
\pgfsetdash{}{0pt}%
\pgfpathmoveto{\pgfqpoint{4.981713in}{1.649173in}}%
\pgfpathlineto{\pgfqpoint{4.990649in}{1.649173in}}%
\pgfpathlineto{\pgfqpoint{4.990649in}{1.302241in}}%
\pgfpathlineto{\pgfqpoint{4.981713in}{1.302241in}}%
\pgfpathlineto{\pgfqpoint{4.981713in}{1.649173in}}%
\pgfpathclose%
\pgfusepath{fill}%
\end{pgfscope}%
\begin{pgfscope}%
\pgfpathrectangle{\pgfqpoint{3.722897in}{0.857143in}}{\pgfqpoint{2.627103in}{1.813434in}}%
\pgfusepath{clip}%
\pgfsetbuttcap%
\pgfsetmiterjoin%
\definecolor{currentfill}{rgb}{0.992771,0.707689,0.712380}%
\pgfsetfillcolor{currentfill}%
\pgfsetlinewidth{0.000000pt}%
\definecolor{currentstroke}{rgb}{0.000000,0.000000,0.000000}%
\pgfsetstrokecolor{currentstroke}%
\pgfsetstrokeopacity{0.000000}%
\pgfsetdash{}{0pt}%
\pgfpathmoveto{\pgfqpoint{4.992883in}{1.631510in}}%
\pgfpathlineto{\pgfqpoint{5.001820in}{1.631510in}}%
\pgfpathlineto{\pgfqpoint{5.001820in}{1.314993in}}%
\pgfpathlineto{\pgfqpoint{4.992883in}{1.314993in}}%
\pgfpathlineto{\pgfqpoint{4.992883in}{1.631510in}}%
\pgfpathclose%
\pgfusepath{fill}%
\end{pgfscope}%
\begin{pgfscope}%
\pgfpathrectangle{\pgfqpoint{3.722897in}{0.857143in}}{\pgfqpoint{2.627103in}{1.813434in}}%
\pgfusepath{clip}%
\pgfsetbuttcap%
\pgfsetmiterjoin%
\definecolor{currentfill}{rgb}{0.992771,0.707689,0.712380}%
\pgfsetfillcolor{currentfill}%
\pgfsetlinewidth{0.000000pt}%
\definecolor{currentstroke}{rgb}{0.000000,0.000000,0.000000}%
\pgfsetstrokecolor{currentstroke}%
\pgfsetstrokeopacity{0.000000}%
\pgfsetdash{}{0pt}%
\pgfpathmoveto{\pgfqpoint{5.004054in}{1.601382in}}%
\pgfpathlineto{\pgfqpoint{5.012990in}{1.601382in}}%
\pgfpathlineto{\pgfqpoint{5.012990in}{1.320846in}}%
\pgfpathlineto{\pgfqpoint{5.004054in}{1.320846in}}%
\pgfpathlineto{\pgfqpoint{5.004054in}{1.601382in}}%
\pgfpathclose%
\pgfusepath{fill}%
\end{pgfscope}%
\begin{pgfscope}%
\pgfpathrectangle{\pgfqpoint{3.722897in}{0.857143in}}{\pgfqpoint{2.627103in}{1.813434in}}%
\pgfusepath{clip}%
\pgfsetbuttcap%
\pgfsetmiterjoin%
\definecolor{currentfill}{rgb}{0.992771,0.707689,0.712380}%
\pgfsetfillcolor{currentfill}%
\pgfsetlinewidth{0.000000pt}%
\definecolor{currentstroke}{rgb}{0.000000,0.000000,0.000000}%
\pgfsetstrokecolor{currentstroke}%
\pgfsetstrokeopacity{0.000000}%
\pgfsetdash{}{0pt}%
\pgfpathmoveto{\pgfqpoint{5.015224in}{1.580469in}}%
\pgfpathlineto{\pgfqpoint{5.024161in}{1.580469in}}%
\pgfpathlineto{\pgfqpoint{5.024161in}{1.336945in}}%
\pgfpathlineto{\pgfqpoint{5.015224in}{1.336945in}}%
\pgfpathlineto{\pgfqpoint{5.015224in}{1.580469in}}%
\pgfpathclose%
\pgfusepath{fill}%
\end{pgfscope}%
\begin{pgfscope}%
\pgfpathrectangle{\pgfqpoint{3.722897in}{0.857143in}}{\pgfqpoint{2.627103in}{1.813434in}}%
\pgfusepath{clip}%
\pgfsetbuttcap%
\pgfsetmiterjoin%
\definecolor{currentfill}{rgb}{0.992771,0.707689,0.712380}%
\pgfsetfillcolor{currentfill}%
\pgfsetlinewidth{0.000000pt}%
\definecolor{currentstroke}{rgb}{0.000000,0.000000,0.000000}%
\pgfsetstrokecolor{currentstroke}%
\pgfsetstrokeopacity{0.000000}%
\pgfsetdash{}{0pt}%
\pgfpathmoveto{\pgfqpoint{5.026395in}{1.531917in}}%
\pgfpathlineto{\pgfqpoint{5.035331in}{1.531917in}}%
\pgfpathlineto{\pgfqpoint{5.035331in}{1.332070in}}%
\pgfpathlineto{\pgfqpoint{5.026395in}{1.332070in}}%
\pgfpathlineto{\pgfqpoint{5.026395in}{1.531917in}}%
\pgfpathclose%
\pgfusepath{fill}%
\end{pgfscope}%
\begin{pgfscope}%
\pgfpathrectangle{\pgfqpoint{3.722897in}{0.857143in}}{\pgfqpoint{2.627103in}{1.813434in}}%
\pgfusepath{clip}%
\pgfsetbuttcap%
\pgfsetmiterjoin%
\definecolor{currentfill}{rgb}{0.992771,0.707689,0.712380}%
\pgfsetfillcolor{currentfill}%
\pgfsetlinewidth{0.000000pt}%
\definecolor{currentstroke}{rgb}{0.000000,0.000000,0.000000}%
\pgfsetstrokecolor{currentstroke}%
\pgfsetstrokeopacity{0.000000}%
\pgfsetdash{}{0pt}%
\pgfpathmoveto{\pgfqpoint{5.037566in}{1.505264in}}%
\pgfpathlineto{\pgfqpoint{5.046502in}{1.505264in}}%
\pgfpathlineto{\pgfqpoint{5.046502in}{1.347917in}}%
\pgfpathlineto{\pgfqpoint{5.037566in}{1.347917in}}%
\pgfpathlineto{\pgfqpoint{5.037566in}{1.505264in}}%
\pgfpathclose%
\pgfusepath{fill}%
\end{pgfscope}%
\begin{pgfscope}%
\pgfpathrectangle{\pgfqpoint{3.722897in}{0.857143in}}{\pgfqpoint{2.627103in}{1.813434in}}%
\pgfusepath{clip}%
\pgfsetbuttcap%
\pgfsetmiterjoin%
\definecolor{currentfill}{rgb}{0.992771,0.707689,0.712380}%
\pgfsetfillcolor{currentfill}%
\pgfsetlinewidth{0.000000pt}%
\definecolor{currentstroke}{rgb}{0.000000,0.000000,0.000000}%
\pgfsetstrokecolor{currentstroke}%
\pgfsetstrokeopacity{0.000000}%
\pgfsetdash{}{0pt}%
\pgfpathmoveto{\pgfqpoint{5.048736in}{1.456315in}}%
\pgfpathlineto{\pgfqpoint{5.057673in}{1.456315in}}%
\pgfpathlineto{\pgfqpoint{5.057673in}{1.347858in}}%
\pgfpathlineto{\pgfqpoint{5.048736in}{1.347858in}}%
\pgfpathlineto{\pgfqpoint{5.048736in}{1.456315in}}%
\pgfpathclose%
\pgfusepath{fill}%
\end{pgfscope}%
\begin{pgfscope}%
\pgfpathrectangle{\pgfqpoint{3.722897in}{0.857143in}}{\pgfqpoint{2.627103in}{1.813434in}}%
\pgfusepath{clip}%
\pgfsetbuttcap%
\pgfsetmiterjoin%
\definecolor{currentfill}{rgb}{0.992771,0.707689,0.712380}%
\pgfsetfillcolor{currentfill}%
\pgfsetlinewidth{0.000000pt}%
\definecolor{currentstroke}{rgb}{0.000000,0.000000,0.000000}%
\pgfsetstrokecolor{currentstroke}%
\pgfsetstrokeopacity{0.000000}%
\pgfsetdash{}{0pt}%
\pgfpathmoveto{\pgfqpoint{5.059907in}{1.420964in}}%
\pgfpathlineto{\pgfqpoint{5.068843in}{1.420964in}}%
\pgfpathlineto{\pgfqpoint{5.068843in}{1.356369in}}%
\pgfpathlineto{\pgfqpoint{5.059907in}{1.356369in}}%
\pgfpathlineto{\pgfqpoint{5.059907in}{1.420964in}}%
\pgfpathclose%
\pgfusepath{fill}%
\end{pgfscope}%
\begin{pgfscope}%
\pgfpathrectangle{\pgfqpoint{3.722897in}{0.857143in}}{\pgfqpoint{2.627103in}{1.813434in}}%
\pgfusepath{clip}%
\pgfsetbuttcap%
\pgfsetmiterjoin%
\definecolor{currentfill}{rgb}{0.992771,0.707689,0.712380}%
\pgfsetfillcolor{currentfill}%
\pgfsetlinewidth{0.000000pt}%
\definecolor{currentstroke}{rgb}{0.000000,0.000000,0.000000}%
\pgfsetstrokecolor{currentstroke}%
\pgfsetstrokeopacity{0.000000}%
\pgfsetdash{}{0pt}%
\pgfpathmoveto{\pgfqpoint{5.071077in}{1.377210in}}%
\pgfpathlineto{\pgfqpoint{5.080014in}{1.377210in}}%
\pgfpathlineto{\pgfqpoint{5.080014in}{1.348263in}}%
\pgfpathlineto{\pgfqpoint{5.071077in}{1.348263in}}%
\pgfpathlineto{\pgfqpoint{5.071077in}{1.377210in}}%
\pgfpathclose%
\pgfusepath{fill}%
\end{pgfscope}%
\begin{pgfscope}%
\pgfpathrectangle{\pgfqpoint{3.722897in}{0.857143in}}{\pgfqpoint{2.627103in}{1.813434in}}%
\pgfusepath{clip}%
\pgfsetbuttcap%
\pgfsetmiterjoin%
\definecolor{currentfill}{rgb}{0.992771,0.707689,0.712380}%
\pgfsetfillcolor{currentfill}%
\pgfsetlinewidth{0.000000pt}%
\definecolor{currentstroke}{rgb}{0.000000,0.000000,0.000000}%
\pgfsetstrokecolor{currentstroke}%
\pgfsetstrokeopacity{0.000000}%
\pgfsetdash{}{0pt}%
\pgfpathmoveto{\pgfqpoint{5.082248in}{1.355131in}}%
\pgfpathlineto{\pgfqpoint{5.091184in}{1.355131in}}%
\pgfpathlineto{\pgfqpoint{5.091184in}{1.345886in}}%
\pgfpathlineto{\pgfqpoint{5.082248in}{1.345886in}}%
\pgfpathlineto{\pgfqpoint{5.082248in}{1.355131in}}%
\pgfpathclose%
\pgfusepath{fill}%
\end{pgfscope}%
\begin{pgfscope}%
\pgfpathrectangle{\pgfqpoint{3.722897in}{0.857143in}}{\pgfqpoint{2.627103in}{1.813434in}}%
\pgfusepath{clip}%
\pgfsetbuttcap%
\pgfsetmiterjoin%
\definecolor{currentfill}{rgb}{0.992771,0.707689,0.712380}%
\pgfsetfillcolor{currentfill}%
\pgfsetlinewidth{0.000000pt}%
\definecolor{currentstroke}{rgb}{0.000000,0.000000,0.000000}%
\pgfsetstrokecolor{currentstroke}%
\pgfsetstrokeopacity{0.000000}%
\pgfsetdash{}{0pt}%
\pgfpathmoveto{\pgfqpoint{5.093419in}{2.406660in}}%
\pgfpathlineto{\pgfqpoint{5.102355in}{2.406660in}}%
\pgfpathlineto{\pgfqpoint{5.102355in}{2.414385in}}%
\pgfpathlineto{\pgfqpoint{5.093419in}{2.414385in}}%
\pgfpathlineto{\pgfqpoint{5.093419in}{2.406660in}}%
\pgfpathclose%
\pgfusepath{fill}%
\end{pgfscope}%
\begin{pgfscope}%
\pgfpathrectangle{\pgfqpoint{3.722897in}{0.857143in}}{\pgfqpoint{2.627103in}{1.813434in}}%
\pgfusepath{clip}%
\pgfsetbuttcap%
\pgfsetmiterjoin%
\definecolor{currentfill}{rgb}{0.992771,0.707689,0.712380}%
\pgfsetfillcolor{currentfill}%
\pgfsetlinewidth{0.000000pt}%
\definecolor{currentstroke}{rgb}{0.000000,0.000000,0.000000}%
\pgfsetstrokecolor{currentstroke}%
\pgfsetstrokeopacity{0.000000}%
\pgfsetdash{}{0pt}%
\pgfpathmoveto{\pgfqpoint{5.104589in}{2.413241in}}%
\pgfpathlineto{\pgfqpoint{5.113526in}{2.413241in}}%
\pgfpathlineto{\pgfqpoint{5.113526in}{2.432765in}}%
\pgfpathlineto{\pgfqpoint{5.104589in}{2.432765in}}%
\pgfpathlineto{\pgfqpoint{5.104589in}{2.413241in}}%
\pgfpathclose%
\pgfusepath{fill}%
\end{pgfscope}%
\begin{pgfscope}%
\pgfpathrectangle{\pgfqpoint{3.722897in}{0.857143in}}{\pgfqpoint{2.627103in}{1.813434in}}%
\pgfusepath{clip}%
\pgfsetbuttcap%
\pgfsetmiterjoin%
\definecolor{currentfill}{rgb}{0.992771,0.707689,0.712380}%
\pgfsetfillcolor{currentfill}%
\pgfsetlinewidth{0.000000pt}%
\definecolor{currentstroke}{rgb}{0.000000,0.000000,0.000000}%
\pgfsetstrokecolor{currentstroke}%
\pgfsetstrokeopacity{0.000000}%
\pgfsetdash{}{0pt}%
\pgfpathmoveto{\pgfqpoint{5.115760in}{2.423484in}}%
\pgfpathlineto{\pgfqpoint{5.124696in}{2.423484in}}%
\pgfpathlineto{\pgfqpoint{5.124696in}{2.452963in}}%
\pgfpathlineto{\pgfqpoint{5.115760in}{2.452963in}}%
\pgfpathlineto{\pgfqpoint{5.115760in}{2.423484in}}%
\pgfpathclose%
\pgfusepath{fill}%
\end{pgfscope}%
\begin{pgfscope}%
\pgfpathrectangle{\pgfqpoint{3.722897in}{0.857143in}}{\pgfqpoint{2.627103in}{1.813434in}}%
\pgfusepath{clip}%
\pgfsetbuttcap%
\pgfsetmiterjoin%
\definecolor{currentfill}{rgb}{0.992771,0.707689,0.712380}%
\pgfsetfillcolor{currentfill}%
\pgfsetlinewidth{0.000000pt}%
\definecolor{currentstroke}{rgb}{0.000000,0.000000,0.000000}%
\pgfsetstrokecolor{currentstroke}%
\pgfsetstrokeopacity{0.000000}%
\pgfsetdash{}{0pt}%
\pgfpathmoveto{\pgfqpoint{5.126930in}{2.440340in}}%
\pgfpathlineto{\pgfqpoint{5.135867in}{2.440340in}}%
\pgfpathlineto{\pgfqpoint{5.135867in}{2.469202in}}%
\pgfpathlineto{\pgfqpoint{5.126930in}{2.469202in}}%
\pgfpathlineto{\pgfqpoint{5.126930in}{2.440340in}}%
\pgfpathclose%
\pgfusepath{fill}%
\end{pgfscope}%
\begin{pgfscope}%
\pgfpathrectangle{\pgfqpoint{3.722897in}{0.857143in}}{\pgfqpoint{2.627103in}{1.813434in}}%
\pgfusepath{clip}%
\pgfsetbuttcap%
\pgfsetmiterjoin%
\definecolor{currentfill}{rgb}{0.992771,0.707689,0.712380}%
\pgfsetfillcolor{currentfill}%
\pgfsetlinewidth{0.000000pt}%
\definecolor{currentstroke}{rgb}{0.000000,0.000000,0.000000}%
\pgfsetstrokecolor{currentstroke}%
\pgfsetstrokeopacity{0.000000}%
\pgfsetdash{}{0pt}%
\pgfpathmoveto{\pgfqpoint{5.138101in}{2.464582in}}%
\pgfpathlineto{\pgfqpoint{5.147038in}{2.464582in}}%
\pgfpathlineto{\pgfqpoint{5.147038in}{2.485306in}}%
\pgfpathlineto{\pgfqpoint{5.138101in}{2.485306in}}%
\pgfpathlineto{\pgfqpoint{5.138101in}{2.464582in}}%
\pgfpathclose%
\pgfusepath{fill}%
\end{pgfscope}%
\begin{pgfscope}%
\pgfpathrectangle{\pgfqpoint{3.722897in}{0.857143in}}{\pgfqpoint{2.627103in}{1.813434in}}%
\pgfusepath{clip}%
\pgfsetbuttcap%
\pgfsetmiterjoin%
\definecolor{currentfill}{rgb}{0.992771,0.707689,0.712380}%
\pgfsetfillcolor{currentfill}%
\pgfsetlinewidth{0.000000pt}%
\definecolor{currentstroke}{rgb}{0.000000,0.000000,0.000000}%
\pgfsetstrokecolor{currentstroke}%
\pgfsetstrokeopacity{0.000000}%
\pgfsetdash{}{0pt}%
\pgfpathmoveto{\pgfqpoint{5.149272in}{2.474402in}}%
\pgfpathlineto{\pgfqpoint{5.158208in}{2.474402in}}%
\pgfpathlineto{\pgfqpoint{5.158208in}{2.475910in}}%
\pgfpathlineto{\pgfqpoint{5.149272in}{2.475910in}}%
\pgfpathlineto{\pgfqpoint{5.149272in}{2.474402in}}%
\pgfpathclose%
\pgfusepath{fill}%
\end{pgfscope}%
\begin{pgfscope}%
\pgfpathrectangle{\pgfqpoint{3.722897in}{0.857143in}}{\pgfqpoint{2.627103in}{1.813434in}}%
\pgfusepath{clip}%
\pgfsetbuttcap%
\pgfsetmiterjoin%
\definecolor{currentfill}{rgb}{0.992771,0.707689,0.712380}%
\pgfsetfillcolor{currentfill}%
\pgfsetlinewidth{0.000000pt}%
\definecolor{currentstroke}{rgb}{0.000000,0.000000,0.000000}%
\pgfsetstrokecolor{currentstroke}%
\pgfsetstrokeopacity{0.000000}%
\pgfsetdash{}{0pt}%
\pgfpathmoveto{\pgfqpoint{5.160442in}{1.249436in}}%
\pgfpathlineto{\pgfqpoint{5.169379in}{1.249436in}}%
\pgfpathlineto{\pgfqpoint{5.169379in}{1.236572in}}%
\pgfpathlineto{\pgfqpoint{5.160442in}{1.236572in}}%
\pgfpathlineto{\pgfqpoint{5.160442in}{1.249436in}}%
\pgfpathclose%
\pgfusepath{fill}%
\end{pgfscope}%
\begin{pgfscope}%
\pgfpathrectangle{\pgfqpoint{3.722897in}{0.857143in}}{\pgfqpoint{2.627103in}{1.813434in}}%
\pgfusepath{clip}%
\pgfsetbuttcap%
\pgfsetmiterjoin%
\definecolor{currentfill}{rgb}{0.992771,0.707689,0.712380}%
\pgfsetfillcolor{currentfill}%
\pgfsetlinewidth{0.000000pt}%
\definecolor{currentstroke}{rgb}{0.000000,0.000000,0.000000}%
\pgfsetstrokecolor{currentstroke}%
\pgfsetstrokeopacity{0.000000}%
\pgfsetdash{}{0pt}%
\pgfpathmoveto{\pgfqpoint{5.171613in}{1.235036in}}%
\pgfpathlineto{\pgfqpoint{5.180549in}{1.235036in}}%
\pgfpathlineto{\pgfqpoint{5.180549in}{1.218179in}}%
\pgfpathlineto{\pgfqpoint{5.171613in}{1.218179in}}%
\pgfpathlineto{\pgfqpoint{5.171613in}{1.235036in}}%
\pgfpathclose%
\pgfusepath{fill}%
\end{pgfscope}%
\begin{pgfscope}%
\pgfpathrectangle{\pgfqpoint{3.722897in}{0.857143in}}{\pgfqpoint{2.627103in}{1.813434in}}%
\pgfusepath{clip}%
\pgfsetbuttcap%
\pgfsetmiterjoin%
\definecolor{currentfill}{rgb}{0.992771,0.707689,0.712380}%
\pgfsetfillcolor{currentfill}%
\pgfsetlinewidth{0.000000pt}%
\definecolor{currentstroke}{rgb}{0.000000,0.000000,0.000000}%
\pgfsetstrokecolor{currentstroke}%
\pgfsetstrokeopacity{0.000000}%
\pgfsetdash{}{0pt}%
\pgfpathmoveto{\pgfqpoint{5.182783in}{1.230818in}}%
\pgfpathlineto{\pgfqpoint{5.191720in}{1.230818in}}%
\pgfpathlineto{\pgfqpoint{5.191720in}{1.199399in}}%
\pgfpathlineto{\pgfqpoint{5.182783in}{1.199399in}}%
\pgfpathlineto{\pgfqpoint{5.182783in}{1.230818in}}%
\pgfpathclose%
\pgfusepath{fill}%
\end{pgfscope}%
\begin{pgfscope}%
\pgfpathrectangle{\pgfqpoint{3.722897in}{0.857143in}}{\pgfqpoint{2.627103in}{1.813434in}}%
\pgfusepath{clip}%
\pgfsetbuttcap%
\pgfsetmiterjoin%
\definecolor{currentfill}{rgb}{0.992771,0.707689,0.712380}%
\pgfsetfillcolor{currentfill}%
\pgfsetlinewidth{0.000000pt}%
\definecolor{currentstroke}{rgb}{0.000000,0.000000,0.000000}%
\pgfsetstrokecolor{currentstroke}%
\pgfsetstrokeopacity{0.000000}%
\pgfsetdash{}{0pt}%
\pgfpathmoveto{\pgfqpoint{5.193954in}{1.237968in}}%
\pgfpathlineto{\pgfqpoint{5.202891in}{1.237968in}}%
\pgfpathlineto{\pgfqpoint{5.202891in}{1.192916in}}%
\pgfpathlineto{\pgfqpoint{5.193954in}{1.192916in}}%
\pgfpathlineto{\pgfqpoint{5.193954in}{1.237968in}}%
\pgfpathclose%
\pgfusepath{fill}%
\end{pgfscope}%
\begin{pgfscope}%
\pgfpathrectangle{\pgfqpoint{3.722897in}{0.857143in}}{\pgfqpoint{2.627103in}{1.813434in}}%
\pgfusepath{clip}%
\pgfsetbuttcap%
\pgfsetmiterjoin%
\definecolor{currentfill}{rgb}{0.992771,0.707689,0.712380}%
\pgfsetfillcolor{currentfill}%
\pgfsetlinewidth{0.000000pt}%
\definecolor{currentstroke}{rgb}{0.000000,0.000000,0.000000}%
\pgfsetstrokecolor{currentstroke}%
\pgfsetstrokeopacity{0.000000}%
\pgfsetdash{}{0pt}%
\pgfpathmoveto{\pgfqpoint{5.205125in}{1.232358in}}%
\pgfpathlineto{\pgfqpoint{5.214061in}{1.232358in}}%
\pgfpathlineto{\pgfqpoint{5.214061in}{1.177926in}}%
\pgfpathlineto{\pgfqpoint{5.205125in}{1.177926in}}%
\pgfpathlineto{\pgfqpoint{5.205125in}{1.232358in}}%
\pgfpathclose%
\pgfusepath{fill}%
\end{pgfscope}%
\begin{pgfscope}%
\pgfpathrectangle{\pgfqpoint{3.722897in}{0.857143in}}{\pgfqpoint{2.627103in}{1.813434in}}%
\pgfusepath{clip}%
\pgfsetbuttcap%
\pgfsetmiterjoin%
\definecolor{currentfill}{rgb}{0.992771,0.707689,0.712380}%
\pgfsetfillcolor{currentfill}%
\pgfsetlinewidth{0.000000pt}%
\definecolor{currentstroke}{rgb}{0.000000,0.000000,0.000000}%
\pgfsetstrokecolor{currentstroke}%
\pgfsetstrokeopacity{0.000000}%
\pgfsetdash{}{0pt}%
\pgfpathmoveto{\pgfqpoint{5.216295in}{1.232948in}}%
\pgfpathlineto{\pgfqpoint{5.225232in}{1.232948in}}%
\pgfpathlineto{\pgfqpoint{5.225232in}{1.156961in}}%
\pgfpathlineto{\pgfqpoint{5.216295in}{1.156961in}}%
\pgfpathlineto{\pgfqpoint{5.216295in}{1.232948in}}%
\pgfpathclose%
\pgfusepath{fill}%
\end{pgfscope}%
\begin{pgfscope}%
\pgfpathrectangle{\pgfqpoint{3.722897in}{0.857143in}}{\pgfqpoint{2.627103in}{1.813434in}}%
\pgfusepath{clip}%
\pgfsetbuttcap%
\pgfsetmiterjoin%
\definecolor{currentfill}{rgb}{0.992771,0.707689,0.712380}%
\pgfsetfillcolor{currentfill}%
\pgfsetlinewidth{0.000000pt}%
\definecolor{currentstroke}{rgb}{0.000000,0.000000,0.000000}%
\pgfsetstrokecolor{currentstroke}%
\pgfsetstrokeopacity{0.000000}%
\pgfsetdash{}{0pt}%
\pgfpathmoveto{\pgfqpoint{5.227466in}{1.259556in}}%
\pgfpathlineto{\pgfqpoint{5.236402in}{1.259556in}}%
\pgfpathlineto{\pgfqpoint{5.236402in}{1.141103in}}%
\pgfpathlineto{\pgfqpoint{5.227466in}{1.141103in}}%
\pgfpathlineto{\pgfqpoint{5.227466in}{1.259556in}}%
\pgfpathclose%
\pgfusepath{fill}%
\end{pgfscope}%
\begin{pgfscope}%
\pgfpathrectangle{\pgfqpoint{3.722897in}{0.857143in}}{\pgfqpoint{2.627103in}{1.813434in}}%
\pgfusepath{clip}%
\pgfsetbuttcap%
\pgfsetmiterjoin%
\definecolor{currentfill}{rgb}{0.992771,0.707689,0.712380}%
\pgfsetfillcolor{currentfill}%
\pgfsetlinewidth{0.000000pt}%
\definecolor{currentstroke}{rgb}{0.000000,0.000000,0.000000}%
\pgfsetstrokecolor{currentstroke}%
\pgfsetstrokeopacity{0.000000}%
\pgfsetdash{}{0pt}%
\pgfpathmoveto{\pgfqpoint{5.238636in}{1.278891in}}%
\pgfpathlineto{\pgfqpoint{5.247573in}{1.278891in}}%
\pgfpathlineto{\pgfqpoint{5.247573in}{1.126835in}}%
\pgfpathlineto{\pgfqpoint{5.238636in}{1.126835in}}%
\pgfpathlineto{\pgfqpoint{5.238636in}{1.278891in}}%
\pgfpathclose%
\pgfusepath{fill}%
\end{pgfscope}%
\begin{pgfscope}%
\pgfpathrectangle{\pgfqpoint{3.722897in}{0.857143in}}{\pgfqpoint{2.627103in}{1.813434in}}%
\pgfusepath{clip}%
\pgfsetbuttcap%
\pgfsetmiterjoin%
\definecolor{currentfill}{rgb}{0.992771,0.707689,0.712380}%
\pgfsetfillcolor{currentfill}%
\pgfsetlinewidth{0.000000pt}%
\definecolor{currentstroke}{rgb}{0.000000,0.000000,0.000000}%
\pgfsetstrokecolor{currentstroke}%
\pgfsetstrokeopacity{0.000000}%
\pgfsetdash{}{0pt}%
\pgfpathmoveto{\pgfqpoint{5.249807in}{1.292992in}}%
\pgfpathlineto{\pgfqpoint{5.258744in}{1.292992in}}%
\pgfpathlineto{\pgfqpoint{5.258744in}{1.109213in}}%
\pgfpathlineto{\pgfqpoint{5.249807in}{1.109213in}}%
\pgfpathlineto{\pgfqpoint{5.249807in}{1.292992in}}%
\pgfpathclose%
\pgfusepath{fill}%
\end{pgfscope}%
\begin{pgfscope}%
\pgfpathrectangle{\pgfqpoint{3.722897in}{0.857143in}}{\pgfqpoint{2.627103in}{1.813434in}}%
\pgfusepath{clip}%
\pgfsetbuttcap%
\pgfsetmiterjoin%
\definecolor{currentfill}{rgb}{0.992771,0.707689,0.712380}%
\pgfsetfillcolor{currentfill}%
\pgfsetlinewidth{0.000000pt}%
\definecolor{currentstroke}{rgb}{0.000000,0.000000,0.000000}%
\pgfsetstrokecolor{currentstroke}%
\pgfsetstrokeopacity{0.000000}%
\pgfsetdash{}{0pt}%
\pgfpathmoveto{\pgfqpoint{5.260978in}{1.315004in}}%
\pgfpathlineto{\pgfqpoint{5.269914in}{1.315004in}}%
\pgfpathlineto{\pgfqpoint{5.269914in}{1.090331in}}%
\pgfpathlineto{\pgfqpoint{5.260978in}{1.090331in}}%
\pgfpathlineto{\pgfqpoint{5.260978in}{1.315004in}}%
\pgfpathclose%
\pgfusepath{fill}%
\end{pgfscope}%
\begin{pgfscope}%
\pgfpathrectangle{\pgfqpoint{3.722897in}{0.857143in}}{\pgfqpoint{2.627103in}{1.813434in}}%
\pgfusepath{clip}%
\pgfsetbuttcap%
\pgfsetmiterjoin%
\definecolor{currentfill}{rgb}{0.992771,0.707689,0.712380}%
\pgfsetfillcolor{currentfill}%
\pgfsetlinewidth{0.000000pt}%
\definecolor{currentstroke}{rgb}{0.000000,0.000000,0.000000}%
\pgfsetstrokecolor{currentstroke}%
\pgfsetstrokeopacity{0.000000}%
\pgfsetdash{}{0pt}%
\pgfpathmoveto{\pgfqpoint{5.272148in}{1.346956in}}%
\pgfpathlineto{\pgfqpoint{5.281085in}{1.346956in}}%
\pgfpathlineto{\pgfqpoint{5.281085in}{1.083175in}}%
\pgfpathlineto{\pgfqpoint{5.272148in}{1.083175in}}%
\pgfpathlineto{\pgfqpoint{5.272148in}{1.346956in}}%
\pgfpathclose%
\pgfusepath{fill}%
\end{pgfscope}%
\begin{pgfscope}%
\pgfpathrectangle{\pgfqpoint{3.722897in}{0.857143in}}{\pgfqpoint{2.627103in}{1.813434in}}%
\pgfusepath{clip}%
\pgfsetbuttcap%
\pgfsetmiterjoin%
\definecolor{currentfill}{rgb}{0.992771,0.707689,0.712380}%
\pgfsetfillcolor{currentfill}%
\pgfsetlinewidth{0.000000pt}%
\definecolor{currentstroke}{rgb}{0.000000,0.000000,0.000000}%
\pgfsetstrokecolor{currentstroke}%
\pgfsetstrokeopacity{0.000000}%
\pgfsetdash{}{0pt}%
\pgfpathmoveto{\pgfqpoint{5.283319in}{1.372542in}}%
\pgfpathlineto{\pgfqpoint{5.292255in}{1.372542in}}%
\pgfpathlineto{\pgfqpoint{5.292255in}{1.073231in}}%
\pgfpathlineto{\pgfqpoint{5.283319in}{1.073231in}}%
\pgfpathlineto{\pgfqpoint{5.283319in}{1.372542in}}%
\pgfpathclose%
\pgfusepath{fill}%
\end{pgfscope}%
\begin{pgfscope}%
\pgfpathrectangle{\pgfqpoint{3.722897in}{0.857143in}}{\pgfqpoint{2.627103in}{1.813434in}}%
\pgfusepath{clip}%
\pgfsetbuttcap%
\pgfsetmiterjoin%
\definecolor{currentfill}{rgb}{0.992771,0.707689,0.712380}%
\pgfsetfillcolor{currentfill}%
\pgfsetlinewidth{0.000000pt}%
\definecolor{currentstroke}{rgb}{0.000000,0.000000,0.000000}%
\pgfsetstrokecolor{currentstroke}%
\pgfsetstrokeopacity{0.000000}%
\pgfsetdash{}{0pt}%
\pgfpathmoveto{\pgfqpoint{5.294489in}{1.398634in}}%
\pgfpathlineto{\pgfqpoint{5.303426in}{1.398634in}}%
\pgfpathlineto{\pgfqpoint{5.303426in}{1.069619in}}%
\pgfpathlineto{\pgfqpoint{5.294489in}{1.069619in}}%
\pgfpathlineto{\pgfqpoint{5.294489in}{1.398634in}}%
\pgfpathclose%
\pgfusepath{fill}%
\end{pgfscope}%
\begin{pgfscope}%
\pgfpathrectangle{\pgfqpoint{3.722897in}{0.857143in}}{\pgfqpoint{2.627103in}{1.813434in}}%
\pgfusepath{clip}%
\pgfsetbuttcap%
\pgfsetmiterjoin%
\definecolor{currentfill}{rgb}{0.992771,0.707689,0.712380}%
\pgfsetfillcolor{currentfill}%
\pgfsetlinewidth{0.000000pt}%
\definecolor{currentstroke}{rgb}{0.000000,0.000000,0.000000}%
\pgfsetstrokecolor{currentstroke}%
\pgfsetstrokeopacity{0.000000}%
\pgfsetdash{}{0pt}%
\pgfpathmoveto{\pgfqpoint{5.305660in}{1.420432in}}%
\pgfpathlineto{\pgfqpoint{5.314597in}{1.420432in}}%
\pgfpathlineto{\pgfqpoint{5.314597in}{1.061995in}}%
\pgfpathlineto{\pgfqpoint{5.305660in}{1.061995in}}%
\pgfpathlineto{\pgfqpoint{5.305660in}{1.420432in}}%
\pgfpathclose%
\pgfusepath{fill}%
\end{pgfscope}%
\begin{pgfscope}%
\pgfpathrectangle{\pgfqpoint{3.722897in}{0.857143in}}{\pgfqpoint{2.627103in}{1.813434in}}%
\pgfusepath{clip}%
\pgfsetbuttcap%
\pgfsetmiterjoin%
\definecolor{currentfill}{rgb}{0.992771,0.707689,0.712380}%
\pgfsetfillcolor{currentfill}%
\pgfsetlinewidth{0.000000pt}%
\definecolor{currentstroke}{rgb}{0.000000,0.000000,0.000000}%
\pgfsetstrokecolor{currentstroke}%
\pgfsetstrokeopacity{0.000000}%
\pgfsetdash{}{0pt}%
\pgfpathmoveto{\pgfqpoint{5.316831in}{1.438569in}}%
\pgfpathlineto{\pgfqpoint{5.325767in}{1.438569in}}%
\pgfpathlineto{\pgfqpoint{5.325767in}{1.054291in}}%
\pgfpathlineto{\pgfqpoint{5.316831in}{1.054291in}}%
\pgfpathlineto{\pgfqpoint{5.316831in}{1.438569in}}%
\pgfpathclose%
\pgfusepath{fill}%
\end{pgfscope}%
\begin{pgfscope}%
\pgfpathrectangle{\pgfqpoint{3.722897in}{0.857143in}}{\pgfqpoint{2.627103in}{1.813434in}}%
\pgfusepath{clip}%
\pgfsetbuttcap%
\pgfsetmiterjoin%
\definecolor{currentfill}{rgb}{0.992771,0.707689,0.712380}%
\pgfsetfillcolor{currentfill}%
\pgfsetlinewidth{0.000000pt}%
\definecolor{currentstroke}{rgb}{0.000000,0.000000,0.000000}%
\pgfsetstrokecolor{currentstroke}%
\pgfsetstrokeopacity{0.000000}%
\pgfsetdash{}{0pt}%
\pgfpathmoveto{\pgfqpoint{5.328001in}{1.476120in}}%
\pgfpathlineto{\pgfqpoint{5.336938in}{1.476120in}}%
\pgfpathlineto{\pgfqpoint{5.336938in}{1.068874in}}%
\pgfpathlineto{\pgfqpoint{5.328001in}{1.068874in}}%
\pgfpathlineto{\pgfqpoint{5.328001in}{1.476120in}}%
\pgfpathclose%
\pgfusepath{fill}%
\end{pgfscope}%
\begin{pgfscope}%
\pgfpathrectangle{\pgfqpoint{3.722897in}{0.857143in}}{\pgfqpoint{2.627103in}{1.813434in}}%
\pgfusepath{clip}%
\pgfsetbuttcap%
\pgfsetmiterjoin%
\definecolor{currentfill}{rgb}{0.992771,0.707689,0.712380}%
\pgfsetfillcolor{currentfill}%
\pgfsetlinewidth{0.000000pt}%
\definecolor{currentstroke}{rgb}{0.000000,0.000000,0.000000}%
\pgfsetstrokecolor{currentstroke}%
\pgfsetstrokeopacity{0.000000}%
\pgfsetdash{}{0pt}%
\pgfpathmoveto{\pgfqpoint{5.339172in}{1.486570in}}%
\pgfpathlineto{\pgfqpoint{5.348108in}{1.486570in}}%
\pgfpathlineto{\pgfqpoint{5.348108in}{1.063583in}}%
\pgfpathlineto{\pgfqpoint{5.339172in}{1.063583in}}%
\pgfpathlineto{\pgfqpoint{5.339172in}{1.486570in}}%
\pgfpathclose%
\pgfusepath{fill}%
\end{pgfscope}%
\begin{pgfscope}%
\pgfpathrectangle{\pgfqpoint{3.722897in}{0.857143in}}{\pgfqpoint{2.627103in}{1.813434in}}%
\pgfusepath{clip}%
\pgfsetbuttcap%
\pgfsetmiterjoin%
\definecolor{currentfill}{rgb}{0.992771,0.707689,0.712380}%
\pgfsetfillcolor{currentfill}%
\pgfsetlinewidth{0.000000pt}%
\definecolor{currentstroke}{rgb}{0.000000,0.000000,0.000000}%
\pgfsetstrokecolor{currentstroke}%
\pgfsetstrokeopacity{0.000000}%
\pgfsetdash{}{0pt}%
\pgfpathmoveto{\pgfqpoint{5.350343in}{1.493654in}}%
\pgfpathlineto{\pgfqpoint{5.359279in}{1.493654in}}%
\pgfpathlineto{\pgfqpoint{5.359279in}{1.043080in}}%
\pgfpathlineto{\pgfqpoint{5.350343in}{1.043080in}}%
\pgfpathlineto{\pgfqpoint{5.350343in}{1.493654in}}%
\pgfpathclose%
\pgfusepath{fill}%
\end{pgfscope}%
\begin{pgfscope}%
\pgfpathrectangle{\pgfqpoint{3.722897in}{0.857143in}}{\pgfqpoint{2.627103in}{1.813434in}}%
\pgfusepath{clip}%
\pgfsetbuttcap%
\pgfsetmiterjoin%
\definecolor{currentfill}{rgb}{0.992771,0.707689,0.712380}%
\pgfsetfillcolor{currentfill}%
\pgfsetlinewidth{0.000000pt}%
\definecolor{currentstroke}{rgb}{0.000000,0.000000,0.000000}%
\pgfsetstrokecolor{currentstroke}%
\pgfsetstrokeopacity{0.000000}%
\pgfsetdash{}{0pt}%
\pgfpathmoveto{\pgfqpoint{5.361513in}{1.519631in}}%
\pgfpathlineto{\pgfqpoint{5.370450in}{1.519631in}}%
\pgfpathlineto{\pgfqpoint{5.370450in}{1.029551in}}%
\pgfpathlineto{\pgfqpoint{5.361513in}{1.029551in}}%
\pgfpathlineto{\pgfqpoint{5.361513in}{1.519631in}}%
\pgfpathclose%
\pgfusepath{fill}%
\end{pgfscope}%
\begin{pgfscope}%
\pgfpathrectangle{\pgfqpoint{3.722897in}{0.857143in}}{\pgfqpoint{2.627103in}{1.813434in}}%
\pgfusepath{clip}%
\pgfsetbuttcap%
\pgfsetmiterjoin%
\definecolor{currentfill}{rgb}{0.992771,0.707689,0.712380}%
\pgfsetfillcolor{currentfill}%
\pgfsetlinewidth{0.000000pt}%
\definecolor{currentstroke}{rgb}{0.000000,0.000000,0.000000}%
\pgfsetstrokecolor{currentstroke}%
\pgfsetstrokeopacity{0.000000}%
\pgfsetdash{}{0pt}%
\pgfpathmoveto{\pgfqpoint{5.372684in}{1.530541in}}%
\pgfpathlineto{\pgfqpoint{5.381620in}{1.530541in}}%
\pgfpathlineto{\pgfqpoint{5.381620in}{1.011110in}}%
\pgfpathlineto{\pgfqpoint{5.372684in}{1.011110in}}%
\pgfpathlineto{\pgfqpoint{5.372684in}{1.530541in}}%
\pgfpathclose%
\pgfusepath{fill}%
\end{pgfscope}%
\begin{pgfscope}%
\pgfpathrectangle{\pgfqpoint{3.722897in}{0.857143in}}{\pgfqpoint{2.627103in}{1.813434in}}%
\pgfusepath{clip}%
\pgfsetbuttcap%
\pgfsetmiterjoin%
\definecolor{currentfill}{rgb}{0.992771,0.707689,0.712380}%
\pgfsetfillcolor{currentfill}%
\pgfsetlinewidth{0.000000pt}%
\definecolor{currentstroke}{rgb}{0.000000,0.000000,0.000000}%
\pgfsetstrokecolor{currentstroke}%
\pgfsetstrokeopacity{0.000000}%
\pgfsetdash{}{0pt}%
\pgfpathmoveto{\pgfqpoint{5.383854in}{1.553306in}}%
\pgfpathlineto{\pgfqpoint{5.392791in}{1.553306in}}%
\pgfpathlineto{\pgfqpoint{5.392791in}{1.018768in}}%
\pgfpathlineto{\pgfqpoint{5.383854in}{1.018768in}}%
\pgfpathlineto{\pgfqpoint{5.383854in}{1.553306in}}%
\pgfpathclose%
\pgfusepath{fill}%
\end{pgfscope}%
\begin{pgfscope}%
\pgfpathrectangle{\pgfqpoint{3.722897in}{0.857143in}}{\pgfqpoint{2.627103in}{1.813434in}}%
\pgfusepath{clip}%
\pgfsetbuttcap%
\pgfsetmiterjoin%
\definecolor{currentfill}{rgb}{0.992771,0.707689,0.712380}%
\pgfsetfillcolor{currentfill}%
\pgfsetlinewidth{0.000000pt}%
\definecolor{currentstroke}{rgb}{0.000000,0.000000,0.000000}%
\pgfsetstrokecolor{currentstroke}%
\pgfsetstrokeopacity{0.000000}%
\pgfsetdash{}{0pt}%
\pgfpathmoveto{\pgfqpoint{5.395025in}{1.542006in}}%
\pgfpathlineto{\pgfqpoint{5.403961in}{1.542006in}}%
\pgfpathlineto{\pgfqpoint{5.403961in}{1.005875in}}%
\pgfpathlineto{\pgfqpoint{5.395025in}{1.005875in}}%
\pgfpathlineto{\pgfqpoint{5.395025in}{1.542006in}}%
\pgfpathclose%
\pgfusepath{fill}%
\end{pgfscope}%
\begin{pgfscope}%
\pgfpathrectangle{\pgfqpoint{3.722897in}{0.857143in}}{\pgfqpoint{2.627103in}{1.813434in}}%
\pgfusepath{clip}%
\pgfsetbuttcap%
\pgfsetmiterjoin%
\definecolor{currentfill}{rgb}{0.992771,0.707689,0.712380}%
\pgfsetfillcolor{currentfill}%
\pgfsetlinewidth{0.000000pt}%
\definecolor{currentstroke}{rgb}{0.000000,0.000000,0.000000}%
\pgfsetstrokecolor{currentstroke}%
\pgfsetstrokeopacity{0.000000}%
\pgfsetdash{}{0pt}%
\pgfpathmoveto{\pgfqpoint{5.406196in}{1.571172in}}%
\pgfpathlineto{\pgfqpoint{5.415132in}{1.571172in}}%
\pgfpathlineto{\pgfqpoint{5.415132in}{1.024226in}}%
\pgfpathlineto{\pgfqpoint{5.406196in}{1.024226in}}%
\pgfpathlineto{\pgfqpoint{5.406196in}{1.571172in}}%
\pgfpathclose%
\pgfusepath{fill}%
\end{pgfscope}%
\begin{pgfscope}%
\pgfpathrectangle{\pgfqpoint{3.722897in}{0.857143in}}{\pgfqpoint{2.627103in}{1.813434in}}%
\pgfusepath{clip}%
\pgfsetbuttcap%
\pgfsetmiterjoin%
\definecolor{currentfill}{rgb}{0.992771,0.707689,0.712380}%
\pgfsetfillcolor{currentfill}%
\pgfsetlinewidth{0.000000pt}%
\definecolor{currentstroke}{rgb}{0.000000,0.000000,0.000000}%
\pgfsetstrokecolor{currentstroke}%
\pgfsetstrokeopacity{0.000000}%
\pgfsetdash{}{0pt}%
\pgfpathmoveto{\pgfqpoint{5.417366in}{1.569398in}}%
\pgfpathlineto{\pgfqpoint{5.426303in}{1.569398in}}%
\pgfpathlineto{\pgfqpoint{5.426303in}{1.047887in}}%
\pgfpathlineto{\pgfqpoint{5.417366in}{1.047887in}}%
\pgfpathlineto{\pgfqpoint{5.417366in}{1.569398in}}%
\pgfpathclose%
\pgfusepath{fill}%
\end{pgfscope}%
\begin{pgfscope}%
\pgfpathrectangle{\pgfqpoint{3.722897in}{0.857143in}}{\pgfqpoint{2.627103in}{1.813434in}}%
\pgfusepath{clip}%
\pgfsetbuttcap%
\pgfsetmiterjoin%
\definecolor{currentfill}{rgb}{0.992771,0.707689,0.712380}%
\pgfsetfillcolor{currentfill}%
\pgfsetlinewidth{0.000000pt}%
\definecolor{currentstroke}{rgb}{0.000000,0.000000,0.000000}%
\pgfsetstrokecolor{currentstroke}%
\pgfsetstrokeopacity{0.000000}%
\pgfsetdash{}{0pt}%
\pgfpathmoveto{\pgfqpoint{5.428537in}{1.555325in}}%
\pgfpathlineto{\pgfqpoint{5.437473in}{1.555325in}}%
\pgfpathlineto{\pgfqpoint{5.437473in}{1.072037in}}%
\pgfpathlineto{\pgfqpoint{5.428537in}{1.072037in}}%
\pgfpathlineto{\pgfqpoint{5.428537in}{1.555325in}}%
\pgfpathclose%
\pgfusepath{fill}%
\end{pgfscope}%
\begin{pgfscope}%
\pgfpathrectangle{\pgfqpoint{3.722897in}{0.857143in}}{\pgfqpoint{2.627103in}{1.813434in}}%
\pgfusepath{clip}%
\pgfsetbuttcap%
\pgfsetmiterjoin%
\definecolor{currentfill}{rgb}{0.992771,0.707689,0.712380}%
\pgfsetfillcolor{currentfill}%
\pgfsetlinewidth{0.000000pt}%
\definecolor{currentstroke}{rgb}{0.000000,0.000000,0.000000}%
\pgfsetstrokecolor{currentstroke}%
\pgfsetstrokeopacity{0.000000}%
\pgfsetdash{}{0pt}%
\pgfpathmoveto{\pgfqpoint{5.439707in}{1.556436in}}%
\pgfpathlineto{\pgfqpoint{5.448644in}{1.556436in}}%
\pgfpathlineto{\pgfqpoint{5.448644in}{1.105498in}}%
\pgfpathlineto{\pgfqpoint{5.439707in}{1.105498in}}%
\pgfpathlineto{\pgfqpoint{5.439707in}{1.556436in}}%
\pgfpathclose%
\pgfusepath{fill}%
\end{pgfscope}%
\begin{pgfscope}%
\pgfpathrectangle{\pgfqpoint{3.722897in}{0.857143in}}{\pgfqpoint{2.627103in}{1.813434in}}%
\pgfusepath{clip}%
\pgfsetbuttcap%
\pgfsetmiterjoin%
\definecolor{currentfill}{rgb}{0.992771,0.707689,0.712380}%
\pgfsetfillcolor{currentfill}%
\pgfsetlinewidth{0.000000pt}%
\definecolor{currentstroke}{rgb}{0.000000,0.000000,0.000000}%
\pgfsetstrokecolor{currentstroke}%
\pgfsetstrokeopacity{0.000000}%
\pgfsetdash{}{0pt}%
\pgfpathmoveto{\pgfqpoint{5.450878in}{1.535147in}}%
\pgfpathlineto{\pgfqpoint{5.459814in}{1.535147in}}%
\pgfpathlineto{\pgfqpoint{5.459814in}{1.131773in}}%
\pgfpathlineto{\pgfqpoint{5.450878in}{1.131773in}}%
\pgfpathlineto{\pgfqpoint{5.450878in}{1.535147in}}%
\pgfpathclose%
\pgfusepath{fill}%
\end{pgfscope}%
\begin{pgfscope}%
\pgfpathrectangle{\pgfqpoint{3.722897in}{0.857143in}}{\pgfqpoint{2.627103in}{1.813434in}}%
\pgfusepath{clip}%
\pgfsetbuttcap%
\pgfsetmiterjoin%
\definecolor{currentfill}{rgb}{0.992771,0.707689,0.712380}%
\pgfsetfillcolor{currentfill}%
\pgfsetlinewidth{0.000000pt}%
\definecolor{currentstroke}{rgb}{0.000000,0.000000,0.000000}%
\pgfsetstrokecolor{currentstroke}%
\pgfsetstrokeopacity{0.000000}%
\pgfsetdash{}{0pt}%
\pgfpathmoveto{\pgfqpoint{5.462049in}{1.523865in}}%
\pgfpathlineto{\pgfqpoint{5.470985in}{1.523865in}}%
\pgfpathlineto{\pgfqpoint{5.470985in}{1.171516in}}%
\pgfpathlineto{\pgfqpoint{5.462049in}{1.171516in}}%
\pgfpathlineto{\pgfqpoint{5.462049in}{1.523865in}}%
\pgfpathclose%
\pgfusepath{fill}%
\end{pgfscope}%
\begin{pgfscope}%
\pgfpathrectangle{\pgfqpoint{3.722897in}{0.857143in}}{\pgfqpoint{2.627103in}{1.813434in}}%
\pgfusepath{clip}%
\pgfsetbuttcap%
\pgfsetmiterjoin%
\definecolor{currentfill}{rgb}{0.992771,0.707689,0.712380}%
\pgfsetfillcolor{currentfill}%
\pgfsetlinewidth{0.000000pt}%
\definecolor{currentstroke}{rgb}{0.000000,0.000000,0.000000}%
\pgfsetstrokecolor{currentstroke}%
\pgfsetstrokeopacity{0.000000}%
\pgfsetdash{}{0pt}%
\pgfpathmoveto{\pgfqpoint{5.473219in}{1.498826in}}%
\pgfpathlineto{\pgfqpoint{5.482156in}{1.498826in}}%
\pgfpathlineto{\pgfqpoint{5.482156in}{1.198941in}}%
\pgfpathlineto{\pgfqpoint{5.473219in}{1.198941in}}%
\pgfpathlineto{\pgfqpoint{5.473219in}{1.498826in}}%
\pgfpathclose%
\pgfusepath{fill}%
\end{pgfscope}%
\begin{pgfscope}%
\pgfpathrectangle{\pgfqpoint{3.722897in}{0.857143in}}{\pgfqpoint{2.627103in}{1.813434in}}%
\pgfusepath{clip}%
\pgfsetbuttcap%
\pgfsetmiterjoin%
\definecolor{currentfill}{rgb}{0.992771,0.707689,0.712380}%
\pgfsetfillcolor{currentfill}%
\pgfsetlinewidth{0.000000pt}%
\definecolor{currentstroke}{rgb}{0.000000,0.000000,0.000000}%
\pgfsetstrokecolor{currentstroke}%
\pgfsetstrokeopacity{0.000000}%
\pgfsetdash{}{0pt}%
\pgfpathmoveto{\pgfqpoint{5.484390in}{1.501242in}}%
\pgfpathlineto{\pgfqpoint{5.493326in}{1.501242in}}%
\pgfpathlineto{\pgfqpoint{5.493326in}{1.248156in}}%
\pgfpathlineto{\pgfqpoint{5.484390in}{1.248156in}}%
\pgfpathlineto{\pgfqpoint{5.484390in}{1.501242in}}%
\pgfpathclose%
\pgfusepath{fill}%
\end{pgfscope}%
\begin{pgfscope}%
\pgfpathrectangle{\pgfqpoint{3.722897in}{0.857143in}}{\pgfqpoint{2.627103in}{1.813434in}}%
\pgfusepath{clip}%
\pgfsetbuttcap%
\pgfsetmiterjoin%
\definecolor{currentfill}{rgb}{0.992771,0.707689,0.712380}%
\pgfsetfillcolor{currentfill}%
\pgfsetlinewidth{0.000000pt}%
\definecolor{currentstroke}{rgb}{0.000000,0.000000,0.000000}%
\pgfsetstrokecolor{currentstroke}%
\pgfsetstrokeopacity{0.000000}%
\pgfsetdash{}{0pt}%
\pgfpathmoveto{\pgfqpoint{5.495560in}{1.478963in}}%
\pgfpathlineto{\pgfqpoint{5.504497in}{1.478963in}}%
\pgfpathlineto{\pgfqpoint{5.504497in}{1.277278in}}%
\pgfpathlineto{\pgfqpoint{5.495560in}{1.277278in}}%
\pgfpathlineto{\pgfqpoint{5.495560in}{1.478963in}}%
\pgfpathclose%
\pgfusepath{fill}%
\end{pgfscope}%
\begin{pgfscope}%
\pgfpathrectangle{\pgfqpoint{3.722897in}{0.857143in}}{\pgfqpoint{2.627103in}{1.813434in}}%
\pgfusepath{clip}%
\pgfsetbuttcap%
\pgfsetmiterjoin%
\definecolor{currentfill}{rgb}{0.992771,0.707689,0.712380}%
\pgfsetfillcolor{currentfill}%
\pgfsetlinewidth{0.000000pt}%
\definecolor{currentstroke}{rgb}{0.000000,0.000000,0.000000}%
\pgfsetstrokecolor{currentstroke}%
\pgfsetstrokeopacity{0.000000}%
\pgfsetdash{}{0pt}%
\pgfpathmoveto{\pgfqpoint{5.506731in}{1.493413in}}%
\pgfpathlineto{\pgfqpoint{5.515667in}{1.493413in}}%
\pgfpathlineto{\pgfqpoint{5.515667in}{1.331393in}}%
\pgfpathlineto{\pgfqpoint{5.506731in}{1.331393in}}%
\pgfpathlineto{\pgfqpoint{5.506731in}{1.493413in}}%
\pgfpathclose%
\pgfusepath{fill}%
\end{pgfscope}%
\begin{pgfscope}%
\pgfpathrectangle{\pgfqpoint{3.722897in}{0.857143in}}{\pgfqpoint{2.627103in}{1.813434in}}%
\pgfusepath{clip}%
\pgfsetbuttcap%
\pgfsetmiterjoin%
\definecolor{currentfill}{rgb}{0.992771,0.707689,0.712380}%
\pgfsetfillcolor{currentfill}%
\pgfsetlinewidth{0.000000pt}%
\definecolor{currentstroke}{rgb}{0.000000,0.000000,0.000000}%
\pgfsetstrokecolor{currentstroke}%
\pgfsetstrokeopacity{0.000000}%
\pgfsetdash{}{0pt}%
\pgfpathmoveto{\pgfqpoint{5.517902in}{1.476244in}}%
\pgfpathlineto{\pgfqpoint{5.526838in}{1.476244in}}%
\pgfpathlineto{\pgfqpoint{5.526838in}{1.357262in}}%
\pgfpathlineto{\pgfqpoint{5.517902in}{1.357262in}}%
\pgfpathlineto{\pgfqpoint{5.517902in}{1.476244in}}%
\pgfpathclose%
\pgfusepath{fill}%
\end{pgfscope}%
\begin{pgfscope}%
\pgfpathrectangle{\pgfqpoint{3.722897in}{0.857143in}}{\pgfqpoint{2.627103in}{1.813434in}}%
\pgfusepath{clip}%
\pgfsetbuttcap%
\pgfsetmiterjoin%
\definecolor{currentfill}{rgb}{0.992771,0.707689,0.712380}%
\pgfsetfillcolor{currentfill}%
\pgfsetlinewidth{0.000000pt}%
\definecolor{currentstroke}{rgb}{0.000000,0.000000,0.000000}%
\pgfsetstrokecolor{currentstroke}%
\pgfsetstrokeopacity{0.000000}%
\pgfsetdash{}{0pt}%
\pgfpathmoveto{\pgfqpoint{5.529072in}{1.471089in}}%
\pgfpathlineto{\pgfqpoint{5.538009in}{1.471089in}}%
\pgfpathlineto{\pgfqpoint{5.538009in}{1.390780in}}%
\pgfpathlineto{\pgfqpoint{5.529072in}{1.390780in}}%
\pgfpathlineto{\pgfqpoint{5.529072in}{1.471089in}}%
\pgfpathclose%
\pgfusepath{fill}%
\end{pgfscope}%
\begin{pgfscope}%
\pgfpathrectangle{\pgfqpoint{3.722897in}{0.857143in}}{\pgfqpoint{2.627103in}{1.813434in}}%
\pgfusepath{clip}%
\pgfsetbuttcap%
\pgfsetmiterjoin%
\definecolor{currentfill}{rgb}{0.992771,0.707689,0.712380}%
\pgfsetfillcolor{currentfill}%
\pgfsetlinewidth{0.000000pt}%
\definecolor{currentstroke}{rgb}{0.000000,0.000000,0.000000}%
\pgfsetstrokecolor{currentstroke}%
\pgfsetstrokeopacity{0.000000}%
\pgfsetdash{}{0pt}%
\pgfpathmoveto{\pgfqpoint{5.540243in}{1.459411in}}%
\pgfpathlineto{\pgfqpoint{5.549179in}{1.459411in}}%
\pgfpathlineto{\pgfqpoint{5.549179in}{1.410713in}}%
\pgfpathlineto{\pgfqpoint{5.540243in}{1.410713in}}%
\pgfpathlineto{\pgfqpoint{5.540243in}{1.459411in}}%
\pgfpathclose%
\pgfusepath{fill}%
\end{pgfscope}%
\begin{pgfscope}%
\pgfpathrectangle{\pgfqpoint{3.722897in}{0.857143in}}{\pgfqpoint{2.627103in}{1.813434in}}%
\pgfusepath{clip}%
\pgfsetbuttcap%
\pgfsetmiterjoin%
\definecolor{currentfill}{rgb}{0.992771,0.707689,0.712380}%
\pgfsetfillcolor{currentfill}%
\pgfsetlinewidth{0.000000pt}%
\definecolor{currentstroke}{rgb}{0.000000,0.000000,0.000000}%
\pgfsetstrokecolor{currentstroke}%
\pgfsetstrokeopacity{0.000000}%
\pgfsetdash{}{0pt}%
\pgfpathmoveto{\pgfqpoint{5.551413in}{1.461673in}}%
\pgfpathlineto{\pgfqpoint{5.560350in}{1.461673in}}%
\pgfpathlineto{\pgfqpoint{5.560350in}{1.440109in}}%
\pgfpathlineto{\pgfqpoint{5.551413in}{1.440109in}}%
\pgfpathlineto{\pgfqpoint{5.551413in}{1.461673in}}%
\pgfpathclose%
\pgfusepath{fill}%
\end{pgfscope}%
\begin{pgfscope}%
\pgfpathrectangle{\pgfqpoint{3.722897in}{0.857143in}}{\pgfqpoint{2.627103in}{1.813434in}}%
\pgfusepath{clip}%
\pgfsetbuttcap%
\pgfsetmiterjoin%
\definecolor{currentfill}{rgb}{0.992771,0.707689,0.712380}%
\pgfsetfillcolor{currentfill}%
\pgfsetlinewidth{0.000000pt}%
\definecolor{currentstroke}{rgb}{0.000000,0.000000,0.000000}%
\pgfsetstrokecolor{currentstroke}%
\pgfsetstrokeopacity{0.000000}%
\pgfsetdash{}{0pt}%
\pgfpathmoveto{\pgfqpoint{5.562584in}{2.106266in}}%
\pgfpathlineto{\pgfqpoint{5.571521in}{2.106266in}}%
\pgfpathlineto{\pgfqpoint{5.571521in}{2.113454in}}%
\pgfpathlineto{\pgfqpoint{5.562584in}{2.113454in}}%
\pgfpathlineto{\pgfqpoint{5.562584in}{2.106266in}}%
\pgfpathclose%
\pgfusepath{fill}%
\end{pgfscope}%
\begin{pgfscope}%
\pgfpathrectangle{\pgfqpoint{3.722897in}{0.857143in}}{\pgfqpoint{2.627103in}{1.813434in}}%
\pgfusepath{clip}%
\pgfsetbuttcap%
\pgfsetmiterjoin%
\definecolor{currentfill}{rgb}{0.992771,0.707689,0.712380}%
\pgfsetfillcolor{currentfill}%
\pgfsetlinewidth{0.000000pt}%
\definecolor{currentstroke}{rgb}{0.000000,0.000000,0.000000}%
\pgfsetstrokecolor{currentstroke}%
\pgfsetstrokeopacity{0.000000}%
\pgfsetdash{}{0pt}%
\pgfpathmoveto{\pgfqpoint{5.573755in}{2.111460in}}%
\pgfpathlineto{\pgfqpoint{5.582691in}{2.111460in}}%
\pgfpathlineto{\pgfqpoint{5.582691in}{2.130256in}}%
\pgfpathlineto{\pgfqpoint{5.573755in}{2.130256in}}%
\pgfpathlineto{\pgfqpoint{5.573755in}{2.111460in}}%
\pgfpathclose%
\pgfusepath{fill}%
\end{pgfscope}%
\begin{pgfscope}%
\pgfpathrectangle{\pgfqpoint{3.722897in}{0.857143in}}{\pgfqpoint{2.627103in}{1.813434in}}%
\pgfusepath{clip}%
\pgfsetbuttcap%
\pgfsetmiterjoin%
\definecolor{currentfill}{rgb}{0.992771,0.707689,0.712380}%
\pgfsetfillcolor{currentfill}%
\pgfsetlinewidth{0.000000pt}%
\definecolor{currentstroke}{rgb}{0.000000,0.000000,0.000000}%
\pgfsetstrokecolor{currentstroke}%
\pgfsetstrokeopacity{0.000000}%
\pgfsetdash{}{0pt}%
\pgfpathmoveto{\pgfqpoint{5.584925in}{2.114176in}}%
\pgfpathlineto{\pgfqpoint{5.593862in}{2.114176in}}%
\pgfpathlineto{\pgfqpoint{5.593862in}{2.138722in}}%
\pgfpathlineto{\pgfqpoint{5.584925in}{2.138722in}}%
\pgfpathlineto{\pgfqpoint{5.584925in}{2.114176in}}%
\pgfpathclose%
\pgfusepath{fill}%
\end{pgfscope}%
\begin{pgfscope}%
\pgfpathrectangle{\pgfqpoint{3.722897in}{0.857143in}}{\pgfqpoint{2.627103in}{1.813434in}}%
\pgfusepath{clip}%
\pgfsetbuttcap%
\pgfsetmiterjoin%
\definecolor{currentfill}{rgb}{0.992771,0.707689,0.712380}%
\pgfsetfillcolor{currentfill}%
\pgfsetlinewidth{0.000000pt}%
\definecolor{currentstroke}{rgb}{0.000000,0.000000,0.000000}%
\pgfsetstrokecolor{currentstroke}%
\pgfsetstrokeopacity{0.000000}%
\pgfsetdash{}{0pt}%
\pgfpathmoveto{\pgfqpoint{5.596096in}{2.118451in}}%
\pgfpathlineto{\pgfqpoint{5.605032in}{2.118451in}}%
\pgfpathlineto{\pgfqpoint{5.605032in}{2.153306in}}%
\pgfpathlineto{\pgfqpoint{5.596096in}{2.153306in}}%
\pgfpathlineto{\pgfqpoint{5.596096in}{2.118451in}}%
\pgfpathclose%
\pgfusepath{fill}%
\end{pgfscope}%
\begin{pgfscope}%
\pgfpathrectangle{\pgfqpoint{3.722897in}{0.857143in}}{\pgfqpoint{2.627103in}{1.813434in}}%
\pgfusepath{clip}%
\pgfsetbuttcap%
\pgfsetmiterjoin%
\definecolor{currentfill}{rgb}{0.992771,0.707689,0.712380}%
\pgfsetfillcolor{currentfill}%
\pgfsetlinewidth{0.000000pt}%
\definecolor{currentstroke}{rgb}{0.000000,0.000000,0.000000}%
\pgfsetstrokecolor{currentstroke}%
\pgfsetstrokeopacity{0.000000}%
\pgfsetdash{}{0pt}%
\pgfpathmoveto{\pgfqpoint{5.607266in}{2.129231in}}%
\pgfpathlineto{\pgfqpoint{5.616203in}{2.129231in}}%
\pgfpathlineto{\pgfqpoint{5.616203in}{2.169164in}}%
\pgfpathlineto{\pgfqpoint{5.607266in}{2.169164in}}%
\pgfpathlineto{\pgfqpoint{5.607266in}{2.129231in}}%
\pgfpathclose%
\pgfusepath{fill}%
\end{pgfscope}%
\begin{pgfscope}%
\pgfpathrectangle{\pgfqpoint{3.722897in}{0.857143in}}{\pgfqpoint{2.627103in}{1.813434in}}%
\pgfusepath{clip}%
\pgfsetbuttcap%
\pgfsetmiterjoin%
\definecolor{currentfill}{rgb}{0.992771,0.707689,0.712380}%
\pgfsetfillcolor{currentfill}%
\pgfsetlinewidth{0.000000pt}%
\definecolor{currentstroke}{rgb}{0.000000,0.000000,0.000000}%
\pgfsetstrokecolor{currentstroke}%
\pgfsetstrokeopacity{0.000000}%
\pgfsetdash{}{0pt}%
\pgfpathmoveto{\pgfqpoint{5.618437in}{2.138868in}}%
\pgfpathlineto{\pgfqpoint{5.627374in}{2.138868in}}%
\pgfpathlineto{\pgfqpoint{5.627374in}{2.164977in}}%
\pgfpathlineto{\pgfqpoint{5.618437in}{2.164977in}}%
\pgfpathlineto{\pgfqpoint{5.618437in}{2.138868in}}%
\pgfpathclose%
\pgfusepath{fill}%
\end{pgfscope}%
\begin{pgfscope}%
\pgfpathrectangle{\pgfqpoint{3.722897in}{0.857143in}}{\pgfqpoint{2.627103in}{1.813434in}}%
\pgfusepath{clip}%
\pgfsetbuttcap%
\pgfsetmiterjoin%
\definecolor{currentfill}{rgb}{0.992771,0.707689,0.712380}%
\pgfsetfillcolor{currentfill}%
\pgfsetlinewidth{0.000000pt}%
\definecolor{currentstroke}{rgb}{0.000000,0.000000,0.000000}%
\pgfsetstrokecolor{currentstroke}%
\pgfsetstrokeopacity{0.000000}%
\pgfsetdash{}{0pt}%
\pgfpathmoveto{\pgfqpoint{5.629608in}{2.144659in}}%
\pgfpathlineto{\pgfqpoint{5.638544in}{2.144659in}}%
\pgfpathlineto{\pgfqpoint{5.638544in}{2.149169in}}%
\pgfpathlineto{\pgfqpoint{5.629608in}{2.149169in}}%
\pgfpathlineto{\pgfqpoint{5.629608in}{2.144659in}}%
\pgfpathclose%
\pgfusepath{fill}%
\end{pgfscope}%
\begin{pgfscope}%
\pgfpathrectangle{\pgfqpoint{3.722897in}{0.857143in}}{\pgfqpoint{2.627103in}{1.813434in}}%
\pgfusepath{clip}%
\pgfsetbuttcap%
\pgfsetmiterjoin%
\definecolor{currentfill}{rgb}{0.992771,0.707689,0.712380}%
\pgfsetfillcolor{currentfill}%
\pgfsetlinewidth{0.000000pt}%
\definecolor{currentstroke}{rgb}{0.000000,0.000000,0.000000}%
\pgfsetstrokecolor{currentstroke}%
\pgfsetstrokeopacity{0.000000}%
\pgfsetdash{}{0pt}%
\pgfpathmoveto{\pgfqpoint{5.640778in}{1.396584in}}%
\pgfpathlineto{\pgfqpoint{5.649715in}{1.396584in}}%
\pgfpathlineto{\pgfqpoint{5.649715in}{1.378573in}}%
\pgfpathlineto{\pgfqpoint{5.640778in}{1.378573in}}%
\pgfpathlineto{\pgfqpoint{5.640778in}{1.396584in}}%
\pgfpathclose%
\pgfusepath{fill}%
\end{pgfscope}%
\begin{pgfscope}%
\pgfpathrectangle{\pgfqpoint{3.722897in}{0.857143in}}{\pgfqpoint{2.627103in}{1.813434in}}%
\pgfusepath{clip}%
\pgfsetbuttcap%
\pgfsetmiterjoin%
\definecolor{currentfill}{rgb}{0.992771,0.707689,0.712380}%
\pgfsetfillcolor{currentfill}%
\pgfsetlinewidth{0.000000pt}%
\definecolor{currentstroke}{rgb}{0.000000,0.000000,0.000000}%
\pgfsetstrokecolor{currentstroke}%
\pgfsetstrokeopacity{0.000000}%
\pgfsetdash{}{0pt}%
\pgfpathmoveto{\pgfqpoint{5.651949in}{1.441826in}}%
\pgfpathlineto{\pgfqpoint{5.660885in}{1.441826in}}%
\pgfpathlineto{\pgfqpoint{5.660885in}{1.399439in}}%
\pgfpathlineto{\pgfqpoint{5.651949in}{1.399439in}}%
\pgfpathlineto{\pgfqpoint{5.651949in}{1.441826in}}%
\pgfpathclose%
\pgfusepath{fill}%
\end{pgfscope}%
\begin{pgfscope}%
\pgfpathrectangle{\pgfqpoint{3.722897in}{0.857143in}}{\pgfqpoint{2.627103in}{1.813434in}}%
\pgfusepath{clip}%
\pgfsetbuttcap%
\pgfsetmiterjoin%
\definecolor{currentfill}{rgb}{0.992771,0.707689,0.712380}%
\pgfsetfillcolor{currentfill}%
\pgfsetlinewidth{0.000000pt}%
\definecolor{currentstroke}{rgb}{0.000000,0.000000,0.000000}%
\pgfsetstrokecolor{currentstroke}%
\pgfsetstrokeopacity{0.000000}%
\pgfsetdash{}{0pt}%
\pgfpathmoveto{\pgfqpoint{5.663119in}{1.435796in}}%
\pgfpathlineto{\pgfqpoint{5.672056in}{1.435796in}}%
\pgfpathlineto{\pgfqpoint{5.672056in}{1.373519in}}%
\pgfpathlineto{\pgfqpoint{5.663119in}{1.373519in}}%
\pgfpathlineto{\pgfqpoint{5.663119in}{1.435796in}}%
\pgfpathclose%
\pgfusepath{fill}%
\end{pgfscope}%
\begin{pgfscope}%
\pgfpathrectangle{\pgfqpoint{3.722897in}{0.857143in}}{\pgfqpoint{2.627103in}{1.813434in}}%
\pgfusepath{clip}%
\pgfsetbuttcap%
\pgfsetmiterjoin%
\definecolor{currentfill}{rgb}{0.992771,0.707689,0.712380}%
\pgfsetfillcolor{currentfill}%
\pgfsetlinewidth{0.000000pt}%
\definecolor{currentstroke}{rgb}{0.000000,0.000000,0.000000}%
\pgfsetstrokecolor{currentstroke}%
\pgfsetstrokeopacity{0.000000}%
\pgfsetdash{}{0pt}%
\pgfpathmoveto{\pgfqpoint{5.674290in}{1.464141in}}%
\pgfpathlineto{\pgfqpoint{5.683227in}{1.464141in}}%
\pgfpathlineto{\pgfqpoint{5.683227in}{1.368459in}}%
\pgfpathlineto{\pgfqpoint{5.674290in}{1.368459in}}%
\pgfpathlineto{\pgfqpoint{5.674290in}{1.464141in}}%
\pgfpathclose%
\pgfusepath{fill}%
\end{pgfscope}%
\begin{pgfscope}%
\pgfpathrectangle{\pgfqpoint{3.722897in}{0.857143in}}{\pgfqpoint{2.627103in}{1.813434in}}%
\pgfusepath{clip}%
\pgfsetbuttcap%
\pgfsetmiterjoin%
\definecolor{currentfill}{rgb}{0.992771,0.707689,0.712380}%
\pgfsetfillcolor{currentfill}%
\pgfsetlinewidth{0.000000pt}%
\definecolor{currentstroke}{rgb}{0.000000,0.000000,0.000000}%
\pgfsetstrokecolor{currentstroke}%
\pgfsetstrokeopacity{0.000000}%
\pgfsetdash{}{0pt}%
\pgfpathmoveto{\pgfqpoint{5.685461in}{1.498334in}}%
\pgfpathlineto{\pgfqpoint{5.694397in}{1.498334in}}%
\pgfpathlineto{\pgfqpoint{5.694397in}{1.377427in}}%
\pgfpathlineto{\pgfqpoint{5.685461in}{1.377427in}}%
\pgfpathlineto{\pgfqpoint{5.685461in}{1.498334in}}%
\pgfpathclose%
\pgfusepath{fill}%
\end{pgfscope}%
\begin{pgfscope}%
\pgfpathrectangle{\pgfqpoint{3.722897in}{0.857143in}}{\pgfqpoint{2.627103in}{1.813434in}}%
\pgfusepath{clip}%
\pgfsetbuttcap%
\pgfsetmiterjoin%
\definecolor{currentfill}{rgb}{0.992771,0.707689,0.712380}%
\pgfsetfillcolor{currentfill}%
\pgfsetlinewidth{0.000000pt}%
\definecolor{currentstroke}{rgb}{0.000000,0.000000,0.000000}%
\pgfsetstrokecolor{currentstroke}%
\pgfsetstrokeopacity{0.000000}%
\pgfsetdash{}{0pt}%
\pgfpathmoveto{\pgfqpoint{5.696631in}{1.505748in}}%
\pgfpathlineto{\pgfqpoint{5.705568in}{1.505748in}}%
\pgfpathlineto{\pgfqpoint{5.705568in}{1.371872in}}%
\pgfpathlineto{\pgfqpoint{5.696631in}{1.371872in}}%
\pgfpathlineto{\pgfqpoint{5.696631in}{1.505748in}}%
\pgfpathclose%
\pgfusepath{fill}%
\end{pgfscope}%
\begin{pgfscope}%
\pgfpathrectangle{\pgfqpoint{3.722897in}{0.857143in}}{\pgfqpoint{2.627103in}{1.813434in}}%
\pgfusepath{clip}%
\pgfsetbuttcap%
\pgfsetmiterjoin%
\definecolor{currentfill}{rgb}{0.992771,0.707689,0.712380}%
\pgfsetfillcolor{currentfill}%
\pgfsetlinewidth{0.000000pt}%
\definecolor{currentstroke}{rgb}{0.000000,0.000000,0.000000}%
\pgfsetstrokecolor{currentstroke}%
\pgfsetstrokeopacity{0.000000}%
\pgfsetdash{}{0pt}%
\pgfpathmoveto{\pgfqpoint{5.707802in}{1.502658in}}%
\pgfpathlineto{\pgfqpoint{5.716738in}{1.502658in}}%
\pgfpathlineto{\pgfqpoint{5.716738in}{1.366024in}}%
\pgfpathlineto{\pgfqpoint{5.707802in}{1.366024in}}%
\pgfpathlineto{\pgfqpoint{5.707802in}{1.502658in}}%
\pgfpathclose%
\pgfusepath{fill}%
\end{pgfscope}%
\begin{pgfscope}%
\pgfpathrectangle{\pgfqpoint{3.722897in}{0.857143in}}{\pgfqpoint{2.627103in}{1.813434in}}%
\pgfusepath{clip}%
\pgfsetbuttcap%
\pgfsetmiterjoin%
\definecolor{currentfill}{rgb}{0.992771,0.707689,0.712380}%
\pgfsetfillcolor{currentfill}%
\pgfsetlinewidth{0.000000pt}%
\definecolor{currentstroke}{rgb}{0.000000,0.000000,0.000000}%
\pgfsetstrokecolor{currentstroke}%
\pgfsetstrokeopacity{0.000000}%
\pgfsetdash{}{0pt}%
\pgfpathmoveto{\pgfqpoint{5.718972in}{1.509960in}}%
\pgfpathlineto{\pgfqpoint{5.727909in}{1.509960in}}%
\pgfpathlineto{\pgfqpoint{5.727909in}{1.390967in}}%
\pgfpathlineto{\pgfqpoint{5.718972in}{1.390967in}}%
\pgfpathlineto{\pgfqpoint{5.718972in}{1.509960in}}%
\pgfpathclose%
\pgfusepath{fill}%
\end{pgfscope}%
\begin{pgfscope}%
\pgfpathrectangle{\pgfqpoint{3.722897in}{0.857143in}}{\pgfqpoint{2.627103in}{1.813434in}}%
\pgfusepath{clip}%
\pgfsetbuttcap%
\pgfsetmiterjoin%
\definecolor{currentfill}{rgb}{0.992771,0.707689,0.712380}%
\pgfsetfillcolor{currentfill}%
\pgfsetlinewidth{0.000000pt}%
\definecolor{currentstroke}{rgb}{0.000000,0.000000,0.000000}%
\pgfsetstrokecolor{currentstroke}%
\pgfsetstrokeopacity{0.000000}%
\pgfsetdash{}{0pt}%
\pgfpathmoveto{\pgfqpoint{5.730143in}{1.492369in}}%
\pgfpathlineto{\pgfqpoint{5.739080in}{1.492369in}}%
\pgfpathlineto{\pgfqpoint{5.739080in}{1.447196in}}%
\pgfpathlineto{\pgfqpoint{5.730143in}{1.447196in}}%
\pgfpathlineto{\pgfqpoint{5.730143in}{1.492369in}}%
\pgfpathclose%
\pgfusepath{fill}%
\end{pgfscope}%
\begin{pgfscope}%
\pgfpathrectangle{\pgfqpoint{3.722897in}{0.857143in}}{\pgfqpoint{2.627103in}{1.813434in}}%
\pgfusepath{clip}%
\pgfsetbuttcap%
\pgfsetmiterjoin%
\definecolor{currentfill}{rgb}{0.992771,0.707689,0.712380}%
\pgfsetfillcolor{currentfill}%
\pgfsetlinewidth{0.000000pt}%
\definecolor{currentstroke}{rgb}{0.000000,0.000000,0.000000}%
\pgfsetstrokecolor{currentstroke}%
\pgfsetstrokeopacity{0.000000}%
\pgfsetdash{}{0pt}%
\pgfpathmoveto{\pgfqpoint{5.741314in}{2.100331in}}%
\pgfpathlineto{\pgfqpoint{5.750250in}{2.100331in}}%
\pgfpathlineto{\pgfqpoint{5.750250in}{2.128641in}}%
\pgfpathlineto{\pgfqpoint{5.741314in}{2.128641in}}%
\pgfpathlineto{\pgfqpoint{5.741314in}{2.100331in}}%
\pgfpathclose%
\pgfusepath{fill}%
\end{pgfscope}%
\begin{pgfscope}%
\pgfpathrectangle{\pgfqpoint{3.722897in}{0.857143in}}{\pgfqpoint{2.627103in}{1.813434in}}%
\pgfusepath{clip}%
\pgfsetbuttcap%
\pgfsetmiterjoin%
\definecolor{currentfill}{rgb}{0.992771,0.707689,0.712380}%
\pgfsetfillcolor{currentfill}%
\pgfsetlinewidth{0.000000pt}%
\definecolor{currentstroke}{rgb}{0.000000,0.000000,0.000000}%
\pgfsetstrokecolor{currentstroke}%
\pgfsetstrokeopacity{0.000000}%
\pgfsetdash{}{0pt}%
\pgfpathmoveto{\pgfqpoint{5.752484in}{2.068005in}}%
\pgfpathlineto{\pgfqpoint{5.761421in}{2.068005in}}%
\pgfpathlineto{\pgfqpoint{5.761421in}{2.151946in}}%
\pgfpathlineto{\pgfqpoint{5.752484in}{2.151946in}}%
\pgfpathlineto{\pgfqpoint{5.752484in}{2.068005in}}%
\pgfpathclose%
\pgfusepath{fill}%
\end{pgfscope}%
\begin{pgfscope}%
\pgfpathrectangle{\pgfqpoint{3.722897in}{0.857143in}}{\pgfqpoint{2.627103in}{1.813434in}}%
\pgfusepath{clip}%
\pgfsetbuttcap%
\pgfsetmiterjoin%
\definecolor{currentfill}{rgb}{0.992771,0.707689,0.712380}%
\pgfsetfillcolor{currentfill}%
\pgfsetlinewidth{0.000000pt}%
\definecolor{currentstroke}{rgb}{0.000000,0.000000,0.000000}%
\pgfsetstrokecolor{currentstroke}%
\pgfsetstrokeopacity{0.000000}%
\pgfsetdash{}{0pt}%
\pgfpathmoveto{\pgfqpoint{5.763655in}{2.028351in}}%
\pgfpathlineto{\pgfqpoint{5.772591in}{2.028351in}}%
\pgfpathlineto{\pgfqpoint{5.772591in}{2.169590in}}%
\pgfpathlineto{\pgfqpoint{5.763655in}{2.169590in}}%
\pgfpathlineto{\pgfqpoint{5.763655in}{2.028351in}}%
\pgfpathclose%
\pgfusepath{fill}%
\end{pgfscope}%
\begin{pgfscope}%
\pgfpathrectangle{\pgfqpoint{3.722897in}{0.857143in}}{\pgfqpoint{2.627103in}{1.813434in}}%
\pgfusepath{clip}%
\pgfsetbuttcap%
\pgfsetmiterjoin%
\definecolor{currentfill}{rgb}{0.992771,0.707689,0.712380}%
\pgfsetfillcolor{currentfill}%
\pgfsetlinewidth{0.000000pt}%
\definecolor{currentstroke}{rgb}{0.000000,0.000000,0.000000}%
\pgfsetstrokecolor{currentstroke}%
\pgfsetstrokeopacity{0.000000}%
\pgfsetdash{}{0pt}%
\pgfpathmoveto{\pgfqpoint{5.774826in}{1.993353in}}%
\pgfpathlineto{\pgfqpoint{5.783762in}{1.993353in}}%
\pgfpathlineto{\pgfqpoint{5.783762in}{2.196508in}}%
\pgfpathlineto{\pgfqpoint{5.774826in}{2.196508in}}%
\pgfpathlineto{\pgfqpoint{5.774826in}{1.993353in}}%
\pgfpathclose%
\pgfusepath{fill}%
\end{pgfscope}%
\begin{pgfscope}%
\pgfpathrectangle{\pgfqpoint{3.722897in}{0.857143in}}{\pgfqpoint{2.627103in}{1.813434in}}%
\pgfusepath{clip}%
\pgfsetbuttcap%
\pgfsetmiterjoin%
\definecolor{currentfill}{rgb}{0.992771,0.707689,0.712380}%
\pgfsetfillcolor{currentfill}%
\pgfsetlinewidth{0.000000pt}%
\definecolor{currentstroke}{rgb}{0.000000,0.000000,0.000000}%
\pgfsetstrokecolor{currentstroke}%
\pgfsetstrokeopacity{0.000000}%
\pgfsetdash{}{0pt}%
\pgfpathmoveto{\pgfqpoint{5.785996in}{1.964114in}}%
\pgfpathlineto{\pgfqpoint{5.794933in}{1.964114in}}%
\pgfpathlineto{\pgfqpoint{5.794933in}{2.227561in}}%
\pgfpathlineto{\pgfqpoint{5.785996in}{2.227561in}}%
\pgfpathlineto{\pgfqpoint{5.785996in}{1.964114in}}%
\pgfpathclose%
\pgfusepath{fill}%
\end{pgfscope}%
\begin{pgfscope}%
\pgfpathrectangle{\pgfqpoint{3.722897in}{0.857143in}}{\pgfqpoint{2.627103in}{1.813434in}}%
\pgfusepath{clip}%
\pgfsetbuttcap%
\pgfsetmiterjoin%
\definecolor{currentfill}{rgb}{0.992771,0.707689,0.712380}%
\pgfsetfillcolor{currentfill}%
\pgfsetlinewidth{0.000000pt}%
\definecolor{currentstroke}{rgb}{0.000000,0.000000,0.000000}%
\pgfsetstrokecolor{currentstroke}%
\pgfsetstrokeopacity{0.000000}%
\pgfsetdash{}{0pt}%
\pgfpathmoveto{\pgfqpoint{5.797167in}{1.925940in}}%
\pgfpathlineto{\pgfqpoint{5.806103in}{1.925940in}}%
\pgfpathlineto{\pgfqpoint{5.806103in}{2.249092in}}%
\pgfpathlineto{\pgfqpoint{5.797167in}{2.249092in}}%
\pgfpathlineto{\pgfqpoint{5.797167in}{1.925940in}}%
\pgfpathclose%
\pgfusepath{fill}%
\end{pgfscope}%
\begin{pgfscope}%
\pgfpathrectangle{\pgfqpoint{3.722897in}{0.857143in}}{\pgfqpoint{2.627103in}{1.813434in}}%
\pgfusepath{clip}%
\pgfsetbuttcap%
\pgfsetmiterjoin%
\definecolor{currentfill}{rgb}{0.992771,0.707689,0.712380}%
\pgfsetfillcolor{currentfill}%
\pgfsetlinewidth{0.000000pt}%
\definecolor{currentstroke}{rgb}{0.000000,0.000000,0.000000}%
\pgfsetstrokecolor{currentstroke}%
\pgfsetstrokeopacity{0.000000}%
\pgfsetdash{}{0pt}%
\pgfpathmoveto{\pgfqpoint{5.808337in}{1.907462in}}%
\pgfpathlineto{\pgfqpoint{5.817274in}{1.907462in}}%
\pgfpathlineto{\pgfqpoint{5.817274in}{2.281836in}}%
\pgfpathlineto{\pgfqpoint{5.808337in}{2.281836in}}%
\pgfpathlineto{\pgfqpoint{5.808337in}{1.907462in}}%
\pgfpathclose%
\pgfusepath{fill}%
\end{pgfscope}%
\begin{pgfscope}%
\pgfpathrectangle{\pgfqpoint{3.722897in}{0.857143in}}{\pgfqpoint{2.627103in}{1.813434in}}%
\pgfusepath{clip}%
\pgfsetbuttcap%
\pgfsetmiterjoin%
\definecolor{currentfill}{rgb}{0.992771,0.707689,0.712380}%
\pgfsetfillcolor{currentfill}%
\pgfsetlinewidth{0.000000pt}%
\definecolor{currentstroke}{rgb}{0.000000,0.000000,0.000000}%
\pgfsetstrokecolor{currentstroke}%
\pgfsetstrokeopacity{0.000000}%
\pgfsetdash{}{0pt}%
\pgfpathmoveto{\pgfqpoint{5.819508in}{1.877348in}}%
\pgfpathlineto{\pgfqpoint{5.828444in}{1.877348in}}%
\pgfpathlineto{\pgfqpoint{5.828444in}{2.295507in}}%
\pgfpathlineto{\pgfqpoint{5.819508in}{2.295507in}}%
\pgfpathlineto{\pgfqpoint{5.819508in}{1.877348in}}%
\pgfpathclose%
\pgfusepath{fill}%
\end{pgfscope}%
\begin{pgfscope}%
\pgfpathrectangle{\pgfqpoint{3.722897in}{0.857143in}}{\pgfqpoint{2.627103in}{1.813434in}}%
\pgfusepath{clip}%
\pgfsetbuttcap%
\pgfsetmiterjoin%
\definecolor{currentfill}{rgb}{0.992771,0.707689,0.712380}%
\pgfsetfillcolor{currentfill}%
\pgfsetlinewidth{0.000000pt}%
\definecolor{currentstroke}{rgb}{0.000000,0.000000,0.000000}%
\pgfsetstrokecolor{currentstroke}%
\pgfsetstrokeopacity{0.000000}%
\pgfsetdash{}{0pt}%
\pgfpathmoveto{\pgfqpoint{5.830679in}{1.872119in}}%
\pgfpathlineto{\pgfqpoint{5.839615in}{1.872119in}}%
\pgfpathlineto{\pgfqpoint{5.839615in}{2.339877in}}%
\pgfpathlineto{\pgfqpoint{5.830679in}{2.339877in}}%
\pgfpathlineto{\pgfqpoint{5.830679in}{1.872119in}}%
\pgfpathclose%
\pgfusepath{fill}%
\end{pgfscope}%
\begin{pgfscope}%
\pgfpathrectangle{\pgfqpoint{3.722897in}{0.857143in}}{\pgfqpoint{2.627103in}{1.813434in}}%
\pgfusepath{clip}%
\pgfsetbuttcap%
\pgfsetmiterjoin%
\definecolor{currentfill}{rgb}{0.992771,0.707689,0.712380}%
\pgfsetfillcolor{currentfill}%
\pgfsetlinewidth{0.000000pt}%
\definecolor{currentstroke}{rgb}{0.000000,0.000000,0.000000}%
\pgfsetstrokecolor{currentstroke}%
\pgfsetstrokeopacity{0.000000}%
\pgfsetdash{}{0pt}%
\pgfpathmoveto{\pgfqpoint{5.841849in}{1.867941in}}%
\pgfpathlineto{\pgfqpoint{5.850786in}{1.867941in}}%
\pgfpathlineto{\pgfqpoint{5.850786in}{2.378290in}}%
\pgfpathlineto{\pgfqpoint{5.841849in}{2.378290in}}%
\pgfpathlineto{\pgfqpoint{5.841849in}{1.867941in}}%
\pgfpathclose%
\pgfusepath{fill}%
\end{pgfscope}%
\begin{pgfscope}%
\pgfpathrectangle{\pgfqpoint{3.722897in}{0.857143in}}{\pgfqpoint{2.627103in}{1.813434in}}%
\pgfusepath{clip}%
\pgfsetbuttcap%
\pgfsetmiterjoin%
\definecolor{currentfill}{rgb}{0.992771,0.707689,0.712380}%
\pgfsetfillcolor{currentfill}%
\pgfsetlinewidth{0.000000pt}%
\definecolor{currentstroke}{rgb}{0.000000,0.000000,0.000000}%
\pgfsetstrokecolor{currentstroke}%
\pgfsetstrokeopacity{0.000000}%
\pgfsetdash{}{0pt}%
\pgfpathmoveto{\pgfqpoint{5.853020in}{1.867959in}}%
\pgfpathlineto{\pgfqpoint{5.861956in}{1.867959in}}%
\pgfpathlineto{\pgfqpoint{5.861956in}{2.421237in}}%
\pgfpathlineto{\pgfqpoint{5.853020in}{2.421237in}}%
\pgfpathlineto{\pgfqpoint{5.853020in}{1.867959in}}%
\pgfpathclose%
\pgfusepath{fill}%
\end{pgfscope}%
\begin{pgfscope}%
\pgfpathrectangle{\pgfqpoint{3.722897in}{0.857143in}}{\pgfqpoint{2.627103in}{1.813434in}}%
\pgfusepath{clip}%
\pgfsetbuttcap%
\pgfsetmiterjoin%
\definecolor{currentfill}{rgb}{0.992771,0.707689,0.712380}%
\pgfsetfillcolor{currentfill}%
\pgfsetlinewidth{0.000000pt}%
\definecolor{currentstroke}{rgb}{0.000000,0.000000,0.000000}%
\pgfsetstrokecolor{currentstroke}%
\pgfsetstrokeopacity{0.000000}%
\pgfsetdash{}{0pt}%
\pgfpathmoveto{\pgfqpoint{5.864190in}{1.877173in}}%
\pgfpathlineto{\pgfqpoint{5.873127in}{1.877173in}}%
\pgfpathlineto{\pgfqpoint{5.873127in}{2.480030in}}%
\pgfpathlineto{\pgfqpoint{5.864190in}{2.480030in}}%
\pgfpathlineto{\pgfqpoint{5.864190in}{1.877173in}}%
\pgfpathclose%
\pgfusepath{fill}%
\end{pgfscope}%
\begin{pgfscope}%
\pgfpathrectangle{\pgfqpoint{3.722897in}{0.857143in}}{\pgfqpoint{2.627103in}{1.813434in}}%
\pgfusepath{clip}%
\pgfsetbuttcap%
\pgfsetmiterjoin%
\definecolor{currentfill}{rgb}{0.992771,0.707689,0.712380}%
\pgfsetfillcolor{currentfill}%
\pgfsetlinewidth{0.000000pt}%
\definecolor{currentstroke}{rgb}{0.000000,0.000000,0.000000}%
\pgfsetstrokecolor{currentstroke}%
\pgfsetstrokeopacity{0.000000}%
\pgfsetdash{}{0pt}%
\pgfpathmoveto{\pgfqpoint{5.875361in}{1.868604in}}%
\pgfpathlineto{\pgfqpoint{5.884297in}{1.868604in}}%
\pgfpathlineto{\pgfqpoint{5.884297in}{2.504623in}}%
\pgfpathlineto{\pgfqpoint{5.875361in}{2.504623in}}%
\pgfpathlineto{\pgfqpoint{5.875361in}{1.868604in}}%
\pgfpathclose%
\pgfusepath{fill}%
\end{pgfscope}%
\begin{pgfscope}%
\pgfpathrectangle{\pgfqpoint{3.722897in}{0.857143in}}{\pgfqpoint{2.627103in}{1.813434in}}%
\pgfusepath{clip}%
\pgfsetbuttcap%
\pgfsetmiterjoin%
\definecolor{currentfill}{rgb}{0.992771,0.707689,0.712380}%
\pgfsetfillcolor{currentfill}%
\pgfsetlinewidth{0.000000pt}%
\definecolor{currentstroke}{rgb}{0.000000,0.000000,0.000000}%
\pgfsetstrokecolor{currentstroke}%
\pgfsetstrokeopacity{0.000000}%
\pgfsetdash{}{0pt}%
\pgfpathmoveto{\pgfqpoint{5.886532in}{1.868885in}}%
\pgfpathlineto{\pgfqpoint{5.895468in}{1.868885in}}%
\pgfpathlineto{\pgfqpoint{5.895468in}{2.533731in}}%
\pgfpathlineto{\pgfqpoint{5.886532in}{2.533731in}}%
\pgfpathlineto{\pgfqpoint{5.886532in}{1.868885in}}%
\pgfpathclose%
\pgfusepath{fill}%
\end{pgfscope}%
\begin{pgfscope}%
\pgfpathrectangle{\pgfqpoint{3.722897in}{0.857143in}}{\pgfqpoint{2.627103in}{1.813434in}}%
\pgfusepath{clip}%
\pgfsetbuttcap%
\pgfsetmiterjoin%
\definecolor{currentfill}{rgb}{0.992771,0.707689,0.712380}%
\pgfsetfillcolor{currentfill}%
\pgfsetlinewidth{0.000000pt}%
\definecolor{currentstroke}{rgb}{0.000000,0.000000,0.000000}%
\pgfsetstrokecolor{currentstroke}%
\pgfsetstrokeopacity{0.000000}%
\pgfsetdash{}{0pt}%
\pgfpathmoveto{\pgfqpoint{5.897702in}{1.867412in}}%
\pgfpathlineto{\pgfqpoint{5.906639in}{1.867412in}}%
\pgfpathlineto{\pgfqpoint{5.906639in}{2.550568in}}%
\pgfpathlineto{\pgfqpoint{5.897702in}{2.550568in}}%
\pgfpathlineto{\pgfqpoint{5.897702in}{1.867412in}}%
\pgfpathclose%
\pgfusepath{fill}%
\end{pgfscope}%
\begin{pgfscope}%
\pgfpathrectangle{\pgfqpoint{3.722897in}{0.857143in}}{\pgfqpoint{2.627103in}{1.813434in}}%
\pgfusepath{clip}%
\pgfsetbuttcap%
\pgfsetmiterjoin%
\definecolor{currentfill}{rgb}{0.992771,0.707689,0.712380}%
\pgfsetfillcolor{currentfill}%
\pgfsetlinewidth{0.000000pt}%
\definecolor{currentstroke}{rgb}{0.000000,0.000000,0.000000}%
\pgfsetstrokecolor{currentstroke}%
\pgfsetstrokeopacity{0.000000}%
\pgfsetdash{}{0pt}%
\pgfpathmoveto{\pgfqpoint{5.908873in}{1.865495in}}%
\pgfpathlineto{\pgfqpoint{5.917809in}{1.865495in}}%
\pgfpathlineto{\pgfqpoint{5.917809in}{2.565161in}}%
\pgfpathlineto{\pgfqpoint{5.908873in}{2.565161in}}%
\pgfpathlineto{\pgfqpoint{5.908873in}{1.865495in}}%
\pgfpathclose%
\pgfusepath{fill}%
\end{pgfscope}%
\begin{pgfscope}%
\pgfpathrectangle{\pgfqpoint{3.722897in}{0.857143in}}{\pgfqpoint{2.627103in}{1.813434in}}%
\pgfusepath{clip}%
\pgfsetbuttcap%
\pgfsetmiterjoin%
\definecolor{currentfill}{rgb}{0.992771,0.707689,0.712380}%
\pgfsetfillcolor{currentfill}%
\pgfsetlinewidth{0.000000pt}%
\definecolor{currentstroke}{rgb}{0.000000,0.000000,0.000000}%
\pgfsetstrokecolor{currentstroke}%
\pgfsetstrokeopacity{0.000000}%
\pgfsetdash{}{0pt}%
\pgfpathmoveto{\pgfqpoint{5.920043in}{1.864821in}}%
\pgfpathlineto{\pgfqpoint{5.928980in}{1.864821in}}%
\pgfpathlineto{\pgfqpoint{5.928980in}{2.585244in}}%
\pgfpathlineto{\pgfqpoint{5.920043in}{2.585244in}}%
\pgfpathlineto{\pgfqpoint{5.920043in}{1.864821in}}%
\pgfpathclose%
\pgfusepath{fill}%
\end{pgfscope}%
\begin{pgfscope}%
\pgfpathrectangle{\pgfqpoint{3.722897in}{0.857143in}}{\pgfqpoint{2.627103in}{1.813434in}}%
\pgfusepath{clip}%
\pgfsetbuttcap%
\pgfsetmiterjoin%
\definecolor{currentfill}{rgb}{0.992771,0.707689,0.712380}%
\pgfsetfillcolor{currentfill}%
\pgfsetlinewidth{0.000000pt}%
\definecolor{currentstroke}{rgb}{0.000000,0.000000,0.000000}%
\pgfsetstrokecolor{currentstroke}%
\pgfsetstrokeopacity{0.000000}%
\pgfsetdash{}{0pt}%
\pgfpathmoveto{\pgfqpoint{5.931214in}{1.863637in}}%
\pgfpathlineto{\pgfqpoint{5.940150in}{1.863637in}}%
\pgfpathlineto{\pgfqpoint{5.940150in}{2.588148in}}%
\pgfpathlineto{\pgfqpoint{5.931214in}{2.588148in}}%
\pgfpathlineto{\pgfqpoint{5.931214in}{1.863637in}}%
\pgfpathclose%
\pgfusepath{fill}%
\end{pgfscope}%
\begin{pgfscope}%
\pgfpathrectangle{\pgfqpoint{3.722897in}{0.857143in}}{\pgfqpoint{2.627103in}{1.813434in}}%
\pgfusepath{clip}%
\pgfsetbuttcap%
\pgfsetmiterjoin%
\definecolor{currentfill}{rgb}{0.992771,0.707689,0.712380}%
\pgfsetfillcolor{currentfill}%
\pgfsetlinewidth{0.000000pt}%
\definecolor{currentstroke}{rgb}{0.000000,0.000000,0.000000}%
\pgfsetstrokecolor{currentstroke}%
\pgfsetstrokeopacity{0.000000}%
\pgfsetdash{}{0pt}%
\pgfpathmoveto{\pgfqpoint{5.942385in}{1.862196in}}%
\pgfpathlineto{\pgfqpoint{5.951321in}{1.862196in}}%
\pgfpathlineto{\pgfqpoint{5.951321in}{2.572214in}}%
\pgfpathlineto{\pgfqpoint{5.942385in}{2.572214in}}%
\pgfpathlineto{\pgfqpoint{5.942385in}{1.862196in}}%
\pgfpathclose%
\pgfusepath{fill}%
\end{pgfscope}%
\begin{pgfscope}%
\pgfpathrectangle{\pgfqpoint{3.722897in}{0.857143in}}{\pgfqpoint{2.627103in}{1.813434in}}%
\pgfusepath{clip}%
\pgfsetbuttcap%
\pgfsetmiterjoin%
\definecolor{currentfill}{rgb}{0.992771,0.707689,0.712380}%
\pgfsetfillcolor{currentfill}%
\pgfsetlinewidth{0.000000pt}%
\definecolor{currentstroke}{rgb}{0.000000,0.000000,0.000000}%
\pgfsetstrokecolor{currentstroke}%
\pgfsetstrokeopacity{0.000000}%
\pgfsetdash{}{0pt}%
\pgfpathmoveto{\pgfqpoint{5.953555in}{1.861665in}}%
\pgfpathlineto{\pgfqpoint{5.962492in}{1.861665in}}%
\pgfpathlineto{\pgfqpoint{5.962492in}{2.563007in}}%
\pgfpathlineto{\pgfqpoint{5.953555in}{2.563007in}}%
\pgfpathlineto{\pgfqpoint{5.953555in}{1.861665in}}%
\pgfpathclose%
\pgfusepath{fill}%
\end{pgfscope}%
\begin{pgfscope}%
\pgfpathrectangle{\pgfqpoint{3.722897in}{0.857143in}}{\pgfqpoint{2.627103in}{1.813434in}}%
\pgfusepath{clip}%
\pgfsetbuttcap%
\pgfsetmiterjoin%
\definecolor{currentfill}{rgb}{0.992771,0.707689,0.712380}%
\pgfsetfillcolor{currentfill}%
\pgfsetlinewidth{0.000000pt}%
\definecolor{currentstroke}{rgb}{0.000000,0.000000,0.000000}%
\pgfsetstrokecolor{currentstroke}%
\pgfsetstrokeopacity{0.000000}%
\pgfsetdash{}{0pt}%
\pgfpathmoveto{\pgfqpoint{5.964726in}{1.858249in}}%
\pgfpathlineto{\pgfqpoint{5.973662in}{1.858249in}}%
\pgfpathlineto{\pgfqpoint{5.973662in}{2.565577in}}%
\pgfpathlineto{\pgfqpoint{5.964726in}{2.565577in}}%
\pgfpathlineto{\pgfqpoint{5.964726in}{1.858249in}}%
\pgfpathclose%
\pgfusepath{fill}%
\end{pgfscope}%
\begin{pgfscope}%
\pgfpathrectangle{\pgfqpoint{3.722897in}{0.857143in}}{\pgfqpoint{2.627103in}{1.813434in}}%
\pgfusepath{clip}%
\pgfsetbuttcap%
\pgfsetmiterjoin%
\definecolor{currentfill}{rgb}{0.992771,0.707689,0.712380}%
\pgfsetfillcolor{currentfill}%
\pgfsetlinewidth{0.000000pt}%
\definecolor{currentstroke}{rgb}{0.000000,0.000000,0.000000}%
\pgfsetstrokecolor{currentstroke}%
\pgfsetstrokeopacity{0.000000}%
\pgfsetdash{}{0pt}%
\pgfpathmoveto{\pgfqpoint{5.975896in}{1.856690in}}%
\pgfpathlineto{\pgfqpoint{5.984833in}{1.856690in}}%
\pgfpathlineto{\pgfqpoint{5.984833in}{2.567547in}}%
\pgfpathlineto{\pgfqpoint{5.975896in}{2.567547in}}%
\pgfpathlineto{\pgfqpoint{5.975896in}{1.856690in}}%
\pgfpathclose%
\pgfusepath{fill}%
\end{pgfscope}%
\begin{pgfscope}%
\pgfpathrectangle{\pgfqpoint{3.722897in}{0.857143in}}{\pgfqpoint{2.627103in}{1.813434in}}%
\pgfusepath{clip}%
\pgfsetbuttcap%
\pgfsetmiterjoin%
\definecolor{currentfill}{rgb}{0.992771,0.707689,0.712380}%
\pgfsetfillcolor{currentfill}%
\pgfsetlinewidth{0.000000pt}%
\definecolor{currentstroke}{rgb}{0.000000,0.000000,0.000000}%
\pgfsetstrokecolor{currentstroke}%
\pgfsetstrokeopacity{0.000000}%
\pgfsetdash{}{0pt}%
\pgfpathmoveto{\pgfqpoint{5.987067in}{1.855321in}}%
\pgfpathlineto{\pgfqpoint{5.996004in}{1.855321in}}%
\pgfpathlineto{\pgfqpoint{5.996004in}{2.553268in}}%
\pgfpathlineto{\pgfqpoint{5.987067in}{2.553268in}}%
\pgfpathlineto{\pgfqpoint{5.987067in}{1.855321in}}%
\pgfpathclose%
\pgfusepath{fill}%
\end{pgfscope}%
\begin{pgfscope}%
\pgfpathrectangle{\pgfqpoint{3.722897in}{0.857143in}}{\pgfqpoint{2.627103in}{1.813434in}}%
\pgfusepath{clip}%
\pgfsetbuttcap%
\pgfsetmiterjoin%
\definecolor{currentfill}{rgb}{0.992771,0.707689,0.712380}%
\pgfsetfillcolor{currentfill}%
\pgfsetlinewidth{0.000000pt}%
\definecolor{currentstroke}{rgb}{0.000000,0.000000,0.000000}%
\pgfsetstrokecolor{currentstroke}%
\pgfsetstrokeopacity{0.000000}%
\pgfsetdash{}{0pt}%
\pgfpathmoveto{\pgfqpoint{5.998238in}{1.855039in}}%
\pgfpathlineto{\pgfqpoint{6.007174in}{1.855039in}}%
\pgfpathlineto{\pgfqpoint{6.007174in}{2.539485in}}%
\pgfpathlineto{\pgfqpoint{5.998238in}{2.539485in}}%
\pgfpathlineto{\pgfqpoint{5.998238in}{1.855039in}}%
\pgfpathclose%
\pgfusepath{fill}%
\end{pgfscope}%
\begin{pgfscope}%
\pgfpathrectangle{\pgfqpoint{3.722897in}{0.857143in}}{\pgfqpoint{2.627103in}{1.813434in}}%
\pgfusepath{clip}%
\pgfsetbuttcap%
\pgfsetmiterjoin%
\definecolor{currentfill}{rgb}{0.992771,0.707689,0.712380}%
\pgfsetfillcolor{currentfill}%
\pgfsetlinewidth{0.000000pt}%
\definecolor{currentstroke}{rgb}{0.000000,0.000000,0.000000}%
\pgfsetstrokecolor{currentstroke}%
\pgfsetstrokeopacity{0.000000}%
\pgfsetdash{}{0pt}%
\pgfpathmoveto{\pgfqpoint{6.009408in}{1.861627in}}%
\pgfpathlineto{\pgfqpoint{6.018345in}{1.861627in}}%
\pgfpathlineto{\pgfqpoint{6.018345in}{2.526992in}}%
\pgfpathlineto{\pgfqpoint{6.009408in}{2.526992in}}%
\pgfpathlineto{\pgfqpoint{6.009408in}{1.861627in}}%
\pgfpathclose%
\pgfusepath{fill}%
\end{pgfscope}%
\begin{pgfscope}%
\pgfpathrectangle{\pgfqpoint{3.722897in}{0.857143in}}{\pgfqpoint{2.627103in}{1.813434in}}%
\pgfusepath{clip}%
\pgfsetbuttcap%
\pgfsetmiterjoin%
\definecolor{currentfill}{rgb}{0.992771,0.707689,0.712380}%
\pgfsetfillcolor{currentfill}%
\pgfsetlinewidth{0.000000pt}%
\definecolor{currentstroke}{rgb}{0.000000,0.000000,0.000000}%
\pgfsetstrokecolor{currentstroke}%
\pgfsetstrokeopacity{0.000000}%
\pgfsetdash{}{0pt}%
\pgfpathmoveto{\pgfqpoint{6.020579in}{1.848339in}}%
\pgfpathlineto{\pgfqpoint{6.029515in}{1.848339in}}%
\pgfpathlineto{\pgfqpoint{6.029515in}{2.479895in}}%
\pgfpathlineto{\pgfqpoint{6.020579in}{2.479895in}}%
\pgfpathlineto{\pgfqpoint{6.020579in}{1.848339in}}%
\pgfpathclose%
\pgfusepath{fill}%
\end{pgfscope}%
\begin{pgfscope}%
\pgfpathrectangle{\pgfqpoint{3.722897in}{0.857143in}}{\pgfqpoint{2.627103in}{1.813434in}}%
\pgfusepath{clip}%
\pgfsetbuttcap%
\pgfsetmiterjoin%
\definecolor{currentfill}{rgb}{0.992771,0.707689,0.712380}%
\pgfsetfillcolor{currentfill}%
\pgfsetlinewidth{0.000000pt}%
\definecolor{currentstroke}{rgb}{0.000000,0.000000,0.000000}%
\pgfsetstrokecolor{currentstroke}%
\pgfsetstrokeopacity{0.000000}%
\pgfsetdash{}{0pt}%
\pgfpathmoveto{\pgfqpoint{6.031749in}{1.849577in}}%
\pgfpathlineto{\pgfqpoint{6.040686in}{1.849577in}}%
\pgfpathlineto{\pgfqpoint{6.040686in}{2.456694in}}%
\pgfpathlineto{\pgfqpoint{6.031749in}{2.456694in}}%
\pgfpathlineto{\pgfqpoint{6.031749in}{1.849577in}}%
\pgfpathclose%
\pgfusepath{fill}%
\end{pgfscope}%
\begin{pgfscope}%
\pgfpathrectangle{\pgfqpoint{3.722897in}{0.857143in}}{\pgfqpoint{2.627103in}{1.813434in}}%
\pgfusepath{clip}%
\pgfsetbuttcap%
\pgfsetmiterjoin%
\definecolor{currentfill}{rgb}{0.992771,0.707689,0.712380}%
\pgfsetfillcolor{currentfill}%
\pgfsetlinewidth{0.000000pt}%
\definecolor{currentstroke}{rgb}{0.000000,0.000000,0.000000}%
\pgfsetstrokecolor{currentstroke}%
\pgfsetstrokeopacity{0.000000}%
\pgfsetdash{}{0pt}%
\pgfpathmoveto{\pgfqpoint{6.042920in}{1.881262in}}%
\pgfpathlineto{\pgfqpoint{6.051857in}{1.881262in}}%
\pgfpathlineto{\pgfqpoint{6.051857in}{2.474270in}}%
\pgfpathlineto{\pgfqpoint{6.042920in}{2.474270in}}%
\pgfpathlineto{\pgfqpoint{6.042920in}{1.881262in}}%
\pgfpathclose%
\pgfusepath{fill}%
\end{pgfscope}%
\begin{pgfscope}%
\pgfpathrectangle{\pgfqpoint{3.722897in}{0.857143in}}{\pgfqpoint{2.627103in}{1.813434in}}%
\pgfusepath{clip}%
\pgfsetbuttcap%
\pgfsetmiterjoin%
\definecolor{currentfill}{rgb}{0.992771,0.707689,0.712380}%
\pgfsetfillcolor{currentfill}%
\pgfsetlinewidth{0.000000pt}%
\definecolor{currentstroke}{rgb}{0.000000,0.000000,0.000000}%
\pgfsetstrokecolor{currentstroke}%
\pgfsetstrokeopacity{0.000000}%
\pgfsetdash{}{0pt}%
\pgfpathmoveto{\pgfqpoint{6.054091in}{1.906859in}}%
\pgfpathlineto{\pgfqpoint{6.063027in}{1.906859in}}%
\pgfpathlineto{\pgfqpoint{6.063027in}{2.500711in}}%
\pgfpathlineto{\pgfqpoint{6.054091in}{2.500711in}}%
\pgfpathlineto{\pgfqpoint{6.054091in}{1.906859in}}%
\pgfpathclose%
\pgfusepath{fill}%
\end{pgfscope}%
\begin{pgfscope}%
\pgfpathrectangle{\pgfqpoint{3.722897in}{0.857143in}}{\pgfqpoint{2.627103in}{1.813434in}}%
\pgfusepath{clip}%
\pgfsetbuttcap%
\pgfsetmiterjoin%
\definecolor{currentfill}{rgb}{0.992771,0.707689,0.712380}%
\pgfsetfillcolor{currentfill}%
\pgfsetlinewidth{0.000000pt}%
\definecolor{currentstroke}{rgb}{0.000000,0.000000,0.000000}%
\pgfsetstrokecolor{currentstroke}%
\pgfsetstrokeopacity{0.000000}%
\pgfsetdash{}{0pt}%
\pgfpathmoveto{\pgfqpoint{6.065261in}{1.908492in}}%
\pgfpathlineto{\pgfqpoint{6.074198in}{1.908492in}}%
\pgfpathlineto{\pgfqpoint{6.074198in}{2.492710in}}%
\pgfpathlineto{\pgfqpoint{6.065261in}{2.492710in}}%
\pgfpathlineto{\pgfqpoint{6.065261in}{1.908492in}}%
\pgfpathclose%
\pgfusepath{fill}%
\end{pgfscope}%
\begin{pgfscope}%
\pgfpathrectangle{\pgfqpoint{3.722897in}{0.857143in}}{\pgfqpoint{2.627103in}{1.813434in}}%
\pgfusepath{clip}%
\pgfsetbuttcap%
\pgfsetmiterjoin%
\definecolor{currentfill}{rgb}{0.992771,0.707689,0.712380}%
\pgfsetfillcolor{currentfill}%
\pgfsetlinewidth{0.000000pt}%
\definecolor{currentstroke}{rgb}{0.000000,0.000000,0.000000}%
\pgfsetstrokecolor{currentstroke}%
\pgfsetstrokeopacity{0.000000}%
\pgfsetdash{}{0pt}%
\pgfpathmoveto{\pgfqpoint{6.076432in}{1.926712in}}%
\pgfpathlineto{\pgfqpoint{6.085368in}{1.926712in}}%
\pgfpathlineto{\pgfqpoint{6.085368in}{2.504351in}}%
\pgfpathlineto{\pgfqpoint{6.076432in}{2.504351in}}%
\pgfpathlineto{\pgfqpoint{6.076432in}{1.926712in}}%
\pgfpathclose%
\pgfusepath{fill}%
\end{pgfscope}%
\begin{pgfscope}%
\pgfpathrectangle{\pgfqpoint{3.722897in}{0.857143in}}{\pgfqpoint{2.627103in}{1.813434in}}%
\pgfusepath{clip}%
\pgfsetbuttcap%
\pgfsetmiterjoin%
\definecolor{currentfill}{rgb}{0.992771,0.707689,0.712380}%
\pgfsetfillcolor{currentfill}%
\pgfsetlinewidth{0.000000pt}%
\definecolor{currentstroke}{rgb}{0.000000,0.000000,0.000000}%
\pgfsetstrokecolor{currentstroke}%
\pgfsetstrokeopacity{0.000000}%
\pgfsetdash{}{0pt}%
\pgfpathmoveto{\pgfqpoint{6.087602in}{1.941574in}}%
\pgfpathlineto{\pgfqpoint{6.096539in}{1.941574in}}%
\pgfpathlineto{\pgfqpoint{6.096539in}{2.512254in}}%
\pgfpathlineto{\pgfqpoint{6.087602in}{2.512254in}}%
\pgfpathlineto{\pgfqpoint{6.087602in}{1.941574in}}%
\pgfpathclose%
\pgfusepath{fill}%
\end{pgfscope}%
\begin{pgfscope}%
\pgfpathrectangle{\pgfqpoint{3.722897in}{0.857143in}}{\pgfqpoint{2.627103in}{1.813434in}}%
\pgfusepath{clip}%
\pgfsetbuttcap%
\pgfsetmiterjoin%
\definecolor{currentfill}{rgb}{0.992771,0.707689,0.712380}%
\pgfsetfillcolor{currentfill}%
\pgfsetlinewidth{0.000000pt}%
\definecolor{currentstroke}{rgb}{0.000000,0.000000,0.000000}%
\pgfsetstrokecolor{currentstroke}%
\pgfsetstrokeopacity{0.000000}%
\pgfsetdash{}{0pt}%
\pgfpathmoveto{\pgfqpoint{6.098773in}{1.956161in}}%
\pgfpathlineto{\pgfqpoint{6.107710in}{1.956161in}}%
\pgfpathlineto{\pgfqpoint{6.107710in}{2.522239in}}%
\pgfpathlineto{\pgfqpoint{6.098773in}{2.522239in}}%
\pgfpathlineto{\pgfqpoint{6.098773in}{1.956161in}}%
\pgfpathclose%
\pgfusepath{fill}%
\end{pgfscope}%
\begin{pgfscope}%
\pgfpathrectangle{\pgfqpoint{3.722897in}{0.857143in}}{\pgfqpoint{2.627103in}{1.813434in}}%
\pgfusepath{clip}%
\pgfsetbuttcap%
\pgfsetmiterjoin%
\definecolor{currentfill}{rgb}{0.992771,0.707689,0.712380}%
\pgfsetfillcolor{currentfill}%
\pgfsetlinewidth{0.000000pt}%
\definecolor{currentstroke}{rgb}{0.000000,0.000000,0.000000}%
\pgfsetstrokecolor{currentstroke}%
\pgfsetstrokeopacity{0.000000}%
\pgfsetdash{}{0pt}%
\pgfpathmoveto{\pgfqpoint{6.109944in}{1.966188in}}%
\pgfpathlineto{\pgfqpoint{6.118880in}{1.966188in}}%
\pgfpathlineto{\pgfqpoint{6.118880in}{2.514579in}}%
\pgfpathlineto{\pgfqpoint{6.109944in}{2.514579in}}%
\pgfpathlineto{\pgfqpoint{6.109944in}{1.966188in}}%
\pgfpathclose%
\pgfusepath{fill}%
\end{pgfscope}%
\begin{pgfscope}%
\pgfpathrectangle{\pgfqpoint{3.722897in}{0.857143in}}{\pgfqpoint{2.627103in}{1.813434in}}%
\pgfusepath{clip}%
\pgfsetbuttcap%
\pgfsetmiterjoin%
\definecolor{currentfill}{rgb}{0.992771,0.707689,0.712380}%
\pgfsetfillcolor{currentfill}%
\pgfsetlinewidth{0.000000pt}%
\definecolor{currentstroke}{rgb}{0.000000,0.000000,0.000000}%
\pgfsetstrokecolor{currentstroke}%
\pgfsetstrokeopacity{0.000000}%
\pgfsetdash{}{0pt}%
\pgfpathmoveto{\pgfqpoint{6.121114in}{1.981146in}}%
\pgfpathlineto{\pgfqpoint{6.130051in}{1.981146in}}%
\pgfpathlineto{\pgfqpoint{6.130051in}{2.513809in}}%
\pgfpathlineto{\pgfqpoint{6.121114in}{2.513809in}}%
\pgfpathlineto{\pgfqpoint{6.121114in}{1.981146in}}%
\pgfpathclose%
\pgfusepath{fill}%
\end{pgfscope}%
\begin{pgfscope}%
\pgfpathrectangle{\pgfqpoint{3.722897in}{0.857143in}}{\pgfqpoint{2.627103in}{1.813434in}}%
\pgfusepath{clip}%
\pgfsetbuttcap%
\pgfsetmiterjoin%
\definecolor{currentfill}{rgb}{0.992771,0.707689,0.712380}%
\pgfsetfillcolor{currentfill}%
\pgfsetlinewidth{0.000000pt}%
\definecolor{currentstroke}{rgb}{0.000000,0.000000,0.000000}%
\pgfsetstrokecolor{currentstroke}%
\pgfsetstrokeopacity{0.000000}%
\pgfsetdash{}{0pt}%
\pgfpathmoveto{\pgfqpoint{6.132285in}{2.003142in}}%
\pgfpathlineto{\pgfqpoint{6.141221in}{2.003142in}}%
\pgfpathlineto{\pgfqpoint{6.141221in}{2.525020in}}%
\pgfpathlineto{\pgfqpoint{6.132285in}{2.525020in}}%
\pgfpathlineto{\pgfqpoint{6.132285in}{2.003142in}}%
\pgfpathclose%
\pgfusepath{fill}%
\end{pgfscope}%
\begin{pgfscope}%
\pgfpathrectangle{\pgfqpoint{3.722897in}{0.857143in}}{\pgfqpoint{2.627103in}{1.813434in}}%
\pgfusepath{clip}%
\pgfsetbuttcap%
\pgfsetmiterjoin%
\definecolor{currentfill}{rgb}{0.992771,0.707689,0.712380}%
\pgfsetfillcolor{currentfill}%
\pgfsetlinewidth{0.000000pt}%
\definecolor{currentstroke}{rgb}{0.000000,0.000000,0.000000}%
\pgfsetstrokecolor{currentstroke}%
\pgfsetstrokeopacity{0.000000}%
\pgfsetdash{}{0pt}%
\pgfpathmoveto{\pgfqpoint{6.143456in}{2.025335in}}%
\pgfpathlineto{\pgfqpoint{6.152392in}{2.025335in}}%
\pgfpathlineto{\pgfqpoint{6.152392in}{2.534816in}}%
\pgfpathlineto{\pgfqpoint{6.143456in}{2.534816in}}%
\pgfpathlineto{\pgfqpoint{6.143456in}{2.025335in}}%
\pgfpathclose%
\pgfusepath{fill}%
\end{pgfscope}%
\begin{pgfscope}%
\pgfpathrectangle{\pgfqpoint{3.722897in}{0.857143in}}{\pgfqpoint{2.627103in}{1.813434in}}%
\pgfusepath{clip}%
\pgfsetbuttcap%
\pgfsetmiterjoin%
\definecolor{currentfill}{rgb}{0.992771,0.707689,0.712380}%
\pgfsetfillcolor{currentfill}%
\pgfsetlinewidth{0.000000pt}%
\definecolor{currentstroke}{rgb}{0.000000,0.000000,0.000000}%
\pgfsetstrokecolor{currentstroke}%
\pgfsetstrokeopacity{0.000000}%
\pgfsetdash{}{0pt}%
\pgfpathmoveto{\pgfqpoint{6.154626in}{2.044678in}}%
\pgfpathlineto{\pgfqpoint{6.163563in}{2.044678in}}%
\pgfpathlineto{\pgfqpoint{6.163563in}{2.546697in}}%
\pgfpathlineto{\pgfqpoint{6.154626in}{2.546697in}}%
\pgfpathlineto{\pgfqpoint{6.154626in}{2.044678in}}%
\pgfpathclose%
\pgfusepath{fill}%
\end{pgfscope}%
\begin{pgfscope}%
\pgfpathrectangle{\pgfqpoint{3.722897in}{0.857143in}}{\pgfqpoint{2.627103in}{1.813434in}}%
\pgfusepath{clip}%
\pgfsetbuttcap%
\pgfsetmiterjoin%
\definecolor{currentfill}{rgb}{0.992771,0.707689,0.712380}%
\pgfsetfillcolor{currentfill}%
\pgfsetlinewidth{0.000000pt}%
\definecolor{currentstroke}{rgb}{0.000000,0.000000,0.000000}%
\pgfsetstrokecolor{currentstroke}%
\pgfsetstrokeopacity{0.000000}%
\pgfsetdash{}{0pt}%
\pgfpathmoveto{\pgfqpoint{6.165797in}{2.066777in}}%
\pgfpathlineto{\pgfqpoint{6.174733in}{2.066777in}}%
\pgfpathlineto{\pgfqpoint{6.174733in}{2.544183in}}%
\pgfpathlineto{\pgfqpoint{6.165797in}{2.544183in}}%
\pgfpathlineto{\pgfqpoint{6.165797in}{2.066777in}}%
\pgfpathclose%
\pgfusepath{fill}%
\end{pgfscope}%
\begin{pgfscope}%
\pgfpathrectangle{\pgfqpoint{3.722897in}{0.857143in}}{\pgfqpoint{2.627103in}{1.813434in}}%
\pgfusepath{clip}%
\pgfsetbuttcap%
\pgfsetmiterjoin%
\definecolor{currentfill}{rgb}{0.992771,0.707689,0.712380}%
\pgfsetfillcolor{currentfill}%
\pgfsetlinewidth{0.000000pt}%
\definecolor{currentstroke}{rgb}{0.000000,0.000000,0.000000}%
\pgfsetstrokecolor{currentstroke}%
\pgfsetstrokeopacity{0.000000}%
\pgfsetdash{}{0pt}%
\pgfpathmoveto{\pgfqpoint{6.176967in}{2.088814in}}%
\pgfpathlineto{\pgfqpoint{6.185904in}{2.088814in}}%
\pgfpathlineto{\pgfqpoint{6.185904in}{2.542668in}}%
\pgfpathlineto{\pgfqpoint{6.176967in}{2.542668in}}%
\pgfpathlineto{\pgfqpoint{6.176967in}{2.088814in}}%
\pgfpathclose%
\pgfusepath{fill}%
\end{pgfscope}%
\begin{pgfscope}%
\pgfpathrectangle{\pgfqpoint{3.722897in}{0.857143in}}{\pgfqpoint{2.627103in}{1.813434in}}%
\pgfusepath{clip}%
\pgfsetbuttcap%
\pgfsetmiterjoin%
\definecolor{currentfill}{rgb}{0.992771,0.707689,0.712380}%
\pgfsetfillcolor{currentfill}%
\pgfsetlinewidth{0.000000pt}%
\definecolor{currentstroke}{rgb}{0.000000,0.000000,0.000000}%
\pgfsetstrokecolor{currentstroke}%
\pgfsetstrokeopacity{0.000000}%
\pgfsetdash{}{0pt}%
\pgfpathmoveto{\pgfqpoint{6.188138in}{2.115964in}}%
\pgfpathlineto{\pgfqpoint{6.197074in}{2.115964in}}%
\pgfpathlineto{\pgfqpoint{6.197074in}{2.540151in}}%
\pgfpathlineto{\pgfqpoint{6.188138in}{2.540151in}}%
\pgfpathlineto{\pgfqpoint{6.188138in}{2.115964in}}%
\pgfpathclose%
\pgfusepath{fill}%
\end{pgfscope}%
\begin{pgfscope}%
\pgfpathrectangle{\pgfqpoint{3.722897in}{0.857143in}}{\pgfqpoint{2.627103in}{1.813434in}}%
\pgfusepath{clip}%
\pgfsetbuttcap%
\pgfsetmiterjoin%
\definecolor{currentfill}{rgb}{0.992771,0.707689,0.712380}%
\pgfsetfillcolor{currentfill}%
\pgfsetlinewidth{0.000000pt}%
\definecolor{currentstroke}{rgb}{0.000000,0.000000,0.000000}%
\pgfsetstrokecolor{currentstroke}%
\pgfsetstrokeopacity{0.000000}%
\pgfsetdash{}{0pt}%
\pgfpathmoveto{\pgfqpoint{6.199309in}{2.149883in}}%
\pgfpathlineto{\pgfqpoint{6.208245in}{2.149883in}}%
\pgfpathlineto{\pgfqpoint{6.208245in}{2.523345in}}%
\pgfpathlineto{\pgfqpoint{6.199309in}{2.523345in}}%
\pgfpathlineto{\pgfqpoint{6.199309in}{2.149883in}}%
\pgfpathclose%
\pgfusepath{fill}%
\end{pgfscope}%
\begin{pgfscope}%
\pgfpathrectangle{\pgfqpoint{3.722897in}{0.857143in}}{\pgfqpoint{2.627103in}{1.813434in}}%
\pgfusepath{clip}%
\pgfsetbuttcap%
\pgfsetmiterjoin%
\definecolor{currentfill}{rgb}{0.992771,0.707689,0.712380}%
\pgfsetfillcolor{currentfill}%
\pgfsetlinewidth{0.000000pt}%
\definecolor{currentstroke}{rgb}{0.000000,0.000000,0.000000}%
\pgfsetstrokecolor{currentstroke}%
\pgfsetstrokeopacity{0.000000}%
\pgfsetdash{}{0pt}%
\pgfpathmoveto{\pgfqpoint{6.210479in}{2.182768in}}%
\pgfpathlineto{\pgfqpoint{6.219416in}{2.182768in}}%
\pgfpathlineto{\pgfqpoint{6.219416in}{2.513922in}}%
\pgfpathlineto{\pgfqpoint{6.210479in}{2.513922in}}%
\pgfpathlineto{\pgfqpoint{6.210479in}{2.182768in}}%
\pgfpathclose%
\pgfusepath{fill}%
\end{pgfscope}%
\begin{pgfscope}%
\pgfpathrectangle{\pgfqpoint{3.722897in}{0.857143in}}{\pgfqpoint{2.627103in}{1.813434in}}%
\pgfusepath{clip}%
\pgfsetbuttcap%
\pgfsetmiterjoin%
\definecolor{currentfill}{rgb}{0.992771,0.707689,0.712380}%
\pgfsetfillcolor{currentfill}%
\pgfsetlinewidth{0.000000pt}%
\definecolor{currentstroke}{rgb}{0.000000,0.000000,0.000000}%
\pgfsetstrokecolor{currentstroke}%
\pgfsetstrokeopacity{0.000000}%
\pgfsetdash{}{0pt}%
\pgfpathmoveto{\pgfqpoint{6.221650in}{2.216388in}}%
\pgfpathlineto{\pgfqpoint{6.230586in}{2.216388in}}%
\pgfpathlineto{\pgfqpoint{6.230586in}{2.516029in}}%
\pgfpathlineto{\pgfqpoint{6.221650in}{2.516029in}}%
\pgfpathlineto{\pgfqpoint{6.221650in}{2.216388in}}%
\pgfpathclose%
\pgfusepath{fill}%
\end{pgfscope}%
\begin{pgfscope}%
\pgfsetbuttcap%
\pgfsetroundjoin%
\definecolor{currentfill}{rgb}{0.000000,0.000000,0.000000}%
\pgfsetfillcolor{currentfill}%
\pgfsetlinewidth{0.803000pt}%
\definecolor{currentstroke}{rgb}{0.000000,0.000000,0.000000}%
\pgfsetstrokecolor{currentstroke}%
\pgfsetdash{}{0pt}%
\pgfsys@defobject{currentmarker}{\pgfqpoint{0.000000in}{-0.048611in}}{\pgfqpoint{0.000000in}{0.000000in}}{%
\pgfpathmoveto{\pgfqpoint{0.000000in}{0.000000in}}%
\pgfpathlineto{\pgfqpoint{0.000000in}{-0.048611in}}%
\pgfusepath{stroke,fill}%
}%
\begin{pgfscope}%
\pgfsys@transformshift{4.382968in}{0.857143in}%
\pgfsys@useobject{currentmarker}{}%
\end{pgfscope}%
\end{pgfscope}%
\begin{pgfscope}%
\definecolor{textcolor}{rgb}{0.000000,0.000000,0.000000}%
\pgfsetstrokecolor{textcolor}%
\pgfsetfillcolor{textcolor}%
\pgftext[x=4.382968in,y=0.759921in,,top]{\color{textcolor}{\rmfamily\fontsize{10.000000}{12.000000}\selectfont\catcode`\^=\active\def^{\ifmmode\sp\else\^{}\fi}\catcode`\%=\active\def%{\%}1978}}%
\end{pgfscope}%
\begin{pgfscope}%
\pgfsetbuttcap%
\pgfsetroundjoin%
\definecolor{currentfill}{rgb}{0.000000,0.000000,0.000000}%
\pgfsetfillcolor{currentfill}%
\pgfsetlinewidth{0.803000pt}%
\definecolor{currentstroke}{rgb}{0.000000,0.000000,0.000000}%
\pgfsetstrokecolor{currentstroke}%
\pgfsetdash{}{0pt}%
\pgfsys@defobject{currentmarker}{\pgfqpoint{0.000000in}{-0.048611in}}{\pgfqpoint{0.000000in}{0.000000in}}{%
\pgfpathmoveto{\pgfqpoint{0.000000in}{0.000000in}}%
\pgfpathlineto{\pgfqpoint{0.000000in}{-0.048611in}}%
\pgfusepath{stroke,fill}%
}%
\begin{pgfscope}%
\pgfsys@transformshift{4.941498in}{0.857143in}%
\pgfsys@useobject{currentmarker}{}%
\end{pgfscope}%
\end{pgfscope}%
\begin{pgfscope}%
\definecolor{textcolor}{rgb}{0.000000,0.000000,0.000000}%
\pgfsetstrokecolor{textcolor}%
\pgfsetfillcolor{textcolor}%
\pgftext[x=4.941498in,y=0.759921in,,top]{\color{textcolor}{\rmfamily\fontsize{10.000000}{12.000000}\selectfont\catcode`\^=\active\def^{\ifmmode\sp\else\^{}\fi}\catcode`\%=\active\def%{\%}1991}}%
\end{pgfscope}%
\begin{pgfscope}%
\pgfsetbuttcap%
\pgfsetroundjoin%
\definecolor{currentfill}{rgb}{0.000000,0.000000,0.000000}%
\pgfsetfillcolor{currentfill}%
\pgfsetlinewidth{0.803000pt}%
\definecolor{currentstroke}{rgb}{0.000000,0.000000,0.000000}%
\pgfsetstrokecolor{currentstroke}%
\pgfsetdash{}{0pt}%
\pgfsys@defobject{currentmarker}{\pgfqpoint{0.000000in}{-0.048611in}}{\pgfqpoint{0.000000in}{0.000000in}}{%
\pgfpathmoveto{\pgfqpoint{0.000000in}{0.000000in}}%
\pgfpathlineto{\pgfqpoint{0.000000in}{-0.048611in}}%
\pgfusepath{stroke,fill}%
}%
\begin{pgfscope}%
\pgfsys@transformshift{5.500029in}{0.857143in}%
\pgfsys@useobject{currentmarker}{}%
\end{pgfscope}%
\end{pgfscope}%
\begin{pgfscope}%
\definecolor{textcolor}{rgb}{0.000000,0.000000,0.000000}%
\pgfsetstrokecolor{textcolor}%
\pgfsetfillcolor{textcolor}%
\pgftext[x=5.500029in,y=0.759921in,,top]{\color{textcolor}{\rmfamily\fontsize{10.000000}{12.000000}\selectfont\catcode`\^=\active\def^{\ifmmode\sp\else\^{}\fi}\catcode`\%=\active\def%{\%}2003}}%
\end{pgfscope}%
\begin{pgfscope}%
\pgfsetbuttcap%
\pgfsetroundjoin%
\definecolor{currentfill}{rgb}{0.000000,0.000000,0.000000}%
\pgfsetfillcolor{currentfill}%
\pgfsetlinewidth{0.803000pt}%
\definecolor{currentstroke}{rgb}{0.000000,0.000000,0.000000}%
\pgfsetstrokecolor{currentstroke}%
\pgfsetdash{}{0pt}%
\pgfsys@defobject{currentmarker}{\pgfqpoint{0.000000in}{-0.048611in}}{\pgfqpoint{0.000000in}{0.000000in}}{%
\pgfpathmoveto{\pgfqpoint{0.000000in}{0.000000in}}%
\pgfpathlineto{\pgfqpoint{0.000000in}{-0.048611in}}%
\pgfusepath{stroke,fill}%
}%
\begin{pgfscope}%
\pgfsys@transformshift{6.058559in}{0.857143in}%
\pgfsys@useobject{currentmarker}{}%
\end{pgfscope}%
\end{pgfscope}%
\begin{pgfscope}%
\definecolor{textcolor}{rgb}{0.000000,0.000000,0.000000}%
\pgfsetstrokecolor{textcolor}%
\pgfsetfillcolor{textcolor}%
\pgftext[x=6.058559in,y=0.759921in,,top]{\color{textcolor}{\rmfamily\fontsize{10.000000}{12.000000}\selectfont\catcode`\^=\active\def^{\ifmmode\sp\else\^{}\fi}\catcode`\%=\active\def%{\%}2016}}%
\end{pgfscope}%
\begin{pgfscope}%
\pgfsetbuttcap%
\pgfsetroundjoin%
\definecolor{currentfill}{rgb}{0.000000,0.000000,0.000000}%
\pgfsetfillcolor{currentfill}%
\pgfsetlinewidth{0.803000pt}%
\definecolor{currentstroke}{rgb}{0.000000,0.000000,0.000000}%
\pgfsetstrokecolor{currentstroke}%
\pgfsetdash{}{0pt}%
\pgfsys@defobject{currentmarker}{\pgfqpoint{-0.048611in}{0.000000in}}{\pgfqpoint{-0.000000in}{0.000000in}}{%
\pgfpathmoveto{\pgfqpoint{-0.000000in}{0.000000in}}%
\pgfpathlineto{\pgfqpoint{-0.048611in}{0.000000in}}%
\pgfusepath{stroke,fill}%
}%
\begin{pgfscope}%
\pgfsys@transformshift{3.722897in}{1.242542in}%
\pgfsys@useobject{currentmarker}{}%
\end{pgfscope}%
\end{pgfscope}%
\begin{pgfscope}%
\definecolor{textcolor}{rgb}{0.000000,0.000000,0.000000}%
\pgfsetstrokecolor{textcolor}%
\pgfsetfillcolor{textcolor}%
\pgftext[x=3.378760in, y=1.189780in, left, base]{\color{textcolor}{\rmfamily\fontsize{10.000000}{12.000000}\selectfont\catcode`\^=\active\def^{\ifmmode\sp\else\^{}\fi}\catcode`\%=\active\def%{\%}$\mathdefault{\ensuremath{-}20}$}}%
\end{pgfscope}%
\begin{pgfscope}%
\pgfsetbuttcap%
\pgfsetroundjoin%
\definecolor{currentfill}{rgb}{0.000000,0.000000,0.000000}%
\pgfsetfillcolor{currentfill}%
\pgfsetlinewidth{0.803000pt}%
\definecolor{currentstroke}{rgb}{0.000000,0.000000,0.000000}%
\pgfsetstrokecolor{currentstroke}%
\pgfsetdash{}{0pt}%
\pgfsys@defobject{currentmarker}{\pgfqpoint{-0.048611in}{0.000000in}}{\pgfqpoint{-0.000000in}{0.000000in}}{%
\pgfpathmoveto{\pgfqpoint{-0.000000in}{0.000000in}}%
\pgfpathlineto{\pgfqpoint{-0.048611in}{0.000000in}}%
\pgfusepath{stroke,fill}%
}%
\begin{pgfscope}%
\pgfsys@transformshift{3.722897in}{1.813884in}%
\pgfsys@useobject{currentmarker}{}%
\end{pgfscope}%
\end{pgfscope}%
\begin{pgfscope}%
\definecolor{textcolor}{rgb}{0.000000,0.000000,0.000000}%
\pgfsetstrokecolor{textcolor}%
\pgfsetfillcolor{textcolor}%
\pgftext[x=3.556230in, y=1.761123in, left, base]{\color{textcolor}{\rmfamily\fontsize{10.000000}{12.000000}\selectfont\catcode`\^=\active\def^{\ifmmode\sp\else\^{}\fi}\catcode`\%=\active\def%{\%}$\mathdefault{0}$}}%
\end{pgfscope}%
\begin{pgfscope}%
\pgfsetbuttcap%
\pgfsetroundjoin%
\definecolor{currentfill}{rgb}{0.000000,0.000000,0.000000}%
\pgfsetfillcolor{currentfill}%
\pgfsetlinewidth{0.803000pt}%
\definecolor{currentstroke}{rgb}{0.000000,0.000000,0.000000}%
\pgfsetstrokecolor{currentstroke}%
\pgfsetdash{}{0pt}%
\pgfsys@defobject{currentmarker}{\pgfqpoint{-0.048611in}{0.000000in}}{\pgfqpoint{-0.000000in}{0.000000in}}{%
\pgfpathmoveto{\pgfqpoint{-0.000000in}{0.000000in}}%
\pgfpathlineto{\pgfqpoint{-0.048611in}{0.000000in}}%
\pgfusepath{stroke,fill}%
}%
\begin{pgfscope}%
\pgfsys@transformshift{3.722897in}{2.385227in}%
\pgfsys@useobject{currentmarker}{}%
\end{pgfscope}%
\end{pgfscope}%
\begin{pgfscope}%
\definecolor{textcolor}{rgb}{0.000000,0.000000,0.000000}%
\pgfsetstrokecolor{textcolor}%
\pgfsetfillcolor{textcolor}%
\pgftext[x=3.486785in, y=2.332465in, left, base]{\color{textcolor}{\rmfamily\fontsize{10.000000}{12.000000}\selectfont\catcode`\^=\active\def^{\ifmmode\sp\else\^{}\fi}\catcode`\%=\active\def%{\%}$\mathdefault{20}$}}%
\end{pgfscope}%
\begin{pgfscope}%
\pgfpathrectangle{\pgfqpoint{3.722897in}{0.857143in}}{\pgfqpoint{2.627103in}{1.813434in}}%
\pgfusepath{clip}%
\pgfsetrectcap%
\pgfsetroundjoin%
\pgfsetlinewidth{1.003750pt}%
\definecolor{currentstroke}{rgb}{0.000000,0.000000,0.000000}%
\pgfsetstrokecolor{currentstroke}%
\pgfsetdash{}{0pt}%
\pgfpathmoveto{\pgfqpoint{3.722897in}{1.813884in}}%
\pgfpathlineto{\pgfqpoint{6.350000in}{1.813884in}}%
\pgfusepath{stroke}%
\end{pgfscope}%
\begin{pgfscope}%
\pgfpathrectangle{\pgfqpoint{3.722897in}{0.857143in}}{\pgfqpoint{2.627103in}{1.813434in}}%
\pgfusepath{clip}%
\pgfsetrectcap%
\pgfsetroundjoin%
\pgfsetlinewidth{1.505625pt}%
\definecolor{currentstroke}{rgb}{0.000000,0.000000,0.000000}%
\pgfsetstrokecolor{currentstroke}%
\pgfsetdash{}{0pt}%
\pgfpathmoveto{\pgfqpoint{3.846779in}{2.026706in}}%
\pgfpathlineto{\pgfqpoint{3.857950in}{2.020937in}}%
\pgfpathlineto{\pgfqpoint{3.869120in}{2.012155in}}%
\pgfpathlineto{\pgfqpoint{3.880291in}{1.981754in}}%
\pgfpathlineto{\pgfqpoint{3.891461in}{1.993765in}}%
\pgfpathlineto{\pgfqpoint{3.902632in}{1.994642in}}%
\pgfpathlineto{\pgfqpoint{3.913803in}{1.987547in}}%
\pgfpathlineto{\pgfqpoint{3.924973in}{1.962617in}}%
\pgfpathlineto{\pgfqpoint{3.936144in}{1.965457in}}%
\pgfpathlineto{\pgfqpoint{3.958485in}{1.939139in}}%
\pgfpathlineto{\pgfqpoint{3.969656in}{1.907797in}}%
\pgfpathlineto{\pgfqpoint{3.991997in}{1.900000in}}%
\pgfpathlineto{\pgfqpoint{4.003168in}{1.887952in}}%
\pgfpathlineto{\pgfqpoint{4.014338in}{1.864906in}}%
\pgfpathlineto{\pgfqpoint{4.036679in}{1.865716in}}%
\pgfpathlineto{\pgfqpoint{4.047850in}{1.849769in}}%
\pgfpathlineto{\pgfqpoint{4.059021in}{1.841111in}}%
\pgfpathlineto{\pgfqpoint{4.070191in}{1.850707in}}%
\pgfpathlineto{\pgfqpoint{4.081362in}{1.855222in}}%
\pgfpathlineto{\pgfqpoint{4.103703in}{1.825417in}}%
\pgfpathlineto{\pgfqpoint{4.114874in}{1.821843in}}%
\pgfpathlineto{\pgfqpoint{4.126044in}{1.826851in}}%
\pgfpathlineto{\pgfqpoint{4.137215in}{1.825499in}}%
\pgfpathlineto{\pgfqpoint{4.148385in}{1.803588in}}%
\pgfpathlineto{\pgfqpoint{4.159556in}{1.785786in}}%
\pgfpathlineto{\pgfqpoint{4.170727in}{1.775343in}}%
\pgfpathlineto{\pgfqpoint{4.181897in}{1.759977in}}%
\pgfpathlineto{\pgfqpoint{4.193068in}{1.735202in}}%
\pgfpathlineto{\pgfqpoint{4.204238in}{1.716180in}}%
\pgfpathlineto{\pgfqpoint{4.215409in}{1.702446in}}%
\pgfpathlineto{\pgfqpoint{4.226580in}{1.701601in}}%
\pgfpathlineto{\pgfqpoint{4.237750in}{1.714360in}}%
\pgfpathlineto{\pgfqpoint{4.248921in}{1.719133in}}%
\pgfpathlineto{\pgfqpoint{4.260091in}{1.726903in}}%
\pgfpathlineto{\pgfqpoint{4.271262in}{1.739223in}}%
\pgfpathlineto{\pgfqpoint{4.282433in}{1.746485in}}%
\pgfpathlineto{\pgfqpoint{4.315944in}{1.739418in}}%
\pgfpathlineto{\pgfqpoint{4.327115in}{1.725813in}}%
\pgfpathlineto{\pgfqpoint{4.338286in}{1.733215in}}%
\pgfpathlineto{\pgfqpoint{4.349456in}{1.728582in}}%
\pgfpathlineto{\pgfqpoint{4.360627in}{1.727675in}}%
\pgfpathlineto{\pgfqpoint{4.371797in}{1.715587in}}%
\pgfpathlineto{\pgfqpoint{4.382968in}{1.715263in}}%
\pgfpathlineto{\pgfqpoint{4.394139in}{1.707290in}}%
\pgfpathlineto{\pgfqpoint{4.405309in}{1.691850in}}%
\pgfpathlineto{\pgfqpoint{4.416480in}{1.672555in}}%
\pgfpathlineto{\pgfqpoint{4.438821in}{1.659492in}}%
\pgfpathlineto{\pgfqpoint{4.449992in}{1.650401in}}%
\pgfpathlineto{\pgfqpoint{4.461162in}{1.635852in}}%
\pgfpathlineto{\pgfqpoint{4.472333in}{1.634518in}}%
\pgfpathlineto{\pgfqpoint{4.483504in}{1.624052in}}%
\pgfpathlineto{\pgfqpoint{4.494674in}{1.622349in}}%
\pgfpathlineto{\pgfqpoint{4.505845in}{1.604243in}}%
\pgfpathlineto{\pgfqpoint{4.517015in}{1.600971in}}%
\pgfpathlineto{\pgfqpoint{4.528186in}{1.600984in}}%
\pgfpathlineto{\pgfqpoint{4.539357in}{1.604449in}}%
\pgfpathlineto{\pgfqpoint{4.550527in}{1.598460in}}%
\pgfpathlineto{\pgfqpoint{4.561698in}{1.619315in}}%
\pgfpathlineto{\pgfqpoint{4.572868in}{1.636193in}}%
\pgfpathlineto{\pgfqpoint{4.584039in}{1.649568in}}%
\pgfpathlineto{\pgfqpoint{4.595210in}{1.677263in}}%
\pgfpathlineto{\pgfqpoint{4.606380in}{1.691067in}}%
\pgfpathlineto{\pgfqpoint{4.617551in}{1.693562in}}%
\pgfpathlineto{\pgfqpoint{4.628721in}{1.703830in}}%
\pgfpathlineto{\pgfqpoint{4.639892in}{1.712367in}}%
\pgfpathlineto{\pgfqpoint{4.651063in}{1.724537in}}%
\pgfpathlineto{\pgfqpoint{4.662233in}{1.749921in}}%
\pgfpathlineto{\pgfqpoint{4.673404in}{1.754230in}}%
\pgfpathlineto{\pgfqpoint{4.684574in}{1.767002in}}%
\pgfpathlineto{\pgfqpoint{4.695745in}{1.772705in}}%
\pgfpathlineto{\pgfqpoint{4.706916in}{1.805905in}}%
\pgfpathlineto{\pgfqpoint{4.718086in}{1.808224in}}%
\pgfpathlineto{\pgfqpoint{4.729257in}{1.822253in}}%
\pgfpathlineto{\pgfqpoint{4.740427in}{1.832670in}}%
\pgfpathlineto{\pgfqpoint{4.751598in}{1.848943in}}%
\pgfpathlineto{\pgfqpoint{4.762769in}{1.845679in}}%
\pgfpathlineto{\pgfqpoint{4.773939in}{1.851465in}}%
\pgfpathlineto{\pgfqpoint{4.785110in}{1.849475in}}%
\pgfpathlineto{\pgfqpoint{4.796280in}{1.858784in}}%
\pgfpathlineto{\pgfqpoint{4.818622in}{1.861710in}}%
\pgfpathlineto{\pgfqpoint{4.829792in}{1.859102in}}%
\pgfpathlineto{\pgfqpoint{4.840963in}{1.865775in}}%
\pgfpathlineto{\pgfqpoint{4.863304in}{1.862953in}}%
\pgfpathlineto{\pgfqpoint{4.874475in}{1.862935in}}%
\pgfpathlineto{\pgfqpoint{4.885645in}{1.871814in}}%
\pgfpathlineto{\pgfqpoint{4.896816in}{1.878571in}}%
\pgfpathlineto{\pgfqpoint{4.919157in}{1.886370in}}%
\pgfpathlineto{\pgfqpoint{4.930328in}{1.899450in}}%
\pgfpathlineto{\pgfqpoint{4.941498in}{1.903698in}}%
\pgfpathlineto{\pgfqpoint{4.952669in}{1.903675in}}%
\pgfpathlineto{\pgfqpoint{4.963840in}{1.913414in}}%
\pgfpathlineto{\pgfqpoint{4.975010in}{1.925200in}}%
\pgfpathlineto{\pgfqpoint{4.986181in}{1.927752in}}%
\pgfpathlineto{\pgfqpoint{4.997351in}{1.932500in}}%
\pgfpathlineto{\pgfqpoint{5.008522in}{1.933149in}}%
\pgfpathlineto{\pgfqpoint{5.019693in}{1.937643in}}%
\pgfpathlineto{\pgfqpoint{5.030863in}{1.933068in}}%
\pgfpathlineto{\pgfqpoint{5.042034in}{1.939483in}}%
\pgfpathlineto{\pgfqpoint{5.053204in}{1.935103in}}%
\pgfpathlineto{\pgfqpoint{5.064375in}{1.941148in}}%
\pgfpathlineto{\pgfqpoint{5.075546in}{1.934568in}}%
\pgfpathlineto{\pgfqpoint{5.086716in}{1.932483in}}%
\pgfpathlineto{\pgfqpoint{5.097887in}{1.925854in}}%
\pgfpathlineto{\pgfqpoint{5.109057in}{1.928375in}}%
\pgfpathlineto{\pgfqpoint{5.120228in}{1.924516in}}%
\pgfpathlineto{\pgfqpoint{5.131399in}{1.924374in}}%
\pgfpathlineto{\pgfqpoint{5.153740in}{1.903254in}}%
\pgfpathlineto{\pgfqpoint{5.164910in}{1.908803in}}%
\pgfpathlineto{\pgfqpoint{5.176081in}{1.902459in}}%
\pgfpathlineto{\pgfqpoint{5.187252in}{1.899172in}}%
\pgfpathlineto{\pgfqpoint{5.198422in}{1.898811in}}%
\pgfpathlineto{\pgfqpoint{5.209593in}{1.892820in}}%
\pgfpathlineto{\pgfqpoint{5.220763in}{1.881345in}}%
\pgfpathlineto{\pgfqpoint{5.231934in}{1.873393in}}%
\pgfpathlineto{\pgfqpoint{5.243105in}{1.872827in}}%
\pgfpathlineto{\pgfqpoint{5.254275in}{1.867574in}}%
\pgfpathlineto{\pgfqpoint{5.265446in}{1.857292in}}%
\pgfpathlineto{\pgfqpoint{5.276617in}{1.841835in}}%
\pgfpathlineto{\pgfqpoint{5.287787in}{1.841372in}}%
\pgfpathlineto{\pgfqpoint{5.298958in}{1.834380in}}%
\pgfpathlineto{\pgfqpoint{5.310128in}{1.820802in}}%
\pgfpathlineto{\pgfqpoint{5.321299in}{1.810797in}}%
\pgfpathlineto{\pgfqpoint{5.332470in}{1.812065in}}%
\pgfpathlineto{\pgfqpoint{5.343640in}{1.797618in}}%
\pgfpathlineto{\pgfqpoint{5.354811in}{1.773072in}}%
\pgfpathlineto{\pgfqpoint{5.377152in}{1.743272in}}%
\pgfpathlineto{\pgfqpoint{5.388323in}{1.743312in}}%
\pgfpathlineto{\pgfqpoint{5.399493in}{1.724105in}}%
\pgfpathlineto{\pgfqpoint{5.410664in}{1.721997in}}%
\pgfpathlineto{\pgfqpoint{5.421834in}{1.726900in}}%
\pgfpathlineto{\pgfqpoint{5.433005in}{1.723030in}}%
\pgfpathlineto{\pgfqpoint{5.444176in}{1.725734in}}%
\pgfpathlineto{\pgfqpoint{5.455346in}{1.724843in}}%
\pgfpathlineto{\pgfqpoint{5.466517in}{1.731454in}}%
\pgfpathlineto{\pgfqpoint{5.477687in}{1.725074in}}%
\pgfpathlineto{\pgfqpoint{5.488858in}{1.736348in}}%
\pgfpathlineto{\pgfqpoint{5.500029in}{1.735083in}}%
\pgfpathlineto{\pgfqpoint{5.511199in}{1.743704in}}%
\pgfpathlineto{\pgfqpoint{5.533540in}{1.742676in}}%
\pgfpathlineto{\pgfqpoint{5.544711in}{1.739181in}}%
\pgfpathlineto{\pgfqpoint{5.555882in}{1.745057in}}%
\pgfpathlineto{\pgfqpoint{5.567052in}{1.747076in}}%
\pgfpathlineto{\pgfqpoint{5.578223in}{1.738489in}}%
\pgfpathlineto{\pgfqpoint{5.589393in}{1.731504in}}%
\pgfpathlineto{\pgfqpoint{5.600564in}{1.737383in}}%
\pgfpathlineto{\pgfqpoint{5.611735in}{1.740585in}}%
\pgfpathlineto{\pgfqpoint{5.622905in}{1.729508in}}%
\pgfpathlineto{\pgfqpoint{5.634076in}{1.722619in}}%
\pgfpathlineto{\pgfqpoint{5.645247in}{1.725077in}}%
\pgfpathlineto{\pgfqpoint{5.656417in}{1.723004in}}%
\pgfpathlineto{\pgfqpoint{5.667588in}{1.710488in}}%
\pgfpathlineto{\pgfqpoint{5.678758in}{1.708532in}}%
\pgfpathlineto{\pgfqpoint{5.701100in}{1.717822in}}%
\pgfpathlineto{\pgfqpoint{5.712270in}{1.709474in}}%
\pgfpathlineto{\pgfqpoint{5.723441in}{1.733938in}}%
\pgfpathlineto{\pgfqpoint{5.734611in}{1.769499in}}%
\pgfpathlineto{\pgfqpoint{5.756953in}{1.807699in}}%
\pgfpathlineto{\pgfqpoint{5.779294in}{1.832198in}}%
\pgfpathlineto{\pgfqpoint{5.790464in}{1.847493in}}%
\pgfpathlineto{\pgfqpoint{5.801635in}{1.858756in}}%
\pgfpathlineto{\pgfqpoint{5.812806in}{1.866871in}}%
\pgfpathlineto{\pgfqpoint{5.823976in}{1.877651in}}%
\pgfpathlineto{\pgfqpoint{5.835147in}{1.877524in}}%
\pgfpathlineto{\pgfqpoint{5.846317in}{1.867747in}}%
\pgfpathlineto{\pgfqpoint{5.857488in}{1.876459in}}%
\pgfpathlineto{\pgfqpoint{5.868659in}{1.887147in}}%
\pgfpathlineto{\pgfqpoint{5.879829in}{1.891944in}}%
\pgfpathlineto{\pgfqpoint{5.891000in}{1.891572in}}%
\pgfpathlineto{\pgfqpoint{5.902170in}{1.888222in}}%
\pgfpathlineto{\pgfqpoint{5.924512in}{1.894720in}}%
\pgfpathlineto{\pgfqpoint{5.946853in}{1.870788in}}%
\pgfpathlineto{\pgfqpoint{5.958023in}{1.875559in}}%
\pgfpathlineto{\pgfqpoint{5.969194in}{1.882317in}}%
\pgfpathlineto{\pgfqpoint{5.980365in}{1.870831in}}%
\pgfpathlineto{\pgfqpoint{5.991535in}{1.866677in}}%
\pgfpathlineto{\pgfqpoint{6.002706in}{1.869455in}}%
\pgfpathlineto{\pgfqpoint{6.013876in}{1.861457in}}%
\pgfpathlineto{\pgfqpoint{6.025047in}{1.848678in}}%
\pgfpathlineto{\pgfqpoint{6.036218in}{1.837812in}}%
\pgfpathlineto{\pgfqpoint{6.047388in}{1.858491in}}%
\pgfpathlineto{\pgfqpoint{6.058559in}{1.863108in}}%
\pgfpathlineto{\pgfqpoint{6.069730in}{1.853879in}}%
\pgfpathlineto{\pgfqpoint{6.080900in}{1.850206in}}%
\pgfpathlineto{\pgfqpoint{6.092071in}{1.852371in}}%
\pgfpathlineto{\pgfqpoint{6.103241in}{1.835411in}}%
\pgfpathlineto{\pgfqpoint{6.114412in}{1.824966in}}%
\pgfpathlineto{\pgfqpoint{6.125583in}{1.826845in}}%
\pgfpathlineto{\pgfqpoint{6.136753in}{1.822388in}}%
\pgfpathlineto{\pgfqpoint{6.147924in}{1.829920in}}%
\pgfpathlineto{\pgfqpoint{6.159094in}{1.820108in}}%
\pgfpathlineto{\pgfqpoint{6.170265in}{1.819198in}}%
\pgfpathlineto{\pgfqpoint{6.181436in}{1.822568in}}%
\pgfpathlineto{\pgfqpoint{6.192606in}{1.813853in}}%
\pgfpathlineto{\pgfqpoint{6.203777in}{1.801810in}}%
\pgfpathlineto{\pgfqpoint{6.214947in}{1.813313in}}%
\pgfpathlineto{\pgfqpoint{6.226118in}{1.817498in}}%
\pgfpathlineto{\pgfqpoint{6.226118in}{1.817498in}}%
\pgfusepath{stroke}%
\end{pgfscope}%
\begin{pgfscope}%
\pgfsetrectcap%
\pgfsetmiterjoin%
\pgfsetlinewidth{0.803000pt}%
\definecolor{currentstroke}{rgb}{0.000000,0.000000,0.000000}%
\pgfsetstrokecolor{currentstroke}%
\pgfsetdash{}{0pt}%
\pgfpathmoveto{\pgfqpoint{3.722897in}{0.857143in}}%
\pgfpathlineto{\pgfqpoint{3.722897in}{2.670576in}}%
\pgfusepath{stroke}%
\end{pgfscope}%
\begin{pgfscope}%
\pgfsetrectcap%
\pgfsetmiterjoin%
\pgfsetlinewidth{0.803000pt}%
\definecolor{currentstroke}{rgb}{0.000000,0.000000,0.000000}%
\pgfsetstrokecolor{currentstroke}%
\pgfsetdash{}{0pt}%
\pgfpathmoveto{\pgfqpoint{6.350000in}{0.857143in}}%
\pgfpathlineto{\pgfqpoint{6.350000in}{2.670576in}}%
\pgfusepath{stroke}%
\end{pgfscope}%
\begin{pgfscope}%
\pgfsetrectcap%
\pgfsetmiterjoin%
\pgfsetlinewidth{0.803000pt}%
\definecolor{currentstroke}{rgb}{0.000000,0.000000,0.000000}%
\pgfsetstrokecolor{currentstroke}%
\pgfsetdash{}{0pt}%
\pgfpathmoveto{\pgfqpoint{3.722897in}{0.857143in}}%
\pgfpathlineto{\pgfqpoint{6.350000in}{0.857143in}}%
\pgfusepath{stroke}%
\end{pgfscope}%
\begin{pgfscope}%
\pgfsetrectcap%
\pgfsetmiterjoin%
\pgfsetlinewidth{0.803000pt}%
\definecolor{currentstroke}{rgb}{0.000000,0.000000,0.000000}%
\pgfsetstrokecolor{currentstroke}%
\pgfsetdash{}{0pt}%
\pgfpathmoveto{\pgfqpoint{3.722897in}{2.670576in}}%
\pgfpathlineto{\pgfqpoint{6.350000in}{2.670576in}}%
\pgfusepath{stroke}%
\end{pgfscope}%
\begin{pgfscope}%
\definecolor{textcolor}{rgb}{0.000000,0.000000,0.000000}%
\pgfsetstrokecolor{textcolor}%
\pgfsetfillcolor{textcolor}%
\pgftext[x=5.036449in,y=2.753910in,,base]{\color{textcolor}{\rmfamily\fontsize{10.000000}{12.000000}\selectfont\catcode`\^=\active\def^{\ifmmode\sp\else\^{}\fi}\catcode`\%=\active\def%{\%}Savings}}%
\end{pgfscope}%
\begin{pgfscope}%
\definecolor{textcolor}{rgb}{0.000000,0.000000,0.000000}%
\pgfsetstrokecolor{textcolor}%
\pgfsetfillcolor{textcolor}%
\pgftext[x=0.235523in, y=0.771037in, left, base,rotate=90.000000]{\color{textcolor}{\rmfamily\fontsize{10.000000}{12.000000}\selectfont\catcode`\^=\active\def^{\ifmmode\sp\else\^{}\fi}\catcode`\%=\active\def%{\%}\% Deviation from SS}}%
\end{pgfscope}%
\begin{pgfscope}%
\pgfsetbuttcap%
\pgfsetmiterjoin%
\definecolor{currentfill}{rgb}{0.066899,0.263188,0.377594}%
\pgfsetfillcolor{currentfill}%
\pgfsetlinewidth{0.000000pt}%
\definecolor{currentstroke}{rgb}{0.000000,0.000000,0.000000}%
\pgfsetstrokecolor{currentstroke}%
\pgfsetstrokeopacity{0.000000}%
\pgfsetdash{}{0pt}%
\pgfpathmoveto{\pgfqpoint{0.552904in}{0.361680in}}%
\pgfpathlineto{\pgfqpoint{0.830682in}{0.361680in}}%
\pgfpathlineto{\pgfqpoint{0.830682in}{0.458903in}}%
\pgfpathlineto{\pgfqpoint{0.552904in}{0.458903in}}%
\pgfpathlineto{\pgfqpoint{0.552904in}{0.361680in}}%
\pgfpathclose%
\pgfusepath{fill}%
\end{pgfscope}%
\begin{pgfscope}%
\definecolor{textcolor}{rgb}{0.000000,0.000000,0.000000}%
\pgfsetstrokecolor{textcolor}%
\pgfsetfillcolor{textcolor}%
\pgftext[x=0.941793in,y=0.361680in,left,base]{\color{textcolor}{\rmfamily\fontsize{10.000000}{12.000000}\selectfont\catcode`\^=\active\def^{\ifmmode\sp\else\^{}\fi}\catcode`\%=\active\def%{\%}TFP}}%
\end{pgfscope}%
\begin{pgfscope}%
\pgfsetbuttcap%
\pgfsetmiterjoin%
\definecolor{currentfill}{rgb}{0.133298,0.375282,0.379395}%
\pgfsetfillcolor{currentfill}%
\pgfsetlinewidth{0.000000pt}%
\definecolor{currentstroke}{rgb}{0.000000,0.000000,0.000000}%
\pgfsetstrokecolor{currentstroke}%
\pgfsetstrokeopacity{0.000000}%
\pgfsetdash{}{0pt}%
\pgfpathmoveto{\pgfqpoint{0.552904in}{0.157823in}}%
\pgfpathlineto{\pgfqpoint{0.830682in}{0.157823in}}%
\pgfpathlineto{\pgfqpoint{0.830682in}{0.255045in}}%
\pgfpathlineto{\pgfqpoint{0.552904in}{0.255045in}}%
\pgfpathlineto{\pgfqpoint{0.552904in}{0.157823in}}%
\pgfpathclose%
\pgfusepath{fill}%
\end{pgfscope}%
\begin{pgfscope}%
\definecolor{textcolor}{rgb}{0.000000,0.000000,0.000000}%
\pgfsetstrokecolor{textcolor}%
\pgfsetfillcolor{textcolor}%
\pgftext[x=0.941793in,y=0.157823in,left,base]{\color{textcolor}{\rmfamily\fontsize{10.000000}{12.000000}\selectfont\catcode`\^=\active\def^{\ifmmode\sp\else\^{}\fi}\catcode`\%=\active\def%{\%}Markup}}%
\end{pgfscope}%
\begin{pgfscope}%
\pgfsetbuttcap%
\pgfsetmiterjoin%
\definecolor{currentfill}{rgb}{0.302379,0.450282,0.300122}%
\pgfsetfillcolor{currentfill}%
\pgfsetlinewidth{0.000000pt}%
\definecolor{currentstroke}{rgb}{0.000000,0.000000,0.000000}%
\pgfsetstrokecolor{currentstroke}%
\pgfsetstrokeopacity{0.000000}%
\pgfsetdash{}{0pt}%
\pgfpathmoveto{\pgfqpoint{1.773498in}{0.361680in}}%
\pgfpathlineto{\pgfqpoint{2.051276in}{0.361680in}}%
\pgfpathlineto{\pgfqpoint{2.051276in}{0.458903in}}%
\pgfpathlineto{\pgfqpoint{1.773498in}{0.458903in}}%
\pgfpathlineto{\pgfqpoint{1.773498in}{0.361680in}}%
\pgfpathclose%
\pgfusepath{fill}%
\end{pgfscope}%
\begin{pgfscope}%
\definecolor{textcolor}{rgb}{0.000000,0.000000,0.000000}%
\pgfsetstrokecolor{textcolor}%
\pgfsetfillcolor{textcolor}%
\pgftext[x=2.162387in,y=0.361680in,left,base]{\color{textcolor}{\rmfamily\fontsize{10.000000}{12.000000}\selectfont\catcode`\^=\active\def^{\ifmmode\sp\else\^{}\fi}\catcode`\%=\active\def%{\%}Wage Markup}}%
\end{pgfscope}%
\begin{pgfscope}%
\pgfsetbuttcap%
\pgfsetmiterjoin%
\definecolor{currentfill}{rgb}{0.511253,0.510898,0.193296}%
\pgfsetfillcolor{currentfill}%
\pgfsetlinewidth{0.000000pt}%
\definecolor{currentstroke}{rgb}{0.000000,0.000000,0.000000}%
\pgfsetstrokecolor{currentstroke}%
\pgfsetstrokeopacity{0.000000}%
\pgfsetdash{}{0pt}%
\pgfpathmoveto{\pgfqpoint{1.773498in}{0.155856in}}%
\pgfpathlineto{\pgfqpoint{2.051276in}{0.155856in}}%
\pgfpathlineto{\pgfqpoint{2.051276in}{0.253079in}}%
\pgfpathlineto{\pgfqpoint{1.773498in}{0.253079in}}%
\pgfpathlineto{\pgfqpoint{1.773498in}{0.155856in}}%
\pgfpathclose%
\pgfusepath{fill}%
\end{pgfscope}%
\begin{pgfscope}%
\definecolor{textcolor}{rgb}{0.000000,0.000000,0.000000}%
\pgfsetstrokecolor{textcolor}%
\pgfsetfillcolor{textcolor}%
\pgftext[x=2.162387in,y=0.155856in,left,base]{\color{textcolor}{\rmfamily\fontsize{10.000000}{12.000000}\selectfont\catcode`\^=\active\def^{\ifmmode\sp\else\^{}\fi}\catcode`\%=\active\def%{\%}Govt. Spending}}%
\end{pgfscope}%
\begin{pgfscope}%
\pgfsetbuttcap%
\pgfsetmiterjoin%
\definecolor{currentfill}{rgb}{0.754268,0.565033,0.211761}%
\pgfsetfillcolor{currentfill}%
\pgfsetlinewidth{0.000000pt}%
\definecolor{currentstroke}{rgb}{0.000000,0.000000,0.000000}%
\pgfsetstrokecolor{currentstroke}%
\pgfsetstrokeopacity{0.000000}%
\pgfsetdash{}{0pt}%
\pgfpathmoveto{\pgfqpoint{3.539408in}{0.361680in}}%
\pgfpathlineto{\pgfqpoint{3.817186in}{0.361680in}}%
\pgfpathlineto{\pgfqpoint{3.817186in}{0.458903in}}%
\pgfpathlineto{\pgfqpoint{3.539408in}{0.458903in}}%
\pgfpathlineto{\pgfqpoint{3.539408in}{0.361680in}}%
\pgfpathclose%
\pgfusepath{fill}%
\end{pgfscope}%
\begin{pgfscope}%
\definecolor{textcolor}{rgb}{0.000000,0.000000,0.000000}%
\pgfsetstrokecolor{textcolor}%
\pgfsetfillcolor{textcolor}%
\pgftext[x=3.928297in,y=0.361680in,left,base]{\color{textcolor}{\rmfamily\fontsize{10.000000}{12.000000}\selectfont\catcode`\^=\active\def^{\ifmmode\sp\else\^{}\fi}\catcode`\%=\active\def%{\%}Mon. Pol.}}%
\end{pgfscope}%
\begin{pgfscope}%
\pgfsetbuttcap%
\pgfsetmiterjoin%
\definecolor{currentfill}{rgb}{0.950697,0.616649,0.428624}%
\pgfsetfillcolor{currentfill}%
\pgfsetlinewidth{0.000000pt}%
\definecolor{currentstroke}{rgb}{0.000000,0.000000,0.000000}%
\pgfsetstrokecolor{currentstroke}%
\pgfsetstrokeopacity{0.000000}%
\pgfsetdash{}{0pt}%
\pgfpathmoveto{\pgfqpoint{3.539408in}{0.157823in}}%
\pgfpathlineto{\pgfqpoint{3.817186in}{0.157823in}}%
\pgfpathlineto{\pgfqpoint{3.817186in}{0.255045in}}%
\pgfpathlineto{\pgfqpoint{3.539408in}{0.255045in}}%
\pgfpathlineto{\pgfqpoint{3.539408in}{0.157823in}}%
\pgfpathclose%
\pgfusepath{fill}%
\end{pgfscope}%
\begin{pgfscope}%
\definecolor{textcolor}{rgb}{0.000000,0.000000,0.000000}%
\pgfsetstrokecolor{textcolor}%
\pgfsetfillcolor{textcolor}%
\pgftext[x=3.928297in,y=0.157823in,left,base]{\color{textcolor}{\rmfamily\fontsize{10.000000}{12.000000}\selectfont\catcode`\^=\active\def^{\ifmmode\sp\else\^{}\fi}\catcode`\%=\active\def%{\%}Tax Prog.}}%
\end{pgfscope}%
\begin{pgfscope}%
\pgfsetbuttcap%
\pgfsetmiterjoin%
\definecolor{currentfill}{rgb}{0.992771,0.707689,0.712380}%
\pgfsetfillcolor{currentfill}%
\pgfsetlinewidth{0.000000pt}%
\definecolor{currentstroke}{rgb}{0.000000,0.000000,0.000000}%
\pgfsetstrokecolor{currentstroke}%
\pgfsetstrokeopacity{0.000000}%
\pgfsetdash{}{0pt}%
\pgfpathmoveto{\pgfqpoint{4.884379in}{0.361680in}}%
\pgfpathlineto{\pgfqpoint{5.162157in}{0.361680in}}%
\pgfpathlineto{\pgfqpoint{5.162157in}{0.458903in}}%
\pgfpathlineto{\pgfqpoint{4.884379in}{0.458903in}}%
\pgfpathlineto{\pgfqpoint{4.884379in}{0.361680in}}%
\pgfpathclose%
\pgfusepath{fill}%
\end{pgfscope}%
\begin{pgfscope}%
\definecolor{textcolor}{rgb}{0.000000,0.000000,0.000000}%
\pgfsetstrokecolor{textcolor}%
\pgfsetfillcolor{textcolor}%
\pgftext[x=5.273268in,y=0.361680in,left,base]{\color{textcolor}{\rmfamily\fontsize{10.000000}{12.000000}\selectfont\catcode`\^=\active\def^{\ifmmode\sp\else\^{}\fi}\catcode`\%=\active\def%{\%}Transfers}}%
\end{pgfscope}%
\end{pgfpicture}%
\makeatother%
\endgroup%


    {\scriptsize \emph{Notes:} NBER-dated recessions highlighted in gray.}
    \label{fig:agg-hist-decomp}
\end{figure}

Looking at specific time periods, price markups increase consumption initially while wage markups decrease consumption. Then, in the 80s the two effects flip-flop and price markups start to decrease consumption while wage markups increase it. Finally, around the time of the Great Recession the two effects switch again until the end of the estimation window. Across almost all periods, the effects from wage markups are slightly stronger than those of price markups, consistent with the variance decomposition in Figure \ref{subfig:hh-agg-aggs}. Instead of price markups, transfers and monetary policy have significant impacts on savings, but the direction of the effects again flips in the early 1980s and late 2000s.

\begin{figure}[t!]
    \centering
    \caption{Historical Decomposition: Household Consumption}
    \input{figures/c_historical_decomp.pgf}
    {\scriptsize \emph{Notes:} Row labeled by wealth percentile. 12 quarter moving average applied. NBER-dated recessions highlighted in gray.}
    \label{fig:cons-hist-decomp}
\end{figure}

Figure \ref{fig:cons-hist-decomp} has historical decompositions for consumption decisions for households across the income and wealth distribution. Like in Section \ref{sec:buis-cycs}, I focus on low, middle, and high income households at the 0th, 50th, 90th, and 99th wealth percentiles. Simulated paths for consumption are very different for households at the 0th wealth percentile than other points along the wealth distribution, which all look very similar. Specifically, low wealth households are effected by a more diverse array of shocks, while the consumption patterns for higher wealth households are almost entirely explained by price and wage markups. Wage markup shocks impact low income households more than price markups, though for high income households price markup shocks are more important.

Tax progressivity shocks have opposite effects on low and high income households --- when tax progressivity shocks cause low income households to consume more they cause high income households to consume less. Despite the decomposition only being fit on aggregate data, the effects of the Reagan-era tax cuts for higher income households are clear within the decomposition \autocite{prasad2012popular}. Before the 80s when the top marginal tax rate in the US was highest, the level of tax progressivity makes higher income households consume less, and it makes lower and middle income households consume more. After the 80s, this flips and tax progressivity has positive effects on the consumption of high income households and negative effects on the consumption of low and middle income households. This lends credence to the use of shock paths fitted on aggregate data to gain understanding of individual-level phenomena within the model.

A similar decomposition for savings decisions is shown in Figure \ref{fig:sav-hist-decomp}. Low and middle income households at the 0th wealth percentile never save, and therefore the decomposition is constant over time. Low and middle income households at other wealth levels do have some variability in their savings decisions, though substantially less than higher income households. This contrasts the consumption decomposition in Figure \ref{fig:cons-hist-decomp}, where within a wealth band the consumption has a similar variability at all income levels. Price markups and tax progressivity are most important for low income households' savings decisions. Middle income household saving is similarly affected by price markups and tax progressivity though also face significant wage markup effects. Changes in savings decisions for high income households, especially at higher wealth levels, are almost entirely caused by wage markup shocks.

\begin{figure}[t!]
    \centering
    \caption{Historical Decomposition: Household Savings}
    \input{figures/a_historical_decomp.pgf}
    {\scriptsize \emph{Notes:} Row labeled by wealth percentile. 12 quarter moving average applied. NBER-dated recessions highlighted in gray.}
    \label{fig:sav-hist-decomp}
\end{figure}


\subsection{Historical Decomposition of Endogenous Effects}

To get a more precise view of what affects households, not just the overall macroeconomic shocks, I separate the decomposed paths into the direct factors that play into household decisions. Like outlined in Section \ref{sec:endog-eff}, shocks affect households through labor supply, wages, interest rates, transfers, and taxes. Therefore, paths for household and aggregate consumption and saving can be explained as the sum of the effects from each of these sources.

Figure \ref{fig:agg-endog-hist-decomp} shows the effect of each of these channels on aggregate consumption and saving. The specific paths for both series are identical to those in Figure \ref{fig:agg-hist-decomp}. Wages and transfers, which move inverse to each other, are the most significant factors affecting consumption, through the labor supply decided by the union also plays an important role. In fact, the transfer and wage effects almost perfectly cancel each other out, so the overall series very nearly follows the path for the labor supply effects. Interest rates and taxes have minimal effects on aggregate consumption within the window.

\begin{figure}[t]
    \centering
    \caption{Historical Endogenous Decomposition: Household Aggregates}
    %% Creator: Matplotlib, PGF backend
%%
%% To include the figure in your LaTeX document, write
%%   \input{<filename>.pgf}
%%
%% Make sure the required packages are loaded in your preamble
%%   \usepackage{pgf}
%%
%% Also ensure that all the required font packages are loaded; for instance,
%% the lmodern package is sometimes necessary when using math font.
%%   \usepackage{lmodern}
%%
%% Figures using additional raster images can only be included by \input if
%% they are in the same directory as the main LaTeX file. For loading figures
%% from other directories you can use the `import` package
%%   \usepackage{import}
%%
%% and then include the figures with
%%   \import{<path to file>}{<filename>.pgf}
%%
%% Matplotlib used the following preamble
%%   \def\mathdefault#1{#1}
%%   \everymath=\expandafter{\the\everymath\displaystyle}
%%   
%%   \ifdefined\pdftexversion\else  % non-pdftex case.
%%     \usepackage{fontspec}
%%     \setmainfont{DejaVuSerif.ttf}[Path=\detokenize{/opt/miniconda3/lib/python3.12/site-packages/matplotlib/mpl-data/fonts/ttf/}]
%%     \setsansfont{DejaVuSans.ttf}[Path=\detokenize{/opt/miniconda3/lib/python3.12/site-packages/matplotlib/mpl-data/fonts/ttf/}]
%%     \setmonofont{DejaVuSansMono.ttf}[Path=\detokenize{/opt/miniconda3/lib/python3.12/site-packages/matplotlib/mpl-data/fonts/ttf/}]
%%   \fi
%%   \makeatletter\@ifpackageloaded{underscore}{}{\usepackage[strings]{underscore}}\makeatother
%%
\begingroup%
\makeatletter%
\begin{pgfpicture}%
\pgfpathrectangle{\pgfpointorigin}{\pgfqpoint{6.500000in}{3.000000in}}%
\pgfusepath{use as bounding box, clip}%
\begin{pgfscope}%
\pgfsetbuttcap%
\pgfsetmiterjoin%
\definecolor{currentfill}{rgb}{1.000000,1.000000,1.000000}%
\pgfsetfillcolor{currentfill}%
\pgfsetlinewidth{0.000000pt}%
\definecolor{currentstroke}{rgb}{1.000000,1.000000,1.000000}%
\pgfsetstrokecolor{currentstroke}%
\pgfsetdash{}{0pt}%
\pgfpathmoveto{\pgfqpoint{0.000000in}{0.000000in}}%
\pgfpathlineto{\pgfqpoint{6.500000in}{0.000000in}}%
\pgfpathlineto{\pgfqpoint{6.500000in}{3.000000in}}%
\pgfpathlineto{\pgfqpoint{0.000000in}{3.000000in}}%
\pgfpathlineto{\pgfqpoint{0.000000in}{0.000000in}}%
\pgfpathclose%
\pgfusepath{fill}%
\end{pgfscope}%
\begin{pgfscope}%
\pgfsetbuttcap%
\pgfsetmiterjoin%
\definecolor{currentfill}{rgb}{1.000000,1.000000,1.000000}%
\pgfsetfillcolor{currentfill}%
\pgfsetlinewidth{0.000000pt}%
\definecolor{currentstroke}{rgb}{0.000000,0.000000,0.000000}%
\pgfsetstrokecolor{currentstroke}%
\pgfsetstrokeopacity{0.000000}%
\pgfsetdash{}{0pt}%
\pgfpathmoveto{\pgfqpoint{0.804646in}{0.600000in}}%
\pgfpathlineto{\pgfqpoint{3.377938in}{0.600000in}}%
\pgfpathlineto{\pgfqpoint{3.377938in}{2.670576in}}%
\pgfpathlineto{\pgfqpoint{0.804646in}{2.670576in}}%
\pgfpathlineto{\pgfqpoint{0.804646in}{0.600000in}}%
\pgfpathclose%
\pgfusepath{fill}%
\end{pgfscope}%
\begin{pgfscope}%
\pgfpathrectangle{\pgfqpoint{0.804646in}{0.600000in}}{\pgfqpoint{2.573292in}{2.070576in}}%
\pgfusepath{clip}%
\pgfsetbuttcap%
\pgfsetmiterjoin%
\definecolor{currentfill}{rgb}{0.066899,0.263188,0.377594}%
\pgfsetfillcolor{currentfill}%
\pgfsetlinewidth{0.000000pt}%
\definecolor{currentstroke}{rgb}{0.000000,0.000000,0.000000}%
\pgfsetstrokecolor{currentstroke}%
\pgfsetstrokeopacity{0.000000}%
\pgfsetdash{}{0pt}%
\pgfpathmoveto{\pgfqpoint{0.921614in}{1.613090in}}%
\pgfpathlineto{\pgfqpoint{0.930367in}{1.613090in}}%
\pgfpathlineto{\pgfqpoint{0.930367in}{1.546761in}}%
\pgfpathlineto{\pgfqpoint{0.921614in}{1.546761in}}%
\pgfpathlineto{\pgfqpoint{0.921614in}{1.613090in}}%
\pgfpathclose%
\pgfusepath{fill}%
\end{pgfscope}%
\begin{pgfscope}%
\pgfpathrectangle{\pgfqpoint{0.804646in}{0.600000in}}{\pgfqpoint{2.573292in}{2.070576in}}%
\pgfusepath{clip}%
\pgfsetbuttcap%
\pgfsetmiterjoin%
\definecolor{currentfill}{rgb}{0.066899,0.263188,0.377594}%
\pgfsetfillcolor{currentfill}%
\pgfsetlinewidth{0.000000pt}%
\definecolor{currentstroke}{rgb}{0.000000,0.000000,0.000000}%
\pgfsetstrokecolor{currentstroke}%
\pgfsetstrokeopacity{0.000000}%
\pgfsetdash{}{0pt}%
\pgfpathmoveto{\pgfqpoint{0.932555in}{1.613090in}}%
\pgfpathlineto{\pgfqpoint{0.941309in}{1.613090in}}%
\pgfpathlineto{\pgfqpoint{0.941309in}{1.537363in}}%
\pgfpathlineto{\pgfqpoint{0.932555in}{1.537363in}}%
\pgfpathlineto{\pgfqpoint{0.932555in}{1.613090in}}%
\pgfpathclose%
\pgfusepath{fill}%
\end{pgfscope}%
\begin{pgfscope}%
\pgfpathrectangle{\pgfqpoint{0.804646in}{0.600000in}}{\pgfqpoint{2.573292in}{2.070576in}}%
\pgfusepath{clip}%
\pgfsetbuttcap%
\pgfsetmiterjoin%
\definecolor{currentfill}{rgb}{0.066899,0.263188,0.377594}%
\pgfsetfillcolor{currentfill}%
\pgfsetlinewidth{0.000000pt}%
\definecolor{currentstroke}{rgb}{0.000000,0.000000,0.000000}%
\pgfsetstrokecolor{currentstroke}%
\pgfsetstrokeopacity{0.000000}%
\pgfsetdash{}{0pt}%
\pgfpathmoveto{\pgfqpoint{0.943497in}{1.613090in}}%
\pgfpathlineto{\pgfqpoint{0.952251in}{1.613090in}}%
\pgfpathlineto{\pgfqpoint{0.952251in}{1.530288in}}%
\pgfpathlineto{\pgfqpoint{0.943497in}{1.530288in}}%
\pgfpathlineto{\pgfqpoint{0.943497in}{1.613090in}}%
\pgfpathclose%
\pgfusepath{fill}%
\end{pgfscope}%
\begin{pgfscope}%
\pgfpathrectangle{\pgfqpoint{0.804646in}{0.600000in}}{\pgfqpoint{2.573292in}{2.070576in}}%
\pgfusepath{clip}%
\pgfsetbuttcap%
\pgfsetmiterjoin%
\definecolor{currentfill}{rgb}{0.066899,0.263188,0.377594}%
\pgfsetfillcolor{currentfill}%
\pgfsetlinewidth{0.000000pt}%
\definecolor{currentstroke}{rgb}{0.000000,0.000000,0.000000}%
\pgfsetstrokecolor{currentstroke}%
\pgfsetstrokeopacity{0.000000}%
\pgfsetdash{}{0pt}%
\pgfpathmoveto{\pgfqpoint{0.954439in}{1.613090in}}%
\pgfpathlineto{\pgfqpoint{0.963192in}{1.613090in}}%
\pgfpathlineto{\pgfqpoint{0.963192in}{1.516242in}}%
\pgfpathlineto{\pgfqpoint{0.954439in}{1.516242in}}%
\pgfpathlineto{\pgfqpoint{0.954439in}{1.613090in}}%
\pgfpathclose%
\pgfusepath{fill}%
\end{pgfscope}%
\begin{pgfscope}%
\pgfpathrectangle{\pgfqpoint{0.804646in}{0.600000in}}{\pgfqpoint{2.573292in}{2.070576in}}%
\pgfusepath{clip}%
\pgfsetbuttcap%
\pgfsetmiterjoin%
\definecolor{currentfill}{rgb}{0.066899,0.263188,0.377594}%
\pgfsetfillcolor{currentfill}%
\pgfsetlinewidth{0.000000pt}%
\definecolor{currentstroke}{rgb}{0.000000,0.000000,0.000000}%
\pgfsetstrokecolor{currentstroke}%
\pgfsetstrokeopacity{0.000000}%
\pgfsetdash{}{0pt}%
\pgfpathmoveto{\pgfqpoint{0.965381in}{1.613090in}}%
\pgfpathlineto{\pgfqpoint{0.974134in}{1.613090in}}%
\pgfpathlineto{\pgfqpoint{0.974134in}{1.488882in}}%
\pgfpathlineto{\pgfqpoint{0.965381in}{1.488882in}}%
\pgfpathlineto{\pgfqpoint{0.965381in}{1.613090in}}%
\pgfpathclose%
\pgfusepath{fill}%
\end{pgfscope}%
\begin{pgfscope}%
\pgfpathrectangle{\pgfqpoint{0.804646in}{0.600000in}}{\pgfqpoint{2.573292in}{2.070576in}}%
\pgfusepath{clip}%
\pgfsetbuttcap%
\pgfsetmiterjoin%
\definecolor{currentfill}{rgb}{0.066899,0.263188,0.377594}%
\pgfsetfillcolor{currentfill}%
\pgfsetlinewidth{0.000000pt}%
\definecolor{currentstroke}{rgb}{0.000000,0.000000,0.000000}%
\pgfsetstrokecolor{currentstroke}%
\pgfsetstrokeopacity{0.000000}%
\pgfsetdash{}{0pt}%
\pgfpathmoveto{\pgfqpoint{0.976323in}{1.613090in}}%
\pgfpathlineto{\pgfqpoint{0.985076in}{1.613090in}}%
\pgfpathlineto{\pgfqpoint{0.985076in}{1.488996in}}%
\pgfpathlineto{\pgfqpoint{0.976323in}{1.488996in}}%
\pgfpathlineto{\pgfqpoint{0.976323in}{1.613090in}}%
\pgfpathclose%
\pgfusepath{fill}%
\end{pgfscope}%
\begin{pgfscope}%
\pgfpathrectangle{\pgfqpoint{0.804646in}{0.600000in}}{\pgfqpoint{2.573292in}{2.070576in}}%
\pgfusepath{clip}%
\pgfsetbuttcap%
\pgfsetmiterjoin%
\definecolor{currentfill}{rgb}{0.066899,0.263188,0.377594}%
\pgfsetfillcolor{currentfill}%
\pgfsetlinewidth{0.000000pt}%
\definecolor{currentstroke}{rgb}{0.000000,0.000000,0.000000}%
\pgfsetstrokecolor{currentstroke}%
\pgfsetstrokeopacity{0.000000}%
\pgfsetdash{}{0pt}%
\pgfpathmoveto{\pgfqpoint{0.987264in}{1.613090in}}%
\pgfpathlineto{\pgfqpoint{0.996018in}{1.613090in}}%
\pgfpathlineto{\pgfqpoint{0.996018in}{1.491706in}}%
\pgfpathlineto{\pgfqpoint{0.987264in}{1.491706in}}%
\pgfpathlineto{\pgfqpoint{0.987264in}{1.613090in}}%
\pgfpathclose%
\pgfusepath{fill}%
\end{pgfscope}%
\begin{pgfscope}%
\pgfpathrectangle{\pgfqpoint{0.804646in}{0.600000in}}{\pgfqpoint{2.573292in}{2.070576in}}%
\pgfusepath{clip}%
\pgfsetbuttcap%
\pgfsetmiterjoin%
\definecolor{currentfill}{rgb}{0.066899,0.263188,0.377594}%
\pgfsetfillcolor{currentfill}%
\pgfsetlinewidth{0.000000pt}%
\definecolor{currentstroke}{rgb}{0.000000,0.000000,0.000000}%
\pgfsetstrokecolor{currentstroke}%
\pgfsetstrokeopacity{0.000000}%
\pgfsetdash{}{0pt}%
\pgfpathmoveto{\pgfqpoint{0.998206in}{1.613090in}}%
\pgfpathlineto{\pgfqpoint{1.006960in}{1.613090in}}%
\pgfpathlineto{\pgfqpoint{1.006960in}{1.516241in}}%
\pgfpathlineto{\pgfqpoint{0.998206in}{1.516241in}}%
\pgfpathlineto{\pgfqpoint{0.998206in}{1.613090in}}%
\pgfpathclose%
\pgfusepath{fill}%
\end{pgfscope}%
\begin{pgfscope}%
\pgfpathrectangle{\pgfqpoint{0.804646in}{0.600000in}}{\pgfqpoint{2.573292in}{2.070576in}}%
\pgfusepath{clip}%
\pgfsetbuttcap%
\pgfsetmiterjoin%
\definecolor{currentfill}{rgb}{0.066899,0.263188,0.377594}%
\pgfsetfillcolor{currentfill}%
\pgfsetlinewidth{0.000000pt}%
\definecolor{currentstroke}{rgb}{0.000000,0.000000,0.000000}%
\pgfsetstrokecolor{currentstroke}%
\pgfsetstrokeopacity{0.000000}%
\pgfsetdash{}{0pt}%
\pgfpathmoveto{\pgfqpoint{1.009148in}{1.613090in}}%
\pgfpathlineto{\pgfqpoint{1.017901in}{1.613090in}}%
\pgfpathlineto{\pgfqpoint{1.017901in}{1.540728in}}%
\pgfpathlineto{\pgfqpoint{1.009148in}{1.540728in}}%
\pgfpathlineto{\pgfqpoint{1.009148in}{1.613090in}}%
\pgfpathclose%
\pgfusepath{fill}%
\end{pgfscope}%
\begin{pgfscope}%
\pgfpathrectangle{\pgfqpoint{0.804646in}{0.600000in}}{\pgfqpoint{2.573292in}{2.070576in}}%
\pgfusepath{clip}%
\pgfsetbuttcap%
\pgfsetmiterjoin%
\definecolor{currentfill}{rgb}{0.066899,0.263188,0.377594}%
\pgfsetfillcolor{currentfill}%
\pgfsetlinewidth{0.000000pt}%
\definecolor{currentstroke}{rgb}{0.000000,0.000000,0.000000}%
\pgfsetstrokecolor{currentstroke}%
\pgfsetstrokeopacity{0.000000}%
\pgfsetdash{}{0pt}%
\pgfpathmoveto{\pgfqpoint{1.020090in}{1.613090in}}%
\pgfpathlineto{\pgfqpoint{1.028843in}{1.613090in}}%
\pgfpathlineto{\pgfqpoint{1.028843in}{1.534028in}}%
\pgfpathlineto{\pgfqpoint{1.020090in}{1.534028in}}%
\pgfpathlineto{\pgfqpoint{1.020090in}{1.613090in}}%
\pgfpathclose%
\pgfusepath{fill}%
\end{pgfscope}%
\begin{pgfscope}%
\pgfpathrectangle{\pgfqpoint{0.804646in}{0.600000in}}{\pgfqpoint{2.573292in}{2.070576in}}%
\pgfusepath{clip}%
\pgfsetbuttcap%
\pgfsetmiterjoin%
\definecolor{currentfill}{rgb}{0.066899,0.263188,0.377594}%
\pgfsetfillcolor{currentfill}%
\pgfsetlinewidth{0.000000pt}%
\definecolor{currentstroke}{rgb}{0.000000,0.000000,0.000000}%
\pgfsetstrokecolor{currentstroke}%
\pgfsetstrokeopacity{0.000000}%
\pgfsetdash{}{0pt}%
\pgfpathmoveto{\pgfqpoint{1.031032in}{1.613090in}}%
\pgfpathlineto{\pgfqpoint{1.039785in}{1.613090in}}%
\pgfpathlineto{\pgfqpoint{1.039785in}{1.556625in}}%
\pgfpathlineto{\pgfqpoint{1.031032in}{1.556625in}}%
\pgfpathlineto{\pgfqpoint{1.031032in}{1.613090in}}%
\pgfpathclose%
\pgfusepath{fill}%
\end{pgfscope}%
\begin{pgfscope}%
\pgfpathrectangle{\pgfqpoint{0.804646in}{0.600000in}}{\pgfqpoint{2.573292in}{2.070576in}}%
\pgfusepath{clip}%
\pgfsetbuttcap%
\pgfsetmiterjoin%
\definecolor{currentfill}{rgb}{0.066899,0.263188,0.377594}%
\pgfsetfillcolor{currentfill}%
\pgfsetlinewidth{0.000000pt}%
\definecolor{currentstroke}{rgb}{0.000000,0.000000,0.000000}%
\pgfsetstrokecolor{currentstroke}%
\pgfsetstrokeopacity{0.000000}%
\pgfsetdash{}{0pt}%
\pgfpathmoveto{\pgfqpoint{1.041973in}{1.613090in}}%
\pgfpathlineto{\pgfqpoint{1.050727in}{1.613090in}}%
\pgfpathlineto{\pgfqpoint{1.050727in}{1.567454in}}%
\pgfpathlineto{\pgfqpoint{1.041973in}{1.567454in}}%
\pgfpathlineto{\pgfqpoint{1.041973in}{1.613090in}}%
\pgfpathclose%
\pgfusepath{fill}%
\end{pgfscope}%
\begin{pgfscope}%
\pgfpathrectangle{\pgfqpoint{0.804646in}{0.600000in}}{\pgfqpoint{2.573292in}{2.070576in}}%
\pgfusepath{clip}%
\pgfsetbuttcap%
\pgfsetmiterjoin%
\definecolor{currentfill}{rgb}{0.066899,0.263188,0.377594}%
\pgfsetfillcolor{currentfill}%
\pgfsetlinewidth{0.000000pt}%
\definecolor{currentstroke}{rgb}{0.000000,0.000000,0.000000}%
\pgfsetstrokecolor{currentstroke}%
\pgfsetstrokeopacity{0.000000}%
\pgfsetdash{}{0pt}%
\pgfpathmoveto{\pgfqpoint{1.052915in}{1.613090in}}%
\pgfpathlineto{\pgfqpoint{1.061669in}{1.613090in}}%
\pgfpathlineto{\pgfqpoint{1.061669in}{1.575650in}}%
\pgfpathlineto{\pgfqpoint{1.052915in}{1.575650in}}%
\pgfpathlineto{\pgfqpoint{1.052915in}{1.613090in}}%
\pgfpathclose%
\pgfusepath{fill}%
\end{pgfscope}%
\begin{pgfscope}%
\pgfpathrectangle{\pgfqpoint{0.804646in}{0.600000in}}{\pgfqpoint{2.573292in}{2.070576in}}%
\pgfusepath{clip}%
\pgfsetbuttcap%
\pgfsetmiterjoin%
\definecolor{currentfill}{rgb}{0.066899,0.263188,0.377594}%
\pgfsetfillcolor{currentfill}%
\pgfsetlinewidth{0.000000pt}%
\definecolor{currentstroke}{rgb}{0.000000,0.000000,0.000000}%
\pgfsetstrokecolor{currentstroke}%
\pgfsetstrokeopacity{0.000000}%
\pgfsetdash{}{0pt}%
\pgfpathmoveto{\pgfqpoint{1.063857in}{1.613090in}}%
\pgfpathlineto{\pgfqpoint{1.072610in}{1.613090in}}%
\pgfpathlineto{\pgfqpoint{1.072610in}{1.567700in}}%
\pgfpathlineto{\pgfqpoint{1.063857in}{1.567700in}}%
\pgfpathlineto{\pgfqpoint{1.063857in}{1.613090in}}%
\pgfpathclose%
\pgfusepath{fill}%
\end{pgfscope}%
\begin{pgfscope}%
\pgfpathrectangle{\pgfqpoint{0.804646in}{0.600000in}}{\pgfqpoint{2.573292in}{2.070576in}}%
\pgfusepath{clip}%
\pgfsetbuttcap%
\pgfsetmiterjoin%
\definecolor{currentfill}{rgb}{0.066899,0.263188,0.377594}%
\pgfsetfillcolor{currentfill}%
\pgfsetlinewidth{0.000000pt}%
\definecolor{currentstroke}{rgb}{0.000000,0.000000,0.000000}%
\pgfsetstrokecolor{currentstroke}%
\pgfsetstrokeopacity{0.000000}%
\pgfsetdash{}{0pt}%
\pgfpathmoveto{\pgfqpoint{1.074799in}{1.613090in}}%
\pgfpathlineto{\pgfqpoint{1.083552in}{1.613090in}}%
\pgfpathlineto{\pgfqpoint{1.083552in}{1.562462in}}%
\pgfpathlineto{\pgfqpoint{1.074799in}{1.562462in}}%
\pgfpathlineto{\pgfqpoint{1.074799in}{1.613090in}}%
\pgfpathclose%
\pgfusepath{fill}%
\end{pgfscope}%
\begin{pgfscope}%
\pgfpathrectangle{\pgfqpoint{0.804646in}{0.600000in}}{\pgfqpoint{2.573292in}{2.070576in}}%
\pgfusepath{clip}%
\pgfsetbuttcap%
\pgfsetmiterjoin%
\definecolor{currentfill}{rgb}{0.066899,0.263188,0.377594}%
\pgfsetfillcolor{currentfill}%
\pgfsetlinewidth{0.000000pt}%
\definecolor{currentstroke}{rgb}{0.000000,0.000000,0.000000}%
\pgfsetstrokecolor{currentstroke}%
\pgfsetstrokeopacity{0.000000}%
\pgfsetdash{}{0pt}%
\pgfpathmoveto{\pgfqpoint{1.085741in}{1.613090in}}%
\pgfpathlineto{\pgfqpoint{1.094494in}{1.613090in}}%
\pgfpathlineto{\pgfqpoint{1.094494in}{1.512132in}}%
\pgfpathlineto{\pgfqpoint{1.085741in}{1.512132in}}%
\pgfpathlineto{\pgfqpoint{1.085741in}{1.613090in}}%
\pgfpathclose%
\pgfusepath{fill}%
\end{pgfscope}%
\begin{pgfscope}%
\pgfpathrectangle{\pgfqpoint{0.804646in}{0.600000in}}{\pgfqpoint{2.573292in}{2.070576in}}%
\pgfusepath{clip}%
\pgfsetbuttcap%
\pgfsetmiterjoin%
\definecolor{currentfill}{rgb}{0.066899,0.263188,0.377594}%
\pgfsetfillcolor{currentfill}%
\pgfsetlinewidth{0.000000pt}%
\definecolor{currentstroke}{rgb}{0.000000,0.000000,0.000000}%
\pgfsetstrokecolor{currentstroke}%
\pgfsetstrokeopacity{0.000000}%
\pgfsetdash{}{0pt}%
\pgfpathmoveto{\pgfqpoint{1.096682in}{1.613090in}}%
\pgfpathlineto{\pgfqpoint{1.105436in}{1.613090in}}%
\pgfpathlineto{\pgfqpoint{1.105436in}{1.493277in}}%
\pgfpathlineto{\pgfqpoint{1.096682in}{1.493277in}}%
\pgfpathlineto{\pgfqpoint{1.096682in}{1.613090in}}%
\pgfpathclose%
\pgfusepath{fill}%
\end{pgfscope}%
\begin{pgfscope}%
\pgfpathrectangle{\pgfqpoint{0.804646in}{0.600000in}}{\pgfqpoint{2.573292in}{2.070576in}}%
\pgfusepath{clip}%
\pgfsetbuttcap%
\pgfsetmiterjoin%
\definecolor{currentfill}{rgb}{0.066899,0.263188,0.377594}%
\pgfsetfillcolor{currentfill}%
\pgfsetlinewidth{0.000000pt}%
\definecolor{currentstroke}{rgb}{0.000000,0.000000,0.000000}%
\pgfsetstrokecolor{currentstroke}%
\pgfsetstrokeopacity{0.000000}%
\pgfsetdash{}{0pt}%
\pgfpathmoveto{\pgfqpoint{1.107624in}{1.613090in}}%
\pgfpathlineto{\pgfqpoint{1.116378in}{1.613090in}}%
\pgfpathlineto{\pgfqpoint{1.116378in}{1.461982in}}%
\pgfpathlineto{\pgfqpoint{1.107624in}{1.461982in}}%
\pgfpathlineto{\pgfqpoint{1.107624in}{1.613090in}}%
\pgfpathclose%
\pgfusepath{fill}%
\end{pgfscope}%
\begin{pgfscope}%
\pgfpathrectangle{\pgfqpoint{0.804646in}{0.600000in}}{\pgfqpoint{2.573292in}{2.070576in}}%
\pgfusepath{clip}%
\pgfsetbuttcap%
\pgfsetmiterjoin%
\definecolor{currentfill}{rgb}{0.066899,0.263188,0.377594}%
\pgfsetfillcolor{currentfill}%
\pgfsetlinewidth{0.000000pt}%
\definecolor{currentstroke}{rgb}{0.000000,0.000000,0.000000}%
\pgfsetstrokecolor{currentstroke}%
\pgfsetstrokeopacity{0.000000}%
\pgfsetdash{}{0pt}%
\pgfpathmoveto{\pgfqpoint{1.118566in}{1.613090in}}%
\pgfpathlineto{\pgfqpoint{1.127319in}{1.613090in}}%
\pgfpathlineto{\pgfqpoint{1.127319in}{1.432183in}}%
\pgfpathlineto{\pgfqpoint{1.118566in}{1.432183in}}%
\pgfpathlineto{\pgfqpoint{1.118566in}{1.613090in}}%
\pgfpathclose%
\pgfusepath{fill}%
\end{pgfscope}%
\begin{pgfscope}%
\pgfpathrectangle{\pgfqpoint{0.804646in}{0.600000in}}{\pgfqpoint{2.573292in}{2.070576in}}%
\pgfusepath{clip}%
\pgfsetbuttcap%
\pgfsetmiterjoin%
\definecolor{currentfill}{rgb}{0.066899,0.263188,0.377594}%
\pgfsetfillcolor{currentfill}%
\pgfsetlinewidth{0.000000pt}%
\definecolor{currentstroke}{rgb}{0.000000,0.000000,0.000000}%
\pgfsetstrokecolor{currentstroke}%
\pgfsetstrokeopacity{0.000000}%
\pgfsetdash{}{0pt}%
\pgfpathmoveto{\pgfqpoint{1.129508in}{1.613090in}}%
\pgfpathlineto{\pgfqpoint{1.138261in}{1.613090in}}%
\pgfpathlineto{\pgfqpoint{1.138261in}{1.442001in}}%
\pgfpathlineto{\pgfqpoint{1.129508in}{1.442001in}}%
\pgfpathlineto{\pgfqpoint{1.129508in}{1.613090in}}%
\pgfpathclose%
\pgfusepath{fill}%
\end{pgfscope}%
\begin{pgfscope}%
\pgfpathrectangle{\pgfqpoint{0.804646in}{0.600000in}}{\pgfqpoint{2.573292in}{2.070576in}}%
\pgfusepath{clip}%
\pgfsetbuttcap%
\pgfsetmiterjoin%
\definecolor{currentfill}{rgb}{0.066899,0.263188,0.377594}%
\pgfsetfillcolor{currentfill}%
\pgfsetlinewidth{0.000000pt}%
\definecolor{currentstroke}{rgb}{0.000000,0.000000,0.000000}%
\pgfsetstrokecolor{currentstroke}%
\pgfsetstrokeopacity{0.000000}%
\pgfsetdash{}{0pt}%
\pgfpathmoveto{\pgfqpoint{1.140450in}{1.613090in}}%
\pgfpathlineto{\pgfqpoint{1.149203in}{1.613090in}}%
\pgfpathlineto{\pgfqpoint{1.149203in}{1.438451in}}%
\pgfpathlineto{\pgfqpoint{1.140450in}{1.438451in}}%
\pgfpathlineto{\pgfqpoint{1.140450in}{1.613090in}}%
\pgfpathclose%
\pgfusepath{fill}%
\end{pgfscope}%
\begin{pgfscope}%
\pgfpathrectangle{\pgfqpoint{0.804646in}{0.600000in}}{\pgfqpoint{2.573292in}{2.070576in}}%
\pgfusepath{clip}%
\pgfsetbuttcap%
\pgfsetmiterjoin%
\definecolor{currentfill}{rgb}{0.066899,0.263188,0.377594}%
\pgfsetfillcolor{currentfill}%
\pgfsetlinewidth{0.000000pt}%
\definecolor{currentstroke}{rgb}{0.000000,0.000000,0.000000}%
\pgfsetstrokecolor{currentstroke}%
\pgfsetstrokeopacity{0.000000}%
\pgfsetdash{}{0pt}%
\pgfpathmoveto{\pgfqpoint{1.151391in}{1.613090in}}%
\pgfpathlineto{\pgfqpoint{1.160145in}{1.613090in}}%
\pgfpathlineto{\pgfqpoint{1.160145in}{1.462150in}}%
\pgfpathlineto{\pgfqpoint{1.151391in}{1.462150in}}%
\pgfpathlineto{\pgfqpoint{1.151391in}{1.613090in}}%
\pgfpathclose%
\pgfusepath{fill}%
\end{pgfscope}%
\begin{pgfscope}%
\pgfpathrectangle{\pgfqpoint{0.804646in}{0.600000in}}{\pgfqpoint{2.573292in}{2.070576in}}%
\pgfusepath{clip}%
\pgfsetbuttcap%
\pgfsetmiterjoin%
\definecolor{currentfill}{rgb}{0.066899,0.263188,0.377594}%
\pgfsetfillcolor{currentfill}%
\pgfsetlinewidth{0.000000pt}%
\definecolor{currentstroke}{rgb}{0.000000,0.000000,0.000000}%
\pgfsetstrokecolor{currentstroke}%
\pgfsetstrokeopacity{0.000000}%
\pgfsetdash{}{0pt}%
\pgfpathmoveto{\pgfqpoint{1.162333in}{1.613090in}}%
\pgfpathlineto{\pgfqpoint{1.171087in}{1.613090in}}%
\pgfpathlineto{\pgfqpoint{1.171087in}{1.435889in}}%
\pgfpathlineto{\pgfqpoint{1.162333in}{1.435889in}}%
\pgfpathlineto{\pgfqpoint{1.162333in}{1.613090in}}%
\pgfpathclose%
\pgfusepath{fill}%
\end{pgfscope}%
\begin{pgfscope}%
\pgfpathrectangle{\pgfqpoint{0.804646in}{0.600000in}}{\pgfqpoint{2.573292in}{2.070576in}}%
\pgfusepath{clip}%
\pgfsetbuttcap%
\pgfsetmiterjoin%
\definecolor{currentfill}{rgb}{0.066899,0.263188,0.377594}%
\pgfsetfillcolor{currentfill}%
\pgfsetlinewidth{0.000000pt}%
\definecolor{currentstroke}{rgb}{0.000000,0.000000,0.000000}%
\pgfsetstrokecolor{currentstroke}%
\pgfsetstrokeopacity{0.000000}%
\pgfsetdash{}{0pt}%
\pgfpathmoveto{\pgfqpoint{1.173275in}{1.613090in}}%
\pgfpathlineto{\pgfqpoint{1.182028in}{1.613090in}}%
\pgfpathlineto{\pgfqpoint{1.182028in}{1.483102in}}%
\pgfpathlineto{\pgfqpoint{1.173275in}{1.483102in}}%
\pgfpathlineto{\pgfqpoint{1.173275in}{1.613090in}}%
\pgfpathclose%
\pgfusepath{fill}%
\end{pgfscope}%
\begin{pgfscope}%
\pgfpathrectangle{\pgfqpoint{0.804646in}{0.600000in}}{\pgfqpoint{2.573292in}{2.070576in}}%
\pgfusepath{clip}%
\pgfsetbuttcap%
\pgfsetmiterjoin%
\definecolor{currentfill}{rgb}{0.066899,0.263188,0.377594}%
\pgfsetfillcolor{currentfill}%
\pgfsetlinewidth{0.000000pt}%
\definecolor{currentstroke}{rgb}{0.000000,0.000000,0.000000}%
\pgfsetstrokecolor{currentstroke}%
\pgfsetstrokeopacity{0.000000}%
\pgfsetdash{}{0pt}%
\pgfpathmoveto{\pgfqpoint{1.184217in}{1.613090in}}%
\pgfpathlineto{\pgfqpoint{1.192970in}{1.613090in}}%
\pgfpathlineto{\pgfqpoint{1.192970in}{1.494695in}}%
\pgfpathlineto{\pgfqpoint{1.184217in}{1.494695in}}%
\pgfpathlineto{\pgfqpoint{1.184217in}{1.613090in}}%
\pgfpathclose%
\pgfusepath{fill}%
\end{pgfscope}%
\begin{pgfscope}%
\pgfpathrectangle{\pgfqpoint{0.804646in}{0.600000in}}{\pgfqpoint{2.573292in}{2.070576in}}%
\pgfusepath{clip}%
\pgfsetbuttcap%
\pgfsetmiterjoin%
\definecolor{currentfill}{rgb}{0.066899,0.263188,0.377594}%
\pgfsetfillcolor{currentfill}%
\pgfsetlinewidth{0.000000pt}%
\definecolor{currentstroke}{rgb}{0.000000,0.000000,0.000000}%
\pgfsetstrokecolor{currentstroke}%
\pgfsetstrokeopacity{0.000000}%
\pgfsetdash{}{0pt}%
\pgfpathmoveto{\pgfqpoint{1.195159in}{1.613090in}}%
\pgfpathlineto{\pgfqpoint{1.203912in}{1.613090in}}%
\pgfpathlineto{\pgfqpoint{1.203912in}{1.502287in}}%
\pgfpathlineto{\pgfqpoint{1.195159in}{1.502287in}}%
\pgfpathlineto{\pgfqpoint{1.195159in}{1.613090in}}%
\pgfpathclose%
\pgfusepath{fill}%
\end{pgfscope}%
\begin{pgfscope}%
\pgfpathrectangle{\pgfqpoint{0.804646in}{0.600000in}}{\pgfqpoint{2.573292in}{2.070576in}}%
\pgfusepath{clip}%
\pgfsetbuttcap%
\pgfsetmiterjoin%
\definecolor{currentfill}{rgb}{0.066899,0.263188,0.377594}%
\pgfsetfillcolor{currentfill}%
\pgfsetlinewidth{0.000000pt}%
\definecolor{currentstroke}{rgb}{0.000000,0.000000,0.000000}%
\pgfsetstrokecolor{currentstroke}%
\pgfsetstrokeopacity{0.000000}%
\pgfsetdash{}{0pt}%
\pgfpathmoveto{\pgfqpoint{1.206100in}{1.613090in}}%
\pgfpathlineto{\pgfqpoint{1.214854in}{1.613090in}}%
\pgfpathlineto{\pgfqpoint{1.214854in}{1.535708in}}%
\pgfpathlineto{\pgfqpoint{1.206100in}{1.535708in}}%
\pgfpathlineto{\pgfqpoint{1.206100in}{1.613090in}}%
\pgfpathclose%
\pgfusepath{fill}%
\end{pgfscope}%
\begin{pgfscope}%
\pgfpathrectangle{\pgfqpoint{0.804646in}{0.600000in}}{\pgfqpoint{2.573292in}{2.070576in}}%
\pgfusepath{clip}%
\pgfsetbuttcap%
\pgfsetmiterjoin%
\definecolor{currentfill}{rgb}{0.066899,0.263188,0.377594}%
\pgfsetfillcolor{currentfill}%
\pgfsetlinewidth{0.000000pt}%
\definecolor{currentstroke}{rgb}{0.000000,0.000000,0.000000}%
\pgfsetstrokecolor{currentstroke}%
\pgfsetstrokeopacity{0.000000}%
\pgfsetdash{}{0pt}%
\pgfpathmoveto{\pgfqpoint{1.217042in}{1.613090in}}%
\pgfpathlineto{\pgfqpoint{1.225796in}{1.613090in}}%
\pgfpathlineto{\pgfqpoint{1.225796in}{1.555280in}}%
\pgfpathlineto{\pgfqpoint{1.217042in}{1.555280in}}%
\pgfpathlineto{\pgfqpoint{1.217042in}{1.613090in}}%
\pgfpathclose%
\pgfusepath{fill}%
\end{pgfscope}%
\begin{pgfscope}%
\pgfpathrectangle{\pgfqpoint{0.804646in}{0.600000in}}{\pgfqpoint{2.573292in}{2.070576in}}%
\pgfusepath{clip}%
\pgfsetbuttcap%
\pgfsetmiterjoin%
\definecolor{currentfill}{rgb}{0.066899,0.263188,0.377594}%
\pgfsetfillcolor{currentfill}%
\pgfsetlinewidth{0.000000pt}%
\definecolor{currentstroke}{rgb}{0.000000,0.000000,0.000000}%
\pgfsetstrokecolor{currentstroke}%
\pgfsetstrokeopacity{0.000000}%
\pgfsetdash{}{0pt}%
\pgfpathmoveto{\pgfqpoint{1.227984in}{1.613090in}}%
\pgfpathlineto{\pgfqpoint{1.236737in}{1.613090in}}%
\pgfpathlineto{\pgfqpoint{1.236737in}{1.577050in}}%
\pgfpathlineto{\pgfqpoint{1.227984in}{1.577050in}}%
\pgfpathlineto{\pgfqpoint{1.227984in}{1.613090in}}%
\pgfpathclose%
\pgfusepath{fill}%
\end{pgfscope}%
\begin{pgfscope}%
\pgfpathrectangle{\pgfqpoint{0.804646in}{0.600000in}}{\pgfqpoint{2.573292in}{2.070576in}}%
\pgfusepath{clip}%
\pgfsetbuttcap%
\pgfsetmiterjoin%
\definecolor{currentfill}{rgb}{0.066899,0.263188,0.377594}%
\pgfsetfillcolor{currentfill}%
\pgfsetlinewidth{0.000000pt}%
\definecolor{currentstroke}{rgb}{0.000000,0.000000,0.000000}%
\pgfsetstrokecolor{currentstroke}%
\pgfsetstrokeopacity{0.000000}%
\pgfsetdash{}{0pt}%
\pgfpathmoveto{\pgfqpoint{1.238926in}{1.613090in}}%
\pgfpathlineto{\pgfqpoint{1.247679in}{1.613090in}}%
\pgfpathlineto{\pgfqpoint{1.247679in}{1.571032in}}%
\pgfpathlineto{\pgfqpoint{1.238926in}{1.571032in}}%
\pgfpathlineto{\pgfqpoint{1.238926in}{1.613090in}}%
\pgfpathclose%
\pgfusepath{fill}%
\end{pgfscope}%
\begin{pgfscope}%
\pgfpathrectangle{\pgfqpoint{0.804646in}{0.600000in}}{\pgfqpoint{2.573292in}{2.070576in}}%
\pgfusepath{clip}%
\pgfsetbuttcap%
\pgfsetmiterjoin%
\definecolor{currentfill}{rgb}{0.066899,0.263188,0.377594}%
\pgfsetfillcolor{currentfill}%
\pgfsetlinewidth{0.000000pt}%
\definecolor{currentstroke}{rgb}{0.000000,0.000000,0.000000}%
\pgfsetstrokecolor{currentstroke}%
\pgfsetstrokeopacity{0.000000}%
\pgfsetdash{}{0pt}%
\pgfpathmoveto{\pgfqpoint{1.249868in}{1.613090in}}%
\pgfpathlineto{\pgfqpoint{1.258621in}{1.613090in}}%
\pgfpathlineto{\pgfqpoint{1.258621in}{1.558615in}}%
\pgfpathlineto{\pgfqpoint{1.249868in}{1.558615in}}%
\pgfpathlineto{\pgfqpoint{1.249868in}{1.613090in}}%
\pgfpathclose%
\pgfusepath{fill}%
\end{pgfscope}%
\begin{pgfscope}%
\pgfpathrectangle{\pgfqpoint{0.804646in}{0.600000in}}{\pgfqpoint{2.573292in}{2.070576in}}%
\pgfusepath{clip}%
\pgfsetbuttcap%
\pgfsetmiterjoin%
\definecolor{currentfill}{rgb}{0.066899,0.263188,0.377594}%
\pgfsetfillcolor{currentfill}%
\pgfsetlinewidth{0.000000pt}%
\definecolor{currentstroke}{rgb}{0.000000,0.000000,0.000000}%
\pgfsetstrokecolor{currentstroke}%
\pgfsetstrokeopacity{0.000000}%
\pgfsetdash{}{0pt}%
\pgfpathmoveto{\pgfqpoint{1.260809in}{1.613090in}}%
\pgfpathlineto{\pgfqpoint{1.269563in}{1.613090in}}%
\pgfpathlineto{\pgfqpoint{1.269563in}{1.560219in}}%
\pgfpathlineto{\pgfqpoint{1.260809in}{1.560219in}}%
\pgfpathlineto{\pgfqpoint{1.260809in}{1.613090in}}%
\pgfpathclose%
\pgfusepath{fill}%
\end{pgfscope}%
\begin{pgfscope}%
\pgfpathrectangle{\pgfqpoint{0.804646in}{0.600000in}}{\pgfqpoint{2.573292in}{2.070576in}}%
\pgfusepath{clip}%
\pgfsetbuttcap%
\pgfsetmiterjoin%
\definecolor{currentfill}{rgb}{0.066899,0.263188,0.377594}%
\pgfsetfillcolor{currentfill}%
\pgfsetlinewidth{0.000000pt}%
\definecolor{currentstroke}{rgb}{0.000000,0.000000,0.000000}%
\pgfsetstrokecolor{currentstroke}%
\pgfsetstrokeopacity{0.000000}%
\pgfsetdash{}{0pt}%
\pgfpathmoveto{\pgfqpoint{1.271751in}{1.613090in}}%
\pgfpathlineto{\pgfqpoint{1.280505in}{1.613090in}}%
\pgfpathlineto{\pgfqpoint{1.280505in}{1.538537in}}%
\pgfpathlineto{\pgfqpoint{1.271751in}{1.538537in}}%
\pgfpathlineto{\pgfqpoint{1.271751in}{1.613090in}}%
\pgfpathclose%
\pgfusepath{fill}%
\end{pgfscope}%
\begin{pgfscope}%
\pgfpathrectangle{\pgfqpoint{0.804646in}{0.600000in}}{\pgfqpoint{2.573292in}{2.070576in}}%
\pgfusepath{clip}%
\pgfsetbuttcap%
\pgfsetmiterjoin%
\definecolor{currentfill}{rgb}{0.066899,0.263188,0.377594}%
\pgfsetfillcolor{currentfill}%
\pgfsetlinewidth{0.000000pt}%
\definecolor{currentstroke}{rgb}{0.000000,0.000000,0.000000}%
\pgfsetstrokecolor{currentstroke}%
\pgfsetstrokeopacity{0.000000}%
\pgfsetdash{}{0pt}%
\pgfpathmoveto{\pgfqpoint{1.282693in}{1.613090in}}%
\pgfpathlineto{\pgfqpoint{1.291446in}{1.613090in}}%
\pgfpathlineto{\pgfqpoint{1.291446in}{1.482861in}}%
\pgfpathlineto{\pgfqpoint{1.282693in}{1.482861in}}%
\pgfpathlineto{\pgfqpoint{1.282693in}{1.613090in}}%
\pgfpathclose%
\pgfusepath{fill}%
\end{pgfscope}%
\begin{pgfscope}%
\pgfpathrectangle{\pgfqpoint{0.804646in}{0.600000in}}{\pgfqpoint{2.573292in}{2.070576in}}%
\pgfusepath{clip}%
\pgfsetbuttcap%
\pgfsetmiterjoin%
\definecolor{currentfill}{rgb}{0.066899,0.263188,0.377594}%
\pgfsetfillcolor{currentfill}%
\pgfsetlinewidth{0.000000pt}%
\definecolor{currentstroke}{rgb}{0.000000,0.000000,0.000000}%
\pgfsetstrokecolor{currentstroke}%
\pgfsetstrokeopacity{0.000000}%
\pgfsetdash{}{0pt}%
\pgfpathmoveto{\pgfqpoint{1.293635in}{1.613090in}}%
\pgfpathlineto{\pgfqpoint{1.302388in}{1.613090in}}%
\pgfpathlineto{\pgfqpoint{1.302388in}{1.417427in}}%
\pgfpathlineto{\pgfqpoint{1.293635in}{1.417427in}}%
\pgfpathlineto{\pgfqpoint{1.293635in}{1.613090in}}%
\pgfpathclose%
\pgfusepath{fill}%
\end{pgfscope}%
\begin{pgfscope}%
\pgfpathrectangle{\pgfqpoint{0.804646in}{0.600000in}}{\pgfqpoint{2.573292in}{2.070576in}}%
\pgfusepath{clip}%
\pgfsetbuttcap%
\pgfsetmiterjoin%
\definecolor{currentfill}{rgb}{0.066899,0.263188,0.377594}%
\pgfsetfillcolor{currentfill}%
\pgfsetlinewidth{0.000000pt}%
\definecolor{currentstroke}{rgb}{0.000000,0.000000,0.000000}%
\pgfsetstrokecolor{currentstroke}%
\pgfsetstrokeopacity{0.000000}%
\pgfsetdash{}{0pt}%
\pgfpathmoveto{\pgfqpoint{1.304577in}{1.613090in}}%
\pgfpathlineto{\pgfqpoint{1.313330in}{1.613090in}}%
\pgfpathlineto{\pgfqpoint{1.313330in}{1.384983in}}%
\pgfpathlineto{\pgfqpoint{1.304577in}{1.384983in}}%
\pgfpathlineto{\pgfqpoint{1.304577in}{1.613090in}}%
\pgfpathclose%
\pgfusepath{fill}%
\end{pgfscope}%
\begin{pgfscope}%
\pgfpathrectangle{\pgfqpoint{0.804646in}{0.600000in}}{\pgfqpoint{2.573292in}{2.070576in}}%
\pgfusepath{clip}%
\pgfsetbuttcap%
\pgfsetmiterjoin%
\definecolor{currentfill}{rgb}{0.066899,0.263188,0.377594}%
\pgfsetfillcolor{currentfill}%
\pgfsetlinewidth{0.000000pt}%
\definecolor{currentstroke}{rgb}{0.000000,0.000000,0.000000}%
\pgfsetstrokecolor{currentstroke}%
\pgfsetstrokeopacity{0.000000}%
\pgfsetdash{}{0pt}%
\pgfpathmoveto{\pgfqpoint{1.315518in}{1.613090in}}%
\pgfpathlineto{\pgfqpoint{1.324272in}{1.613090in}}%
\pgfpathlineto{\pgfqpoint{1.324272in}{1.401926in}}%
\pgfpathlineto{\pgfqpoint{1.315518in}{1.401926in}}%
\pgfpathlineto{\pgfqpoint{1.315518in}{1.613090in}}%
\pgfpathclose%
\pgfusepath{fill}%
\end{pgfscope}%
\begin{pgfscope}%
\pgfpathrectangle{\pgfqpoint{0.804646in}{0.600000in}}{\pgfqpoint{2.573292in}{2.070576in}}%
\pgfusepath{clip}%
\pgfsetbuttcap%
\pgfsetmiterjoin%
\definecolor{currentfill}{rgb}{0.066899,0.263188,0.377594}%
\pgfsetfillcolor{currentfill}%
\pgfsetlinewidth{0.000000pt}%
\definecolor{currentstroke}{rgb}{0.000000,0.000000,0.000000}%
\pgfsetstrokecolor{currentstroke}%
\pgfsetstrokeopacity{0.000000}%
\pgfsetdash{}{0pt}%
\pgfpathmoveto{\pgfqpoint{1.326460in}{1.613090in}}%
\pgfpathlineto{\pgfqpoint{1.335214in}{1.613090in}}%
\pgfpathlineto{\pgfqpoint{1.335214in}{1.420210in}}%
\pgfpathlineto{\pgfqpoint{1.326460in}{1.420210in}}%
\pgfpathlineto{\pgfqpoint{1.326460in}{1.613090in}}%
\pgfpathclose%
\pgfusepath{fill}%
\end{pgfscope}%
\begin{pgfscope}%
\pgfpathrectangle{\pgfqpoint{0.804646in}{0.600000in}}{\pgfqpoint{2.573292in}{2.070576in}}%
\pgfusepath{clip}%
\pgfsetbuttcap%
\pgfsetmiterjoin%
\definecolor{currentfill}{rgb}{0.066899,0.263188,0.377594}%
\pgfsetfillcolor{currentfill}%
\pgfsetlinewidth{0.000000pt}%
\definecolor{currentstroke}{rgb}{0.000000,0.000000,0.000000}%
\pgfsetstrokecolor{currentstroke}%
\pgfsetstrokeopacity{0.000000}%
\pgfsetdash{}{0pt}%
\pgfpathmoveto{\pgfqpoint{1.337402in}{1.613090in}}%
\pgfpathlineto{\pgfqpoint{1.346155in}{1.613090in}}%
\pgfpathlineto{\pgfqpoint{1.346155in}{1.461598in}}%
\pgfpathlineto{\pgfqpoint{1.337402in}{1.461598in}}%
\pgfpathlineto{\pgfqpoint{1.337402in}{1.613090in}}%
\pgfpathclose%
\pgfusepath{fill}%
\end{pgfscope}%
\begin{pgfscope}%
\pgfpathrectangle{\pgfqpoint{0.804646in}{0.600000in}}{\pgfqpoint{2.573292in}{2.070576in}}%
\pgfusepath{clip}%
\pgfsetbuttcap%
\pgfsetmiterjoin%
\definecolor{currentfill}{rgb}{0.066899,0.263188,0.377594}%
\pgfsetfillcolor{currentfill}%
\pgfsetlinewidth{0.000000pt}%
\definecolor{currentstroke}{rgb}{0.000000,0.000000,0.000000}%
\pgfsetstrokecolor{currentstroke}%
\pgfsetstrokeopacity{0.000000}%
\pgfsetdash{}{0pt}%
\pgfpathmoveto{\pgfqpoint{1.348344in}{1.613090in}}%
\pgfpathlineto{\pgfqpoint{1.357097in}{1.613090in}}%
\pgfpathlineto{\pgfqpoint{1.357097in}{1.455611in}}%
\pgfpathlineto{\pgfqpoint{1.348344in}{1.455611in}}%
\pgfpathlineto{\pgfqpoint{1.348344in}{1.613090in}}%
\pgfpathclose%
\pgfusepath{fill}%
\end{pgfscope}%
\begin{pgfscope}%
\pgfpathrectangle{\pgfqpoint{0.804646in}{0.600000in}}{\pgfqpoint{2.573292in}{2.070576in}}%
\pgfusepath{clip}%
\pgfsetbuttcap%
\pgfsetmiterjoin%
\definecolor{currentfill}{rgb}{0.066899,0.263188,0.377594}%
\pgfsetfillcolor{currentfill}%
\pgfsetlinewidth{0.000000pt}%
\definecolor{currentstroke}{rgb}{0.000000,0.000000,0.000000}%
\pgfsetstrokecolor{currentstroke}%
\pgfsetstrokeopacity{0.000000}%
\pgfsetdash{}{0pt}%
\pgfpathmoveto{\pgfqpoint{1.359286in}{1.613090in}}%
\pgfpathlineto{\pgfqpoint{1.368039in}{1.613090in}}%
\pgfpathlineto{\pgfqpoint{1.368039in}{1.458254in}}%
\pgfpathlineto{\pgfqpoint{1.359286in}{1.458254in}}%
\pgfpathlineto{\pgfqpoint{1.359286in}{1.613090in}}%
\pgfpathclose%
\pgfusepath{fill}%
\end{pgfscope}%
\begin{pgfscope}%
\pgfpathrectangle{\pgfqpoint{0.804646in}{0.600000in}}{\pgfqpoint{2.573292in}{2.070576in}}%
\pgfusepath{clip}%
\pgfsetbuttcap%
\pgfsetmiterjoin%
\definecolor{currentfill}{rgb}{0.066899,0.263188,0.377594}%
\pgfsetfillcolor{currentfill}%
\pgfsetlinewidth{0.000000pt}%
\definecolor{currentstroke}{rgb}{0.000000,0.000000,0.000000}%
\pgfsetstrokecolor{currentstroke}%
\pgfsetstrokeopacity{0.000000}%
\pgfsetdash{}{0pt}%
\pgfpathmoveto{\pgfqpoint{1.370227in}{1.613090in}}%
\pgfpathlineto{\pgfqpoint{1.378981in}{1.613090in}}%
\pgfpathlineto{\pgfqpoint{1.378981in}{1.448702in}}%
\pgfpathlineto{\pgfqpoint{1.370227in}{1.448702in}}%
\pgfpathlineto{\pgfqpoint{1.370227in}{1.613090in}}%
\pgfpathclose%
\pgfusepath{fill}%
\end{pgfscope}%
\begin{pgfscope}%
\pgfpathrectangle{\pgfqpoint{0.804646in}{0.600000in}}{\pgfqpoint{2.573292in}{2.070576in}}%
\pgfusepath{clip}%
\pgfsetbuttcap%
\pgfsetmiterjoin%
\definecolor{currentfill}{rgb}{0.066899,0.263188,0.377594}%
\pgfsetfillcolor{currentfill}%
\pgfsetlinewidth{0.000000pt}%
\definecolor{currentstroke}{rgb}{0.000000,0.000000,0.000000}%
\pgfsetstrokecolor{currentstroke}%
\pgfsetstrokeopacity{0.000000}%
\pgfsetdash{}{0pt}%
\pgfpathmoveto{\pgfqpoint{1.381169in}{1.613090in}}%
\pgfpathlineto{\pgfqpoint{1.389923in}{1.613090in}}%
\pgfpathlineto{\pgfqpoint{1.389923in}{1.469662in}}%
\pgfpathlineto{\pgfqpoint{1.381169in}{1.469662in}}%
\pgfpathlineto{\pgfqpoint{1.381169in}{1.613090in}}%
\pgfpathclose%
\pgfusepath{fill}%
\end{pgfscope}%
\begin{pgfscope}%
\pgfpathrectangle{\pgfqpoint{0.804646in}{0.600000in}}{\pgfqpoint{2.573292in}{2.070576in}}%
\pgfusepath{clip}%
\pgfsetbuttcap%
\pgfsetmiterjoin%
\definecolor{currentfill}{rgb}{0.066899,0.263188,0.377594}%
\pgfsetfillcolor{currentfill}%
\pgfsetlinewidth{0.000000pt}%
\definecolor{currentstroke}{rgb}{0.000000,0.000000,0.000000}%
\pgfsetstrokecolor{currentstroke}%
\pgfsetstrokeopacity{0.000000}%
\pgfsetdash{}{0pt}%
\pgfpathmoveto{\pgfqpoint{1.392111in}{1.613090in}}%
\pgfpathlineto{\pgfqpoint{1.400864in}{1.613090in}}%
\pgfpathlineto{\pgfqpoint{1.400864in}{1.505376in}}%
\pgfpathlineto{\pgfqpoint{1.392111in}{1.505376in}}%
\pgfpathlineto{\pgfqpoint{1.392111in}{1.613090in}}%
\pgfpathclose%
\pgfusepath{fill}%
\end{pgfscope}%
\begin{pgfscope}%
\pgfpathrectangle{\pgfqpoint{0.804646in}{0.600000in}}{\pgfqpoint{2.573292in}{2.070576in}}%
\pgfusepath{clip}%
\pgfsetbuttcap%
\pgfsetmiterjoin%
\definecolor{currentfill}{rgb}{0.066899,0.263188,0.377594}%
\pgfsetfillcolor{currentfill}%
\pgfsetlinewidth{0.000000pt}%
\definecolor{currentstroke}{rgb}{0.000000,0.000000,0.000000}%
\pgfsetstrokecolor{currentstroke}%
\pgfsetstrokeopacity{0.000000}%
\pgfsetdash{}{0pt}%
\pgfpathmoveto{\pgfqpoint{1.403053in}{1.613090in}}%
\pgfpathlineto{\pgfqpoint{1.411806in}{1.613090in}}%
\pgfpathlineto{\pgfqpoint{1.411806in}{1.525212in}}%
\pgfpathlineto{\pgfqpoint{1.403053in}{1.525212in}}%
\pgfpathlineto{\pgfqpoint{1.403053in}{1.613090in}}%
\pgfpathclose%
\pgfusepath{fill}%
\end{pgfscope}%
\begin{pgfscope}%
\pgfpathrectangle{\pgfqpoint{0.804646in}{0.600000in}}{\pgfqpoint{2.573292in}{2.070576in}}%
\pgfusepath{clip}%
\pgfsetbuttcap%
\pgfsetmiterjoin%
\definecolor{currentfill}{rgb}{0.066899,0.263188,0.377594}%
\pgfsetfillcolor{currentfill}%
\pgfsetlinewidth{0.000000pt}%
\definecolor{currentstroke}{rgb}{0.000000,0.000000,0.000000}%
\pgfsetstrokecolor{currentstroke}%
\pgfsetstrokeopacity{0.000000}%
\pgfsetdash{}{0pt}%
\pgfpathmoveto{\pgfqpoint{1.413995in}{1.613090in}}%
\pgfpathlineto{\pgfqpoint{1.422748in}{1.613090in}}%
\pgfpathlineto{\pgfqpoint{1.422748in}{1.536211in}}%
\pgfpathlineto{\pgfqpoint{1.413995in}{1.536211in}}%
\pgfpathlineto{\pgfqpoint{1.413995in}{1.613090in}}%
\pgfpathclose%
\pgfusepath{fill}%
\end{pgfscope}%
\begin{pgfscope}%
\pgfpathrectangle{\pgfqpoint{0.804646in}{0.600000in}}{\pgfqpoint{2.573292in}{2.070576in}}%
\pgfusepath{clip}%
\pgfsetbuttcap%
\pgfsetmiterjoin%
\definecolor{currentfill}{rgb}{0.066899,0.263188,0.377594}%
\pgfsetfillcolor{currentfill}%
\pgfsetlinewidth{0.000000pt}%
\definecolor{currentstroke}{rgb}{0.000000,0.000000,0.000000}%
\pgfsetstrokecolor{currentstroke}%
\pgfsetstrokeopacity{0.000000}%
\pgfsetdash{}{0pt}%
\pgfpathmoveto{\pgfqpoint{1.424936in}{1.613090in}}%
\pgfpathlineto{\pgfqpoint{1.433690in}{1.613090in}}%
\pgfpathlineto{\pgfqpoint{1.433690in}{1.541848in}}%
\pgfpathlineto{\pgfqpoint{1.424936in}{1.541848in}}%
\pgfpathlineto{\pgfqpoint{1.424936in}{1.613090in}}%
\pgfpathclose%
\pgfusepath{fill}%
\end{pgfscope}%
\begin{pgfscope}%
\pgfpathrectangle{\pgfqpoint{0.804646in}{0.600000in}}{\pgfqpoint{2.573292in}{2.070576in}}%
\pgfusepath{clip}%
\pgfsetbuttcap%
\pgfsetmiterjoin%
\definecolor{currentfill}{rgb}{0.066899,0.263188,0.377594}%
\pgfsetfillcolor{currentfill}%
\pgfsetlinewidth{0.000000pt}%
\definecolor{currentstroke}{rgb}{0.000000,0.000000,0.000000}%
\pgfsetstrokecolor{currentstroke}%
\pgfsetstrokeopacity{0.000000}%
\pgfsetdash{}{0pt}%
\pgfpathmoveto{\pgfqpoint{1.435878in}{1.613090in}}%
\pgfpathlineto{\pgfqpoint{1.444632in}{1.613090in}}%
\pgfpathlineto{\pgfqpoint{1.444632in}{1.593553in}}%
\pgfpathlineto{\pgfqpoint{1.435878in}{1.593553in}}%
\pgfpathlineto{\pgfqpoint{1.435878in}{1.613090in}}%
\pgfpathclose%
\pgfusepath{fill}%
\end{pgfscope}%
\begin{pgfscope}%
\pgfpathrectangle{\pgfqpoint{0.804646in}{0.600000in}}{\pgfqpoint{2.573292in}{2.070576in}}%
\pgfusepath{clip}%
\pgfsetbuttcap%
\pgfsetmiterjoin%
\definecolor{currentfill}{rgb}{0.066899,0.263188,0.377594}%
\pgfsetfillcolor{currentfill}%
\pgfsetlinewidth{0.000000pt}%
\definecolor{currentstroke}{rgb}{0.000000,0.000000,0.000000}%
\pgfsetstrokecolor{currentstroke}%
\pgfsetstrokeopacity{0.000000}%
\pgfsetdash{}{0pt}%
\pgfpathmoveto{\pgfqpoint{1.446820in}{1.613090in}}%
\pgfpathlineto{\pgfqpoint{1.455573in}{1.613090in}}%
\pgfpathlineto{\pgfqpoint{1.455573in}{1.613063in}}%
\pgfpathlineto{\pgfqpoint{1.446820in}{1.613063in}}%
\pgfpathlineto{\pgfqpoint{1.446820in}{1.613090in}}%
\pgfpathclose%
\pgfusepath{fill}%
\end{pgfscope}%
\begin{pgfscope}%
\pgfpathrectangle{\pgfqpoint{0.804646in}{0.600000in}}{\pgfqpoint{2.573292in}{2.070576in}}%
\pgfusepath{clip}%
\pgfsetbuttcap%
\pgfsetmiterjoin%
\definecolor{currentfill}{rgb}{0.066899,0.263188,0.377594}%
\pgfsetfillcolor{currentfill}%
\pgfsetlinewidth{0.000000pt}%
\definecolor{currentstroke}{rgb}{0.000000,0.000000,0.000000}%
\pgfsetstrokecolor{currentstroke}%
\pgfsetstrokeopacity{0.000000}%
\pgfsetdash{}{0pt}%
\pgfpathmoveto{\pgfqpoint{1.457762in}{1.613090in}}%
\pgfpathlineto{\pgfqpoint{1.466515in}{1.613090in}}%
\pgfpathlineto{\pgfqpoint{1.466515in}{1.630865in}}%
\pgfpathlineto{\pgfqpoint{1.457762in}{1.630865in}}%
\pgfpathlineto{\pgfqpoint{1.457762in}{1.613090in}}%
\pgfpathclose%
\pgfusepath{fill}%
\end{pgfscope}%
\begin{pgfscope}%
\pgfpathrectangle{\pgfqpoint{0.804646in}{0.600000in}}{\pgfqpoint{2.573292in}{2.070576in}}%
\pgfusepath{clip}%
\pgfsetbuttcap%
\pgfsetmiterjoin%
\definecolor{currentfill}{rgb}{0.066899,0.263188,0.377594}%
\pgfsetfillcolor{currentfill}%
\pgfsetlinewidth{0.000000pt}%
\definecolor{currentstroke}{rgb}{0.000000,0.000000,0.000000}%
\pgfsetstrokecolor{currentstroke}%
\pgfsetstrokeopacity{0.000000}%
\pgfsetdash{}{0pt}%
\pgfpathmoveto{\pgfqpoint{1.468704in}{1.613090in}}%
\pgfpathlineto{\pgfqpoint{1.477457in}{1.613090in}}%
\pgfpathlineto{\pgfqpoint{1.477457in}{1.654393in}}%
\pgfpathlineto{\pgfqpoint{1.468704in}{1.654393in}}%
\pgfpathlineto{\pgfqpoint{1.468704in}{1.613090in}}%
\pgfpathclose%
\pgfusepath{fill}%
\end{pgfscope}%
\begin{pgfscope}%
\pgfpathrectangle{\pgfqpoint{0.804646in}{0.600000in}}{\pgfqpoint{2.573292in}{2.070576in}}%
\pgfusepath{clip}%
\pgfsetbuttcap%
\pgfsetmiterjoin%
\definecolor{currentfill}{rgb}{0.066899,0.263188,0.377594}%
\pgfsetfillcolor{currentfill}%
\pgfsetlinewidth{0.000000pt}%
\definecolor{currentstroke}{rgb}{0.000000,0.000000,0.000000}%
\pgfsetstrokecolor{currentstroke}%
\pgfsetstrokeopacity{0.000000}%
\pgfsetdash{}{0pt}%
\pgfpathmoveto{\pgfqpoint{1.479645in}{1.613090in}}%
\pgfpathlineto{\pgfqpoint{1.488399in}{1.613090in}}%
\pgfpathlineto{\pgfqpoint{1.488399in}{1.652966in}}%
\pgfpathlineto{\pgfqpoint{1.479645in}{1.652966in}}%
\pgfpathlineto{\pgfqpoint{1.479645in}{1.613090in}}%
\pgfpathclose%
\pgfusepath{fill}%
\end{pgfscope}%
\begin{pgfscope}%
\pgfpathrectangle{\pgfqpoint{0.804646in}{0.600000in}}{\pgfqpoint{2.573292in}{2.070576in}}%
\pgfusepath{clip}%
\pgfsetbuttcap%
\pgfsetmiterjoin%
\definecolor{currentfill}{rgb}{0.066899,0.263188,0.377594}%
\pgfsetfillcolor{currentfill}%
\pgfsetlinewidth{0.000000pt}%
\definecolor{currentstroke}{rgb}{0.000000,0.000000,0.000000}%
\pgfsetstrokecolor{currentstroke}%
\pgfsetstrokeopacity{0.000000}%
\pgfsetdash{}{0pt}%
\pgfpathmoveto{\pgfqpoint{1.490587in}{1.613090in}}%
\pgfpathlineto{\pgfqpoint{1.499341in}{1.613090in}}%
\pgfpathlineto{\pgfqpoint{1.499341in}{1.658262in}}%
\pgfpathlineto{\pgfqpoint{1.490587in}{1.658262in}}%
\pgfpathlineto{\pgfqpoint{1.490587in}{1.613090in}}%
\pgfpathclose%
\pgfusepath{fill}%
\end{pgfscope}%
\begin{pgfscope}%
\pgfpathrectangle{\pgfqpoint{0.804646in}{0.600000in}}{\pgfqpoint{2.573292in}{2.070576in}}%
\pgfusepath{clip}%
\pgfsetbuttcap%
\pgfsetmiterjoin%
\definecolor{currentfill}{rgb}{0.066899,0.263188,0.377594}%
\pgfsetfillcolor{currentfill}%
\pgfsetlinewidth{0.000000pt}%
\definecolor{currentstroke}{rgb}{0.000000,0.000000,0.000000}%
\pgfsetstrokecolor{currentstroke}%
\pgfsetstrokeopacity{0.000000}%
\pgfsetdash{}{0pt}%
\pgfpathmoveto{\pgfqpoint{1.501529in}{1.613090in}}%
\pgfpathlineto{\pgfqpoint{1.510282in}{1.613090in}}%
\pgfpathlineto{\pgfqpoint{1.510282in}{1.695371in}}%
\pgfpathlineto{\pgfqpoint{1.501529in}{1.695371in}}%
\pgfpathlineto{\pgfqpoint{1.501529in}{1.613090in}}%
\pgfpathclose%
\pgfusepath{fill}%
\end{pgfscope}%
\begin{pgfscope}%
\pgfpathrectangle{\pgfqpoint{0.804646in}{0.600000in}}{\pgfqpoint{2.573292in}{2.070576in}}%
\pgfusepath{clip}%
\pgfsetbuttcap%
\pgfsetmiterjoin%
\definecolor{currentfill}{rgb}{0.066899,0.263188,0.377594}%
\pgfsetfillcolor{currentfill}%
\pgfsetlinewidth{0.000000pt}%
\definecolor{currentstroke}{rgb}{0.000000,0.000000,0.000000}%
\pgfsetstrokecolor{currentstroke}%
\pgfsetstrokeopacity{0.000000}%
\pgfsetdash{}{0pt}%
\pgfpathmoveto{\pgfqpoint{1.512471in}{1.613090in}}%
\pgfpathlineto{\pgfqpoint{1.521224in}{1.613090in}}%
\pgfpathlineto{\pgfqpoint{1.521224in}{1.686490in}}%
\pgfpathlineto{\pgfqpoint{1.512471in}{1.686490in}}%
\pgfpathlineto{\pgfqpoint{1.512471in}{1.613090in}}%
\pgfpathclose%
\pgfusepath{fill}%
\end{pgfscope}%
\begin{pgfscope}%
\pgfpathrectangle{\pgfqpoint{0.804646in}{0.600000in}}{\pgfqpoint{2.573292in}{2.070576in}}%
\pgfusepath{clip}%
\pgfsetbuttcap%
\pgfsetmiterjoin%
\definecolor{currentfill}{rgb}{0.066899,0.263188,0.377594}%
\pgfsetfillcolor{currentfill}%
\pgfsetlinewidth{0.000000pt}%
\definecolor{currentstroke}{rgb}{0.000000,0.000000,0.000000}%
\pgfsetstrokecolor{currentstroke}%
\pgfsetstrokeopacity{0.000000}%
\pgfsetdash{}{0pt}%
\pgfpathmoveto{\pgfqpoint{1.523413in}{1.613090in}}%
\pgfpathlineto{\pgfqpoint{1.532166in}{1.613090in}}%
\pgfpathlineto{\pgfqpoint{1.532166in}{1.637552in}}%
\pgfpathlineto{\pgfqpoint{1.523413in}{1.637552in}}%
\pgfpathlineto{\pgfqpoint{1.523413in}{1.613090in}}%
\pgfpathclose%
\pgfusepath{fill}%
\end{pgfscope}%
\begin{pgfscope}%
\pgfpathrectangle{\pgfqpoint{0.804646in}{0.600000in}}{\pgfqpoint{2.573292in}{2.070576in}}%
\pgfusepath{clip}%
\pgfsetbuttcap%
\pgfsetmiterjoin%
\definecolor{currentfill}{rgb}{0.066899,0.263188,0.377594}%
\pgfsetfillcolor{currentfill}%
\pgfsetlinewidth{0.000000pt}%
\definecolor{currentstroke}{rgb}{0.000000,0.000000,0.000000}%
\pgfsetstrokecolor{currentstroke}%
\pgfsetstrokeopacity{0.000000}%
\pgfsetdash{}{0pt}%
\pgfpathmoveto{\pgfqpoint{1.534354in}{1.613090in}}%
\pgfpathlineto{\pgfqpoint{1.543108in}{1.613090in}}%
\pgfpathlineto{\pgfqpoint{1.543108in}{1.586673in}}%
\pgfpathlineto{\pgfqpoint{1.534354in}{1.586673in}}%
\pgfpathlineto{\pgfqpoint{1.534354in}{1.613090in}}%
\pgfpathclose%
\pgfusepath{fill}%
\end{pgfscope}%
\begin{pgfscope}%
\pgfpathrectangle{\pgfqpoint{0.804646in}{0.600000in}}{\pgfqpoint{2.573292in}{2.070576in}}%
\pgfusepath{clip}%
\pgfsetbuttcap%
\pgfsetmiterjoin%
\definecolor{currentfill}{rgb}{0.066899,0.263188,0.377594}%
\pgfsetfillcolor{currentfill}%
\pgfsetlinewidth{0.000000pt}%
\definecolor{currentstroke}{rgb}{0.000000,0.000000,0.000000}%
\pgfsetstrokecolor{currentstroke}%
\pgfsetstrokeopacity{0.000000}%
\pgfsetdash{}{0pt}%
\pgfpathmoveto{\pgfqpoint{1.545296in}{1.613090in}}%
\pgfpathlineto{\pgfqpoint{1.554050in}{1.613090in}}%
\pgfpathlineto{\pgfqpoint{1.554050in}{1.613149in}}%
\pgfpathlineto{\pgfqpoint{1.545296in}{1.613149in}}%
\pgfpathlineto{\pgfqpoint{1.545296in}{1.613090in}}%
\pgfpathclose%
\pgfusepath{fill}%
\end{pgfscope}%
\begin{pgfscope}%
\pgfpathrectangle{\pgfqpoint{0.804646in}{0.600000in}}{\pgfqpoint{2.573292in}{2.070576in}}%
\pgfusepath{clip}%
\pgfsetbuttcap%
\pgfsetmiterjoin%
\definecolor{currentfill}{rgb}{0.066899,0.263188,0.377594}%
\pgfsetfillcolor{currentfill}%
\pgfsetlinewidth{0.000000pt}%
\definecolor{currentstroke}{rgb}{0.000000,0.000000,0.000000}%
\pgfsetstrokecolor{currentstroke}%
\pgfsetstrokeopacity{0.000000}%
\pgfsetdash{}{0pt}%
\pgfpathmoveto{\pgfqpoint{1.556238in}{1.613090in}}%
\pgfpathlineto{\pgfqpoint{1.564991in}{1.613090in}}%
\pgfpathlineto{\pgfqpoint{1.564991in}{1.643797in}}%
\pgfpathlineto{\pgfqpoint{1.556238in}{1.643797in}}%
\pgfpathlineto{\pgfqpoint{1.556238in}{1.613090in}}%
\pgfpathclose%
\pgfusepath{fill}%
\end{pgfscope}%
\begin{pgfscope}%
\pgfpathrectangle{\pgfqpoint{0.804646in}{0.600000in}}{\pgfqpoint{2.573292in}{2.070576in}}%
\pgfusepath{clip}%
\pgfsetbuttcap%
\pgfsetmiterjoin%
\definecolor{currentfill}{rgb}{0.066899,0.263188,0.377594}%
\pgfsetfillcolor{currentfill}%
\pgfsetlinewidth{0.000000pt}%
\definecolor{currentstroke}{rgb}{0.000000,0.000000,0.000000}%
\pgfsetstrokecolor{currentstroke}%
\pgfsetstrokeopacity{0.000000}%
\pgfsetdash{}{0pt}%
\pgfpathmoveto{\pgfqpoint{1.567180in}{1.613090in}}%
\pgfpathlineto{\pgfqpoint{1.575933in}{1.613090in}}%
\pgfpathlineto{\pgfqpoint{1.575933in}{1.638389in}}%
\pgfpathlineto{\pgfqpoint{1.567180in}{1.638389in}}%
\pgfpathlineto{\pgfqpoint{1.567180in}{1.613090in}}%
\pgfpathclose%
\pgfusepath{fill}%
\end{pgfscope}%
\begin{pgfscope}%
\pgfpathrectangle{\pgfqpoint{0.804646in}{0.600000in}}{\pgfqpoint{2.573292in}{2.070576in}}%
\pgfusepath{clip}%
\pgfsetbuttcap%
\pgfsetmiterjoin%
\definecolor{currentfill}{rgb}{0.066899,0.263188,0.377594}%
\pgfsetfillcolor{currentfill}%
\pgfsetlinewidth{0.000000pt}%
\definecolor{currentstroke}{rgb}{0.000000,0.000000,0.000000}%
\pgfsetstrokecolor{currentstroke}%
\pgfsetstrokeopacity{0.000000}%
\pgfsetdash{}{0pt}%
\pgfpathmoveto{\pgfqpoint{1.578122in}{1.613090in}}%
\pgfpathlineto{\pgfqpoint{1.586875in}{1.613090in}}%
\pgfpathlineto{\pgfqpoint{1.586875in}{1.640942in}}%
\pgfpathlineto{\pgfqpoint{1.578122in}{1.640942in}}%
\pgfpathlineto{\pgfqpoint{1.578122in}{1.613090in}}%
\pgfpathclose%
\pgfusepath{fill}%
\end{pgfscope}%
\begin{pgfscope}%
\pgfpathrectangle{\pgfqpoint{0.804646in}{0.600000in}}{\pgfqpoint{2.573292in}{2.070576in}}%
\pgfusepath{clip}%
\pgfsetbuttcap%
\pgfsetmiterjoin%
\definecolor{currentfill}{rgb}{0.066899,0.263188,0.377594}%
\pgfsetfillcolor{currentfill}%
\pgfsetlinewidth{0.000000pt}%
\definecolor{currentstroke}{rgb}{0.000000,0.000000,0.000000}%
\pgfsetstrokecolor{currentstroke}%
\pgfsetstrokeopacity{0.000000}%
\pgfsetdash{}{0pt}%
\pgfpathmoveto{\pgfqpoint{1.589063in}{1.613090in}}%
\pgfpathlineto{\pgfqpoint{1.597817in}{1.613090in}}%
\pgfpathlineto{\pgfqpoint{1.597817in}{1.601884in}}%
\pgfpathlineto{\pgfqpoint{1.589063in}{1.601884in}}%
\pgfpathlineto{\pgfqpoint{1.589063in}{1.613090in}}%
\pgfpathclose%
\pgfusepath{fill}%
\end{pgfscope}%
\begin{pgfscope}%
\pgfpathrectangle{\pgfqpoint{0.804646in}{0.600000in}}{\pgfqpoint{2.573292in}{2.070576in}}%
\pgfusepath{clip}%
\pgfsetbuttcap%
\pgfsetmiterjoin%
\definecolor{currentfill}{rgb}{0.066899,0.263188,0.377594}%
\pgfsetfillcolor{currentfill}%
\pgfsetlinewidth{0.000000pt}%
\definecolor{currentstroke}{rgb}{0.000000,0.000000,0.000000}%
\pgfsetstrokecolor{currentstroke}%
\pgfsetstrokeopacity{0.000000}%
\pgfsetdash{}{0pt}%
\pgfpathmoveto{\pgfqpoint{1.600005in}{1.613090in}}%
\pgfpathlineto{\pgfqpoint{1.608759in}{1.613090in}}%
\pgfpathlineto{\pgfqpoint{1.608759in}{1.559848in}}%
\pgfpathlineto{\pgfqpoint{1.600005in}{1.559848in}}%
\pgfpathlineto{\pgfqpoint{1.600005in}{1.613090in}}%
\pgfpathclose%
\pgfusepath{fill}%
\end{pgfscope}%
\begin{pgfscope}%
\pgfpathrectangle{\pgfqpoint{0.804646in}{0.600000in}}{\pgfqpoint{2.573292in}{2.070576in}}%
\pgfusepath{clip}%
\pgfsetbuttcap%
\pgfsetmiterjoin%
\definecolor{currentfill}{rgb}{0.066899,0.263188,0.377594}%
\pgfsetfillcolor{currentfill}%
\pgfsetlinewidth{0.000000pt}%
\definecolor{currentstroke}{rgb}{0.000000,0.000000,0.000000}%
\pgfsetstrokecolor{currentstroke}%
\pgfsetstrokeopacity{0.000000}%
\pgfsetdash{}{0pt}%
\pgfpathmoveto{\pgfqpoint{1.610947in}{1.613090in}}%
\pgfpathlineto{\pgfqpoint{1.619700in}{1.613090in}}%
\pgfpathlineto{\pgfqpoint{1.619700in}{1.580351in}}%
\pgfpathlineto{\pgfqpoint{1.610947in}{1.580351in}}%
\pgfpathlineto{\pgfqpoint{1.610947in}{1.613090in}}%
\pgfpathclose%
\pgfusepath{fill}%
\end{pgfscope}%
\begin{pgfscope}%
\pgfpathrectangle{\pgfqpoint{0.804646in}{0.600000in}}{\pgfqpoint{2.573292in}{2.070576in}}%
\pgfusepath{clip}%
\pgfsetbuttcap%
\pgfsetmiterjoin%
\definecolor{currentfill}{rgb}{0.066899,0.263188,0.377594}%
\pgfsetfillcolor{currentfill}%
\pgfsetlinewidth{0.000000pt}%
\definecolor{currentstroke}{rgb}{0.000000,0.000000,0.000000}%
\pgfsetstrokecolor{currentstroke}%
\pgfsetstrokeopacity{0.000000}%
\pgfsetdash{}{0pt}%
\pgfpathmoveto{\pgfqpoint{1.621889in}{1.613090in}}%
\pgfpathlineto{\pgfqpoint{1.630642in}{1.613090in}}%
\pgfpathlineto{\pgfqpoint{1.630642in}{1.535069in}}%
\pgfpathlineto{\pgfqpoint{1.621889in}{1.535069in}}%
\pgfpathlineto{\pgfqpoint{1.621889in}{1.613090in}}%
\pgfpathclose%
\pgfusepath{fill}%
\end{pgfscope}%
\begin{pgfscope}%
\pgfpathrectangle{\pgfqpoint{0.804646in}{0.600000in}}{\pgfqpoint{2.573292in}{2.070576in}}%
\pgfusepath{clip}%
\pgfsetbuttcap%
\pgfsetmiterjoin%
\definecolor{currentfill}{rgb}{0.066899,0.263188,0.377594}%
\pgfsetfillcolor{currentfill}%
\pgfsetlinewidth{0.000000pt}%
\definecolor{currentstroke}{rgb}{0.000000,0.000000,0.000000}%
\pgfsetstrokecolor{currentstroke}%
\pgfsetstrokeopacity{0.000000}%
\pgfsetdash{}{0pt}%
\pgfpathmoveto{\pgfqpoint{1.632831in}{1.613090in}}%
\pgfpathlineto{\pgfqpoint{1.641584in}{1.613090in}}%
\pgfpathlineto{\pgfqpoint{1.641584in}{1.515545in}}%
\pgfpathlineto{\pgfqpoint{1.632831in}{1.515545in}}%
\pgfpathlineto{\pgfqpoint{1.632831in}{1.613090in}}%
\pgfpathclose%
\pgfusepath{fill}%
\end{pgfscope}%
\begin{pgfscope}%
\pgfpathrectangle{\pgfqpoint{0.804646in}{0.600000in}}{\pgfqpoint{2.573292in}{2.070576in}}%
\pgfusepath{clip}%
\pgfsetbuttcap%
\pgfsetmiterjoin%
\definecolor{currentfill}{rgb}{0.066899,0.263188,0.377594}%
\pgfsetfillcolor{currentfill}%
\pgfsetlinewidth{0.000000pt}%
\definecolor{currentstroke}{rgb}{0.000000,0.000000,0.000000}%
\pgfsetstrokecolor{currentstroke}%
\pgfsetstrokeopacity{0.000000}%
\pgfsetdash{}{0pt}%
\pgfpathmoveto{\pgfqpoint{1.643772in}{1.613090in}}%
\pgfpathlineto{\pgfqpoint{1.652526in}{1.613090in}}%
\pgfpathlineto{\pgfqpoint{1.652526in}{1.513681in}}%
\pgfpathlineto{\pgfqpoint{1.643772in}{1.513681in}}%
\pgfpathlineto{\pgfqpoint{1.643772in}{1.613090in}}%
\pgfpathclose%
\pgfusepath{fill}%
\end{pgfscope}%
\begin{pgfscope}%
\pgfpathrectangle{\pgfqpoint{0.804646in}{0.600000in}}{\pgfqpoint{2.573292in}{2.070576in}}%
\pgfusepath{clip}%
\pgfsetbuttcap%
\pgfsetmiterjoin%
\definecolor{currentfill}{rgb}{0.066899,0.263188,0.377594}%
\pgfsetfillcolor{currentfill}%
\pgfsetlinewidth{0.000000pt}%
\definecolor{currentstroke}{rgb}{0.000000,0.000000,0.000000}%
\pgfsetstrokecolor{currentstroke}%
\pgfsetstrokeopacity{0.000000}%
\pgfsetdash{}{0pt}%
\pgfpathmoveto{\pgfqpoint{1.654714in}{1.613090in}}%
\pgfpathlineto{\pgfqpoint{1.663468in}{1.613090in}}%
\pgfpathlineto{\pgfqpoint{1.663468in}{1.533359in}}%
\pgfpathlineto{\pgfqpoint{1.654714in}{1.533359in}}%
\pgfpathlineto{\pgfqpoint{1.654714in}{1.613090in}}%
\pgfpathclose%
\pgfusepath{fill}%
\end{pgfscope}%
\begin{pgfscope}%
\pgfpathrectangle{\pgfqpoint{0.804646in}{0.600000in}}{\pgfqpoint{2.573292in}{2.070576in}}%
\pgfusepath{clip}%
\pgfsetbuttcap%
\pgfsetmiterjoin%
\definecolor{currentfill}{rgb}{0.066899,0.263188,0.377594}%
\pgfsetfillcolor{currentfill}%
\pgfsetlinewidth{0.000000pt}%
\definecolor{currentstroke}{rgb}{0.000000,0.000000,0.000000}%
\pgfsetstrokecolor{currentstroke}%
\pgfsetstrokeopacity{0.000000}%
\pgfsetdash{}{0pt}%
\pgfpathmoveto{\pgfqpoint{1.665656in}{1.613090in}}%
\pgfpathlineto{\pgfqpoint{1.674409in}{1.613090in}}%
\pgfpathlineto{\pgfqpoint{1.674409in}{1.571384in}}%
\pgfpathlineto{\pgfqpoint{1.665656in}{1.571384in}}%
\pgfpathlineto{\pgfqpoint{1.665656in}{1.613090in}}%
\pgfpathclose%
\pgfusepath{fill}%
\end{pgfscope}%
\begin{pgfscope}%
\pgfpathrectangle{\pgfqpoint{0.804646in}{0.600000in}}{\pgfqpoint{2.573292in}{2.070576in}}%
\pgfusepath{clip}%
\pgfsetbuttcap%
\pgfsetmiterjoin%
\definecolor{currentfill}{rgb}{0.066899,0.263188,0.377594}%
\pgfsetfillcolor{currentfill}%
\pgfsetlinewidth{0.000000pt}%
\definecolor{currentstroke}{rgb}{0.000000,0.000000,0.000000}%
\pgfsetstrokecolor{currentstroke}%
\pgfsetstrokeopacity{0.000000}%
\pgfsetdash{}{0pt}%
\pgfpathmoveto{\pgfqpoint{1.676598in}{1.613090in}}%
\pgfpathlineto{\pgfqpoint{1.685351in}{1.613090in}}%
\pgfpathlineto{\pgfqpoint{1.685351in}{1.608655in}}%
\pgfpathlineto{\pgfqpoint{1.676598in}{1.608655in}}%
\pgfpathlineto{\pgfqpoint{1.676598in}{1.613090in}}%
\pgfpathclose%
\pgfusepath{fill}%
\end{pgfscope}%
\begin{pgfscope}%
\pgfpathrectangle{\pgfqpoint{0.804646in}{0.600000in}}{\pgfqpoint{2.573292in}{2.070576in}}%
\pgfusepath{clip}%
\pgfsetbuttcap%
\pgfsetmiterjoin%
\definecolor{currentfill}{rgb}{0.066899,0.263188,0.377594}%
\pgfsetfillcolor{currentfill}%
\pgfsetlinewidth{0.000000pt}%
\definecolor{currentstroke}{rgb}{0.000000,0.000000,0.000000}%
\pgfsetstrokecolor{currentstroke}%
\pgfsetstrokeopacity{0.000000}%
\pgfsetdash{}{0pt}%
\pgfpathmoveto{\pgfqpoint{1.687540in}{1.613090in}}%
\pgfpathlineto{\pgfqpoint{1.696293in}{1.613090in}}%
\pgfpathlineto{\pgfqpoint{1.696293in}{1.642787in}}%
\pgfpathlineto{\pgfqpoint{1.687540in}{1.642787in}}%
\pgfpathlineto{\pgfqpoint{1.687540in}{1.613090in}}%
\pgfpathclose%
\pgfusepath{fill}%
\end{pgfscope}%
\begin{pgfscope}%
\pgfpathrectangle{\pgfqpoint{0.804646in}{0.600000in}}{\pgfqpoint{2.573292in}{2.070576in}}%
\pgfusepath{clip}%
\pgfsetbuttcap%
\pgfsetmiterjoin%
\definecolor{currentfill}{rgb}{0.066899,0.263188,0.377594}%
\pgfsetfillcolor{currentfill}%
\pgfsetlinewidth{0.000000pt}%
\definecolor{currentstroke}{rgb}{0.000000,0.000000,0.000000}%
\pgfsetstrokecolor{currentstroke}%
\pgfsetstrokeopacity{0.000000}%
\pgfsetdash{}{0pt}%
\pgfpathmoveto{\pgfqpoint{1.698481in}{1.613090in}}%
\pgfpathlineto{\pgfqpoint{1.707235in}{1.613090in}}%
\pgfpathlineto{\pgfqpoint{1.707235in}{1.665844in}}%
\pgfpathlineto{\pgfqpoint{1.698481in}{1.665844in}}%
\pgfpathlineto{\pgfqpoint{1.698481in}{1.613090in}}%
\pgfpathclose%
\pgfusepath{fill}%
\end{pgfscope}%
\begin{pgfscope}%
\pgfpathrectangle{\pgfqpoint{0.804646in}{0.600000in}}{\pgfqpoint{2.573292in}{2.070576in}}%
\pgfusepath{clip}%
\pgfsetbuttcap%
\pgfsetmiterjoin%
\definecolor{currentfill}{rgb}{0.066899,0.263188,0.377594}%
\pgfsetfillcolor{currentfill}%
\pgfsetlinewidth{0.000000pt}%
\definecolor{currentstroke}{rgb}{0.000000,0.000000,0.000000}%
\pgfsetstrokecolor{currentstroke}%
\pgfsetstrokeopacity{0.000000}%
\pgfsetdash{}{0pt}%
\pgfpathmoveto{\pgfqpoint{1.709423in}{1.613090in}}%
\pgfpathlineto{\pgfqpoint{1.718177in}{1.613090in}}%
\pgfpathlineto{\pgfqpoint{1.718177in}{1.673057in}}%
\pgfpathlineto{\pgfqpoint{1.709423in}{1.673057in}}%
\pgfpathlineto{\pgfqpoint{1.709423in}{1.613090in}}%
\pgfpathclose%
\pgfusepath{fill}%
\end{pgfscope}%
\begin{pgfscope}%
\pgfpathrectangle{\pgfqpoint{0.804646in}{0.600000in}}{\pgfqpoint{2.573292in}{2.070576in}}%
\pgfusepath{clip}%
\pgfsetbuttcap%
\pgfsetmiterjoin%
\definecolor{currentfill}{rgb}{0.066899,0.263188,0.377594}%
\pgfsetfillcolor{currentfill}%
\pgfsetlinewidth{0.000000pt}%
\definecolor{currentstroke}{rgb}{0.000000,0.000000,0.000000}%
\pgfsetstrokecolor{currentstroke}%
\pgfsetstrokeopacity{0.000000}%
\pgfsetdash{}{0pt}%
\pgfpathmoveto{\pgfqpoint{1.720365in}{1.613090in}}%
\pgfpathlineto{\pgfqpoint{1.729118in}{1.613090in}}%
\pgfpathlineto{\pgfqpoint{1.729118in}{1.676797in}}%
\pgfpathlineto{\pgfqpoint{1.720365in}{1.676797in}}%
\pgfpathlineto{\pgfqpoint{1.720365in}{1.613090in}}%
\pgfpathclose%
\pgfusepath{fill}%
\end{pgfscope}%
\begin{pgfscope}%
\pgfpathrectangle{\pgfqpoint{0.804646in}{0.600000in}}{\pgfqpoint{2.573292in}{2.070576in}}%
\pgfusepath{clip}%
\pgfsetbuttcap%
\pgfsetmiterjoin%
\definecolor{currentfill}{rgb}{0.066899,0.263188,0.377594}%
\pgfsetfillcolor{currentfill}%
\pgfsetlinewidth{0.000000pt}%
\definecolor{currentstroke}{rgb}{0.000000,0.000000,0.000000}%
\pgfsetstrokecolor{currentstroke}%
\pgfsetstrokeopacity{0.000000}%
\pgfsetdash{}{0pt}%
\pgfpathmoveto{\pgfqpoint{1.731307in}{1.613090in}}%
\pgfpathlineto{\pgfqpoint{1.740060in}{1.613090in}}%
\pgfpathlineto{\pgfqpoint{1.740060in}{1.688340in}}%
\pgfpathlineto{\pgfqpoint{1.731307in}{1.688340in}}%
\pgfpathlineto{\pgfqpoint{1.731307in}{1.613090in}}%
\pgfpathclose%
\pgfusepath{fill}%
\end{pgfscope}%
\begin{pgfscope}%
\pgfpathrectangle{\pgfqpoint{0.804646in}{0.600000in}}{\pgfqpoint{2.573292in}{2.070576in}}%
\pgfusepath{clip}%
\pgfsetbuttcap%
\pgfsetmiterjoin%
\definecolor{currentfill}{rgb}{0.066899,0.263188,0.377594}%
\pgfsetfillcolor{currentfill}%
\pgfsetlinewidth{0.000000pt}%
\definecolor{currentstroke}{rgb}{0.000000,0.000000,0.000000}%
\pgfsetstrokecolor{currentstroke}%
\pgfsetstrokeopacity{0.000000}%
\pgfsetdash{}{0pt}%
\pgfpathmoveto{\pgfqpoint{1.742249in}{1.613090in}}%
\pgfpathlineto{\pgfqpoint{1.751002in}{1.613090in}}%
\pgfpathlineto{\pgfqpoint{1.751002in}{1.697497in}}%
\pgfpathlineto{\pgfqpoint{1.742249in}{1.697497in}}%
\pgfpathlineto{\pgfqpoint{1.742249in}{1.613090in}}%
\pgfpathclose%
\pgfusepath{fill}%
\end{pgfscope}%
\begin{pgfscope}%
\pgfpathrectangle{\pgfqpoint{0.804646in}{0.600000in}}{\pgfqpoint{2.573292in}{2.070576in}}%
\pgfusepath{clip}%
\pgfsetbuttcap%
\pgfsetmiterjoin%
\definecolor{currentfill}{rgb}{0.066899,0.263188,0.377594}%
\pgfsetfillcolor{currentfill}%
\pgfsetlinewidth{0.000000pt}%
\definecolor{currentstroke}{rgb}{0.000000,0.000000,0.000000}%
\pgfsetstrokecolor{currentstroke}%
\pgfsetstrokeopacity{0.000000}%
\pgfsetdash{}{0pt}%
\pgfpathmoveto{\pgfqpoint{1.753190in}{1.613090in}}%
\pgfpathlineto{\pgfqpoint{1.761944in}{1.613090in}}%
\pgfpathlineto{\pgfqpoint{1.761944in}{1.703588in}}%
\pgfpathlineto{\pgfqpoint{1.753190in}{1.703588in}}%
\pgfpathlineto{\pgfqpoint{1.753190in}{1.613090in}}%
\pgfpathclose%
\pgfusepath{fill}%
\end{pgfscope}%
\begin{pgfscope}%
\pgfpathrectangle{\pgfqpoint{0.804646in}{0.600000in}}{\pgfqpoint{2.573292in}{2.070576in}}%
\pgfusepath{clip}%
\pgfsetbuttcap%
\pgfsetmiterjoin%
\definecolor{currentfill}{rgb}{0.066899,0.263188,0.377594}%
\pgfsetfillcolor{currentfill}%
\pgfsetlinewidth{0.000000pt}%
\definecolor{currentstroke}{rgb}{0.000000,0.000000,0.000000}%
\pgfsetstrokecolor{currentstroke}%
\pgfsetstrokeopacity{0.000000}%
\pgfsetdash{}{0pt}%
\pgfpathmoveto{\pgfqpoint{1.764132in}{1.613090in}}%
\pgfpathlineto{\pgfqpoint{1.772886in}{1.613090in}}%
\pgfpathlineto{\pgfqpoint{1.772886in}{1.705964in}}%
\pgfpathlineto{\pgfqpoint{1.764132in}{1.705964in}}%
\pgfpathlineto{\pgfqpoint{1.764132in}{1.613090in}}%
\pgfpathclose%
\pgfusepath{fill}%
\end{pgfscope}%
\begin{pgfscope}%
\pgfpathrectangle{\pgfqpoint{0.804646in}{0.600000in}}{\pgfqpoint{2.573292in}{2.070576in}}%
\pgfusepath{clip}%
\pgfsetbuttcap%
\pgfsetmiterjoin%
\definecolor{currentfill}{rgb}{0.066899,0.263188,0.377594}%
\pgfsetfillcolor{currentfill}%
\pgfsetlinewidth{0.000000pt}%
\definecolor{currentstroke}{rgb}{0.000000,0.000000,0.000000}%
\pgfsetstrokecolor{currentstroke}%
\pgfsetstrokeopacity{0.000000}%
\pgfsetdash{}{0pt}%
\pgfpathmoveto{\pgfqpoint{1.775074in}{1.613090in}}%
\pgfpathlineto{\pgfqpoint{1.783827in}{1.613090in}}%
\pgfpathlineto{\pgfqpoint{1.783827in}{1.710696in}}%
\pgfpathlineto{\pgfqpoint{1.775074in}{1.710696in}}%
\pgfpathlineto{\pgfqpoint{1.775074in}{1.613090in}}%
\pgfpathclose%
\pgfusepath{fill}%
\end{pgfscope}%
\begin{pgfscope}%
\pgfpathrectangle{\pgfqpoint{0.804646in}{0.600000in}}{\pgfqpoint{2.573292in}{2.070576in}}%
\pgfusepath{clip}%
\pgfsetbuttcap%
\pgfsetmiterjoin%
\definecolor{currentfill}{rgb}{0.066899,0.263188,0.377594}%
\pgfsetfillcolor{currentfill}%
\pgfsetlinewidth{0.000000pt}%
\definecolor{currentstroke}{rgb}{0.000000,0.000000,0.000000}%
\pgfsetstrokecolor{currentstroke}%
\pgfsetstrokeopacity{0.000000}%
\pgfsetdash{}{0pt}%
\pgfpathmoveto{\pgfqpoint{1.786016in}{1.613090in}}%
\pgfpathlineto{\pgfqpoint{1.794769in}{1.613090in}}%
\pgfpathlineto{\pgfqpoint{1.794769in}{1.692934in}}%
\pgfpathlineto{\pgfqpoint{1.786016in}{1.692934in}}%
\pgfpathlineto{\pgfqpoint{1.786016in}{1.613090in}}%
\pgfpathclose%
\pgfusepath{fill}%
\end{pgfscope}%
\begin{pgfscope}%
\pgfpathrectangle{\pgfqpoint{0.804646in}{0.600000in}}{\pgfqpoint{2.573292in}{2.070576in}}%
\pgfusepath{clip}%
\pgfsetbuttcap%
\pgfsetmiterjoin%
\definecolor{currentfill}{rgb}{0.066899,0.263188,0.377594}%
\pgfsetfillcolor{currentfill}%
\pgfsetlinewidth{0.000000pt}%
\definecolor{currentstroke}{rgb}{0.000000,0.000000,0.000000}%
\pgfsetstrokecolor{currentstroke}%
\pgfsetstrokeopacity{0.000000}%
\pgfsetdash{}{0pt}%
\pgfpathmoveto{\pgfqpoint{1.796958in}{1.613090in}}%
\pgfpathlineto{\pgfqpoint{1.805711in}{1.613090in}}%
\pgfpathlineto{\pgfqpoint{1.805711in}{1.696648in}}%
\pgfpathlineto{\pgfqpoint{1.796958in}{1.696648in}}%
\pgfpathlineto{\pgfqpoint{1.796958in}{1.613090in}}%
\pgfpathclose%
\pgfusepath{fill}%
\end{pgfscope}%
\begin{pgfscope}%
\pgfpathrectangle{\pgfqpoint{0.804646in}{0.600000in}}{\pgfqpoint{2.573292in}{2.070576in}}%
\pgfusepath{clip}%
\pgfsetbuttcap%
\pgfsetmiterjoin%
\definecolor{currentfill}{rgb}{0.066899,0.263188,0.377594}%
\pgfsetfillcolor{currentfill}%
\pgfsetlinewidth{0.000000pt}%
\definecolor{currentstroke}{rgb}{0.000000,0.000000,0.000000}%
\pgfsetstrokecolor{currentstroke}%
\pgfsetstrokeopacity{0.000000}%
\pgfsetdash{}{0pt}%
\pgfpathmoveto{\pgfqpoint{1.807899in}{1.613090in}}%
\pgfpathlineto{\pgfqpoint{1.816653in}{1.613090in}}%
\pgfpathlineto{\pgfqpoint{1.816653in}{1.701329in}}%
\pgfpathlineto{\pgfqpoint{1.807899in}{1.701329in}}%
\pgfpathlineto{\pgfqpoint{1.807899in}{1.613090in}}%
\pgfpathclose%
\pgfusepath{fill}%
\end{pgfscope}%
\begin{pgfscope}%
\pgfpathrectangle{\pgfqpoint{0.804646in}{0.600000in}}{\pgfqpoint{2.573292in}{2.070576in}}%
\pgfusepath{clip}%
\pgfsetbuttcap%
\pgfsetmiterjoin%
\definecolor{currentfill}{rgb}{0.066899,0.263188,0.377594}%
\pgfsetfillcolor{currentfill}%
\pgfsetlinewidth{0.000000pt}%
\definecolor{currentstroke}{rgb}{0.000000,0.000000,0.000000}%
\pgfsetstrokecolor{currentstroke}%
\pgfsetstrokeopacity{0.000000}%
\pgfsetdash{}{0pt}%
\pgfpathmoveto{\pgfqpoint{1.818841in}{1.613090in}}%
\pgfpathlineto{\pgfqpoint{1.827595in}{1.613090in}}%
\pgfpathlineto{\pgfqpoint{1.827595in}{1.714730in}}%
\pgfpathlineto{\pgfqpoint{1.818841in}{1.714730in}}%
\pgfpathlineto{\pgfqpoint{1.818841in}{1.613090in}}%
\pgfpathclose%
\pgfusepath{fill}%
\end{pgfscope}%
\begin{pgfscope}%
\pgfpathrectangle{\pgfqpoint{0.804646in}{0.600000in}}{\pgfqpoint{2.573292in}{2.070576in}}%
\pgfusepath{clip}%
\pgfsetbuttcap%
\pgfsetmiterjoin%
\definecolor{currentfill}{rgb}{0.066899,0.263188,0.377594}%
\pgfsetfillcolor{currentfill}%
\pgfsetlinewidth{0.000000pt}%
\definecolor{currentstroke}{rgb}{0.000000,0.000000,0.000000}%
\pgfsetstrokecolor{currentstroke}%
\pgfsetstrokeopacity{0.000000}%
\pgfsetdash{}{0pt}%
\pgfpathmoveto{\pgfqpoint{1.829783in}{1.613090in}}%
\pgfpathlineto{\pgfqpoint{1.838536in}{1.613090in}}%
\pgfpathlineto{\pgfqpoint{1.838536in}{1.735219in}}%
\pgfpathlineto{\pgfqpoint{1.829783in}{1.735219in}}%
\pgfpathlineto{\pgfqpoint{1.829783in}{1.613090in}}%
\pgfpathclose%
\pgfusepath{fill}%
\end{pgfscope}%
\begin{pgfscope}%
\pgfpathrectangle{\pgfqpoint{0.804646in}{0.600000in}}{\pgfqpoint{2.573292in}{2.070576in}}%
\pgfusepath{clip}%
\pgfsetbuttcap%
\pgfsetmiterjoin%
\definecolor{currentfill}{rgb}{0.066899,0.263188,0.377594}%
\pgfsetfillcolor{currentfill}%
\pgfsetlinewidth{0.000000pt}%
\definecolor{currentstroke}{rgb}{0.000000,0.000000,0.000000}%
\pgfsetstrokecolor{currentstroke}%
\pgfsetstrokeopacity{0.000000}%
\pgfsetdash{}{0pt}%
\pgfpathmoveto{\pgfqpoint{1.840725in}{1.613090in}}%
\pgfpathlineto{\pgfqpoint{1.849478in}{1.613090in}}%
\pgfpathlineto{\pgfqpoint{1.849478in}{1.750020in}}%
\pgfpathlineto{\pgfqpoint{1.840725in}{1.750020in}}%
\pgfpathlineto{\pgfqpoint{1.840725in}{1.613090in}}%
\pgfpathclose%
\pgfusepath{fill}%
\end{pgfscope}%
\begin{pgfscope}%
\pgfpathrectangle{\pgfqpoint{0.804646in}{0.600000in}}{\pgfqpoint{2.573292in}{2.070576in}}%
\pgfusepath{clip}%
\pgfsetbuttcap%
\pgfsetmiterjoin%
\definecolor{currentfill}{rgb}{0.066899,0.263188,0.377594}%
\pgfsetfillcolor{currentfill}%
\pgfsetlinewidth{0.000000pt}%
\definecolor{currentstroke}{rgb}{0.000000,0.000000,0.000000}%
\pgfsetstrokecolor{currentstroke}%
\pgfsetstrokeopacity{0.000000}%
\pgfsetdash{}{0pt}%
\pgfpathmoveto{\pgfqpoint{1.851667in}{1.613090in}}%
\pgfpathlineto{\pgfqpoint{1.860420in}{1.613090in}}%
\pgfpathlineto{\pgfqpoint{1.860420in}{1.753163in}}%
\pgfpathlineto{\pgfqpoint{1.851667in}{1.753163in}}%
\pgfpathlineto{\pgfqpoint{1.851667in}{1.613090in}}%
\pgfpathclose%
\pgfusepath{fill}%
\end{pgfscope}%
\begin{pgfscope}%
\pgfpathrectangle{\pgfqpoint{0.804646in}{0.600000in}}{\pgfqpoint{2.573292in}{2.070576in}}%
\pgfusepath{clip}%
\pgfsetbuttcap%
\pgfsetmiterjoin%
\definecolor{currentfill}{rgb}{0.066899,0.263188,0.377594}%
\pgfsetfillcolor{currentfill}%
\pgfsetlinewidth{0.000000pt}%
\definecolor{currentstroke}{rgb}{0.000000,0.000000,0.000000}%
\pgfsetstrokecolor{currentstroke}%
\pgfsetstrokeopacity{0.000000}%
\pgfsetdash{}{0pt}%
\pgfpathmoveto{\pgfqpoint{1.862608in}{1.613090in}}%
\pgfpathlineto{\pgfqpoint{1.871362in}{1.613090in}}%
\pgfpathlineto{\pgfqpoint{1.871362in}{1.762478in}}%
\pgfpathlineto{\pgfqpoint{1.862608in}{1.762478in}}%
\pgfpathlineto{\pgfqpoint{1.862608in}{1.613090in}}%
\pgfpathclose%
\pgfusepath{fill}%
\end{pgfscope}%
\begin{pgfscope}%
\pgfpathrectangle{\pgfqpoint{0.804646in}{0.600000in}}{\pgfqpoint{2.573292in}{2.070576in}}%
\pgfusepath{clip}%
\pgfsetbuttcap%
\pgfsetmiterjoin%
\definecolor{currentfill}{rgb}{0.066899,0.263188,0.377594}%
\pgfsetfillcolor{currentfill}%
\pgfsetlinewidth{0.000000pt}%
\definecolor{currentstroke}{rgb}{0.000000,0.000000,0.000000}%
\pgfsetstrokecolor{currentstroke}%
\pgfsetstrokeopacity{0.000000}%
\pgfsetdash{}{0pt}%
\pgfpathmoveto{\pgfqpoint{1.873550in}{1.613090in}}%
\pgfpathlineto{\pgfqpoint{1.882304in}{1.613090in}}%
\pgfpathlineto{\pgfqpoint{1.882304in}{1.781010in}}%
\pgfpathlineto{\pgfqpoint{1.873550in}{1.781010in}}%
\pgfpathlineto{\pgfqpoint{1.873550in}{1.613090in}}%
\pgfpathclose%
\pgfusepath{fill}%
\end{pgfscope}%
\begin{pgfscope}%
\pgfpathrectangle{\pgfqpoint{0.804646in}{0.600000in}}{\pgfqpoint{2.573292in}{2.070576in}}%
\pgfusepath{clip}%
\pgfsetbuttcap%
\pgfsetmiterjoin%
\definecolor{currentfill}{rgb}{0.066899,0.263188,0.377594}%
\pgfsetfillcolor{currentfill}%
\pgfsetlinewidth{0.000000pt}%
\definecolor{currentstroke}{rgb}{0.000000,0.000000,0.000000}%
\pgfsetstrokecolor{currentstroke}%
\pgfsetstrokeopacity{0.000000}%
\pgfsetdash{}{0pt}%
\pgfpathmoveto{\pgfqpoint{1.884492in}{1.613090in}}%
\pgfpathlineto{\pgfqpoint{1.893245in}{1.613090in}}%
\pgfpathlineto{\pgfqpoint{1.893245in}{1.785874in}}%
\pgfpathlineto{\pgfqpoint{1.884492in}{1.785874in}}%
\pgfpathlineto{\pgfqpoint{1.884492in}{1.613090in}}%
\pgfpathclose%
\pgfusepath{fill}%
\end{pgfscope}%
\begin{pgfscope}%
\pgfpathrectangle{\pgfqpoint{0.804646in}{0.600000in}}{\pgfqpoint{2.573292in}{2.070576in}}%
\pgfusepath{clip}%
\pgfsetbuttcap%
\pgfsetmiterjoin%
\definecolor{currentfill}{rgb}{0.066899,0.263188,0.377594}%
\pgfsetfillcolor{currentfill}%
\pgfsetlinewidth{0.000000pt}%
\definecolor{currentstroke}{rgb}{0.000000,0.000000,0.000000}%
\pgfsetstrokecolor{currentstroke}%
\pgfsetstrokeopacity{0.000000}%
\pgfsetdash{}{0pt}%
\pgfpathmoveto{\pgfqpoint{1.895434in}{1.613090in}}%
\pgfpathlineto{\pgfqpoint{1.904187in}{1.613090in}}%
\pgfpathlineto{\pgfqpoint{1.904187in}{1.808212in}}%
\pgfpathlineto{\pgfqpoint{1.895434in}{1.808212in}}%
\pgfpathlineto{\pgfqpoint{1.895434in}{1.613090in}}%
\pgfpathclose%
\pgfusepath{fill}%
\end{pgfscope}%
\begin{pgfscope}%
\pgfpathrectangle{\pgfqpoint{0.804646in}{0.600000in}}{\pgfqpoint{2.573292in}{2.070576in}}%
\pgfusepath{clip}%
\pgfsetbuttcap%
\pgfsetmiterjoin%
\definecolor{currentfill}{rgb}{0.066899,0.263188,0.377594}%
\pgfsetfillcolor{currentfill}%
\pgfsetlinewidth{0.000000pt}%
\definecolor{currentstroke}{rgb}{0.000000,0.000000,0.000000}%
\pgfsetstrokecolor{currentstroke}%
\pgfsetstrokeopacity{0.000000}%
\pgfsetdash{}{0pt}%
\pgfpathmoveto{\pgfqpoint{1.906376in}{1.613090in}}%
\pgfpathlineto{\pgfqpoint{1.915129in}{1.613090in}}%
\pgfpathlineto{\pgfqpoint{1.915129in}{1.830853in}}%
\pgfpathlineto{\pgfqpoint{1.906376in}{1.830853in}}%
\pgfpathlineto{\pgfqpoint{1.906376in}{1.613090in}}%
\pgfpathclose%
\pgfusepath{fill}%
\end{pgfscope}%
\begin{pgfscope}%
\pgfpathrectangle{\pgfqpoint{0.804646in}{0.600000in}}{\pgfqpoint{2.573292in}{2.070576in}}%
\pgfusepath{clip}%
\pgfsetbuttcap%
\pgfsetmiterjoin%
\definecolor{currentfill}{rgb}{0.066899,0.263188,0.377594}%
\pgfsetfillcolor{currentfill}%
\pgfsetlinewidth{0.000000pt}%
\definecolor{currentstroke}{rgb}{0.000000,0.000000,0.000000}%
\pgfsetstrokecolor{currentstroke}%
\pgfsetstrokeopacity{0.000000}%
\pgfsetdash{}{0pt}%
\pgfpathmoveto{\pgfqpoint{1.917317in}{1.613090in}}%
\pgfpathlineto{\pgfqpoint{1.926071in}{1.613090in}}%
\pgfpathlineto{\pgfqpoint{1.926071in}{1.836046in}}%
\pgfpathlineto{\pgfqpoint{1.917317in}{1.836046in}}%
\pgfpathlineto{\pgfqpoint{1.917317in}{1.613090in}}%
\pgfpathclose%
\pgfusepath{fill}%
\end{pgfscope}%
\begin{pgfscope}%
\pgfpathrectangle{\pgfqpoint{0.804646in}{0.600000in}}{\pgfqpoint{2.573292in}{2.070576in}}%
\pgfusepath{clip}%
\pgfsetbuttcap%
\pgfsetmiterjoin%
\definecolor{currentfill}{rgb}{0.066899,0.263188,0.377594}%
\pgfsetfillcolor{currentfill}%
\pgfsetlinewidth{0.000000pt}%
\definecolor{currentstroke}{rgb}{0.000000,0.000000,0.000000}%
\pgfsetstrokecolor{currentstroke}%
\pgfsetstrokeopacity{0.000000}%
\pgfsetdash{}{0pt}%
\pgfpathmoveto{\pgfqpoint{1.928259in}{1.613090in}}%
\pgfpathlineto{\pgfqpoint{1.937013in}{1.613090in}}%
\pgfpathlineto{\pgfqpoint{1.937013in}{1.836643in}}%
\pgfpathlineto{\pgfqpoint{1.928259in}{1.836643in}}%
\pgfpathlineto{\pgfqpoint{1.928259in}{1.613090in}}%
\pgfpathclose%
\pgfusepath{fill}%
\end{pgfscope}%
\begin{pgfscope}%
\pgfpathrectangle{\pgfqpoint{0.804646in}{0.600000in}}{\pgfqpoint{2.573292in}{2.070576in}}%
\pgfusepath{clip}%
\pgfsetbuttcap%
\pgfsetmiterjoin%
\definecolor{currentfill}{rgb}{0.066899,0.263188,0.377594}%
\pgfsetfillcolor{currentfill}%
\pgfsetlinewidth{0.000000pt}%
\definecolor{currentstroke}{rgb}{0.000000,0.000000,0.000000}%
\pgfsetstrokecolor{currentstroke}%
\pgfsetstrokeopacity{0.000000}%
\pgfsetdash{}{0pt}%
\pgfpathmoveto{\pgfqpoint{1.939201in}{1.613090in}}%
\pgfpathlineto{\pgfqpoint{1.947954in}{1.613090in}}%
\pgfpathlineto{\pgfqpoint{1.947954in}{1.826575in}}%
\pgfpathlineto{\pgfqpoint{1.939201in}{1.826575in}}%
\pgfpathlineto{\pgfqpoint{1.939201in}{1.613090in}}%
\pgfpathclose%
\pgfusepath{fill}%
\end{pgfscope}%
\begin{pgfscope}%
\pgfpathrectangle{\pgfqpoint{0.804646in}{0.600000in}}{\pgfqpoint{2.573292in}{2.070576in}}%
\pgfusepath{clip}%
\pgfsetbuttcap%
\pgfsetmiterjoin%
\definecolor{currentfill}{rgb}{0.066899,0.263188,0.377594}%
\pgfsetfillcolor{currentfill}%
\pgfsetlinewidth{0.000000pt}%
\definecolor{currentstroke}{rgb}{0.000000,0.000000,0.000000}%
\pgfsetstrokecolor{currentstroke}%
\pgfsetstrokeopacity{0.000000}%
\pgfsetdash{}{0pt}%
\pgfpathmoveto{\pgfqpoint{1.950143in}{1.613090in}}%
\pgfpathlineto{\pgfqpoint{1.958896in}{1.613090in}}%
\pgfpathlineto{\pgfqpoint{1.958896in}{1.823406in}}%
\pgfpathlineto{\pgfqpoint{1.950143in}{1.823406in}}%
\pgfpathlineto{\pgfqpoint{1.950143in}{1.613090in}}%
\pgfpathclose%
\pgfusepath{fill}%
\end{pgfscope}%
\begin{pgfscope}%
\pgfpathrectangle{\pgfqpoint{0.804646in}{0.600000in}}{\pgfqpoint{2.573292in}{2.070576in}}%
\pgfusepath{clip}%
\pgfsetbuttcap%
\pgfsetmiterjoin%
\definecolor{currentfill}{rgb}{0.066899,0.263188,0.377594}%
\pgfsetfillcolor{currentfill}%
\pgfsetlinewidth{0.000000pt}%
\definecolor{currentstroke}{rgb}{0.000000,0.000000,0.000000}%
\pgfsetstrokecolor{currentstroke}%
\pgfsetstrokeopacity{0.000000}%
\pgfsetdash{}{0pt}%
\pgfpathmoveto{\pgfqpoint{1.961085in}{1.613090in}}%
\pgfpathlineto{\pgfqpoint{1.969838in}{1.613090in}}%
\pgfpathlineto{\pgfqpoint{1.969838in}{1.803284in}}%
\pgfpathlineto{\pgfqpoint{1.961085in}{1.803284in}}%
\pgfpathlineto{\pgfqpoint{1.961085in}{1.613090in}}%
\pgfpathclose%
\pgfusepath{fill}%
\end{pgfscope}%
\begin{pgfscope}%
\pgfpathrectangle{\pgfqpoint{0.804646in}{0.600000in}}{\pgfqpoint{2.573292in}{2.070576in}}%
\pgfusepath{clip}%
\pgfsetbuttcap%
\pgfsetmiterjoin%
\definecolor{currentfill}{rgb}{0.066899,0.263188,0.377594}%
\pgfsetfillcolor{currentfill}%
\pgfsetlinewidth{0.000000pt}%
\definecolor{currentstroke}{rgb}{0.000000,0.000000,0.000000}%
\pgfsetstrokecolor{currentstroke}%
\pgfsetstrokeopacity{0.000000}%
\pgfsetdash{}{0pt}%
\pgfpathmoveto{\pgfqpoint{1.972026in}{1.613090in}}%
\pgfpathlineto{\pgfqpoint{1.980780in}{1.613090in}}%
\pgfpathlineto{\pgfqpoint{1.980780in}{1.788585in}}%
\pgfpathlineto{\pgfqpoint{1.972026in}{1.788585in}}%
\pgfpathlineto{\pgfqpoint{1.972026in}{1.613090in}}%
\pgfpathclose%
\pgfusepath{fill}%
\end{pgfscope}%
\begin{pgfscope}%
\pgfpathrectangle{\pgfqpoint{0.804646in}{0.600000in}}{\pgfqpoint{2.573292in}{2.070576in}}%
\pgfusepath{clip}%
\pgfsetbuttcap%
\pgfsetmiterjoin%
\definecolor{currentfill}{rgb}{0.066899,0.263188,0.377594}%
\pgfsetfillcolor{currentfill}%
\pgfsetlinewidth{0.000000pt}%
\definecolor{currentstroke}{rgb}{0.000000,0.000000,0.000000}%
\pgfsetstrokecolor{currentstroke}%
\pgfsetstrokeopacity{0.000000}%
\pgfsetdash{}{0pt}%
\pgfpathmoveto{\pgfqpoint{1.982968in}{1.613090in}}%
\pgfpathlineto{\pgfqpoint{1.991722in}{1.613090in}}%
\pgfpathlineto{\pgfqpoint{1.991722in}{1.777965in}}%
\pgfpathlineto{\pgfqpoint{1.982968in}{1.777965in}}%
\pgfpathlineto{\pgfqpoint{1.982968in}{1.613090in}}%
\pgfpathclose%
\pgfusepath{fill}%
\end{pgfscope}%
\begin{pgfscope}%
\pgfpathrectangle{\pgfqpoint{0.804646in}{0.600000in}}{\pgfqpoint{2.573292in}{2.070576in}}%
\pgfusepath{clip}%
\pgfsetbuttcap%
\pgfsetmiterjoin%
\definecolor{currentfill}{rgb}{0.066899,0.263188,0.377594}%
\pgfsetfillcolor{currentfill}%
\pgfsetlinewidth{0.000000pt}%
\definecolor{currentstroke}{rgb}{0.000000,0.000000,0.000000}%
\pgfsetstrokecolor{currentstroke}%
\pgfsetstrokeopacity{0.000000}%
\pgfsetdash{}{0pt}%
\pgfpathmoveto{\pgfqpoint{1.993910in}{1.613090in}}%
\pgfpathlineto{\pgfqpoint{2.002663in}{1.613090in}}%
\pgfpathlineto{\pgfqpoint{2.002663in}{1.728675in}}%
\pgfpathlineto{\pgfqpoint{1.993910in}{1.728675in}}%
\pgfpathlineto{\pgfqpoint{1.993910in}{1.613090in}}%
\pgfpathclose%
\pgfusepath{fill}%
\end{pgfscope}%
\begin{pgfscope}%
\pgfpathrectangle{\pgfqpoint{0.804646in}{0.600000in}}{\pgfqpoint{2.573292in}{2.070576in}}%
\pgfusepath{clip}%
\pgfsetbuttcap%
\pgfsetmiterjoin%
\definecolor{currentfill}{rgb}{0.066899,0.263188,0.377594}%
\pgfsetfillcolor{currentfill}%
\pgfsetlinewidth{0.000000pt}%
\definecolor{currentstroke}{rgb}{0.000000,0.000000,0.000000}%
\pgfsetstrokecolor{currentstroke}%
\pgfsetstrokeopacity{0.000000}%
\pgfsetdash{}{0pt}%
\pgfpathmoveto{\pgfqpoint{2.004852in}{1.613090in}}%
\pgfpathlineto{\pgfqpoint{2.013605in}{1.613090in}}%
\pgfpathlineto{\pgfqpoint{2.013605in}{1.705319in}}%
\pgfpathlineto{\pgfqpoint{2.004852in}{1.705319in}}%
\pgfpathlineto{\pgfqpoint{2.004852in}{1.613090in}}%
\pgfpathclose%
\pgfusepath{fill}%
\end{pgfscope}%
\begin{pgfscope}%
\pgfpathrectangle{\pgfqpoint{0.804646in}{0.600000in}}{\pgfqpoint{2.573292in}{2.070576in}}%
\pgfusepath{clip}%
\pgfsetbuttcap%
\pgfsetmiterjoin%
\definecolor{currentfill}{rgb}{0.066899,0.263188,0.377594}%
\pgfsetfillcolor{currentfill}%
\pgfsetlinewidth{0.000000pt}%
\definecolor{currentstroke}{rgb}{0.000000,0.000000,0.000000}%
\pgfsetstrokecolor{currentstroke}%
\pgfsetstrokeopacity{0.000000}%
\pgfsetdash{}{0pt}%
\pgfpathmoveto{\pgfqpoint{2.015794in}{1.613090in}}%
\pgfpathlineto{\pgfqpoint{2.024547in}{1.613090in}}%
\pgfpathlineto{\pgfqpoint{2.024547in}{1.693241in}}%
\pgfpathlineto{\pgfqpoint{2.015794in}{1.693241in}}%
\pgfpathlineto{\pgfqpoint{2.015794in}{1.613090in}}%
\pgfpathclose%
\pgfusepath{fill}%
\end{pgfscope}%
\begin{pgfscope}%
\pgfpathrectangle{\pgfqpoint{0.804646in}{0.600000in}}{\pgfqpoint{2.573292in}{2.070576in}}%
\pgfusepath{clip}%
\pgfsetbuttcap%
\pgfsetmiterjoin%
\definecolor{currentfill}{rgb}{0.066899,0.263188,0.377594}%
\pgfsetfillcolor{currentfill}%
\pgfsetlinewidth{0.000000pt}%
\definecolor{currentstroke}{rgb}{0.000000,0.000000,0.000000}%
\pgfsetstrokecolor{currentstroke}%
\pgfsetstrokeopacity{0.000000}%
\pgfsetdash{}{0pt}%
\pgfpathmoveto{\pgfqpoint{2.026735in}{1.613090in}}%
\pgfpathlineto{\pgfqpoint{2.035489in}{1.613090in}}%
\pgfpathlineto{\pgfqpoint{2.035489in}{1.676186in}}%
\pgfpathlineto{\pgfqpoint{2.026735in}{1.676186in}}%
\pgfpathlineto{\pgfqpoint{2.026735in}{1.613090in}}%
\pgfpathclose%
\pgfusepath{fill}%
\end{pgfscope}%
\begin{pgfscope}%
\pgfpathrectangle{\pgfqpoint{0.804646in}{0.600000in}}{\pgfqpoint{2.573292in}{2.070576in}}%
\pgfusepath{clip}%
\pgfsetbuttcap%
\pgfsetmiterjoin%
\definecolor{currentfill}{rgb}{0.066899,0.263188,0.377594}%
\pgfsetfillcolor{currentfill}%
\pgfsetlinewidth{0.000000pt}%
\definecolor{currentstroke}{rgb}{0.000000,0.000000,0.000000}%
\pgfsetstrokecolor{currentstroke}%
\pgfsetstrokeopacity{0.000000}%
\pgfsetdash{}{0pt}%
\pgfpathmoveto{\pgfqpoint{2.037677in}{1.613090in}}%
\pgfpathlineto{\pgfqpoint{2.046431in}{1.613090in}}%
\pgfpathlineto{\pgfqpoint{2.046431in}{1.662747in}}%
\pgfpathlineto{\pgfqpoint{2.037677in}{1.662747in}}%
\pgfpathlineto{\pgfqpoint{2.037677in}{1.613090in}}%
\pgfpathclose%
\pgfusepath{fill}%
\end{pgfscope}%
\begin{pgfscope}%
\pgfpathrectangle{\pgfqpoint{0.804646in}{0.600000in}}{\pgfqpoint{2.573292in}{2.070576in}}%
\pgfusepath{clip}%
\pgfsetbuttcap%
\pgfsetmiterjoin%
\definecolor{currentfill}{rgb}{0.066899,0.263188,0.377594}%
\pgfsetfillcolor{currentfill}%
\pgfsetlinewidth{0.000000pt}%
\definecolor{currentstroke}{rgb}{0.000000,0.000000,0.000000}%
\pgfsetstrokecolor{currentstroke}%
\pgfsetstrokeopacity{0.000000}%
\pgfsetdash{}{0pt}%
\pgfpathmoveto{\pgfqpoint{2.048619in}{1.613090in}}%
\pgfpathlineto{\pgfqpoint{2.057372in}{1.613090in}}%
\pgfpathlineto{\pgfqpoint{2.057372in}{1.652593in}}%
\pgfpathlineto{\pgfqpoint{2.048619in}{1.652593in}}%
\pgfpathlineto{\pgfqpoint{2.048619in}{1.613090in}}%
\pgfpathclose%
\pgfusepath{fill}%
\end{pgfscope}%
\begin{pgfscope}%
\pgfpathrectangle{\pgfqpoint{0.804646in}{0.600000in}}{\pgfqpoint{2.573292in}{2.070576in}}%
\pgfusepath{clip}%
\pgfsetbuttcap%
\pgfsetmiterjoin%
\definecolor{currentfill}{rgb}{0.066899,0.263188,0.377594}%
\pgfsetfillcolor{currentfill}%
\pgfsetlinewidth{0.000000pt}%
\definecolor{currentstroke}{rgb}{0.000000,0.000000,0.000000}%
\pgfsetstrokecolor{currentstroke}%
\pgfsetstrokeopacity{0.000000}%
\pgfsetdash{}{0pt}%
\pgfpathmoveto{\pgfqpoint{2.059561in}{1.613090in}}%
\pgfpathlineto{\pgfqpoint{2.068314in}{1.613090in}}%
\pgfpathlineto{\pgfqpoint{2.068314in}{1.650287in}}%
\pgfpathlineto{\pgfqpoint{2.059561in}{1.650287in}}%
\pgfpathlineto{\pgfqpoint{2.059561in}{1.613090in}}%
\pgfpathclose%
\pgfusepath{fill}%
\end{pgfscope}%
\begin{pgfscope}%
\pgfpathrectangle{\pgfqpoint{0.804646in}{0.600000in}}{\pgfqpoint{2.573292in}{2.070576in}}%
\pgfusepath{clip}%
\pgfsetbuttcap%
\pgfsetmiterjoin%
\definecolor{currentfill}{rgb}{0.066899,0.263188,0.377594}%
\pgfsetfillcolor{currentfill}%
\pgfsetlinewidth{0.000000pt}%
\definecolor{currentstroke}{rgb}{0.000000,0.000000,0.000000}%
\pgfsetstrokecolor{currentstroke}%
\pgfsetstrokeopacity{0.000000}%
\pgfsetdash{}{0pt}%
\pgfpathmoveto{\pgfqpoint{2.070503in}{1.613090in}}%
\pgfpathlineto{\pgfqpoint{2.079256in}{1.613090in}}%
\pgfpathlineto{\pgfqpoint{2.079256in}{1.651274in}}%
\pgfpathlineto{\pgfqpoint{2.070503in}{1.651274in}}%
\pgfpathlineto{\pgfqpoint{2.070503in}{1.613090in}}%
\pgfpathclose%
\pgfusepath{fill}%
\end{pgfscope}%
\begin{pgfscope}%
\pgfpathrectangle{\pgfqpoint{0.804646in}{0.600000in}}{\pgfqpoint{2.573292in}{2.070576in}}%
\pgfusepath{clip}%
\pgfsetbuttcap%
\pgfsetmiterjoin%
\definecolor{currentfill}{rgb}{0.066899,0.263188,0.377594}%
\pgfsetfillcolor{currentfill}%
\pgfsetlinewidth{0.000000pt}%
\definecolor{currentstroke}{rgb}{0.000000,0.000000,0.000000}%
\pgfsetstrokecolor{currentstroke}%
\pgfsetstrokeopacity{0.000000}%
\pgfsetdash{}{0pt}%
\pgfpathmoveto{\pgfqpoint{2.081444in}{1.613090in}}%
\pgfpathlineto{\pgfqpoint{2.090198in}{1.613090in}}%
\pgfpathlineto{\pgfqpoint{2.090198in}{1.665110in}}%
\pgfpathlineto{\pgfqpoint{2.081444in}{1.665110in}}%
\pgfpathlineto{\pgfqpoint{2.081444in}{1.613090in}}%
\pgfpathclose%
\pgfusepath{fill}%
\end{pgfscope}%
\begin{pgfscope}%
\pgfpathrectangle{\pgfqpoint{0.804646in}{0.600000in}}{\pgfqpoint{2.573292in}{2.070576in}}%
\pgfusepath{clip}%
\pgfsetbuttcap%
\pgfsetmiterjoin%
\definecolor{currentfill}{rgb}{0.066899,0.263188,0.377594}%
\pgfsetfillcolor{currentfill}%
\pgfsetlinewidth{0.000000pt}%
\definecolor{currentstroke}{rgb}{0.000000,0.000000,0.000000}%
\pgfsetstrokecolor{currentstroke}%
\pgfsetstrokeopacity{0.000000}%
\pgfsetdash{}{0pt}%
\pgfpathmoveto{\pgfqpoint{2.092386in}{1.613090in}}%
\pgfpathlineto{\pgfqpoint{2.101140in}{1.613090in}}%
\pgfpathlineto{\pgfqpoint{2.101140in}{1.680773in}}%
\pgfpathlineto{\pgfqpoint{2.092386in}{1.680773in}}%
\pgfpathlineto{\pgfqpoint{2.092386in}{1.613090in}}%
\pgfpathclose%
\pgfusepath{fill}%
\end{pgfscope}%
\begin{pgfscope}%
\pgfpathrectangle{\pgfqpoint{0.804646in}{0.600000in}}{\pgfqpoint{2.573292in}{2.070576in}}%
\pgfusepath{clip}%
\pgfsetbuttcap%
\pgfsetmiterjoin%
\definecolor{currentfill}{rgb}{0.066899,0.263188,0.377594}%
\pgfsetfillcolor{currentfill}%
\pgfsetlinewidth{0.000000pt}%
\definecolor{currentstroke}{rgb}{0.000000,0.000000,0.000000}%
\pgfsetstrokecolor{currentstroke}%
\pgfsetstrokeopacity{0.000000}%
\pgfsetdash{}{0pt}%
\pgfpathmoveto{\pgfqpoint{2.103328in}{1.613090in}}%
\pgfpathlineto{\pgfqpoint{2.112081in}{1.613090in}}%
\pgfpathlineto{\pgfqpoint{2.112081in}{1.690668in}}%
\pgfpathlineto{\pgfqpoint{2.103328in}{1.690668in}}%
\pgfpathlineto{\pgfqpoint{2.103328in}{1.613090in}}%
\pgfpathclose%
\pgfusepath{fill}%
\end{pgfscope}%
\begin{pgfscope}%
\pgfpathrectangle{\pgfqpoint{0.804646in}{0.600000in}}{\pgfqpoint{2.573292in}{2.070576in}}%
\pgfusepath{clip}%
\pgfsetbuttcap%
\pgfsetmiterjoin%
\definecolor{currentfill}{rgb}{0.066899,0.263188,0.377594}%
\pgfsetfillcolor{currentfill}%
\pgfsetlinewidth{0.000000pt}%
\definecolor{currentstroke}{rgb}{0.000000,0.000000,0.000000}%
\pgfsetstrokecolor{currentstroke}%
\pgfsetstrokeopacity{0.000000}%
\pgfsetdash{}{0pt}%
\pgfpathmoveto{\pgfqpoint{2.114270in}{1.613090in}}%
\pgfpathlineto{\pgfqpoint{2.123023in}{1.613090in}}%
\pgfpathlineto{\pgfqpoint{2.123023in}{1.697110in}}%
\pgfpathlineto{\pgfqpoint{2.114270in}{1.697110in}}%
\pgfpathlineto{\pgfqpoint{2.114270in}{1.613090in}}%
\pgfpathclose%
\pgfusepath{fill}%
\end{pgfscope}%
\begin{pgfscope}%
\pgfpathrectangle{\pgfqpoint{0.804646in}{0.600000in}}{\pgfqpoint{2.573292in}{2.070576in}}%
\pgfusepath{clip}%
\pgfsetbuttcap%
\pgfsetmiterjoin%
\definecolor{currentfill}{rgb}{0.066899,0.263188,0.377594}%
\pgfsetfillcolor{currentfill}%
\pgfsetlinewidth{0.000000pt}%
\definecolor{currentstroke}{rgb}{0.000000,0.000000,0.000000}%
\pgfsetstrokecolor{currentstroke}%
\pgfsetstrokeopacity{0.000000}%
\pgfsetdash{}{0pt}%
\pgfpathmoveto{\pgfqpoint{2.125212in}{1.613090in}}%
\pgfpathlineto{\pgfqpoint{2.133965in}{1.613090in}}%
\pgfpathlineto{\pgfqpoint{2.133965in}{1.704455in}}%
\pgfpathlineto{\pgfqpoint{2.125212in}{1.704455in}}%
\pgfpathlineto{\pgfqpoint{2.125212in}{1.613090in}}%
\pgfpathclose%
\pgfusepath{fill}%
\end{pgfscope}%
\begin{pgfscope}%
\pgfpathrectangle{\pgfqpoint{0.804646in}{0.600000in}}{\pgfqpoint{2.573292in}{2.070576in}}%
\pgfusepath{clip}%
\pgfsetbuttcap%
\pgfsetmiterjoin%
\definecolor{currentfill}{rgb}{0.066899,0.263188,0.377594}%
\pgfsetfillcolor{currentfill}%
\pgfsetlinewidth{0.000000pt}%
\definecolor{currentstroke}{rgb}{0.000000,0.000000,0.000000}%
\pgfsetstrokecolor{currentstroke}%
\pgfsetstrokeopacity{0.000000}%
\pgfsetdash{}{0pt}%
\pgfpathmoveto{\pgfqpoint{2.136153in}{1.613090in}}%
\pgfpathlineto{\pgfqpoint{2.144907in}{1.613090in}}%
\pgfpathlineto{\pgfqpoint{2.144907in}{1.737539in}}%
\pgfpathlineto{\pgfqpoint{2.136153in}{1.737539in}}%
\pgfpathlineto{\pgfqpoint{2.136153in}{1.613090in}}%
\pgfpathclose%
\pgfusepath{fill}%
\end{pgfscope}%
\begin{pgfscope}%
\pgfpathrectangle{\pgfqpoint{0.804646in}{0.600000in}}{\pgfqpoint{2.573292in}{2.070576in}}%
\pgfusepath{clip}%
\pgfsetbuttcap%
\pgfsetmiterjoin%
\definecolor{currentfill}{rgb}{0.066899,0.263188,0.377594}%
\pgfsetfillcolor{currentfill}%
\pgfsetlinewidth{0.000000pt}%
\definecolor{currentstroke}{rgb}{0.000000,0.000000,0.000000}%
\pgfsetstrokecolor{currentstroke}%
\pgfsetstrokeopacity{0.000000}%
\pgfsetdash{}{0pt}%
\pgfpathmoveto{\pgfqpoint{2.147095in}{1.613090in}}%
\pgfpathlineto{\pgfqpoint{2.155849in}{1.613090in}}%
\pgfpathlineto{\pgfqpoint{2.155849in}{1.755343in}}%
\pgfpathlineto{\pgfqpoint{2.147095in}{1.755343in}}%
\pgfpathlineto{\pgfqpoint{2.147095in}{1.613090in}}%
\pgfpathclose%
\pgfusepath{fill}%
\end{pgfscope}%
\begin{pgfscope}%
\pgfpathrectangle{\pgfqpoint{0.804646in}{0.600000in}}{\pgfqpoint{2.573292in}{2.070576in}}%
\pgfusepath{clip}%
\pgfsetbuttcap%
\pgfsetmiterjoin%
\definecolor{currentfill}{rgb}{0.066899,0.263188,0.377594}%
\pgfsetfillcolor{currentfill}%
\pgfsetlinewidth{0.000000pt}%
\definecolor{currentstroke}{rgb}{0.000000,0.000000,0.000000}%
\pgfsetstrokecolor{currentstroke}%
\pgfsetstrokeopacity{0.000000}%
\pgfsetdash{}{0pt}%
\pgfpathmoveto{\pgfqpoint{2.158037in}{1.613090in}}%
\pgfpathlineto{\pgfqpoint{2.166790in}{1.613090in}}%
\pgfpathlineto{\pgfqpoint{2.166790in}{1.761235in}}%
\pgfpathlineto{\pgfqpoint{2.158037in}{1.761235in}}%
\pgfpathlineto{\pgfqpoint{2.158037in}{1.613090in}}%
\pgfpathclose%
\pgfusepath{fill}%
\end{pgfscope}%
\begin{pgfscope}%
\pgfpathrectangle{\pgfqpoint{0.804646in}{0.600000in}}{\pgfqpoint{2.573292in}{2.070576in}}%
\pgfusepath{clip}%
\pgfsetbuttcap%
\pgfsetmiterjoin%
\definecolor{currentfill}{rgb}{0.066899,0.263188,0.377594}%
\pgfsetfillcolor{currentfill}%
\pgfsetlinewidth{0.000000pt}%
\definecolor{currentstroke}{rgb}{0.000000,0.000000,0.000000}%
\pgfsetstrokecolor{currentstroke}%
\pgfsetstrokeopacity{0.000000}%
\pgfsetdash{}{0pt}%
\pgfpathmoveto{\pgfqpoint{2.168979in}{1.613090in}}%
\pgfpathlineto{\pgfqpoint{2.177732in}{1.613090in}}%
\pgfpathlineto{\pgfqpoint{2.177732in}{1.769385in}}%
\pgfpathlineto{\pgfqpoint{2.168979in}{1.769385in}}%
\pgfpathlineto{\pgfqpoint{2.168979in}{1.613090in}}%
\pgfpathclose%
\pgfusepath{fill}%
\end{pgfscope}%
\begin{pgfscope}%
\pgfpathrectangle{\pgfqpoint{0.804646in}{0.600000in}}{\pgfqpoint{2.573292in}{2.070576in}}%
\pgfusepath{clip}%
\pgfsetbuttcap%
\pgfsetmiterjoin%
\definecolor{currentfill}{rgb}{0.066899,0.263188,0.377594}%
\pgfsetfillcolor{currentfill}%
\pgfsetlinewidth{0.000000pt}%
\definecolor{currentstroke}{rgb}{0.000000,0.000000,0.000000}%
\pgfsetstrokecolor{currentstroke}%
\pgfsetstrokeopacity{0.000000}%
\pgfsetdash{}{0pt}%
\pgfpathmoveto{\pgfqpoint{2.179921in}{1.613090in}}%
\pgfpathlineto{\pgfqpoint{2.188674in}{1.613090in}}%
\pgfpathlineto{\pgfqpoint{2.188674in}{1.759715in}}%
\pgfpathlineto{\pgfqpoint{2.179921in}{1.759715in}}%
\pgfpathlineto{\pgfqpoint{2.179921in}{1.613090in}}%
\pgfpathclose%
\pgfusepath{fill}%
\end{pgfscope}%
\begin{pgfscope}%
\pgfpathrectangle{\pgfqpoint{0.804646in}{0.600000in}}{\pgfqpoint{2.573292in}{2.070576in}}%
\pgfusepath{clip}%
\pgfsetbuttcap%
\pgfsetmiterjoin%
\definecolor{currentfill}{rgb}{0.066899,0.263188,0.377594}%
\pgfsetfillcolor{currentfill}%
\pgfsetlinewidth{0.000000pt}%
\definecolor{currentstroke}{rgb}{0.000000,0.000000,0.000000}%
\pgfsetstrokecolor{currentstroke}%
\pgfsetstrokeopacity{0.000000}%
\pgfsetdash{}{0pt}%
\pgfpathmoveto{\pgfqpoint{2.190862in}{1.613090in}}%
\pgfpathlineto{\pgfqpoint{2.199616in}{1.613090in}}%
\pgfpathlineto{\pgfqpoint{2.199616in}{1.775309in}}%
\pgfpathlineto{\pgfqpoint{2.190862in}{1.775309in}}%
\pgfpathlineto{\pgfqpoint{2.190862in}{1.613090in}}%
\pgfpathclose%
\pgfusepath{fill}%
\end{pgfscope}%
\begin{pgfscope}%
\pgfpathrectangle{\pgfqpoint{0.804646in}{0.600000in}}{\pgfqpoint{2.573292in}{2.070576in}}%
\pgfusepath{clip}%
\pgfsetbuttcap%
\pgfsetmiterjoin%
\definecolor{currentfill}{rgb}{0.066899,0.263188,0.377594}%
\pgfsetfillcolor{currentfill}%
\pgfsetlinewidth{0.000000pt}%
\definecolor{currentstroke}{rgb}{0.000000,0.000000,0.000000}%
\pgfsetstrokecolor{currentstroke}%
\pgfsetstrokeopacity{0.000000}%
\pgfsetdash{}{0pt}%
\pgfpathmoveto{\pgfqpoint{2.201804in}{1.613090in}}%
\pgfpathlineto{\pgfqpoint{2.210558in}{1.613090in}}%
\pgfpathlineto{\pgfqpoint{2.210558in}{1.773096in}}%
\pgfpathlineto{\pgfqpoint{2.201804in}{1.773096in}}%
\pgfpathlineto{\pgfqpoint{2.201804in}{1.613090in}}%
\pgfpathclose%
\pgfusepath{fill}%
\end{pgfscope}%
\begin{pgfscope}%
\pgfpathrectangle{\pgfqpoint{0.804646in}{0.600000in}}{\pgfqpoint{2.573292in}{2.070576in}}%
\pgfusepath{clip}%
\pgfsetbuttcap%
\pgfsetmiterjoin%
\definecolor{currentfill}{rgb}{0.066899,0.263188,0.377594}%
\pgfsetfillcolor{currentfill}%
\pgfsetlinewidth{0.000000pt}%
\definecolor{currentstroke}{rgb}{0.000000,0.000000,0.000000}%
\pgfsetstrokecolor{currentstroke}%
\pgfsetstrokeopacity{0.000000}%
\pgfsetdash{}{0pt}%
\pgfpathmoveto{\pgfqpoint{2.212746in}{1.613090in}}%
\pgfpathlineto{\pgfqpoint{2.221499in}{1.613090in}}%
\pgfpathlineto{\pgfqpoint{2.221499in}{1.766378in}}%
\pgfpathlineto{\pgfqpoint{2.212746in}{1.766378in}}%
\pgfpathlineto{\pgfqpoint{2.212746in}{1.613090in}}%
\pgfpathclose%
\pgfusepath{fill}%
\end{pgfscope}%
\begin{pgfscope}%
\pgfpathrectangle{\pgfqpoint{0.804646in}{0.600000in}}{\pgfqpoint{2.573292in}{2.070576in}}%
\pgfusepath{clip}%
\pgfsetbuttcap%
\pgfsetmiterjoin%
\definecolor{currentfill}{rgb}{0.066899,0.263188,0.377594}%
\pgfsetfillcolor{currentfill}%
\pgfsetlinewidth{0.000000pt}%
\definecolor{currentstroke}{rgb}{0.000000,0.000000,0.000000}%
\pgfsetstrokecolor{currentstroke}%
\pgfsetstrokeopacity{0.000000}%
\pgfsetdash{}{0pt}%
\pgfpathmoveto{\pgfqpoint{2.223688in}{1.613090in}}%
\pgfpathlineto{\pgfqpoint{2.232441in}{1.613090in}}%
\pgfpathlineto{\pgfqpoint{2.232441in}{1.786895in}}%
\pgfpathlineto{\pgfqpoint{2.223688in}{1.786895in}}%
\pgfpathlineto{\pgfqpoint{2.223688in}{1.613090in}}%
\pgfpathclose%
\pgfusepath{fill}%
\end{pgfscope}%
\begin{pgfscope}%
\pgfpathrectangle{\pgfqpoint{0.804646in}{0.600000in}}{\pgfqpoint{2.573292in}{2.070576in}}%
\pgfusepath{clip}%
\pgfsetbuttcap%
\pgfsetmiterjoin%
\definecolor{currentfill}{rgb}{0.066899,0.263188,0.377594}%
\pgfsetfillcolor{currentfill}%
\pgfsetlinewidth{0.000000pt}%
\definecolor{currentstroke}{rgb}{0.000000,0.000000,0.000000}%
\pgfsetstrokecolor{currentstroke}%
\pgfsetstrokeopacity{0.000000}%
\pgfsetdash{}{0pt}%
\pgfpathmoveto{\pgfqpoint{2.234630in}{1.613090in}}%
\pgfpathlineto{\pgfqpoint{2.243383in}{1.613090in}}%
\pgfpathlineto{\pgfqpoint{2.243383in}{1.794286in}}%
\pgfpathlineto{\pgfqpoint{2.234630in}{1.794286in}}%
\pgfpathlineto{\pgfqpoint{2.234630in}{1.613090in}}%
\pgfpathclose%
\pgfusepath{fill}%
\end{pgfscope}%
\begin{pgfscope}%
\pgfpathrectangle{\pgfqpoint{0.804646in}{0.600000in}}{\pgfqpoint{2.573292in}{2.070576in}}%
\pgfusepath{clip}%
\pgfsetbuttcap%
\pgfsetmiterjoin%
\definecolor{currentfill}{rgb}{0.066899,0.263188,0.377594}%
\pgfsetfillcolor{currentfill}%
\pgfsetlinewidth{0.000000pt}%
\definecolor{currentstroke}{rgb}{0.000000,0.000000,0.000000}%
\pgfsetstrokecolor{currentstroke}%
\pgfsetstrokeopacity{0.000000}%
\pgfsetdash{}{0pt}%
\pgfpathmoveto{\pgfqpoint{2.245571in}{1.613090in}}%
\pgfpathlineto{\pgfqpoint{2.254325in}{1.613090in}}%
\pgfpathlineto{\pgfqpoint{2.254325in}{1.806358in}}%
\pgfpathlineto{\pgfqpoint{2.245571in}{1.806358in}}%
\pgfpathlineto{\pgfqpoint{2.245571in}{1.613090in}}%
\pgfpathclose%
\pgfusepath{fill}%
\end{pgfscope}%
\begin{pgfscope}%
\pgfpathrectangle{\pgfqpoint{0.804646in}{0.600000in}}{\pgfqpoint{2.573292in}{2.070576in}}%
\pgfusepath{clip}%
\pgfsetbuttcap%
\pgfsetmiterjoin%
\definecolor{currentfill}{rgb}{0.066899,0.263188,0.377594}%
\pgfsetfillcolor{currentfill}%
\pgfsetlinewidth{0.000000pt}%
\definecolor{currentstroke}{rgb}{0.000000,0.000000,0.000000}%
\pgfsetstrokecolor{currentstroke}%
\pgfsetstrokeopacity{0.000000}%
\pgfsetdash{}{0pt}%
\pgfpathmoveto{\pgfqpoint{2.256513in}{1.613090in}}%
\pgfpathlineto{\pgfqpoint{2.265267in}{1.613090in}}%
\pgfpathlineto{\pgfqpoint{2.265267in}{1.815807in}}%
\pgfpathlineto{\pgfqpoint{2.256513in}{1.815807in}}%
\pgfpathlineto{\pgfqpoint{2.256513in}{1.613090in}}%
\pgfpathclose%
\pgfusepath{fill}%
\end{pgfscope}%
\begin{pgfscope}%
\pgfpathrectangle{\pgfqpoint{0.804646in}{0.600000in}}{\pgfqpoint{2.573292in}{2.070576in}}%
\pgfusepath{clip}%
\pgfsetbuttcap%
\pgfsetmiterjoin%
\definecolor{currentfill}{rgb}{0.066899,0.263188,0.377594}%
\pgfsetfillcolor{currentfill}%
\pgfsetlinewidth{0.000000pt}%
\definecolor{currentstroke}{rgb}{0.000000,0.000000,0.000000}%
\pgfsetstrokecolor{currentstroke}%
\pgfsetstrokeopacity{0.000000}%
\pgfsetdash{}{0pt}%
\pgfpathmoveto{\pgfqpoint{2.267455in}{1.613090in}}%
\pgfpathlineto{\pgfqpoint{2.276208in}{1.613090in}}%
\pgfpathlineto{\pgfqpoint{2.276208in}{1.827492in}}%
\pgfpathlineto{\pgfqpoint{2.267455in}{1.827492in}}%
\pgfpathlineto{\pgfqpoint{2.267455in}{1.613090in}}%
\pgfpathclose%
\pgfusepath{fill}%
\end{pgfscope}%
\begin{pgfscope}%
\pgfpathrectangle{\pgfqpoint{0.804646in}{0.600000in}}{\pgfqpoint{2.573292in}{2.070576in}}%
\pgfusepath{clip}%
\pgfsetbuttcap%
\pgfsetmiterjoin%
\definecolor{currentfill}{rgb}{0.066899,0.263188,0.377594}%
\pgfsetfillcolor{currentfill}%
\pgfsetlinewidth{0.000000pt}%
\definecolor{currentstroke}{rgb}{0.000000,0.000000,0.000000}%
\pgfsetstrokecolor{currentstroke}%
\pgfsetstrokeopacity{0.000000}%
\pgfsetdash{}{0pt}%
\pgfpathmoveto{\pgfqpoint{2.278397in}{1.613090in}}%
\pgfpathlineto{\pgfqpoint{2.287150in}{1.613090in}}%
\pgfpathlineto{\pgfqpoint{2.287150in}{1.828114in}}%
\pgfpathlineto{\pgfqpoint{2.278397in}{1.828114in}}%
\pgfpathlineto{\pgfqpoint{2.278397in}{1.613090in}}%
\pgfpathclose%
\pgfusepath{fill}%
\end{pgfscope}%
\begin{pgfscope}%
\pgfpathrectangle{\pgfqpoint{0.804646in}{0.600000in}}{\pgfqpoint{2.573292in}{2.070576in}}%
\pgfusepath{clip}%
\pgfsetbuttcap%
\pgfsetmiterjoin%
\definecolor{currentfill}{rgb}{0.066899,0.263188,0.377594}%
\pgfsetfillcolor{currentfill}%
\pgfsetlinewidth{0.000000pt}%
\definecolor{currentstroke}{rgb}{0.000000,0.000000,0.000000}%
\pgfsetstrokecolor{currentstroke}%
\pgfsetstrokeopacity{0.000000}%
\pgfsetdash{}{0pt}%
\pgfpathmoveto{\pgfqpoint{2.289339in}{1.613090in}}%
\pgfpathlineto{\pgfqpoint{2.298092in}{1.613090in}}%
\pgfpathlineto{\pgfqpoint{2.298092in}{1.830409in}}%
\pgfpathlineto{\pgfqpoint{2.289339in}{1.830409in}}%
\pgfpathlineto{\pgfqpoint{2.289339in}{1.613090in}}%
\pgfpathclose%
\pgfusepath{fill}%
\end{pgfscope}%
\begin{pgfscope}%
\pgfpathrectangle{\pgfqpoint{0.804646in}{0.600000in}}{\pgfqpoint{2.573292in}{2.070576in}}%
\pgfusepath{clip}%
\pgfsetbuttcap%
\pgfsetmiterjoin%
\definecolor{currentfill}{rgb}{0.066899,0.263188,0.377594}%
\pgfsetfillcolor{currentfill}%
\pgfsetlinewidth{0.000000pt}%
\definecolor{currentstroke}{rgb}{0.000000,0.000000,0.000000}%
\pgfsetstrokecolor{currentstroke}%
\pgfsetstrokeopacity{0.000000}%
\pgfsetdash{}{0pt}%
\pgfpathmoveto{\pgfqpoint{2.300280in}{1.613090in}}%
\pgfpathlineto{\pgfqpoint{2.309034in}{1.613090in}}%
\pgfpathlineto{\pgfqpoint{2.309034in}{1.842360in}}%
\pgfpathlineto{\pgfqpoint{2.300280in}{1.842360in}}%
\pgfpathlineto{\pgfqpoint{2.300280in}{1.613090in}}%
\pgfpathclose%
\pgfusepath{fill}%
\end{pgfscope}%
\begin{pgfscope}%
\pgfpathrectangle{\pgfqpoint{0.804646in}{0.600000in}}{\pgfqpoint{2.573292in}{2.070576in}}%
\pgfusepath{clip}%
\pgfsetbuttcap%
\pgfsetmiterjoin%
\definecolor{currentfill}{rgb}{0.066899,0.263188,0.377594}%
\pgfsetfillcolor{currentfill}%
\pgfsetlinewidth{0.000000pt}%
\definecolor{currentstroke}{rgb}{0.000000,0.000000,0.000000}%
\pgfsetstrokecolor{currentstroke}%
\pgfsetstrokeopacity{0.000000}%
\pgfsetdash{}{0pt}%
\pgfpathmoveto{\pgfqpoint{2.311222in}{1.613090in}}%
\pgfpathlineto{\pgfqpoint{2.319976in}{1.613090in}}%
\pgfpathlineto{\pgfqpoint{2.319976in}{1.850858in}}%
\pgfpathlineto{\pgfqpoint{2.311222in}{1.850858in}}%
\pgfpathlineto{\pgfqpoint{2.311222in}{1.613090in}}%
\pgfpathclose%
\pgfusepath{fill}%
\end{pgfscope}%
\begin{pgfscope}%
\pgfpathrectangle{\pgfqpoint{0.804646in}{0.600000in}}{\pgfqpoint{2.573292in}{2.070576in}}%
\pgfusepath{clip}%
\pgfsetbuttcap%
\pgfsetmiterjoin%
\definecolor{currentfill}{rgb}{0.066899,0.263188,0.377594}%
\pgfsetfillcolor{currentfill}%
\pgfsetlinewidth{0.000000pt}%
\definecolor{currentstroke}{rgb}{0.000000,0.000000,0.000000}%
\pgfsetstrokecolor{currentstroke}%
\pgfsetstrokeopacity{0.000000}%
\pgfsetdash{}{0pt}%
\pgfpathmoveto{\pgfqpoint{2.322164in}{1.613090in}}%
\pgfpathlineto{\pgfqpoint{2.330917in}{1.613090in}}%
\pgfpathlineto{\pgfqpoint{2.330917in}{1.852527in}}%
\pgfpathlineto{\pgfqpoint{2.322164in}{1.852527in}}%
\pgfpathlineto{\pgfqpoint{2.322164in}{1.613090in}}%
\pgfpathclose%
\pgfusepath{fill}%
\end{pgfscope}%
\begin{pgfscope}%
\pgfpathrectangle{\pgfqpoint{0.804646in}{0.600000in}}{\pgfqpoint{2.573292in}{2.070576in}}%
\pgfusepath{clip}%
\pgfsetbuttcap%
\pgfsetmiterjoin%
\definecolor{currentfill}{rgb}{0.066899,0.263188,0.377594}%
\pgfsetfillcolor{currentfill}%
\pgfsetlinewidth{0.000000pt}%
\definecolor{currentstroke}{rgb}{0.000000,0.000000,0.000000}%
\pgfsetstrokecolor{currentstroke}%
\pgfsetstrokeopacity{0.000000}%
\pgfsetdash{}{0pt}%
\pgfpathmoveto{\pgfqpoint{2.333106in}{1.613090in}}%
\pgfpathlineto{\pgfqpoint{2.341859in}{1.613090in}}%
\pgfpathlineto{\pgfqpoint{2.341859in}{1.864829in}}%
\pgfpathlineto{\pgfqpoint{2.333106in}{1.864829in}}%
\pgfpathlineto{\pgfqpoint{2.333106in}{1.613090in}}%
\pgfpathclose%
\pgfusepath{fill}%
\end{pgfscope}%
\begin{pgfscope}%
\pgfpathrectangle{\pgfqpoint{0.804646in}{0.600000in}}{\pgfqpoint{2.573292in}{2.070576in}}%
\pgfusepath{clip}%
\pgfsetbuttcap%
\pgfsetmiterjoin%
\definecolor{currentfill}{rgb}{0.066899,0.263188,0.377594}%
\pgfsetfillcolor{currentfill}%
\pgfsetlinewidth{0.000000pt}%
\definecolor{currentstroke}{rgb}{0.000000,0.000000,0.000000}%
\pgfsetstrokecolor{currentstroke}%
\pgfsetstrokeopacity{0.000000}%
\pgfsetdash{}{0pt}%
\pgfpathmoveto{\pgfqpoint{2.344048in}{1.613090in}}%
\pgfpathlineto{\pgfqpoint{2.352801in}{1.613090in}}%
\pgfpathlineto{\pgfqpoint{2.352801in}{1.855013in}}%
\pgfpathlineto{\pgfqpoint{2.344048in}{1.855013in}}%
\pgfpathlineto{\pgfqpoint{2.344048in}{1.613090in}}%
\pgfpathclose%
\pgfusepath{fill}%
\end{pgfscope}%
\begin{pgfscope}%
\pgfpathrectangle{\pgfqpoint{0.804646in}{0.600000in}}{\pgfqpoint{2.573292in}{2.070576in}}%
\pgfusepath{clip}%
\pgfsetbuttcap%
\pgfsetmiterjoin%
\definecolor{currentfill}{rgb}{0.066899,0.263188,0.377594}%
\pgfsetfillcolor{currentfill}%
\pgfsetlinewidth{0.000000pt}%
\definecolor{currentstroke}{rgb}{0.000000,0.000000,0.000000}%
\pgfsetstrokecolor{currentstroke}%
\pgfsetstrokeopacity{0.000000}%
\pgfsetdash{}{0pt}%
\pgfpathmoveto{\pgfqpoint{2.354989in}{1.613090in}}%
\pgfpathlineto{\pgfqpoint{2.363743in}{1.613090in}}%
\pgfpathlineto{\pgfqpoint{2.363743in}{1.869566in}}%
\pgfpathlineto{\pgfqpoint{2.354989in}{1.869566in}}%
\pgfpathlineto{\pgfqpoint{2.354989in}{1.613090in}}%
\pgfpathclose%
\pgfusepath{fill}%
\end{pgfscope}%
\begin{pgfscope}%
\pgfpathrectangle{\pgfqpoint{0.804646in}{0.600000in}}{\pgfqpoint{2.573292in}{2.070576in}}%
\pgfusepath{clip}%
\pgfsetbuttcap%
\pgfsetmiterjoin%
\definecolor{currentfill}{rgb}{0.066899,0.263188,0.377594}%
\pgfsetfillcolor{currentfill}%
\pgfsetlinewidth{0.000000pt}%
\definecolor{currentstroke}{rgb}{0.000000,0.000000,0.000000}%
\pgfsetstrokecolor{currentstroke}%
\pgfsetstrokeopacity{0.000000}%
\pgfsetdash{}{0pt}%
\pgfpathmoveto{\pgfqpoint{2.365931in}{1.613090in}}%
\pgfpathlineto{\pgfqpoint{2.374685in}{1.613090in}}%
\pgfpathlineto{\pgfqpoint{2.374685in}{1.874698in}}%
\pgfpathlineto{\pgfqpoint{2.365931in}{1.874698in}}%
\pgfpathlineto{\pgfqpoint{2.365931in}{1.613090in}}%
\pgfpathclose%
\pgfusepath{fill}%
\end{pgfscope}%
\begin{pgfscope}%
\pgfpathrectangle{\pgfqpoint{0.804646in}{0.600000in}}{\pgfqpoint{2.573292in}{2.070576in}}%
\pgfusepath{clip}%
\pgfsetbuttcap%
\pgfsetmiterjoin%
\definecolor{currentfill}{rgb}{0.066899,0.263188,0.377594}%
\pgfsetfillcolor{currentfill}%
\pgfsetlinewidth{0.000000pt}%
\definecolor{currentstroke}{rgb}{0.000000,0.000000,0.000000}%
\pgfsetstrokecolor{currentstroke}%
\pgfsetstrokeopacity{0.000000}%
\pgfsetdash{}{0pt}%
\pgfpathmoveto{\pgfqpoint{2.376873in}{1.613090in}}%
\pgfpathlineto{\pgfqpoint{2.385626in}{1.613090in}}%
\pgfpathlineto{\pgfqpoint{2.385626in}{1.874494in}}%
\pgfpathlineto{\pgfqpoint{2.376873in}{1.874494in}}%
\pgfpathlineto{\pgfqpoint{2.376873in}{1.613090in}}%
\pgfpathclose%
\pgfusepath{fill}%
\end{pgfscope}%
\begin{pgfscope}%
\pgfpathrectangle{\pgfqpoint{0.804646in}{0.600000in}}{\pgfqpoint{2.573292in}{2.070576in}}%
\pgfusepath{clip}%
\pgfsetbuttcap%
\pgfsetmiterjoin%
\definecolor{currentfill}{rgb}{0.066899,0.263188,0.377594}%
\pgfsetfillcolor{currentfill}%
\pgfsetlinewidth{0.000000pt}%
\definecolor{currentstroke}{rgb}{0.000000,0.000000,0.000000}%
\pgfsetstrokecolor{currentstroke}%
\pgfsetstrokeopacity{0.000000}%
\pgfsetdash{}{0pt}%
\pgfpathmoveto{\pgfqpoint{2.387815in}{1.613090in}}%
\pgfpathlineto{\pgfqpoint{2.396568in}{1.613090in}}%
\pgfpathlineto{\pgfqpoint{2.396568in}{1.887766in}}%
\pgfpathlineto{\pgfqpoint{2.387815in}{1.887766in}}%
\pgfpathlineto{\pgfqpoint{2.387815in}{1.613090in}}%
\pgfpathclose%
\pgfusepath{fill}%
\end{pgfscope}%
\begin{pgfscope}%
\pgfpathrectangle{\pgfqpoint{0.804646in}{0.600000in}}{\pgfqpoint{2.573292in}{2.070576in}}%
\pgfusepath{clip}%
\pgfsetbuttcap%
\pgfsetmiterjoin%
\definecolor{currentfill}{rgb}{0.066899,0.263188,0.377594}%
\pgfsetfillcolor{currentfill}%
\pgfsetlinewidth{0.000000pt}%
\definecolor{currentstroke}{rgb}{0.000000,0.000000,0.000000}%
\pgfsetstrokecolor{currentstroke}%
\pgfsetstrokeopacity{0.000000}%
\pgfsetdash{}{0pt}%
\pgfpathmoveto{\pgfqpoint{2.398757in}{1.613090in}}%
\pgfpathlineto{\pgfqpoint{2.407510in}{1.613090in}}%
\pgfpathlineto{\pgfqpoint{2.407510in}{1.890729in}}%
\pgfpathlineto{\pgfqpoint{2.398757in}{1.890729in}}%
\pgfpathlineto{\pgfqpoint{2.398757in}{1.613090in}}%
\pgfpathclose%
\pgfusepath{fill}%
\end{pgfscope}%
\begin{pgfscope}%
\pgfpathrectangle{\pgfqpoint{0.804646in}{0.600000in}}{\pgfqpoint{2.573292in}{2.070576in}}%
\pgfusepath{clip}%
\pgfsetbuttcap%
\pgfsetmiterjoin%
\definecolor{currentfill}{rgb}{0.066899,0.263188,0.377594}%
\pgfsetfillcolor{currentfill}%
\pgfsetlinewidth{0.000000pt}%
\definecolor{currentstroke}{rgb}{0.000000,0.000000,0.000000}%
\pgfsetstrokecolor{currentstroke}%
\pgfsetstrokeopacity{0.000000}%
\pgfsetdash{}{0pt}%
\pgfpathmoveto{\pgfqpoint{2.409698in}{1.613090in}}%
\pgfpathlineto{\pgfqpoint{2.418452in}{1.613090in}}%
\pgfpathlineto{\pgfqpoint{2.418452in}{1.889496in}}%
\pgfpathlineto{\pgfqpoint{2.409698in}{1.889496in}}%
\pgfpathlineto{\pgfqpoint{2.409698in}{1.613090in}}%
\pgfpathclose%
\pgfusepath{fill}%
\end{pgfscope}%
\begin{pgfscope}%
\pgfpathrectangle{\pgfqpoint{0.804646in}{0.600000in}}{\pgfqpoint{2.573292in}{2.070576in}}%
\pgfusepath{clip}%
\pgfsetbuttcap%
\pgfsetmiterjoin%
\definecolor{currentfill}{rgb}{0.066899,0.263188,0.377594}%
\pgfsetfillcolor{currentfill}%
\pgfsetlinewidth{0.000000pt}%
\definecolor{currentstroke}{rgb}{0.000000,0.000000,0.000000}%
\pgfsetstrokecolor{currentstroke}%
\pgfsetstrokeopacity{0.000000}%
\pgfsetdash{}{0pt}%
\pgfpathmoveto{\pgfqpoint{2.420640in}{1.613090in}}%
\pgfpathlineto{\pgfqpoint{2.429394in}{1.613090in}}%
\pgfpathlineto{\pgfqpoint{2.429394in}{1.878100in}}%
\pgfpathlineto{\pgfqpoint{2.420640in}{1.878100in}}%
\pgfpathlineto{\pgfqpoint{2.420640in}{1.613090in}}%
\pgfpathclose%
\pgfusepath{fill}%
\end{pgfscope}%
\begin{pgfscope}%
\pgfpathrectangle{\pgfqpoint{0.804646in}{0.600000in}}{\pgfqpoint{2.573292in}{2.070576in}}%
\pgfusepath{clip}%
\pgfsetbuttcap%
\pgfsetmiterjoin%
\definecolor{currentfill}{rgb}{0.066899,0.263188,0.377594}%
\pgfsetfillcolor{currentfill}%
\pgfsetlinewidth{0.000000pt}%
\definecolor{currentstroke}{rgb}{0.000000,0.000000,0.000000}%
\pgfsetstrokecolor{currentstroke}%
\pgfsetstrokeopacity{0.000000}%
\pgfsetdash{}{0pt}%
\pgfpathmoveto{\pgfqpoint{2.431582in}{1.613090in}}%
\pgfpathlineto{\pgfqpoint{2.440335in}{1.613090in}}%
\pgfpathlineto{\pgfqpoint{2.440335in}{1.870504in}}%
\pgfpathlineto{\pgfqpoint{2.431582in}{1.870504in}}%
\pgfpathlineto{\pgfqpoint{2.431582in}{1.613090in}}%
\pgfpathclose%
\pgfusepath{fill}%
\end{pgfscope}%
\begin{pgfscope}%
\pgfpathrectangle{\pgfqpoint{0.804646in}{0.600000in}}{\pgfqpoint{2.573292in}{2.070576in}}%
\pgfusepath{clip}%
\pgfsetbuttcap%
\pgfsetmiterjoin%
\definecolor{currentfill}{rgb}{0.066899,0.263188,0.377594}%
\pgfsetfillcolor{currentfill}%
\pgfsetlinewidth{0.000000pt}%
\definecolor{currentstroke}{rgb}{0.000000,0.000000,0.000000}%
\pgfsetstrokecolor{currentstroke}%
\pgfsetstrokeopacity{0.000000}%
\pgfsetdash{}{0pt}%
\pgfpathmoveto{\pgfqpoint{2.442524in}{1.613090in}}%
\pgfpathlineto{\pgfqpoint{2.451277in}{1.613090in}}%
\pgfpathlineto{\pgfqpoint{2.451277in}{1.829026in}}%
\pgfpathlineto{\pgfqpoint{2.442524in}{1.829026in}}%
\pgfpathlineto{\pgfqpoint{2.442524in}{1.613090in}}%
\pgfpathclose%
\pgfusepath{fill}%
\end{pgfscope}%
\begin{pgfscope}%
\pgfpathrectangle{\pgfqpoint{0.804646in}{0.600000in}}{\pgfqpoint{2.573292in}{2.070576in}}%
\pgfusepath{clip}%
\pgfsetbuttcap%
\pgfsetmiterjoin%
\definecolor{currentfill}{rgb}{0.066899,0.263188,0.377594}%
\pgfsetfillcolor{currentfill}%
\pgfsetlinewidth{0.000000pt}%
\definecolor{currentstroke}{rgb}{0.000000,0.000000,0.000000}%
\pgfsetstrokecolor{currentstroke}%
\pgfsetstrokeopacity{0.000000}%
\pgfsetdash{}{0pt}%
\pgfpathmoveto{\pgfqpoint{2.453466in}{1.613090in}}%
\pgfpathlineto{\pgfqpoint{2.462219in}{1.613090in}}%
\pgfpathlineto{\pgfqpoint{2.462219in}{1.800951in}}%
\pgfpathlineto{\pgfqpoint{2.453466in}{1.800951in}}%
\pgfpathlineto{\pgfqpoint{2.453466in}{1.613090in}}%
\pgfpathclose%
\pgfusepath{fill}%
\end{pgfscope}%
\begin{pgfscope}%
\pgfpathrectangle{\pgfqpoint{0.804646in}{0.600000in}}{\pgfqpoint{2.573292in}{2.070576in}}%
\pgfusepath{clip}%
\pgfsetbuttcap%
\pgfsetmiterjoin%
\definecolor{currentfill}{rgb}{0.066899,0.263188,0.377594}%
\pgfsetfillcolor{currentfill}%
\pgfsetlinewidth{0.000000pt}%
\definecolor{currentstroke}{rgb}{0.000000,0.000000,0.000000}%
\pgfsetstrokecolor{currentstroke}%
\pgfsetstrokeopacity{0.000000}%
\pgfsetdash{}{0pt}%
\pgfpathmoveto{\pgfqpoint{2.464407in}{1.613090in}}%
\pgfpathlineto{\pgfqpoint{2.473161in}{1.613090in}}%
\pgfpathlineto{\pgfqpoint{2.473161in}{1.768657in}}%
\pgfpathlineto{\pgfqpoint{2.464407in}{1.768657in}}%
\pgfpathlineto{\pgfqpoint{2.464407in}{1.613090in}}%
\pgfpathclose%
\pgfusepath{fill}%
\end{pgfscope}%
\begin{pgfscope}%
\pgfpathrectangle{\pgfqpoint{0.804646in}{0.600000in}}{\pgfqpoint{2.573292in}{2.070576in}}%
\pgfusepath{clip}%
\pgfsetbuttcap%
\pgfsetmiterjoin%
\definecolor{currentfill}{rgb}{0.066899,0.263188,0.377594}%
\pgfsetfillcolor{currentfill}%
\pgfsetlinewidth{0.000000pt}%
\definecolor{currentstroke}{rgb}{0.000000,0.000000,0.000000}%
\pgfsetstrokecolor{currentstroke}%
\pgfsetstrokeopacity{0.000000}%
\pgfsetdash{}{0pt}%
\pgfpathmoveto{\pgfqpoint{2.475349in}{1.613090in}}%
\pgfpathlineto{\pgfqpoint{2.484103in}{1.613090in}}%
\pgfpathlineto{\pgfqpoint{2.484103in}{1.729166in}}%
\pgfpathlineto{\pgfqpoint{2.475349in}{1.729166in}}%
\pgfpathlineto{\pgfqpoint{2.475349in}{1.613090in}}%
\pgfpathclose%
\pgfusepath{fill}%
\end{pgfscope}%
\begin{pgfscope}%
\pgfpathrectangle{\pgfqpoint{0.804646in}{0.600000in}}{\pgfqpoint{2.573292in}{2.070576in}}%
\pgfusepath{clip}%
\pgfsetbuttcap%
\pgfsetmiterjoin%
\definecolor{currentfill}{rgb}{0.066899,0.263188,0.377594}%
\pgfsetfillcolor{currentfill}%
\pgfsetlinewidth{0.000000pt}%
\definecolor{currentstroke}{rgb}{0.000000,0.000000,0.000000}%
\pgfsetstrokecolor{currentstroke}%
\pgfsetstrokeopacity{0.000000}%
\pgfsetdash{}{0pt}%
\pgfpathmoveto{\pgfqpoint{2.486291in}{1.613090in}}%
\pgfpathlineto{\pgfqpoint{2.495044in}{1.613090in}}%
\pgfpathlineto{\pgfqpoint{2.495044in}{1.732241in}}%
\pgfpathlineto{\pgfqpoint{2.486291in}{1.732241in}}%
\pgfpathlineto{\pgfqpoint{2.486291in}{1.613090in}}%
\pgfpathclose%
\pgfusepath{fill}%
\end{pgfscope}%
\begin{pgfscope}%
\pgfpathrectangle{\pgfqpoint{0.804646in}{0.600000in}}{\pgfqpoint{2.573292in}{2.070576in}}%
\pgfusepath{clip}%
\pgfsetbuttcap%
\pgfsetmiterjoin%
\definecolor{currentfill}{rgb}{0.066899,0.263188,0.377594}%
\pgfsetfillcolor{currentfill}%
\pgfsetlinewidth{0.000000pt}%
\definecolor{currentstroke}{rgb}{0.000000,0.000000,0.000000}%
\pgfsetstrokecolor{currentstroke}%
\pgfsetstrokeopacity{0.000000}%
\pgfsetdash{}{0pt}%
\pgfpathmoveto{\pgfqpoint{2.497233in}{1.613090in}}%
\pgfpathlineto{\pgfqpoint{2.505986in}{1.613090in}}%
\pgfpathlineto{\pgfqpoint{2.505986in}{1.711068in}}%
\pgfpathlineto{\pgfqpoint{2.497233in}{1.711068in}}%
\pgfpathlineto{\pgfqpoint{2.497233in}{1.613090in}}%
\pgfpathclose%
\pgfusepath{fill}%
\end{pgfscope}%
\begin{pgfscope}%
\pgfpathrectangle{\pgfqpoint{0.804646in}{0.600000in}}{\pgfqpoint{2.573292in}{2.070576in}}%
\pgfusepath{clip}%
\pgfsetbuttcap%
\pgfsetmiterjoin%
\definecolor{currentfill}{rgb}{0.066899,0.263188,0.377594}%
\pgfsetfillcolor{currentfill}%
\pgfsetlinewidth{0.000000pt}%
\definecolor{currentstroke}{rgb}{0.000000,0.000000,0.000000}%
\pgfsetstrokecolor{currentstroke}%
\pgfsetstrokeopacity{0.000000}%
\pgfsetdash{}{0pt}%
\pgfpathmoveto{\pgfqpoint{2.508174in}{1.613090in}}%
\pgfpathlineto{\pgfqpoint{2.516928in}{1.613090in}}%
\pgfpathlineto{\pgfqpoint{2.516928in}{1.701047in}}%
\pgfpathlineto{\pgfqpoint{2.508174in}{1.701047in}}%
\pgfpathlineto{\pgfqpoint{2.508174in}{1.613090in}}%
\pgfpathclose%
\pgfusepath{fill}%
\end{pgfscope}%
\begin{pgfscope}%
\pgfpathrectangle{\pgfqpoint{0.804646in}{0.600000in}}{\pgfqpoint{2.573292in}{2.070576in}}%
\pgfusepath{clip}%
\pgfsetbuttcap%
\pgfsetmiterjoin%
\definecolor{currentfill}{rgb}{0.066899,0.263188,0.377594}%
\pgfsetfillcolor{currentfill}%
\pgfsetlinewidth{0.000000pt}%
\definecolor{currentstroke}{rgb}{0.000000,0.000000,0.000000}%
\pgfsetstrokecolor{currentstroke}%
\pgfsetstrokeopacity{0.000000}%
\pgfsetdash{}{0pt}%
\pgfpathmoveto{\pgfqpoint{2.519116in}{1.613090in}}%
\pgfpathlineto{\pgfqpoint{2.527870in}{1.613090in}}%
\pgfpathlineto{\pgfqpoint{2.527870in}{1.688431in}}%
\pgfpathlineto{\pgfqpoint{2.519116in}{1.688431in}}%
\pgfpathlineto{\pgfqpoint{2.519116in}{1.613090in}}%
\pgfpathclose%
\pgfusepath{fill}%
\end{pgfscope}%
\begin{pgfscope}%
\pgfpathrectangle{\pgfqpoint{0.804646in}{0.600000in}}{\pgfqpoint{2.573292in}{2.070576in}}%
\pgfusepath{clip}%
\pgfsetbuttcap%
\pgfsetmiterjoin%
\definecolor{currentfill}{rgb}{0.066899,0.263188,0.377594}%
\pgfsetfillcolor{currentfill}%
\pgfsetlinewidth{0.000000pt}%
\definecolor{currentstroke}{rgb}{0.000000,0.000000,0.000000}%
\pgfsetstrokecolor{currentstroke}%
\pgfsetstrokeopacity{0.000000}%
\pgfsetdash{}{0pt}%
\pgfpathmoveto{\pgfqpoint{2.530058in}{1.613090in}}%
\pgfpathlineto{\pgfqpoint{2.538812in}{1.613090in}}%
\pgfpathlineto{\pgfqpoint{2.538812in}{1.678223in}}%
\pgfpathlineto{\pgfqpoint{2.530058in}{1.678223in}}%
\pgfpathlineto{\pgfqpoint{2.530058in}{1.613090in}}%
\pgfpathclose%
\pgfusepath{fill}%
\end{pgfscope}%
\begin{pgfscope}%
\pgfpathrectangle{\pgfqpoint{0.804646in}{0.600000in}}{\pgfqpoint{2.573292in}{2.070576in}}%
\pgfusepath{clip}%
\pgfsetbuttcap%
\pgfsetmiterjoin%
\definecolor{currentfill}{rgb}{0.066899,0.263188,0.377594}%
\pgfsetfillcolor{currentfill}%
\pgfsetlinewidth{0.000000pt}%
\definecolor{currentstroke}{rgb}{0.000000,0.000000,0.000000}%
\pgfsetstrokecolor{currentstroke}%
\pgfsetstrokeopacity{0.000000}%
\pgfsetdash{}{0pt}%
\pgfpathmoveto{\pgfqpoint{2.541000in}{1.613090in}}%
\pgfpathlineto{\pgfqpoint{2.549753in}{1.613090in}}%
\pgfpathlineto{\pgfqpoint{2.549753in}{1.663859in}}%
\pgfpathlineto{\pgfqpoint{2.541000in}{1.663859in}}%
\pgfpathlineto{\pgfqpoint{2.541000in}{1.613090in}}%
\pgfpathclose%
\pgfusepath{fill}%
\end{pgfscope}%
\begin{pgfscope}%
\pgfpathrectangle{\pgfqpoint{0.804646in}{0.600000in}}{\pgfqpoint{2.573292in}{2.070576in}}%
\pgfusepath{clip}%
\pgfsetbuttcap%
\pgfsetmiterjoin%
\definecolor{currentfill}{rgb}{0.066899,0.263188,0.377594}%
\pgfsetfillcolor{currentfill}%
\pgfsetlinewidth{0.000000pt}%
\definecolor{currentstroke}{rgb}{0.000000,0.000000,0.000000}%
\pgfsetstrokecolor{currentstroke}%
\pgfsetstrokeopacity{0.000000}%
\pgfsetdash{}{0pt}%
\pgfpathmoveto{\pgfqpoint{2.551942in}{1.613090in}}%
\pgfpathlineto{\pgfqpoint{2.560695in}{1.613090in}}%
\pgfpathlineto{\pgfqpoint{2.560695in}{1.670769in}}%
\pgfpathlineto{\pgfqpoint{2.551942in}{1.670769in}}%
\pgfpathlineto{\pgfqpoint{2.551942in}{1.613090in}}%
\pgfpathclose%
\pgfusepath{fill}%
\end{pgfscope}%
\begin{pgfscope}%
\pgfpathrectangle{\pgfqpoint{0.804646in}{0.600000in}}{\pgfqpoint{2.573292in}{2.070576in}}%
\pgfusepath{clip}%
\pgfsetbuttcap%
\pgfsetmiterjoin%
\definecolor{currentfill}{rgb}{0.066899,0.263188,0.377594}%
\pgfsetfillcolor{currentfill}%
\pgfsetlinewidth{0.000000pt}%
\definecolor{currentstroke}{rgb}{0.000000,0.000000,0.000000}%
\pgfsetstrokecolor{currentstroke}%
\pgfsetstrokeopacity{0.000000}%
\pgfsetdash{}{0pt}%
\pgfpathmoveto{\pgfqpoint{2.562883in}{1.613090in}}%
\pgfpathlineto{\pgfqpoint{2.571637in}{1.613090in}}%
\pgfpathlineto{\pgfqpoint{2.571637in}{1.670895in}}%
\pgfpathlineto{\pgfqpoint{2.562883in}{1.670895in}}%
\pgfpathlineto{\pgfqpoint{2.562883in}{1.613090in}}%
\pgfpathclose%
\pgfusepath{fill}%
\end{pgfscope}%
\begin{pgfscope}%
\pgfpathrectangle{\pgfqpoint{0.804646in}{0.600000in}}{\pgfqpoint{2.573292in}{2.070576in}}%
\pgfusepath{clip}%
\pgfsetbuttcap%
\pgfsetmiterjoin%
\definecolor{currentfill}{rgb}{0.066899,0.263188,0.377594}%
\pgfsetfillcolor{currentfill}%
\pgfsetlinewidth{0.000000pt}%
\definecolor{currentstroke}{rgb}{0.000000,0.000000,0.000000}%
\pgfsetstrokecolor{currentstroke}%
\pgfsetstrokeopacity{0.000000}%
\pgfsetdash{}{0pt}%
\pgfpathmoveto{\pgfqpoint{2.573825in}{1.613090in}}%
\pgfpathlineto{\pgfqpoint{2.582579in}{1.613090in}}%
\pgfpathlineto{\pgfqpoint{2.582579in}{1.664125in}}%
\pgfpathlineto{\pgfqpoint{2.573825in}{1.664125in}}%
\pgfpathlineto{\pgfqpoint{2.573825in}{1.613090in}}%
\pgfpathclose%
\pgfusepath{fill}%
\end{pgfscope}%
\begin{pgfscope}%
\pgfpathrectangle{\pgfqpoint{0.804646in}{0.600000in}}{\pgfqpoint{2.573292in}{2.070576in}}%
\pgfusepath{clip}%
\pgfsetbuttcap%
\pgfsetmiterjoin%
\definecolor{currentfill}{rgb}{0.066899,0.263188,0.377594}%
\pgfsetfillcolor{currentfill}%
\pgfsetlinewidth{0.000000pt}%
\definecolor{currentstroke}{rgb}{0.000000,0.000000,0.000000}%
\pgfsetstrokecolor{currentstroke}%
\pgfsetstrokeopacity{0.000000}%
\pgfsetdash{}{0pt}%
\pgfpathmoveto{\pgfqpoint{2.584767in}{1.613090in}}%
\pgfpathlineto{\pgfqpoint{2.593521in}{1.613090in}}%
\pgfpathlineto{\pgfqpoint{2.593521in}{1.671711in}}%
\pgfpathlineto{\pgfqpoint{2.584767in}{1.671711in}}%
\pgfpathlineto{\pgfqpoint{2.584767in}{1.613090in}}%
\pgfpathclose%
\pgfusepath{fill}%
\end{pgfscope}%
\begin{pgfscope}%
\pgfpathrectangle{\pgfqpoint{0.804646in}{0.600000in}}{\pgfqpoint{2.573292in}{2.070576in}}%
\pgfusepath{clip}%
\pgfsetbuttcap%
\pgfsetmiterjoin%
\definecolor{currentfill}{rgb}{0.066899,0.263188,0.377594}%
\pgfsetfillcolor{currentfill}%
\pgfsetlinewidth{0.000000pt}%
\definecolor{currentstroke}{rgb}{0.000000,0.000000,0.000000}%
\pgfsetstrokecolor{currentstroke}%
\pgfsetstrokeopacity{0.000000}%
\pgfsetdash{}{0pt}%
\pgfpathmoveto{\pgfqpoint{2.595709in}{1.613090in}}%
\pgfpathlineto{\pgfqpoint{2.604462in}{1.613090in}}%
\pgfpathlineto{\pgfqpoint{2.604462in}{1.676341in}}%
\pgfpathlineto{\pgfqpoint{2.595709in}{1.676341in}}%
\pgfpathlineto{\pgfqpoint{2.595709in}{1.613090in}}%
\pgfpathclose%
\pgfusepath{fill}%
\end{pgfscope}%
\begin{pgfscope}%
\pgfpathrectangle{\pgfqpoint{0.804646in}{0.600000in}}{\pgfqpoint{2.573292in}{2.070576in}}%
\pgfusepath{clip}%
\pgfsetbuttcap%
\pgfsetmiterjoin%
\definecolor{currentfill}{rgb}{0.066899,0.263188,0.377594}%
\pgfsetfillcolor{currentfill}%
\pgfsetlinewidth{0.000000pt}%
\definecolor{currentstroke}{rgb}{0.000000,0.000000,0.000000}%
\pgfsetstrokecolor{currentstroke}%
\pgfsetstrokeopacity{0.000000}%
\pgfsetdash{}{0pt}%
\pgfpathmoveto{\pgfqpoint{2.606651in}{1.613090in}}%
\pgfpathlineto{\pgfqpoint{2.615404in}{1.613090in}}%
\pgfpathlineto{\pgfqpoint{2.615404in}{1.676290in}}%
\pgfpathlineto{\pgfqpoint{2.606651in}{1.676290in}}%
\pgfpathlineto{\pgfqpoint{2.606651in}{1.613090in}}%
\pgfpathclose%
\pgfusepath{fill}%
\end{pgfscope}%
\begin{pgfscope}%
\pgfpathrectangle{\pgfqpoint{0.804646in}{0.600000in}}{\pgfqpoint{2.573292in}{2.070576in}}%
\pgfusepath{clip}%
\pgfsetbuttcap%
\pgfsetmiterjoin%
\definecolor{currentfill}{rgb}{0.066899,0.263188,0.377594}%
\pgfsetfillcolor{currentfill}%
\pgfsetlinewidth{0.000000pt}%
\definecolor{currentstroke}{rgb}{0.000000,0.000000,0.000000}%
\pgfsetstrokecolor{currentstroke}%
\pgfsetstrokeopacity{0.000000}%
\pgfsetdash{}{0pt}%
\pgfpathmoveto{\pgfqpoint{2.617592in}{1.613090in}}%
\pgfpathlineto{\pgfqpoint{2.626346in}{1.613090in}}%
\pgfpathlineto{\pgfqpoint{2.626346in}{1.690828in}}%
\pgfpathlineto{\pgfqpoint{2.617592in}{1.690828in}}%
\pgfpathlineto{\pgfqpoint{2.617592in}{1.613090in}}%
\pgfpathclose%
\pgfusepath{fill}%
\end{pgfscope}%
\begin{pgfscope}%
\pgfpathrectangle{\pgfqpoint{0.804646in}{0.600000in}}{\pgfqpoint{2.573292in}{2.070576in}}%
\pgfusepath{clip}%
\pgfsetbuttcap%
\pgfsetmiterjoin%
\definecolor{currentfill}{rgb}{0.066899,0.263188,0.377594}%
\pgfsetfillcolor{currentfill}%
\pgfsetlinewidth{0.000000pt}%
\definecolor{currentstroke}{rgb}{0.000000,0.000000,0.000000}%
\pgfsetstrokecolor{currentstroke}%
\pgfsetstrokeopacity{0.000000}%
\pgfsetdash{}{0pt}%
\pgfpathmoveto{\pgfqpoint{2.628534in}{1.613090in}}%
\pgfpathlineto{\pgfqpoint{2.637288in}{1.613090in}}%
\pgfpathlineto{\pgfqpoint{2.637288in}{1.688808in}}%
\pgfpathlineto{\pgfqpoint{2.628534in}{1.688808in}}%
\pgfpathlineto{\pgfqpoint{2.628534in}{1.613090in}}%
\pgfpathclose%
\pgfusepath{fill}%
\end{pgfscope}%
\begin{pgfscope}%
\pgfpathrectangle{\pgfqpoint{0.804646in}{0.600000in}}{\pgfqpoint{2.573292in}{2.070576in}}%
\pgfusepath{clip}%
\pgfsetbuttcap%
\pgfsetmiterjoin%
\definecolor{currentfill}{rgb}{0.066899,0.263188,0.377594}%
\pgfsetfillcolor{currentfill}%
\pgfsetlinewidth{0.000000pt}%
\definecolor{currentstroke}{rgb}{0.000000,0.000000,0.000000}%
\pgfsetstrokecolor{currentstroke}%
\pgfsetstrokeopacity{0.000000}%
\pgfsetdash{}{0pt}%
\pgfpathmoveto{\pgfqpoint{2.639476in}{1.613090in}}%
\pgfpathlineto{\pgfqpoint{2.648230in}{1.613090in}}%
\pgfpathlineto{\pgfqpoint{2.648230in}{1.693757in}}%
\pgfpathlineto{\pgfqpoint{2.639476in}{1.693757in}}%
\pgfpathlineto{\pgfqpoint{2.639476in}{1.613090in}}%
\pgfpathclose%
\pgfusepath{fill}%
\end{pgfscope}%
\begin{pgfscope}%
\pgfpathrectangle{\pgfqpoint{0.804646in}{0.600000in}}{\pgfqpoint{2.573292in}{2.070576in}}%
\pgfusepath{clip}%
\pgfsetbuttcap%
\pgfsetmiterjoin%
\definecolor{currentfill}{rgb}{0.066899,0.263188,0.377594}%
\pgfsetfillcolor{currentfill}%
\pgfsetlinewidth{0.000000pt}%
\definecolor{currentstroke}{rgb}{0.000000,0.000000,0.000000}%
\pgfsetstrokecolor{currentstroke}%
\pgfsetstrokeopacity{0.000000}%
\pgfsetdash{}{0pt}%
\pgfpathmoveto{\pgfqpoint{2.650418in}{1.613090in}}%
\pgfpathlineto{\pgfqpoint{2.659171in}{1.613090in}}%
\pgfpathlineto{\pgfqpoint{2.659171in}{1.710928in}}%
\pgfpathlineto{\pgfqpoint{2.650418in}{1.710928in}}%
\pgfpathlineto{\pgfqpoint{2.650418in}{1.613090in}}%
\pgfpathclose%
\pgfusepath{fill}%
\end{pgfscope}%
\begin{pgfscope}%
\pgfpathrectangle{\pgfqpoint{0.804646in}{0.600000in}}{\pgfqpoint{2.573292in}{2.070576in}}%
\pgfusepath{clip}%
\pgfsetbuttcap%
\pgfsetmiterjoin%
\definecolor{currentfill}{rgb}{0.066899,0.263188,0.377594}%
\pgfsetfillcolor{currentfill}%
\pgfsetlinewidth{0.000000pt}%
\definecolor{currentstroke}{rgb}{0.000000,0.000000,0.000000}%
\pgfsetstrokecolor{currentstroke}%
\pgfsetstrokeopacity{0.000000}%
\pgfsetdash{}{0pt}%
\pgfpathmoveto{\pgfqpoint{2.661360in}{1.613090in}}%
\pgfpathlineto{\pgfqpoint{2.670113in}{1.613090in}}%
\pgfpathlineto{\pgfqpoint{2.670113in}{1.705056in}}%
\pgfpathlineto{\pgfqpoint{2.661360in}{1.705056in}}%
\pgfpathlineto{\pgfqpoint{2.661360in}{1.613090in}}%
\pgfpathclose%
\pgfusepath{fill}%
\end{pgfscope}%
\begin{pgfscope}%
\pgfpathrectangle{\pgfqpoint{0.804646in}{0.600000in}}{\pgfqpoint{2.573292in}{2.070576in}}%
\pgfusepath{clip}%
\pgfsetbuttcap%
\pgfsetmiterjoin%
\definecolor{currentfill}{rgb}{0.066899,0.263188,0.377594}%
\pgfsetfillcolor{currentfill}%
\pgfsetlinewidth{0.000000pt}%
\definecolor{currentstroke}{rgb}{0.000000,0.000000,0.000000}%
\pgfsetstrokecolor{currentstroke}%
\pgfsetstrokeopacity{0.000000}%
\pgfsetdash{}{0pt}%
\pgfpathmoveto{\pgfqpoint{2.672301in}{1.613090in}}%
\pgfpathlineto{\pgfqpoint{2.681055in}{1.613090in}}%
\pgfpathlineto{\pgfqpoint{2.681055in}{1.720062in}}%
\pgfpathlineto{\pgfqpoint{2.672301in}{1.720062in}}%
\pgfpathlineto{\pgfqpoint{2.672301in}{1.613090in}}%
\pgfpathclose%
\pgfusepath{fill}%
\end{pgfscope}%
\begin{pgfscope}%
\pgfpathrectangle{\pgfqpoint{0.804646in}{0.600000in}}{\pgfqpoint{2.573292in}{2.070576in}}%
\pgfusepath{clip}%
\pgfsetbuttcap%
\pgfsetmiterjoin%
\definecolor{currentfill}{rgb}{0.066899,0.263188,0.377594}%
\pgfsetfillcolor{currentfill}%
\pgfsetlinewidth{0.000000pt}%
\definecolor{currentstroke}{rgb}{0.000000,0.000000,0.000000}%
\pgfsetstrokecolor{currentstroke}%
\pgfsetstrokeopacity{0.000000}%
\pgfsetdash{}{0pt}%
\pgfpathmoveto{\pgfqpoint{2.683243in}{1.613090in}}%
\pgfpathlineto{\pgfqpoint{2.691997in}{1.613090in}}%
\pgfpathlineto{\pgfqpoint{2.691997in}{1.718405in}}%
\pgfpathlineto{\pgfqpoint{2.683243in}{1.718405in}}%
\pgfpathlineto{\pgfqpoint{2.683243in}{1.613090in}}%
\pgfpathclose%
\pgfusepath{fill}%
\end{pgfscope}%
\begin{pgfscope}%
\pgfpathrectangle{\pgfqpoint{0.804646in}{0.600000in}}{\pgfqpoint{2.573292in}{2.070576in}}%
\pgfusepath{clip}%
\pgfsetbuttcap%
\pgfsetmiterjoin%
\definecolor{currentfill}{rgb}{0.066899,0.263188,0.377594}%
\pgfsetfillcolor{currentfill}%
\pgfsetlinewidth{0.000000pt}%
\definecolor{currentstroke}{rgb}{0.000000,0.000000,0.000000}%
\pgfsetstrokecolor{currentstroke}%
\pgfsetstrokeopacity{0.000000}%
\pgfsetdash{}{0pt}%
\pgfpathmoveto{\pgfqpoint{2.694185in}{1.613090in}}%
\pgfpathlineto{\pgfqpoint{2.702939in}{1.613090in}}%
\pgfpathlineto{\pgfqpoint{2.702939in}{1.704400in}}%
\pgfpathlineto{\pgfqpoint{2.694185in}{1.704400in}}%
\pgfpathlineto{\pgfqpoint{2.694185in}{1.613090in}}%
\pgfpathclose%
\pgfusepath{fill}%
\end{pgfscope}%
\begin{pgfscope}%
\pgfpathrectangle{\pgfqpoint{0.804646in}{0.600000in}}{\pgfqpoint{2.573292in}{2.070576in}}%
\pgfusepath{clip}%
\pgfsetbuttcap%
\pgfsetmiterjoin%
\definecolor{currentfill}{rgb}{0.066899,0.263188,0.377594}%
\pgfsetfillcolor{currentfill}%
\pgfsetlinewidth{0.000000pt}%
\definecolor{currentstroke}{rgb}{0.000000,0.000000,0.000000}%
\pgfsetstrokecolor{currentstroke}%
\pgfsetstrokeopacity{0.000000}%
\pgfsetdash{}{0pt}%
\pgfpathmoveto{\pgfqpoint{2.705127in}{1.613090in}}%
\pgfpathlineto{\pgfqpoint{2.713880in}{1.613090in}}%
\pgfpathlineto{\pgfqpoint{2.713880in}{1.707867in}}%
\pgfpathlineto{\pgfqpoint{2.705127in}{1.707867in}}%
\pgfpathlineto{\pgfqpoint{2.705127in}{1.613090in}}%
\pgfpathclose%
\pgfusepath{fill}%
\end{pgfscope}%
\begin{pgfscope}%
\pgfpathrectangle{\pgfqpoint{0.804646in}{0.600000in}}{\pgfqpoint{2.573292in}{2.070576in}}%
\pgfusepath{clip}%
\pgfsetbuttcap%
\pgfsetmiterjoin%
\definecolor{currentfill}{rgb}{0.066899,0.263188,0.377594}%
\pgfsetfillcolor{currentfill}%
\pgfsetlinewidth{0.000000pt}%
\definecolor{currentstroke}{rgb}{0.000000,0.000000,0.000000}%
\pgfsetstrokecolor{currentstroke}%
\pgfsetstrokeopacity{0.000000}%
\pgfsetdash{}{0pt}%
\pgfpathmoveto{\pgfqpoint{2.716069in}{1.613090in}}%
\pgfpathlineto{\pgfqpoint{2.724822in}{1.613090in}}%
\pgfpathlineto{\pgfqpoint{2.724822in}{1.694489in}}%
\pgfpathlineto{\pgfqpoint{2.716069in}{1.694489in}}%
\pgfpathlineto{\pgfqpoint{2.716069in}{1.613090in}}%
\pgfpathclose%
\pgfusepath{fill}%
\end{pgfscope}%
\begin{pgfscope}%
\pgfpathrectangle{\pgfqpoint{0.804646in}{0.600000in}}{\pgfqpoint{2.573292in}{2.070576in}}%
\pgfusepath{clip}%
\pgfsetbuttcap%
\pgfsetmiterjoin%
\definecolor{currentfill}{rgb}{0.066899,0.263188,0.377594}%
\pgfsetfillcolor{currentfill}%
\pgfsetlinewidth{0.000000pt}%
\definecolor{currentstroke}{rgb}{0.000000,0.000000,0.000000}%
\pgfsetstrokecolor{currentstroke}%
\pgfsetstrokeopacity{0.000000}%
\pgfsetdash{}{0pt}%
\pgfpathmoveto{\pgfqpoint{2.727010in}{1.613090in}}%
\pgfpathlineto{\pgfqpoint{2.735764in}{1.613090in}}%
\pgfpathlineto{\pgfqpoint{2.735764in}{1.686306in}}%
\pgfpathlineto{\pgfqpoint{2.727010in}{1.686306in}}%
\pgfpathlineto{\pgfqpoint{2.727010in}{1.613090in}}%
\pgfpathclose%
\pgfusepath{fill}%
\end{pgfscope}%
\begin{pgfscope}%
\pgfpathrectangle{\pgfqpoint{0.804646in}{0.600000in}}{\pgfqpoint{2.573292in}{2.070576in}}%
\pgfusepath{clip}%
\pgfsetbuttcap%
\pgfsetmiterjoin%
\definecolor{currentfill}{rgb}{0.066899,0.263188,0.377594}%
\pgfsetfillcolor{currentfill}%
\pgfsetlinewidth{0.000000pt}%
\definecolor{currentstroke}{rgb}{0.000000,0.000000,0.000000}%
\pgfsetstrokecolor{currentstroke}%
\pgfsetstrokeopacity{0.000000}%
\pgfsetdash{}{0pt}%
\pgfpathmoveto{\pgfqpoint{2.737952in}{1.613090in}}%
\pgfpathlineto{\pgfqpoint{2.746706in}{1.613090in}}%
\pgfpathlineto{\pgfqpoint{2.746706in}{1.670691in}}%
\pgfpathlineto{\pgfqpoint{2.737952in}{1.670691in}}%
\pgfpathlineto{\pgfqpoint{2.737952in}{1.613090in}}%
\pgfpathclose%
\pgfusepath{fill}%
\end{pgfscope}%
\begin{pgfscope}%
\pgfpathrectangle{\pgfqpoint{0.804646in}{0.600000in}}{\pgfqpoint{2.573292in}{2.070576in}}%
\pgfusepath{clip}%
\pgfsetbuttcap%
\pgfsetmiterjoin%
\definecolor{currentfill}{rgb}{0.066899,0.263188,0.377594}%
\pgfsetfillcolor{currentfill}%
\pgfsetlinewidth{0.000000pt}%
\definecolor{currentstroke}{rgb}{0.000000,0.000000,0.000000}%
\pgfsetstrokecolor{currentstroke}%
\pgfsetstrokeopacity{0.000000}%
\pgfsetdash{}{0pt}%
\pgfpathmoveto{\pgfqpoint{2.748894in}{1.613090in}}%
\pgfpathlineto{\pgfqpoint{2.757648in}{1.613090in}}%
\pgfpathlineto{\pgfqpoint{2.757648in}{1.637733in}}%
\pgfpathlineto{\pgfqpoint{2.748894in}{1.637733in}}%
\pgfpathlineto{\pgfqpoint{2.748894in}{1.613090in}}%
\pgfpathclose%
\pgfusepath{fill}%
\end{pgfscope}%
\begin{pgfscope}%
\pgfpathrectangle{\pgfqpoint{0.804646in}{0.600000in}}{\pgfqpoint{2.573292in}{2.070576in}}%
\pgfusepath{clip}%
\pgfsetbuttcap%
\pgfsetmiterjoin%
\definecolor{currentfill}{rgb}{0.066899,0.263188,0.377594}%
\pgfsetfillcolor{currentfill}%
\pgfsetlinewidth{0.000000pt}%
\definecolor{currentstroke}{rgb}{0.000000,0.000000,0.000000}%
\pgfsetstrokecolor{currentstroke}%
\pgfsetstrokeopacity{0.000000}%
\pgfsetdash{}{0pt}%
\pgfpathmoveto{\pgfqpoint{2.759836in}{1.613090in}}%
\pgfpathlineto{\pgfqpoint{2.768589in}{1.613090in}}%
\pgfpathlineto{\pgfqpoint{2.768589in}{1.599842in}}%
\pgfpathlineto{\pgfqpoint{2.759836in}{1.599842in}}%
\pgfpathlineto{\pgfqpoint{2.759836in}{1.613090in}}%
\pgfpathclose%
\pgfusepath{fill}%
\end{pgfscope}%
\begin{pgfscope}%
\pgfpathrectangle{\pgfqpoint{0.804646in}{0.600000in}}{\pgfqpoint{2.573292in}{2.070576in}}%
\pgfusepath{clip}%
\pgfsetbuttcap%
\pgfsetmiterjoin%
\definecolor{currentfill}{rgb}{0.066899,0.263188,0.377594}%
\pgfsetfillcolor{currentfill}%
\pgfsetlinewidth{0.000000pt}%
\definecolor{currentstroke}{rgb}{0.000000,0.000000,0.000000}%
\pgfsetstrokecolor{currentstroke}%
\pgfsetstrokeopacity{0.000000}%
\pgfsetdash{}{0pt}%
\pgfpathmoveto{\pgfqpoint{2.770778in}{1.613090in}}%
\pgfpathlineto{\pgfqpoint{2.779531in}{1.613090in}}%
\pgfpathlineto{\pgfqpoint{2.779531in}{1.533675in}}%
\pgfpathlineto{\pgfqpoint{2.770778in}{1.533675in}}%
\pgfpathlineto{\pgfqpoint{2.770778in}{1.613090in}}%
\pgfpathclose%
\pgfusepath{fill}%
\end{pgfscope}%
\begin{pgfscope}%
\pgfpathrectangle{\pgfqpoint{0.804646in}{0.600000in}}{\pgfqpoint{2.573292in}{2.070576in}}%
\pgfusepath{clip}%
\pgfsetbuttcap%
\pgfsetmiterjoin%
\definecolor{currentfill}{rgb}{0.066899,0.263188,0.377594}%
\pgfsetfillcolor{currentfill}%
\pgfsetlinewidth{0.000000pt}%
\definecolor{currentstroke}{rgb}{0.000000,0.000000,0.000000}%
\pgfsetstrokecolor{currentstroke}%
\pgfsetstrokeopacity{0.000000}%
\pgfsetdash{}{0pt}%
\pgfpathmoveto{\pgfqpoint{2.781719in}{1.613090in}}%
\pgfpathlineto{\pgfqpoint{2.790473in}{1.613090in}}%
\pgfpathlineto{\pgfqpoint{2.790473in}{1.459301in}}%
\pgfpathlineto{\pgfqpoint{2.781719in}{1.459301in}}%
\pgfpathlineto{\pgfqpoint{2.781719in}{1.613090in}}%
\pgfpathclose%
\pgfusepath{fill}%
\end{pgfscope}%
\begin{pgfscope}%
\pgfpathrectangle{\pgfqpoint{0.804646in}{0.600000in}}{\pgfqpoint{2.573292in}{2.070576in}}%
\pgfusepath{clip}%
\pgfsetbuttcap%
\pgfsetmiterjoin%
\definecolor{currentfill}{rgb}{0.066899,0.263188,0.377594}%
\pgfsetfillcolor{currentfill}%
\pgfsetlinewidth{0.000000pt}%
\definecolor{currentstroke}{rgb}{0.000000,0.000000,0.000000}%
\pgfsetstrokecolor{currentstroke}%
\pgfsetstrokeopacity{0.000000}%
\pgfsetdash{}{0pt}%
\pgfpathmoveto{\pgfqpoint{2.792661in}{1.613090in}}%
\pgfpathlineto{\pgfqpoint{2.801415in}{1.613090in}}%
\pgfpathlineto{\pgfqpoint{2.801415in}{1.394626in}}%
\pgfpathlineto{\pgfqpoint{2.792661in}{1.394626in}}%
\pgfpathlineto{\pgfqpoint{2.792661in}{1.613090in}}%
\pgfpathclose%
\pgfusepath{fill}%
\end{pgfscope}%
\begin{pgfscope}%
\pgfpathrectangle{\pgfqpoint{0.804646in}{0.600000in}}{\pgfqpoint{2.573292in}{2.070576in}}%
\pgfusepath{clip}%
\pgfsetbuttcap%
\pgfsetmiterjoin%
\definecolor{currentfill}{rgb}{0.066899,0.263188,0.377594}%
\pgfsetfillcolor{currentfill}%
\pgfsetlinewidth{0.000000pt}%
\definecolor{currentstroke}{rgb}{0.000000,0.000000,0.000000}%
\pgfsetstrokecolor{currentstroke}%
\pgfsetstrokeopacity{0.000000}%
\pgfsetdash{}{0pt}%
\pgfpathmoveto{\pgfqpoint{2.803603in}{1.613090in}}%
\pgfpathlineto{\pgfqpoint{2.812357in}{1.613090in}}%
\pgfpathlineto{\pgfqpoint{2.812357in}{1.364800in}}%
\pgfpathlineto{\pgfqpoint{2.803603in}{1.364800in}}%
\pgfpathlineto{\pgfqpoint{2.803603in}{1.613090in}}%
\pgfpathclose%
\pgfusepath{fill}%
\end{pgfscope}%
\begin{pgfscope}%
\pgfpathrectangle{\pgfqpoint{0.804646in}{0.600000in}}{\pgfqpoint{2.573292in}{2.070576in}}%
\pgfusepath{clip}%
\pgfsetbuttcap%
\pgfsetmiterjoin%
\definecolor{currentfill}{rgb}{0.066899,0.263188,0.377594}%
\pgfsetfillcolor{currentfill}%
\pgfsetlinewidth{0.000000pt}%
\definecolor{currentstroke}{rgb}{0.000000,0.000000,0.000000}%
\pgfsetstrokecolor{currentstroke}%
\pgfsetstrokeopacity{0.000000}%
\pgfsetdash{}{0pt}%
\pgfpathmoveto{\pgfqpoint{2.814545in}{1.613090in}}%
\pgfpathlineto{\pgfqpoint{2.823298in}{1.613090in}}%
\pgfpathlineto{\pgfqpoint{2.823298in}{1.344433in}}%
\pgfpathlineto{\pgfqpoint{2.814545in}{1.344433in}}%
\pgfpathlineto{\pgfqpoint{2.814545in}{1.613090in}}%
\pgfpathclose%
\pgfusepath{fill}%
\end{pgfscope}%
\begin{pgfscope}%
\pgfpathrectangle{\pgfqpoint{0.804646in}{0.600000in}}{\pgfqpoint{2.573292in}{2.070576in}}%
\pgfusepath{clip}%
\pgfsetbuttcap%
\pgfsetmiterjoin%
\definecolor{currentfill}{rgb}{0.066899,0.263188,0.377594}%
\pgfsetfillcolor{currentfill}%
\pgfsetlinewidth{0.000000pt}%
\definecolor{currentstroke}{rgb}{0.000000,0.000000,0.000000}%
\pgfsetstrokecolor{currentstroke}%
\pgfsetstrokeopacity{0.000000}%
\pgfsetdash{}{0pt}%
\pgfpathmoveto{\pgfqpoint{2.825487in}{1.613090in}}%
\pgfpathlineto{\pgfqpoint{2.834240in}{1.613090in}}%
\pgfpathlineto{\pgfqpoint{2.834240in}{1.344663in}}%
\pgfpathlineto{\pgfqpoint{2.825487in}{1.344663in}}%
\pgfpathlineto{\pgfqpoint{2.825487in}{1.613090in}}%
\pgfpathclose%
\pgfusepath{fill}%
\end{pgfscope}%
\begin{pgfscope}%
\pgfpathrectangle{\pgfqpoint{0.804646in}{0.600000in}}{\pgfqpoint{2.573292in}{2.070576in}}%
\pgfusepath{clip}%
\pgfsetbuttcap%
\pgfsetmiterjoin%
\definecolor{currentfill}{rgb}{0.066899,0.263188,0.377594}%
\pgfsetfillcolor{currentfill}%
\pgfsetlinewidth{0.000000pt}%
\definecolor{currentstroke}{rgb}{0.000000,0.000000,0.000000}%
\pgfsetstrokecolor{currentstroke}%
\pgfsetstrokeopacity{0.000000}%
\pgfsetdash{}{0pt}%
\pgfpathmoveto{\pgfqpoint{2.836428in}{1.613090in}}%
\pgfpathlineto{\pgfqpoint{2.845182in}{1.613090in}}%
\pgfpathlineto{\pgfqpoint{2.845182in}{1.354251in}}%
\pgfpathlineto{\pgfqpoint{2.836428in}{1.354251in}}%
\pgfpathlineto{\pgfqpoint{2.836428in}{1.613090in}}%
\pgfpathclose%
\pgfusepath{fill}%
\end{pgfscope}%
\begin{pgfscope}%
\pgfpathrectangle{\pgfqpoint{0.804646in}{0.600000in}}{\pgfqpoint{2.573292in}{2.070576in}}%
\pgfusepath{clip}%
\pgfsetbuttcap%
\pgfsetmiterjoin%
\definecolor{currentfill}{rgb}{0.066899,0.263188,0.377594}%
\pgfsetfillcolor{currentfill}%
\pgfsetlinewidth{0.000000pt}%
\definecolor{currentstroke}{rgb}{0.000000,0.000000,0.000000}%
\pgfsetstrokecolor{currentstroke}%
\pgfsetstrokeopacity{0.000000}%
\pgfsetdash{}{0pt}%
\pgfpathmoveto{\pgfqpoint{2.847370in}{1.613090in}}%
\pgfpathlineto{\pgfqpoint{2.856124in}{1.613090in}}%
\pgfpathlineto{\pgfqpoint{2.856124in}{1.357429in}}%
\pgfpathlineto{\pgfqpoint{2.847370in}{1.357429in}}%
\pgfpathlineto{\pgfqpoint{2.847370in}{1.613090in}}%
\pgfpathclose%
\pgfusepath{fill}%
\end{pgfscope}%
\begin{pgfscope}%
\pgfpathrectangle{\pgfqpoint{0.804646in}{0.600000in}}{\pgfqpoint{2.573292in}{2.070576in}}%
\pgfusepath{clip}%
\pgfsetbuttcap%
\pgfsetmiterjoin%
\definecolor{currentfill}{rgb}{0.066899,0.263188,0.377594}%
\pgfsetfillcolor{currentfill}%
\pgfsetlinewidth{0.000000pt}%
\definecolor{currentstroke}{rgb}{0.000000,0.000000,0.000000}%
\pgfsetstrokecolor{currentstroke}%
\pgfsetstrokeopacity{0.000000}%
\pgfsetdash{}{0pt}%
\pgfpathmoveto{\pgfqpoint{2.858312in}{1.613090in}}%
\pgfpathlineto{\pgfqpoint{2.867066in}{1.613090in}}%
\pgfpathlineto{\pgfqpoint{2.867066in}{1.350546in}}%
\pgfpathlineto{\pgfqpoint{2.858312in}{1.350546in}}%
\pgfpathlineto{\pgfqpoint{2.858312in}{1.613090in}}%
\pgfpathclose%
\pgfusepath{fill}%
\end{pgfscope}%
\begin{pgfscope}%
\pgfpathrectangle{\pgfqpoint{0.804646in}{0.600000in}}{\pgfqpoint{2.573292in}{2.070576in}}%
\pgfusepath{clip}%
\pgfsetbuttcap%
\pgfsetmiterjoin%
\definecolor{currentfill}{rgb}{0.066899,0.263188,0.377594}%
\pgfsetfillcolor{currentfill}%
\pgfsetlinewidth{0.000000pt}%
\definecolor{currentstroke}{rgb}{0.000000,0.000000,0.000000}%
\pgfsetstrokecolor{currentstroke}%
\pgfsetstrokeopacity{0.000000}%
\pgfsetdash{}{0pt}%
\pgfpathmoveto{\pgfqpoint{2.869254in}{1.613090in}}%
\pgfpathlineto{\pgfqpoint{2.878007in}{1.613090in}}%
\pgfpathlineto{\pgfqpoint{2.878007in}{1.355365in}}%
\pgfpathlineto{\pgfqpoint{2.869254in}{1.355365in}}%
\pgfpathlineto{\pgfqpoint{2.869254in}{1.613090in}}%
\pgfpathclose%
\pgfusepath{fill}%
\end{pgfscope}%
\begin{pgfscope}%
\pgfpathrectangle{\pgfqpoint{0.804646in}{0.600000in}}{\pgfqpoint{2.573292in}{2.070576in}}%
\pgfusepath{clip}%
\pgfsetbuttcap%
\pgfsetmiterjoin%
\definecolor{currentfill}{rgb}{0.066899,0.263188,0.377594}%
\pgfsetfillcolor{currentfill}%
\pgfsetlinewidth{0.000000pt}%
\definecolor{currentstroke}{rgb}{0.000000,0.000000,0.000000}%
\pgfsetstrokecolor{currentstroke}%
\pgfsetstrokeopacity{0.000000}%
\pgfsetdash{}{0pt}%
\pgfpathmoveto{\pgfqpoint{2.880196in}{1.613090in}}%
\pgfpathlineto{\pgfqpoint{2.888949in}{1.613090in}}%
\pgfpathlineto{\pgfqpoint{2.888949in}{1.374634in}}%
\pgfpathlineto{\pgfqpoint{2.880196in}{1.374634in}}%
\pgfpathlineto{\pgfqpoint{2.880196in}{1.613090in}}%
\pgfpathclose%
\pgfusepath{fill}%
\end{pgfscope}%
\begin{pgfscope}%
\pgfpathrectangle{\pgfqpoint{0.804646in}{0.600000in}}{\pgfqpoint{2.573292in}{2.070576in}}%
\pgfusepath{clip}%
\pgfsetbuttcap%
\pgfsetmiterjoin%
\definecolor{currentfill}{rgb}{0.066899,0.263188,0.377594}%
\pgfsetfillcolor{currentfill}%
\pgfsetlinewidth{0.000000pt}%
\definecolor{currentstroke}{rgb}{0.000000,0.000000,0.000000}%
\pgfsetstrokecolor{currentstroke}%
\pgfsetstrokeopacity{0.000000}%
\pgfsetdash{}{0pt}%
\pgfpathmoveto{\pgfqpoint{2.891137in}{1.613090in}}%
\pgfpathlineto{\pgfqpoint{2.899891in}{1.613090in}}%
\pgfpathlineto{\pgfqpoint{2.899891in}{1.365643in}}%
\pgfpathlineto{\pgfqpoint{2.891137in}{1.365643in}}%
\pgfpathlineto{\pgfqpoint{2.891137in}{1.613090in}}%
\pgfpathclose%
\pgfusepath{fill}%
\end{pgfscope}%
\begin{pgfscope}%
\pgfpathrectangle{\pgfqpoint{0.804646in}{0.600000in}}{\pgfqpoint{2.573292in}{2.070576in}}%
\pgfusepath{clip}%
\pgfsetbuttcap%
\pgfsetmiterjoin%
\definecolor{currentfill}{rgb}{0.066899,0.263188,0.377594}%
\pgfsetfillcolor{currentfill}%
\pgfsetlinewidth{0.000000pt}%
\definecolor{currentstroke}{rgb}{0.000000,0.000000,0.000000}%
\pgfsetstrokecolor{currentstroke}%
\pgfsetstrokeopacity{0.000000}%
\pgfsetdash{}{0pt}%
\pgfpathmoveto{\pgfqpoint{2.902079in}{1.613090in}}%
\pgfpathlineto{\pgfqpoint{2.910833in}{1.613090in}}%
\pgfpathlineto{\pgfqpoint{2.910833in}{1.381538in}}%
\pgfpathlineto{\pgfqpoint{2.902079in}{1.381538in}}%
\pgfpathlineto{\pgfqpoint{2.902079in}{1.613090in}}%
\pgfpathclose%
\pgfusepath{fill}%
\end{pgfscope}%
\begin{pgfscope}%
\pgfpathrectangle{\pgfqpoint{0.804646in}{0.600000in}}{\pgfqpoint{2.573292in}{2.070576in}}%
\pgfusepath{clip}%
\pgfsetbuttcap%
\pgfsetmiterjoin%
\definecolor{currentfill}{rgb}{0.066899,0.263188,0.377594}%
\pgfsetfillcolor{currentfill}%
\pgfsetlinewidth{0.000000pt}%
\definecolor{currentstroke}{rgb}{0.000000,0.000000,0.000000}%
\pgfsetstrokecolor{currentstroke}%
\pgfsetstrokeopacity{0.000000}%
\pgfsetdash{}{0pt}%
\pgfpathmoveto{\pgfqpoint{2.913021in}{1.613090in}}%
\pgfpathlineto{\pgfqpoint{2.921774in}{1.613090in}}%
\pgfpathlineto{\pgfqpoint{2.921774in}{1.385936in}}%
\pgfpathlineto{\pgfqpoint{2.913021in}{1.385936in}}%
\pgfpathlineto{\pgfqpoint{2.913021in}{1.613090in}}%
\pgfpathclose%
\pgfusepath{fill}%
\end{pgfscope}%
\begin{pgfscope}%
\pgfpathrectangle{\pgfqpoint{0.804646in}{0.600000in}}{\pgfqpoint{2.573292in}{2.070576in}}%
\pgfusepath{clip}%
\pgfsetbuttcap%
\pgfsetmiterjoin%
\definecolor{currentfill}{rgb}{0.066899,0.263188,0.377594}%
\pgfsetfillcolor{currentfill}%
\pgfsetlinewidth{0.000000pt}%
\definecolor{currentstroke}{rgb}{0.000000,0.000000,0.000000}%
\pgfsetstrokecolor{currentstroke}%
\pgfsetstrokeopacity{0.000000}%
\pgfsetdash{}{0pt}%
\pgfpathmoveto{\pgfqpoint{2.923963in}{1.613090in}}%
\pgfpathlineto{\pgfqpoint{2.932716in}{1.613090in}}%
\pgfpathlineto{\pgfqpoint{2.932716in}{1.384478in}}%
\pgfpathlineto{\pgfqpoint{2.923963in}{1.384478in}}%
\pgfpathlineto{\pgfqpoint{2.923963in}{1.613090in}}%
\pgfpathclose%
\pgfusepath{fill}%
\end{pgfscope}%
\begin{pgfscope}%
\pgfpathrectangle{\pgfqpoint{0.804646in}{0.600000in}}{\pgfqpoint{2.573292in}{2.070576in}}%
\pgfusepath{clip}%
\pgfsetbuttcap%
\pgfsetmiterjoin%
\definecolor{currentfill}{rgb}{0.066899,0.263188,0.377594}%
\pgfsetfillcolor{currentfill}%
\pgfsetlinewidth{0.000000pt}%
\definecolor{currentstroke}{rgb}{0.000000,0.000000,0.000000}%
\pgfsetstrokecolor{currentstroke}%
\pgfsetstrokeopacity{0.000000}%
\pgfsetdash{}{0pt}%
\pgfpathmoveto{\pgfqpoint{2.934905in}{1.613090in}}%
\pgfpathlineto{\pgfqpoint{2.943658in}{1.613090in}}%
\pgfpathlineto{\pgfqpoint{2.943658in}{1.384184in}}%
\pgfpathlineto{\pgfqpoint{2.934905in}{1.384184in}}%
\pgfpathlineto{\pgfqpoint{2.934905in}{1.613090in}}%
\pgfpathclose%
\pgfusepath{fill}%
\end{pgfscope}%
\begin{pgfscope}%
\pgfpathrectangle{\pgfqpoint{0.804646in}{0.600000in}}{\pgfqpoint{2.573292in}{2.070576in}}%
\pgfusepath{clip}%
\pgfsetbuttcap%
\pgfsetmiterjoin%
\definecolor{currentfill}{rgb}{0.066899,0.263188,0.377594}%
\pgfsetfillcolor{currentfill}%
\pgfsetlinewidth{0.000000pt}%
\definecolor{currentstroke}{rgb}{0.000000,0.000000,0.000000}%
\pgfsetstrokecolor{currentstroke}%
\pgfsetstrokeopacity{0.000000}%
\pgfsetdash{}{0pt}%
\pgfpathmoveto{\pgfqpoint{2.945846in}{1.613090in}}%
\pgfpathlineto{\pgfqpoint{2.954600in}{1.613090in}}%
\pgfpathlineto{\pgfqpoint{2.954600in}{1.392581in}}%
\pgfpathlineto{\pgfqpoint{2.945846in}{1.392581in}}%
\pgfpathlineto{\pgfqpoint{2.945846in}{1.613090in}}%
\pgfpathclose%
\pgfusepath{fill}%
\end{pgfscope}%
\begin{pgfscope}%
\pgfpathrectangle{\pgfqpoint{0.804646in}{0.600000in}}{\pgfqpoint{2.573292in}{2.070576in}}%
\pgfusepath{clip}%
\pgfsetbuttcap%
\pgfsetmiterjoin%
\definecolor{currentfill}{rgb}{0.066899,0.263188,0.377594}%
\pgfsetfillcolor{currentfill}%
\pgfsetlinewidth{0.000000pt}%
\definecolor{currentstroke}{rgb}{0.000000,0.000000,0.000000}%
\pgfsetstrokecolor{currentstroke}%
\pgfsetstrokeopacity{0.000000}%
\pgfsetdash{}{0pt}%
\pgfpathmoveto{\pgfqpoint{2.956788in}{1.613090in}}%
\pgfpathlineto{\pgfqpoint{2.965542in}{1.613090in}}%
\pgfpathlineto{\pgfqpoint{2.965542in}{1.395421in}}%
\pgfpathlineto{\pgfqpoint{2.956788in}{1.395421in}}%
\pgfpathlineto{\pgfqpoint{2.956788in}{1.613090in}}%
\pgfpathclose%
\pgfusepath{fill}%
\end{pgfscope}%
\begin{pgfscope}%
\pgfpathrectangle{\pgfqpoint{0.804646in}{0.600000in}}{\pgfqpoint{2.573292in}{2.070576in}}%
\pgfusepath{clip}%
\pgfsetbuttcap%
\pgfsetmiterjoin%
\definecolor{currentfill}{rgb}{0.066899,0.263188,0.377594}%
\pgfsetfillcolor{currentfill}%
\pgfsetlinewidth{0.000000pt}%
\definecolor{currentstroke}{rgb}{0.000000,0.000000,0.000000}%
\pgfsetstrokecolor{currentstroke}%
\pgfsetstrokeopacity{0.000000}%
\pgfsetdash{}{0pt}%
\pgfpathmoveto{\pgfqpoint{2.967730in}{1.613090in}}%
\pgfpathlineto{\pgfqpoint{2.976483in}{1.613090in}}%
\pgfpathlineto{\pgfqpoint{2.976483in}{1.403690in}}%
\pgfpathlineto{\pgfqpoint{2.967730in}{1.403690in}}%
\pgfpathlineto{\pgfqpoint{2.967730in}{1.613090in}}%
\pgfpathclose%
\pgfusepath{fill}%
\end{pgfscope}%
\begin{pgfscope}%
\pgfpathrectangle{\pgfqpoint{0.804646in}{0.600000in}}{\pgfqpoint{2.573292in}{2.070576in}}%
\pgfusepath{clip}%
\pgfsetbuttcap%
\pgfsetmiterjoin%
\definecolor{currentfill}{rgb}{0.066899,0.263188,0.377594}%
\pgfsetfillcolor{currentfill}%
\pgfsetlinewidth{0.000000pt}%
\definecolor{currentstroke}{rgb}{0.000000,0.000000,0.000000}%
\pgfsetstrokecolor{currentstroke}%
\pgfsetstrokeopacity{0.000000}%
\pgfsetdash{}{0pt}%
\pgfpathmoveto{\pgfqpoint{2.978672in}{1.613090in}}%
\pgfpathlineto{\pgfqpoint{2.987425in}{1.613090in}}%
\pgfpathlineto{\pgfqpoint{2.987425in}{1.400539in}}%
\pgfpathlineto{\pgfqpoint{2.978672in}{1.400539in}}%
\pgfpathlineto{\pgfqpoint{2.978672in}{1.613090in}}%
\pgfpathclose%
\pgfusepath{fill}%
\end{pgfscope}%
\begin{pgfscope}%
\pgfpathrectangle{\pgfqpoint{0.804646in}{0.600000in}}{\pgfqpoint{2.573292in}{2.070576in}}%
\pgfusepath{clip}%
\pgfsetbuttcap%
\pgfsetmiterjoin%
\definecolor{currentfill}{rgb}{0.066899,0.263188,0.377594}%
\pgfsetfillcolor{currentfill}%
\pgfsetlinewidth{0.000000pt}%
\definecolor{currentstroke}{rgb}{0.000000,0.000000,0.000000}%
\pgfsetstrokecolor{currentstroke}%
\pgfsetstrokeopacity{0.000000}%
\pgfsetdash{}{0pt}%
\pgfpathmoveto{\pgfqpoint{2.989614in}{1.613090in}}%
\pgfpathlineto{\pgfqpoint{2.998367in}{1.613090in}}%
\pgfpathlineto{\pgfqpoint{2.998367in}{1.396830in}}%
\pgfpathlineto{\pgfqpoint{2.989614in}{1.396830in}}%
\pgfpathlineto{\pgfqpoint{2.989614in}{1.613090in}}%
\pgfpathclose%
\pgfusepath{fill}%
\end{pgfscope}%
\begin{pgfscope}%
\pgfpathrectangle{\pgfqpoint{0.804646in}{0.600000in}}{\pgfqpoint{2.573292in}{2.070576in}}%
\pgfusepath{clip}%
\pgfsetbuttcap%
\pgfsetmiterjoin%
\definecolor{currentfill}{rgb}{0.066899,0.263188,0.377594}%
\pgfsetfillcolor{currentfill}%
\pgfsetlinewidth{0.000000pt}%
\definecolor{currentstroke}{rgb}{0.000000,0.000000,0.000000}%
\pgfsetstrokecolor{currentstroke}%
\pgfsetstrokeopacity{0.000000}%
\pgfsetdash{}{0pt}%
\pgfpathmoveto{\pgfqpoint{3.000555in}{1.613090in}}%
\pgfpathlineto{\pgfqpoint{3.009309in}{1.613090in}}%
\pgfpathlineto{\pgfqpoint{3.009309in}{1.402168in}}%
\pgfpathlineto{\pgfqpoint{3.000555in}{1.402168in}}%
\pgfpathlineto{\pgfqpoint{3.000555in}{1.613090in}}%
\pgfpathclose%
\pgfusepath{fill}%
\end{pgfscope}%
\begin{pgfscope}%
\pgfpathrectangle{\pgfqpoint{0.804646in}{0.600000in}}{\pgfqpoint{2.573292in}{2.070576in}}%
\pgfusepath{clip}%
\pgfsetbuttcap%
\pgfsetmiterjoin%
\definecolor{currentfill}{rgb}{0.066899,0.263188,0.377594}%
\pgfsetfillcolor{currentfill}%
\pgfsetlinewidth{0.000000pt}%
\definecolor{currentstroke}{rgb}{0.000000,0.000000,0.000000}%
\pgfsetstrokecolor{currentstroke}%
\pgfsetstrokeopacity{0.000000}%
\pgfsetdash{}{0pt}%
\pgfpathmoveto{\pgfqpoint{3.011497in}{1.613090in}}%
\pgfpathlineto{\pgfqpoint{3.020251in}{1.613090in}}%
\pgfpathlineto{\pgfqpoint{3.020251in}{1.421115in}}%
\pgfpathlineto{\pgfqpoint{3.011497in}{1.421115in}}%
\pgfpathlineto{\pgfqpoint{3.011497in}{1.613090in}}%
\pgfpathclose%
\pgfusepath{fill}%
\end{pgfscope}%
\begin{pgfscope}%
\pgfpathrectangle{\pgfqpoint{0.804646in}{0.600000in}}{\pgfqpoint{2.573292in}{2.070576in}}%
\pgfusepath{clip}%
\pgfsetbuttcap%
\pgfsetmiterjoin%
\definecolor{currentfill}{rgb}{0.066899,0.263188,0.377594}%
\pgfsetfillcolor{currentfill}%
\pgfsetlinewidth{0.000000pt}%
\definecolor{currentstroke}{rgb}{0.000000,0.000000,0.000000}%
\pgfsetstrokecolor{currentstroke}%
\pgfsetstrokeopacity{0.000000}%
\pgfsetdash{}{0pt}%
\pgfpathmoveto{\pgfqpoint{3.022439in}{1.613090in}}%
\pgfpathlineto{\pgfqpoint{3.031192in}{1.613090in}}%
\pgfpathlineto{\pgfqpoint{3.031192in}{1.434544in}}%
\pgfpathlineto{\pgfqpoint{3.022439in}{1.434544in}}%
\pgfpathlineto{\pgfqpoint{3.022439in}{1.613090in}}%
\pgfpathclose%
\pgfusepath{fill}%
\end{pgfscope}%
\begin{pgfscope}%
\pgfpathrectangle{\pgfqpoint{0.804646in}{0.600000in}}{\pgfqpoint{2.573292in}{2.070576in}}%
\pgfusepath{clip}%
\pgfsetbuttcap%
\pgfsetmiterjoin%
\definecolor{currentfill}{rgb}{0.066899,0.263188,0.377594}%
\pgfsetfillcolor{currentfill}%
\pgfsetlinewidth{0.000000pt}%
\definecolor{currentstroke}{rgb}{0.000000,0.000000,0.000000}%
\pgfsetstrokecolor{currentstroke}%
\pgfsetstrokeopacity{0.000000}%
\pgfsetdash{}{0pt}%
\pgfpathmoveto{\pgfqpoint{3.033381in}{1.613090in}}%
\pgfpathlineto{\pgfqpoint{3.042134in}{1.613090in}}%
\pgfpathlineto{\pgfqpoint{3.042134in}{1.458934in}}%
\pgfpathlineto{\pgfqpoint{3.033381in}{1.458934in}}%
\pgfpathlineto{\pgfqpoint{3.033381in}{1.613090in}}%
\pgfpathclose%
\pgfusepath{fill}%
\end{pgfscope}%
\begin{pgfscope}%
\pgfpathrectangle{\pgfqpoint{0.804646in}{0.600000in}}{\pgfqpoint{2.573292in}{2.070576in}}%
\pgfusepath{clip}%
\pgfsetbuttcap%
\pgfsetmiterjoin%
\definecolor{currentfill}{rgb}{0.066899,0.263188,0.377594}%
\pgfsetfillcolor{currentfill}%
\pgfsetlinewidth{0.000000pt}%
\definecolor{currentstroke}{rgb}{0.000000,0.000000,0.000000}%
\pgfsetstrokecolor{currentstroke}%
\pgfsetstrokeopacity{0.000000}%
\pgfsetdash{}{0pt}%
\pgfpathmoveto{\pgfqpoint{3.044323in}{1.613090in}}%
\pgfpathlineto{\pgfqpoint{3.053076in}{1.613090in}}%
\pgfpathlineto{\pgfqpoint{3.053076in}{1.462277in}}%
\pgfpathlineto{\pgfqpoint{3.044323in}{1.462277in}}%
\pgfpathlineto{\pgfqpoint{3.044323in}{1.613090in}}%
\pgfpathclose%
\pgfusepath{fill}%
\end{pgfscope}%
\begin{pgfscope}%
\pgfpathrectangle{\pgfqpoint{0.804646in}{0.600000in}}{\pgfqpoint{2.573292in}{2.070576in}}%
\pgfusepath{clip}%
\pgfsetbuttcap%
\pgfsetmiterjoin%
\definecolor{currentfill}{rgb}{0.066899,0.263188,0.377594}%
\pgfsetfillcolor{currentfill}%
\pgfsetlinewidth{0.000000pt}%
\definecolor{currentstroke}{rgb}{0.000000,0.000000,0.000000}%
\pgfsetstrokecolor{currentstroke}%
\pgfsetstrokeopacity{0.000000}%
\pgfsetdash{}{0pt}%
\pgfpathmoveto{\pgfqpoint{3.055264in}{1.613090in}}%
\pgfpathlineto{\pgfqpoint{3.064018in}{1.613090in}}%
\pgfpathlineto{\pgfqpoint{3.064018in}{1.456297in}}%
\pgfpathlineto{\pgfqpoint{3.055264in}{1.456297in}}%
\pgfpathlineto{\pgfqpoint{3.055264in}{1.613090in}}%
\pgfpathclose%
\pgfusepath{fill}%
\end{pgfscope}%
\begin{pgfscope}%
\pgfpathrectangle{\pgfqpoint{0.804646in}{0.600000in}}{\pgfqpoint{2.573292in}{2.070576in}}%
\pgfusepath{clip}%
\pgfsetbuttcap%
\pgfsetmiterjoin%
\definecolor{currentfill}{rgb}{0.066899,0.263188,0.377594}%
\pgfsetfillcolor{currentfill}%
\pgfsetlinewidth{0.000000pt}%
\definecolor{currentstroke}{rgb}{0.000000,0.000000,0.000000}%
\pgfsetstrokecolor{currentstroke}%
\pgfsetstrokeopacity{0.000000}%
\pgfsetdash{}{0pt}%
\pgfpathmoveto{\pgfqpoint{3.066206in}{1.613090in}}%
\pgfpathlineto{\pgfqpoint{3.074960in}{1.613090in}}%
\pgfpathlineto{\pgfqpoint{3.074960in}{1.463272in}}%
\pgfpathlineto{\pgfqpoint{3.066206in}{1.463272in}}%
\pgfpathlineto{\pgfqpoint{3.066206in}{1.613090in}}%
\pgfpathclose%
\pgfusepath{fill}%
\end{pgfscope}%
\begin{pgfscope}%
\pgfpathrectangle{\pgfqpoint{0.804646in}{0.600000in}}{\pgfqpoint{2.573292in}{2.070576in}}%
\pgfusepath{clip}%
\pgfsetbuttcap%
\pgfsetmiterjoin%
\definecolor{currentfill}{rgb}{0.066899,0.263188,0.377594}%
\pgfsetfillcolor{currentfill}%
\pgfsetlinewidth{0.000000pt}%
\definecolor{currentstroke}{rgb}{0.000000,0.000000,0.000000}%
\pgfsetstrokecolor{currentstroke}%
\pgfsetstrokeopacity{0.000000}%
\pgfsetdash{}{0pt}%
\pgfpathmoveto{\pgfqpoint{3.077148in}{1.613090in}}%
\pgfpathlineto{\pgfqpoint{3.085901in}{1.613090in}}%
\pgfpathlineto{\pgfqpoint{3.085901in}{1.468094in}}%
\pgfpathlineto{\pgfqpoint{3.077148in}{1.468094in}}%
\pgfpathlineto{\pgfqpoint{3.077148in}{1.613090in}}%
\pgfpathclose%
\pgfusepath{fill}%
\end{pgfscope}%
\begin{pgfscope}%
\pgfpathrectangle{\pgfqpoint{0.804646in}{0.600000in}}{\pgfqpoint{2.573292in}{2.070576in}}%
\pgfusepath{clip}%
\pgfsetbuttcap%
\pgfsetmiterjoin%
\definecolor{currentfill}{rgb}{0.066899,0.263188,0.377594}%
\pgfsetfillcolor{currentfill}%
\pgfsetlinewidth{0.000000pt}%
\definecolor{currentstroke}{rgb}{0.000000,0.000000,0.000000}%
\pgfsetstrokecolor{currentstroke}%
\pgfsetstrokeopacity{0.000000}%
\pgfsetdash{}{0pt}%
\pgfpathmoveto{\pgfqpoint{3.088090in}{1.613090in}}%
\pgfpathlineto{\pgfqpoint{3.096843in}{1.613090in}}%
\pgfpathlineto{\pgfqpoint{3.096843in}{1.476013in}}%
\pgfpathlineto{\pgfqpoint{3.088090in}{1.476013in}}%
\pgfpathlineto{\pgfqpoint{3.088090in}{1.613090in}}%
\pgfpathclose%
\pgfusepath{fill}%
\end{pgfscope}%
\begin{pgfscope}%
\pgfpathrectangle{\pgfqpoint{0.804646in}{0.600000in}}{\pgfqpoint{2.573292in}{2.070576in}}%
\pgfusepath{clip}%
\pgfsetbuttcap%
\pgfsetmiterjoin%
\definecolor{currentfill}{rgb}{0.066899,0.263188,0.377594}%
\pgfsetfillcolor{currentfill}%
\pgfsetlinewidth{0.000000pt}%
\definecolor{currentstroke}{rgb}{0.000000,0.000000,0.000000}%
\pgfsetstrokecolor{currentstroke}%
\pgfsetstrokeopacity{0.000000}%
\pgfsetdash{}{0pt}%
\pgfpathmoveto{\pgfqpoint{3.099032in}{1.613090in}}%
\pgfpathlineto{\pgfqpoint{3.107785in}{1.613090in}}%
\pgfpathlineto{\pgfqpoint{3.107785in}{1.468138in}}%
\pgfpathlineto{\pgfqpoint{3.099032in}{1.468138in}}%
\pgfpathlineto{\pgfqpoint{3.099032in}{1.613090in}}%
\pgfpathclose%
\pgfusepath{fill}%
\end{pgfscope}%
\begin{pgfscope}%
\pgfpathrectangle{\pgfqpoint{0.804646in}{0.600000in}}{\pgfqpoint{2.573292in}{2.070576in}}%
\pgfusepath{clip}%
\pgfsetbuttcap%
\pgfsetmiterjoin%
\definecolor{currentfill}{rgb}{0.066899,0.263188,0.377594}%
\pgfsetfillcolor{currentfill}%
\pgfsetlinewidth{0.000000pt}%
\definecolor{currentstroke}{rgb}{0.000000,0.000000,0.000000}%
\pgfsetstrokecolor{currentstroke}%
\pgfsetstrokeopacity{0.000000}%
\pgfsetdash{}{0pt}%
\pgfpathmoveto{\pgfqpoint{3.109973in}{1.613090in}}%
\pgfpathlineto{\pgfqpoint{3.118727in}{1.613090in}}%
\pgfpathlineto{\pgfqpoint{3.118727in}{1.482022in}}%
\pgfpathlineto{\pgfqpoint{3.109973in}{1.482022in}}%
\pgfpathlineto{\pgfqpoint{3.109973in}{1.613090in}}%
\pgfpathclose%
\pgfusepath{fill}%
\end{pgfscope}%
\begin{pgfscope}%
\pgfpathrectangle{\pgfqpoint{0.804646in}{0.600000in}}{\pgfqpoint{2.573292in}{2.070576in}}%
\pgfusepath{clip}%
\pgfsetbuttcap%
\pgfsetmiterjoin%
\definecolor{currentfill}{rgb}{0.066899,0.263188,0.377594}%
\pgfsetfillcolor{currentfill}%
\pgfsetlinewidth{0.000000pt}%
\definecolor{currentstroke}{rgb}{0.000000,0.000000,0.000000}%
\pgfsetstrokecolor{currentstroke}%
\pgfsetstrokeopacity{0.000000}%
\pgfsetdash{}{0pt}%
\pgfpathmoveto{\pgfqpoint{3.120915in}{1.613090in}}%
\pgfpathlineto{\pgfqpoint{3.129669in}{1.613090in}}%
\pgfpathlineto{\pgfqpoint{3.129669in}{1.473845in}}%
\pgfpathlineto{\pgfqpoint{3.120915in}{1.473845in}}%
\pgfpathlineto{\pgfqpoint{3.120915in}{1.613090in}}%
\pgfpathclose%
\pgfusepath{fill}%
\end{pgfscope}%
\begin{pgfscope}%
\pgfpathrectangle{\pgfqpoint{0.804646in}{0.600000in}}{\pgfqpoint{2.573292in}{2.070576in}}%
\pgfusepath{clip}%
\pgfsetbuttcap%
\pgfsetmiterjoin%
\definecolor{currentfill}{rgb}{0.066899,0.263188,0.377594}%
\pgfsetfillcolor{currentfill}%
\pgfsetlinewidth{0.000000pt}%
\definecolor{currentstroke}{rgb}{0.000000,0.000000,0.000000}%
\pgfsetstrokecolor{currentstroke}%
\pgfsetstrokeopacity{0.000000}%
\pgfsetdash{}{0pt}%
\pgfpathmoveto{\pgfqpoint{3.131857in}{1.613090in}}%
\pgfpathlineto{\pgfqpoint{3.140610in}{1.613090in}}%
\pgfpathlineto{\pgfqpoint{3.140610in}{1.486110in}}%
\pgfpathlineto{\pgfqpoint{3.131857in}{1.486110in}}%
\pgfpathlineto{\pgfqpoint{3.131857in}{1.613090in}}%
\pgfpathclose%
\pgfusepath{fill}%
\end{pgfscope}%
\begin{pgfscope}%
\pgfpathrectangle{\pgfqpoint{0.804646in}{0.600000in}}{\pgfqpoint{2.573292in}{2.070576in}}%
\pgfusepath{clip}%
\pgfsetbuttcap%
\pgfsetmiterjoin%
\definecolor{currentfill}{rgb}{0.066899,0.263188,0.377594}%
\pgfsetfillcolor{currentfill}%
\pgfsetlinewidth{0.000000pt}%
\definecolor{currentstroke}{rgb}{0.000000,0.000000,0.000000}%
\pgfsetstrokecolor{currentstroke}%
\pgfsetstrokeopacity{0.000000}%
\pgfsetdash{}{0pt}%
\pgfpathmoveto{\pgfqpoint{3.142799in}{1.613090in}}%
\pgfpathlineto{\pgfqpoint{3.151552in}{1.613090in}}%
\pgfpathlineto{\pgfqpoint{3.151552in}{1.497533in}}%
\pgfpathlineto{\pgfqpoint{3.142799in}{1.497533in}}%
\pgfpathlineto{\pgfqpoint{3.142799in}{1.613090in}}%
\pgfpathclose%
\pgfusepath{fill}%
\end{pgfscope}%
\begin{pgfscope}%
\pgfpathrectangle{\pgfqpoint{0.804646in}{0.600000in}}{\pgfqpoint{2.573292in}{2.070576in}}%
\pgfusepath{clip}%
\pgfsetbuttcap%
\pgfsetmiterjoin%
\definecolor{currentfill}{rgb}{0.066899,0.263188,0.377594}%
\pgfsetfillcolor{currentfill}%
\pgfsetlinewidth{0.000000pt}%
\definecolor{currentstroke}{rgb}{0.000000,0.000000,0.000000}%
\pgfsetstrokecolor{currentstroke}%
\pgfsetstrokeopacity{0.000000}%
\pgfsetdash{}{0pt}%
\pgfpathmoveto{\pgfqpoint{3.153741in}{1.613090in}}%
\pgfpathlineto{\pgfqpoint{3.162494in}{1.613090in}}%
\pgfpathlineto{\pgfqpoint{3.162494in}{1.490958in}}%
\pgfpathlineto{\pgfqpoint{3.153741in}{1.490958in}}%
\pgfpathlineto{\pgfqpoint{3.153741in}{1.613090in}}%
\pgfpathclose%
\pgfusepath{fill}%
\end{pgfscope}%
\begin{pgfscope}%
\pgfpathrectangle{\pgfqpoint{0.804646in}{0.600000in}}{\pgfqpoint{2.573292in}{2.070576in}}%
\pgfusepath{clip}%
\pgfsetbuttcap%
\pgfsetmiterjoin%
\definecolor{currentfill}{rgb}{0.066899,0.263188,0.377594}%
\pgfsetfillcolor{currentfill}%
\pgfsetlinewidth{0.000000pt}%
\definecolor{currentstroke}{rgb}{0.000000,0.000000,0.000000}%
\pgfsetstrokecolor{currentstroke}%
\pgfsetstrokeopacity{0.000000}%
\pgfsetdash{}{0pt}%
\pgfpathmoveto{\pgfqpoint{3.164682in}{1.613090in}}%
\pgfpathlineto{\pgfqpoint{3.173436in}{1.613090in}}%
\pgfpathlineto{\pgfqpoint{3.173436in}{1.501169in}}%
\pgfpathlineto{\pgfqpoint{3.164682in}{1.501169in}}%
\pgfpathlineto{\pgfqpoint{3.164682in}{1.613090in}}%
\pgfpathclose%
\pgfusepath{fill}%
\end{pgfscope}%
\begin{pgfscope}%
\pgfpathrectangle{\pgfqpoint{0.804646in}{0.600000in}}{\pgfqpoint{2.573292in}{2.070576in}}%
\pgfusepath{clip}%
\pgfsetbuttcap%
\pgfsetmiterjoin%
\definecolor{currentfill}{rgb}{0.066899,0.263188,0.377594}%
\pgfsetfillcolor{currentfill}%
\pgfsetlinewidth{0.000000pt}%
\definecolor{currentstroke}{rgb}{0.000000,0.000000,0.000000}%
\pgfsetstrokecolor{currentstroke}%
\pgfsetstrokeopacity{0.000000}%
\pgfsetdash{}{0pt}%
\pgfpathmoveto{\pgfqpoint{3.175624in}{1.613090in}}%
\pgfpathlineto{\pgfqpoint{3.184378in}{1.613090in}}%
\pgfpathlineto{\pgfqpoint{3.184378in}{1.507657in}}%
\pgfpathlineto{\pgfqpoint{3.175624in}{1.507657in}}%
\pgfpathlineto{\pgfqpoint{3.175624in}{1.613090in}}%
\pgfpathclose%
\pgfusepath{fill}%
\end{pgfscope}%
\begin{pgfscope}%
\pgfpathrectangle{\pgfqpoint{0.804646in}{0.600000in}}{\pgfqpoint{2.573292in}{2.070576in}}%
\pgfusepath{clip}%
\pgfsetbuttcap%
\pgfsetmiterjoin%
\definecolor{currentfill}{rgb}{0.066899,0.263188,0.377594}%
\pgfsetfillcolor{currentfill}%
\pgfsetlinewidth{0.000000pt}%
\definecolor{currentstroke}{rgb}{0.000000,0.000000,0.000000}%
\pgfsetstrokecolor{currentstroke}%
\pgfsetstrokeopacity{0.000000}%
\pgfsetdash{}{0pt}%
\pgfpathmoveto{\pgfqpoint{3.186566in}{1.613090in}}%
\pgfpathlineto{\pgfqpoint{3.195319in}{1.613090in}}%
\pgfpathlineto{\pgfqpoint{3.195319in}{1.518734in}}%
\pgfpathlineto{\pgfqpoint{3.186566in}{1.518734in}}%
\pgfpathlineto{\pgfqpoint{3.186566in}{1.613090in}}%
\pgfpathclose%
\pgfusepath{fill}%
\end{pgfscope}%
\begin{pgfscope}%
\pgfpathrectangle{\pgfqpoint{0.804646in}{0.600000in}}{\pgfqpoint{2.573292in}{2.070576in}}%
\pgfusepath{clip}%
\pgfsetbuttcap%
\pgfsetmiterjoin%
\definecolor{currentfill}{rgb}{0.066899,0.263188,0.377594}%
\pgfsetfillcolor{currentfill}%
\pgfsetlinewidth{0.000000pt}%
\definecolor{currentstroke}{rgb}{0.000000,0.000000,0.000000}%
\pgfsetstrokecolor{currentstroke}%
\pgfsetstrokeopacity{0.000000}%
\pgfsetdash{}{0pt}%
\pgfpathmoveto{\pgfqpoint{3.197508in}{1.613090in}}%
\pgfpathlineto{\pgfqpoint{3.206261in}{1.613090in}}%
\pgfpathlineto{\pgfqpoint{3.206261in}{1.527694in}}%
\pgfpathlineto{\pgfqpoint{3.197508in}{1.527694in}}%
\pgfpathlineto{\pgfqpoint{3.197508in}{1.613090in}}%
\pgfpathclose%
\pgfusepath{fill}%
\end{pgfscope}%
\begin{pgfscope}%
\pgfpathrectangle{\pgfqpoint{0.804646in}{0.600000in}}{\pgfqpoint{2.573292in}{2.070576in}}%
\pgfusepath{clip}%
\pgfsetbuttcap%
\pgfsetmiterjoin%
\definecolor{currentfill}{rgb}{0.066899,0.263188,0.377594}%
\pgfsetfillcolor{currentfill}%
\pgfsetlinewidth{0.000000pt}%
\definecolor{currentstroke}{rgb}{0.000000,0.000000,0.000000}%
\pgfsetstrokecolor{currentstroke}%
\pgfsetstrokeopacity{0.000000}%
\pgfsetdash{}{0pt}%
\pgfpathmoveto{\pgfqpoint{3.208450in}{1.613090in}}%
\pgfpathlineto{\pgfqpoint{3.217203in}{1.613090in}}%
\pgfpathlineto{\pgfqpoint{3.217203in}{1.529349in}}%
\pgfpathlineto{\pgfqpoint{3.208450in}{1.529349in}}%
\pgfpathlineto{\pgfqpoint{3.208450in}{1.613090in}}%
\pgfpathclose%
\pgfusepath{fill}%
\end{pgfscope}%
\begin{pgfscope}%
\pgfpathrectangle{\pgfqpoint{0.804646in}{0.600000in}}{\pgfqpoint{2.573292in}{2.070576in}}%
\pgfusepath{clip}%
\pgfsetbuttcap%
\pgfsetmiterjoin%
\definecolor{currentfill}{rgb}{0.066899,0.263188,0.377594}%
\pgfsetfillcolor{currentfill}%
\pgfsetlinewidth{0.000000pt}%
\definecolor{currentstroke}{rgb}{0.000000,0.000000,0.000000}%
\pgfsetstrokecolor{currentstroke}%
\pgfsetstrokeopacity{0.000000}%
\pgfsetdash{}{0pt}%
\pgfpathmoveto{\pgfqpoint{3.219391in}{1.613090in}}%
\pgfpathlineto{\pgfqpoint{3.228145in}{1.613090in}}%
\pgfpathlineto{\pgfqpoint{3.228145in}{1.522532in}}%
\pgfpathlineto{\pgfqpoint{3.219391in}{1.522532in}}%
\pgfpathlineto{\pgfqpoint{3.219391in}{1.613090in}}%
\pgfpathclose%
\pgfusepath{fill}%
\end{pgfscope}%
\begin{pgfscope}%
\pgfpathrectangle{\pgfqpoint{0.804646in}{0.600000in}}{\pgfqpoint{2.573292in}{2.070576in}}%
\pgfusepath{clip}%
\pgfsetbuttcap%
\pgfsetmiterjoin%
\definecolor{currentfill}{rgb}{0.066899,0.263188,0.377594}%
\pgfsetfillcolor{currentfill}%
\pgfsetlinewidth{0.000000pt}%
\definecolor{currentstroke}{rgb}{0.000000,0.000000,0.000000}%
\pgfsetstrokecolor{currentstroke}%
\pgfsetstrokeopacity{0.000000}%
\pgfsetdash{}{0pt}%
\pgfpathmoveto{\pgfqpoint{3.230333in}{1.613090in}}%
\pgfpathlineto{\pgfqpoint{3.239087in}{1.613090in}}%
\pgfpathlineto{\pgfqpoint{3.239087in}{1.529762in}}%
\pgfpathlineto{\pgfqpoint{3.230333in}{1.529762in}}%
\pgfpathlineto{\pgfqpoint{3.230333in}{1.613090in}}%
\pgfpathclose%
\pgfusepath{fill}%
\end{pgfscope}%
\begin{pgfscope}%
\pgfpathrectangle{\pgfqpoint{0.804646in}{0.600000in}}{\pgfqpoint{2.573292in}{2.070576in}}%
\pgfusepath{clip}%
\pgfsetbuttcap%
\pgfsetmiterjoin%
\definecolor{currentfill}{rgb}{0.066899,0.263188,0.377594}%
\pgfsetfillcolor{currentfill}%
\pgfsetlinewidth{0.000000pt}%
\definecolor{currentstroke}{rgb}{0.000000,0.000000,0.000000}%
\pgfsetstrokecolor{currentstroke}%
\pgfsetstrokeopacity{0.000000}%
\pgfsetdash{}{0pt}%
\pgfpathmoveto{\pgfqpoint{3.241275in}{1.613090in}}%
\pgfpathlineto{\pgfqpoint{3.250028in}{1.613090in}}%
\pgfpathlineto{\pgfqpoint{3.250028in}{1.531941in}}%
\pgfpathlineto{\pgfqpoint{3.241275in}{1.531941in}}%
\pgfpathlineto{\pgfqpoint{3.241275in}{1.613090in}}%
\pgfpathclose%
\pgfusepath{fill}%
\end{pgfscope}%
\begin{pgfscope}%
\pgfpathrectangle{\pgfqpoint{0.804646in}{0.600000in}}{\pgfqpoint{2.573292in}{2.070576in}}%
\pgfusepath{clip}%
\pgfsetbuttcap%
\pgfsetmiterjoin%
\definecolor{currentfill}{rgb}{0.066899,0.263188,0.377594}%
\pgfsetfillcolor{currentfill}%
\pgfsetlinewidth{0.000000pt}%
\definecolor{currentstroke}{rgb}{0.000000,0.000000,0.000000}%
\pgfsetstrokecolor{currentstroke}%
\pgfsetstrokeopacity{0.000000}%
\pgfsetdash{}{0pt}%
\pgfpathmoveto{\pgfqpoint{3.252217in}{1.613090in}}%
\pgfpathlineto{\pgfqpoint{3.260970in}{1.613090in}}%
\pgfpathlineto{\pgfqpoint{3.260970in}{1.521480in}}%
\pgfpathlineto{\pgfqpoint{3.252217in}{1.521480in}}%
\pgfpathlineto{\pgfqpoint{3.252217in}{1.613090in}}%
\pgfpathclose%
\pgfusepath{fill}%
\end{pgfscope}%
\begin{pgfscope}%
\pgfpathrectangle{\pgfqpoint{0.804646in}{0.600000in}}{\pgfqpoint{2.573292in}{2.070576in}}%
\pgfusepath{clip}%
\pgfsetbuttcap%
\pgfsetmiterjoin%
\definecolor{currentfill}{rgb}{0.133298,0.375282,0.379395}%
\pgfsetfillcolor{currentfill}%
\pgfsetlinewidth{0.000000pt}%
\definecolor{currentstroke}{rgb}{0.000000,0.000000,0.000000}%
\pgfsetstrokecolor{currentstroke}%
\pgfsetstrokeopacity{0.000000}%
\pgfsetdash{}{0pt}%
\pgfpathmoveto{\pgfqpoint{0.921614in}{1.613090in}}%
\pgfpathlineto{\pgfqpoint{0.930367in}{1.613090in}}%
\pgfpathlineto{\pgfqpoint{0.930367in}{1.679738in}}%
\pgfpathlineto{\pgfqpoint{0.921614in}{1.679738in}}%
\pgfpathlineto{\pgfqpoint{0.921614in}{1.613090in}}%
\pgfpathclose%
\pgfusepath{fill}%
\end{pgfscope}%
\begin{pgfscope}%
\pgfpathrectangle{\pgfqpoint{0.804646in}{0.600000in}}{\pgfqpoint{2.573292in}{2.070576in}}%
\pgfusepath{clip}%
\pgfsetbuttcap%
\pgfsetmiterjoin%
\definecolor{currentfill}{rgb}{0.133298,0.375282,0.379395}%
\pgfsetfillcolor{currentfill}%
\pgfsetlinewidth{0.000000pt}%
\definecolor{currentstroke}{rgb}{0.000000,0.000000,0.000000}%
\pgfsetstrokecolor{currentstroke}%
\pgfsetstrokeopacity{0.000000}%
\pgfsetdash{}{0pt}%
\pgfpathmoveto{\pgfqpoint{0.932555in}{1.613090in}}%
\pgfpathlineto{\pgfqpoint{0.941309in}{1.613090in}}%
\pgfpathlineto{\pgfqpoint{0.941309in}{1.705593in}}%
\pgfpathlineto{\pgfqpoint{0.932555in}{1.705593in}}%
\pgfpathlineto{\pgfqpoint{0.932555in}{1.613090in}}%
\pgfpathclose%
\pgfusepath{fill}%
\end{pgfscope}%
\begin{pgfscope}%
\pgfpathrectangle{\pgfqpoint{0.804646in}{0.600000in}}{\pgfqpoint{2.573292in}{2.070576in}}%
\pgfusepath{clip}%
\pgfsetbuttcap%
\pgfsetmiterjoin%
\definecolor{currentfill}{rgb}{0.133298,0.375282,0.379395}%
\pgfsetfillcolor{currentfill}%
\pgfsetlinewidth{0.000000pt}%
\definecolor{currentstroke}{rgb}{0.000000,0.000000,0.000000}%
\pgfsetstrokecolor{currentstroke}%
\pgfsetstrokeopacity{0.000000}%
\pgfsetdash{}{0pt}%
\pgfpathmoveto{\pgfqpoint{0.943497in}{1.613090in}}%
\pgfpathlineto{\pgfqpoint{0.952251in}{1.613090in}}%
\pgfpathlineto{\pgfqpoint{0.952251in}{1.777818in}}%
\pgfpathlineto{\pgfqpoint{0.943497in}{1.777818in}}%
\pgfpathlineto{\pgfqpoint{0.943497in}{1.613090in}}%
\pgfpathclose%
\pgfusepath{fill}%
\end{pgfscope}%
\begin{pgfscope}%
\pgfpathrectangle{\pgfqpoint{0.804646in}{0.600000in}}{\pgfqpoint{2.573292in}{2.070576in}}%
\pgfusepath{clip}%
\pgfsetbuttcap%
\pgfsetmiterjoin%
\definecolor{currentfill}{rgb}{0.133298,0.375282,0.379395}%
\pgfsetfillcolor{currentfill}%
\pgfsetlinewidth{0.000000pt}%
\definecolor{currentstroke}{rgb}{0.000000,0.000000,0.000000}%
\pgfsetstrokecolor{currentstroke}%
\pgfsetstrokeopacity{0.000000}%
\pgfsetdash{}{0pt}%
\pgfpathmoveto{\pgfqpoint{0.954439in}{1.613090in}}%
\pgfpathlineto{\pgfqpoint{0.963192in}{1.613090in}}%
\pgfpathlineto{\pgfqpoint{0.963192in}{1.768375in}}%
\pgfpathlineto{\pgfqpoint{0.954439in}{1.768375in}}%
\pgfpathlineto{\pgfqpoint{0.954439in}{1.613090in}}%
\pgfpathclose%
\pgfusepath{fill}%
\end{pgfscope}%
\begin{pgfscope}%
\pgfpathrectangle{\pgfqpoint{0.804646in}{0.600000in}}{\pgfqpoint{2.573292in}{2.070576in}}%
\pgfusepath{clip}%
\pgfsetbuttcap%
\pgfsetmiterjoin%
\definecolor{currentfill}{rgb}{0.133298,0.375282,0.379395}%
\pgfsetfillcolor{currentfill}%
\pgfsetlinewidth{0.000000pt}%
\definecolor{currentstroke}{rgb}{0.000000,0.000000,0.000000}%
\pgfsetstrokecolor{currentstroke}%
\pgfsetstrokeopacity{0.000000}%
\pgfsetdash{}{0pt}%
\pgfpathmoveto{\pgfqpoint{0.965381in}{1.613090in}}%
\pgfpathlineto{\pgfqpoint{0.974134in}{1.613090in}}%
\pgfpathlineto{\pgfqpoint{0.974134in}{1.761115in}}%
\pgfpathlineto{\pgfqpoint{0.965381in}{1.761115in}}%
\pgfpathlineto{\pgfqpoint{0.965381in}{1.613090in}}%
\pgfpathclose%
\pgfusepath{fill}%
\end{pgfscope}%
\begin{pgfscope}%
\pgfpathrectangle{\pgfqpoint{0.804646in}{0.600000in}}{\pgfqpoint{2.573292in}{2.070576in}}%
\pgfusepath{clip}%
\pgfsetbuttcap%
\pgfsetmiterjoin%
\definecolor{currentfill}{rgb}{0.133298,0.375282,0.379395}%
\pgfsetfillcolor{currentfill}%
\pgfsetlinewidth{0.000000pt}%
\definecolor{currentstroke}{rgb}{0.000000,0.000000,0.000000}%
\pgfsetstrokecolor{currentstroke}%
\pgfsetstrokeopacity{0.000000}%
\pgfsetdash{}{0pt}%
\pgfpathmoveto{\pgfqpoint{0.976323in}{1.613090in}}%
\pgfpathlineto{\pgfqpoint{0.985076in}{1.613090in}}%
\pgfpathlineto{\pgfqpoint{0.985076in}{1.731548in}}%
\pgfpathlineto{\pgfqpoint{0.976323in}{1.731548in}}%
\pgfpathlineto{\pgfqpoint{0.976323in}{1.613090in}}%
\pgfpathclose%
\pgfusepath{fill}%
\end{pgfscope}%
\begin{pgfscope}%
\pgfpathrectangle{\pgfqpoint{0.804646in}{0.600000in}}{\pgfqpoint{2.573292in}{2.070576in}}%
\pgfusepath{clip}%
\pgfsetbuttcap%
\pgfsetmiterjoin%
\definecolor{currentfill}{rgb}{0.133298,0.375282,0.379395}%
\pgfsetfillcolor{currentfill}%
\pgfsetlinewidth{0.000000pt}%
\definecolor{currentstroke}{rgb}{0.000000,0.000000,0.000000}%
\pgfsetstrokecolor{currentstroke}%
\pgfsetstrokeopacity{0.000000}%
\pgfsetdash{}{0pt}%
\pgfpathmoveto{\pgfqpoint{0.987264in}{1.613090in}}%
\pgfpathlineto{\pgfqpoint{0.996018in}{1.613090in}}%
\pgfpathlineto{\pgfqpoint{0.996018in}{1.798276in}}%
\pgfpathlineto{\pgfqpoint{0.987264in}{1.798276in}}%
\pgfpathlineto{\pgfqpoint{0.987264in}{1.613090in}}%
\pgfpathclose%
\pgfusepath{fill}%
\end{pgfscope}%
\begin{pgfscope}%
\pgfpathrectangle{\pgfqpoint{0.804646in}{0.600000in}}{\pgfqpoint{2.573292in}{2.070576in}}%
\pgfusepath{clip}%
\pgfsetbuttcap%
\pgfsetmiterjoin%
\definecolor{currentfill}{rgb}{0.133298,0.375282,0.379395}%
\pgfsetfillcolor{currentfill}%
\pgfsetlinewidth{0.000000pt}%
\definecolor{currentstroke}{rgb}{0.000000,0.000000,0.000000}%
\pgfsetstrokecolor{currentstroke}%
\pgfsetstrokeopacity{0.000000}%
\pgfsetdash{}{0pt}%
\pgfpathmoveto{\pgfqpoint{0.998206in}{1.613090in}}%
\pgfpathlineto{\pgfqpoint{1.006960in}{1.613090in}}%
\pgfpathlineto{\pgfqpoint{1.006960in}{1.859805in}}%
\pgfpathlineto{\pgfqpoint{0.998206in}{1.859805in}}%
\pgfpathlineto{\pgfqpoint{0.998206in}{1.613090in}}%
\pgfpathclose%
\pgfusepath{fill}%
\end{pgfscope}%
\begin{pgfscope}%
\pgfpathrectangle{\pgfqpoint{0.804646in}{0.600000in}}{\pgfqpoint{2.573292in}{2.070576in}}%
\pgfusepath{clip}%
\pgfsetbuttcap%
\pgfsetmiterjoin%
\definecolor{currentfill}{rgb}{0.133298,0.375282,0.379395}%
\pgfsetfillcolor{currentfill}%
\pgfsetlinewidth{0.000000pt}%
\definecolor{currentstroke}{rgb}{0.000000,0.000000,0.000000}%
\pgfsetstrokecolor{currentstroke}%
\pgfsetstrokeopacity{0.000000}%
\pgfsetdash{}{0pt}%
\pgfpathmoveto{\pgfqpoint{1.009148in}{1.613090in}}%
\pgfpathlineto{\pgfqpoint{1.017901in}{1.613090in}}%
\pgfpathlineto{\pgfqpoint{1.017901in}{1.871745in}}%
\pgfpathlineto{\pgfqpoint{1.009148in}{1.871745in}}%
\pgfpathlineto{\pgfqpoint{1.009148in}{1.613090in}}%
\pgfpathclose%
\pgfusepath{fill}%
\end{pgfscope}%
\begin{pgfscope}%
\pgfpathrectangle{\pgfqpoint{0.804646in}{0.600000in}}{\pgfqpoint{2.573292in}{2.070576in}}%
\pgfusepath{clip}%
\pgfsetbuttcap%
\pgfsetmiterjoin%
\definecolor{currentfill}{rgb}{0.133298,0.375282,0.379395}%
\pgfsetfillcolor{currentfill}%
\pgfsetlinewidth{0.000000pt}%
\definecolor{currentstroke}{rgb}{0.000000,0.000000,0.000000}%
\pgfsetstrokecolor{currentstroke}%
\pgfsetstrokeopacity{0.000000}%
\pgfsetdash{}{0pt}%
\pgfpathmoveto{\pgfqpoint{1.020090in}{1.613090in}}%
\pgfpathlineto{\pgfqpoint{1.028843in}{1.613090in}}%
\pgfpathlineto{\pgfqpoint{1.028843in}{1.865596in}}%
\pgfpathlineto{\pgfqpoint{1.020090in}{1.865596in}}%
\pgfpathlineto{\pgfqpoint{1.020090in}{1.613090in}}%
\pgfpathclose%
\pgfusepath{fill}%
\end{pgfscope}%
\begin{pgfscope}%
\pgfpathrectangle{\pgfqpoint{0.804646in}{0.600000in}}{\pgfqpoint{2.573292in}{2.070576in}}%
\pgfusepath{clip}%
\pgfsetbuttcap%
\pgfsetmiterjoin%
\definecolor{currentfill}{rgb}{0.133298,0.375282,0.379395}%
\pgfsetfillcolor{currentfill}%
\pgfsetlinewidth{0.000000pt}%
\definecolor{currentstroke}{rgb}{0.000000,0.000000,0.000000}%
\pgfsetstrokecolor{currentstroke}%
\pgfsetstrokeopacity{0.000000}%
\pgfsetdash{}{0pt}%
\pgfpathmoveto{\pgfqpoint{1.031032in}{1.613090in}}%
\pgfpathlineto{\pgfqpoint{1.039785in}{1.613090in}}%
\pgfpathlineto{\pgfqpoint{1.039785in}{1.902239in}}%
\pgfpathlineto{\pgfqpoint{1.031032in}{1.902239in}}%
\pgfpathlineto{\pgfqpoint{1.031032in}{1.613090in}}%
\pgfpathclose%
\pgfusepath{fill}%
\end{pgfscope}%
\begin{pgfscope}%
\pgfpathrectangle{\pgfqpoint{0.804646in}{0.600000in}}{\pgfqpoint{2.573292in}{2.070576in}}%
\pgfusepath{clip}%
\pgfsetbuttcap%
\pgfsetmiterjoin%
\definecolor{currentfill}{rgb}{0.133298,0.375282,0.379395}%
\pgfsetfillcolor{currentfill}%
\pgfsetlinewidth{0.000000pt}%
\definecolor{currentstroke}{rgb}{0.000000,0.000000,0.000000}%
\pgfsetstrokecolor{currentstroke}%
\pgfsetstrokeopacity{0.000000}%
\pgfsetdash{}{0pt}%
\pgfpathmoveto{\pgfqpoint{1.041973in}{1.613090in}}%
\pgfpathlineto{\pgfqpoint{1.050727in}{1.613090in}}%
\pgfpathlineto{\pgfqpoint{1.050727in}{1.958597in}}%
\pgfpathlineto{\pgfqpoint{1.041973in}{1.958597in}}%
\pgfpathlineto{\pgfqpoint{1.041973in}{1.613090in}}%
\pgfpathclose%
\pgfusepath{fill}%
\end{pgfscope}%
\begin{pgfscope}%
\pgfpathrectangle{\pgfqpoint{0.804646in}{0.600000in}}{\pgfqpoint{2.573292in}{2.070576in}}%
\pgfusepath{clip}%
\pgfsetbuttcap%
\pgfsetmiterjoin%
\definecolor{currentfill}{rgb}{0.133298,0.375282,0.379395}%
\pgfsetfillcolor{currentfill}%
\pgfsetlinewidth{0.000000pt}%
\definecolor{currentstroke}{rgb}{0.000000,0.000000,0.000000}%
\pgfsetstrokecolor{currentstroke}%
\pgfsetstrokeopacity{0.000000}%
\pgfsetdash{}{0pt}%
\pgfpathmoveto{\pgfqpoint{1.052915in}{1.613090in}}%
\pgfpathlineto{\pgfqpoint{1.061669in}{1.613090in}}%
\pgfpathlineto{\pgfqpoint{1.061669in}{1.910876in}}%
\pgfpathlineto{\pgfqpoint{1.052915in}{1.910876in}}%
\pgfpathlineto{\pgfqpoint{1.052915in}{1.613090in}}%
\pgfpathclose%
\pgfusepath{fill}%
\end{pgfscope}%
\begin{pgfscope}%
\pgfpathrectangle{\pgfqpoint{0.804646in}{0.600000in}}{\pgfqpoint{2.573292in}{2.070576in}}%
\pgfusepath{clip}%
\pgfsetbuttcap%
\pgfsetmiterjoin%
\definecolor{currentfill}{rgb}{0.133298,0.375282,0.379395}%
\pgfsetfillcolor{currentfill}%
\pgfsetlinewidth{0.000000pt}%
\definecolor{currentstroke}{rgb}{0.000000,0.000000,0.000000}%
\pgfsetstrokecolor{currentstroke}%
\pgfsetstrokeopacity{0.000000}%
\pgfsetdash{}{0pt}%
\pgfpathmoveto{\pgfqpoint{1.063857in}{1.613090in}}%
\pgfpathlineto{\pgfqpoint{1.072610in}{1.613090in}}%
\pgfpathlineto{\pgfqpoint{1.072610in}{1.977739in}}%
\pgfpathlineto{\pgfqpoint{1.063857in}{1.977739in}}%
\pgfpathlineto{\pgfqpoint{1.063857in}{1.613090in}}%
\pgfpathclose%
\pgfusepath{fill}%
\end{pgfscope}%
\begin{pgfscope}%
\pgfpathrectangle{\pgfqpoint{0.804646in}{0.600000in}}{\pgfqpoint{2.573292in}{2.070576in}}%
\pgfusepath{clip}%
\pgfsetbuttcap%
\pgfsetmiterjoin%
\definecolor{currentfill}{rgb}{0.133298,0.375282,0.379395}%
\pgfsetfillcolor{currentfill}%
\pgfsetlinewidth{0.000000pt}%
\definecolor{currentstroke}{rgb}{0.000000,0.000000,0.000000}%
\pgfsetstrokecolor{currentstroke}%
\pgfsetstrokeopacity{0.000000}%
\pgfsetdash{}{0pt}%
\pgfpathmoveto{\pgfqpoint{1.074799in}{1.613090in}}%
\pgfpathlineto{\pgfqpoint{1.083552in}{1.613090in}}%
\pgfpathlineto{\pgfqpoint{1.083552in}{1.899421in}}%
\pgfpathlineto{\pgfqpoint{1.074799in}{1.899421in}}%
\pgfpathlineto{\pgfqpoint{1.074799in}{1.613090in}}%
\pgfpathclose%
\pgfusepath{fill}%
\end{pgfscope}%
\begin{pgfscope}%
\pgfpathrectangle{\pgfqpoint{0.804646in}{0.600000in}}{\pgfqpoint{2.573292in}{2.070576in}}%
\pgfusepath{clip}%
\pgfsetbuttcap%
\pgfsetmiterjoin%
\definecolor{currentfill}{rgb}{0.133298,0.375282,0.379395}%
\pgfsetfillcolor{currentfill}%
\pgfsetlinewidth{0.000000pt}%
\definecolor{currentstroke}{rgb}{0.000000,0.000000,0.000000}%
\pgfsetstrokecolor{currentstroke}%
\pgfsetstrokeopacity{0.000000}%
\pgfsetdash{}{0pt}%
\pgfpathmoveto{\pgfqpoint{1.085741in}{1.613090in}}%
\pgfpathlineto{\pgfqpoint{1.094494in}{1.613090in}}%
\pgfpathlineto{\pgfqpoint{1.094494in}{1.914920in}}%
\pgfpathlineto{\pgfqpoint{1.085741in}{1.914920in}}%
\pgfpathlineto{\pgfqpoint{1.085741in}{1.613090in}}%
\pgfpathclose%
\pgfusepath{fill}%
\end{pgfscope}%
\begin{pgfscope}%
\pgfpathrectangle{\pgfqpoint{0.804646in}{0.600000in}}{\pgfqpoint{2.573292in}{2.070576in}}%
\pgfusepath{clip}%
\pgfsetbuttcap%
\pgfsetmiterjoin%
\definecolor{currentfill}{rgb}{0.133298,0.375282,0.379395}%
\pgfsetfillcolor{currentfill}%
\pgfsetlinewidth{0.000000pt}%
\definecolor{currentstroke}{rgb}{0.000000,0.000000,0.000000}%
\pgfsetstrokecolor{currentstroke}%
\pgfsetstrokeopacity{0.000000}%
\pgfsetdash{}{0pt}%
\pgfpathmoveto{\pgfqpoint{1.096682in}{1.613090in}}%
\pgfpathlineto{\pgfqpoint{1.105436in}{1.613090in}}%
\pgfpathlineto{\pgfqpoint{1.105436in}{1.984351in}}%
\pgfpathlineto{\pgfqpoint{1.096682in}{1.984351in}}%
\pgfpathlineto{\pgfqpoint{1.096682in}{1.613090in}}%
\pgfpathclose%
\pgfusepath{fill}%
\end{pgfscope}%
\begin{pgfscope}%
\pgfpathrectangle{\pgfqpoint{0.804646in}{0.600000in}}{\pgfqpoint{2.573292in}{2.070576in}}%
\pgfusepath{clip}%
\pgfsetbuttcap%
\pgfsetmiterjoin%
\definecolor{currentfill}{rgb}{0.133298,0.375282,0.379395}%
\pgfsetfillcolor{currentfill}%
\pgfsetlinewidth{0.000000pt}%
\definecolor{currentstroke}{rgb}{0.000000,0.000000,0.000000}%
\pgfsetstrokecolor{currentstroke}%
\pgfsetstrokeopacity{0.000000}%
\pgfsetdash{}{0pt}%
\pgfpathmoveto{\pgfqpoint{1.107624in}{1.613090in}}%
\pgfpathlineto{\pgfqpoint{1.116378in}{1.613090in}}%
\pgfpathlineto{\pgfqpoint{1.116378in}{1.909332in}}%
\pgfpathlineto{\pgfqpoint{1.107624in}{1.909332in}}%
\pgfpathlineto{\pgfqpoint{1.107624in}{1.613090in}}%
\pgfpathclose%
\pgfusepath{fill}%
\end{pgfscope}%
\begin{pgfscope}%
\pgfpathrectangle{\pgfqpoint{0.804646in}{0.600000in}}{\pgfqpoint{2.573292in}{2.070576in}}%
\pgfusepath{clip}%
\pgfsetbuttcap%
\pgfsetmiterjoin%
\definecolor{currentfill}{rgb}{0.133298,0.375282,0.379395}%
\pgfsetfillcolor{currentfill}%
\pgfsetlinewidth{0.000000pt}%
\definecolor{currentstroke}{rgb}{0.000000,0.000000,0.000000}%
\pgfsetstrokecolor{currentstroke}%
\pgfsetstrokeopacity{0.000000}%
\pgfsetdash{}{0pt}%
\pgfpathmoveto{\pgfqpoint{1.118566in}{1.613090in}}%
\pgfpathlineto{\pgfqpoint{1.127319in}{1.613090in}}%
\pgfpathlineto{\pgfqpoint{1.127319in}{1.973166in}}%
\pgfpathlineto{\pgfqpoint{1.118566in}{1.973166in}}%
\pgfpathlineto{\pgfqpoint{1.118566in}{1.613090in}}%
\pgfpathclose%
\pgfusepath{fill}%
\end{pgfscope}%
\begin{pgfscope}%
\pgfpathrectangle{\pgfqpoint{0.804646in}{0.600000in}}{\pgfqpoint{2.573292in}{2.070576in}}%
\pgfusepath{clip}%
\pgfsetbuttcap%
\pgfsetmiterjoin%
\definecolor{currentfill}{rgb}{0.133298,0.375282,0.379395}%
\pgfsetfillcolor{currentfill}%
\pgfsetlinewidth{0.000000pt}%
\definecolor{currentstroke}{rgb}{0.000000,0.000000,0.000000}%
\pgfsetstrokecolor{currentstroke}%
\pgfsetstrokeopacity{0.000000}%
\pgfsetdash{}{0pt}%
\pgfpathmoveto{\pgfqpoint{1.129508in}{1.613090in}}%
\pgfpathlineto{\pgfqpoint{1.138261in}{1.613090in}}%
\pgfpathlineto{\pgfqpoint{1.138261in}{1.960073in}}%
\pgfpathlineto{\pgfqpoint{1.129508in}{1.960073in}}%
\pgfpathlineto{\pgfqpoint{1.129508in}{1.613090in}}%
\pgfpathclose%
\pgfusepath{fill}%
\end{pgfscope}%
\begin{pgfscope}%
\pgfpathrectangle{\pgfqpoint{0.804646in}{0.600000in}}{\pgfqpoint{2.573292in}{2.070576in}}%
\pgfusepath{clip}%
\pgfsetbuttcap%
\pgfsetmiterjoin%
\definecolor{currentfill}{rgb}{0.133298,0.375282,0.379395}%
\pgfsetfillcolor{currentfill}%
\pgfsetlinewidth{0.000000pt}%
\definecolor{currentstroke}{rgb}{0.000000,0.000000,0.000000}%
\pgfsetstrokecolor{currentstroke}%
\pgfsetstrokeopacity{0.000000}%
\pgfsetdash{}{0pt}%
\pgfpathmoveto{\pgfqpoint{1.140450in}{1.613090in}}%
\pgfpathlineto{\pgfqpoint{1.149203in}{1.613090in}}%
\pgfpathlineto{\pgfqpoint{1.149203in}{2.004166in}}%
\pgfpathlineto{\pgfqpoint{1.140450in}{2.004166in}}%
\pgfpathlineto{\pgfqpoint{1.140450in}{1.613090in}}%
\pgfpathclose%
\pgfusepath{fill}%
\end{pgfscope}%
\begin{pgfscope}%
\pgfpathrectangle{\pgfqpoint{0.804646in}{0.600000in}}{\pgfqpoint{2.573292in}{2.070576in}}%
\pgfusepath{clip}%
\pgfsetbuttcap%
\pgfsetmiterjoin%
\definecolor{currentfill}{rgb}{0.133298,0.375282,0.379395}%
\pgfsetfillcolor{currentfill}%
\pgfsetlinewidth{0.000000pt}%
\definecolor{currentstroke}{rgb}{0.000000,0.000000,0.000000}%
\pgfsetstrokecolor{currentstroke}%
\pgfsetstrokeopacity{0.000000}%
\pgfsetdash{}{0pt}%
\pgfpathmoveto{\pgfqpoint{1.151391in}{1.613090in}}%
\pgfpathlineto{\pgfqpoint{1.160145in}{1.613090in}}%
\pgfpathlineto{\pgfqpoint{1.160145in}{2.058582in}}%
\pgfpathlineto{\pgfqpoint{1.151391in}{2.058582in}}%
\pgfpathlineto{\pgfqpoint{1.151391in}{1.613090in}}%
\pgfpathclose%
\pgfusepath{fill}%
\end{pgfscope}%
\begin{pgfscope}%
\pgfpathrectangle{\pgfqpoint{0.804646in}{0.600000in}}{\pgfqpoint{2.573292in}{2.070576in}}%
\pgfusepath{clip}%
\pgfsetbuttcap%
\pgfsetmiterjoin%
\definecolor{currentfill}{rgb}{0.133298,0.375282,0.379395}%
\pgfsetfillcolor{currentfill}%
\pgfsetlinewidth{0.000000pt}%
\definecolor{currentstroke}{rgb}{0.000000,0.000000,0.000000}%
\pgfsetstrokecolor{currentstroke}%
\pgfsetstrokeopacity{0.000000}%
\pgfsetdash{}{0pt}%
\pgfpathmoveto{\pgfqpoint{1.162333in}{1.613090in}}%
\pgfpathlineto{\pgfqpoint{1.171087in}{1.613090in}}%
\pgfpathlineto{\pgfqpoint{1.171087in}{2.183762in}}%
\pgfpathlineto{\pgfqpoint{1.162333in}{2.183762in}}%
\pgfpathlineto{\pgfqpoint{1.162333in}{1.613090in}}%
\pgfpathclose%
\pgfusepath{fill}%
\end{pgfscope}%
\begin{pgfscope}%
\pgfpathrectangle{\pgfqpoint{0.804646in}{0.600000in}}{\pgfqpoint{2.573292in}{2.070576in}}%
\pgfusepath{clip}%
\pgfsetbuttcap%
\pgfsetmiterjoin%
\definecolor{currentfill}{rgb}{0.133298,0.375282,0.379395}%
\pgfsetfillcolor{currentfill}%
\pgfsetlinewidth{0.000000pt}%
\definecolor{currentstroke}{rgb}{0.000000,0.000000,0.000000}%
\pgfsetstrokecolor{currentstroke}%
\pgfsetstrokeopacity{0.000000}%
\pgfsetdash{}{0pt}%
\pgfpathmoveto{\pgfqpoint{1.173275in}{1.613090in}}%
\pgfpathlineto{\pgfqpoint{1.182028in}{1.613090in}}%
\pgfpathlineto{\pgfqpoint{1.182028in}{2.279656in}}%
\pgfpathlineto{\pgfqpoint{1.173275in}{2.279656in}}%
\pgfpathlineto{\pgfqpoint{1.173275in}{1.613090in}}%
\pgfpathclose%
\pgfusepath{fill}%
\end{pgfscope}%
\begin{pgfscope}%
\pgfpathrectangle{\pgfqpoint{0.804646in}{0.600000in}}{\pgfqpoint{2.573292in}{2.070576in}}%
\pgfusepath{clip}%
\pgfsetbuttcap%
\pgfsetmiterjoin%
\definecolor{currentfill}{rgb}{0.133298,0.375282,0.379395}%
\pgfsetfillcolor{currentfill}%
\pgfsetlinewidth{0.000000pt}%
\definecolor{currentstroke}{rgb}{0.000000,0.000000,0.000000}%
\pgfsetstrokecolor{currentstroke}%
\pgfsetstrokeopacity{0.000000}%
\pgfsetdash{}{0pt}%
\pgfpathmoveto{\pgfqpoint{1.184217in}{1.613090in}}%
\pgfpathlineto{\pgfqpoint{1.192970in}{1.613090in}}%
\pgfpathlineto{\pgfqpoint{1.192970in}{2.242173in}}%
\pgfpathlineto{\pgfqpoint{1.184217in}{2.242173in}}%
\pgfpathlineto{\pgfqpoint{1.184217in}{1.613090in}}%
\pgfpathclose%
\pgfusepath{fill}%
\end{pgfscope}%
\begin{pgfscope}%
\pgfpathrectangle{\pgfqpoint{0.804646in}{0.600000in}}{\pgfqpoint{2.573292in}{2.070576in}}%
\pgfusepath{clip}%
\pgfsetbuttcap%
\pgfsetmiterjoin%
\definecolor{currentfill}{rgb}{0.133298,0.375282,0.379395}%
\pgfsetfillcolor{currentfill}%
\pgfsetlinewidth{0.000000pt}%
\definecolor{currentstroke}{rgb}{0.000000,0.000000,0.000000}%
\pgfsetstrokecolor{currentstroke}%
\pgfsetstrokeopacity{0.000000}%
\pgfsetdash{}{0pt}%
\pgfpathmoveto{\pgfqpoint{1.195159in}{1.613090in}}%
\pgfpathlineto{\pgfqpoint{1.203912in}{1.613090in}}%
\pgfpathlineto{\pgfqpoint{1.203912in}{2.276815in}}%
\pgfpathlineto{\pgfqpoint{1.195159in}{2.276815in}}%
\pgfpathlineto{\pgfqpoint{1.195159in}{1.613090in}}%
\pgfpathclose%
\pgfusepath{fill}%
\end{pgfscope}%
\begin{pgfscope}%
\pgfpathrectangle{\pgfqpoint{0.804646in}{0.600000in}}{\pgfqpoint{2.573292in}{2.070576in}}%
\pgfusepath{clip}%
\pgfsetbuttcap%
\pgfsetmiterjoin%
\definecolor{currentfill}{rgb}{0.133298,0.375282,0.379395}%
\pgfsetfillcolor{currentfill}%
\pgfsetlinewidth{0.000000pt}%
\definecolor{currentstroke}{rgb}{0.000000,0.000000,0.000000}%
\pgfsetstrokecolor{currentstroke}%
\pgfsetstrokeopacity{0.000000}%
\pgfsetdash{}{0pt}%
\pgfpathmoveto{\pgfqpoint{1.206100in}{1.613090in}}%
\pgfpathlineto{\pgfqpoint{1.214854in}{1.613090in}}%
\pgfpathlineto{\pgfqpoint{1.214854in}{2.313304in}}%
\pgfpathlineto{\pgfqpoint{1.206100in}{2.313304in}}%
\pgfpathlineto{\pgfqpoint{1.206100in}{1.613090in}}%
\pgfpathclose%
\pgfusepath{fill}%
\end{pgfscope}%
\begin{pgfscope}%
\pgfpathrectangle{\pgfqpoint{0.804646in}{0.600000in}}{\pgfqpoint{2.573292in}{2.070576in}}%
\pgfusepath{clip}%
\pgfsetbuttcap%
\pgfsetmiterjoin%
\definecolor{currentfill}{rgb}{0.133298,0.375282,0.379395}%
\pgfsetfillcolor{currentfill}%
\pgfsetlinewidth{0.000000pt}%
\definecolor{currentstroke}{rgb}{0.000000,0.000000,0.000000}%
\pgfsetstrokecolor{currentstroke}%
\pgfsetstrokeopacity{0.000000}%
\pgfsetdash{}{0pt}%
\pgfpathmoveto{\pgfqpoint{1.217042in}{1.613090in}}%
\pgfpathlineto{\pgfqpoint{1.225796in}{1.613090in}}%
\pgfpathlineto{\pgfqpoint{1.225796in}{2.292827in}}%
\pgfpathlineto{\pgfqpoint{1.217042in}{2.292827in}}%
\pgfpathlineto{\pgfqpoint{1.217042in}{1.613090in}}%
\pgfpathclose%
\pgfusepath{fill}%
\end{pgfscope}%
\begin{pgfscope}%
\pgfpathrectangle{\pgfqpoint{0.804646in}{0.600000in}}{\pgfqpoint{2.573292in}{2.070576in}}%
\pgfusepath{clip}%
\pgfsetbuttcap%
\pgfsetmiterjoin%
\definecolor{currentfill}{rgb}{0.133298,0.375282,0.379395}%
\pgfsetfillcolor{currentfill}%
\pgfsetlinewidth{0.000000pt}%
\definecolor{currentstroke}{rgb}{0.000000,0.000000,0.000000}%
\pgfsetstrokecolor{currentstroke}%
\pgfsetstrokeopacity{0.000000}%
\pgfsetdash{}{0pt}%
\pgfpathmoveto{\pgfqpoint{1.227984in}{1.613090in}}%
\pgfpathlineto{\pgfqpoint{1.236737in}{1.613090in}}%
\pgfpathlineto{\pgfqpoint{1.236737in}{2.239364in}}%
\pgfpathlineto{\pgfqpoint{1.227984in}{2.239364in}}%
\pgfpathlineto{\pgfqpoint{1.227984in}{1.613090in}}%
\pgfpathclose%
\pgfusepath{fill}%
\end{pgfscope}%
\begin{pgfscope}%
\pgfpathrectangle{\pgfqpoint{0.804646in}{0.600000in}}{\pgfqpoint{2.573292in}{2.070576in}}%
\pgfusepath{clip}%
\pgfsetbuttcap%
\pgfsetmiterjoin%
\definecolor{currentfill}{rgb}{0.133298,0.375282,0.379395}%
\pgfsetfillcolor{currentfill}%
\pgfsetlinewidth{0.000000pt}%
\definecolor{currentstroke}{rgb}{0.000000,0.000000,0.000000}%
\pgfsetstrokecolor{currentstroke}%
\pgfsetstrokeopacity{0.000000}%
\pgfsetdash{}{0pt}%
\pgfpathmoveto{\pgfqpoint{1.238926in}{1.613090in}}%
\pgfpathlineto{\pgfqpoint{1.247679in}{1.613090in}}%
\pgfpathlineto{\pgfqpoint{1.247679in}{2.190723in}}%
\pgfpathlineto{\pgfqpoint{1.238926in}{2.190723in}}%
\pgfpathlineto{\pgfqpoint{1.238926in}{1.613090in}}%
\pgfpathclose%
\pgfusepath{fill}%
\end{pgfscope}%
\begin{pgfscope}%
\pgfpathrectangle{\pgfqpoint{0.804646in}{0.600000in}}{\pgfqpoint{2.573292in}{2.070576in}}%
\pgfusepath{clip}%
\pgfsetbuttcap%
\pgfsetmiterjoin%
\definecolor{currentfill}{rgb}{0.133298,0.375282,0.379395}%
\pgfsetfillcolor{currentfill}%
\pgfsetlinewidth{0.000000pt}%
\definecolor{currentstroke}{rgb}{0.000000,0.000000,0.000000}%
\pgfsetstrokecolor{currentstroke}%
\pgfsetstrokeopacity{0.000000}%
\pgfsetdash{}{0pt}%
\pgfpathmoveto{\pgfqpoint{1.249868in}{1.613090in}}%
\pgfpathlineto{\pgfqpoint{1.258621in}{1.613090in}}%
\pgfpathlineto{\pgfqpoint{1.258621in}{2.153157in}}%
\pgfpathlineto{\pgfqpoint{1.249868in}{2.153157in}}%
\pgfpathlineto{\pgfqpoint{1.249868in}{1.613090in}}%
\pgfpathclose%
\pgfusepath{fill}%
\end{pgfscope}%
\begin{pgfscope}%
\pgfpathrectangle{\pgfqpoint{0.804646in}{0.600000in}}{\pgfqpoint{2.573292in}{2.070576in}}%
\pgfusepath{clip}%
\pgfsetbuttcap%
\pgfsetmiterjoin%
\definecolor{currentfill}{rgb}{0.133298,0.375282,0.379395}%
\pgfsetfillcolor{currentfill}%
\pgfsetlinewidth{0.000000pt}%
\definecolor{currentstroke}{rgb}{0.000000,0.000000,0.000000}%
\pgfsetstrokecolor{currentstroke}%
\pgfsetstrokeopacity{0.000000}%
\pgfsetdash{}{0pt}%
\pgfpathmoveto{\pgfqpoint{1.260809in}{1.613090in}}%
\pgfpathlineto{\pgfqpoint{1.269563in}{1.613090in}}%
\pgfpathlineto{\pgfqpoint{1.269563in}{2.040519in}}%
\pgfpathlineto{\pgfqpoint{1.260809in}{2.040519in}}%
\pgfpathlineto{\pgfqpoint{1.260809in}{1.613090in}}%
\pgfpathclose%
\pgfusepath{fill}%
\end{pgfscope}%
\begin{pgfscope}%
\pgfpathrectangle{\pgfqpoint{0.804646in}{0.600000in}}{\pgfqpoint{2.573292in}{2.070576in}}%
\pgfusepath{clip}%
\pgfsetbuttcap%
\pgfsetmiterjoin%
\definecolor{currentfill}{rgb}{0.133298,0.375282,0.379395}%
\pgfsetfillcolor{currentfill}%
\pgfsetlinewidth{0.000000pt}%
\definecolor{currentstroke}{rgb}{0.000000,0.000000,0.000000}%
\pgfsetstrokecolor{currentstroke}%
\pgfsetstrokeopacity{0.000000}%
\pgfsetdash{}{0pt}%
\pgfpathmoveto{\pgfqpoint{1.271751in}{1.613090in}}%
\pgfpathlineto{\pgfqpoint{1.280505in}{1.613090in}}%
\pgfpathlineto{\pgfqpoint{1.280505in}{2.024034in}}%
\pgfpathlineto{\pgfqpoint{1.271751in}{2.024034in}}%
\pgfpathlineto{\pgfqpoint{1.271751in}{1.613090in}}%
\pgfpathclose%
\pgfusepath{fill}%
\end{pgfscope}%
\begin{pgfscope}%
\pgfpathrectangle{\pgfqpoint{0.804646in}{0.600000in}}{\pgfqpoint{2.573292in}{2.070576in}}%
\pgfusepath{clip}%
\pgfsetbuttcap%
\pgfsetmiterjoin%
\definecolor{currentfill}{rgb}{0.133298,0.375282,0.379395}%
\pgfsetfillcolor{currentfill}%
\pgfsetlinewidth{0.000000pt}%
\definecolor{currentstroke}{rgb}{0.000000,0.000000,0.000000}%
\pgfsetstrokecolor{currentstroke}%
\pgfsetstrokeopacity{0.000000}%
\pgfsetdash{}{0pt}%
\pgfpathmoveto{\pgfqpoint{1.282693in}{1.613090in}}%
\pgfpathlineto{\pgfqpoint{1.291446in}{1.613090in}}%
\pgfpathlineto{\pgfqpoint{1.291446in}{1.956887in}}%
\pgfpathlineto{\pgfqpoint{1.282693in}{1.956887in}}%
\pgfpathlineto{\pgfqpoint{1.282693in}{1.613090in}}%
\pgfpathclose%
\pgfusepath{fill}%
\end{pgfscope}%
\begin{pgfscope}%
\pgfpathrectangle{\pgfqpoint{0.804646in}{0.600000in}}{\pgfqpoint{2.573292in}{2.070576in}}%
\pgfusepath{clip}%
\pgfsetbuttcap%
\pgfsetmiterjoin%
\definecolor{currentfill}{rgb}{0.133298,0.375282,0.379395}%
\pgfsetfillcolor{currentfill}%
\pgfsetlinewidth{0.000000pt}%
\definecolor{currentstroke}{rgb}{0.000000,0.000000,0.000000}%
\pgfsetstrokecolor{currentstroke}%
\pgfsetstrokeopacity{0.000000}%
\pgfsetdash{}{0pt}%
\pgfpathmoveto{\pgfqpoint{1.293635in}{1.613090in}}%
\pgfpathlineto{\pgfqpoint{1.302388in}{1.613090in}}%
\pgfpathlineto{\pgfqpoint{1.302388in}{1.837028in}}%
\pgfpathlineto{\pgfqpoint{1.293635in}{1.837028in}}%
\pgfpathlineto{\pgfqpoint{1.293635in}{1.613090in}}%
\pgfpathclose%
\pgfusepath{fill}%
\end{pgfscope}%
\begin{pgfscope}%
\pgfpathrectangle{\pgfqpoint{0.804646in}{0.600000in}}{\pgfqpoint{2.573292in}{2.070576in}}%
\pgfusepath{clip}%
\pgfsetbuttcap%
\pgfsetmiterjoin%
\definecolor{currentfill}{rgb}{0.133298,0.375282,0.379395}%
\pgfsetfillcolor{currentfill}%
\pgfsetlinewidth{0.000000pt}%
\definecolor{currentstroke}{rgb}{0.000000,0.000000,0.000000}%
\pgfsetstrokecolor{currentstroke}%
\pgfsetstrokeopacity{0.000000}%
\pgfsetdash{}{0pt}%
\pgfpathmoveto{\pgfqpoint{1.304577in}{1.613090in}}%
\pgfpathlineto{\pgfqpoint{1.313330in}{1.613090in}}%
\pgfpathlineto{\pgfqpoint{1.313330in}{1.851730in}}%
\pgfpathlineto{\pgfqpoint{1.304577in}{1.851730in}}%
\pgfpathlineto{\pgfqpoint{1.304577in}{1.613090in}}%
\pgfpathclose%
\pgfusepath{fill}%
\end{pgfscope}%
\begin{pgfscope}%
\pgfpathrectangle{\pgfqpoint{0.804646in}{0.600000in}}{\pgfqpoint{2.573292in}{2.070576in}}%
\pgfusepath{clip}%
\pgfsetbuttcap%
\pgfsetmiterjoin%
\definecolor{currentfill}{rgb}{0.133298,0.375282,0.379395}%
\pgfsetfillcolor{currentfill}%
\pgfsetlinewidth{0.000000pt}%
\definecolor{currentstroke}{rgb}{0.000000,0.000000,0.000000}%
\pgfsetstrokecolor{currentstroke}%
\pgfsetstrokeopacity{0.000000}%
\pgfsetdash{}{0pt}%
\pgfpathmoveto{\pgfqpoint{1.315518in}{1.613090in}}%
\pgfpathlineto{\pgfqpoint{1.324272in}{1.613090in}}%
\pgfpathlineto{\pgfqpoint{1.324272in}{1.802593in}}%
\pgfpathlineto{\pgfqpoint{1.315518in}{1.802593in}}%
\pgfpathlineto{\pgfqpoint{1.315518in}{1.613090in}}%
\pgfpathclose%
\pgfusepath{fill}%
\end{pgfscope}%
\begin{pgfscope}%
\pgfpathrectangle{\pgfqpoint{0.804646in}{0.600000in}}{\pgfqpoint{2.573292in}{2.070576in}}%
\pgfusepath{clip}%
\pgfsetbuttcap%
\pgfsetmiterjoin%
\definecolor{currentfill}{rgb}{0.133298,0.375282,0.379395}%
\pgfsetfillcolor{currentfill}%
\pgfsetlinewidth{0.000000pt}%
\definecolor{currentstroke}{rgb}{0.000000,0.000000,0.000000}%
\pgfsetstrokecolor{currentstroke}%
\pgfsetstrokeopacity{0.000000}%
\pgfsetdash{}{0pt}%
\pgfpathmoveto{\pgfqpoint{1.326460in}{1.613090in}}%
\pgfpathlineto{\pgfqpoint{1.335214in}{1.613090in}}%
\pgfpathlineto{\pgfqpoint{1.335214in}{1.801845in}}%
\pgfpathlineto{\pgfqpoint{1.326460in}{1.801845in}}%
\pgfpathlineto{\pgfqpoint{1.326460in}{1.613090in}}%
\pgfpathclose%
\pgfusepath{fill}%
\end{pgfscope}%
\begin{pgfscope}%
\pgfpathrectangle{\pgfqpoint{0.804646in}{0.600000in}}{\pgfqpoint{2.573292in}{2.070576in}}%
\pgfusepath{clip}%
\pgfsetbuttcap%
\pgfsetmiterjoin%
\definecolor{currentfill}{rgb}{0.133298,0.375282,0.379395}%
\pgfsetfillcolor{currentfill}%
\pgfsetlinewidth{0.000000pt}%
\definecolor{currentstroke}{rgb}{0.000000,0.000000,0.000000}%
\pgfsetstrokecolor{currentstroke}%
\pgfsetstrokeopacity{0.000000}%
\pgfsetdash{}{0pt}%
\pgfpathmoveto{\pgfqpoint{1.337402in}{1.613090in}}%
\pgfpathlineto{\pgfqpoint{1.346155in}{1.613090in}}%
\pgfpathlineto{\pgfqpoint{1.346155in}{1.874175in}}%
\pgfpathlineto{\pgfqpoint{1.337402in}{1.874175in}}%
\pgfpathlineto{\pgfqpoint{1.337402in}{1.613090in}}%
\pgfpathclose%
\pgfusepath{fill}%
\end{pgfscope}%
\begin{pgfscope}%
\pgfpathrectangle{\pgfqpoint{0.804646in}{0.600000in}}{\pgfqpoint{2.573292in}{2.070576in}}%
\pgfusepath{clip}%
\pgfsetbuttcap%
\pgfsetmiterjoin%
\definecolor{currentfill}{rgb}{0.133298,0.375282,0.379395}%
\pgfsetfillcolor{currentfill}%
\pgfsetlinewidth{0.000000pt}%
\definecolor{currentstroke}{rgb}{0.000000,0.000000,0.000000}%
\pgfsetstrokecolor{currentstroke}%
\pgfsetstrokeopacity{0.000000}%
\pgfsetdash{}{0pt}%
\pgfpathmoveto{\pgfqpoint{1.348344in}{1.613090in}}%
\pgfpathlineto{\pgfqpoint{1.357097in}{1.613090in}}%
\pgfpathlineto{\pgfqpoint{1.357097in}{1.909217in}}%
\pgfpathlineto{\pgfqpoint{1.348344in}{1.909217in}}%
\pgfpathlineto{\pgfqpoint{1.348344in}{1.613090in}}%
\pgfpathclose%
\pgfusepath{fill}%
\end{pgfscope}%
\begin{pgfscope}%
\pgfpathrectangle{\pgfqpoint{0.804646in}{0.600000in}}{\pgfqpoint{2.573292in}{2.070576in}}%
\pgfusepath{clip}%
\pgfsetbuttcap%
\pgfsetmiterjoin%
\definecolor{currentfill}{rgb}{0.133298,0.375282,0.379395}%
\pgfsetfillcolor{currentfill}%
\pgfsetlinewidth{0.000000pt}%
\definecolor{currentstroke}{rgb}{0.000000,0.000000,0.000000}%
\pgfsetstrokecolor{currentstroke}%
\pgfsetstrokeopacity{0.000000}%
\pgfsetdash{}{0pt}%
\pgfpathmoveto{\pgfqpoint{1.359286in}{1.613090in}}%
\pgfpathlineto{\pgfqpoint{1.368039in}{1.613090in}}%
\pgfpathlineto{\pgfqpoint{1.368039in}{1.903057in}}%
\pgfpathlineto{\pgfqpoint{1.359286in}{1.903057in}}%
\pgfpathlineto{\pgfqpoint{1.359286in}{1.613090in}}%
\pgfpathclose%
\pgfusepath{fill}%
\end{pgfscope}%
\begin{pgfscope}%
\pgfpathrectangle{\pgfqpoint{0.804646in}{0.600000in}}{\pgfqpoint{2.573292in}{2.070576in}}%
\pgfusepath{clip}%
\pgfsetbuttcap%
\pgfsetmiterjoin%
\definecolor{currentfill}{rgb}{0.133298,0.375282,0.379395}%
\pgfsetfillcolor{currentfill}%
\pgfsetlinewidth{0.000000pt}%
\definecolor{currentstroke}{rgb}{0.000000,0.000000,0.000000}%
\pgfsetstrokecolor{currentstroke}%
\pgfsetstrokeopacity{0.000000}%
\pgfsetdash{}{0pt}%
\pgfpathmoveto{\pgfqpoint{1.370227in}{1.613090in}}%
\pgfpathlineto{\pgfqpoint{1.378981in}{1.613090in}}%
\pgfpathlineto{\pgfqpoint{1.378981in}{1.876695in}}%
\pgfpathlineto{\pgfqpoint{1.370227in}{1.876695in}}%
\pgfpathlineto{\pgfqpoint{1.370227in}{1.613090in}}%
\pgfpathclose%
\pgfusepath{fill}%
\end{pgfscope}%
\begin{pgfscope}%
\pgfpathrectangle{\pgfqpoint{0.804646in}{0.600000in}}{\pgfqpoint{2.573292in}{2.070576in}}%
\pgfusepath{clip}%
\pgfsetbuttcap%
\pgfsetmiterjoin%
\definecolor{currentfill}{rgb}{0.133298,0.375282,0.379395}%
\pgfsetfillcolor{currentfill}%
\pgfsetlinewidth{0.000000pt}%
\definecolor{currentstroke}{rgb}{0.000000,0.000000,0.000000}%
\pgfsetstrokecolor{currentstroke}%
\pgfsetstrokeopacity{0.000000}%
\pgfsetdash{}{0pt}%
\pgfpathmoveto{\pgfqpoint{1.381169in}{1.613090in}}%
\pgfpathlineto{\pgfqpoint{1.389923in}{1.613090in}}%
\pgfpathlineto{\pgfqpoint{1.389923in}{1.903494in}}%
\pgfpathlineto{\pgfqpoint{1.381169in}{1.903494in}}%
\pgfpathlineto{\pgfqpoint{1.381169in}{1.613090in}}%
\pgfpathclose%
\pgfusepath{fill}%
\end{pgfscope}%
\begin{pgfscope}%
\pgfpathrectangle{\pgfqpoint{0.804646in}{0.600000in}}{\pgfqpoint{2.573292in}{2.070576in}}%
\pgfusepath{clip}%
\pgfsetbuttcap%
\pgfsetmiterjoin%
\definecolor{currentfill}{rgb}{0.133298,0.375282,0.379395}%
\pgfsetfillcolor{currentfill}%
\pgfsetlinewidth{0.000000pt}%
\definecolor{currentstroke}{rgb}{0.000000,0.000000,0.000000}%
\pgfsetstrokecolor{currentstroke}%
\pgfsetstrokeopacity{0.000000}%
\pgfsetdash{}{0pt}%
\pgfpathmoveto{\pgfqpoint{1.392111in}{1.613090in}}%
\pgfpathlineto{\pgfqpoint{1.400864in}{1.613090in}}%
\pgfpathlineto{\pgfqpoint{1.400864in}{1.949455in}}%
\pgfpathlineto{\pgfqpoint{1.392111in}{1.949455in}}%
\pgfpathlineto{\pgfqpoint{1.392111in}{1.613090in}}%
\pgfpathclose%
\pgfusepath{fill}%
\end{pgfscope}%
\begin{pgfscope}%
\pgfpathrectangle{\pgfqpoint{0.804646in}{0.600000in}}{\pgfqpoint{2.573292in}{2.070576in}}%
\pgfusepath{clip}%
\pgfsetbuttcap%
\pgfsetmiterjoin%
\definecolor{currentfill}{rgb}{0.133298,0.375282,0.379395}%
\pgfsetfillcolor{currentfill}%
\pgfsetlinewidth{0.000000pt}%
\definecolor{currentstroke}{rgb}{0.000000,0.000000,0.000000}%
\pgfsetstrokecolor{currentstroke}%
\pgfsetstrokeopacity{0.000000}%
\pgfsetdash{}{0pt}%
\pgfpathmoveto{\pgfqpoint{1.403053in}{1.613090in}}%
\pgfpathlineto{\pgfqpoint{1.411806in}{1.613090in}}%
\pgfpathlineto{\pgfqpoint{1.411806in}{1.998947in}}%
\pgfpathlineto{\pgfqpoint{1.403053in}{1.998947in}}%
\pgfpathlineto{\pgfqpoint{1.403053in}{1.613090in}}%
\pgfpathclose%
\pgfusepath{fill}%
\end{pgfscope}%
\begin{pgfscope}%
\pgfpathrectangle{\pgfqpoint{0.804646in}{0.600000in}}{\pgfqpoint{2.573292in}{2.070576in}}%
\pgfusepath{clip}%
\pgfsetbuttcap%
\pgfsetmiterjoin%
\definecolor{currentfill}{rgb}{0.133298,0.375282,0.379395}%
\pgfsetfillcolor{currentfill}%
\pgfsetlinewidth{0.000000pt}%
\definecolor{currentstroke}{rgb}{0.000000,0.000000,0.000000}%
\pgfsetstrokecolor{currentstroke}%
\pgfsetstrokeopacity{0.000000}%
\pgfsetdash{}{0pt}%
\pgfpathmoveto{\pgfqpoint{1.413995in}{1.613090in}}%
\pgfpathlineto{\pgfqpoint{1.422748in}{1.613090in}}%
\pgfpathlineto{\pgfqpoint{1.422748in}{1.883528in}}%
\pgfpathlineto{\pgfqpoint{1.413995in}{1.883528in}}%
\pgfpathlineto{\pgfqpoint{1.413995in}{1.613090in}}%
\pgfpathclose%
\pgfusepath{fill}%
\end{pgfscope}%
\begin{pgfscope}%
\pgfpathrectangle{\pgfqpoint{0.804646in}{0.600000in}}{\pgfqpoint{2.573292in}{2.070576in}}%
\pgfusepath{clip}%
\pgfsetbuttcap%
\pgfsetmiterjoin%
\definecolor{currentfill}{rgb}{0.133298,0.375282,0.379395}%
\pgfsetfillcolor{currentfill}%
\pgfsetlinewidth{0.000000pt}%
\definecolor{currentstroke}{rgb}{0.000000,0.000000,0.000000}%
\pgfsetstrokecolor{currentstroke}%
\pgfsetstrokeopacity{0.000000}%
\pgfsetdash{}{0pt}%
\pgfpathmoveto{\pgfqpoint{1.424936in}{1.613090in}}%
\pgfpathlineto{\pgfqpoint{1.433690in}{1.613090in}}%
\pgfpathlineto{\pgfqpoint{1.433690in}{1.979430in}}%
\pgfpathlineto{\pgfqpoint{1.424936in}{1.979430in}}%
\pgfpathlineto{\pgfqpoint{1.424936in}{1.613090in}}%
\pgfpathclose%
\pgfusepath{fill}%
\end{pgfscope}%
\begin{pgfscope}%
\pgfpathrectangle{\pgfqpoint{0.804646in}{0.600000in}}{\pgfqpoint{2.573292in}{2.070576in}}%
\pgfusepath{clip}%
\pgfsetbuttcap%
\pgfsetmiterjoin%
\definecolor{currentfill}{rgb}{0.133298,0.375282,0.379395}%
\pgfsetfillcolor{currentfill}%
\pgfsetlinewidth{0.000000pt}%
\definecolor{currentstroke}{rgb}{0.000000,0.000000,0.000000}%
\pgfsetstrokecolor{currentstroke}%
\pgfsetstrokeopacity{0.000000}%
\pgfsetdash{}{0pt}%
\pgfpathmoveto{\pgfqpoint{1.435878in}{1.613090in}}%
\pgfpathlineto{\pgfqpoint{1.444632in}{1.613090in}}%
\pgfpathlineto{\pgfqpoint{1.444632in}{1.979630in}}%
\pgfpathlineto{\pgfqpoint{1.435878in}{1.979630in}}%
\pgfpathlineto{\pgfqpoint{1.435878in}{1.613090in}}%
\pgfpathclose%
\pgfusepath{fill}%
\end{pgfscope}%
\begin{pgfscope}%
\pgfpathrectangle{\pgfqpoint{0.804646in}{0.600000in}}{\pgfqpoint{2.573292in}{2.070576in}}%
\pgfusepath{clip}%
\pgfsetbuttcap%
\pgfsetmiterjoin%
\definecolor{currentfill}{rgb}{0.133298,0.375282,0.379395}%
\pgfsetfillcolor{currentfill}%
\pgfsetlinewidth{0.000000pt}%
\definecolor{currentstroke}{rgb}{0.000000,0.000000,0.000000}%
\pgfsetstrokecolor{currentstroke}%
\pgfsetstrokeopacity{0.000000}%
\pgfsetdash{}{0pt}%
\pgfpathmoveto{\pgfqpoint{1.446820in}{1.613090in}}%
\pgfpathlineto{\pgfqpoint{1.455573in}{1.613090in}}%
\pgfpathlineto{\pgfqpoint{1.455573in}{1.991622in}}%
\pgfpathlineto{\pgfqpoint{1.446820in}{1.991622in}}%
\pgfpathlineto{\pgfqpoint{1.446820in}{1.613090in}}%
\pgfpathclose%
\pgfusepath{fill}%
\end{pgfscope}%
\begin{pgfscope}%
\pgfpathrectangle{\pgfqpoint{0.804646in}{0.600000in}}{\pgfqpoint{2.573292in}{2.070576in}}%
\pgfusepath{clip}%
\pgfsetbuttcap%
\pgfsetmiterjoin%
\definecolor{currentfill}{rgb}{0.133298,0.375282,0.379395}%
\pgfsetfillcolor{currentfill}%
\pgfsetlinewidth{0.000000pt}%
\definecolor{currentstroke}{rgb}{0.000000,0.000000,0.000000}%
\pgfsetstrokecolor{currentstroke}%
\pgfsetstrokeopacity{0.000000}%
\pgfsetdash{}{0pt}%
\pgfpathmoveto{\pgfqpoint{1.457762in}{1.630865in}}%
\pgfpathlineto{\pgfqpoint{1.466515in}{1.630865in}}%
\pgfpathlineto{\pgfqpoint{1.466515in}{1.986485in}}%
\pgfpathlineto{\pgfqpoint{1.457762in}{1.986485in}}%
\pgfpathlineto{\pgfqpoint{1.457762in}{1.630865in}}%
\pgfpathclose%
\pgfusepath{fill}%
\end{pgfscope}%
\begin{pgfscope}%
\pgfpathrectangle{\pgfqpoint{0.804646in}{0.600000in}}{\pgfqpoint{2.573292in}{2.070576in}}%
\pgfusepath{clip}%
\pgfsetbuttcap%
\pgfsetmiterjoin%
\definecolor{currentfill}{rgb}{0.133298,0.375282,0.379395}%
\pgfsetfillcolor{currentfill}%
\pgfsetlinewidth{0.000000pt}%
\definecolor{currentstroke}{rgb}{0.000000,0.000000,0.000000}%
\pgfsetstrokecolor{currentstroke}%
\pgfsetstrokeopacity{0.000000}%
\pgfsetdash{}{0pt}%
\pgfpathmoveto{\pgfqpoint{1.468704in}{1.654393in}}%
\pgfpathlineto{\pgfqpoint{1.477457in}{1.654393in}}%
\pgfpathlineto{\pgfqpoint{1.477457in}{2.021886in}}%
\pgfpathlineto{\pgfqpoint{1.468704in}{2.021886in}}%
\pgfpathlineto{\pgfqpoint{1.468704in}{1.654393in}}%
\pgfpathclose%
\pgfusepath{fill}%
\end{pgfscope}%
\begin{pgfscope}%
\pgfpathrectangle{\pgfqpoint{0.804646in}{0.600000in}}{\pgfqpoint{2.573292in}{2.070576in}}%
\pgfusepath{clip}%
\pgfsetbuttcap%
\pgfsetmiterjoin%
\definecolor{currentfill}{rgb}{0.133298,0.375282,0.379395}%
\pgfsetfillcolor{currentfill}%
\pgfsetlinewidth{0.000000pt}%
\definecolor{currentstroke}{rgb}{0.000000,0.000000,0.000000}%
\pgfsetstrokecolor{currentstroke}%
\pgfsetstrokeopacity{0.000000}%
\pgfsetdash{}{0pt}%
\pgfpathmoveto{\pgfqpoint{1.479645in}{1.652966in}}%
\pgfpathlineto{\pgfqpoint{1.488399in}{1.652966in}}%
\pgfpathlineto{\pgfqpoint{1.488399in}{1.861769in}}%
\pgfpathlineto{\pgfqpoint{1.479645in}{1.861769in}}%
\pgfpathlineto{\pgfqpoint{1.479645in}{1.652966in}}%
\pgfpathclose%
\pgfusepath{fill}%
\end{pgfscope}%
\begin{pgfscope}%
\pgfpathrectangle{\pgfqpoint{0.804646in}{0.600000in}}{\pgfqpoint{2.573292in}{2.070576in}}%
\pgfusepath{clip}%
\pgfsetbuttcap%
\pgfsetmiterjoin%
\definecolor{currentfill}{rgb}{0.133298,0.375282,0.379395}%
\pgfsetfillcolor{currentfill}%
\pgfsetlinewidth{0.000000pt}%
\definecolor{currentstroke}{rgb}{0.000000,0.000000,0.000000}%
\pgfsetstrokecolor{currentstroke}%
\pgfsetstrokeopacity{0.000000}%
\pgfsetdash{}{0pt}%
\pgfpathmoveto{\pgfqpoint{1.490587in}{1.658262in}}%
\pgfpathlineto{\pgfqpoint{1.499341in}{1.658262in}}%
\pgfpathlineto{\pgfqpoint{1.499341in}{1.918839in}}%
\pgfpathlineto{\pgfqpoint{1.490587in}{1.918839in}}%
\pgfpathlineto{\pgfqpoint{1.490587in}{1.658262in}}%
\pgfpathclose%
\pgfusepath{fill}%
\end{pgfscope}%
\begin{pgfscope}%
\pgfpathrectangle{\pgfqpoint{0.804646in}{0.600000in}}{\pgfqpoint{2.573292in}{2.070576in}}%
\pgfusepath{clip}%
\pgfsetbuttcap%
\pgfsetmiterjoin%
\definecolor{currentfill}{rgb}{0.133298,0.375282,0.379395}%
\pgfsetfillcolor{currentfill}%
\pgfsetlinewidth{0.000000pt}%
\definecolor{currentstroke}{rgb}{0.000000,0.000000,0.000000}%
\pgfsetstrokecolor{currentstroke}%
\pgfsetstrokeopacity{0.000000}%
\pgfsetdash{}{0pt}%
\pgfpathmoveto{\pgfqpoint{1.501529in}{1.695371in}}%
\pgfpathlineto{\pgfqpoint{1.510282in}{1.695371in}}%
\pgfpathlineto{\pgfqpoint{1.510282in}{1.939074in}}%
\pgfpathlineto{\pgfqpoint{1.501529in}{1.939074in}}%
\pgfpathlineto{\pgfqpoint{1.501529in}{1.695371in}}%
\pgfpathclose%
\pgfusepath{fill}%
\end{pgfscope}%
\begin{pgfscope}%
\pgfpathrectangle{\pgfqpoint{0.804646in}{0.600000in}}{\pgfqpoint{2.573292in}{2.070576in}}%
\pgfusepath{clip}%
\pgfsetbuttcap%
\pgfsetmiterjoin%
\definecolor{currentfill}{rgb}{0.133298,0.375282,0.379395}%
\pgfsetfillcolor{currentfill}%
\pgfsetlinewidth{0.000000pt}%
\definecolor{currentstroke}{rgb}{0.000000,0.000000,0.000000}%
\pgfsetstrokecolor{currentstroke}%
\pgfsetstrokeopacity{0.000000}%
\pgfsetdash{}{0pt}%
\pgfpathmoveto{\pgfqpoint{1.512471in}{1.686490in}}%
\pgfpathlineto{\pgfqpoint{1.521224in}{1.686490in}}%
\pgfpathlineto{\pgfqpoint{1.521224in}{1.875335in}}%
\pgfpathlineto{\pgfqpoint{1.512471in}{1.875335in}}%
\pgfpathlineto{\pgfqpoint{1.512471in}{1.686490in}}%
\pgfpathclose%
\pgfusepath{fill}%
\end{pgfscope}%
\begin{pgfscope}%
\pgfpathrectangle{\pgfqpoint{0.804646in}{0.600000in}}{\pgfqpoint{2.573292in}{2.070576in}}%
\pgfusepath{clip}%
\pgfsetbuttcap%
\pgfsetmiterjoin%
\definecolor{currentfill}{rgb}{0.133298,0.375282,0.379395}%
\pgfsetfillcolor{currentfill}%
\pgfsetlinewidth{0.000000pt}%
\definecolor{currentstroke}{rgb}{0.000000,0.000000,0.000000}%
\pgfsetstrokecolor{currentstroke}%
\pgfsetstrokeopacity{0.000000}%
\pgfsetdash{}{0pt}%
\pgfpathmoveto{\pgfqpoint{1.523413in}{1.637552in}}%
\pgfpathlineto{\pgfqpoint{1.532166in}{1.637552in}}%
\pgfpathlineto{\pgfqpoint{1.532166in}{1.814166in}}%
\pgfpathlineto{\pgfqpoint{1.523413in}{1.814166in}}%
\pgfpathlineto{\pgfqpoint{1.523413in}{1.637552in}}%
\pgfpathclose%
\pgfusepath{fill}%
\end{pgfscope}%
\begin{pgfscope}%
\pgfpathrectangle{\pgfqpoint{0.804646in}{0.600000in}}{\pgfqpoint{2.573292in}{2.070576in}}%
\pgfusepath{clip}%
\pgfsetbuttcap%
\pgfsetmiterjoin%
\definecolor{currentfill}{rgb}{0.133298,0.375282,0.379395}%
\pgfsetfillcolor{currentfill}%
\pgfsetlinewidth{0.000000pt}%
\definecolor{currentstroke}{rgb}{0.000000,0.000000,0.000000}%
\pgfsetstrokecolor{currentstroke}%
\pgfsetstrokeopacity{0.000000}%
\pgfsetdash{}{0pt}%
\pgfpathmoveto{\pgfqpoint{1.534354in}{1.613090in}}%
\pgfpathlineto{\pgfqpoint{1.543108in}{1.613090in}}%
\pgfpathlineto{\pgfqpoint{1.543108in}{1.727465in}}%
\pgfpathlineto{\pgfqpoint{1.534354in}{1.727465in}}%
\pgfpathlineto{\pgfqpoint{1.534354in}{1.613090in}}%
\pgfpathclose%
\pgfusepath{fill}%
\end{pgfscope}%
\begin{pgfscope}%
\pgfpathrectangle{\pgfqpoint{0.804646in}{0.600000in}}{\pgfqpoint{2.573292in}{2.070576in}}%
\pgfusepath{clip}%
\pgfsetbuttcap%
\pgfsetmiterjoin%
\definecolor{currentfill}{rgb}{0.133298,0.375282,0.379395}%
\pgfsetfillcolor{currentfill}%
\pgfsetlinewidth{0.000000pt}%
\definecolor{currentstroke}{rgb}{0.000000,0.000000,0.000000}%
\pgfsetstrokecolor{currentstroke}%
\pgfsetstrokeopacity{0.000000}%
\pgfsetdash{}{0pt}%
\pgfpathmoveto{\pgfqpoint{1.545296in}{1.613149in}}%
\pgfpathlineto{\pgfqpoint{1.554050in}{1.613149in}}%
\pgfpathlineto{\pgfqpoint{1.554050in}{1.671423in}}%
\pgfpathlineto{\pgfqpoint{1.545296in}{1.671423in}}%
\pgfpathlineto{\pgfqpoint{1.545296in}{1.613149in}}%
\pgfpathclose%
\pgfusepath{fill}%
\end{pgfscope}%
\begin{pgfscope}%
\pgfpathrectangle{\pgfqpoint{0.804646in}{0.600000in}}{\pgfqpoint{2.573292in}{2.070576in}}%
\pgfusepath{clip}%
\pgfsetbuttcap%
\pgfsetmiterjoin%
\definecolor{currentfill}{rgb}{0.133298,0.375282,0.379395}%
\pgfsetfillcolor{currentfill}%
\pgfsetlinewidth{0.000000pt}%
\definecolor{currentstroke}{rgb}{0.000000,0.000000,0.000000}%
\pgfsetstrokecolor{currentstroke}%
\pgfsetstrokeopacity{0.000000}%
\pgfsetdash{}{0pt}%
\pgfpathmoveto{\pgfqpoint{1.556238in}{1.643797in}}%
\pgfpathlineto{\pgfqpoint{1.564991in}{1.643797in}}%
\pgfpathlineto{\pgfqpoint{1.564991in}{1.692111in}}%
\pgfpathlineto{\pgfqpoint{1.556238in}{1.692111in}}%
\pgfpathlineto{\pgfqpoint{1.556238in}{1.643797in}}%
\pgfpathclose%
\pgfusepath{fill}%
\end{pgfscope}%
\begin{pgfscope}%
\pgfpathrectangle{\pgfqpoint{0.804646in}{0.600000in}}{\pgfqpoint{2.573292in}{2.070576in}}%
\pgfusepath{clip}%
\pgfsetbuttcap%
\pgfsetmiterjoin%
\definecolor{currentfill}{rgb}{0.133298,0.375282,0.379395}%
\pgfsetfillcolor{currentfill}%
\pgfsetlinewidth{0.000000pt}%
\definecolor{currentstroke}{rgb}{0.000000,0.000000,0.000000}%
\pgfsetstrokecolor{currentstroke}%
\pgfsetstrokeopacity{0.000000}%
\pgfsetdash{}{0pt}%
\pgfpathmoveto{\pgfqpoint{1.567180in}{1.638389in}}%
\pgfpathlineto{\pgfqpoint{1.575933in}{1.638389in}}%
\pgfpathlineto{\pgfqpoint{1.575933in}{1.725026in}}%
\pgfpathlineto{\pgfqpoint{1.567180in}{1.725026in}}%
\pgfpathlineto{\pgfqpoint{1.567180in}{1.638389in}}%
\pgfpathclose%
\pgfusepath{fill}%
\end{pgfscope}%
\begin{pgfscope}%
\pgfpathrectangle{\pgfqpoint{0.804646in}{0.600000in}}{\pgfqpoint{2.573292in}{2.070576in}}%
\pgfusepath{clip}%
\pgfsetbuttcap%
\pgfsetmiterjoin%
\definecolor{currentfill}{rgb}{0.133298,0.375282,0.379395}%
\pgfsetfillcolor{currentfill}%
\pgfsetlinewidth{0.000000pt}%
\definecolor{currentstroke}{rgb}{0.000000,0.000000,0.000000}%
\pgfsetstrokecolor{currentstroke}%
\pgfsetstrokeopacity{0.000000}%
\pgfsetdash{}{0pt}%
\pgfpathmoveto{\pgfqpoint{1.578122in}{1.640942in}}%
\pgfpathlineto{\pgfqpoint{1.586875in}{1.640942in}}%
\pgfpathlineto{\pgfqpoint{1.586875in}{1.684702in}}%
\pgfpathlineto{\pgfqpoint{1.578122in}{1.684702in}}%
\pgfpathlineto{\pgfqpoint{1.578122in}{1.640942in}}%
\pgfpathclose%
\pgfusepath{fill}%
\end{pgfscope}%
\begin{pgfscope}%
\pgfpathrectangle{\pgfqpoint{0.804646in}{0.600000in}}{\pgfqpoint{2.573292in}{2.070576in}}%
\pgfusepath{clip}%
\pgfsetbuttcap%
\pgfsetmiterjoin%
\definecolor{currentfill}{rgb}{0.133298,0.375282,0.379395}%
\pgfsetfillcolor{currentfill}%
\pgfsetlinewidth{0.000000pt}%
\definecolor{currentstroke}{rgb}{0.000000,0.000000,0.000000}%
\pgfsetstrokecolor{currentstroke}%
\pgfsetstrokeopacity{0.000000}%
\pgfsetdash{}{0pt}%
\pgfpathmoveto{\pgfqpoint{1.589063in}{1.613090in}}%
\pgfpathlineto{\pgfqpoint{1.597817in}{1.613090in}}%
\pgfpathlineto{\pgfqpoint{1.597817in}{1.652185in}}%
\pgfpathlineto{\pgfqpoint{1.589063in}{1.652185in}}%
\pgfpathlineto{\pgfqpoint{1.589063in}{1.613090in}}%
\pgfpathclose%
\pgfusepath{fill}%
\end{pgfscope}%
\begin{pgfscope}%
\pgfpathrectangle{\pgfqpoint{0.804646in}{0.600000in}}{\pgfqpoint{2.573292in}{2.070576in}}%
\pgfusepath{clip}%
\pgfsetbuttcap%
\pgfsetmiterjoin%
\definecolor{currentfill}{rgb}{0.133298,0.375282,0.379395}%
\pgfsetfillcolor{currentfill}%
\pgfsetlinewidth{0.000000pt}%
\definecolor{currentstroke}{rgb}{0.000000,0.000000,0.000000}%
\pgfsetstrokecolor{currentstroke}%
\pgfsetstrokeopacity{0.000000}%
\pgfsetdash{}{0pt}%
\pgfpathmoveto{\pgfqpoint{1.600005in}{1.613090in}}%
\pgfpathlineto{\pgfqpoint{1.608759in}{1.613090in}}%
\pgfpathlineto{\pgfqpoint{1.608759in}{1.657650in}}%
\pgfpathlineto{\pgfqpoint{1.600005in}{1.657650in}}%
\pgfpathlineto{\pgfqpoint{1.600005in}{1.613090in}}%
\pgfpathclose%
\pgfusepath{fill}%
\end{pgfscope}%
\begin{pgfscope}%
\pgfpathrectangle{\pgfqpoint{0.804646in}{0.600000in}}{\pgfqpoint{2.573292in}{2.070576in}}%
\pgfusepath{clip}%
\pgfsetbuttcap%
\pgfsetmiterjoin%
\definecolor{currentfill}{rgb}{0.133298,0.375282,0.379395}%
\pgfsetfillcolor{currentfill}%
\pgfsetlinewidth{0.000000pt}%
\definecolor{currentstroke}{rgb}{0.000000,0.000000,0.000000}%
\pgfsetstrokecolor{currentstroke}%
\pgfsetstrokeopacity{0.000000}%
\pgfsetdash{}{0pt}%
\pgfpathmoveto{\pgfqpoint{1.610947in}{1.580351in}}%
\pgfpathlineto{\pgfqpoint{1.619700in}{1.580351in}}%
\pgfpathlineto{\pgfqpoint{1.619700in}{1.539082in}}%
\pgfpathlineto{\pgfqpoint{1.610947in}{1.539082in}}%
\pgfpathlineto{\pgfqpoint{1.610947in}{1.580351in}}%
\pgfpathclose%
\pgfusepath{fill}%
\end{pgfscope}%
\begin{pgfscope}%
\pgfpathrectangle{\pgfqpoint{0.804646in}{0.600000in}}{\pgfqpoint{2.573292in}{2.070576in}}%
\pgfusepath{clip}%
\pgfsetbuttcap%
\pgfsetmiterjoin%
\definecolor{currentfill}{rgb}{0.133298,0.375282,0.379395}%
\pgfsetfillcolor{currentfill}%
\pgfsetlinewidth{0.000000pt}%
\definecolor{currentstroke}{rgb}{0.000000,0.000000,0.000000}%
\pgfsetstrokecolor{currentstroke}%
\pgfsetstrokeopacity{0.000000}%
\pgfsetdash{}{0pt}%
\pgfpathmoveto{\pgfqpoint{1.621889in}{1.535069in}}%
\pgfpathlineto{\pgfqpoint{1.630642in}{1.535069in}}%
\pgfpathlineto{\pgfqpoint{1.630642in}{1.493240in}}%
\pgfpathlineto{\pgfqpoint{1.621889in}{1.493240in}}%
\pgfpathlineto{\pgfqpoint{1.621889in}{1.535069in}}%
\pgfpathclose%
\pgfusepath{fill}%
\end{pgfscope}%
\begin{pgfscope}%
\pgfpathrectangle{\pgfqpoint{0.804646in}{0.600000in}}{\pgfqpoint{2.573292in}{2.070576in}}%
\pgfusepath{clip}%
\pgfsetbuttcap%
\pgfsetmiterjoin%
\definecolor{currentfill}{rgb}{0.133298,0.375282,0.379395}%
\pgfsetfillcolor{currentfill}%
\pgfsetlinewidth{0.000000pt}%
\definecolor{currentstroke}{rgb}{0.000000,0.000000,0.000000}%
\pgfsetstrokecolor{currentstroke}%
\pgfsetstrokeopacity{0.000000}%
\pgfsetdash{}{0pt}%
\pgfpathmoveto{\pgfqpoint{1.632831in}{1.515545in}}%
\pgfpathlineto{\pgfqpoint{1.641584in}{1.515545in}}%
\pgfpathlineto{\pgfqpoint{1.641584in}{1.417043in}}%
\pgfpathlineto{\pgfqpoint{1.632831in}{1.417043in}}%
\pgfpathlineto{\pgfqpoint{1.632831in}{1.515545in}}%
\pgfpathclose%
\pgfusepath{fill}%
\end{pgfscope}%
\begin{pgfscope}%
\pgfpathrectangle{\pgfqpoint{0.804646in}{0.600000in}}{\pgfqpoint{2.573292in}{2.070576in}}%
\pgfusepath{clip}%
\pgfsetbuttcap%
\pgfsetmiterjoin%
\definecolor{currentfill}{rgb}{0.133298,0.375282,0.379395}%
\pgfsetfillcolor{currentfill}%
\pgfsetlinewidth{0.000000pt}%
\definecolor{currentstroke}{rgb}{0.000000,0.000000,0.000000}%
\pgfsetstrokecolor{currentstroke}%
\pgfsetstrokeopacity{0.000000}%
\pgfsetdash{}{0pt}%
\pgfpathmoveto{\pgfqpoint{1.643772in}{1.513681in}}%
\pgfpathlineto{\pgfqpoint{1.652526in}{1.513681in}}%
\pgfpathlineto{\pgfqpoint{1.652526in}{1.459636in}}%
\pgfpathlineto{\pgfqpoint{1.643772in}{1.459636in}}%
\pgfpathlineto{\pgfqpoint{1.643772in}{1.513681in}}%
\pgfpathclose%
\pgfusepath{fill}%
\end{pgfscope}%
\begin{pgfscope}%
\pgfpathrectangle{\pgfqpoint{0.804646in}{0.600000in}}{\pgfqpoint{2.573292in}{2.070576in}}%
\pgfusepath{clip}%
\pgfsetbuttcap%
\pgfsetmiterjoin%
\definecolor{currentfill}{rgb}{0.133298,0.375282,0.379395}%
\pgfsetfillcolor{currentfill}%
\pgfsetlinewidth{0.000000pt}%
\definecolor{currentstroke}{rgb}{0.000000,0.000000,0.000000}%
\pgfsetstrokecolor{currentstroke}%
\pgfsetstrokeopacity{0.000000}%
\pgfsetdash{}{0pt}%
\pgfpathmoveto{\pgfqpoint{1.654714in}{1.533359in}}%
\pgfpathlineto{\pgfqpoint{1.663468in}{1.533359in}}%
\pgfpathlineto{\pgfqpoint{1.663468in}{1.455460in}}%
\pgfpathlineto{\pgfqpoint{1.654714in}{1.455460in}}%
\pgfpathlineto{\pgfqpoint{1.654714in}{1.533359in}}%
\pgfpathclose%
\pgfusepath{fill}%
\end{pgfscope}%
\begin{pgfscope}%
\pgfpathrectangle{\pgfqpoint{0.804646in}{0.600000in}}{\pgfqpoint{2.573292in}{2.070576in}}%
\pgfusepath{clip}%
\pgfsetbuttcap%
\pgfsetmiterjoin%
\definecolor{currentfill}{rgb}{0.133298,0.375282,0.379395}%
\pgfsetfillcolor{currentfill}%
\pgfsetlinewidth{0.000000pt}%
\definecolor{currentstroke}{rgb}{0.000000,0.000000,0.000000}%
\pgfsetstrokecolor{currentstroke}%
\pgfsetstrokeopacity{0.000000}%
\pgfsetdash{}{0pt}%
\pgfpathmoveto{\pgfqpoint{1.665656in}{1.571384in}}%
\pgfpathlineto{\pgfqpoint{1.674409in}{1.571384in}}%
\pgfpathlineto{\pgfqpoint{1.674409in}{1.471958in}}%
\pgfpathlineto{\pgfqpoint{1.665656in}{1.471958in}}%
\pgfpathlineto{\pgfqpoint{1.665656in}{1.571384in}}%
\pgfpathclose%
\pgfusepath{fill}%
\end{pgfscope}%
\begin{pgfscope}%
\pgfpathrectangle{\pgfqpoint{0.804646in}{0.600000in}}{\pgfqpoint{2.573292in}{2.070576in}}%
\pgfusepath{clip}%
\pgfsetbuttcap%
\pgfsetmiterjoin%
\definecolor{currentfill}{rgb}{0.133298,0.375282,0.379395}%
\pgfsetfillcolor{currentfill}%
\pgfsetlinewidth{0.000000pt}%
\definecolor{currentstroke}{rgb}{0.000000,0.000000,0.000000}%
\pgfsetstrokecolor{currentstroke}%
\pgfsetstrokeopacity{0.000000}%
\pgfsetdash{}{0pt}%
\pgfpathmoveto{\pgfqpoint{1.676598in}{1.608655in}}%
\pgfpathlineto{\pgfqpoint{1.685351in}{1.608655in}}%
\pgfpathlineto{\pgfqpoint{1.685351in}{1.549999in}}%
\pgfpathlineto{\pgfqpoint{1.676598in}{1.549999in}}%
\pgfpathlineto{\pgfqpoint{1.676598in}{1.608655in}}%
\pgfpathclose%
\pgfusepath{fill}%
\end{pgfscope}%
\begin{pgfscope}%
\pgfpathrectangle{\pgfqpoint{0.804646in}{0.600000in}}{\pgfqpoint{2.573292in}{2.070576in}}%
\pgfusepath{clip}%
\pgfsetbuttcap%
\pgfsetmiterjoin%
\definecolor{currentfill}{rgb}{0.133298,0.375282,0.379395}%
\pgfsetfillcolor{currentfill}%
\pgfsetlinewidth{0.000000pt}%
\definecolor{currentstroke}{rgb}{0.000000,0.000000,0.000000}%
\pgfsetstrokecolor{currentstroke}%
\pgfsetstrokeopacity{0.000000}%
\pgfsetdash{}{0pt}%
\pgfpathmoveto{\pgfqpoint{1.687540in}{1.613090in}}%
\pgfpathlineto{\pgfqpoint{1.696293in}{1.613090in}}%
\pgfpathlineto{\pgfqpoint{1.696293in}{1.497530in}}%
\pgfpathlineto{\pgfqpoint{1.687540in}{1.497530in}}%
\pgfpathlineto{\pgfqpoint{1.687540in}{1.613090in}}%
\pgfpathclose%
\pgfusepath{fill}%
\end{pgfscope}%
\begin{pgfscope}%
\pgfpathrectangle{\pgfqpoint{0.804646in}{0.600000in}}{\pgfqpoint{2.573292in}{2.070576in}}%
\pgfusepath{clip}%
\pgfsetbuttcap%
\pgfsetmiterjoin%
\definecolor{currentfill}{rgb}{0.133298,0.375282,0.379395}%
\pgfsetfillcolor{currentfill}%
\pgfsetlinewidth{0.000000pt}%
\definecolor{currentstroke}{rgb}{0.000000,0.000000,0.000000}%
\pgfsetstrokecolor{currentstroke}%
\pgfsetstrokeopacity{0.000000}%
\pgfsetdash{}{0pt}%
\pgfpathmoveto{\pgfqpoint{1.698481in}{1.613090in}}%
\pgfpathlineto{\pgfqpoint{1.707235in}{1.613090in}}%
\pgfpathlineto{\pgfqpoint{1.707235in}{1.496495in}}%
\pgfpathlineto{\pgfqpoint{1.698481in}{1.496495in}}%
\pgfpathlineto{\pgfqpoint{1.698481in}{1.613090in}}%
\pgfpathclose%
\pgfusepath{fill}%
\end{pgfscope}%
\begin{pgfscope}%
\pgfpathrectangle{\pgfqpoint{0.804646in}{0.600000in}}{\pgfqpoint{2.573292in}{2.070576in}}%
\pgfusepath{clip}%
\pgfsetbuttcap%
\pgfsetmiterjoin%
\definecolor{currentfill}{rgb}{0.133298,0.375282,0.379395}%
\pgfsetfillcolor{currentfill}%
\pgfsetlinewidth{0.000000pt}%
\definecolor{currentstroke}{rgb}{0.000000,0.000000,0.000000}%
\pgfsetstrokecolor{currentstroke}%
\pgfsetstrokeopacity{0.000000}%
\pgfsetdash{}{0pt}%
\pgfpathmoveto{\pgfqpoint{1.709423in}{1.613090in}}%
\pgfpathlineto{\pgfqpoint{1.718177in}{1.613090in}}%
\pgfpathlineto{\pgfqpoint{1.718177in}{1.463842in}}%
\pgfpathlineto{\pgfqpoint{1.709423in}{1.463842in}}%
\pgfpathlineto{\pgfqpoint{1.709423in}{1.613090in}}%
\pgfpathclose%
\pgfusepath{fill}%
\end{pgfscope}%
\begin{pgfscope}%
\pgfpathrectangle{\pgfqpoint{0.804646in}{0.600000in}}{\pgfqpoint{2.573292in}{2.070576in}}%
\pgfusepath{clip}%
\pgfsetbuttcap%
\pgfsetmiterjoin%
\definecolor{currentfill}{rgb}{0.133298,0.375282,0.379395}%
\pgfsetfillcolor{currentfill}%
\pgfsetlinewidth{0.000000pt}%
\definecolor{currentstroke}{rgb}{0.000000,0.000000,0.000000}%
\pgfsetstrokecolor{currentstroke}%
\pgfsetstrokeopacity{0.000000}%
\pgfsetdash{}{0pt}%
\pgfpathmoveto{\pgfqpoint{1.720365in}{1.613090in}}%
\pgfpathlineto{\pgfqpoint{1.729118in}{1.613090in}}%
\pgfpathlineto{\pgfqpoint{1.729118in}{1.419587in}}%
\pgfpathlineto{\pgfqpoint{1.720365in}{1.419587in}}%
\pgfpathlineto{\pgfqpoint{1.720365in}{1.613090in}}%
\pgfpathclose%
\pgfusepath{fill}%
\end{pgfscope}%
\begin{pgfscope}%
\pgfpathrectangle{\pgfqpoint{0.804646in}{0.600000in}}{\pgfqpoint{2.573292in}{2.070576in}}%
\pgfusepath{clip}%
\pgfsetbuttcap%
\pgfsetmiterjoin%
\definecolor{currentfill}{rgb}{0.133298,0.375282,0.379395}%
\pgfsetfillcolor{currentfill}%
\pgfsetlinewidth{0.000000pt}%
\definecolor{currentstroke}{rgb}{0.000000,0.000000,0.000000}%
\pgfsetstrokecolor{currentstroke}%
\pgfsetstrokeopacity{0.000000}%
\pgfsetdash{}{0pt}%
\pgfpathmoveto{\pgfqpoint{1.731307in}{1.613090in}}%
\pgfpathlineto{\pgfqpoint{1.740060in}{1.613090in}}%
\pgfpathlineto{\pgfqpoint{1.740060in}{1.363792in}}%
\pgfpathlineto{\pgfqpoint{1.731307in}{1.363792in}}%
\pgfpathlineto{\pgfqpoint{1.731307in}{1.613090in}}%
\pgfpathclose%
\pgfusepath{fill}%
\end{pgfscope}%
\begin{pgfscope}%
\pgfpathrectangle{\pgfqpoint{0.804646in}{0.600000in}}{\pgfqpoint{2.573292in}{2.070576in}}%
\pgfusepath{clip}%
\pgfsetbuttcap%
\pgfsetmiterjoin%
\definecolor{currentfill}{rgb}{0.133298,0.375282,0.379395}%
\pgfsetfillcolor{currentfill}%
\pgfsetlinewidth{0.000000pt}%
\definecolor{currentstroke}{rgb}{0.000000,0.000000,0.000000}%
\pgfsetstrokecolor{currentstroke}%
\pgfsetstrokeopacity{0.000000}%
\pgfsetdash{}{0pt}%
\pgfpathmoveto{\pgfqpoint{1.742249in}{1.613090in}}%
\pgfpathlineto{\pgfqpoint{1.751002in}{1.613090in}}%
\pgfpathlineto{\pgfqpoint{1.751002in}{1.393793in}}%
\pgfpathlineto{\pgfqpoint{1.742249in}{1.393793in}}%
\pgfpathlineto{\pgfqpoint{1.742249in}{1.613090in}}%
\pgfpathclose%
\pgfusepath{fill}%
\end{pgfscope}%
\begin{pgfscope}%
\pgfpathrectangle{\pgfqpoint{0.804646in}{0.600000in}}{\pgfqpoint{2.573292in}{2.070576in}}%
\pgfusepath{clip}%
\pgfsetbuttcap%
\pgfsetmiterjoin%
\definecolor{currentfill}{rgb}{0.133298,0.375282,0.379395}%
\pgfsetfillcolor{currentfill}%
\pgfsetlinewidth{0.000000pt}%
\definecolor{currentstroke}{rgb}{0.000000,0.000000,0.000000}%
\pgfsetstrokecolor{currentstroke}%
\pgfsetstrokeopacity{0.000000}%
\pgfsetdash{}{0pt}%
\pgfpathmoveto{\pgfqpoint{1.753190in}{1.613090in}}%
\pgfpathlineto{\pgfqpoint{1.761944in}{1.613090in}}%
\pgfpathlineto{\pgfqpoint{1.761944in}{1.367067in}}%
\pgfpathlineto{\pgfqpoint{1.753190in}{1.367067in}}%
\pgfpathlineto{\pgfqpoint{1.753190in}{1.613090in}}%
\pgfpathclose%
\pgfusepath{fill}%
\end{pgfscope}%
\begin{pgfscope}%
\pgfpathrectangle{\pgfqpoint{0.804646in}{0.600000in}}{\pgfqpoint{2.573292in}{2.070576in}}%
\pgfusepath{clip}%
\pgfsetbuttcap%
\pgfsetmiterjoin%
\definecolor{currentfill}{rgb}{0.133298,0.375282,0.379395}%
\pgfsetfillcolor{currentfill}%
\pgfsetlinewidth{0.000000pt}%
\definecolor{currentstroke}{rgb}{0.000000,0.000000,0.000000}%
\pgfsetstrokecolor{currentstroke}%
\pgfsetstrokeopacity{0.000000}%
\pgfsetdash{}{0pt}%
\pgfpathmoveto{\pgfqpoint{1.764132in}{1.613090in}}%
\pgfpathlineto{\pgfqpoint{1.772886in}{1.613090in}}%
\pgfpathlineto{\pgfqpoint{1.772886in}{1.343109in}}%
\pgfpathlineto{\pgfqpoint{1.764132in}{1.343109in}}%
\pgfpathlineto{\pgfqpoint{1.764132in}{1.613090in}}%
\pgfpathclose%
\pgfusepath{fill}%
\end{pgfscope}%
\begin{pgfscope}%
\pgfpathrectangle{\pgfqpoint{0.804646in}{0.600000in}}{\pgfqpoint{2.573292in}{2.070576in}}%
\pgfusepath{clip}%
\pgfsetbuttcap%
\pgfsetmiterjoin%
\definecolor{currentfill}{rgb}{0.133298,0.375282,0.379395}%
\pgfsetfillcolor{currentfill}%
\pgfsetlinewidth{0.000000pt}%
\definecolor{currentstroke}{rgb}{0.000000,0.000000,0.000000}%
\pgfsetstrokecolor{currentstroke}%
\pgfsetstrokeopacity{0.000000}%
\pgfsetdash{}{0pt}%
\pgfpathmoveto{\pgfqpoint{1.775074in}{1.613090in}}%
\pgfpathlineto{\pgfqpoint{1.783827in}{1.613090in}}%
\pgfpathlineto{\pgfqpoint{1.783827in}{1.359140in}}%
\pgfpathlineto{\pgfqpoint{1.775074in}{1.359140in}}%
\pgfpathlineto{\pgfqpoint{1.775074in}{1.613090in}}%
\pgfpathclose%
\pgfusepath{fill}%
\end{pgfscope}%
\begin{pgfscope}%
\pgfpathrectangle{\pgfqpoint{0.804646in}{0.600000in}}{\pgfqpoint{2.573292in}{2.070576in}}%
\pgfusepath{clip}%
\pgfsetbuttcap%
\pgfsetmiterjoin%
\definecolor{currentfill}{rgb}{0.133298,0.375282,0.379395}%
\pgfsetfillcolor{currentfill}%
\pgfsetlinewidth{0.000000pt}%
\definecolor{currentstroke}{rgb}{0.000000,0.000000,0.000000}%
\pgfsetstrokecolor{currentstroke}%
\pgfsetstrokeopacity{0.000000}%
\pgfsetdash{}{0pt}%
\pgfpathmoveto{\pgfqpoint{1.786016in}{1.613090in}}%
\pgfpathlineto{\pgfqpoint{1.794769in}{1.613090in}}%
\pgfpathlineto{\pgfqpoint{1.794769in}{1.327623in}}%
\pgfpathlineto{\pgfqpoint{1.786016in}{1.327623in}}%
\pgfpathlineto{\pgfqpoint{1.786016in}{1.613090in}}%
\pgfpathclose%
\pgfusepath{fill}%
\end{pgfscope}%
\begin{pgfscope}%
\pgfpathrectangle{\pgfqpoint{0.804646in}{0.600000in}}{\pgfqpoint{2.573292in}{2.070576in}}%
\pgfusepath{clip}%
\pgfsetbuttcap%
\pgfsetmiterjoin%
\definecolor{currentfill}{rgb}{0.133298,0.375282,0.379395}%
\pgfsetfillcolor{currentfill}%
\pgfsetlinewidth{0.000000pt}%
\definecolor{currentstroke}{rgb}{0.000000,0.000000,0.000000}%
\pgfsetstrokecolor{currentstroke}%
\pgfsetstrokeopacity{0.000000}%
\pgfsetdash{}{0pt}%
\pgfpathmoveto{\pgfqpoint{1.796958in}{1.613090in}}%
\pgfpathlineto{\pgfqpoint{1.805711in}{1.613090in}}%
\pgfpathlineto{\pgfqpoint{1.805711in}{1.298546in}}%
\pgfpathlineto{\pgfqpoint{1.796958in}{1.298546in}}%
\pgfpathlineto{\pgfqpoint{1.796958in}{1.613090in}}%
\pgfpathclose%
\pgfusepath{fill}%
\end{pgfscope}%
\begin{pgfscope}%
\pgfpathrectangle{\pgfqpoint{0.804646in}{0.600000in}}{\pgfqpoint{2.573292in}{2.070576in}}%
\pgfusepath{clip}%
\pgfsetbuttcap%
\pgfsetmiterjoin%
\definecolor{currentfill}{rgb}{0.133298,0.375282,0.379395}%
\pgfsetfillcolor{currentfill}%
\pgfsetlinewidth{0.000000pt}%
\definecolor{currentstroke}{rgb}{0.000000,0.000000,0.000000}%
\pgfsetstrokecolor{currentstroke}%
\pgfsetstrokeopacity{0.000000}%
\pgfsetdash{}{0pt}%
\pgfpathmoveto{\pgfqpoint{1.807899in}{1.613090in}}%
\pgfpathlineto{\pgfqpoint{1.816653in}{1.613090in}}%
\pgfpathlineto{\pgfqpoint{1.816653in}{1.272571in}}%
\pgfpathlineto{\pgfqpoint{1.807899in}{1.272571in}}%
\pgfpathlineto{\pgfqpoint{1.807899in}{1.613090in}}%
\pgfpathclose%
\pgfusepath{fill}%
\end{pgfscope}%
\begin{pgfscope}%
\pgfpathrectangle{\pgfqpoint{0.804646in}{0.600000in}}{\pgfqpoint{2.573292in}{2.070576in}}%
\pgfusepath{clip}%
\pgfsetbuttcap%
\pgfsetmiterjoin%
\definecolor{currentfill}{rgb}{0.133298,0.375282,0.379395}%
\pgfsetfillcolor{currentfill}%
\pgfsetlinewidth{0.000000pt}%
\definecolor{currentstroke}{rgb}{0.000000,0.000000,0.000000}%
\pgfsetstrokecolor{currentstroke}%
\pgfsetstrokeopacity{0.000000}%
\pgfsetdash{}{0pt}%
\pgfpathmoveto{\pgfqpoint{1.818841in}{1.613090in}}%
\pgfpathlineto{\pgfqpoint{1.827595in}{1.613090in}}%
\pgfpathlineto{\pgfqpoint{1.827595in}{1.264099in}}%
\pgfpathlineto{\pgfqpoint{1.818841in}{1.264099in}}%
\pgfpathlineto{\pgfqpoint{1.818841in}{1.613090in}}%
\pgfpathclose%
\pgfusepath{fill}%
\end{pgfscope}%
\begin{pgfscope}%
\pgfpathrectangle{\pgfqpoint{0.804646in}{0.600000in}}{\pgfqpoint{2.573292in}{2.070576in}}%
\pgfusepath{clip}%
\pgfsetbuttcap%
\pgfsetmiterjoin%
\definecolor{currentfill}{rgb}{0.133298,0.375282,0.379395}%
\pgfsetfillcolor{currentfill}%
\pgfsetlinewidth{0.000000pt}%
\definecolor{currentstroke}{rgb}{0.000000,0.000000,0.000000}%
\pgfsetstrokecolor{currentstroke}%
\pgfsetstrokeopacity{0.000000}%
\pgfsetdash{}{0pt}%
\pgfpathmoveto{\pgfqpoint{1.829783in}{1.613090in}}%
\pgfpathlineto{\pgfqpoint{1.838536in}{1.613090in}}%
\pgfpathlineto{\pgfqpoint{1.838536in}{1.239888in}}%
\pgfpathlineto{\pgfqpoint{1.829783in}{1.239888in}}%
\pgfpathlineto{\pgfqpoint{1.829783in}{1.613090in}}%
\pgfpathclose%
\pgfusepath{fill}%
\end{pgfscope}%
\begin{pgfscope}%
\pgfpathrectangle{\pgfqpoint{0.804646in}{0.600000in}}{\pgfqpoint{2.573292in}{2.070576in}}%
\pgfusepath{clip}%
\pgfsetbuttcap%
\pgfsetmiterjoin%
\definecolor{currentfill}{rgb}{0.133298,0.375282,0.379395}%
\pgfsetfillcolor{currentfill}%
\pgfsetlinewidth{0.000000pt}%
\definecolor{currentstroke}{rgb}{0.000000,0.000000,0.000000}%
\pgfsetstrokecolor{currentstroke}%
\pgfsetstrokeopacity{0.000000}%
\pgfsetdash{}{0pt}%
\pgfpathmoveto{\pgfqpoint{1.840725in}{1.613090in}}%
\pgfpathlineto{\pgfqpoint{1.849478in}{1.613090in}}%
\pgfpathlineto{\pgfqpoint{1.849478in}{1.192029in}}%
\pgfpathlineto{\pgfqpoint{1.840725in}{1.192029in}}%
\pgfpathlineto{\pgfqpoint{1.840725in}{1.613090in}}%
\pgfpathclose%
\pgfusepath{fill}%
\end{pgfscope}%
\begin{pgfscope}%
\pgfpathrectangle{\pgfqpoint{0.804646in}{0.600000in}}{\pgfqpoint{2.573292in}{2.070576in}}%
\pgfusepath{clip}%
\pgfsetbuttcap%
\pgfsetmiterjoin%
\definecolor{currentfill}{rgb}{0.133298,0.375282,0.379395}%
\pgfsetfillcolor{currentfill}%
\pgfsetlinewidth{0.000000pt}%
\definecolor{currentstroke}{rgb}{0.000000,0.000000,0.000000}%
\pgfsetstrokecolor{currentstroke}%
\pgfsetstrokeopacity{0.000000}%
\pgfsetdash{}{0pt}%
\pgfpathmoveto{\pgfqpoint{1.851667in}{1.613090in}}%
\pgfpathlineto{\pgfqpoint{1.860420in}{1.613090in}}%
\pgfpathlineto{\pgfqpoint{1.860420in}{1.214183in}}%
\pgfpathlineto{\pgfqpoint{1.851667in}{1.214183in}}%
\pgfpathlineto{\pgfqpoint{1.851667in}{1.613090in}}%
\pgfpathclose%
\pgfusepath{fill}%
\end{pgfscope}%
\begin{pgfscope}%
\pgfpathrectangle{\pgfqpoint{0.804646in}{0.600000in}}{\pgfqpoint{2.573292in}{2.070576in}}%
\pgfusepath{clip}%
\pgfsetbuttcap%
\pgfsetmiterjoin%
\definecolor{currentfill}{rgb}{0.133298,0.375282,0.379395}%
\pgfsetfillcolor{currentfill}%
\pgfsetlinewidth{0.000000pt}%
\definecolor{currentstroke}{rgb}{0.000000,0.000000,0.000000}%
\pgfsetstrokecolor{currentstroke}%
\pgfsetstrokeopacity{0.000000}%
\pgfsetdash{}{0pt}%
\pgfpathmoveto{\pgfqpoint{1.862608in}{1.613090in}}%
\pgfpathlineto{\pgfqpoint{1.871362in}{1.613090in}}%
\pgfpathlineto{\pgfqpoint{1.871362in}{1.190065in}}%
\pgfpathlineto{\pgfqpoint{1.862608in}{1.190065in}}%
\pgfpathlineto{\pgfqpoint{1.862608in}{1.613090in}}%
\pgfpathclose%
\pgfusepath{fill}%
\end{pgfscope}%
\begin{pgfscope}%
\pgfpathrectangle{\pgfqpoint{0.804646in}{0.600000in}}{\pgfqpoint{2.573292in}{2.070576in}}%
\pgfusepath{clip}%
\pgfsetbuttcap%
\pgfsetmiterjoin%
\definecolor{currentfill}{rgb}{0.133298,0.375282,0.379395}%
\pgfsetfillcolor{currentfill}%
\pgfsetlinewidth{0.000000pt}%
\definecolor{currentstroke}{rgb}{0.000000,0.000000,0.000000}%
\pgfsetstrokecolor{currentstroke}%
\pgfsetstrokeopacity{0.000000}%
\pgfsetdash{}{0pt}%
\pgfpathmoveto{\pgfqpoint{1.873550in}{1.613090in}}%
\pgfpathlineto{\pgfqpoint{1.882304in}{1.613090in}}%
\pgfpathlineto{\pgfqpoint{1.882304in}{1.142819in}}%
\pgfpathlineto{\pgfqpoint{1.873550in}{1.142819in}}%
\pgfpathlineto{\pgfqpoint{1.873550in}{1.613090in}}%
\pgfpathclose%
\pgfusepath{fill}%
\end{pgfscope}%
\begin{pgfscope}%
\pgfpathrectangle{\pgfqpoint{0.804646in}{0.600000in}}{\pgfqpoint{2.573292in}{2.070576in}}%
\pgfusepath{clip}%
\pgfsetbuttcap%
\pgfsetmiterjoin%
\definecolor{currentfill}{rgb}{0.133298,0.375282,0.379395}%
\pgfsetfillcolor{currentfill}%
\pgfsetlinewidth{0.000000pt}%
\definecolor{currentstroke}{rgb}{0.000000,0.000000,0.000000}%
\pgfsetstrokecolor{currentstroke}%
\pgfsetstrokeopacity{0.000000}%
\pgfsetdash{}{0pt}%
\pgfpathmoveto{\pgfqpoint{1.884492in}{1.613090in}}%
\pgfpathlineto{\pgfqpoint{1.893245in}{1.613090in}}%
\pgfpathlineto{\pgfqpoint{1.893245in}{1.103020in}}%
\pgfpathlineto{\pgfqpoint{1.884492in}{1.103020in}}%
\pgfpathlineto{\pgfqpoint{1.884492in}{1.613090in}}%
\pgfpathclose%
\pgfusepath{fill}%
\end{pgfscope}%
\begin{pgfscope}%
\pgfpathrectangle{\pgfqpoint{0.804646in}{0.600000in}}{\pgfqpoint{2.573292in}{2.070576in}}%
\pgfusepath{clip}%
\pgfsetbuttcap%
\pgfsetmiterjoin%
\definecolor{currentfill}{rgb}{0.133298,0.375282,0.379395}%
\pgfsetfillcolor{currentfill}%
\pgfsetlinewidth{0.000000pt}%
\definecolor{currentstroke}{rgb}{0.000000,0.000000,0.000000}%
\pgfsetstrokecolor{currentstroke}%
\pgfsetstrokeopacity{0.000000}%
\pgfsetdash{}{0pt}%
\pgfpathmoveto{\pgfqpoint{1.895434in}{1.613090in}}%
\pgfpathlineto{\pgfqpoint{1.904187in}{1.613090in}}%
\pgfpathlineto{\pgfqpoint{1.904187in}{1.143349in}}%
\pgfpathlineto{\pgfqpoint{1.895434in}{1.143349in}}%
\pgfpathlineto{\pgfqpoint{1.895434in}{1.613090in}}%
\pgfpathclose%
\pgfusepath{fill}%
\end{pgfscope}%
\begin{pgfscope}%
\pgfpathrectangle{\pgfqpoint{0.804646in}{0.600000in}}{\pgfqpoint{2.573292in}{2.070576in}}%
\pgfusepath{clip}%
\pgfsetbuttcap%
\pgfsetmiterjoin%
\definecolor{currentfill}{rgb}{0.133298,0.375282,0.379395}%
\pgfsetfillcolor{currentfill}%
\pgfsetlinewidth{0.000000pt}%
\definecolor{currentstroke}{rgb}{0.000000,0.000000,0.000000}%
\pgfsetstrokecolor{currentstroke}%
\pgfsetstrokeopacity{0.000000}%
\pgfsetdash{}{0pt}%
\pgfpathmoveto{\pgfqpoint{1.906376in}{1.613090in}}%
\pgfpathlineto{\pgfqpoint{1.915129in}{1.613090in}}%
\pgfpathlineto{\pgfqpoint{1.915129in}{1.108065in}}%
\pgfpathlineto{\pgfqpoint{1.906376in}{1.108065in}}%
\pgfpathlineto{\pgfqpoint{1.906376in}{1.613090in}}%
\pgfpathclose%
\pgfusepath{fill}%
\end{pgfscope}%
\begin{pgfscope}%
\pgfpathrectangle{\pgfqpoint{0.804646in}{0.600000in}}{\pgfqpoint{2.573292in}{2.070576in}}%
\pgfusepath{clip}%
\pgfsetbuttcap%
\pgfsetmiterjoin%
\definecolor{currentfill}{rgb}{0.133298,0.375282,0.379395}%
\pgfsetfillcolor{currentfill}%
\pgfsetlinewidth{0.000000pt}%
\definecolor{currentstroke}{rgb}{0.000000,0.000000,0.000000}%
\pgfsetstrokecolor{currentstroke}%
\pgfsetstrokeopacity{0.000000}%
\pgfsetdash{}{0pt}%
\pgfpathmoveto{\pgfqpoint{1.917317in}{1.613090in}}%
\pgfpathlineto{\pgfqpoint{1.926071in}{1.613090in}}%
\pgfpathlineto{\pgfqpoint{1.926071in}{1.095045in}}%
\pgfpathlineto{\pgfqpoint{1.917317in}{1.095045in}}%
\pgfpathlineto{\pgfqpoint{1.917317in}{1.613090in}}%
\pgfpathclose%
\pgfusepath{fill}%
\end{pgfscope}%
\begin{pgfscope}%
\pgfpathrectangle{\pgfqpoint{0.804646in}{0.600000in}}{\pgfqpoint{2.573292in}{2.070576in}}%
\pgfusepath{clip}%
\pgfsetbuttcap%
\pgfsetmiterjoin%
\definecolor{currentfill}{rgb}{0.133298,0.375282,0.379395}%
\pgfsetfillcolor{currentfill}%
\pgfsetlinewidth{0.000000pt}%
\definecolor{currentstroke}{rgb}{0.000000,0.000000,0.000000}%
\pgfsetstrokecolor{currentstroke}%
\pgfsetstrokeopacity{0.000000}%
\pgfsetdash{}{0pt}%
\pgfpathmoveto{\pgfqpoint{1.928259in}{1.613090in}}%
\pgfpathlineto{\pgfqpoint{1.937013in}{1.613090in}}%
\pgfpathlineto{\pgfqpoint{1.937013in}{1.102347in}}%
\pgfpathlineto{\pgfqpoint{1.928259in}{1.102347in}}%
\pgfpathlineto{\pgfqpoint{1.928259in}{1.613090in}}%
\pgfpathclose%
\pgfusepath{fill}%
\end{pgfscope}%
\begin{pgfscope}%
\pgfpathrectangle{\pgfqpoint{0.804646in}{0.600000in}}{\pgfqpoint{2.573292in}{2.070576in}}%
\pgfusepath{clip}%
\pgfsetbuttcap%
\pgfsetmiterjoin%
\definecolor{currentfill}{rgb}{0.133298,0.375282,0.379395}%
\pgfsetfillcolor{currentfill}%
\pgfsetlinewidth{0.000000pt}%
\definecolor{currentstroke}{rgb}{0.000000,0.000000,0.000000}%
\pgfsetstrokecolor{currentstroke}%
\pgfsetstrokeopacity{0.000000}%
\pgfsetdash{}{0pt}%
\pgfpathmoveto{\pgfqpoint{1.939201in}{1.613090in}}%
\pgfpathlineto{\pgfqpoint{1.947954in}{1.613090in}}%
\pgfpathlineto{\pgfqpoint{1.947954in}{1.096854in}}%
\pgfpathlineto{\pgfqpoint{1.939201in}{1.096854in}}%
\pgfpathlineto{\pgfqpoint{1.939201in}{1.613090in}}%
\pgfpathclose%
\pgfusepath{fill}%
\end{pgfscope}%
\begin{pgfscope}%
\pgfpathrectangle{\pgfqpoint{0.804646in}{0.600000in}}{\pgfqpoint{2.573292in}{2.070576in}}%
\pgfusepath{clip}%
\pgfsetbuttcap%
\pgfsetmiterjoin%
\definecolor{currentfill}{rgb}{0.133298,0.375282,0.379395}%
\pgfsetfillcolor{currentfill}%
\pgfsetlinewidth{0.000000pt}%
\definecolor{currentstroke}{rgb}{0.000000,0.000000,0.000000}%
\pgfsetstrokecolor{currentstroke}%
\pgfsetstrokeopacity{0.000000}%
\pgfsetdash{}{0pt}%
\pgfpathmoveto{\pgfqpoint{1.950143in}{1.613090in}}%
\pgfpathlineto{\pgfqpoint{1.958896in}{1.613090in}}%
\pgfpathlineto{\pgfqpoint{1.958896in}{1.045357in}}%
\pgfpathlineto{\pgfqpoint{1.950143in}{1.045357in}}%
\pgfpathlineto{\pgfqpoint{1.950143in}{1.613090in}}%
\pgfpathclose%
\pgfusepath{fill}%
\end{pgfscope}%
\begin{pgfscope}%
\pgfpathrectangle{\pgfqpoint{0.804646in}{0.600000in}}{\pgfqpoint{2.573292in}{2.070576in}}%
\pgfusepath{clip}%
\pgfsetbuttcap%
\pgfsetmiterjoin%
\definecolor{currentfill}{rgb}{0.133298,0.375282,0.379395}%
\pgfsetfillcolor{currentfill}%
\pgfsetlinewidth{0.000000pt}%
\definecolor{currentstroke}{rgb}{0.000000,0.000000,0.000000}%
\pgfsetstrokecolor{currentstroke}%
\pgfsetstrokeopacity{0.000000}%
\pgfsetdash{}{0pt}%
\pgfpathmoveto{\pgfqpoint{1.961085in}{1.613090in}}%
\pgfpathlineto{\pgfqpoint{1.969838in}{1.613090in}}%
\pgfpathlineto{\pgfqpoint{1.969838in}{1.015047in}}%
\pgfpathlineto{\pgfqpoint{1.961085in}{1.015047in}}%
\pgfpathlineto{\pgfqpoint{1.961085in}{1.613090in}}%
\pgfpathclose%
\pgfusepath{fill}%
\end{pgfscope}%
\begin{pgfscope}%
\pgfpathrectangle{\pgfqpoint{0.804646in}{0.600000in}}{\pgfqpoint{2.573292in}{2.070576in}}%
\pgfusepath{clip}%
\pgfsetbuttcap%
\pgfsetmiterjoin%
\definecolor{currentfill}{rgb}{0.133298,0.375282,0.379395}%
\pgfsetfillcolor{currentfill}%
\pgfsetlinewidth{0.000000pt}%
\definecolor{currentstroke}{rgb}{0.000000,0.000000,0.000000}%
\pgfsetstrokecolor{currentstroke}%
\pgfsetstrokeopacity{0.000000}%
\pgfsetdash{}{0pt}%
\pgfpathmoveto{\pgfqpoint{1.972026in}{1.613090in}}%
\pgfpathlineto{\pgfqpoint{1.980780in}{1.613090in}}%
\pgfpathlineto{\pgfqpoint{1.980780in}{1.027400in}}%
\pgfpathlineto{\pgfqpoint{1.972026in}{1.027400in}}%
\pgfpathlineto{\pgfqpoint{1.972026in}{1.613090in}}%
\pgfpathclose%
\pgfusepath{fill}%
\end{pgfscope}%
\begin{pgfscope}%
\pgfpathrectangle{\pgfqpoint{0.804646in}{0.600000in}}{\pgfqpoint{2.573292in}{2.070576in}}%
\pgfusepath{clip}%
\pgfsetbuttcap%
\pgfsetmiterjoin%
\definecolor{currentfill}{rgb}{0.133298,0.375282,0.379395}%
\pgfsetfillcolor{currentfill}%
\pgfsetlinewidth{0.000000pt}%
\definecolor{currentstroke}{rgb}{0.000000,0.000000,0.000000}%
\pgfsetstrokecolor{currentstroke}%
\pgfsetstrokeopacity{0.000000}%
\pgfsetdash{}{0pt}%
\pgfpathmoveto{\pgfqpoint{1.982968in}{1.613090in}}%
\pgfpathlineto{\pgfqpoint{1.991722in}{1.613090in}}%
\pgfpathlineto{\pgfqpoint{1.991722in}{1.012173in}}%
\pgfpathlineto{\pgfqpoint{1.982968in}{1.012173in}}%
\pgfpathlineto{\pgfqpoint{1.982968in}{1.613090in}}%
\pgfpathclose%
\pgfusepath{fill}%
\end{pgfscope}%
\begin{pgfscope}%
\pgfpathrectangle{\pgfqpoint{0.804646in}{0.600000in}}{\pgfqpoint{2.573292in}{2.070576in}}%
\pgfusepath{clip}%
\pgfsetbuttcap%
\pgfsetmiterjoin%
\definecolor{currentfill}{rgb}{0.133298,0.375282,0.379395}%
\pgfsetfillcolor{currentfill}%
\pgfsetlinewidth{0.000000pt}%
\definecolor{currentstroke}{rgb}{0.000000,0.000000,0.000000}%
\pgfsetstrokecolor{currentstroke}%
\pgfsetstrokeopacity{0.000000}%
\pgfsetdash{}{0pt}%
\pgfpathmoveto{\pgfqpoint{1.993910in}{1.613090in}}%
\pgfpathlineto{\pgfqpoint{2.002663in}{1.613090in}}%
\pgfpathlineto{\pgfqpoint{2.002663in}{0.946937in}}%
\pgfpathlineto{\pgfqpoint{1.993910in}{0.946937in}}%
\pgfpathlineto{\pgfqpoint{1.993910in}{1.613090in}}%
\pgfpathclose%
\pgfusepath{fill}%
\end{pgfscope}%
\begin{pgfscope}%
\pgfpathrectangle{\pgfqpoint{0.804646in}{0.600000in}}{\pgfqpoint{2.573292in}{2.070576in}}%
\pgfusepath{clip}%
\pgfsetbuttcap%
\pgfsetmiterjoin%
\definecolor{currentfill}{rgb}{0.133298,0.375282,0.379395}%
\pgfsetfillcolor{currentfill}%
\pgfsetlinewidth{0.000000pt}%
\definecolor{currentstroke}{rgb}{0.000000,0.000000,0.000000}%
\pgfsetstrokecolor{currentstroke}%
\pgfsetstrokeopacity{0.000000}%
\pgfsetdash{}{0pt}%
\pgfpathmoveto{\pgfqpoint{2.004852in}{1.613090in}}%
\pgfpathlineto{\pgfqpoint{2.013605in}{1.613090in}}%
\pgfpathlineto{\pgfqpoint{2.013605in}{0.957439in}}%
\pgfpathlineto{\pgfqpoint{2.004852in}{0.957439in}}%
\pgfpathlineto{\pgfqpoint{2.004852in}{1.613090in}}%
\pgfpathclose%
\pgfusepath{fill}%
\end{pgfscope}%
\begin{pgfscope}%
\pgfpathrectangle{\pgfqpoint{0.804646in}{0.600000in}}{\pgfqpoint{2.573292in}{2.070576in}}%
\pgfusepath{clip}%
\pgfsetbuttcap%
\pgfsetmiterjoin%
\definecolor{currentfill}{rgb}{0.133298,0.375282,0.379395}%
\pgfsetfillcolor{currentfill}%
\pgfsetlinewidth{0.000000pt}%
\definecolor{currentstroke}{rgb}{0.000000,0.000000,0.000000}%
\pgfsetstrokecolor{currentstroke}%
\pgfsetstrokeopacity{0.000000}%
\pgfsetdash{}{0pt}%
\pgfpathmoveto{\pgfqpoint{2.015794in}{1.613090in}}%
\pgfpathlineto{\pgfqpoint{2.024547in}{1.613090in}}%
\pgfpathlineto{\pgfqpoint{2.024547in}{0.926495in}}%
\pgfpathlineto{\pgfqpoint{2.015794in}{0.926495in}}%
\pgfpathlineto{\pgfqpoint{2.015794in}{1.613090in}}%
\pgfpathclose%
\pgfusepath{fill}%
\end{pgfscope}%
\begin{pgfscope}%
\pgfpathrectangle{\pgfqpoint{0.804646in}{0.600000in}}{\pgfqpoint{2.573292in}{2.070576in}}%
\pgfusepath{clip}%
\pgfsetbuttcap%
\pgfsetmiterjoin%
\definecolor{currentfill}{rgb}{0.133298,0.375282,0.379395}%
\pgfsetfillcolor{currentfill}%
\pgfsetlinewidth{0.000000pt}%
\definecolor{currentstroke}{rgb}{0.000000,0.000000,0.000000}%
\pgfsetstrokecolor{currentstroke}%
\pgfsetstrokeopacity{0.000000}%
\pgfsetdash{}{0pt}%
\pgfpathmoveto{\pgfqpoint{2.026735in}{1.613090in}}%
\pgfpathlineto{\pgfqpoint{2.035489in}{1.613090in}}%
\pgfpathlineto{\pgfqpoint{2.035489in}{0.903314in}}%
\pgfpathlineto{\pgfqpoint{2.026735in}{0.903314in}}%
\pgfpathlineto{\pgfqpoint{2.026735in}{1.613090in}}%
\pgfpathclose%
\pgfusepath{fill}%
\end{pgfscope}%
\begin{pgfscope}%
\pgfpathrectangle{\pgfqpoint{0.804646in}{0.600000in}}{\pgfqpoint{2.573292in}{2.070576in}}%
\pgfusepath{clip}%
\pgfsetbuttcap%
\pgfsetmiterjoin%
\definecolor{currentfill}{rgb}{0.133298,0.375282,0.379395}%
\pgfsetfillcolor{currentfill}%
\pgfsetlinewidth{0.000000pt}%
\definecolor{currentstroke}{rgb}{0.000000,0.000000,0.000000}%
\pgfsetstrokecolor{currentstroke}%
\pgfsetstrokeopacity{0.000000}%
\pgfsetdash{}{0pt}%
\pgfpathmoveto{\pgfqpoint{2.037677in}{1.613090in}}%
\pgfpathlineto{\pgfqpoint{2.046431in}{1.613090in}}%
\pgfpathlineto{\pgfqpoint{2.046431in}{0.916423in}}%
\pgfpathlineto{\pgfqpoint{2.037677in}{0.916423in}}%
\pgfpathlineto{\pgfqpoint{2.037677in}{1.613090in}}%
\pgfpathclose%
\pgfusepath{fill}%
\end{pgfscope}%
\begin{pgfscope}%
\pgfpathrectangle{\pgfqpoint{0.804646in}{0.600000in}}{\pgfqpoint{2.573292in}{2.070576in}}%
\pgfusepath{clip}%
\pgfsetbuttcap%
\pgfsetmiterjoin%
\definecolor{currentfill}{rgb}{0.133298,0.375282,0.379395}%
\pgfsetfillcolor{currentfill}%
\pgfsetlinewidth{0.000000pt}%
\definecolor{currentstroke}{rgb}{0.000000,0.000000,0.000000}%
\pgfsetstrokecolor{currentstroke}%
\pgfsetstrokeopacity{0.000000}%
\pgfsetdash{}{0pt}%
\pgfpathmoveto{\pgfqpoint{2.048619in}{1.613090in}}%
\pgfpathlineto{\pgfqpoint{2.057372in}{1.613090in}}%
\pgfpathlineto{\pgfqpoint{2.057372in}{0.884639in}}%
\pgfpathlineto{\pgfqpoint{2.048619in}{0.884639in}}%
\pgfpathlineto{\pgfqpoint{2.048619in}{1.613090in}}%
\pgfpathclose%
\pgfusepath{fill}%
\end{pgfscope}%
\begin{pgfscope}%
\pgfpathrectangle{\pgfqpoint{0.804646in}{0.600000in}}{\pgfqpoint{2.573292in}{2.070576in}}%
\pgfusepath{clip}%
\pgfsetbuttcap%
\pgfsetmiterjoin%
\definecolor{currentfill}{rgb}{0.133298,0.375282,0.379395}%
\pgfsetfillcolor{currentfill}%
\pgfsetlinewidth{0.000000pt}%
\definecolor{currentstroke}{rgb}{0.000000,0.000000,0.000000}%
\pgfsetstrokecolor{currentstroke}%
\pgfsetstrokeopacity{0.000000}%
\pgfsetdash{}{0pt}%
\pgfpathmoveto{\pgfqpoint{2.059561in}{1.613090in}}%
\pgfpathlineto{\pgfqpoint{2.068314in}{1.613090in}}%
\pgfpathlineto{\pgfqpoint{2.068314in}{0.879382in}}%
\pgfpathlineto{\pgfqpoint{2.059561in}{0.879382in}}%
\pgfpathlineto{\pgfqpoint{2.059561in}{1.613090in}}%
\pgfpathclose%
\pgfusepath{fill}%
\end{pgfscope}%
\begin{pgfscope}%
\pgfpathrectangle{\pgfqpoint{0.804646in}{0.600000in}}{\pgfqpoint{2.573292in}{2.070576in}}%
\pgfusepath{clip}%
\pgfsetbuttcap%
\pgfsetmiterjoin%
\definecolor{currentfill}{rgb}{0.133298,0.375282,0.379395}%
\pgfsetfillcolor{currentfill}%
\pgfsetlinewidth{0.000000pt}%
\definecolor{currentstroke}{rgb}{0.000000,0.000000,0.000000}%
\pgfsetstrokecolor{currentstroke}%
\pgfsetstrokeopacity{0.000000}%
\pgfsetdash{}{0pt}%
\pgfpathmoveto{\pgfqpoint{2.070503in}{1.613090in}}%
\pgfpathlineto{\pgfqpoint{2.079256in}{1.613090in}}%
\pgfpathlineto{\pgfqpoint{2.079256in}{0.846008in}}%
\pgfpathlineto{\pgfqpoint{2.070503in}{0.846008in}}%
\pgfpathlineto{\pgfqpoint{2.070503in}{1.613090in}}%
\pgfpathclose%
\pgfusepath{fill}%
\end{pgfscope}%
\begin{pgfscope}%
\pgfpathrectangle{\pgfqpoint{0.804646in}{0.600000in}}{\pgfqpoint{2.573292in}{2.070576in}}%
\pgfusepath{clip}%
\pgfsetbuttcap%
\pgfsetmiterjoin%
\definecolor{currentfill}{rgb}{0.133298,0.375282,0.379395}%
\pgfsetfillcolor{currentfill}%
\pgfsetlinewidth{0.000000pt}%
\definecolor{currentstroke}{rgb}{0.000000,0.000000,0.000000}%
\pgfsetstrokecolor{currentstroke}%
\pgfsetstrokeopacity{0.000000}%
\pgfsetdash{}{0pt}%
\pgfpathmoveto{\pgfqpoint{2.081444in}{1.613090in}}%
\pgfpathlineto{\pgfqpoint{2.090198in}{1.613090in}}%
\pgfpathlineto{\pgfqpoint{2.090198in}{0.861357in}}%
\pgfpathlineto{\pgfqpoint{2.081444in}{0.861357in}}%
\pgfpathlineto{\pgfqpoint{2.081444in}{1.613090in}}%
\pgfpathclose%
\pgfusepath{fill}%
\end{pgfscope}%
\begin{pgfscope}%
\pgfpathrectangle{\pgfqpoint{0.804646in}{0.600000in}}{\pgfqpoint{2.573292in}{2.070576in}}%
\pgfusepath{clip}%
\pgfsetbuttcap%
\pgfsetmiterjoin%
\definecolor{currentfill}{rgb}{0.133298,0.375282,0.379395}%
\pgfsetfillcolor{currentfill}%
\pgfsetlinewidth{0.000000pt}%
\definecolor{currentstroke}{rgb}{0.000000,0.000000,0.000000}%
\pgfsetstrokecolor{currentstroke}%
\pgfsetstrokeopacity{0.000000}%
\pgfsetdash{}{0pt}%
\pgfpathmoveto{\pgfqpoint{2.092386in}{1.613090in}}%
\pgfpathlineto{\pgfqpoint{2.101140in}{1.613090in}}%
\pgfpathlineto{\pgfqpoint{2.101140in}{0.832246in}}%
\pgfpathlineto{\pgfqpoint{2.092386in}{0.832246in}}%
\pgfpathlineto{\pgfqpoint{2.092386in}{1.613090in}}%
\pgfpathclose%
\pgfusepath{fill}%
\end{pgfscope}%
\begin{pgfscope}%
\pgfpathrectangle{\pgfqpoint{0.804646in}{0.600000in}}{\pgfqpoint{2.573292in}{2.070576in}}%
\pgfusepath{clip}%
\pgfsetbuttcap%
\pgfsetmiterjoin%
\definecolor{currentfill}{rgb}{0.133298,0.375282,0.379395}%
\pgfsetfillcolor{currentfill}%
\pgfsetlinewidth{0.000000pt}%
\definecolor{currentstroke}{rgb}{0.000000,0.000000,0.000000}%
\pgfsetstrokecolor{currentstroke}%
\pgfsetstrokeopacity{0.000000}%
\pgfsetdash{}{0pt}%
\pgfpathmoveto{\pgfqpoint{2.103328in}{1.613090in}}%
\pgfpathlineto{\pgfqpoint{2.112081in}{1.613090in}}%
\pgfpathlineto{\pgfqpoint{2.112081in}{0.818772in}}%
\pgfpathlineto{\pgfqpoint{2.103328in}{0.818772in}}%
\pgfpathlineto{\pgfqpoint{2.103328in}{1.613090in}}%
\pgfpathclose%
\pgfusepath{fill}%
\end{pgfscope}%
\begin{pgfscope}%
\pgfpathrectangle{\pgfqpoint{0.804646in}{0.600000in}}{\pgfqpoint{2.573292in}{2.070576in}}%
\pgfusepath{clip}%
\pgfsetbuttcap%
\pgfsetmiterjoin%
\definecolor{currentfill}{rgb}{0.133298,0.375282,0.379395}%
\pgfsetfillcolor{currentfill}%
\pgfsetlinewidth{0.000000pt}%
\definecolor{currentstroke}{rgb}{0.000000,0.000000,0.000000}%
\pgfsetstrokecolor{currentstroke}%
\pgfsetstrokeopacity{0.000000}%
\pgfsetdash{}{0pt}%
\pgfpathmoveto{\pgfqpoint{2.114270in}{1.613090in}}%
\pgfpathlineto{\pgfqpoint{2.123023in}{1.613090in}}%
\pgfpathlineto{\pgfqpoint{2.123023in}{0.823211in}}%
\pgfpathlineto{\pgfqpoint{2.114270in}{0.823211in}}%
\pgfpathlineto{\pgfqpoint{2.114270in}{1.613090in}}%
\pgfpathclose%
\pgfusepath{fill}%
\end{pgfscope}%
\begin{pgfscope}%
\pgfpathrectangle{\pgfqpoint{0.804646in}{0.600000in}}{\pgfqpoint{2.573292in}{2.070576in}}%
\pgfusepath{clip}%
\pgfsetbuttcap%
\pgfsetmiterjoin%
\definecolor{currentfill}{rgb}{0.133298,0.375282,0.379395}%
\pgfsetfillcolor{currentfill}%
\pgfsetlinewidth{0.000000pt}%
\definecolor{currentstroke}{rgb}{0.000000,0.000000,0.000000}%
\pgfsetstrokecolor{currentstroke}%
\pgfsetstrokeopacity{0.000000}%
\pgfsetdash{}{0pt}%
\pgfpathmoveto{\pgfqpoint{2.125212in}{1.613090in}}%
\pgfpathlineto{\pgfqpoint{2.133965in}{1.613090in}}%
\pgfpathlineto{\pgfqpoint{2.133965in}{0.837325in}}%
\pgfpathlineto{\pgfqpoint{2.125212in}{0.837325in}}%
\pgfpathlineto{\pgfqpoint{2.125212in}{1.613090in}}%
\pgfpathclose%
\pgfusepath{fill}%
\end{pgfscope}%
\begin{pgfscope}%
\pgfpathrectangle{\pgfqpoint{0.804646in}{0.600000in}}{\pgfqpoint{2.573292in}{2.070576in}}%
\pgfusepath{clip}%
\pgfsetbuttcap%
\pgfsetmiterjoin%
\definecolor{currentfill}{rgb}{0.133298,0.375282,0.379395}%
\pgfsetfillcolor{currentfill}%
\pgfsetlinewidth{0.000000pt}%
\definecolor{currentstroke}{rgb}{0.000000,0.000000,0.000000}%
\pgfsetstrokecolor{currentstroke}%
\pgfsetstrokeopacity{0.000000}%
\pgfsetdash{}{0pt}%
\pgfpathmoveto{\pgfqpoint{2.136153in}{1.613090in}}%
\pgfpathlineto{\pgfqpoint{2.144907in}{1.613090in}}%
\pgfpathlineto{\pgfqpoint{2.144907in}{0.824255in}}%
\pgfpathlineto{\pgfqpoint{2.136153in}{0.824255in}}%
\pgfpathlineto{\pgfqpoint{2.136153in}{1.613090in}}%
\pgfpathclose%
\pgfusepath{fill}%
\end{pgfscope}%
\begin{pgfscope}%
\pgfpathrectangle{\pgfqpoint{0.804646in}{0.600000in}}{\pgfqpoint{2.573292in}{2.070576in}}%
\pgfusepath{clip}%
\pgfsetbuttcap%
\pgfsetmiterjoin%
\definecolor{currentfill}{rgb}{0.133298,0.375282,0.379395}%
\pgfsetfillcolor{currentfill}%
\pgfsetlinewidth{0.000000pt}%
\definecolor{currentstroke}{rgb}{0.000000,0.000000,0.000000}%
\pgfsetstrokecolor{currentstroke}%
\pgfsetstrokeopacity{0.000000}%
\pgfsetdash{}{0pt}%
\pgfpathmoveto{\pgfqpoint{2.147095in}{1.613090in}}%
\pgfpathlineto{\pgfqpoint{2.155849in}{1.613090in}}%
\pgfpathlineto{\pgfqpoint{2.155849in}{0.816083in}}%
\pgfpathlineto{\pgfqpoint{2.147095in}{0.816083in}}%
\pgfpathlineto{\pgfqpoint{2.147095in}{1.613090in}}%
\pgfpathclose%
\pgfusepath{fill}%
\end{pgfscope}%
\begin{pgfscope}%
\pgfpathrectangle{\pgfqpoint{0.804646in}{0.600000in}}{\pgfqpoint{2.573292in}{2.070576in}}%
\pgfusepath{clip}%
\pgfsetbuttcap%
\pgfsetmiterjoin%
\definecolor{currentfill}{rgb}{0.133298,0.375282,0.379395}%
\pgfsetfillcolor{currentfill}%
\pgfsetlinewidth{0.000000pt}%
\definecolor{currentstroke}{rgb}{0.000000,0.000000,0.000000}%
\pgfsetstrokecolor{currentstroke}%
\pgfsetstrokeopacity{0.000000}%
\pgfsetdash{}{0pt}%
\pgfpathmoveto{\pgfqpoint{2.158037in}{1.613090in}}%
\pgfpathlineto{\pgfqpoint{2.166790in}{1.613090in}}%
\pgfpathlineto{\pgfqpoint{2.166790in}{0.819333in}}%
\pgfpathlineto{\pgfqpoint{2.158037in}{0.819333in}}%
\pgfpathlineto{\pgfqpoint{2.158037in}{1.613090in}}%
\pgfpathclose%
\pgfusepath{fill}%
\end{pgfscope}%
\begin{pgfscope}%
\pgfpathrectangle{\pgfqpoint{0.804646in}{0.600000in}}{\pgfqpoint{2.573292in}{2.070576in}}%
\pgfusepath{clip}%
\pgfsetbuttcap%
\pgfsetmiterjoin%
\definecolor{currentfill}{rgb}{0.133298,0.375282,0.379395}%
\pgfsetfillcolor{currentfill}%
\pgfsetlinewidth{0.000000pt}%
\definecolor{currentstroke}{rgb}{0.000000,0.000000,0.000000}%
\pgfsetstrokecolor{currentstroke}%
\pgfsetstrokeopacity{0.000000}%
\pgfsetdash{}{0pt}%
\pgfpathmoveto{\pgfqpoint{2.168979in}{1.613090in}}%
\pgfpathlineto{\pgfqpoint{2.177732in}{1.613090in}}%
\pgfpathlineto{\pgfqpoint{2.177732in}{0.805306in}}%
\pgfpathlineto{\pgfqpoint{2.168979in}{0.805306in}}%
\pgfpathlineto{\pgfqpoint{2.168979in}{1.613090in}}%
\pgfpathclose%
\pgfusepath{fill}%
\end{pgfscope}%
\begin{pgfscope}%
\pgfpathrectangle{\pgfqpoint{0.804646in}{0.600000in}}{\pgfqpoint{2.573292in}{2.070576in}}%
\pgfusepath{clip}%
\pgfsetbuttcap%
\pgfsetmiterjoin%
\definecolor{currentfill}{rgb}{0.133298,0.375282,0.379395}%
\pgfsetfillcolor{currentfill}%
\pgfsetlinewidth{0.000000pt}%
\definecolor{currentstroke}{rgb}{0.000000,0.000000,0.000000}%
\pgfsetstrokecolor{currentstroke}%
\pgfsetstrokeopacity{0.000000}%
\pgfsetdash{}{0pt}%
\pgfpathmoveto{\pgfqpoint{2.179921in}{1.613090in}}%
\pgfpathlineto{\pgfqpoint{2.188674in}{1.613090in}}%
\pgfpathlineto{\pgfqpoint{2.188674in}{0.818514in}}%
\pgfpathlineto{\pgfqpoint{2.179921in}{0.818514in}}%
\pgfpathlineto{\pgfqpoint{2.179921in}{1.613090in}}%
\pgfpathclose%
\pgfusepath{fill}%
\end{pgfscope}%
\begin{pgfscope}%
\pgfpathrectangle{\pgfqpoint{0.804646in}{0.600000in}}{\pgfqpoint{2.573292in}{2.070576in}}%
\pgfusepath{clip}%
\pgfsetbuttcap%
\pgfsetmiterjoin%
\definecolor{currentfill}{rgb}{0.133298,0.375282,0.379395}%
\pgfsetfillcolor{currentfill}%
\pgfsetlinewidth{0.000000pt}%
\definecolor{currentstroke}{rgb}{0.000000,0.000000,0.000000}%
\pgfsetstrokecolor{currentstroke}%
\pgfsetstrokeopacity{0.000000}%
\pgfsetdash{}{0pt}%
\pgfpathmoveto{\pgfqpoint{2.190862in}{1.613090in}}%
\pgfpathlineto{\pgfqpoint{2.199616in}{1.613090in}}%
\pgfpathlineto{\pgfqpoint{2.199616in}{0.849907in}}%
\pgfpathlineto{\pgfqpoint{2.190862in}{0.849907in}}%
\pgfpathlineto{\pgfqpoint{2.190862in}{1.613090in}}%
\pgfpathclose%
\pgfusepath{fill}%
\end{pgfscope}%
\begin{pgfscope}%
\pgfpathrectangle{\pgfqpoint{0.804646in}{0.600000in}}{\pgfqpoint{2.573292in}{2.070576in}}%
\pgfusepath{clip}%
\pgfsetbuttcap%
\pgfsetmiterjoin%
\definecolor{currentfill}{rgb}{0.133298,0.375282,0.379395}%
\pgfsetfillcolor{currentfill}%
\pgfsetlinewidth{0.000000pt}%
\definecolor{currentstroke}{rgb}{0.000000,0.000000,0.000000}%
\pgfsetstrokecolor{currentstroke}%
\pgfsetstrokeopacity{0.000000}%
\pgfsetdash{}{0pt}%
\pgfpathmoveto{\pgfqpoint{2.201804in}{1.613090in}}%
\pgfpathlineto{\pgfqpoint{2.210558in}{1.613090in}}%
\pgfpathlineto{\pgfqpoint{2.210558in}{0.844020in}}%
\pgfpathlineto{\pgfqpoint{2.201804in}{0.844020in}}%
\pgfpathlineto{\pgfqpoint{2.201804in}{1.613090in}}%
\pgfpathclose%
\pgfusepath{fill}%
\end{pgfscope}%
\begin{pgfscope}%
\pgfpathrectangle{\pgfqpoint{0.804646in}{0.600000in}}{\pgfqpoint{2.573292in}{2.070576in}}%
\pgfusepath{clip}%
\pgfsetbuttcap%
\pgfsetmiterjoin%
\definecolor{currentfill}{rgb}{0.133298,0.375282,0.379395}%
\pgfsetfillcolor{currentfill}%
\pgfsetlinewidth{0.000000pt}%
\definecolor{currentstroke}{rgb}{0.000000,0.000000,0.000000}%
\pgfsetstrokecolor{currentstroke}%
\pgfsetstrokeopacity{0.000000}%
\pgfsetdash{}{0pt}%
\pgfpathmoveto{\pgfqpoint{2.212746in}{1.613090in}}%
\pgfpathlineto{\pgfqpoint{2.221499in}{1.613090in}}%
\pgfpathlineto{\pgfqpoint{2.221499in}{0.864624in}}%
\pgfpathlineto{\pgfqpoint{2.212746in}{0.864624in}}%
\pgfpathlineto{\pgfqpoint{2.212746in}{1.613090in}}%
\pgfpathclose%
\pgfusepath{fill}%
\end{pgfscope}%
\begin{pgfscope}%
\pgfpathrectangle{\pgfqpoint{0.804646in}{0.600000in}}{\pgfqpoint{2.573292in}{2.070576in}}%
\pgfusepath{clip}%
\pgfsetbuttcap%
\pgfsetmiterjoin%
\definecolor{currentfill}{rgb}{0.133298,0.375282,0.379395}%
\pgfsetfillcolor{currentfill}%
\pgfsetlinewidth{0.000000pt}%
\definecolor{currentstroke}{rgb}{0.000000,0.000000,0.000000}%
\pgfsetstrokecolor{currentstroke}%
\pgfsetstrokeopacity{0.000000}%
\pgfsetdash{}{0pt}%
\pgfpathmoveto{\pgfqpoint{2.223688in}{1.613090in}}%
\pgfpathlineto{\pgfqpoint{2.232441in}{1.613090in}}%
\pgfpathlineto{\pgfqpoint{2.232441in}{0.881179in}}%
\pgfpathlineto{\pgfqpoint{2.223688in}{0.881179in}}%
\pgfpathlineto{\pgfqpoint{2.223688in}{1.613090in}}%
\pgfpathclose%
\pgfusepath{fill}%
\end{pgfscope}%
\begin{pgfscope}%
\pgfpathrectangle{\pgfqpoint{0.804646in}{0.600000in}}{\pgfqpoint{2.573292in}{2.070576in}}%
\pgfusepath{clip}%
\pgfsetbuttcap%
\pgfsetmiterjoin%
\definecolor{currentfill}{rgb}{0.133298,0.375282,0.379395}%
\pgfsetfillcolor{currentfill}%
\pgfsetlinewidth{0.000000pt}%
\definecolor{currentstroke}{rgb}{0.000000,0.000000,0.000000}%
\pgfsetstrokecolor{currentstroke}%
\pgfsetstrokeopacity{0.000000}%
\pgfsetdash{}{0pt}%
\pgfpathmoveto{\pgfqpoint{2.234630in}{1.613090in}}%
\pgfpathlineto{\pgfqpoint{2.243383in}{1.613090in}}%
\pgfpathlineto{\pgfqpoint{2.243383in}{0.916336in}}%
\pgfpathlineto{\pgfqpoint{2.234630in}{0.916336in}}%
\pgfpathlineto{\pgfqpoint{2.234630in}{1.613090in}}%
\pgfpathclose%
\pgfusepath{fill}%
\end{pgfscope}%
\begin{pgfscope}%
\pgfpathrectangle{\pgfqpoint{0.804646in}{0.600000in}}{\pgfqpoint{2.573292in}{2.070576in}}%
\pgfusepath{clip}%
\pgfsetbuttcap%
\pgfsetmiterjoin%
\definecolor{currentfill}{rgb}{0.133298,0.375282,0.379395}%
\pgfsetfillcolor{currentfill}%
\pgfsetlinewidth{0.000000pt}%
\definecolor{currentstroke}{rgb}{0.000000,0.000000,0.000000}%
\pgfsetstrokecolor{currentstroke}%
\pgfsetstrokeopacity{0.000000}%
\pgfsetdash{}{0pt}%
\pgfpathmoveto{\pgfqpoint{2.245571in}{1.613090in}}%
\pgfpathlineto{\pgfqpoint{2.254325in}{1.613090in}}%
\pgfpathlineto{\pgfqpoint{2.254325in}{0.910819in}}%
\pgfpathlineto{\pgfqpoint{2.245571in}{0.910819in}}%
\pgfpathlineto{\pgfqpoint{2.245571in}{1.613090in}}%
\pgfpathclose%
\pgfusepath{fill}%
\end{pgfscope}%
\begin{pgfscope}%
\pgfpathrectangle{\pgfqpoint{0.804646in}{0.600000in}}{\pgfqpoint{2.573292in}{2.070576in}}%
\pgfusepath{clip}%
\pgfsetbuttcap%
\pgfsetmiterjoin%
\definecolor{currentfill}{rgb}{0.133298,0.375282,0.379395}%
\pgfsetfillcolor{currentfill}%
\pgfsetlinewidth{0.000000pt}%
\definecolor{currentstroke}{rgb}{0.000000,0.000000,0.000000}%
\pgfsetstrokecolor{currentstroke}%
\pgfsetstrokeopacity{0.000000}%
\pgfsetdash{}{0pt}%
\pgfpathmoveto{\pgfqpoint{2.256513in}{1.613090in}}%
\pgfpathlineto{\pgfqpoint{2.265267in}{1.613090in}}%
\pgfpathlineto{\pgfqpoint{2.265267in}{0.949211in}}%
\pgfpathlineto{\pgfqpoint{2.256513in}{0.949211in}}%
\pgfpathlineto{\pgfqpoint{2.256513in}{1.613090in}}%
\pgfpathclose%
\pgfusepath{fill}%
\end{pgfscope}%
\begin{pgfscope}%
\pgfpathrectangle{\pgfqpoint{0.804646in}{0.600000in}}{\pgfqpoint{2.573292in}{2.070576in}}%
\pgfusepath{clip}%
\pgfsetbuttcap%
\pgfsetmiterjoin%
\definecolor{currentfill}{rgb}{0.133298,0.375282,0.379395}%
\pgfsetfillcolor{currentfill}%
\pgfsetlinewidth{0.000000pt}%
\definecolor{currentstroke}{rgb}{0.000000,0.000000,0.000000}%
\pgfsetstrokecolor{currentstroke}%
\pgfsetstrokeopacity{0.000000}%
\pgfsetdash{}{0pt}%
\pgfpathmoveto{\pgfqpoint{2.267455in}{1.613090in}}%
\pgfpathlineto{\pgfqpoint{2.276208in}{1.613090in}}%
\pgfpathlineto{\pgfqpoint{2.276208in}{1.002654in}}%
\pgfpathlineto{\pgfqpoint{2.267455in}{1.002654in}}%
\pgfpathlineto{\pgfqpoint{2.267455in}{1.613090in}}%
\pgfpathclose%
\pgfusepath{fill}%
\end{pgfscope}%
\begin{pgfscope}%
\pgfpathrectangle{\pgfqpoint{0.804646in}{0.600000in}}{\pgfqpoint{2.573292in}{2.070576in}}%
\pgfusepath{clip}%
\pgfsetbuttcap%
\pgfsetmiterjoin%
\definecolor{currentfill}{rgb}{0.133298,0.375282,0.379395}%
\pgfsetfillcolor{currentfill}%
\pgfsetlinewidth{0.000000pt}%
\definecolor{currentstroke}{rgb}{0.000000,0.000000,0.000000}%
\pgfsetstrokecolor{currentstroke}%
\pgfsetstrokeopacity{0.000000}%
\pgfsetdash{}{0pt}%
\pgfpathmoveto{\pgfqpoint{2.278397in}{1.613090in}}%
\pgfpathlineto{\pgfqpoint{2.287150in}{1.613090in}}%
\pgfpathlineto{\pgfqpoint{2.287150in}{1.020994in}}%
\pgfpathlineto{\pgfqpoint{2.278397in}{1.020994in}}%
\pgfpathlineto{\pgfqpoint{2.278397in}{1.613090in}}%
\pgfpathclose%
\pgfusepath{fill}%
\end{pgfscope}%
\begin{pgfscope}%
\pgfpathrectangle{\pgfqpoint{0.804646in}{0.600000in}}{\pgfqpoint{2.573292in}{2.070576in}}%
\pgfusepath{clip}%
\pgfsetbuttcap%
\pgfsetmiterjoin%
\definecolor{currentfill}{rgb}{0.133298,0.375282,0.379395}%
\pgfsetfillcolor{currentfill}%
\pgfsetlinewidth{0.000000pt}%
\definecolor{currentstroke}{rgb}{0.000000,0.000000,0.000000}%
\pgfsetstrokecolor{currentstroke}%
\pgfsetstrokeopacity{0.000000}%
\pgfsetdash{}{0pt}%
\pgfpathmoveto{\pgfqpoint{2.289339in}{1.613090in}}%
\pgfpathlineto{\pgfqpoint{2.298092in}{1.613090in}}%
\pgfpathlineto{\pgfqpoint{2.298092in}{1.099162in}}%
\pgfpathlineto{\pgfqpoint{2.289339in}{1.099162in}}%
\pgfpathlineto{\pgfqpoint{2.289339in}{1.613090in}}%
\pgfpathclose%
\pgfusepath{fill}%
\end{pgfscope}%
\begin{pgfscope}%
\pgfpathrectangle{\pgfqpoint{0.804646in}{0.600000in}}{\pgfqpoint{2.573292in}{2.070576in}}%
\pgfusepath{clip}%
\pgfsetbuttcap%
\pgfsetmiterjoin%
\definecolor{currentfill}{rgb}{0.133298,0.375282,0.379395}%
\pgfsetfillcolor{currentfill}%
\pgfsetlinewidth{0.000000pt}%
\definecolor{currentstroke}{rgb}{0.000000,0.000000,0.000000}%
\pgfsetstrokecolor{currentstroke}%
\pgfsetstrokeopacity{0.000000}%
\pgfsetdash{}{0pt}%
\pgfpathmoveto{\pgfqpoint{2.300280in}{1.613090in}}%
\pgfpathlineto{\pgfqpoint{2.309034in}{1.613090in}}%
\pgfpathlineto{\pgfqpoint{2.309034in}{1.169971in}}%
\pgfpathlineto{\pgfqpoint{2.300280in}{1.169971in}}%
\pgfpathlineto{\pgfqpoint{2.300280in}{1.613090in}}%
\pgfpathclose%
\pgfusepath{fill}%
\end{pgfscope}%
\begin{pgfscope}%
\pgfpathrectangle{\pgfqpoint{0.804646in}{0.600000in}}{\pgfqpoint{2.573292in}{2.070576in}}%
\pgfusepath{clip}%
\pgfsetbuttcap%
\pgfsetmiterjoin%
\definecolor{currentfill}{rgb}{0.133298,0.375282,0.379395}%
\pgfsetfillcolor{currentfill}%
\pgfsetlinewidth{0.000000pt}%
\definecolor{currentstroke}{rgb}{0.000000,0.000000,0.000000}%
\pgfsetstrokecolor{currentstroke}%
\pgfsetstrokeopacity{0.000000}%
\pgfsetdash{}{0pt}%
\pgfpathmoveto{\pgfqpoint{2.311222in}{1.613090in}}%
\pgfpathlineto{\pgfqpoint{2.319976in}{1.613090in}}%
\pgfpathlineto{\pgfqpoint{2.319976in}{1.217435in}}%
\pgfpathlineto{\pgfqpoint{2.311222in}{1.217435in}}%
\pgfpathlineto{\pgfqpoint{2.311222in}{1.613090in}}%
\pgfpathclose%
\pgfusepath{fill}%
\end{pgfscope}%
\begin{pgfscope}%
\pgfpathrectangle{\pgfqpoint{0.804646in}{0.600000in}}{\pgfqpoint{2.573292in}{2.070576in}}%
\pgfusepath{clip}%
\pgfsetbuttcap%
\pgfsetmiterjoin%
\definecolor{currentfill}{rgb}{0.133298,0.375282,0.379395}%
\pgfsetfillcolor{currentfill}%
\pgfsetlinewidth{0.000000pt}%
\definecolor{currentstroke}{rgb}{0.000000,0.000000,0.000000}%
\pgfsetstrokecolor{currentstroke}%
\pgfsetstrokeopacity{0.000000}%
\pgfsetdash{}{0pt}%
\pgfpathmoveto{\pgfqpoint{2.322164in}{1.613090in}}%
\pgfpathlineto{\pgfqpoint{2.330917in}{1.613090in}}%
\pgfpathlineto{\pgfqpoint{2.330917in}{1.202649in}}%
\pgfpathlineto{\pgfqpoint{2.322164in}{1.202649in}}%
\pgfpathlineto{\pgfqpoint{2.322164in}{1.613090in}}%
\pgfpathclose%
\pgfusepath{fill}%
\end{pgfscope}%
\begin{pgfscope}%
\pgfpathrectangle{\pgfqpoint{0.804646in}{0.600000in}}{\pgfqpoint{2.573292in}{2.070576in}}%
\pgfusepath{clip}%
\pgfsetbuttcap%
\pgfsetmiterjoin%
\definecolor{currentfill}{rgb}{0.133298,0.375282,0.379395}%
\pgfsetfillcolor{currentfill}%
\pgfsetlinewidth{0.000000pt}%
\definecolor{currentstroke}{rgb}{0.000000,0.000000,0.000000}%
\pgfsetstrokecolor{currentstroke}%
\pgfsetstrokeopacity{0.000000}%
\pgfsetdash{}{0pt}%
\pgfpathmoveto{\pgfqpoint{2.333106in}{1.613090in}}%
\pgfpathlineto{\pgfqpoint{2.341859in}{1.613090in}}%
\pgfpathlineto{\pgfqpoint{2.341859in}{1.275424in}}%
\pgfpathlineto{\pgfqpoint{2.333106in}{1.275424in}}%
\pgfpathlineto{\pgfqpoint{2.333106in}{1.613090in}}%
\pgfpathclose%
\pgfusepath{fill}%
\end{pgfscope}%
\begin{pgfscope}%
\pgfpathrectangle{\pgfqpoint{0.804646in}{0.600000in}}{\pgfqpoint{2.573292in}{2.070576in}}%
\pgfusepath{clip}%
\pgfsetbuttcap%
\pgfsetmiterjoin%
\definecolor{currentfill}{rgb}{0.133298,0.375282,0.379395}%
\pgfsetfillcolor{currentfill}%
\pgfsetlinewidth{0.000000pt}%
\definecolor{currentstroke}{rgb}{0.000000,0.000000,0.000000}%
\pgfsetstrokecolor{currentstroke}%
\pgfsetstrokeopacity{0.000000}%
\pgfsetdash{}{0pt}%
\pgfpathmoveto{\pgfqpoint{2.344048in}{1.613090in}}%
\pgfpathlineto{\pgfqpoint{2.352801in}{1.613090in}}%
\pgfpathlineto{\pgfqpoint{2.352801in}{1.314476in}}%
\pgfpathlineto{\pgfqpoint{2.344048in}{1.314476in}}%
\pgfpathlineto{\pgfqpoint{2.344048in}{1.613090in}}%
\pgfpathclose%
\pgfusepath{fill}%
\end{pgfscope}%
\begin{pgfscope}%
\pgfpathrectangle{\pgfqpoint{0.804646in}{0.600000in}}{\pgfqpoint{2.573292in}{2.070576in}}%
\pgfusepath{clip}%
\pgfsetbuttcap%
\pgfsetmiterjoin%
\definecolor{currentfill}{rgb}{0.133298,0.375282,0.379395}%
\pgfsetfillcolor{currentfill}%
\pgfsetlinewidth{0.000000pt}%
\definecolor{currentstroke}{rgb}{0.000000,0.000000,0.000000}%
\pgfsetstrokecolor{currentstroke}%
\pgfsetstrokeopacity{0.000000}%
\pgfsetdash{}{0pt}%
\pgfpathmoveto{\pgfqpoint{2.354989in}{1.613090in}}%
\pgfpathlineto{\pgfqpoint{2.363743in}{1.613090in}}%
\pgfpathlineto{\pgfqpoint{2.363743in}{1.339168in}}%
\pgfpathlineto{\pgfqpoint{2.354989in}{1.339168in}}%
\pgfpathlineto{\pgfqpoint{2.354989in}{1.613090in}}%
\pgfpathclose%
\pgfusepath{fill}%
\end{pgfscope}%
\begin{pgfscope}%
\pgfpathrectangle{\pgfqpoint{0.804646in}{0.600000in}}{\pgfqpoint{2.573292in}{2.070576in}}%
\pgfusepath{clip}%
\pgfsetbuttcap%
\pgfsetmiterjoin%
\definecolor{currentfill}{rgb}{0.133298,0.375282,0.379395}%
\pgfsetfillcolor{currentfill}%
\pgfsetlinewidth{0.000000pt}%
\definecolor{currentstroke}{rgb}{0.000000,0.000000,0.000000}%
\pgfsetstrokecolor{currentstroke}%
\pgfsetstrokeopacity{0.000000}%
\pgfsetdash{}{0pt}%
\pgfpathmoveto{\pgfqpoint{2.365931in}{1.613090in}}%
\pgfpathlineto{\pgfqpoint{2.374685in}{1.613090in}}%
\pgfpathlineto{\pgfqpoint{2.374685in}{1.389443in}}%
\pgfpathlineto{\pgfqpoint{2.365931in}{1.389443in}}%
\pgfpathlineto{\pgfqpoint{2.365931in}{1.613090in}}%
\pgfpathclose%
\pgfusepath{fill}%
\end{pgfscope}%
\begin{pgfscope}%
\pgfpathrectangle{\pgfqpoint{0.804646in}{0.600000in}}{\pgfqpoint{2.573292in}{2.070576in}}%
\pgfusepath{clip}%
\pgfsetbuttcap%
\pgfsetmiterjoin%
\definecolor{currentfill}{rgb}{0.133298,0.375282,0.379395}%
\pgfsetfillcolor{currentfill}%
\pgfsetlinewidth{0.000000pt}%
\definecolor{currentstroke}{rgb}{0.000000,0.000000,0.000000}%
\pgfsetstrokecolor{currentstroke}%
\pgfsetstrokeopacity{0.000000}%
\pgfsetdash{}{0pt}%
\pgfpathmoveto{\pgfqpoint{2.376873in}{1.613090in}}%
\pgfpathlineto{\pgfqpoint{2.385626in}{1.613090in}}%
\pgfpathlineto{\pgfqpoint{2.385626in}{1.399048in}}%
\pgfpathlineto{\pgfqpoint{2.376873in}{1.399048in}}%
\pgfpathlineto{\pgfqpoint{2.376873in}{1.613090in}}%
\pgfpathclose%
\pgfusepath{fill}%
\end{pgfscope}%
\begin{pgfscope}%
\pgfpathrectangle{\pgfqpoint{0.804646in}{0.600000in}}{\pgfqpoint{2.573292in}{2.070576in}}%
\pgfusepath{clip}%
\pgfsetbuttcap%
\pgfsetmiterjoin%
\definecolor{currentfill}{rgb}{0.133298,0.375282,0.379395}%
\pgfsetfillcolor{currentfill}%
\pgfsetlinewidth{0.000000pt}%
\definecolor{currentstroke}{rgb}{0.000000,0.000000,0.000000}%
\pgfsetstrokecolor{currentstroke}%
\pgfsetstrokeopacity{0.000000}%
\pgfsetdash{}{0pt}%
\pgfpathmoveto{\pgfqpoint{2.387815in}{1.613090in}}%
\pgfpathlineto{\pgfqpoint{2.396568in}{1.613090in}}%
\pgfpathlineto{\pgfqpoint{2.396568in}{1.402040in}}%
\pgfpathlineto{\pgfqpoint{2.387815in}{1.402040in}}%
\pgfpathlineto{\pgfqpoint{2.387815in}{1.613090in}}%
\pgfpathclose%
\pgfusepath{fill}%
\end{pgfscope}%
\begin{pgfscope}%
\pgfpathrectangle{\pgfqpoint{0.804646in}{0.600000in}}{\pgfqpoint{2.573292in}{2.070576in}}%
\pgfusepath{clip}%
\pgfsetbuttcap%
\pgfsetmiterjoin%
\definecolor{currentfill}{rgb}{0.133298,0.375282,0.379395}%
\pgfsetfillcolor{currentfill}%
\pgfsetlinewidth{0.000000pt}%
\definecolor{currentstroke}{rgb}{0.000000,0.000000,0.000000}%
\pgfsetstrokecolor{currentstroke}%
\pgfsetstrokeopacity{0.000000}%
\pgfsetdash{}{0pt}%
\pgfpathmoveto{\pgfqpoint{2.398757in}{1.613090in}}%
\pgfpathlineto{\pgfqpoint{2.407510in}{1.613090in}}%
\pgfpathlineto{\pgfqpoint{2.407510in}{1.444873in}}%
\pgfpathlineto{\pgfqpoint{2.398757in}{1.444873in}}%
\pgfpathlineto{\pgfqpoint{2.398757in}{1.613090in}}%
\pgfpathclose%
\pgfusepath{fill}%
\end{pgfscope}%
\begin{pgfscope}%
\pgfpathrectangle{\pgfqpoint{0.804646in}{0.600000in}}{\pgfqpoint{2.573292in}{2.070576in}}%
\pgfusepath{clip}%
\pgfsetbuttcap%
\pgfsetmiterjoin%
\definecolor{currentfill}{rgb}{0.133298,0.375282,0.379395}%
\pgfsetfillcolor{currentfill}%
\pgfsetlinewidth{0.000000pt}%
\definecolor{currentstroke}{rgb}{0.000000,0.000000,0.000000}%
\pgfsetstrokecolor{currentstroke}%
\pgfsetstrokeopacity{0.000000}%
\pgfsetdash{}{0pt}%
\pgfpathmoveto{\pgfqpoint{2.409698in}{1.613090in}}%
\pgfpathlineto{\pgfqpoint{2.418452in}{1.613090in}}%
\pgfpathlineto{\pgfqpoint{2.418452in}{1.469018in}}%
\pgfpathlineto{\pgfqpoint{2.409698in}{1.469018in}}%
\pgfpathlineto{\pgfqpoint{2.409698in}{1.613090in}}%
\pgfpathclose%
\pgfusepath{fill}%
\end{pgfscope}%
\begin{pgfscope}%
\pgfpathrectangle{\pgfqpoint{0.804646in}{0.600000in}}{\pgfqpoint{2.573292in}{2.070576in}}%
\pgfusepath{clip}%
\pgfsetbuttcap%
\pgfsetmiterjoin%
\definecolor{currentfill}{rgb}{0.133298,0.375282,0.379395}%
\pgfsetfillcolor{currentfill}%
\pgfsetlinewidth{0.000000pt}%
\definecolor{currentstroke}{rgb}{0.000000,0.000000,0.000000}%
\pgfsetstrokecolor{currentstroke}%
\pgfsetstrokeopacity{0.000000}%
\pgfsetdash{}{0pt}%
\pgfpathmoveto{\pgfqpoint{2.420640in}{1.613090in}}%
\pgfpathlineto{\pgfqpoint{2.429394in}{1.613090in}}%
\pgfpathlineto{\pgfqpoint{2.429394in}{1.513197in}}%
\pgfpathlineto{\pgfqpoint{2.420640in}{1.513197in}}%
\pgfpathlineto{\pgfqpoint{2.420640in}{1.613090in}}%
\pgfpathclose%
\pgfusepath{fill}%
\end{pgfscope}%
\begin{pgfscope}%
\pgfpathrectangle{\pgfqpoint{0.804646in}{0.600000in}}{\pgfqpoint{2.573292in}{2.070576in}}%
\pgfusepath{clip}%
\pgfsetbuttcap%
\pgfsetmiterjoin%
\definecolor{currentfill}{rgb}{0.133298,0.375282,0.379395}%
\pgfsetfillcolor{currentfill}%
\pgfsetlinewidth{0.000000pt}%
\definecolor{currentstroke}{rgb}{0.000000,0.000000,0.000000}%
\pgfsetstrokecolor{currentstroke}%
\pgfsetstrokeopacity{0.000000}%
\pgfsetdash{}{0pt}%
\pgfpathmoveto{\pgfqpoint{2.431582in}{1.613090in}}%
\pgfpathlineto{\pgfqpoint{2.440335in}{1.613090in}}%
\pgfpathlineto{\pgfqpoint{2.440335in}{1.499663in}}%
\pgfpathlineto{\pgfqpoint{2.431582in}{1.499663in}}%
\pgfpathlineto{\pgfqpoint{2.431582in}{1.613090in}}%
\pgfpathclose%
\pgfusepath{fill}%
\end{pgfscope}%
\begin{pgfscope}%
\pgfpathrectangle{\pgfqpoint{0.804646in}{0.600000in}}{\pgfqpoint{2.573292in}{2.070576in}}%
\pgfusepath{clip}%
\pgfsetbuttcap%
\pgfsetmiterjoin%
\definecolor{currentfill}{rgb}{0.133298,0.375282,0.379395}%
\pgfsetfillcolor{currentfill}%
\pgfsetlinewidth{0.000000pt}%
\definecolor{currentstroke}{rgb}{0.000000,0.000000,0.000000}%
\pgfsetstrokecolor{currentstroke}%
\pgfsetstrokeopacity{0.000000}%
\pgfsetdash{}{0pt}%
\pgfpathmoveto{\pgfqpoint{2.442524in}{1.613090in}}%
\pgfpathlineto{\pgfqpoint{2.451277in}{1.613090in}}%
\pgfpathlineto{\pgfqpoint{2.451277in}{1.546152in}}%
\pgfpathlineto{\pgfqpoint{2.442524in}{1.546152in}}%
\pgfpathlineto{\pgfqpoint{2.442524in}{1.613090in}}%
\pgfpathclose%
\pgfusepath{fill}%
\end{pgfscope}%
\begin{pgfscope}%
\pgfpathrectangle{\pgfqpoint{0.804646in}{0.600000in}}{\pgfqpoint{2.573292in}{2.070576in}}%
\pgfusepath{clip}%
\pgfsetbuttcap%
\pgfsetmiterjoin%
\definecolor{currentfill}{rgb}{0.133298,0.375282,0.379395}%
\pgfsetfillcolor{currentfill}%
\pgfsetlinewidth{0.000000pt}%
\definecolor{currentstroke}{rgb}{0.000000,0.000000,0.000000}%
\pgfsetstrokecolor{currentstroke}%
\pgfsetstrokeopacity{0.000000}%
\pgfsetdash{}{0pt}%
\pgfpathmoveto{\pgfqpoint{2.453466in}{1.613090in}}%
\pgfpathlineto{\pgfqpoint{2.462219in}{1.613090in}}%
\pgfpathlineto{\pgfqpoint{2.462219in}{1.554201in}}%
\pgfpathlineto{\pgfqpoint{2.453466in}{1.554201in}}%
\pgfpathlineto{\pgfqpoint{2.453466in}{1.613090in}}%
\pgfpathclose%
\pgfusepath{fill}%
\end{pgfscope}%
\begin{pgfscope}%
\pgfpathrectangle{\pgfqpoint{0.804646in}{0.600000in}}{\pgfqpoint{2.573292in}{2.070576in}}%
\pgfusepath{clip}%
\pgfsetbuttcap%
\pgfsetmiterjoin%
\definecolor{currentfill}{rgb}{0.133298,0.375282,0.379395}%
\pgfsetfillcolor{currentfill}%
\pgfsetlinewidth{0.000000pt}%
\definecolor{currentstroke}{rgb}{0.000000,0.000000,0.000000}%
\pgfsetstrokecolor{currentstroke}%
\pgfsetstrokeopacity{0.000000}%
\pgfsetdash{}{0pt}%
\pgfpathmoveto{\pgfqpoint{2.464407in}{1.613090in}}%
\pgfpathlineto{\pgfqpoint{2.473161in}{1.613090in}}%
\pgfpathlineto{\pgfqpoint{2.473161in}{1.579645in}}%
\pgfpathlineto{\pgfqpoint{2.464407in}{1.579645in}}%
\pgfpathlineto{\pgfqpoint{2.464407in}{1.613090in}}%
\pgfpathclose%
\pgfusepath{fill}%
\end{pgfscope}%
\begin{pgfscope}%
\pgfpathrectangle{\pgfqpoint{0.804646in}{0.600000in}}{\pgfqpoint{2.573292in}{2.070576in}}%
\pgfusepath{clip}%
\pgfsetbuttcap%
\pgfsetmiterjoin%
\definecolor{currentfill}{rgb}{0.133298,0.375282,0.379395}%
\pgfsetfillcolor{currentfill}%
\pgfsetlinewidth{0.000000pt}%
\definecolor{currentstroke}{rgb}{0.000000,0.000000,0.000000}%
\pgfsetstrokecolor{currentstroke}%
\pgfsetstrokeopacity{0.000000}%
\pgfsetdash{}{0pt}%
\pgfpathmoveto{\pgfqpoint{2.475349in}{1.613090in}}%
\pgfpathlineto{\pgfqpoint{2.484103in}{1.613090in}}%
\pgfpathlineto{\pgfqpoint{2.484103in}{1.598018in}}%
\pgfpathlineto{\pgfqpoint{2.475349in}{1.598018in}}%
\pgfpathlineto{\pgfqpoint{2.475349in}{1.613090in}}%
\pgfpathclose%
\pgfusepath{fill}%
\end{pgfscope}%
\begin{pgfscope}%
\pgfpathrectangle{\pgfqpoint{0.804646in}{0.600000in}}{\pgfqpoint{2.573292in}{2.070576in}}%
\pgfusepath{clip}%
\pgfsetbuttcap%
\pgfsetmiterjoin%
\definecolor{currentfill}{rgb}{0.133298,0.375282,0.379395}%
\pgfsetfillcolor{currentfill}%
\pgfsetlinewidth{0.000000pt}%
\definecolor{currentstroke}{rgb}{0.000000,0.000000,0.000000}%
\pgfsetstrokecolor{currentstroke}%
\pgfsetstrokeopacity{0.000000}%
\pgfsetdash{}{0pt}%
\pgfpathmoveto{\pgfqpoint{2.486291in}{1.613090in}}%
\pgfpathlineto{\pgfqpoint{2.495044in}{1.613090in}}%
\pgfpathlineto{\pgfqpoint{2.495044in}{1.600006in}}%
\pgfpathlineto{\pgfqpoint{2.486291in}{1.600006in}}%
\pgfpathlineto{\pgfqpoint{2.486291in}{1.613090in}}%
\pgfpathclose%
\pgfusepath{fill}%
\end{pgfscope}%
\begin{pgfscope}%
\pgfpathrectangle{\pgfqpoint{0.804646in}{0.600000in}}{\pgfqpoint{2.573292in}{2.070576in}}%
\pgfusepath{clip}%
\pgfsetbuttcap%
\pgfsetmiterjoin%
\definecolor{currentfill}{rgb}{0.133298,0.375282,0.379395}%
\pgfsetfillcolor{currentfill}%
\pgfsetlinewidth{0.000000pt}%
\definecolor{currentstroke}{rgb}{0.000000,0.000000,0.000000}%
\pgfsetstrokecolor{currentstroke}%
\pgfsetstrokeopacity{0.000000}%
\pgfsetdash{}{0pt}%
\pgfpathmoveto{\pgfqpoint{2.497233in}{1.711068in}}%
\pgfpathlineto{\pgfqpoint{2.505986in}{1.711068in}}%
\pgfpathlineto{\pgfqpoint{2.505986in}{1.725968in}}%
\pgfpathlineto{\pgfqpoint{2.497233in}{1.725968in}}%
\pgfpathlineto{\pgfqpoint{2.497233in}{1.711068in}}%
\pgfpathclose%
\pgfusepath{fill}%
\end{pgfscope}%
\begin{pgfscope}%
\pgfpathrectangle{\pgfqpoint{0.804646in}{0.600000in}}{\pgfqpoint{2.573292in}{2.070576in}}%
\pgfusepath{clip}%
\pgfsetbuttcap%
\pgfsetmiterjoin%
\definecolor{currentfill}{rgb}{0.133298,0.375282,0.379395}%
\pgfsetfillcolor{currentfill}%
\pgfsetlinewidth{0.000000pt}%
\definecolor{currentstroke}{rgb}{0.000000,0.000000,0.000000}%
\pgfsetstrokecolor{currentstroke}%
\pgfsetstrokeopacity{0.000000}%
\pgfsetdash{}{0pt}%
\pgfpathmoveto{\pgfqpoint{2.508174in}{1.701047in}}%
\pgfpathlineto{\pgfqpoint{2.516928in}{1.701047in}}%
\pgfpathlineto{\pgfqpoint{2.516928in}{1.732824in}}%
\pgfpathlineto{\pgfqpoint{2.508174in}{1.732824in}}%
\pgfpathlineto{\pgfqpoint{2.508174in}{1.701047in}}%
\pgfpathclose%
\pgfusepath{fill}%
\end{pgfscope}%
\begin{pgfscope}%
\pgfpathrectangle{\pgfqpoint{0.804646in}{0.600000in}}{\pgfqpoint{2.573292in}{2.070576in}}%
\pgfusepath{clip}%
\pgfsetbuttcap%
\pgfsetmiterjoin%
\definecolor{currentfill}{rgb}{0.133298,0.375282,0.379395}%
\pgfsetfillcolor{currentfill}%
\pgfsetlinewidth{0.000000pt}%
\definecolor{currentstroke}{rgb}{0.000000,0.000000,0.000000}%
\pgfsetstrokecolor{currentstroke}%
\pgfsetstrokeopacity{0.000000}%
\pgfsetdash{}{0pt}%
\pgfpathmoveto{\pgfqpoint{2.519116in}{1.688431in}}%
\pgfpathlineto{\pgfqpoint{2.527870in}{1.688431in}}%
\pgfpathlineto{\pgfqpoint{2.527870in}{1.739486in}}%
\pgfpathlineto{\pgfqpoint{2.519116in}{1.739486in}}%
\pgfpathlineto{\pgfqpoint{2.519116in}{1.688431in}}%
\pgfpathclose%
\pgfusepath{fill}%
\end{pgfscope}%
\begin{pgfscope}%
\pgfpathrectangle{\pgfqpoint{0.804646in}{0.600000in}}{\pgfqpoint{2.573292in}{2.070576in}}%
\pgfusepath{clip}%
\pgfsetbuttcap%
\pgfsetmiterjoin%
\definecolor{currentfill}{rgb}{0.133298,0.375282,0.379395}%
\pgfsetfillcolor{currentfill}%
\pgfsetlinewidth{0.000000pt}%
\definecolor{currentstroke}{rgb}{0.000000,0.000000,0.000000}%
\pgfsetstrokecolor{currentstroke}%
\pgfsetstrokeopacity{0.000000}%
\pgfsetdash{}{0pt}%
\pgfpathmoveto{\pgfqpoint{2.530058in}{1.678223in}}%
\pgfpathlineto{\pgfqpoint{2.538812in}{1.678223in}}%
\pgfpathlineto{\pgfqpoint{2.538812in}{1.722688in}}%
\pgfpathlineto{\pgfqpoint{2.530058in}{1.722688in}}%
\pgfpathlineto{\pgfqpoint{2.530058in}{1.678223in}}%
\pgfpathclose%
\pgfusepath{fill}%
\end{pgfscope}%
\begin{pgfscope}%
\pgfpathrectangle{\pgfqpoint{0.804646in}{0.600000in}}{\pgfqpoint{2.573292in}{2.070576in}}%
\pgfusepath{clip}%
\pgfsetbuttcap%
\pgfsetmiterjoin%
\definecolor{currentfill}{rgb}{0.133298,0.375282,0.379395}%
\pgfsetfillcolor{currentfill}%
\pgfsetlinewidth{0.000000pt}%
\definecolor{currentstroke}{rgb}{0.000000,0.000000,0.000000}%
\pgfsetstrokecolor{currentstroke}%
\pgfsetstrokeopacity{0.000000}%
\pgfsetdash{}{0pt}%
\pgfpathmoveto{\pgfqpoint{2.541000in}{1.663859in}}%
\pgfpathlineto{\pgfqpoint{2.549753in}{1.663859in}}%
\pgfpathlineto{\pgfqpoint{2.549753in}{1.718259in}}%
\pgfpathlineto{\pgfqpoint{2.541000in}{1.718259in}}%
\pgfpathlineto{\pgfqpoint{2.541000in}{1.663859in}}%
\pgfpathclose%
\pgfusepath{fill}%
\end{pgfscope}%
\begin{pgfscope}%
\pgfpathrectangle{\pgfqpoint{0.804646in}{0.600000in}}{\pgfqpoint{2.573292in}{2.070576in}}%
\pgfusepath{clip}%
\pgfsetbuttcap%
\pgfsetmiterjoin%
\definecolor{currentfill}{rgb}{0.133298,0.375282,0.379395}%
\pgfsetfillcolor{currentfill}%
\pgfsetlinewidth{0.000000pt}%
\definecolor{currentstroke}{rgb}{0.000000,0.000000,0.000000}%
\pgfsetstrokecolor{currentstroke}%
\pgfsetstrokeopacity{0.000000}%
\pgfsetdash{}{0pt}%
\pgfpathmoveto{\pgfqpoint{2.551942in}{1.613090in}}%
\pgfpathlineto{\pgfqpoint{2.560695in}{1.613090in}}%
\pgfpathlineto{\pgfqpoint{2.560695in}{1.601758in}}%
\pgfpathlineto{\pgfqpoint{2.551942in}{1.601758in}}%
\pgfpathlineto{\pgfqpoint{2.551942in}{1.613090in}}%
\pgfpathclose%
\pgfusepath{fill}%
\end{pgfscope}%
\begin{pgfscope}%
\pgfpathrectangle{\pgfqpoint{0.804646in}{0.600000in}}{\pgfqpoint{2.573292in}{2.070576in}}%
\pgfusepath{clip}%
\pgfsetbuttcap%
\pgfsetmiterjoin%
\definecolor{currentfill}{rgb}{0.133298,0.375282,0.379395}%
\pgfsetfillcolor{currentfill}%
\pgfsetlinewidth{0.000000pt}%
\definecolor{currentstroke}{rgb}{0.000000,0.000000,0.000000}%
\pgfsetstrokecolor{currentstroke}%
\pgfsetstrokeopacity{0.000000}%
\pgfsetdash{}{0pt}%
\pgfpathmoveto{\pgfqpoint{2.562883in}{1.613090in}}%
\pgfpathlineto{\pgfqpoint{2.571637in}{1.613090in}}%
\pgfpathlineto{\pgfqpoint{2.571637in}{1.595029in}}%
\pgfpathlineto{\pgfqpoint{2.562883in}{1.595029in}}%
\pgfpathlineto{\pgfqpoint{2.562883in}{1.613090in}}%
\pgfpathclose%
\pgfusepath{fill}%
\end{pgfscope}%
\begin{pgfscope}%
\pgfpathrectangle{\pgfqpoint{0.804646in}{0.600000in}}{\pgfqpoint{2.573292in}{2.070576in}}%
\pgfusepath{clip}%
\pgfsetbuttcap%
\pgfsetmiterjoin%
\definecolor{currentfill}{rgb}{0.133298,0.375282,0.379395}%
\pgfsetfillcolor{currentfill}%
\pgfsetlinewidth{0.000000pt}%
\definecolor{currentstroke}{rgb}{0.000000,0.000000,0.000000}%
\pgfsetstrokecolor{currentstroke}%
\pgfsetstrokeopacity{0.000000}%
\pgfsetdash{}{0pt}%
\pgfpathmoveto{\pgfqpoint{2.573825in}{1.613090in}}%
\pgfpathlineto{\pgfqpoint{2.582579in}{1.613090in}}%
\pgfpathlineto{\pgfqpoint{2.582579in}{1.557572in}}%
\pgfpathlineto{\pgfqpoint{2.573825in}{1.557572in}}%
\pgfpathlineto{\pgfqpoint{2.573825in}{1.613090in}}%
\pgfpathclose%
\pgfusepath{fill}%
\end{pgfscope}%
\begin{pgfscope}%
\pgfpathrectangle{\pgfqpoint{0.804646in}{0.600000in}}{\pgfqpoint{2.573292in}{2.070576in}}%
\pgfusepath{clip}%
\pgfsetbuttcap%
\pgfsetmiterjoin%
\definecolor{currentfill}{rgb}{0.133298,0.375282,0.379395}%
\pgfsetfillcolor{currentfill}%
\pgfsetlinewidth{0.000000pt}%
\definecolor{currentstroke}{rgb}{0.000000,0.000000,0.000000}%
\pgfsetstrokecolor{currentstroke}%
\pgfsetstrokeopacity{0.000000}%
\pgfsetdash{}{0pt}%
\pgfpathmoveto{\pgfqpoint{2.584767in}{1.613090in}}%
\pgfpathlineto{\pgfqpoint{2.593521in}{1.613090in}}%
\pgfpathlineto{\pgfqpoint{2.593521in}{1.571129in}}%
\pgfpathlineto{\pgfqpoint{2.584767in}{1.571129in}}%
\pgfpathlineto{\pgfqpoint{2.584767in}{1.613090in}}%
\pgfpathclose%
\pgfusepath{fill}%
\end{pgfscope}%
\begin{pgfscope}%
\pgfpathrectangle{\pgfqpoint{0.804646in}{0.600000in}}{\pgfqpoint{2.573292in}{2.070576in}}%
\pgfusepath{clip}%
\pgfsetbuttcap%
\pgfsetmiterjoin%
\definecolor{currentfill}{rgb}{0.133298,0.375282,0.379395}%
\pgfsetfillcolor{currentfill}%
\pgfsetlinewidth{0.000000pt}%
\definecolor{currentstroke}{rgb}{0.000000,0.000000,0.000000}%
\pgfsetstrokecolor{currentstroke}%
\pgfsetstrokeopacity{0.000000}%
\pgfsetdash{}{0pt}%
\pgfpathmoveto{\pgfqpoint{2.595709in}{1.613090in}}%
\pgfpathlineto{\pgfqpoint{2.604462in}{1.613090in}}%
\pgfpathlineto{\pgfqpoint{2.604462in}{1.555425in}}%
\pgfpathlineto{\pgfqpoint{2.595709in}{1.555425in}}%
\pgfpathlineto{\pgfqpoint{2.595709in}{1.613090in}}%
\pgfpathclose%
\pgfusepath{fill}%
\end{pgfscope}%
\begin{pgfscope}%
\pgfpathrectangle{\pgfqpoint{0.804646in}{0.600000in}}{\pgfqpoint{2.573292in}{2.070576in}}%
\pgfusepath{clip}%
\pgfsetbuttcap%
\pgfsetmiterjoin%
\definecolor{currentfill}{rgb}{0.133298,0.375282,0.379395}%
\pgfsetfillcolor{currentfill}%
\pgfsetlinewidth{0.000000pt}%
\definecolor{currentstroke}{rgb}{0.000000,0.000000,0.000000}%
\pgfsetstrokecolor{currentstroke}%
\pgfsetstrokeopacity{0.000000}%
\pgfsetdash{}{0pt}%
\pgfpathmoveto{\pgfqpoint{2.606651in}{1.613090in}}%
\pgfpathlineto{\pgfqpoint{2.615404in}{1.613090in}}%
\pgfpathlineto{\pgfqpoint{2.615404in}{1.527391in}}%
\pgfpathlineto{\pgfqpoint{2.606651in}{1.527391in}}%
\pgfpathlineto{\pgfqpoint{2.606651in}{1.613090in}}%
\pgfpathclose%
\pgfusepath{fill}%
\end{pgfscope}%
\begin{pgfscope}%
\pgfpathrectangle{\pgfqpoint{0.804646in}{0.600000in}}{\pgfqpoint{2.573292in}{2.070576in}}%
\pgfusepath{clip}%
\pgfsetbuttcap%
\pgfsetmiterjoin%
\definecolor{currentfill}{rgb}{0.133298,0.375282,0.379395}%
\pgfsetfillcolor{currentfill}%
\pgfsetlinewidth{0.000000pt}%
\definecolor{currentstroke}{rgb}{0.000000,0.000000,0.000000}%
\pgfsetstrokecolor{currentstroke}%
\pgfsetstrokeopacity{0.000000}%
\pgfsetdash{}{0pt}%
\pgfpathmoveto{\pgfqpoint{2.617592in}{1.613090in}}%
\pgfpathlineto{\pgfqpoint{2.626346in}{1.613090in}}%
\pgfpathlineto{\pgfqpoint{2.626346in}{1.533898in}}%
\pgfpathlineto{\pgfqpoint{2.617592in}{1.533898in}}%
\pgfpathlineto{\pgfqpoint{2.617592in}{1.613090in}}%
\pgfpathclose%
\pgfusepath{fill}%
\end{pgfscope}%
\begin{pgfscope}%
\pgfpathrectangle{\pgfqpoint{0.804646in}{0.600000in}}{\pgfqpoint{2.573292in}{2.070576in}}%
\pgfusepath{clip}%
\pgfsetbuttcap%
\pgfsetmiterjoin%
\definecolor{currentfill}{rgb}{0.133298,0.375282,0.379395}%
\pgfsetfillcolor{currentfill}%
\pgfsetlinewidth{0.000000pt}%
\definecolor{currentstroke}{rgb}{0.000000,0.000000,0.000000}%
\pgfsetstrokecolor{currentstroke}%
\pgfsetstrokeopacity{0.000000}%
\pgfsetdash{}{0pt}%
\pgfpathmoveto{\pgfqpoint{2.628534in}{1.613090in}}%
\pgfpathlineto{\pgfqpoint{2.637288in}{1.613090in}}%
\pgfpathlineto{\pgfqpoint{2.637288in}{1.514359in}}%
\pgfpathlineto{\pgfqpoint{2.628534in}{1.514359in}}%
\pgfpathlineto{\pgfqpoint{2.628534in}{1.613090in}}%
\pgfpathclose%
\pgfusepath{fill}%
\end{pgfscope}%
\begin{pgfscope}%
\pgfpathrectangle{\pgfqpoint{0.804646in}{0.600000in}}{\pgfqpoint{2.573292in}{2.070576in}}%
\pgfusepath{clip}%
\pgfsetbuttcap%
\pgfsetmiterjoin%
\definecolor{currentfill}{rgb}{0.133298,0.375282,0.379395}%
\pgfsetfillcolor{currentfill}%
\pgfsetlinewidth{0.000000pt}%
\definecolor{currentstroke}{rgb}{0.000000,0.000000,0.000000}%
\pgfsetstrokecolor{currentstroke}%
\pgfsetstrokeopacity{0.000000}%
\pgfsetdash{}{0pt}%
\pgfpathmoveto{\pgfqpoint{2.639476in}{1.613090in}}%
\pgfpathlineto{\pgfqpoint{2.648230in}{1.613090in}}%
\pgfpathlineto{\pgfqpoint{2.648230in}{1.507549in}}%
\pgfpathlineto{\pgfqpoint{2.639476in}{1.507549in}}%
\pgfpathlineto{\pgfqpoint{2.639476in}{1.613090in}}%
\pgfpathclose%
\pgfusepath{fill}%
\end{pgfscope}%
\begin{pgfscope}%
\pgfpathrectangle{\pgfqpoint{0.804646in}{0.600000in}}{\pgfqpoint{2.573292in}{2.070576in}}%
\pgfusepath{clip}%
\pgfsetbuttcap%
\pgfsetmiterjoin%
\definecolor{currentfill}{rgb}{0.133298,0.375282,0.379395}%
\pgfsetfillcolor{currentfill}%
\pgfsetlinewidth{0.000000pt}%
\definecolor{currentstroke}{rgb}{0.000000,0.000000,0.000000}%
\pgfsetstrokecolor{currentstroke}%
\pgfsetstrokeopacity{0.000000}%
\pgfsetdash{}{0pt}%
\pgfpathmoveto{\pgfqpoint{2.650418in}{1.613090in}}%
\pgfpathlineto{\pgfqpoint{2.659171in}{1.613090in}}%
\pgfpathlineto{\pgfqpoint{2.659171in}{1.517832in}}%
\pgfpathlineto{\pgfqpoint{2.650418in}{1.517832in}}%
\pgfpathlineto{\pgfqpoint{2.650418in}{1.613090in}}%
\pgfpathclose%
\pgfusepath{fill}%
\end{pgfscope}%
\begin{pgfscope}%
\pgfpathrectangle{\pgfqpoint{0.804646in}{0.600000in}}{\pgfqpoint{2.573292in}{2.070576in}}%
\pgfusepath{clip}%
\pgfsetbuttcap%
\pgfsetmiterjoin%
\definecolor{currentfill}{rgb}{0.133298,0.375282,0.379395}%
\pgfsetfillcolor{currentfill}%
\pgfsetlinewidth{0.000000pt}%
\definecolor{currentstroke}{rgb}{0.000000,0.000000,0.000000}%
\pgfsetstrokecolor{currentstroke}%
\pgfsetstrokeopacity{0.000000}%
\pgfsetdash{}{0pt}%
\pgfpathmoveto{\pgfqpoint{2.661360in}{1.613090in}}%
\pgfpathlineto{\pgfqpoint{2.670113in}{1.613090in}}%
\pgfpathlineto{\pgfqpoint{2.670113in}{1.554084in}}%
\pgfpathlineto{\pgfqpoint{2.661360in}{1.554084in}}%
\pgfpathlineto{\pgfqpoint{2.661360in}{1.613090in}}%
\pgfpathclose%
\pgfusepath{fill}%
\end{pgfscope}%
\begin{pgfscope}%
\pgfpathrectangle{\pgfqpoint{0.804646in}{0.600000in}}{\pgfqpoint{2.573292in}{2.070576in}}%
\pgfusepath{clip}%
\pgfsetbuttcap%
\pgfsetmiterjoin%
\definecolor{currentfill}{rgb}{0.133298,0.375282,0.379395}%
\pgfsetfillcolor{currentfill}%
\pgfsetlinewidth{0.000000pt}%
\definecolor{currentstroke}{rgb}{0.000000,0.000000,0.000000}%
\pgfsetstrokecolor{currentstroke}%
\pgfsetstrokeopacity{0.000000}%
\pgfsetdash{}{0pt}%
\pgfpathmoveto{\pgfqpoint{2.672301in}{1.613090in}}%
\pgfpathlineto{\pgfqpoint{2.681055in}{1.613090in}}%
\pgfpathlineto{\pgfqpoint{2.681055in}{1.556576in}}%
\pgfpathlineto{\pgfqpoint{2.672301in}{1.556576in}}%
\pgfpathlineto{\pgfqpoint{2.672301in}{1.613090in}}%
\pgfpathclose%
\pgfusepath{fill}%
\end{pgfscope}%
\begin{pgfscope}%
\pgfpathrectangle{\pgfqpoint{0.804646in}{0.600000in}}{\pgfqpoint{2.573292in}{2.070576in}}%
\pgfusepath{clip}%
\pgfsetbuttcap%
\pgfsetmiterjoin%
\definecolor{currentfill}{rgb}{0.133298,0.375282,0.379395}%
\pgfsetfillcolor{currentfill}%
\pgfsetlinewidth{0.000000pt}%
\definecolor{currentstroke}{rgb}{0.000000,0.000000,0.000000}%
\pgfsetstrokecolor{currentstroke}%
\pgfsetstrokeopacity{0.000000}%
\pgfsetdash{}{0pt}%
\pgfpathmoveto{\pgfqpoint{2.683243in}{1.718405in}}%
\pgfpathlineto{\pgfqpoint{2.691997in}{1.718405in}}%
\pgfpathlineto{\pgfqpoint{2.691997in}{1.723966in}}%
\pgfpathlineto{\pgfqpoint{2.683243in}{1.723966in}}%
\pgfpathlineto{\pgfqpoint{2.683243in}{1.718405in}}%
\pgfpathclose%
\pgfusepath{fill}%
\end{pgfscope}%
\begin{pgfscope}%
\pgfpathrectangle{\pgfqpoint{0.804646in}{0.600000in}}{\pgfqpoint{2.573292in}{2.070576in}}%
\pgfusepath{clip}%
\pgfsetbuttcap%
\pgfsetmiterjoin%
\definecolor{currentfill}{rgb}{0.133298,0.375282,0.379395}%
\pgfsetfillcolor{currentfill}%
\pgfsetlinewidth{0.000000pt}%
\definecolor{currentstroke}{rgb}{0.000000,0.000000,0.000000}%
\pgfsetstrokecolor{currentstroke}%
\pgfsetstrokeopacity{0.000000}%
\pgfsetdash{}{0pt}%
\pgfpathmoveto{\pgfqpoint{2.694185in}{1.613090in}}%
\pgfpathlineto{\pgfqpoint{2.702939in}{1.613090in}}%
\pgfpathlineto{\pgfqpoint{2.702939in}{1.562505in}}%
\pgfpathlineto{\pgfqpoint{2.694185in}{1.562505in}}%
\pgfpathlineto{\pgfqpoint{2.694185in}{1.613090in}}%
\pgfpathclose%
\pgfusepath{fill}%
\end{pgfscope}%
\begin{pgfscope}%
\pgfpathrectangle{\pgfqpoint{0.804646in}{0.600000in}}{\pgfqpoint{2.573292in}{2.070576in}}%
\pgfusepath{clip}%
\pgfsetbuttcap%
\pgfsetmiterjoin%
\definecolor{currentfill}{rgb}{0.133298,0.375282,0.379395}%
\pgfsetfillcolor{currentfill}%
\pgfsetlinewidth{0.000000pt}%
\definecolor{currentstroke}{rgb}{0.000000,0.000000,0.000000}%
\pgfsetstrokecolor{currentstroke}%
\pgfsetstrokeopacity{0.000000}%
\pgfsetdash{}{0pt}%
\pgfpathmoveto{\pgfqpoint{2.705127in}{1.613090in}}%
\pgfpathlineto{\pgfqpoint{2.713880in}{1.613090in}}%
\pgfpathlineto{\pgfqpoint{2.713880in}{1.607014in}}%
\pgfpathlineto{\pgfqpoint{2.705127in}{1.607014in}}%
\pgfpathlineto{\pgfqpoint{2.705127in}{1.613090in}}%
\pgfpathclose%
\pgfusepath{fill}%
\end{pgfscope}%
\begin{pgfscope}%
\pgfpathrectangle{\pgfqpoint{0.804646in}{0.600000in}}{\pgfqpoint{2.573292in}{2.070576in}}%
\pgfusepath{clip}%
\pgfsetbuttcap%
\pgfsetmiterjoin%
\definecolor{currentfill}{rgb}{0.133298,0.375282,0.379395}%
\pgfsetfillcolor{currentfill}%
\pgfsetlinewidth{0.000000pt}%
\definecolor{currentstroke}{rgb}{0.000000,0.000000,0.000000}%
\pgfsetstrokecolor{currentstroke}%
\pgfsetstrokeopacity{0.000000}%
\pgfsetdash{}{0pt}%
\pgfpathmoveto{\pgfqpoint{2.716069in}{1.694489in}}%
\pgfpathlineto{\pgfqpoint{2.724822in}{1.694489in}}%
\pgfpathlineto{\pgfqpoint{2.724822in}{1.729155in}}%
\pgfpathlineto{\pgfqpoint{2.716069in}{1.729155in}}%
\pgfpathlineto{\pgfqpoint{2.716069in}{1.694489in}}%
\pgfpathclose%
\pgfusepath{fill}%
\end{pgfscope}%
\begin{pgfscope}%
\pgfpathrectangle{\pgfqpoint{0.804646in}{0.600000in}}{\pgfqpoint{2.573292in}{2.070576in}}%
\pgfusepath{clip}%
\pgfsetbuttcap%
\pgfsetmiterjoin%
\definecolor{currentfill}{rgb}{0.133298,0.375282,0.379395}%
\pgfsetfillcolor{currentfill}%
\pgfsetlinewidth{0.000000pt}%
\definecolor{currentstroke}{rgb}{0.000000,0.000000,0.000000}%
\pgfsetstrokecolor{currentstroke}%
\pgfsetstrokeopacity{0.000000}%
\pgfsetdash{}{0pt}%
\pgfpathmoveto{\pgfqpoint{2.727010in}{1.686306in}}%
\pgfpathlineto{\pgfqpoint{2.735764in}{1.686306in}}%
\pgfpathlineto{\pgfqpoint{2.735764in}{1.726580in}}%
\pgfpathlineto{\pgfqpoint{2.727010in}{1.726580in}}%
\pgfpathlineto{\pgfqpoint{2.727010in}{1.686306in}}%
\pgfpathclose%
\pgfusepath{fill}%
\end{pgfscope}%
\begin{pgfscope}%
\pgfpathrectangle{\pgfqpoint{0.804646in}{0.600000in}}{\pgfqpoint{2.573292in}{2.070576in}}%
\pgfusepath{clip}%
\pgfsetbuttcap%
\pgfsetmiterjoin%
\definecolor{currentfill}{rgb}{0.133298,0.375282,0.379395}%
\pgfsetfillcolor{currentfill}%
\pgfsetlinewidth{0.000000pt}%
\definecolor{currentstroke}{rgb}{0.000000,0.000000,0.000000}%
\pgfsetstrokecolor{currentstroke}%
\pgfsetstrokeopacity{0.000000}%
\pgfsetdash{}{0pt}%
\pgfpathmoveto{\pgfqpoint{2.737952in}{1.670691in}}%
\pgfpathlineto{\pgfqpoint{2.746706in}{1.670691in}}%
\pgfpathlineto{\pgfqpoint{2.746706in}{1.740454in}}%
\pgfpathlineto{\pgfqpoint{2.737952in}{1.740454in}}%
\pgfpathlineto{\pgfqpoint{2.737952in}{1.670691in}}%
\pgfpathclose%
\pgfusepath{fill}%
\end{pgfscope}%
\begin{pgfscope}%
\pgfpathrectangle{\pgfqpoint{0.804646in}{0.600000in}}{\pgfqpoint{2.573292in}{2.070576in}}%
\pgfusepath{clip}%
\pgfsetbuttcap%
\pgfsetmiterjoin%
\definecolor{currentfill}{rgb}{0.133298,0.375282,0.379395}%
\pgfsetfillcolor{currentfill}%
\pgfsetlinewidth{0.000000pt}%
\definecolor{currentstroke}{rgb}{0.000000,0.000000,0.000000}%
\pgfsetstrokecolor{currentstroke}%
\pgfsetstrokeopacity{0.000000}%
\pgfsetdash{}{0pt}%
\pgfpathmoveto{\pgfqpoint{2.748894in}{1.637733in}}%
\pgfpathlineto{\pgfqpoint{2.757648in}{1.637733in}}%
\pgfpathlineto{\pgfqpoint{2.757648in}{1.728886in}}%
\pgfpathlineto{\pgfqpoint{2.748894in}{1.728886in}}%
\pgfpathlineto{\pgfqpoint{2.748894in}{1.637733in}}%
\pgfpathclose%
\pgfusepath{fill}%
\end{pgfscope}%
\begin{pgfscope}%
\pgfpathrectangle{\pgfqpoint{0.804646in}{0.600000in}}{\pgfqpoint{2.573292in}{2.070576in}}%
\pgfusepath{clip}%
\pgfsetbuttcap%
\pgfsetmiterjoin%
\definecolor{currentfill}{rgb}{0.133298,0.375282,0.379395}%
\pgfsetfillcolor{currentfill}%
\pgfsetlinewidth{0.000000pt}%
\definecolor{currentstroke}{rgb}{0.000000,0.000000,0.000000}%
\pgfsetstrokecolor{currentstroke}%
\pgfsetstrokeopacity{0.000000}%
\pgfsetdash{}{0pt}%
\pgfpathmoveto{\pgfqpoint{2.759836in}{1.613090in}}%
\pgfpathlineto{\pgfqpoint{2.768589in}{1.613090in}}%
\pgfpathlineto{\pgfqpoint{2.768589in}{1.682606in}}%
\pgfpathlineto{\pgfqpoint{2.759836in}{1.682606in}}%
\pgfpathlineto{\pgfqpoint{2.759836in}{1.613090in}}%
\pgfpathclose%
\pgfusepath{fill}%
\end{pgfscope}%
\begin{pgfscope}%
\pgfpathrectangle{\pgfqpoint{0.804646in}{0.600000in}}{\pgfqpoint{2.573292in}{2.070576in}}%
\pgfusepath{clip}%
\pgfsetbuttcap%
\pgfsetmiterjoin%
\definecolor{currentfill}{rgb}{0.133298,0.375282,0.379395}%
\pgfsetfillcolor{currentfill}%
\pgfsetlinewidth{0.000000pt}%
\definecolor{currentstroke}{rgb}{0.000000,0.000000,0.000000}%
\pgfsetstrokecolor{currentstroke}%
\pgfsetstrokeopacity{0.000000}%
\pgfsetdash{}{0pt}%
\pgfpathmoveto{\pgfqpoint{2.770778in}{1.613090in}}%
\pgfpathlineto{\pgfqpoint{2.779531in}{1.613090in}}%
\pgfpathlineto{\pgfqpoint{2.779531in}{1.741929in}}%
\pgfpathlineto{\pgfqpoint{2.770778in}{1.741929in}}%
\pgfpathlineto{\pgfqpoint{2.770778in}{1.613090in}}%
\pgfpathclose%
\pgfusepath{fill}%
\end{pgfscope}%
\begin{pgfscope}%
\pgfpathrectangle{\pgfqpoint{0.804646in}{0.600000in}}{\pgfqpoint{2.573292in}{2.070576in}}%
\pgfusepath{clip}%
\pgfsetbuttcap%
\pgfsetmiterjoin%
\definecolor{currentfill}{rgb}{0.133298,0.375282,0.379395}%
\pgfsetfillcolor{currentfill}%
\pgfsetlinewidth{0.000000pt}%
\definecolor{currentstroke}{rgb}{0.000000,0.000000,0.000000}%
\pgfsetstrokecolor{currentstroke}%
\pgfsetstrokeopacity{0.000000}%
\pgfsetdash{}{0pt}%
\pgfpathmoveto{\pgfqpoint{2.781719in}{1.613090in}}%
\pgfpathlineto{\pgfqpoint{2.790473in}{1.613090in}}%
\pgfpathlineto{\pgfqpoint{2.790473in}{1.820095in}}%
\pgfpathlineto{\pgfqpoint{2.781719in}{1.820095in}}%
\pgfpathlineto{\pgfqpoint{2.781719in}{1.613090in}}%
\pgfpathclose%
\pgfusepath{fill}%
\end{pgfscope}%
\begin{pgfscope}%
\pgfpathrectangle{\pgfqpoint{0.804646in}{0.600000in}}{\pgfqpoint{2.573292in}{2.070576in}}%
\pgfusepath{clip}%
\pgfsetbuttcap%
\pgfsetmiterjoin%
\definecolor{currentfill}{rgb}{0.133298,0.375282,0.379395}%
\pgfsetfillcolor{currentfill}%
\pgfsetlinewidth{0.000000pt}%
\definecolor{currentstroke}{rgb}{0.000000,0.000000,0.000000}%
\pgfsetstrokecolor{currentstroke}%
\pgfsetstrokeopacity{0.000000}%
\pgfsetdash{}{0pt}%
\pgfpathmoveto{\pgfqpoint{2.792661in}{1.613090in}}%
\pgfpathlineto{\pgfqpoint{2.801415in}{1.613090in}}%
\pgfpathlineto{\pgfqpoint{2.801415in}{1.882435in}}%
\pgfpathlineto{\pgfqpoint{2.792661in}{1.882435in}}%
\pgfpathlineto{\pgfqpoint{2.792661in}{1.613090in}}%
\pgfpathclose%
\pgfusepath{fill}%
\end{pgfscope}%
\begin{pgfscope}%
\pgfpathrectangle{\pgfqpoint{0.804646in}{0.600000in}}{\pgfqpoint{2.573292in}{2.070576in}}%
\pgfusepath{clip}%
\pgfsetbuttcap%
\pgfsetmiterjoin%
\definecolor{currentfill}{rgb}{0.133298,0.375282,0.379395}%
\pgfsetfillcolor{currentfill}%
\pgfsetlinewidth{0.000000pt}%
\definecolor{currentstroke}{rgb}{0.000000,0.000000,0.000000}%
\pgfsetstrokecolor{currentstroke}%
\pgfsetstrokeopacity{0.000000}%
\pgfsetdash{}{0pt}%
\pgfpathmoveto{\pgfqpoint{2.803603in}{1.613090in}}%
\pgfpathlineto{\pgfqpoint{2.812357in}{1.613090in}}%
\pgfpathlineto{\pgfqpoint{2.812357in}{1.867777in}}%
\pgfpathlineto{\pgfqpoint{2.803603in}{1.867777in}}%
\pgfpathlineto{\pgfqpoint{2.803603in}{1.613090in}}%
\pgfpathclose%
\pgfusepath{fill}%
\end{pgfscope}%
\begin{pgfscope}%
\pgfpathrectangle{\pgfqpoint{0.804646in}{0.600000in}}{\pgfqpoint{2.573292in}{2.070576in}}%
\pgfusepath{clip}%
\pgfsetbuttcap%
\pgfsetmiterjoin%
\definecolor{currentfill}{rgb}{0.133298,0.375282,0.379395}%
\pgfsetfillcolor{currentfill}%
\pgfsetlinewidth{0.000000pt}%
\definecolor{currentstroke}{rgb}{0.000000,0.000000,0.000000}%
\pgfsetstrokecolor{currentstroke}%
\pgfsetstrokeopacity{0.000000}%
\pgfsetdash{}{0pt}%
\pgfpathmoveto{\pgfqpoint{2.814545in}{1.613090in}}%
\pgfpathlineto{\pgfqpoint{2.823298in}{1.613090in}}%
\pgfpathlineto{\pgfqpoint{2.823298in}{1.898692in}}%
\pgfpathlineto{\pgfqpoint{2.814545in}{1.898692in}}%
\pgfpathlineto{\pgfqpoint{2.814545in}{1.613090in}}%
\pgfpathclose%
\pgfusepath{fill}%
\end{pgfscope}%
\begin{pgfscope}%
\pgfpathrectangle{\pgfqpoint{0.804646in}{0.600000in}}{\pgfqpoint{2.573292in}{2.070576in}}%
\pgfusepath{clip}%
\pgfsetbuttcap%
\pgfsetmiterjoin%
\definecolor{currentfill}{rgb}{0.133298,0.375282,0.379395}%
\pgfsetfillcolor{currentfill}%
\pgfsetlinewidth{0.000000pt}%
\definecolor{currentstroke}{rgb}{0.000000,0.000000,0.000000}%
\pgfsetstrokecolor{currentstroke}%
\pgfsetstrokeopacity{0.000000}%
\pgfsetdash{}{0pt}%
\pgfpathmoveto{\pgfqpoint{2.825487in}{1.613090in}}%
\pgfpathlineto{\pgfqpoint{2.834240in}{1.613090in}}%
\pgfpathlineto{\pgfqpoint{2.834240in}{1.931853in}}%
\pgfpathlineto{\pgfqpoint{2.825487in}{1.931853in}}%
\pgfpathlineto{\pgfqpoint{2.825487in}{1.613090in}}%
\pgfpathclose%
\pgfusepath{fill}%
\end{pgfscope}%
\begin{pgfscope}%
\pgfpathrectangle{\pgfqpoint{0.804646in}{0.600000in}}{\pgfqpoint{2.573292in}{2.070576in}}%
\pgfusepath{clip}%
\pgfsetbuttcap%
\pgfsetmiterjoin%
\definecolor{currentfill}{rgb}{0.133298,0.375282,0.379395}%
\pgfsetfillcolor{currentfill}%
\pgfsetlinewidth{0.000000pt}%
\definecolor{currentstroke}{rgb}{0.000000,0.000000,0.000000}%
\pgfsetstrokecolor{currentstroke}%
\pgfsetstrokeopacity{0.000000}%
\pgfsetdash{}{0pt}%
\pgfpathmoveto{\pgfqpoint{2.836428in}{1.613090in}}%
\pgfpathlineto{\pgfqpoint{2.845182in}{1.613090in}}%
\pgfpathlineto{\pgfqpoint{2.845182in}{1.902484in}}%
\pgfpathlineto{\pgfqpoint{2.836428in}{1.902484in}}%
\pgfpathlineto{\pgfqpoint{2.836428in}{1.613090in}}%
\pgfpathclose%
\pgfusepath{fill}%
\end{pgfscope}%
\begin{pgfscope}%
\pgfpathrectangle{\pgfqpoint{0.804646in}{0.600000in}}{\pgfqpoint{2.573292in}{2.070576in}}%
\pgfusepath{clip}%
\pgfsetbuttcap%
\pgfsetmiterjoin%
\definecolor{currentfill}{rgb}{0.133298,0.375282,0.379395}%
\pgfsetfillcolor{currentfill}%
\pgfsetlinewidth{0.000000pt}%
\definecolor{currentstroke}{rgb}{0.000000,0.000000,0.000000}%
\pgfsetstrokecolor{currentstroke}%
\pgfsetstrokeopacity{0.000000}%
\pgfsetdash{}{0pt}%
\pgfpathmoveto{\pgfqpoint{2.847370in}{1.613090in}}%
\pgfpathlineto{\pgfqpoint{2.856124in}{1.613090in}}%
\pgfpathlineto{\pgfqpoint{2.856124in}{1.924146in}}%
\pgfpathlineto{\pgfqpoint{2.847370in}{1.924146in}}%
\pgfpathlineto{\pgfqpoint{2.847370in}{1.613090in}}%
\pgfpathclose%
\pgfusepath{fill}%
\end{pgfscope}%
\begin{pgfscope}%
\pgfpathrectangle{\pgfqpoint{0.804646in}{0.600000in}}{\pgfqpoint{2.573292in}{2.070576in}}%
\pgfusepath{clip}%
\pgfsetbuttcap%
\pgfsetmiterjoin%
\definecolor{currentfill}{rgb}{0.133298,0.375282,0.379395}%
\pgfsetfillcolor{currentfill}%
\pgfsetlinewidth{0.000000pt}%
\definecolor{currentstroke}{rgb}{0.000000,0.000000,0.000000}%
\pgfsetstrokecolor{currentstroke}%
\pgfsetstrokeopacity{0.000000}%
\pgfsetdash{}{0pt}%
\pgfpathmoveto{\pgfqpoint{2.858312in}{1.613090in}}%
\pgfpathlineto{\pgfqpoint{2.867066in}{1.613090in}}%
\pgfpathlineto{\pgfqpoint{2.867066in}{1.929941in}}%
\pgfpathlineto{\pgfqpoint{2.858312in}{1.929941in}}%
\pgfpathlineto{\pgfqpoint{2.858312in}{1.613090in}}%
\pgfpathclose%
\pgfusepath{fill}%
\end{pgfscope}%
\begin{pgfscope}%
\pgfpathrectangle{\pgfqpoint{0.804646in}{0.600000in}}{\pgfqpoint{2.573292in}{2.070576in}}%
\pgfusepath{clip}%
\pgfsetbuttcap%
\pgfsetmiterjoin%
\definecolor{currentfill}{rgb}{0.133298,0.375282,0.379395}%
\pgfsetfillcolor{currentfill}%
\pgfsetlinewidth{0.000000pt}%
\definecolor{currentstroke}{rgb}{0.000000,0.000000,0.000000}%
\pgfsetstrokecolor{currentstroke}%
\pgfsetstrokeopacity{0.000000}%
\pgfsetdash{}{0pt}%
\pgfpathmoveto{\pgfqpoint{2.869254in}{1.613090in}}%
\pgfpathlineto{\pgfqpoint{2.878007in}{1.613090in}}%
\pgfpathlineto{\pgfqpoint{2.878007in}{1.944193in}}%
\pgfpathlineto{\pgfqpoint{2.869254in}{1.944193in}}%
\pgfpathlineto{\pgfqpoint{2.869254in}{1.613090in}}%
\pgfpathclose%
\pgfusepath{fill}%
\end{pgfscope}%
\begin{pgfscope}%
\pgfpathrectangle{\pgfqpoint{0.804646in}{0.600000in}}{\pgfqpoint{2.573292in}{2.070576in}}%
\pgfusepath{clip}%
\pgfsetbuttcap%
\pgfsetmiterjoin%
\definecolor{currentfill}{rgb}{0.133298,0.375282,0.379395}%
\pgfsetfillcolor{currentfill}%
\pgfsetlinewidth{0.000000pt}%
\definecolor{currentstroke}{rgb}{0.000000,0.000000,0.000000}%
\pgfsetstrokecolor{currentstroke}%
\pgfsetstrokeopacity{0.000000}%
\pgfsetdash{}{0pt}%
\pgfpathmoveto{\pgfqpoint{2.880196in}{1.613090in}}%
\pgfpathlineto{\pgfqpoint{2.888949in}{1.613090in}}%
\pgfpathlineto{\pgfqpoint{2.888949in}{1.878645in}}%
\pgfpathlineto{\pgfqpoint{2.880196in}{1.878645in}}%
\pgfpathlineto{\pgfqpoint{2.880196in}{1.613090in}}%
\pgfpathclose%
\pgfusepath{fill}%
\end{pgfscope}%
\begin{pgfscope}%
\pgfpathrectangle{\pgfqpoint{0.804646in}{0.600000in}}{\pgfqpoint{2.573292in}{2.070576in}}%
\pgfusepath{clip}%
\pgfsetbuttcap%
\pgfsetmiterjoin%
\definecolor{currentfill}{rgb}{0.133298,0.375282,0.379395}%
\pgfsetfillcolor{currentfill}%
\pgfsetlinewidth{0.000000pt}%
\definecolor{currentstroke}{rgb}{0.000000,0.000000,0.000000}%
\pgfsetstrokecolor{currentstroke}%
\pgfsetstrokeopacity{0.000000}%
\pgfsetdash{}{0pt}%
\pgfpathmoveto{\pgfqpoint{2.891137in}{1.613090in}}%
\pgfpathlineto{\pgfqpoint{2.899891in}{1.613090in}}%
\pgfpathlineto{\pgfqpoint{2.899891in}{1.879060in}}%
\pgfpathlineto{\pgfqpoint{2.891137in}{1.879060in}}%
\pgfpathlineto{\pgfqpoint{2.891137in}{1.613090in}}%
\pgfpathclose%
\pgfusepath{fill}%
\end{pgfscope}%
\begin{pgfscope}%
\pgfpathrectangle{\pgfqpoint{0.804646in}{0.600000in}}{\pgfqpoint{2.573292in}{2.070576in}}%
\pgfusepath{clip}%
\pgfsetbuttcap%
\pgfsetmiterjoin%
\definecolor{currentfill}{rgb}{0.133298,0.375282,0.379395}%
\pgfsetfillcolor{currentfill}%
\pgfsetlinewidth{0.000000pt}%
\definecolor{currentstroke}{rgb}{0.000000,0.000000,0.000000}%
\pgfsetstrokecolor{currentstroke}%
\pgfsetstrokeopacity{0.000000}%
\pgfsetdash{}{0pt}%
\pgfpathmoveto{\pgfqpoint{2.902079in}{1.613090in}}%
\pgfpathlineto{\pgfqpoint{2.910833in}{1.613090in}}%
\pgfpathlineto{\pgfqpoint{2.910833in}{1.933329in}}%
\pgfpathlineto{\pgfqpoint{2.902079in}{1.933329in}}%
\pgfpathlineto{\pgfqpoint{2.902079in}{1.613090in}}%
\pgfpathclose%
\pgfusepath{fill}%
\end{pgfscope}%
\begin{pgfscope}%
\pgfpathrectangle{\pgfqpoint{0.804646in}{0.600000in}}{\pgfqpoint{2.573292in}{2.070576in}}%
\pgfusepath{clip}%
\pgfsetbuttcap%
\pgfsetmiterjoin%
\definecolor{currentfill}{rgb}{0.133298,0.375282,0.379395}%
\pgfsetfillcolor{currentfill}%
\pgfsetlinewidth{0.000000pt}%
\definecolor{currentstroke}{rgb}{0.000000,0.000000,0.000000}%
\pgfsetstrokecolor{currentstroke}%
\pgfsetstrokeopacity{0.000000}%
\pgfsetdash{}{0pt}%
\pgfpathmoveto{\pgfqpoint{2.913021in}{1.613090in}}%
\pgfpathlineto{\pgfqpoint{2.921774in}{1.613090in}}%
\pgfpathlineto{\pgfqpoint{2.921774in}{1.849709in}}%
\pgfpathlineto{\pgfqpoint{2.913021in}{1.849709in}}%
\pgfpathlineto{\pgfqpoint{2.913021in}{1.613090in}}%
\pgfpathclose%
\pgfusepath{fill}%
\end{pgfscope}%
\begin{pgfscope}%
\pgfpathrectangle{\pgfqpoint{0.804646in}{0.600000in}}{\pgfqpoint{2.573292in}{2.070576in}}%
\pgfusepath{clip}%
\pgfsetbuttcap%
\pgfsetmiterjoin%
\definecolor{currentfill}{rgb}{0.133298,0.375282,0.379395}%
\pgfsetfillcolor{currentfill}%
\pgfsetlinewidth{0.000000pt}%
\definecolor{currentstroke}{rgb}{0.000000,0.000000,0.000000}%
\pgfsetstrokecolor{currentstroke}%
\pgfsetstrokeopacity{0.000000}%
\pgfsetdash{}{0pt}%
\pgfpathmoveto{\pgfqpoint{2.923963in}{1.613090in}}%
\pgfpathlineto{\pgfqpoint{2.932716in}{1.613090in}}%
\pgfpathlineto{\pgfqpoint{2.932716in}{1.869095in}}%
\pgfpathlineto{\pgfqpoint{2.923963in}{1.869095in}}%
\pgfpathlineto{\pgfqpoint{2.923963in}{1.613090in}}%
\pgfpathclose%
\pgfusepath{fill}%
\end{pgfscope}%
\begin{pgfscope}%
\pgfpathrectangle{\pgfqpoint{0.804646in}{0.600000in}}{\pgfqpoint{2.573292in}{2.070576in}}%
\pgfusepath{clip}%
\pgfsetbuttcap%
\pgfsetmiterjoin%
\definecolor{currentfill}{rgb}{0.133298,0.375282,0.379395}%
\pgfsetfillcolor{currentfill}%
\pgfsetlinewidth{0.000000pt}%
\definecolor{currentstroke}{rgb}{0.000000,0.000000,0.000000}%
\pgfsetstrokecolor{currentstroke}%
\pgfsetstrokeopacity{0.000000}%
\pgfsetdash{}{0pt}%
\pgfpathmoveto{\pgfqpoint{2.934905in}{1.613090in}}%
\pgfpathlineto{\pgfqpoint{2.943658in}{1.613090in}}%
\pgfpathlineto{\pgfqpoint{2.943658in}{1.828240in}}%
\pgfpathlineto{\pgfqpoint{2.934905in}{1.828240in}}%
\pgfpathlineto{\pgfqpoint{2.934905in}{1.613090in}}%
\pgfpathclose%
\pgfusepath{fill}%
\end{pgfscope}%
\begin{pgfscope}%
\pgfpathrectangle{\pgfqpoint{0.804646in}{0.600000in}}{\pgfqpoint{2.573292in}{2.070576in}}%
\pgfusepath{clip}%
\pgfsetbuttcap%
\pgfsetmiterjoin%
\definecolor{currentfill}{rgb}{0.133298,0.375282,0.379395}%
\pgfsetfillcolor{currentfill}%
\pgfsetlinewidth{0.000000pt}%
\definecolor{currentstroke}{rgb}{0.000000,0.000000,0.000000}%
\pgfsetstrokecolor{currentstroke}%
\pgfsetstrokeopacity{0.000000}%
\pgfsetdash{}{0pt}%
\pgfpathmoveto{\pgfqpoint{2.945846in}{1.613090in}}%
\pgfpathlineto{\pgfqpoint{2.954600in}{1.613090in}}%
\pgfpathlineto{\pgfqpoint{2.954600in}{1.799606in}}%
\pgfpathlineto{\pgfqpoint{2.945846in}{1.799606in}}%
\pgfpathlineto{\pgfqpoint{2.945846in}{1.613090in}}%
\pgfpathclose%
\pgfusepath{fill}%
\end{pgfscope}%
\begin{pgfscope}%
\pgfpathrectangle{\pgfqpoint{0.804646in}{0.600000in}}{\pgfqpoint{2.573292in}{2.070576in}}%
\pgfusepath{clip}%
\pgfsetbuttcap%
\pgfsetmiterjoin%
\definecolor{currentfill}{rgb}{0.133298,0.375282,0.379395}%
\pgfsetfillcolor{currentfill}%
\pgfsetlinewidth{0.000000pt}%
\definecolor{currentstroke}{rgb}{0.000000,0.000000,0.000000}%
\pgfsetstrokecolor{currentstroke}%
\pgfsetstrokeopacity{0.000000}%
\pgfsetdash{}{0pt}%
\pgfpathmoveto{\pgfqpoint{2.956788in}{1.613090in}}%
\pgfpathlineto{\pgfqpoint{2.965542in}{1.613090in}}%
\pgfpathlineto{\pgfqpoint{2.965542in}{1.839660in}}%
\pgfpathlineto{\pgfqpoint{2.956788in}{1.839660in}}%
\pgfpathlineto{\pgfqpoint{2.956788in}{1.613090in}}%
\pgfpathclose%
\pgfusepath{fill}%
\end{pgfscope}%
\begin{pgfscope}%
\pgfpathrectangle{\pgfqpoint{0.804646in}{0.600000in}}{\pgfqpoint{2.573292in}{2.070576in}}%
\pgfusepath{clip}%
\pgfsetbuttcap%
\pgfsetmiterjoin%
\definecolor{currentfill}{rgb}{0.133298,0.375282,0.379395}%
\pgfsetfillcolor{currentfill}%
\pgfsetlinewidth{0.000000pt}%
\definecolor{currentstroke}{rgb}{0.000000,0.000000,0.000000}%
\pgfsetstrokecolor{currentstroke}%
\pgfsetstrokeopacity{0.000000}%
\pgfsetdash{}{0pt}%
\pgfpathmoveto{\pgfqpoint{2.967730in}{1.613090in}}%
\pgfpathlineto{\pgfqpoint{2.976483in}{1.613090in}}%
\pgfpathlineto{\pgfqpoint{2.976483in}{1.864265in}}%
\pgfpathlineto{\pgfqpoint{2.967730in}{1.864265in}}%
\pgfpathlineto{\pgfqpoint{2.967730in}{1.613090in}}%
\pgfpathclose%
\pgfusepath{fill}%
\end{pgfscope}%
\begin{pgfscope}%
\pgfpathrectangle{\pgfqpoint{0.804646in}{0.600000in}}{\pgfqpoint{2.573292in}{2.070576in}}%
\pgfusepath{clip}%
\pgfsetbuttcap%
\pgfsetmiterjoin%
\definecolor{currentfill}{rgb}{0.133298,0.375282,0.379395}%
\pgfsetfillcolor{currentfill}%
\pgfsetlinewidth{0.000000pt}%
\definecolor{currentstroke}{rgb}{0.000000,0.000000,0.000000}%
\pgfsetstrokecolor{currentstroke}%
\pgfsetstrokeopacity{0.000000}%
\pgfsetdash{}{0pt}%
\pgfpathmoveto{\pgfqpoint{2.978672in}{1.613090in}}%
\pgfpathlineto{\pgfqpoint{2.987425in}{1.613090in}}%
\pgfpathlineto{\pgfqpoint{2.987425in}{1.833276in}}%
\pgfpathlineto{\pgfqpoint{2.978672in}{1.833276in}}%
\pgfpathlineto{\pgfqpoint{2.978672in}{1.613090in}}%
\pgfpathclose%
\pgfusepath{fill}%
\end{pgfscope}%
\begin{pgfscope}%
\pgfpathrectangle{\pgfqpoint{0.804646in}{0.600000in}}{\pgfqpoint{2.573292in}{2.070576in}}%
\pgfusepath{clip}%
\pgfsetbuttcap%
\pgfsetmiterjoin%
\definecolor{currentfill}{rgb}{0.133298,0.375282,0.379395}%
\pgfsetfillcolor{currentfill}%
\pgfsetlinewidth{0.000000pt}%
\definecolor{currentstroke}{rgb}{0.000000,0.000000,0.000000}%
\pgfsetstrokecolor{currentstroke}%
\pgfsetstrokeopacity{0.000000}%
\pgfsetdash{}{0pt}%
\pgfpathmoveto{\pgfqpoint{2.989614in}{1.613090in}}%
\pgfpathlineto{\pgfqpoint{2.998367in}{1.613090in}}%
\pgfpathlineto{\pgfqpoint{2.998367in}{1.816201in}}%
\pgfpathlineto{\pgfqpoint{2.989614in}{1.816201in}}%
\pgfpathlineto{\pgfqpoint{2.989614in}{1.613090in}}%
\pgfpathclose%
\pgfusepath{fill}%
\end{pgfscope}%
\begin{pgfscope}%
\pgfpathrectangle{\pgfqpoint{0.804646in}{0.600000in}}{\pgfqpoint{2.573292in}{2.070576in}}%
\pgfusepath{clip}%
\pgfsetbuttcap%
\pgfsetmiterjoin%
\definecolor{currentfill}{rgb}{0.133298,0.375282,0.379395}%
\pgfsetfillcolor{currentfill}%
\pgfsetlinewidth{0.000000pt}%
\definecolor{currentstroke}{rgb}{0.000000,0.000000,0.000000}%
\pgfsetstrokecolor{currentstroke}%
\pgfsetstrokeopacity{0.000000}%
\pgfsetdash{}{0pt}%
\pgfpathmoveto{\pgfqpoint{3.000555in}{1.613090in}}%
\pgfpathlineto{\pgfqpoint{3.009309in}{1.613090in}}%
\pgfpathlineto{\pgfqpoint{3.009309in}{1.845445in}}%
\pgfpathlineto{\pgfqpoint{3.000555in}{1.845445in}}%
\pgfpathlineto{\pgfqpoint{3.000555in}{1.613090in}}%
\pgfpathclose%
\pgfusepath{fill}%
\end{pgfscope}%
\begin{pgfscope}%
\pgfpathrectangle{\pgfqpoint{0.804646in}{0.600000in}}{\pgfqpoint{2.573292in}{2.070576in}}%
\pgfusepath{clip}%
\pgfsetbuttcap%
\pgfsetmiterjoin%
\definecolor{currentfill}{rgb}{0.133298,0.375282,0.379395}%
\pgfsetfillcolor{currentfill}%
\pgfsetlinewidth{0.000000pt}%
\definecolor{currentstroke}{rgb}{0.000000,0.000000,0.000000}%
\pgfsetstrokecolor{currentstroke}%
\pgfsetstrokeopacity{0.000000}%
\pgfsetdash{}{0pt}%
\pgfpathmoveto{\pgfqpoint{3.011497in}{1.613090in}}%
\pgfpathlineto{\pgfqpoint{3.020251in}{1.613090in}}%
\pgfpathlineto{\pgfqpoint{3.020251in}{1.827486in}}%
\pgfpathlineto{\pgfqpoint{3.011497in}{1.827486in}}%
\pgfpathlineto{\pgfqpoint{3.011497in}{1.613090in}}%
\pgfpathclose%
\pgfusepath{fill}%
\end{pgfscope}%
\begin{pgfscope}%
\pgfpathrectangle{\pgfqpoint{0.804646in}{0.600000in}}{\pgfqpoint{2.573292in}{2.070576in}}%
\pgfusepath{clip}%
\pgfsetbuttcap%
\pgfsetmiterjoin%
\definecolor{currentfill}{rgb}{0.133298,0.375282,0.379395}%
\pgfsetfillcolor{currentfill}%
\pgfsetlinewidth{0.000000pt}%
\definecolor{currentstroke}{rgb}{0.000000,0.000000,0.000000}%
\pgfsetstrokecolor{currentstroke}%
\pgfsetstrokeopacity{0.000000}%
\pgfsetdash{}{0pt}%
\pgfpathmoveto{\pgfqpoint{3.022439in}{1.613090in}}%
\pgfpathlineto{\pgfqpoint{3.031192in}{1.613090in}}%
\pgfpathlineto{\pgfqpoint{3.031192in}{1.830138in}}%
\pgfpathlineto{\pgfqpoint{3.022439in}{1.830138in}}%
\pgfpathlineto{\pgfqpoint{3.022439in}{1.613090in}}%
\pgfpathclose%
\pgfusepath{fill}%
\end{pgfscope}%
\begin{pgfscope}%
\pgfpathrectangle{\pgfqpoint{0.804646in}{0.600000in}}{\pgfqpoint{2.573292in}{2.070576in}}%
\pgfusepath{clip}%
\pgfsetbuttcap%
\pgfsetmiterjoin%
\definecolor{currentfill}{rgb}{0.133298,0.375282,0.379395}%
\pgfsetfillcolor{currentfill}%
\pgfsetlinewidth{0.000000pt}%
\definecolor{currentstroke}{rgb}{0.000000,0.000000,0.000000}%
\pgfsetstrokecolor{currentstroke}%
\pgfsetstrokeopacity{0.000000}%
\pgfsetdash{}{0pt}%
\pgfpathmoveto{\pgfqpoint{3.033381in}{1.613090in}}%
\pgfpathlineto{\pgfqpoint{3.042134in}{1.613090in}}%
\pgfpathlineto{\pgfqpoint{3.042134in}{1.866598in}}%
\pgfpathlineto{\pgfqpoint{3.033381in}{1.866598in}}%
\pgfpathlineto{\pgfqpoint{3.033381in}{1.613090in}}%
\pgfpathclose%
\pgfusepath{fill}%
\end{pgfscope}%
\begin{pgfscope}%
\pgfpathrectangle{\pgfqpoint{0.804646in}{0.600000in}}{\pgfqpoint{2.573292in}{2.070576in}}%
\pgfusepath{clip}%
\pgfsetbuttcap%
\pgfsetmiterjoin%
\definecolor{currentfill}{rgb}{0.133298,0.375282,0.379395}%
\pgfsetfillcolor{currentfill}%
\pgfsetlinewidth{0.000000pt}%
\definecolor{currentstroke}{rgb}{0.000000,0.000000,0.000000}%
\pgfsetstrokecolor{currentstroke}%
\pgfsetstrokeopacity{0.000000}%
\pgfsetdash{}{0pt}%
\pgfpathmoveto{\pgfqpoint{3.044323in}{1.613090in}}%
\pgfpathlineto{\pgfqpoint{3.053076in}{1.613090in}}%
\pgfpathlineto{\pgfqpoint{3.053076in}{1.914480in}}%
\pgfpathlineto{\pgfqpoint{3.044323in}{1.914480in}}%
\pgfpathlineto{\pgfqpoint{3.044323in}{1.613090in}}%
\pgfpathclose%
\pgfusepath{fill}%
\end{pgfscope}%
\begin{pgfscope}%
\pgfpathrectangle{\pgfqpoint{0.804646in}{0.600000in}}{\pgfqpoint{2.573292in}{2.070576in}}%
\pgfusepath{clip}%
\pgfsetbuttcap%
\pgfsetmiterjoin%
\definecolor{currentfill}{rgb}{0.133298,0.375282,0.379395}%
\pgfsetfillcolor{currentfill}%
\pgfsetlinewidth{0.000000pt}%
\definecolor{currentstroke}{rgb}{0.000000,0.000000,0.000000}%
\pgfsetstrokecolor{currentstroke}%
\pgfsetstrokeopacity{0.000000}%
\pgfsetdash{}{0pt}%
\pgfpathmoveto{\pgfqpoint{3.055264in}{1.613090in}}%
\pgfpathlineto{\pgfqpoint{3.064018in}{1.613090in}}%
\pgfpathlineto{\pgfqpoint{3.064018in}{1.859742in}}%
\pgfpathlineto{\pgfqpoint{3.055264in}{1.859742in}}%
\pgfpathlineto{\pgfqpoint{3.055264in}{1.613090in}}%
\pgfpathclose%
\pgfusepath{fill}%
\end{pgfscope}%
\begin{pgfscope}%
\pgfpathrectangle{\pgfqpoint{0.804646in}{0.600000in}}{\pgfqpoint{2.573292in}{2.070576in}}%
\pgfusepath{clip}%
\pgfsetbuttcap%
\pgfsetmiterjoin%
\definecolor{currentfill}{rgb}{0.133298,0.375282,0.379395}%
\pgfsetfillcolor{currentfill}%
\pgfsetlinewidth{0.000000pt}%
\definecolor{currentstroke}{rgb}{0.000000,0.000000,0.000000}%
\pgfsetstrokecolor{currentstroke}%
\pgfsetstrokeopacity{0.000000}%
\pgfsetdash{}{0pt}%
\pgfpathmoveto{\pgfqpoint{3.066206in}{1.613090in}}%
\pgfpathlineto{\pgfqpoint{3.074960in}{1.613090in}}%
\pgfpathlineto{\pgfqpoint{3.074960in}{1.892030in}}%
\pgfpathlineto{\pgfqpoint{3.066206in}{1.892030in}}%
\pgfpathlineto{\pgfqpoint{3.066206in}{1.613090in}}%
\pgfpathclose%
\pgfusepath{fill}%
\end{pgfscope}%
\begin{pgfscope}%
\pgfpathrectangle{\pgfqpoint{0.804646in}{0.600000in}}{\pgfqpoint{2.573292in}{2.070576in}}%
\pgfusepath{clip}%
\pgfsetbuttcap%
\pgfsetmiterjoin%
\definecolor{currentfill}{rgb}{0.133298,0.375282,0.379395}%
\pgfsetfillcolor{currentfill}%
\pgfsetlinewidth{0.000000pt}%
\definecolor{currentstroke}{rgb}{0.000000,0.000000,0.000000}%
\pgfsetstrokecolor{currentstroke}%
\pgfsetstrokeopacity{0.000000}%
\pgfsetdash{}{0pt}%
\pgfpathmoveto{\pgfqpoint{3.077148in}{1.613090in}}%
\pgfpathlineto{\pgfqpoint{3.085901in}{1.613090in}}%
\pgfpathlineto{\pgfqpoint{3.085901in}{1.957425in}}%
\pgfpathlineto{\pgfqpoint{3.077148in}{1.957425in}}%
\pgfpathlineto{\pgfqpoint{3.077148in}{1.613090in}}%
\pgfpathclose%
\pgfusepath{fill}%
\end{pgfscope}%
\begin{pgfscope}%
\pgfpathrectangle{\pgfqpoint{0.804646in}{0.600000in}}{\pgfqpoint{2.573292in}{2.070576in}}%
\pgfusepath{clip}%
\pgfsetbuttcap%
\pgfsetmiterjoin%
\definecolor{currentfill}{rgb}{0.133298,0.375282,0.379395}%
\pgfsetfillcolor{currentfill}%
\pgfsetlinewidth{0.000000pt}%
\definecolor{currentstroke}{rgb}{0.000000,0.000000,0.000000}%
\pgfsetstrokecolor{currentstroke}%
\pgfsetstrokeopacity{0.000000}%
\pgfsetdash{}{0pt}%
\pgfpathmoveto{\pgfqpoint{3.088090in}{1.613090in}}%
\pgfpathlineto{\pgfqpoint{3.096843in}{1.613090in}}%
\pgfpathlineto{\pgfqpoint{3.096843in}{2.006791in}}%
\pgfpathlineto{\pgfqpoint{3.088090in}{2.006791in}}%
\pgfpathlineto{\pgfqpoint{3.088090in}{1.613090in}}%
\pgfpathclose%
\pgfusepath{fill}%
\end{pgfscope}%
\begin{pgfscope}%
\pgfpathrectangle{\pgfqpoint{0.804646in}{0.600000in}}{\pgfqpoint{2.573292in}{2.070576in}}%
\pgfusepath{clip}%
\pgfsetbuttcap%
\pgfsetmiterjoin%
\definecolor{currentfill}{rgb}{0.133298,0.375282,0.379395}%
\pgfsetfillcolor{currentfill}%
\pgfsetlinewidth{0.000000pt}%
\definecolor{currentstroke}{rgb}{0.000000,0.000000,0.000000}%
\pgfsetstrokecolor{currentstroke}%
\pgfsetstrokeopacity{0.000000}%
\pgfsetdash{}{0pt}%
\pgfpathmoveto{\pgfqpoint{3.099032in}{1.613090in}}%
\pgfpathlineto{\pgfqpoint{3.107785in}{1.613090in}}%
\pgfpathlineto{\pgfqpoint{3.107785in}{1.950376in}}%
\pgfpathlineto{\pgfqpoint{3.099032in}{1.950376in}}%
\pgfpathlineto{\pgfqpoint{3.099032in}{1.613090in}}%
\pgfpathclose%
\pgfusepath{fill}%
\end{pgfscope}%
\begin{pgfscope}%
\pgfpathrectangle{\pgfqpoint{0.804646in}{0.600000in}}{\pgfqpoint{2.573292in}{2.070576in}}%
\pgfusepath{clip}%
\pgfsetbuttcap%
\pgfsetmiterjoin%
\definecolor{currentfill}{rgb}{0.133298,0.375282,0.379395}%
\pgfsetfillcolor{currentfill}%
\pgfsetlinewidth{0.000000pt}%
\definecolor{currentstroke}{rgb}{0.000000,0.000000,0.000000}%
\pgfsetstrokecolor{currentstroke}%
\pgfsetstrokeopacity{0.000000}%
\pgfsetdash{}{0pt}%
\pgfpathmoveto{\pgfqpoint{3.109973in}{1.613090in}}%
\pgfpathlineto{\pgfqpoint{3.118727in}{1.613090in}}%
\pgfpathlineto{\pgfqpoint{3.118727in}{1.993549in}}%
\pgfpathlineto{\pgfqpoint{3.109973in}{1.993549in}}%
\pgfpathlineto{\pgfqpoint{3.109973in}{1.613090in}}%
\pgfpathclose%
\pgfusepath{fill}%
\end{pgfscope}%
\begin{pgfscope}%
\pgfpathrectangle{\pgfqpoint{0.804646in}{0.600000in}}{\pgfqpoint{2.573292in}{2.070576in}}%
\pgfusepath{clip}%
\pgfsetbuttcap%
\pgfsetmiterjoin%
\definecolor{currentfill}{rgb}{0.133298,0.375282,0.379395}%
\pgfsetfillcolor{currentfill}%
\pgfsetlinewidth{0.000000pt}%
\definecolor{currentstroke}{rgb}{0.000000,0.000000,0.000000}%
\pgfsetstrokecolor{currentstroke}%
\pgfsetstrokeopacity{0.000000}%
\pgfsetdash{}{0pt}%
\pgfpathmoveto{\pgfqpoint{3.120915in}{1.613090in}}%
\pgfpathlineto{\pgfqpoint{3.129669in}{1.613090in}}%
\pgfpathlineto{\pgfqpoint{3.129669in}{1.975502in}}%
\pgfpathlineto{\pgfqpoint{3.120915in}{1.975502in}}%
\pgfpathlineto{\pgfqpoint{3.120915in}{1.613090in}}%
\pgfpathclose%
\pgfusepath{fill}%
\end{pgfscope}%
\begin{pgfscope}%
\pgfpathrectangle{\pgfqpoint{0.804646in}{0.600000in}}{\pgfqpoint{2.573292in}{2.070576in}}%
\pgfusepath{clip}%
\pgfsetbuttcap%
\pgfsetmiterjoin%
\definecolor{currentfill}{rgb}{0.133298,0.375282,0.379395}%
\pgfsetfillcolor{currentfill}%
\pgfsetlinewidth{0.000000pt}%
\definecolor{currentstroke}{rgb}{0.000000,0.000000,0.000000}%
\pgfsetstrokecolor{currentstroke}%
\pgfsetstrokeopacity{0.000000}%
\pgfsetdash{}{0pt}%
\pgfpathmoveto{\pgfqpoint{3.131857in}{1.613090in}}%
\pgfpathlineto{\pgfqpoint{3.140610in}{1.613090in}}%
\pgfpathlineto{\pgfqpoint{3.140610in}{1.966033in}}%
\pgfpathlineto{\pgfqpoint{3.131857in}{1.966033in}}%
\pgfpathlineto{\pgfqpoint{3.131857in}{1.613090in}}%
\pgfpathclose%
\pgfusepath{fill}%
\end{pgfscope}%
\begin{pgfscope}%
\pgfpathrectangle{\pgfqpoint{0.804646in}{0.600000in}}{\pgfqpoint{2.573292in}{2.070576in}}%
\pgfusepath{clip}%
\pgfsetbuttcap%
\pgfsetmiterjoin%
\definecolor{currentfill}{rgb}{0.133298,0.375282,0.379395}%
\pgfsetfillcolor{currentfill}%
\pgfsetlinewidth{0.000000pt}%
\definecolor{currentstroke}{rgb}{0.000000,0.000000,0.000000}%
\pgfsetstrokecolor{currentstroke}%
\pgfsetstrokeopacity{0.000000}%
\pgfsetdash{}{0pt}%
\pgfpathmoveto{\pgfqpoint{3.142799in}{1.613090in}}%
\pgfpathlineto{\pgfqpoint{3.151552in}{1.613090in}}%
\pgfpathlineto{\pgfqpoint{3.151552in}{2.006038in}}%
\pgfpathlineto{\pgfqpoint{3.142799in}{2.006038in}}%
\pgfpathlineto{\pgfqpoint{3.142799in}{1.613090in}}%
\pgfpathclose%
\pgfusepath{fill}%
\end{pgfscope}%
\begin{pgfscope}%
\pgfpathrectangle{\pgfqpoint{0.804646in}{0.600000in}}{\pgfqpoint{2.573292in}{2.070576in}}%
\pgfusepath{clip}%
\pgfsetbuttcap%
\pgfsetmiterjoin%
\definecolor{currentfill}{rgb}{0.133298,0.375282,0.379395}%
\pgfsetfillcolor{currentfill}%
\pgfsetlinewidth{0.000000pt}%
\definecolor{currentstroke}{rgb}{0.000000,0.000000,0.000000}%
\pgfsetstrokecolor{currentstroke}%
\pgfsetstrokeopacity{0.000000}%
\pgfsetdash{}{0pt}%
\pgfpathmoveto{\pgfqpoint{3.153741in}{1.613090in}}%
\pgfpathlineto{\pgfqpoint{3.162494in}{1.613090in}}%
\pgfpathlineto{\pgfqpoint{3.162494in}{1.985216in}}%
\pgfpathlineto{\pgfqpoint{3.153741in}{1.985216in}}%
\pgfpathlineto{\pgfqpoint{3.153741in}{1.613090in}}%
\pgfpathclose%
\pgfusepath{fill}%
\end{pgfscope}%
\begin{pgfscope}%
\pgfpathrectangle{\pgfqpoint{0.804646in}{0.600000in}}{\pgfqpoint{2.573292in}{2.070576in}}%
\pgfusepath{clip}%
\pgfsetbuttcap%
\pgfsetmiterjoin%
\definecolor{currentfill}{rgb}{0.133298,0.375282,0.379395}%
\pgfsetfillcolor{currentfill}%
\pgfsetlinewidth{0.000000pt}%
\definecolor{currentstroke}{rgb}{0.000000,0.000000,0.000000}%
\pgfsetstrokecolor{currentstroke}%
\pgfsetstrokeopacity{0.000000}%
\pgfsetdash{}{0pt}%
\pgfpathmoveto{\pgfqpoint{3.164682in}{1.613090in}}%
\pgfpathlineto{\pgfqpoint{3.173436in}{1.613090in}}%
\pgfpathlineto{\pgfqpoint{3.173436in}{1.953977in}}%
\pgfpathlineto{\pgfqpoint{3.164682in}{1.953977in}}%
\pgfpathlineto{\pgfqpoint{3.164682in}{1.613090in}}%
\pgfpathclose%
\pgfusepath{fill}%
\end{pgfscope}%
\begin{pgfscope}%
\pgfpathrectangle{\pgfqpoint{0.804646in}{0.600000in}}{\pgfqpoint{2.573292in}{2.070576in}}%
\pgfusepath{clip}%
\pgfsetbuttcap%
\pgfsetmiterjoin%
\definecolor{currentfill}{rgb}{0.133298,0.375282,0.379395}%
\pgfsetfillcolor{currentfill}%
\pgfsetlinewidth{0.000000pt}%
\definecolor{currentstroke}{rgb}{0.000000,0.000000,0.000000}%
\pgfsetstrokecolor{currentstroke}%
\pgfsetstrokeopacity{0.000000}%
\pgfsetdash{}{0pt}%
\pgfpathmoveto{\pgfqpoint{3.175624in}{1.613090in}}%
\pgfpathlineto{\pgfqpoint{3.184378in}{1.613090in}}%
\pgfpathlineto{\pgfqpoint{3.184378in}{1.973189in}}%
\pgfpathlineto{\pgfqpoint{3.175624in}{1.973189in}}%
\pgfpathlineto{\pgfqpoint{3.175624in}{1.613090in}}%
\pgfpathclose%
\pgfusepath{fill}%
\end{pgfscope}%
\begin{pgfscope}%
\pgfpathrectangle{\pgfqpoint{0.804646in}{0.600000in}}{\pgfqpoint{2.573292in}{2.070576in}}%
\pgfusepath{clip}%
\pgfsetbuttcap%
\pgfsetmiterjoin%
\definecolor{currentfill}{rgb}{0.133298,0.375282,0.379395}%
\pgfsetfillcolor{currentfill}%
\pgfsetlinewidth{0.000000pt}%
\definecolor{currentstroke}{rgb}{0.000000,0.000000,0.000000}%
\pgfsetstrokecolor{currentstroke}%
\pgfsetstrokeopacity{0.000000}%
\pgfsetdash{}{0pt}%
\pgfpathmoveto{\pgfqpoint{3.186566in}{1.613090in}}%
\pgfpathlineto{\pgfqpoint{3.195319in}{1.613090in}}%
\pgfpathlineto{\pgfqpoint{3.195319in}{1.962810in}}%
\pgfpathlineto{\pgfqpoint{3.186566in}{1.962810in}}%
\pgfpathlineto{\pgfqpoint{3.186566in}{1.613090in}}%
\pgfpathclose%
\pgfusepath{fill}%
\end{pgfscope}%
\begin{pgfscope}%
\pgfpathrectangle{\pgfqpoint{0.804646in}{0.600000in}}{\pgfqpoint{2.573292in}{2.070576in}}%
\pgfusepath{clip}%
\pgfsetbuttcap%
\pgfsetmiterjoin%
\definecolor{currentfill}{rgb}{0.133298,0.375282,0.379395}%
\pgfsetfillcolor{currentfill}%
\pgfsetlinewidth{0.000000pt}%
\definecolor{currentstroke}{rgb}{0.000000,0.000000,0.000000}%
\pgfsetstrokecolor{currentstroke}%
\pgfsetstrokeopacity{0.000000}%
\pgfsetdash{}{0pt}%
\pgfpathmoveto{\pgfqpoint{3.197508in}{1.613090in}}%
\pgfpathlineto{\pgfqpoint{3.206261in}{1.613090in}}%
\pgfpathlineto{\pgfqpoint{3.206261in}{1.994396in}}%
\pgfpathlineto{\pgfqpoint{3.197508in}{1.994396in}}%
\pgfpathlineto{\pgfqpoint{3.197508in}{1.613090in}}%
\pgfpathclose%
\pgfusepath{fill}%
\end{pgfscope}%
\begin{pgfscope}%
\pgfpathrectangle{\pgfqpoint{0.804646in}{0.600000in}}{\pgfqpoint{2.573292in}{2.070576in}}%
\pgfusepath{clip}%
\pgfsetbuttcap%
\pgfsetmiterjoin%
\definecolor{currentfill}{rgb}{0.133298,0.375282,0.379395}%
\pgfsetfillcolor{currentfill}%
\pgfsetlinewidth{0.000000pt}%
\definecolor{currentstroke}{rgb}{0.000000,0.000000,0.000000}%
\pgfsetstrokecolor{currentstroke}%
\pgfsetstrokeopacity{0.000000}%
\pgfsetdash{}{0pt}%
\pgfpathmoveto{\pgfqpoint{3.208450in}{1.613090in}}%
\pgfpathlineto{\pgfqpoint{3.217203in}{1.613090in}}%
\pgfpathlineto{\pgfqpoint{3.217203in}{2.024407in}}%
\pgfpathlineto{\pgfqpoint{3.208450in}{2.024407in}}%
\pgfpathlineto{\pgfqpoint{3.208450in}{1.613090in}}%
\pgfpathclose%
\pgfusepath{fill}%
\end{pgfscope}%
\begin{pgfscope}%
\pgfpathrectangle{\pgfqpoint{0.804646in}{0.600000in}}{\pgfqpoint{2.573292in}{2.070576in}}%
\pgfusepath{clip}%
\pgfsetbuttcap%
\pgfsetmiterjoin%
\definecolor{currentfill}{rgb}{0.133298,0.375282,0.379395}%
\pgfsetfillcolor{currentfill}%
\pgfsetlinewidth{0.000000pt}%
\definecolor{currentstroke}{rgb}{0.000000,0.000000,0.000000}%
\pgfsetstrokecolor{currentstroke}%
\pgfsetstrokeopacity{0.000000}%
\pgfsetdash{}{0pt}%
\pgfpathmoveto{\pgfqpoint{3.219391in}{1.613090in}}%
\pgfpathlineto{\pgfqpoint{3.228145in}{1.613090in}}%
\pgfpathlineto{\pgfqpoint{3.228145in}{2.093302in}}%
\pgfpathlineto{\pgfqpoint{3.219391in}{2.093302in}}%
\pgfpathlineto{\pgfqpoint{3.219391in}{1.613090in}}%
\pgfpathclose%
\pgfusepath{fill}%
\end{pgfscope}%
\begin{pgfscope}%
\pgfpathrectangle{\pgfqpoint{0.804646in}{0.600000in}}{\pgfqpoint{2.573292in}{2.070576in}}%
\pgfusepath{clip}%
\pgfsetbuttcap%
\pgfsetmiterjoin%
\definecolor{currentfill}{rgb}{0.133298,0.375282,0.379395}%
\pgfsetfillcolor{currentfill}%
\pgfsetlinewidth{0.000000pt}%
\definecolor{currentstroke}{rgb}{0.000000,0.000000,0.000000}%
\pgfsetstrokecolor{currentstroke}%
\pgfsetstrokeopacity{0.000000}%
\pgfsetdash{}{0pt}%
\pgfpathmoveto{\pgfqpoint{3.230333in}{1.613090in}}%
\pgfpathlineto{\pgfqpoint{3.239087in}{1.613090in}}%
\pgfpathlineto{\pgfqpoint{3.239087in}{2.087018in}}%
\pgfpathlineto{\pgfqpoint{3.230333in}{2.087018in}}%
\pgfpathlineto{\pgfqpoint{3.230333in}{1.613090in}}%
\pgfpathclose%
\pgfusepath{fill}%
\end{pgfscope}%
\begin{pgfscope}%
\pgfpathrectangle{\pgfqpoint{0.804646in}{0.600000in}}{\pgfqpoint{2.573292in}{2.070576in}}%
\pgfusepath{clip}%
\pgfsetbuttcap%
\pgfsetmiterjoin%
\definecolor{currentfill}{rgb}{0.133298,0.375282,0.379395}%
\pgfsetfillcolor{currentfill}%
\pgfsetlinewidth{0.000000pt}%
\definecolor{currentstroke}{rgb}{0.000000,0.000000,0.000000}%
\pgfsetstrokecolor{currentstroke}%
\pgfsetstrokeopacity{0.000000}%
\pgfsetdash{}{0pt}%
\pgfpathmoveto{\pgfqpoint{3.241275in}{1.613090in}}%
\pgfpathlineto{\pgfqpoint{3.250028in}{1.613090in}}%
\pgfpathlineto{\pgfqpoint{3.250028in}{2.137110in}}%
\pgfpathlineto{\pgfqpoint{3.241275in}{2.137110in}}%
\pgfpathlineto{\pgfqpoint{3.241275in}{1.613090in}}%
\pgfpathclose%
\pgfusepath{fill}%
\end{pgfscope}%
\begin{pgfscope}%
\pgfpathrectangle{\pgfqpoint{0.804646in}{0.600000in}}{\pgfqpoint{2.573292in}{2.070576in}}%
\pgfusepath{clip}%
\pgfsetbuttcap%
\pgfsetmiterjoin%
\definecolor{currentfill}{rgb}{0.133298,0.375282,0.379395}%
\pgfsetfillcolor{currentfill}%
\pgfsetlinewidth{0.000000pt}%
\definecolor{currentstroke}{rgb}{0.000000,0.000000,0.000000}%
\pgfsetstrokecolor{currentstroke}%
\pgfsetstrokeopacity{0.000000}%
\pgfsetdash{}{0pt}%
\pgfpathmoveto{\pgfqpoint{3.252217in}{1.613090in}}%
\pgfpathlineto{\pgfqpoint{3.260970in}{1.613090in}}%
\pgfpathlineto{\pgfqpoint{3.260970in}{2.168003in}}%
\pgfpathlineto{\pgfqpoint{3.252217in}{2.168003in}}%
\pgfpathlineto{\pgfqpoint{3.252217in}{1.613090in}}%
\pgfpathclose%
\pgfusepath{fill}%
\end{pgfscope}%
\begin{pgfscope}%
\pgfpathrectangle{\pgfqpoint{0.804646in}{0.600000in}}{\pgfqpoint{2.573292in}{2.070576in}}%
\pgfusepath{clip}%
\pgfsetbuttcap%
\pgfsetmiterjoin%
\definecolor{currentfill}{rgb}{0.302379,0.450282,0.300122}%
\pgfsetfillcolor{currentfill}%
\pgfsetlinewidth{0.000000pt}%
\definecolor{currentstroke}{rgb}{0.000000,0.000000,0.000000}%
\pgfsetstrokecolor{currentstroke}%
\pgfsetstrokeopacity{0.000000}%
\pgfsetdash{}{0pt}%
\pgfpathmoveto{\pgfqpoint{0.921614in}{1.679738in}}%
\pgfpathlineto{\pgfqpoint{0.930367in}{1.679738in}}%
\pgfpathlineto{\pgfqpoint{0.930367in}{1.684817in}}%
\pgfpathlineto{\pgfqpoint{0.921614in}{1.684817in}}%
\pgfpathlineto{\pgfqpoint{0.921614in}{1.679738in}}%
\pgfpathclose%
\pgfusepath{fill}%
\end{pgfscope}%
\begin{pgfscope}%
\pgfpathrectangle{\pgfqpoint{0.804646in}{0.600000in}}{\pgfqpoint{2.573292in}{2.070576in}}%
\pgfusepath{clip}%
\pgfsetbuttcap%
\pgfsetmiterjoin%
\definecolor{currentfill}{rgb}{0.302379,0.450282,0.300122}%
\pgfsetfillcolor{currentfill}%
\pgfsetlinewidth{0.000000pt}%
\definecolor{currentstroke}{rgb}{0.000000,0.000000,0.000000}%
\pgfsetstrokecolor{currentstroke}%
\pgfsetstrokeopacity{0.000000}%
\pgfsetdash{}{0pt}%
\pgfpathmoveto{\pgfqpoint{0.932555in}{1.705593in}}%
\pgfpathlineto{\pgfqpoint{0.941309in}{1.705593in}}%
\pgfpathlineto{\pgfqpoint{0.941309in}{1.708137in}}%
\pgfpathlineto{\pgfqpoint{0.932555in}{1.708137in}}%
\pgfpathlineto{\pgfqpoint{0.932555in}{1.705593in}}%
\pgfpathclose%
\pgfusepath{fill}%
\end{pgfscope}%
\begin{pgfscope}%
\pgfpathrectangle{\pgfqpoint{0.804646in}{0.600000in}}{\pgfqpoint{2.573292in}{2.070576in}}%
\pgfusepath{clip}%
\pgfsetbuttcap%
\pgfsetmiterjoin%
\definecolor{currentfill}{rgb}{0.302379,0.450282,0.300122}%
\pgfsetfillcolor{currentfill}%
\pgfsetlinewidth{0.000000pt}%
\definecolor{currentstroke}{rgb}{0.000000,0.000000,0.000000}%
\pgfsetstrokecolor{currentstroke}%
\pgfsetstrokeopacity{0.000000}%
\pgfsetdash{}{0pt}%
\pgfpathmoveto{\pgfqpoint{0.943497in}{1.777818in}}%
\pgfpathlineto{\pgfqpoint{0.952251in}{1.777818in}}%
\pgfpathlineto{\pgfqpoint{0.952251in}{1.789338in}}%
\pgfpathlineto{\pgfqpoint{0.943497in}{1.789338in}}%
\pgfpathlineto{\pgfqpoint{0.943497in}{1.777818in}}%
\pgfpathclose%
\pgfusepath{fill}%
\end{pgfscope}%
\begin{pgfscope}%
\pgfpathrectangle{\pgfqpoint{0.804646in}{0.600000in}}{\pgfqpoint{2.573292in}{2.070576in}}%
\pgfusepath{clip}%
\pgfsetbuttcap%
\pgfsetmiterjoin%
\definecolor{currentfill}{rgb}{0.302379,0.450282,0.300122}%
\pgfsetfillcolor{currentfill}%
\pgfsetlinewidth{0.000000pt}%
\definecolor{currentstroke}{rgb}{0.000000,0.000000,0.000000}%
\pgfsetstrokecolor{currentstroke}%
\pgfsetstrokeopacity{0.000000}%
\pgfsetdash{}{0pt}%
\pgfpathmoveto{\pgfqpoint{0.954439in}{1.768375in}}%
\pgfpathlineto{\pgfqpoint{0.963192in}{1.768375in}}%
\pgfpathlineto{\pgfqpoint{0.963192in}{1.785045in}}%
\pgfpathlineto{\pgfqpoint{0.954439in}{1.785045in}}%
\pgfpathlineto{\pgfqpoint{0.954439in}{1.768375in}}%
\pgfpathclose%
\pgfusepath{fill}%
\end{pgfscope}%
\begin{pgfscope}%
\pgfpathrectangle{\pgfqpoint{0.804646in}{0.600000in}}{\pgfqpoint{2.573292in}{2.070576in}}%
\pgfusepath{clip}%
\pgfsetbuttcap%
\pgfsetmiterjoin%
\definecolor{currentfill}{rgb}{0.302379,0.450282,0.300122}%
\pgfsetfillcolor{currentfill}%
\pgfsetlinewidth{0.000000pt}%
\definecolor{currentstroke}{rgb}{0.000000,0.000000,0.000000}%
\pgfsetstrokecolor{currentstroke}%
\pgfsetstrokeopacity{0.000000}%
\pgfsetdash{}{0pt}%
\pgfpathmoveto{\pgfqpoint{0.965381in}{1.761115in}}%
\pgfpathlineto{\pgfqpoint{0.974134in}{1.761115in}}%
\pgfpathlineto{\pgfqpoint{0.974134in}{1.781970in}}%
\pgfpathlineto{\pgfqpoint{0.965381in}{1.781970in}}%
\pgfpathlineto{\pgfqpoint{0.965381in}{1.761115in}}%
\pgfpathclose%
\pgfusepath{fill}%
\end{pgfscope}%
\begin{pgfscope}%
\pgfpathrectangle{\pgfqpoint{0.804646in}{0.600000in}}{\pgfqpoint{2.573292in}{2.070576in}}%
\pgfusepath{clip}%
\pgfsetbuttcap%
\pgfsetmiterjoin%
\definecolor{currentfill}{rgb}{0.302379,0.450282,0.300122}%
\pgfsetfillcolor{currentfill}%
\pgfsetlinewidth{0.000000pt}%
\definecolor{currentstroke}{rgb}{0.000000,0.000000,0.000000}%
\pgfsetstrokecolor{currentstroke}%
\pgfsetstrokeopacity{0.000000}%
\pgfsetdash{}{0pt}%
\pgfpathmoveto{\pgfqpoint{0.976323in}{1.731548in}}%
\pgfpathlineto{\pgfqpoint{0.985076in}{1.731548in}}%
\pgfpathlineto{\pgfqpoint{0.985076in}{1.745748in}}%
\pgfpathlineto{\pgfqpoint{0.976323in}{1.745748in}}%
\pgfpathlineto{\pgfqpoint{0.976323in}{1.731548in}}%
\pgfpathclose%
\pgfusepath{fill}%
\end{pgfscope}%
\begin{pgfscope}%
\pgfpathrectangle{\pgfqpoint{0.804646in}{0.600000in}}{\pgfqpoint{2.573292in}{2.070576in}}%
\pgfusepath{clip}%
\pgfsetbuttcap%
\pgfsetmiterjoin%
\definecolor{currentfill}{rgb}{0.302379,0.450282,0.300122}%
\pgfsetfillcolor{currentfill}%
\pgfsetlinewidth{0.000000pt}%
\definecolor{currentstroke}{rgb}{0.000000,0.000000,0.000000}%
\pgfsetstrokecolor{currentstroke}%
\pgfsetstrokeopacity{0.000000}%
\pgfsetdash{}{0pt}%
\pgfpathmoveto{\pgfqpoint{0.987264in}{1.798276in}}%
\pgfpathlineto{\pgfqpoint{0.996018in}{1.798276in}}%
\pgfpathlineto{\pgfqpoint{0.996018in}{1.803624in}}%
\pgfpathlineto{\pgfqpoint{0.987264in}{1.803624in}}%
\pgfpathlineto{\pgfqpoint{0.987264in}{1.798276in}}%
\pgfpathclose%
\pgfusepath{fill}%
\end{pgfscope}%
\begin{pgfscope}%
\pgfpathrectangle{\pgfqpoint{0.804646in}{0.600000in}}{\pgfqpoint{2.573292in}{2.070576in}}%
\pgfusepath{clip}%
\pgfsetbuttcap%
\pgfsetmiterjoin%
\definecolor{currentfill}{rgb}{0.302379,0.450282,0.300122}%
\pgfsetfillcolor{currentfill}%
\pgfsetlinewidth{0.000000pt}%
\definecolor{currentstroke}{rgb}{0.000000,0.000000,0.000000}%
\pgfsetstrokecolor{currentstroke}%
\pgfsetstrokeopacity{0.000000}%
\pgfsetdash{}{0pt}%
\pgfpathmoveto{\pgfqpoint{0.998206in}{1.516241in}}%
\pgfpathlineto{\pgfqpoint{1.006960in}{1.516241in}}%
\pgfpathlineto{\pgfqpoint{1.006960in}{1.511042in}}%
\pgfpathlineto{\pgfqpoint{0.998206in}{1.511042in}}%
\pgfpathlineto{\pgfqpoint{0.998206in}{1.516241in}}%
\pgfpathclose%
\pgfusepath{fill}%
\end{pgfscope}%
\begin{pgfscope}%
\pgfpathrectangle{\pgfqpoint{0.804646in}{0.600000in}}{\pgfqpoint{2.573292in}{2.070576in}}%
\pgfusepath{clip}%
\pgfsetbuttcap%
\pgfsetmiterjoin%
\definecolor{currentfill}{rgb}{0.302379,0.450282,0.300122}%
\pgfsetfillcolor{currentfill}%
\pgfsetlinewidth{0.000000pt}%
\definecolor{currentstroke}{rgb}{0.000000,0.000000,0.000000}%
\pgfsetstrokecolor{currentstroke}%
\pgfsetstrokeopacity{0.000000}%
\pgfsetdash{}{0pt}%
\pgfpathmoveto{\pgfqpoint{1.009148in}{1.540728in}}%
\pgfpathlineto{\pgfqpoint{1.017901in}{1.540728in}}%
\pgfpathlineto{\pgfqpoint{1.017901in}{1.535467in}}%
\pgfpathlineto{\pgfqpoint{1.009148in}{1.535467in}}%
\pgfpathlineto{\pgfqpoint{1.009148in}{1.540728in}}%
\pgfpathclose%
\pgfusepath{fill}%
\end{pgfscope}%
\begin{pgfscope}%
\pgfpathrectangle{\pgfqpoint{0.804646in}{0.600000in}}{\pgfqpoint{2.573292in}{2.070576in}}%
\pgfusepath{clip}%
\pgfsetbuttcap%
\pgfsetmiterjoin%
\definecolor{currentfill}{rgb}{0.302379,0.450282,0.300122}%
\pgfsetfillcolor{currentfill}%
\pgfsetlinewidth{0.000000pt}%
\definecolor{currentstroke}{rgb}{0.000000,0.000000,0.000000}%
\pgfsetstrokecolor{currentstroke}%
\pgfsetstrokeopacity{0.000000}%
\pgfsetdash{}{0pt}%
\pgfpathmoveto{\pgfqpoint{1.020090in}{1.534028in}}%
\pgfpathlineto{\pgfqpoint{1.028843in}{1.534028in}}%
\pgfpathlineto{\pgfqpoint{1.028843in}{1.528114in}}%
\pgfpathlineto{\pgfqpoint{1.020090in}{1.528114in}}%
\pgfpathlineto{\pgfqpoint{1.020090in}{1.534028in}}%
\pgfpathclose%
\pgfusepath{fill}%
\end{pgfscope}%
\begin{pgfscope}%
\pgfpathrectangle{\pgfqpoint{0.804646in}{0.600000in}}{\pgfqpoint{2.573292in}{2.070576in}}%
\pgfusepath{clip}%
\pgfsetbuttcap%
\pgfsetmiterjoin%
\definecolor{currentfill}{rgb}{0.302379,0.450282,0.300122}%
\pgfsetfillcolor{currentfill}%
\pgfsetlinewidth{0.000000pt}%
\definecolor{currentstroke}{rgb}{0.000000,0.000000,0.000000}%
\pgfsetstrokecolor{currentstroke}%
\pgfsetstrokeopacity{0.000000}%
\pgfsetdash{}{0pt}%
\pgfpathmoveto{\pgfqpoint{1.031032in}{1.556625in}}%
\pgfpathlineto{\pgfqpoint{1.039785in}{1.556625in}}%
\pgfpathlineto{\pgfqpoint{1.039785in}{1.548456in}}%
\pgfpathlineto{\pgfqpoint{1.031032in}{1.548456in}}%
\pgfpathlineto{\pgfqpoint{1.031032in}{1.556625in}}%
\pgfpathclose%
\pgfusepath{fill}%
\end{pgfscope}%
\begin{pgfscope}%
\pgfpathrectangle{\pgfqpoint{0.804646in}{0.600000in}}{\pgfqpoint{2.573292in}{2.070576in}}%
\pgfusepath{clip}%
\pgfsetbuttcap%
\pgfsetmiterjoin%
\definecolor{currentfill}{rgb}{0.302379,0.450282,0.300122}%
\pgfsetfillcolor{currentfill}%
\pgfsetlinewidth{0.000000pt}%
\definecolor{currentstroke}{rgb}{0.000000,0.000000,0.000000}%
\pgfsetstrokecolor{currentstroke}%
\pgfsetstrokeopacity{0.000000}%
\pgfsetdash{}{0pt}%
\pgfpathmoveto{\pgfqpoint{1.041973in}{1.567454in}}%
\pgfpathlineto{\pgfqpoint{1.050727in}{1.567454in}}%
\pgfpathlineto{\pgfqpoint{1.050727in}{1.553284in}}%
\pgfpathlineto{\pgfqpoint{1.041973in}{1.553284in}}%
\pgfpathlineto{\pgfqpoint{1.041973in}{1.567454in}}%
\pgfpathclose%
\pgfusepath{fill}%
\end{pgfscope}%
\begin{pgfscope}%
\pgfpathrectangle{\pgfqpoint{0.804646in}{0.600000in}}{\pgfqpoint{2.573292in}{2.070576in}}%
\pgfusepath{clip}%
\pgfsetbuttcap%
\pgfsetmiterjoin%
\definecolor{currentfill}{rgb}{0.302379,0.450282,0.300122}%
\pgfsetfillcolor{currentfill}%
\pgfsetlinewidth{0.000000pt}%
\definecolor{currentstroke}{rgb}{0.000000,0.000000,0.000000}%
\pgfsetstrokecolor{currentstroke}%
\pgfsetstrokeopacity{0.000000}%
\pgfsetdash{}{0pt}%
\pgfpathmoveto{\pgfqpoint{1.052915in}{1.575650in}}%
\pgfpathlineto{\pgfqpoint{1.061669in}{1.575650in}}%
\pgfpathlineto{\pgfqpoint{1.061669in}{1.552799in}}%
\pgfpathlineto{\pgfqpoint{1.052915in}{1.552799in}}%
\pgfpathlineto{\pgfqpoint{1.052915in}{1.575650in}}%
\pgfpathclose%
\pgfusepath{fill}%
\end{pgfscope}%
\begin{pgfscope}%
\pgfpathrectangle{\pgfqpoint{0.804646in}{0.600000in}}{\pgfqpoint{2.573292in}{2.070576in}}%
\pgfusepath{clip}%
\pgfsetbuttcap%
\pgfsetmiterjoin%
\definecolor{currentfill}{rgb}{0.302379,0.450282,0.300122}%
\pgfsetfillcolor{currentfill}%
\pgfsetlinewidth{0.000000pt}%
\definecolor{currentstroke}{rgb}{0.000000,0.000000,0.000000}%
\pgfsetstrokecolor{currentstroke}%
\pgfsetstrokeopacity{0.000000}%
\pgfsetdash{}{0pt}%
\pgfpathmoveto{\pgfqpoint{1.063857in}{1.567700in}}%
\pgfpathlineto{\pgfqpoint{1.072610in}{1.567700in}}%
\pgfpathlineto{\pgfqpoint{1.072610in}{1.548503in}}%
\pgfpathlineto{\pgfqpoint{1.063857in}{1.548503in}}%
\pgfpathlineto{\pgfqpoint{1.063857in}{1.567700in}}%
\pgfpathclose%
\pgfusepath{fill}%
\end{pgfscope}%
\begin{pgfscope}%
\pgfpathrectangle{\pgfqpoint{0.804646in}{0.600000in}}{\pgfqpoint{2.573292in}{2.070576in}}%
\pgfusepath{clip}%
\pgfsetbuttcap%
\pgfsetmiterjoin%
\definecolor{currentfill}{rgb}{0.302379,0.450282,0.300122}%
\pgfsetfillcolor{currentfill}%
\pgfsetlinewidth{0.000000pt}%
\definecolor{currentstroke}{rgb}{0.000000,0.000000,0.000000}%
\pgfsetstrokecolor{currentstroke}%
\pgfsetstrokeopacity{0.000000}%
\pgfsetdash{}{0pt}%
\pgfpathmoveto{\pgfqpoint{1.074799in}{1.562462in}}%
\pgfpathlineto{\pgfqpoint{1.083552in}{1.562462in}}%
\pgfpathlineto{\pgfqpoint{1.083552in}{1.541520in}}%
\pgfpathlineto{\pgfqpoint{1.074799in}{1.541520in}}%
\pgfpathlineto{\pgfqpoint{1.074799in}{1.562462in}}%
\pgfpathclose%
\pgfusepath{fill}%
\end{pgfscope}%
\begin{pgfscope}%
\pgfpathrectangle{\pgfqpoint{0.804646in}{0.600000in}}{\pgfqpoint{2.573292in}{2.070576in}}%
\pgfusepath{clip}%
\pgfsetbuttcap%
\pgfsetmiterjoin%
\definecolor{currentfill}{rgb}{0.302379,0.450282,0.300122}%
\pgfsetfillcolor{currentfill}%
\pgfsetlinewidth{0.000000pt}%
\definecolor{currentstroke}{rgb}{0.000000,0.000000,0.000000}%
\pgfsetstrokecolor{currentstroke}%
\pgfsetstrokeopacity{0.000000}%
\pgfsetdash{}{0pt}%
\pgfpathmoveto{\pgfqpoint{1.085741in}{1.512132in}}%
\pgfpathlineto{\pgfqpoint{1.094494in}{1.512132in}}%
\pgfpathlineto{\pgfqpoint{1.094494in}{1.500327in}}%
\pgfpathlineto{\pgfqpoint{1.085741in}{1.500327in}}%
\pgfpathlineto{\pgfqpoint{1.085741in}{1.512132in}}%
\pgfpathclose%
\pgfusepath{fill}%
\end{pgfscope}%
\begin{pgfscope}%
\pgfpathrectangle{\pgfqpoint{0.804646in}{0.600000in}}{\pgfqpoint{2.573292in}{2.070576in}}%
\pgfusepath{clip}%
\pgfsetbuttcap%
\pgfsetmiterjoin%
\definecolor{currentfill}{rgb}{0.302379,0.450282,0.300122}%
\pgfsetfillcolor{currentfill}%
\pgfsetlinewidth{0.000000pt}%
\definecolor{currentstroke}{rgb}{0.000000,0.000000,0.000000}%
\pgfsetstrokecolor{currentstroke}%
\pgfsetstrokeopacity{0.000000}%
\pgfsetdash{}{0pt}%
\pgfpathmoveto{\pgfqpoint{1.096682in}{1.984351in}}%
\pgfpathlineto{\pgfqpoint{1.105436in}{1.984351in}}%
\pgfpathlineto{\pgfqpoint{1.105436in}{1.987956in}}%
\pgfpathlineto{\pgfqpoint{1.096682in}{1.987956in}}%
\pgfpathlineto{\pgfqpoint{1.096682in}{1.984351in}}%
\pgfpathclose%
\pgfusepath{fill}%
\end{pgfscope}%
\begin{pgfscope}%
\pgfpathrectangle{\pgfqpoint{0.804646in}{0.600000in}}{\pgfqpoint{2.573292in}{2.070576in}}%
\pgfusepath{clip}%
\pgfsetbuttcap%
\pgfsetmiterjoin%
\definecolor{currentfill}{rgb}{0.302379,0.450282,0.300122}%
\pgfsetfillcolor{currentfill}%
\pgfsetlinewidth{0.000000pt}%
\definecolor{currentstroke}{rgb}{0.000000,0.000000,0.000000}%
\pgfsetstrokecolor{currentstroke}%
\pgfsetstrokeopacity{0.000000}%
\pgfsetdash{}{0pt}%
\pgfpathmoveto{\pgfqpoint{1.107624in}{1.909332in}}%
\pgfpathlineto{\pgfqpoint{1.116378in}{1.909332in}}%
\pgfpathlineto{\pgfqpoint{1.116378in}{1.919378in}}%
\pgfpathlineto{\pgfqpoint{1.107624in}{1.919378in}}%
\pgfpathlineto{\pgfqpoint{1.107624in}{1.909332in}}%
\pgfpathclose%
\pgfusepath{fill}%
\end{pgfscope}%
\begin{pgfscope}%
\pgfpathrectangle{\pgfqpoint{0.804646in}{0.600000in}}{\pgfqpoint{2.573292in}{2.070576in}}%
\pgfusepath{clip}%
\pgfsetbuttcap%
\pgfsetmiterjoin%
\definecolor{currentfill}{rgb}{0.302379,0.450282,0.300122}%
\pgfsetfillcolor{currentfill}%
\pgfsetlinewidth{0.000000pt}%
\definecolor{currentstroke}{rgb}{0.000000,0.000000,0.000000}%
\pgfsetstrokecolor{currentstroke}%
\pgfsetstrokeopacity{0.000000}%
\pgfsetdash{}{0pt}%
\pgfpathmoveto{\pgfqpoint{1.118566in}{1.973166in}}%
\pgfpathlineto{\pgfqpoint{1.127319in}{1.973166in}}%
\pgfpathlineto{\pgfqpoint{1.127319in}{1.997124in}}%
\pgfpathlineto{\pgfqpoint{1.118566in}{1.997124in}}%
\pgfpathlineto{\pgfqpoint{1.118566in}{1.973166in}}%
\pgfpathclose%
\pgfusepath{fill}%
\end{pgfscope}%
\begin{pgfscope}%
\pgfpathrectangle{\pgfqpoint{0.804646in}{0.600000in}}{\pgfqpoint{2.573292in}{2.070576in}}%
\pgfusepath{clip}%
\pgfsetbuttcap%
\pgfsetmiterjoin%
\definecolor{currentfill}{rgb}{0.302379,0.450282,0.300122}%
\pgfsetfillcolor{currentfill}%
\pgfsetlinewidth{0.000000pt}%
\definecolor{currentstroke}{rgb}{0.000000,0.000000,0.000000}%
\pgfsetstrokecolor{currentstroke}%
\pgfsetstrokeopacity{0.000000}%
\pgfsetdash{}{0pt}%
\pgfpathmoveto{\pgfqpoint{1.129508in}{1.960073in}}%
\pgfpathlineto{\pgfqpoint{1.138261in}{1.960073in}}%
\pgfpathlineto{\pgfqpoint{1.138261in}{1.980108in}}%
\pgfpathlineto{\pgfqpoint{1.129508in}{1.980108in}}%
\pgfpathlineto{\pgfqpoint{1.129508in}{1.960073in}}%
\pgfpathclose%
\pgfusepath{fill}%
\end{pgfscope}%
\begin{pgfscope}%
\pgfpathrectangle{\pgfqpoint{0.804646in}{0.600000in}}{\pgfqpoint{2.573292in}{2.070576in}}%
\pgfusepath{clip}%
\pgfsetbuttcap%
\pgfsetmiterjoin%
\definecolor{currentfill}{rgb}{0.302379,0.450282,0.300122}%
\pgfsetfillcolor{currentfill}%
\pgfsetlinewidth{0.000000pt}%
\definecolor{currentstroke}{rgb}{0.000000,0.000000,0.000000}%
\pgfsetstrokecolor{currentstroke}%
\pgfsetstrokeopacity{0.000000}%
\pgfsetdash{}{0pt}%
\pgfpathmoveto{\pgfqpoint{1.140450in}{2.004166in}}%
\pgfpathlineto{\pgfqpoint{1.149203in}{2.004166in}}%
\pgfpathlineto{\pgfqpoint{1.149203in}{2.015502in}}%
\pgfpathlineto{\pgfqpoint{1.140450in}{2.015502in}}%
\pgfpathlineto{\pgfqpoint{1.140450in}{2.004166in}}%
\pgfpathclose%
\pgfusepath{fill}%
\end{pgfscope}%
\begin{pgfscope}%
\pgfpathrectangle{\pgfqpoint{0.804646in}{0.600000in}}{\pgfqpoint{2.573292in}{2.070576in}}%
\pgfusepath{clip}%
\pgfsetbuttcap%
\pgfsetmiterjoin%
\definecolor{currentfill}{rgb}{0.302379,0.450282,0.300122}%
\pgfsetfillcolor{currentfill}%
\pgfsetlinewidth{0.000000pt}%
\definecolor{currentstroke}{rgb}{0.000000,0.000000,0.000000}%
\pgfsetstrokecolor{currentstroke}%
\pgfsetstrokeopacity{0.000000}%
\pgfsetdash{}{0pt}%
\pgfpathmoveto{\pgfqpoint{1.151391in}{2.058582in}}%
\pgfpathlineto{\pgfqpoint{1.160145in}{2.058582in}}%
\pgfpathlineto{\pgfqpoint{1.160145in}{2.072681in}}%
\pgfpathlineto{\pgfqpoint{1.151391in}{2.072681in}}%
\pgfpathlineto{\pgfqpoint{1.151391in}{2.058582in}}%
\pgfpathclose%
\pgfusepath{fill}%
\end{pgfscope}%
\begin{pgfscope}%
\pgfpathrectangle{\pgfqpoint{0.804646in}{0.600000in}}{\pgfqpoint{2.573292in}{2.070576in}}%
\pgfusepath{clip}%
\pgfsetbuttcap%
\pgfsetmiterjoin%
\definecolor{currentfill}{rgb}{0.302379,0.450282,0.300122}%
\pgfsetfillcolor{currentfill}%
\pgfsetlinewidth{0.000000pt}%
\definecolor{currentstroke}{rgb}{0.000000,0.000000,0.000000}%
\pgfsetstrokecolor{currentstroke}%
\pgfsetstrokeopacity{0.000000}%
\pgfsetdash{}{0pt}%
\pgfpathmoveto{\pgfqpoint{1.162333in}{2.183762in}}%
\pgfpathlineto{\pgfqpoint{1.171087in}{2.183762in}}%
\pgfpathlineto{\pgfqpoint{1.171087in}{2.213361in}}%
\pgfpathlineto{\pgfqpoint{1.162333in}{2.213361in}}%
\pgfpathlineto{\pgfqpoint{1.162333in}{2.183762in}}%
\pgfpathclose%
\pgfusepath{fill}%
\end{pgfscope}%
\begin{pgfscope}%
\pgfpathrectangle{\pgfqpoint{0.804646in}{0.600000in}}{\pgfqpoint{2.573292in}{2.070576in}}%
\pgfusepath{clip}%
\pgfsetbuttcap%
\pgfsetmiterjoin%
\definecolor{currentfill}{rgb}{0.302379,0.450282,0.300122}%
\pgfsetfillcolor{currentfill}%
\pgfsetlinewidth{0.000000pt}%
\definecolor{currentstroke}{rgb}{0.000000,0.000000,0.000000}%
\pgfsetstrokecolor{currentstroke}%
\pgfsetstrokeopacity{0.000000}%
\pgfsetdash{}{0pt}%
\pgfpathmoveto{\pgfqpoint{1.173275in}{2.279656in}}%
\pgfpathlineto{\pgfqpoint{1.182028in}{2.279656in}}%
\pgfpathlineto{\pgfqpoint{1.182028in}{2.303936in}}%
\pgfpathlineto{\pgfqpoint{1.173275in}{2.303936in}}%
\pgfpathlineto{\pgfqpoint{1.173275in}{2.279656in}}%
\pgfpathclose%
\pgfusepath{fill}%
\end{pgfscope}%
\begin{pgfscope}%
\pgfpathrectangle{\pgfqpoint{0.804646in}{0.600000in}}{\pgfqpoint{2.573292in}{2.070576in}}%
\pgfusepath{clip}%
\pgfsetbuttcap%
\pgfsetmiterjoin%
\definecolor{currentfill}{rgb}{0.302379,0.450282,0.300122}%
\pgfsetfillcolor{currentfill}%
\pgfsetlinewidth{0.000000pt}%
\definecolor{currentstroke}{rgb}{0.000000,0.000000,0.000000}%
\pgfsetstrokecolor{currentstroke}%
\pgfsetstrokeopacity{0.000000}%
\pgfsetdash{}{0pt}%
\pgfpathmoveto{\pgfqpoint{1.184217in}{2.242173in}}%
\pgfpathlineto{\pgfqpoint{1.192970in}{2.242173in}}%
\pgfpathlineto{\pgfqpoint{1.192970in}{2.258912in}}%
\pgfpathlineto{\pgfqpoint{1.184217in}{2.258912in}}%
\pgfpathlineto{\pgfqpoint{1.184217in}{2.242173in}}%
\pgfpathclose%
\pgfusepath{fill}%
\end{pgfscope}%
\begin{pgfscope}%
\pgfpathrectangle{\pgfqpoint{0.804646in}{0.600000in}}{\pgfqpoint{2.573292in}{2.070576in}}%
\pgfusepath{clip}%
\pgfsetbuttcap%
\pgfsetmiterjoin%
\definecolor{currentfill}{rgb}{0.302379,0.450282,0.300122}%
\pgfsetfillcolor{currentfill}%
\pgfsetlinewidth{0.000000pt}%
\definecolor{currentstroke}{rgb}{0.000000,0.000000,0.000000}%
\pgfsetstrokecolor{currentstroke}%
\pgfsetstrokeopacity{0.000000}%
\pgfsetdash{}{0pt}%
\pgfpathmoveto{\pgfqpoint{1.195159in}{2.276815in}}%
\pgfpathlineto{\pgfqpoint{1.203912in}{2.276815in}}%
\pgfpathlineto{\pgfqpoint{1.203912in}{2.287781in}}%
\pgfpathlineto{\pgfqpoint{1.195159in}{2.287781in}}%
\pgfpathlineto{\pgfqpoint{1.195159in}{2.276815in}}%
\pgfpathclose%
\pgfusepath{fill}%
\end{pgfscope}%
\begin{pgfscope}%
\pgfpathrectangle{\pgfqpoint{0.804646in}{0.600000in}}{\pgfqpoint{2.573292in}{2.070576in}}%
\pgfusepath{clip}%
\pgfsetbuttcap%
\pgfsetmiterjoin%
\definecolor{currentfill}{rgb}{0.302379,0.450282,0.300122}%
\pgfsetfillcolor{currentfill}%
\pgfsetlinewidth{0.000000pt}%
\definecolor{currentstroke}{rgb}{0.000000,0.000000,0.000000}%
\pgfsetstrokecolor{currentstroke}%
\pgfsetstrokeopacity{0.000000}%
\pgfsetdash{}{0pt}%
\pgfpathmoveto{\pgfqpoint{1.206100in}{2.313304in}}%
\pgfpathlineto{\pgfqpoint{1.214854in}{2.313304in}}%
\pgfpathlineto{\pgfqpoint{1.214854in}{2.314269in}}%
\pgfpathlineto{\pgfqpoint{1.206100in}{2.314269in}}%
\pgfpathlineto{\pgfqpoint{1.206100in}{2.313304in}}%
\pgfpathclose%
\pgfusepath{fill}%
\end{pgfscope}%
\begin{pgfscope}%
\pgfpathrectangle{\pgfqpoint{0.804646in}{0.600000in}}{\pgfqpoint{2.573292in}{2.070576in}}%
\pgfusepath{clip}%
\pgfsetbuttcap%
\pgfsetmiterjoin%
\definecolor{currentfill}{rgb}{0.302379,0.450282,0.300122}%
\pgfsetfillcolor{currentfill}%
\pgfsetlinewidth{0.000000pt}%
\definecolor{currentstroke}{rgb}{0.000000,0.000000,0.000000}%
\pgfsetstrokecolor{currentstroke}%
\pgfsetstrokeopacity{0.000000}%
\pgfsetdash{}{0pt}%
\pgfpathmoveto{\pgfqpoint{1.217042in}{1.555280in}}%
\pgfpathlineto{\pgfqpoint{1.225796in}{1.555280in}}%
\pgfpathlineto{\pgfqpoint{1.225796in}{1.541524in}}%
\pgfpathlineto{\pgfqpoint{1.217042in}{1.541524in}}%
\pgfpathlineto{\pgfqpoint{1.217042in}{1.555280in}}%
\pgfpathclose%
\pgfusepath{fill}%
\end{pgfscope}%
\begin{pgfscope}%
\pgfpathrectangle{\pgfqpoint{0.804646in}{0.600000in}}{\pgfqpoint{2.573292in}{2.070576in}}%
\pgfusepath{clip}%
\pgfsetbuttcap%
\pgfsetmiterjoin%
\definecolor{currentfill}{rgb}{0.302379,0.450282,0.300122}%
\pgfsetfillcolor{currentfill}%
\pgfsetlinewidth{0.000000pt}%
\definecolor{currentstroke}{rgb}{0.000000,0.000000,0.000000}%
\pgfsetstrokecolor{currentstroke}%
\pgfsetstrokeopacity{0.000000}%
\pgfsetdash{}{0pt}%
\pgfpathmoveto{\pgfqpoint{1.227984in}{1.577050in}}%
\pgfpathlineto{\pgfqpoint{1.236737in}{1.577050in}}%
\pgfpathlineto{\pgfqpoint{1.236737in}{1.534934in}}%
\pgfpathlineto{\pgfqpoint{1.227984in}{1.534934in}}%
\pgfpathlineto{\pgfqpoint{1.227984in}{1.577050in}}%
\pgfpathclose%
\pgfusepath{fill}%
\end{pgfscope}%
\begin{pgfscope}%
\pgfpathrectangle{\pgfqpoint{0.804646in}{0.600000in}}{\pgfqpoint{2.573292in}{2.070576in}}%
\pgfusepath{clip}%
\pgfsetbuttcap%
\pgfsetmiterjoin%
\definecolor{currentfill}{rgb}{0.302379,0.450282,0.300122}%
\pgfsetfillcolor{currentfill}%
\pgfsetlinewidth{0.000000pt}%
\definecolor{currentstroke}{rgb}{0.000000,0.000000,0.000000}%
\pgfsetstrokecolor{currentstroke}%
\pgfsetstrokeopacity{0.000000}%
\pgfsetdash{}{0pt}%
\pgfpathmoveto{\pgfqpoint{1.238926in}{1.571032in}}%
\pgfpathlineto{\pgfqpoint{1.247679in}{1.571032in}}%
\pgfpathlineto{\pgfqpoint{1.247679in}{1.530828in}}%
\pgfpathlineto{\pgfqpoint{1.238926in}{1.530828in}}%
\pgfpathlineto{\pgfqpoint{1.238926in}{1.571032in}}%
\pgfpathclose%
\pgfusepath{fill}%
\end{pgfscope}%
\begin{pgfscope}%
\pgfpathrectangle{\pgfqpoint{0.804646in}{0.600000in}}{\pgfqpoint{2.573292in}{2.070576in}}%
\pgfusepath{clip}%
\pgfsetbuttcap%
\pgfsetmiterjoin%
\definecolor{currentfill}{rgb}{0.302379,0.450282,0.300122}%
\pgfsetfillcolor{currentfill}%
\pgfsetlinewidth{0.000000pt}%
\definecolor{currentstroke}{rgb}{0.000000,0.000000,0.000000}%
\pgfsetstrokecolor{currentstroke}%
\pgfsetstrokeopacity{0.000000}%
\pgfsetdash{}{0pt}%
\pgfpathmoveto{\pgfqpoint{1.249868in}{1.558615in}}%
\pgfpathlineto{\pgfqpoint{1.258621in}{1.558615in}}%
\pgfpathlineto{\pgfqpoint{1.258621in}{1.522874in}}%
\pgfpathlineto{\pgfqpoint{1.249868in}{1.522874in}}%
\pgfpathlineto{\pgfqpoint{1.249868in}{1.558615in}}%
\pgfpathclose%
\pgfusepath{fill}%
\end{pgfscope}%
\begin{pgfscope}%
\pgfpathrectangle{\pgfqpoint{0.804646in}{0.600000in}}{\pgfqpoint{2.573292in}{2.070576in}}%
\pgfusepath{clip}%
\pgfsetbuttcap%
\pgfsetmiterjoin%
\definecolor{currentfill}{rgb}{0.302379,0.450282,0.300122}%
\pgfsetfillcolor{currentfill}%
\pgfsetlinewidth{0.000000pt}%
\definecolor{currentstroke}{rgb}{0.000000,0.000000,0.000000}%
\pgfsetstrokecolor{currentstroke}%
\pgfsetstrokeopacity{0.000000}%
\pgfsetdash{}{0pt}%
\pgfpathmoveto{\pgfqpoint{1.260809in}{1.560219in}}%
\pgfpathlineto{\pgfqpoint{1.269563in}{1.560219in}}%
\pgfpathlineto{\pgfqpoint{1.269563in}{1.512039in}}%
\pgfpathlineto{\pgfqpoint{1.260809in}{1.512039in}}%
\pgfpathlineto{\pgfqpoint{1.260809in}{1.560219in}}%
\pgfpathclose%
\pgfusepath{fill}%
\end{pgfscope}%
\begin{pgfscope}%
\pgfpathrectangle{\pgfqpoint{0.804646in}{0.600000in}}{\pgfqpoint{2.573292in}{2.070576in}}%
\pgfusepath{clip}%
\pgfsetbuttcap%
\pgfsetmiterjoin%
\definecolor{currentfill}{rgb}{0.302379,0.450282,0.300122}%
\pgfsetfillcolor{currentfill}%
\pgfsetlinewidth{0.000000pt}%
\definecolor{currentstroke}{rgb}{0.000000,0.000000,0.000000}%
\pgfsetstrokecolor{currentstroke}%
\pgfsetstrokeopacity{0.000000}%
\pgfsetdash{}{0pt}%
\pgfpathmoveto{\pgfqpoint{1.271751in}{1.538537in}}%
\pgfpathlineto{\pgfqpoint{1.280505in}{1.538537in}}%
\pgfpathlineto{\pgfqpoint{1.280505in}{1.473024in}}%
\pgfpathlineto{\pgfqpoint{1.271751in}{1.473024in}}%
\pgfpathlineto{\pgfqpoint{1.271751in}{1.538537in}}%
\pgfpathclose%
\pgfusepath{fill}%
\end{pgfscope}%
\begin{pgfscope}%
\pgfpathrectangle{\pgfqpoint{0.804646in}{0.600000in}}{\pgfqpoint{2.573292in}{2.070576in}}%
\pgfusepath{clip}%
\pgfsetbuttcap%
\pgfsetmiterjoin%
\definecolor{currentfill}{rgb}{0.302379,0.450282,0.300122}%
\pgfsetfillcolor{currentfill}%
\pgfsetlinewidth{0.000000pt}%
\definecolor{currentstroke}{rgb}{0.000000,0.000000,0.000000}%
\pgfsetstrokecolor{currentstroke}%
\pgfsetstrokeopacity{0.000000}%
\pgfsetdash{}{0pt}%
\pgfpathmoveto{\pgfqpoint{1.282693in}{1.482861in}}%
\pgfpathlineto{\pgfqpoint{1.291446in}{1.482861in}}%
\pgfpathlineto{\pgfqpoint{1.291446in}{1.437865in}}%
\pgfpathlineto{\pgfqpoint{1.282693in}{1.437865in}}%
\pgfpathlineto{\pgfqpoint{1.282693in}{1.482861in}}%
\pgfpathclose%
\pgfusepath{fill}%
\end{pgfscope}%
\begin{pgfscope}%
\pgfpathrectangle{\pgfqpoint{0.804646in}{0.600000in}}{\pgfqpoint{2.573292in}{2.070576in}}%
\pgfusepath{clip}%
\pgfsetbuttcap%
\pgfsetmiterjoin%
\definecolor{currentfill}{rgb}{0.302379,0.450282,0.300122}%
\pgfsetfillcolor{currentfill}%
\pgfsetlinewidth{0.000000pt}%
\definecolor{currentstroke}{rgb}{0.000000,0.000000,0.000000}%
\pgfsetstrokecolor{currentstroke}%
\pgfsetstrokeopacity{0.000000}%
\pgfsetdash{}{0pt}%
\pgfpathmoveto{\pgfqpoint{1.293635in}{1.417427in}}%
\pgfpathlineto{\pgfqpoint{1.302388in}{1.417427in}}%
\pgfpathlineto{\pgfqpoint{1.302388in}{1.391210in}}%
\pgfpathlineto{\pgfqpoint{1.293635in}{1.391210in}}%
\pgfpathlineto{\pgfqpoint{1.293635in}{1.417427in}}%
\pgfpathclose%
\pgfusepath{fill}%
\end{pgfscope}%
\begin{pgfscope}%
\pgfpathrectangle{\pgfqpoint{0.804646in}{0.600000in}}{\pgfqpoint{2.573292in}{2.070576in}}%
\pgfusepath{clip}%
\pgfsetbuttcap%
\pgfsetmiterjoin%
\definecolor{currentfill}{rgb}{0.302379,0.450282,0.300122}%
\pgfsetfillcolor{currentfill}%
\pgfsetlinewidth{0.000000pt}%
\definecolor{currentstroke}{rgb}{0.000000,0.000000,0.000000}%
\pgfsetstrokecolor{currentstroke}%
\pgfsetstrokeopacity{0.000000}%
\pgfsetdash{}{0pt}%
\pgfpathmoveto{\pgfqpoint{1.304577in}{1.384983in}}%
\pgfpathlineto{\pgfqpoint{1.313330in}{1.384983in}}%
\pgfpathlineto{\pgfqpoint{1.313330in}{1.373521in}}%
\pgfpathlineto{\pgfqpoint{1.304577in}{1.373521in}}%
\pgfpathlineto{\pgfqpoint{1.304577in}{1.384983in}}%
\pgfpathclose%
\pgfusepath{fill}%
\end{pgfscope}%
\begin{pgfscope}%
\pgfpathrectangle{\pgfqpoint{0.804646in}{0.600000in}}{\pgfqpoint{2.573292in}{2.070576in}}%
\pgfusepath{clip}%
\pgfsetbuttcap%
\pgfsetmiterjoin%
\definecolor{currentfill}{rgb}{0.302379,0.450282,0.300122}%
\pgfsetfillcolor{currentfill}%
\pgfsetlinewidth{0.000000pt}%
\definecolor{currentstroke}{rgb}{0.000000,0.000000,0.000000}%
\pgfsetstrokecolor{currentstroke}%
\pgfsetstrokeopacity{0.000000}%
\pgfsetdash{}{0pt}%
\pgfpathmoveto{\pgfqpoint{1.315518in}{1.401926in}}%
\pgfpathlineto{\pgfqpoint{1.324272in}{1.401926in}}%
\pgfpathlineto{\pgfqpoint{1.324272in}{1.379122in}}%
\pgfpathlineto{\pgfqpoint{1.315518in}{1.379122in}}%
\pgfpathlineto{\pgfqpoint{1.315518in}{1.401926in}}%
\pgfpathclose%
\pgfusepath{fill}%
\end{pgfscope}%
\begin{pgfscope}%
\pgfpathrectangle{\pgfqpoint{0.804646in}{0.600000in}}{\pgfqpoint{2.573292in}{2.070576in}}%
\pgfusepath{clip}%
\pgfsetbuttcap%
\pgfsetmiterjoin%
\definecolor{currentfill}{rgb}{0.302379,0.450282,0.300122}%
\pgfsetfillcolor{currentfill}%
\pgfsetlinewidth{0.000000pt}%
\definecolor{currentstroke}{rgb}{0.000000,0.000000,0.000000}%
\pgfsetstrokecolor{currentstroke}%
\pgfsetstrokeopacity{0.000000}%
\pgfsetdash{}{0pt}%
\pgfpathmoveto{\pgfqpoint{1.326460in}{1.420210in}}%
\pgfpathlineto{\pgfqpoint{1.335214in}{1.420210in}}%
\pgfpathlineto{\pgfqpoint{1.335214in}{1.398562in}}%
\pgfpathlineto{\pgfqpoint{1.326460in}{1.398562in}}%
\pgfpathlineto{\pgfqpoint{1.326460in}{1.420210in}}%
\pgfpathclose%
\pgfusepath{fill}%
\end{pgfscope}%
\begin{pgfscope}%
\pgfpathrectangle{\pgfqpoint{0.804646in}{0.600000in}}{\pgfqpoint{2.573292in}{2.070576in}}%
\pgfusepath{clip}%
\pgfsetbuttcap%
\pgfsetmiterjoin%
\definecolor{currentfill}{rgb}{0.302379,0.450282,0.300122}%
\pgfsetfillcolor{currentfill}%
\pgfsetlinewidth{0.000000pt}%
\definecolor{currentstroke}{rgb}{0.000000,0.000000,0.000000}%
\pgfsetstrokecolor{currentstroke}%
\pgfsetstrokeopacity{0.000000}%
\pgfsetdash{}{0pt}%
\pgfpathmoveto{\pgfqpoint{1.337402in}{1.461598in}}%
\pgfpathlineto{\pgfqpoint{1.346155in}{1.461598in}}%
\pgfpathlineto{\pgfqpoint{1.346155in}{1.450069in}}%
\pgfpathlineto{\pgfqpoint{1.337402in}{1.450069in}}%
\pgfpathlineto{\pgfqpoint{1.337402in}{1.461598in}}%
\pgfpathclose%
\pgfusepath{fill}%
\end{pgfscope}%
\begin{pgfscope}%
\pgfpathrectangle{\pgfqpoint{0.804646in}{0.600000in}}{\pgfqpoint{2.573292in}{2.070576in}}%
\pgfusepath{clip}%
\pgfsetbuttcap%
\pgfsetmiterjoin%
\definecolor{currentfill}{rgb}{0.302379,0.450282,0.300122}%
\pgfsetfillcolor{currentfill}%
\pgfsetlinewidth{0.000000pt}%
\definecolor{currentstroke}{rgb}{0.000000,0.000000,0.000000}%
\pgfsetstrokecolor{currentstroke}%
\pgfsetstrokeopacity{0.000000}%
\pgfsetdash{}{0pt}%
\pgfpathmoveto{\pgfqpoint{1.348344in}{1.455611in}}%
\pgfpathlineto{\pgfqpoint{1.357097in}{1.455611in}}%
\pgfpathlineto{\pgfqpoint{1.357097in}{1.446229in}}%
\pgfpathlineto{\pgfqpoint{1.348344in}{1.446229in}}%
\pgfpathlineto{\pgfqpoint{1.348344in}{1.455611in}}%
\pgfpathclose%
\pgfusepath{fill}%
\end{pgfscope}%
\begin{pgfscope}%
\pgfpathrectangle{\pgfqpoint{0.804646in}{0.600000in}}{\pgfqpoint{2.573292in}{2.070576in}}%
\pgfusepath{clip}%
\pgfsetbuttcap%
\pgfsetmiterjoin%
\definecolor{currentfill}{rgb}{0.302379,0.450282,0.300122}%
\pgfsetfillcolor{currentfill}%
\pgfsetlinewidth{0.000000pt}%
\definecolor{currentstroke}{rgb}{0.000000,0.000000,0.000000}%
\pgfsetstrokecolor{currentstroke}%
\pgfsetstrokeopacity{0.000000}%
\pgfsetdash{}{0pt}%
\pgfpathmoveto{\pgfqpoint{1.359286in}{1.458254in}}%
\pgfpathlineto{\pgfqpoint{1.368039in}{1.458254in}}%
\pgfpathlineto{\pgfqpoint{1.368039in}{1.442991in}}%
\pgfpathlineto{\pgfqpoint{1.359286in}{1.442991in}}%
\pgfpathlineto{\pgfqpoint{1.359286in}{1.458254in}}%
\pgfpathclose%
\pgfusepath{fill}%
\end{pgfscope}%
\begin{pgfscope}%
\pgfpathrectangle{\pgfqpoint{0.804646in}{0.600000in}}{\pgfqpoint{2.573292in}{2.070576in}}%
\pgfusepath{clip}%
\pgfsetbuttcap%
\pgfsetmiterjoin%
\definecolor{currentfill}{rgb}{0.302379,0.450282,0.300122}%
\pgfsetfillcolor{currentfill}%
\pgfsetlinewidth{0.000000pt}%
\definecolor{currentstroke}{rgb}{0.000000,0.000000,0.000000}%
\pgfsetstrokecolor{currentstroke}%
\pgfsetstrokeopacity{0.000000}%
\pgfsetdash{}{0pt}%
\pgfpathmoveto{\pgfqpoint{1.370227in}{1.448702in}}%
\pgfpathlineto{\pgfqpoint{1.378981in}{1.448702in}}%
\pgfpathlineto{\pgfqpoint{1.378981in}{1.432015in}}%
\pgfpathlineto{\pgfqpoint{1.370227in}{1.432015in}}%
\pgfpathlineto{\pgfqpoint{1.370227in}{1.448702in}}%
\pgfpathclose%
\pgfusepath{fill}%
\end{pgfscope}%
\begin{pgfscope}%
\pgfpathrectangle{\pgfqpoint{0.804646in}{0.600000in}}{\pgfqpoint{2.573292in}{2.070576in}}%
\pgfusepath{clip}%
\pgfsetbuttcap%
\pgfsetmiterjoin%
\definecolor{currentfill}{rgb}{0.302379,0.450282,0.300122}%
\pgfsetfillcolor{currentfill}%
\pgfsetlinewidth{0.000000pt}%
\definecolor{currentstroke}{rgb}{0.000000,0.000000,0.000000}%
\pgfsetstrokecolor{currentstroke}%
\pgfsetstrokeopacity{0.000000}%
\pgfsetdash{}{0pt}%
\pgfpathmoveto{\pgfqpoint{1.381169in}{1.469662in}}%
\pgfpathlineto{\pgfqpoint{1.389923in}{1.469662in}}%
\pgfpathlineto{\pgfqpoint{1.389923in}{1.454196in}}%
\pgfpathlineto{\pgfqpoint{1.381169in}{1.454196in}}%
\pgfpathlineto{\pgfqpoint{1.381169in}{1.469662in}}%
\pgfpathclose%
\pgfusepath{fill}%
\end{pgfscope}%
\begin{pgfscope}%
\pgfpathrectangle{\pgfqpoint{0.804646in}{0.600000in}}{\pgfqpoint{2.573292in}{2.070576in}}%
\pgfusepath{clip}%
\pgfsetbuttcap%
\pgfsetmiterjoin%
\definecolor{currentfill}{rgb}{0.302379,0.450282,0.300122}%
\pgfsetfillcolor{currentfill}%
\pgfsetlinewidth{0.000000pt}%
\definecolor{currentstroke}{rgb}{0.000000,0.000000,0.000000}%
\pgfsetstrokecolor{currentstroke}%
\pgfsetstrokeopacity{0.000000}%
\pgfsetdash{}{0pt}%
\pgfpathmoveto{\pgfqpoint{1.392111in}{1.505376in}}%
\pgfpathlineto{\pgfqpoint{1.400864in}{1.505376in}}%
\pgfpathlineto{\pgfqpoint{1.400864in}{1.487831in}}%
\pgfpathlineto{\pgfqpoint{1.392111in}{1.487831in}}%
\pgfpathlineto{\pgfqpoint{1.392111in}{1.505376in}}%
\pgfpathclose%
\pgfusepath{fill}%
\end{pgfscope}%
\begin{pgfscope}%
\pgfpathrectangle{\pgfqpoint{0.804646in}{0.600000in}}{\pgfqpoint{2.573292in}{2.070576in}}%
\pgfusepath{clip}%
\pgfsetbuttcap%
\pgfsetmiterjoin%
\definecolor{currentfill}{rgb}{0.302379,0.450282,0.300122}%
\pgfsetfillcolor{currentfill}%
\pgfsetlinewidth{0.000000pt}%
\definecolor{currentstroke}{rgb}{0.000000,0.000000,0.000000}%
\pgfsetstrokecolor{currentstroke}%
\pgfsetstrokeopacity{0.000000}%
\pgfsetdash{}{0pt}%
\pgfpathmoveto{\pgfqpoint{1.403053in}{1.525212in}}%
\pgfpathlineto{\pgfqpoint{1.411806in}{1.525212in}}%
\pgfpathlineto{\pgfqpoint{1.411806in}{1.504632in}}%
\pgfpathlineto{\pgfqpoint{1.403053in}{1.504632in}}%
\pgfpathlineto{\pgfqpoint{1.403053in}{1.525212in}}%
\pgfpathclose%
\pgfusepath{fill}%
\end{pgfscope}%
\begin{pgfscope}%
\pgfpathrectangle{\pgfqpoint{0.804646in}{0.600000in}}{\pgfqpoint{2.573292in}{2.070576in}}%
\pgfusepath{clip}%
\pgfsetbuttcap%
\pgfsetmiterjoin%
\definecolor{currentfill}{rgb}{0.302379,0.450282,0.300122}%
\pgfsetfillcolor{currentfill}%
\pgfsetlinewidth{0.000000pt}%
\definecolor{currentstroke}{rgb}{0.000000,0.000000,0.000000}%
\pgfsetstrokecolor{currentstroke}%
\pgfsetstrokeopacity{0.000000}%
\pgfsetdash{}{0pt}%
\pgfpathmoveto{\pgfqpoint{1.413995in}{1.536211in}}%
\pgfpathlineto{\pgfqpoint{1.422748in}{1.536211in}}%
\pgfpathlineto{\pgfqpoint{1.422748in}{1.498440in}}%
\pgfpathlineto{\pgfqpoint{1.413995in}{1.498440in}}%
\pgfpathlineto{\pgfqpoint{1.413995in}{1.536211in}}%
\pgfpathclose%
\pgfusepath{fill}%
\end{pgfscope}%
\begin{pgfscope}%
\pgfpathrectangle{\pgfqpoint{0.804646in}{0.600000in}}{\pgfqpoint{2.573292in}{2.070576in}}%
\pgfusepath{clip}%
\pgfsetbuttcap%
\pgfsetmiterjoin%
\definecolor{currentfill}{rgb}{0.302379,0.450282,0.300122}%
\pgfsetfillcolor{currentfill}%
\pgfsetlinewidth{0.000000pt}%
\definecolor{currentstroke}{rgb}{0.000000,0.000000,0.000000}%
\pgfsetstrokecolor{currentstroke}%
\pgfsetstrokeopacity{0.000000}%
\pgfsetdash{}{0pt}%
\pgfpathmoveto{\pgfqpoint{1.424936in}{1.541848in}}%
\pgfpathlineto{\pgfqpoint{1.433690in}{1.541848in}}%
\pgfpathlineto{\pgfqpoint{1.433690in}{1.507902in}}%
\pgfpathlineto{\pgfqpoint{1.424936in}{1.507902in}}%
\pgfpathlineto{\pgfqpoint{1.424936in}{1.541848in}}%
\pgfpathclose%
\pgfusepath{fill}%
\end{pgfscope}%
\begin{pgfscope}%
\pgfpathrectangle{\pgfqpoint{0.804646in}{0.600000in}}{\pgfqpoint{2.573292in}{2.070576in}}%
\pgfusepath{clip}%
\pgfsetbuttcap%
\pgfsetmiterjoin%
\definecolor{currentfill}{rgb}{0.302379,0.450282,0.300122}%
\pgfsetfillcolor{currentfill}%
\pgfsetlinewidth{0.000000pt}%
\definecolor{currentstroke}{rgb}{0.000000,0.000000,0.000000}%
\pgfsetstrokecolor{currentstroke}%
\pgfsetstrokeopacity{0.000000}%
\pgfsetdash{}{0pt}%
\pgfpathmoveto{\pgfqpoint{1.435878in}{1.593553in}}%
\pgfpathlineto{\pgfqpoint{1.444632in}{1.593553in}}%
\pgfpathlineto{\pgfqpoint{1.444632in}{1.554041in}}%
\pgfpathlineto{\pgfqpoint{1.435878in}{1.554041in}}%
\pgfpathlineto{\pgfqpoint{1.435878in}{1.593553in}}%
\pgfpathclose%
\pgfusepath{fill}%
\end{pgfscope}%
\begin{pgfscope}%
\pgfpathrectangle{\pgfqpoint{0.804646in}{0.600000in}}{\pgfqpoint{2.573292in}{2.070576in}}%
\pgfusepath{clip}%
\pgfsetbuttcap%
\pgfsetmiterjoin%
\definecolor{currentfill}{rgb}{0.302379,0.450282,0.300122}%
\pgfsetfillcolor{currentfill}%
\pgfsetlinewidth{0.000000pt}%
\definecolor{currentstroke}{rgb}{0.000000,0.000000,0.000000}%
\pgfsetstrokecolor{currentstroke}%
\pgfsetstrokeopacity{0.000000}%
\pgfsetdash{}{0pt}%
\pgfpathmoveto{\pgfqpoint{1.446820in}{1.613063in}}%
\pgfpathlineto{\pgfqpoint{1.455573in}{1.613063in}}%
\pgfpathlineto{\pgfqpoint{1.455573in}{1.568310in}}%
\pgfpathlineto{\pgfqpoint{1.446820in}{1.568310in}}%
\pgfpathlineto{\pgfqpoint{1.446820in}{1.613063in}}%
\pgfpathclose%
\pgfusepath{fill}%
\end{pgfscope}%
\begin{pgfscope}%
\pgfpathrectangle{\pgfqpoint{0.804646in}{0.600000in}}{\pgfqpoint{2.573292in}{2.070576in}}%
\pgfusepath{clip}%
\pgfsetbuttcap%
\pgfsetmiterjoin%
\definecolor{currentfill}{rgb}{0.302379,0.450282,0.300122}%
\pgfsetfillcolor{currentfill}%
\pgfsetlinewidth{0.000000pt}%
\definecolor{currentstroke}{rgb}{0.000000,0.000000,0.000000}%
\pgfsetstrokecolor{currentstroke}%
\pgfsetstrokeopacity{0.000000}%
\pgfsetdash{}{0pt}%
\pgfpathmoveto{\pgfqpoint{1.457762in}{1.613090in}}%
\pgfpathlineto{\pgfqpoint{1.466515in}{1.613090in}}%
\pgfpathlineto{\pgfqpoint{1.466515in}{1.557697in}}%
\pgfpathlineto{\pgfqpoint{1.457762in}{1.557697in}}%
\pgfpathlineto{\pgfqpoint{1.457762in}{1.613090in}}%
\pgfpathclose%
\pgfusepath{fill}%
\end{pgfscope}%
\begin{pgfscope}%
\pgfpathrectangle{\pgfqpoint{0.804646in}{0.600000in}}{\pgfqpoint{2.573292in}{2.070576in}}%
\pgfusepath{clip}%
\pgfsetbuttcap%
\pgfsetmiterjoin%
\definecolor{currentfill}{rgb}{0.302379,0.450282,0.300122}%
\pgfsetfillcolor{currentfill}%
\pgfsetlinewidth{0.000000pt}%
\definecolor{currentstroke}{rgb}{0.000000,0.000000,0.000000}%
\pgfsetstrokecolor{currentstroke}%
\pgfsetstrokeopacity{0.000000}%
\pgfsetdash{}{0pt}%
\pgfpathmoveto{\pgfqpoint{1.468704in}{1.613090in}}%
\pgfpathlineto{\pgfqpoint{1.477457in}{1.613090in}}%
\pgfpathlineto{\pgfqpoint{1.477457in}{1.552003in}}%
\pgfpathlineto{\pgfqpoint{1.468704in}{1.552003in}}%
\pgfpathlineto{\pgfqpoint{1.468704in}{1.613090in}}%
\pgfpathclose%
\pgfusepath{fill}%
\end{pgfscope}%
\begin{pgfscope}%
\pgfpathrectangle{\pgfqpoint{0.804646in}{0.600000in}}{\pgfqpoint{2.573292in}{2.070576in}}%
\pgfusepath{clip}%
\pgfsetbuttcap%
\pgfsetmiterjoin%
\definecolor{currentfill}{rgb}{0.302379,0.450282,0.300122}%
\pgfsetfillcolor{currentfill}%
\pgfsetlinewidth{0.000000pt}%
\definecolor{currentstroke}{rgb}{0.000000,0.000000,0.000000}%
\pgfsetstrokecolor{currentstroke}%
\pgfsetstrokeopacity{0.000000}%
\pgfsetdash{}{0pt}%
\pgfpathmoveto{\pgfqpoint{1.479645in}{1.613090in}}%
\pgfpathlineto{\pgfqpoint{1.488399in}{1.613090in}}%
\pgfpathlineto{\pgfqpoint{1.488399in}{1.546180in}}%
\pgfpathlineto{\pgfqpoint{1.479645in}{1.546180in}}%
\pgfpathlineto{\pgfqpoint{1.479645in}{1.613090in}}%
\pgfpathclose%
\pgfusepath{fill}%
\end{pgfscope}%
\begin{pgfscope}%
\pgfpathrectangle{\pgfqpoint{0.804646in}{0.600000in}}{\pgfqpoint{2.573292in}{2.070576in}}%
\pgfusepath{clip}%
\pgfsetbuttcap%
\pgfsetmiterjoin%
\definecolor{currentfill}{rgb}{0.302379,0.450282,0.300122}%
\pgfsetfillcolor{currentfill}%
\pgfsetlinewidth{0.000000pt}%
\definecolor{currentstroke}{rgb}{0.000000,0.000000,0.000000}%
\pgfsetstrokecolor{currentstroke}%
\pgfsetstrokeopacity{0.000000}%
\pgfsetdash{}{0pt}%
\pgfpathmoveto{\pgfqpoint{1.490587in}{1.613090in}}%
\pgfpathlineto{\pgfqpoint{1.499341in}{1.613090in}}%
\pgfpathlineto{\pgfqpoint{1.499341in}{1.549959in}}%
\pgfpathlineto{\pgfqpoint{1.490587in}{1.549959in}}%
\pgfpathlineto{\pgfqpoint{1.490587in}{1.613090in}}%
\pgfpathclose%
\pgfusepath{fill}%
\end{pgfscope}%
\begin{pgfscope}%
\pgfpathrectangle{\pgfqpoint{0.804646in}{0.600000in}}{\pgfqpoint{2.573292in}{2.070576in}}%
\pgfusepath{clip}%
\pgfsetbuttcap%
\pgfsetmiterjoin%
\definecolor{currentfill}{rgb}{0.302379,0.450282,0.300122}%
\pgfsetfillcolor{currentfill}%
\pgfsetlinewidth{0.000000pt}%
\definecolor{currentstroke}{rgb}{0.000000,0.000000,0.000000}%
\pgfsetstrokecolor{currentstroke}%
\pgfsetstrokeopacity{0.000000}%
\pgfsetdash{}{0pt}%
\pgfpathmoveto{\pgfqpoint{1.501529in}{1.613090in}}%
\pgfpathlineto{\pgfqpoint{1.510282in}{1.613090in}}%
\pgfpathlineto{\pgfqpoint{1.510282in}{1.527904in}}%
\pgfpathlineto{\pgfqpoint{1.501529in}{1.527904in}}%
\pgfpathlineto{\pgfqpoint{1.501529in}{1.613090in}}%
\pgfpathclose%
\pgfusepath{fill}%
\end{pgfscope}%
\begin{pgfscope}%
\pgfpathrectangle{\pgfqpoint{0.804646in}{0.600000in}}{\pgfqpoint{2.573292in}{2.070576in}}%
\pgfusepath{clip}%
\pgfsetbuttcap%
\pgfsetmiterjoin%
\definecolor{currentfill}{rgb}{0.302379,0.450282,0.300122}%
\pgfsetfillcolor{currentfill}%
\pgfsetlinewidth{0.000000pt}%
\definecolor{currentstroke}{rgb}{0.000000,0.000000,0.000000}%
\pgfsetstrokecolor{currentstroke}%
\pgfsetstrokeopacity{0.000000}%
\pgfsetdash{}{0pt}%
\pgfpathmoveto{\pgfqpoint{1.512471in}{1.613090in}}%
\pgfpathlineto{\pgfqpoint{1.521224in}{1.613090in}}%
\pgfpathlineto{\pgfqpoint{1.521224in}{1.531418in}}%
\pgfpathlineto{\pgfqpoint{1.512471in}{1.531418in}}%
\pgfpathlineto{\pgfqpoint{1.512471in}{1.613090in}}%
\pgfpathclose%
\pgfusepath{fill}%
\end{pgfscope}%
\begin{pgfscope}%
\pgfpathrectangle{\pgfqpoint{0.804646in}{0.600000in}}{\pgfqpoint{2.573292in}{2.070576in}}%
\pgfusepath{clip}%
\pgfsetbuttcap%
\pgfsetmiterjoin%
\definecolor{currentfill}{rgb}{0.302379,0.450282,0.300122}%
\pgfsetfillcolor{currentfill}%
\pgfsetlinewidth{0.000000pt}%
\definecolor{currentstroke}{rgb}{0.000000,0.000000,0.000000}%
\pgfsetstrokecolor{currentstroke}%
\pgfsetstrokeopacity{0.000000}%
\pgfsetdash{}{0pt}%
\pgfpathmoveto{\pgfqpoint{1.523413in}{1.613090in}}%
\pgfpathlineto{\pgfqpoint{1.532166in}{1.613090in}}%
\pgfpathlineto{\pgfqpoint{1.532166in}{1.512675in}}%
\pgfpathlineto{\pgfqpoint{1.523413in}{1.512675in}}%
\pgfpathlineto{\pgfqpoint{1.523413in}{1.613090in}}%
\pgfpathclose%
\pgfusepath{fill}%
\end{pgfscope}%
\begin{pgfscope}%
\pgfpathrectangle{\pgfqpoint{0.804646in}{0.600000in}}{\pgfqpoint{2.573292in}{2.070576in}}%
\pgfusepath{clip}%
\pgfsetbuttcap%
\pgfsetmiterjoin%
\definecolor{currentfill}{rgb}{0.302379,0.450282,0.300122}%
\pgfsetfillcolor{currentfill}%
\pgfsetlinewidth{0.000000pt}%
\definecolor{currentstroke}{rgb}{0.000000,0.000000,0.000000}%
\pgfsetstrokecolor{currentstroke}%
\pgfsetstrokeopacity{0.000000}%
\pgfsetdash{}{0pt}%
\pgfpathmoveto{\pgfqpoint{1.534354in}{1.586673in}}%
\pgfpathlineto{\pgfqpoint{1.543108in}{1.586673in}}%
\pgfpathlineto{\pgfqpoint{1.543108in}{1.561304in}}%
\pgfpathlineto{\pgfqpoint{1.534354in}{1.561304in}}%
\pgfpathlineto{\pgfqpoint{1.534354in}{1.586673in}}%
\pgfpathclose%
\pgfusepath{fill}%
\end{pgfscope}%
\begin{pgfscope}%
\pgfpathrectangle{\pgfqpoint{0.804646in}{0.600000in}}{\pgfqpoint{2.573292in}{2.070576in}}%
\pgfusepath{clip}%
\pgfsetbuttcap%
\pgfsetmiterjoin%
\definecolor{currentfill}{rgb}{0.302379,0.450282,0.300122}%
\pgfsetfillcolor{currentfill}%
\pgfsetlinewidth{0.000000pt}%
\definecolor{currentstroke}{rgb}{0.000000,0.000000,0.000000}%
\pgfsetstrokecolor{currentstroke}%
\pgfsetstrokeopacity{0.000000}%
\pgfsetdash{}{0pt}%
\pgfpathmoveto{\pgfqpoint{1.545296in}{1.613090in}}%
\pgfpathlineto{\pgfqpoint{1.554050in}{1.613090in}}%
\pgfpathlineto{\pgfqpoint{1.554050in}{1.558022in}}%
\pgfpathlineto{\pgfqpoint{1.545296in}{1.558022in}}%
\pgfpathlineto{\pgfqpoint{1.545296in}{1.613090in}}%
\pgfpathclose%
\pgfusepath{fill}%
\end{pgfscope}%
\begin{pgfscope}%
\pgfpathrectangle{\pgfqpoint{0.804646in}{0.600000in}}{\pgfqpoint{2.573292in}{2.070576in}}%
\pgfusepath{clip}%
\pgfsetbuttcap%
\pgfsetmiterjoin%
\definecolor{currentfill}{rgb}{0.302379,0.450282,0.300122}%
\pgfsetfillcolor{currentfill}%
\pgfsetlinewidth{0.000000pt}%
\definecolor{currentstroke}{rgb}{0.000000,0.000000,0.000000}%
\pgfsetstrokecolor{currentstroke}%
\pgfsetstrokeopacity{0.000000}%
\pgfsetdash{}{0pt}%
\pgfpathmoveto{\pgfqpoint{1.556238in}{1.613090in}}%
\pgfpathlineto{\pgfqpoint{1.564991in}{1.613090in}}%
\pgfpathlineto{\pgfqpoint{1.564991in}{1.510643in}}%
\pgfpathlineto{\pgfqpoint{1.556238in}{1.510643in}}%
\pgfpathlineto{\pgfqpoint{1.556238in}{1.613090in}}%
\pgfpathclose%
\pgfusepath{fill}%
\end{pgfscope}%
\begin{pgfscope}%
\pgfpathrectangle{\pgfqpoint{0.804646in}{0.600000in}}{\pgfqpoint{2.573292in}{2.070576in}}%
\pgfusepath{clip}%
\pgfsetbuttcap%
\pgfsetmiterjoin%
\definecolor{currentfill}{rgb}{0.302379,0.450282,0.300122}%
\pgfsetfillcolor{currentfill}%
\pgfsetlinewidth{0.000000pt}%
\definecolor{currentstroke}{rgb}{0.000000,0.000000,0.000000}%
\pgfsetstrokecolor{currentstroke}%
\pgfsetstrokeopacity{0.000000}%
\pgfsetdash{}{0pt}%
\pgfpathmoveto{\pgfqpoint{1.567180in}{1.613090in}}%
\pgfpathlineto{\pgfqpoint{1.575933in}{1.613090in}}%
\pgfpathlineto{\pgfqpoint{1.575933in}{1.550733in}}%
\pgfpathlineto{\pgfqpoint{1.567180in}{1.550733in}}%
\pgfpathlineto{\pgfqpoint{1.567180in}{1.613090in}}%
\pgfpathclose%
\pgfusepath{fill}%
\end{pgfscope}%
\begin{pgfscope}%
\pgfpathrectangle{\pgfqpoint{0.804646in}{0.600000in}}{\pgfqpoint{2.573292in}{2.070576in}}%
\pgfusepath{clip}%
\pgfsetbuttcap%
\pgfsetmiterjoin%
\definecolor{currentfill}{rgb}{0.302379,0.450282,0.300122}%
\pgfsetfillcolor{currentfill}%
\pgfsetlinewidth{0.000000pt}%
\definecolor{currentstroke}{rgb}{0.000000,0.000000,0.000000}%
\pgfsetstrokecolor{currentstroke}%
\pgfsetstrokeopacity{0.000000}%
\pgfsetdash{}{0pt}%
\pgfpathmoveto{\pgfqpoint{1.578122in}{1.613090in}}%
\pgfpathlineto{\pgfqpoint{1.586875in}{1.613090in}}%
\pgfpathlineto{\pgfqpoint{1.586875in}{1.536840in}}%
\pgfpathlineto{\pgfqpoint{1.578122in}{1.536840in}}%
\pgfpathlineto{\pgfqpoint{1.578122in}{1.613090in}}%
\pgfpathclose%
\pgfusepath{fill}%
\end{pgfscope}%
\begin{pgfscope}%
\pgfpathrectangle{\pgfqpoint{0.804646in}{0.600000in}}{\pgfqpoint{2.573292in}{2.070576in}}%
\pgfusepath{clip}%
\pgfsetbuttcap%
\pgfsetmiterjoin%
\definecolor{currentfill}{rgb}{0.302379,0.450282,0.300122}%
\pgfsetfillcolor{currentfill}%
\pgfsetlinewidth{0.000000pt}%
\definecolor{currentstroke}{rgb}{0.000000,0.000000,0.000000}%
\pgfsetstrokecolor{currentstroke}%
\pgfsetstrokeopacity{0.000000}%
\pgfsetdash{}{0pt}%
\pgfpathmoveto{\pgfqpoint{1.589063in}{1.601884in}}%
\pgfpathlineto{\pgfqpoint{1.597817in}{1.601884in}}%
\pgfpathlineto{\pgfqpoint{1.597817in}{1.572573in}}%
\pgfpathlineto{\pgfqpoint{1.589063in}{1.572573in}}%
\pgfpathlineto{\pgfqpoint{1.589063in}{1.601884in}}%
\pgfpathclose%
\pgfusepath{fill}%
\end{pgfscope}%
\begin{pgfscope}%
\pgfpathrectangle{\pgfqpoint{0.804646in}{0.600000in}}{\pgfqpoint{2.573292in}{2.070576in}}%
\pgfusepath{clip}%
\pgfsetbuttcap%
\pgfsetmiterjoin%
\definecolor{currentfill}{rgb}{0.302379,0.450282,0.300122}%
\pgfsetfillcolor{currentfill}%
\pgfsetlinewidth{0.000000pt}%
\definecolor{currentstroke}{rgb}{0.000000,0.000000,0.000000}%
\pgfsetstrokecolor{currentstroke}%
\pgfsetstrokeopacity{0.000000}%
\pgfsetdash{}{0pt}%
\pgfpathmoveto{\pgfqpoint{1.600005in}{1.559848in}}%
\pgfpathlineto{\pgfqpoint{1.608759in}{1.559848in}}%
\pgfpathlineto{\pgfqpoint{1.608759in}{1.559619in}}%
\pgfpathlineto{\pgfqpoint{1.600005in}{1.559619in}}%
\pgfpathlineto{\pgfqpoint{1.600005in}{1.559848in}}%
\pgfpathclose%
\pgfusepath{fill}%
\end{pgfscope}%
\begin{pgfscope}%
\pgfpathrectangle{\pgfqpoint{0.804646in}{0.600000in}}{\pgfqpoint{2.573292in}{2.070576in}}%
\pgfusepath{clip}%
\pgfsetbuttcap%
\pgfsetmiterjoin%
\definecolor{currentfill}{rgb}{0.302379,0.450282,0.300122}%
\pgfsetfillcolor{currentfill}%
\pgfsetlinewidth{0.000000pt}%
\definecolor{currentstroke}{rgb}{0.000000,0.000000,0.000000}%
\pgfsetstrokecolor{currentstroke}%
\pgfsetstrokeopacity{0.000000}%
\pgfsetdash{}{0pt}%
\pgfpathmoveto{\pgfqpoint{1.610947in}{1.539082in}}%
\pgfpathlineto{\pgfqpoint{1.619700in}{1.539082in}}%
\pgfpathlineto{\pgfqpoint{1.619700in}{1.528874in}}%
\pgfpathlineto{\pgfqpoint{1.610947in}{1.528874in}}%
\pgfpathlineto{\pgfqpoint{1.610947in}{1.539082in}}%
\pgfpathclose%
\pgfusepath{fill}%
\end{pgfscope}%
\begin{pgfscope}%
\pgfpathrectangle{\pgfqpoint{0.804646in}{0.600000in}}{\pgfqpoint{2.573292in}{2.070576in}}%
\pgfusepath{clip}%
\pgfsetbuttcap%
\pgfsetmiterjoin%
\definecolor{currentfill}{rgb}{0.302379,0.450282,0.300122}%
\pgfsetfillcolor{currentfill}%
\pgfsetlinewidth{0.000000pt}%
\definecolor{currentstroke}{rgb}{0.000000,0.000000,0.000000}%
\pgfsetstrokecolor{currentstroke}%
\pgfsetstrokeopacity{0.000000}%
\pgfsetdash{}{0pt}%
\pgfpathmoveto{\pgfqpoint{1.621889in}{1.613090in}}%
\pgfpathlineto{\pgfqpoint{1.630642in}{1.613090in}}%
\pgfpathlineto{\pgfqpoint{1.630642in}{1.632477in}}%
\pgfpathlineto{\pgfqpoint{1.621889in}{1.632477in}}%
\pgfpathlineto{\pgfqpoint{1.621889in}{1.613090in}}%
\pgfpathclose%
\pgfusepath{fill}%
\end{pgfscope}%
\begin{pgfscope}%
\pgfpathrectangle{\pgfqpoint{0.804646in}{0.600000in}}{\pgfqpoint{2.573292in}{2.070576in}}%
\pgfusepath{clip}%
\pgfsetbuttcap%
\pgfsetmiterjoin%
\definecolor{currentfill}{rgb}{0.302379,0.450282,0.300122}%
\pgfsetfillcolor{currentfill}%
\pgfsetlinewidth{0.000000pt}%
\definecolor{currentstroke}{rgb}{0.000000,0.000000,0.000000}%
\pgfsetstrokecolor{currentstroke}%
\pgfsetstrokeopacity{0.000000}%
\pgfsetdash{}{0pt}%
\pgfpathmoveto{\pgfqpoint{1.632831in}{1.613090in}}%
\pgfpathlineto{\pgfqpoint{1.641584in}{1.613090in}}%
\pgfpathlineto{\pgfqpoint{1.641584in}{1.657941in}}%
\pgfpathlineto{\pgfqpoint{1.632831in}{1.657941in}}%
\pgfpathlineto{\pgfqpoint{1.632831in}{1.613090in}}%
\pgfpathclose%
\pgfusepath{fill}%
\end{pgfscope}%
\begin{pgfscope}%
\pgfpathrectangle{\pgfqpoint{0.804646in}{0.600000in}}{\pgfqpoint{2.573292in}{2.070576in}}%
\pgfusepath{clip}%
\pgfsetbuttcap%
\pgfsetmiterjoin%
\definecolor{currentfill}{rgb}{0.302379,0.450282,0.300122}%
\pgfsetfillcolor{currentfill}%
\pgfsetlinewidth{0.000000pt}%
\definecolor{currentstroke}{rgb}{0.000000,0.000000,0.000000}%
\pgfsetstrokecolor{currentstroke}%
\pgfsetstrokeopacity{0.000000}%
\pgfsetdash{}{0pt}%
\pgfpathmoveto{\pgfqpoint{1.643772in}{1.613090in}}%
\pgfpathlineto{\pgfqpoint{1.652526in}{1.613090in}}%
\pgfpathlineto{\pgfqpoint{1.652526in}{1.671873in}}%
\pgfpathlineto{\pgfqpoint{1.643772in}{1.671873in}}%
\pgfpathlineto{\pgfqpoint{1.643772in}{1.613090in}}%
\pgfpathclose%
\pgfusepath{fill}%
\end{pgfscope}%
\begin{pgfscope}%
\pgfpathrectangle{\pgfqpoint{0.804646in}{0.600000in}}{\pgfqpoint{2.573292in}{2.070576in}}%
\pgfusepath{clip}%
\pgfsetbuttcap%
\pgfsetmiterjoin%
\definecolor{currentfill}{rgb}{0.302379,0.450282,0.300122}%
\pgfsetfillcolor{currentfill}%
\pgfsetlinewidth{0.000000pt}%
\definecolor{currentstroke}{rgb}{0.000000,0.000000,0.000000}%
\pgfsetstrokecolor{currentstroke}%
\pgfsetstrokeopacity{0.000000}%
\pgfsetdash{}{0pt}%
\pgfpathmoveto{\pgfqpoint{1.654714in}{1.613090in}}%
\pgfpathlineto{\pgfqpoint{1.663468in}{1.613090in}}%
\pgfpathlineto{\pgfqpoint{1.663468in}{1.671449in}}%
\pgfpathlineto{\pgfqpoint{1.654714in}{1.671449in}}%
\pgfpathlineto{\pgfqpoint{1.654714in}{1.613090in}}%
\pgfpathclose%
\pgfusepath{fill}%
\end{pgfscope}%
\begin{pgfscope}%
\pgfpathrectangle{\pgfqpoint{0.804646in}{0.600000in}}{\pgfqpoint{2.573292in}{2.070576in}}%
\pgfusepath{clip}%
\pgfsetbuttcap%
\pgfsetmiterjoin%
\definecolor{currentfill}{rgb}{0.302379,0.450282,0.300122}%
\pgfsetfillcolor{currentfill}%
\pgfsetlinewidth{0.000000pt}%
\definecolor{currentstroke}{rgb}{0.000000,0.000000,0.000000}%
\pgfsetstrokecolor{currentstroke}%
\pgfsetstrokeopacity{0.000000}%
\pgfsetdash{}{0pt}%
\pgfpathmoveto{\pgfqpoint{1.665656in}{1.613090in}}%
\pgfpathlineto{\pgfqpoint{1.674409in}{1.613090in}}%
\pgfpathlineto{\pgfqpoint{1.674409in}{1.664329in}}%
\pgfpathlineto{\pgfqpoint{1.665656in}{1.664329in}}%
\pgfpathlineto{\pgfqpoint{1.665656in}{1.613090in}}%
\pgfpathclose%
\pgfusepath{fill}%
\end{pgfscope}%
\begin{pgfscope}%
\pgfpathrectangle{\pgfqpoint{0.804646in}{0.600000in}}{\pgfqpoint{2.573292in}{2.070576in}}%
\pgfusepath{clip}%
\pgfsetbuttcap%
\pgfsetmiterjoin%
\definecolor{currentfill}{rgb}{0.302379,0.450282,0.300122}%
\pgfsetfillcolor{currentfill}%
\pgfsetlinewidth{0.000000pt}%
\definecolor{currentstroke}{rgb}{0.000000,0.000000,0.000000}%
\pgfsetstrokecolor{currentstroke}%
\pgfsetstrokeopacity{0.000000}%
\pgfsetdash{}{0pt}%
\pgfpathmoveto{\pgfqpoint{1.676598in}{1.613090in}}%
\pgfpathlineto{\pgfqpoint{1.685351in}{1.613090in}}%
\pgfpathlineto{\pgfqpoint{1.685351in}{1.665681in}}%
\pgfpathlineto{\pgfqpoint{1.676598in}{1.665681in}}%
\pgfpathlineto{\pgfqpoint{1.676598in}{1.613090in}}%
\pgfpathclose%
\pgfusepath{fill}%
\end{pgfscope}%
\begin{pgfscope}%
\pgfpathrectangle{\pgfqpoint{0.804646in}{0.600000in}}{\pgfqpoint{2.573292in}{2.070576in}}%
\pgfusepath{clip}%
\pgfsetbuttcap%
\pgfsetmiterjoin%
\definecolor{currentfill}{rgb}{0.302379,0.450282,0.300122}%
\pgfsetfillcolor{currentfill}%
\pgfsetlinewidth{0.000000pt}%
\definecolor{currentstroke}{rgb}{0.000000,0.000000,0.000000}%
\pgfsetstrokecolor{currentstroke}%
\pgfsetstrokeopacity{0.000000}%
\pgfsetdash{}{0pt}%
\pgfpathmoveto{\pgfqpoint{1.687540in}{1.642787in}}%
\pgfpathlineto{\pgfqpoint{1.696293in}{1.642787in}}%
\pgfpathlineto{\pgfqpoint{1.696293in}{1.693557in}}%
\pgfpathlineto{\pgfqpoint{1.687540in}{1.693557in}}%
\pgfpathlineto{\pgfqpoint{1.687540in}{1.642787in}}%
\pgfpathclose%
\pgfusepath{fill}%
\end{pgfscope}%
\begin{pgfscope}%
\pgfpathrectangle{\pgfqpoint{0.804646in}{0.600000in}}{\pgfqpoint{2.573292in}{2.070576in}}%
\pgfusepath{clip}%
\pgfsetbuttcap%
\pgfsetmiterjoin%
\definecolor{currentfill}{rgb}{0.302379,0.450282,0.300122}%
\pgfsetfillcolor{currentfill}%
\pgfsetlinewidth{0.000000pt}%
\definecolor{currentstroke}{rgb}{0.000000,0.000000,0.000000}%
\pgfsetstrokecolor{currentstroke}%
\pgfsetstrokeopacity{0.000000}%
\pgfsetdash{}{0pt}%
\pgfpathmoveto{\pgfqpoint{1.698481in}{1.665844in}}%
\pgfpathlineto{\pgfqpoint{1.707235in}{1.665844in}}%
\pgfpathlineto{\pgfqpoint{1.707235in}{1.712399in}}%
\pgfpathlineto{\pgfqpoint{1.698481in}{1.712399in}}%
\pgfpathlineto{\pgfqpoint{1.698481in}{1.665844in}}%
\pgfpathclose%
\pgfusepath{fill}%
\end{pgfscope}%
\begin{pgfscope}%
\pgfpathrectangle{\pgfqpoint{0.804646in}{0.600000in}}{\pgfqpoint{2.573292in}{2.070576in}}%
\pgfusepath{clip}%
\pgfsetbuttcap%
\pgfsetmiterjoin%
\definecolor{currentfill}{rgb}{0.302379,0.450282,0.300122}%
\pgfsetfillcolor{currentfill}%
\pgfsetlinewidth{0.000000pt}%
\definecolor{currentstroke}{rgb}{0.000000,0.000000,0.000000}%
\pgfsetstrokecolor{currentstroke}%
\pgfsetstrokeopacity{0.000000}%
\pgfsetdash{}{0pt}%
\pgfpathmoveto{\pgfqpoint{1.709423in}{1.673057in}}%
\pgfpathlineto{\pgfqpoint{1.718177in}{1.673057in}}%
\pgfpathlineto{\pgfqpoint{1.718177in}{1.713715in}}%
\pgfpathlineto{\pgfqpoint{1.709423in}{1.713715in}}%
\pgfpathlineto{\pgfqpoint{1.709423in}{1.673057in}}%
\pgfpathclose%
\pgfusepath{fill}%
\end{pgfscope}%
\begin{pgfscope}%
\pgfpathrectangle{\pgfqpoint{0.804646in}{0.600000in}}{\pgfqpoint{2.573292in}{2.070576in}}%
\pgfusepath{clip}%
\pgfsetbuttcap%
\pgfsetmiterjoin%
\definecolor{currentfill}{rgb}{0.302379,0.450282,0.300122}%
\pgfsetfillcolor{currentfill}%
\pgfsetlinewidth{0.000000pt}%
\definecolor{currentstroke}{rgb}{0.000000,0.000000,0.000000}%
\pgfsetstrokecolor{currentstroke}%
\pgfsetstrokeopacity{0.000000}%
\pgfsetdash{}{0pt}%
\pgfpathmoveto{\pgfqpoint{1.720365in}{1.676797in}}%
\pgfpathlineto{\pgfqpoint{1.729118in}{1.676797in}}%
\pgfpathlineto{\pgfqpoint{1.729118in}{1.730460in}}%
\pgfpathlineto{\pgfqpoint{1.720365in}{1.730460in}}%
\pgfpathlineto{\pgfqpoint{1.720365in}{1.676797in}}%
\pgfpathclose%
\pgfusepath{fill}%
\end{pgfscope}%
\begin{pgfscope}%
\pgfpathrectangle{\pgfqpoint{0.804646in}{0.600000in}}{\pgfqpoint{2.573292in}{2.070576in}}%
\pgfusepath{clip}%
\pgfsetbuttcap%
\pgfsetmiterjoin%
\definecolor{currentfill}{rgb}{0.302379,0.450282,0.300122}%
\pgfsetfillcolor{currentfill}%
\pgfsetlinewidth{0.000000pt}%
\definecolor{currentstroke}{rgb}{0.000000,0.000000,0.000000}%
\pgfsetstrokecolor{currentstroke}%
\pgfsetstrokeopacity{0.000000}%
\pgfsetdash{}{0pt}%
\pgfpathmoveto{\pgfqpoint{1.731307in}{1.688340in}}%
\pgfpathlineto{\pgfqpoint{1.740060in}{1.688340in}}%
\pgfpathlineto{\pgfqpoint{1.740060in}{1.752099in}}%
\pgfpathlineto{\pgfqpoint{1.731307in}{1.752099in}}%
\pgfpathlineto{\pgfqpoint{1.731307in}{1.688340in}}%
\pgfpathclose%
\pgfusepath{fill}%
\end{pgfscope}%
\begin{pgfscope}%
\pgfpathrectangle{\pgfqpoint{0.804646in}{0.600000in}}{\pgfqpoint{2.573292in}{2.070576in}}%
\pgfusepath{clip}%
\pgfsetbuttcap%
\pgfsetmiterjoin%
\definecolor{currentfill}{rgb}{0.302379,0.450282,0.300122}%
\pgfsetfillcolor{currentfill}%
\pgfsetlinewidth{0.000000pt}%
\definecolor{currentstroke}{rgb}{0.000000,0.000000,0.000000}%
\pgfsetstrokecolor{currentstroke}%
\pgfsetstrokeopacity{0.000000}%
\pgfsetdash{}{0pt}%
\pgfpathmoveto{\pgfqpoint{1.742249in}{1.697497in}}%
\pgfpathlineto{\pgfqpoint{1.751002in}{1.697497in}}%
\pgfpathlineto{\pgfqpoint{1.751002in}{1.766114in}}%
\pgfpathlineto{\pgfqpoint{1.742249in}{1.766114in}}%
\pgfpathlineto{\pgfqpoint{1.742249in}{1.697497in}}%
\pgfpathclose%
\pgfusepath{fill}%
\end{pgfscope}%
\begin{pgfscope}%
\pgfpathrectangle{\pgfqpoint{0.804646in}{0.600000in}}{\pgfqpoint{2.573292in}{2.070576in}}%
\pgfusepath{clip}%
\pgfsetbuttcap%
\pgfsetmiterjoin%
\definecolor{currentfill}{rgb}{0.302379,0.450282,0.300122}%
\pgfsetfillcolor{currentfill}%
\pgfsetlinewidth{0.000000pt}%
\definecolor{currentstroke}{rgb}{0.000000,0.000000,0.000000}%
\pgfsetstrokecolor{currentstroke}%
\pgfsetstrokeopacity{0.000000}%
\pgfsetdash{}{0pt}%
\pgfpathmoveto{\pgfqpoint{1.753190in}{1.703588in}}%
\pgfpathlineto{\pgfqpoint{1.761944in}{1.703588in}}%
\pgfpathlineto{\pgfqpoint{1.761944in}{1.772835in}}%
\pgfpathlineto{\pgfqpoint{1.753190in}{1.772835in}}%
\pgfpathlineto{\pgfqpoint{1.753190in}{1.703588in}}%
\pgfpathclose%
\pgfusepath{fill}%
\end{pgfscope}%
\begin{pgfscope}%
\pgfpathrectangle{\pgfqpoint{0.804646in}{0.600000in}}{\pgfqpoint{2.573292in}{2.070576in}}%
\pgfusepath{clip}%
\pgfsetbuttcap%
\pgfsetmiterjoin%
\definecolor{currentfill}{rgb}{0.302379,0.450282,0.300122}%
\pgfsetfillcolor{currentfill}%
\pgfsetlinewidth{0.000000pt}%
\definecolor{currentstroke}{rgb}{0.000000,0.000000,0.000000}%
\pgfsetstrokecolor{currentstroke}%
\pgfsetstrokeopacity{0.000000}%
\pgfsetdash{}{0pt}%
\pgfpathmoveto{\pgfqpoint{1.764132in}{1.705964in}}%
\pgfpathlineto{\pgfqpoint{1.772886in}{1.705964in}}%
\pgfpathlineto{\pgfqpoint{1.772886in}{1.776808in}}%
\pgfpathlineto{\pgfqpoint{1.764132in}{1.776808in}}%
\pgfpathlineto{\pgfqpoint{1.764132in}{1.705964in}}%
\pgfpathclose%
\pgfusepath{fill}%
\end{pgfscope}%
\begin{pgfscope}%
\pgfpathrectangle{\pgfqpoint{0.804646in}{0.600000in}}{\pgfqpoint{2.573292in}{2.070576in}}%
\pgfusepath{clip}%
\pgfsetbuttcap%
\pgfsetmiterjoin%
\definecolor{currentfill}{rgb}{0.302379,0.450282,0.300122}%
\pgfsetfillcolor{currentfill}%
\pgfsetlinewidth{0.000000pt}%
\definecolor{currentstroke}{rgb}{0.000000,0.000000,0.000000}%
\pgfsetstrokecolor{currentstroke}%
\pgfsetstrokeopacity{0.000000}%
\pgfsetdash{}{0pt}%
\pgfpathmoveto{\pgfqpoint{1.775074in}{1.710696in}}%
\pgfpathlineto{\pgfqpoint{1.783827in}{1.710696in}}%
\pgfpathlineto{\pgfqpoint{1.783827in}{1.778232in}}%
\pgfpathlineto{\pgfqpoint{1.775074in}{1.778232in}}%
\pgfpathlineto{\pgfqpoint{1.775074in}{1.710696in}}%
\pgfpathclose%
\pgfusepath{fill}%
\end{pgfscope}%
\begin{pgfscope}%
\pgfpathrectangle{\pgfqpoint{0.804646in}{0.600000in}}{\pgfqpoint{2.573292in}{2.070576in}}%
\pgfusepath{clip}%
\pgfsetbuttcap%
\pgfsetmiterjoin%
\definecolor{currentfill}{rgb}{0.302379,0.450282,0.300122}%
\pgfsetfillcolor{currentfill}%
\pgfsetlinewidth{0.000000pt}%
\definecolor{currentstroke}{rgb}{0.000000,0.000000,0.000000}%
\pgfsetstrokecolor{currentstroke}%
\pgfsetstrokeopacity{0.000000}%
\pgfsetdash{}{0pt}%
\pgfpathmoveto{\pgfqpoint{1.786016in}{1.692934in}}%
\pgfpathlineto{\pgfqpoint{1.794769in}{1.692934in}}%
\pgfpathlineto{\pgfqpoint{1.794769in}{1.770641in}}%
\pgfpathlineto{\pgfqpoint{1.786016in}{1.770641in}}%
\pgfpathlineto{\pgfqpoint{1.786016in}{1.692934in}}%
\pgfpathclose%
\pgfusepath{fill}%
\end{pgfscope}%
\begin{pgfscope}%
\pgfpathrectangle{\pgfqpoint{0.804646in}{0.600000in}}{\pgfqpoint{2.573292in}{2.070576in}}%
\pgfusepath{clip}%
\pgfsetbuttcap%
\pgfsetmiterjoin%
\definecolor{currentfill}{rgb}{0.302379,0.450282,0.300122}%
\pgfsetfillcolor{currentfill}%
\pgfsetlinewidth{0.000000pt}%
\definecolor{currentstroke}{rgb}{0.000000,0.000000,0.000000}%
\pgfsetstrokecolor{currentstroke}%
\pgfsetstrokeopacity{0.000000}%
\pgfsetdash{}{0pt}%
\pgfpathmoveto{\pgfqpoint{1.796958in}{1.696648in}}%
\pgfpathlineto{\pgfqpoint{1.805711in}{1.696648in}}%
\pgfpathlineto{\pgfqpoint{1.805711in}{1.776183in}}%
\pgfpathlineto{\pgfqpoint{1.796958in}{1.776183in}}%
\pgfpathlineto{\pgfqpoint{1.796958in}{1.696648in}}%
\pgfpathclose%
\pgfusepath{fill}%
\end{pgfscope}%
\begin{pgfscope}%
\pgfpathrectangle{\pgfqpoint{0.804646in}{0.600000in}}{\pgfqpoint{2.573292in}{2.070576in}}%
\pgfusepath{clip}%
\pgfsetbuttcap%
\pgfsetmiterjoin%
\definecolor{currentfill}{rgb}{0.302379,0.450282,0.300122}%
\pgfsetfillcolor{currentfill}%
\pgfsetlinewidth{0.000000pt}%
\definecolor{currentstroke}{rgb}{0.000000,0.000000,0.000000}%
\pgfsetstrokecolor{currentstroke}%
\pgfsetstrokeopacity{0.000000}%
\pgfsetdash{}{0pt}%
\pgfpathmoveto{\pgfqpoint{1.807899in}{1.701329in}}%
\pgfpathlineto{\pgfqpoint{1.816653in}{1.701329in}}%
\pgfpathlineto{\pgfqpoint{1.816653in}{1.785204in}}%
\pgfpathlineto{\pgfqpoint{1.807899in}{1.785204in}}%
\pgfpathlineto{\pgfqpoint{1.807899in}{1.701329in}}%
\pgfpathclose%
\pgfusepath{fill}%
\end{pgfscope}%
\begin{pgfscope}%
\pgfpathrectangle{\pgfqpoint{0.804646in}{0.600000in}}{\pgfqpoint{2.573292in}{2.070576in}}%
\pgfusepath{clip}%
\pgfsetbuttcap%
\pgfsetmiterjoin%
\definecolor{currentfill}{rgb}{0.302379,0.450282,0.300122}%
\pgfsetfillcolor{currentfill}%
\pgfsetlinewidth{0.000000pt}%
\definecolor{currentstroke}{rgb}{0.000000,0.000000,0.000000}%
\pgfsetstrokecolor{currentstroke}%
\pgfsetstrokeopacity{0.000000}%
\pgfsetdash{}{0pt}%
\pgfpathmoveto{\pgfqpoint{1.818841in}{1.714730in}}%
\pgfpathlineto{\pgfqpoint{1.827595in}{1.714730in}}%
\pgfpathlineto{\pgfqpoint{1.827595in}{1.791260in}}%
\pgfpathlineto{\pgfqpoint{1.818841in}{1.791260in}}%
\pgfpathlineto{\pgfqpoint{1.818841in}{1.714730in}}%
\pgfpathclose%
\pgfusepath{fill}%
\end{pgfscope}%
\begin{pgfscope}%
\pgfpathrectangle{\pgfqpoint{0.804646in}{0.600000in}}{\pgfqpoint{2.573292in}{2.070576in}}%
\pgfusepath{clip}%
\pgfsetbuttcap%
\pgfsetmiterjoin%
\definecolor{currentfill}{rgb}{0.302379,0.450282,0.300122}%
\pgfsetfillcolor{currentfill}%
\pgfsetlinewidth{0.000000pt}%
\definecolor{currentstroke}{rgb}{0.000000,0.000000,0.000000}%
\pgfsetstrokecolor{currentstroke}%
\pgfsetstrokeopacity{0.000000}%
\pgfsetdash{}{0pt}%
\pgfpathmoveto{\pgfqpoint{1.829783in}{1.735219in}}%
\pgfpathlineto{\pgfqpoint{1.838536in}{1.735219in}}%
\pgfpathlineto{\pgfqpoint{1.838536in}{1.808390in}}%
\pgfpathlineto{\pgfqpoint{1.829783in}{1.808390in}}%
\pgfpathlineto{\pgfqpoint{1.829783in}{1.735219in}}%
\pgfpathclose%
\pgfusepath{fill}%
\end{pgfscope}%
\begin{pgfscope}%
\pgfpathrectangle{\pgfqpoint{0.804646in}{0.600000in}}{\pgfqpoint{2.573292in}{2.070576in}}%
\pgfusepath{clip}%
\pgfsetbuttcap%
\pgfsetmiterjoin%
\definecolor{currentfill}{rgb}{0.302379,0.450282,0.300122}%
\pgfsetfillcolor{currentfill}%
\pgfsetlinewidth{0.000000pt}%
\definecolor{currentstroke}{rgb}{0.000000,0.000000,0.000000}%
\pgfsetstrokecolor{currentstroke}%
\pgfsetstrokeopacity{0.000000}%
\pgfsetdash{}{0pt}%
\pgfpathmoveto{\pgfqpoint{1.840725in}{1.750020in}}%
\pgfpathlineto{\pgfqpoint{1.849478in}{1.750020in}}%
\pgfpathlineto{\pgfqpoint{1.849478in}{1.816280in}}%
\pgfpathlineto{\pgfqpoint{1.840725in}{1.816280in}}%
\pgfpathlineto{\pgfqpoint{1.840725in}{1.750020in}}%
\pgfpathclose%
\pgfusepath{fill}%
\end{pgfscope}%
\begin{pgfscope}%
\pgfpathrectangle{\pgfqpoint{0.804646in}{0.600000in}}{\pgfqpoint{2.573292in}{2.070576in}}%
\pgfusepath{clip}%
\pgfsetbuttcap%
\pgfsetmiterjoin%
\definecolor{currentfill}{rgb}{0.302379,0.450282,0.300122}%
\pgfsetfillcolor{currentfill}%
\pgfsetlinewidth{0.000000pt}%
\definecolor{currentstroke}{rgb}{0.000000,0.000000,0.000000}%
\pgfsetstrokecolor{currentstroke}%
\pgfsetstrokeopacity{0.000000}%
\pgfsetdash{}{0pt}%
\pgfpathmoveto{\pgfqpoint{1.851667in}{1.753163in}}%
\pgfpathlineto{\pgfqpoint{1.860420in}{1.753163in}}%
\pgfpathlineto{\pgfqpoint{1.860420in}{1.814448in}}%
\pgfpathlineto{\pgfqpoint{1.851667in}{1.814448in}}%
\pgfpathlineto{\pgfqpoint{1.851667in}{1.753163in}}%
\pgfpathclose%
\pgfusepath{fill}%
\end{pgfscope}%
\begin{pgfscope}%
\pgfpathrectangle{\pgfqpoint{0.804646in}{0.600000in}}{\pgfqpoint{2.573292in}{2.070576in}}%
\pgfusepath{clip}%
\pgfsetbuttcap%
\pgfsetmiterjoin%
\definecolor{currentfill}{rgb}{0.302379,0.450282,0.300122}%
\pgfsetfillcolor{currentfill}%
\pgfsetlinewidth{0.000000pt}%
\definecolor{currentstroke}{rgb}{0.000000,0.000000,0.000000}%
\pgfsetstrokecolor{currentstroke}%
\pgfsetstrokeopacity{0.000000}%
\pgfsetdash{}{0pt}%
\pgfpathmoveto{\pgfqpoint{1.862608in}{1.762478in}}%
\pgfpathlineto{\pgfqpoint{1.871362in}{1.762478in}}%
\pgfpathlineto{\pgfqpoint{1.871362in}{1.824109in}}%
\pgfpathlineto{\pgfqpoint{1.862608in}{1.824109in}}%
\pgfpathlineto{\pgfqpoint{1.862608in}{1.762478in}}%
\pgfpathclose%
\pgfusepath{fill}%
\end{pgfscope}%
\begin{pgfscope}%
\pgfpathrectangle{\pgfqpoint{0.804646in}{0.600000in}}{\pgfqpoint{2.573292in}{2.070576in}}%
\pgfusepath{clip}%
\pgfsetbuttcap%
\pgfsetmiterjoin%
\definecolor{currentfill}{rgb}{0.302379,0.450282,0.300122}%
\pgfsetfillcolor{currentfill}%
\pgfsetlinewidth{0.000000pt}%
\definecolor{currentstroke}{rgb}{0.000000,0.000000,0.000000}%
\pgfsetstrokecolor{currentstroke}%
\pgfsetstrokeopacity{0.000000}%
\pgfsetdash{}{0pt}%
\pgfpathmoveto{\pgfqpoint{1.873550in}{1.781010in}}%
\pgfpathlineto{\pgfqpoint{1.882304in}{1.781010in}}%
\pgfpathlineto{\pgfqpoint{1.882304in}{1.838828in}}%
\pgfpathlineto{\pgfqpoint{1.873550in}{1.838828in}}%
\pgfpathlineto{\pgfqpoint{1.873550in}{1.781010in}}%
\pgfpathclose%
\pgfusepath{fill}%
\end{pgfscope}%
\begin{pgfscope}%
\pgfpathrectangle{\pgfqpoint{0.804646in}{0.600000in}}{\pgfqpoint{2.573292in}{2.070576in}}%
\pgfusepath{clip}%
\pgfsetbuttcap%
\pgfsetmiterjoin%
\definecolor{currentfill}{rgb}{0.302379,0.450282,0.300122}%
\pgfsetfillcolor{currentfill}%
\pgfsetlinewidth{0.000000pt}%
\definecolor{currentstroke}{rgb}{0.000000,0.000000,0.000000}%
\pgfsetstrokecolor{currentstroke}%
\pgfsetstrokeopacity{0.000000}%
\pgfsetdash{}{0pt}%
\pgfpathmoveto{\pgfqpoint{1.884492in}{1.785874in}}%
\pgfpathlineto{\pgfqpoint{1.893245in}{1.785874in}}%
\pgfpathlineto{\pgfqpoint{1.893245in}{1.832169in}}%
\pgfpathlineto{\pgfqpoint{1.884492in}{1.832169in}}%
\pgfpathlineto{\pgfqpoint{1.884492in}{1.785874in}}%
\pgfpathclose%
\pgfusepath{fill}%
\end{pgfscope}%
\begin{pgfscope}%
\pgfpathrectangle{\pgfqpoint{0.804646in}{0.600000in}}{\pgfqpoint{2.573292in}{2.070576in}}%
\pgfusepath{clip}%
\pgfsetbuttcap%
\pgfsetmiterjoin%
\definecolor{currentfill}{rgb}{0.302379,0.450282,0.300122}%
\pgfsetfillcolor{currentfill}%
\pgfsetlinewidth{0.000000pt}%
\definecolor{currentstroke}{rgb}{0.000000,0.000000,0.000000}%
\pgfsetstrokecolor{currentstroke}%
\pgfsetstrokeopacity{0.000000}%
\pgfsetdash{}{0pt}%
\pgfpathmoveto{\pgfqpoint{1.895434in}{1.808212in}}%
\pgfpathlineto{\pgfqpoint{1.904187in}{1.808212in}}%
\pgfpathlineto{\pgfqpoint{1.904187in}{1.852849in}}%
\pgfpathlineto{\pgfqpoint{1.895434in}{1.852849in}}%
\pgfpathlineto{\pgfqpoint{1.895434in}{1.808212in}}%
\pgfpathclose%
\pgfusepath{fill}%
\end{pgfscope}%
\begin{pgfscope}%
\pgfpathrectangle{\pgfqpoint{0.804646in}{0.600000in}}{\pgfqpoint{2.573292in}{2.070576in}}%
\pgfusepath{clip}%
\pgfsetbuttcap%
\pgfsetmiterjoin%
\definecolor{currentfill}{rgb}{0.302379,0.450282,0.300122}%
\pgfsetfillcolor{currentfill}%
\pgfsetlinewidth{0.000000pt}%
\definecolor{currentstroke}{rgb}{0.000000,0.000000,0.000000}%
\pgfsetstrokecolor{currentstroke}%
\pgfsetstrokeopacity{0.000000}%
\pgfsetdash{}{0pt}%
\pgfpathmoveto{\pgfqpoint{1.906376in}{1.830853in}}%
\pgfpathlineto{\pgfqpoint{1.915129in}{1.830853in}}%
\pgfpathlineto{\pgfqpoint{1.915129in}{1.867323in}}%
\pgfpathlineto{\pgfqpoint{1.906376in}{1.867323in}}%
\pgfpathlineto{\pgfqpoint{1.906376in}{1.830853in}}%
\pgfpathclose%
\pgfusepath{fill}%
\end{pgfscope}%
\begin{pgfscope}%
\pgfpathrectangle{\pgfqpoint{0.804646in}{0.600000in}}{\pgfqpoint{2.573292in}{2.070576in}}%
\pgfusepath{clip}%
\pgfsetbuttcap%
\pgfsetmiterjoin%
\definecolor{currentfill}{rgb}{0.302379,0.450282,0.300122}%
\pgfsetfillcolor{currentfill}%
\pgfsetlinewidth{0.000000pt}%
\definecolor{currentstroke}{rgb}{0.000000,0.000000,0.000000}%
\pgfsetstrokecolor{currentstroke}%
\pgfsetstrokeopacity{0.000000}%
\pgfsetdash{}{0pt}%
\pgfpathmoveto{\pgfqpoint{1.917317in}{1.836046in}}%
\pgfpathlineto{\pgfqpoint{1.926071in}{1.836046in}}%
\pgfpathlineto{\pgfqpoint{1.926071in}{1.867898in}}%
\pgfpathlineto{\pgfqpoint{1.917317in}{1.867898in}}%
\pgfpathlineto{\pgfqpoint{1.917317in}{1.836046in}}%
\pgfpathclose%
\pgfusepath{fill}%
\end{pgfscope}%
\begin{pgfscope}%
\pgfpathrectangle{\pgfqpoint{0.804646in}{0.600000in}}{\pgfqpoint{2.573292in}{2.070576in}}%
\pgfusepath{clip}%
\pgfsetbuttcap%
\pgfsetmiterjoin%
\definecolor{currentfill}{rgb}{0.302379,0.450282,0.300122}%
\pgfsetfillcolor{currentfill}%
\pgfsetlinewidth{0.000000pt}%
\definecolor{currentstroke}{rgb}{0.000000,0.000000,0.000000}%
\pgfsetstrokecolor{currentstroke}%
\pgfsetstrokeopacity{0.000000}%
\pgfsetdash{}{0pt}%
\pgfpathmoveto{\pgfqpoint{1.928259in}{1.836643in}}%
\pgfpathlineto{\pgfqpoint{1.937013in}{1.836643in}}%
\pgfpathlineto{\pgfqpoint{1.937013in}{1.877282in}}%
\pgfpathlineto{\pgfqpoint{1.928259in}{1.877282in}}%
\pgfpathlineto{\pgfqpoint{1.928259in}{1.836643in}}%
\pgfpathclose%
\pgfusepath{fill}%
\end{pgfscope}%
\begin{pgfscope}%
\pgfpathrectangle{\pgfqpoint{0.804646in}{0.600000in}}{\pgfqpoint{2.573292in}{2.070576in}}%
\pgfusepath{clip}%
\pgfsetbuttcap%
\pgfsetmiterjoin%
\definecolor{currentfill}{rgb}{0.302379,0.450282,0.300122}%
\pgfsetfillcolor{currentfill}%
\pgfsetlinewidth{0.000000pt}%
\definecolor{currentstroke}{rgb}{0.000000,0.000000,0.000000}%
\pgfsetstrokecolor{currentstroke}%
\pgfsetstrokeopacity{0.000000}%
\pgfsetdash{}{0pt}%
\pgfpathmoveto{\pgfqpoint{1.939201in}{1.826575in}}%
\pgfpathlineto{\pgfqpoint{1.947954in}{1.826575in}}%
\pgfpathlineto{\pgfqpoint{1.947954in}{1.875321in}}%
\pgfpathlineto{\pgfqpoint{1.939201in}{1.875321in}}%
\pgfpathlineto{\pgfqpoint{1.939201in}{1.826575in}}%
\pgfpathclose%
\pgfusepath{fill}%
\end{pgfscope}%
\begin{pgfscope}%
\pgfpathrectangle{\pgfqpoint{0.804646in}{0.600000in}}{\pgfqpoint{2.573292in}{2.070576in}}%
\pgfusepath{clip}%
\pgfsetbuttcap%
\pgfsetmiterjoin%
\definecolor{currentfill}{rgb}{0.302379,0.450282,0.300122}%
\pgfsetfillcolor{currentfill}%
\pgfsetlinewidth{0.000000pt}%
\definecolor{currentstroke}{rgb}{0.000000,0.000000,0.000000}%
\pgfsetstrokecolor{currentstroke}%
\pgfsetstrokeopacity{0.000000}%
\pgfsetdash{}{0pt}%
\pgfpathmoveto{\pgfqpoint{1.950143in}{1.823406in}}%
\pgfpathlineto{\pgfqpoint{1.958896in}{1.823406in}}%
\pgfpathlineto{\pgfqpoint{1.958896in}{1.874804in}}%
\pgfpathlineto{\pgfqpoint{1.950143in}{1.874804in}}%
\pgfpathlineto{\pgfqpoint{1.950143in}{1.823406in}}%
\pgfpathclose%
\pgfusepath{fill}%
\end{pgfscope}%
\begin{pgfscope}%
\pgfpathrectangle{\pgfqpoint{0.804646in}{0.600000in}}{\pgfqpoint{2.573292in}{2.070576in}}%
\pgfusepath{clip}%
\pgfsetbuttcap%
\pgfsetmiterjoin%
\definecolor{currentfill}{rgb}{0.302379,0.450282,0.300122}%
\pgfsetfillcolor{currentfill}%
\pgfsetlinewidth{0.000000pt}%
\definecolor{currentstroke}{rgb}{0.000000,0.000000,0.000000}%
\pgfsetstrokecolor{currentstroke}%
\pgfsetstrokeopacity{0.000000}%
\pgfsetdash{}{0pt}%
\pgfpathmoveto{\pgfqpoint{1.961085in}{1.803284in}}%
\pgfpathlineto{\pgfqpoint{1.969838in}{1.803284in}}%
\pgfpathlineto{\pgfqpoint{1.969838in}{1.853024in}}%
\pgfpathlineto{\pgfqpoint{1.961085in}{1.853024in}}%
\pgfpathlineto{\pgfqpoint{1.961085in}{1.803284in}}%
\pgfpathclose%
\pgfusepath{fill}%
\end{pgfscope}%
\begin{pgfscope}%
\pgfpathrectangle{\pgfqpoint{0.804646in}{0.600000in}}{\pgfqpoint{2.573292in}{2.070576in}}%
\pgfusepath{clip}%
\pgfsetbuttcap%
\pgfsetmiterjoin%
\definecolor{currentfill}{rgb}{0.302379,0.450282,0.300122}%
\pgfsetfillcolor{currentfill}%
\pgfsetlinewidth{0.000000pt}%
\definecolor{currentstroke}{rgb}{0.000000,0.000000,0.000000}%
\pgfsetstrokecolor{currentstroke}%
\pgfsetstrokeopacity{0.000000}%
\pgfsetdash{}{0pt}%
\pgfpathmoveto{\pgfqpoint{1.972026in}{1.788585in}}%
\pgfpathlineto{\pgfqpoint{1.980780in}{1.788585in}}%
\pgfpathlineto{\pgfqpoint{1.980780in}{1.840926in}}%
\pgfpathlineto{\pgfqpoint{1.972026in}{1.840926in}}%
\pgfpathlineto{\pgfqpoint{1.972026in}{1.788585in}}%
\pgfpathclose%
\pgfusepath{fill}%
\end{pgfscope}%
\begin{pgfscope}%
\pgfpathrectangle{\pgfqpoint{0.804646in}{0.600000in}}{\pgfqpoint{2.573292in}{2.070576in}}%
\pgfusepath{clip}%
\pgfsetbuttcap%
\pgfsetmiterjoin%
\definecolor{currentfill}{rgb}{0.302379,0.450282,0.300122}%
\pgfsetfillcolor{currentfill}%
\pgfsetlinewidth{0.000000pt}%
\definecolor{currentstroke}{rgb}{0.000000,0.000000,0.000000}%
\pgfsetstrokecolor{currentstroke}%
\pgfsetstrokeopacity{0.000000}%
\pgfsetdash{}{0pt}%
\pgfpathmoveto{\pgfqpoint{1.982968in}{1.777965in}}%
\pgfpathlineto{\pgfqpoint{1.991722in}{1.777965in}}%
\pgfpathlineto{\pgfqpoint{1.991722in}{1.833597in}}%
\pgfpathlineto{\pgfqpoint{1.982968in}{1.833597in}}%
\pgfpathlineto{\pgfqpoint{1.982968in}{1.777965in}}%
\pgfpathclose%
\pgfusepath{fill}%
\end{pgfscope}%
\begin{pgfscope}%
\pgfpathrectangle{\pgfqpoint{0.804646in}{0.600000in}}{\pgfqpoint{2.573292in}{2.070576in}}%
\pgfusepath{clip}%
\pgfsetbuttcap%
\pgfsetmiterjoin%
\definecolor{currentfill}{rgb}{0.302379,0.450282,0.300122}%
\pgfsetfillcolor{currentfill}%
\pgfsetlinewidth{0.000000pt}%
\definecolor{currentstroke}{rgb}{0.000000,0.000000,0.000000}%
\pgfsetstrokecolor{currentstroke}%
\pgfsetstrokeopacity{0.000000}%
\pgfsetdash{}{0pt}%
\pgfpathmoveto{\pgfqpoint{1.993910in}{1.728675in}}%
\pgfpathlineto{\pgfqpoint{2.002663in}{1.728675in}}%
\pgfpathlineto{\pgfqpoint{2.002663in}{1.791296in}}%
\pgfpathlineto{\pgfqpoint{1.993910in}{1.791296in}}%
\pgfpathlineto{\pgfqpoint{1.993910in}{1.728675in}}%
\pgfpathclose%
\pgfusepath{fill}%
\end{pgfscope}%
\begin{pgfscope}%
\pgfpathrectangle{\pgfqpoint{0.804646in}{0.600000in}}{\pgfqpoint{2.573292in}{2.070576in}}%
\pgfusepath{clip}%
\pgfsetbuttcap%
\pgfsetmiterjoin%
\definecolor{currentfill}{rgb}{0.302379,0.450282,0.300122}%
\pgfsetfillcolor{currentfill}%
\pgfsetlinewidth{0.000000pt}%
\definecolor{currentstroke}{rgb}{0.000000,0.000000,0.000000}%
\pgfsetstrokecolor{currentstroke}%
\pgfsetstrokeopacity{0.000000}%
\pgfsetdash{}{0pt}%
\pgfpathmoveto{\pgfqpoint{2.004852in}{1.705319in}}%
\pgfpathlineto{\pgfqpoint{2.013605in}{1.705319in}}%
\pgfpathlineto{\pgfqpoint{2.013605in}{1.774628in}}%
\pgfpathlineto{\pgfqpoint{2.004852in}{1.774628in}}%
\pgfpathlineto{\pgfqpoint{2.004852in}{1.705319in}}%
\pgfpathclose%
\pgfusepath{fill}%
\end{pgfscope}%
\begin{pgfscope}%
\pgfpathrectangle{\pgfqpoint{0.804646in}{0.600000in}}{\pgfqpoint{2.573292in}{2.070576in}}%
\pgfusepath{clip}%
\pgfsetbuttcap%
\pgfsetmiterjoin%
\definecolor{currentfill}{rgb}{0.302379,0.450282,0.300122}%
\pgfsetfillcolor{currentfill}%
\pgfsetlinewidth{0.000000pt}%
\definecolor{currentstroke}{rgb}{0.000000,0.000000,0.000000}%
\pgfsetstrokecolor{currentstroke}%
\pgfsetstrokeopacity{0.000000}%
\pgfsetdash{}{0pt}%
\pgfpathmoveto{\pgfqpoint{2.015794in}{1.693241in}}%
\pgfpathlineto{\pgfqpoint{2.024547in}{1.693241in}}%
\pgfpathlineto{\pgfqpoint{2.024547in}{1.760903in}}%
\pgfpathlineto{\pgfqpoint{2.015794in}{1.760903in}}%
\pgfpathlineto{\pgfqpoint{2.015794in}{1.693241in}}%
\pgfpathclose%
\pgfusepath{fill}%
\end{pgfscope}%
\begin{pgfscope}%
\pgfpathrectangle{\pgfqpoint{0.804646in}{0.600000in}}{\pgfqpoint{2.573292in}{2.070576in}}%
\pgfusepath{clip}%
\pgfsetbuttcap%
\pgfsetmiterjoin%
\definecolor{currentfill}{rgb}{0.302379,0.450282,0.300122}%
\pgfsetfillcolor{currentfill}%
\pgfsetlinewidth{0.000000pt}%
\definecolor{currentstroke}{rgb}{0.000000,0.000000,0.000000}%
\pgfsetstrokecolor{currentstroke}%
\pgfsetstrokeopacity{0.000000}%
\pgfsetdash{}{0pt}%
\pgfpathmoveto{\pgfqpoint{2.026735in}{1.676186in}}%
\pgfpathlineto{\pgfqpoint{2.035489in}{1.676186in}}%
\pgfpathlineto{\pgfqpoint{2.035489in}{1.746362in}}%
\pgfpathlineto{\pgfqpoint{2.026735in}{1.746362in}}%
\pgfpathlineto{\pgfqpoint{2.026735in}{1.676186in}}%
\pgfpathclose%
\pgfusepath{fill}%
\end{pgfscope}%
\begin{pgfscope}%
\pgfpathrectangle{\pgfqpoint{0.804646in}{0.600000in}}{\pgfqpoint{2.573292in}{2.070576in}}%
\pgfusepath{clip}%
\pgfsetbuttcap%
\pgfsetmiterjoin%
\definecolor{currentfill}{rgb}{0.302379,0.450282,0.300122}%
\pgfsetfillcolor{currentfill}%
\pgfsetlinewidth{0.000000pt}%
\definecolor{currentstroke}{rgb}{0.000000,0.000000,0.000000}%
\pgfsetstrokecolor{currentstroke}%
\pgfsetstrokeopacity{0.000000}%
\pgfsetdash{}{0pt}%
\pgfpathmoveto{\pgfqpoint{2.037677in}{1.662747in}}%
\pgfpathlineto{\pgfqpoint{2.046431in}{1.662747in}}%
\pgfpathlineto{\pgfqpoint{2.046431in}{1.738497in}}%
\pgfpathlineto{\pgfqpoint{2.037677in}{1.738497in}}%
\pgfpathlineto{\pgfqpoint{2.037677in}{1.662747in}}%
\pgfpathclose%
\pgfusepath{fill}%
\end{pgfscope}%
\begin{pgfscope}%
\pgfpathrectangle{\pgfqpoint{0.804646in}{0.600000in}}{\pgfqpoint{2.573292in}{2.070576in}}%
\pgfusepath{clip}%
\pgfsetbuttcap%
\pgfsetmiterjoin%
\definecolor{currentfill}{rgb}{0.302379,0.450282,0.300122}%
\pgfsetfillcolor{currentfill}%
\pgfsetlinewidth{0.000000pt}%
\definecolor{currentstroke}{rgb}{0.000000,0.000000,0.000000}%
\pgfsetstrokecolor{currentstroke}%
\pgfsetstrokeopacity{0.000000}%
\pgfsetdash{}{0pt}%
\pgfpathmoveto{\pgfqpoint{2.048619in}{1.652593in}}%
\pgfpathlineto{\pgfqpoint{2.057372in}{1.652593in}}%
\pgfpathlineto{\pgfqpoint{2.057372in}{1.726079in}}%
\pgfpathlineto{\pgfqpoint{2.048619in}{1.726079in}}%
\pgfpathlineto{\pgfqpoint{2.048619in}{1.652593in}}%
\pgfpathclose%
\pgfusepath{fill}%
\end{pgfscope}%
\begin{pgfscope}%
\pgfpathrectangle{\pgfqpoint{0.804646in}{0.600000in}}{\pgfqpoint{2.573292in}{2.070576in}}%
\pgfusepath{clip}%
\pgfsetbuttcap%
\pgfsetmiterjoin%
\definecolor{currentfill}{rgb}{0.302379,0.450282,0.300122}%
\pgfsetfillcolor{currentfill}%
\pgfsetlinewidth{0.000000pt}%
\definecolor{currentstroke}{rgb}{0.000000,0.000000,0.000000}%
\pgfsetstrokecolor{currentstroke}%
\pgfsetstrokeopacity{0.000000}%
\pgfsetdash{}{0pt}%
\pgfpathmoveto{\pgfqpoint{2.059561in}{1.650287in}}%
\pgfpathlineto{\pgfqpoint{2.068314in}{1.650287in}}%
\pgfpathlineto{\pgfqpoint{2.068314in}{1.722396in}}%
\pgfpathlineto{\pgfqpoint{2.059561in}{1.722396in}}%
\pgfpathlineto{\pgfqpoint{2.059561in}{1.650287in}}%
\pgfpathclose%
\pgfusepath{fill}%
\end{pgfscope}%
\begin{pgfscope}%
\pgfpathrectangle{\pgfqpoint{0.804646in}{0.600000in}}{\pgfqpoint{2.573292in}{2.070576in}}%
\pgfusepath{clip}%
\pgfsetbuttcap%
\pgfsetmiterjoin%
\definecolor{currentfill}{rgb}{0.302379,0.450282,0.300122}%
\pgfsetfillcolor{currentfill}%
\pgfsetlinewidth{0.000000pt}%
\definecolor{currentstroke}{rgb}{0.000000,0.000000,0.000000}%
\pgfsetstrokecolor{currentstroke}%
\pgfsetstrokeopacity{0.000000}%
\pgfsetdash{}{0pt}%
\pgfpathmoveto{\pgfqpoint{2.070503in}{1.651274in}}%
\pgfpathlineto{\pgfqpoint{2.079256in}{1.651274in}}%
\pgfpathlineto{\pgfqpoint{2.079256in}{1.718339in}}%
\pgfpathlineto{\pgfqpoint{2.070503in}{1.718339in}}%
\pgfpathlineto{\pgfqpoint{2.070503in}{1.651274in}}%
\pgfpathclose%
\pgfusepath{fill}%
\end{pgfscope}%
\begin{pgfscope}%
\pgfpathrectangle{\pgfqpoint{0.804646in}{0.600000in}}{\pgfqpoint{2.573292in}{2.070576in}}%
\pgfusepath{clip}%
\pgfsetbuttcap%
\pgfsetmiterjoin%
\definecolor{currentfill}{rgb}{0.302379,0.450282,0.300122}%
\pgfsetfillcolor{currentfill}%
\pgfsetlinewidth{0.000000pt}%
\definecolor{currentstroke}{rgb}{0.000000,0.000000,0.000000}%
\pgfsetstrokecolor{currentstroke}%
\pgfsetstrokeopacity{0.000000}%
\pgfsetdash{}{0pt}%
\pgfpathmoveto{\pgfqpoint{2.081444in}{1.665110in}}%
\pgfpathlineto{\pgfqpoint{2.090198in}{1.665110in}}%
\pgfpathlineto{\pgfqpoint{2.090198in}{1.729465in}}%
\pgfpathlineto{\pgfqpoint{2.081444in}{1.729465in}}%
\pgfpathlineto{\pgfqpoint{2.081444in}{1.665110in}}%
\pgfpathclose%
\pgfusepath{fill}%
\end{pgfscope}%
\begin{pgfscope}%
\pgfpathrectangle{\pgfqpoint{0.804646in}{0.600000in}}{\pgfqpoint{2.573292in}{2.070576in}}%
\pgfusepath{clip}%
\pgfsetbuttcap%
\pgfsetmiterjoin%
\definecolor{currentfill}{rgb}{0.302379,0.450282,0.300122}%
\pgfsetfillcolor{currentfill}%
\pgfsetlinewidth{0.000000pt}%
\definecolor{currentstroke}{rgb}{0.000000,0.000000,0.000000}%
\pgfsetstrokecolor{currentstroke}%
\pgfsetstrokeopacity{0.000000}%
\pgfsetdash{}{0pt}%
\pgfpathmoveto{\pgfqpoint{2.092386in}{1.680773in}}%
\pgfpathlineto{\pgfqpoint{2.101140in}{1.680773in}}%
\pgfpathlineto{\pgfqpoint{2.101140in}{1.741537in}}%
\pgfpathlineto{\pgfqpoint{2.092386in}{1.741537in}}%
\pgfpathlineto{\pgfqpoint{2.092386in}{1.680773in}}%
\pgfpathclose%
\pgfusepath{fill}%
\end{pgfscope}%
\begin{pgfscope}%
\pgfpathrectangle{\pgfqpoint{0.804646in}{0.600000in}}{\pgfqpoint{2.573292in}{2.070576in}}%
\pgfusepath{clip}%
\pgfsetbuttcap%
\pgfsetmiterjoin%
\definecolor{currentfill}{rgb}{0.302379,0.450282,0.300122}%
\pgfsetfillcolor{currentfill}%
\pgfsetlinewidth{0.000000pt}%
\definecolor{currentstroke}{rgb}{0.000000,0.000000,0.000000}%
\pgfsetstrokecolor{currentstroke}%
\pgfsetstrokeopacity{0.000000}%
\pgfsetdash{}{0pt}%
\pgfpathmoveto{\pgfqpoint{2.103328in}{1.690668in}}%
\pgfpathlineto{\pgfqpoint{2.112081in}{1.690668in}}%
\pgfpathlineto{\pgfqpoint{2.112081in}{1.744953in}}%
\pgfpathlineto{\pgfqpoint{2.103328in}{1.744953in}}%
\pgfpathlineto{\pgfqpoint{2.103328in}{1.690668in}}%
\pgfpathclose%
\pgfusepath{fill}%
\end{pgfscope}%
\begin{pgfscope}%
\pgfpathrectangle{\pgfqpoint{0.804646in}{0.600000in}}{\pgfqpoint{2.573292in}{2.070576in}}%
\pgfusepath{clip}%
\pgfsetbuttcap%
\pgfsetmiterjoin%
\definecolor{currentfill}{rgb}{0.302379,0.450282,0.300122}%
\pgfsetfillcolor{currentfill}%
\pgfsetlinewidth{0.000000pt}%
\definecolor{currentstroke}{rgb}{0.000000,0.000000,0.000000}%
\pgfsetstrokecolor{currentstroke}%
\pgfsetstrokeopacity{0.000000}%
\pgfsetdash{}{0pt}%
\pgfpathmoveto{\pgfqpoint{2.114270in}{1.697110in}}%
\pgfpathlineto{\pgfqpoint{2.123023in}{1.697110in}}%
\pgfpathlineto{\pgfqpoint{2.123023in}{1.749155in}}%
\pgfpathlineto{\pgfqpoint{2.114270in}{1.749155in}}%
\pgfpathlineto{\pgfqpoint{2.114270in}{1.697110in}}%
\pgfpathclose%
\pgfusepath{fill}%
\end{pgfscope}%
\begin{pgfscope}%
\pgfpathrectangle{\pgfqpoint{0.804646in}{0.600000in}}{\pgfqpoint{2.573292in}{2.070576in}}%
\pgfusepath{clip}%
\pgfsetbuttcap%
\pgfsetmiterjoin%
\definecolor{currentfill}{rgb}{0.302379,0.450282,0.300122}%
\pgfsetfillcolor{currentfill}%
\pgfsetlinewidth{0.000000pt}%
\definecolor{currentstroke}{rgb}{0.000000,0.000000,0.000000}%
\pgfsetstrokecolor{currentstroke}%
\pgfsetstrokeopacity{0.000000}%
\pgfsetdash{}{0pt}%
\pgfpathmoveto{\pgfqpoint{2.125212in}{1.704455in}}%
\pgfpathlineto{\pgfqpoint{2.133965in}{1.704455in}}%
\pgfpathlineto{\pgfqpoint{2.133965in}{1.752347in}}%
\pgfpathlineto{\pgfqpoint{2.125212in}{1.752347in}}%
\pgfpathlineto{\pgfqpoint{2.125212in}{1.704455in}}%
\pgfpathclose%
\pgfusepath{fill}%
\end{pgfscope}%
\begin{pgfscope}%
\pgfpathrectangle{\pgfqpoint{0.804646in}{0.600000in}}{\pgfqpoint{2.573292in}{2.070576in}}%
\pgfusepath{clip}%
\pgfsetbuttcap%
\pgfsetmiterjoin%
\definecolor{currentfill}{rgb}{0.302379,0.450282,0.300122}%
\pgfsetfillcolor{currentfill}%
\pgfsetlinewidth{0.000000pt}%
\definecolor{currentstroke}{rgb}{0.000000,0.000000,0.000000}%
\pgfsetstrokecolor{currentstroke}%
\pgfsetstrokeopacity{0.000000}%
\pgfsetdash{}{0pt}%
\pgfpathmoveto{\pgfqpoint{2.136153in}{1.737539in}}%
\pgfpathlineto{\pgfqpoint{2.144907in}{1.737539in}}%
\pgfpathlineto{\pgfqpoint{2.144907in}{1.778142in}}%
\pgfpathlineto{\pgfqpoint{2.136153in}{1.778142in}}%
\pgfpathlineto{\pgfqpoint{2.136153in}{1.737539in}}%
\pgfpathclose%
\pgfusepath{fill}%
\end{pgfscope}%
\begin{pgfscope}%
\pgfpathrectangle{\pgfqpoint{0.804646in}{0.600000in}}{\pgfqpoint{2.573292in}{2.070576in}}%
\pgfusepath{clip}%
\pgfsetbuttcap%
\pgfsetmiterjoin%
\definecolor{currentfill}{rgb}{0.302379,0.450282,0.300122}%
\pgfsetfillcolor{currentfill}%
\pgfsetlinewidth{0.000000pt}%
\definecolor{currentstroke}{rgb}{0.000000,0.000000,0.000000}%
\pgfsetstrokecolor{currentstroke}%
\pgfsetstrokeopacity{0.000000}%
\pgfsetdash{}{0pt}%
\pgfpathmoveto{\pgfqpoint{2.147095in}{1.755343in}}%
\pgfpathlineto{\pgfqpoint{2.155849in}{1.755343in}}%
\pgfpathlineto{\pgfqpoint{2.155849in}{1.788126in}}%
\pgfpathlineto{\pgfqpoint{2.147095in}{1.788126in}}%
\pgfpathlineto{\pgfqpoint{2.147095in}{1.755343in}}%
\pgfpathclose%
\pgfusepath{fill}%
\end{pgfscope}%
\begin{pgfscope}%
\pgfpathrectangle{\pgfqpoint{0.804646in}{0.600000in}}{\pgfqpoint{2.573292in}{2.070576in}}%
\pgfusepath{clip}%
\pgfsetbuttcap%
\pgfsetmiterjoin%
\definecolor{currentfill}{rgb}{0.302379,0.450282,0.300122}%
\pgfsetfillcolor{currentfill}%
\pgfsetlinewidth{0.000000pt}%
\definecolor{currentstroke}{rgb}{0.000000,0.000000,0.000000}%
\pgfsetstrokecolor{currentstroke}%
\pgfsetstrokeopacity{0.000000}%
\pgfsetdash{}{0pt}%
\pgfpathmoveto{\pgfqpoint{2.158037in}{1.761235in}}%
\pgfpathlineto{\pgfqpoint{2.166790in}{1.761235in}}%
\pgfpathlineto{\pgfqpoint{2.166790in}{1.788972in}}%
\pgfpathlineto{\pgfqpoint{2.158037in}{1.788972in}}%
\pgfpathlineto{\pgfqpoint{2.158037in}{1.761235in}}%
\pgfpathclose%
\pgfusepath{fill}%
\end{pgfscope}%
\begin{pgfscope}%
\pgfpathrectangle{\pgfqpoint{0.804646in}{0.600000in}}{\pgfqpoint{2.573292in}{2.070576in}}%
\pgfusepath{clip}%
\pgfsetbuttcap%
\pgfsetmiterjoin%
\definecolor{currentfill}{rgb}{0.302379,0.450282,0.300122}%
\pgfsetfillcolor{currentfill}%
\pgfsetlinewidth{0.000000pt}%
\definecolor{currentstroke}{rgb}{0.000000,0.000000,0.000000}%
\pgfsetstrokecolor{currentstroke}%
\pgfsetstrokeopacity{0.000000}%
\pgfsetdash{}{0pt}%
\pgfpathmoveto{\pgfqpoint{2.168979in}{1.769385in}}%
\pgfpathlineto{\pgfqpoint{2.177732in}{1.769385in}}%
\pgfpathlineto{\pgfqpoint{2.177732in}{1.790120in}}%
\pgfpathlineto{\pgfqpoint{2.168979in}{1.790120in}}%
\pgfpathlineto{\pgfqpoint{2.168979in}{1.769385in}}%
\pgfpathclose%
\pgfusepath{fill}%
\end{pgfscope}%
\begin{pgfscope}%
\pgfpathrectangle{\pgfqpoint{0.804646in}{0.600000in}}{\pgfqpoint{2.573292in}{2.070576in}}%
\pgfusepath{clip}%
\pgfsetbuttcap%
\pgfsetmiterjoin%
\definecolor{currentfill}{rgb}{0.302379,0.450282,0.300122}%
\pgfsetfillcolor{currentfill}%
\pgfsetlinewidth{0.000000pt}%
\definecolor{currentstroke}{rgb}{0.000000,0.000000,0.000000}%
\pgfsetstrokecolor{currentstroke}%
\pgfsetstrokeopacity{0.000000}%
\pgfsetdash{}{0pt}%
\pgfpathmoveto{\pgfqpoint{2.179921in}{1.759715in}}%
\pgfpathlineto{\pgfqpoint{2.188674in}{1.759715in}}%
\pgfpathlineto{\pgfqpoint{2.188674in}{1.778668in}}%
\pgfpathlineto{\pgfqpoint{2.179921in}{1.778668in}}%
\pgfpathlineto{\pgfqpoint{2.179921in}{1.759715in}}%
\pgfpathclose%
\pgfusepath{fill}%
\end{pgfscope}%
\begin{pgfscope}%
\pgfpathrectangle{\pgfqpoint{0.804646in}{0.600000in}}{\pgfqpoint{2.573292in}{2.070576in}}%
\pgfusepath{clip}%
\pgfsetbuttcap%
\pgfsetmiterjoin%
\definecolor{currentfill}{rgb}{0.302379,0.450282,0.300122}%
\pgfsetfillcolor{currentfill}%
\pgfsetlinewidth{0.000000pt}%
\definecolor{currentstroke}{rgb}{0.000000,0.000000,0.000000}%
\pgfsetstrokecolor{currentstroke}%
\pgfsetstrokeopacity{0.000000}%
\pgfsetdash{}{0pt}%
\pgfpathmoveto{\pgfqpoint{2.190862in}{1.775309in}}%
\pgfpathlineto{\pgfqpoint{2.199616in}{1.775309in}}%
\pgfpathlineto{\pgfqpoint{2.199616in}{1.798811in}}%
\pgfpathlineto{\pgfqpoint{2.190862in}{1.798811in}}%
\pgfpathlineto{\pgfqpoint{2.190862in}{1.775309in}}%
\pgfpathclose%
\pgfusepath{fill}%
\end{pgfscope}%
\begin{pgfscope}%
\pgfpathrectangle{\pgfqpoint{0.804646in}{0.600000in}}{\pgfqpoint{2.573292in}{2.070576in}}%
\pgfusepath{clip}%
\pgfsetbuttcap%
\pgfsetmiterjoin%
\definecolor{currentfill}{rgb}{0.302379,0.450282,0.300122}%
\pgfsetfillcolor{currentfill}%
\pgfsetlinewidth{0.000000pt}%
\definecolor{currentstroke}{rgb}{0.000000,0.000000,0.000000}%
\pgfsetstrokecolor{currentstroke}%
\pgfsetstrokeopacity{0.000000}%
\pgfsetdash{}{0pt}%
\pgfpathmoveto{\pgfqpoint{2.201804in}{1.773096in}}%
\pgfpathlineto{\pgfqpoint{2.210558in}{1.773096in}}%
\pgfpathlineto{\pgfqpoint{2.210558in}{1.798683in}}%
\pgfpathlineto{\pgfqpoint{2.201804in}{1.798683in}}%
\pgfpathlineto{\pgfqpoint{2.201804in}{1.773096in}}%
\pgfpathclose%
\pgfusepath{fill}%
\end{pgfscope}%
\begin{pgfscope}%
\pgfpathrectangle{\pgfqpoint{0.804646in}{0.600000in}}{\pgfqpoint{2.573292in}{2.070576in}}%
\pgfusepath{clip}%
\pgfsetbuttcap%
\pgfsetmiterjoin%
\definecolor{currentfill}{rgb}{0.302379,0.450282,0.300122}%
\pgfsetfillcolor{currentfill}%
\pgfsetlinewidth{0.000000pt}%
\definecolor{currentstroke}{rgb}{0.000000,0.000000,0.000000}%
\pgfsetstrokecolor{currentstroke}%
\pgfsetstrokeopacity{0.000000}%
\pgfsetdash{}{0pt}%
\pgfpathmoveto{\pgfqpoint{2.212746in}{1.766378in}}%
\pgfpathlineto{\pgfqpoint{2.221499in}{1.766378in}}%
\pgfpathlineto{\pgfqpoint{2.221499in}{1.797384in}}%
\pgfpathlineto{\pgfqpoint{2.212746in}{1.797384in}}%
\pgfpathlineto{\pgfqpoint{2.212746in}{1.766378in}}%
\pgfpathclose%
\pgfusepath{fill}%
\end{pgfscope}%
\begin{pgfscope}%
\pgfpathrectangle{\pgfqpoint{0.804646in}{0.600000in}}{\pgfqpoint{2.573292in}{2.070576in}}%
\pgfusepath{clip}%
\pgfsetbuttcap%
\pgfsetmiterjoin%
\definecolor{currentfill}{rgb}{0.302379,0.450282,0.300122}%
\pgfsetfillcolor{currentfill}%
\pgfsetlinewidth{0.000000pt}%
\definecolor{currentstroke}{rgb}{0.000000,0.000000,0.000000}%
\pgfsetstrokecolor{currentstroke}%
\pgfsetstrokeopacity{0.000000}%
\pgfsetdash{}{0pt}%
\pgfpathmoveto{\pgfqpoint{2.223688in}{1.786895in}}%
\pgfpathlineto{\pgfqpoint{2.232441in}{1.786895in}}%
\pgfpathlineto{\pgfqpoint{2.232441in}{1.820136in}}%
\pgfpathlineto{\pgfqpoint{2.223688in}{1.820136in}}%
\pgfpathlineto{\pgfqpoint{2.223688in}{1.786895in}}%
\pgfpathclose%
\pgfusepath{fill}%
\end{pgfscope}%
\begin{pgfscope}%
\pgfpathrectangle{\pgfqpoint{0.804646in}{0.600000in}}{\pgfqpoint{2.573292in}{2.070576in}}%
\pgfusepath{clip}%
\pgfsetbuttcap%
\pgfsetmiterjoin%
\definecolor{currentfill}{rgb}{0.302379,0.450282,0.300122}%
\pgfsetfillcolor{currentfill}%
\pgfsetlinewidth{0.000000pt}%
\definecolor{currentstroke}{rgb}{0.000000,0.000000,0.000000}%
\pgfsetstrokecolor{currentstroke}%
\pgfsetstrokeopacity{0.000000}%
\pgfsetdash{}{0pt}%
\pgfpathmoveto{\pgfqpoint{2.234630in}{1.794286in}}%
\pgfpathlineto{\pgfqpoint{2.243383in}{1.794286in}}%
\pgfpathlineto{\pgfqpoint{2.243383in}{1.829753in}}%
\pgfpathlineto{\pgfqpoint{2.234630in}{1.829753in}}%
\pgfpathlineto{\pgfqpoint{2.234630in}{1.794286in}}%
\pgfpathclose%
\pgfusepath{fill}%
\end{pgfscope}%
\begin{pgfscope}%
\pgfpathrectangle{\pgfqpoint{0.804646in}{0.600000in}}{\pgfqpoint{2.573292in}{2.070576in}}%
\pgfusepath{clip}%
\pgfsetbuttcap%
\pgfsetmiterjoin%
\definecolor{currentfill}{rgb}{0.302379,0.450282,0.300122}%
\pgfsetfillcolor{currentfill}%
\pgfsetlinewidth{0.000000pt}%
\definecolor{currentstroke}{rgb}{0.000000,0.000000,0.000000}%
\pgfsetstrokecolor{currentstroke}%
\pgfsetstrokeopacity{0.000000}%
\pgfsetdash{}{0pt}%
\pgfpathmoveto{\pgfqpoint{2.245571in}{1.806358in}}%
\pgfpathlineto{\pgfqpoint{2.254325in}{1.806358in}}%
\pgfpathlineto{\pgfqpoint{2.254325in}{1.843586in}}%
\pgfpathlineto{\pgfqpoint{2.245571in}{1.843586in}}%
\pgfpathlineto{\pgfqpoint{2.245571in}{1.806358in}}%
\pgfpathclose%
\pgfusepath{fill}%
\end{pgfscope}%
\begin{pgfscope}%
\pgfpathrectangle{\pgfqpoint{0.804646in}{0.600000in}}{\pgfqpoint{2.573292in}{2.070576in}}%
\pgfusepath{clip}%
\pgfsetbuttcap%
\pgfsetmiterjoin%
\definecolor{currentfill}{rgb}{0.302379,0.450282,0.300122}%
\pgfsetfillcolor{currentfill}%
\pgfsetlinewidth{0.000000pt}%
\definecolor{currentstroke}{rgb}{0.000000,0.000000,0.000000}%
\pgfsetstrokecolor{currentstroke}%
\pgfsetstrokeopacity{0.000000}%
\pgfsetdash{}{0pt}%
\pgfpathmoveto{\pgfqpoint{2.256513in}{1.815807in}}%
\pgfpathlineto{\pgfqpoint{2.265267in}{1.815807in}}%
\pgfpathlineto{\pgfqpoint{2.265267in}{1.854722in}}%
\pgfpathlineto{\pgfqpoint{2.256513in}{1.854722in}}%
\pgfpathlineto{\pgfqpoint{2.256513in}{1.815807in}}%
\pgfpathclose%
\pgfusepath{fill}%
\end{pgfscope}%
\begin{pgfscope}%
\pgfpathrectangle{\pgfqpoint{0.804646in}{0.600000in}}{\pgfqpoint{2.573292in}{2.070576in}}%
\pgfusepath{clip}%
\pgfsetbuttcap%
\pgfsetmiterjoin%
\definecolor{currentfill}{rgb}{0.302379,0.450282,0.300122}%
\pgfsetfillcolor{currentfill}%
\pgfsetlinewidth{0.000000pt}%
\definecolor{currentstroke}{rgb}{0.000000,0.000000,0.000000}%
\pgfsetstrokecolor{currentstroke}%
\pgfsetstrokeopacity{0.000000}%
\pgfsetdash{}{0pt}%
\pgfpathmoveto{\pgfqpoint{2.267455in}{1.827492in}}%
\pgfpathlineto{\pgfqpoint{2.276208in}{1.827492in}}%
\pgfpathlineto{\pgfqpoint{2.276208in}{1.866268in}}%
\pgfpathlineto{\pgfqpoint{2.267455in}{1.866268in}}%
\pgfpathlineto{\pgfqpoint{2.267455in}{1.827492in}}%
\pgfpathclose%
\pgfusepath{fill}%
\end{pgfscope}%
\begin{pgfscope}%
\pgfpathrectangle{\pgfqpoint{0.804646in}{0.600000in}}{\pgfqpoint{2.573292in}{2.070576in}}%
\pgfusepath{clip}%
\pgfsetbuttcap%
\pgfsetmiterjoin%
\definecolor{currentfill}{rgb}{0.302379,0.450282,0.300122}%
\pgfsetfillcolor{currentfill}%
\pgfsetlinewidth{0.000000pt}%
\definecolor{currentstroke}{rgb}{0.000000,0.000000,0.000000}%
\pgfsetstrokecolor{currentstroke}%
\pgfsetstrokeopacity{0.000000}%
\pgfsetdash{}{0pt}%
\pgfpathmoveto{\pgfqpoint{2.278397in}{1.828114in}}%
\pgfpathlineto{\pgfqpoint{2.287150in}{1.828114in}}%
\pgfpathlineto{\pgfqpoint{2.287150in}{1.868234in}}%
\pgfpathlineto{\pgfqpoint{2.278397in}{1.868234in}}%
\pgfpathlineto{\pgfqpoint{2.278397in}{1.828114in}}%
\pgfpathclose%
\pgfusepath{fill}%
\end{pgfscope}%
\begin{pgfscope}%
\pgfpathrectangle{\pgfqpoint{0.804646in}{0.600000in}}{\pgfqpoint{2.573292in}{2.070576in}}%
\pgfusepath{clip}%
\pgfsetbuttcap%
\pgfsetmiterjoin%
\definecolor{currentfill}{rgb}{0.302379,0.450282,0.300122}%
\pgfsetfillcolor{currentfill}%
\pgfsetlinewidth{0.000000pt}%
\definecolor{currentstroke}{rgb}{0.000000,0.000000,0.000000}%
\pgfsetstrokecolor{currentstroke}%
\pgfsetstrokeopacity{0.000000}%
\pgfsetdash{}{0pt}%
\pgfpathmoveto{\pgfqpoint{2.289339in}{1.830409in}}%
\pgfpathlineto{\pgfqpoint{2.298092in}{1.830409in}}%
\pgfpathlineto{\pgfqpoint{2.298092in}{1.876210in}}%
\pgfpathlineto{\pgfqpoint{2.289339in}{1.876210in}}%
\pgfpathlineto{\pgfqpoint{2.289339in}{1.830409in}}%
\pgfpathclose%
\pgfusepath{fill}%
\end{pgfscope}%
\begin{pgfscope}%
\pgfpathrectangle{\pgfqpoint{0.804646in}{0.600000in}}{\pgfqpoint{2.573292in}{2.070576in}}%
\pgfusepath{clip}%
\pgfsetbuttcap%
\pgfsetmiterjoin%
\definecolor{currentfill}{rgb}{0.302379,0.450282,0.300122}%
\pgfsetfillcolor{currentfill}%
\pgfsetlinewidth{0.000000pt}%
\definecolor{currentstroke}{rgb}{0.000000,0.000000,0.000000}%
\pgfsetstrokecolor{currentstroke}%
\pgfsetstrokeopacity{0.000000}%
\pgfsetdash{}{0pt}%
\pgfpathmoveto{\pgfqpoint{2.300280in}{1.842360in}}%
\pgfpathlineto{\pgfqpoint{2.309034in}{1.842360in}}%
\pgfpathlineto{\pgfqpoint{2.309034in}{1.892155in}}%
\pgfpathlineto{\pgfqpoint{2.300280in}{1.892155in}}%
\pgfpathlineto{\pgfqpoint{2.300280in}{1.842360in}}%
\pgfpathclose%
\pgfusepath{fill}%
\end{pgfscope}%
\begin{pgfscope}%
\pgfpathrectangle{\pgfqpoint{0.804646in}{0.600000in}}{\pgfqpoint{2.573292in}{2.070576in}}%
\pgfusepath{clip}%
\pgfsetbuttcap%
\pgfsetmiterjoin%
\definecolor{currentfill}{rgb}{0.302379,0.450282,0.300122}%
\pgfsetfillcolor{currentfill}%
\pgfsetlinewidth{0.000000pt}%
\definecolor{currentstroke}{rgb}{0.000000,0.000000,0.000000}%
\pgfsetstrokecolor{currentstroke}%
\pgfsetstrokeopacity{0.000000}%
\pgfsetdash{}{0pt}%
\pgfpathmoveto{\pgfqpoint{2.311222in}{1.850858in}}%
\pgfpathlineto{\pgfqpoint{2.319976in}{1.850858in}}%
\pgfpathlineto{\pgfqpoint{2.319976in}{1.904025in}}%
\pgfpathlineto{\pgfqpoint{2.311222in}{1.904025in}}%
\pgfpathlineto{\pgfqpoint{2.311222in}{1.850858in}}%
\pgfpathclose%
\pgfusepath{fill}%
\end{pgfscope}%
\begin{pgfscope}%
\pgfpathrectangle{\pgfqpoint{0.804646in}{0.600000in}}{\pgfqpoint{2.573292in}{2.070576in}}%
\pgfusepath{clip}%
\pgfsetbuttcap%
\pgfsetmiterjoin%
\definecolor{currentfill}{rgb}{0.302379,0.450282,0.300122}%
\pgfsetfillcolor{currentfill}%
\pgfsetlinewidth{0.000000pt}%
\definecolor{currentstroke}{rgb}{0.000000,0.000000,0.000000}%
\pgfsetstrokecolor{currentstroke}%
\pgfsetstrokeopacity{0.000000}%
\pgfsetdash{}{0pt}%
\pgfpathmoveto{\pgfqpoint{2.322164in}{1.852527in}}%
\pgfpathlineto{\pgfqpoint{2.330917in}{1.852527in}}%
\pgfpathlineto{\pgfqpoint{2.330917in}{1.902213in}}%
\pgfpathlineto{\pgfqpoint{2.322164in}{1.902213in}}%
\pgfpathlineto{\pgfqpoint{2.322164in}{1.852527in}}%
\pgfpathclose%
\pgfusepath{fill}%
\end{pgfscope}%
\begin{pgfscope}%
\pgfpathrectangle{\pgfqpoint{0.804646in}{0.600000in}}{\pgfqpoint{2.573292in}{2.070576in}}%
\pgfusepath{clip}%
\pgfsetbuttcap%
\pgfsetmiterjoin%
\definecolor{currentfill}{rgb}{0.302379,0.450282,0.300122}%
\pgfsetfillcolor{currentfill}%
\pgfsetlinewidth{0.000000pt}%
\definecolor{currentstroke}{rgb}{0.000000,0.000000,0.000000}%
\pgfsetstrokecolor{currentstroke}%
\pgfsetstrokeopacity{0.000000}%
\pgfsetdash{}{0pt}%
\pgfpathmoveto{\pgfqpoint{2.333106in}{1.864829in}}%
\pgfpathlineto{\pgfqpoint{2.341859in}{1.864829in}}%
\pgfpathlineto{\pgfqpoint{2.341859in}{1.922844in}}%
\pgfpathlineto{\pgfqpoint{2.333106in}{1.922844in}}%
\pgfpathlineto{\pgfqpoint{2.333106in}{1.864829in}}%
\pgfpathclose%
\pgfusepath{fill}%
\end{pgfscope}%
\begin{pgfscope}%
\pgfpathrectangle{\pgfqpoint{0.804646in}{0.600000in}}{\pgfqpoint{2.573292in}{2.070576in}}%
\pgfusepath{clip}%
\pgfsetbuttcap%
\pgfsetmiterjoin%
\definecolor{currentfill}{rgb}{0.302379,0.450282,0.300122}%
\pgfsetfillcolor{currentfill}%
\pgfsetlinewidth{0.000000pt}%
\definecolor{currentstroke}{rgb}{0.000000,0.000000,0.000000}%
\pgfsetstrokecolor{currentstroke}%
\pgfsetstrokeopacity{0.000000}%
\pgfsetdash{}{0pt}%
\pgfpathmoveto{\pgfqpoint{2.344048in}{1.855013in}}%
\pgfpathlineto{\pgfqpoint{2.352801in}{1.855013in}}%
\pgfpathlineto{\pgfqpoint{2.352801in}{1.916701in}}%
\pgfpathlineto{\pgfqpoint{2.344048in}{1.916701in}}%
\pgfpathlineto{\pgfqpoint{2.344048in}{1.855013in}}%
\pgfpathclose%
\pgfusepath{fill}%
\end{pgfscope}%
\begin{pgfscope}%
\pgfpathrectangle{\pgfqpoint{0.804646in}{0.600000in}}{\pgfqpoint{2.573292in}{2.070576in}}%
\pgfusepath{clip}%
\pgfsetbuttcap%
\pgfsetmiterjoin%
\definecolor{currentfill}{rgb}{0.302379,0.450282,0.300122}%
\pgfsetfillcolor{currentfill}%
\pgfsetlinewidth{0.000000pt}%
\definecolor{currentstroke}{rgb}{0.000000,0.000000,0.000000}%
\pgfsetstrokecolor{currentstroke}%
\pgfsetstrokeopacity{0.000000}%
\pgfsetdash{}{0pt}%
\pgfpathmoveto{\pgfqpoint{2.354989in}{1.869566in}}%
\pgfpathlineto{\pgfqpoint{2.363743in}{1.869566in}}%
\pgfpathlineto{\pgfqpoint{2.363743in}{1.929066in}}%
\pgfpathlineto{\pgfqpoint{2.354989in}{1.929066in}}%
\pgfpathlineto{\pgfqpoint{2.354989in}{1.869566in}}%
\pgfpathclose%
\pgfusepath{fill}%
\end{pgfscope}%
\begin{pgfscope}%
\pgfpathrectangle{\pgfqpoint{0.804646in}{0.600000in}}{\pgfqpoint{2.573292in}{2.070576in}}%
\pgfusepath{clip}%
\pgfsetbuttcap%
\pgfsetmiterjoin%
\definecolor{currentfill}{rgb}{0.302379,0.450282,0.300122}%
\pgfsetfillcolor{currentfill}%
\pgfsetlinewidth{0.000000pt}%
\definecolor{currentstroke}{rgb}{0.000000,0.000000,0.000000}%
\pgfsetstrokecolor{currentstroke}%
\pgfsetstrokeopacity{0.000000}%
\pgfsetdash{}{0pt}%
\pgfpathmoveto{\pgfqpoint{2.365931in}{1.874698in}}%
\pgfpathlineto{\pgfqpoint{2.374685in}{1.874698in}}%
\pgfpathlineto{\pgfqpoint{2.374685in}{1.933041in}}%
\pgfpathlineto{\pgfqpoint{2.365931in}{1.933041in}}%
\pgfpathlineto{\pgfqpoint{2.365931in}{1.874698in}}%
\pgfpathclose%
\pgfusepath{fill}%
\end{pgfscope}%
\begin{pgfscope}%
\pgfpathrectangle{\pgfqpoint{0.804646in}{0.600000in}}{\pgfqpoint{2.573292in}{2.070576in}}%
\pgfusepath{clip}%
\pgfsetbuttcap%
\pgfsetmiterjoin%
\definecolor{currentfill}{rgb}{0.302379,0.450282,0.300122}%
\pgfsetfillcolor{currentfill}%
\pgfsetlinewidth{0.000000pt}%
\definecolor{currentstroke}{rgb}{0.000000,0.000000,0.000000}%
\pgfsetstrokecolor{currentstroke}%
\pgfsetstrokeopacity{0.000000}%
\pgfsetdash{}{0pt}%
\pgfpathmoveto{\pgfqpoint{2.376873in}{1.874494in}}%
\pgfpathlineto{\pgfqpoint{2.385626in}{1.874494in}}%
\pgfpathlineto{\pgfqpoint{2.385626in}{1.929476in}}%
\pgfpathlineto{\pgfqpoint{2.376873in}{1.929476in}}%
\pgfpathlineto{\pgfqpoint{2.376873in}{1.874494in}}%
\pgfpathclose%
\pgfusepath{fill}%
\end{pgfscope}%
\begin{pgfscope}%
\pgfpathrectangle{\pgfqpoint{0.804646in}{0.600000in}}{\pgfqpoint{2.573292in}{2.070576in}}%
\pgfusepath{clip}%
\pgfsetbuttcap%
\pgfsetmiterjoin%
\definecolor{currentfill}{rgb}{0.302379,0.450282,0.300122}%
\pgfsetfillcolor{currentfill}%
\pgfsetlinewidth{0.000000pt}%
\definecolor{currentstroke}{rgb}{0.000000,0.000000,0.000000}%
\pgfsetstrokecolor{currentstroke}%
\pgfsetstrokeopacity{0.000000}%
\pgfsetdash{}{0pt}%
\pgfpathmoveto{\pgfqpoint{2.387815in}{1.887766in}}%
\pgfpathlineto{\pgfqpoint{2.396568in}{1.887766in}}%
\pgfpathlineto{\pgfqpoint{2.396568in}{1.937362in}}%
\pgfpathlineto{\pgfqpoint{2.387815in}{1.937362in}}%
\pgfpathlineto{\pgfqpoint{2.387815in}{1.887766in}}%
\pgfpathclose%
\pgfusepath{fill}%
\end{pgfscope}%
\begin{pgfscope}%
\pgfpathrectangle{\pgfqpoint{0.804646in}{0.600000in}}{\pgfqpoint{2.573292in}{2.070576in}}%
\pgfusepath{clip}%
\pgfsetbuttcap%
\pgfsetmiterjoin%
\definecolor{currentfill}{rgb}{0.302379,0.450282,0.300122}%
\pgfsetfillcolor{currentfill}%
\pgfsetlinewidth{0.000000pt}%
\definecolor{currentstroke}{rgb}{0.000000,0.000000,0.000000}%
\pgfsetstrokecolor{currentstroke}%
\pgfsetstrokeopacity{0.000000}%
\pgfsetdash{}{0pt}%
\pgfpathmoveto{\pgfqpoint{2.398757in}{1.890729in}}%
\pgfpathlineto{\pgfqpoint{2.407510in}{1.890729in}}%
\pgfpathlineto{\pgfqpoint{2.407510in}{1.933728in}}%
\pgfpathlineto{\pgfqpoint{2.398757in}{1.933728in}}%
\pgfpathlineto{\pgfqpoint{2.398757in}{1.890729in}}%
\pgfpathclose%
\pgfusepath{fill}%
\end{pgfscope}%
\begin{pgfscope}%
\pgfpathrectangle{\pgfqpoint{0.804646in}{0.600000in}}{\pgfqpoint{2.573292in}{2.070576in}}%
\pgfusepath{clip}%
\pgfsetbuttcap%
\pgfsetmiterjoin%
\definecolor{currentfill}{rgb}{0.302379,0.450282,0.300122}%
\pgfsetfillcolor{currentfill}%
\pgfsetlinewidth{0.000000pt}%
\definecolor{currentstroke}{rgb}{0.000000,0.000000,0.000000}%
\pgfsetstrokecolor{currentstroke}%
\pgfsetstrokeopacity{0.000000}%
\pgfsetdash{}{0pt}%
\pgfpathmoveto{\pgfqpoint{2.409698in}{1.889496in}}%
\pgfpathlineto{\pgfqpoint{2.418452in}{1.889496in}}%
\pgfpathlineto{\pgfqpoint{2.418452in}{1.930337in}}%
\pgfpathlineto{\pgfqpoint{2.409698in}{1.930337in}}%
\pgfpathlineto{\pgfqpoint{2.409698in}{1.889496in}}%
\pgfpathclose%
\pgfusepath{fill}%
\end{pgfscope}%
\begin{pgfscope}%
\pgfpathrectangle{\pgfqpoint{0.804646in}{0.600000in}}{\pgfqpoint{2.573292in}{2.070576in}}%
\pgfusepath{clip}%
\pgfsetbuttcap%
\pgfsetmiterjoin%
\definecolor{currentfill}{rgb}{0.302379,0.450282,0.300122}%
\pgfsetfillcolor{currentfill}%
\pgfsetlinewidth{0.000000pt}%
\definecolor{currentstroke}{rgb}{0.000000,0.000000,0.000000}%
\pgfsetstrokecolor{currentstroke}%
\pgfsetstrokeopacity{0.000000}%
\pgfsetdash{}{0pt}%
\pgfpathmoveto{\pgfqpoint{2.420640in}{1.878100in}}%
\pgfpathlineto{\pgfqpoint{2.429394in}{1.878100in}}%
\pgfpathlineto{\pgfqpoint{2.429394in}{1.921126in}}%
\pgfpathlineto{\pgfqpoint{2.420640in}{1.921126in}}%
\pgfpathlineto{\pgfqpoint{2.420640in}{1.878100in}}%
\pgfpathclose%
\pgfusepath{fill}%
\end{pgfscope}%
\begin{pgfscope}%
\pgfpathrectangle{\pgfqpoint{0.804646in}{0.600000in}}{\pgfqpoint{2.573292in}{2.070576in}}%
\pgfusepath{clip}%
\pgfsetbuttcap%
\pgfsetmiterjoin%
\definecolor{currentfill}{rgb}{0.302379,0.450282,0.300122}%
\pgfsetfillcolor{currentfill}%
\pgfsetlinewidth{0.000000pt}%
\definecolor{currentstroke}{rgb}{0.000000,0.000000,0.000000}%
\pgfsetstrokecolor{currentstroke}%
\pgfsetstrokeopacity{0.000000}%
\pgfsetdash{}{0pt}%
\pgfpathmoveto{\pgfqpoint{2.431582in}{1.870504in}}%
\pgfpathlineto{\pgfqpoint{2.440335in}{1.870504in}}%
\pgfpathlineto{\pgfqpoint{2.440335in}{1.919255in}}%
\pgfpathlineto{\pgfqpoint{2.431582in}{1.919255in}}%
\pgfpathlineto{\pgfqpoint{2.431582in}{1.870504in}}%
\pgfpathclose%
\pgfusepath{fill}%
\end{pgfscope}%
\begin{pgfscope}%
\pgfpathrectangle{\pgfqpoint{0.804646in}{0.600000in}}{\pgfqpoint{2.573292in}{2.070576in}}%
\pgfusepath{clip}%
\pgfsetbuttcap%
\pgfsetmiterjoin%
\definecolor{currentfill}{rgb}{0.302379,0.450282,0.300122}%
\pgfsetfillcolor{currentfill}%
\pgfsetlinewidth{0.000000pt}%
\definecolor{currentstroke}{rgb}{0.000000,0.000000,0.000000}%
\pgfsetstrokecolor{currentstroke}%
\pgfsetstrokeopacity{0.000000}%
\pgfsetdash{}{0pt}%
\pgfpathmoveto{\pgfqpoint{2.442524in}{1.829026in}}%
\pgfpathlineto{\pgfqpoint{2.451277in}{1.829026in}}%
\pgfpathlineto{\pgfqpoint{2.451277in}{1.886546in}}%
\pgfpathlineto{\pgfqpoint{2.442524in}{1.886546in}}%
\pgfpathlineto{\pgfqpoint{2.442524in}{1.829026in}}%
\pgfpathclose%
\pgfusepath{fill}%
\end{pgfscope}%
\begin{pgfscope}%
\pgfpathrectangle{\pgfqpoint{0.804646in}{0.600000in}}{\pgfqpoint{2.573292in}{2.070576in}}%
\pgfusepath{clip}%
\pgfsetbuttcap%
\pgfsetmiterjoin%
\definecolor{currentfill}{rgb}{0.302379,0.450282,0.300122}%
\pgfsetfillcolor{currentfill}%
\pgfsetlinewidth{0.000000pt}%
\definecolor{currentstroke}{rgb}{0.000000,0.000000,0.000000}%
\pgfsetstrokecolor{currentstroke}%
\pgfsetstrokeopacity{0.000000}%
\pgfsetdash{}{0pt}%
\pgfpathmoveto{\pgfqpoint{2.453466in}{1.800951in}}%
\pgfpathlineto{\pgfqpoint{2.462219in}{1.800951in}}%
\pgfpathlineto{\pgfqpoint{2.462219in}{1.867025in}}%
\pgfpathlineto{\pgfqpoint{2.453466in}{1.867025in}}%
\pgfpathlineto{\pgfqpoint{2.453466in}{1.800951in}}%
\pgfpathclose%
\pgfusepath{fill}%
\end{pgfscope}%
\begin{pgfscope}%
\pgfpathrectangle{\pgfqpoint{0.804646in}{0.600000in}}{\pgfqpoint{2.573292in}{2.070576in}}%
\pgfusepath{clip}%
\pgfsetbuttcap%
\pgfsetmiterjoin%
\definecolor{currentfill}{rgb}{0.302379,0.450282,0.300122}%
\pgfsetfillcolor{currentfill}%
\pgfsetlinewidth{0.000000pt}%
\definecolor{currentstroke}{rgb}{0.000000,0.000000,0.000000}%
\pgfsetstrokecolor{currentstroke}%
\pgfsetstrokeopacity{0.000000}%
\pgfsetdash{}{0pt}%
\pgfpathmoveto{\pgfqpoint{2.464407in}{1.768657in}}%
\pgfpathlineto{\pgfqpoint{2.473161in}{1.768657in}}%
\pgfpathlineto{\pgfqpoint{2.473161in}{1.842987in}}%
\pgfpathlineto{\pgfqpoint{2.464407in}{1.842987in}}%
\pgfpathlineto{\pgfqpoint{2.464407in}{1.768657in}}%
\pgfpathclose%
\pgfusepath{fill}%
\end{pgfscope}%
\begin{pgfscope}%
\pgfpathrectangle{\pgfqpoint{0.804646in}{0.600000in}}{\pgfqpoint{2.573292in}{2.070576in}}%
\pgfusepath{clip}%
\pgfsetbuttcap%
\pgfsetmiterjoin%
\definecolor{currentfill}{rgb}{0.302379,0.450282,0.300122}%
\pgfsetfillcolor{currentfill}%
\pgfsetlinewidth{0.000000pt}%
\definecolor{currentstroke}{rgb}{0.000000,0.000000,0.000000}%
\pgfsetstrokecolor{currentstroke}%
\pgfsetstrokeopacity{0.000000}%
\pgfsetdash{}{0pt}%
\pgfpathmoveto{\pgfqpoint{2.475349in}{1.729166in}}%
\pgfpathlineto{\pgfqpoint{2.484103in}{1.729166in}}%
\pgfpathlineto{\pgfqpoint{2.484103in}{1.804860in}}%
\pgfpathlineto{\pgfqpoint{2.475349in}{1.804860in}}%
\pgfpathlineto{\pgfqpoint{2.475349in}{1.729166in}}%
\pgfpathclose%
\pgfusepath{fill}%
\end{pgfscope}%
\begin{pgfscope}%
\pgfpathrectangle{\pgfqpoint{0.804646in}{0.600000in}}{\pgfqpoint{2.573292in}{2.070576in}}%
\pgfusepath{clip}%
\pgfsetbuttcap%
\pgfsetmiterjoin%
\definecolor{currentfill}{rgb}{0.302379,0.450282,0.300122}%
\pgfsetfillcolor{currentfill}%
\pgfsetlinewidth{0.000000pt}%
\definecolor{currentstroke}{rgb}{0.000000,0.000000,0.000000}%
\pgfsetstrokecolor{currentstroke}%
\pgfsetstrokeopacity{0.000000}%
\pgfsetdash{}{0pt}%
\pgfpathmoveto{\pgfqpoint{2.486291in}{1.732241in}}%
\pgfpathlineto{\pgfqpoint{2.495044in}{1.732241in}}%
\pgfpathlineto{\pgfqpoint{2.495044in}{1.802446in}}%
\pgfpathlineto{\pgfqpoint{2.486291in}{1.802446in}}%
\pgfpathlineto{\pgfqpoint{2.486291in}{1.732241in}}%
\pgfpathclose%
\pgfusepath{fill}%
\end{pgfscope}%
\begin{pgfscope}%
\pgfpathrectangle{\pgfqpoint{0.804646in}{0.600000in}}{\pgfqpoint{2.573292in}{2.070576in}}%
\pgfusepath{clip}%
\pgfsetbuttcap%
\pgfsetmiterjoin%
\definecolor{currentfill}{rgb}{0.302379,0.450282,0.300122}%
\pgfsetfillcolor{currentfill}%
\pgfsetlinewidth{0.000000pt}%
\definecolor{currentstroke}{rgb}{0.000000,0.000000,0.000000}%
\pgfsetstrokecolor{currentstroke}%
\pgfsetstrokeopacity{0.000000}%
\pgfsetdash{}{0pt}%
\pgfpathmoveto{\pgfqpoint{2.497233in}{1.725968in}}%
\pgfpathlineto{\pgfqpoint{2.505986in}{1.725968in}}%
\pgfpathlineto{\pgfqpoint{2.505986in}{1.791248in}}%
\pgfpathlineto{\pgfqpoint{2.497233in}{1.791248in}}%
\pgfpathlineto{\pgfqpoint{2.497233in}{1.725968in}}%
\pgfpathclose%
\pgfusepath{fill}%
\end{pgfscope}%
\begin{pgfscope}%
\pgfpathrectangle{\pgfqpoint{0.804646in}{0.600000in}}{\pgfqpoint{2.573292in}{2.070576in}}%
\pgfusepath{clip}%
\pgfsetbuttcap%
\pgfsetmiterjoin%
\definecolor{currentfill}{rgb}{0.302379,0.450282,0.300122}%
\pgfsetfillcolor{currentfill}%
\pgfsetlinewidth{0.000000pt}%
\definecolor{currentstroke}{rgb}{0.000000,0.000000,0.000000}%
\pgfsetstrokecolor{currentstroke}%
\pgfsetstrokeopacity{0.000000}%
\pgfsetdash{}{0pt}%
\pgfpathmoveto{\pgfqpoint{2.508174in}{1.732824in}}%
\pgfpathlineto{\pgfqpoint{2.516928in}{1.732824in}}%
\pgfpathlineto{\pgfqpoint{2.516928in}{1.794143in}}%
\pgfpathlineto{\pgfqpoint{2.508174in}{1.794143in}}%
\pgfpathlineto{\pgfqpoint{2.508174in}{1.732824in}}%
\pgfpathclose%
\pgfusepath{fill}%
\end{pgfscope}%
\begin{pgfscope}%
\pgfpathrectangle{\pgfqpoint{0.804646in}{0.600000in}}{\pgfqpoint{2.573292in}{2.070576in}}%
\pgfusepath{clip}%
\pgfsetbuttcap%
\pgfsetmiterjoin%
\definecolor{currentfill}{rgb}{0.302379,0.450282,0.300122}%
\pgfsetfillcolor{currentfill}%
\pgfsetlinewidth{0.000000pt}%
\definecolor{currentstroke}{rgb}{0.000000,0.000000,0.000000}%
\pgfsetstrokecolor{currentstroke}%
\pgfsetstrokeopacity{0.000000}%
\pgfsetdash{}{0pt}%
\pgfpathmoveto{\pgfqpoint{2.519116in}{1.739486in}}%
\pgfpathlineto{\pgfqpoint{2.527870in}{1.739486in}}%
\pgfpathlineto{\pgfqpoint{2.527870in}{1.797758in}}%
\pgfpathlineto{\pgfqpoint{2.519116in}{1.797758in}}%
\pgfpathlineto{\pgfqpoint{2.519116in}{1.739486in}}%
\pgfpathclose%
\pgfusepath{fill}%
\end{pgfscope}%
\begin{pgfscope}%
\pgfpathrectangle{\pgfqpoint{0.804646in}{0.600000in}}{\pgfqpoint{2.573292in}{2.070576in}}%
\pgfusepath{clip}%
\pgfsetbuttcap%
\pgfsetmiterjoin%
\definecolor{currentfill}{rgb}{0.302379,0.450282,0.300122}%
\pgfsetfillcolor{currentfill}%
\pgfsetlinewidth{0.000000pt}%
\definecolor{currentstroke}{rgb}{0.000000,0.000000,0.000000}%
\pgfsetstrokecolor{currentstroke}%
\pgfsetstrokeopacity{0.000000}%
\pgfsetdash{}{0pt}%
\pgfpathmoveto{\pgfqpoint{2.530058in}{1.722688in}}%
\pgfpathlineto{\pgfqpoint{2.538812in}{1.722688in}}%
\pgfpathlineto{\pgfqpoint{2.538812in}{1.775408in}}%
\pgfpathlineto{\pgfqpoint{2.530058in}{1.775408in}}%
\pgfpathlineto{\pgfqpoint{2.530058in}{1.722688in}}%
\pgfpathclose%
\pgfusepath{fill}%
\end{pgfscope}%
\begin{pgfscope}%
\pgfpathrectangle{\pgfqpoint{0.804646in}{0.600000in}}{\pgfqpoint{2.573292in}{2.070576in}}%
\pgfusepath{clip}%
\pgfsetbuttcap%
\pgfsetmiterjoin%
\definecolor{currentfill}{rgb}{0.302379,0.450282,0.300122}%
\pgfsetfillcolor{currentfill}%
\pgfsetlinewidth{0.000000pt}%
\definecolor{currentstroke}{rgb}{0.000000,0.000000,0.000000}%
\pgfsetstrokecolor{currentstroke}%
\pgfsetstrokeopacity{0.000000}%
\pgfsetdash{}{0pt}%
\pgfpathmoveto{\pgfqpoint{2.541000in}{1.718259in}}%
\pgfpathlineto{\pgfqpoint{2.549753in}{1.718259in}}%
\pgfpathlineto{\pgfqpoint{2.549753in}{1.764748in}}%
\pgfpathlineto{\pgfqpoint{2.541000in}{1.764748in}}%
\pgfpathlineto{\pgfqpoint{2.541000in}{1.718259in}}%
\pgfpathclose%
\pgfusepath{fill}%
\end{pgfscope}%
\begin{pgfscope}%
\pgfpathrectangle{\pgfqpoint{0.804646in}{0.600000in}}{\pgfqpoint{2.573292in}{2.070576in}}%
\pgfusepath{clip}%
\pgfsetbuttcap%
\pgfsetmiterjoin%
\definecolor{currentfill}{rgb}{0.302379,0.450282,0.300122}%
\pgfsetfillcolor{currentfill}%
\pgfsetlinewidth{0.000000pt}%
\definecolor{currentstroke}{rgb}{0.000000,0.000000,0.000000}%
\pgfsetstrokecolor{currentstroke}%
\pgfsetstrokeopacity{0.000000}%
\pgfsetdash{}{0pt}%
\pgfpathmoveto{\pgfqpoint{2.551942in}{1.670769in}}%
\pgfpathlineto{\pgfqpoint{2.560695in}{1.670769in}}%
\pgfpathlineto{\pgfqpoint{2.560695in}{1.707445in}}%
\pgfpathlineto{\pgfqpoint{2.551942in}{1.707445in}}%
\pgfpathlineto{\pgfqpoint{2.551942in}{1.670769in}}%
\pgfpathclose%
\pgfusepath{fill}%
\end{pgfscope}%
\begin{pgfscope}%
\pgfpathrectangle{\pgfqpoint{0.804646in}{0.600000in}}{\pgfqpoint{2.573292in}{2.070576in}}%
\pgfusepath{clip}%
\pgfsetbuttcap%
\pgfsetmiterjoin%
\definecolor{currentfill}{rgb}{0.302379,0.450282,0.300122}%
\pgfsetfillcolor{currentfill}%
\pgfsetlinewidth{0.000000pt}%
\definecolor{currentstroke}{rgb}{0.000000,0.000000,0.000000}%
\pgfsetstrokecolor{currentstroke}%
\pgfsetstrokeopacity{0.000000}%
\pgfsetdash{}{0pt}%
\pgfpathmoveto{\pgfqpoint{2.562883in}{1.670895in}}%
\pgfpathlineto{\pgfqpoint{2.571637in}{1.670895in}}%
\pgfpathlineto{\pgfqpoint{2.571637in}{1.700045in}}%
\pgfpathlineto{\pgfqpoint{2.562883in}{1.700045in}}%
\pgfpathlineto{\pgfqpoint{2.562883in}{1.670895in}}%
\pgfpathclose%
\pgfusepath{fill}%
\end{pgfscope}%
\begin{pgfscope}%
\pgfpathrectangle{\pgfqpoint{0.804646in}{0.600000in}}{\pgfqpoint{2.573292in}{2.070576in}}%
\pgfusepath{clip}%
\pgfsetbuttcap%
\pgfsetmiterjoin%
\definecolor{currentfill}{rgb}{0.302379,0.450282,0.300122}%
\pgfsetfillcolor{currentfill}%
\pgfsetlinewidth{0.000000pt}%
\definecolor{currentstroke}{rgb}{0.000000,0.000000,0.000000}%
\pgfsetstrokecolor{currentstroke}%
\pgfsetstrokeopacity{0.000000}%
\pgfsetdash{}{0pt}%
\pgfpathmoveto{\pgfqpoint{2.573825in}{1.664125in}}%
\pgfpathlineto{\pgfqpoint{2.582579in}{1.664125in}}%
\pgfpathlineto{\pgfqpoint{2.582579in}{1.684570in}}%
\pgfpathlineto{\pgfqpoint{2.573825in}{1.684570in}}%
\pgfpathlineto{\pgfqpoint{2.573825in}{1.664125in}}%
\pgfpathclose%
\pgfusepath{fill}%
\end{pgfscope}%
\begin{pgfscope}%
\pgfpathrectangle{\pgfqpoint{0.804646in}{0.600000in}}{\pgfqpoint{2.573292in}{2.070576in}}%
\pgfusepath{clip}%
\pgfsetbuttcap%
\pgfsetmiterjoin%
\definecolor{currentfill}{rgb}{0.302379,0.450282,0.300122}%
\pgfsetfillcolor{currentfill}%
\pgfsetlinewidth{0.000000pt}%
\definecolor{currentstroke}{rgb}{0.000000,0.000000,0.000000}%
\pgfsetstrokecolor{currentstroke}%
\pgfsetstrokeopacity{0.000000}%
\pgfsetdash{}{0pt}%
\pgfpathmoveto{\pgfqpoint{2.584767in}{1.671711in}}%
\pgfpathlineto{\pgfqpoint{2.593521in}{1.671711in}}%
\pgfpathlineto{\pgfqpoint{2.593521in}{1.684842in}}%
\pgfpathlineto{\pgfqpoint{2.584767in}{1.684842in}}%
\pgfpathlineto{\pgfqpoint{2.584767in}{1.671711in}}%
\pgfpathclose%
\pgfusepath{fill}%
\end{pgfscope}%
\begin{pgfscope}%
\pgfpathrectangle{\pgfqpoint{0.804646in}{0.600000in}}{\pgfqpoint{2.573292in}{2.070576in}}%
\pgfusepath{clip}%
\pgfsetbuttcap%
\pgfsetmiterjoin%
\definecolor{currentfill}{rgb}{0.302379,0.450282,0.300122}%
\pgfsetfillcolor{currentfill}%
\pgfsetlinewidth{0.000000pt}%
\definecolor{currentstroke}{rgb}{0.000000,0.000000,0.000000}%
\pgfsetstrokecolor{currentstroke}%
\pgfsetstrokeopacity{0.000000}%
\pgfsetdash{}{0pt}%
\pgfpathmoveto{\pgfqpoint{2.595709in}{1.676341in}}%
\pgfpathlineto{\pgfqpoint{2.604462in}{1.676341in}}%
\pgfpathlineto{\pgfqpoint{2.604462in}{1.680306in}}%
\pgfpathlineto{\pgfqpoint{2.595709in}{1.680306in}}%
\pgfpathlineto{\pgfqpoint{2.595709in}{1.676341in}}%
\pgfpathclose%
\pgfusepath{fill}%
\end{pgfscope}%
\begin{pgfscope}%
\pgfpathrectangle{\pgfqpoint{0.804646in}{0.600000in}}{\pgfqpoint{2.573292in}{2.070576in}}%
\pgfusepath{clip}%
\pgfsetbuttcap%
\pgfsetmiterjoin%
\definecolor{currentfill}{rgb}{0.302379,0.450282,0.300122}%
\pgfsetfillcolor{currentfill}%
\pgfsetlinewidth{0.000000pt}%
\definecolor{currentstroke}{rgb}{0.000000,0.000000,0.000000}%
\pgfsetstrokecolor{currentstroke}%
\pgfsetstrokeopacity{0.000000}%
\pgfsetdash{}{0pt}%
\pgfpathmoveto{\pgfqpoint{2.606651in}{1.527391in}}%
\pgfpathlineto{\pgfqpoint{2.615404in}{1.527391in}}%
\pgfpathlineto{\pgfqpoint{2.615404in}{1.520629in}}%
\pgfpathlineto{\pgfqpoint{2.606651in}{1.520629in}}%
\pgfpathlineto{\pgfqpoint{2.606651in}{1.527391in}}%
\pgfpathclose%
\pgfusepath{fill}%
\end{pgfscope}%
\begin{pgfscope}%
\pgfpathrectangle{\pgfqpoint{0.804646in}{0.600000in}}{\pgfqpoint{2.573292in}{2.070576in}}%
\pgfusepath{clip}%
\pgfsetbuttcap%
\pgfsetmiterjoin%
\definecolor{currentfill}{rgb}{0.302379,0.450282,0.300122}%
\pgfsetfillcolor{currentfill}%
\pgfsetlinewidth{0.000000pt}%
\definecolor{currentstroke}{rgb}{0.000000,0.000000,0.000000}%
\pgfsetstrokecolor{currentstroke}%
\pgfsetstrokeopacity{0.000000}%
\pgfsetdash{}{0pt}%
\pgfpathmoveto{\pgfqpoint{2.617592in}{1.533898in}}%
\pgfpathlineto{\pgfqpoint{2.626346in}{1.533898in}}%
\pgfpathlineto{\pgfqpoint{2.626346in}{1.519638in}}%
\pgfpathlineto{\pgfqpoint{2.617592in}{1.519638in}}%
\pgfpathlineto{\pgfqpoint{2.617592in}{1.533898in}}%
\pgfpathclose%
\pgfusepath{fill}%
\end{pgfscope}%
\begin{pgfscope}%
\pgfpathrectangle{\pgfqpoint{0.804646in}{0.600000in}}{\pgfqpoint{2.573292in}{2.070576in}}%
\pgfusepath{clip}%
\pgfsetbuttcap%
\pgfsetmiterjoin%
\definecolor{currentfill}{rgb}{0.302379,0.450282,0.300122}%
\pgfsetfillcolor{currentfill}%
\pgfsetlinewidth{0.000000pt}%
\definecolor{currentstroke}{rgb}{0.000000,0.000000,0.000000}%
\pgfsetstrokecolor{currentstroke}%
\pgfsetstrokeopacity{0.000000}%
\pgfsetdash{}{0pt}%
\pgfpathmoveto{\pgfqpoint{2.628534in}{1.514359in}}%
\pgfpathlineto{\pgfqpoint{2.637288in}{1.514359in}}%
\pgfpathlineto{\pgfqpoint{2.637288in}{1.492884in}}%
\pgfpathlineto{\pgfqpoint{2.628534in}{1.492884in}}%
\pgfpathlineto{\pgfqpoint{2.628534in}{1.514359in}}%
\pgfpathclose%
\pgfusepath{fill}%
\end{pgfscope}%
\begin{pgfscope}%
\pgfpathrectangle{\pgfqpoint{0.804646in}{0.600000in}}{\pgfqpoint{2.573292in}{2.070576in}}%
\pgfusepath{clip}%
\pgfsetbuttcap%
\pgfsetmiterjoin%
\definecolor{currentfill}{rgb}{0.302379,0.450282,0.300122}%
\pgfsetfillcolor{currentfill}%
\pgfsetlinewidth{0.000000pt}%
\definecolor{currentstroke}{rgb}{0.000000,0.000000,0.000000}%
\pgfsetstrokecolor{currentstroke}%
\pgfsetstrokeopacity{0.000000}%
\pgfsetdash{}{0pt}%
\pgfpathmoveto{\pgfqpoint{2.639476in}{1.507549in}}%
\pgfpathlineto{\pgfqpoint{2.648230in}{1.507549in}}%
\pgfpathlineto{\pgfqpoint{2.648230in}{1.482266in}}%
\pgfpathlineto{\pgfqpoint{2.639476in}{1.482266in}}%
\pgfpathlineto{\pgfqpoint{2.639476in}{1.507549in}}%
\pgfpathclose%
\pgfusepath{fill}%
\end{pgfscope}%
\begin{pgfscope}%
\pgfpathrectangle{\pgfqpoint{0.804646in}{0.600000in}}{\pgfqpoint{2.573292in}{2.070576in}}%
\pgfusepath{clip}%
\pgfsetbuttcap%
\pgfsetmiterjoin%
\definecolor{currentfill}{rgb}{0.302379,0.450282,0.300122}%
\pgfsetfillcolor{currentfill}%
\pgfsetlinewidth{0.000000pt}%
\definecolor{currentstroke}{rgb}{0.000000,0.000000,0.000000}%
\pgfsetstrokecolor{currentstroke}%
\pgfsetstrokeopacity{0.000000}%
\pgfsetdash{}{0pt}%
\pgfpathmoveto{\pgfqpoint{2.650418in}{1.517832in}}%
\pgfpathlineto{\pgfqpoint{2.659171in}{1.517832in}}%
\pgfpathlineto{\pgfqpoint{2.659171in}{1.488676in}}%
\pgfpathlineto{\pgfqpoint{2.650418in}{1.488676in}}%
\pgfpathlineto{\pgfqpoint{2.650418in}{1.517832in}}%
\pgfpathclose%
\pgfusepath{fill}%
\end{pgfscope}%
\begin{pgfscope}%
\pgfpathrectangle{\pgfqpoint{0.804646in}{0.600000in}}{\pgfqpoint{2.573292in}{2.070576in}}%
\pgfusepath{clip}%
\pgfsetbuttcap%
\pgfsetmiterjoin%
\definecolor{currentfill}{rgb}{0.302379,0.450282,0.300122}%
\pgfsetfillcolor{currentfill}%
\pgfsetlinewidth{0.000000pt}%
\definecolor{currentstroke}{rgb}{0.000000,0.000000,0.000000}%
\pgfsetstrokecolor{currentstroke}%
\pgfsetstrokeopacity{0.000000}%
\pgfsetdash{}{0pt}%
\pgfpathmoveto{\pgfqpoint{2.661360in}{1.554084in}}%
\pgfpathlineto{\pgfqpoint{2.670113in}{1.554084in}}%
\pgfpathlineto{\pgfqpoint{2.670113in}{1.523013in}}%
\pgfpathlineto{\pgfqpoint{2.661360in}{1.523013in}}%
\pgfpathlineto{\pgfqpoint{2.661360in}{1.554084in}}%
\pgfpathclose%
\pgfusepath{fill}%
\end{pgfscope}%
\begin{pgfscope}%
\pgfpathrectangle{\pgfqpoint{0.804646in}{0.600000in}}{\pgfqpoint{2.573292in}{2.070576in}}%
\pgfusepath{clip}%
\pgfsetbuttcap%
\pgfsetmiterjoin%
\definecolor{currentfill}{rgb}{0.302379,0.450282,0.300122}%
\pgfsetfillcolor{currentfill}%
\pgfsetlinewidth{0.000000pt}%
\definecolor{currentstroke}{rgb}{0.000000,0.000000,0.000000}%
\pgfsetstrokecolor{currentstroke}%
\pgfsetstrokeopacity{0.000000}%
\pgfsetdash{}{0pt}%
\pgfpathmoveto{\pgfqpoint{2.672301in}{1.556576in}}%
\pgfpathlineto{\pgfqpoint{2.681055in}{1.556576in}}%
\pgfpathlineto{\pgfqpoint{2.681055in}{1.524526in}}%
\pgfpathlineto{\pgfqpoint{2.672301in}{1.524526in}}%
\pgfpathlineto{\pgfqpoint{2.672301in}{1.556576in}}%
\pgfpathclose%
\pgfusepath{fill}%
\end{pgfscope}%
\begin{pgfscope}%
\pgfpathrectangle{\pgfqpoint{0.804646in}{0.600000in}}{\pgfqpoint{2.573292in}{2.070576in}}%
\pgfusepath{clip}%
\pgfsetbuttcap%
\pgfsetmiterjoin%
\definecolor{currentfill}{rgb}{0.302379,0.450282,0.300122}%
\pgfsetfillcolor{currentfill}%
\pgfsetlinewidth{0.000000pt}%
\definecolor{currentstroke}{rgb}{0.000000,0.000000,0.000000}%
\pgfsetstrokecolor{currentstroke}%
\pgfsetstrokeopacity{0.000000}%
\pgfsetdash{}{0pt}%
\pgfpathmoveto{\pgfqpoint{2.683243in}{1.613090in}}%
\pgfpathlineto{\pgfqpoint{2.691997in}{1.613090in}}%
\pgfpathlineto{\pgfqpoint{2.691997in}{1.588838in}}%
\pgfpathlineto{\pgfqpoint{2.683243in}{1.588838in}}%
\pgfpathlineto{\pgfqpoint{2.683243in}{1.613090in}}%
\pgfpathclose%
\pgfusepath{fill}%
\end{pgfscope}%
\begin{pgfscope}%
\pgfpathrectangle{\pgfqpoint{0.804646in}{0.600000in}}{\pgfqpoint{2.573292in}{2.070576in}}%
\pgfusepath{clip}%
\pgfsetbuttcap%
\pgfsetmiterjoin%
\definecolor{currentfill}{rgb}{0.302379,0.450282,0.300122}%
\pgfsetfillcolor{currentfill}%
\pgfsetlinewidth{0.000000pt}%
\definecolor{currentstroke}{rgb}{0.000000,0.000000,0.000000}%
\pgfsetstrokecolor{currentstroke}%
\pgfsetstrokeopacity{0.000000}%
\pgfsetdash{}{0pt}%
\pgfpathmoveto{\pgfqpoint{2.694185in}{1.562505in}}%
\pgfpathlineto{\pgfqpoint{2.702939in}{1.562505in}}%
\pgfpathlineto{\pgfqpoint{2.702939in}{1.537021in}}%
\pgfpathlineto{\pgfqpoint{2.694185in}{1.537021in}}%
\pgfpathlineto{\pgfqpoint{2.694185in}{1.562505in}}%
\pgfpathclose%
\pgfusepath{fill}%
\end{pgfscope}%
\begin{pgfscope}%
\pgfpathrectangle{\pgfqpoint{0.804646in}{0.600000in}}{\pgfqpoint{2.573292in}{2.070576in}}%
\pgfusepath{clip}%
\pgfsetbuttcap%
\pgfsetmiterjoin%
\definecolor{currentfill}{rgb}{0.302379,0.450282,0.300122}%
\pgfsetfillcolor{currentfill}%
\pgfsetlinewidth{0.000000pt}%
\definecolor{currentstroke}{rgb}{0.000000,0.000000,0.000000}%
\pgfsetstrokecolor{currentstroke}%
\pgfsetstrokeopacity{0.000000}%
\pgfsetdash{}{0pt}%
\pgfpathmoveto{\pgfqpoint{2.705127in}{1.607014in}}%
\pgfpathlineto{\pgfqpoint{2.713880in}{1.607014in}}%
\pgfpathlineto{\pgfqpoint{2.713880in}{1.585000in}}%
\pgfpathlineto{\pgfqpoint{2.705127in}{1.585000in}}%
\pgfpathlineto{\pgfqpoint{2.705127in}{1.607014in}}%
\pgfpathclose%
\pgfusepath{fill}%
\end{pgfscope}%
\begin{pgfscope}%
\pgfpathrectangle{\pgfqpoint{0.804646in}{0.600000in}}{\pgfqpoint{2.573292in}{2.070576in}}%
\pgfusepath{clip}%
\pgfsetbuttcap%
\pgfsetmiterjoin%
\definecolor{currentfill}{rgb}{0.302379,0.450282,0.300122}%
\pgfsetfillcolor{currentfill}%
\pgfsetlinewidth{0.000000pt}%
\definecolor{currentstroke}{rgb}{0.000000,0.000000,0.000000}%
\pgfsetstrokecolor{currentstroke}%
\pgfsetstrokeopacity{0.000000}%
\pgfsetdash{}{0pt}%
\pgfpathmoveto{\pgfqpoint{2.716069in}{1.613090in}}%
\pgfpathlineto{\pgfqpoint{2.724822in}{1.613090in}}%
\pgfpathlineto{\pgfqpoint{2.724822in}{1.596456in}}%
\pgfpathlineto{\pgfqpoint{2.716069in}{1.596456in}}%
\pgfpathlineto{\pgfqpoint{2.716069in}{1.613090in}}%
\pgfpathclose%
\pgfusepath{fill}%
\end{pgfscope}%
\begin{pgfscope}%
\pgfpathrectangle{\pgfqpoint{0.804646in}{0.600000in}}{\pgfqpoint{2.573292in}{2.070576in}}%
\pgfusepath{clip}%
\pgfsetbuttcap%
\pgfsetmiterjoin%
\definecolor{currentfill}{rgb}{0.302379,0.450282,0.300122}%
\pgfsetfillcolor{currentfill}%
\pgfsetlinewidth{0.000000pt}%
\definecolor{currentstroke}{rgb}{0.000000,0.000000,0.000000}%
\pgfsetstrokecolor{currentstroke}%
\pgfsetstrokeopacity{0.000000}%
\pgfsetdash{}{0pt}%
\pgfpathmoveto{\pgfqpoint{2.727010in}{1.613090in}}%
\pgfpathlineto{\pgfqpoint{2.735764in}{1.613090in}}%
\pgfpathlineto{\pgfqpoint{2.735764in}{1.603503in}}%
\pgfpathlineto{\pgfqpoint{2.727010in}{1.603503in}}%
\pgfpathlineto{\pgfqpoint{2.727010in}{1.613090in}}%
\pgfpathclose%
\pgfusepath{fill}%
\end{pgfscope}%
\begin{pgfscope}%
\pgfpathrectangle{\pgfqpoint{0.804646in}{0.600000in}}{\pgfqpoint{2.573292in}{2.070576in}}%
\pgfusepath{clip}%
\pgfsetbuttcap%
\pgfsetmiterjoin%
\definecolor{currentfill}{rgb}{0.302379,0.450282,0.300122}%
\pgfsetfillcolor{currentfill}%
\pgfsetlinewidth{0.000000pt}%
\definecolor{currentstroke}{rgb}{0.000000,0.000000,0.000000}%
\pgfsetstrokecolor{currentstroke}%
\pgfsetstrokeopacity{0.000000}%
\pgfsetdash{}{0pt}%
\pgfpathmoveto{\pgfqpoint{2.737952in}{1.740454in}}%
\pgfpathlineto{\pgfqpoint{2.746706in}{1.740454in}}%
\pgfpathlineto{\pgfqpoint{2.746706in}{1.741979in}}%
\pgfpathlineto{\pgfqpoint{2.737952in}{1.741979in}}%
\pgfpathlineto{\pgfqpoint{2.737952in}{1.740454in}}%
\pgfpathclose%
\pgfusepath{fill}%
\end{pgfscope}%
\begin{pgfscope}%
\pgfpathrectangle{\pgfqpoint{0.804646in}{0.600000in}}{\pgfqpoint{2.573292in}{2.070576in}}%
\pgfusepath{clip}%
\pgfsetbuttcap%
\pgfsetmiterjoin%
\definecolor{currentfill}{rgb}{0.302379,0.450282,0.300122}%
\pgfsetfillcolor{currentfill}%
\pgfsetlinewidth{0.000000pt}%
\definecolor{currentstroke}{rgb}{0.000000,0.000000,0.000000}%
\pgfsetstrokecolor{currentstroke}%
\pgfsetstrokeopacity{0.000000}%
\pgfsetdash{}{0pt}%
\pgfpathmoveto{\pgfqpoint{2.748894in}{1.728886in}}%
\pgfpathlineto{\pgfqpoint{2.757648in}{1.728886in}}%
\pgfpathlineto{\pgfqpoint{2.757648in}{1.743554in}}%
\pgfpathlineto{\pgfqpoint{2.748894in}{1.743554in}}%
\pgfpathlineto{\pgfqpoint{2.748894in}{1.728886in}}%
\pgfpathclose%
\pgfusepath{fill}%
\end{pgfscope}%
\begin{pgfscope}%
\pgfpathrectangle{\pgfqpoint{0.804646in}{0.600000in}}{\pgfqpoint{2.573292in}{2.070576in}}%
\pgfusepath{clip}%
\pgfsetbuttcap%
\pgfsetmiterjoin%
\definecolor{currentfill}{rgb}{0.302379,0.450282,0.300122}%
\pgfsetfillcolor{currentfill}%
\pgfsetlinewidth{0.000000pt}%
\definecolor{currentstroke}{rgb}{0.000000,0.000000,0.000000}%
\pgfsetstrokecolor{currentstroke}%
\pgfsetstrokeopacity{0.000000}%
\pgfsetdash{}{0pt}%
\pgfpathmoveto{\pgfqpoint{2.759836in}{1.682606in}}%
\pgfpathlineto{\pgfqpoint{2.768589in}{1.682606in}}%
\pgfpathlineto{\pgfqpoint{2.768589in}{1.699399in}}%
\pgfpathlineto{\pgfqpoint{2.759836in}{1.699399in}}%
\pgfpathlineto{\pgfqpoint{2.759836in}{1.682606in}}%
\pgfpathclose%
\pgfusepath{fill}%
\end{pgfscope}%
\begin{pgfscope}%
\pgfpathrectangle{\pgfqpoint{0.804646in}{0.600000in}}{\pgfqpoint{2.573292in}{2.070576in}}%
\pgfusepath{clip}%
\pgfsetbuttcap%
\pgfsetmiterjoin%
\definecolor{currentfill}{rgb}{0.302379,0.450282,0.300122}%
\pgfsetfillcolor{currentfill}%
\pgfsetlinewidth{0.000000pt}%
\definecolor{currentstroke}{rgb}{0.000000,0.000000,0.000000}%
\pgfsetstrokecolor{currentstroke}%
\pgfsetstrokeopacity{0.000000}%
\pgfsetdash{}{0pt}%
\pgfpathmoveto{\pgfqpoint{2.770778in}{1.741929in}}%
\pgfpathlineto{\pgfqpoint{2.779531in}{1.741929in}}%
\pgfpathlineto{\pgfqpoint{2.779531in}{1.770813in}}%
\pgfpathlineto{\pgfqpoint{2.770778in}{1.770813in}}%
\pgfpathlineto{\pgfqpoint{2.770778in}{1.741929in}}%
\pgfpathclose%
\pgfusepath{fill}%
\end{pgfscope}%
\begin{pgfscope}%
\pgfpathrectangle{\pgfqpoint{0.804646in}{0.600000in}}{\pgfqpoint{2.573292in}{2.070576in}}%
\pgfusepath{clip}%
\pgfsetbuttcap%
\pgfsetmiterjoin%
\definecolor{currentfill}{rgb}{0.302379,0.450282,0.300122}%
\pgfsetfillcolor{currentfill}%
\pgfsetlinewidth{0.000000pt}%
\definecolor{currentstroke}{rgb}{0.000000,0.000000,0.000000}%
\pgfsetstrokecolor{currentstroke}%
\pgfsetstrokeopacity{0.000000}%
\pgfsetdash{}{0pt}%
\pgfpathmoveto{\pgfqpoint{2.781719in}{1.820095in}}%
\pgfpathlineto{\pgfqpoint{2.790473in}{1.820095in}}%
\pgfpathlineto{\pgfqpoint{2.790473in}{1.857005in}}%
\pgfpathlineto{\pgfqpoint{2.781719in}{1.857005in}}%
\pgfpathlineto{\pgfqpoint{2.781719in}{1.820095in}}%
\pgfpathclose%
\pgfusepath{fill}%
\end{pgfscope}%
\begin{pgfscope}%
\pgfpathrectangle{\pgfqpoint{0.804646in}{0.600000in}}{\pgfqpoint{2.573292in}{2.070576in}}%
\pgfusepath{clip}%
\pgfsetbuttcap%
\pgfsetmiterjoin%
\definecolor{currentfill}{rgb}{0.302379,0.450282,0.300122}%
\pgfsetfillcolor{currentfill}%
\pgfsetlinewidth{0.000000pt}%
\definecolor{currentstroke}{rgb}{0.000000,0.000000,0.000000}%
\pgfsetstrokecolor{currentstroke}%
\pgfsetstrokeopacity{0.000000}%
\pgfsetdash{}{0pt}%
\pgfpathmoveto{\pgfqpoint{2.792661in}{1.882435in}}%
\pgfpathlineto{\pgfqpoint{2.801415in}{1.882435in}}%
\pgfpathlineto{\pgfqpoint{2.801415in}{1.919258in}}%
\pgfpathlineto{\pgfqpoint{2.792661in}{1.919258in}}%
\pgfpathlineto{\pgfqpoint{2.792661in}{1.882435in}}%
\pgfpathclose%
\pgfusepath{fill}%
\end{pgfscope}%
\begin{pgfscope}%
\pgfpathrectangle{\pgfqpoint{0.804646in}{0.600000in}}{\pgfqpoint{2.573292in}{2.070576in}}%
\pgfusepath{clip}%
\pgfsetbuttcap%
\pgfsetmiterjoin%
\definecolor{currentfill}{rgb}{0.302379,0.450282,0.300122}%
\pgfsetfillcolor{currentfill}%
\pgfsetlinewidth{0.000000pt}%
\definecolor{currentstroke}{rgb}{0.000000,0.000000,0.000000}%
\pgfsetstrokecolor{currentstroke}%
\pgfsetstrokeopacity{0.000000}%
\pgfsetdash{}{0pt}%
\pgfpathmoveto{\pgfqpoint{2.803603in}{1.867777in}}%
\pgfpathlineto{\pgfqpoint{2.812357in}{1.867777in}}%
\pgfpathlineto{\pgfqpoint{2.812357in}{1.897781in}}%
\pgfpathlineto{\pgfqpoint{2.803603in}{1.897781in}}%
\pgfpathlineto{\pgfqpoint{2.803603in}{1.867777in}}%
\pgfpathclose%
\pgfusepath{fill}%
\end{pgfscope}%
\begin{pgfscope}%
\pgfpathrectangle{\pgfqpoint{0.804646in}{0.600000in}}{\pgfqpoint{2.573292in}{2.070576in}}%
\pgfusepath{clip}%
\pgfsetbuttcap%
\pgfsetmiterjoin%
\definecolor{currentfill}{rgb}{0.302379,0.450282,0.300122}%
\pgfsetfillcolor{currentfill}%
\pgfsetlinewidth{0.000000pt}%
\definecolor{currentstroke}{rgb}{0.000000,0.000000,0.000000}%
\pgfsetstrokecolor{currentstroke}%
\pgfsetstrokeopacity{0.000000}%
\pgfsetdash{}{0pt}%
\pgfpathmoveto{\pgfqpoint{2.814545in}{1.898692in}}%
\pgfpathlineto{\pgfqpoint{2.823298in}{1.898692in}}%
\pgfpathlineto{\pgfqpoint{2.823298in}{1.925070in}}%
\pgfpathlineto{\pgfqpoint{2.814545in}{1.925070in}}%
\pgfpathlineto{\pgfqpoint{2.814545in}{1.898692in}}%
\pgfpathclose%
\pgfusepath{fill}%
\end{pgfscope}%
\begin{pgfscope}%
\pgfpathrectangle{\pgfqpoint{0.804646in}{0.600000in}}{\pgfqpoint{2.573292in}{2.070576in}}%
\pgfusepath{clip}%
\pgfsetbuttcap%
\pgfsetmiterjoin%
\definecolor{currentfill}{rgb}{0.302379,0.450282,0.300122}%
\pgfsetfillcolor{currentfill}%
\pgfsetlinewidth{0.000000pt}%
\definecolor{currentstroke}{rgb}{0.000000,0.000000,0.000000}%
\pgfsetstrokecolor{currentstroke}%
\pgfsetstrokeopacity{0.000000}%
\pgfsetdash{}{0pt}%
\pgfpathmoveto{\pgfqpoint{2.825487in}{1.931853in}}%
\pgfpathlineto{\pgfqpoint{2.834240in}{1.931853in}}%
\pgfpathlineto{\pgfqpoint{2.834240in}{1.955804in}}%
\pgfpathlineto{\pgfqpoint{2.825487in}{1.955804in}}%
\pgfpathlineto{\pgfqpoint{2.825487in}{1.931853in}}%
\pgfpathclose%
\pgfusepath{fill}%
\end{pgfscope}%
\begin{pgfscope}%
\pgfpathrectangle{\pgfqpoint{0.804646in}{0.600000in}}{\pgfqpoint{2.573292in}{2.070576in}}%
\pgfusepath{clip}%
\pgfsetbuttcap%
\pgfsetmiterjoin%
\definecolor{currentfill}{rgb}{0.302379,0.450282,0.300122}%
\pgfsetfillcolor{currentfill}%
\pgfsetlinewidth{0.000000pt}%
\definecolor{currentstroke}{rgb}{0.000000,0.000000,0.000000}%
\pgfsetstrokecolor{currentstroke}%
\pgfsetstrokeopacity{0.000000}%
\pgfsetdash{}{0pt}%
\pgfpathmoveto{\pgfqpoint{2.836428in}{1.902484in}}%
\pgfpathlineto{\pgfqpoint{2.845182in}{1.902484in}}%
\pgfpathlineto{\pgfqpoint{2.845182in}{1.918060in}}%
\pgfpathlineto{\pgfqpoint{2.836428in}{1.918060in}}%
\pgfpathlineto{\pgfqpoint{2.836428in}{1.902484in}}%
\pgfpathclose%
\pgfusepath{fill}%
\end{pgfscope}%
\begin{pgfscope}%
\pgfpathrectangle{\pgfqpoint{0.804646in}{0.600000in}}{\pgfqpoint{2.573292in}{2.070576in}}%
\pgfusepath{clip}%
\pgfsetbuttcap%
\pgfsetmiterjoin%
\definecolor{currentfill}{rgb}{0.302379,0.450282,0.300122}%
\pgfsetfillcolor{currentfill}%
\pgfsetlinewidth{0.000000pt}%
\definecolor{currentstroke}{rgb}{0.000000,0.000000,0.000000}%
\pgfsetstrokecolor{currentstroke}%
\pgfsetstrokeopacity{0.000000}%
\pgfsetdash{}{0pt}%
\pgfpathmoveto{\pgfqpoint{2.847370in}{1.924146in}}%
\pgfpathlineto{\pgfqpoint{2.856124in}{1.924146in}}%
\pgfpathlineto{\pgfqpoint{2.856124in}{1.935852in}}%
\pgfpathlineto{\pgfqpoint{2.847370in}{1.935852in}}%
\pgfpathlineto{\pgfqpoint{2.847370in}{1.924146in}}%
\pgfpathclose%
\pgfusepath{fill}%
\end{pgfscope}%
\begin{pgfscope}%
\pgfpathrectangle{\pgfqpoint{0.804646in}{0.600000in}}{\pgfqpoint{2.573292in}{2.070576in}}%
\pgfusepath{clip}%
\pgfsetbuttcap%
\pgfsetmiterjoin%
\definecolor{currentfill}{rgb}{0.302379,0.450282,0.300122}%
\pgfsetfillcolor{currentfill}%
\pgfsetlinewidth{0.000000pt}%
\definecolor{currentstroke}{rgb}{0.000000,0.000000,0.000000}%
\pgfsetstrokecolor{currentstroke}%
\pgfsetstrokeopacity{0.000000}%
\pgfsetdash{}{0pt}%
\pgfpathmoveto{\pgfqpoint{2.858312in}{1.929941in}}%
\pgfpathlineto{\pgfqpoint{2.867066in}{1.929941in}}%
\pgfpathlineto{\pgfqpoint{2.867066in}{1.937570in}}%
\pgfpathlineto{\pgfqpoint{2.858312in}{1.937570in}}%
\pgfpathlineto{\pgfqpoint{2.858312in}{1.929941in}}%
\pgfpathclose%
\pgfusepath{fill}%
\end{pgfscope}%
\begin{pgfscope}%
\pgfpathrectangle{\pgfqpoint{0.804646in}{0.600000in}}{\pgfqpoint{2.573292in}{2.070576in}}%
\pgfusepath{clip}%
\pgfsetbuttcap%
\pgfsetmiterjoin%
\definecolor{currentfill}{rgb}{0.302379,0.450282,0.300122}%
\pgfsetfillcolor{currentfill}%
\pgfsetlinewidth{0.000000pt}%
\definecolor{currentstroke}{rgb}{0.000000,0.000000,0.000000}%
\pgfsetstrokecolor{currentstroke}%
\pgfsetstrokeopacity{0.000000}%
\pgfsetdash{}{0pt}%
\pgfpathmoveto{\pgfqpoint{2.869254in}{1.944193in}}%
\pgfpathlineto{\pgfqpoint{2.878007in}{1.944193in}}%
\pgfpathlineto{\pgfqpoint{2.878007in}{1.947546in}}%
\pgfpathlineto{\pgfqpoint{2.869254in}{1.947546in}}%
\pgfpathlineto{\pgfqpoint{2.869254in}{1.944193in}}%
\pgfpathclose%
\pgfusepath{fill}%
\end{pgfscope}%
\begin{pgfscope}%
\pgfpathrectangle{\pgfqpoint{0.804646in}{0.600000in}}{\pgfqpoint{2.573292in}{2.070576in}}%
\pgfusepath{clip}%
\pgfsetbuttcap%
\pgfsetmiterjoin%
\definecolor{currentfill}{rgb}{0.302379,0.450282,0.300122}%
\pgfsetfillcolor{currentfill}%
\pgfsetlinewidth{0.000000pt}%
\definecolor{currentstroke}{rgb}{0.000000,0.000000,0.000000}%
\pgfsetstrokecolor{currentstroke}%
\pgfsetstrokeopacity{0.000000}%
\pgfsetdash{}{0pt}%
\pgfpathmoveto{\pgfqpoint{2.880196in}{1.374634in}}%
\pgfpathlineto{\pgfqpoint{2.888949in}{1.374634in}}%
\pgfpathlineto{\pgfqpoint{2.888949in}{1.369010in}}%
\pgfpathlineto{\pgfqpoint{2.880196in}{1.369010in}}%
\pgfpathlineto{\pgfqpoint{2.880196in}{1.374634in}}%
\pgfpathclose%
\pgfusepath{fill}%
\end{pgfscope}%
\begin{pgfscope}%
\pgfpathrectangle{\pgfqpoint{0.804646in}{0.600000in}}{\pgfqpoint{2.573292in}{2.070576in}}%
\pgfusepath{clip}%
\pgfsetbuttcap%
\pgfsetmiterjoin%
\definecolor{currentfill}{rgb}{0.302379,0.450282,0.300122}%
\pgfsetfillcolor{currentfill}%
\pgfsetlinewidth{0.000000pt}%
\definecolor{currentstroke}{rgb}{0.000000,0.000000,0.000000}%
\pgfsetstrokecolor{currentstroke}%
\pgfsetstrokeopacity{0.000000}%
\pgfsetdash{}{0pt}%
\pgfpathmoveto{\pgfqpoint{2.891137in}{1.365643in}}%
\pgfpathlineto{\pgfqpoint{2.899891in}{1.365643in}}%
\pgfpathlineto{\pgfqpoint{2.899891in}{1.358106in}}%
\pgfpathlineto{\pgfqpoint{2.891137in}{1.358106in}}%
\pgfpathlineto{\pgfqpoint{2.891137in}{1.365643in}}%
\pgfpathclose%
\pgfusepath{fill}%
\end{pgfscope}%
\begin{pgfscope}%
\pgfpathrectangle{\pgfqpoint{0.804646in}{0.600000in}}{\pgfqpoint{2.573292in}{2.070576in}}%
\pgfusepath{clip}%
\pgfsetbuttcap%
\pgfsetmiterjoin%
\definecolor{currentfill}{rgb}{0.302379,0.450282,0.300122}%
\pgfsetfillcolor{currentfill}%
\pgfsetlinewidth{0.000000pt}%
\definecolor{currentstroke}{rgb}{0.000000,0.000000,0.000000}%
\pgfsetstrokecolor{currentstroke}%
\pgfsetstrokeopacity{0.000000}%
\pgfsetdash{}{0pt}%
\pgfpathmoveto{\pgfqpoint{2.902079in}{1.381538in}}%
\pgfpathlineto{\pgfqpoint{2.910833in}{1.381538in}}%
\pgfpathlineto{\pgfqpoint{2.910833in}{1.374011in}}%
\pgfpathlineto{\pgfqpoint{2.902079in}{1.374011in}}%
\pgfpathlineto{\pgfqpoint{2.902079in}{1.381538in}}%
\pgfpathclose%
\pgfusepath{fill}%
\end{pgfscope}%
\begin{pgfscope}%
\pgfpathrectangle{\pgfqpoint{0.804646in}{0.600000in}}{\pgfqpoint{2.573292in}{2.070576in}}%
\pgfusepath{clip}%
\pgfsetbuttcap%
\pgfsetmiterjoin%
\definecolor{currentfill}{rgb}{0.302379,0.450282,0.300122}%
\pgfsetfillcolor{currentfill}%
\pgfsetlinewidth{0.000000pt}%
\definecolor{currentstroke}{rgb}{0.000000,0.000000,0.000000}%
\pgfsetstrokecolor{currentstroke}%
\pgfsetstrokeopacity{0.000000}%
\pgfsetdash{}{0pt}%
\pgfpathmoveto{\pgfqpoint{2.913021in}{1.385936in}}%
\pgfpathlineto{\pgfqpoint{2.921774in}{1.385936in}}%
\pgfpathlineto{\pgfqpoint{2.921774in}{1.369645in}}%
\pgfpathlineto{\pgfqpoint{2.913021in}{1.369645in}}%
\pgfpathlineto{\pgfqpoint{2.913021in}{1.385936in}}%
\pgfpathclose%
\pgfusepath{fill}%
\end{pgfscope}%
\begin{pgfscope}%
\pgfpathrectangle{\pgfqpoint{0.804646in}{0.600000in}}{\pgfqpoint{2.573292in}{2.070576in}}%
\pgfusepath{clip}%
\pgfsetbuttcap%
\pgfsetmiterjoin%
\definecolor{currentfill}{rgb}{0.302379,0.450282,0.300122}%
\pgfsetfillcolor{currentfill}%
\pgfsetlinewidth{0.000000pt}%
\definecolor{currentstroke}{rgb}{0.000000,0.000000,0.000000}%
\pgfsetstrokecolor{currentstroke}%
\pgfsetstrokeopacity{0.000000}%
\pgfsetdash{}{0pt}%
\pgfpathmoveto{\pgfqpoint{2.923963in}{1.384478in}}%
\pgfpathlineto{\pgfqpoint{2.932716in}{1.384478in}}%
\pgfpathlineto{\pgfqpoint{2.932716in}{1.364709in}}%
\pgfpathlineto{\pgfqpoint{2.923963in}{1.364709in}}%
\pgfpathlineto{\pgfqpoint{2.923963in}{1.384478in}}%
\pgfpathclose%
\pgfusepath{fill}%
\end{pgfscope}%
\begin{pgfscope}%
\pgfpathrectangle{\pgfqpoint{0.804646in}{0.600000in}}{\pgfqpoint{2.573292in}{2.070576in}}%
\pgfusepath{clip}%
\pgfsetbuttcap%
\pgfsetmiterjoin%
\definecolor{currentfill}{rgb}{0.302379,0.450282,0.300122}%
\pgfsetfillcolor{currentfill}%
\pgfsetlinewidth{0.000000pt}%
\definecolor{currentstroke}{rgb}{0.000000,0.000000,0.000000}%
\pgfsetstrokecolor{currentstroke}%
\pgfsetstrokeopacity{0.000000}%
\pgfsetdash{}{0pt}%
\pgfpathmoveto{\pgfqpoint{2.934905in}{1.384184in}}%
\pgfpathlineto{\pgfqpoint{2.943658in}{1.384184in}}%
\pgfpathlineto{\pgfqpoint{2.943658in}{1.359650in}}%
\pgfpathlineto{\pgfqpoint{2.934905in}{1.359650in}}%
\pgfpathlineto{\pgfqpoint{2.934905in}{1.384184in}}%
\pgfpathclose%
\pgfusepath{fill}%
\end{pgfscope}%
\begin{pgfscope}%
\pgfpathrectangle{\pgfqpoint{0.804646in}{0.600000in}}{\pgfqpoint{2.573292in}{2.070576in}}%
\pgfusepath{clip}%
\pgfsetbuttcap%
\pgfsetmiterjoin%
\definecolor{currentfill}{rgb}{0.302379,0.450282,0.300122}%
\pgfsetfillcolor{currentfill}%
\pgfsetlinewidth{0.000000pt}%
\definecolor{currentstroke}{rgb}{0.000000,0.000000,0.000000}%
\pgfsetstrokecolor{currentstroke}%
\pgfsetstrokeopacity{0.000000}%
\pgfsetdash{}{0pt}%
\pgfpathmoveto{\pgfqpoint{2.945846in}{1.392581in}}%
\pgfpathlineto{\pgfqpoint{2.954600in}{1.392581in}}%
\pgfpathlineto{\pgfqpoint{2.954600in}{1.364689in}}%
\pgfpathlineto{\pgfqpoint{2.945846in}{1.364689in}}%
\pgfpathlineto{\pgfqpoint{2.945846in}{1.392581in}}%
\pgfpathclose%
\pgfusepath{fill}%
\end{pgfscope}%
\begin{pgfscope}%
\pgfpathrectangle{\pgfqpoint{0.804646in}{0.600000in}}{\pgfqpoint{2.573292in}{2.070576in}}%
\pgfusepath{clip}%
\pgfsetbuttcap%
\pgfsetmiterjoin%
\definecolor{currentfill}{rgb}{0.302379,0.450282,0.300122}%
\pgfsetfillcolor{currentfill}%
\pgfsetlinewidth{0.000000pt}%
\definecolor{currentstroke}{rgb}{0.000000,0.000000,0.000000}%
\pgfsetstrokecolor{currentstroke}%
\pgfsetstrokeopacity{0.000000}%
\pgfsetdash{}{0pt}%
\pgfpathmoveto{\pgfqpoint{2.956788in}{1.395421in}}%
\pgfpathlineto{\pgfqpoint{2.965542in}{1.395421in}}%
\pgfpathlineto{\pgfqpoint{2.965542in}{1.367302in}}%
\pgfpathlineto{\pgfqpoint{2.956788in}{1.367302in}}%
\pgfpathlineto{\pgfqpoint{2.956788in}{1.395421in}}%
\pgfpathclose%
\pgfusepath{fill}%
\end{pgfscope}%
\begin{pgfscope}%
\pgfpathrectangle{\pgfqpoint{0.804646in}{0.600000in}}{\pgfqpoint{2.573292in}{2.070576in}}%
\pgfusepath{clip}%
\pgfsetbuttcap%
\pgfsetmiterjoin%
\definecolor{currentfill}{rgb}{0.302379,0.450282,0.300122}%
\pgfsetfillcolor{currentfill}%
\pgfsetlinewidth{0.000000pt}%
\definecolor{currentstroke}{rgb}{0.000000,0.000000,0.000000}%
\pgfsetstrokecolor{currentstroke}%
\pgfsetstrokeopacity{0.000000}%
\pgfsetdash{}{0pt}%
\pgfpathmoveto{\pgfqpoint{2.967730in}{1.403690in}}%
\pgfpathlineto{\pgfqpoint{2.976483in}{1.403690in}}%
\pgfpathlineto{\pgfqpoint{2.976483in}{1.374519in}}%
\pgfpathlineto{\pgfqpoint{2.967730in}{1.374519in}}%
\pgfpathlineto{\pgfqpoint{2.967730in}{1.403690in}}%
\pgfpathclose%
\pgfusepath{fill}%
\end{pgfscope}%
\begin{pgfscope}%
\pgfpathrectangle{\pgfqpoint{0.804646in}{0.600000in}}{\pgfqpoint{2.573292in}{2.070576in}}%
\pgfusepath{clip}%
\pgfsetbuttcap%
\pgfsetmiterjoin%
\definecolor{currentfill}{rgb}{0.302379,0.450282,0.300122}%
\pgfsetfillcolor{currentfill}%
\pgfsetlinewidth{0.000000pt}%
\definecolor{currentstroke}{rgb}{0.000000,0.000000,0.000000}%
\pgfsetstrokecolor{currentstroke}%
\pgfsetstrokeopacity{0.000000}%
\pgfsetdash{}{0pt}%
\pgfpathmoveto{\pgfqpoint{2.978672in}{1.400539in}}%
\pgfpathlineto{\pgfqpoint{2.987425in}{1.400539in}}%
\pgfpathlineto{\pgfqpoint{2.987425in}{1.367724in}}%
\pgfpathlineto{\pgfqpoint{2.978672in}{1.367724in}}%
\pgfpathlineto{\pgfqpoint{2.978672in}{1.400539in}}%
\pgfpathclose%
\pgfusepath{fill}%
\end{pgfscope}%
\begin{pgfscope}%
\pgfpathrectangle{\pgfqpoint{0.804646in}{0.600000in}}{\pgfqpoint{2.573292in}{2.070576in}}%
\pgfusepath{clip}%
\pgfsetbuttcap%
\pgfsetmiterjoin%
\definecolor{currentfill}{rgb}{0.302379,0.450282,0.300122}%
\pgfsetfillcolor{currentfill}%
\pgfsetlinewidth{0.000000pt}%
\definecolor{currentstroke}{rgb}{0.000000,0.000000,0.000000}%
\pgfsetstrokecolor{currentstroke}%
\pgfsetstrokeopacity{0.000000}%
\pgfsetdash{}{0pt}%
\pgfpathmoveto{\pgfqpoint{2.989614in}{1.396830in}}%
\pgfpathlineto{\pgfqpoint{2.998367in}{1.396830in}}%
\pgfpathlineto{\pgfqpoint{2.998367in}{1.361690in}}%
\pgfpathlineto{\pgfqpoint{2.989614in}{1.361690in}}%
\pgfpathlineto{\pgfqpoint{2.989614in}{1.396830in}}%
\pgfpathclose%
\pgfusepath{fill}%
\end{pgfscope}%
\begin{pgfscope}%
\pgfpathrectangle{\pgfqpoint{0.804646in}{0.600000in}}{\pgfqpoint{2.573292in}{2.070576in}}%
\pgfusepath{clip}%
\pgfsetbuttcap%
\pgfsetmiterjoin%
\definecolor{currentfill}{rgb}{0.302379,0.450282,0.300122}%
\pgfsetfillcolor{currentfill}%
\pgfsetlinewidth{0.000000pt}%
\definecolor{currentstroke}{rgb}{0.000000,0.000000,0.000000}%
\pgfsetstrokecolor{currentstroke}%
\pgfsetstrokeopacity{0.000000}%
\pgfsetdash{}{0pt}%
\pgfpathmoveto{\pgfqpoint{3.000555in}{1.402168in}}%
\pgfpathlineto{\pgfqpoint{3.009309in}{1.402168in}}%
\pgfpathlineto{\pgfqpoint{3.009309in}{1.368301in}}%
\pgfpathlineto{\pgfqpoint{3.000555in}{1.368301in}}%
\pgfpathlineto{\pgfqpoint{3.000555in}{1.402168in}}%
\pgfpathclose%
\pgfusepath{fill}%
\end{pgfscope}%
\begin{pgfscope}%
\pgfpathrectangle{\pgfqpoint{0.804646in}{0.600000in}}{\pgfqpoint{2.573292in}{2.070576in}}%
\pgfusepath{clip}%
\pgfsetbuttcap%
\pgfsetmiterjoin%
\definecolor{currentfill}{rgb}{0.302379,0.450282,0.300122}%
\pgfsetfillcolor{currentfill}%
\pgfsetlinewidth{0.000000pt}%
\definecolor{currentstroke}{rgb}{0.000000,0.000000,0.000000}%
\pgfsetstrokecolor{currentstroke}%
\pgfsetstrokeopacity{0.000000}%
\pgfsetdash{}{0pt}%
\pgfpathmoveto{\pgfqpoint{3.011497in}{1.421115in}}%
\pgfpathlineto{\pgfqpoint{3.020251in}{1.421115in}}%
\pgfpathlineto{\pgfqpoint{3.020251in}{1.381445in}}%
\pgfpathlineto{\pgfqpoint{3.011497in}{1.381445in}}%
\pgfpathlineto{\pgfqpoint{3.011497in}{1.421115in}}%
\pgfpathclose%
\pgfusepath{fill}%
\end{pgfscope}%
\begin{pgfscope}%
\pgfpathrectangle{\pgfqpoint{0.804646in}{0.600000in}}{\pgfqpoint{2.573292in}{2.070576in}}%
\pgfusepath{clip}%
\pgfsetbuttcap%
\pgfsetmiterjoin%
\definecolor{currentfill}{rgb}{0.302379,0.450282,0.300122}%
\pgfsetfillcolor{currentfill}%
\pgfsetlinewidth{0.000000pt}%
\definecolor{currentstroke}{rgb}{0.000000,0.000000,0.000000}%
\pgfsetstrokecolor{currentstroke}%
\pgfsetstrokeopacity{0.000000}%
\pgfsetdash{}{0pt}%
\pgfpathmoveto{\pgfqpoint{3.022439in}{1.434544in}}%
\pgfpathlineto{\pgfqpoint{3.031192in}{1.434544in}}%
\pgfpathlineto{\pgfqpoint{3.031192in}{1.393100in}}%
\pgfpathlineto{\pgfqpoint{3.022439in}{1.393100in}}%
\pgfpathlineto{\pgfqpoint{3.022439in}{1.434544in}}%
\pgfpathclose%
\pgfusepath{fill}%
\end{pgfscope}%
\begin{pgfscope}%
\pgfpathrectangle{\pgfqpoint{0.804646in}{0.600000in}}{\pgfqpoint{2.573292in}{2.070576in}}%
\pgfusepath{clip}%
\pgfsetbuttcap%
\pgfsetmiterjoin%
\definecolor{currentfill}{rgb}{0.302379,0.450282,0.300122}%
\pgfsetfillcolor{currentfill}%
\pgfsetlinewidth{0.000000pt}%
\definecolor{currentstroke}{rgb}{0.000000,0.000000,0.000000}%
\pgfsetstrokecolor{currentstroke}%
\pgfsetstrokeopacity{0.000000}%
\pgfsetdash{}{0pt}%
\pgfpathmoveto{\pgfqpoint{3.033381in}{1.458934in}}%
\pgfpathlineto{\pgfqpoint{3.042134in}{1.458934in}}%
\pgfpathlineto{\pgfqpoint{3.042134in}{1.419792in}}%
\pgfpathlineto{\pgfqpoint{3.033381in}{1.419792in}}%
\pgfpathlineto{\pgfqpoint{3.033381in}{1.458934in}}%
\pgfpathclose%
\pgfusepath{fill}%
\end{pgfscope}%
\begin{pgfscope}%
\pgfpathrectangle{\pgfqpoint{0.804646in}{0.600000in}}{\pgfqpoint{2.573292in}{2.070576in}}%
\pgfusepath{clip}%
\pgfsetbuttcap%
\pgfsetmiterjoin%
\definecolor{currentfill}{rgb}{0.302379,0.450282,0.300122}%
\pgfsetfillcolor{currentfill}%
\pgfsetlinewidth{0.000000pt}%
\definecolor{currentstroke}{rgb}{0.000000,0.000000,0.000000}%
\pgfsetstrokecolor{currentstroke}%
\pgfsetstrokeopacity{0.000000}%
\pgfsetdash{}{0pt}%
\pgfpathmoveto{\pgfqpoint{3.044323in}{1.462277in}}%
\pgfpathlineto{\pgfqpoint{3.053076in}{1.462277in}}%
\pgfpathlineto{\pgfqpoint{3.053076in}{1.424867in}}%
\pgfpathlineto{\pgfqpoint{3.044323in}{1.424867in}}%
\pgfpathlineto{\pgfqpoint{3.044323in}{1.462277in}}%
\pgfpathclose%
\pgfusepath{fill}%
\end{pgfscope}%
\begin{pgfscope}%
\pgfpathrectangle{\pgfqpoint{0.804646in}{0.600000in}}{\pgfqpoint{2.573292in}{2.070576in}}%
\pgfusepath{clip}%
\pgfsetbuttcap%
\pgfsetmiterjoin%
\definecolor{currentfill}{rgb}{0.302379,0.450282,0.300122}%
\pgfsetfillcolor{currentfill}%
\pgfsetlinewidth{0.000000pt}%
\definecolor{currentstroke}{rgb}{0.000000,0.000000,0.000000}%
\pgfsetstrokecolor{currentstroke}%
\pgfsetstrokeopacity{0.000000}%
\pgfsetdash{}{0pt}%
\pgfpathmoveto{\pgfqpoint{3.055264in}{1.456297in}}%
\pgfpathlineto{\pgfqpoint{3.064018in}{1.456297in}}%
\pgfpathlineto{\pgfqpoint{3.064018in}{1.413695in}}%
\pgfpathlineto{\pgfqpoint{3.055264in}{1.413695in}}%
\pgfpathlineto{\pgfqpoint{3.055264in}{1.456297in}}%
\pgfpathclose%
\pgfusepath{fill}%
\end{pgfscope}%
\begin{pgfscope}%
\pgfpathrectangle{\pgfqpoint{0.804646in}{0.600000in}}{\pgfqpoint{2.573292in}{2.070576in}}%
\pgfusepath{clip}%
\pgfsetbuttcap%
\pgfsetmiterjoin%
\definecolor{currentfill}{rgb}{0.302379,0.450282,0.300122}%
\pgfsetfillcolor{currentfill}%
\pgfsetlinewidth{0.000000pt}%
\definecolor{currentstroke}{rgb}{0.000000,0.000000,0.000000}%
\pgfsetstrokecolor{currentstroke}%
\pgfsetstrokeopacity{0.000000}%
\pgfsetdash{}{0pt}%
\pgfpathmoveto{\pgfqpoint{3.066206in}{1.463272in}}%
\pgfpathlineto{\pgfqpoint{3.074960in}{1.463272in}}%
\pgfpathlineto{\pgfqpoint{3.074960in}{1.420796in}}%
\pgfpathlineto{\pgfqpoint{3.066206in}{1.420796in}}%
\pgfpathlineto{\pgfqpoint{3.066206in}{1.463272in}}%
\pgfpathclose%
\pgfusepath{fill}%
\end{pgfscope}%
\begin{pgfscope}%
\pgfpathrectangle{\pgfqpoint{0.804646in}{0.600000in}}{\pgfqpoint{2.573292in}{2.070576in}}%
\pgfusepath{clip}%
\pgfsetbuttcap%
\pgfsetmiterjoin%
\definecolor{currentfill}{rgb}{0.302379,0.450282,0.300122}%
\pgfsetfillcolor{currentfill}%
\pgfsetlinewidth{0.000000pt}%
\definecolor{currentstroke}{rgb}{0.000000,0.000000,0.000000}%
\pgfsetstrokecolor{currentstroke}%
\pgfsetstrokeopacity{0.000000}%
\pgfsetdash{}{0pt}%
\pgfpathmoveto{\pgfqpoint{3.077148in}{1.468094in}}%
\pgfpathlineto{\pgfqpoint{3.085901in}{1.468094in}}%
\pgfpathlineto{\pgfqpoint{3.085901in}{1.433343in}}%
\pgfpathlineto{\pgfqpoint{3.077148in}{1.433343in}}%
\pgfpathlineto{\pgfqpoint{3.077148in}{1.468094in}}%
\pgfpathclose%
\pgfusepath{fill}%
\end{pgfscope}%
\begin{pgfscope}%
\pgfpathrectangle{\pgfqpoint{0.804646in}{0.600000in}}{\pgfqpoint{2.573292in}{2.070576in}}%
\pgfusepath{clip}%
\pgfsetbuttcap%
\pgfsetmiterjoin%
\definecolor{currentfill}{rgb}{0.302379,0.450282,0.300122}%
\pgfsetfillcolor{currentfill}%
\pgfsetlinewidth{0.000000pt}%
\definecolor{currentstroke}{rgb}{0.000000,0.000000,0.000000}%
\pgfsetstrokecolor{currentstroke}%
\pgfsetstrokeopacity{0.000000}%
\pgfsetdash{}{0pt}%
\pgfpathmoveto{\pgfqpoint{3.088090in}{1.476013in}}%
\pgfpathlineto{\pgfqpoint{3.096843in}{1.476013in}}%
\pgfpathlineto{\pgfqpoint{3.096843in}{1.441024in}}%
\pgfpathlineto{\pgfqpoint{3.088090in}{1.441024in}}%
\pgfpathlineto{\pgfqpoint{3.088090in}{1.476013in}}%
\pgfpathclose%
\pgfusepath{fill}%
\end{pgfscope}%
\begin{pgfscope}%
\pgfpathrectangle{\pgfqpoint{0.804646in}{0.600000in}}{\pgfqpoint{2.573292in}{2.070576in}}%
\pgfusepath{clip}%
\pgfsetbuttcap%
\pgfsetmiterjoin%
\definecolor{currentfill}{rgb}{0.302379,0.450282,0.300122}%
\pgfsetfillcolor{currentfill}%
\pgfsetlinewidth{0.000000pt}%
\definecolor{currentstroke}{rgb}{0.000000,0.000000,0.000000}%
\pgfsetstrokecolor{currentstroke}%
\pgfsetstrokeopacity{0.000000}%
\pgfsetdash{}{0pt}%
\pgfpathmoveto{\pgfqpoint{3.099032in}{1.468138in}}%
\pgfpathlineto{\pgfqpoint{3.107785in}{1.468138in}}%
\pgfpathlineto{\pgfqpoint{3.107785in}{1.427055in}}%
\pgfpathlineto{\pgfqpoint{3.099032in}{1.427055in}}%
\pgfpathlineto{\pgfqpoint{3.099032in}{1.468138in}}%
\pgfpathclose%
\pgfusepath{fill}%
\end{pgfscope}%
\begin{pgfscope}%
\pgfpathrectangle{\pgfqpoint{0.804646in}{0.600000in}}{\pgfqpoint{2.573292in}{2.070576in}}%
\pgfusepath{clip}%
\pgfsetbuttcap%
\pgfsetmiterjoin%
\definecolor{currentfill}{rgb}{0.302379,0.450282,0.300122}%
\pgfsetfillcolor{currentfill}%
\pgfsetlinewidth{0.000000pt}%
\definecolor{currentstroke}{rgb}{0.000000,0.000000,0.000000}%
\pgfsetstrokecolor{currentstroke}%
\pgfsetstrokeopacity{0.000000}%
\pgfsetdash{}{0pt}%
\pgfpathmoveto{\pgfqpoint{3.109973in}{1.482022in}}%
\pgfpathlineto{\pgfqpoint{3.118727in}{1.482022in}}%
\pgfpathlineto{\pgfqpoint{3.118727in}{1.441817in}}%
\pgfpathlineto{\pgfqpoint{3.109973in}{1.441817in}}%
\pgfpathlineto{\pgfqpoint{3.109973in}{1.482022in}}%
\pgfpathclose%
\pgfusepath{fill}%
\end{pgfscope}%
\begin{pgfscope}%
\pgfpathrectangle{\pgfqpoint{0.804646in}{0.600000in}}{\pgfqpoint{2.573292in}{2.070576in}}%
\pgfusepath{clip}%
\pgfsetbuttcap%
\pgfsetmiterjoin%
\definecolor{currentfill}{rgb}{0.302379,0.450282,0.300122}%
\pgfsetfillcolor{currentfill}%
\pgfsetlinewidth{0.000000pt}%
\definecolor{currentstroke}{rgb}{0.000000,0.000000,0.000000}%
\pgfsetstrokecolor{currentstroke}%
\pgfsetstrokeopacity{0.000000}%
\pgfsetdash{}{0pt}%
\pgfpathmoveto{\pgfqpoint{3.120915in}{1.473845in}}%
\pgfpathlineto{\pgfqpoint{3.129669in}{1.473845in}}%
\pgfpathlineto{\pgfqpoint{3.129669in}{1.431597in}}%
\pgfpathlineto{\pgfqpoint{3.120915in}{1.431597in}}%
\pgfpathlineto{\pgfqpoint{3.120915in}{1.473845in}}%
\pgfpathclose%
\pgfusepath{fill}%
\end{pgfscope}%
\begin{pgfscope}%
\pgfpathrectangle{\pgfqpoint{0.804646in}{0.600000in}}{\pgfqpoint{2.573292in}{2.070576in}}%
\pgfusepath{clip}%
\pgfsetbuttcap%
\pgfsetmiterjoin%
\definecolor{currentfill}{rgb}{0.302379,0.450282,0.300122}%
\pgfsetfillcolor{currentfill}%
\pgfsetlinewidth{0.000000pt}%
\definecolor{currentstroke}{rgb}{0.000000,0.000000,0.000000}%
\pgfsetstrokecolor{currentstroke}%
\pgfsetstrokeopacity{0.000000}%
\pgfsetdash{}{0pt}%
\pgfpathmoveto{\pgfqpoint{3.131857in}{1.486110in}}%
\pgfpathlineto{\pgfqpoint{3.140610in}{1.486110in}}%
\pgfpathlineto{\pgfqpoint{3.140610in}{1.437527in}}%
\pgfpathlineto{\pgfqpoint{3.131857in}{1.437527in}}%
\pgfpathlineto{\pgfqpoint{3.131857in}{1.486110in}}%
\pgfpathclose%
\pgfusepath{fill}%
\end{pgfscope}%
\begin{pgfscope}%
\pgfpathrectangle{\pgfqpoint{0.804646in}{0.600000in}}{\pgfqpoint{2.573292in}{2.070576in}}%
\pgfusepath{clip}%
\pgfsetbuttcap%
\pgfsetmiterjoin%
\definecolor{currentfill}{rgb}{0.302379,0.450282,0.300122}%
\pgfsetfillcolor{currentfill}%
\pgfsetlinewidth{0.000000pt}%
\definecolor{currentstroke}{rgb}{0.000000,0.000000,0.000000}%
\pgfsetstrokecolor{currentstroke}%
\pgfsetstrokeopacity{0.000000}%
\pgfsetdash{}{0pt}%
\pgfpathmoveto{\pgfqpoint{3.142799in}{1.497533in}}%
\pgfpathlineto{\pgfqpoint{3.151552in}{1.497533in}}%
\pgfpathlineto{\pgfqpoint{3.151552in}{1.449359in}}%
\pgfpathlineto{\pgfqpoint{3.142799in}{1.449359in}}%
\pgfpathlineto{\pgfqpoint{3.142799in}{1.497533in}}%
\pgfpathclose%
\pgfusepath{fill}%
\end{pgfscope}%
\begin{pgfscope}%
\pgfpathrectangle{\pgfqpoint{0.804646in}{0.600000in}}{\pgfqpoint{2.573292in}{2.070576in}}%
\pgfusepath{clip}%
\pgfsetbuttcap%
\pgfsetmiterjoin%
\definecolor{currentfill}{rgb}{0.302379,0.450282,0.300122}%
\pgfsetfillcolor{currentfill}%
\pgfsetlinewidth{0.000000pt}%
\definecolor{currentstroke}{rgb}{0.000000,0.000000,0.000000}%
\pgfsetstrokecolor{currentstroke}%
\pgfsetstrokeopacity{0.000000}%
\pgfsetdash{}{0pt}%
\pgfpathmoveto{\pgfqpoint{3.153741in}{1.490958in}}%
\pgfpathlineto{\pgfqpoint{3.162494in}{1.490958in}}%
\pgfpathlineto{\pgfqpoint{3.162494in}{1.440076in}}%
\pgfpathlineto{\pgfqpoint{3.153741in}{1.440076in}}%
\pgfpathlineto{\pgfqpoint{3.153741in}{1.490958in}}%
\pgfpathclose%
\pgfusepath{fill}%
\end{pgfscope}%
\begin{pgfscope}%
\pgfpathrectangle{\pgfqpoint{0.804646in}{0.600000in}}{\pgfqpoint{2.573292in}{2.070576in}}%
\pgfusepath{clip}%
\pgfsetbuttcap%
\pgfsetmiterjoin%
\definecolor{currentfill}{rgb}{0.302379,0.450282,0.300122}%
\pgfsetfillcolor{currentfill}%
\pgfsetlinewidth{0.000000pt}%
\definecolor{currentstroke}{rgb}{0.000000,0.000000,0.000000}%
\pgfsetstrokecolor{currentstroke}%
\pgfsetstrokeopacity{0.000000}%
\pgfsetdash{}{0pt}%
\pgfpathmoveto{\pgfqpoint{3.164682in}{1.501169in}}%
\pgfpathlineto{\pgfqpoint{3.173436in}{1.501169in}}%
\pgfpathlineto{\pgfqpoint{3.173436in}{1.446336in}}%
\pgfpathlineto{\pgfqpoint{3.164682in}{1.446336in}}%
\pgfpathlineto{\pgfqpoint{3.164682in}{1.501169in}}%
\pgfpathclose%
\pgfusepath{fill}%
\end{pgfscope}%
\begin{pgfscope}%
\pgfpathrectangle{\pgfqpoint{0.804646in}{0.600000in}}{\pgfqpoint{2.573292in}{2.070576in}}%
\pgfusepath{clip}%
\pgfsetbuttcap%
\pgfsetmiterjoin%
\definecolor{currentfill}{rgb}{0.302379,0.450282,0.300122}%
\pgfsetfillcolor{currentfill}%
\pgfsetlinewidth{0.000000pt}%
\definecolor{currentstroke}{rgb}{0.000000,0.000000,0.000000}%
\pgfsetstrokecolor{currentstroke}%
\pgfsetstrokeopacity{0.000000}%
\pgfsetdash{}{0pt}%
\pgfpathmoveto{\pgfqpoint{3.175624in}{1.507657in}}%
\pgfpathlineto{\pgfqpoint{3.184378in}{1.507657in}}%
\pgfpathlineto{\pgfqpoint{3.184378in}{1.451744in}}%
\pgfpathlineto{\pgfqpoint{3.175624in}{1.451744in}}%
\pgfpathlineto{\pgfqpoint{3.175624in}{1.507657in}}%
\pgfpathclose%
\pgfusepath{fill}%
\end{pgfscope}%
\begin{pgfscope}%
\pgfpathrectangle{\pgfqpoint{0.804646in}{0.600000in}}{\pgfqpoint{2.573292in}{2.070576in}}%
\pgfusepath{clip}%
\pgfsetbuttcap%
\pgfsetmiterjoin%
\definecolor{currentfill}{rgb}{0.302379,0.450282,0.300122}%
\pgfsetfillcolor{currentfill}%
\pgfsetlinewidth{0.000000pt}%
\definecolor{currentstroke}{rgb}{0.000000,0.000000,0.000000}%
\pgfsetstrokecolor{currentstroke}%
\pgfsetstrokeopacity{0.000000}%
\pgfsetdash{}{0pt}%
\pgfpathmoveto{\pgfqpoint{3.186566in}{1.518734in}}%
\pgfpathlineto{\pgfqpoint{3.195319in}{1.518734in}}%
\pgfpathlineto{\pgfqpoint{3.195319in}{1.457906in}}%
\pgfpathlineto{\pgfqpoint{3.186566in}{1.457906in}}%
\pgfpathlineto{\pgfqpoint{3.186566in}{1.518734in}}%
\pgfpathclose%
\pgfusepath{fill}%
\end{pgfscope}%
\begin{pgfscope}%
\pgfpathrectangle{\pgfqpoint{0.804646in}{0.600000in}}{\pgfqpoint{2.573292in}{2.070576in}}%
\pgfusepath{clip}%
\pgfsetbuttcap%
\pgfsetmiterjoin%
\definecolor{currentfill}{rgb}{0.302379,0.450282,0.300122}%
\pgfsetfillcolor{currentfill}%
\pgfsetlinewidth{0.000000pt}%
\definecolor{currentstroke}{rgb}{0.000000,0.000000,0.000000}%
\pgfsetstrokecolor{currentstroke}%
\pgfsetstrokeopacity{0.000000}%
\pgfsetdash{}{0pt}%
\pgfpathmoveto{\pgfqpoint{3.197508in}{1.527694in}}%
\pgfpathlineto{\pgfqpoint{3.206261in}{1.527694in}}%
\pgfpathlineto{\pgfqpoint{3.206261in}{1.468150in}}%
\pgfpathlineto{\pgfqpoint{3.197508in}{1.468150in}}%
\pgfpathlineto{\pgfqpoint{3.197508in}{1.527694in}}%
\pgfpathclose%
\pgfusepath{fill}%
\end{pgfscope}%
\begin{pgfscope}%
\pgfpathrectangle{\pgfqpoint{0.804646in}{0.600000in}}{\pgfqpoint{2.573292in}{2.070576in}}%
\pgfusepath{clip}%
\pgfsetbuttcap%
\pgfsetmiterjoin%
\definecolor{currentfill}{rgb}{0.302379,0.450282,0.300122}%
\pgfsetfillcolor{currentfill}%
\pgfsetlinewidth{0.000000pt}%
\definecolor{currentstroke}{rgb}{0.000000,0.000000,0.000000}%
\pgfsetstrokecolor{currentstroke}%
\pgfsetstrokeopacity{0.000000}%
\pgfsetdash{}{0pt}%
\pgfpathmoveto{\pgfqpoint{3.208450in}{1.529349in}}%
\pgfpathlineto{\pgfqpoint{3.217203in}{1.529349in}}%
\pgfpathlineto{\pgfqpoint{3.217203in}{1.471629in}}%
\pgfpathlineto{\pgfqpoint{3.208450in}{1.471629in}}%
\pgfpathlineto{\pgfqpoint{3.208450in}{1.529349in}}%
\pgfpathclose%
\pgfusepath{fill}%
\end{pgfscope}%
\begin{pgfscope}%
\pgfpathrectangle{\pgfqpoint{0.804646in}{0.600000in}}{\pgfqpoint{2.573292in}{2.070576in}}%
\pgfusepath{clip}%
\pgfsetbuttcap%
\pgfsetmiterjoin%
\definecolor{currentfill}{rgb}{0.302379,0.450282,0.300122}%
\pgfsetfillcolor{currentfill}%
\pgfsetlinewidth{0.000000pt}%
\definecolor{currentstroke}{rgb}{0.000000,0.000000,0.000000}%
\pgfsetstrokecolor{currentstroke}%
\pgfsetstrokeopacity{0.000000}%
\pgfsetdash{}{0pt}%
\pgfpathmoveto{\pgfqpoint{3.219391in}{1.522532in}}%
\pgfpathlineto{\pgfqpoint{3.228145in}{1.522532in}}%
\pgfpathlineto{\pgfqpoint{3.228145in}{1.467859in}}%
\pgfpathlineto{\pgfqpoint{3.219391in}{1.467859in}}%
\pgfpathlineto{\pgfqpoint{3.219391in}{1.522532in}}%
\pgfpathclose%
\pgfusepath{fill}%
\end{pgfscope}%
\begin{pgfscope}%
\pgfpathrectangle{\pgfqpoint{0.804646in}{0.600000in}}{\pgfqpoint{2.573292in}{2.070576in}}%
\pgfusepath{clip}%
\pgfsetbuttcap%
\pgfsetmiterjoin%
\definecolor{currentfill}{rgb}{0.302379,0.450282,0.300122}%
\pgfsetfillcolor{currentfill}%
\pgfsetlinewidth{0.000000pt}%
\definecolor{currentstroke}{rgb}{0.000000,0.000000,0.000000}%
\pgfsetstrokecolor{currentstroke}%
\pgfsetstrokeopacity{0.000000}%
\pgfsetdash{}{0pt}%
\pgfpathmoveto{\pgfqpoint{3.230333in}{1.529762in}}%
\pgfpathlineto{\pgfqpoint{3.239087in}{1.529762in}}%
\pgfpathlineto{\pgfqpoint{3.239087in}{1.475161in}}%
\pgfpathlineto{\pgfqpoint{3.230333in}{1.475161in}}%
\pgfpathlineto{\pgfqpoint{3.230333in}{1.529762in}}%
\pgfpathclose%
\pgfusepath{fill}%
\end{pgfscope}%
\begin{pgfscope}%
\pgfpathrectangle{\pgfqpoint{0.804646in}{0.600000in}}{\pgfqpoint{2.573292in}{2.070576in}}%
\pgfusepath{clip}%
\pgfsetbuttcap%
\pgfsetmiterjoin%
\definecolor{currentfill}{rgb}{0.302379,0.450282,0.300122}%
\pgfsetfillcolor{currentfill}%
\pgfsetlinewidth{0.000000pt}%
\definecolor{currentstroke}{rgb}{0.000000,0.000000,0.000000}%
\pgfsetstrokecolor{currentstroke}%
\pgfsetstrokeopacity{0.000000}%
\pgfsetdash{}{0pt}%
\pgfpathmoveto{\pgfqpoint{3.241275in}{1.531941in}}%
\pgfpathlineto{\pgfqpoint{3.250028in}{1.531941in}}%
\pgfpathlineto{\pgfqpoint{3.250028in}{1.483188in}}%
\pgfpathlineto{\pgfqpoint{3.241275in}{1.483188in}}%
\pgfpathlineto{\pgfqpoint{3.241275in}{1.531941in}}%
\pgfpathclose%
\pgfusepath{fill}%
\end{pgfscope}%
\begin{pgfscope}%
\pgfpathrectangle{\pgfqpoint{0.804646in}{0.600000in}}{\pgfqpoint{2.573292in}{2.070576in}}%
\pgfusepath{clip}%
\pgfsetbuttcap%
\pgfsetmiterjoin%
\definecolor{currentfill}{rgb}{0.302379,0.450282,0.300122}%
\pgfsetfillcolor{currentfill}%
\pgfsetlinewidth{0.000000pt}%
\definecolor{currentstroke}{rgb}{0.000000,0.000000,0.000000}%
\pgfsetstrokecolor{currentstroke}%
\pgfsetstrokeopacity{0.000000}%
\pgfsetdash{}{0pt}%
\pgfpathmoveto{\pgfqpoint{3.252217in}{1.521480in}}%
\pgfpathlineto{\pgfqpoint{3.260970in}{1.521480in}}%
\pgfpathlineto{\pgfqpoint{3.260970in}{1.479405in}}%
\pgfpathlineto{\pgfqpoint{3.252217in}{1.479405in}}%
\pgfpathlineto{\pgfqpoint{3.252217in}{1.521480in}}%
\pgfpathclose%
\pgfusepath{fill}%
\end{pgfscope}%
\begin{pgfscope}%
\pgfpathrectangle{\pgfqpoint{0.804646in}{0.600000in}}{\pgfqpoint{2.573292in}{2.070576in}}%
\pgfusepath{clip}%
\pgfsetbuttcap%
\pgfsetmiterjoin%
\definecolor{currentfill}{rgb}{0.511253,0.510898,0.193296}%
\pgfsetfillcolor{currentfill}%
\pgfsetlinewidth{0.000000pt}%
\definecolor{currentstroke}{rgb}{0.000000,0.000000,0.000000}%
\pgfsetstrokecolor{currentstroke}%
\pgfsetstrokeopacity{0.000000}%
\pgfsetdash{}{0pt}%
\pgfpathmoveto{\pgfqpoint{0.921614in}{1.546761in}}%
\pgfpathlineto{\pgfqpoint{0.930367in}{1.546761in}}%
\pgfpathlineto{\pgfqpoint{0.930367in}{1.507733in}}%
\pgfpathlineto{\pgfqpoint{0.921614in}{1.507733in}}%
\pgfpathlineto{\pgfqpoint{0.921614in}{1.546761in}}%
\pgfpathclose%
\pgfusepath{fill}%
\end{pgfscope}%
\begin{pgfscope}%
\pgfpathrectangle{\pgfqpoint{0.804646in}{0.600000in}}{\pgfqpoint{2.573292in}{2.070576in}}%
\pgfusepath{clip}%
\pgfsetbuttcap%
\pgfsetmiterjoin%
\definecolor{currentfill}{rgb}{0.511253,0.510898,0.193296}%
\pgfsetfillcolor{currentfill}%
\pgfsetlinewidth{0.000000pt}%
\definecolor{currentstroke}{rgb}{0.000000,0.000000,0.000000}%
\pgfsetstrokecolor{currentstroke}%
\pgfsetstrokeopacity{0.000000}%
\pgfsetdash{}{0pt}%
\pgfpathmoveto{\pgfqpoint{0.932555in}{1.537363in}}%
\pgfpathlineto{\pgfqpoint{0.941309in}{1.537363in}}%
\pgfpathlineto{\pgfqpoint{0.941309in}{1.450786in}}%
\pgfpathlineto{\pgfqpoint{0.932555in}{1.450786in}}%
\pgfpathlineto{\pgfqpoint{0.932555in}{1.537363in}}%
\pgfpathclose%
\pgfusepath{fill}%
\end{pgfscope}%
\begin{pgfscope}%
\pgfpathrectangle{\pgfqpoint{0.804646in}{0.600000in}}{\pgfqpoint{2.573292in}{2.070576in}}%
\pgfusepath{clip}%
\pgfsetbuttcap%
\pgfsetmiterjoin%
\definecolor{currentfill}{rgb}{0.511253,0.510898,0.193296}%
\pgfsetfillcolor{currentfill}%
\pgfsetlinewidth{0.000000pt}%
\definecolor{currentstroke}{rgb}{0.000000,0.000000,0.000000}%
\pgfsetstrokecolor{currentstroke}%
\pgfsetstrokeopacity{0.000000}%
\pgfsetdash{}{0pt}%
\pgfpathmoveto{\pgfqpoint{0.943497in}{1.530288in}}%
\pgfpathlineto{\pgfqpoint{0.952251in}{1.530288in}}%
\pgfpathlineto{\pgfqpoint{0.952251in}{1.333991in}}%
\pgfpathlineto{\pgfqpoint{0.943497in}{1.333991in}}%
\pgfpathlineto{\pgfqpoint{0.943497in}{1.530288in}}%
\pgfpathclose%
\pgfusepath{fill}%
\end{pgfscope}%
\begin{pgfscope}%
\pgfpathrectangle{\pgfqpoint{0.804646in}{0.600000in}}{\pgfqpoint{2.573292in}{2.070576in}}%
\pgfusepath{clip}%
\pgfsetbuttcap%
\pgfsetmiterjoin%
\definecolor{currentfill}{rgb}{0.511253,0.510898,0.193296}%
\pgfsetfillcolor{currentfill}%
\pgfsetlinewidth{0.000000pt}%
\definecolor{currentstroke}{rgb}{0.000000,0.000000,0.000000}%
\pgfsetstrokecolor{currentstroke}%
\pgfsetstrokeopacity{0.000000}%
\pgfsetdash{}{0pt}%
\pgfpathmoveto{\pgfqpoint{0.954439in}{1.516242in}}%
\pgfpathlineto{\pgfqpoint{0.963192in}{1.516242in}}%
\pgfpathlineto{\pgfqpoint{0.963192in}{1.321329in}}%
\pgfpathlineto{\pgfqpoint{0.954439in}{1.321329in}}%
\pgfpathlineto{\pgfqpoint{0.954439in}{1.516242in}}%
\pgfpathclose%
\pgfusepath{fill}%
\end{pgfscope}%
\begin{pgfscope}%
\pgfpathrectangle{\pgfqpoint{0.804646in}{0.600000in}}{\pgfqpoint{2.573292in}{2.070576in}}%
\pgfusepath{clip}%
\pgfsetbuttcap%
\pgfsetmiterjoin%
\definecolor{currentfill}{rgb}{0.511253,0.510898,0.193296}%
\pgfsetfillcolor{currentfill}%
\pgfsetlinewidth{0.000000pt}%
\definecolor{currentstroke}{rgb}{0.000000,0.000000,0.000000}%
\pgfsetstrokecolor{currentstroke}%
\pgfsetstrokeopacity{0.000000}%
\pgfsetdash{}{0pt}%
\pgfpathmoveto{\pgfqpoint{0.965381in}{1.488882in}}%
\pgfpathlineto{\pgfqpoint{0.974134in}{1.488882in}}%
\pgfpathlineto{\pgfqpoint{0.974134in}{1.400496in}}%
\pgfpathlineto{\pgfqpoint{0.965381in}{1.400496in}}%
\pgfpathlineto{\pgfqpoint{0.965381in}{1.488882in}}%
\pgfpathclose%
\pgfusepath{fill}%
\end{pgfscope}%
\begin{pgfscope}%
\pgfpathrectangle{\pgfqpoint{0.804646in}{0.600000in}}{\pgfqpoint{2.573292in}{2.070576in}}%
\pgfusepath{clip}%
\pgfsetbuttcap%
\pgfsetmiterjoin%
\definecolor{currentfill}{rgb}{0.511253,0.510898,0.193296}%
\pgfsetfillcolor{currentfill}%
\pgfsetlinewidth{0.000000pt}%
\definecolor{currentstroke}{rgb}{0.000000,0.000000,0.000000}%
\pgfsetstrokecolor{currentstroke}%
\pgfsetstrokeopacity{0.000000}%
\pgfsetdash{}{0pt}%
\pgfpathmoveto{\pgfqpoint{0.976323in}{1.488996in}}%
\pgfpathlineto{\pgfqpoint{0.985076in}{1.488996in}}%
\pgfpathlineto{\pgfqpoint{0.985076in}{1.414323in}}%
\pgfpathlineto{\pgfqpoint{0.976323in}{1.414323in}}%
\pgfpathlineto{\pgfqpoint{0.976323in}{1.488996in}}%
\pgfpathclose%
\pgfusepath{fill}%
\end{pgfscope}%
\begin{pgfscope}%
\pgfpathrectangle{\pgfqpoint{0.804646in}{0.600000in}}{\pgfqpoint{2.573292in}{2.070576in}}%
\pgfusepath{clip}%
\pgfsetbuttcap%
\pgfsetmiterjoin%
\definecolor{currentfill}{rgb}{0.511253,0.510898,0.193296}%
\pgfsetfillcolor{currentfill}%
\pgfsetlinewidth{0.000000pt}%
\definecolor{currentstroke}{rgb}{0.000000,0.000000,0.000000}%
\pgfsetstrokecolor{currentstroke}%
\pgfsetstrokeopacity{0.000000}%
\pgfsetdash{}{0pt}%
\pgfpathmoveto{\pgfqpoint{0.987264in}{1.491706in}}%
\pgfpathlineto{\pgfqpoint{0.996018in}{1.491706in}}%
\pgfpathlineto{\pgfqpoint{0.996018in}{1.363716in}}%
\pgfpathlineto{\pgfqpoint{0.987264in}{1.363716in}}%
\pgfpathlineto{\pgfqpoint{0.987264in}{1.491706in}}%
\pgfpathclose%
\pgfusepath{fill}%
\end{pgfscope}%
\begin{pgfscope}%
\pgfpathrectangle{\pgfqpoint{0.804646in}{0.600000in}}{\pgfqpoint{2.573292in}{2.070576in}}%
\pgfusepath{clip}%
\pgfsetbuttcap%
\pgfsetmiterjoin%
\definecolor{currentfill}{rgb}{0.511253,0.510898,0.193296}%
\pgfsetfillcolor{currentfill}%
\pgfsetlinewidth{0.000000pt}%
\definecolor{currentstroke}{rgb}{0.000000,0.000000,0.000000}%
\pgfsetstrokecolor{currentstroke}%
\pgfsetstrokeopacity{0.000000}%
\pgfsetdash{}{0pt}%
\pgfpathmoveto{\pgfqpoint{0.998206in}{1.511042in}}%
\pgfpathlineto{\pgfqpoint{1.006960in}{1.511042in}}%
\pgfpathlineto{\pgfqpoint{1.006960in}{1.264192in}}%
\pgfpathlineto{\pgfqpoint{0.998206in}{1.264192in}}%
\pgfpathlineto{\pgfqpoint{0.998206in}{1.511042in}}%
\pgfpathclose%
\pgfusepath{fill}%
\end{pgfscope}%
\begin{pgfscope}%
\pgfpathrectangle{\pgfqpoint{0.804646in}{0.600000in}}{\pgfqpoint{2.573292in}{2.070576in}}%
\pgfusepath{clip}%
\pgfsetbuttcap%
\pgfsetmiterjoin%
\definecolor{currentfill}{rgb}{0.511253,0.510898,0.193296}%
\pgfsetfillcolor{currentfill}%
\pgfsetlinewidth{0.000000pt}%
\definecolor{currentstroke}{rgb}{0.000000,0.000000,0.000000}%
\pgfsetstrokecolor{currentstroke}%
\pgfsetstrokeopacity{0.000000}%
\pgfsetdash{}{0pt}%
\pgfpathmoveto{\pgfqpoint{1.009148in}{1.535467in}}%
\pgfpathlineto{\pgfqpoint{1.017901in}{1.535467in}}%
\pgfpathlineto{\pgfqpoint{1.017901in}{1.334159in}}%
\pgfpathlineto{\pgfqpoint{1.009148in}{1.334159in}}%
\pgfpathlineto{\pgfqpoint{1.009148in}{1.535467in}}%
\pgfpathclose%
\pgfusepath{fill}%
\end{pgfscope}%
\begin{pgfscope}%
\pgfpathrectangle{\pgfqpoint{0.804646in}{0.600000in}}{\pgfqpoint{2.573292in}{2.070576in}}%
\pgfusepath{clip}%
\pgfsetbuttcap%
\pgfsetmiterjoin%
\definecolor{currentfill}{rgb}{0.511253,0.510898,0.193296}%
\pgfsetfillcolor{currentfill}%
\pgfsetlinewidth{0.000000pt}%
\definecolor{currentstroke}{rgb}{0.000000,0.000000,0.000000}%
\pgfsetstrokecolor{currentstroke}%
\pgfsetstrokeopacity{0.000000}%
\pgfsetdash{}{0pt}%
\pgfpathmoveto{\pgfqpoint{1.020090in}{1.528114in}}%
\pgfpathlineto{\pgfqpoint{1.028843in}{1.528114in}}%
\pgfpathlineto{\pgfqpoint{1.028843in}{1.300572in}}%
\pgfpathlineto{\pgfqpoint{1.020090in}{1.300572in}}%
\pgfpathlineto{\pgfqpoint{1.020090in}{1.528114in}}%
\pgfpathclose%
\pgfusepath{fill}%
\end{pgfscope}%
\begin{pgfscope}%
\pgfpathrectangle{\pgfqpoint{0.804646in}{0.600000in}}{\pgfqpoint{2.573292in}{2.070576in}}%
\pgfusepath{clip}%
\pgfsetbuttcap%
\pgfsetmiterjoin%
\definecolor{currentfill}{rgb}{0.511253,0.510898,0.193296}%
\pgfsetfillcolor{currentfill}%
\pgfsetlinewidth{0.000000pt}%
\definecolor{currentstroke}{rgb}{0.000000,0.000000,0.000000}%
\pgfsetstrokecolor{currentstroke}%
\pgfsetstrokeopacity{0.000000}%
\pgfsetdash{}{0pt}%
\pgfpathmoveto{\pgfqpoint{1.031032in}{1.548456in}}%
\pgfpathlineto{\pgfqpoint{1.039785in}{1.548456in}}%
\pgfpathlineto{\pgfqpoint{1.039785in}{1.255788in}}%
\pgfpathlineto{\pgfqpoint{1.031032in}{1.255788in}}%
\pgfpathlineto{\pgfqpoint{1.031032in}{1.548456in}}%
\pgfpathclose%
\pgfusepath{fill}%
\end{pgfscope}%
\begin{pgfscope}%
\pgfpathrectangle{\pgfqpoint{0.804646in}{0.600000in}}{\pgfqpoint{2.573292in}{2.070576in}}%
\pgfusepath{clip}%
\pgfsetbuttcap%
\pgfsetmiterjoin%
\definecolor{currentfill}{rgb}{0.511253,0.510898,0.193296}%
\pgfsetfillcolor{currentfill}%
\pgfsetlinewidth{0.000000pt}%
\definecolor{currentstroke}{rgb}{0.000000,0.000000,0.000000}%
\pgfsetstrokecolor{currentstroke}%
\pgfsetstrokeopacity{0.000000}%
\pgfsetdash{}{0pt}%
\pgfpathmoveto{\pgfqpoint{1.041973in}{1.553284in}}%
\pgfpathlineto{\pgfqpoint{1.050727in}{1.553284in}}%
\pgfpathlineto{\pgfqpoint{1.050727in}{1.138669in}}%
\pgfpathlineto{\pgfqpoint{1.041973in}{1.138669in}}%
\pgfpathlineto{\pgfqpoint{1.041973in}{1.553284in}}%
\pgfpathclose%
\pgfusepath{fill}%
\end{pgfscope}%
\begin{pgfscope}%
\pgfpathrectangle{\pgfqpoint{0.804646in}{0.600000in}}{\pgfqpoint{2.573292in}{2.070576in}}%
\pgfusepath{clip}%
\pgfsetbuttcap%
\pgfsetmiterjoin%
\definecolor{currentfill}{rgb}{0.511253,0.510898,0.193296}%
\pgfsetfillcolor{currentfill}%
\pgfsetlinewidth{0.000000pt}%
\definecolor{currentstroke}{rgb}{0.000000,0.000000,0.000000}%
\pgfsetstrokecolor{currentstroke}%
\pgfsetstrokeopacity{0.000000}%
\pgfsetdash{}{0pt}%
\pgfpathmoveto{\pgfqpoint{1.052915in}{1.552799in}}%
\pgfpathlineto{\pgfqpoint{1.061669in}{1.552799in}}%
\pgfpathlineto{\pgfqpoint{1.061669in}{1.226240in}}%
\pgfpathlineto{\pgfqpoint{1.052915in}{1.226240in}}%
\pgfpathlineto{\pgfqpoint{1.052915in}{1.552799in}}%
\pgfpathclose%
\pgfusepath{fill}%
\end{pgfscope}%
\begin{pgfscope}%
\pgfpathrectangle{\pgfqpoint{0.804646in}{0.600000in}}{\pgfqpoint{2.573292in}{2.070576in}}%
\pgfusepath{clip}%
\pgfsetbuttcap%
\pgfsetmiterjoin%
\definecolor{currentfill}{rgb}{0.511253,0.510898,0.193296}%
\pgfsetfillcolor{currentfill}%
\pgfsetlinewidth{0.000000pt}%
\definecolor{currentstroke}{rgb}{0.000000,0.000000,0.000000}%
\pgfsetstrokecolor{currentstroke}%
\pgfsetstrokeopacity{0.000000}%
\pgfsetdash{}{0pt}%
\pgfpathmoveto{\pgfqpoint{1.063857in}{1.548503in}}%
\pgfpathlineto{\pgfqpoint{1.072610in}{1.548503in}}%
\pgfpathlineto{\pgfqpoint{1.072610in}{1.135714in}}%
\pgfpathlineto{\pgfqpoint{1.063857in}{1.135714in}}%
\pgfpathlineto{\pgfqpoint{1.063857in}{1.548503in}}%
\pgfpathclose%
\pgfusepath{fill}%
\end{pgfscope}%
\begin{pgfscope}%
\pgfpathrectangle{\pgfqpoint{0.804646in}{0.600000in}}{\pgfqpoint{2.573292in}{2.070576in}}%
\pgfusepath{clip}%
\pgfsetbuttcap%
\pgfsetmiterjoin%
\definecolor{currentfill}{rgb}{0.511253,0.510898,0.193296}%
\pgfsetfillcolor{currentfill}%
\pgfsetlinewidth{0.000000pt}%
\definecolor{currentstroke}{rgb}{0.000000,0.000000,0.000000}%
\pgfsetstrokecolor{currentstroke}%
\pgfsetstrokeopacity{0.000000}%
\pgfsetdash{}{0pt}%
\pgfpathmoveto{\pgfqpoint{1.074799in}{1.541520in}}%
\pgfpathlineto{\pgfqpoint{1.083552in}{1.541520in}}%
\pgfpathlineto{\pgfqpoint{1.083552in}{1.218543in}}%
\pgfpathlineto{\pgfqpoint{1.074799in}{1.218543in}}%
\pgfpathlineto{\pgfqpoint{1.074799in}{1.541520in}}%
\pgfpathclose%
\pgfusepath{fill}%
\end{pgfscope}%
\begin{pgfscope}%
\pgfpathrectangle{\pgfqpoint{0.804646in}{0.600000in}}{\pgfqpoint{2.573292in}{2.070576in}}%
\pgfusepath{clip}%
\pgfsetbuttcap%
\pgfsetmiterjoin%
\definecolor{currentfill}{rgb}{0.511253,0.510898,0.193296}%
\pgfsetfillcolor{currentfill}%
\pgfsetlinewidth{0.000000pt}%
\definecolor{currentstroke}{rgb}{0.000000,0.000000,0.000000}%
\pgfsetstrokecolor{currentstroke}%
\pgfsetstrokeopacity{0.000000}%
\pgfsetdash{}{0pt}%
\pgfpathmoveto{\pgfqpoint{1.085741in}{1.500327in}}%
\pgfpathlineto{\pgfqpoint{1.094494in}{1.500327in}}%
\pgfpathlineto{\pgfqpoint{1.094494in}{1.149838in}}%
\pgfpathlineto{\pgfqpoint{1.085741in}{1.149838in}}%
\pgfpathlineto{\pgfqpoint{1.085741in}{1.500327in}}%
\pgfpathclose%
\pgfusepath{fill}%
\end{pgfscope}%
\begin{pgfscope}%
\pgfpathrectangle{\pgfqpoint{0.804646in}{0.600000in}}{\pgfqpoint{2.573292in}{2.070576in}}%
\pgfusepath{clip}%
\pgfsetbuttcap%
\pgfsetmiterjoin%
\definecolor{currentfill}{rgb}{0.511253,0.510898,0.193296}%
\pgfsetfillcolor{currentfill}%
\pgfsetlinewidth{0.000000pt}%
\definecolor{currentstroke}{rgb}{0.000000,0.000000,0.000000}%
\pgfsetstrokecolor{currentstroke}%
\pgfsetstrokeopacity{0.000000}%
\pgfsetdash{}{0pt}%
\pgfpathmoveto{\pgfqpoint{1.096682in}{1.493277in}}%
\pgfpathlineto{\pgfqpoint{1.105436in}{1.493277in}}%
\pgfpathlineto{\pgfqpoint{1.105436in}{1.110070in}}%
\pgfpathlineto{\pgfqpoint{1.096682in}{1.110070in}}%
\pgfpathlineto{\pgfqpoint{1.096682in}{1.493277in}}%
\pgfpathclose%
\pgfusepath{fill}%
\end{pgfscope}%
\begin{pgfscope}%
\pgfpathrectangle{\pgfqpoint{0.804646in}{0.600000in}}{\pgfqpoint{2.573292in}{2.070576in}}%
\pgfusepath{clip}%
\pgfsetbuttcap%
\pgfsetmiterjoin%
\definecolor{currentfill}{rgb}{0.511253,0.510898,0.193296}%
\pgfsetfillcolor{currentfill}%
\pgfsetlinewidth{0.000000pt}%
\definecolor{currentstroke}{rgb}{0.000000,0.000000,0.000000}%
\pgfsetstrokecolor{currentstroke}%
\pgfsetstrokeopacity{0.000000}%
\pgfsetdash{}{0pt}%
\pgfpathmoveto{\pgfqpoint{1.107624in}{1.461982in}}%
\pgfpathlineto{\pgfqpoint{1.116378in}{1.461982in}}%
\pgfpathlineto{\pgfqpoint{1.116378in}{1.212922in}}%
\pgfpathlineto{\pgfqpoint{1.107624in}{1.212922in}}%
\pgfpathlineto{\pgfqpoint{1.107624in}{1.461982in}}%
\pgfpathclose%
\pgfusepath{fill}%
\end{pgfscope}%
\begin{pgfscope}%
\pgfpathrectangle{\pgfqpoint{0.804646in}{0.600000in}}{\pgfqpoint{2.573292in}{2.070576in}}%
\pgfusepath{clip}%
\pgfsetbuttcap%
\pgfsetmiterjoin%
\definecolor{currentfill}{rgb}{0.511253,0.510898,0.193296}%
\pgfsetfillcolor{currentfill}%
\pgfsetlinewidth{0.000000pt}%
\definecolor{currentstroke}{rgb}{0.000000,0.000000,0.000000}%
\pgfsetstrokecolor{currentstroke}%
\pgfsetstrokeopacity{0.000000}%
\pgfsetdash{}{0pt}%
\pgfpathmoveto{\pgfqpoint{1.118566in}{1.432183in}}%
\pgfpathlineto{\pgfqpoint{1.127319in}{1.432183in}}%
\pgfpathlineto{\pgfqpoint{1.127319in}{1.107201in}}%
\pgfpathlineto{\pgfqpoint{1.118566in}{1.107201in}}%
\pgfpathlineto{\pgfqpoint{1.118566in}{1.432183in}}%
\pgfpathclose%
\pgfusepath{fill}%
\end{pgfscope}%
\begin{pgfscope}%
\pgfpathrectangle{\pgfqpoint{0.804646in}{0.600000in}}{\pgfqpoint{2.573292in}{2.070576in}}%
\pgfusepath{clip}%
\pgfsetbuttcap%
\pgfsetmiterjoin%
\definecolor{currentfill}{rgb}{0.511253,0.510898,0.193296}%
\pgfsetfillcolor{currentfill}%
\pgfsetlinewidth{0.000000pt}%
\definecolor{currentstroke}{rgb}{0.000000,0.000000,0.000000}%
\pgfsetstrokecolor{currentstroke}%
\pgfsetstrokeopacity{0.000000}%
\pgfsetdash{}{0pt}%
\pgfpathmoveto{\pgfqpoint{1.129508in}{1.442001in}}%
\pgfpathlineto{\pgfqpoint{1.138261in}{1.442001in}}%
\pgfpathlineto{\pgfqpoint{1.138261in}{1.162842in}}%
\pgfpathlineto{\pgfqpoint{1.129508in}{1.162842in}}%
\pgfpathlineto{\pgfqpoint{1.129508in}{1.442001in}}%
\pgfpathclose%
\pgfusepath{fill}%
\end{pgfscope}%
\begin{pgfscope}%
\pgfpathrectangle{\pgfqpoint{0.804646in}{0.600000in}}{\pgfqpoint{2.573292in}{2.070576in}}%
\pgfusepath{clip}%
\pgfsetbuttcap%
\pgfsetmiterjoin%
\definecolor{currentfill}{rgb}{0.511253,0.510898,0.193296}%
\pgfsetfillcolor{currentfill}%
\pgfsetlinewidth{0.000000pt}%
\definecolor{currentstroke}{rgb}{0.000000,0.000000,0.000000}%
\pgfsetstrokecolor{currentstroke}%
\pgfsetstrokeopacity{0.000000}%
\pgfsetdash{}{0pt}%
\pgfpathmoveto{\pgfqpoint{1.140450in}{1.438451in}}%
\pgfpathlineto{\pgfqpoint{1.149203in}{1.438451in}}%
\pgfpathlineto{\pgfqpoint{1.149203in}{1.132005in}}%
\pgfpathlineto{\pgfqpoint{1.140450in}{1.132005in}}%
\pgfpathlineto{\pgfqpoint{1.140450in}{1.438451in}}%
\pgfpathclose%
\pgfusepath{fill}%
\end{pgfscope}%
\begin{pgfscope}%
\pgfpathrectangle{\pgfqpoint{0.804646in}{0.600000in}}{\pgfqpoint{2.573292in}{2.070576in}}%
\pgfusepath{clip}%
\pgfsetbuttcap%
\pgfsetmiterjoin%
\definecolor{currentfill}{rgb}{0.511253,0.510898,0.193296}%
\pgfsetfillcolor{currentfill}%
\pgfsetlinewidth{0.000000pt}%
\definecolor{currentstroke}{rgb}{0.000000,0.000000,0.000000}%
\pgfsetstrokecolor{currentstroke}%
\pgfsetstrokeopacity{0.000000}%
\pgfsetdash{}{0pt}%
\pgfpathmoveto{\pgfqpoint{1.151391in}{1.462150in}}%
\pgfpathlineto{\pgfqpoint{1.160145in}{1.462150in}}%
\pgfpathlineto{\pgfqpoint{1.160145in}{1.054456in}}%
\pgfpathlineto{\pgfqpoint{1.151391in}{1.054456in}}%
\pgfpathlineto{\pgfqpoint{1.151391in}{1.462150in}}%
\pgfpathclose%
\pgfusepath{fill}%
\end{pgfscope}%
\begin{pgfscope}%
\pgfpathrectangle{\pgfqpoint{0.804646in}{0.600000in}}{\pgfqpoint{2.573292in}{2.070576in}}%
\pgfusepath{clip}%
\pgfsetbuttcap%
\pgfsetmiterjoin%
\definecolor{currentfill}{rgb}{0.511253,0.510898,0.193296}%
\pgfsetfillcolor{currentfill}%
\pgfsetlinewidth{0.000000pt}%
\definecolor{currentstroke}{rgb}{0.000000,0.000000,0.000000}%
\pgfsetstrokecolor{currentstroke}%
\pgfsetstrokeopacity{0.000000}%
\pgfsetdash{}{0pt}%
\pgfpathmoveto{\pgfqpoint{1.162333in}{1.435889in}}%
\pgfpathlineto{\pgfqpoint{1.171087in}{1.435889in}}%
\pgfpathlineto{\pgfqpoint{1.171087in}{0.847387in}}%
\pgfpathlineto{\pgfqpoint{1.162333in}{0.847387in}}%
\pgfpathlineto{\pgfqpoint{1.162333in}{1.435889in}}%
\pgfpathclose%
\pgfusepath{fill}%
\end{pgfscope}%
\begin{pgfscope}%
\pgfpathrectangle{\pgfqpoint{0.804646in}{0.600000in}}{\pgfqpoint{2.573292in}{2.070576in}}%
\pgfusepath{clip}%
\pgfsetbuttcap%
\pgfsetmiterjoin%
\definecolor{currentfill}{rgb}{0.511253,0.510898,0.193296}%
\pgfsetfillcolor{currentfill}%
\pgfsetlinewidth{0.000000pt}%
\definecolor{currentstroke}{rgb}{0.000000,0.000000,0.000000}%
\pgfsetstrokecolor{currentstroke}%
\pgfsetstrokeopacity{0.000000}%
\pgfsetdash{}{0pt}%
\pgfpathmoveto{\pgfqpoint{1.173275in}{1.483102in}}%
\pgfpathlineto{\pgfqpoint{1.182028in}{1.483102in}}%
\pgfpathlineto{\pgfqpoint{1.182028in}{0.786972in}}%
\pgfpathlineto{\pgfqpoint{1.173275in}{0.786972in}}%
\pgfpathlineto{\pgfqpoint{1.173275in}{1.483102in}}%
\pgfpathclose%
\pgfusepath{fill}%
\end{pgfscope}%
\begin{pgfscope}%
\pgfpathrectangle{\pgfqpoint{0.804646in}{0.600000in}}{\pgfqpoint{2.573292in}{2.070576in}}%
\pgfusepath{clip}%
\pgfsetbuttcap%
\pgfsetmiterjoin%
\definecolor{currentfill}{rgb}{0.511253,0.510898,0.193296}%
\pgfsetfillcolor{currentfill}%
\pgfsetlinewidth{0.000000pt}%
\definecolor{currentstroke}{rgb}{0.000000,0.000000,0.000000}%
\pgfsetstrokecolor{currentstroke}%
\pgfsetstrokeopacity{0.000000}%
\pgfsetdash{}{0pt}%
\pgfpathmoveto{\pgfqpoint{1.184217in}{1.494695in}}%
\pgfpathlineto{\pgfqpoint{1.192970in}{1.494695in}}%
\pgfpathlineto{\pgfqpoint{1.192970in}{0.887362in}}%
\pgfpathlineto{\pgfqpoint{1.184217in}{0.887362in}}%
\pgfpathlineto{\pgfqpoint{1.184217in}{1.494695in}}%
\pgfpathclose%
\pgfusepath{fill}%
\end{pgfscope}%
\begin{pgfscope}%
\pgfpathrectangle{\pgfqpoint{0.804646in}{0.600000in}}{\pgfqpoint{2.573292in}{2.070576in}}%
\pgfusepath{clip}%
\pgfsetbuttcap%
\pgfsetmiterjoin%
\definecolor{currentfill}{rgb}{0.511253,0.510898,0.193296}%
\pgfsetfillcolor{currentfill}%
\pgfsetlinewidth{0.000000pt}%
\definecolor{currentstroke}{rgb}{0.000000,0.000000,0.000000}%
\pgfsetstrokecolor{currentstroke}%
\pgfsetstrokeopacity{0.000000}%
\pgfsetdash{}{0pt}%
\pgfpathmoveto{\pgfqpoint{1.195159in}{1.502287in}}%
\pgfpathlineto{\pgfqpoint{1.203912in}{1.502287in}}%
\pgfpathlineto{\pgfqpoint{1.203912in}{0.894937in}}%
\pgfpathlineto{\pgfqpoint{1.195159in}{0.894937in}}%
\pgfpathlineto{\pgfqpoint{1.195159in}{1.502287in}}%
\pgfpathclose%
\pgfusepath{fill}%
\end{pgfscope}%
\begin{pgfscope}%
\pgfpathrectangle{\pgfqpoint{0.804646in}{0.600000in}}{\pgfqpoint{2.573292in}{2.070576in}}%
\pgfusepath{clip}%
\pgfsetbuttcap%
\pgfsetmiterjoin%
\definecolor{currentfill}{rgb}{0.511253,0.510898,0.193296}%
\pgfsetfillcolor{currentfill}%
\pgfsetlinewidth{0.000000pt}%
\definecolor{currentstroke}{rgb}{0.000000,0.000000,0.000000}%
\pgfsetstrokecolor{currentstroke}%
\pgfsetstrokeopacity{0.000000}%
\pgfsetdash{}{0pt}%
\pgfpathmoveto{\pgfqpoint{1.206100in}{1.535708in}}%
\pgfpathlineto{\pgfqpoint{1.214854in}{1.535708in}}%
\pgfpathlineto{\pgfqpoint{1.214854in}{0.864851in}}%
\pgfpathlineto{\pgfqpoint{1.206100in}{0.864851in}}%
\pgfpathlineto{\pgfqpoint{1.206100in}{1.535708in}}%
\pgfpathclose%
\pgfusepath{fill}%
\end{pgfscope}%
\begin{pgfscope}%
\pgfpathrectangle{\pgfqpoint{0.804646in}{0.600000in}}{\pgfqpoint{2.573292in}{2.070576in}}%
\pgfusepath{clip}%
\pgfsetbuttcap%
\pgfsetmiterjoin%
\definecolor{currentfill}{rgb}{0.511253,0.510898,0.193296}%
\pgfsetfillcolor{currentfill}%
\pgfsetlinewidth{0.000000pt}%
\definecolor{currentstroke}{rgb}{0.000000,0.000000,0.000000}%
\pgfsetstrokecolor{currentstroke}%
\pgfsetstrokeopacity{0.000000}%
\pgfsetdash{}{0pt}%
\pgfpathmoveto{\pgfqpoint{1.217042in}{1.541524in}}%
\pgfpathlineto{\pgfqpoint{1.225796in}{1.541524in}}%
\pgfpathlineto{\pgfqpoint{1.225796in}{0.856539in}}%
\pgfpathlineto{\pgfqpoint{1.217042in}{0.856539in}}%
\pgfpathlineto{\pgfqpoint{1.217042in}{1.541524in}}%
\pgfpathclose%
\pgfusepath{fill}%
\end{pgfscope}%
\begin{pgfscope}%
\pgfpathrectangle{\pgfqpoint{0.804646in}{0.600000in}}{\pgfqpoint{2.573292in}{2.070576in}}%
\pgfusepath{clip}%
\pgfsetbuttcap%
\pgfsetmiterjoin%
\definecolor{currentfill}{rgb}{0.511253,0.510898,0.193296}%
\pgfsetfillcolor{currentfill}%
\pgfsetlinewidth{0.000000pt}%
\definecolor{currentstroke}{rgb}{0.000000,0.000000,0.000000}%
\pgfsetstrokecolor{currentstroke}%
\pgfsetstrokeopacity{0.000000}%
\pgfsetdash{}{0pt}%
\pgfpathmoveto{\pgfqpoint{1.227984in}{1.534934in}}%
\pgfpathlineto{\pgfqpoint{1.236737in}{1.534934in}}%
\pgfpathlineto{\pgfqpoint{1.236737in}{0.911162in}}%
\pgfpathlineto{\pgfqpoint{1.227984in}{0.911162in}}%
\pgfpathlineto{\pgfqpoint{1.227984in}{1.534934in}}%
\pgfpathclose%
\pgfusepath{fill}%
\end{pgfscope}%
\begin{pgfscope}%
\pgfpathrectangle{\pgfqpoint{0.804646in}{0.600000in}}{\pgfqpoint{2.573292in}{2.070576in}}%
\pgfusepath{clip}%
\pgfsetbuttcap%
\pgfsetmiterjoin%
\definecolor{currentfill}{rgb}{0.511253,0.510898,0.193296}%
\pgfsetfillcolor{currentfill}%
\pgfsetlinewidth{0.000000pt}%
\definecolor{currentstroke}{rgb}{0.000000,0.000000,0.000000}%
\pgfsetstrokecolor{currentstroke}%
\pgfsetstrokeopacity{0.000000}%
\pgfsetdash{}{0pt}%
\pgfpathmoveto{\pgfqpoint{1.238926in}{1.530828in}}%
\pgfpathlineto{\pgfqpoint{1.247679in}{1.530828in}}%
\pgfpathlineto{\pgfqpoint{1.247679in}{0.919349in}}%
\pgfpathlineto{\pgfqpoint{1.238926in}{0.919349in}}%
\pgfpathlineto{\pgfqpoint{1.238926in}{1.530828in}}%
\pgfpathclose%
\pgfusepath{fill}%
\end{pgfscope}%
\begin{pgfscope}%
\pgfpathrectangle{\pgfqpoint{0.804646in}{0.600000in}}{\pgfqpoint{2.573292in}{2.070576in}}%
\pgfusepath{clip}%
\pgfsetbuttcap%
\pgfsetmiterjoin%
\definecolor{currentfill}{rgb}{0.511253,0.510898,0.193296}%
\pgfsetfillcolor{currentfill}%
\pgfsetlinewidth{0.000000pt}%
\definecolor{currentstroke}{rgb}{0.000000,0.000000,0.000000}%
\pgfsetstrokecolor{currentstroke}%
\pgfsetstrokeopacity{0.000000}%
\pgfsetdash{}{0pt}%
\pgfpathmoveto{\pgfqpoint{1.249868in}{1.522874in}}%
\pgfpathlineto{\pgfqpoint{1.258621in}{1.522874in}}%
\pgfpathlineto{\pgfqpoint{1.258621in}{0.943366in}}%
\pgfpathlineto{\pgfqpoint{1.249868in}{0.943366in}}%
\pgfpathlineto{\pgfqpoint{1.249868in}{1.522874in}}%
\pgfpathclose%
\pgfusepath{fill}%
\end{pgfscope}%
\begin{pgfscope}%
\pgfpathrectangle{\pgfqpoint{0.804646in}{0.600000in}}{\pgfqpoint{2.573292in}{2.070576in}}%
\pgfusepath{clip}%
\pgfsetbuttcap%
\pgfsetmiterjoin%
\definecolor{currentfill}{rgb}{0.511253,0.510898,0.193296}%
\pgfsetfillcolor{currentfill}%
\pgfsetlinewidth{0.000000pt}%
\definecolor{currentstroke}{rgb}{0.000000,0.000000,0.000000}%
\pgfsetstrokecolor{currentstroke}%
\pgfsetstrokeopacity{0.000000}%
\pgfsetdash{}{0pt}%
\pgfpathmoveto{\pgfqpoint{1.260809in}{1.512039in}}%
\pgfpathlineto{\pgfqpoint{1.269563in}{1.512039in}}%
\pgfpathlineto{\pgfqpoint{1.269563in}{1.079402in}}%
\pgfpathlineto{\pgfqpoint{1.260809in}{1.079402in}}%
\pgfpathlineto{\pgfqpoint{1.260809in}{1.512039in}}%
\pgfpathclose%
\pgfusepath{fill}%
\end{pgfscope}%
\begin{pgfscope}%
\pgfpathrectangle{\pgfqpoint{0.804646in}{0.600000in}}{\pgfqpoint{2.573292in}{2.070576in}}%
\pgfusepath{clip}%
\pgfsetbuttcap%
\pgfsetmiterjoin%
\definecolor{currentfill}{rgb}{0.511253,0.510898,0.193296}%
\pgfsetfillcolor{currentfill}%
\pgfsetlinewidth{0.000000pt}%
\definecolor{currentstroke}{rgb}{0.000000,0.000000,0.000000}%
\pgfsetstrokecolor{currentstroke}%
\pgfsetstrokeopacity{0.000000}%
\pgfsetdash{}{0pt}%
\pgfpathmoveto{\pgfqpoint{1.271751in}{1.473024in}}%
\pgfpathlineto{\pgfqpoint{1.280505in}{1.473024in}}%
\pgfpathlineto{\pgfqpoint{1.280505in}{1.077507in}}%
\pgfpathlineto{\pgfqpoint{1.271751in}{1.077507in}}%
\pgfpathlineto{\pgfqpoint{1.271751in}{1.473024in}}%
\pgfpathclose%
\pgfusepath{fill}%
\end{pgfscope}%
\begin{pgfscope}%
\pgfpathrectangle{\pgfqpoint{0.804646in}{0.600000in}}{\pgfqpoint{2.573292in}{2.070576in}}%
\pgfusepath{clip}%
\pgfsetbuttcap%
\pgfsetmiterjoin%
\definecolor{currentfill}{rgb}{0.511253,0.510898,0.193296}%
\pgfsetfillcolor{currentfill}%
\pgfsetlinewidth{0.000000pt}%
\definecolor{currentstroke}{rgb}{0.000000,0.000000,0.000000}%
\pgfsetstrokecolor{currentstroke}%
\pgfsetstrokeopacity{0.000000}%
\pgfsetdash{}{0pt}%
\pgfpathmoveto{\pgfqpoint{1.282693in}{1.437865in}}%
\pgfpathlineto{\pgfqpoint{1.291446in}{1.437865in}}%
\pgfpathlineto{\pgfqpoint{1.291446in}{1.113049in}}%
\pgfpathlineto{\pgfqpoint{1.282693in}{1.113049in}}%
\pgfpathlineto{\pgfqpoint{1.282693in}{1.437865in}}%
\pgfpathclose%
\pgfusepath{fill}%
\end{pgfscope}%
\begin{pgfscope}%
\pgfpathrectangle{\pgfqpoint{0.804646in}{0.600000in}}{\pgfqpoint{2.573292in}{2.070576in}}%
\pgfusepath{clip}%
\pgfsetbuttcap%
\pgfsetmiterjoin%
\definecolor{currentfill}{rgb}{0.511253,0.510898,0.193296}%
\pgfsetfillcolor{currentfill}%
\pgfsetlinewidth{0.000000pt}%
\definecolor{currentstroke}{rgb}{0.000000,0.000000,0.000000}%
\pgfsetstrokecolor{currentstroke}%
\pgfsetstrokeopacity{0.000000}%
\pgfsetdash{}{0pt}%
\pgfpathmoveto{\pgfqpoint{1.293635in}{1.391210in}}%
\pgfpathlineto{\pgfqpoint{1.302388in}{1.391210in}}%
\pgfpathlineto{\pgfqpoint{1.302388in}{1.292228in}}%
\pgfpathlineto{\pgfqpoint{1.293635in}{1.292228in}}%
\pgfpathlineto{\pgfqpoint{1.293635in}{1.391210in}}%
\pgfpathclose%
\pgfusepath{fill}%
\end{pgfscope}%
\begin{pgfscope}%
\pgfpathrectangle{\pgfqpoint{0.804646in}{0.600000in}}{\pgfqpoint{2.573292in}{2.070576in}}%
\pgfusepath{clip}%
\pgfsetbuttcap%
\pgfsetmiterjoin%
\definecolor{currentfill}{rgb}{0.511253,0.510898,0.193296}%
\pgfsetfillcolor{currentfill}%
\pgfsetlinewidth{0.000000pt}%
\definecolor{currentstroke}{rgb}{0.000000,0.000000,0.000000}%
\pgfsetstrokecolor{currentstroke}%
\pgfsetstrokeopacity{0.000000}%
\pgfsetdash{}{0pt}%
\pgfpathmoveto{\pgfqpoint{1.304577in}{1.373521in}}%
\pgfpathlineto{\pgfqpoint{1.313330in}{1.373521in}}%
\pgfpathlineto{\pgfqpoint{1.313330in}{1.305331in}}%
\pgfpathlineto{\pgfqpoint{1.304577in}{1.305331in}}%
\pgfpathlineto{\pgfqpoint{1.304577in}{1.373521in}}%
\pgfpathclose%
\pgfusepath{fill}%
\end{pgfscope}%
\begin{pgfscope}%
\pgfpathrectangle{\pgfqpoint{0.804646in}{0.600000in}}{\pgfqpoint{2.573292in}{2.070576in}}%
\pgfusepath{clip}%
\pgfsetbuttcap%
\pgfsetmiterjoin%
\definecolor{currentfill}{rgb}{0.511253,0.510898,0.193296}%
\pgfsetfillcolor{currentfill}%
\pgfsetlinewidth{0.000000pt}%
\definecolor{currentstroke}{rgb}{0.000000,0.000000,0.000000}%
\pgfsetstrokecolor{currentstroke}%
\pgfsetstrokeopacity{0.000000}%
\pgfsetdash{}{0pt}%
\pgfpathmoveto{\pgfqpoint{1.315518in}{1.802593in}}%
\pgfpathlineto{\pgfqpoint{1.324272in}{1.802593in}}%
\pgfpathlineto{\pgfqpoint{1.324272in}{1.845394in}}%
\pgfpathlineto{\pgfqpoint{1.315518in}{1.845394in}}%
\pgfpathlineto{\pgfqpoint{1.315518in}{1.802593in}}%
\pgfpathclose%
\pgfusepath{fill}%
\end{pgfscope}%
\begin{pgfscope}%
\pgfpathrectangle{\pgfqpoint{0.804646in}{0.600000in}}{\pgfqpoint{2.573292in}{2.070576in}}%
\pgfusepath{clip}%
\pgfsetbuttcap%
\pgfsetmiterjoin%
\definecolor{currentfill}{rgb}{0.511253,0.510898,0.193296}%
\pgfsetfillcolor{currentfill}%
\pgfsetlinewidth{0.000000pt}%
\definecolor{currentstroke}{rgb}{0.000000,0.000000,0.000000}%
\pgfsetstrokecolor{currentstroke}%
\pgfsetstrokeopacity{0.000000}%
\pgfsetdash{}{0pt}%
\pgfpathmoveto{\pgfqpoint{1.326460in}{1.398562in}}%
\pgfpathlineto{\pgfqpoint{1.335214in}{1.398562in}}%
\pgfpathlineto{\pgfqpoint{1.335214in}{1.396613in}}%
\pgfpathlineto{\pgfqpoint{1.326460in}{1.396613in}}%
\pgfpathlineto{\pgfqpoint{1.326460in}{1.398562in}}%
\pgfpathclose%
\pgfusepath{fill}%
\end{pgfscope}%
\begin{pgfscope}%
\pgfpathrectangle{\pgfqpoint{0.804646in}{0.600000in}}{\pgfqpoint{2.573292in}{2.070576in}}%
\pgfusepath{clip}%
\pgfsetbuttcap%
\pgfsetmiterjoin%
\definecolor{currentfill}{rgb}{0.511253,0.510898,0.193296}%
\pgfsetfillcolor{currentfill}%
\pgfsetlinewidth{0.000000pt}%
\definecolor{currentstroke}{rgb}{0.000000,0.000000,0.000000}%
\pgfsetstrokecolor{currentstroke}%
\pgfsetstrokeopacity{0.000000}%
\pgfsetdash{}{0pt}%
\pgfpathmoveto{\pgfqpoint{1.337402in}{1.450069in}}%
\pgfpathlineto{\pgfqpoint{1.346155in}{1.450069in}}%
\pgfpathlineto{\pgfqpoint{1.346155in}{1.362958in}}%
\pgfpathlineto{\pgfqpoint{1.337402in}{1.362958in}}%
\pgfpathlineto{\pgfqpoint{1.337402in}{1.450069in}}%
\pgfpathclose%
\pgfusepath{fill}%
\end{pgfscope}%
\begin{pgfscope}%
\pgfpathrectangle{\pgfqpoint{0.804646in}{0.600000in}}{\pgfqpoint{2.573292in}{2.070576in}}%
\pgfusepath{clip}%
\pgfsetbuttcap%
\pgfsetmiterjoin%
\definecolor{currentfill}{rgb}{0.511253,0.510898,0.193296}%
\pgfsetfillcolor{currentfill}%
\pgfsetlinewidth{0.000000pt}%
\definecolor{currentstroke}{rgb}{0.000000,0.000000,0.000000}%
\pgfsetstrokecolor{currentstroke}%
\pgfsetstrokeopacity{0.000000}%
\pgfsetdash{}{0pt}%
\pgfpathmoveto{\pgfqpoint{1.348344in}{1.446229in}}%
\pgfpathlineto{\pgfqpoint{1.357097in}{1.446229in}}%
\pgfpathlineto{\pgfqpoint{1.357097in}{1.316484in}}%
\pgfpathlineto{\pgfqpoint{1.348344in}{1.316484in}}%
\pgfpathlineto{\pgfqpoint{1.348344in}{1.446229in}}%
\pgfpathclose%
\pgfusepath{fill}%
\end{pgfscope}%
\begin{pgfscope}%
\pgfpathrectangle{\pgfqpoint{0.804646in}{0.600000in}}{\pgfqpoint{2.573292in}{2.070576in}}%
\pgfusepath{clip}%
\pgfsetbuttcap%
\pgfsetmiterjoin%
\definecolor{currentfill}{rgb}{0.511253,0.510898,0.193296}%
\pgfsetfillcolor{currentfill}%
\pgfsetlinewidth{0.000000pt}%
\definecolor{currentstroke}{rgb}{0.000000,0.000000,0.000000}%
\pgfsetstrokecolor{currentstroke}%
\pgfsetstrokeopacity{0.000000}%
\pgfsetdash{}{0pt}%
\pgfpathmoveto{\pgfqpoint{1.359286in}{1.442991in}}%
\pgfpathlineto{\pgfqpoint{1.368039in}{1.442991in}}%
\pgfpathlineto{\pgfqpoint{1.368039in}{1.314360in}}%
\pgfpathlineto{\pgfqpoint{1.359286in}{1.314360in}}%
\pgfpathlineto{\pgfqpoint{1.359286in}{1.442991in}}%
\pgfpathclose%
\pgfusepath{fill}%
\end{pgfscope}%
\begin{pgfscope}%
\pgfpathrectangle{\pgfqpoint{0.804646in}{0.600000in}}{\pgfqpoint{2.573292in}{2.070576in}}%
\pgfusepath{clip}%
\pgfsetbuttcap%
\pgfsetmiterjoin%
\definecolor{currentfill}{rgb}{0.511253,0.510898,0.193296}%
\pgfsetfillcolor{currentfill}%
\pgfsetlinewidth{0.000000pt}%
\definecolor{currentstroke}{rgb}{0.000000,0.000000,0.000000}%
\pgfsetstrokecolor{currentstroke}%
\pgfsetstrokeopacity{0.000000}%
\pgfsetdash{}{0pt}%
\pgfpathmoveto{\pgfqpoint{1.370227in}{1.432015in}}%
\pgfpathlineto{\pgfqpoint{1.378981in}{1.432015in}}%
\pgfpathlineto{\pgfqpoint{1.378981in}{1.342759in}}%
\pgfpathlineto{\pgfqpoint{1.370227in}{1.342759in}}%
\pgfpathlineto{\pgfqpoint{1.370227in}{1.432015in}}%
\pgfpathclose%
\pgfusepath{fill}%
\end{pgfscope}%
\begin{pgfscope}%
\pgfpathrectangle{\pgfqpoint{0.804646in}{0.600000in}}{\pgfqpoint{2.573292in}{2.070576in}}%
\pgfusepath{clip}%
\pgfsetbuttcap%
\pgfsetmiterjoin%
\definecolor{currentfill}{rgb}{0.511253,0.510898,0.193296}%
\pgfsetfillcolor{currentfill}%
\pgfsetlinewidth{0.000000pt}%
\definecolor{currentstroke}{rgb}{0.000000,0.000000,0.000000}%
\pgfsetstrokecolor{currentstroke}%
\pgfsetstrokeopacity{0.000000}%
\pgfsetdash{}{0pt}%
\pgfpathmoveto{\pgfqpoint{1.381169in}{1.454196in}}%
\pgfpathlineto{\pgfqpoint{1.389923in}{1.454196in}}%
\pgfpathlineto{\pgfqpoint{1.389923in}{1.321919in}}%
\pgfpathlineto{\pgfqpoint{1.381169in}{1.321919in}}%
\pgfpathlineto{\pgfqpoint{1.381169in}{1.454196in}}%
\pgfpathclose%
\pgfusepath{fill}%
\end{pgfscope}%
\begin{pgfscope}%
\pgfpathrectangle{\pgfqpoint{0.804646in}{0.600000in}}{\pgfqpoint{2.573292in}{2.070576in}}%
\pgfusepath{clip}%
\pgfsetbuttcap%
\pgfsetmiterjoin%
\definecolor{currentfill}{rgb}{0.511253,0.510898,0.193296}%
\pgfsetfillcolor{currentfill}%
\pgfsetlinewidth{0.000000pt}%
\definecolor{currentstroke}{rgb}{0.000000,0.000000,0.000000}%
\pgfsetstrokecolor{currentstroke}%
\pgfsetstrokeopacity{0.000000}%
\pgfsetdash{}{0pt}%
\pgfpathmoveto{\pgfqpoint{1.392111in}{1.487831in}}%
\pgfpathlineto{\pgfqpoint{1.400864in}{1.487831in}}%
\pgfpathlineto{\pgfqpoint{1.400864in}{1.251226in}}%
\pgfpathlineto{\pgfqpoint{1.392111in}{1.251226in}}%
\pgfpathlineto{\pgfqpoint{1.392111in}{1.487831in}}%
\pgfpathclose%
\pgfusepath{fill}%
\end{pgfscope}%
\begin{pgfscope}%
\pgfpathrectangle{\pgfqpoint{0.804646in}{0.600000in}}{\pgfqpoint{2.573292in}{2.070576in}}%
\pgfusepath{clip}%
\pgfsetbuttcap%
\pgfsetmiterjoin%
\definecolor{currentfill}{rgb}{0.511253,0.510898,0.193296}%
\pgfsetfillcolor{currentfill}%
\pgfsetlinewidth{0.000000pt}%
\definecolor{currentstroke}{rgb}{0.000000,0.000000,0.000000}%
\pgfsetstrokecolor{currentstroke}%
\pgfsetstrokeopacity{0.000000}%
\pgfsetdash{}{0pt}%
\pgfpathmoveto{\pgfqpoint{1.403053in}{1.504632in}}%
\pgfpathlineto{\pgfqpoint{1.411806in}{1.504632in}}%
\pgfpathlineto{\pgfqpoint{1.411806in}{1.233540in}}%
\pgfpathlineto{\pgfqpoint{1.403053in}{1.233540in}}%
\pgfpathlineto{\pgfqpoint{1.403053in}{1.504632in}}%
\pgfpathclose%
\pgfusepath{fill}%
\end{pgfscope}%
\begin{pgfscope}%
\pgfpathrectangle{\pgfqpoint{0.804646in}{0.600000in}}{\pgfqpoint{2.573292in}{2.070576in}}%
\pgfusepath{clip}%
\pgfsetbuttcap%
\pgfsetmiterjoin%
\definecolor{currentfill}{rgb}{0.511253,0.510898,0.193296}%
\pgfsetfillcolor{currentfill}%
\pgfsetlinewidth{0.000000pt}%
\definecolor{currentstroke}{rgb}{0.000000,0.000000,0.000000}%
\pgfsetstrokecolor{currentstroke}%
\pgfsetstrokeopacity{0.000000}%
\pgfsetdash{}{0pt}%
\pgfpathmoveto{\pgfqpoint{1.413995in}{1.498440in}}%
\pgfpathlineto{\pgfqpoint{1.422748in}{1.498440in}}%
\pgfpathlineto{\pgfqpoint{1.422748in}{1.356689in}}%
\pgfpathlineto{\pgfqpoint{1.413995in}{1.356689in}}%
\pgfpathlineto{\pgfqpoint{1.413995in}{1.498440in}}%
\pgfpathclose%
\pgfusepath{fill}%
\end{pgfscope}%
\begin{pgfscope}%
\pgfpathrectangle{\pgfqpoint{0.804646in}{0.600000in}}{\pgfqpoint{2.573292in}{2.070576in}}%
\pgfusepath{clip}%
\pgfsetbuttcap%
\pgfsetmiterjoin%
\definecolor{currentfill}{rgb}{0.511253,0.510898,0.193296}%
\pgfsetfillcolor{currentfill}%
\pgfsetlinewidth{0.000000pt}%
\definecolor{currentstroke}{rgb}{0.000000,0.000000,0.000000}%
\pgfsetstrokecolor{currentstroke}%
\pgfsetstrokeopacity{0.000000}%
\pgfsetdash{}{0pt}%
\pgfpathmoveto{\pgfqpoint{1.424936in}{1.507902in}}%
\pgfpathlineto{\pgfqpoint{1.433690in}{1.507902in}}%
\pgfpathlineto{\pgfqpoint{1.433690in}{1.182162in}}%
\pgfpathlineto{\pgfqpoint{1.424936in}{1.182162in}}%
\pgfpathlineto{\pgfqpoint{1.424936in}{1.507902in}}%
\pgfpathclose%
\pgfusepath{fill}%
\end{pgfscope}%
\begin{pgfscope}%
\pgfpathrectangle{\pgfqpoint{0.804646in}{0.600000in}}{\pgfqpoint{2.573292in}{2.070576in}}%
\pgfusepath{clip}%
\pgfsetbuttcap%
\pgfsetmiterjoin%
\definecolor{currentfill}{rgb}{0.511253,0.510898,0.193296}%
\pgfsetfillcolor{currentfill}%
\pgfsetlinewidth{0.000000pt}%
\definecolor{currentstroke}{rgb}{0.000000,0.000000,0.000000}%
\pgfsetstrokecolor{currentstroke}%
\pgfsetstrokeopacity{0.000000}%
\pgfsetdash{}{0pt}%
\pgfpathmoveto{\pgfqpoint{1.435878in}{1.554041in}}%
\pgfpathlineto{\pgfqpoint{1.444632in}{1.554041in}}%
\pgfpathlineto{\pgfqpoint{1.444632in}{1.222121in}}%
\pgfpathlineto{\pgfqpoint{1.435878in}{1.222121in}}%
\pgfpathlineto{\pgfqpoint{1.435878in}{1.554041in}}%
\pgfpathclose%
\pgfusepath{fill}%
\end{pgfscope}%
\begin{pgfscope}%
\pgfpathrectangle{\pgfqpoint{0.804646in}{0.600000in}}{\pgfqpoint{2.573292in}{2.070576in}}%
\pgfusepath{clip}%
\pgfsetbuttcap%
\pgfsetmiterjoin%
\definecolor{currentfill}{rgb}{0.511253,0.510898,0.193296}%
\pgfsetfillcolor{currentfill}%
\pgfsetlinewidth{0.000000pt}%
\definecolor{currentstroke}{rgb}{0.000000,0.000000,0.000000}%
\pgfsetstrokecolor{currentstroke}%
\pgfsetstrokeopacity{0.000000}%
\pgfsetdash{}{0pt}%
\pgfpathmoveto{\pgfqpoint{1.446820in}{1.568310in}}%
\pgfpathlineto{\pgfqpoint{1.455573in}{1.568310in}}%
\pgfpathlineto{\pgfqpoint{1.455573in}{1.212604in}}%
\pgfpathlineto{\pgfqpoint{1.446820in}{1.212604in}}%
\pgfpathlineto{\pgfqpoint{1.446820in}{1.568310in}}%
\pgfpathclose%
\pgfusepath{fill}%
\end{pgfscope}%
\begin{pgfscope}%
\pgfpathrectangle{\pgfqpoint{0.804646in}{0.600000in}}{\pgfqpoint{2.573292in}{2.070576in}}%
\pgfusepath{clip}%
\pgfsetbuttcap%
\pgfsetmiterjoin%
\definecolor{currentfill}{rgb}{0.511253,0.510898,0.193296}%
\pgfsetfillcolor{currentfill}%
\pgfsetlinewidth{0.000000pt}%
\definecolor{currentstroke}{rgb}{0.000000,0.000000,0.000000}%
\pgfsetstrokecolor{currentstroke}%
\pgfsetstrokeopacity{0.000000}%
\pgfsetdash{}{0pt}%
\pgfpathmoveto{\pgfqpoint{1.457762in}{1.557697in}}%
\pgfpathlineto{\pgfqpoint{1.466515in}{1.557697in}}%
\pgfpathlineto{\pgfqpoint{1.466515in}{1.200871in}}%
\pgfpathlineto{\pgfqpoint{1.457762in}{1.200871in}}%
\pgfpathlineto{\pgfqpoint{1.457762in}{1.557697in}}%
\pgfpathclose%
\pgfusepath{fill}%
\end{pgfscope}%
\begin{pgfscope}%
\pgfpathrectangle{\pgfqpoint{0.804646in}{0.600000in}}{\pgfqpoint{2.573292in}{2.070576in}}%
\pgfusepath{clip}%
\pgfsetbuttcap%
\pgfsetmiterjoin%
\definecolor{currentfill}{rgb}{0.511253,0.510898,0.193296}%
\pgfsetfillcolor{currentfill}%
\pgfsetlinewidth{0.000000pt}%
\definecolor{currentstroke}{rgb}{0.000000,0.000000,0.000000}%
\pgfsetstrokecolor{currentstroke}%
\pgfsetstrokeopacity{0.000000}%
\pgfsetdash{}{0pt}%
\pgfpathmoveto{\pgfqpoint{1.468704in}{1.552003in}}%
\pgfpathlineto{\pgfqpoint{1.477457in}{1.552003in}}%
\pgfpathlineto{\pgfqpoint{1.477457in}{1.122321in}}%
\pgfpathlineto{\pgfqpoint{1.468704in}{1.122321in}}%
\pgfpathlineto{\pgfqpoint{1.468704in}{1.552003in}}%
\pgfpathclose%
\pgfusepath{fill}%
\end{pgfscope}%
\begin{pgfscope}%
\pgfpathrectangle{\pgfqpoint{0.804646in}{0.600000in}}{\pgfqpoint{2.573292in}{2.070576in}}%
\pgfusepath{clip}%
\pgfsetbuttcap%
\pgfsetmiterjoin%
\definecolor{currentfill}{rgb}{0.511253,0.510898,0.193296}%
\pgfsetfillcolor{currentfill}%
\pgfsetlinewidth{0.000000pt}%
\definecolor{currentstroke}{rgb}{0.000000,0.000000,0.000000}%
\pgfsetstrokecolor{currentstroke}%
\pgfsetstrokeopacity{0.000000}%
\pgfsetdash{}{0pt}%
\pgfpathmoveto{\pgfqpoint{1.479645in}{1.546180in}}%
\pgfpathlineto{\pgfqpoint{1.488399in}{1.546180in}}%
\pgfpathlineto{\pgfqpoint{1.488399in}{1.299674in}}%
\pgfpathlineto{\pgfqpoint{1.479645in}{1.299674in}}%
\pgfpathlineto{\pgfqpoint{1.479645in}{1.546180in}}%
\pgfpathclose%
\pgfusepath{fill}%
\end{pgfscope}%
\begin{pgfscope}%
\pgfpathrectangle{\pgfqpoint{0.804646in}{0.600000in}}{\pgfqpoint{2.573292in}{2.070576in}}%
\pgfusepath{clip}%
\pgfsetbuttcap%
\pgfsetmiterjoin%
\definecolor{currentfill}{rgb}{0.511253,0.510898,0.193296}%
\pgfsetfillcolor{currentfill}%
\pgfsetlinewidth{0.000000pt}%
\definecolor{currentstroke}{rgb}{0.000000,0.000000,0.000000}%
\pgfsetstrokecolor{currentstroke}%
\pgfsetstrokeopacity{0.000000}%
\pgfsetdash{}{0pt}%
\pgfpathmoveto{\pgfqpoint{1.490587in}{1.549959in}}%
\pgfpathlineto{\pgfqpoint{1.499341in}{1.549959in}}%
\pgfpathlineto{\pgfqpoint{1.499341in}{1.232303in}}%
\pgfpathlineto{\pgfqpoint{1.490587in}{1.232303in}}%
\pgfpathlineto{\pgfqpoint{1.490587in}{1.549959in}}%
\pgfpathclose%
\pgfusepath{fill}%
\end{pgfscope}%
\begin{pgfscope}%
\pgfpathrectangle{\pgfqpoint{0.804646in}{0.600000in}}{\pgfqpoint{2.573292in}{2.070576in}}%
\pgfusepath{clip}%
\pgfsetbuttcap%
\pgfsetmiterjoin%
\definecolor{currentfill}{rgb}{0.511253,0.510898,0.193296}%
\pgfsetfillcolor{currentfill}%
\pgfsetlinewidth{0.000000pt}%
\definecolor{currentstroke}{rgb}{0.000000,0.000000,0.000000}%
\pgfsetstrokecolor{currentstroke}%
\pgfsetstrokeopacity{0.000000}%
\pgfsetdash{}{0pt}%
\pgfpathmoveto{\pgfqpoint{1.501529in}{1.527904in}}%
\pgfpathlineto{\pgfqpoint{1.510282in}{1.527904in}}%
\pgfpathlineto{\pgfqpoint{1.510282in}{1.225206in}}%
\pgfpathlineto{\pgfqpoint{1.501529in}{1.225206in}}%
\pgfpathlineto{\pgfqpoint{1.501529in}{1.527904in}}%
\pgfpathclose%
\pgfusepath{fill}%
\end{pgfscope}%
\begin{pgfscope}%
\pgfpathrectangle{\pgfqpoint{0.804646in}{0.600000in}}{\pgfqpoint{2.573292in}{2.070576in}}%
\pgfusepath{clip}%
\pgfsetbuttcap%
\pgfsetmiterjoin%
\definecolor{currentfill}{rgb}{0.511253,0.510898,0.193296}%
\pgfsetfillcolor{currentfill}%
\pgfsetlinewidth{0.000000pt}%
\definecolor{currentstroke}{rgb}{0.000000,0.000000,0.000000}%
\pgfsetstrokecolor{currentstroke}%
\pgfsetstrokeopacity{0.000000}%
\pgfsetdash{}{0pt}%
\pgfpathmoveto{\pgfqpoint{1.512471in}{1.531418in}}%
\pgfpathlineto{\pgfqpoint{1.521224in}{1.531418in}}%
\pgfpathlineto{\pgfqpoint{1.521224in}{1.276524in}}%
\pgfpathlineto{\pgfqpoint{1.512471in}{1.276524in}}%
\pgfpathlineto{\pgfqpoint{1.512471in}{1.531418in}}%
\pgfpathclose%
\pgfusepath{fill}%
\end{pgfscope}%
\begin{pgfscope}%
\pgfpathrectangle{\pgfqpoint{0.804646in}{0.600000in}}{\pgfqpoint{2.573292in}{2.070576in}}%
\pgfusepath{clip}%
\pgfsetbuttcap%
\pgfsetmiterjoin%
\definecolor{currentfill}{rgb}{0.511253,0.510898,0.193296}%
\pgfsetfillcolor{currentfill}%
\pgfsetlinewidth{0.000000pt}%
\definecolor{currentstroke}{rgb}{0.000000,0.000000,0.000000}%
\pgfsetstrokecolor{currentstroke}%
\pgfsetstrokeopacity{0.000000}%
\pgfsetdash{}{0pt}%
\pgfpathmoveto{\pgfqpoint{1.523413in}{1.512675in}}%
\pgfpathlineto{\pgfqpoint{1.532166in}{1.512675in}}%
\pgfpathlineto{\pgfqpoint{1.532166in}{1.241731in}}%
\pgfpathlineto{\pgfqpoint{1.523413in}{1.241731in}}%
\pgfpathlineto{\pgfqpoint{1.523413in}{1.512675in}}%
\pgfpathclose%
\pgfusepath{fill}%
\end{pgfscope}%
\begin{pgfscope}%
\pgfpathrectangle{\pgfqpoint{0.804646in}{0.600000in}}{\pgfqpoint{2.573292in}{2.070576in}}%
\pgfusepath{clip}%
\pgfsetbuttcap%
\pgfsetmiterjoin%
\definecolor{currentfill}{rgb}{0.511253,0.510898,0.193296}%
\pgfsetfillcolor{currentfill}%
\pgfsetlinewidth{0.000000pt}%
\definecolor{currentstroke}{rgb}{0.000000,0.000000,0.000000}%
\pgfsetstrokecolor{currentstroke}%
\pgfsetstrokeopacity{0.000000}%
\pgfsetdash{}{0pt}%
\pgfpathmoveto{\pgfqpoint{1.534354in}{1.561304in}}%
\pgfpathlineto{\pgfqpoint{1.543108in}{1.561304in}}%
\pgfpathlineto{\pgfqpoint{1.543108in}{1.331466in}}%
\pgfpathlineto{\pgfqpoint{1.534354in}{1.331466in}}%
\pgfpathlineto{\pgfqpoint{1.534354in}{1.561304in}}%
\pgfpathclose%
\pgfusepath{fill}%
\end{pgfscope}%
\begin{pgfscope}%
\pgfpathrectangle{\pgfqpoint{0.804646in}{0.600000in}}{\pgfqpoint{2.573292in}{2.070576in}}%
\pgfusepath{clip}%
\pgfsetbuttcap%
\pgfsetmiterjoin%
\definecolor{currentfill}{rgb}{0.511253,0.510898,0.193296}%
\pgfsetfillcolor{currentfill}%
\pgfsetlinewidth{0.000000pt}%
\definecolor{currentstroke}{rgb}{0.000000,0.000000,0.000000}%
\pgfsetstrokecolor{currentstroke}%
\pgfsetstrokeopacity{0.000000}%
\pgfsetdash{}{0pt}%
\pgfpathmoveto{\pgfqpoint{1.545296in}{1.558022in}}%
\pgfpathlineto{\pgfqpoint{1.554050in}{1.558022in}}%
\pgfpathlineto{\pgfqpoint{1.554050in}{1.501791in}}%
\pgfpathlineto{\pgfqpoint{1.545296in}{1.501791in}}%
\pgfpathlineto{\pgfqpoint{1.545296in}{1.558022in}}%
\pgfpathclose%
\pgfusepath{fill}%
\end{pgfscope}%
\begin{pgfscope}%
\pgfpathrectangle{\pgfqpoint{0.804646in}{0.600000in}}{\pgfqpoint{2.573292in}{2.070576in}}%
\pgfusepath{clip}%
\pgfsetbuttcap%
\pgfsetmiterjoin%
\definecolor{currentfill}{rgb}{0.511253,0.510898,0.193296}%
\pgfsetfillcolor{currentfill}%
\pgfsetlinewidth{0.000000pt}%
\definecolor{currentstroke}{rgb}{0.000000,0.000000,0.000000}%
\pgfsetstrokecolor{currentstroke}%
\pgfsetstrokeopacity{0.000000}%
\pgfsetdash{}{0pt}%
\pgfpathmoveto{\pgfqpoint{1.556238in}{1.510643in}}%
\pgfpathlineto{\pgfqpoint{1.564991in}{1.510643in}}%
\pgfpathlineto{\pgfqpoint{1.564991in}{1.430897in}}%
\pgfpathlineto{\pgfqpoint{1.556238in}{1.430897in}}%
\pgfpathlineto{\pgfqpoint{1.556238in}{1.510643in}}%
\pgfpathclose%
\pgfusepath{fill}%
\end{pgfscope}%
\begin{pgfscope}%
\pgfpathrectangle{\pgfqpoint{0.804646in}{0.600000in}}{\pgfqpoint{2.573292in}{2.070576in}}%
\pgfusepath{clip}%
\pgfsetbuttcap%
\pgfsetmiterjoin%
\definecolor{currentfill}{rgb}{0.511253,0.510898,0.193296}%
\pgfsetfillcolor{currentfill}%
\pgfsetlinewidth{0.000000pt}%
\definecolor{currentstroke}{rgb}{0.000000,0.000000,0.000000}%
\pgfsetstrokecolor{currentstroke}%
\pgfsetstrokeopacity{0.000000}%
\pgfsetdash{}{0pt}%
\pgfpathmoveto{\pgfqpoint{1.567180in}{1.550733in}}%
\pgfpathlineto{\pgfqpoint{1.575933in}{1.550733in}}%
\pgfpathlineto{\pgfqpoint{1.575933in}{1.245710in}}%
\pgfpathlineto{\pgfqpoint{1.567180in}{1.245710in}}%
\pgfpathlineto{\pgfqpoint{1.567180in}{1.550733in}}%
\pgfpathclose%
\pgfusepath{fill}%
\end{pgfscope}%
\begin{pgfscope}%
\pgfpathrectangle{\pgfqpoint{0.804646in}{0.600000in}}{\pgfqpoint{2.573292in}{2.070576in}}%
\pgfusepath{clip}%
\pgfsetbuttcap%
\pgfsetmiterjoin%
\definecolor{currentfill}{rgb}{0.511253,0.510898,0.193296}%
\pgfsetfillcolor{currentfill}%
\pgfsetlinewidth{0.000000pt}%
\definecolor{currentstroke}{rgb}{0.000000,0.000000,0.000000}%
\pgfsetstrokecolor{currentstroke}%
\pgfsetstrokeopacity{0.000000}%
\pgfsetdash{}{0pt}%
\pgfpathmoveto{\pgfqpoint{1.578122in}{1.536840in}}%
\pgfpathlineto{\pgfqpoint{1.586875in}{1.536840in}}%
\pgfpathlineto{\pgfqpoint{1.586875in}{1.343473in}}%
\pgfpathlineto{\pgfqpoint{1.578122in}{1.343473in}}%
\pgfpathlineto{\pgfqpoint{1.578122in}{1.536840in}}%
\pgfpathclose%
\pgfusepath{fill}%
\end{pgfscope}%
\begin{pgfscope}%
\pgfpathrectangle{\pgfqpoint{0.804646in}{0.600000in}}{\pgfqpoint{2.573292in}{2.070576in}}%
\pgfusepath{clip}%
\pgfsetbuttcap%
\pgfsetmiterjoin%
\definecolor{currentfill}{rgb}{0.511253,0.510898,0.193296}%
\pgfsetfillcolor{currentfill}%
\pgfsetlinewidth{0.000000pt}%
\definecolor{currentstroke}{rgb}{0.000000,0.000000,0.000000}%
\pgfsetstrokecolor{currentstroke}%
\pgfsetstrokeopacity{0.000000}%
\pgfsetdash{}{0pt}%
\pgfpathmoveto{\pgfqpoint{1.589063in}{1.572573in}}%
\pgfpathlineto{\pgfqpoint{1.597817in}{1.572573in}}%
\pgfpathlineto{\pgfqpoint{1.597817in}{1.304102in}}%
\pgfpathlineto{\pgfqpoint{1.589063in}{1.304102in}}%
\pgfpathlineto{\pgfqpoint{1.589063in}{1.572573in}}%
\pgfpathclose%
\pgfusepath{fill}%
\end{pgfscope}%
\begin{pgfscope}%
\pgfpathrectangle{\pgfqpoint{0.804646in}{0.600000in}}{\pgfqpoint{2.573292in}{2.070576in}}%
\pgfusepath{clip}%
\pgfsetbuttcap%
\pgfsetmiterjoin%
\definecolor{currentfill}{rgb}{0.511253,0.510898,0.193296}%
\pgfsetfillcolor{currentfill}%
\pgfsetlinewidth{0.000000pt}%
\definecolor{currentstroke}{rgb}{0.000000,0.000000,0.000000}%
\pgfsetstrokecolor{currentstroke}%
\pgfsetstrokeopacity{0.000000}%
\pgfsetdash{}{0pt}%
\pgfpathmoveto{\pgfqpoint{1.600005in}{1.559619in}}%
\pgfpathlineto{\pgfqpoint{1.608759in}{1.559619in}}%
\pgfpathlineto{\pgfqpoint{1.608759in}{1.339011in}}%
\pgfpathlineto{\pgfqpoint{1.600005in}{1.339011in}}%
\pgfpathlineto{\pgfqpoint{1.600005in}{1.559619in}}%
\pgfpathclose%
\pgfusepath{fill}%
\end{pgfscope}%
\begin{pgfscope}%
\pgfpathrectangle{\pgfqpoint{0.804646in}{0.600000in}}{\pgfqpoint{2.573292in}{2.070576in}}%
\pgfusepath{clip}%
\pgfsetbuttcap%
\pgfsetmiterjoin%
\definecolor{currentfill}{rgb}{0.511253,0.510898,0.193296}%
\pgfsetfillcolor{currentfill}%
\pgfsetlinewidth{0.000000pt}%
\definecolor{currentstroke}{rgb}{0.000000,0.000000,0.000000}%
\pgfsetstrokecolor{currentstroke}%
\pgfsetstrokeopacity{0.000000}%
\pgfsetdash{}{0pt}%
\pgfpathmoveto{\pgfqpoint{1.610947in}{1.528874in}}%
\pgfpathlineto{\pgfqpoint{1.619700in}{1.528874in}}%
\pgfpathlineto{\pgfqpoint{1.619700in}{1.399904in}}%
\pgfpathlineto{\pgfqpoint{1.610947in}{1.399904in}}%
\pgfpathlineto{\pgfqpoint{1.610947in}{1.528874in}}%
\pgfpathclose%
\pgfusepath{fill}%
\end{pgfscope}%
\begin{pgfscope}%
\pgfpathrectangle{\pgfqpoint{0.804646in}{0.600000in}}{\pgfqpoint{2.573292in}{2.070576in}}%
\pgfusepath{clip}%
\pgfsetbuttcap%
\pgfsetmiterjoin%
\definecolor{currentfill}{rgb}{0.511253,0.510898,0.193296}%
\pgfsetfillcolor{currentfill}%
\pgfsetlinewidth{0.000000pt}%
\definecolor{currentstroke}{rgb}{0.000000,0.000000,0.000000}%
\pgfsetstrokecolor{currentstroke}%
\pgfsetstrokeopacity{0.000000}%
\pgfsetdash{}{0pt}%
\pgfpathmoveto{\pgfqpoint{1.621889in}{1.493240in}}%
\pgfpathlineto{\pgfqpoint{1.630642in}{1.493240in}}%
\pgfpathlineto{\pgfqpoint{1.630642in}{1.406214in}}%
\pgfpathlineto{\pgfqpoint{1.621889in}{1.406214in}}%
\pgfpathlineto{\pgfqpoint{1.621889in}{1.493240in}}%
\pgfpathclose%
\pgfusepath{fill}%
\end{pgfscope}%
\begin{pgfscope}%
\pgfpathrectangle{\pgfqpoint{0.804646in}{0.600000in}}{\pgfqpoint{2.573292in}{2.070576in}}%
\pgfusepath{clip}%
\pgfsetbuttcap%
\pgfsetmiterjoin%
\definecolor{currentfill}{rgb}{0.511253,0.510898,0.193296}%
\pgfsetfillcolor{currentfill}%
\pgfsetlinewidth{0.000000pt}%
\definecolor{currentstroke}{rgb}{0.000000,0.000000,0.000000}%
\pgfsetstrokecolor{currentstroke}%
\pgfsetstrokeopacity{0.000000}%
\pgfsetdash{}{0pt}%
\pgfpathmoveto{\pgfqpoint{1.632831in}{1.657941in}}%
\pgfpathlineto{\pgfqpoint{1.641584in}{1.657941in}}%
\pgfpathlineto{\pgfqpoint{1.641584in}{1.683467in}}%
\pgfpathlineto{\pgfqpoint{1.632831in}{1.683467in}}%
\pgfpathlineto{\pgfqpoint{1.632831in}{1.657941in}}%
\pgfpathclose%
\pgfusepath{fill}%
\end{pgfscope}%
\begin{pgfscope}%
\pgfpathrectangle{\pgfqpoint{0.804646in}{0.600000in}}{\pgfqpoint{2.573292in}{2.070576in}}%
\pgfusepath{clip}%
\pgfsetbuttcap%
\pgfsetmiterjoin%
\definecolor{currentfill}{rgb}{0.511253,0.510898,0.193296}%
\pgfsetfillcolor{currentfill}%
\pgfsetlinewidth{0.000000pt}%
\definecolor{currentstroke}{rgb}{0.000000,0.000000,0.000000}%
\pgfsetstrokecolor{currentstroke}%
\pgfsetstrokeopacity{0.000000}%
\pgfsetdash{}{0pt}%
\pgfpathmoveto{\pgfqpoint{1.643772in}{1.459636in}}%
\pgfpathlineto{\pgfqpoint{1.652526in}{1.459636in}}%
\pgfpathlineto{\pgfqpoint{1.652526in}{1.451867in}}%
\pgfpathlineto{\pgfqpoint{1.643772in}{1.451867in}}%
\pgfpathlineto{\pgfqpoint{1.643772in}{1.459636in}}%
\pgfpathclose%
\pgfusepath{fill}%
\end{pgfscope}%
\begin{pgfscope}%
\pgfpathrectangle{\pgfqpoint{0.804646in}{0.600000in}}{\pgfqpoint{2.573292in}{2.070576in}}%
\pgfusepath{clip}%
\pgfsetbuttcap%
\pgfsetmiterjoin%
\definecolor{currentfill}{rgb}{0.511253,0.510898,0.193296}%
\pgfsetfillcolor{currentfill}%
\pgfsetlinewidth{0.000000pt}%
\definecolor{currentstroke}{rgb}{0.000000,0.000000,0.000000}%
\pgfsetstrokecolor{currentstroke}%
\pgfsetstrokeopacity{0.000000}%
\pgfsetdash{}{0pt}%
\pgfpathmoveto{\pgfqpoint{1.654714in}{1.671449in}}%
\pgfpathlineto{\pgfqpoint{1.663468in}{1.671449in}}%
\pgfpathlineto{\pgfqpoint{1.663468in}{1.738550in}}%
\pgfpathlineto{\pgfqpoint{1.654714in}{1.738550in}}%
\pgfpathlineto{\pgfqpoint{1.654714in}{1.671449in}}%
\pgfpathclose%
\pgfusepath{fill}%
\end{pgfscope}%
\begin{pgfscope}%
\pgfpathrectangle{\pgfqpoint{0.804646in}{0.600000in}}{\pgfqpoint{2.573292in}{2.070576in}}%
\pgfusepath{clip}%
\pgfsetbuttcap%
\pgfsetmiterjoin%
\definecolor{currentfill}{rgb}{0.511253,0.510898,0.193296}%
\pgfsetfillcolor{currentfill}%
\pgfsetlinewidth{0.000000pt}%
\definecolor{currentstroke}{rgb}{0.000000,0.000000,0.000000}%
\pgfsetstrokecolor{currentstroke}%
\pgfsetstrokeopacity{0.000000}%
\pgfsetdash{}{0pt}%
\pgfpathmoveto{\pgfqpoint{1.665656in}{1.664329in}}%
\pgfpathlineto{\pgfqpoint{1.674409in}{1.664329in}}%
\pgfpathlineto{\pgfqpoint{1.674409in}{1.745671in}}%
\pgfpathlineto{\pgfqpoint{1.665656in}{1.745671in}}%
\pgfpathlineto{\pgfqpoint{1.665656in}{1.664329in}}%
\pgfpathclose%
\pgfusepath{fill}%
\end{pgfscope}%
\begin{pgfscope}%
\pgfpathrectangle{\pgfqpoint{0.804646in}{0.600000in}}{\pgfqpoint{2.573292in}{2.070576in}}%
\pgfusepath{clip}%
\pgfsetbuttcap%
\pgfsetmiterjoin%
\definecolor{currentfill}{rgb}{0.511253,0.510898,0.193296}%
\pgfsetfillcolor{currentfill}%
\pgfsetlinewidth{0.000000pt}%
\definecolor{currentstroke}{rgb}{0.000000,0.000000,0.000000}%
\pgfsetstrokecolor{currentstroke}%
\pgfsetstrokeopacity{0.000000}%
\pgfsetdash{}{0pt}%
\pgfpathmoveto{\pgfqpoint{1.676598in}{1.549999in}}%
\pgfpathlineto{\pgfqpoint{1.685351in}{1.549999in}}%
\pgfpathlineto{\pgfqpoint{1.685351in}{1.519958in}}%
\pgfpathlineto{\pgfqpoint{1.676598in}{1.519958in}}%
\pgfpathlineto{\pgfqpoint{1.676598in}{1.549999in}}%
\pgfpathclose%
\pgfusepath{fill}%
\end{pgfscope}%
\begin{pgfscope}%
\pgfpathrectangle{\pgfqpoint{0.804646in}{0.600000in}}{\pgfqpoint{2.573292in}{2.070576in}}%
\pgfusepath{clip}%
\pgfsetbuttcap%
\pgfsetmiterjoin%
\definecolor{currentfill}{rgb}{0.511253,0.510898,0.193296}%
\pgfsetfillcolor{currentfill}%
\pgfsetlinewidth{0.000000pt}%
\definecolor{currentstroke}{rgb}{0.000000,0.000000,0.000000}%
\pgfsetstrokecolor{currentstroke}%
\pgfsetstrokeopacity{0.000000}%
\pgfsetdash{}{0pt}%
\pgfpathmoveto{\pgfqpoint{1.687540in}{1.693557in}}%
\pgfpathlineto{\pgfqpoint{1.696293in}{1.693557in}}%
\pgfpathlineto{\pgfqpoint{1.696293in}{1.741886in}}%
\pgfpathlineto{\pgfqpoint{1.687540in}{1.741886in}}%
\pgfpathlineto{\pgfqpoint{1.687540in}{1.693557in}}%
\pgfpathclose%
\pgfusepath{fill}%
\end{pgfscope}%
\begin{pgfscope}%
\pgfpathrectangle{\pgfqpoint{0.804646in}{0.600000in}}{\pgfqpoint{2.573292in}{2.070576in}}%
\pgfusepath{clip}%
\pgfsetbuttcap%
\pgfsetmiterjoin%
\definecolor{currentfill}{rgb}{0.511253,0.510898,0.193296}%
\pgfsetfillcolor{currentfill}%
\pgfsetlinewidth{0.000000pt}%
\definecolor{currentstroke}{rgb}{0.000000,0.000000,0.000000}%
\pgfsetstrokecolor{currentstroke}%
\pgfsetstrokeopacity{0.000000}%
\pgfsetdash{}{0pt}%
\pgfpathmoveto{\pgfqpoint{1.698481in}{1.712399in}}%
\pgfpathlineto{\pgfqpoint{1.707235in}{1.712399in}}%
\pgfpathlineto{\pgfqpoint{1.707235in}{1.728962in}}%
\pgfpathlineto{\pgfqpoint{1.698481in}{1.728962in}}%
\pgfpathlineto{\pgfqpoint{1.698481in}{1.712399in}}%
\pgfpathclose%
\pgfusepath{fill}%
\end{pgfscope}%
\begin{pgfscope}%
\pgfpathrectangle{\pgfqpoint{0.804646in}{0.600000in}}{\pgfqpoint{2.573292in}{2.070576in}}%
\pgfusepath{clip}%
\pgfsetbuttcap%
\pgfsetmiterjoin%
\definecolor{currentfill}{rgb}{0.511253,0.510898,0.193296}%
\pgfsetfillcolor{currentfill}%
\pgfsetlinewidth{0.000000pt}%
\definecolor{currentstroke}{rgb}{0.000000,0.000000,0.000000}%
\pgfsetstrokecolor{currentstroke}%
\pgfsetstrokeopacity{0.000000}%
\pgfsetdash{}{0pt}%
\pgfpathmoveto{\pgfqpoint{1.709423in}{1.713715in}}%
\pgfpathlineto{\pgfqpoint{1.718177in}{1.713715in}}%
\pgfpathlineto{\pgfqpoint{1.718177in}{1.762912in}}%
\pgfpathlineto{\pgfqpoint{1.709423in}{1.762912in}}%
\pgfpathlineto{\pgfqpoint{1.709423in}{1.713715in}}%
\pgfpathclose%
\pgfusepath{fill}%
\end{pgfscope}%
\begin{pgfscope}%
\pgfpathrectangle{\pgfqpoint{0.804646in}{0.600000in}}{\pgfqpoint{2.573292in}{2.070576in}}%
\pgfusepath{clip}%
\pgfsetbuttcap%
\pgfsetmiterjoin%
\definecolor{currentfill}{rgb}{0.511253,0.510898,0.193296}%
\pgfsetfillcolor{currentfill}%
\pgfsetlinewidth{0.000000pt}%
\definecolor{currentstroke}{rgb}{0.000000,0.000000,0.000000}%
\pgfsetstrokecolor{currentstroke}%
\pgfsetstrokeopacity{0.000000}%
\pgfsetdash{}{0pt}%
\pgfpathmoveto{\pgfqpoint{1.720365in}{1.730460in}}%
\pgfpathlineto{\pgfqpoint{1.729118in}{1.730460in}}%
\pgfpathlineto{\pgfqpoint{1.729118in}{1.827928in}}%
\pgfpathlineto{\pgfqpoint{1.720365in}{1.827928in}}%
\pgfpathlineto{\pgfqpoint{1.720365in}{1.730460in}}%
\pgfpathclose%
\pgfusepath{fill}%
\end{pgfscope}%
\begin{pgfscope}%
\pgfpathrectangle{\pgfqpoint{0.804646in}{0.600000in}}{\pgfqpoint{2.573292in}{2.070576in}}%
\pgfusepath{clip}%
\pgfsetbuttcap%
\pgfsetmiterjoin%
\definecolor{currentfill}{rgb}{0.511253,0.510898,0.193296}%
\pgfsetfillcolor{currentfill}%
\pgfsetlinewidth{0.000000pt}%
\definecolor{currentstroke}{rgb}{0.000000,0.000000,0.000000}%
\pgfsetstrokecolor{currentstroke}%
\pgfsetstrokeopacity{0.000000}%
\pgfsetdash{}{0pt}%
\pgfpathmoveto{\pgfqpoint{1.731307in}{1.752099in}}%
\pgfpathlineto{\pgfqpoint{1.740060in}{1.752099in}}%
\pgfpathlineto{\pgfqpoint{1.740060in}{1.923672in}}%
\pgfpathlineto{\pgfqpoint{1.731307in}{1.923672in}}%
\pgfpathlineto{\pgfqpoint{1.731307in}{1.752099in}}%
\pgfpathclose%
\pgfusepath{fill}%
\end{pgfscope}%
\begin{pgfscope}%
\pgfpathrectangle{\pgfqpoint{0.804646in}{0.600000in}}{\pgfqpoint{2.573292in}{2.070576in}}%
\pgfusepath{clip}%
\pgfsetbuttcap%
\pgfsetmiterjoin%
\definecolor{currentfill}{rgb}{0.511253,0.510898,0.193296}%
\pgfsetfillcolor{currentfill}%
\pgfsetlinewidth{0.000000pt}%
\definecolor{currentstroke}{rgb}{0.000000,0.000000,0.000000}%
\pgfsetstrokecolor{currentstroke}%
\pgfsetstrokeopacity{0.000000}%
\pgfsetdash{}{0pt}%
\pgfpathmoveto{\pgfqpoint{1.742249in}{1.766114in}}%
\pgfpathlineto{\pgfqpoint{1.751002in}{1.766114in}}%
\pgfpathlineto{\pgfqpoint{1.751002in}{1.909777in}}%
\pgfpathlineto{\pgfqpoint{1.742249in}{1.909777in}}%
\pgfpathlineto{\pgfqpoint{1.742249in}{1.766114in}}%
\pgfpathclose%
\pgfusepath{fill}%
\end{pgfscope}%
\begin{pgfscope}%
\pgfpathrectangle{\pgfqpoint{0.804646in}{0.600000in}}{\pgfqpoint{2.573292in}{2.070576in}}%
\pgfusepath{clip}%
\pgfsetbuttcap%
\pgfsetmiterjoin%
\definecolor{currentfill}{rgb}{0.511253,0.510898,0.193296}%
\pgfsetfillcolor{currentfill}%
\pgfsetlinewidth{0.000000pt}%
\definecolor{currentstroke}{rgb}{0.000000,0.000000,0.000000}%
\pgfsetstrokecolor{currentstroke}%
\pgfsetstrokeopacity{0.000000}%
\pgfsetdash{}{0pt}%
\pgfpathmoveto{\pgfqpoint{1.753190in}{1.772835in}}%
\pgfpathlineto{\pgfqpoint{1.761944in}{1.772835in}}%
\pgfpathlineto{\pgfqpoint{1.761944in}{1.951021in}}%
\pgfpathlineto{\pgfqpoint{1.753190in}{1.951021in}}%
\pgfpathlineto{\pgfqpoint{1.753190in}{1.772835in}}%
\pgfpathclose%
\pgfusepath{fill}%
\end{pgfscope}%
\begin{pgfscope}%
\pgfpathrectangle{\pgfqpoint{0.804646in}{0.600000in}}{\pgfqpoint{2.573292in}{2.070576in}}%
\pgfusepath{clip}%
\pgfsetbuttcap%
\pgfsetmiterjoin%
\definecolor{currentfill}{rgb}{0.511253,0.510898,0.193296}%
\pgfsetfillcolor{currentfill}%
\pgfsetlinewidth{0.000000pt}%
\definecolor{currentstroke}{rgb}{0.000000,0.000000,0.000000}%
\pgfsetstrokecolor{currentstroke}%
\pgfsetstrokeopacity{0.000000}%
\pgfsetdash{}{0pt}%
\pgfpathmoveto{\pgfqpoint{1.764132in}{1.776808in}}%
\pgfpathlineto{\pgfqpoint{1.772886in}{1.776808in}}%
\pgfpathlineto{\pgfqpoint{1.772886in}{2.035118in}}%
\pgfpathlineto{\pgfqpoint{1.764132in}{2.035118in}}%
\pgfpathlineto{\pgfqpoint{1.764132in}{1.776808in}}%
\pgfpathclose%
\pgfusepath{fill}%
\end{pgfscope}%
\begin{pgfscope}%
\pgfpathrectangle{\pgfqpoint{0.804646in}{0.600000in}}{\pgfqpoint{2.573292in}{2.070576in}}%
\pgfusepath{clip}%
\pgfsetbuttcap%
\pgfsetmiterjoin%
\definecolor{currentfill}{rgb}{0.511253,0.510898,0.193296}%
\pgfsetfillcolor{currentfill}%
\pgfsetlinewidth{0.000000pt}%
\definecolor{currentstroke}{rgb}{0.000000,0.000000,0.000000}%
\pgfsetstrokecolor{currentstroke}%
\pgfsetstrokeopacity{0.000000}%
\pgfsetdash{}{0pt}%
\pgfpathmoveto{\pgfqpoint{1.775074in}{1.778232in}}%
\pgfpathlineto{\pgfqpoint{1.783827in}{1.778232in}}%
\pgfpathlineto{\pgfqpoint{1.783827in}{1.989881in}}%
\pgfpathlineto{\pgfqpoint{1.775074in}{1.989881in}}%
\pgfpathlineto{\pgfqpoint{1.775074in}{1.778232in}}%
\pgfpathclose%
\pgfusepath{fill}%
\end{pgfscope}%
\begin{pgfscope}%
\pgfpathrectangle{\pgfqpoint{0.804646in}{0.600000in}}{\pgfqpoint{2.573292in}{2.070576in}}%
\pgfusepath{clip}%
\pgfsetbuttcap%
\pgfsetmiterjoin%
\definecolor{currentfill}{rgb}{0.511253,0.510898,0.193296}%
\pgfsetfillcolor{currentfill}%
\pgfsetlinewidth{0.000000pt}%
\definecolor{currentstroke}{rgb}{0.000000,0.000000,0.000000}%
\pgfsetstrokecolor{currentstroke}%
\pgfsetstrokeopacity{0.000000}%
\pgfsetdash{}{0pt}%
\pgfpathmoveto{\pgfqpoint{1.786016in}{1.770641in}}%
\pgfpathlineto{\pgfqpoint{1.794769in}{1.770641in}}%
\pgfpathlineto{\pgfqpoint{1.794769in}{2.026284in}}%
\pgfpathlineto{\pgfqpoint{1.786016in}{2.026284in}}%
\pgfpathlineto{\pgfqpoint{1.786016in}{1.770641in}}%
\pgfpathclose%
\pgfusepath{fill}%
\end{pgfscope}%
\begin{pgfscope}%
\pgfpathrectangle{\pgfqpoint{0.804646in}{0.600000in}}{\pgfqpoint{2.573292in}{2.070576in}}%
\pgfusepath{clip}%
\pgfsetbuttcap%
\pgfsetmiterjoin%
\definecolor{currentfill}{rgb}{0.511253,0.510898,0.193296}%
\pgfsetfillcolor{currentfill}%
\pgfsetlinewidth{0.000000pt}%
\definecolor{currentstroke}{rgb}{0.000000,0.000000,0.000000}%
\pgfsetstrokecolor{currentstroke}%
\pgfsetstrokeopacity{0.000000}%
\pgfsetdash{}{0pt}%
\pgfpathmoveto{\pgfqpoint{1.796958in}{1.776183in}}%
\pgfpathlineto{\pgfqpoint{1.805711in}{1.776183in}}%
\pgfpathlineto{\pgfqpoint{1.805711in}{2.079656in}}%
\pgfpathlineto{\pgfqpoint{1.796958in}{2.079656in}}%
\pgfpathlineto{\pgfqpoint{1.796958in}{1.776183in}}%
\pgfpathclose%
\pgfusepath{fill}%
\end{pgfscope}%
\begin{pgfscope}%
\pgfpathrectangle{\pgfqpoint{0.804646in}{0.600000in}}{\pgfqpoint{2.573292in}{2.070576in}}%
\pgfusepath{clip}%
\pgfsetbuttcap%
\pgfsetmiterjoin%
\definecolor{currentfill}{rgb}{0.511253,0.510898,0.193296}%
\pgfsetfillcolor{currentfill}%
\pgfsetlinewidth{0.000000pt}%
\definecolor{currentstroke}{rgb}{0.000000,0.000000,0.000000}%
\pgfsetstrokecolor{currentstroke}%
\pgfsetstrokeopacity{0.000000}%
\pgfsetdash{}{0pt}%
\pgfpathmoveto{\pgfqpoint{1.807899in}{1.785204in}}%
\pgfpathlineto{\pgfqpoint{1.816653in}{1.785204in}}%
\pgfpathlineto{\pgfqpoint{1.816653in}{2.137965in}}%
\pgfpathlineto{\pgfqpoint{1.807899in}{2.137965in}}%
\pgfpathlineto{\pgfqpoint{1.807899in}{1.785204in}}%
\pgfpathclose%
\pgfusepath{fill}%
\end{pgfscope}%
\begin{pgfscope}%
\pgfpathrectangle{\pgfqpoint{0.804646in}{0.600000in}}{\pgfqpoint{2.573292in}{2.070576in}}%
\pgfusepath{clip}%
\pgfsetbuttcap%
\pgfsetmiterjoin%
\definecolor{currentfill}{rgb}{0.511253,0.510898,0.193296}%
\pgfsetfillcolor{currentfill}%
\pgfsetlinewidth{0.000000pt}%
\definecolor{currentstroke}{rgb}{0.000000,0.000000,0.000000}%
\pgfsetstrokecolor{currentstroke}%
\pgfsetstrokeopacity{0.000000}%
\pgfsetdash{}{0pt}%
\pgfpathmoveto{\pgfqpoint{1.818841in}{1.791260in}}%
\pgfpathlineto{\pgfqpoint{1.827595in}{1.791260in}}%
\pgfpathlineto{\pgfqpoint{1.827595in}{2.080723in}}%
\pgfpathlineto{\pgfqpoint{1.818841in}{2.080723in}}%
\pgfpathlineto{\pgfqpoint{1.818841in}{1.791260in}}%
\pgfpathclose%
\pgfusepath{fill}%
\end{pgfscope}%
\begin{pgfscope}%
\pgfpathrectangle{\pgfqpoint{0.804646in}{0.600000in}}{\pgfqpoint{2.573292in}{2.070576in}}%
\pgfusepath{clip}%
\pgfsetbuttcap%
\pgfsetmiterjoin%
\definecolor{currentfill}{rgb}{0.511253,0.510898,0.193296}%
\pgfsetfillcolor{currentfill}%
\pgfsetlinewidth{0.000000pt}%
\definecolor{currentstroke}{rgb}{0.000000,0.000000,0.000000}%
\pgfsetstrokecolor{currentstroke}%
\pgfsetstrokeopacity{0.000000}%
\pgfsetdash{}{0pt}%
\pgfpathmoveto{\pgfqpoint{1.829783in}{1.808390in}}%
\pgfpathlineto{\pgfqpoint{1.838536in}{1.808390in}}%
\pgfpathlineto{\pgfqpoint{1.838536in}{2.171561in}}%
\pgfpathlineto{\pgfqpoint{1.829783in}{2.171561in}}%
\pgfpathlineto{\pgfqpoint{1.829783in}{1.808390in}}%
\pgfpathclose%
\pgfusepath{fill}%
\end{pgfscope}%
\begin{pgfscope}%
\pgfpathrectangle{\pgfqpoint{0.804646in}{0.600000in}}{\pgfqpoint{2.573292in}{2.070576in}}%
\pgfusepath{clip}%
\pgfsetbuttcap%
\pgfsetmiterjoin%
\definecolor{currentfill}{rgb}{0.511253,0.510898,0.193296}%
\pgfsetfillcolor{currentfill}%
\pgfsetlinewidth{0.000000pt}%
\definecolor{currentstroke}{rgb}{0.000000,0.000000,0.000000}%
\pgfsetstrokecolor{currentstroke}%
\pgfsetstrokeopacity{0.000000}%
\pgfsetdash{}{0pt}%
\pgfpathmoveto{\pgfqpoint{1.840725in}{1.816280in}}%
\pgfpathlineto{\pgfqpoint{1.849478in}{1.816280in}}%
\pgfpathlineto{\pgfqpoint{1.849478in}{2.224031in}}%
\pgfpathlineto{\pgfqpoint{1.840725in}{2.224031in}}%
\pgfpathlineto{\pgfqpoint{1.840725in}{1.816280in}}%
\pgfpathclose%
\pgfusepath{fill}%
\end{pgfscope}%
\begin{pgfscope}%
\pgfpathrectangle{\pgfqpoint{0.804646in}{0.600000in}}{\pgfqpoint{2.573292in}{2.070576in}}%
\pgfusepath{clip}%
\pgfsetbuttcap%
\pgfsetmiterjoin%
\definecolor{currentfill}{rgb}{0.511253,0.510898,0.193296}%
\pgfsetfillcolor{currentfill}%
\pgfsetlinewidth{0.000000pt}%
\definecolor{currentstroke}{rgb}{0.000000,0.000000,0.000000}%
\pgfsetstrokecolor{currentstroke}%
\pgfsetstrokeopacity{0.000000}%
\pgfsetdash{}{0pt}%
\pgfpathmoveto{\pgfqpoint{1.851667in}{1.814448in}}%
\pgfpathlineto{\pgfqpoint{1.860420in}{1.814448in}}%
\pgfpathlineto{\pgfqpoint{1.860420in}{2.192217in}}%
\pgfpathlineto{\pgfqpoint{1.851667in}{2.192217in}}%
\pgfpathlineto{\pgfqpoint{1.851667in}{1.814448in}}%
\pgfpathclose%
\pgfusepath{fill}%
\end{pgfscope}%
\begin{pgfscope}%
\pgfpathrectangle{\pgfqpoint{0.804646in}{0.600000in}}{\pgfqpoint{2.573292in}{2.070576in}}%
\pgfusepath{clip}%
\pgfsetbuttcap%
\pgfsetmiterjoin%
\definecolor{currentfill}{rgb}{0.511253,0.510898,0.193296}%
\pgfsetfillcolor{currentfill}%
\pgfsetlinewidth{0.000000pt}%
\definecolor{currentstroke}{rgb}{0.000000,0.000000,0.000000}%
\pgfsetstrokecolor{currentstroke}%
\pgfsetstrokeopacity{0.000000}%
\pgfsetdash{}{0pt}%
\pgfpathmoveto{\pgfqpoint{1.862608in}{1.824109in}}%
\pgfpathlineto{\pgfqpoint{1.871362in}{1.824109in}}%
\pgfpathlineto{\pgfqpoint{1.871362in}{2.226929in}}%
\pgfpathlineto{\pgfqpoint{1.862608in}{2.226929in}}%
\pgfpathlineto{\pgfqpoint{1.862608in}{1.824109in}}%
\pgfpathclose%
\pgfusepath{fill}%
\end{pgfscope}%
\begin{pgfscope}%
\pgfpathrectangle{\pgfqpoint{0.804646in}{0.600000in}}{\pgfqpoint{2.573292in}{2.070576in}}%
\pgfusepath{clip}%
\pgfsetbuttcap%
\pgfsetmiterjoin%
\definecolor{currentfill}{rgb}{0.511253,0.510898,0.193296}%
\pgfsetfillcolor{currentfill}%
\pgfsetlinewidth{0.000000pt}%
\definecolor{currentstroke}{rgb}{0.000000,0.000000,0.000000}%
\pgfsetstrokecolor{currentstroke}%
\pgfsetstrokeopacity{0.000000}%
\pgfsetdash{}{0pt}%
\pgfpathmoveto{\pgfqpoint{1.873550in}{1.838828in}}%
\pgfpathlineto{\pgfqpoint{1.882304in}{1.838828in}}%
\pgfpathlineto{\pgfqpoint{1.882304in}{2.283878in}}%
\pgfpathlineto{\pgfqpoint{1.873550in}{2.283878in}}%
\pgfpathlineto{\pgfqpoint{1.873550in}{1.838828in}}%
\pgfpathclose%
\pgfusepath{fill}%
\end{pgfscope}%
\begin{pgfscope}%
\pgfpathrectangle{\pgfqpoint{0.804646in}{0.600000in}}{\pgfqpoint{2.573292in}{2.070576in}}%
\pgfusepath{clip}%
\pgfsetbuttcap%
\pgfsetmiterjoin%
\definecolor{currentfill}{rgb}{0.511253,0.510898,0.193296}%
\pgfsetfillcolor{currentfill}%
\pgfsetlinewidth{0.000000pt}%
\definecolor{currentstroke}{rgb}{0.000000,0.000000,0.000000}%
\pgfsetstrokecolor{currentstroke}%
\pgfsetstrokeopacity{0.000000}%
\pgfsetdash{}{0pt}%
\pgfpathmoveto{\pgfqpoint{1.884492in}{1.832169in}}%
\pgfpathlineto{\pgfqpoint{1.893245in}{1.832169in}}%
\pgfpathlineto{\pgfqpoint{1.893245in}{2.335592in}}%
\pgfpathlineto{\pgfqpoint{1.884492in}{2.335592in}}%
\pgfpathlineto{\pgfqpoint{1.884492in}{1.832169in}}%
\pgfpathclose%
\pgfusepath{fill}%
\end{pgfscope}%
\begin{pgfscope}%
\pgfpathrectangle{\pgfqpoint{0.804646in}{0.600000in}}{\pgfqpoint{2.573292in}{2.070576in}}%
\pgfusepath{clip}%
\pgfsetbuttcap%
\pgfsetmiterjoin%
\definecolor{currentfill}{rgb}{0.511253,0.510898,0.193296}%
\pgfsetfillcolor{currentfill}%
\pgfsetlinewidth{0.000000pt}%
\definecolor{currentstroke}{rgb}{0.000000,0.000000,0.000000}%
\pgfsetstrokecolor{currentstroke}%
\pgfsetstrokeopacity{0.000000}%
\pgfsetdash{}{0pt}%
\pgfpathmoveto{\pgfqpoint{1.895434in}{1.852849in}}%
\pgfpathlineto{\pgfqpoint{1.904187in}{1.852849in}}%
\pgfpathlineto{\pgfqpoint{1.904187in}{2.284622in}}%
\pgfpathlineto{\pgfqpoint{1.895434in}{2.284622in}}%
\pgfpathlineto{\pgfqpoint{1.895434in}{1.852849in}}%
\pgfpathclose%
\pgfusepath{fill}%
\end{pgfscope}%
\begin{pgfscope}%
\pgfpathrectangle{\pgfqpoint{0.804646in}{0.600000in}}{\pgfqpoint{2.573292in}{2.070576in}}%
\pgfusepath{clip}%
\pgfsetbuttcap%
\pgfsetmiterjoin%
\definecolor{currentfill}{rgb}{0.511253,0.510898,0.193296}%
\pgfsetfillcolor{currentfill}%
\pgfsetlinewidth{0.000000pt}%
\definecolor{currentstroke}{rgb}{0.000000,0.000000,0.000000}%
\pgfsetstrokecolor{currentstroke}%
\pgfsetstrokeopacity{0.000000}%
\pgfsetdash{}{0pt}%
\pgfpathmoveto{\pgfqpoint{1.906376in}{1.867323in}}%
\pgfpathlineto{\pgfqpoint{1.915129in}{1.867323in}}%
\pgfpathlineto{\pgfqpoint{1.915129in}{2.315125in}}%
\pgfpathlineto{\pgfqpoint{1.906376in}{2.315125in}}%
\pgfpathlineto{\pgfqpoint{1.906376in}{1.867323in}}%
\pgfpathclose%
\pgfusepath{fill}%
\end{pgfscope}%
\begin{pgfscope}%
\pgfpathrectangle{\pgfqpoint{0.804646in}{0.600000in}}{\pgfqpoint{2.573292in}{2.070576in}}%
\pgfusepath{clip}%
\pgfsetbuttcap%
\pgfsetmiterjoin%
\definecolor{currentfill}{rgb}{0.511253,0.510898,0.193296}%
\pgfsetfillcolor{currentfill}%
\pgfsetlinewidth{0.000000pt}%
\definecolor{currentstroke}{rgb}{0.000000,0.000000,0.000000}%
\pgfsetstrokecolor{currentstroke}%
\pgfsetstrokeopacity{0.000000}%
\pgfsetdash{}{0pt}%
\pgfpathmoveto{\pgfqpoint{1.917317in}{1.867898in}}%
\pgfpathlineto{\pgfqpoint{1.926071in}{1.867898in}}%
\pgfpathlineto{\pgfqpoint{1.926071in}{2.312933in}}%
\pgfpathlineto{\pgfqpoint{1.917317in}{2.312933in}}%
\pgfpathlineto{\pgfqpoint{1.917317in}{1.867898in}}%
\pgfpathclose%
\pgfusepath{fill}%
\end{pgfscope}%
\begin{pgfscope}%
\pgfpathrectangle{\pgfqpoint{0.804646in}{0.600000in}}{\pgfqpoint{2.573292in}{2.070576in}}%
\pgfusepath{clip}%
\pgfsetbuttcap%
\pgfsetmiterjoin%
\definecolor{currentfill}{rgb}{0.511253,0.510898,0.193296}%
\pgfsetfillcolor{currentfill}%
\pgfsetlinewidth{0.000000pt}%
\definecolor{currentstroke}{rgb}{0.000000,0.000000,0.000000}%
\pgfsetstrokecolor{currentstroke}%
\pgfsetstrokeopacity{0.000000}%
\pgfsetdash{}{0pt}%
\pgfpathmoveto{\pgfqpoint{1.928259in}{1.877282in}}%
\pgfpathlineto{\pgfqpoint{1.937013in}{1.877282in}}%
\pgfpathlineto{\pgfqpoint{1.937013in}{2.278981in}}%
\pgfpathlineto{\pgfqpoint{1.928259in}{2.278981in}}%
\pgfpathlineto{\pgfqpoint{1.928259in}{1.877282in}}%
\pgfpathclose%
\pgfusepath{fill}%
\end{pgfscope}%
\begin{pgfscope}%
\pgfpathrectangle{\pgfqpoint{0.804646in}{0.600000in}}{\pgfqpoint{2.573292in}{2.070576in}}%
\pgfusepath{clip}%
\pgfsetbuttcap%
\pgfsetmiterjoin%
\definecolor{currentfill}{rgb}{0.511253,0.510898,0.193296}%
\pgfsetfillcolor{currentfill}%
\pgfsetlinewidth{0.000000pt}%
\definecolor{currentstroke}{rgb}{0.000000,0.000000,0.000000}%
\pgfsetstrokecolor{currentstroke}%
\pgfsetstrokeopacity{0.000000}%
\pgfsetdash{}{0pt}%
\pgfpathmoveto{\pgfqpoint{1.939201in}{1.875321in}}%
\pgfpathlineto{\pgfqpoint{1.947954in}{1.875321in}}%
\pgfpathlineto{\pgfqpoint{1.947954in}{2.300889in}}%
\pgfpathlineto{\pgfqpoint{1.939201in}{2.300889in}}%
\pgfpathlineto{\pgfqpoint{1.939201in}{1.875321in}}%
\pgfpathclose%
\pgfusepath{fill}%
\end{pgfscope}%
\begin{pgfscope}%
\pgfpathrectangle{\pgfqpoint{0.804646in}{0.600000in}}{\pgfqpoint{2.573292in}{2.070576in}}%
\pgfusepath{clip}%
\pgfsetbuttcap%
\pgfsetmiterjoin%
\definecolor{currentfill}{rgb}{0.511253,0.510898,0.193296}%
\pgfsetfillcolor{currentfill}%
\pgfsetlinewidth{0.000000pt}%
\definecolor{currentstroke}{rgb}{0.000000,0.000000,0.000000}%
\pgfsetstrokecolor{currentstroke}%
\pgfsetstrokeopacity{0.000000}%
\pgfsetdash{}{0pt}%
\pgfpathmoveto{\pgfqpoint{1.950143in}{1.874804in}}%
\pgfpathlineto{\pgfqpoint{1.958896in}{1.874804in}}%
\pgfpathlineto{\pgfqpoint{1.958896in}{2.395950in}}%
\pgfpathlineto{\pgfqpoint{1.950143in}{2.395950in}}%
\pgfpathlineto{\pgfqpoint{1.950143in}{1.874804in}}%
\pgfpathclose%
\pgfusepath{fill}%
\end{pgfscope}%
\begin{pgfscope}%
\pgfpathrectangle{\pgfqpoint{0.804646in}{0.600000in}}{\pgfqpoint{2.573292in}{2.070576in}}%
\pgfusepath{clip}%
\pgfsetbuttcap%
\pgfsetmiterjoin%
\definecolor{currentfill}{rgb}{0.511253,0.510898,0.193296}%
\pgfsetfillcolor{currentfill}%
\pgfsetlinewidth{0.000000pt}%
\definecolor{currentstroke}{rgb}{0.000000,0.000000,0.000000}%
\pgfsetstrokecolor{currentstroke}%
\pgfsetstrokeopacity{0.000000}%
\pgfsetdash{}{0pt}%
\pgfpathmoveto{\pgfqpoint{1.961085in}{1.853024in}}%
\pgfpathlineto{\pgfqpoint{1.969838in}{1.853024in}}%
\pgfpathlineto{\pgfqpoint{1.969838in}{2.411733in}}%
\pgfpathlineto{\pgfqpoint{1.961085in}{2.411733in}}%
\pgfpathlineto{\pgfqpoint{1.961085in}{1.853024in}}%
\pgfpathclose%
\pgfusepath{fill}%
\end{pgfscope}%
\begin{pgfscope}%
\pgfpathrectangle{\pgfqpoint{0.804646in}{0.600000in}}{\pgfqpoint{2.573292in}{2.070576in}}%
\pgfusepath{clip}%
\pgfsetbuttcap%
\pgfsetmiterjoin%
\definecolor{currentfill}{rgb}{0.511253,0.510898,0.193296}%
\pgfsetfillcolor{currentfill}%
\pgfsetlinewidth{0.000000pt}%
\definecolor{currentstroke}{rgb}{0.000000,0.000000,0.000000}%
\pgfsetstrokecolor{currentstroke}%
\pgfsetstrokeopacity{0.000000}%
\pgfsetdash{}{0pt}%
\pgfpathmoveto{\pgfqpoint{1.972026in}{1.840926in}}%
\pgfpathlineto{\pgfqpoint{1.980780in}{1.840926in}}%
\pgfpathlineto{\pgfqpoint{1.980780in}{2.378060in}}%
\pgfpathlineto{\pgfqpoint{1.972026in}{2.378060in}}%
\pgfpathlineto{\pgfqpoint{1.972026in}{1.840926in}}%
\pgfpathclose%
\pgfusepath{fill}%
\end{pgfscope}%
\begin{pgfscope}%
\pgfpathrectangle{\pgfqpoint{0.804646in}{0.600000in}}{\pgfqpoint{2.573292in}{2.070576in}}%
\pgfusepath{clip}%
\pgfsetbuttcap%
\pgfsetmiterjoin%
\definecolor{currentfill}{rgb}{0.511253,0.510898,0.193296}%
\pgfsetfillcolor{currentfill}%
\pgfsetlinewidth{0.000000pt}%
\definecolor{currentstroke}{rgb}{0.000000,0.000000,0.000000}%
\pgfsetstrokecolor{currentstroke}%
\pgfsetstrokeopacity{0.000000}%
\pgfsetdash{}{0pt}%
\pgfpathmoveto{\pgfqpoint{1.982968in}{1.833597in}}%
\pgfpathlineto{\pgfqpoint{1.991722in}{1.833597in}}%
\pgfpathlineto{\pgfqpoint{1.991722in}{2.403307in}}%
\pgfpathlineto{\pgfqpoint{1.982968in}{2.403307in}}%
\pgfpathlineto{\pgfqpoint{1.982968in}{1.833597in}}%
\pgfpathclose%
\pgfusepath{fill}%
\end{pgfscope}%
\begin{pgfscope}%
\pgfpathrectangle{\pgfqpoint{0.804646in}{0.600000in}}{\pgfqpoint{2.573292in}{2.070576in}}%
\pgfusepath{clip}%
\pgfsetbuttcap%
\pgfsetmiterjoin%
\definecolor{currentfill}{rgb}{0.511253,0.510898,0.193296}%
\pgfsetfillcolor{currentfill}%
\pgfsetlinewidth{0.000000pt}%
\definecolor{currentstroke}{rgb}{0.000000,0.000000,0.000000}%
\pgfsetstrokecolor{currentstroke}%
\pgfsetstrokeopacity{0.000000}%
\pgfsetdash{}{0pt}%
\pgfpathmoveto{\pgfqpoint{1.993910in}{1.791296in}}%
\pgfpathlineto{\pgfqpoint{2.002663in}{1.791296in}}%
\pgfpathlineto{\pgfqpoint{2.002663in}{2.407944in}}%
\pgfpathlineto{\pgfqpoint{1.993910in}{2.407944in}}%
\pgfpathlineto{\pgfqpoint{1.993910in}{1.791296in}}%
\pgfpathclose%
\pgfusepath{fill}%
\end{pgfscope}%
\begin{pgfscope}%
\pgfpathrectangle{\pgfqpoint{0.804646in}{0.600000in}}{\pgfqpoint{2.573292in}{2.070576in}}%
\pgfusepath{clip}%
\pgfsetbuttcap%
\pgfsetmiterjoin%
\definecolor{currentfill}{rgb}{0.511253,0.510898,0.193296}%
\pgfsetfillcolor{currentfill}%
\pgfsetlinewidth{0.000000pt}%
\definecolor{currentstroke}{rgb}{0.000000,0.000000,0.000000}%
\pgfsetstrokecolor{currentstroke}%
\pgfsetstrokeopacity{0.000000}%
\pgfsetdash{}{0pt}%
\pgfpathmoveto{\pgfqpoint{2.004852in}{1.774628in}}%
\pgfpathlineto{\pgfqpoint{2.013605in}{1.774628in}}%
\pgfpathlineto{\pgfqpoint{2.013605in}{2.395178in}}%
\pgfpathlineto{\pgfqpoint{2.004852in}{2.395178in}}%
\pgfpathlineto{\pgfqpoint{2.004852in}{1.774628in}}%
\pgfpathclose%
\pgfusepath{fill}%
\end{pgfscope}%
\begin{pgfscope}%
\pgfpathrectangle{\pgfqpoint{0.804646in}{0.600000in}}{\pgfqpoint{2.573292in}{2.070576in}}%
\pgfusepath{clip}%
\pgfsetbuttcap%
\pgfsetmiterjoin%
\definecolor{currentfill}{rgb}{0.511253,0.510898,0.193296}%
\pgfsetfillcolor{currentfill}%
\pgfsetlinewidth{0.000000pt}%
\definecolor{currentstroke}{rgb}{0.000000,0.000000,0.000000}%
\pgfsetstrokecolor{currentstroke}%
\pgfsetstrokeopacity{0.000000}%
\pgfsetdash{}{0pt}%
\pgfpathmoveto{\pgfqpoint{2.015794in}{1.760903in}}%
\pgfpathlineto{\pgfqpoint{2.024547in}{1.760903in}}%
\pgfpathlineto{\pgfqpoint{2.024547in}{2.458379in}}%
\pgfpathlineto{\pgfqpoint{2.015794in}{2.458379in}}%
\pgfpathlineto{\pgfqpoint{2.015794in}{1.760903in}}%
\pgfpathclose%
\pgfusepath{fill}%
\end{pgfscope}%
\begin{pgfscope}%
\pgfpathrectangle{\pgfqpoint{0.804646in}{0.600000in}}{\pgfqpoint{2.573292in}{2.070576in}}%
\pgfusepath{clip}%
\pgfsetbuttcap%
\pgfsetmiterjoin%
\definecolor{currentfill}{rgb}{0.511253,0.510898,0.193296}%
\pgfsetfillcolor{currentfill}%
\pgfsetlinewidth{0.000000pt}%
\definecolor{currentstroke}{rgb}{0.000000,0.000000,0.000000}%
\pgfsetstrokecolor{currentstroke}%
\pgfsetstrokeopacity{0.000000}%
\pgfsetdash{}{0pt}%
\pgfpathmoveto{\pgfqpoint{2.026735in}{1.746362in}}%
\pgfpathlineto{\pgfqpoint{2.035489in}{1.746362in}}%
\pgfpathlineto{\pgfqpoint{2.035489in}{2.454326in}}%
\pgfpathlineto{\pgfqpoint{2.026735in}{2.454326in}}%
\pgfpathlineto{\pgfqpoint{2.026735in}{1.746362in}}%
\pgfpathclose%
\pgfusepath{fill}%
\end{pgfscope}%
\begin{pgfscope}%
\pgfpathrectangle{\pgfqpoint{0.804646in}{0.600000in}}{\pgfqpoint{2.573292in}{2.070576in}}%
\pgfusepath{clip}%
\pgfsetbuttcap%
\pgfsetmiterjoin%
\definecolor{currentfill}{rgb}{0.511253,0.510898,0.193296}%
\pgfsetfillcolor{currentfill}%
\pgfsetlinewidth{0.000000pt}%
\definecolor{currentstroke}{rgb}{0.000000,0.000000,0.000000}%
\pgfsetstrokecolor{currentstroke}%
\pgfsetstrokeopacity{0.000000}%
\pgfsetdash{}{0pt}%
\pgfpathmoveto{\pgfqpoint{2.037677in}{1.738497in}}%
\pgfpathlineto{\pgfqpoint{2.046431in}{1.738497in}}%
\pgfpathlineto{\pgfqpoint{2.046431in}{2.460856in}}%
\pgfpathlineto{\pgfqpoint{2.037677in}{2.460856in}}%
\pgfpathlineto{\pgfqpoint{2.037677in}{1.738497in}}%
\pgfpathclose%
\pgfusepath{fill}%
\end{pgfscope}%
\begin{pgfscope}%
\pgfpathrectangle{\pgfqpoint{0.804646in}{0.600000in}}{\pgfqpoint{2.573292in}{2.070576in}}%
\pgfusepath{clip}%
\pgfsetbuttcap%
\pgfsetmiterjoin%
\definecolor{currentfill}{rgb}{0.511253,0.510898,0.193296}%
\pgfsetfillcolor{currentfill}%
\pgfsetlinewidth{0.000000pt}%
\definecolor{currentstroke}{rgb}{0.000000,0.000000,0.000000}%
\pgfsetstrokecolor{currentstroke}%
\pgfsetstrokeopacity{0.000000}%
\pgfsetdash{}{0pt}%
\pgfpathmoveto{\pgfqpoint{2.048619in}{1.726079in}}%
\pgfpathlineto{\pgfqpoint{2.057372in}{1.726079in}}%
\pgfpathlineto{\pgfqpoint{2.057372in}{2.510297in}}%
\pgfpathlineto{\pgfqpoint{2.048619in}{2.510297in}}%
\pgfpathlineto{\pgfqpoint{2.048619in}{1.726079in}}%
\pgfpathclose%
\pgfusepath{fill}%
\end{pgfscope}%
\begin{pgfscope}%
\pgfpathrectangle{\pgfqpoint{0.804646in}{0.600000in}}{\pgfqpoint{2.573292in}{2.070576in}}%
\pgfusepath{clip}%
\pgfsetbuttcap%
\pgfsetmiterjoin%
\definecolor{currentfill}{rgb}{0.511253,0.510898,0.193296}%
\pgfsetfillcolor{currentfill}%
\pgfsetlinewidth{0.000000pt}%
\definecolor{currentstroke}{rgb}{0.000000,0.000000,0.000000}%
\pgfsetstrokecolor{currentstroke}%
\pgfsetstrokeopacity{0.000000}%
\pgfsetdash{}{0pt}%
\pgfpathmoveto{\pgfqpoint{2.059561in}{1.722396in}}%
\pgfpathlineto{\pgfqpoint{2.068314in}{1.722396in}}%
\pgfpathlineto{\pgfqpoint{2.068314in}{2.516143in}}%
\pgfpathlineto{\pgfqpoint{2.059561in}{2.516143in}}%
\pgfpathlineto{\pgfqpoint{2.059561in}{1.722396in}}%
\pgfpathclose%
\pgfusepath{fill}%
\end{pgfscope}%
\begin{pgfscope}%
\pgfpathrectangle{\pgfqpoint{0.804646in}{0.600000in}}{\pgfqpoint{2.573292in}{2.070576in}}%
\pgfusepath{clip}%
\pgfsetbuttcap%
\pgfsetmiterjoin%
\definecolor{currentfill}{rgb}{0.511253,0.510898,0.193296}%
\pgfsetfillcolor{currentfill}%
\pgfsetlinewidth{0.000000pt}%
\definecolor{currentstroke}{rgb}{0.000000,0.000000,0.000000}%
\pgfsetstrokecolor{currentstroke}%
\pgfsetstrokeopacity{0.000000}%
\pgfsetdash{}{0pt}%
\pgfpathmoveto{\pgfqpoint{2.070503in}{1.718339in}}%
\pgfpathlineto{\pgfqpoint{2.079256in}{1.718339in}}%
\pgfpathlineto{\pgfqpoint{2.079256in}{2.576459in}}%
\pgfpathlineto{\pgfqpoint{2.070503in}{2.576459in}}%
\pgfpathlineto{\pgfqpoint{2.070503in}{1.718339in}}%
\pgfpathclose%
\pgfusepath{fill}%
\end{pgfscope}%
\begin{pgfscope}%
\pgfpathrectangle{\pgfqpoint{0.804646in}{0.600000in}}{\pgfqpoint{2.573292in}{2.070576in}}%
\pgfusepath{clip}%
\pgfsetbuttcap%
\pgfsetmiterjoin%
\definecolor{currentfill}{rgb}{0.511253,0.510898,0.193296}%
\pgfsetfillcolor{currentfill}%
\pgfsetlinewidth{0.000000pt}%
\definecolor{currentstroke}{rgb}{0.000000,0.000000,0.000000}%
\pgfsetstrokecolor{currentstroke}%
\pgfsetstrokeopacity{0.000000}%
\pgfsetdash{}{0pt}%
\pgfpathmoveto{\pgfqpoint{2.081444in}{1.729465in}}%
\pgfpathlineto{\pgfqpoint{2.090198in}{1.729465in}}%
\pgfpathlineto{\pgfqpoint{2.090198in}{2.527227in}}%
\pgfpathlineto{\pgfqpoint{2.081444in}{2.527227in}}%
\pgfpathlineto{\pgfqpoint{2.081444in}{1.729465in}}%
\pgfpathclose%
\pgfusepath{fill}%
\end{pgfscope}%
\begin{pgfscope}%
\pgfpathrectangle{\pgfqpoint{0.804646in}{0.600000in}}{\pgfqpoint{2.573292in}{2.070576in}}%
\pgfusepath{clip}%
\pgfsetbuttcap%
\pgfsetmiterjoin%
\definecolor{currentfill}{rgb}{0.511253,0.510898,0.193296}%
\pgfsetfillcolor{currentfill}%
\pgfsetlinewidth{0.000000pt}%
\definecolor{currentstroke}{rgb}{0.000000,0.000000,0.000000}%
\pgfsetstrokecolor{currentstroke}%
\pgfsetstrokeopacity{0.000000}%
\pgfsetdash{}{0pt}%
\pgfpathmoveto{\pgfqpoint{2.092386in}{1.741537in}}%
\pgfpathlineto{\pgfqpoint{2.101140in}{1.741537in}}%
\pgfpathlineto{\pgfqpoint{2.101140in}{2.573558in}}%
\pgfpathlineto{\pgfqpoint{2.092386in}{2.573558in}}%
\pgfpathlineto{\pgfqpoint{2.092386in}{1.741537in}}%
\pgfpathclose%
\pgfusepath{fill}%
\end{pgfscope}%
\begin{pgfscope}%
\pgfpathrectangle{\pgfqpoint{0.804646in}{0.600000in}}{\pgfqpoint{2.573292in}{2.070576in}}%
\pgfusepath{clip}%
\pgfsetbuttcap%
\pgfsetmiterjoin%
\definecolor{currentfill}{rgb}{0.511253,0.510898,0.193296}%
\pgfsetfillcolor{currentfill}%
\pgfsetlinewidth{0.000000pt}%
\definecolor{currentstroke}{rgb}{0.000000,0.000000,0.000000}%
\pgfsetstrokecolor{currentstroke}%
\pgfsetstrokeopacity{0.000000}%
\pgfsetdash{}{0pt}%
\pgfpathmoveto{\pgfqpoint{2.103328in}{1.744953in}}%
\pgfpathlineto{\pgfqpoint{2.112081in}{1.744953in}}%
\pgfpathlineto{\pgfqpoint{2.112081in}{2.571119in}}%
\pgfpathlineto{\pgfqpoint{2.103328in}{2.571119in}}%
\pgfpathlineto{\pgfqpoint{2.103328in}{1.744953in}}%
\pgfpathclose%
\pgfusepath{fill}%
\end{pgfscope}%
\begin{pgfscope}%
\pgfpathrectangle{\pgfqpoint{0.804646in}{0.600000in}}{\pgfqpoint{2.573292in}{2.070576in}}%
\pgfusepath{clip}%
\pgfsetbuttcap%
\pgfsetmiterjoin%
\definecolor{currentfill}{rgb}{0.511253,0.510898,0.193296}%
\pgfsetfillcolor{currentfill}%
\pgfsetlinewidth{0.000000pt}%
\definecolor{currentstroke}{rgb}{0.000000,0.000000,0.000000}%
\pgfsetstrokecolor{currentstroke}%
\pgfsetstrokeopacity{0.000000}%
\pgfsetdash{}{0pt}%
\pgfpathmoveto{\pgfqpoint{2.114270in}{1.749155in}}%
\pgfpathlineto{\pgfqpoint{2.123023in}{1.749155in}}%
\pgfpathlineto{\pgfqpoint{2.123023in}{2.572496in}}%
\pgfpathlineto{\pgfqpoint{2.114270in}{2.572496in}}%
\pgfpathlineto{\pgfqpoint{2.114270in}{1.749155in}}%
\pgfpathclose%
\pgfusepath{fill}%
\end{pgfscope}%
\begin{pgfscope}%
\pgfpathrectangle{\pgfqpoint{0.804646in}{0.600000in}}{\pgfqpoint{2.573292in}{2.070576in}}%
\pgfusepath{clip}%
\pgfsetbuttcap%
\pgfsetmiterjoin%
\definecolor{currentfill}{rgb}{0.511253,0.510898,0.193296}%
\pgfsetfillcolor{currentfill}%
\pgfsetlinewidth{0.000000pt}%
\definecolor{currentstroke}{rgb}{0.000000,0.000000,0.000000}%
\pgfsetstrokecolor{currentstroke}%
\pgfsetstrokeopacity{0.000000}%
\pgfsetdash{}{0pt}%
\pgfpathmoveto{\pgfqpoint{2.125212in}{1.752347in}}%
\pgfpathlineto{\pgfqpoint{2.133965in}{1.752347in}}%
\pgfpathlineto{\pgfqpoint{2.133965in}{2.520470in}}%
\pgfpathlineto{\pgfqpoint{2.125212in}{2.520470in}}%
\pgfpathlineto{\pgfqpoint{2.125212in}{1.752347in}}%
\pgfpathclose%
\pgfusepath{fill}%
\end{pgfscope}%
\begin{pgfscope}%
\pgfpathrectangle{\pgfqpoint{0.804646in}{0.600000in}}{\pgfqpoint{2.573292in}{2.070576in}}%
\pgfusepath{clip}%
\pgfsetbuttcap%
\pgfsetmiterjoin%
\definecolor{currentfill}{rgb}{0.511253,0.510898,0.193296}%
\pgfsetfillcolor{currentfill}%
\pgfsetlinewidth{0.000000pt}%
\definecolor{currentstroke}{rgb}{0.000000,0.000000,0.000000}%
\pgfsetstrokecolor{currentstroke}%
\pgfsetstrokeopacity{0.000000}%
\pgfsetdash{}{0pt}%
\pgfpathmoveto{\pgfqpoint{2.136153in}{1.778142in}}%
\pgfpathlineto{\pgfqpoint{2.144907in}{1.778142in}}%
\pgfpathlineto{\pgfqpoint{2.144907in}{2.551791in}}%
\pgfpathlineto{\pgfqpoint{2.136153in}{2.551791in}}%
\pgfpathlineto{\pgfqpoint{2.136153in}{1.778142in}}%
\pgfpathclose%
\pgfusepath{fill}%
\end{pgfscope}%
\begin{pgfscope}%
\pgfpathrectangle{\pgfqpoint{0.804646in}{0.600000in}}{\pgfqpoint{2.573292in}{2.070576in}}%
\pgfusepath{clip}%
\pgfsetbuttcap%
\pgfsetmiterjoin%
\definecolor{currentfill}{rgb}{0.511253,0.510898,0.193296}%
\pgfsetfillcolor{currentfill}%
\pgfsetlinewidth{0.000000pt}%
\definecolor{currentstroke}{rgb}{0.000000,0.000000,0.000000}%
\pgfsetstrokecolor{currentstroke}%
\pgfsetstrokeopacity{0.000000}%
\pgfsetdash{}{0pt}%
\pgfpathmoveto{\pgfqpoint{2.147095in}{1.788126in}}%
\pgfpathlineto{\pgfqpoint{2.155849in}{1.788126in}}%
\pgfpathlineto{\pgfqpoint{2.155849in}{2.530526in}}%
\pgfpathlineto{\pgfqpoint{2.147095in}{2.530526in}}%
\pgfpathlineto{\pgfqpoint{2.147095in}{1.788126in}}%
\pgfpathclose%
\pgfusepath{fill}%
\end{pgfscope}%
\begin{pgfscope}%
\pgfpathrectangle{\pgfqpoint{0.804646in}{0.600000in}}{\pgfqpoint{2.573292in}{2.070576in}}%
\pgfusepath{clip}%
\pgfsetbuttcap%
\pgfsetmiterjoin%
\definecolor{currentfill}{rgb}{0.511253,0.510898,0.193296}%
\pgfsetfillcolor{currentfill}%
\pgfsetlinewidth{0.000000pt}%
\definecolor{currentstroke}{rgb}{0.000000,0.000000,0.000000}%
\pgfsetstrokecolor{currentstroke}%
\pgfsetstrokeopacity{0.000000}%
\pgfsetdash{}{0pt}%
\pgfpathmoveto{\pgfqpoint{2.158037in}{1.788972in}}%
\pgfpathlineto{\pgfqpoint{2.166790in}{1.788972in}}%
\pgfpathlineto{\pgfqpoint{2.166790in}{2.541900in}}%
\pgfpathlineto{\pgfqpoint{2.158037in}{2.541900in}}%
\pgfpathlineto{\pgfqpoint{2.158037in}{1.788972in}}%
\pgfpathclose%
\pgfusepath{fill}%
\end{pgfscope}%
\begin{pgfscope}%
\pgfpathrectangle{\pgfqpoint{0.804646in}{0.600000in}}{\pgfqpoint{2.573292in}{2.070576in}}%
\pgfusepath{clip}%
\pgfsetbuttcap%
\pgfsetmiterjoin%
\definecolor{currentfill}{rgb}{0.511253,0.510898,0.193296}%
\pgfsetfillcolor{currentfill}%
\pgfsetlinewidth{0.000000pt}%
\definecolor{currentstroke}{rgb}{0.000000,0.000000,0.000000}%
\pgfsetstrokecolor{currentstroke}%
\pgfsetstrokeopacity{0.000000}%
\pgfsetdash{}{0pt}%
\pgfpathmoveto{\pgfqpoint{2.168979in}{1.790120in}}%
\pgfpathlineto{\pgfqpoint{2.177732in}{1.790120in}}%
\pgfpathlineto{\pgfqpoint{2.177732in}{2.523255in}}%
\pgfpathlineto{\pgfqpoint{2.168979in}{2.523255in}}%
\pgfpathlineto{\pgfqpoint{2.168979in}{1.790120in}}%
\pgfpathclose%
\pgfusepath{fill}%
\end{pgfscope}%
\begin{pgfscope}%
\pgfpathrectangle{\pgfqpoint{0.804646in}{0.600000in}}{\pgfqpoint{2.573292in}{2.070576in}}%
\pgfusepath{clip}%
\pgfsetbuttcap%
\pgfsetmiterjoin%
\definecolor{currentfill}{rgb}{0.511253,0.510898,0.193296}%
\pgfsetfillcolor{currentfill}%
\pgfsetlinewidth{0.000000pt}%
\definecolor{currentstroke}{rgb}{0.000000,0.000000,0.000000}%
\pgfsetstrokecolor{currentstroke}%
\pgfsetstrokeopacity{0.000000}%
\pgfsetdash{}{0pt}%
\pgfpathmoveto{\pgfqpoint{2.179921in}{1.778668in}}%
\pgfpathlineto{\pgfqpoint{2.188674in}{1.778668in}}%
\pgfpathlineto{\pgfqpoint{2.188674in}{2.483957in}}%
\pgfpathlineto{\pgfqpoint{2.179921in}{2.483957in}}%
\pgfpathlineto{\pgfqpoint{2.179921in}{1.778668in}}%
\pgfpathclose%
\pgfusepath{fill}%
\end{pgfscope}%
\begin{pgfscope}%
\pgfpathrectangle{\pgfqpoint{0.804646in}{0.600000in}}{\pgfqpoint{2.573292in}{2.070576in}}%
\pgfusepath{clip}%
\pgfsetbuttcap%
\pgfsetmiterjoin%
\definecolor{currentfill}{rgb}{0.511253,0.510898,0.193296}%
\pgfsetfillcolor{currentfill}%
\pgfsetlinewidth{0.000000pt}%
\definecolor{currentstroke}{rgb}{0.000000,0.000000,0.000000}%
\pgfsetstrokecolor{currentstroke}%
\pgfsetstrokeopacity{0.000000}%
\pgfsetdash{}{0pt}%
\pgfpathmoveto{\pgfqpoint{2.190862in}{1.798811in}}%
\pgfpathlineto{\pgfqpoint{2.199616in}{1.798811in}}%
\pgfpathlineto{\pgfqpoint{2.199616in}{2.420952in}}%
\pgfpathlineto{\pgfqpoint{2.190862in}{2.420952in}}%
\pgfpathlineto{\pgfqpoint{2.190862in}{1.798811in}}%
\pgfpathclose%
\pgfusepath{fill}%
\end{pgfscope}%
\begin{pgfscope}%
\pgfpathrectangle{\pgfqpoint{0.804646in}{0.600000in}}{\pgfqpoint{2.573292in}{2.070576in}}%
\pgfusepath{clip}%
\pgfsetbuttcap%
\pgfsetmiterjoin%
\definecolor{currentfill}{rgb}{0.511253,0.510898,0.193296}%
\pgfsetfillcolor{currentfill}%
\pgfsetlinewidth{0.000000pt}%
\definecolor{currentstroke}{rgb}{0.000000,0.000000,0.000000}%
\pgfsetstrokecolor{currentstroke}%
\pgfsetstrokeopacity{0.000000}%
\pgfsetdash{}{0pt}%
\pgfpathmoveto{\pgfqpoint{2.201804in}{1.798683in}}%
\pgfpathlineto{\pgfqpoint{2.210558in}{1.798683in}}%
\pgfpathlineto{\pgfqpoint{2.210558in}{2.415015in}}%
\pgfpathlineto{\pgfqpoint{2.201804in}{2.415015in}}%
\pgfpathlineto{\pgfqpoint{2.201804in}{1.798683in}}%
\pgfpathclose%
\pgfusepath{fill}%
\end{pgfscope}%
\begin{pgfscope}%
\pgfpathrectangle{\pgfqpoint{0.804646in}{0.600000in}}{\pgfqpoint{2.573292in}{2.070576in}}%
\pgfusepath{clip}%
\pgfsetbuttcap%
\pgfsetmiterjoin%
\definecolor{currentfill}{rgb}{0.511253,0.510898,0.193296}%
\pgfsetfillcolor{currentfill}%
\pgfsetlinewidth{0.000000pt}%
\definecolor{currentstroke}{rgb}{0.000000,0.000000,0.000000}%
\pgfsetstrokecolor{currentstroke}%
\pgfsetstrokeopacity{0.000000}%
\pgfsetdash{}{0pt}%
\pgfpathmoveto{\pgfqpoint{2.212746in}{1.797384in}}%
\pgfpathlineto{\pgfqpoint{2.221499in}{1.797384in}}%
\pgfpathlineto{\pgfqpoint{2.221499in}{2.420455in}}%
\pgfpathlineto{\pgfqpoint{2.212746in}{2.420455in}}%
\pgfpathlineto{\pgfqpoint{2.212746in}{1.797384in}}%
\pgfpathclose%
\pgfusepath{fill}%
\end{pgfscope}%
\begin{pgfscope}%
\pgfpathrectangle{\pgfqpoint{0.804646in}{0.600000in}}{\pgfqpoint{2.573292in}{2.070576in}}%
\pgfusepath{clip}%
\pgfsetbuttcap%
\pgfsetmiterjoin%
\definecolor{currentfill}{rgb}{0.511253,0.510898,0.193296}%
\pgfsetfillcolor{currentfill}%
\pgfsetlinewidth{0.000000pt}%
\definecolor{currentstroke}{rgb}{0.000000,0.000000,0.000000}%
\pgfsetstrokecolor{currentstroke}%
\pgfsetstrokeopacity{0.000000}%
\pgfsetdash{}{0pt}%
\pgfpathmoveto{\pgfqpoint{2.223688in}{1.820136in}}%
\pgfpathlineto{\pgfqpoint{2.232441in}{1.820136in}}%
\pgfpathlineto{\pgfqpoint{2.232441in}{2.410059in}}%
\pgfpathlineto{\pgfqpoint{2.223688in}{2.410059in}}%
\pgfpathlineto{\pgfqpoint{2.223688in}{1.820136in}}%
\pgfpathclose%
\pgfusepath{fill}%
\end{pgfscope}%
\begin{pgfscope}%
\pgfpathrectangle{\pgfqpoint{0.804646in}{0.600000in}}{\pgfqpoint{2.573292in}{2.070576in}}%
\pgfusepath{clip}%
\pgfsetbuttcap%
\pgfsetmiterjoin%
\definecolor{currentfill}{rgb}{0.511253,0.510898,0.193296}%
\pgfsetfillcolor{currentfill}%
\pgfsetlinewidth{0.000000pt}%
\definecolor{currentstroke}{rgb}{0.000000,0.000000,0.000000}%
\pgfsetstrokecolor{currentstroke}%
\pgfsetstrokeopacity{0.000000}%
\pgfsetdash{}{0pt}%
\pgfpathmoveto{\pgfqpoint{2.234630in}{1.829753in}}%
\pgfpathlineto{\pgfqpoint{2.243383in}{1.829753in}}%
\pgfpathlineto{\pgfqpoint{2.243383in}{2.360319in}}%
\pgfpathlineto{\pgfqpoint{2.234630in}{2.360319in}}%
\pgfpathlineto{\pgfqpoint{2.234630in}{1.829753in}}%
\pgfpathclose%
\pgfusepath{fill}%
\end{pgfscope}%
\begin{pgfscope}%
\pgfpathrectangle{\pgfqpoint{0.804646in}{0.600000in}}{\pgfqpoint{2.573292in}{2.070576in}}%
\pgfusepath{clip}%
\pgfsetbuttcap%
\pgfsetmiterjoin%
\definecolor{currentfill}{rgb}{0.511253,0.510898,0.193296}%
\pgfsetfillcolor{currentfill}%
\pgfsetlinewidth{0.000000pt}%
\definecolor{currentstroke}{rgb}{0.000000,0.000000,0.000000}%
\pgfsetstrokecolor{currentstroke}%
\pgfsetstrokeopacity{0.000000}%
\pgfsetdash{}{0pt}%
\pgfpathmoveto{\pgfqpoint{2.245571in}{1.843586in}}%
\pgfpathlineto{\pgfqpoint{2.254325in}{1.843586in}}%
\pgfpathlineto{\pgfqpoint{2.254325in}{2.375227in}}%
\pgfpathlineto{\pgfqpoint{2.245571in}{2.375227in}}%
\pgfpathlineto{\pgfqpoint{2.245571in}{1.843586in}}%
\pgfpathclose%
\pgfusepath{fill}%
\end{pgfscope}%
\begin{pgfscope}%
\pgfpathrectangle{\pgfqpoint{0.804646in}{0.600000in}}{\pgfqpoint{2.573292in}{2.070576in}}%
\pgfusepath{clip}%
\pgfsetbuttcap%
\pgfsetmiterjoin%
\definecolor{currentfill}{rgb}{0.511253,0.510898,0.193296}%
\pgfsetfillcolor{currentfill}%
\pgfsetlinewidth{0.000000pt}%
\definecolor{currentstroke}{rgb}{0.000000,0.000000,0.000000}%
\pgfsetstrokecolor{currentstroke}%
\pgfsetstrokeopacity{0.000000}%
\pgfsetdash{}{0pt}%
\pgfpathmoveto{\pgfqpoint{2.256513in}{1.854722in}}%
\pgfpathlineto{\pgfqpoint{2.265267in}{1.854722in}}%
\pgfpathlineto{\pgfqpoint{2.265267in}{2.319209in}}%
\pgfpathlineto{\pgfqpoint{2.256513in}{2.319209in}}%
\pgfpathlineto{\pgfqpoint{2.256513in}{1.854722in}}%
\pgfpathclose%
\pgfusepath{fill}%
\end{pgfscope}%
\begin{pgfscope}%
\pgfpathrectangle{\pgfqpoint{0.804646in}{0.600000in}}{\pgfqpoint{2.573292in}{2.070576in}}%
\pgfusepath{clip}%
\pgfsetbuttcap%
\pgfsetmiterjoin%
\definecolor{currentfill}{rgb}{0.511253,0.510898,0.193296}%
\pgfsetfillcolor{currentfill}%
\pgfsetlinewidth{0.000000pt}%
\definecolor{currentstroke}{rgb}{0.000000,0.000000,0.000000}%
\pgfsetstrokecolor{currentstroke}%
\pgfsetstrokeopacity{0.000000}%
\pgfsetdash{}{0pt}%
\pgfpathmoveto{\pgfqpoint{2.267455in}{1.866268in}}%
\pgfpathlineto{\pgfqpoint{2.276208in}{1.866268in}}%
\pgfpathlineto{\pgfqpoint{2.276208in}{2.231370in}}%
\pgfpathlineto{\pgfqpoint{2.267455in}{2.231370in}}%
\pgfpathlineto{\pgfqpoint{2.267455in}{1.866268in}}%
\pgfpathclose%
\pgfusepath{fill}%
\end{pgfscope}%
\begin{pgfscope}%
\pgfpathrectangle{\pgfqpoint{0.804646in}{0.600000in}}{\pgfqpoint{2.573292in}{2.070576in}}%
\pgfusepath{clip}%
\pgfsetbuttcap%
\pgfsetmiterjoin%
\definecolor{currentfill}{rgb}{0.511253,0.510898,0.193296}%
\pgfsetfillcolor{currentfill}%
\pgfsetlinewidth{0.000000pt}%
\definecolor{currentstroke}{rgb}{0.000000,0.000000,0.000000}%
\pgfsetstrokecolor{currentstroke}%
\pgfsetstrokeopacity{0.000000}%
\pgfsetdash{}{0pt}%
\pgfpathmoveto{\pgfqpoint{2.278397in}{1.868234in}}%
\pgfpathlineto{\pgfqpoint{2.287150in}{1.868234in}}%
\pgfpathlineto{\pgfqpoint{2.287150in}{2.224362in}}%
\pgfpathlineto{\pgfqpoint{2.278397in}{2.224362in}}%
\pgfpathlineto{\pgfqpoint{2.278397in}{1.868234in}}%
\pgfpathclose%
\pgfusepath{fill}%
\end{pgfscope}%
\begin{pgfscope}%
\pgfpathrectangle{\pgfqpoint{0.804646in}{0.600000in}}{\pgfqpoint{2.573292in}{2.070576in}}%
\pgfusepath{clip}%
\pgfsetbuttcap%
\pgfsetmiterjoin%
\definecolor{currentfill}{rgb}{0.511253,0.510898,0.193296}%
\pgfsetfillcolor{currentfill}%
\pgfsetlinewidth{0.000000pt}%
\definecolor{currentstroke}{rgb}{0.000000,0.000000,0.000000}%
\pgfsetstrokecolor{currentstroke}%
\pgfsetstrokeopacity{0.000000}%
\pgfsetdash{}{0pt}%
\pgfpathmoveto{\pgfqpoint{2.289339in}{1.876210in}}%
\pgfpathlineto{\pgfqpoint{2.298092in}{1.876210in}}%
\pgfpathlineto{\pgfqpoint{2.298092in}{2.145759in}}%
\pgfpathlineto{\pgfqpoint{2.289339in}{2.145759in}}%
\pgfpathlineto{\pgfqpoint{2.289339in}{1.876210in}}%
\pgfpathclose%
\pgfusepath{fill}%
\end{pgfscope}%
\begin{pgfscope}%
\pgfpathrectangle{\pgfqpoint{0.804646in}{0.600000in}}{\pgfqpoint{2.573292in}{2.070576in}}%
\pgfusepath{clip}%
\pgfsetbuttcap%
\pgfsetmiterjoin%
\definecolor{currentfill}{rgb}{0.511253,0.510898,0.193296}%
\pgfsetfillcolor{currentfill}%
\pgfsetlinewidth{0.000000pt}%
\definecolor{currentstroke}{rgb}{0.000000,0.000000,0.000000}%
\pgfsetstrokecolor{currentstroke}%
\pgfsetstrokeopacity{0.000000}%
\pgfsetdash{}{0pt}%
\pgfpathmoveto{\pgfqpoint{2.300280in}{1.892155in}}%
\pgfpathlineto{\pgfqpoint{2.309034in}{1.892155in}}%
\pgfpathlineto{\pgfqpoint{2.309034in}{2.048288in}}%
\pgfpathlineto{\pgfqpoint{2.300280in}{2.048288in}}%
\pgfpathlineto{\pgfqpoint{2.300280in}{1.892155in}}%
\pgfpathclose%
\pgfusepath{fill}%
\end{pgfscope}%
\begin{pgfscope}%
\pgfpathrectangle{\pgfqpoint{0.804646in}{0.600000in}}{\pgfqpoint{2.573292in}{2.070576in}}%
\pgfusepath{clip}%
\pgfsetbuttcap%
\pgfsetmiterjoin%
\definecolor{currentfill}{rgb}{0.511253,0.510898,0.193296}%
\pgfsetfillcolor{currentfill}%
\pgfsetlinewidth{0.000000pt}%
\definecolor{currentstroke}{rgb}{0.000000,0.000000,0.000000}%
\pgfsetstrokecolor{currentstroke}%
\pgfsetstrokeopacity{0.000000}%
\pgfsetdash{}{0pt}%
\pgfpathmoveto{\pgfqpoint{2.311222in}{1.904025in}}%
\pgfpathlineto{\pgfqpoint{2.319976in}{1.904025in}}%
\pgfpathlineto{\pgfqpoint{2.319976in}{2.003593in}}%
\pgfpathlineto{\pgfqpoint{2.311222in}{2.003593in}}%
\pgfpathlineto{\pgfqpoint{2.311222in}{1.904025in}}%
\pgfpathclose%
\pgfusepath{fill}%
\end{pgfscope}%
\begin{pgfscope}%
\pgfpathrectangle{\pgfqpoint{0.804646in}{0.600000in}}{\pgfqpoint{2.573292in}{2.070576in}}%
\pgfusepath{clip}%
\pgfsetbuttcap%
\pgfsetmiterjoin%
\definecolor{currentfill}{rgb}{0.511253,0.510898,0.193296}%
\pgfsetfillcolor{currentfill}%
\pgfsetlinewidth{0.000000pt}%
\definecolor{currentstroke}{rgb}{0.000000,0.000000,0.000000}%
\pgfsetstrokecolor{currentstroke}%
\pgfsetstrokeopacity{0.000000}%
\pgfsetdash{}{0pt}%
\pgfpathmoveto{\pgfqpoint{2.322164in}{1.902213in}}%
\pgfpathlineto{\pgfqpoint{2.330917in}{1.902213in}}%
\pgfpathlineto{\pgfqpoint{2.330917in}{2.040048in}}%
\pgfpathlineto{\pgfqpoint{2.322164in}{2.040048in}}%
\pgfpathlineto{\pgfqpoint{2.322164in}{1.902213in}}%
\pgfpathclose%
\pgfusepath{fill}%
\end{pgfscope}%
\begin{pgfscope}%
\pgfpathrectangle{\pgfqpoint{0.804646in}{0.600000in}}{\pgfqpoint{2.573292in}{2.070576in}}%
\pgfusepath{clip}%
\pgfsetbuttcap%
\pgfsetmiterjoin%
\definecolor{currentfill}{rgb}{0.511253,0.510898,0.193296}%
\pgfsetfillcolor{currentfill}%
\pgfsetlinewidth{0.000000pt}%
\definecolor{currentstroke}{rgb}{0.000000,0.000000,0.000000}%
\pgfsetstrokecolor{currentstroke}%
\pgfsetstrokeopacity{0.000000}%
\pgfsetdash{}{0pt}%
\pgfpathmoveto{\pgfqpoint{2.333106in}{1.922844in}}%
\pgfpathlineto{\pgfqpoint{2.341859in}{1.922844in}}%
\pgfpathlineto{\pgfqpoint{2.341859in}{1.987160in}}%
\pgfpathlineto{\pgfqpoint{2.333106in}{1.987160in}}%
\pgfpathlineto{\pgfqpoint{2.333106in}{1.922844in}}%
\pgfpathclose%
\pgfusepath{fill}%
\end{pgfscope}%
\begin{pgfscope}%
\pgfpathrectangle{\pgfqpoint{0.804646in}{0.600000in}}{\pgfqpoint{2.573292in}{2.070576in}}%
\pgfusepath{clip}%
\pgfsetbuttcap%
\pgfsetmiterjoin%
\definecolor{currentfill}{rgb}{0.511253,0.510898,0.193296}%
\pgfsetfillcolor{currentfill}%
\pgfsetlinewidth{0.000000pt}%
\definecolor{currentstroke}{rgb}{0.000000,0.000000,0.000000}%
\pgfsetstrokecolor{currentstroke}%
\pgfsetstrokeopacity{0.000000}%
\pgfsetdash{}{0pt}%
\pgfpathmoveto{\pgfqpoint{2.344048in}{1.916701in}}%
\pgfpathlineto{\pgfqpoint{2.352801in}{1.916701in}}%
\pgfpathlineto{\pgfqpoint{2.352801in}{1.944229in}}%
\pgfpathlineto{\pgfqpoint{2.344048in}{1.944229in}}%
\pgfpathlineto{\pgfqpoint{2.344048in}{1.916701in}}%
\pgfpathclose%
\pgfusepath{fill}%
\end{pgfscope}%
\begin{pgfscope}%
\pgfpathrectangle{\pgfqpoint{0.804646in}{0.600000in}}{\pgfqpoint{2.573292in}{2.070576in}}%
\pgfusepath{clip}%
\pgfsetbuttcap%
\pgfsetmiterjoin%
\definecolor{currentfill}{rgb}{0.511253,0.510898,0.193296}%
\pgfsetfillcolor{currentfill}%
\pgfsetlinewidth{0.000000pt}%
\definecolor{currentstroke}{rgb}{0.000000,0.000000,0.000000}%
\pgfsetstrokecolor{currentstroke}%
\pgfsetstrokeopacity{0.000000}%
\pgfsetdash{}{0pt}%
\pgfpathmoveto{\pgfqpoint{2.354989in}{1.929066in}}%
\pgfpathlineto{\pgfqpoint{2.363743in}{1.929066in}}%
\pgfpathlineto{\pgfqpoint{2.363743in}{1.937340in}}%
\pgfpathlineto{\pgfqpoint{2.354989in}{1.937340in}}%
\pgfpathlineto{\pgfqpoint{2.354989in}{1.929066in}}%
\pgfpathclose%
\pgfusepath{fill}%
\end{pgfscope}%
\begin{pgfscope}%
\pgfpathrectangle{\pgfqpoint{0.804646in}{0.600000in}}{\pgfqpoint{2.573292in}{2.070576in}}%
\pgfusepath{clip}%
\pgfsetbuttcap%
\pgfsetmiterjoin%
\definecolor{currentfill}{rgb}{0.511253,0.510898,0.193296}%
\pgfsetfillcolor{currentfill}%
\pgfsetlinewidth{0.000000pt}%
\definecolor{currentstroke}{rgb}{0.000000,0.000000,0.000000}%
\pgfsetstrokecolor{currentstroke}%
\pgfsetstrokeopacity{0.000000}%
\pgfsetdash{}{0pt}%
\pgfpathmoveto{\pgfqpoint{2.365931in}{1.389443in}}%
\pgfpathlineto{\pgfqpoint{2.374685in}{1.389443in}}%
\pgfpathlineto{\pgfqpoint{2.374685in}{1.342476in}}%
\pgfpathlineto{\pgfqpoint{2.365931in}{1.342476in}}%
\pgfpathlineto{\pgfqpoint{2.365931in}{1.389443in}}%
\pgfpathclose%
\pgfusepath{fill}%
\end{pgfscope}%
\begin{pgfscope}%
\pgfpathrectangle{\pgfqpoint{0.804646in}{0.600000in}}{\pgfqpoint{2.573292in}{2.070576in}}%
\pgfusepath{clip}%
\pgfsetbuttcap%
\pgfsetmiterjoin%
\definecolor{currentfill}{rgb}{0.511253,0.510898,0.193296}%
\pgfsetfillcolor{currentfill}%
\pgfsetlinewidth{0.000000pt}%
\definecolor{currentstroke}{rgb}{0.000000,0.000000,0.000000}%
\pgfsetstrokecolor{currentstroke}%
\pgfsetstrokeopacity{0.000000}%
\pgfsetdash{}{0pt}%
\pgfpathmoveto{\pgfqpoint{2.376873in}{1.399048in}}%
\pgfpathlineto{\pgfqpoint{2.385626in}{1.399048in}}%
\pgfpathlineto{\pgfqpoint{2.385626in}{1.386806in}}%
\pgfpathlineto{\pgfqpoint{2.376873in}{1.386806in}}%
\pgfpathlineto{\pgfqpoint{2.376873in}{1.399048in}}%
\pgfpathclose%
\pgfusepath{fill}%
\end{pgfscope}%
\begin{pgfscope}%
\pgfpathrectangle{\pgfqpoint{0.804646in}{0.600000in}}{\pgfqpoint{2.573292in}{2.070576in}}%
\pgfusepath{clip}%
\pgfsetbuttcap%
\pgfsetmiterjoin%
\definecolor{currentfill}{rgb}{0.511253,0.510898,0.193296}%
\pgfsetfillcolor{currentfill}%
\pgfsetlinewidth{0.000000pt}%
\definecolor{currentstroke}{rgb}{0.000000,0.000000,0.000000}%
\pgfsetstrokecolor{currentstroke}%
\pgfsetstrokeopacity{0.000000}%
\pgfsetdash{}{0pt}%
\pgfpathmoveto{\pgfqpoint{2.387815in}{1.402040in}}%
\pgfpathlineto{\pgfqpoint{2.396568in}{1.402040in}}%
\pgfpathlineto{\pgfqpoint{2.396568in}{1.371754in}}%
\pgfpathlineto{\pgfqpoint{2.387815in}{1.371754in}}%
\pgfpathlineto{\pgfqpoint{2.387815in}{1.402040in}}%
\pgfpathclose%
\pgfusepath{fill}%
\end{pgfscope}%
\begin{pgfscope}%
\pgfpathrectangle{\pgfqpoint{0.804646in}{0.600000in}}{\pgfqpoint{2.573292in}{2.070576in}}%
\pgfusepath{clip}%
\pgfsetbuttcap%
\pgfsetmiterjoin%
\definecolor{currentfill}{rgb}{0.511253,0.510898,0.193296}%
\pgfsetfillcolor{currentfill}%
\pgfsetlinewidth{0.000000pt}%
\definecolor{currentstroke}{rgb}{0.000000,0.000000,0.000000}%
\pgfsetstrokecolor{currentstroke}%
\pgfsetstrokeopacity{0.000000}%
\pgfsetdash{}{0pt}%
\pgfpathmoveto{\pgfqpoint{2.398757in}{1.444873in}}%
\pgfpathlineto{\pgfqpoint{2.407510in}{1.444873in}}%
\pgfpathlineto{\pgfqpoint{2.407510in}{1.359949in}}%
\pgfpathlineto{\pgfqpoint{2.398757in}{1.359949in}}%
\pgfpathlineto{\pgfqpoint{2.398757in}{1.444873in}}%
\pgfpathclose%
\pgfusepath{fill}%
\end{pgfscope}%
\begin{pgfscope}%
\pgfpathrectangle{\pgfqpoint{0.804646in}{0.600000in}}{\pgfqpoint{2.573292in}{2.070576in}}%
\pgfusepath{clip}%
\pgfsetbuttcap%
\pgfsetmiterjoin%
\definecolor{currentfill}{rgb}{0.511253,0.510898,0.193296}%
\pgfsetfillcolor{currentfill}%
\pgfsetlinewidth{0.000000pt}%
\definecolor{currentstroke}{rgb}{0.000000,0.000000,0.000000}%
\pgfsetstrokecolor{currentstroke}%
\pgfsetstrokeopacity{0.000000}%
\pgfsetdash{}{0pt}%
\pgfpathmoveto{\pgfqpoint{2.409698in}{1.469018in}}%
\pgfpathlineto{\pgfqpoint{2.418452in}{1.469018in}}%
\pgfpathlineto{\pgfqpoint{2.418452in}{1.369271in}}%
\pgfpathlineto{\pgfqpoint{2.409698in}{1.369271in}}%
\pgfpathlineto{\pgfqpoint{2.409698in}{1.469018in}}%
\pgfpathclose%
\pgfusepath{fill}%
\end{pgfscope}%
\begin{pgfscope}%
\pgfpathrectangle{\pgfqpoint{0.804646in}{0.600000in}}{\pgfqpoint{2.573292in}{2.070576in}}%
\pgfusepath{clip}%
\pgfsetbuttcap%
\pgfsetmiterjoin%
\definecolor{currentfill}{rgb}{0.511253,0.510898,0.193296}%
\pgfsetfillcolor{currentfill}%
\pgfsetlinewidth{0.000000pt}%
\definecolor{currentstroke}{rgb}{0.000000,0.000000,0.000000}%
\pgfsetstrokecolor{currentstroke}%
\pgfsetstrokeopacity{0.000000}%
\pgfsetdash{}{0pt}%
\pgfpathmoveto{\pgfqpoint{2.420640in}{1.513197in}}%
\pgfpathlineto{\pgfqpoint{2.429394in}{1.513197in}}%
\pgfpathlineto{\pgfqpoint{2.429394in}{1.377886in}}%
\pgfpathlineto{\pgfqpoint{2.420640in}{1.377886in}}%
\pgfpathlineto{\pgfqpoint{2.420640in}{1.513197in}}%
\pgfpathclose%
\pgfusepath{fill}%
\end{pgfscope}%
\begin{pgfscope}%
\pgfpathrectangle{\pgfqpoint{0.804646in}{0.600000in}}{\pgfqpoint{2.573292in}{2.070576in}}%
\pgfusepath{clip}%
\pgfsetbuttcap%
\pgfsetmiterjoin%
\definecolor{currentfill}{rgb}{0.511253,0.510898,0.193296}%
\pgfsetfillcolor{currentfill}%
\pgfsetlinewidth{0.000000pt}%
\definecolor{currentstroke}{rgb}{0.000000,0.000000,0.000000}%
\pgfsetstrokecolor{currentstroke}%
\pgfsetstrokeopacity{0.000000}%
\pgfsetdash{}{0pt}%
\pgfpathmoveto{\pgfqpoint{2.431582in}{1.499663in}}%
\pgfpathlineto{\pgfqpoint{2.440335in}{1.499663in}}%
\pgfpathlineto{\pgfqpoint{2.440335in}{1.420947in}}%
\pgfpathlineto{\pgfqpoint{2.431582in}{1.420947in}}%
\pgfpathlineto{\pgfqpoint{2.431582in}{1.499663in}}%
\pgfpathclose%
\pgfusepath{fill}%
\end{pgfscope}%
\begin{pgfscope}%
\pgfpathrectangle{\pgfqpoint{0.804646in}{0.600000in}}{\pgfqpoint{2.573292in}{2.070576in}}%
\pgfusepath{clip}%
\pgfsetbuttcap%
\pgfsetmiterjoin%
\definecolor{currentfill}{rgb}{0.511253,0.510898,0.193296}%
\pgfsetfillcolor{currentfill}%
\pgfsetlinewidth{0.000000pt}%
\definecolor{currentstroke}{rgb}{0.000000,0.000000,0.000000}%
\pgfsetstrokecolor{currentstroke}%
\pgfsetstrokeopacity{0.000000}%
\pgfsetdash{}{0pt}%
\pgfpathmoveto{\pgfqpoint{2.442524in}{1.546152in}}%
\pgfpathlineto{\pgfqpoint{2.451277in}{1.546152in}}%
\pgfpathlineto{\pgfqpoint{2.451277in}{1.387536in}}%
\pgfpathlineto{\pgfqpoint{2.442524in}{1.387536in}}%
\pgfpathlineto{\pgfqpoint{2.442524in}{1.546152in}}%
\pgfpathclose%
\pgfusepath{fill}%
\end{pgfscope}%
\begin{pgfscope}%
\pgfpathrectangle{\pgfqpoint{0.804646in}{0.600000in}}{\pgfqpoint{2.573292in}{2.070576in}}%
\pgfusepath{clip}%
\pgfsetbuttcap%
\pgfsetmiterjoin%
\definecolor{currentfill}{rgb}{0.511253,0.510898,0.193296}%
\pgfsetfillcolor{currentfill}%
\pgfsetlinewidth{0.000000pt}%
\definecolor{currentstroke}{rgb}{0.000000,0.000000,0.000000}%
\pgfsetstrokecolor{currentstroke}%
\pgfsetstrokeopacity{0.000000}%
\pgfsetdash{}{0pt}%
\pgfpathmoveto{\pgfqpoint{2.453466in}{1.554201in}}%
\pgfpathlineto{\pgfqpoint{2.462219in}{1.554201in}}%
\pgfpathlineto{\pgfqpoint{2.462219in}{1.456200in}}%
\pgfpathlineto{\pgfqpoint{2.453466in}{1.456200in}}%
\pgfpathlineto{\pgfqpoint{2.453466in}{1.554201in}}%
\pgfpathclose%
\pgfusepath{fill}%
\end{pgfscope}%
\begin{pgfscope}%
\pgfpathrectangle{\pgfqpoint{0.804646in}{0.600000in}}{\pgfqpoint{2.573292in}{2.070576in}}%
\pgfusepath{clip}%
\pgfsetbuttcap%
\pgfsetmiterjoin%
\definecolor{currentfill}{rgb}{0.511253,0.510898,0.193296}%
\pgfsetfillcolor{currentfill}%
\pgfsetlinewidth{0.000000pt}%
\definecolor{currentstroke}{rgb}{0.000000,0.000000,0.000000}%
\pgfsetstrokecolor{currentstroke}%
\pgfsetstrokeopacity{0.000000}%
\pgfsetdash{}{0pt}%
\pgfpathmoveto{\pgfqpoint{2.464407in}{1.579645in}}%
\pgfpathlineto{\pgfqpoint{2.473161in}{1.579645in}}%
\pgfpathlineto{\pgfqpoint{2.473161in}{1.525919in}}%
\pgfpathlineto{\pgfqpoint{2.464407in}{1.525919in}}%
\pgfpathlineto{\pgfqpoint{2.464407in}{1.579645in}}%
\pgfpathclose%
\pgfusepath{fill}%
\end{pgfscope}%
\begin{pgfscope}%
\pgfpathrectangle{\pgfqpoint{0.804646in}{0.600000in}}{\pgfqpoint{2.573292in}{2.070576in}}%
\pgfusepath{clip}%
\pgfsetbuttcap%
\pgfsetmiterjoin%
\definecolor{currentfill}{rgb}{0.511253,0.510898,0.193296}%
\pgfsetfillcolor{currentfill}%
\pgfsetlinewidth{0.000000pt}%
\definecolor{currentstroke}{rgb}{0.000000,0.000000,0.000000}%
\pgfsetstrokecolor{currentstroke}%
\pgfsetstrokeopacity{0.000000}%
\pgfsetdash{}{0pt}%
\pgfpathmoveto{\pgfqpoint{2.475349in}{1.598018in}}%
\pgfpathlineto{\pgfqpoint{2.484103in}{1.598018in}}%
\pgfpathlineto{\pgfqpoint{2.484103in}{1.528004in}}%
\pgfpathlineto{\pgfqpoint{2.475349in}{1.528004in}}%
\pgfpathlineto{\pgfqpoint{2.475349in}{1.598018in}}%
\pgfpathclose%
\pgfusepath{fill}%
\end{pgfscope}%
\begin{pgfscope}%
\pgfpathrectangle{\pgfqpoint{0.804646in}{0.600000in}}{\pgfqpoint{2.573292in}{2.070576in}}%
\pgfusepath{clip}%
\pgfsetbuttcap%
\pgfsetmiterjoin%
\definecolor{currentfill}{rgb}{0.511253,0.510898,0.193296}%
\pgfsetfillcolor{currentfill}%
\pgfsetlinewidth{0.000000pt}%
\definecolor{currentstroke}{rgb}{0.000000,0.000000,0.000000}%
\pgfsetstrokecolor{currentstroke}%
\pgfsetstrokeopacity{0.000000}%
\pgfsetdash{}{0pt}%
\pgfpathmoveto{\pgfqpoint{2.486291in}{1.600006in}}%
\pgfpathlineto{\pgfqpoint{2.495044in}{1.600006in}}%
\pgfpathlineto{\pgfqpoint{2.495044in}{1.575581in}}%
\pgfpathlineto{\pgfqpoint{2.486291in}{1.575581in}}%
\pgfpathlineto{\pgfqpoint{2.486291in}{1.600006in}}%
\pgfpathclose%
\pgfusepath{fill}%
\end{pgfscope}%
\begin{pgfscope}%
\pgfpathrectangle{\pgfqpoint{0.804646in}{0.600000in}}{\pgfqpoint{2.573292in}{2.070576in}}%
\pgfusepath{clip}%
\pgfsetbuttcap%
\pgfsetmiterjoin%
\definecolor{currentfill}{rgb}{0.511253,0.510898,0.193296}%
\pgfsetfillcolor{currentfill}%
\pgfsetlinewidth{0.000000pt}%
\definecolor{currentstroke}{rgb}{0.000000,0.000000,0.000000}%
\pgfsetstrokecolor{currentstroke}%
\pgfsetstrokeopacity{0.000000}%
\pgfsetdash{}{0pt}%
\pgfpathmoveto{\pgfqpoint{2.497233in}{1.613090in}}%
\pgfpathlineto{\pgfqpoint{2.505986in}{1.613090in}}%
\pgfpathlineto{\pgfqpoint{2.505986in}{1.571681in}}%
\pgfpathlineto{\pgfqpoint{2.497233in}{1.571681in}}%
\pgfpathlineto{\pgfqpoint{2.497233in}{1.613090in}}%
\pgfpathclose%
\pgfusepath{fill}%
\end{pgfscope}%
\begin{pgfscope}%
\pgfpathrectangle{\pgfqpoint{0.804646in}{0.600000in}}{\pgfqpoint{2.573292in}{2.070576in}}%
\pgfusepath{clip}%
\pgfsetbuttcap%
\pgfsetmiterjoin%
\definecolor{currentfill}{rgb}{0.511253,0.510898,0.193296}%
\pgfsetfillcolor{currentfill}%
\pgfsetlinewidth{0.000000pt}%
\definecolor{currentstroke}{rgb}{0.000000,0.000000,0.000000}%
\pgfsetstrokecolor{currentstroke}%
\pgfsetstrokeopacity{0.000000}%
\pgfsetdash{}{0pt}%
\pgfpathmoveto{\pgfqpoint{2.508174in}{1.613090in}}%
\pgfpathlineto{\pgfqpoint{2.516928in}{1.613090in}}%
\pgfpathlineto{\pgfqpoint{2.516928in}{1.565269in}}%
\pgfpathlineto{\pgfqpoint{2.508174in}{1.565269in}}%
\pgfpathlineto{\pgfqpoint{2.508174in}{1.613090in}}%
\pgfpathclose%
\pgfusepath{fill}%
\end{pgfscope}%
\begin{pgfscope}%
\pgfpathrectangle{\pgfqpoint{0.804646in}{0.600000in}}{\pgfqpoint{2.573292in}{2.070576in}}%
\pgfusepath{clip}%
\pgfsetbuttcap%
\pgfsetmiterjoin%
\definecolor{currentfill}{rgb}{0.511253,0.510898,0.193296}%
\pgfsetfillcolor{currentfill}%
\pgfsetlinewidth{0.000000pt}%
\definecolor{currentstroke}{rgb}{0.000000,0.000000,0.000000}%
\pgfsetstrokecolor{currentstroke}%
\pgfsetstrokeopacity{0.000000}%
\pgfsetdash{}{0pt}%
\pgfpathmoveto{\pgfqpoint{2.519116in}{1.613090in}}%
\pgfpathlineto{\pgfqpoint{2.527870in}{1.613090in}}%
\pgfpathlineto{\pgfqpoint{2.527870in}{1.543855in}}%
\pgfpathlineto{\pgfqpoint{2.519116in}{1.543855in}}%
\pgfpathlineto{\pgfqpoint{2.519116in}{1.613090in}}%
\pgfpathclose%
\pgfusepath{fill}%
\end{pgfscope}%
\begin{pgfscope}%
\pgfpathrectangle{\pgfqpoint{0.804646in}{0.600000in}}{\pgfqpoint{2.573292in}{2.070576in}}%
\pgfusepath{clip}%
\pgfsetbuttcap%
\pgfsetmiterjoin%
\definecolor{currentfill}{rgb}{0.511253,0.510898,0.193296}%
\pgfsetfillcolor{currentfill}%
\pgfsetlinewidth{0.000000pt}%
\definecolor{currentstroke}{rgb}{0.000000,0.000000,0.000000}%
\pgfsetstrokecolor{currentstroke}%
\pgfsetstrokeopacity{0.000000}%
\pgfsetdash{}{0pt}%
\pgfpathmoveto{\pgfqpoint{2.530058in}{1.775408in}}%
\pgfpathlineto{\pgfqpoint{2.538812in}{1.775408in}}%
\pgfpathlineto{\pgfqpoint{2.538812in}{1.776614in}}%
\pgfpathlineto{\pgfqpoint{2.530058in}{1.776614in}}%
\pgfpathlineto{\pgfqpoint{2.530058in}{1.775408in}}%
\pgfpathclose%
\pgfusepath{fill}%
\end{pgfscope}%
\begin{pgfscope}%
\pgfpathrectangle{\pgfqpoint{0.804646in}{0.600000in}}{\pgfqpoint{2.573292in}{2.070576in}}%
\pgfusepath{clip}%
\pgfsetbuttcap%
\pgfsetmiterjoin%
\definecolor{currentfill}{rgb}{0.511253,0.510898,0.193296}%
\pgfsetfillcolor{currentfill}%
\pgfsetlinewidth{0.000000pt}%
\definecolor{currentstroke}{rgb}{0.000000,0.000000,0.000000}%
\pgfsetstrokecolor{currentstroke}%
\pgfsetstrokeopacity{0.000000}%
\pgfsetdash{}{0pt}%
\pgfpathmoveto{\pgfqpoint{2.541000in}{1.613090in}}%
\pgfpathlineto{\pgfqpoint{2.549753in}{1.613090in}}%
\pgfpathlineto{\pgfqpoint{2.549753in}{1.612128in}}%
\pgfpathlineto{\pgfqpoint{2.541000in}{1.612128in}}%
\pgfpathlineto{\pgfqpoint{2.541000in}{1.613090in}}%
\pgfpathclose%
\pgfusepath{fill}%
\end{pgfscope}%
\begin{pgfscope}%
\pgfpathrectangle{\pgfqpoint{0.804646in}{0.600000in}}{\pgfqpoint{2.573292in}{2.070576in}}%
\pgfusepath{clip}%
\pgfsetbuttcap%
\pgfsetmiterjoin%
\definecolor{currentfill}{rgb}{0.511253,0.510898,0.193296}%
\pgfsetfillcolor{currentfill}%
\pgfsetlinewidth{0.000000pt}%
\definecolor{currentstroke}{rgb}{0.000000,0.000000,0.000000}%
\pgfsetstrokecolor{currentstroke}%
\pgfsetstrokeopacity{0.000000}%
\pgfsetdash{}{0pt}%
\pgfpathmoveto{\pgfqpoint{2.551942in}{1.707445in}}%
\pgfpathlineto{\pgfqpoint{2.560695in}{1.707445in}}%
\pgfpathlineto{\pgfqpoint{2.560695in}{1.823300in}}%
\pgfpathlineto{\pgfqpoint{2.551942in}{1.823300in}}%
\pgfpathlineto{\pgfqpoint{2.551942in}{1.707445in}}%
\pgfpathclose%
\pgfusepath{fill}%
\end{pgfscope}%
\begin{pgfscope}%
\pgfpathrectangle{\pgfqpoint{0.804646in}{0.600000in}}{\pgfqpoint{2.573292in}{2.070576in}}%
\pgfusepath{clip}%
\pgfsetbuttcap%
\pgfsetmiterjoin%
\definecolor{currentfill}{rgb}{0.511253,0.510898,0.193296}%
\pgfsetfillcolor{currentfill}%
\pgfsetlinewidth{0.000000pt}%
\definecolor{currentstroke}{rgb}{0.000000,0.000000,0.000000}%
\pgfsetstrokecolor{currentstroke}%
\pgfsetstrokeopacity{0.000000}%
\pgfsetdash{}{0pt}%
\pgfpathmoveto{\pgfqpoint{2.562883in}{1.700045in}}%
\pgfpathlineto{\pgfqpoint{2.571637in}{1.700045in}}%
\pgfpathlineto{\pgfqpoint{2.571637in}{1.822356in}}%
\pgfpathlineto{\pgfqpoint{2.562883in}{1.822356in}}%
\pgfpathlineto{\pgfqpoint{2.562883in}{1.700045in}}%
\pgfpathclose%
\pgfusepath{fill}%
\end{pgfscope}%
\begin{pgfscope}%
\pgfpathrectangle{\pgfqpoint{0.804646in}{0.600000in}}{\pgfqpoint{2.573292in}{2.070576in}}%
\pgfusepath{clip}%
\pgfsetbuttcap%
\pgfsetmiterjoin%
\definecolor{currentfill}{rgb}{0.511253,0.510898,0.193296}%
\pgfsetfillcolor{currentfill}%
\pgfsetlinewidth{0.000000pt}%
\definecolor{currentstroke}{rgb}{0.000000,0.000000,0.000000}%
\pgfsetstrokecolor{currentstroke}%
\pgfsetstrokeopacity{0.000000}%
\pgfsetdash{}{0pt}%
\pgfpathmoveto{\pgfqpoint{2.573825in}{1.684570in}}%
\pgfpathlineto{\pgfqpoint{2.582579in}{1.684570in}}%
\pgfpathlineto{\pgfqpoint{2.582579in}{1.864838in}}%
\pgfpathlineto{\pgfqpoint{2.573825in}{1.864838in}}%
\pgfpathlineto{\pgfqpoint{2.573825in}{1.684570in}}%
\pgfpathclose%
\pgfusepath{fill}%
\end{pgfscope}%
\begin{pgfscope}%
\pgfpathrectangle{\pgfqpoint{0.804646in}{0.600000in}}{\pgfqpoint{2.573292in}{2.070576in}}%
\pgfusepath{clip}%
\pgfsetbuttcap%
\pgfsetmiterjoin%
\definecolor{currentfill}{rgb}{0.511253,0.510898,0.193296}%
\pgfsetfillcolor{currentfill}%
\pgfsetlinewidth{0.000000pt}%
\definecolor{currentstroke}{rgb}{0.000000,0.000000,0.000000}%
\pgfsetstrokecolor{currentstroke}%
\pgfsetstrokeopacity{0.000000}%
\pgfsetdash{}{0pt}%
\pgfpathmoveto{\pgfqpoint{2.584767in}{1.684842in}}%
\pgfpathlineto{\pgfqpoint{2.593521in}{1.684842in}}%
\pgfpathlineto{\pgfqpoint{2.593521in}{1.839406in}}%
\pgfpathlineto{\pgfqpoint{2.584767in}{1.839406in}}%
\pgfpathlineto{\pgfqpoint{2.584767in}{1.684842in}}%
\pgfpathclose%
\pgfusepath{fill}%
\end{pgfscope}%
\begin{pgfscope}%
\pgfpathrectangle{\pgfqpoint{0.804646in}{0.600000in}}{\pgfqpoint{2.573292in}{2.070576in}}%
\pgfusepath{clip}%
\pgfsetbuttcap%
\pgfsetmiterjoin%
\definecolor{currentfill}{rgb}{0.511253,0.510898,0.193296}%
\pgfsetfillcolor{currentfill}%
\pgfsetlinewidth{0.000000pt}%
\definecolor{currentstroke}{rgb}{0.000000,0.000000,0.000000}%
\pgfsetstrokecolor{currentstroke}%
\pgfsetstrokeopacity{0.000000}%
\pgfsetdash{}{0pt}%
\pgfpathmoveto{\pgfqpoint{2.595709in}{1.680306in}}%
\pgfpathlineto{\pgfqpoint{2.604462in}{1.680306in}}%
\pgfpathlineto{\pgfqpoint{2.604462in}{1.883846in}}%
\pgfpathlineto{\pgfqpoint{2.595709in}{1.883846in}}%
\pgfpathlineto{\pgfqpoint{2.595709in}{1.680306in}}%
\pgfpathclose%
\pgfusepath{fill}%
\end{pgfscope}%
\begin{pgfscope}%
\pgfpathrectangle{\pgfqpoint{0.804646in}{0.600000in}}{\pgfqpoint{2.573292in}{2.070576in}}%
\pgfusepath{clip}%
\pgfsetbuttcap%
\pgfsetmiterjoin%
\definecolor{currentfill}{rgb}{0.511253,0.510898,0.193296}%
\pgfsetfillcolor{currentfill}%
\pgfsetlinewidth{0.000000pt}%
\definecolor{currentstroke}{rgb}{0.000000,0.000000,0.000000}%
\pgfsetstrokecolor{currentstroke}%
\pgfsetstrokeopacity{0.000000}%
\pgfsetdash{}{0pt}%
\pgfpathmoveto{\pgfqpoint{2.606651in}{1.676290in}}%
\pgfpathlineto{\pgfqpoint{2.615404in}{1.676290in}}%
\pgfpathlineto{\pgfqpoint{2.615404in}{1.904963in}}%
\pgfpathlineto{\pgfqpoint{2.606651in}{1.904963in}}%
\pgfpathlineto{\pgfqpoint{2.606651in}{1.676290in}}%
\pgfpathclose%
\pgfusepath{fill}%
\end{pgfscope}%
\begin{pgfscope}%
\pgfpathrectangle{\pgfqpoint{0.804646in}{0.600000in}}{\pgfqpoint{2.573292in}{2.070576in}}%
\pgfusepath{clip}%
\pgfsetbuttcap%
\pgfsetmiterjoin%
\definecolor{currentfill}{rgb}{0.511253,0.510898,0.193296}%
\pgfsetfillcolor{currentfill}%
\pgfsetlinewidth{0.000000pt}%
\definecolor{currentstroke}{rgb}{0.000000,0.000000,0.000000}%
\pgfsetstrokecolor{currentstroke}%
\pgfsetstrokeopacity{0.000000}%
\pgfsetdash{}{0pt}%
\pgfpathmoveto{\pgfqpoint{2.617592in}{1.690828in}}%
\pgfpathlineto{\pgfqpoint{2.626346in}{1.690828in}}%
\pgfpathlineto{\pgfqpoint{2.626346in}{1.890252in}}%
\pgfpathlineto{\pgfqpoint{2.617592in}{1.890252in}}%
\pgfpathlineto{\pgfqpoint{2.617592in}{1.690828in}}%
\pgfpathclose%
\pgfusepath{fill}%
\end{pgfscope}%
\begin{pgfscope}%
\pgfpathrectangle{\pgfqpoint{0.804646in}{0.600000in}}{\pgfqpoint{2.573292in}{2.070576in}}%
\pgfusepath{clip}%
\pgfsetbuttcap%
\pgfsetmiterjoin%
\definecolor{currentfill}{rgb}{0.511253,0.510898,0.193296}%
\pgfsetfillcolor{currentfill}%
\pgfsetlinewidth{0.000000pt}%
\definecolor{currentstroke}{rgb}{0.000000,0.000000,0.000000}%
\pgfsetstrokecolor{currentstroke}%
\pgfsetstrokeopacity{0.000000}%
\pgfsetdash{}{0pt}%
\pgfpathmoveto{\pgfqpoint{2.628534in}{1.688808in}}%
\pgfpathlineto{\pgfqpoint{2.637288in}{1.688808in}}%
\pgfpathlineto{\pgfqpoint{2.637288in}{1.928086in}}%
\pgfpathlineto{\pgfqpoint{2.628534in}{1.928086in}}%
\pgfpathlineto{\pgfqpoint{2.628534in}{1.688808in}}%
\pgfpathclose%
\pgfusepath{fill}%
\end{pgfscope}%
\begin{pgfscope}%
\pgfpathrectangle{\pgfqpoint{0.804646in}{0.600000in}}{\pgfqpoint{2.573292in}{2.070576in}}%
\pgfusepath{clip}%
\pgfsetbuttcap%
\pgfsetmiterjoin%
\definecolor{currentfill}{rgb}{0.511253,0.510898,0.193296}%
\pgfsetfillcolor{currentfill}%
\pgfsetlinewidth{0.000000pt}%
\definecolor{currentstroke}{rgb}{0.000000,0.000000,0.000000}%
\pgfsetstrokecolor{currentstroke}%
\pgfsetstrokeopacity{0.000000}%
\pgfsetdash{}{0pt}%
\pgfpathmoveto{\pgfqpoint{2.639476in}{1.693757in}}%
\pgfpathlineto{\pgfqpoint{2.648230in}{1.693757in}}%
\pgfpathlineto{\pgfqpoint{2.648230in}{1.935934in}}%
\pgfpathlineto{\pgfqpoint{2.639476in}{1.935934in}}%
\pgfpathlineto{\pgfqpoint{2.639476in}{1.693757in}}%
\pgfpathclose%
\pgfusepath{fill}%
\end{pgfscope}%
\begin{pgfscope}%
\pgfpathrectangle{\pgfqpoint{0.804646in}{0.600000in}}{\pgfqpoint{2.573292in}{2.070576in}}%
\pgfusepath{clip}%
\pgfsetbuttcap%
\pgfsetmiterjoin%
\definecolor{currentfill}{rgb}{0.511253,0.510898,0.193296}%
\pgfsetfillcolor{currentfill}%
\pgfsetlinewidth{0.000000pt}%
\definecolor{currentstroke}{rgb}{0.000000,0.000000,0.000000}%
\pgfsetstrokecolor{currentstroke}%
\pgfsetstrokeopacity{0.000000}%
\pgfsetdash{}{0pt}%
\pgfpathmoveto{\pgfqpoint{2.650418in}{1.710928in}}%
\pgfpathlineto{\pgfqpoint{2.659171in}{1.710928in}}%
\pgfpathlineto{\pgfqpoint{2.659171in}{1.910910in}}%
\pgfpathlineto{\pgfqpoint{2.650418in}{1.910910in}}%
\pgfpathlineto{\pgfqpoint{2.650418in}{1.710928in}}%
\pgfpathclose%
\pgfusepath{fill}%
\end{pgfscope}%
\begin{pgfscope}%
\pgfpathrectangle{\pgfqpoint{0.804646in}{0.600000in}}{\pgfqpoint{2.573292in}{2.070576in}}%
\pgfusepath{clip}%
\pgfsetbuttcap%
\pgfsetmiterjoin%
\definecolor{currentfill}{rgb}{0.511253,0.510898,0.193296}%
\pgfsetfillcolor{currentfill}%
\pgfsetlinewidth{0.000000pt}%
\definecolor{currentstroke}{rgb}{0.000000,0.000000,0.000000}%
\pgfsetstrokecolor{currentstroke}%
\pgfsetstrokeopacity{0.000000}%
\pgfsetdash{}{0pt}%
\pgfpathmoveto{\pgfqpoint{2.661360in}{1.705056in}}%
\pgfpathlineto{\pgfqpoint{2.670113in}{1.705056in}}%
\pgfpathlineto{\pgfqpoint{2.670113in}{1.824781in}}%
\pgfpathlineto{\pgfqpoint{2.661360in}{1.824781in}}%
\pgfpathlineto{\pgfqpoint{2.661360in}{1.705056in}}%
\pgfpathclose%
\pgfusepath{fill}%
\end{pgfscope}%
\begin{pgfscope}%
\pgfpathrectangle{\pgfqpoint{0.804646in}{0.600000in}}{\pgfqpoint{2.573292in}{2.070576in}}%
\pgfusepath{clip}%
\pgfsetbuttcap%
\pgfsetmiterjoin%
\definecolor{currentfill}{rgb}{0.511253,0.510898,0.193296}%
\pgfsetfillcolor{currentfill}%
\pgfsetlinewidth{0.000000pt}%
\definecolor{currentstroke}{rgb}{0.000000,0.000000,0.000000}%
\pgfsetstrokecolor{currentstroke}%
\pgfsetstrokeopacity{0.000000}%
\pgfsetdash{}{0pt}%
\pgfpathmoveto{\pgfqpoint{2.672301in}{1.720062in}}%
\pgfpathlineto{\pgfqpoint{2.681055in}{1.720062in}}%
\pgfpathlineto{\pgfqpoint{2.681055in}{1.827927in}}%
\pgfpathlineto{\pgfqpoint{2.672301in}{1.827927in}}%
\pgfpathlineto{\pgfqpoint{2.672301in}{1.720062in}}%
\pgfpathclose%
\pgfusepath{fill}%
\end{pgfscope}%
\begin{pgfscope}%
\pgfpathrectangle{\pgfqpoint{0.804646in}{0.600000in}}{\pgfqpoint{2.573292in}{2.070576in}}%
\pgfusepath{clip}%
\pgfsetbuttcap%
\pgfsetmiterjoin%
\definecolor{currentfill}{rgb}{0.511253,0.510898,0.193296}%
\pgfsetfillcolor{currentfill}%
\pgfsetlinewidth{0.000000pt}%
\definecolor{currentstroke}{rgb}{0.000000,0.000000,0.000000}%
\pgfsetstrokecolor{currentstroke}%
\pgfsetstrokeopacity{0.000000}%
\pgfsetdash{}{0pt}%
\pgfpathmoveto{\pgfqpoint{2.683243in}{1.723966in}}%
\pgfpathlineto{\pgfqpoint{2.691997in}{1.723966in}}%
\pgfpathlineto{\pgfqpoint{2.691997in}{1.732406in}}%
\pgfpathlineto{\pgfqpoint{2.683243in}{1.732406in}}%
\pgfpathlineto{\pgfqpoint{2.683243in}{1.723966in}}%
\pgfpathclose%
\pgfusepath{fill}%
\end{pgfscope}%
\begin{pgfscope}%
\pgfpathrectangle{\pgfqpoint{0.804646in}{0.600000in}}{\pgfqpoint{2.573292in}{2.070576in}}%
\pgfusepath{clip}%
\pgfsetbuttcap%
\pgfsetmiterjoin%
\definecolor{currentfill}{rgb}{0.511253,0.510898,0.193296}%
\pgfsetfillcolor{currentfill}%
\pgfsetlinewidth{0.000000pt}%
\definecolor{currentstroke}{rgb}{0.000000,0.000000,0.000000}%
\pgfsetstrokecolor{currentstroke}%
\pgfsetstrokeopacity{0.000000}%
\pgfsetdash{}{0pt}%
\pgfpathmoveto{\pgfqpoint{2.694185in}{1.704400in}}%
\pgfpathlineto{\pgfqpoint{2.702939in}{1.704400in}}%
\pgfpathlineto{\pgfqpoint{2.702939in}{1.804584in}}%
\pgfpathlineto{\pgfqpoint{2.694185in}{1.804584in}}%
\pgfpathlineto{\pgfqpoint{2.694185in}{1.704400in}}%
\pgfpathclose%
\pgfusepath{fill}%
\end{pgfscope}%
\begin{pgfscope}%
\pgfpathrectangle{\pgfqpoint{0.804646in}{0.600000in}}{\pgfqpoint{2.573292in}{2.070576in}}%
\pgfusepath{clip}%
\pgfsetbuttcap%
\pgfsetmiterjoin%
\definecolor{currentfill}{rgb}{0.511253,0.510898,0.193296}%
\pgfsetfillcolor{currentfill}%
\pgfsetlinewidth{0.000000pt}%
\definecolor{currentstroke}{rgb}{0.000000,0.000000,0.000000}%
\pgfsetstrokecolor{currentstroke}%
\pgfsetstrokeopacity{0.000000}%
\pgfsetdash{}{0pt}%
\pgfpathmoveto{\pgfqpoint{2.705127in}{1.707867in}}%
\pgfpathlineto{\pgfqpoint{2.713880in}{1.707867in}}%
\pgfpathlineto{\pgfqpoint{2.713880in}{1.708719in}}%
\pgfpathlineto{\pgfqpoint{2.705127in}{1.708719in}}%
\pgfpathlineto{\pgfqpoint{2.705127in}{1.707867in}}%
\pgfpathclose%
\pgfusepath{fill}%
\end{pgfscope}%
\begin{pgfscope}%
\pgfpathrectangle{\pgfqpoint{0.804646in}{0.600000in}}{\pgfqpoint{2.573292in}{2.070576in}}%
\pgfusepath{clip}%
\pgfsetbuttcap%
\pgfsetmiterjoin%
\definecolor{currentfill}{rgb}{0.511253,0.510898,0.193296}%
\pgfsetfillcolor{currentfill}%
\pgfsetlinewidth{0.000000pt}%
\definecolor{currentstroke}{rgb}{0.000000,0.000000,0.000000}%
\pgfsetstrokecolor{currentstroke}%
\pgfsetstrokeopacity{0.000000}%
\pgfsetdash{}{0pt}%
\pgfpathmoveto{\pgfqpoint{2.716069in}{1.596456in}}%
\pgfpathlineto{\pgfqpoint{2.724822in}{1.596456in}}%
\pgfpathlineto{\pgfqpoint{2.724822in}{1.565276in}}%
\pgfpathlineto{\pgfqpoint{2.716069in}{1.565276in}}%
\pgfpathlineto{\pgfqpoint{2.716069in}{1.596456in}}%
\pgfpathclose%
\pgfusepath{fill}%
\end{pgfscope}%
\begin{pgfscope}%
\pgfpathrectangle{\pgfqpoint{0.804646in}{0.600000in}}{\pgfqpoint{2.573292in}{2.070576in}}%
\pgfusepath{clip}%
\pgfsetbuttcap%
\pgfsetmiterjoin%
\definecolor{currentfill}{rgb}{0.511253,0.510898,0.193296}%
\pgfsetfillcolor{currentfill}%
\pgfsetlinewidth{0.000000pt}%
\definecolor{currentstroke}{rgb}{0.000000,0.000000,0.000000}%
\pgfsetstrokecolor{currentstroke}%
\pgfsetstrokeopacity{0.000000}%
\pgfsetdash{}{0pt}%
\pgfpathmoveto{\pgfqpoint{2.727010in}{1.603503in}}%
\pgfpathlineto{\pgfqpoint{2.735764in}{1.603503in}}%
\pgfpathlineto{\pgfqpoint{2.735764in}{1.588316in}}%
\pgfpathlineto{\pgfqpoint{2.727010in}{1.588316in}}%
\pgfpathlineto{\pgfqpoint{2.727010in}{1.603503in}}%
\pgfpathclose%
\pgfusepath{fill}%
\end{pgfscope}%
\begin{pgfscope}%
\pgfpathrectangle{\pgfqpoint{0.804646in}{0.600000in}}{\pgfqpoint{2.573292in}{2.070576in}}%
\pgfusepath{clip}%
\pgfsetbuttcap%
\pgfsetmiterjoin%
\definecolor{currentfill}{rgb}{0.511253,0.510898,0.193296}%
\pgfsetfillcolor{currentfill}%
\pgfsetlinewidth{0.000000pt}%
\definecolor{currentstroke}{rgb}{0.000000,0.000000,0.000000}%
\pgfsetstrokecolor{currentstroke}%
\pgfsetstrokeopacity{0.000000}%
\pgfsetdash{}{0pt}%
\pgfpathmoveto{\pgfqpoint{2.737952in}{1.613090in}}%
\pgfpathlineto{\pgfqpoint{2.746706in}{1.613090in}}%
\pgfpathlineto{\pgfqpoint{2.746706in}{1.573273in}}%
\pgfpathlineto{\pgfqpoint{2.737952in}{1.573273in}}%
\pgfpathlineto{\pgfqpoint{2.737952in}{1.613090in}}%
\pgfpathclose%
\pgfusepath{fill}%
\end{pgfscope}%
\begin{pgfscope}%
\pgfpathrectangle{\pgfqpoint{0.804646in}{0.600000in}}{\pgfqpoint{2.573292in}{2.070576in}}%
\pgfusepath{clip}%
\pgfsetbuttcap%
\pgfsetmiterjoin%
\definecolor{currentfill}{rgb}{0.511253,0.510898,0.193296}%
\pgfsetfillcolor{currentfill}%
\pgfsetlinewidth{0.000000pt}%
\definecolor{currentstroke}{rgb}{0.000000,0.000000,0.000000}%
\pgfsetstrokecolor{currentstroke}%
\pgfsetstrokeopacity{0.000000}%
\pgfsetdash{}{0pt}%
\pgfpathmoveto{\pgfqpoint{2.748894in}{1.613090in}}%
\pgfpathlineto{\pgfqpoint{2.757648in}{1.613090in}}%
\pgfpathlineto{\pgfqpoint{2.757648in}{1.547778in}}%
\pgfpathlineto{\pgfqpoint{2.748894in}{1.547778in}}%
\pgfpathlineto{\pgfqpoint{2.748894in}{1.613090in}}%
\pgfpathclose%
\pgfusepath{fill}%
\end{pgfscope}%
\begin{pgfscope}%
\pgfpathrectangle{\pgfqpoint{0.804646in}{0.600000in}}{\pgfqpoint{2.573292in}{2.070576in}}%
\pgfusepath{clip}%
\pgfsetbuttcap%
\pgfsetmiterjoin%
\definecolor{currentfill}{rgb}{0.511253,0.510898,0.193296}%
\pgfsetfillcolor{currentfill}%
\pgfsetlinewidth{0.000000pt}%
\definecolor{currentstroke}{rgb}{0.000000,0.000000,0.000000}%
\pgfsetstrokecolor{currentstroke}%
\pgfsetstrokeopacity{0.000000}%
\pgfsetdash{}{0pt}%
\pgfpathmoveto{\pgfqpoint{2.759836in}{1.699399in}}%
\pgfpathlineto{\pgfqpoint{2.768589in}{1.699399in}}%
\pgfpathlineto{\pgfqpoint{2.768589in}{1.780058in}}%
\pgfpathlineto{\pgfqpoint{2.759836in}{1.780058in}}%
\pgfpathlineto{\pgfqpoint{2.759836in}{1.699399in}}%
\pgfpathclose%
\pgfusepath{fill}%
\end{pgfscope}%
\begin{pgfscope}%
\pgfpathrectangle{\pgfqpoint{0.804646in}{0.600000in}}{\pgfqpoint{2.573292in}{2.070576in}}%
\pgfusepath{clip}%
\pgfsetbuttcap%
\pgfsetmiterjoin%
\definecolor{currentfill}{rgb}{0.511253,0.510898,0.193296}%
\pgfsetfillcolor{currentfill}%
\pgfsetlinewidth{0.000000pt}%
\definecolor{currentstroke}{rgb}{0.000000,0.000000,0.000000}%
\pgfsetstrokecolor{currentstroke}%
\pgfsetstrokeopacity{0.000000}%
\pgfsetdash{}{0pt}%
\pgfpathmoveto{\pgfqpoint{2.770778in}{1.770813in}}%
\pgfpathlineto{\pgfqpoint{2.779531in}{1.770813in}}%
\pgfpathlineto{\pgfqpoint{2.779531in}{1.789534in}}%
\pgfpathlineto{\pgfqpoint{2.770778in}{1.789534in}}%
\pgfpathlineto{\pgfqpoint{2.770778in}{1.770813in}}%
\pgfpathclose%
\pgfusepath{fill}%
\end{pgfscope}%
\begin{pgfscope}%
\pgfpathrectangle{\pgfqpoint{0.804646in}{0.600000in}}{\pgfqpoint{2.573292in}{2.070576in}}%
\pgfusepath{clip}%
\pgfsetbuttcap%
\pgfsetmiterjoin%
\definecolor{currentfill}{rgb}{0.511253,0.510898,0.193296}%
\pgfsetfillcolor{currentfill}%
\pgfsetlinewidth{0.000000pt}%
\definecolor{currentstroke}{rgb}{0.000000,0.000000,0.000000}%
\pgfsetstrokecolor{currentstroke}%
\pgfsetstrokeopacity{0.000000}%
\pgfsetdash{}{0pt}%
\pgfpathmoveto{\pgfqpoint{2.781719in}{1.459301in}}%
\pgfpathlineto{\pgfqpoint{2.790473in}{1.459301in}}%
\pgfpathlineto{\pgfqpoint{2.790473in}{1.367788in}}%
\pgfpathlineto{\pgfqpoint{2.781719in}{1.367788in}}%
\pgfpathlineto{\pgfqpoint{2.781719in}{1.459301in}}%
\pgfpathclose%
\pgfusepath{fill}%
\end{pgfscope}%
\begin{pgfscope}%
\pgfpathrectangle{\pgfqpoint{0.804646in}{0.600000in}}{\pgfqpoint{2.573292in}{2.070576in}}%
\pgfusepath{clip}%
\pgfsetbuttcap%
\pgfsetmiterjoin%
\definecolor{currentfill}{rgb}{0.511253,0.510898,0.193296}%
\pgfsetfillcolor{currentfill}%
\pgfsetlinewidth{0.000000pt}%
\definecolor{currentstroke}{rgb}{0.000000,0.000000,0.000000}%
\pgfsetstrokecolor{currentstroke}%
\pgfsetstrokeopacity{0.000000}%
\pgfsetdash{}{0pt}%
\pgfpathmoveto{\pgfqpoint{2.792661in}{1.394626in}}%
\pgfpathlineto{\pgfqpoint{2.801415in}{1.394626in}}%
\pgfpathlineto{\pgfqpoint{2.801415in}{1.272819in}}%
\pgfpathlineto{\pgfqpoint{2.792661in}{1.272819in}}%
\pgfpathlineto{\pgfqpoint{2.792661in}{1.394626in}}%
\pgfpathclose%
\pgfusepath{fill}%
\end{pgfscope}%
\begin{pgfscope}%
\pgfpathrectangle{\pgfqpoint{0.804646in}{0.600000in}}{\pgfqpoint{2.573292in}{2.070576in}}%
\pgfusepath{clip}%
\pgfsetbuttcap%
\pgfsetmiterjoin%
\definecolor{currentfill}{rgb}{0.511253,0.510898,0.193296}%
\pgfsetfillcolor{currentfill}%
\pgfsetlinewidth{0.000000pt}%
\definecolor{currentstroke}{rgb}{0.000000,0.000000,0.000000}%
\pgfsetstrokecolor{currentstroke}%
\pgfsetstrokeopacity{0.000000}%
\pgfsetdash{}{0pt}%
\pgfpathmoveto{\pgfqpoint{2.803603in}{1.364800in}}%
\pgfpathlineto{\pgfqpoint{2.812357in}{1.364800in}}%
\pgfpathlineto{\pgfqpoint{2.812357in}{1.314975in}}%
\pgfpathlineto{\pgfqpoint{2.803603in}{1.314975in}}%
\pgfpathlineto{\pgfqpoint{2.803603in}{1.364800in}}%
\pgfpathclose%
\pgfusepath{fill}%
\end{pgfscope}%
\begin{pgfscope}%
\pgfpathrectangle{\pgfqpoint{0.804646in}{0.600000in}}{\pgfqpoint{2.573292in}{2.070576in}}%
\pgfusepath{clip}%
\pgfsetbuttcap%
\pgfsetmiterjoin%
\definecolor{currentfill}{rgb}{0.511253,0.510898,0.193296}%
\pgfsetfillcolor{currentfill}%
\pgfsetlinewidth{0.000000pt}%
\definecolor{currentstroke}{rgb}{0.000000,0.000000,0.000000}%
\pgfsetstrokecolor{currentstroke}%
\pgfsetstrokeopacity{0.000000}%
\pgfsetdash{}{0pt}%
\pgfpathmoveto{\pgfqpoint{2.814545in}{1.344433in}}%
\pgfpathlineto{\pgfqpoint{2.823298in}{1.344433in}}%
\pgfpathlineto{\pgfqpoint{2.823298in}{1.280827in}}%
\pgfpathlineto{\pgfqpoint{2.814545in}{1.280827in}}%
\pgfpathlineto{\pgfqpoint{2.814545in}{1.344433in}}%
\pgfpathclose%
\pgfusepath{fill}%
\end{pgfscope}%
\begin{pgfscope}%
\pgfpathrectangle{\pgfqpoint{0.804646in}{0.600000in}}{\pgfqpoint{2.573292in}{2.070576in}}%
\pgfusepath{clip}%
\pgfsetbuttcap%
\pgfsetmiterjoin%
\definecolor{currentfill}{rgb}{0.511253,0.510898,0.193296}%
\pgfsetfillcolor{currentfill}%
\pgfsetlinewidth{0.000000pt}%
\definecolor{currentstroke}{rgb}{0.000000,0.000000,0.000000}%
\pgfsetstrokecolor{currentstroke}%
\pgfsetstrokeopacity{0.000000}%
\pgfsetdash{}{0pt}%
\pgfpathmoveto{\pgfqpoint{2.825487in}{1.344663in}}%
\pgfpathlineto{\pgfqpoint{2.834240in}{1.344663in}}%
\pgfpathlineto{\pgfqpoint{2.834240in}{1.246157in}}%
\pgfpathlineto{\pgfqpoint{2.825487in}{1.246157in}}%
\pgfpathlineto{\pgfqpoint{2.825487in}{1.344663in}}%
\pgfpathclose%
\pgfusepath{fill}%
\end{pgfscope}%
\begin{pgfscope}%
\pgfpathrectangle{\pgfqpoint{0.804646in}{0.600000in}}{\pgfqpoint{2.573292in}{2.070576in}}%
\pgfusepath{clip}%
\pgfsetbuttcap%
\pgfsetmiterjoin%
\definecolor{currentfill}{rgb}{0.511253,0.510898,0.193296}%
\pgfsetfillcolor{currentfill}%
\pgfsetlinewidth{0.000000pt}%
\definecolor{currentstroke}{rgb}{0.000000,0.000000,0.000000}%
\pgfsetstrokecolor{currentstroke}%
\pgfsetstrokeopacity{0.000000}%
\pgfsetdash{}{0pt}%
\pgfpathmoveto{\pgfqpoint{2.836428in}{1.354251in}}%
\pgfpathlineto{\pgfqpoint{2.845182in}{1.354251in}}%
\pgfpathlineto{\pgfqpoint{2.845182in}{1.297926in}}%
\pgfpathlineto{\pgfqpoint{2.836428in}{1.297926in}}%
\pgfpathlineto{\pgfqpoint{2.836428in}{1.354251in}}%
\pgfpathclose%
\pgfusepath{fill}%
\end{pgfscope}%
\begin{pgfscope}%
\pgfpathrectangle{\pgfqpoint{0.804646in}{0.600000in}}{\pgfqpoint{2.573292in}{2.070576in}}%
\pgfusepath{clip}%
\pgfsetbuttcap%
\pgfsetmiterjoin%
\definecolor{currentfill}{rgb}{0.511253,0.510898,0.193296}%
\pgfsetfillcolor{currentfill}%
\pgfsetlinewidth{0.000000pt}%
\definecolor{currentstroke}{rgb}{0.000000,0.000000,0.000000}%
\pgfsetstrokecolor{currentstroke}%
\pgfsetstrokeopacity{0.000000}%
\pgfsetdash{}{0pt}%
\pgfpathmoveto{\pgfqpoint{2.847370in}{1.357429in}}%
\pgfpathlineto{\pgfqpoint{2.856124in}{1.357429in}}%
\pgfpathlineto{\pgfqpoint{2.856124in}{1.260588in}}%
\pgfpathlineto{\pgfqpoint{2.847370in}{1.260588in}}%
\pgfpathlineto{\pgfqpoint{2.847370in}{1.357429in}}%
\pgfpathclose%
\pgfusepath{fill}%
\end{pgfscope}%
\begin{pgfscope}%
\pgfpathrectangle{\pgfqpoint{0.804646in}{0.600000in}}{\pgfqpoint{2.573292in}{2.070576in}}%
\pgfusepath{clip}%
\pgfsetbuttcap%
\pgfsetmiterjoin%
\definecolor{currentfill}{rgb}{0.511253,0.510898,0.193296}%
\pgfsetfillcolor{currentfill}%
\pgfsetlinewidth{0.000000pt}%
\definecolor{currentstroke}{rgb}{0.000000,0.000000,0.000000}%
\pgfsetstrokecolor{currentstroke}%
\pgfsetstrokeopacity{0.000000}%
\pgfsetdash{}{0pt}%
\pgfpathmoveto{\pgfqpoint{2.858312in}{1.350546in}}%
\pgfpathlineto{\pgfqpoint{2.867066in}{1.350546in}}%
\pgfpathlineto{\pgfqpoint{2.867066in}{1.263614in}}%
\pgfpathlineto{\pgfqpoint{2.858312in}{1.263614in}}%
\pgfpathlineto{\pgfqpoint{2.858312in}{1.350546in}}%
\pgfpathclose%
\pgfusepath{fill}%
\end{pgfscope}%
\begin{pgfscope}%
\pgfpathrectangle{\pgfqpoint{0.804646in}{0.600000in}}{\pgfqpoint{2.573292in}{2.070576in}}%
\pgfusepath{clip}%
\pgfsetbuttcap%
\pgfsetmiterjoin%
\definecolor{currentfill}{rgb}{0.511253,0.510898,0.193296}%
\pgfsetfillcolor{currentfill}%
\pgfsetlinewidth{0.000000pt}%
\definecolor{currentstroke}{rgb}{0.000000,0.000000,0.000000}%
\pgfsetstrokecolor{currentstroke}%
\pgfsetstrokeopacity{0.000000}%
\pgfsetdash{}{0pt}%
\pgfpathmoveto{\pgfqpoint{2.869254in}{1.355365in}}%
\pgfpathlineto{\pgfqpoint{2.878007in}{1.355365in}}%
\pgfpathlineto{\pgfqpoint{2.878007in}{1.233079in}}%
\pgfpathlineto{\pgfqpoint{2.869254in}{1.233079in}}%
\pgfpathlineto{\pgfqpoint{2.869254in}{1.355365in}}%
\pgfpathclose%
\pgfusepath{fill}%
\end{pgfscope}%
\begin{pgfscope}%
\pgfpathrectangle{\pgfqpoint{0.804646in}{0.600000in}}{\pgfqpoint{2.573292in}{2.070576in}}%
\pgfusepath{clip}%
\pgfsetbuttcap%
\pgfsetmiterjoin%
\definecolor{currentfill}{rgb}{0.511253,0.510898,0.193296}%
\pgfsetfillcolor{currentfill}%
\pgfsetlinewidth{0.000000pt}%
\definecolor{currentstroke}{rgb}{0.000000,0.000000,0.000000}%
\pgfsetstrokecolor{currentstroke}%
\pgfsetstrokeopacity{0.000000}%
\pgfsetdash{}{0pt}%
\pgfpathmoveto{\pgfqpoint{2.880196in}{1.369010in}}%
\pgfpathlineto{\pgfqpoint{2.888949in}{1.369010in}}%
\pgfpathlineto{\pgfqpoint{2.888949in}{1.308210in}}%
\pgfpathlineto{\pgfqpoint{2.880196in}{1.308210in}}%
\pgfpathlineto{\pgfqpoint{2.880196in}{1.369010in}}%
\pgfpathclose%
\pgfusepath{fill}%
\end{pgfscope}%
\begin{pgfscope}%
\pgfpathrectangle{\pgfqpoint{0.804646in}{0.600000in}}{\pgfqpoint{2.573292in}{2.070576in}}%
\pgfusepath{clip}%
\pgfsetbuttcap%
\pgfsetmiterjoin%
\definecolor{currentfill}{rgb}{0.511253,0.510898,0.193296}%
\pgfsetfillcolor{currentfill}%
\pgfsetlinewidth{0.000000pt}%
\definecolor{currentstroke}{rgb}{0.000000,0.000000,0.000000}%
\pgfsetstrokecolor{currentstroke}%
\pgfsetstrokeopacity{0.000000}%
\pgfsetdash{}{0pt}%
\pgfpathmoveto{\pgfqpoint{2.891137in}{1.358106in}}%
\pgfpathlineto{\pgfqpoint{2.899891in}{1.358106in}}%
\pgfpathlineto{\pgfqpoint{2.899891in}{1.319906in}}%
\pgfpathlineto{\pgfqpoint{2.891137in}{1.319906in}}%
\pgfpathlineto{\pgfqpoint{2.891137in}{1.358106in}}%
\pgfpathclose%
\pgfusepath{fill}%
\end{pgfscope}%
\begin{pgfscope}%
\pgfpathrectangle{\pgfqpoint{0.804646in}{0.600000in}}{\pgfqpoint{2.573292in}{2.070576in}}%
\pgfusepath{clip}%
\pgfsetbuttcap%
\pgfsetmiterjoin%
\definecolor{currentfill}{rgb}{0.511253,0.510898,0.193296}%
\pgfsetfillcolor{currentfill}%
\pgfsetlinewidth{0.000000pt}%
\definecolor{currentstroke}{rgb}{0.000000,0.000000,0.000000}%
\pgfsetstrokecolor{currentstroke}%
\pgfsetstrokeopacity{0.000000}%
\pgfsetdash{}{0pt}%
\pgfpathmoveto{\pgfqpoint{2.902079in}{1.374011in}}%
\pgfpathlineto{\pgfqpoint{2.910833in}{1.374011in}}%
\pgfpathlineto{\pgfqpoint{2.910833in}{1.238649in}}%
\pgfpathlineto{\pgfqpoint{2.902079in}{1.238649in}}%
\pgfpathlineto{\pgfqpoint{2.902079in}{1.374011in}}%
\pgfpathclose%
\pgfusepath{fill}%
\end{pgfscope}%
\begin{pgfscope}%
\pgfpathrectangle{\pgfqpoint{0.804646in}{0.600000in}}{\pgfqpoint{2.573292in}{2.070576in}}%
\pgfusepath{clip}%
\pgfsetbuttcap%
\pgfsetmiterjoin%
\definecolor{currentfill}{rgb}{0.511253,0.510898,0.193296}%
\pgfsetfillcolor{currentfill}%
\pgfsetlinewidth{0.000000pt}%
\definecolor{currentstroke}{rgb}{0.000000,0.000000,0.000000}%
\pgfsetstrokecolor{currentstroke}%
\pgfsetstrokeopacity{0.000000}%
\pgfsetdash{}{0pt}%
\pgfpathmoveto{\pgfqpoint{2.913021in}{1.369645in}}%
\pgfpathlineto{\pgfqpoint{2.921774in}{1.369645in}}%
\pgfpathlineto{\pgfqpoint{2.921774in}{1.337766in}}%
\pgfpathlineto{\pgfqpoint{2.913021in}{1.337766in}}%
\pgfpathlineto{\pgfqpoint{2.913021in}{1.369645in}}%
\pgfpathclose%
\pgfusepath{fill}%
\end{pgfscope}%
\begin{pgfscope}%
\pgfpathrectangle{\pgfqpoint{0.804646in}{0.600000in}}{\pgfqpoint{2.573292in}{2.070576in}}%
\pgfusepath{clip}%
\pgfsetbuttcap%
\pgfsetmiterjoin%
\definecolor{currentfill}{rgb}{0.511253,0.510898,0.193296}%
\pgfsetfillcolor{currentfill}%
\pgfsetlinewidth{0.000000pt}%
\definecolor{currentstroke}{rgb}{0.000000,0.000000,0.000000}%
\pgfsetstrokecolor{currentstroke}%
\pgfsetstrokeopacity{0.000000}%
\pgfsetdash{}{0pt}%
\pgfpathmoveto{\pgfqpoint{2.923963in}{1.364709in}}%
\pgfpathlineto{\pgfqpoint{2.932716in}{1.364709in}}%
\pgfpathlineto{\pgfqpoint{2.932716in}{1.295673in}}%
\pgfpathlineto{\pgfqpoint{2.923963in}{1.295673in}}%
\pgfpathlineto{\pgfqpoint{2.923963in}{1.364709in}}%
\pgfpathclose%
\pgfusepath{fill}%
\end{pgfscope}%
\begin{pgfscope}%
\pgfpathrectangle{\pgfqpoint{0.804646in}{0.600000in}}{\pgfqpoint{2.573292in}{2.070576in}}%
\pgfusepath{clip}%
\pgfsetbuttcap%
\pgfsetmiterjoin%
\definecolor{currentfill}{rgb}{0.511253,0.510898,0.193296}%
\pgfsetfillcolor{currentfill}%
\pgfsetlinewidth{0.000000pt}%
\definecolor{currentstroke}{rgb}{0.000000,0.000000,0.000000}%
\pgfsetstrokecolor{currentstroke}%
\pgfsetstrokeopacity{0.000000}%
\pgfsetdash{}{0pt}%
\pgfpathmoveto{\pgfqpoint{2.934905in}{1.359650in}}%
\pgfpathlineto{\pgfqpoint{2.943658in}{1.359650in}}%
\pgfpathlineto{\pgfqpoint{2.943658in}{1.309241in}}%
\pgfpathlineto{\pgfqpoint{2.934905in}{1.309241in}}%
\pgfpathlineto{\pgfqpoint{2.934905in}{1.359650in}}%
\pgfpathclose%
\pgfusepath{fill}%
\end{pgfscope}%
\begin{pgfscope}%
\pgfpathrectangle{\pgfqpoint{0.804646in}{0.600000in}}{\pgfqpoint{2.573292in}{2.070576in}}%
\pgfusepath{clip}%
\pgfsetbuttcap%
\pgfsetmiterjoin%
\definecolor{currentfill}{rgb}{0.511253,0.510898,0.193296}%
\pgfsetfillcolor{currentfill}%
\pgfsetlinewidth{0.000000pt}%
\definecolor{currentstroke}{rgb}{0.000000,0.000000,0.000000}%
\pgfsetstrokecolor{currentstroke}%
\pgfsetstrokeopacity{0.000000}%
\pgfsetdash{}{0pt}%
\pgfpathmoveto{\pgfqpoint{2.945846in}{1.364689in}}%
\pgfpathlineto{\pgfqpoint{2.954600in}{1.364689in}}%
\pgfpathlineto{\pgfqpoint{2.954600in}{1.350061in}}%
\pgfpathlineto{\pgfqpoint{2.945846in}{1.350061in}}%
\pgfpathlineto{\pgfqpoint{2.945846in}{1.364689in}}%
\pgfpathclose%
\pgfusepath{fill}%
\end{pgfscope}%
\begin{pgfscope}%
\pgfpathrectangle{\pgfqpoint{0.804646in}{0.600000in}}{\pgfqpoint{2.573292in}{2.070576in}}%
\pgfusepath{clip}%
\pgfsetbuttcap%
\pgfsetmiterjoin%
\definecolor{currentfill}{rgb}{0.511253,0.510898,0.193296}%
\pgfsetfillcolor{currentfill}%
\pgfsetlinewidth{0.000000pt}%
\definecolor{currentstroke}{rgb}{0.000000,0.000000,0.000000}%
\pgfsetstrokecolor{currentstroke}%
\pgfsetstrokeopacity{0.000000}%
\pgfsetdash{}{0pt}%
\pgfpathmoveto{\pgfqpoint{2.956788in}{1.367302in}}%
\pgfpathlineto{\pgfqpoint{2.965542in}{1.367302in}}%
\pgfpathlineto{\pgfqpoint{2.965542in}{1.286815in}}%
\pgfpathlineto{\pgfqpoint{2.956788in}{1.286815in}}%
\pgfpathlineto{\pgfqpoint{2.956788in}{1.367302in}}%
\pgfpathclose%
\pgfusepath{fill}%
\end{pgfscope}%
\begin{pgfscope}%
\pgfpathrectangle{\pgfqpoint{0.804646in}{0.600000in}}{\pgfqpoint{2.573292in}{2.070576in}}%
\pgfusepath{clip}%
\pgfsetbuttcap%
\pgfsetmiterjoin%
\definecolor{currentfill}{rgb}{0.511253,0.510898,0.193296}%
\pgfsetfillcolor{currentfill}%
\pgfsetlinewidth{0.000000pt}%
\definecolor{currentstroke}{rgb}{0.000000,0.000000,0.000000}%
\pgfsetstrokecolor{currentstroke}%
\pgfsetstrokeopacity{0.000000}%
\pgfsetdash{}{0pt}%
\pgfpathmoveto{\pgfqpoint{2.967730in}{1.374519in}}%
\pgfpathlineto{\pgfqpoint{2.976483in}{1.374519in}}%
\pgfpathlineto{\pgfqpoint{2.976483in}{1.208218in}}%
\pgfpathlineto{\pgfqpoint{2.967730in}{1.208218in}}%
\pgfpathlineto{\pgfqpoint{2.967730in}{1.374519in}}%
\pgfpathclose%
\pgfusepath{fill}%
\end{pgfscope}%
\begin{pgfscope}%
\pgfpathrectangle{\pgfqpoint{0.804646in}{0.600000in}}{\pgfqpoint{2.573292in}{2.070576in}}%
\pgfusepath{clip}%
\pgfsetbuttcap%
\pgfsetmiterjoin%
\definecolor{currentfill}{rgb}{0.511253,0.510898,0.193296}%
\pgfsetfillcolor{currentfill}%
\pgfsetlinewidth{0.000000pt}%
\definecolor{currentstroke}{rgb}{0.000000,0.000000,0.000000}%
\pgfsetstrokecolor{currentstroke}%
\pgfsetstrokeopacity{0.000000}%
\pgfsetdash{}{0pt}%
\pgfpathmoveto{\pgfqpoint{2.978672in}{1.367724in}}%
\pgfpathlineto{\pgfqpoint{2.987425in}{1.367724in}}%
\pgfpathlineto{\pgfqpoint{2.987425in}{1.239020in}}%
\pgfpathlineto{\pgfqpoint{2.978672in}{1.239020in}}%
\pgfpathlineto{\pgfqpoint{2.978672in}{1.367724in}}%
\pgfpathclose%
\pgfusepath{fill}%
\end{pgfscope}%
\begin{pgfscope}%
\pgfpathrectangle{\pgfqpoint{0.804646in}{0.600000in}}{\pgfqpoint{2.573292in}{2.070576in}}%
\pgfusepath{clip}%
\pgfsetbuttcap%
\pgfsetmiterjoin%
\definecolor{currentfill}{rgb}{0.511253,0.510898,0.193296}%
\pgfsetfillcolor{currentfill}%
\pgfsetlinewidth{0.000000pt}%
\definecolor{currentstroke}{rgb}{0.000000,0.000000,0.000000}%
\pgfsetstrokecolor{currentstroke}%
\pgfsetstrokeopacity{0.000000}%
\pgfsetdash{}{0pt}%
\pgfpathmoveto{\pgfqpoint{2.989614in}{1.361690in}}%
\pgfpathlineto{\pgfqpoint{2.998367in}{1.361690in}}%
\pgfpathlineto{\pgfqpoint{2.998367in}{1.286095in}}%
\pgfpathlineto{\pgfqpoint{2.989614in}{1.286095in}}%
\pgfpathlineto{\pgfqpoint{2.989614in}{1.361690in}}%
\pgfpathclose%
\pgfusepath{fill}%
\end{pgfscope}%
\begin{pgfscope}%
\pgfpathrectangle{\pgfqpoint{0.804646in}{0.600000in}}{\pgfqpoint{2.573292in}{2.070576in}}%
\pgfusepath{clip}%
\pgfsetbuttcap%
\pgfsetmiterjoin%
\definecolor{currentfill}{rgb}{0.511253,0.510898,0.193296}%
\pgfsetfillcolor{currentfill}%
\pgfsetlinewidth{0.000000pt}%
\definecolor{currentstroke}{rgb}{0.000000,0.000000,0.000000}%
\pgfsetstrokecolor{currentstroke}%
\pgfsetstrokeopacity{0.000000}%
\pgfsetdash{}{0pt}%
\pgfpathmoveto{\pgfqpoint{3.000555in}{1.368301in}}%
\pgfpathlineto{\pgfqpoint{3.009309in}{1.368301in}}%
\pgfpathlineto{\pgfqpoint{3.009309in}{1.236249in}}%
\pgfpathlineto{\pgfqpoint{3.000555in}{1.236249in}}%
\pgfpathlineto{\pgfqpoint{3.000555in}{1.368301in}}%
\pgfpathclose%
\pgfusepath{fill}%
\end{pgfscope}%
\begin{pgfscope}%
\pgfpathrectangle{\pgfqpoint{0.804646in}{0.600000in}}{\pgfqpoint{2.573292in}{2.070576in}}%
\pgfusepath{clip}%
\pgfsetbuttcap%
\pgfsetmiterjoin%
\definecolor{currentfill}{rgb}{0.511253,0.510898,0.193296}%
\pgfsetfillcolor{currentfill}%
\pgfsetlinewidth{0.000000pt}%
\definecolor{currentstroke}{rgb}{0.000000,0.000000,0.000000}%
\pgfsetstrokecolor{currentstroke}%
\pgfsetstrokeopacity{0.000000}%
\pgfsetdash{}{0pt}%
\pgfpathmoveto{\pgfqpoint{3.011497in}{1.381445in}}%
\pgfpathlineto{\pgfqpoint{3.020251in}{1.381445in}}%
\pgfpathlineto{\pgfqpoint{3.020251in}{1.237534in}}%
\pgfpathlineto{\pgfqpoint{3.011497in}{1.237534in}}%
\pgfpathlineto{\pgfqpoint{3.011497in}{1.381445in}}%
\pgfpathclose%
\pgfusepath{fill}%
\end{pgfscope}%
\begin{pgfscope}%
\pgfpathrectangle{\pgfqpoint{0.804646in}{0.600000in}}{\pgfqpoint{2.573292in}{2.070576in}}%
\pgfusepath{clip}%
\pgfsetbuttcap%
\pgfsetmiterjoin%
\definecolor{currentfill}{rgb}{0.511253,0.510898,0.193296}%
\pgfsetfillcolor{currentfill}%
\pgfsetlinewidth{0.000000pt}%
\definecolor{currentstroke}{rgb}{0.000000,0.000000,0.000000}%
\pgfsetstrokecolor{currentstroke}%
\pgfsetstrokeopacity{0.000000}%
\pgfsetdash{}{0pt}%
\pgfpathmoveto{\pgfqpoint{3.022439in}{1.393100in}}%
\pgfpathlineto{\pgfqpoint{3.031192in}{1.393100in}}%
\pgfpathlineto{\pgfqpoint{3.031192in}{1.242077in}}%
\pgfpathlineto{\pgfqpoint{3.022439in}{1.242077in}}%
\pgfpathlineto{\pgfqpoint{3.022439in}{1.393100in}}%
\pgfpathclose%
\pgfusepath{fill}%
\end{pgfscope}%
\begin{pgfscope}%
\pgfpathrectangle{\pgfqpoint{0.804646in}{0.600000in}}{\pgfqpoint{2.573292in}{2.070576in}}%
\pgfusepath{clip}%
\pgfsetbuttcap%
\pgfsetmiterjoin%
\definecolor{currentfill}{rgb}{0.511253,0.510898,0.193296}%
\pgfsetfillcolor{currentfill}%
\pgfsetlinewidth{0.000000pt}%
\definecolor{currentstroke}{rgb}{0.000000,0.000000,0.000000}%
\pgfsetstrokecolor{currentstroke}%
\pgfsetstrokeopacity{0.000000}%
\pgfsetdash{}{0pt}%
\pgfpathmoveto{\pgfqpoint{3.033381in}{1.419792in}}%
\pgfpathlineto{\pgfqpoint{3.042134in}{1.419792in}}%
\pgfpathlineto{\pgfqpoint{3.042134in}{1.211386in}}%
\pgfpathlineto{\pgfqpoint{3.033381in}{1.211386in}}%
\pgfpathlineto{\pgfqpoint{3.033381in}{1.419792in}}%
\pgfpathclose%
\pgfusepath{fill}%
\end{pgfscope}%
\begin{pgfscope}%
\pgfpathrectangle{\pgfqpoint{0.804646in}{0.600000in}}{\pgfqpoint{2.573292in}{2.070576in}}%
\pgfusepath{clip}%
\pgfsetbuttcap%
\pgfsetmiterjoin%
\definecolor{currentfill}{rgb}{0.511253,0.510898,0.193296}%
\pgfsetfillcolor{currentfill}%
\pgfsetlinewidth{0.000000pt}%
\definecolor{currentstroke}{rgb}{0.000000,0.000000,0.000000}%
\pgfsetstrokecolor{currentstroke}%
\pgfsetstrokeopacity{0.000000}%
\pgfsetdash{}{0pt}%
\pgfpathmoveto{\pgfqpoint{3.044323in}{1.424867in}}%
\pgfpathlineto{\pgfqpoint{3.053076in}{1.424867in}}%
\pgfpathlineto{\pgfqpoint{3.053076in}{1.108593in}}%
\pgfpathlineto{\pgfqpoint{3.044323in}{1.108593in}}%
\pgfpathlineto{\pgfqpoint{3.044323in}{1.424867in}}%
\pgfpathclose%
\pgfusepath{fill}%
\end{pgfscope}%
\begin{pgfscope}%
\pgfpathrectangle{\pgfqpoint{0.804646in}{0.600000in}}{\pgfqpoint{2.573292in}{2.070576in}}%
\pgfusepath{clip}%
\pgfsetbuttcap%
\pgfsetmiterjoin%
\definecolor{currentfill}{rgb}{0.511253,0.510898,0.193296}%
\pgfsetfillcolor{currentfill}%
\pgfsetlinewidth{0.000000pt}%
\definecolor{currentstroke}{rgb}{0.000000,0.000000,0.000000}%
\pgfsetstrokecolor{currentstroke}%
\pgfsetstrokeopacity{0.000000}%
\pgfsetdash{}{0pt}%
\pgfpathmoveto{\pgfqpoint{3.055264in}{1.413695in}}%
\pgfpathlineto{\pgfqpoint{3.064018in}{1.413695in}}%
\pgfpathlineto{\pgfqpoint{3.064018in}{1.180528in}}%
\pgfpathlineto{\pgfqpoint{3.055264in}{1.180528in}}%
\pgfpathlineto{\pgfqpoint{3.055264in}{1.413695in}}%
\pgfpathclose%
\pgfusepath{fill}%
\end{pgfscope}%
\begin{pgfscope}%
\pgfpathrectangle{\pgfqpoint{0.804646in}{0.600000in}}{\pgfqpoint{2.573292in}{2.070576in}}%
\pgfusepath{clip}%
\pgfsetbuttcap%
\pgfsetmiterjoin%
\definecolor{currentfill}{rgb}{0.511253,0.510898,0.193296}%
\pgfsetfillcolor{currentfill}%
\pgfsetlinewidth{0.000000pt}%
\definecolor{currentstroke}{rgb}{0.000000,0.000000,0.000000}%
\pgfsetstrokecolor{currentstroke}%
\pgfsetstrokeopacity{0.000000}%
\pgfsetdash{}{0pt}%
\pgfpathmoveto{\pgfqpoint{3.066206in}{1.420796in}}%
\pgfpathlineto{\pgfqpoint{3.074960in}{1.420796in}}%
\pgfpathlineto{\pgfqpoint{3.074960in}{1.141130in}}%
\pgfpathlineto{\pgfqpoint{3.066206in}{1.141130in}}%
\pgfpathlineto{\pgfqpoint{3.066206in}{1.420796in}}%
\pgfpathclose%
\pgfusepath{fill}%
\end{pgfscope}%
\begin{pgfscope}%
\pgfpathrectangle{\pgfqpoint{0.804646in}{0.600000in}}{\pgfqpoint{2.573292in}{2.070576in}}%
\pgfusepath{clip}%
\pgfsetbuttcap%
\pgfsetmiterjoin%
\definecolor{currentfill}{rgb}{0.511253,0.510898,0.193296}%
\pgfsetfillcolor{currentfill}%
\pgfsetlinewidth{0.000000pt}%
\definecolor{currentstroke}{rgb}{0.000000,0.000000,0.000000}%
\pgfsetstrokecolor{currentstroke}%
\pgfsetstrokeopacity{0.000000}%
\pgfsetdash{}{0pt}%
\pgfpathmoveto{\pgfqpoint{3.077148in}{1.433343in}}%
\pgfpathlineto{\pgfqpoint{3.085901in}{1.433343in}}%
\pgfpathlineto{\pgfqpoint{3.085901in}{1.093077in}}%
\pgfpathlineto{\pgfqpoint{3.077148in}{1.093077in}}%
\pgfpathlineto{\pgfqpoint{3.077148in}{1.433343in}}%
\pgfpathclose%
\pgfusepath{fill}%
\end{pgfscope}%
\begin{pgfscope}%
\pgfpathrectangle{\pgfqpoint{0.804646in}{0.600000in}}{\pgfqpoint{2.573292in}{2.070576in}}%
\pgfusepath{clip}%
\pgfsetbuttcap%
\pgfsetmiterjoin%
\definecolor{currentfill}{rgb}{0.511253,0.510898,0.193296}%
\pgfsetfillcolor{currentfill}%
\pgfsetlinewidth{0.000000pt}%
\definecolor{currentstroke}{rgb}{0.000000,0.000000,0.000000}%
\pgfsetstrokecolor{currentstroke}%
\pgfsetstrokeopacity{0.000000}%
\pgfsetdash{}{0pt}%
\pgfpathmoveto{\pgfqpoint{3.088090in}{1.441024in}}%
\pgfpathlineto{\pgfqpoint{3.096843in}{1.441024in}}%
\pgfpathlineto{\pgfqpoint{3.096843in}{1.007124in}}%
\pgfpathlineto{\pgfqpoint{3.088090in}{1.007124in}}%
\pgfpathlineto{\pgfqpoint{3.088090in}{1.441024in}}%
\pgfpathclose%
\pgfusepath{fill}%
\end{pgfscope}%
\begin{pgfscope}%
\pgfpathrectangle{\pgfqpoint{0.804646in}{0.600000in}}{\pgfqpoint{2.573292in}{2.070576in}}%
\pgfusepath{clip}%
\pgfsetbuttcap%
\pgfsetmiterjoin%
\definecolor{currentfill}{rgb}{0.511253,0.510898,0.193296}%
\pgfsetfillcolor{currentfill}%
\pgfsetlinewidth{0.000000pt}%
\definecolor{currentstroke}{rgb}{0.000000,0.000000,0.000000}%
\pgfsetstrokecolor{currentstroke}%
\pgfsetstrokeopacity{0.000000}%
\pgfsetdash{}{0pt}%
\pgfpathmoveto{\pgfqpoint{3.099032in}{1.427055in}}%
\pgfpathlineto{\pgfqpoint{3.107785in}{1.427055in}}%
\pgfpathlineto{\pgfqpoint{3.107785in}{1.063028in}}%
\pgfpathlineto{\pgfqpoint{3.099032in}{1.063028in}}%
\pgfpathlineto{\pgfqpoint{3.099032in}{1.427055in}}%
\pgfpathclose%
\pgfusepath{fill}%
\end{pgfscope}%
\begin{pgfscope}%
\pgfpathrectangle{\pgfqpoint{0.804646in}{0.600000in}}{\pgfqpoint{2.573292in}{2.070576in}}%
\pgfusepath{clip}%
\pgfsetbuttcap%
\pgfsetmiterjoin%
\definecolor{currentfill}{rgb}{0.511253,0.510898,0.193296}%
\pgfsetfillcolor{currentfill}%
\pgfsetlinewidth{0.000000pt}%
\definecolor{currentstroke}{rgb}{0.000000,0.000000,0.000000}%
\pgfsetstrokecolor{currentstroke}%
\pgfsetstrokeopacity{0.000000}%
\pgfsetdash{}{0pt}%
\pgfpathmoveto{\pgfqpoint{3.109973in}{1.441817in}}%
\pgfpathlineto{\pgfqpoint{3.118727in}{1.441817in}}%
\pgfpathlineto{\pgfqpoint{3.118727in}{1.013142in}}%
\pgfpathlineto{\pgfqpoint{3.109973in}{1.013142in}}%
\pgfpathlineto{\pgfqpoint{3.109973in}{1.441817in}}%
\pgfpathclose%
\pgfusepath{fill}%
\end{pgfscope}%
\begin{pgfscope}%
\pgfpathrectangle{\pgfqpoint{0.804646in}{0.600000in}}{\pgfqpoint{2.573292in}{2.070576in}}%
\pgfusepath{clip}%
\pgfsetbuttcap%
\pgfsetmiterjoin%
\definecolor{currentfill}{rgb}{0.511253,0.510898,0.193296}%
\pgfsetfillcolor{currentfill}%
\pgfsetlinewidth{0.000000pt}%
\definecolor{currentstroke}{rgb}{0.000000,0.000000,0.000000}%
\pgfsetstrokecolor{currentstroke}%
\pgfsetstrokeopacity{0.000000}%
\pgfsetdash{}{0pt}%
\pgfpathmoveto{\pgfqpoint{3.120915in}{1.431597in}}%
\pgfpathlineto{\pgfqpoint{3.129669in}{1.431597in}}%
\pgfpathlineto{\pgfqpoint{3.129669in}{1.044488in}}%
\pgfpathlineto{\pgfqpoint{3.120915in}{1.044488in}}%
\pgfpathlineto{\pgfqpoint{3.120915in}{1.431597in}}%
\pgfpathclose%
\pgfusepath{fill}%
\end{pgfscope}%
\begin{pgfscope}%
\pgfpathrectangle{\pgfqpoint{0.804646in}{0.600000in}}{\pgfqpoint{2.573292in}{2.070576in}}%
\pgfusepath{clip}%
\pgfsetbuttcap%
\pgfsetmiterjoin%
\definecolor{currentfill}{rgb}{0.511253,0.510898,0.193296}%
\pgfsetfillcolor{currentfill}%
\pgfsetlinewidth{0.000000pt}%
\definecolor{currentstroke}{rgb}{0.000000,0.000000,0.000000}%
\pgfsetstrokecolor{currentstroke}%
\pgfsetstrokeopacity{0.000000}%
\pgfsetdash{}{0pt}%
\pgfpathmoveto{\pgfqpoint{3.131857in}{1.437527in}}%
\pgfpathlineto{\pgfqpoint{3.140610in}{1.437527in}}%
\pgfpathlineto{\pgfqpoint{3.140610in}{1.032208in}}%
\pgfpathlineto{\pgfqpoint{3.131857in}{1.032208in}}%
\pgfpathlineto{\pgfqpoint{3.131857in}{1.437527in}}%
\pgfpathclose%
\pgfusepath{fill}%
\end{pgfscope}%
\begin{pgfscope}%
\pgfpathrectangle{\pgfqpoint{0.804646in}{0.600000in}}{\pgfqpoint{2.573292in}{2.070576in}}%
\pgfusepath{clip}%
\pgfsetbuttcap%
\pgfsetmiterjoin%
\definecolor{currentfill}{rgb}{0.511253,0.510898,0.193296}%
\pgfsetfillcolor{currentfill}%
\pgfsetlinewidth{0.000000pt}%
\definecolor{currentstroke}{rgb}{0.000000,0.000000,0.000000}%
\pgfsetstrokecolor{currentstroke}%
\pgfsetstrokeopacity{0.000000}%
\pgfsetdash{}{0pt}%
\pgfpathmoveto{\pgfqpoint{3.142799in}{1.449359in}}%
\pgfpathlineto{\pgfqpoint{3.151552in}{1.449359in}}%
\pgfpathlineto{\pgfqpoint{3.151552in}{0.976831in}}%
\pgfpathlineto{\pgfqpoint{3.142799in}{0.976831in}}%
\pgfpathlineto{\pgfqpoint{3.142799in}{1.449359in}}%
\pgfpathclose%
\pgfusepath{fill}%
\end{pgfscope}%
\begin{pgfscope}%
\pgfpathrectangle{\pgfqpoint{0.804646in}{0.600000in}}{\pgfqpoint{2.573292in}{2.070576in}}%
\pgfusepath{clip}%
\pgfsetbuttcap%
\pgfsetmiterjoin%
\definecolor{currentfill}{rgb}{0.511253,0.510898,0.193296}%
\pgfsetfillcolor{currentfill}%
\pgfsetlinewidth{0.000000pt}%
\definecolor{currentstroke}{rgb}{0.000000,0.000000,0.000000}%
\pgfsetstrokecolor{currentstroke}%
\pgfsetstrokeopacity{0.000000}%
\pgfsetdash{}{0pt}%
\pgfpathmoveto{\pgfqpoint{3.153741in}{1.440076in}}%
\pgfpathlineto{\pgfqpoint{3.162494in}{1.440076in}}%
\pgfpathlineto{\pgfqpoint{3.162494in}{1.015976in}}%
\pgfpathlineto{\pgfqpoint{3.153741in}{1.015976in}}%
\pgfpathlineto{\pgfqpoint{3.153741in}{1.440076in}}%
\pgfpathclose%
\pgfusepath{fill}%
\end{pgfscope}%
\begin{pgfscope}%
\pgfpathrectangle{\pgfqpoint{0.804646in}{0.600000in}}{\pgfqpoint{2.573292in}{2.070576in}}%
\pgfusepath{clip}%
\pgfsetbuttcap%
\pgfsetmiterjoin%
\definecolor{currentfill}{rgb}{0.511253,0.510898,0.193296}%
\pgfsetfillcolor{currentfill}%
\pgfsetlinewidth{0.000000pt}%
\definecolor{currentstroke}{rgb}{0.000000,0.000000,0.000000}%
\pgfsetstrokecolor{currentstroke}%
\pgfsetstrokeopacity{0.000000}%
\pgfsetdash{}{0pt}%
\pgfpathmoveto{\pgfqpoint{3.164682in}{1.446336in}}%
\pgfpathlineto{\pgfqpoint{3.173436in}{1.446336in}}%
\pgfpathlineto{\pgfqpoint{3.173436in}{1.047440in}}%
\pgfpathlineto{\pgfqpoint{3.164682in}{1.047440in}}%
\pgfpathlineto{\pgfqpoint{3.164682in}{1.446336in}}%
\pgfpathclose%
\pgfusepath{fill}%
\end{pgfscope}%
\begin{pgfscope}%
\pgfpathrectangle{\pgfqpoint{0.804646in}{0.600000in}}{\pgfqpoint{2.573292in}{2.070576in}}%
\pgfusepath{clip}%
\pgfsetbuttcap%
\pgfsetmiterjoin%
\definecolor{currentfill}{rgb}{0.511253,0.510898,0.193296}%
\pgfsetfillcolor{currentfill}%
\pgfsetlinewidth{0.000000pt}%
\definecolor{currentstroke}{rgb}{0.000000,0.000000,0.000000}%
\pgfsetstrokecolor{currentstroke}%
\pgfsetstrokeopacity{0.000000}%
\pgfsetdash{}{0pt}%
\pgfpathmoveto{\pgfqpoint{3.175624in}{1.451744in}}%
\pgfpathlineto{\pgfqpoint{3.184378in}{1.451744in}}%
\pgfpathlineto{\pgfqpoint{3.184378in}{1.057260in}}%
\pgfpathlineto{\pgfqpoint{3.175624in}{1.057260in}}%
\pgfpathlineto{\pgfqpoint{3.175624in}{1.451744in}}%
\pgfpathclose%
\pgfusepath{fill}%
\end{pgfscope}%
\begin{pgfscope}%
\pgfpathrectangle{\pgfqpoint{0.804646in}{0.600000in}}{\pgfqpoint{2.573292in}{2.070576in}}%
\pgfusepath{clip}%
\pgfsetbuttcap%
\pgfsetmiterjoin%
\definecolor{currentfill}{rgb}{0.511253,0.510898,0.193296}%
\pgfsetfillcolor{currentfill}%
\pgfsetlinewidth{0.000000pt}%
\definecolor{currentstroke}{rgb}{0.000000,0.000000,0.000000}%
\pgfsetstrokecolor{currentstroke}%
\pgfsetstrokeopacity{0.000000}%
\pgfsetdash{}{0pt}%
\pgfpathmoveto{\pgfqpoint{3.186566in}{1.457906in}}%
\pgfpathlineto{\pgfqpoint{3.195319in}{1.457906in}}%
\pgfpathlineto{\pgfqpoint{3.195319in}{1.029818in}}%
\pgfpathlineto{\pgfqpoint{3.186566in}{1.029818in}}%
\pgfpathlineto{\pgfqpoint{3.186566in}{1.457906in}}%
\pgfpathclose%
\pgfusepath{fill}%
\end{pgfscope}%
\begin{pgfscope}%
\pgfpathrectangle{\pgfqpoint{0.804646in}{0.600000in}}{\pgfqpoint{2.573292in}{2.070576in}}%
\pgfusepath{clip}%
\pgfsetbuttcap%
\pgfsetmiterjoin%
\definecolor{currentfill}{rgb}{0.511253,0.510898,0.193296}%
\pgfsetfillcolor{currentfill}%
\pgfsetlinewidth{0.000000pt}%
\definecolor{currentstroke}{rgb}{0.000000,0.000000,0.000000}%
\pgfsetstrokecolor{currentstroke}%
\pgfsetstrokeopacity{0.000000}%
\pgfsetdash{}{0pt}%
\pgfpathmoveto{\pgfqpoint{3.197508in}{1.468150in}}%
\pgfpathlineto{\pgfqpoint{3.206261in}{1.468150in}}%
\pgfpathlineto{\pgfqpoint{3.206261in}{0.991738in}}%
\pgfpathlineto{\pgfqpoint{3.197508in}{0.991738in}}%
\pgfpathlineto{\pgfqpoint{3.197508in}{1.468150in}}%
\pgfpathclose%
\pgfusepath{fill}%
\end{pgfscope}%
\begin{pgfscope}%
\pgfpathrectangle{\pgfqpoint{0.804646in}{0.600000in}}{\pgfqpoint{2.573292in}{2.070576in}}%
\pgfusepath{clip}%
\pgfsetbuttcap%
\pgfsetmiterjoin%
\definecolor{currentfill}{rgb}{0.511253,0.510898,0.193296}%
\pgfsetfillcolor{currentfill}%
\pgfsetlinewidth{0.000000pt}%
\definecolor{currentstroke}{rgb}{0.000000,0.000000,0.000000}%
\pgfsetstrokecolor{currentstroke}%
\pgfsetstrokeopacity{0.000000}%
\pgfsetdash{}{0pt}%
\pgfpathmoveto{\pgfqpoint{3.208450in}{1.471629in}}%
\pgfpathlineto{\pgfqpoint{3.217203in}{1.471629in}}%
\pgfpathlineto{\pgfqpoint{3.217203in}{0.949877in}}%
\pgfpathlineto{\pgfqpoint{3.208450in}{0.949877in}}%
\pgfpathlineto{\pgfqpoint{3.208450in}{1.471629in}}%
\pgfpathclose%
\pgfusepath{fill}%
\end{pgfscope}%
\begin{pgfscope}%
\pgfpathrectangle{\pgfqpoint{0.804646in}{0.600000in}}{\pgfqpoint{2.573292in}{2.070576in}}%
\pgfusepath{clip}%
\pgfsetbuttcap%
\pgfsetmiterjoin%
\definecolor{currentfill}{rgb}{0.511253,0.510898,0.193296}%
\pgfsetfillcolor{currentfill}%
\pgfsetlinewidth{0.000000pt}%
\definecolor{currentstroke}{rgb}{0.000000,0.000000,0.000000}%
\pgfsetstrokecolor{currentstroke}%
\pgfsetstrokeopacity{0.000000}%
\pgfsetdash{}{0pt}%
\pgfpathmoveto{\pgfqpoint{3.219391in}{1.467859in}}%
\pgfpathlineto{\pgfqpoint{3.228145in}{1.467859in}}%
\pgfpathlineto{\pgfqpoint{3.228145in}{0.810285in}}%
\pgfpathlineto{\pgfqpoint{3.219391in}{0.810285in}}%
\pgfpathlineto{\pgfqpoint{3.219391in}{1.467859in}}%
\pgfpathclose%
\pgfusepath{fill}%
\end{pgfscope}%
\begin{pgfscope}%
\pgfpathrectangle{\pgfqpoint{0.804646in}{0.600000in}}{\pgfqpoint{2.573292in}{2.070576in}}%
\pgfusepath{clip}%
\pgfsetbuttcap%
\pgfsetmiterjoin%
\definecolor{currentfill}{rgb}{0.511253,0.510898,0.193296}%
\pgfsetfillcolor{currentfill}%
\pgfsetlinewidth{0.000000pt}%
\definecolor{currentstroke}{rgb}{0.000000,0.000000,0.000000}%
\pgfsetstrokecolor{currentstroke}%
\pgfsetstrokeopacity{0.000000}%
\pgfsetdash{}{0pt}%
\pgfpathmoveto{\pgfqpoint{3.230333in}{1.475161in}}%
\pgfpathlineto{\pgfqpoint{3.239087in}{1.475161in}}%
\pgfpathlineto{\pgfqpoint{3.239087in}{0.827998in}}%
\pgfpathlineto{\pgfqpoint{3.230333in}{0.827998in}}%
\pgfpathlineto{\pgfqpoint{3.230333in}{1.475161in}}%
\pgfpathclose%
\pgfusepath{fill}%
\end{pgfscope}%
\begin{pgfscope}%
\pgfpathrectangle{\pgfqpoint{0.804646in}{0.600000in}}{\pgfqpoint{2.573292in}{2.070576in}}%
\pgfusepath{clip}%
\pgfsetbuttcap%
\pgfsetmiterjoin%
\definecolor{currentfill}{rgb}{0.511253,0.510898,0.193296}%
\pgfsetfillcolor{currentfill}%
\pgfsetlinewidth{0.000000pt}%
\definecolor{currentstroke}{rgb}{0.000000,0.000000,0.000000}%
\pgfsetstrokecolor{currentstroke}%
\pgfsetstrokeopacity{0.000000}%
\pgfsetdash{}{0pt}%
\pgfpathmoveto{\pgfqpoint{3.241275in}{1.483188in}}%
\pgfpathlineto{\pgfqpoint{3.250028in}{1.483188in}}%
\pgfpathlineto{\pgfqpoint{3.250028in}{0.816918in}}%
\pgfpathlineto{\pgfqpoint{3.241275in}{0.816918in}}%
\pgfpathlineto{\pgfqpoint{3.241275in}{1.483188in}}%
\pgfpathclose%
\pgfusepath{fill}%
\end{pgfscope}%
\begin{pgfscope}%
\pgfpathrectangle{\pgfqpoint{0.804646in}{0.600000in}}{\pgfqpoint{2.573292in}{2.070576in}}%
\pgfusepath{clip}%
\pgfsetbuttcap%
\pgfsetmiterjoin%
\definecolor{currentfill}{rgb}{0.511253,0.510898,0.193296}%
\pgfsetfillcolor{currentfill}%
\pgfsetlinewidth{0.000000pt}%
\definecolor{currentstroke}{rgb}{0.000000,0.000000,0.000000}%
\pgfsetstrokecolor{currentstroke}%
\pgfsetstrokeopacity{0.000000}%
\pgfsetdash{}{0pt}%
\pgfpathmoveto{\pgfqpoint{3.252217in}{1.479405in}}%
\pgfpathlineto{\pgfqpoint{3.260970in}{1.479405in}}%
\pgfpathlineto{\pgfqpoint{3.260970in}{0.769512in}}%
\pgfpathlineto{\pgfqpoint{3.252217in}{0.769512in}}%
\pgfpathlineto{\pgfqpoint{3.252217in}{1.479405in}}%
\pgfpathclose%
\pgfusepath{fill}%
\end{pgfscope}%
\begin{pgfscope}%
\pgfpathrectangle{\pgfqpoint{0.804646in}{0.600000in}}{\pgfqpoint{2.573292in}{2.070576in}}%
\pgfusepath{clip}%
\pgfsetbuttcap%
\pgfsetmiterjoin%
\definecolor{currentfill}{rgb}{0.754268,0.565033,0.211761}%
\pgfsetfillcolor{currentfill}%
\pgfsetlinewidth{0.000000pt}%
\definecolor{currentstroke}{rgb}{0.000000,0.000000,0.000000}%
\pgfsetstrokecolor{currentstroke}%
\pgfsetstrokeopacity{0.000000}%
\pgfsetdash{}{0pt}%
\pgfpathmoveto{\pgfqpoint{0.921614in}{1.507733in}}%
\pgfpathlineto{\pgfqpoint{0.930367in}{1.507733in}}%
\pgfpathlineto{\pgfqpoint{0.930367in}{1.467492in}}%
\pgfpathlineto{\pgfqpoint{0.921614in}{1.467492in}}%
\pgfpathlineto{\pgfqpoint{0.921614in}{1.507733in}}%
\pgfpathclose%
\pgfusepath{fill}%
\end{pgfscope}%
\begin{pgfscope}%
\pgfpathrectangle{\pgfqpoint{0.804646in}{0.600000in}}{\pgfqpoint{2.573292in}{2.070576in}}%
\pgfusepath{clip}%
\pgfsetbuttcap%
\pgfsetmiterjoin%
\definecolor{currentfill}{rgb}{0.754268,0.565033,0.211761}%
\pgfsetfillcolor{currentfill}%
\pgfsetlinewidth{0.000000pt}%
\definecolor{currentstroke}{rgb}{0.000000,0.000000,0.000000}%
\pgfsetstrokecolor{currentstroke}%
\pgfsetstrokeopacity{0.000000}%
\pgfsetdash{}{0pt}%
\pgfpathmoveto{\pgfqpoint{0.932555in}{1.450786in}}%
\pgfpathlineto{\pgfqpoint{0.941309in}{1.450786in}}%
\pgfpathlineto{\pgfqpoint{0.941309in}{1.441443in}}%
\pgfpathlineto{\pgfqpoint{0.932555in}{1.441443in}}%
\pgfpathlineto{\pgfqpoint{0.932555in}{1.450786in}}%
\pgfpathclose%
\pgfusepath{fill}%
\end{pgfscope}%
\begin{pgfscope}%
\pgfpathrectangle{\pgfqpoint{0.804646in}{0.600000in}}{\pgfqpoint{2.573292in}{2.070576in}}%
\pgfusepath{clip}%
\pgfsetbuttcap%
\pgfsetmiterjoin%
\definecolor{currentfill}{rgb}{0.754268,0.565033,0.211761}%
\pgfsetfillcolor{currentfill}%
\pgfsetlinewidth{0.000000pt}%
\definecolor{currentstroke}{rgb}{0.000000,0.000000,0.000000}%
\pgfsetstrokecolor{currentstroke}%
\pgfsetstrokeopacity{0.000000}%
\pgfsetdash{}{0pt}%
\pgfpathmoveto{\pgfqpoint{0.943497in}{1.789338in}}%
\pgfpathlineto{\pgfqpoint{0.952251in}{1.789338in}}%
\pgfpathlineto{\pgfqpoint{0.952251in}{1.815349in}}%
\pgfpathlineto{\pgfqpoint{0.943497in}{1.815349in}}%
\pgfpathlineto{\pgfqpoint{0.943497in}{1.789338in}}%
\pgfpathclose%
\pgfusepath{fill}%
\end{pgfscope}%
\begin{pgfscope}%
\pgfpathrectangle{\pgfqpoint{0.804646in}{0.600000in}}{\pgfqpoint{2.573292in}{2.070576in}}%
\pgfusepath{clip}%
\pgfsetbuttcap%
\pgfsetmiterjoin%
\definecolor{currentfill}{rgb}{0.754268,0.565033,0.211761}%
\pgfsetfillcolor{currentfill}%
\pgfsetlinewidth{0.000000pt}%
\definecolor{currentstroke}{rgb}{0.000000,0.000000,0.000000}%
\pgfsetstrokecolor{currentstroke}%
\pgfsetstrokeopacity{0.000000}%
\pgfsetdash{}{0pt}%
\pgfpathmoveto{\pgfqpoint{0.954439in}{1.785045in}}%
\pgfpathlineto{\pgfqpoint{0.963192in}{1.785045in}}%
\pgfpathlineto{\pgfqpoint{0.963192in}{1.831855in}}%
\pgfpathlineto{\pgfqpoint{0.954439in}{1.831855in}}%
\pgfpathlineto{\pgfqpoint{0.954439in}{1.785045in}}%
\pgfpathclose%
\pgfusepath{fill}%
\end{pgfscope}%
\begin{pgfscope}%
\pgfpathrectangle{\pgfqpoint{0.804646in}{0.600000in}}{\pgfqpoint{2.573292in}{2.070576in}}%
\pgfusepath{clip}%
\pgfsetbuttcap%
\pgfsetmiterjoin%
\definecolor{currentfill}{rgb}{0.754268,0.565033,0.211761}%
\pgfsetfillcolor{currentfill}%
\pgfsetlinewidth{0.000000pt}%
\definecolor{currentstroke}{rgb}{0.000000,0.000000,0.000000}%
\pgfsetstrokecolor{currentstroke}%
\pgfsetstrokeopacity{0.000000}%
\pgfsetdash{}{0pt}%
\pgfpathmoveto{\pgfqpoint{0.965381in}{1.400496in}}%
\pgfpathlineto{\pgfqpoint{0.974134in}{1.400496in}}%
\pgfpathlineto{\pgfqpoint{0.974134in}{1.353242in}}%
\pgfpathlineto{\pgfqpoint{0.965381in}{1.353242in}}%
\pgfpathlineto{\pgfqpoint{0.965381in}{1.400496in}}%
\pgfpathclose%
\pgfusepath{fill}%
\end{pgfscope}%
\begin{pgfscope}%
\pgfpathrectangle{\pgfqpoint{0.804646in}{0.600000in}}{\pgfqpoint{2.573292in}{2.070576in}}%
\pgfusepath{clip}%
\pgfsetbuttcap%
\pgfsetmiterjoin%
\definecolor{currentfill}{rgb}{0.754268,0.565033,0.211761}%
\pgfsetfillcolor{currentfill}%
\pgfsetlinewidth{0.000000pt}%
\definecolor{currentstroke}{rgb}{0.000000,0.000000,0.000000}%
\pgfsetstrokecolor{currentstroke}%
\pgfsetstrokeopacity{0.000000}%
\pgfsetdash{}{0pt}%
\pgfpathmoveto{\pgfqpoint{0.976323in}{1.414323in}}%
\pgfpathlineto{\pgfqpoint{0.985076in}{1.414323in}}%
\pgfpathlineto{\pgfqpoint{0.985076in}{1.366765in}}%
\pgfpathlineto{\pgfqpoint{0.976323in}{1.366765in}}%
\pgfpathlineto{\pgfqpoint{0.976323in}{1.414323in}}%
\pgfpathclose%
\pgfusepath{fill}%
\end{pgfscope}%
\begin{pgfscope}%
\pgfpathrectangle{\pgfqpoint{0.804646in}{0.600000in}}{\pgfqpoint{2.573292in}{2.070576in}}%
\pgfusepath{clip}%
\pgfsetbuttcap%
\pgfsetmiterjoin%
\definecolor{currentfill}{rgb}{0.754268,0.565033,0.211761}%
\pgfsetfillcolor{currentfill}%
\pgfsetlinewidth{0.000000pt}%
\definecolor{currentstroke}{rgb}{0.000000,0.000000,0.000000}%
\pgfsetstrokecolor{currentstroke}%
\pgfsetstrokeopacity{0.000000}%
\pgfsetdash{}{0pt}%
\pgfpathmoveto{\pgfqpoint{0.987264in}{1.363716in}}%
\pgfpathlineto{\pgfqpoint{0.996018in}{1.363716in}}%
\pgfpathlineto{\pgfqpoint{0.996018in}{1.354173in}}%
\pgfpathlineto{\pgfqpoint{0.987264in}{1.354173in}}%
\pgfpathlineto{\pgfqpoint{0.987264in}{1.363716in}}%
\pgfpathclose%
\pgfusepath{fill}%
\end{pgfscope}%
\begin{pgfscope}%
\pgfpathrectangle{\pgfqpoint{0.804646in}{0.600000in}}{\pgfqpoint{2.573292in}{2.070576in}}%
\pgfusepath{clip}%
\pgfsetbuttcap%
\pgfsetmiterjoin%
\definecolor{currentfill}{rgb}{0.754268,0.565033,0.211761}%
\pgfsetfillcolor{currentfill}%
\pgfsetlinewidth{0.000000pt}%
\definecolor{currentstroke}{rgb}{0.000000,0.000000,0.000000}%
\pgfsetstrokecolor{currentstroke}%
\pgfsetstrokeopacity{0.000000}%
\pgfsetdash{}{0pt}%
\pgfpathmoveto{\pgfqpoint{0.998206in}{1.859805in}}%
\pgfpathlineto{\pgfqpoint{1.006960in}{1.859805in}}%
\pgfpathlineto{\pgfqpoint{1.006960in}{1.910484in}}%
\pgfpathlineto{\pgfqpoint{0.998206in}{1.910484in}}%
\pgfpathlineto{\pgfqpoint{0.998206in}{1.859805in}}%
\pgfpathclose%
\pgfusepath{fill}%
\end{pgfscope}%
\begin{pgfscope}%
\pgfpathrectangle{\pgfqpoint{0.804646in}{0.600000in}}{\pgfqpoint{2.573292in}{2.070576in}}%
\pgfusepath{clip}%
\pgfsetbuttcap%
\pgfsetmiterjoin%
\definecolor{currentfill}{rgb}{0.754268,0.565033,0.211761}%
\pgfsetfillcolor{currentfill}%
\pgfsetlinewidth{0.000000pt}%
\definecolor{currentstroke}{rgb}{0.000000,0.000000,0.000000}%
\pgfsetstrokecolor{currentstroke}%
\pgfsetstrokeopacity{0.000000}%
\pgfsetdash{}{0pt}%
\pgfpathmoveto{\pgfqpoint{1.009148in}{1.871745in}}%
\pgfpathlineto{\pgfqpoint{1.017901in}{1.871745in}}%
\pgfpathlineto{\pgfqpoint{1.017901in}{1.885317in}}%
\pgfpathlineto{\pgfqpoint{1.009148in}{1.885317in}}%
\pgfpathlineto{\pgfqpoint{1.009148in}{1.871745in}}%
\pgfpathclose%
\pgfusepath{fill}%
\end{pgfscope}%
\begin{pgfscope}%
\pgfpathrectangle{\pgfqpoint{0.804646in}{0.600000in}}{\pgfqpoint{2.573292in}{2.070576in}}%
\pgfusepath{clip}%
\pgfsetbuttcap%
\pgfsetmiterjoin%
\definecolor{currentfill}{rgb}{0.754268,0.565033,0.211761}%
\pgfsetfillcolor{currentfill}%
\pgfsetlinewidth{0.000000pt}%
\definecolor{currentstroke}{rgb}{0.000000,0.000000,0.000000}%
\pgfsetstrokecolor{currentstroke}%
\pgfsetstrokeopacity{0.000000}%
\pgfsetdash{}{0pt}%
\pgfpathmoveto{\pgfqpoint{1.020090in}{1.865596in}}%
\pgfpathlineto{\pgfqpoint{1.028843in}{1.865596in}}%
\pgfpathlineto{\pgfqpoint{1.028843in}{1.899335in}}%
\pgfpathlineto{\pgfqpoint{1.020090in}{1.899335in}}%
\pgfpathlineto{\pgfqpoint{1.020090in}{1.865596in}}%
\pgfpathclose%
\pgfusepath{fill}%
\end{pgfscope}%
\begin{pgfscope}%
\pgfpathrectangle{\pgfqpoint{0.804646in}{0.600000in}}{\pgfqpoint{2.573292in}{2.070576in}}%
\pgfusepath{clip}%
\pgfsetbuttcap%
\pgfsetmiterjoin%
\definecolor{currentfill}{rgb}{0.754268,0.565033,0.211761}%
\pgfsetfillcolor{currentfill}%
\pgfsetlinewidth{0.000000pt}%
\definecolor{currentstroke}{rgb}{0.000000,0.000000,0.000000}%
\pgfsetstrokecolor{currentstroke}%
\pgfsetstrokeopacity{0.000000}%
\pgfsetdash{}{0pt}%
\pgfpathmoveto{\pgfqpoint{1.031032in}{1.902239in}}%
\pgfpathlineto{\pgfqpoint{1.039785in}{1.902239in}}%
\pgfpathlineto{\pgfqpoint{1.039785in}{1.951469in}}%
\pgfpathlineto{\pgfqpoint{1.031032in}{1.951469in}}%
\pgfpathlineto{\pgfqpoint{1.031032in}{1.902239in}}%
\pgfpathclose%
\pgfusepath{fill}%
\end{pgfscope}%
\begin{pgfscope}%
\pgfpathrectangle{\pgfqpoint{0.804646in}{0.600000in}}{\pgfqpoint{2.573292in}{2.070576in}}%
\pgfusepath{clip}%
\pgfsetbuttcap%
\pgfsetmiterjoin%
\definecolor{currentfill}{rgb}{0.754268,0.565033,0.211761}%
\pgfsetfillcolor{currentfill}%
\pgfsetlinewidth{0.000000pt}%
\definecolor{currentstroke}{rgb}{0.000000,0.000000,0.000000}%
\pgfsetstrokecolor{currentstroke}%
\pgfsetstrokeopacity{0.000000}%
\pgfsetdash{}{0pt}%
\pgfpathmoveto{\pgfqpoint{1.041973in}{1.958597in}}%
\pgfpathlineto{\pgfqpoint{1.050727in}{1.958597in}}%
\pgfpathlineto{\pgfqpoint{1.050727in}{2.062207in}}%
\pgfpathlineto{\pgfqpoint{1.041973in}{2.062207in}}%
\pgfpathlineto{\pgfqpoint{1.041973in}{1.958597in}}%
\pgfpathclose%
\pgfusepath{fill}%
\end{pgfscope}%
\begin{pgfscope}%
\pgfpathrectangle{\pgfqpoint{0.804646in}{0.600000in}}{\pgfqpoint{2.573292in}{2.070576in}}%
\pgfusepath{clip}%
\pgfsetbuttcap%
\pgfsetmiterjoin%
\definecolor{currentfill}{rgb}{0.754268,0.565033,0.211761}%
\pgfsetfillcolor{currentfill}%
\pgfsetlinewidth{0.000000pt}%
\definecolor{currentstroke}{rgb}{0.000000,0.000000,0.000000}%
\pgfsetstrokecolor{currentstroke}%
\pgfsetstrokeopacity{0.000000}%
\pgfsetdash{}{0pt}%
\pgfpathmoveto{\pgfqpoint{1.052915in}{1.910876in}}%
\pgfpathlineto{\pgfqpoint{1.061669in}{1.910876in}}%
\pgfpathlineto{\pgfqpoint{1.061669in}{1.955553in}}%
\pgfpathlineto{\pgfqpoint{1.052915in}{1.955553in}}%
\pgfpathlineto{\pgfqpoint{1.052915in}{1.910876in}}%
\pgfpathclose%
\pgfusepath{fill}%
\end{pgfscope}%
\begin{pgfscope}%
\pgfpathrectangle{\pgfqpoint{0.804646in}{0.600000in}}{\pgfqpoint{2.573292in}{2.070576in}}%
\pgfusepath{clip}%
\pgfsetbuttcap%
\pgfsetmiterjoin%
\definecolor{currentfill}{rgb}{0.754268,0.565033,0.211761}%
\pgfsetfillcolor{currentfill}%
\pgfsetlinewidth{0.000000pt}%
\definecolor{currentstroke}{rgb}{0.000000,0.000000,0.000000}%
\pgfsetstrokecolor{currentstroke}%
\pgfsetstrokeopacity{0.000000}%
\pgfsetdash{}{0pt}%
\pgfpathmoveto{\pgfqpoint{1.063857in}{1.977739in}}%
\pgfpathlineto{\pgfqpoint{1.072610in}{1.977739in}}%
\pgfpathlineto{\pgfqpoint{1.072610in}{2.058297in}}%
\pgfpathlineto{\pgfqpoint{1.063857in}{2.058297in}}%
\pgfpathlineto{\pgfqpoint{1.063857in}{1.977739in}}%
\pgfpathclose%
\pgfusepath{fill}%
\end{pgfscope}%
\begin{pgfscope}%
\pgfpathrectangle{\pgfqpoint{0.804646in}{0.600000in}}{\pgfqpoint{2.573292in}{2.070576in}}%
\pgfusepath{clip}%
\pgfsetbuttcap%
\pgfsetmiterjoin%
\definecolor{currentfill}{rgb}{0.754268,0.565033,0.211761}%
\pgfsetfillcolor{currentfill}%
\pgfsetlinewidth{0.000000pt}%
\definecolor{currentstroke}{rgb}{0.000000,0.000000,0.000000}%
\pgfsetstrokecolor{currentstroke}%
\pgfsetstrokeopacity{0.000000}%
\pgfsetdash{}{0pt}%
\pgfpathmoveto{\pgfqpoint{1.074799in}{1.899421in}}%
\pgfpathlineto{\pgfqpoint{1.083552in}{1.899421in}}%
\pgfpathlineto{\pgfqpoint{1.083552in}{1.960710in}}%
\pgfpathlineto{\pgfqpoint{1.074799in}{1.960710in}}%
\pgfpathlineto{\pgfqpoint{1.074799in}{1.899421in}}%
\pgfpathclose%
\pgfusepath{fill}%
\end{pgfscope}%
\begin{pgfscope}%
\pgfpathrectangle{\pgfqpoint{0.804646in}{0.600000in}}{\pgfqpoint{2.573292in}{2.070576in}}%
\pgfusepath{clip}%
\pgfsetbuttcap%
\pgfsetmiterjoin%
\definecolor{currentfill}{rgb}{0.754268,0.565033,0.211761}%
\pgfsetfillcolor{currentfill}%
\pgfsetlinewidth{0.000000pt}%
\definecolor{currentstroke}{rgb}{0.000000,0.000000,0.000000}%
\pgfsetstrokecolor{currentstroke}%
\pgfsetstrokeopacity{0.000000}%
\pgfsetdash{}{0pt}%
\pgfpathmoveto{\pgfqpoint{1.085741in}{1.914920in}}%
\pgfpathlineto{\pgfqpoint{1.094494in}{1.914920in}}%
\pgfpathlineto{\pgfqpoint{1.094494in}{2.001901in}}%
\pgfpathlineto{\pgfqpoint{1.085741in}{2.001901in}}%
\pgfpathlineto{\pgfqpoint{1.085741in}{1.914920in}}%
\pgfpathclose%
\pgfusepath{fill}%
\end{pgfscope}%
\begin{pgfscope}%
\pgfpathrectangle{\pgfqpoint{0.804646in}{0.600000in}}{\pgfqpoint{2.573292in}{2.070576in}}%
\pgfusepath{clip}%
\pgfsetbuttcap%
\pgfsetmiterjoin%
\definecolor{currentfill}{rgb}{0.754268,0.565033,0.211761}%
\pgfsetfillcolor{currentfill}%
\pgfsetlinewidth{0.000000pt}%
\definecolor{currentstroke}{rgb}{0.000000,0.000000,0.000000}%
\pgfsetstrokecolor{currentstroke}%
\pgfsetstrokeopacity{0.000000}%
\pgfsetdash{}{0pt}%
\pgfpathmoveto{\pgfqpoint{1.096682in}{1.987956in}}%
\pgfpathlineto{\pgfqpoint{1.105436in}{1.987956in}}%
\pgfpathlineto{\pgfqpoint{1.105436in}{2.048697in}}%
\pgfpathlineto{\pgfqpoint{1.096682in}{2.048697in}}%
\pgfpathlineto{\pgfqpoint{1.096682in}{1.987956in}}%
\pgfpathclose%
\pgfusepath{fill}%
\end{pgfscope}%
\begin{pgfscope}%
\pgfpathrectangle{\pgfqpoint{0.804646in}{0.600000in}}{\pgfqpoint{2.573292in}{2.070576in}}%
\pgfusepath{clip}%
\pgfsetbuttcap%
\pgfsetmiterjoin%
\definecolor{currentfill}{rgb}{0.754268,0.565033,0.211761}%
\pgfsetfillcolor{currentfill}%
\pgfsetlinewidth{0.000000pt}%
\definecolor{currentstroke}{rgb}{0.000000,0.000000,0.000000}%
\pgfsetstrokecolor{currentstroke}%
\pgfsetstrokeopacity{0.000000}%
\pgfsetdash{}{0pt}%
\pgfpathmoveto{\pgfqpoint{1.107624in}{1.919378in}}%
\pgfpathlineto{\pgfqpoint{1.116378in}{1.919378in}}%
\pgfpathlineto{\pgfqpoint{1.116378in}{1.927279in}}%
\pgfpathlineto{\pgfqpoint{1.107624in}{1.927279in}}%
\pgfpathlineto{\pgfqpoint{1.107624in}{1.919378in}}%
\pgfpathclose%
\pgfusepath{fill}%
\end{pgfscope}%
\begin{pgfscope}%
\pgfpathrectangle{\pgfqpoint{0.804646in}{0.600000in}}{\pgfqpoint{2.573292in}{2.070576in}}%
\pgfusepath{clip}%
\pgfsetbuttcap%
\pgfsetmiterjoin%
\definecolor{currentfill}{rgb}{0.754268,0.565033,0.211761}%
\pgfsetfillcolor{currentfill}%
\pgfsetlinewidth{0.000000pt}%
\definecolor{currentstroke}{rgb}{0.000000,0.000000,0.000000}%
\pgfsetstrokecolor{currentstroke}%
\pgfsetstrokeopacity{0.000000}%
\pgfsetdash{}{0pt}%
\pgfpathmoveto{\pgfqpoint{1.118566in}{1.997124in}}%
\pgfpathlineto{\pgfqpoint{1.127319in}{1.997124in}}%
\pgfpathlineto{\pgfqpoint{1.127319in}{2.041029in}}%
\pgfpathlineto{\pgfqpoint{1.118566in}{2.041029in}}%
\pgfpathlineto{\pgfqpoint{1.118566in}{1.997124in}}%
\pgfpathclose%
\pgfusepath{fill}%
\end{pgfscope}%
\begin{pgfscope}%
\pgfpathrectangle{\pgfqpoint{0.804646in}{0.600000in}}{\pgfqpoint{2.573292in}{2.070576in}}%
\pgfusepath{clip}%
\pgfsetbuttcap%
\pgfsetmiterjoin%
\definecolor{currentfill}{rgb}{0.754268,0.565033,0.211761}%
\pgfsetfillcolor{currentfill}%
\pgfsetlinewidth{0.000000pt}%
\definecolor{currentstroke}{rgb}{0.000000,0.000000,0.000000}%
\pgfsetstrokecolor{currentstroke}%
\pgfsetstrokeopacity{0.000000}%
\pgfsetdash{}{0pt}%
\pgfpathmoveto{\pgfqpoint{1.129508in}{1.162842in}}%
\pgfpathlineto{\pgfqpoint{1.138261in}{1.162842in}}%
\pgfpathlineto{\pgfqpoint{1.138261in}{1.161068in}}%
\pgfpathlineto{\pgfqpoint{1.129508in}{1.161068in}}%
\pgfpathlineto{\pgfqpoint{1.129508in}{1.162842in}}%
\pgfpathclose%
\pgfusepath{fill}%
\end{pgfscope}%
\begin{pgfscope}%
\pgfpathrectangle{\pgfqpoint{0.804646in}{0.600000in}}{\pgfqpoint{2.573292in}{2.070576in}}%
\pgfusepath{clip}%
\pgfsetbuttcap%
\pgfsetmiterjoin%
\definecolor{currentfill}{rgb}{0.754268,0.565033,0.211761}%
\pgfsetfillcolor{currentfill}%
\pgfsetlinewidth{0.000000pt}%
\definecolor{currentstroke}{rgb}{0.000000,0.000000,0.000000}%
\pgfsetstrokecolor{currentstroke}%
\pgfsetstrokeopacity{0.000000}%
\pgfsetdash{}{0pt}%
\pgfpathmoveto{\pgfqpoint{1.140450in}{1.132005in}}%
\pgfpathlineto{\pgfqpoint{1.149203in}{1.132005in}}%
\pgfpathlineto{\pgfqpoint{1.149203in}{1.115528in}}%
\pgfpathlineto{\pgfqpoint{1.140450in}{1.115528in}}%
\pgfpathlineto{\pgfqpoint{1.140450in}{1.132005in}}%
\pgfpathclose%
\pgfusepath{fill}%
\end{pgfscope}%
\begin{pgfscope}%
\pgfpathrectangle{\pgfqpoint{0.804646in}{0.600000in}}{\pgfqpoint{2.573292in}{2.070576in}}%
\pgfusepath{clip}%
\pgfsetbuttcap%
\pgfsetmiterjoin%
\definecolor{currentfill}{rgb}{0.754268,0.565033,0.211761}%
\pgfsetfillcolor{currentfill}%
\pgfsetlinewidth{0.000000pt}%
\definecolor{currentstroke}{rgb}{0.000000,0.000000,0.000000}%
\pgfsetstrokecolor{currentstroke}%
\pgfsetstrokeopacity{0.000000}%
\pgfsetdash{}{0pt}%
\pgfpathmoveto{\pgfqpoint{1.151391in}{2.072681in}}%
\pgfpathlineto{\pgfqpoint{1.160145in}{2.072681in}}%
\pgfpathlineto{\pgfqpoint{1.160145in}{2.094941in}}%
\pgfpathlineto{\pgfqpoint{1.151391in}{2.094941in}}%
\pgfpathlineto{\pgfqpoint{1.151391in}{2.072681in}}%
\pgfpathclose%
\pgfusepath{fill}%
\end{pgfscope}%
\begin{pgfscope}%
\pgfpathrectangle{\pgfqpoint{0.804646in}{0.600000in}}{\pgfqpoint{2.573292in}{2.070576in}}%
\pgfusepath{clip}%
\pgfsetbuttcap%
\pgfsetmiterjoin%
\definecolor{currentfill}{rgb}{0.754268,0.565033,0.211761}%
\pgfsetfillcolor{currentfill}%
\pgfsetlinewidth{0.000000pt}%
\definecolor{currentstroke}{rgb}{0.000000,0.000000,0.000000}%
\pgfsetstrokecolor{currentstroke}%
\pgfsetstrokeopacity{0.000000}%
\pgfsetdash{}{0pt}%
\pgfpathmoveto{\pgfqpoint{1.162333in}{2.213361in}}%
\pgfpathlineto{\pgfqpoint{1.171087in}{2.213361in}}%
\pgfpathlineto{\pgfqpoint{1.171087in}{2.298414in}}%
\pgfpathlineto{\pgfqpoint{1.162333in}{2.298414in}}%
\pgfpathlineto{\pgfqpoint{1.162333in}{2.213361in}}%
\pgfpathclose%
\pgfusepath{fill}%
\end{pgfscope}%
\begin{pgfscope}%
\pgfpathrectangle{\pgfqpoint{0.804646in}{0.600000in}}{\pgfqpoint{2.573292in}{2.070576in}}%
\pgfusepath{clip}%
\pgfsetbuttcap%
\pgfsetmiterjoin%
\definecolor{currentfill}{rgb}{0.754268,0.565033,0.211761}%
\pgfsetfillcolor{currentfill}%
\pgfsetlinewidth{0.000000pt}%
\definecolor{currentstroke}{rgb}{0.000000,0.000000,0.000000}%
\pgfsetstrokecolor{currentstroke}%
\pgfsetstrokeopacity{0.000000}%
\pgfsetdash{}{0pt}%
\pgfpathmoveto{\pgfqpoint{1.173275in}{2.303936in}}%
\pgfpathlineto{\pgfqpoint{1.182028in}{2.303936in}}%
\pgfpathlineto{\pgfqpoint{1.182028in}{2.396936in}}%
\pgfpathlineto{\pgfqpoint{1.173275in}{2.396936in}}%
\pgfpathlineto{\pgfqpoint{1.173275in}{2.303936in}}%
\pgfpathclose%
\pgfusepath{fill}%
\end{pgfscope}%
\begin{pgfscope}%
\pgfpathrectangle{\pgfqpoint{0.804646in}{0.600000in}}{\pgfqpoint{2.573292in}{2.070576in}}%
\pgfusepath{clip}%
\pgfsetbuttcap%
\pgfsetmiterjoin%
\definecolor{currentfill}{rgb}{0.754268,0.565033,0.211761}%
\pgfsetfillcolor{currentfill}%
\pgfsetlinewidth{0.000000pt}%
\definecolor{currentstroke}{rgb}{0.000000,0.000000,0.000000}%
\pgfsetstrokecolor{currentstroke}%
\pgfsetstrokeopacity{0.000000}%
\pgfsetdash{}{0pt}%
\pgfpathmoveto{\pgfqpoint{1.184217in}{2.258912in}}%
\pgfpathlineto{\pgfqpoint{1.192970in}{2.258912in}}%
\pgfpathlineto{\pgfqpoint{1.192970in}{2.311089in}}%
\pgfpathlineto{\pgfqpoint{1.184217in}{2.311089in}}%
\pgfpathlineto{\pgfqpoint{1.184217in}{2.258912in}}%
\pgfpathclose%
\pgfusepath{fill}%
\end{pgfscope}%
\begin{pgfscope}%
\pgfpathrectangle{\pgfqpoint{0.804646in}{0.600000in}}{\pgfqpoint{2.573292in}{2.070576in}}%
\pgfusepath{clip}%
\pgfsetbuttcap%
\pgfsetmiterjoin%
\definecolor{currentfill}{rgb}{0.754268,0.565033,0.211761}%
\pgfsetfillcolor{currentfill}%
\pgfsetlinewidth{0.000000pt}%
\definecolor{currentstroke}{rgb}{0.000000,0.000000,0.000000}%
\pgfsetstrokecolor{currentstroke}%
\pgfsetstrokeopacity{0.000000}%
\pgfsetdash{}{0pt}%
\pgfpathmoveto{\pgfqpoint{1.195159in}{2.287781in}}%
\pgfpathlineto{\pgfqpoint{1.203912in}{2.287781in}}%
\pgfpathlineto{\pgfqpoint{1.203912in}{2.332924in}}%
\pgfpathlineto{\pgfqpoint{1.195159in}{2.332924in}}%
\pgfpathlineto{\pgfqpoint{1.195159in}{2.287781in}}%
\pgfpathclose%
\pgfusepath{fill}%
\end{pgfscope}%
\begin{pgfscope}%
\pgfpathrectangle{\pgfqpoint{0.804646in}{0.600000in}}{\pgfqpoint{2.573292in}{2.070576in}}%
\pgfusepath{clip}%
\pgfsetbuttcap%
\pgfsetmiterjoin%
\definecolor{currentfill}{rgb}{0.754268,0.565033,0.211761}%
\pgfsetfillcolor{currentfill}%
\pgfsetlinewidth{0.000000pt}%
\definecolor{currentstroke}{rgb}{0.000000,0.000000,0.000000}%
\pgfsetstrokecolor{currentstroke}%
\pgfsetstrokeopacity{0.000000}%
\pgfsetdash{}{0pt}%
\pgfpathmoveto{\pgfqpoint{1.206100in}{2.314269in}}%
\pgfpathlineto{\pgfqpoint{1.214854in}{2.314269in}}%
\pgfpathlineto{\pgfqpoint{1.214854in}{2.383296in}}%
\pgfpathlineto{\pgfqpoint{1.206100in}{2.383296in}}%
\pgfpathlineto{\pgfqpoint{1.206100in}{2.314269in}}%
\pgfpathclose%
\pgfusepath{fill}%
\end{pgfscope}%
\begin{pgfscope}%
\pgfpathrectangle{\pgfqpoint{0.804646in}{0.600000in}}{\pgfqpoint{2.573292in}{2.070576in}}%
\pgfusepath{clip}%
\pgfsetbuttcap%
\pgfsetmiterjoin%
\definecolor{currentfill}{rgb}{0.754268,0.565033,0.211761}%
\pgfsetfillcolor{currentfill}%
\pgfsetlinewidth{0.000000pt}%
\definecolor{currentstroke}{rgb}{0.000000,0.000000,0.000000}%
\pgfsetstrokecolor{currentstroke}%
\pgfsetstrokeopacity{0.000000}%
\pgfsetdash{}{0pt}%
\pgfpathmoveto{\pgfqpoint{1.217042in}{2.292827in}}%
\pgfpathlineto{\pgfqpoint{1.225796in}{2.292827in}}%
\pgfpathlineto{\pgfqpoint{1.225796in}{2.392932in}}%
\pgfpathlineto{\pgfqpoint{1.217042in}{2.392932in}}%
\pgfpathlineto{\pgfqpoint{1.217042in}{2.292827in}}%
\pgfpathclose%
\pgfusepath{fill}%
\end{pgfscope}%
\begin{pgfscope}%
\pgfpathrectangle{\pgfqpoint{0.804646in}{0.600000in}}{\pgfqpoint{2.573292in}{2.070576in}}%
\pgfusepath{clip}%
\pgfsetbuttcap%
\pgfsetmiterjoin%
\definecolor{currentfill}{rgb}{0.754268,0.565033,0.211761}%
\pgfsetfillcolor{currentfill}%
\pgfsetlinewidth{0.000000pt}%
\definecolor{currentstroke}{rgb}{0.000000,0.000000,0.000000}%
\pgfsetstrokecolor{currentstroke}%
\pgfsetstrokeopacity{0.000000}%
\pgfsetdash{}{0pt}%
\pgfpathmoveto{\pgfqpoint{1.227984in}{2.239364in}}%
\pgfpathlineto{\pgfqpoint{1.236737in}{2.239364in}}%
\pgfpathlineto{\pgfqpoint{1.236737in}{2.324592in}}%
\pgfpathlineto{\pgfqpoint{1.227984in}{2.324592in}}%
\pgfpathlineto{\pgfqpoint{1.227984in}{2.239364in}}%
\pgfpathclose%
\pgfusepath{fill}%
\end{pgfscope}%
\begin{pgfscope}%
\pgfpathrectangle{\pgfqpoint{0.804646in}{0.600000in}}{\pgfqpoint{2.573292in}{2.070576in}}%
\pgfusepath{clip}%
\pgfsetbuttcap%
\pgfsetmiterjoin%
\definecolor{currentfill}{rgb}{0.754268,0.565033,0.211761}%
\pgfsetfillcolor{currentfill}%
\pgfsetlinewidth{0.000000pt}%
\definecolor{currentstroke}{rgb}{0.000000,0.000000,0.000000}%
\pgfsetstrokecolor{currentstroke}%
\pgfsetstrokeopacity{0.000000}%
\pgfsetdash{}{0pt}%
\pgfpathmoveto{\pgfqpoint{1.238926in}{2.190723in}}%
\pgfpathlineto{\pgfqpoint{1.247679in}{2.190723in}}%
\pgfpathlineto{\pgfqpoint{1.247679in}{2.286586in}}%
\pgfpathlineto{\pgfqpoint{1.238926in}{2.286586in}}%
\pgfpathlineto{\pgfqpoint{1.238926in}{2.190723in}}%
\pgfpathclose%
\pgfusepath{fill}%
\end{pgfscope}%
\begin{pgfscope}%
\pgfpathrectangle{\pgfqpoint{0.804646in}{0.600000in}}{\pgfqpoint{2.573292in}{2.070576in}}%
\pgfusepath{clip}%
\pgfsetbuttcap%
\pgfsetmiterjoin%
\definecolor{currentfill}{rgb}{0.754268,0.565033,0.211761}%
\pgfsetfillcolor{currentfill}%
\pgfsetlinewidth{0.000000pt}%
\definecolor{currentstroke}{rgb}{0.000000,0.000000,0.000000}%
\pgfsetstrokecolor{currentstroke}%
\pgfsetstrokeopacity{0.000000}%
\pgfsetdash{}{0pt}%
\pgfpathmoveto{\pgfqpoint{1.249868in}{2.153157in}}%
\pgfpathlineto{\pgfqpoint{1.258621in}{2.153157in}}%
\pgfpathlineto{\pgfqpoint{1.258621in}{2.242671in}}%
\pgfpathlineto{\pgfqpoint{1.249868in}{2.242671in}}%
\pgfpathlineto{\pgfqpoint{1.249868in}{2.153157in}}%
\pgfpathclose%
\pgfusepath{fill}%
\end{pgfscope}%
\begin{pgfscope}%
\pgfpathrectangle{\pgfqpoint{0.804646in}{0.600000in}}{\pgfqpoint{2.573292in}{2.070576in}}%
\pgfusepath{clip}%
\pgfsetbuttcap%
\pgfsetmiterjoin%
\definecolor{currentfill}{rgb}{0.754268,0.565033,0.211761}%
\pgfsetfillcolor{currentfill}%
\pgfsetlinewidth{0.000000pt}%
\definecolor{currentstroke}{rgb}{0.000000,0.000000,0.000000}%
\pgfsetstrokecolor{currentstroke}%
\pgfsetstrokeopacity{0.000000}%
\pgfsetdash{}{0pt}%
\pgfpathmoveto{\pgfqpoint{1.260809in}{2.040519in}}%
\pgfpathlineto{\pgfqpoint{1.269563in}{2.040519in}}%
\pgfpathlineto{\pgfqpoint{1.269563in}{2.102827in}}%
\pgfpathlineto{\pgfqpoint{1.260809in}{2.102827in}}%
\pgfpathlineto{\pgfqpoint{1.260809in}{2.040519in}}%
\pgfpathclose%
\pgfusepath{fill}%
\end{pgfscope}%
\begin{pgfscope}%
\pgfpathrectangle{\pgfqpoint{0.804646in}{0.600000in}}{\pgfqpoint{2.573292in}{2.070576in}}%
\pgfusepath{clip}%
\pgfsetbuttcap%
\pgfsetmiterjoin%
\definecolor{currentfill}{rgb}{0.754268,0.565033,0.211761}%
\pgfsetfillcolor{currentfill}%
\pgfsetlinewidth{0.000000pt}%
\definecolor{currentstroke}{rgb}{0.000000,0.000000,0.000000}%
\pgfsetstrokecolor{currentstroke}%
\pgfsetstrokeopacity{0.000000}%
\pgfsetdash{}{0pt}%
\pgfpathmoveto{\pgfqpoint{1.271751in}{2.024034in}}%
\pgfpathlineto{\pgfqpoint{1.280505in}{2.024034in}}%
\pgfpathlineto{\pgfqpoint{1.280505in}{2.085436in}}%
\pgfpathlineto{\pgfqpoint{1.271751in}{2.085436in}}%
\pgfpathlineto{\pgfqpoint{1.271751in}{2.024034in}}%
\pgfpathclose%
\pgfusepath{fill}%
\end{pgfscope}%
\begin{pgfscope}%
\pgfpathrectangle{\pgfqpoint{0.804646in}{0.600000in}}{\pgfqpoint{2.573292in}{2.070576in}}%
\pgfusepath{clip}%
\pgfsetbuttcap%
\pgfsetmiterjoin%
\definecolor{currentfill}{rgb}{0.754268,0.565033,0.211761}%
\pgfsetfillcolor{currentfill}%
\pgfsetlinewidth{0.000000pt}%
\definecolor{currentstroke}{rgb}{0.000000,0.000000,0.000000}%
\pgfsetstrokecolor{currentstroke}%
\pgfsetstrokeopacity{0.000000}%
\pgfsetdash{}{0pt}%
\pgfpathmoveto{\pgfqpoint{1.282693in}{1.956887in}}%
\pgfpathlineto{\pgfqpoint{1.291446in}{1.956887in}}%
\pgfpathlineto{\pgfqpoint{1.291446in}{2.011326in}}%
\pgfpathlineto{\pgfqpoint{1.282693in}{2.011326in}}%
\pgfpathlineto{\pgfqpoint{1.282693in}{1.956887in}}%
\pgfpathclose%
\pgfusepath{fill}%
\end{pgfscope}%
\begin{pgfscope}%
\pgfpathrectangle{\pgfqpoint{0.804646in}{0.600000in}}{\pgfqpoint{2.573292in}{2.070576in}}%
\pgfusepath{clip}%
\pgfsetbuttcap%
\pgfsetmiterjoin%
\definecolor{currentfill}{rgb}{0.754268,0.565033,0.211761}%
\pgfsetfillcolor{currentfill}%
\pgfsetlinewidth{0.000000pt}%
\definecolor{currentstroke}{rgb}{0.000000,0.000000,0.000000}%
\pgfsetstrokecolor{currentstroke}%
\pgfsetstrokeopacity{0.000000}%
\pgfsetdash{}{0pt}%
\pgfpathmoveto{\pgfqpoint{1.293635in}{1.292228in}}%
\pgfpathlineto{\pgfqpoint{1.302388in}{1.292228in}}%
\pgfpathlineto{\pgfqpoint{1.302388in}{1.253055in}}%
\pgfpathlineto{\pgfqpoint{1.293635in}{1.253055in}}%
\pgfpathlineto{\pgfqpoint{1.293635in}{1.292228in}}%
\pgfpathclose%
\pgfusepath{fill}%
\end{pgfscope}%
\begin{pgfscope}%
\pgfpathrectangle{\pgfqpoint{0.804646in}{0.600000in}}{\pgfqpoint{2.573292in}{2.070576in}}%
\pgfusepath{clip}%
\pgfsetbuttcap%
\pgfsetmiterjoin%
\definecolor{currentfill}{rgb}{0.754268,0.565033,0.211761}%
\pgfsetfillcolor{currentfill}%
\pgfsetlinewidth{0.000000pt}%
\definecolor{currentstroke}{rgb}{0.000000,0.000000,0.000000}%
\pgfsetstrokecolor{currentstroke}%
\pgfsetstrokeopacity{0.000000}%
\pgfsetdash{}{0pt}%
\pgfpathmoveto{\pgfqpoint{1.304577in}{1.305331in}}%
\pgfpathlineto{\pgfqpoint{1.313330in}{1.305331in}}%
\pgfpathlineto{\pgfqpoint{1.313330in}{1.231894in}}%
\pgfpathlineto{\pgfqpoint{1.304577in}{1.231894in}}%
\pgfpathlineto{\pgfqpoint{1.304577in}{1.305331in}}%
\pgfpathclose%
\pgfusepath{fill}%
\end{pgfscope}%
\begin{pgfscope}%
\pgfpathrectangle{\pgfqpoint{0.804646in}{0.600000in}}{\pgfqpoint{2.573292in}{2.070576in}}%
\pgfusepath{clip}%
\pgfsetbuttcap%
\pgfsetmiterjoin%
\definecolor{currentfill}{rgb}{0.754268,0.565033,0.211761}%
\pgfsetfillcolor{currentfill}%
\pgfsetlinewidth{0.000000pt}%
\definecolor{currentstroke}{rgb}{0.000000,0.000000,0.000000}%
\pgfsetstrokecolor{currentstroke}%
\pgfsetstrokeopacity{0.000000}%
\pgfsetdash{}{0pt}%
\pgfpathmoveto{\pgfqpoint{1.315518in}{1.379122in}}%
\pgfpathlineto{\pgfqpoint{1.324272in}{1.379122in}}%
\pgfpathlineto{\pgfqpoint{1.324272in}{1.286165in}}%
\pgfpathlineto{\pgfqpoint{1.315518in}{1.286165in}}%
\pgfpathlineto{\pgfqpoint{1.315518in}{1.379122in}}%
\pgfpathclose%
\pgfusepath{fill}%
\end{pgfscope}%
\begin{pgfscope}%
\pgfpathrectangle{\pgfqpoint{0.804646in}{0.600000in}}{\pgfqpoint{2.573292in}{2.070576in}}%
\pgfusepath{clip}%
\pgfsetbuttcap%
\pgfsetmiterjoin%
\definecolor{currentfill}{rgb}{0.754268,0.565033,0.211761}%
\pgfsetfillcolor{currentfill}%
\pgfsetlinewidth{0.000000pt}%
\definecolor{currentstroke}{rgb}{0.000000,0.000000,0.000000}%
\pgfsetstrokecolor{currentstroke}%
\pgfsetstrokeopacity{0.000000}%
\pgfsetdash{}{0pt}%
\pgfpathmoveto{\pgfqpoint{1.326460in}{1.396613in}}%
\pgfpathlineto{\pgfqpoint{1.335214in}{1.396613in}}%
\pgfpathlineto{\pgfqpoint{1.335214in}{1.317562in}}%
\pgfpathlineto{\pgfqpoint{1.326460in}{1.317562in}}%
\pgfpathlineto{\pgfqpoint{1.326460in}{1.396613in}}%
\pgfpathclose%
\pgfusepath{fill}%
\end{pgfscope}%
\begin{pgfscope}%
\pgfpathrectangle{\pgfqpoint{0.804646in}{0.600000in}}{\pgfqpoint{2.573292in}{2.070576in}}%
\pgfusepath{clip}%
\pgfsetbuttcap%
\pgfsetmiterjoin%
\definecolor{currentfill}{rgb}{0.754268,0.565033,0.211761}%
\pgfsetfillcolor{currentfill}%
\pgfsetlinewidth{0.000000pt}%
\definecolor{currentstroke}{rgb}{0.000000,0.000000,0.000000}%
\pgfsetstrokecolor{currentstroke}%
\pgfsetstrokeopacity{0.000000}%
\pgfsetdash{}{0pt}%
\pgfpathmoveto{\pgfqpoint{1.337402in}{1.362958in}}%
\pgfpathlineto{\pgfqpoint{1.346155in}{1.362958in}}%
\pgfpathlineto{\pgfqpoint{1.346155in}{1.309747in}}%
\pgfpathlineto{\pgfqpoint{1.337402in}{1.309747in}}%
\pgfpathlineto{\pgfqpoint{1.337402in}{1.362958in}}%
\pgfpathclose%
\pgfusepath{fill}%
\end{pgfscope}%
\begin{pgfscope}%
\pgfpathrectangle{\pgfqpoint{0.804646in}{0.600000in}}{\pgfqpoint{2.573292in}{2.070576in}}%
\pgfusepath{clip}%
\pgfsetbuttcap%
\pgfsetmiterjoin%
\definecolor{currentfill}{rgb}{0.754268,0.565033,0.211761}%
\pgfsetfillcolor{currentfill}%
\pgfsetlinewidth{0.000000pt}%
\definecolor{currentstroke}{rgb}{0.000000,0.000000,0.000000}%
\pgfsetstrokecolor{currentstroke}%
\pgfsetstrokeopacity{0.000000}%
\pgfsetdash{}{0pt}%
\pgfpathmoveto{\pgfqpoint{1.348344in}{1.316484in}}%
\pgfpathlineto{\pgfqpoint{1.357097in}{1.316484in}}%
\pgfpathlineto{\pgfqpoint{1.357097in}{1.272118in}}%
\pgfpathlineto{\pgfqpoint{1.348344in}{1.272118in}}%
\pgfpathlineto{\pgfqpoint{1.348344in}{1.316484in}}%
\pgfpathclose%
\pgfusepath{fill}%
\end{pgfscope}%
\begin{pgfscope}%
\pgfpathrectangle{\pgfqpoint{0.804646in}{0.600000in}}{\pgfqpoint{2.573292in}{2.070576in}}%
\pgfusepath{clip}%
\pgfsetbuttcap%
\pgfsetmiterjoin%
\definecolor{currentfill}{rgb}{0.754268,0.565033,0.211761}%
\pgfsetfillcolor{currentfill}%
\pgfsetlinewidth{0.000000pt}%
\definecolor{currentstroke}{rgb}{0.000000,0.000000,0.000000}%
\pgfsetstrokecolor{currentstroke}%
\pgfsetstrokeopacity{0.000000}%
\pgfsetdash{}{0pt}%
\pgfpathmoveto{\pgfqpoint{1.359286in}{1.314360in}}%
\pgfpathlineto{\pgfqpoint{1.368039in}{1.314360in}}%
\pgfpathlineto{\pgfqpoint{1.368039in}{1.280009in}}%
\pgfpathlineto{\pgfqpoint{1.359286in}{1.280009in}}%
\pgfpathlineto{\pgfqpoint{1.359286in}{1.314360in}}%
\pgfpathclose%
\pgfusepath{fill}%
\end{pgfscope}%
\begin{pgfscope}%
\pgfpathrectangle{\pgfqpoint{0.804646in}{0.600000in}}{\pgfqpoint{2.573292in}{2.070576in}}%
\pgfusepath{clip}%
\pgfsetbuttcap%
\pgfsetmiterjoin%
\definecolor{currentfill}{rgb}{0.754268,0.565033,0.211761}%
\pgfsetfillcolor{currentfill}%
\pgfsetlinewidth{0.000000pt}%
\definecolor{currentstroke}{rgb}{0.000000,0.000000,0.000000}%
\pgfsetstrokecolor{currentstroke}%
\pgfsetstrokeopacity{0.000000}%
\pgfsetdash{}{0pt}%
\pgfpathmoveto{\pgfqpoint{1.370227in}{1.342759in}}%
\pgfpathlineto{\pgfqpoint{1.378981in}{1.342759in}}%
\pgfpathlineto{\pgfqpoint{1.378981in}{1.299774in}}%
\pgfpathlineto{\pgfqpoint{1.370227in}{1.299774in}}%
\pgfpathlineto{\pgfqpoint{1.370227in}{1.342759in}}%
\pgfpathclose%
\pgfusepath{fill}%
\end{pgfscope}%
\begin{pgfscope}%
\pgfpathrectangle{\pgfqpoint{0.804646in}{0.600000in}}{\pgfqpoint{2.573292in}{2.070576in}}%
\pgfusepath{clip}%
\pgfsetbuttcap%
\pgfsetmiterjoin%
\definecolor{currentfill}{rgb}{0.754268,0.565033,0.211761}%
\pgfsetfillcolor{currentfill}%
\pgfsetlinewidth{0.000000pt}%
\definecolor{currentstroke}{rgb}{0.000000,0.000000,0.000000}%
\pgfsetstrokecolor{currentstroke}%
\pgfsetstrokeopacity{0.000000}%
\pgfsetdash{}{0pt}%
\pgfpathmoveto{\pgfqpoint{1.381169in}{1.321919in}}%
\pgfpathlineto{\pgfqpoint{1.389923in}{1.321919in}}%
\pgfpathlineto{\pgfqpoint{1.389923in}{1.294724in}}%
\pgfpathlineto{\pgfqpoint{1.381169in}{1.294724in}}%
\pgfpathlineto{\pgfqpoint{1.381169in}{1.321919in}}%
\pgfpathclose%
\pgfusepath{fill}%
\end{pgfscope}%
\begin{pgfscope}%
\pgfpathrectangle{\pgfqpoint{0.804646in}{0.600000in}}{\pgfqpoint{2.573292in}{2.070576in}}%
\pgfusepath{clip}%
\pgfsetbuttcap%
\pgfsetmiterjoin%
\definecolor{currentfill}{rgb}{0.754268,0.565033,0.211761}%
\pgfsetfillcolor{currentfill}%
\pgfsetlinewidth{0.000000pt}%
\definecolor{currentstroke}{rgb}{0.000000,0.000000,0.000000}%
\pgfsetstrokecolor{currentstroke}%
\pgfsetstrokeopacity{0.000000}%
\pgfsetdash{}{0pt}%
\pgfpathmoveto{\pgfqpoint{1.392111in}{1.949455in}}%
\pgfpathlineto{\pgfqpoint{1.400864in}{1.949455in}}%
\pgfpathlineto{\pgfqpoint{1.400864in}{1.958966in}}%
\pgfpathlineto{\pgfqpoint{1.392111in}{1.958966in}}%
\pgfpathlineto{\pgfqpoint{1.392111in}{1.949455in}}%
\pgfpathclose%
\pgfusepath{fill}%
\end{pgfscope}%
\begin{pgfscope}%
\pgfpathrectangle{\pgfqpoint{0.804646in}{0.600000in}}{\pgfqpoint{2.573292in}{2.070576in}}%
\pgfusepath{clip}%
\pgfsetbuttcap%
\pgfsetmiterjoin%
\definecolor{currentfill}{rgb}{0.754268,0.565033,0.211761}%
\pgfsetfillcolor{currentfill}%
\pgfsetlinewidth{0.000000pt}%
\definecolor{currentstroke}{rgb}{0.000000,0.000000,0.000000}%
\pgfsetstrokecolor{currentstroke}%
\pgfsetstrokeopacity{0.000000}%
\pgfsetdash{}{0pt}%
\pgfpathmoveto{\pgfqpoint{1.403053in}{1.233540in}}%
\pgfpathlineto{\pgfqpoint{1.411806in}{1.233540in}}%
\pgfpathlineto{\pgfqpoint{1.411806in}{1.227574in}}%
\pgfpathlineto{\pgfqpoint{1.403053in}{1.227574in}}%
\pgfpathlineto{\pgfqpoint{1.403053in}{1.233540in}}%
\pgfpathclose%
\pgfusepath{fill}%
\end{pgfscope}%
\begin{pgfscope}%
\pgfpathrectangle{\pgfqpoint{0.804646in}{0.600000in}}{\pgfqpoint{2.573292in}{2.070576in}}%
\pgfusepath{clip}%
\pgfsetbuttcap%
\pgfsetmiterjoin%
\definecolor{currentfill}{rgb}{0.754268,0.565033,0.211761}%
\pgfsetfillcolor{currentfill}%
\pgfsetlinewidth{0.000000pt}%
\definecolor{currentstroke}{rgb}{0.000000,0.000000,0.000000}%
\pgfsetstrokecolor{currentstroke}%
\pgfsetstrokeopacity{0.000000}%
\pgfsetdash{}{0pt}%
\pgfpathmoveto{\pgfqpoint{1.413995in}{1.356689in}}%
\pgfpathlineto{\pgfqpoint{1.422748in}{1.356689in}}%
\pgfpathlineto{\pgfqpoint{1.422748in}{1.327512in}}%
\pgfpathlineto{\pgfqpoint{1.413995in}{1.327512in}}%
\pgfpathlineto{\pgfqpoint{1.413995in}{1.356689in}}%
\pgfpathclose%
\pgfusepath{fill}%
\end{pgfscope}%
\begin{pgfscope}%
\pgfpathrectangle{\pgfqpoint{0.804646in}{0.600000in}}{\pgfqpoint{2.573292in}{2.070576in}}%
\pgfusepath{clip}%
\pgfsetbuttcap%
\pgfsetmiterjoin%
\definecolor{currentfill}{rgb}{0.754268,0.565033,0.211761}%
\pgfsetfillcolor{currentfill}%
\pgfsetlinewidth{0.000000pt}%
\definecolor{currentstroke}{rgb}{0.000000,0.000000,0.000000}%
\pgfsetstrokecolor{currentstroke}%
\pgfsetstrokeopacity{0.000000}%
\pgfsetdash{}{0pt}%
\pgfpathmoveto{\pgfqpoint{1.424936in}{1.979430in}}%
\pgfpathlineto{\pgfqpoint{1.433690in}{1.979430in}}%
\pgfpathlineto{\pgfqpoint{1.433690in}{2.006403in}}%
\pgfpathlineto{\pgfqpoint{1.424936in}{2.006403in}}%
\pgfpathlineto{\pgfqpoint{1.424936in}{1.979430in}}%
\pgfpathclose%
\pgfusepath{fill}%
\end{pgfscope}%
\begin{pgfscope}%
\pgfpathrectangle{\pgfqpoint{0.804646in}{0.600000in}}{\pgfqpoint{2.573292in}{2.070576in}}%
\pgfusepath{clip}%
\pgfsetbuttcap%
\pgfsetmiterjoin%
\definecolor{currentfill}{rgb}{0.754268,0.565033,0.211761}%
\pgfsetfillcolor{currentfill}%
\pgfsetlinewidth{0.000000pt}%
\definecolor{currentstroke}{rgb}{0.000000,0.000000,0.000000}%
\pgfsetstrokecolor{currentstroke}%
\pgfsetstrokeopacity{0.000000}%
\pgfsetdash{}{0pt}%
\pgfpathmoveto{\pgfqpoint{1.435878in}{1.979630in}}%
\pgfpathlineto{\pgfqpoint{1.444632in}{1.979630in}}%
\pgfpathlineto{\pgfqpoint{1.444632in}{2.025450in}}%
\pgfpathlineto{\pgfqpoint{1.435878in}{2.025450in}}%
\pgfpathlineto{\pgfqpoint{1.435878in}{1.979630in}}%
\pgfpathclose%
\pgfusepath{fill}%
\end{pgfscope}%
\begin{pgfscope}%
\pgfpathrectangle{\pgfqpoint{0.804646in}{0.600000in}}{\pgfqpoint{2.573292in}{2.070576in}}%
\pgfusepath{clip}%
\pgfsetbuttcap%
\pgfsetmiterjoin%
\definecolor{currentfill}{rgb}{0.754268,0.565033,0.211761}%
\pgfsetfillcolor{currentfill}%
\pgfsetlinewidth{0.000000pt}%
\definecolor{currentstroke}{rgb}{0.000000,0.000000,0.000000}%
\pgfsetstrokecolor{currentstroke}%
\pgfsetstrokeopacity{0.000000}%
\pgfsetdash{}{0pt}%
\pgfpathmoveto{\pgfqpoint{1.446820in}{1.991622in}}%
\pgfpathlineto{\pgfqpoint{1.455573in}{1.991622in}}%
\pgfpathlineto{\pgfqpoint{1.455573in}{2.025975in}}%
\pgfpathlineto{\pgfqpoint{1.446820in}{2.025975in}}%
\pgfpathlineto{\pgfqpoint{1.446820in}{1.991622in}}%
\pgfpathclose%
\pgfusepath{fill}%
\end{pgfscope}%
\begin{pgfscope}%
\pgfpathrectangle{\pgfqpoint{0.804646in}{0.600000in}}{\pgfqpoint{2.573292in}{2.070576in}}%
\pgfusepath{clip}%
\pgfsetbuttcap%
\pgfsetmiterjoin%
\definecolor{currentfill}{rgb}{0.754268,0.565033,0.211761}%
\pgfsetfillcolor{currentfill}%
\pgfsetlinewidth{0.000000pt}%
\definecolor{currentstroke}{rgb}{0.000000,0.000000,0.000000}%
\pgfsetstrokecolor{currentstroke}%
\pgfsetstrokeopacity{0.000000}%
\pgfsetdash{}{0pt}%
\pgfpathmoveto{\pgfqpoint{1.457762in}{1.986485in}}%
\pgfpathlineto{\pgfqpoint{1.466515in}{1.986485in}}%
\pgfpathlineto{\pgfqpoint{1.466515in}{2.035190in}}%
\pgfpathlineto{\pgfqpoint{1.457762in}{2.035190in}}%
\pgfpathlineto{\pgfqpoint{1.457762in}{1.986485in}}%
\pgfpathclose%
\pgfusepath{fill}%
\end{pgfscope}%
\begin{pgfscope}%
\pgfpathrectangle{\pgfqpoint{0.804646in}{0.600000in}}{\pgfqpoint{2.573292in}{2.070576in}}%
\pgfusepath{clip}%
\pgfsetbuttcap%
\pgfsetmiterjoin%
\definecolor{currentfill}{rgb}{0.754268,0.565033,0.211761}%
\pgfsetfillcolor{currentfill}%
\pgfsetlinewidth{0.000000pt}%
\definecolor{currentstroke}{rgb}{0.000000,0.000000,0.000000}%
\pgfsetstrokecolor{currentstroke}%
\pgfsetstrokeopacity{0.000000}%
\pgfsetdash{}{0pt}%
\pgfpathmoveto{\pgfqpoint{1.468704in}{2.021886in}}%
\pgfpathlineto{\pgfqpoint{1.477457in}{2.021886in}}%
\pgfpathlineto{\pgfqpoint{1.477457in}{2.108086in}}%
\pgfpathlineto{\pgfqpoint{1.468704in}{2.108086in}}%
\pgfpathlineto{\pgfqpoint{1.468704in}{2.021886in}}%
\pgfpathclose%
\pgfusepath{fill}%
\end{pgfscope}%
\begin{pgfscope}%
\pgfpathrectangle{\pgfqpoint{0.804646in}{0.600000in}}{\pgfqpoint{2.573292in}{2.070576in}}%
\pgfusepath{clip}%
\pgfsetbuttcap%
\pgfsetmiterjoin%
\definecolor{currentfill}{rgb}{0.754268,0.565033,0.211761}%
\pgfsetfillcolor{currentfill}%
\pgfsetlinewidth{0.000000pt}%
\definecolor{currentstroke}{rgb}{0.000000,0.000000,0.000000}%
\pgfsetstrokecolor{currentstroke}%
\pgfsetstrokeopacity{0.000000}%
\pgfsetdash{}{0pt}%
\pgfpathmoveto{\pgfqpoint{1.479645in}{1.861769in}}%
\pgfpathlineto{\pgfqpoint{1.488399in}{1.861769in}}%
\pgfpathlineto{\pgfqpoint{1.488399in}{1.908728in}}%
\pgfpathlineto{\pgfqpoint{1.479645in}{1.908728in}}%
\pgfpathlineto{\pgfqpoint{1.479645in}{1.861769in}}%
\pgfpathclose%
\pgfusepath{fill}%
\end{pgfscope}%
\begin{pgfscope}%
\pgfpathrectangle{\pgfqpoint{0.804646in}{0.600000in}}{\pgfqpoint{2.573292in}{2.070576in}}%
\pgfusepath{clip}%
\pgfsetbuttcap%
\pgfsetmiterjoin%
\definecolor{currentfill}{rgb}{0.754268,0.565033,0.211761}%
\pgfsetfillcolor{currentfill}%
\pgfsetlinewidth{0.000000pt}%
\definecolor{currentstroke}{rgb}{0.000000,0.000000,0.000000}%
\pgfsetstrokecolor{currentstroke}%
\pgfsetstrokeopacity{0.000000}%
\pgfsetdash{}{0pt}%
\pgfpathmoveto{\pgfqpoint{1.490587in}{1.918839in}}%
\pgfpathlineto{\pgfqpoint{1.499341in}{1.918839in}}%
\pgfpathlineto{\pgfqpoint{1.499341in}{1.977894in}}%
\pgfpathlineto{\pgfqpoint{1.490587in}{1.977894in}}%
\pgfpathlineto{\pgfqpoint{1.490587in}{1.918839in}}%
\pgfpathclose%
\pgfusepath{fill}%
\end{pgfscope}%
\begin{pgfscope}%
\pgfpathrectangle{\pgfqpoint{0.804646in}{0.600000in}}{\pgfqpoint{2.573292in}{2.070576in}}%
\pgfusepath{clip}%
\pgfsetbuttcap%
\pgfsetmiterjoin%
\definecolor{currentfill}{rgb}{0.754268,0.565033,0.211761}%
\pgfsetfillcolor{currentfill}%
\pgfsetlinewidth{0.000000pt}%
\definecolor{currentstroke}{rgb}{0.000000,0.000000,0.000000}%
\pgfsetstrokecolor{currentstroke}%
\pgfsetstrokeopacity{0.000000}%
\pgfsetdash{}{0pt}%
\pgfpathmoveto{\pgfqpoint{1.501529in}{1.939074in}}%
\pgfpathlineto{\pgfqpoint{1.510282in}{1.939074in}}%
\pgfpathlineto{\pgfqpoint{1.510282in}{2.004664in}}%
\pgfpathlineto{\pgfqpoint{1.501529in}{2.004664in}}%
\pgfpathlineto{\pgfqpoint{1.501529in}{1.939074in}}%
\pgfpathclose%
\pgfusepath{fill}%
\end{pgfscope}%
\begin{pgfscope}%
\pgfpathrectangle{\pgfqpoint{0.804646in}{0.600000in}}{\pgfqpoint{2.573292in}{2.070576in}}%
\pgfusepath{clip}%
\pgfsetbuttcap%
\pgfsetmiterjoin%
\definecolor{currentfill}{rgb}{0.754268,0.565033,0.211761}%
\pgfsetfillcolor{currentfill}%
\pgfsetlinewidth{0.000000pt}%
\definecolor{currentstroke}{rgb}{0.000000,0.000000,0.000000}%
\pgfsetstrokecolor{currentstroke}%
\pgfsetstrokeopacity{0.000000}%
\pgfsetdash{}{0pt}%
\pgfpathmoveto{\pgfqpoint{1.512471in}{1.875335in}}%
\pgfpathlineto{\pgfqpoint{1.521224in}{1.875335in}}%
\pgfpathlineto{\pgfqpoint{1.521224in}{1.964674in}}%
\pgfpathlineto{\pgfqpoint{1.512471in}{1.964674in}}%
\pgfpathlineto{\pgfqpoint{1.512471in}{1.875335in}}%
\pgfpathclose%
\pgfusepath{fill}%
\end{pgfscope}%
\begin{pgfscope}%
\pgfpathrectangle{\pgfqpoint{0.804646in}{0.600000in}}{\pgfqpoint{2.573292in}{2.070576in}}%
\pgfusepath{clip}%
\pgfsetbuttcap%
\pgfsetmiterjoin%
\definecolor{currentfill}{rgb}{0.754268,0.565033,0.211761}%
\pgfsetfillcolor{currentfill}%
\pgfsetlinewidth{0.000000pt}%
\definecolor{currentstroke}{rgb}{0.000000,0.000000,0.000000}%
\pgfsetstrokecolor{currentstroke}%
\pgfsetstrokeopacity{0.000000}%
\pgfsetdash{}{0pt}%
\pgfpathmoveto{\pgfqpoint{1.523413in}{1.814166in}}%
\pgfpathlineto{\pgfqpoint{1.532166in}{1.814166in}}%
\pgfpathlineto{\pgfqpoint{1.532166in}{1.913184in}}%
\pgfpathlineto{\pgfqpoint{1.523413in}{1.913184in}}%
\pgfpathlineto{\pgfqpoint{1.523413in}{1.814166in}}%
\pgfpathclose%
\pgfusepath{fill}%
\end{pgfscope}%
\begin{pgfscope}%
\pgfpathrectangle{\pgfqpoint{0.804646in}{0.600000in}}{\pgfqpoint{2.573292in}{2.070576in}}%
\pgfusepath{clip}%
\pgfsetbuttcap%
\pgfsetmiterjoin%
\definecolor{currentfill}{rgb}{0.754268,0.565033,0.211761}%
\pgfsetfillcolor{currentfill}%
\pgfsetlinewidth{0.000000pt}%
\definecolor{currentstroke}{rgb}{0.000000,0.000000,0.000000}%
\pgfsetstrokecolor{currentstroke}%
\pgfsetstrokeopacity{0.000000}%
\pgfsetdash{}{0pt}%
\pgfpathmoveto{\pgfqpoint{1.534354in}{1.727465in}}%
\pgfpathlineto{\pgfqpoint{1.543108in}{1.727465in}}%
\pgfpathlineto{\pgfqpoint{1.543108in}{1.822676in}}%
\pgfpathlineto{\pgfqpoint{1.534354in}{1.822676in}}%
\pgfpathlineto{\pgfqpoint{1.534354in}{1.727465in}}%
\pgfpathclose%
\pgfusepath{fill}%
\end{pgfscope}%
\begin{pgfscope}%
\pgfpathrectangle{\pgfqpoint{0.804646in}{0.600000in}}{\pgfqpoint{2.573292in}{2.070576in}}%
\pgfusepath{clip}%
\pgfsetbuttcap%
\pgfsetmiterjoin%
\definecolor{currentfill}{rgb}{0.754268,0.565033,0.211761}%
\pgfsetfillcolor{currentfill}%
\pgfsetlinewidth{0.000000pt}%
\definecolor{currentstroke}{rgb}{0.000000,0.000000,0.000000}%
\pgfsetstrokecolor{currentstroke}%
\pgfsetstrokeopacity{0.000000}%
\pgfsetdash{}{0pt}%
\pgfpathmoveto{\pgfqpoint{1.545296in}{1.501791in}}%
\pgfpathlineto{\pgfqpoint{1.554050in}{1.501791in}}%
\pgfpathlineto{\pgfqpoint{1.554050in}{1.496575in}}%
\pgfpathlineto{\pgfqpoint{1.545296in}{1.496575in}}%
\pgfpathlineto{\pgfqpoint{1.545296in}{1.501791in}}%
\pgfpathclose%
\pgfusepath{fill}%
\end{pgfscope}%
\begin{pgfscope}%
\pgfpathrectangle{\pgfqpoint{0.804646in}{0.600000in}}{\pgfqpoint{2.573292in}{2.070576in}}%
\pgfusepath{clip}%
\pgfsetbuttcap%
\pgfsetmiterjoin%
\definecolor{currentfill}{rgb}{0.754268,0.565033,0.211761}%
\pgfsetfillcolor{currentfill}%
\pgfsetlinewidth{0.000000pt}%
\definecolor{currentstroke}{rgb}{0.000000,0.000000,0.000000}%
\pgfsetstrokecolor{currentstroke}%
\pgfsetstrokeopacity{0.000000}%
\pgfsetdash{}{0pt}%
\pgfpathmoveto{\pgfqpoint{1.556238in}{1.692111in}}%
\pgfpathlineto{\pgfqpoint{1.564991in}{1.692111in}}%
\pgfpathlineto{\pgfqpoint{1.564991in}{1.724473in}}%
\pgfpathlineto{\pgfqpoint{1.556238in}{1.724473in}}%
\pgfpathlineto{\pgfqpoint{1.556238in}{1.692111in}}%
\pgfpathclose%
\pgfusepath{fill}%
\end{pgfscope}%
\begin{pgfscope}%
\pgfpathrectangle{\pgfqpoint{0.804646in}{0.600000in}}{\pgfqpoint{2.573292in}{2.070576in}}%
\pgfusepath{clip}%
\pgfsetbuttcap%
\pgfsetmiterjoin%
\definecolor{currentfill}{rgb}{0.754268,0.565033,0.211761}%
\pgfsetfillcolor{currentfill}%
\pgfsetlinewidth{0.000000pt}%
\definecolor{currentstroke}{rgb}{0.000000,0.000000,0.000000}%
\pgfsetstrokecolor{currentstroke}%
\pgfsetstrokeopacity{0.000000}%
\pgfsetdash{}{0pt}%
\pgfpathmoveto{\pgfqpoint{1.567180in}{1.725026in}}%
\pgfpathlineto{\pgfqpoint{1.575933in}{1.725026in}}%
\pgfpathlineto{\pgfqpoint{1.575933in}{1.886316in}}%
\pgfpathlineto{\pgfqpoint{1.567180in}{1.886316in}}%
\pgfpathlineto{\pgfqpoint{1.567180in}{1.725026in}}%
\pgfpathclose%
\pgfusepath{fill}%
\end{pgfscope}%
\begin{pgfscope}%
\pgfpathrectangle{\pgfqpoint{0.804646in}{0.600000in}}{\pgfqpoint{2.573292in}{2.070576in}}%
\pgfusepath{clip}%
\pgfsetbuttcap%
\pgfsetmiterjoin%
\definecolor{currentfill}{rgb}{0.754268,0.565033,0.211761}%
\pgfsetfillcolor{currentfill}%
\pgfsetlinewidth{0.000000pt}%
\definecolor{currentstroke}{rgb}{0.000000,0.000000,0.000000}%
\pgfsetstrokecolor{currentstroke}%
\pgfsetstrokeopacity{0.000000}%
\pgfsetdash{}{0pt}%
\pgfpathmoveto{\pgfqpoint{1.578122in}{1.684702in}}%
\pgfpathlineto{\pgfqpoint{1.586875in}{1.684702in}}%
\pgfpathlineto{\pgfqpoint{1.586875in}{1.770229in}}%
\pgfpathlineto{\pgfqpoint{1.578122in}{1.770229in}}%
\pgfpathlineto{\pgfqpoint{1.578122in}{1.684702in}}%
\pgfpathclose%
\pgfusepath{fill}%
\end{pgfscope}%
\begin{pgfscope}%
\pgfpathrectangle{\pgfqpoint{0.804646in}{0.600000in}}{\pgfqpoint{2.573292in}{2.070576in}}%
\pgfusepath{clip}%
\pgfsetbuttcap%
\pgfsetmiterjoin%
\definecolor{currentfill}{rgb}{0.754268,0.565033,0.211761}%
\pgfsetfillcolor{currentfill}%
\pgfsetlinewidth{0.000000pt}%
\definecolor{currentstroke}{rgb}{0.000000,0.000000,0.000000}%
\pgfsetstrokecolor{currentstroke}%
\pgfsetstrokeopacity{0.000000}%
\pgfsetdash{}{0pt}%
\pgfpathmoveto{\pgfqpoint{1.589063in}{1.652185in}}%
\pgfpathlineto{\pgfqpoint{1.597817in}{1.652185in}}%
\pgfpathlineto{\pgfqpoint{1.597817in}{1.772286in}}%
\pgfpathlineto{\pgfqpoint{1.589063in}{1.772286in}}%
\pgfpathlineto{\pgfqpoint{1.589063in}{1.652185in}}%
\pgfpathclose%
\pgfusepath{fill}%
\end{pgfscope}%
\begin{pgfscope}%
\pgfpathrectangle{\pgfqpoint{0.804646in}{0.600000in}}{\pgfqpoint{2.573292in}{2.070576in}}%
\pgfusepath{clip}%
\pgfsetbuttcap%
\pgfsetmiterjoin%
\definecolor{currentfill}{rgb}{0.754268,0.565033,0.211761}%
\pgfsetfillcolor{currentfill}%
\pgfsetlinewidth{0.000000pt}%
\definecolor{currentstroke}{rgb}{0.000000,0.000000,0.000000}%
\pgfsetstrokecolor{currentstroke}%
\pgfsetstrokeopacity{0.000000}%
\pgfsetdash{}{0pt}%
\pgfpathmoveto{\pgfqpoint{1.600005in}{1.657650in}}%
\pgfpathlineto{\pgfqpoint{1.608759in}{1.657650in}}%
\pgfpathlineto{\pgfqpoint{1.608759in}{1.725710in}}%
\pgfpathlineto{\pgfqpoint{1.600005in}{1.725710in}}%
\pgfpathlineto{\pgfqpoint{1.600005in}{1.657650in}}%
\pgfpathclose%
\pgfusepath{fill}%
\end{pgfscope}%
\begin{pgfscope}%
\pgfpathrectangle{\pgfqpoint{0.804646in}{0.600000in}}{\pgfqpoint{2.573292in}{2.070576in}}%
\pgfusepath{clip}%
\pgfsetbuttcap%
\pgfsetmiterjoin%
\definecolor{currentfill}{rgb}{0.754268,0.565033,0.211761}%
\pgfsetfillcolor{currentfill}%
\pgfsetlinewidth{0.000000pt}%
\definecolor{currentstroke}{rgb}{0.000000,0.000000,0.000000}%
\pgfsetstrokecolor{currentstroke}%
\pgfsetstrokeopacity{0.000000}%
\pgfsetdash{}{0pt}%
\pgfpathmoveto{\pgfqpoint{1.610947in}{1.613090in}}%
\pgfpathlineto{\pgfqpoint{1.619700in}{1.613090in}}%
\pgfpathlineto{\pgfqpoint{1.619700in}{1.649357in}}%
\pgfpathlineto{\pgfqpoint{1.610947in}{1.649357in}}%
\pgfpathlineto{\pgfqpoint{1.610947in}{1.613090in}}%
\pgfpathclose%
\pgfusepath{fill}%
\end{pgfscope}%
\begin{pgfscope}%
\pgfpathrectangle{\pgfqpoint{0.804646in}{0.600000in}}{\pgfqpoint{2.573292in}{2.070576in}}%
\pgfusepath{clip}%
\pgfsetbuttcap%
\pgfsetmiterjoin%
\definecolor{currentfill}{rgb}{0.754268,0.565033,0.211761}%
\pgfsetfillcolor{currentfill}%
\pgfsetlinewidth{0.000000pt}%
\definecolor{currentstroke}{rgb}{0.000000,0.000000,0.000000}%
\pgfsetstrokecolor{currentstroke}%
\pgfsetstrokeopacity{0.000000}%
\pgfsetdash{}{0pt}%
\pgfpathmoveto{\pgfqpoint{1.621889in}{1.632477in}}%
\pgfpathlineto{\pgfqpoint{1.630642in}{1.632477in}}%
\pgfpathlineto{\pgfqpoint{1.630642in}{1.644450in}}%
\pgfpathlineto{\pgfqpoint{1.621889in}{1.644450in}}%
\pgfpathlineto{\pgfqpoint{1.621889in}{1.632477in}}%
\pgfpathclose%
\pgfusepath{fill}%
\end{pgfscope}%
\begin{pgfscope}%
\pgfpathrectangle{\pgfqpoint{0.804646in}{0.600000in}}{\pgfqpoint{2.573292in}{2.070576in}}%
\pgfusepath{clip}%
\pgfsetbuttcap%
\pgfsetmiterjoin%
\definecolor{currentfill}{rgb}{0.754268,0.565033,0.211761}%
\pgfsetfillcolor{currentfill}%
\pgfsetlinewidth{0.000000pt}%
\definecolor{currentstroke}{rgb}{0.000000,0.000000,0.000000}%
\pgfsetstrokecolor{currentstroke}%
\pgfsetstrokeopacity{0.000000}%
\pgfsetdash{}{0pt}%
\pgfpathmoveto{\pgfqpoint{1.632831in}{1.417043in}}%
\pgfpathlineto{\pgfqpoint{1.641584in}{1.417043in}}%
\pgfpathlineto{\pgfqpoint{1.641584in}{1.386235in}}%
\pgfpathlineto{\pgfqpoint{1.632831in}{1.386235in}}%
\pgfpathlineto{\pgfqpoint{1.632831in}{1.417043in}}%
\pgfpathclose%
\pgfusepath{fill}%
\end{pgfscope}%
\begin{pgfscope}%
\pgfpathrectangle{\pgfqpoint{0.804646in}{0.600000in}}{\pgfqpoint{2.573292in}{2.070576in}}%
\pgfusepath{clip}%
\pgfsetbuttcap%
\pgfsetmiterjoin%
\definecolor{currentfill}{rgb}{0.754268,0.565033,0.211761}%
\pgfsetfillcolor{currentfill}%
\pgfsetlinewidth{0.000000pt}%
\definecolor{currentstroke}{rgb}{0.000000,0.000000,0.000000}%
\pgfsetstrokecolor{currentstroke}%
\pgfsetstrokeopacity{0.000000}%
\pgfsetdash{}{0pt}%
\pgfpathmoveto{\pgfqpoint{1.643772in}{1.451867in}}%
\pgfpathlineto{\pgfqpoint{1.652526in}{1.451867in}}%
\pgfpathlineto{\pgfqpoint{1.652526in}{1.417392in}}%
\pgfpathlineto{\pgfqpoint{1.643772in}{1.417392in}}%
\pgfpathlineto{\pgfqpoint{1.643772in}{1.451867in}}%
\pgfpathclose%
\pgfusepath{fill}%
\end{pgfscope}%
\begin{pgfscope}%
\pgfpathrectangle{\pgfqpoint{0.804646in}{0.600000in}}{\pgfqpoint{2.573292in}{2.070576in}}%
\pgfusepath{clip}%
\pgfsetbuttcap%
\pgfsetmiterjoin%
\definecolor{currentfill}{rgb}{0.754268,0.565033,0.211761}%
\pgfsetfillcolor{currentfill}%
\pgfsetlinewidth{0.000000pt}%
\definecolor{currentstroke}{rgb}{0.000000,0.000000,0.000000}%
\pgfsetstrokecolor{currentstroke}%
\pgfsetstrokeopacity{0.000000}%
\pgfsetdash{}{0pt}%
\pgfpathmoveto{\pgfqpoint{1.654714in}{1.455460in}}%
\pgfpathlineto{\pgfqpoint{1.663468in}{1.455460in}}%
\pgfpathlineto{\pgfqpoint{1.663468in}{1.374778in}}%
\pgfpathlineto{\pgfqpoint{1.654714in}{1.374778in}}%
\pgfpathlineto{\pgfqpoint{1.654714in}{1.455460in}}%
\pgfpathclose%
\pgfusepath{fill}%
\end{pgfscope}%
\begin{pgfscope}%
\pgfpathrectangle{\pgfqpoint{0.804646in}{0.600000in}}{\pgfqpoint{2.573292in}{2.070576in}}%
\pgfusepath{clip}%
\pgfsetbuttcap%
\pgfsetmiterjoin%
\definecolor{currentfill}{rgb}{0.754268,0.565033,0.211761}%
\pgfsetfillcolor{currentfill}%
\pgfsetlinewidth{0.000000pt}%
\definecolor{currentstroke}{rgb}{0.000000,0.000000,0.000000}%
\pgfsetstrokecolor{currentstroke}%
\pgfsetstrokeopacity{0.000000}%
\pgfsetdash{}{0pt}%
\pgfpathmoveto{\pgfqpoint{1.665656in}{1.471958in}}%
\pgfpathlineto{\pgfqpoint{1.674409in}{1.471958in}}%
\pgfpathlineto{\pgfqpoint{1.674409in}{1.410182in}}%
\pgfpathlineto{\pgfqpoint{1.665656in}{1.410182in}}%
\pgfpathlineto{\pgfqpoint{1.665656in}{1.471958in}}%
\pgfpathclose%
\pgfusepath{fill}%
\end{pgfscope}%
\begin{pgfscope}%
\pgfpathrectangle{\pgfqpoint{0.804646in}{0.600000in}}{\pgfqpoint{2.573292in}{2.070576in}}%
\pgfusepath{clip}%
\pgfsetbuttcap%
\pgfsetmiterjoin%
\definecolor{currentfill}{rgb}{0.754268,0.565033,0.211761}%
\pgfsetfillcolor{currentfill}%
\pgfsetlinewidth{0.000000pt}%
\definecolor{currentstroke}{rgb}{0.000000,0.000000,0.000000}%
\pgfsetstrokecolor{currentstroke}%
\pgfsetstrokeopacity{0.000000}%
\pgfsetdash{}{0pt}%
\pgfpathmoveto{\pgfqpoint{1.676598in}{1.519958in}}%
\pgfpathlineto{\pgfqpoint{1.685351in}{1.519958in}}%
\pgfpathlineto{\pgfqpoint{1.685351in}{1.509706in}}%
\pgfpathlineto{\pgfqpoint{1.676598in}{1.509706in}}%
\pgfpathlineto{\pgfqpoint{1.676598in}{1.519958in}}%
\pgfpathclose%
\pgfusepath{fill}%
\end{pgfscope}%
\begin{pgfscope}%
\pgfpathrectangle{\pgfqpoint{0.804646in}{0.600000in}}{\pgfqpoint{2.573292in}{2.070576in}}%
\pgfusepath{clip}%
\pgfsetbuttcap%
\pgfsetmiterjoin%
\definecolor{currentfill}{rgb}{0.754268,0.565033,0.211761}%
\pgfsetfillcolor{currentfill}%
\pgfsetlinewidth{0.000000pt}%
\definecolor{currentstroke}{rgb}{0.000000,0.000000,0.000000}%
\pgfsetstrokecolor{currentstroke}%
\pgfsetstrokeopacity{0.000000}%
\pgfsetdash{}{0pt}%
\pgfpathmoveto{\pgfqpoint{1.687540in}{1.497530in}}%
\pgfpathlineto{\pgfqpoint{1.696293in}{1.497530in}}%
\pgfpathlineto{\pgfqpoint{1.696293in}{1.455815in}}%
\pgfpathlineto{\pgfqpoint{1.687540in}{1.455815in}}%
\pgfpathlineto{\pgfqpoint{1.687540in}{1.497530in}}%
\pgfpathclose%
\pgfusepath{fill}%
\end{pgfscope}%
\begin{pgfscope}%
\pgfpathrectangle{\pgfqpoint{0.804646in}{0.600000in}}{\pgfqpoint{2.573292in}{2.070576in}}%
\pgfusepath{clip}%
\pgfsetbuttcap%
\pgfsetmiterjoin%
\definecolor{currentfill}{rgb}{0.754268,0.565033,0.211761}%
\pgfsetfillcolor{currentfill}%
\pgfsetlinewidth{0.000000pt}%
\definecolor{currentstroke}{rgb}{0.000000,0.000000,0.000000}%
\pgfsetstrokecolor{currentstroke}%
\pgfsetstrokeopacity{0.000000}%
\pgfsetdash{}{0pt}%
\pgfpathmoveto{\pgfqpoint{1.698481in}{1.496495in}}%
\pgfpathlineto{\pgfqpoint{1.707235in}{1.496495in}}%
\pgfpathlineto{\pgfqpoint{1.707235in}{1.468223in}}%
\pgfpathlineto{\pgfqpoint{1.698481in}{1.468223in}}%
\pgfpathlineto{\pgfqpoint{1.698481in}{1.496495in}}%
\pgfpathclose%
\pgfusepath{fill}%
\end{pgfscope}%
\begin{pgfscope}%
\pgfpathrectangle{\pgfqpoint{0.804646in}{0.600000in}}{\pgfqpoint{2.573292in}{2.070576in}}%
\pgfusepath{clip}%
\pgfsetbuttcap%
\pgfsetmiterjoin%
\definecolor{currentfill}{rgb}{0.754268,0.565033,0.211761}%
\pgfsetfillcolor{currentfill}%
\pgfsetlinewidth{0.000000pt}%
\definecolor{currentstroke}{rgb}{0.000000,0.000000,0.000000}%
\pgfsetstrokecolor{currentstroke}%
\pgfsetstrokeopacity{0.000000}%
\pgfsetdash{}{0pt}%
\pgfpathmoveto{\pgfqpoint{1.709423in}{1.463842in}}%
\pgfpathlineto{\pgfqpoint{1.718177in}{1.463842in}}%
\pgfpathlineto{\pgfqpoint{1.718177in}{1.430390in}}%
\pgfpathlineto{\pgfqpoint{1.709423in}{1.430390in}}%
\pgfpathlineto{\pgfqpoint{1.709423in}{1.463842in}}%
\pgfpathclose%
\pgfusepath{fill}%
\end{pgfscope}%
\begin{pgfscope}%
\pgfpathrectangle{\pgfqpoint{0.804646in}{0.600000in}}{\pgfqpoint{2.573292in}{2.070576in}}%
\pgfusepath{clip}%
\pgfsetbuttcap%
\pgfsetmiterjoin%
\definecolor{currentfill}{rgb}{0.754268,0.565033,0.211761}%
\pgfsetfillcolor{currentfill}%
\pgfsetlinewidth{0.000000pt}%
\definecolor{currentstroke}{rgb}{0.000000,0.000000,0.000000}%
\pgfsetstrokecolor{currentstroke}%
\pgfsetstrokeopacity{0.000000}%
\pgfsetdash{}{0pt}%
\pgfpathmoveto{\pgfqpoint{1.720365in}{1.419587in}}%
\pgfpathlineto{\pgfqpoint{1.729118in}{1.419587in}}%
\pgfpathlineto{\pgfqpoint{1.729118in}{1.363239in}}%
\pgfpathlineto{\pgfqpoint{1.720365in}{1.363239in}}%
\pgfpathlineto{\pgfqpoint{1.720365in}{1.419587in}}%
\pgfpathclose%
\pgfusepath{fill}%
\end{pgfscope}%
\begin{pgfscope}%
\pgfpathrectangle{\pgfqpoint{0.804646in}{0.600000in}}{\pgfqpoint{2.573292in}{2.070576in}}%
\pgfusepath{clip}%
\pgfsetbuttcap%
\pgfsetmiterjoin%
\definecolor{currentfill}{rgb}{0.754268,0.565033,0.211761}%
\pgfsetfillcolor{currentfill}%
\pgfsetlinewidth{0.000000pt}%
\definecolor{currentstroke}{rgb}{0.000000,0.000000,0.000000}%
\pgfsetstrokecolor{currentstroke}%
\pgfsetstrokeopacity{0.000000}%
\pgfsetdash{}{0pt}%
\pgfpathmoveto{\pgfqpoint{1.731307in}{1.363792in}}%
\pgfpathlineto{\pgfqpoint{1.740060in}{1.363792in}}%
\pgfpathlineto{\pgfqpoint{1.740060in}{1.311814in}}%
\pgfpathlineto{\pgfqpoint{1.731307in}{1.311814in}}%
\pgfpathlineto{\pgfqpoint{1.731307in}{1.363792in}}%
\pgfpathclose%
\pgfusepath{fill}%
\end{pgfscope}%
\begin{pgfscope}%
\pgfpathrectangle{\pgfqpoint{0.804646in}{0.600000in}}{\pgfqpoint{2.573292in}{2.070576in}}%
\pgfusepath{clip}%
\pgfsetbuttcap%
\pgfsetmiterjoin%
\definecolor{currentfill}{rgb}{0.754268,0.565033,0.211761}%
\pgfsetfillcolor{currentfill}%
\pgfsetlinewidth{0.000000pt}%
\definecolor{currentstroke}{rgb}{0.000000,0.000000,0.000000}%
\pgfsetstrokecolor{currentstroke}%
\pgfsetstrokeopacity{0.000000}%
\pgfsetdash{}{0pt}%
\pgfpathmoveto{\pgfqpoint{1.742249in}{1.393793in}}%
\pgfpathlineto{\pgfqpoint{1.751002in}{1.393793in}}%
\pgfpathlineto{\pgfqpoint{1.751002in}{1.330330in}}%
\pgfpathlineto{\pgfqpoint{1.742249in}{1.330330in}}%
\pgfpathlineto{\pgfqpoint{1.742249in}{1.393793in}}%
\pgfpathclose%
\pgfusepath{fill}%
\end{pgfscope}%
\begin{pgfscope}%
\pgfpathrectangle{\pgfqpoint{0.804646in}{0.600000in}}{\pgfqpoint{2.573292in}{2.070576in}}%
\pgfusepath{clip}%
\pgfsetbuttcap%
\pgfsetmiterjoin%
\definecolor{currentfill}{rgb}{0.754268,0.565033,0.211761}%
\pgfsetfillcolor{currentfill}%
\pgfsetlinewidth{0.000000pt}%
\definecolor{currentstroke}{rgb}{0.000000,0.000000,0.000000}%
\pgfsetstrokecolor{currentstroke}%
\pgfsetstrokeopacity{0.000000}%
\pgfsetdash{}{0pt}%
\pgfpathmoveto{\pgfqpoint{1.753190in}{1.367067in}}%
\pgfpathlineto{\pgfqpoint{1.761944in}{1.367067in}}%
\pgfpathlineto{\pgfqpoint{1.761944in}{1.307736in}}%
\pgfpathlineto{\pgfqpoint{1.753190in}{1.307736in}}%
\pgfpathlineto{\pgfqpoint{1.753190in}{1.367067in}}%
\pgfpathclose%
\pgfusepath{fill}%
\end{pgfscope}%
\begin{pgfscope}%
\pgfpathrectangle{\pgfqpoint{0.804646in}{0.600000in}}{\pgfqpoint{2.573292in}{2.070576in}}%
\pgfusepath{clip}%
\pgfsetbuttcap%
\pgfsetmiterjoin%
\definecolor{currentfill}{rgb}{0.754268,0.565033,0.211761}%
\pgfsetfillcolor{currentfill}%
\pgfsetlinewidth{0.000000pt}%
\definecolor{currentstroke}{rgb}{0.000000,0.000000,0.000000}%
\pgfsetstrokecolor{currentstroke}%
\pgfsetstrokeopacity{0.000000}%
\pgfsetdash{}{0pt}%
\pgfpathmoveto{\pgfqpoint{1.764132in}{1.343109in}}%
\pgfpathlineto{\pgfqpoint{1.772886in}{1.343109in}}%
\pgfpathlineto{\pgfqpoint{1.772886in}{1.226444in}}%
\pgfpathlineto{\pgfqpoint{1.764132in}{1.226444in}}%
\pgfpathlineto{\pgfqpoint{1.764132in}{1.343109in}}%
\pgfpathclose%
\pgfusepath{fill}%
\end{pgfscope}%
\begin{pgfscope}%
\pgfpathrectangle{\pgfqpoint{0.804646in}{0.600000in}}{\pgfqpoint{2.573292in}{2.070576in}}%
\pgfusepath{clip}%
\pgfsetbuttcap%
\pgfsetmiterjoin%
\definecolor{currentfill}{rgb}{0.754268,0.565033,0.211761}%
\pgfsetfillcolor{currentfill}%
\pgfsetlinewidth{0.000000pt}%
\definecolor{currentstroke}{rgb}{0.000000,0.000000,0.000000}%
\pgfsetstrokecolor{currentstroke}%
\pgfsetstrokeopacity{0.000000}%
\pgfsetdash{}{0pt}%
\pgfpathmoveto{\pgfqpoint{1.775074in}{1.359140in}}%
\pgfpathlineto{\pgfqpoint{1.783827in}{1.359140in}}%
\pgfpathlineto{\pgfqpoint{1.783827in}{1.306267in}}%
\pgfpathlineto{\pgfqpoint{1.775074in}{1.306267in}}%
\pgfpathlineto{\pgfqpoint{1.775074in}{1.359140in}}%
\pgfpathclose%
\pgfusepath{fill}%
\end{pgfscope}%
\begin{pgfscope}%
\pgfpathrectangle{\pgfqpoint{0.804646in}{0.600000in}}{\pgfqpoint{2.573292in}{2.070576in}}%
\pgfusepath{clip}%
\pgfsetbuttcap%
\pgfsetmiterjoin%
\definecolor{currentfill}{rgb}{0.754268,0.565033,0.211761}%
\pgfsetfillcolor{currentfill}%
\pgfsetlinewidth{0.000000pt}%
\definecolor{currentstroke}{rgb}{0.000000,0.000000,0.000000}%
\pgfsetstrokecolor{currentstroke}%
\pgfsetstrokeopacity{0.000000}%
\pgfsetdash{}{0pt}%
\pgfpathmoveto{\pgfqpoint{1.786016in}{1.327623in}}%
\pgfpathlineto{\pgfqpoint{1.794769in}{1.327623in}}%
\pgfpathlineto{\pgfqpoint{1.794769in}{1.248190in}}%
\pgfpathlineto{\pgfqpoint{1.786016in}{1.248190in}}%
\pgfpathlineto{\pgfqpoint{1.786016in}{1.327623in}}%
\pgfpathclose%
\pgfusepath{fill}%
\end{pgfscope}%
\begin{pgfscope}%
\pgfpathrectangle{\pgfqpoint{0.804646in}{0.600000in}}{\pgfqpoint{2.573292in}{2.070576in}}%
\pgfusepath{clip}%
\pgfsetbuttcap%
\pgfsetmiterjoin%
\definecolor{currentfill}{rgb}{0.754268,0.565033,0.211761}%
\pgfsetfillcolor{currentfill}%
\pgfsetlinewidth{0.000000pt}%
\definecolor{currentstroke}{rgb}{0.000000,0.000000,0.000000}%
\pgfsetstrokecolor{currentstroke}%
\pgfsetstrokeopacity{0.000000}%
\pgfsetdash{}{0pt}%
\pgfpathmoveto{\pgfqpoint{1.796958in}{1.298546in}}%
\pgfpathlineto{\pgfqpoint{1.805711in}{1.298546in}}%
\pgfpathlineto{\pgfqpoint{1.805711in}{1.206095in}}%
\pgfpathlineto{\pgfqpoint{1.796958in}{1.206095in}}%
\pgfpathlineto{\pgfqpoint{1.796958in}{1.298546in}}%
\pgfpathclose%
\pgfusepath{fill}%
\end{pgfscope}%
\begin{pgfscope}%
\pgfpathrectangle{\pgfqpoint{0.804646in}{0.600000in}}{\pgfqpoint{2.573292in}{2.070576in}}%
\pgfusepath{clip}%
\pgfsetbuttcap%
\pgfsetmiterjoin%
\definecolor{currentfill}{rgb}{0.754268,0.565033,0.211761}%
\pgfsetfillcolor{currentfill}%
\pgfsetlinewidth{0.000000pt}%
\definecolor{currentstroke}{rgb}{0.000000,0.000000,0.000000}%
\pgfsetstrokecolor{currentstroke}%
\pgfsetstrokeopacity{0.000000}%
\pgfsetdash{}{0pt}%
\pgfpathmoveto{\pgfqpoint{1.807899in}{1.272571in}}%
\pgfpathlineto{\pgfqpoint{1.816653in}{1.272571in}}%
\pgfpathlineto{\pgfqpoint{1.816653in}{1.159466in}}%
\pgfpathlineto{\pgfqpoint{1.807899in}{1.159466in}}%
\pgfpathlineto{\pgfqpoint{1.807899in}{1.272571in}}%
\pgfpathclose%
\pgfusepath{fill}%
\end{pgfscope}%
\begin{pgfscope}%
\pgfpathrectangle{\pgfqpoint{0.804646in}{0.600000in}}{\pgfqpoint{2.573292in}{2.070576in}}%
\pgfusepath{clip}%
\pgfsetbuttcap%
\pgfsetmiterjoin%
\definecolor{currentfill}{rgb}{0.754268,0.565033,0.211761}%
\pgfsetfillcolor{currentfill}%
\pgfsetlinewidth{0.000000pt}%
\definecolor{currentstroke}{rgb}{0.000000,0.000000,0.000000}%
\pgfsetstrokecolor{currentstroke}%
\pgfsetstrokeopacity{0.000000}%
\pgfsetdash{}{0pt}%
\pgfpathmoveto{\pgfqpoint{1.818841in}{1.264099in}}%
\pgfpathlineto{\pgfqpoint{1.827595in}{1.264099in}}%
\pgfpathlineto{\pgfqpoint{1.827595in}{1.182847in}}%
\pgfpathlineto{\pgfqpoint{1.818841in}{1.182847in}}%
\pgfpathlineto{\pgfqpoint{1.818841in}{1.264099in}}%
\pgfpathclose%
\pgfusepath{fill}%
\end{pgfscope}%
\begin{pgfscope}%
\pgfpathrectangle{\pgfqpoint{0.804646in}{0.600000in}}{\pgfqpoint{2.573292in}{2.070576in}}%
\pgfusepath{clip}%
\pgfsetbuttcap%
\pgfsetmiterjoin%
\definecolor{currentfill}{rgb}{0.754268,0.565033,0.211761}%
\pgfsetfillcolor{currentfill}%
\pgfsetlinewidth{0.000000pt}%
\definecolor{currentstroke}{rgb}{0.000000,0.000000,0.000000}%
\pgfsetstrokecolor{currentstroke}%
\pgfsetstrokeopacity{0.000000}%
\pgfsetdash{}{0pt}%
\pgfpathmoveto{\pgfqpoint{1.829783in}{1.239888in}}%
\pgfpathlineto{\pgfqpoint{1.838536in}{1.239888in}}%
\pgfpathlineto{\pgfqpoint{1.838536in}{1.145596in}}%
\pgfpathlineto{\pgfqpoint{1.829783in}{1.145596in}}%
\pgfpathlineto{\pgfqpoint{1.829783in}{1.239888in}}%
\pgfpathclose%
\pgfusepath{fill}%
\end{pgfscope}%
\begin{pgfscope}%
\pgfpathrectangle{\pgfqpoint{0.804646in}{0.600000in}}{\pgfqpoint{2.573292in}{2.070576in}}%
\pgfusepath{clip}%
\pgfsetbuttcap%
\pgfsetmiterjoin%
\definecolor{currentfill}{rgb}{0.754268,0.565033,0.211761}%
\pgfsetfillcolor{currentfill}%
\pgfsetlinewidth{0.000000pt}%
\definecolor{currentstroke}{rgb}{0.000000,0.000000,0.000000}%
\pgfsetstrokecolor{currentstroke}%
\pgfsetstrokeopacity{0.000000}%
\pgfsetdash{}{0pt}%
\pgfpathmoveto{\pgfqpoint{1.840725in}{1.192029in}}%
\pgfpathlineto{\pgfqpoint{1.849478in}{1.192029in}}%
\pgfpathlineto{\pgfqpoint{1.849478in}{1.100967in}}%
\pgfpathlineto{\pgfqpoint{1.840725in}{1.100967in}}%
\pgfpathlineto{\pgfqpoint{1.840725in}{1.192029in}}%
\pgfpathclose%
\pgfusepath{fill}%
\end{pgfscope}%
\begin{pgfscope}%
\pgfpathrectangle{\pgfqpoint{0.804646in}{0.600000in}}{\pgfqpoint{2.573292in}{2.070576in}}%
\pgfusepath{clip}%
\pgfsetbuttcap%
\pgfsetmiterjoin%
\definecolor{currentfill}{rgb}{0.754268,0.565033,0.211761}%
\pgfsetfillcolor{currentfill}%
\pgfsetlinewidth{0.000000pt}%
\definecolor{currentstroke}{rgb}{0.000000,0.000000,0.000000}%
\pgfsetstrokecolor{currentstroke}%
\pgfsetstrokeopacity{0.000000}%
\pgfsetdash{}{0pt}%
\pgfpathmoveto{\pgfqpoint{1.851667in}{1.214183in}}%
\pgfpathlineto{\pgfqpoint{1.860420in}{1.214183in}}%
\pgfpathlineto{\pgfqpoint{1.860420in}{1.124550in}}%
\pgfpathlineto{\pgfqpoint{1.851667in}{1.124550in}}%
\pgfpathlineto{\pgfqpoint{1.851667in}{1.214183in}}%
\pgfpathclose%
\pgfusepath{fill}%
\end{pgfscope}%
\begin{pgfscope}%
\pgfpathrectangle{\pgfqpoint{0.804646in}{0.600000in}}{\pgfqpoint{2.573292in}{2.070576in}}%
\pgfusepath{clip}%
\pgfsetbuttcap%
\pgfsetmiterjoin%
\definecolor{currentfill}{rgb}{0.754268,0.565033,0.211761}%
\pgfsetfillcolor{currentfill}%
\pgfsetlinewidth{0.000000pt}%
\definecolor{currentstroke}{rgb}{0.000000,0.000000,0.000000}%
\pgfsetstrokecolor{currentstroke}%
\pgfsetstrokeopacity{0.000000}%
\pgfsetdash{}{0pt}%
\pgfpathmoveto{\pgfqpoint{1.862608in}{1.190065in}}%
\pgfpathlineto{\pgfqpoint{1.871362in}{1.190065in}}%
\pgfpathlineto{\pgfqpoint{1.871362in}{1.115371in}}%
\pgfpathlineto{\pgfqpoint{1.862608in}{1.115371in}}%
\pgfpathlineto{\pgfqpoint{1.862608in}{1.190065in}}%
\pgfpathclose%
\pgfusepath{fill}%
\end{pgfscope}%
\begin{pgfscope}%
\pgfpathrectangle{\pgfqpoint{0.804646in}{0.600000in}}{\pgfqpoint{2.573292in}{2.070576in}}%
\pgfusepath{clip}%
\pgfsetbuttcap%
\pgfsetmiterjoin%
\definecolor{currentfill}{rgb}{0.754268,0.565033,0.211761}%
\pgfsetfillcolor{currentfill}%
\pgfsetlinewidth{0.000000pt}%
\definecolor{currentstroke}{rgb}{0.000000,0.000000,0.000000}%
\pgfsetstrokecolor{currentstroke}%
\pgfsetstrokeopacity{0.000000}%
\pgfsetdash{}{0pt}%
\pgfpathmoveto{\pgfqpoint{1.873550in}{1.142819in}}%
\pgfpathlineto{\pgfqpoint{1.882304in}{1.142819in}}%
\pgfpathlineto{\pgfqpoint{1.882304in}{1.053828in}}%
\pgfpathlineto{\pgfqpoint{1.873550in}{1.053828in}}%
\pgfpathlineto{\pgfqpoint{1.873550in}{1.142819in}}%
\pgfpathclose%
\pgfusepath{fill}%
\end{pgfscope}%
\begin{pgfscope}%
\pgfpathrectangle{\pgfqpoint{0.804646in}{0.600000in}}{\pgfqpoint{2.573292in}{2.070576in}}%
\pgfusepath{clip}%
\pgfsetbuttcap%
\pgfsetmiterjoin%
\definecolor{currentfill}{rgb}{0.754268,0.565033,0.211761}%
\pgfsetfillcolor{currentfill}%
\pgfsetlinewidth{0.000000pt}%
\definecolor{currentstroke}{rgb}{0.000000,0.000000,0.000000}%
\pgfsetstrokecolor{currentstroke}%
\pgfsetstrokeopacity{0.000000}%
\pgfsetdash{}{0pt}%
\pgfpathmoveto{\pgfqpoint{1.884492in}{1.103020in}}%
\pgfpathlineto{\pgfqpoint{1.893245in}{1.103020in}}%
\pgfpathlineto{\pgfqpoint{1.893245in}{1.013779in}}%
\pgfpathlineto{\pgfqpoint{1.884492in}{1.013779in}}%
\pgfpathlineto{\pgfqpoint{1.884492in}{1.103020in}}%
\pgfpathclose%
\pgfusepath{fill}%
\end{pgfscope}%
\begin{pgfscope}%
\pgfpathrectangle{\pgfqpoint{0.804646in}{0.600000in}}{\pgfqpoint{2.573292in}{2.070576in}}%
\pgfusepath{clip}%
\pgfsetbuttcap%
\pgfsetmiterjoin%
\definecolor{currentfill}{rgb}{0.754268,0.565033,0.211761}%
\pgfsetfillcolor{currentfill}%
\pgfsetlinewidth{0.000000pt}%
\definecolor{currentstroke}{rgb}{0.000000,0.000000,0.000000}%
\pgfsetstrokecolor{currentstroke}%
\pgfsetstrokeopacity{0.000000}%
\pgfsetdash{}{0pt}%
\pgfpathmoveto{\pgfqpoint{1.895434in}{1.143349in}}%
\pgfpathlineto{\pgfqpoint{1.904187in}{1.143349in}}%
\pgfpathlineto{\pgfqpoint{1.904187in}{1.076299in}}%
\pgfpathlineto{\pgfqpoint{1.895434in}{1.076299in}}%
\pgfpathlineto{\pgfqpoint{1.895434in}{1.143349in}}%
\pgfpathclose%
\pgfusepath{fill}%
\end{pgfscope}%
\begin{pgfscope}%
\pgfpathrectangle{\pgfqpoint{0.804646in}{0.600000in}}{\pgfqpoint{2.573292in}{2.070576in}}%
\pgfusepath{clip}%
\pgfsetbuttcap%
\pgfsetmiterjoin%
\definecolor{currentfill}{rgb}{0.754268,0.565033,0.211761}%
\pgfsetfillcolor{currentfill}%
\pgfsetlinewidth{0.000000pt}%
\definecolor{currentstroke}{rgb}{0.000000,0.000000,0.000000}%
\pgfsetstrokecolor{currentstroke}%
\pgfsetstrokeopacity{0.000000}%
\pgfsetdash{}{0pt}%
\pgfpathmoveto{\pgfqpoint{1.906376in}{1.108065in}}%
\pgfpathlineto{\pgfqpoint{1.915129in}{1.108065in}}%
\pgfpathlineto{\pgfqpoint{1.915129in}{1.052361in}}%
\pgfpathlineto{\pgfqpoint{1.906376in}{1.052361in}}%
\pgfpathlineto{\pgfqpoint{1.906376in}{1.108065in}}%
\pgfpathclose%
\pgfusepath{fill}%
\end{pgfscope}%
\begin{pgfscope}%
\pgfpathrectangle{\pgfqpoint{0.804646in}{0.600000in}}{\pgfqpoint{2.573292in}{2.070576in}}%
\pgfusepath{clip}%
\pgfsetbuttcap%
\pgfsetmiterjoin%
\definecolor{currentfill}{rgb}{0.754268,0.565033,0.211761}%
\pgfsetfillcolor{currentfill}%
\pgfsetlinewidth{0.000000pt}%
\definecolor{currentstroke}{rgb}{0.000000,0.000000,0.000000}%
\pgfsetstrokecolor{currentstroke}%
\pgfsetstrokeopacity{0.000000}%
\pgfsetdash{}{0pt}%
\pgfpathmoveto{\pgfqpoint{1.917317in}{1.095045in}}%
\pgfpathlineto{\pgfqpoint{1.926071in}{1.095045in}}%
\pgfpathlineto{\pgfqpoint{1.926071in}{1.051513in}}%
\pgfpathlineto{\pgfqpoint{1.917317in}{1.051513in}}%
\pgfpathlineto{\pgfqpoint{1.917317in}{1.095045in}}%
\pgfpathclose%
\pgfusepath{fill}%
\end{pgfscope}%
\begin{pgfscope}%
\pgfpathrectangle{\pgfqpoint{0.804646in}{0.600000in}}{\pgfqpoint{2.573292in}{2.070576in}}%
\pgfusepath{clip}%
\pgfsetbuttcap%
\pgfsetmiterjoin%
\definecolor{currentfill}{rgb}{0.754268,0.565033,0.211761}%
\pgfsetfillcolor{currentfill}%
\pgfsetlinewidth{0.000000pt}%
\definecolor{currentstroke}{rgb}{0.000000,0.000000,0.000000}%
\pgfsetstrokecolor{currentstroke}%
\pgfsetstrokeopacity{0.000000}%
\pgfsetdash{}{0pt}%
\pgfpathmoveto{\pgfqpoint{1.928259in}{1.102347in}}%
\pgfpathlineto{\pgfqpoint{1.937013in}{1.102347in}}%
\pgfpathlineto{\pgfqpoint{1.937013in}{1.076903in}}%
\pgfpathlineto{\pgfqpoint{1.928259in}{1.076903in}}%
\pgfpathlineto{\pgfqpoint{1.928259in}{1.102347in}}%
\pgfpathclose%
\pgfusepath{fill}%
\end{pgfscope}%
\begin{pgfscope}%
\pgfpathrectangle{\pgfqpoint{0.804646in}{0.600000in}}{\pgfqpoint{2.573292in}{2.070576in}}%
\pgfusepath{clip}%
\pgfsetbuttcap%
\pgfsetmiterjoin%
\definecolor{currentfill}{rgb}{0.754268,0.565033,0.211761}%
\pgfsetfillcolor{currentfill}%
\pgfsetlinewidth{0.000000pt}%
\definecolor{currentstroke}{rgb}{0.000000,0.000000,0.000000}%
\pgfsetstrokecolor{currentstroke}%
\pgfsetstrokeopacity{0.000000}%
\pgfsetdash{}{0pt}%
\pgfpathmoveto{\pgfqpoint{1.939201in}{1.096854in}}%
\pgfpathlineto{\pgfqpoint{1.947954in}{1.096854in}}%
\pgfpathlineto{\pgfqpoint{1.947954in}{1.049544in}}%
\pgfpathlineto{\pgfqpoint{1.939201in}{1.049544in}}%
\pgfpathlineto{\pgfqpoint{1.939201in}{1.096854in}}%
\pgfpathclose%
\pgfusepath{fill}%
\end{pgfscope}%
\begin{pgfscope}%
\pgfpathrectangle{\pgfqpoint{0.804646in}{0.600000in}}{\pgfqpoint{2.573292in}{2.070576in}}%
\pgfusepath{clip}%
\pgfsetbuttcap%
\pgfsetmiterjoin%
\definecolor{currentfill}{rgb}{0.754268,0.565033,0.211761}%
\pgfsetfillcolor{currentfill}%
\pgfsetlinewidth{0.000000pt}%
\definecolor{currentstroke}{rgb}{0.000000,0.000000,0.000000}%
\pgfsetstrokecolor{currentstroke}%
\pgfsetstrokeopacity{0.000000}%
\pgfsetdash{}{0pt}%
\pgfpathmoveto{\pgfqpoint{1.950143in}{1.045357in}}%
\pgfpathlineto{\pgfqpoint{1.958896in}{1.045357in}}%
\pgfpathlineto{\pgfqpoint{1.958896in}{0.975546in}}%
\pgfpathlineto{\pgfqpoint{1.950143in}{0.975546in}}%
\pgfpathlineto{\pgfqpoint{1.950143in}{1.045357in}}%
\pgfpathclose%
\pgfusepath{fill}%
\end{pgfscope}%
\begin{pgfscope}%
\pgfpathrectangle{\pgfqpoint{0.804646in}{0.600000in}}{\pgfqpoint{2.573292in}{2.070576in}}%
\pgfusepath{clip}%
\pgfsetbuttcap%
\pgfsetmiterjoin%
\definecolor{currentfill}{rgb}{0.754268,0.565033,0.211761}%
\pgfsetfillcolor{currentfill}%
\pgfsetlinewidth{0.000000pt}%
\definecolor{currentstroke}{rgb}{0.000000,0.000000,0.000000}%
\pgfsetstrokecolor{currentstroke}%
\pgfsetstrokeopacity{0.000000}%
\pgfsetdash{}{0pt}%
\pgfpathmoveto{\pgfqpoint{1.961085in}{1.015047in}}%
\pgfpathlineto{\pgfqpoint{1.969838in}{1.015047in}}%
\pgfpathlineto{\pgfqpoint{1.969838in}{0.936584in}}%
\pgfpathlineto{\pgfqpoint{1.961085in}{0.936584in}}%
\pgfpathlineto{\pgfqpoint{1.961085in}{1.015047in}}%
\pgfpathclose%
\pgfusepath{fill}%
\end{pgfscope}%
\begin{pgfscope}%
\pgfpathrectangle{\pgfqpoint{0.804646in}{0.600000in}}{\pgfqpoint{2.573292in}{2.070576in}}%
\pgfusepath{clip}%
\pgfsetbuttcap%
\pgfsetmiterjoin%
\definecolor{currentfill}{rgb}{0.754268,0.565033,0.211761}%
\pgfsetfillcolor{currentfill}%
\pgfsetlinewidth{0.000000pt}%
\definecolor{currentstroke}{rgb}{0.000000,0.000000,0.000000}%
\pgfsetstrokecolor{currentstroke}%
\pgfsetstrokeopacity{0.000000}%
\pgfsetdash{}{0pt}%
\pgfpathmoveto{\pgfqpoint{1.972026in}{1.027400in}}%
\pgfpathlineto{\pgfqpoint{1.980780in}{1.027400in}}%
\pgfpathlineto{\pgfqpoint{1.980780in}{0.960630in}}%
\pgfpathlineto{\pgfqpoint{1.972026in}{0.960630in}}%
\pgfpathlineto{\pgfqpoint{1.972026in}{1.027400in}}%
\pgfpathclose%
\pgfusepath{fill}%
\end{pgfscope}%
\begin{pgfscope}%
\pgfpathrectangle{\pgfqpoint{0.804646in}{0.600000in}}{\pgfqpoint{2.573292in}{2.070576in}}%
\pgfusepath{clip}%
\pgfsetbuttcap%
\pgfsetmiterjoin%
\definecolor{currentfill}{rgb}{0.754268,0.565033,0.211761}%
\pgfsetfillcolor{currentfill}%
\pgfsetlinewidth{0.000000pt}%
\definecolor{currentstroke}{rgb}{0.000000,0.000000,0.000000}%
\pgfsetstrokecolor{currentstroke}%
\pgfsetstrokeopacity{0.000000}%
\pgfsetdash{}{0pt}%
\pgfpathmoveto{\pgfqpoint{1.982968in}{1.012173in}}%
\pgfpathlineto{\pgfqpoint{1.991722in}{1.012173in}}%
\pgfpathlineto{\pgfqpoint{1.991722in}{0.925604in}}%
\pgfpathlineto{\pgfqpoint{1.982968in}{0.925604in}}%
\pgfpathlineto{\pgfqpoint{1.982968in}{1.012173in}}%
\pgfpathclose%
\pgfusepath{fill}%
\end{pgfscope}%
\begin{pgfscope}%
\pgfpathrectangle{\pgfqpoint{0.804646in}{0.600000in}}{\pgfqpoint{2.573292in}{2.070576in}}%
\pgfusepath{clip}%
\pgfsetbuttcap%
\pgfsetmiterjoin%
\definecolor{currentfill}{rgb}{0.754268,0.565033,0.211761}%
\pgfsetfillcolor{currentfill}%
\pgfsetlinewidth{0.000000pt}%
\definecolor{currentstroke}{rgb}{0.000000,0.000000,0.000000}%
\pgfsetstrokecolor{currentstroke}%
\pgfsetstrokeopacity{0.000000}%
\pgfsetdash{}{0pt}%
\pgfpathmoveto{\pgfqpoint{1.993910in}{0.946937in}}%
\pgfpathlineto{\pgfqpoint{2.002663in}{0.946937in}}%
\pgfpathlineto{\pgfqpoint{2.002663in}{0.852488in}}%
\pgfpathlineto{\pgfqpoint{1.993910in}{0.852488in}}%
\pgfpathlineto{\pgfqpoint{1.993910in}{0.946937in}}%
\pgfpathclose%
\pgfusepath{fill}%
\end{pgfscope}%
\begin{pgfscope}%
\pgfpathrectangle{\pgfqpoint{0.804646in}{0.600000in}}{\pgfqpoint{2.573292in}{2.070576in}}%
\pgfusepath{clip}%
\pgfsetbuttcap%
\pgfsetmiterjoin%
\definecolor{currentfill}{rgb}{0.754268,0.565033,0.211761}%
\pgfsetfillcolor{currentfill}%
\pgfsetlinewidth{0.000000pt}%
\definecolor{currentstroke}{rgb}{0.000000,0.000000,0.000000}%
\pgfsetstrokecolor{currentstroke}%
\pgfsetstrokeopacity{0.000000}%
\pgfsetdash{}{0pt}%
\pgfpathmoveto{\pgfqpoint{2.004852in}{0.957439in}}%
\pgfpathlineto{\pgfqpoint{2.013605in}{0.957439in}}%
\pgfpathlineto{\pgfqpoint{2.013605in}{0.868848in}}%
\pgfpathlineto{\pgfqpoint{2.004852in}{0.868848in}}%
\pgfpathlineto{\pgfqpoint{2.004852in}{0.957439in}}%
\pgfpathclose%
\pgfusepath{fill}%
\end{pgfscope}%
\begin{pgfscope}%
\pgfpathrectangle{\pgfqpoint{0.804646in}{0.600000in}}{\pgfqpoint{2.573292in}{2.070576in}}%
\pgfusepath{clip}%
\pgfsetbuttcap%
\pgfsetmiterjoin%
\definecolor{currentfill}{rgb}{0.754268,0.565033,0.211761}%
\pgfsetfillcolor{currentfill}%
\pgfsetlinewidth{0.000000pt}%
\definecolor{currentstroke}{rgb}{0.000000,0.000000,0.000000}%
\pgfsetstrokecolor{currentstroke}%
\pgfsetstrokeopacity{0.000000}%
\pgfsetdash{}{0pt}%
\pgfpathmoveto{\pgfqpoint{2.015794in}{0.926495in}}%
\pgfpathlineto{\pgfqpoint{2.024547in}{0.926495in}}%
\pgfpathlineto{\pgfqpoint{2.024547in}{0.801358in}}%
\pgfpathlineto{\pgfqpoint{2.015794in}{0.801358in}}%
\pgfpathlineto{\pgfqpoint{2.015794in}{0.926495in}}%
\pgfpathclose%
\pgfusepath{fill}%
\end{pgfscope}%
\begin{pgfscope}%
\pgfpathrectangle{\pgfqpoint{0.804646in}{0.600000in}}{\pgfqpoint{2.573292in}{2.070576in}}%
\pgfusepath{clip}%
\pgfsetbuttcap%
\pgfsetmiterjoin%
\definecolor{currentfill}{rgb}{0.754268,0.565033,0.211761}%
\pgfsetfillcolor{currentfill}%
\pgfsetlinewidth{0.000000pt}%
\definecolor{currentstroke}{rgb}{0.000000,0.000000,0.000000}%
\pgfsetstrokecolor{currentstroke}%
\pgfsetstrokeopacity{0.000000}%
\pgfsetdash{}{0pt}%
\pgfpathmoveto{\pgfqpoint{2.026735in}{0.903314in}}%
\pgfpathlineto{\pgfqpoint{2.035489in}{0.903314in}}%
\pgfpathlineto{\pgfqpoint{2.035489in}{0.772775in}}%
\pgfpathlineto{\pgfqpoint{2.026735in}{0.772775in}}%
\pgfpathlineto{\pgfqpoint{2.026735in}{0.903314in}}%
\pgfpathclose%
\pgfusepath{fill}%
\end{pgfscope}%
\begin{pgfscope}%
\pgfpathrectangle{\pgfqpoint{0.804646in}{0.600000in}}{\pgfqpoint{2.573292in}{2.070576in}}%
\pgfusepath{clip}%
\pgfsetbuttcap%
\pgfsetmiterjoin%
\definecolor{currentfill}{rgb}{0.754268,0.565033,0.211761}%
\pgfsetfillcolor{currentfill}%
\pgfsetlinewidth{0.000000pt}%
\definecolor{currentstroke}{rgb}{0.000000,0.000000,0.000000}%
\pgfsetstrokecolor{currentstroke}%
\pgfsetstrokeopacity{0.000000}%
\pgfsetdash{}{0pt}%
\pgfpathmoveto{\pgfqpoint{2.037677in}{0.916423in}}%
\pgfpathlineto{\pgfqpoint{2.046431in}{0.916423in}}%
\pgfpathlineto{\pgfqpoint{2.046431in}{0.805630in}}%
\pgfpathlineto{\pgfqpoint{2.037677in}{0.805630in}}%
\pgfpathlineto{\pgfqpoint{2.037677in}{0.916423in}}%
\pgfpathclose%
\pgfusepath{fill}%
\end{pgfscope}%
\begin{pgfscope}%
\pgfpathrectangle{\pgfqpoint{0.804646in}{0.600000in}}{\pgfqpoint{2.573292in}{2.070576in}}%
\pgfusepath{clip}%
\pgfsetbuttcap%
\pgfsetmiterjoin%
\definecolor{currentfill}{rgb}{0.754268,0.565033,0.211761}%
\pgfsetfillcolor{currentfill}%
\pgfsetlinewidth{0.000000pt}%
\definecolor{currentstroke}{rgb}{0.000000,0.000000,0.000000}%
\pgfsetstrokecolor{currentstroke}%
\pgfsetstrokeopacity{0.000000}%
\pgfsetdash{}{0pt}%
\pgfpathmoveto{\pgfqpoint{2.048619in}{0.884639in}}%
\pgfpathlineto{\pgfqpoint{2.057372in}{0.884639in}}%
\pgfpathlineto{\pgfqpoint{2.057372in}{0.747974in}}%
\pgfpathlineto{\pgfqpoint{2.048619in}{0.747974in}}%
\pgfpathlineto{\pgfqpoint{2.048619in}{0.884639in}}%
\pgfpathclose%
\pgfusepath{fill}%
\end{pgfscope}%
\begin{pgfscope}%
\pgfpathrectangle{\pgfqpoint{0.804646in}{0.600000in}}{\pgfqpoint{2.573292in}{2.070576in}}%
\pgfusepath{clip}%
\pgfsetbuttcap%
\pgfsetmiterjoin%
\definecolor{currentfill}{rgb}{0.754268,0.565033,0.211761}%
\pgfsetfillcolor{currentfill}%
\pgfsetlinewidth{0.000000pt}%
\definecolor{currentstroke}{rgb}{0.000000,0.000000,0.000000}%
\pgfsetstrokecolor{currentstroke}%
\pgfsetstrokeopacity{0.000000}%
\pgfsetdash{}{0pt}%
\pgfpathmoveto{\pgfqpoint{2.059561in}{0.879382in}}%
\pgfpathlineto{\pgfqpoint{2.068314in}{0.879382in}}%
\pgfpathlineto{\pgfqpoint{2.068314in}{0.752011in}}%
\pgfpathlineto{\pgfqpoint{2.059561in}{0.752011in}}%
\pgfpathlineto{\pgfqpoint{2.059561in}{0.879382in}}%
\pgfpathclose%
\pgfusepath{fill}%
\end{pgfscope}%
\begin{pgfscope}%
\pgfpathrectangle{\pgfqpoint{0.804646in}{0.600000in}}{\pgfqpoint{2.573292in}{2.070576in}}%
\pgfusepath{clip}%
\pgfsetbuttcap%
\pgfsetmiterjoin%
\definecolor{currentfill}{rgb}{0.754268,0.565033,0.211761}%
\pgfsetfillcolor{currentfill}%
\pgfsetlinewidth{0.000000pt}%
\definecolor{currentstroke}{rgb}{0.000000,0.000000,0.000000}%
\pgfsetstrokecolor{currentstroke}%
\pgfsetstrokeopacity{0.000000}%
\pgfsetdash{}{0pt}%
\pgfpathmoveto{\pgfqpoint{2.070503in}{0.846008in}}%
\pgfpathlineto{\pgfqpoint{2.079256in}{0.846008in}}%
\pgfpathlineto{\pgfqpoint{2.079256in}{0.699509in}}%
\pgfpathlineto{\pgfqpoint{2.070503in}{0.699509in}}%
\pgfpathlineto{\pgfqpoint{2.070503in}{0.846008in}}%
\pgfpathclose%
\pgfusepath{fill}%
\end{pgfscope}%
\begin{pgfscope}%
\pgfpathrectangle{\pgfqpoint{0.804646in}{0.600000in}}{\pgfqpoint{2.573292in}{2.070576in}}%
\pgfusepath{clip}%
\pgfsetbuttcap%
\pgfsetmiterjoin%
\definecolor{currentfill}{rgb}{0.754268,0.565033,0.211761}%
\pgfsetfillcolor{currentfill}%
\pgfsetlinewidth{0.000000pt}%
\definecolor{currentstroke}{rgb}{0.000000,0.000000,0.000000}%
\pgfsetstrokecolor{currentstroke}%
\pgfsetstrokeopacity{0.000000}%
\pgfsetdash{}{0pt}%
\pgfpathmoveto{\pgfqpoint{2.081444in}{0.861357in}}%
\pgfpathlineto{\pgfqpoint{2.090198in}{0.861357in}}%
\pgfpathlineto{\pgfqpoint{2.090198in}{0.746005in}}%
\pgfpathlineto{\pgfqpoint{2.081444in}{0.746005in}}%
\pgfpathlineto{\pgfqpoint{2.081444in}{0.861357in}}%
\pgfpathclose%
\pgfusepath{fill}%
\end{pgfscope}%
\begin{pgfscope}%
\pgfpathrectangle{\pgfqpoint{0.804646in}{0.600000in}}{\pgfqpoint{2.573292in}{2.070576in}}%
\pgfusepath{clip}%
\pgfsetbuttcap%
\pgfsetmiterjoin%
\definecolor{currentfill}{rgb}{0.754268,0.565033,0.211761}%
\pgfsetfillcolor{currentfill}%
\pgfsetlinewidth{0.000000pt}%
\definecolor{currentstroke}{rgb}{0.000000,0.000000,0.000000}%
\pgfsetstrokecolor{currentstroke}%
\pgfsetstrokeopacity{0.000000}%
\pgfsetdash{}{0pt}%
\pgfpathmoveto{\pgfqpoint{2.092386in}{0.832246in}}%
\pgfpathlineto{\pgfqpoint{2.101140in}{0.832246in}}%
\pgfpathlineto{\pgfqpoint{2.101140in}{0.694117in}}%
\pgfpathlineto{\pgfqpoint{2.092386in}{0.694117in}}%
\pgfpathlineto{\pgfqpoint{2.092386in}{0.832246in}}%
\pgfpathclose%
\pgfusepath{fill}%
\end{pgfscope}%
\begin{pgfscope}%
\pgfpathrectangle{\pgfqpoint{0.804646in}{0.600000in}}{\pgfqpoint{2.573292in}{2.070576in}}%
\pgfusepath{clip}%
\pgfsetbuttcap%
\pgfsetmiterjoin%
\definecolor{currentfill}{rgb}{0.754268,0.565033,0.211761}%
\pgfsetfillcolor{currentfill}%
\pgfsetlinewidth{0.000000pt}%
\definecolor{currentstroke}{rgb}{0.000000,0.000000,0.000000}%
\pgfsetstrokecolor{currentstroke}%
\pgfsetstrokeopacity{0.000000}%
\pgfsetdash{}{0pt}%
\pgfpathmoveto{\pgfqpoint{2.103328in}{0.818772in}}%
\pgfpathlineto{\pgfqpoint{2.112081in}{0.818772in}}%
\pgfpathlineto{\pgfqpoint{2.112081in}{0.702993in}}%
\pgfpathlineto{\pgfqpoint{2.103328in}{0.702993in}}%
\pgfpathlineto{\pgfqpoint{2.103328in}{0.818772in}}%
\pgfpathclose%
\pgfusepath{fill}%
\end{pgfscope}%
\begin{pgfscope}%
\pgfpathrectangle{\pgfqpoint{0.804646in}{0.600000in}}{\pgfqpoint{2.573292in}{2.070576in}}%
\pgfusepath{clip}%
\pgfsetbuttcap%
\pgfsetmiterjoin%
\definecolor{currentfill}{rgb}{0.754268,0.565033,0.211761}%
\pgfsetfillcolor{currentfill}%
\pgfsetlinewidth{0.000000pt}%
\definecolor{currentstroke}{rgb}{0.000000,0.000000,0.000000}%
\pgfsetstrokecolor{currentstroke}%
\pgfsetstrokeopacity{0.000000}%
\pgfsetdash{}{0pt}%
\pgfpathmoveto{\pgfqpoint{2.114270in}{0.823211in}}%
\pgfpathlineto{\pgfqpoint{2.123023in}{0.823211in}}%
\pgfpathlineto{\pgfqpoint{2.123023in}{0.701678in}}%
\pgfpathlineto{\pgfqpoint{2.114270in}{0.701678in}}%
\pgfpathlineto{\pgfqpoint{2.114270in}{0.823211in}}%
\pgfpathclose%
\pgfusepath{fill}%
\end{pgfscope}%
\begin{pgfscope}%
\pgfpathrectangle{\pgfqpoint{0.804646in}{0.600000in}}{\pgfqpoint{2.573292in}{2.070576in}}%
\pgfusepath{clip}%
\pgfsetbuttcap%
\pgfsetmiterjoin%
\definecolor{currentfill}{rgb}{0.754268,0.565033,0.211761}%
\pgfsetfillcolor{currentfill}%
\pgfsetlinewidth{0.000000pt}%
\definecolor{currentstroke}{rgb}{0.000000,0.000000,0.000000}%
\pgfsetstrokecolor{currentstroke}%
\pgfsetstrokeopacity{0.000000}%
\pgfsetdash{}{0pt}%
\pgfpathmoveto{\pgfqpoint{2.125212in}{0.837325in}}%
\pgfpathlineto{\pgfqpoint{2.133965in}{0.837325in}}%
\pgfpathlineto{\pgfqpoint{2.133965in}{0.749922in}}%
\pgfpathlineto{\pgfqpoint{2.125212in}{0.749922in}}%
\pgfpathlineto{\pgfqpoint{2.125212in}{0.837325in}}%
\pgfpathclose%
\pgfusepath{fill}%
\end{pgfscope}%
\begin{pgfscope}%
\pgfpathrectangle{\pgfqpoint{0.804646in}{0.600000in}}{\pgfqpoint{2.573292in}{2.070576in}}%
\pgfusepath{clip}%
\pgfsetbuttcap%
\pgfsetmiterjoin%
\definecolor{currentfill}{rgb}{0.754268,0.565033,0.211761}%
\pgfsetfillcolor{currentfill}%
\pgfsetlinewidth{0.000000pt}%
\definecolor{currentstroke}{rgb}{0.000000,0.000000,0.000000}%
\pgfsetstrokecolor{currentstroke}%
\pgfsetstrokeopacity{0.000000}%
\pgfsetdash{}{0pt}%
\pgfpathmoveto{\pgfqpoint{2.136153in}{0.824255in}}%
\pgfpathlineto{\pgfqpoint{2.144907in}{0.824255in}}%
\pgfpathlineto{\pgfqpoint{2.144907in}{0.733408in}}%
\pgfpathlineto{\pgfqpoint{2.136153in}{0.733408in}}%
\pgfpathlineto{\pgfqpoint{2.136153in}{0.824255in}}%
\pgfpathclose%
\pgfusepath{fill}%
\end{pgfscope}%
\begin{pgfscope}%
\pgfpathrectangle{\pgfqpoint{0.804646in}{0.600000in}}{\pgfqpoint{2.573292in}{2.070576in}}%
\pgfusepath{clip}%
\pgfsetbuttcap%
\pgfsetmiterjoin%
\definecolor{currentfill}{rgb}{0.754268,0.565033,0.211761}%
\pgfsetfillcolor{currentfill}%
\pgfsetlinewidth{0.000000pt}%
\definecolor{currentstroke}{rgb}{0.000000,0.000000,0.000000}%
\pgfsetstrokecolor{currentstroke}%
\pgfsetstrokeopacity{0.000000}%
\pgfsetdash{}{0pt}%
\pgfpathmoveto{\pgfqpoint{2.147095in}{0.816083in}}%
\pgfpathlineto{\pgfqpoint{2.155849in}{0.816083in}}%
\pgfpathlineto{\pgfqpoint{2.155849in}{0.745003in}}%
\pgfpathlineto{\pgfqpoint{2.147095in}{0.745003in}}%
\pgfpathlineto{\pgfqpoint{2.147095in}{0.816083in}}%
\pgfpathclose%
\pgfusepath{fill}%
\end{pgfscope}%
\begin{pgfscope}%
\pgfpathrectangle{\pgfqpoint{0.804646in}{0.600000in}}{\pgfqpoint{2.573292in}{2.070576in}}%
\pgfusepath{clip}%
\pgfsetbuttcap%
\pgfsetmiterjoin%
\definecolor{currentfill}{rgb}{0.754268,0.565033,0.211761}%
\pgfsetfillcolor{currentfill}%
\pgfsetlinewidth{0.000000pt}%
\definecolor{currentstroke}{rgb}{0.000000,0.000000,0.000000}%
\pgfsetstrokecolor{currentstroke}%
\pgfsetstrokeopacity{0.000000}%
\pgfsetdash{}{0pt}%
\pgfpathmoveto{\pgfqpoint{2.158037in}{0.819333in}}%
\pgfpathlineto{\pgfqpoint{2.166790in}{0.819333in}}%
\pgfpathlineto{\pgfqpoint{2.166790in}{0.747831in}}%
\pgfpathlineto{\pgfqpoint{2.158037in}{0.747831in}}%
\pgfpathlineto{\pgfqpoint{2.158037in}{0.819333in}}%
\pgfpathclose%
\pgfusepath{fill}%
\end{pgfscope}%
\begin{pgfscope}%
\pgfpathrectangle{\pgfqpoint{0.804646in}{0.600000in}}{\pgfqpoint{2.573292in}{2.070576in}}%
\pgfusepath{clip}%
\pgfsetbuttcap%
\pgfsetmiterjoin%
\definecolor{currentfill}{rgb}{0.754268,0.565033,0.211761}%
\pgfsetfillcolor{currentfill}%
\pgfsetlinewidth{0.000000pt}%
\definecolor{currentstroke}{rgb}{0.000000,0.000000,0.000000}%
\pgfsetstrokecolor{currentstroke}%
\pgfsetstrokeopacity{0.000000}%
\pgfsetdash{}{0pt}%
\pgfpathmoveto{\pgfqpoint{2.168979in}{0.805306in}}%
\pgfpathlineto{\pgfqpoint{2.177732in}{0.805306in}}%
\pgfpathlineto{\pgfqpoint{2.177732in}{0.751753in}}%
\pgfpathlineto{\pgfqpoint{2.168979in}{0.751753in}}%
\pgfpathlineto{\pgfqpoint{2.168979in}{0.805306in}}%
\pgfpathclose%
\pgfusepath{fill}%
\end{pgfscope}%
\begin{pgfscope}%
\pgfpathrectangle{\pgfqpoint{0.804646in}{0.600000in}}{\pgfqpoint{2.573292in}{2.070576in}}%
\pgfusepath{clip}%
\pgfsetbuttcap%
\pgfsetmiterjoin%
\definecolor{currentfill}{rgb}{0.754268,0.565033,0.211761}%
\pgfsetfillcolor{currentfill}%
\pgfsetlinewidth{0.000000pt}%
\definecolor{currentstroke}{rgb}{0.000000,0.000000,0.000000}%
\pgfsetstrokecolor{currentstroke}%
\pgfsetstrokeopacity{0.000000}%
\pgfsetdash{}{0pt}%
\pgfpathmoveto{\pgfqpoint{2.179921in}{0.818514in}}%
\pgfpathlineto{\pgfqpoint{2.188674in}{0.818514in}}%
\pgfpathlineto{\pgfqpoint{2.188674in}{0.778286in}}%
\pgfpathlineto{\pgfqpoint{2.179921in}{0.778286in}}%
\pgfpathlineto{\pgfqpoint{2.179921in}{0.818514in}}%
\pgfpathclose%
\pgfusepath{fill}%
\end{pgfscope}%
\begin{pgfscope}%
\pgfpathrectangle{\pgfqpoint{0.804646in}{0.600000in}}{\pgfqpoint{2.573292in}{2.070576in}}%
\pgfusepath{clip}%
\pgfsetbuttcap%
\pgfsetmiterjoin%
\definecolor{currentfill}{rgb}{0.754268,0.565033,0.211761}%
\pgfsetfillcolor{currentfill}%
\pgfsetlinewidth{0.000000pt}%
\definecolor{currentstroke}{rgb}{0.000000,0.000000,0.000000}%
\pgfsetstrokecolor{currentstroke}%
\pgfsetstrokeopacity{0.000000}%
\pgfsetdash{}{0pt}%
\pgfpathmoveto{\pgfqpoint{2.190862in}{0.849907in}}%
\pgfpathlineto{\pgfqpoint{2.199616in}{0.849907in}}%
\pgfpathlineto{\pgfqpoint{2.199616in}{0.848847in}}%
\pgfpathlineto{\pgfqpoint{2.190862in}{0.848847in}}%
\pgfpathlineto{\pgfqpoint{2.190862in}{0.849907in}}%
\pgfpathclose%
\pgfusepath{fill}%
\end{pgfscope}%
\begin{pgfscope}%
\pgfpathrectangle{\pgfqpoint{0.804646in}{0.600000in}}{\pgfqpoint{2.573292in}{2.070576in}}%
\pgfusepath{clip}%
\pgfsetbuttcap%
\pgfsetmiterjoin%
\definecolor{currentfill}{rgb}{0.754268,0.565033,0.211761}%
\pgfsetfillcolor{currentfill}%
\pgfsetlinewidth{0.000000pt}%
\definecolor{currentstroke}{rgb}{0.000000,0.000000,0.000000}%
\pgfsetstrokecolor{currentstroke}%
\pgfsetstrokeopacity{0.000000}%
\pgfsetdash{}{0pt}%
\pgfpathmoveto{\pgfqpoint{2.201804in}{0.844020in}}%
\pgfpathlineto{\pgfqpoint{2.210558in}{0.844020in}}%
\pgfpathlineto{\pgfqpoint{2.210558in}{0.842368in}}%
\pgfpathlineto{\pgfqpoint{2.201804in}{0.842368in}}%
\pgfpathlineto{\pgfqpoint{2.201804in}{0.844020in}}%
\pgfpathclose%
\pgfusepath{fill}%
\end{pgfscope}%
\begin{pgfscope}%
\pgfpathrectangle{\pgfqpoint{0.804646in}{0.600000in}}{\pgfqpoint{2.573292in}{2.070576in}}%
\pgfusepath{clip}%
\pgfsetbuttcap%
\pgfsetmiterjoin%
\definecolor{currentfill}{rgb}{0.754268,0.565033,0.211761}%
\pgfsetfillcolor{currentfill}%
\pgfsetlinewidth{0.000000pt}%
\definecolor{currentstroke}{rgb}{0.000000,0.000000,0.000000}%
\pgfsetstrokecolor{currentstroke}%
\pgfsetstrokeopacity{0.000000}%
\pgfsetdash{}{0pt}%
\pgfpathmoveto{\pgfqpoint{2.212746in}{0.864624in}}%
\pgfpathlineto{\pgfqpoint{2.221499in}{0.864624in}}%
\pgfpathlineto{\pgfqpoint{2.221499in}{0.841038in}}%
\pgfpathlineto{\pgfqpoint{2.212746in}{0.841038in}}%
\pgfpathlineto{\pgfqpoint{2.212746in}{0.864624in}}%
\pgfpathclose%
\pgfusepath{fill}%
\end{pgfscope}%
\begin{pgfscope}%
\pgfpathrectangle{\pgfqpoint{0.804646in}{0.600000in}}{\pgfqpoint{2.573292in}{2.070576in}}%
\pgfusepath{clip}%
\pgfsetbuttcap%
\pgfsetmiterjoin%
\definecolor{currentfill}{rgb}{0.754268,0.565033,0.211761}%
\pgfsetfillcolor{currentfill}%
\pgfsetlinewidth{0.000000pt}%
\definecolor{currentstroke}{rgb}{0.000000,0.000000,0.000000}%
\pgfsetstrokecolor{currentstroke}%
\pgfsetstrokeopacity{0.000000}%
\pgfsetdash{}{0pt}%
\pgfpathmoveto{\pgfqpoint{2.223688in}{2.410059in}}%
\pgfpathlineto{\pgfqpoint{2.232441in}{2.410059in}}%
\pgfpathlineto{\pgfqpoint{2.232441in}{2.410829in}}%
\pgfpathlineto{\pgfqpoint{2.223688in}{2.410829in}}%
\pgfpathlineto{\pgfqpoint{2.223688in}{2.410059in}}%
\pgfpathclose%
\pgfusepath{fill}%
\end{pgfscope}%
\begin{pgfscope}%
\pgfpathrectangle{\pgfqpoint{0.804646in}{0.600000in}}{\pgfqpoint{2.573292in}{2.070576in}}%
\pgfusepath{clip}%
\pgfsetbuttcap%
\pgfsetmiterjoin%
\definecolor{currentfill}{rgb}{0.754268,0.565033,0.211761}%
\pgfsetfillcolor{currentfill}%
\pgfsetlinewidth{0.000000pt}%
\definecolor{currentstroke}{rgb}{0.000000,0.000000,0.000000}%
\pgfsetstrokecolor{currentstroke}%
\pgfsetstrokeopacity{0.000000}%
\pgfsetdash{}{0pt}%
\pgfpathmoveto{\pgfqpoint{2.234630in}{2.360319in}}%
\pgfpathlineto{\pgfqpoint{2.243383in}{2.360319in}}%
\pgfpathlineto{\pgfqpoint{2.243383in}{2.366830in}}%
\pgfpathlineto{\pgfqpoint{2.234630in}{2.366830in}}%
\pgfpathlineto{\pgfqpoint{2.234630in}{2.360319in}}%
\pgfpathclose%
\pgfusepath{fill}%
\end{pgfscope}%
\begin{pgfscope}%
\pgfpathrectangle{\pgfqpoint{0.804646in}{0.600000in}}{\pgfqpoint{2.573292in}{2.070576in}}%
\pgfusepath{clip}%
\pgfsetbuttcap%
\pgfsetmiterjoin%
\definecolor{currentfill}{rgb}{0.754268,0.565033,0.211761}%
\pgfsetfillcolor{currentfill}%
\pgfsetlinewidth{0.000000pt}%
\definecolor{currentstroke}{rgb}{0.000000,0.000000,0.000000}%
\pgfsetstrokecolor{currentstroke}%
\pgfsetstrokeopacity{0.000000}%
\pgfsetdash{}{0pt}%
\pgfpathmoveto{\pgfqpoint{2.245571in}{0.910819in}}%
\pgfpathlineto{\pgfqpoint{2.254325in}{0.910819in}}%
\pgfpathlineto{\pgfqpoint{2.254325in}{0.909121in}}%
\pgfpathlineto{\pgfqpoint{2.245571in}{0.909121in}}%
\pgfpathlineto{\pgfqpoint{2.245571in}{0.910819in}}%
\pgfpathclose%
\pgfusepath{fill}%
\end{pgfscope}%
\begin{pgfscope}%
\pgfpathrectangle{\pgfqpoint{0.804646in}{0.600000in}}{\pgfqpoint{2.573292in}{2.070576in}}%
\pgfusepath{clip}%
\pgfsetbuttcap%
\pgfsetmiterjoin%
\definecolor{currentfill}{rgb}{0.754268,0.565033,0.211761}%
\pgfsetfillcolor{currentfill}%
\pgfsetlinewidth{0.000000pt}%
\definecolor{currentstroke}{rgb}{0.000000,0.000000,0.000000}%
\pgfsetstrokecolor{currentstroke}%
\pgfsetstrokeopacity{0.000000}%
\pgfsetdash{}{0pt}%
\pgfpathmoveto{\pgfqpoint{2.256513in}{2.319209in}}%
\pgfpathlineto{\pgfqpoint{2.265267in}{2.319209in}}%
\pgfpathlineto{\pgfqpoint{2.265267in}{2.338655in}}%
\pgfpathlineto{\pgfqpoint{2.256513in}{2.338655in}}%
\pgfpathlineto{\pgfqpoint{2.256513in}{2.319209in}}%
\pgfpathclose%
\pgfusepath{fill}%
\end{pgfscope}%
\begin{pgfscope}%
\pgfpathrectangle{\pgfqpoint{0.804646in}{0.600000in}}{\pgfqpoint{2.573292in}{2.070576in}}%
\pgfusepath{clip}%
\pgfsetbuttcap%
\pgfsetmiterjoin%
\definecolor{currentfill}{rgb}{0.754268,0.565033,0.211761}%
\pgfsetfillcolor{currentfill}%
\pgfsetlinewidth{0.000000pt}%
\definecolor{currentstroke}{rgb}{0.000000,0.000000,0.000000}%
\pgfsetstrokecolor{currentstroke}%
\pgfsetstrokeopacity{0.000000}%
\pgfsetdash{}{0pt}%
\pgfpathmoveto{\pgfqpoint{2.267455in}{2.231370in}}%
\pgfpathlineto{\pgfqpoint{2.276208in}{2.231370in}}%
\pgfpathlineto{\pgfqpoint{2.276208in}{2.282070in}}%
\pgfpathlineto{\pgfqpoint{2.267455in}{2.282070in}}%
\pgfpathlineto{\pgfqpoint{2.267455in}{2.231370in}}%
\pgfpathclose%
\pgfusepath{fill}%
\end{pgfscope}%
\begin{pgfscope}%
\pgfpathrectangle{\pgfqpoint{0.804646in}{0.600000in}}{\pgfqpoint{2.573292in}{2.070576in}}%
\pgfusepath{clip}%
\pgfsetbuttcap%
\pgfsetmiterjoin%
\definecolor{currentfill}{rgb}{0.754268,0.565033,0.211761}%
\pgfsetfillcolor{currentfill}%
\pgfsetlinewidth{0.000000pt}%
\definecolor{currentstroke}{rgb}{0.000000,0.000000,0.000000}%
\pgfsetstrokecolor{currentstroke}%
\pgfsetstrokeopacity{0.000000}%
\pgfsetdash{}{0pt}%
\pgfpathmoveto{\pgfqpoint{2.278397in}{2.224362in}}%
\pgfpathlineto{\pgfqpoint{2.287150in}{2.224362in}}%
\pgfpathlineto{\pgfqpoint{2.287150in}{2.272669in}}%
\pgfpathlineto{\pgfqpoint{2.278397in}{2.272669in}}%
\pgfpathlineto{\pgfqpoint{2.278397in}{2.224362in}}%
\pgfpathclose%
\pgfusepath{fill}%
\end{pgfscope}%
\begin{pgfscope}%
\pgfpathrectangle{\pgfqpoint{0.804646in}{0.600000in}}{\pgfqpoint{2.573292in}{2.070576in}}%
\pgfusepath{clip}%
\pgfsetbuttcap%
\pgfsetmiterjoin%
\definecolor{currentfill}{rgb}{0.754268,0.565033,0.211761}%
\pgfsetfillcolor{currentfill}%
\pgfsetlinewidth{0.000000pt}%
\definecolor{currentstroke}{rgb}{0.000000,0.000000,0.000000}%
\pgfsetstrokecolor{currentstroke}%
\pgfsetstrokeopacity{0.000000}%
\pgfsetdash{}{0pt}%
\pgfpathmoveto{\pgfqpoint{2.289339in}{2.145759in}}%
\pgfpathlineto{\pgfqpoint{2.298092in}{2.145759in}}%
\pgfpathlineto{\pgfqpoint{2.298092in}{2.204714in}}%
\pgfpathlineto{\pgfqpoint{2.289339in}{2.204714in}}%
\pgfpathlineto{\pgfqpoint{2.289339in}{2.145759in}}%
\pgfpathclose%
\pgfusepath{fill}%
\end{pgfscope}%
\begin{pgfscope}%
\pgfpathrectangle{\pgfqpoint{0.804646in}{0.600000in}}{\pgfqpoint{2.573292in}{2.070576in}}%
\pgfusepath{clip}%
\pgfsetbuttcap%
\pgfsetmiterjoin%
\definecolor{currentfill}{rgb}{0.754268,0.565033,0.211761}%
\pgfsetfillcolor{currentfill}%
\pgfsetlinewidth{0.000000pt}%
\definecolor{currentstroke}{rgb}{0.000000,0.000000,0.000000}%
\pgfsetstrokecolor{currentstroke}%
\pgfsetstrokeopacity{0.000000}%
\pgfsetdash{}{0pt}%
\pgfpathmoveto{\pgfqpoint{2.300280in}{2.048288in}}%
\pgfpathlineto{\pgfqpoint{2.309034in}{2.048288in}}%
\pgfpathlineto{\pgfqpoint{2.309034in}{2.133211in}}%
\pgfpathlineto{\pgfqpoint{2.300280in}{2.133211in}}%
\pgfpathlineto{\pgfqpoint{2.300280in}{2.048288in}}%
\pgfpathclose%
\pgfusepath{fill}%
\end{pgfscope}%
\begin{pgfscope}%
\pgfpathrectangle{\pgfqpoint{0.804646in}{0.600000in}}{\pgfqpoint{2.573292in}{2.070576in}}%
\pgfusepath{clip}%
\pgfsetbuttcap%
\pgfsetmiterjoin%
\definecolor{currentfill}{rgb}{0.754268,0.565033,0.211761}%
\pgfsetfillcolor{currentfill}%
\pgfsetlinewidth{0.000000pt}%
\definecolor{currentstroke}{rgb}{0.000000,0.000000,0.000000}%
\pgfsetstrokecolor{currentstroke}%
\pgfsetstrokeopacity{0.000000}%
\pgfsetdash{}{0pt}%
\pgfpathmoveto{\pgfqpoint{2.311222in}{2.003593in}}%
\pgfpathlineto{\pgfqpoint{2.319976in}{2.003593in}}%
\pgfpathlineto{\pgfqpoint{2.319976in}{2.103780in}}%
\pgfpathlineto{\pgfqpoint{2.311222in}{2.103780in}}%
\pgfpathlineto{\pgfqpoint{2.311222in}{2.003593in}}%
\pgfpathclose%
\pgfusepath{fill}%
\end{pgfscope}%
\begin{pgfscope}%
\pgfpathrectangle{\pgfqpoint{0.804646in}{0.600000in}}{\pgfqpoint{2.573292in}{2.070576in}}%
\pgfusepath{clip}%
\pgfsetbuttcap%
\pgfsetmiterjoin%
\definecolor{currentfill}{rgb}{0.754268,0.565033,0.211761}%
\pgfsetfillcolor{currentfill}%
\pgfsetlinewidth{0.000000pt}%
\definecolor{currentstroke}{rgb}{0.000000,0.000000,0.000000}%
\pgfsetstrokecolor{currentstroke}%
\pgfsetstrokeopacity{0.000000}%
\pgfsetdash{}{0pt}%
\pgfpathmoveto{\pgfqpoint{2.322164in}{2.040048in}}%
\pgfpathlineto{\pgfqpoint{2.330917in}{2.040048in}}%
\pgfpathlineto{\pgfqpoint{2.330917in}{2.131372in}}%
\pgfpathlineto{\pgfqpoint{2.322164in}{2.131372in}}%
\pgfpathlineto{\pgfqpoint{2.322164in}{2.040048in}}%
\pgfpathclose%
\pgfusepath{fill}%
\end{pgfscope}%
\begin{pgfscope}%
\pgfpathrectangle{\pgfqpoint{0.804646in}{0.600000in}}{\pgfqpoint{2.573292in}{2.070576in}}%
\pgfusepath{clip}%
\pgfsetbuttcap%
\pgfsetmiterjoin%
\definecolor{currentfill}{rgb}{0.754268,0.565033,0.211761}%
\pgfsetfillcolor{currentfill}%
\pgfsetlinewidth{0.000000pt}%
\definecolor{currentstroke}{rgb}{0.000000,0.000000,0.000000}%
\pgfsetstrokecolor{currentstroke}%
\pgfsetstrokeopacity{0.000000}%
\pgfsetdash{}{0pt}%
\pgfpathmoveto{\pgfqpoint{2.333106in}{1.987160in}}%
\pgfpathlineto{\pgfqpoint{2.341859in}{1.987160in}}%
\pgfpathlineto{\pgfqpoint{2.341859in}{2.076978in}}%
\pgfpathlineto{\pgfqpoint{2.333106in}{2.076978in}}%
\pgfpathlineto{\pgfqpoint{2.333106in}{1.987160in}}%
\pgfpathclose%
\pgfusepath{fill}%
\end{pgfscope}%
\begin{pgfscope}%
\pgfpathrectangle{\pgfqpoint{0.804646in}{0.600000in}}{\pgfqpoint{2.573292in}{2.070576in}}%
\pgfusepath{clip}%
\pgfsetbuttcap%
\pgfsetmiterjoin%
\definecolor{currentfill}{rgb}{0.754268,0.565033,0.211761}%
\pgfsetfillcolor{currentfill}%
\pgfsetlinewidth{0.000000pt}%
\definecolor{currentstroke}{rgb}{0.000000,0.000000,0.000000}%
\pgfsetstrokecolor{currentstroke}%
\pgfsetstrokeopacity{0.000000}%
\pgfsetdash{}{0pt}%
\pgfpathmoveto{\pgfqpoint{2.344048in}{1.944229in}}%
\pgfpathlineto{\pgfqpoint{2.352801in}{1.944229in}}%
\pgfpathlineto{\pgfqpoint{2.352801in}{2.040924in}}%
\pgfpathlineto{\pgfqpoint{2.344048in}{2.040924in}}%
\pgfpathlineto{\pgfqpoint{2.344048in}{1.944229in}}%
\pgfpathclose%
\pgfusepath{fill}%
\end{pgfscope}%
\begin{pgfscope}%
\pgfpathrectangle{\pgfqpoint{0.804646in}{0.600000in}}{\pgfqpoint{2.573292in}{2.070576in}}%
\pgfusepath{clip}%
\pgfsetbuttcap%
\pgfsetmiterjoin%
\definecolor{currentfill}{rgb}{0.754268,0.565033,0.211761}%
\pgfsetfillcolor{currentfill}%
\pgfsetlinewidth{0.000000pt}%
\definecolor{currentstroke}{rgb}{0.000000,0.000000,0.000000}%
\pgfsetstrokecolor{currentstroke}%
\pgfsetstrokeopacity{0.000000}%
\pgfsetdash{}{0pt}%
\pgfpathmoveto{\pgfqpoint{2.354989in}{1.937340in}}%
\pgfpathlineto{\pgfqpoint{2.363743in}{1.937340in}}%
\pgfpathlineto{\pgfqpoint{2.363743in}{2.039556in}}%
\pgfpathlineto{\pgfqpoint{2.354989in}{2.039556in}}%
\pgfpathlineto{\pgfqpoint{2.354989in}{1.937340in}}%
\pgfpathclose%
\pgfusepath{fill}%
\end{pgfscope}%
\begin{pgfscope}%
\pgfpathrectangle{\pgfqpoint{0.804646in}{0.600000in}}{\pgfqpoint{2.573292in}{2.070576in}}%
\pgfusepath{clip}%
\pgfsetbuttcap%
\pgfsetmiterjoin%
\definecolor{currentfill}{rgb}{0.754268,0.565033,0.211761}%
\pgfsetfillcolor{currentfill}%
\pgfsetlinewidth{0.000000pt}%
\definecolor{currentstroke}{rgb}{0.000000,0.000000,0.000000}%
\pgfsetstrokecolor{currentstroke}%
\pgfsetstrokeopacity{0.000000}%
\pgfsetdash{}{0pt}%
\pgfpathmoveto{\pgfqpoint{2.365931in}{1.933041in}}%
\pgfpathlineto{\pgfqpoint{2.374685in}{1.933041in}}%
\pgfpathlineto{\pgfqpoint{2.374685in}{2.041859in}}%
\pgfpathlineto{\pgfqpoint{2.365931in}{2.041859in}}%
\pgfpathlineto{\pgfqpoint{2.365931in}{1.933041in}}%
\pgfpathclose%
\pgfusepath{fill}%
\end{pgfscope}%
\begin{pgfscope}%
\pgfpathrectangle{\pgfqpoint{0.804646in}{0.600000in}}{\pgfqpoint{2.573292in}{2.070576in}}%
\pgfusepath{clip}%
\pgfsetbuttcap%
\pgfsetmiterjoin%
\definecolor{currentfill}{rgb}{0.754268,0.565033,0.211761}%
\pgfsetfillcolor{currentfill}%
\pgfsetlinewidth{0.000000pt}%
\definecolor{currentstroke}{rgb}{0.000000,0.000000,0.000000}%
\pgfsetstrokecolor{currentstroke}%
\pgfsetstrokeopacity{0.000000}%
\pgfsetdash{}{0pt}%
\pgfpathmoveto{\pgfqpoint{2.376873in}{1.929476in}}%
\pgfpathlineto{\pgfqpoint{2.385626in}{1.929476in}}%
\pgfpathlineto{\pgfqpoint{2.385626in}{2.011852in}}%
\pgfpathlineto{\pgfqpoint{2.376873in}{2.011852in}}%
\pgfpathlineto{\pgfqpoint{2.376873in}{1.929476in}}%
\pgfpathclose%
\pgfusepath{fill}%
\end{pgfscope}%
\begin{pgfscope}%
\pgfpathrectangle{\pgfqpoint{0.804646in}{0.600000in}}{\pgfqpoint{2.573292in}{2.070576in}}%
\pgfusepath{clip}%
\pgfsetbuttcap%
\pgfsetmiterjoin%
\definecolor{currentfill}{rgb}{0.754268,0.565033,0.211761}%
\pgfsetfillcolor{currentfill}%
\pgfsetlinewidth{0.000000pt}%
\definecolor{currentstroke}{rgb}{0.000000,0.000000,0.000000}%
\pgfsetstrokecolor{currentstroke}%
\pgfsetstrokeopacity{0.000000}%
\pgfsetdash{}{0pt}%
\pgfpathmoveto{\pgfqpoint{2.387815in}{1.937362in}}%
\pgfpathlineto{\pgfqpoint{2.396568in}{1.937362in}}%
\pgfpathlineto{\pgfqpoint{2.396568in}{2.041831in}}%
\pgfpathlineto{\pgfqpoint{2.387815in}{2.041831in}}%
\pgfpathlineto{\pgfqpoint{2.387815in}{1.937362in}}%
\pgfpathclose%
\pgfusepath{fill}%
\end{pgfscope}%
\begin{pgfscope}%
\pgfpathrectangle{\pgfqpoint{0.804646in}{0.600000in}}{\pgfqpoint{2.573292in}{2.070576in}}%
\pgfusepath{clip}%
\pgfsetbuttcap%
\pgfsetmiterjoin%
\definecolor{currentfill}{rgb}{0.754268,0.565033,0.211761}%
\pgfsetfillcolor{currentfill}%
\pgfsetlinewidth{0.000000pt}%
\definecolor{currentstroke}{rgb}{0.000000,0.000000,0.000000}%
\pgfsetstrokecolor{currentstroke}%
\pgfsetstrokeopacity{0.000000}%
\pgfsetdash{}{0pt}%
\pgfpathmoveto{\pgfqpoint{2.398757in}{1.933728in}}%
\pgfpathlineto{\pgfqpoint{2.407510in}{1.933728in}}%
\pgfpathlineto{\pgfqpoint{2.407510in}{2.071084in}}%
\pgfpathlineto{\pgfqpoint{2.398757in}{2.071084in}}%
\pgfpathlineto{\pgfqpoint{2.398757in}{1.933728in}}%
\pgfpathclose%
\pgfusepath{fill}%
\end{pgfscope}%
\begin{pgfscope}%
\pgfpathrectangle{\pgfqpoint{0.804646in}{0.600000in}}{\pgfqpoint{2.573292in}{2.070576in}}%
\pgfusepath{clip}%
\pgfsetbuttcap%
\pgfsetmiterjoin%
\definecolor{currentfill}{rgb}{0.754268,0.565033,0.211761}%
\pgfsetfillcolor{currentfill}%
\pgfsetlinewidth{0.000000pt}%
\definecolor{currentstroke}{rgb}{0.000000,0.000000,0.000000}%
\pgfsetstrokecolor{currentstroke}%
\pgfsetstrokeopacity{0.000000}%
\pgfsetdash{}{0pt}%
\pgfpathmoveto{\pgfqpoint{2.409698in}{1.930337in}}%
\pgfpathlineto{\pgfqpoint{2.418452in}{1.930337in}}%
\pgfpathlineto{\pgfqpoint{2.418452in}{2.060411in}}%
\pgfpathlineto{\pgfqpoint{2.409698in}{2.060411in}}%
\pgfpathlineto{\pgfqpoint{2.409698in}{1.930337in}}%
\pgfpathclose%
\pgfusepath{fill}%
\end{pgfscope}%
\begin{pgfscope}%
\pgfpathrectangle{\pgfqpoint{0.804646in}{0.600000in}}{\pgfqpoint{2.573292in}{2.070576in}}%
\pgfusepath{clip}%
\pgfsetbuttcap%
\pgfsetmiterjoin%
\definecolor{currentfill}{rgb}{0.754268,0.565033,0.211761}%
\pgfsetfillcolor{currentfill}%
\pgfsetlinewidth{0.000000pt}%
\definecolor{currentstroke}{rgb}{0.000000,0.000000,0.000000}%
\pgfsetstrokecolor{currentstroke}%
\pgfsetstrokeopacity{0.000000}%
\pgfsetdash{}{0pt}%
\pgfpathmoveto{\pgfqpoint{2.420640in}{1.921126in}}%
\pgfpathlineto{\pgfqpoint{2.429394in}{1.921126in}}%
\pgfpathlineto{\pgfqpoint{2.429394in}{2.064590in}}%
\pgfpathlineto{\pgfqpoint{2.420640in}{2.064590in}}%
\pgfpathlineto{\pgfqpoint{2.420640in}{1.921126in}}%
\pgfpathclose%
\pgfusepath{fill}%
\end{pgfscope}%
\begin{pgfscope}%
\pgfpathrectangle{\pgfqpoint{0.804646in}{0.600000in}}{\pgfqpoint{2.573292in}{2.070576in}}%
\pgfusepath{clip}%
\pgfsetbuttcap%
\pgfsetmiterjoin%
\definecolor{currentfill}{rgb}{0.754268,0.565033,0.211761}%
\pgfsetfillcolor{currentfill}%
\pgfsetlinewidth{0.000000pt}%
\definecolor{currentstroke}{rgb}{0.000000,0.000000,0.000000}%
\pgfsetstrokecolor{currentstroke}%
\pgfsetstrokeopacity{0.000000}%
\pgfsetdash{}{0pt}%
\pgfpathmoveto{\pgfqpoint{2.431582in}{1.919255in}}%
\pgfpathlineto{\pgfqpoint{2.440335in}{1.919255in}}%
\pgfpathlineto{\pgfqpoint{2.440335in}{2.019322in}}%
\pgfpathlineto{\pgfqpoint{2.431582in}{2.019322in}}%
\pgfpathlineto{\pgfqpoint{2.431582in}{1.919255in}}%
\pgfpathclose%
\pgfusepath{fill}%
\end{pgfscope}%
\begin{pgfscope}%
\pgfpathrectangle{\pgfqpoint{0.804646in}{0.600000in}}{\pgfqpoint{2.573292in}{2.070576in}}%
\pgfusepath{clip}%
\pgfsetbuttcap%
\pgfsetmiterjoin%
\definecolor{currentfill}{rgb}{0.754268,0.565033,0.211761}%
\pgfsetfillcolor{currentfill}%
\pgfsetlinewidth{0.000000pt}%
\definecolor{currentstroke}{rgb}{0.000000,0.000000,0.000000}%
\pgfsetstrokecolor{currentstroke}%
\pgfsetstrokeopacity{0.000000}%
\pgfsetdash{}{0pt}%
\pgfpathmoveto{\pgfqpoint{2.442524in}{1.886546in}}%
\pgfpathlineto{\pgfqpoint{2.451277in}{1.886546in}}%
\pgfpathlineto{\pgfqpoint{2.451277in}{2.023214in}}%
\pgfpathlineto{\pgfqpoint{2.442524in}{2.023214in}}%
\pgfpathlineto{\pgfqpoint{2.442524in}{1.886546in}}%
\pgfpathclose%
\pgfusepath{fill}%
\end{pgfscope}%
\begin{pgfscope}%
\pgfpathrectangle{\pgfqpoint{0.804646in}{0.600000in}}{\pgfqpoint{2.573292in}{2.070576in}}%
\pgfusepath{clip}%
\pgfsetbuttcap%
\pgfsetmiterjoin%
\definecolor{currentfill}{rgb}{0.754268,0.565033,0.211761}%
\pgfsetfillcolor{currentfill}%
\pgfsetlinewidth{0.000000pt}%
\definecolor{currentstroke}{rgb}{0.000000,0.000000,0.000000}%
\pgfsetstrokecolor{currentstroke}%
\pgfsetstrokeopacity{0.000000}%
\pgfsetdash{}{0pt}%
\pgfpathmoveto{\pgfqpoint{2.453466in}{1.867025in}}%
\pgfpathlineto{\pgfqpoint{2.462219in}{1.867025in}}%
\pgfpathlineto{\pgfqpoint{2.462219in}{1.952650in}}%
\pgfpathlineto{\pgfqpoint{2.453466in}{1.952650in}}%
\pgfpathlineto{\pgfqpoint{2.453466in}{1.867025in}}%
\pgfpathclose%
\pgfusepath{fill}%
\end{pgfscope}%
\begin{pgfscope}%
\pgfpathrectangle{\pgfqpoint{0.804646in}{0.600000in}}{\pgfqpoint{2.573292in}{2.070576in}}%
\pgfusepath{clip}%
\pgfsetbuttcap%
\pgfsetmiterjoin%
\definecolor{currentfill}{rgb}{0.754268,0.565033,0.211761}%
\pgfsetfillcolor{currentfill}%
\pgfsetlinewidth{0.000000pt}%
\definecolor{currentstroke}{rgb}{0.000000,0.000000,0.000000}%
\pgfsetstrokecolor{currentstroke}%
\pgfsetstrokeopacity{0.000000}%
\pgfsetdash{}{0pt}%
\pgfpathmoveto{\pgfqpoint{2.464407in}{1.842987in}}%
\pgfpathlineto{\pgfqpoint{2.473161in}{1.842987in}}%
\pgfpathlineto{\pgfqpoint{2.473161in}{1.901685in}}%
\pgfpathlineto{\pgfqpoint{2.464407in}{1.901685in}}%
\pgfpathlineto{\pgfqpoint{2.464407in}{1.842987in}}%
\pgfpathclose%
\pgfusepath{fill}%
\end{pgfscope}%
\begin{pgfscope}%
\pgfpathrectangle{\pgfqpoint{0.804646in}{0.600000in}}{\pgfqpoint{2.573292in}{2.070576in}}%
\pgfusepath{clip}%
\pgfsetbuttcap%
\pgfsetmiterjoin%
\definecolor{currentfill}{rgb}{0.754268,0.565033,0.211761}%
\pgfsetfillcolor{currentfill}%
\pgfsetlinewidth{0.000000pt}%
\definecolor{currentstroke}{rgb}{0.000000,0.000000,0.000000}%
\pgfsetstrokecolor{currentstroke}%
\pgfsetstrokeopacity{0.000000}%
\pgfsetdash{}{0pt}%
\pgfpathmoveto{\pgfqpoint{2.475349in}{1.804860in}}%
\pgfpathlineto{\pgfqpoint{2.484103in}{1.804860in}}%
\pgfpathlineto{\pgfqpoint{2.484103in}{1.860875in}}%
\pgfpathlineto{\pgfqpoint{2.475349in}{1.860875in}}%
\pgfpathlineto{\pgfqpoint{2.475349in}{1.804860in}}%
\pgfpathclose%
\pgfusepath{fill}%
\end{pgfscope}%
\begin{pgfscope}%
\pgfpathrectangle{\pgfqpoint{0.804646in}{0.600000in}}{\pgfqpoint{2.573292in}{2.070576in}}%
\pgfusepath{clip}%
\pgfsetbuttcap%
\pgfsetmiterjoin%
\definecolor{currentfill}{rgb}{0.754268,0.565033,0.211761}%
\pgfsetfillcolor{currentfill}%
\pgfsetlinewidth{0.000000pt}%
\definecolor{currentstroke}{rgb}{0.000000,0.000000,0.000000}%
\pgfsetstrokecolor{currentstroke}%
\pgfsetstrokeopacity{0.000000}%
\pgfsetdash{}{0pt}%
\pgfpathmoveto{\pgfqpoint{2.486291in}{1.802446in}}%
\pgfpathlineto{\pgfqpoint{2.495044in}{1.802446in}}%
\pgfpathlineto{\pgfqpoint{2.495044in}{1.830954in}}%
\pgfpathlineto{\pgfqpoint{2.486291in}{1.830954in}}%
\pgfpathlineto{\pgfqpoint{2.486291in}{1.802446in}}%
\pgfpathclose%
\pgfusepath{fill}%
\end{pgfscope}%
\begin{pgfscope}%
\pgfpathrectangle{\pgfqpoint{0.804646in}{0.600000in}}{\pgfqpoint{2.573292in}{2.070576in}}%
\pgfusepath{clip}%
\pgfsetbuttcap%
\pgfsetmiterjoin%
\definecolor{currentfill}{rgb}{0.754268,0.565033,0.211761}%
\pgfsetfillcolor{currentfill}%
\pgfsetlinewidth{0.000000pt}%
\definecolor{currentstroke}{rgb}{0.000000,0.000000,0.000000}%
\pgfsetstrokecolor{currentstroke}%
\pgfsetstrokeopacity{0.000000}%
\pgfsetdash{}{0pt}%
\pgfpathmoveto{\pgfqpoint{2.497233in}{1.791248in}}%
\pgfpathlineto{\pgfqpoint{2.505986in}{1.791248in}}%
\pgfpathlineto{\pgfqpoint{2.505986in}{1.828480in}}%
\pgfpathlineto{\pgfqpoint{2.497233in}{1.828480in}}%
\pgfpathlineto{\pgfqpoint{2.497233in}{1.791248in}}%
\pgfpathclose%
\pgfusepath{fill}%
\end{pgfscope}%
\begin{pgfscope}%
\pgfpathrectangle{\pgfqpoint{0.804646in}{0.600000in}}{\pgfqpoint{2.573292in}{2.070576in}}%
\pgfusepath{clip}%
\pgfsetbuttcap%
\pgfsetmiterjoin%
\definecolor{currentfill}{rgb}{0.754268,0.565033,0.211761}%
\pgfsetfillcolor{currentfill}%
\pgfsetlinewidth{0.000000pt}%
\definecolor{currentstroke}{rgb}{0.000000,0.000000,0.000000}%
\pgfsetstrokecolor{currentstroke}%
\pgfsetstrokeopacity{0.000000}%
\pgfsetdash{}{0pt}%
\pgfpathmoveto{\pgfqpoint{2.508174in}{1.794143in}}%
\pgfpathlineto{\pgfqpoint{2.516928in}{1.794143in}}%
\pgfpathlineto{\pgfqpoint{2.516928in}{1.815789in}}%
\pgfpathlineto{\pgfqpoint{2.508174in}{1.815789in}}%
\pgfpathlineto{\pgfqpoint{2.508174in}{1.794143in}}%
\pgfpathclose%
\pgfusepath{fill}%
\end{pgfscope}%
\begin{pgfscope}%
\pgfpathrectangle{\pgfqpoint{0.804646in}{0.600000in}}{\pgfqpoint{2.573292in}{2.070576in}}%
\pgfusepath{clip}%
\pgfsetbuttcap%
\pgfsetmiterjoin%
\definecolor{currentfill}{rgb}{0.754268,0.565033,0.211761}%
\pgfsetfillcolor{currentfill}%
\pgfsetlinewidth{0.000000pt}%
\definecolor{currentstroke}{rgb}{0.000000,0.000000,0.000000}%
\pgfsetstrokecolor{currentstroke}%
\pgfsetstrokeopacity{0.000000}%
\pgfsetdash{}{0pt}%
\pgfpathmoveto{\pgfqpoint{2.519116in}{1.797758in}}%
\pgfpathlineto{\pgfqpoint{2.527870in}{1.797758in}}%
\pgfpathlineto{\pgfqpoint{2.527870in}{1.841950in}}%
\pgfpathlineto{\pgfqpoint{2.519116in}{1.841950in}}%
\pgfpathlineto{\pgfqpoint{2.519116in}{1.797758in}}%
\pgfpathclose%
\pgfusepath{fill}%
\end{pgfscope}%
\begin{pgfscope}%
\pgfpathrectangle{\pgfqpoint{0.804646in}{0.600000in}}{\pgfqpoint{2.573292in}{2.070576in}}%
\pgfusepath{clip}%
\pgfsetbuttcap%
\pgfsetmiterjoin%
\definecolor{currentfill}{rgb}{0.754268,0.565033,0.211761}%
\pgfsetfillcolor{currentfill}%
\pgfsetlinewidth{0.000000pt}%
\definecolor{currentstroke}{rgb}{0.000000,0.000000,0.000000}%
\pgfsetstrokecolor{currentstroke}%
\pgfsetstrokeopacity{0.000000}%
\pgfsetdash{}{0pt}%
\pgfpathmoveto{\pgfqpoint{2.530058in}{1.613090in}}%
\pgfpathlineto{\pgfqpoint{2.538812in}{1.613090in}}%
\pgfpathlineto{\pgfqpoint{2.538812in}{1.612632in}}%
\pgfpathlineto{\pgfqpoint{2.530058in}{1.612632in}}%
\pgfpathlineto{\pgfqpoint{2.530058in}{1.613090in}}%
\pgfpathclose%
\pgfusepath{fill}%
\end{pgfscope}%
\begin{pgfscope}%
\pgfpathrectangle{\pgfqpoint{0.804646in}{0.600000in}}{\pgfqpoint{2.573292in}{2.070576in}}%
\pgfusepath{clip}%
\pgfsetbuttcap%
\pgfsetmiterjoin%
\definecolor{currentfill}{rgb}{0.754268,0.565033,0.211761}%
\pgfsetfillcolor{currentfill}%
\pgfsetlinewidth{0.000000pt}%
\definecolor{currentstroke}{rgb}{0.000000,0.000000,0.000000}%
\pgfsetstrokecolor{currentstroke}%
\pgfsetstrokeopacity{0.000000}%
\pgfsetdash{}{0pt}%
\pgfpathmoveto{\pgfqpoint{2.541000in}{1.764748in}}%
\pgfpathlineto{\pgfqpoint{2.549753in}{1.764748in}}%
\pgfpathlineto{\pgfqpoint{2.549753in}{1.783554in}}%
\pgfpathlineto{\pgfqpoint{2.541000in}{1.783554in}}%
\pgfpathlineto{\pgfqpoint{2.541000in}{1.764748in}}%
\pgfpathclose%
\pgfusepath{fill}%
\end{pgfscope}%
\begin{pgfscope}%
\pgfpathrectangle{\pgfqpoint{0.804646in}{0.600000in}}{\pgfqpoint{2.573292in}{2.070576in}}%
\pgfusepath{clip}%
\pgfsetbuttcap%
\pgfsetmiterjoin%
\definecolor{currentfill}{rgb}{0.754268,0.565033,0.211761}%
\pgfsetfillcolor{currentfill}%
\pgfsetlinewidth{0.000000pt}%
\definecolor{currentstroke}{rgb}{0.000000,0.000000,0.000000}%
\pgfsetstrokecolor{currentstroke}%
\pgfsetstrokeopacity{0.000000}%
\pgfsetdash{}{0pt}%
\pgfpathmoveto{\pgfqpoint{2.551942in}{1.601758in}}%
\pgfpathlineto{\pgfqpoint{2.560695in}{1.601758in}}%
\pgfpathlineto{\pgfqpoint{2.560695in}{1.572845in}}%
\pgfpathlineto{\pgfqpoint{2.551942in}{1.572845in}}%
\pgfpathlineto{\pgfqpoint{2.551942in}{1.601758in}}%
\pgfpathclose%
\pgfusepath{fill}%
\end{pgfscope}%
\begin{pgfscope}%
\pgfpathrectangle{\pgfqpoint{0.804646in}{0.600000in}}{\pgfqpoint{2.573292in}{2.070576in}}%
\pgfusepath{clip}%
\pgfsetbuttcap%
\pgfsetmiterjoin%
\definecolor{currentfill}{rgb}{0.754268,0.565033,0.211761}%
\pgfsetfillcolor{currentfill}%
\pgfsetlinewidth{0.000000pt}%
\definecolor{currentstroke}{rgb}{0.000000,0.000000,0.000000}%
\pgfsetstrokecolor{currentstroke}%
\pgfsetstrokeopacity{0.000000}%
\pgfsetdash{}{0pt}%
\pgfpathmoveto{\pgfqpoint{2.562883in}{1.595029in}}%
\pgfpathlineto{\pgfqpoint{2.571637in}{1.595029in}}%
\pgfpathlineto{\pgfqpoint{2.571637in}{1.582060in}}%
\pgfpathlineto{\pgfqpoint{2.562883in}{1.582060in}}%
\pgfpathlineto{\pgfqpoint{2.562883in}{1.595029in}}%
\pgfpathclose%
\pgfusepath{fill}%
\end{pgfscope}%
\begin{pgfscope}%
\pgfpathrectangle{\pgfqpoint{0.804646in}{0.600000in}}{\pgfqpoint{2.573292in}{2.070576in}}%
\pgfusepath{clip}%
\pgfsetbuttcap%
\pgfsetmiterjoin%
\definecolor{currentfill}{rgb}{0.754268,0.565033,0.211761}%
\pgfsetfillcolor{currentfill}%
\pgfsetlinewidth{0.000000pt}%
\definecolor{currentstroke}{rgb}{0.000000,0.000000,0.000000}%
\pgfsetstrokecolor{currentstroke}%
\pgfsetstrokeopacity{0.000000}%
\pgfsetdash{}{0pt}%
\pgfpathmoveto{\pgfqpoint{2.573825in}{1.557572in}}%
\pgfpathlineto{\pgfqpoint{2.582579in}{1.557572in}}%
\pgfpathlineto{\pgfqpoint{2.582579in}{1.531583in}}%
\pgfpathlineto{\pgfqpoint{2.573825in}{1.531583in}}%
\pgfpathlineto{\pgfqpoint{2.573825in}{1.557572in}}%
\pgfpathclose%
\pgfusepath{fill}%
\end{pgfscope}%
\begin{pgfscope}%
\pgfpathrectangle{\pgfqpoint{0.804646in}{0.600000in}}{\pgfqpoint{2.573292in}{2.070576in}}%
\pgfusepath{clip}%
\pgfsetbuttcap%
\pgfsetmiterjoin%
\definecolor{currentfill}{rgb}{0.754268,0.565033,0.211761}%
\pgfsetfillcolor{currentfill}%
\pgfsetlinewidth{0.000000pt}%
\definecolor{currentstroke}{rgb}{0.000000,0.000000,0.000000}%
\pgfsetstrokecolor{currentstroke}%
\pgfsetstrokeopacity{0.000000}%
\pgfsetdash{}{0pt}%
\pgfpathmoveto{\pgfqpoint{2.584767in}{1.571129in}}%
\pgfpathlineto{\pgfqpoint{2.593521in}{1.571129in}}%
\pgfpathlineto{\pgfqpoint{2.593521in}{1.561783in}}%
\pgfpathlineto{\pgfqpoint{2.584767in}{1.561783in}}%
\pgfpathlineto{\pgfqpoint{2.584767in}{1.571129in}}%
\pgfpathclose%
\pgfusepath{fill}%
\end{pgfscope}%
\begin{pgfscope}%
\pgfpathrectangle{\pgfqpoint{0.804646in}{0.600000in}}{\pgfqpoint{2.573292in}{2.070576in}}%
\pgfusepath{clip}%
\pgfsetbuttcap%
\pgfsetmiterjoin%
\definecolor{currentfill}{rgb}{0.754268,0.565033,0.211761}%
\pgfsetfillcolor{currentfill}%
\pgfsetlinewidth{0.000000pt}%
\definecolor{currentstroke}{rgb}{0.000000,0.000000,0.000000}%
\pgfsetstrokecolor{currentstroke}%
\pgfsetstrokeopacity{0.000000}%
\pgfsetdash{}{0pt}%
\pgfpathmoveto{\pgfqpoint{2.595709in}{1.555425in}}%
\pgfpathlineto{\pgfqpoint{2.604462in}{1.555425in}}%
\pgfpathlineto{\pgfqpoint{2.604462in}{1.526879in}}%
\pgfpathlineto{\pgfqpoint{2.595709in}{1.526879in}}%
\pgfpathlineto{\pgfqpoint{2.595709in}{1.555425in}}%
\pgfpathclose%
\pgfusepath{fill}%
\end{pgfscope}%
\begin{pgfscope}%
\pgfpathrectangle{\pgfqpoint{0.804646in}{0.600000in}}{\pgfqpoint{2.573292in}{2.070576in}}%
\pgfusepath{clip}%
\pgfsetbuttcap%
\pgfsetmiterjoin%
\definecolor{currentfill}{rgb}{0.754268,0.565033,0.211761}%
\pgfsetfillcolor{currentfill}%
\pgfsetlinewidth{0.000000pt}%
\definecolor{currentstroke}{rgb}{0.000000,0.000000,0.000000}%
\pgfsetstrokecolor{currentstroke}%
\pgfsetstrokeopacity{0.000000}%
\pgfsetdash{}{0pt}%
\pgfpathmoveto{\pgfqpoint{2.606651in}{1.520629in}}%
\pgfpathlineto{\pgfqpoint{2.615404in}{1.520629in}}%
\pgfpathlineto{\pgfqpoint{2.615404in}{1.496120in}}%
\pgfpathlineto{\pgfqpoint{2.606651in}{1.496120in}}%
\pgfpathlineto{\pgfqpoint{2.606651in}{1.520629in}}%
\pgfpathclose%
\pgfusepath{fill}%
\end{pgfscope}%
\begin{pgfscope}%
\pgfpathrectangle{\pgfqpoint{0.804646in}{0.600000in}}{\pgfqpoint{2.573292in}{2.070576in}}%
\pgfusepath{clip}%
\pgfsetbuttcap%
\pgfsetmiterjoin%
\definecolor{currentfill}{rgb}{0.754268,0.565033,0.211761}%
\pgfsetfillcolor{currentfill}%
\pgfsetlinewidth{0.000000pt}%
\definecolor{currentstroke}{rgb}{0.000000,0.000000,0.000000}%
\pgfsetstrokecolor{currentstroke}%
\pgfsetstrokeopacity{0.000000}%
\pgfsetdash{}{0pt}%
\pgfpathmoveto{\pgfqpoint{2.617592in}{1.890252in}}%
\pgfpathlineto{\pgfqpoint{2.626346in}{1.890252in}}%
\pgfpathlineto{\pgfqpoint{2.626346in}{1.898603in}}%
\pgfpathlineto{\pgfqpoint{2.617592in}{1.898603in}}%
\pgfpathlineto{\pgfqpoint{2.617592in}{1.890252in}}%
\pgfpathclose%
\pgfusepath{fill}%
\end{pgfscope}%
\begin{pgfscope}%
\pgfpathrectangle{\pgfqpoint{0.804646in}{0.600000in}}{\pgfqpoint{2.573292in}{2.070576in}}%
\pgfusepath{clip}%
\pgfsetbuttcap%
\pgfsetmiterjoin%
\definecolor{currentfill}{rgb}{0.754268,0.565033,0.211761}%
\pgfsetfillcolor{currentfill}%
\pgfsetlinewidth{0.000000pt}%
\definecolor{currentstroke}{rgb}{0.000000,0.000000,0.000000}%
\pgfsetstrokecolor{currentstroke}%
\pgfsetstrokeopacity{0.000000}%
\pgfsetdash{}{0pt}%
\pgfpathmoveto{\pgfqpoint{2.628534in}{1.928086in}}%
\pgfpathlineto{\pgfqpoint{2.637288in}{1.928086in}}%
\pgfpathlineto{\pgfqpoint{2.637288in}{1.933662in}}%
\pgfpathlineto{\pgfqpoint{2.628534in}{1.933662in}}%
\pgfpathlineto{\pgfqpoint{2.628534in}{1.928086in}}%
\pgfpathclose%
\pgfusepath{fill}%
\end{pgfscope}%
\begin{pgfscope}%
\pgfpathrectangle{\pgfqpoint{0.804646in}{0.600000in}}{\pgfqpoint{2.573292in}{2.070576in}}%
\pgfusepath{clip}%
\pgfsetbuttcap%
\pgfsetmiterjoin%
\definecolor{currentfill}{rgb}{0.754268,0.565033,0.211761}%
\pgfsetfillcolor{currentfill}%
\pgfsetlinewidth{0.000000pt}%
\definecolor{currentstroke}{rgb}{0.000000,0.000000,0.000000}%
\pgfsetstrokecolor{currentstroke}%
\pgfsetstrokeopacity{0.000000}%
\pgfsetdash{}{0pt}%
\pgfpathmoveto{\pgfqpoint{2.639476in}{1.482266in}}%
\pgfpathlineto{\pgfqpoint{2.648230in}{1.482266in}}%
\pgfpathlineto{\pgfqpoint{2.648230in}{1.473104in}}%
\pgfpathlineto{\pgfqpoint{2.639476in}{1.473104in}}%
\pgfpathlineto{\pgfqpoint{2.639476in}{1.482266in}}%
\pgfpathclose%
\pgfusepath{fill}%
\end{pgfscope}%
\begin{pgfscope}%
\pgfpathrectangle{\pgfqpoint{0.804646in}{0.600000in}}{\pgfqpoint{2.573292in}{2.070576in}}%
\pgfusepath{clip}%
\pgfsetbuttcap%
\pgfsetmiterjoin%
\definecolor{currentfill}{rgb}{0.754268,0.565033,0.211761}%
\pgfsetfillcolor{currentfill}%
\pgfsetlinewidth{0.000000pt}%
\definecolor{currentstroke}{rgb}{0.000000,0.000000,0.000000}%
\pgfsetstrokecolor{currentstroke}%
\pgfsetstrokeopacity{0.000000}%
\pgfsetdash{}{0pt}%
\pgfpathmoveto{\pgfqpoint{2.650418in}{1.910910in}}%
\pgfpathlineto{\pgfqpoint{2.659171in}{1.910910in}}%
\pgfpathlineto{\pgfqpoint{2.659171in}{1.920671in}}%
\pgfpathlineto{\pgfqpoint{2.650418in}{1.920671in}}%
\pgfpathlineto{\pgfqpoint{2.650418in}{1.910910in}}%
\pgfpathclose%
\pgfusepath{fill}%
\end{pgfscope}%
\begin{pgfscope}%
\pgfpathrectangle{\pgfqpoint{0.804646in}{0.600000in}}{\pgfqpoint{2.573292in}{2.070576in}}%
\pgfusepath{clip}%
\pgfsetbuttcap%
\pgfsetmiterjoin%
\definecolor{currentfill}{rgb}{0.754268,0.565033,0.211761}%
\pgfsetfillcolor{currentfill}%
\pgfsetlinewidth{0.000000pt}%
\definecolor{currentstroke}{rgb}{0.000000,0.000000,0.000000}%
\pgfsetstrokecolor{currentstroke}%
\pgfsetstrokeopacity{0.000000}%
\pgfsetdash{}{0pt}%
\pgfpathmoveto{\pgfqpoint{2.661360in}{1.824781in}}%
\pgfpathlineto{\pgfqpoint{2.670113in}{1.824781in}}%
\pgfpathlineto{\pgfqpoint{2.670113in}{1.878258in}}%
\pgfpathlineto{\pgfqpoint{2.661360in}{1.878258in}}%
\pgfpathlineto{\pgfqpoint{2.661360in}{1.824781in}}%
\pgfpathclose%
\pgfusepath{fill}%
\end{pgfscope}%
\begin{pgfscope}%
\pgfpathrectangle{\pgfqpoint{0.804646in}{0.600000in}}{\pgfqpoint{2.573292in}{2.070576in}}%
\pgfusepath{clip}%
\pgfsetbuttcap%
\pgfsetmiterjoin%
\definecolor{currentfill}{rgb}{0.754268,0.565033,0.211761}%
\pgfsetfillcolor{currentfill}%
\pgfsetlinewidth{0.000000pt}%
\definecolor{currentstroke}{rgb}{0.000000,0.000000,0.000000}%
\pgfsetstrokecolor{currentstroke}%
\pgfsetstrokeopacity{0.000000}%
\pgfsetdash{}{0pt}%
\pgfpathmoveto{\pgfqpoint{2.672301in}{1.827927in}}%
\pgfpathlineto{\pgfqpoint{2.681055in}{1.827927in}}%
\pgfpathlineto{\pgfqpoint{2.681055in}{1.881039in}}%
\pgfpathlineto{\pgfqpoint{2.672301in}{1.881039in}}%
\pgfpathlineto{\pgfqpoint{2.672301in}{1.827927in}}%
\pgfpathclose%
\pgfusepath{fill}%
\end{pgfscope}%
\begin{pgfscope}%
\pgfpathrectangle{\pgfqpoint{0.804646in}{0.600000in}}{\pgfqpoint{2.573292in}{2.070576in}}%
\pgfusepath{clip}%
\pgfsetbuttcap%
\pgfsetmiterjoin%
\definecolor{currentfill}{rgb}{0.754268,0.565033,0.211761}%
\pgfsetfillcolor{currentfill}%
\pgfsetlinewidth{0.000000pt}%
\definecolor{currentstroke}{rgb}{0.000000,0.000000,0.000000}%
\pgfsetstrokecolor{currentstroke}%
\pgfsetstrokeopacity{0.000000}%
\pgfsetdash{}{0pt}%
\pgfpathmoveto{\pgfqpoint{2.683243in}{1.732406in}}%
\pgfpathlineto{\pgfqpoint{2.691997in}{1.732406in}}%
\pgfpathlineto{\pgfqpoint{2.691997in}{1.800188in}}%
\pgfpathlineto{\pgfqpoint{2.683243in}{1.800188in}}%
\pgfpathlineto{\pgfqpoint{2.683243in}{1.732406in}}%
\pgfpathclose%
\pgfusepath{fill}%
\end{pgfscope}%
\begin{pgfscope}%
\pgfpathrectangle{\pgfqpoint{0.804646in}{0.600000in}}{\pgfqpoint{2.573292in}{2.070576in}}%
\pgfusepath{clip}%
\pgfsetbuttcap%
\pgfsetmiterjoin%
\definecolor{currentfill}{rgb}{0.754268,0.565033,0.211761}%
\pgfsetfillcolor{currentfill}%
\pgfsetlinewidth{0.000000pt}%
\definecolor{currentstroke}{rgb}{0.000000,0.000000,0.000000}%
\pgfsetstrokecolor{currentstroke}%
\pgfsetstrokeopacity{0.000000}%
\pgfsetdash{}{0pt}%
\pgfpathmoveto{\pgfqpoint{2.694185in}{1.804584in}}%
\pgfpathlineto{\pgfqpoint{2.702939in}{1.804584in}}%
\pgfpathlineto{\pgfqpoint{2.702939in}{1.847316in}}%
\pgfpathlineto{\pgfqpoint{2.694185in}{1.847316in}}%
\pgfpathlineto{\pgfqpoint{2.694185in}{1.804584in}}%
\pgfpathclose%
\pgfusepath{fill}%
\end{pgfscope}%
\begin{pgfscope}%
\pgfpathrectangle{\pgfqpoint{0.804646in}{0.600000in}}{\pgfqpoint{2.573292in}{2.070576in}}%
\pgfusepath{clip}%
\pgfsetbuttcap%
\pgfsetmiterjoin%
\definecolor{currentfill}{rgb}{0.754268,0.565033,0.211761}%
\pgfsetfillcolor{currentfill}%
\pgfsetlinewidth{0.000000pt}%
\definecolor{currentstroke}{rgb}{0.000000,0.000000,0.000000}%
\pgfsetstrokecolor{currentstroke}%
\pgfsetstrokeopacity{0.000000}%
\pgfsetdash{}{0pt}%
\pgfpathmoveto{\pgfqpoint{2.705127in}{1.708719in}}%
\pgfpathlineto{\pgfqpoint{2.713880in}{1.708719in}}%
\pgfpathlineto{\pgfqpoint{2.713880in}{1.790129in}}%
\pgfpathlineto{\pgfqpoint{2.705127in}{1.790129in}}%
\pgfpathlineto{\pgfqpoint{2.705127in}{1.708719in}}%
\pgfpathclose%
\pgfusepath{fill}%
\end{pgfscope}%
\begin{pgfscope}%
\pgfpathrectangle{\pgfqpoint{0.804646in}{0.600000in}}{\pgfqpoint{2.573292in}{2.070576in}}%
\pgfusepath{clip}%
\pgfsetbuttcap%
\pgfsetmiterjoin%
\definecolor{currentfill}{rgb}{0.754268,0.565033,0.211761}%
\pgfsetfillcolor{currentfill}%
\pgfsetlinewidth{0.000000pt}%
\definecolor{currentstroke}{rgb}{0.000000,0.000000,0.000000}%
\pgfsetstrokecolor{currentstroke}%
\pgfsetstrokeopacity{0.000000}%
\pgfsetdash{}{0pt}%
\pgfpathmoveto{\pgfqpoint{2.716069in}{1.729155in}}%
\pgfpathlineto{\pgfqpoint{2.724822in}{1.729155in}}%
\pgfpathlineto{\pgfqpoint{2.724822in}{1.804085in}}%
\pgfpathlineto{\pgfqpoint{2.716069in}{1.804085in}}%
\pgfpathlineto{\pgfqpoint{2.716069in}{1.729155in}}%
\pgfpathclose%
\pgfusepath{fill}%
\end{pgfscope}%
\begin{pgfscope}%
\pgfpathrectangle{\pgfqpoint{0.804646in}{0.600000in}}{\pgfqpoint{2.573292in}{2.070576in}}%
\pgfusepath{clip}%
\pgfsetbuttcap%
\pgfsetmiterjoin%
\definecolor{currentfill}{rgb}{0.754268,0.565033,0.211761}%
\pgfsetfillcolor{currentfill}%
\pgfsetlinewidth{0.000000pt}%
\definecolor{currentstroke}{rgb}{0.000000,0.000000,0.000000}%
\pgfsetstrokecolor{currentstroke}%
\pgfsetstrokeopacity{0.000000}%
\pgfsetdash{}{0pt}%
\pgfpathmoveto{\pgfqpoint{2.727010in}{1.726580in}}%
\pgfpathlineto{\pgfqpoint{2.735764in}{1.726580in}}%
\pgfpathlineto{\pgfqpoint{2.735764in}{1.784636in}}%
\pgfpathlineto{\pgfqpoint{2.727010in}{1.784636in}}%
\pgfpathlineto{\pgfqpoint{2.727010in}{1.726580in}}%
\pgfpathclose%
\pgfusepath{fill}%
\end{pgfscope}%
\begin{pgfscope}%
\pgfpathrectangle{\pgfqpoint{0.804646in}{0.600000in}}{\pgfqpoint{2.573292in}{2.070576in}}%
\pgfusepath{clip}%
\pgfsetbuttcap%
\pgfsetmiterjoin%
\definecolor{currentfill}{rgb}{0.754268,0.565033,0.211761}%
\pgfsetfillcolor{currentfill}%
\pgfsetlinewidth{0.000000pt}%
\definecolor{currentstroke}{rgb}{0.000000,0.000000,0.000000}%
\pgfsetstrokecolor{currentstroke}%
\pgfsetstrokeopacity{0.000000}%
\pgfsetdash{}{0pt}%
\pgfpathmoveto{\pgfqpoint{2.737952in}{1.741979in}}%
\pgfpathlineto{\pgfqpoint{2.746706in}{1.741979in}}%
\pgfpathlineto{\pgfqpoint{2.746706in}{1.799981in}}%
\pgfpathlineto{\pgfqpoint{2.737952in}{1.799981in}}%
\pgfpathlineto{\pgfqpoint{2.737952in}{1.741979in}}%
\pgfpathclose%
\pgfusepath{fill}%
\end{pgfscope}%
\begin{pgfscope}%
\pgfpathrectangle{\pgfqpoint{0.804646in}{0.600000in}}{\pgfqpoint{2.573292in}{2.070576in}}%
\pgfusepath{clip}%
\pgfsetbuttcap%
\pgfsetmiterjoin%
\definecolor{currentfill}{rgb}{0.754268,0.565033,0.211761}%
\pgfsetfillcolor{currentfill}%
\pgfsetlinewidth{0.000000pt}%
\definecolor{currentstroke}{rgb}{0.000000,0.000000,0.000000}%
\pgfsetstrokecolor{currentstroke}%
\pgfsetstrokeopacity{0.000000}%
\pgfsetdash{}{0pt}%
\pgfpathmoveto{\pgfqpoint{2.748894in}{1.743554in}}%
\pgfpathlineto{\pgfqpoint{2.757648in}{1.743554in}}%
\pgfpathlineto{\pgfqpoint{2.757648in}{1.812442in}}%
\pgfpathlineto{\pgfqpoint{2.748894in}{1.812442in}}%
\pgfpathlineto{\pgfqpoint{2.748894in}{1.743554in}}%
\pgfpathclose%
\pgfusepath{fill}%
\end{pgfscope}%
\begin{pgfscope}%
\pgfpathrectangle{\pgfqpoint{0.804646in}{0.600000in}}{\pgfqpoint{2.573292in}{2.070576in}}%
\pgfusepath{clip}%
\pgfsetbuttcap%
\pgfsetmiterjoin%
\definecolor{currentfill}{rgb}{0.754268,0.565033,0.211761}%
\pgfsetfillcolor{currentfill}%
\pgfsetlinewidth{0.000000pt}%
\definecolor{currentstroke}{rgb}{0.000000,0.000000,0.000000}%
\pgfsetstrokecolor{currentstroke}%
\pgfsetstrokeopacity{0.000000}%
\pgfsetdash{}{0pt}%
\pgfpathmoveto{\pgfqpoint{2.759836in}{1.599842in}}%
\pgfpathlineto{\pgfqpoint{2.768589in}{1.599842in}}%
\pgfpathlineto{\pgfqpoint{2.768589in}{1.569919in}}%
\pgfpathlineto{\pgfqpoint{2.759836in}{1.569919in}}%
\pgfpathlineto{\pgfqpoint{2.759836in}{1.599842in}}%
\pgfpathclose%
\pgfusepath{fill}%
\end{pgfscope}%
\begin{pgfscope}%
\pgfpathrectangle{\pgfqpoint{0.804646in}{0.600000in}}{\pgfqpoint{2.573292in}{2.070576in}}%
\pgfusepath{clip}%
\pgfsetbuttcap%
\pgfsetmiterjoin%
\definecolor{currentfill}{rgb}{0.754268,0.565033,0.211761}%
\pgfsetfillcolor{currentfill}%
\pgfsetlinewidth{0.000000pt}%
\definecolor{currentstroke}{rgb}{0.000000,0.000000,0.000000}%
\pgfsetstrokecolor{currentstroke}%
\pgfsetstrokeopacity{0.000000}%
\pgfsetdash{}{0pt}%
\pgfpathmoveto{\pgfqpoint{2.770778in}{1.533675in}}%
\pgfpathlineto{\pgfqpoint{2.779531in}{1.533675in}}%
\pgfpathlineto{\pgfqpoint{2.779531in}{1.494089in}}%
\pgfpathlineto{\pgfqpoint{2.770778in}{1.494089in}}%
\pgfpathlineto{\pgfqpoint{2.770778in}{1.533675in}}%
\pgfpathclose%
\pgfusepath{fill}%
\end{pgfscope}%
\begin{pgfscope}%
\pgfpathrectangle{\pgfqpoint{0.804646in}{0.600000in}}{\pgfqpoint{2.573292in}{2.070576in}}%
\pgfusepath{clip}%
\pgfsetbuttcap%
\pgfsetmiterjoin%
\definecolor{currentfill}{rgb}{0.754268,0.565033,0.211761}%
\pgfsetfillcolor{currentfill}%
\pgfsetlinewidth{0.000000pt}%
\definecolor{currentstroke}{rgb}{0.000000,0.000000,0.000000}%
\pgfsetstrokecolor{currentstroke}%
\pgfsetstrokeopacity{0.000000}%
\pgfsetdash{}{0pt}%
\pgfpathmoveto{\pgfqpoint{2.781719in}{1.367788in}}%
\pgfpathlineto{\pgfqpoint{2.790473in}{1.367788in}}%
\pgfpathlineto{\pgfqpoint{2.790473in}{1.359625in}}%
\pgfpathlineto{\pgfqpoint{2.781719in}{1.359625in}}%
\pgfpathlineto{\pgfqpoint{2.781719in}{1.367788in}}%
\pgfpathclose%
\pgfusepath{fill}%
\end{pgfscope}%
\begin{pgfscope}%
\pgfpathrectangle{\pgfqpoint{0.804646in}{0.600000in}}{\pgfqpoint{2.573292in}{2.070576in}}%
\pgfusepath{clip}%
\pgfsetbuttcap%
\pgfsetmiterjoin%
\definecolor{currentfill}{rgb}{0.754268,0.565033,0.211761}%
\pgfsetfillcolor{currentfill}%
\pgfsetlinewidth{0.000000pt}%
\definecolor{currentstroke}{rgb}{0.000000,0.000000,0.000000}%
\pgfsetstrokecolor{currentstroke}%
\pgfsetstrokeopacity{0.000000}%
\pgfsetdash{}{0pt}%
\pgfpathmoveto{\pgfqpoint{2.792661in}{1.272819in}}%
\pgfpathlineto{\pgfqpoint{2.801415in}{1.272819in}}%
\pgfpathlineto{\pgfqpoint{2.801415in}{1.262346in}}%
\pgfpathlineto{\pgfqpoint{2.792661in}{1.262346in}}%
\pgfpathlineto{\pgfqpoint{2.792661in}{1.272819in}}%
\pgfpathclose%
\pgfusepath{fill}%
\end{pgfscope}%
\begin{pgfscope}%
\pgfpathrectangle{\pgfqpoint{0.804646in}{0.600000in}}{\pgfqpoint{2.573292in}{2.070576in}}%
\pgfusepath{clip}%
\pgfsetbuttcap%
\pgfsetmiterjoin%
\definecolor{currentfill}{rgb}{0.754268,0.565033,0.211761}%
\pgfsetfillcolor{currentfill}%
\pgfsetlinewidth{0.000000pt}%
\definecolor{currentstroke}{rgb}{0.000000,0.000000,0.000000}%
\pgfsetstrokecolor{currentstroke}%
\pgfsetstrokeopacity{0.000000}%
\pgfsetdash{}{0pt}%
\pgfpathmoveto{\pgfqpoint{2.803603in}{1.314975in}}%
\pgfpathlineto{\pgfqpoint{2.812357in}{1.314975in}}%
\pgfpathlineto{\pgfqpoint{2.812357in}{1.291284in}}%
\pgfpathlineto{\pgfqpoint{2.803603in}{1.291284in}}%
\pgfpathlineto{\pgfqpoint{2.803603in}{1.314975in}}%
\pgfpathclose%
\pgfusepath{fill}%
\end{pgfscope}%
\begin{pgfscope}%
\pgfpathrectangle{\pgfqpoint{0.804646in}{0.600000in}}{\pgfqpoint{2.573292in}{2.070576in}}%
\pgfusepath{clip}%
\pgfsetbuttcap%
\pgfsetmiterjoin%
\definecolor{currentfill}{rgb}{0.754268,0.565033,0.211761}%
\pgfsetfillcolor{currentfill}%
\pgfsetlinewidth{0.000000pt}%
\definecolor{currentstroke}{rgb}{0.000000,0.000000,0.000000}%
\pgfsetstrokecolor{currentstroke}%
\pgfsetstrokeopacity{0.000000}%
\pgfsetdash{}{0pt}%
\pgfpathmoveto{\pgfqpoint{2.814545in}{1.280827in}}%
\pgfpathlineto{\pgfqpoint{2.823298in}{1.280827in}}%
\pgfpathlineto{\pgfqpoint{2.823298in}{1.259140in}}%
\pgfpathlineto{\pgfqpoint{2.814545in}{1.259140in}}%
\pgfpathlineto{\pgfqpoint{2.814545in}{1.280827in}}%
\pgfpathclose%
\pgfusepath{fill}%
\end{pgfscope}%
\begin{pgfscope}%
\pgfpathrectangle{\pgfqpoint{0.804646in}{0.600000in}}{\pgfqpoint{2.573292in}{2.070576in}}%
\pgfusepath{clip}%
\pgfsetbuttcap%
\pgfsetmiterjoin%
\definecolor{currentfill}{rgb}{0.754268,0.565033,0.211761}%
\pgfsetfillcolor{currentfill}%
\pgfsetlinewidth{0.000000pt}%
\definecolor{currentstroke}{rgb}{0.000000,0.000000,0.000000}%
\pgfsetstrokecolor{currentstroke}%
\pgfsetstrokeopacity{0.000000}%
\pgfsetdash{}{0pt}%
\pgfpathmoveto{\pgfqpoint{2.825487in}{1.246157in}}%
\pgfpathlineto{\pgfqpoint{2.834240in}{1.246157in}}%
\pgfpathlineto{\pgfqpoint{2.834240in}{1.222208in}}%
\pgfpathlineto{\pgfqpoint{2.825487in}{1.222208in}}%
\pgfpathlineto{\pgfqpoint{2.825487in}{1.246157in}}%
\pgfpathclose%
\pgfusepath{fill}%
\end{pgfscope}%
\begin{pgfscope}%
\pgfpathrectangle{\pgfqpoint{0.804646in}{0.600000in}}{\pgfqpoint{2.573292in}{2.070576in}}%
\pgfusepath{clip}%
\pgfsetbuttcap%
\pgfsetmiterjoin%
\definecolor{currentfill}{rgb}{0.754268,0.565033,0.211761}%
\pgfsetfillcolor{currentfill}%
\pgfsetlinewidth{0.000000pt}%
\definecolor{currentstroke}{rgb}{0.000000,0.000000,0.000000}%
\pgfsetstrokecolor{currentstroke}%
\pgfsetstrokeopacity{0.000000}%
\pgfsetdash{}{0pt}%
\pgfpathmoveto{\pgfqpoint{2.836428in}{1.297926in}}%
\pgfpathlineto{\pgfqpoint{2.845182in}{1.297926in}}%
\pgfpathlineto{\pgfqpoint{2.845182in}{1.260898in}}%
\pgfpathlineto{\pgfqpoint{2.836428in}{1.260898in}}%
\pgfpathlineto{\pgfqpoint{2.836428in}{1.297926in}}%
\pgfpathclose%
\pgfusepath{fill}%
\end{pgfscope}%
\begin{pgfscope}%
\pgfpathrectangle{\pgfqpoint{0.804646in}{0.600000in}}{\pgfqpoint{2.573292in}{2.070576in}}%
\pgfusepath{clip}%
\pgfsetbuttcap%
\pgfsetmiterjoin%
\definecolor{currentfill}{rgb}{0.754268,0.565033,0.211761}%
\pgfsetfillcolor{currentfill}%
\pgfsetlinewidth{0.000000pt}%
\definecolor{currentstroke}{rgb}{0.000000,0.000000,0.000000}%
\pgfsetstrokecolor{currentstroke}%
\pgfsetstrokeopacity{0.000000}%
\pgfsetdash{}{0pt}%
\pgfpathmoveto{\pgfqpoint{2.847370in}{1.260588in}}%
\pgfpathlineto{\pgfqpoint{2.856124in}{1.260588in}}%
\pgfpathlineto{\pgfqpoint{2.856124in}{1.237804in}}%
\pgfpathlineto{\pgfqpoint{2.847370in}{1.237804in}}%
\pgfpathlineto{\pgfqpoint{2.847370in}{1.260588in}}%
\pgfpathclose%
\pgfusepath{fill}%
\end{pgfscope}%
\begin{pgfscope}%
\pgfpathrectangle{\pgfqpoint{0.804646in}{0.600000in}}{\pgfqpoint{2.573292in}{2.070576in}}%
\pgfusepath{clip}%
\pgfsetbuttcap%
\pgfsetmiterjoin%
\definecolor{currentfill}{rgb}{0.754268,0.565033,0.211761}%
\pgfsetfillcolor{currentfill}%
\pgfsetlinewidth{0.000000pt}%
\definecolor{currentstroke}{rgb}{0.000000,0.000000,0.000000}%
\pgfsetstrokecolor{currentstroke}%
\pgfsetstrokeopacity{0.000000}%
\pgfsetdash{}{0pt}%
\pgfpathmoveto{\pgfqpoint{2.858312in}{1.263614in}}%
\pgfpathlineto{\pgfqpoint{2.867066in}{1.263614in}}%
\pgfpathlineto{\pgfqpoint{2.867066in}{1.230979in}}%
\pgfpathlineto{\pgfqpoint{2.858312in}{1.230979in}}%
\pgfpathlineto{\pgfqpoint{2.858312in}{1.263614in}}%
\pgfpathclose%
\pgfusepath{fill}%
\end{pgfscope}%
\begin{pgfscope}%
\pgfpathrectangle{\pgfqpoint{0.804646in}{0.600000in}}{\pgfqpoint{2.573292in}{2.070576in}}%
\pgfusepath{clip}%
\pgfsetbuttcap%
\pgfsetmiterjoin%
\definecolor{currentfill}{rgb}{0.754268,0.565033,0.211761}%
\pgfsetfillcolor{currentfill}%
\pgfsetlinewidth{0.000000pt}%
\definecolor{currentstroke}{rgb}{0.000000,0.000000,0.000000}%
\pgfsetstrokecolor{currentstroke}%
\pgfsetstrokeopacity{0.000000}%
\pgfsetdash{}{0pt}%
\pgfpathmoveto{\pgfqpoint{2.869254in}{1.233079in}}%
\pgfpathlineto{\pgfqpoint{2.878007in}{1.233079in}}%
\pgfpathlineto{\pgfqpoint{2.878007in}{1.220680in}}%
\pgfpathlineto{\pgfqpoint{2.869254in}{1.220680in}}%
\pgfpathlineto{\pgfqpoint{2.869254in}{1.233079in}}%
\pgfpathclose%
\pgfusepath{fill}%
\end{pgfscope}%
\begin{pgfscope}%
\pgfpathrectangle{\pgfqpoint{0.804646in}{0.600000in}}{\pgfqpoint{2.573292in}{2.070576in}}%
\pgfusepath{clip}%
\pgfsetbuttcap%
\pgfsetmiterjoin%
\definecolor{currentfill}{rgb}{0.754268,0.565033,0.211761}%
\pgfsetfillcolor{currentfill}%
\pgfsetlinewidth{0.000000pt}%
\definecolor{currentstroke}{rgb}{0.000000,0.000000,0.000000}%
\pgfsetstrokecolor{currentstroke}%
\pgfsetstrokeopacity{0.000000}%
\pgfsetdash{}{0pt}%
\pgfpathmoveto{\pgfqpoint{2.880196in}{1.308210in}}%
\pgfpathlineto{\pgfqpoint{2.888949in}{1.308210in}}%
\pgfpathlineto{\pgfqpoint{2.888949in}{1.287884in}}%
\pgfpathlineto{\pgfqpoint{2.880196in}{1.287884in}}%
\pgfpathlineto{\pgfqpoint{2.880196in}{1.308210in}}%
\pgfpathclose%
\pgfusepath{fill}%
\end{pgfscope}%
\begin{pgfscope}%
\pgfpathrectangle{\pgfqpoint{0.804646in}{0.600000in}}{\pgfqpoint{2.573292in}{2.070576in}}%
\pgfusepath{clip}%
\pgfsetbuttcap%
\pgfsetmiterjoin%
\definecolor{currentfill}{rgb}{0.754268,0.565033,0.211761}%
\pgfsetfillcolor{currentfill}%
\pgfsetlinewidth{0.000000pt}%
\definecolor{currentstroke}{rgb}{0.000000,0.000000,0.000000}%
\pgfsetstrokecolor{currentstroke}%
\pgfsetstrokeopacity{0.000000}%
\pgfsetdash{}{0pt}%
\pgfpathmoveto{\pgfqpoint{2.891137in}{1.319906in}}%
\pgfpathlineto{\pgfqpoint{2.899891in}{1.319906in}}%
\pgfpathlineto{\pgfqpoint{2.899891in}{1.271805in}}%
\pgfpathlineto{\pgfqpoint{2.891137in}{1.271805in}}%
\pgfpathlineto{\pgfqpoint{2.891137in}{1.319906in}}%
\pgfpathclose%
\pgfusepath{fill}%
\end{pgfscope}%
\begin{pgfscope}%
\pgfpathrectangle{\pgfqpoint{0.804646in}{0.600000in}}{\pgfqpoint{2.573292in}{2.070576in}}%
\pgfusepath{clip}%
\pgfsetbuttcap%
\pgfsetmiterjoin%
\definecolor{currentfill}{rgb}{0.754268,0.565033,0.211761}%
\pgfsetfillcolor{currentfill}%
\pgfsetlinewidth{0.000000pt}%
\definecolor{currentstroke}{rgb}{0.000000,0.000000,0.000000}%
\pgfsetstrokecolor{currentstroke}%
\pgfsetstrokeopacity{0.000000}%
\pgfsetdash{}{0pt}%
\pgfpathmoveto{\pgfqpoint{2.902079in}{1.238649in}}%
\pgfpathlineto{\pgfqpoint{2.910833in}{1.238649in}}%
\pgfpathlineto{\pgfqpoint{2.910833in}{1.211961in}}%
\pgfpathlineto{\pgfqpoint{2.902079in}{1.211961in}}%
\pgfpathlineto{\pgfqpoint{2.902079in}{1.238649in}}%
\pgfpathclose%
\pgfusepath{fill}%
\end{pgfscope}%
\begin{pgfscope}%
\pgfpathrectangle{\pgfqpoint{0.804646in}{0.600000in}}{\pgfqpoint{2.573292in}{2.070576in}}%
\pgfusepath{clip}%
\pgfsetbuttcap%
\pgfsetmiterjoin%
\definecolor{currentfill}{rgb}{0.754268,0.565033,0.211761}%
\pgfsetfillcolor{currentfill}%
\pgfsetlinewidth{0.000000pt}%
\definecolor{currentstroke}{rgb}{0.000000,0.000000,0.000000}%
\pgfsetstrokecolor{currentstroke}%
\pgfsetstrokeopacity{0.000000}%
\pgfsetdash{}{0pt}%
\pgfpathmoveto{\pgfqpoint{2.913021in}{1.337766in}}%
\pgfpathlineto{\pgfqpoint{2.921774in}{1.337766in}}%
\pgfpathlineto{\pgfqpoint{2.921774in}{1.284018in}}%
\pgfpathlineto{\pgfqpoint{2.913021in}{1.284018in}}%
\pgfpathlineto{\pgfqpoint{2.913021in}{1.337766in}}%
\pgfpathclose%
\pgfusepath{fill}%
\end{pgfscope}%
\begin{pgfscope}%
\pgfpathrectangle{\pgfqpoint{0.804646in}{0.600000in}}{\pgfqpoint{2.573292in}{2.070576in}}%
\pgfusepath{clip}%
\pgfsetbuttcap%
\pgfsetmiterjoin%
\definecolor{currentfill}{rgb}{0.754268,0.565033,0.211761}%
\pgfsetfillcolor{currentfill}%
\pgfsetlinewidth{0.000000pt}%
\definecolor{currentstroke}{rgb}{0.000000,0.000000,0.000000}%
\pgfsetstrokecolor{currentstroke}%
\pgfsetstrokeopacity{0.000000}%
\pgfsetdash{}{0pt}%
\pgfpathmoveto{\pgfqpoint{2.923963in}{1.295673in}}%
\pgfpathlineto{\pgfqpoint{2.932716in}{1.295673in}}%
\pgfpathlineto{\pgfqpoint{2.932716in}{1.264404in}}%
\pgfpathlineto{\pgfqpoint{2.923963in}{1.264404in}}%
\pgfpathlineto{\pgfqpoint{2.923963in}{1.295673in}}%
\pgfpathclose%
\pgfusepath{fill}%
\end{pgfscope}%
\begin{pgfscope}%
\pgfpathrectangle{\pgfqpoint{0.804646in}{0.600000in}}{\pgfqpoint{2.573292in}{2.070576in}}%
\pgfusepath{clip}%
\pgfsetbuttcap%
\pgfsetmiterjoin%
\definecolor{currentfill}{rgb}{0.754268,0.565033,0.211761}%
\pgfsetfillcolor{currentfill}%
\pgfsetlinewidth{0.000000pt}%
\definecolor{currentstroke}{rgb}{0.000000,0.000000,0.000000}%
\pgfsetstrokecolor{currentstroke}%
\pgfsetstrokeopacity{0.000000}%
\pgfsetdash{}{0pt}%
\pgfpathmoveto{\pgfqpoint{2.934905in}{1.309241in}}%
\pgfpathlineto{\pgfqpoint{2.943658in}{1.309241in}}%
\pgfpathlineto{\pgfqpoint{2.943658in}{1.271139in}}%
\pgfpathlineto{\pgfqpoint{2.934905in}{1.271139in}}%
\pgfpathlineto{\pgfqpoint{2.934905in}{1.309241in}}%
\pgfpathclose%
\pgfusepath{fill}%
\end{pgfscope}%
\begin{pgfscope}%
\pgfpathrectangle{\pgfqpoint{0.804646in}{0.600000in}}{\pgfqpoint{2.573292in}{2.070576in}}%
\pgfusepath{clip}%
\pgfsetbuttcap%
\pgfsetmiterjoin%
\definecolor{currentfill}{rgb}{0.754268,0.565033,0.211761}%
\pgfsetfillcolor{currentfill}%
\pgfsetlinewidth{0.000000pt}%
\definecolor{currentstroke}{rgb}{0.000000,0.000000,0.000000}%
\pgfsetstrokecolor{currentstroke}%
\pgfsetstrokeopacity{0.000000}%
\pgfsetdash{}{0pt}%
\pgfpathmoveto{\pgfqpoint{2.945846in}{1.350061in}}%
\pgfpathlineto{\pgfqpoint{2.954600in}{1.350061in}}%
\pgfpathlineto{\pgfqpoint{2.954600in}{1.295431in}}%
\pgfpathlineto{\pgfqpoint{2.945846in}{1.295431in}}%
\pgfpathlineto{\pgfqpoint{2.945846in}{1.350061in}}%
\pgfpathclose%
\pgfusepath{fill}%
\end{pgfscope}%
\begin{pgfscope}%
\pgfpathrectangle{\pgfqpoint{0.804646in}{0.600000in}}{\pgfqpoint{2.573292in}{2.070576in}}%
\pgfusepath{clip}%
\pgfsetbuttcap%
\pgfsetmiterjoin%
\definecolor{currentfill}{rgb}{0.754268,0.565033,0.211761}%
\pgfsetfillcolor{currentfill}%
\pgfsetlinewidth{0.000000pt}%
\definecolor{currentstroke}{rgb}{0.000000,0.000000,0.000000}%
\pgfsetstrokecolor{currentstroke}%
\pgfsetstrokeopacity{0.000000}%
\pgfsetdash{}{0pt}%
\pgfpathmoveto{\pgfqpoint{2.956788in}{1.286815in}}%
\pgfpathlineto{\pgfqpoint{2.965542in}{1.286815in}}%
\pgfpathlineto{\pgfqpoint{2.965542in}{1.253280in}}%
\pgfpathlineto{\pgfqpoint{2.956788in}{1.253280in}}%
\pgfpathlineto{\pgfqpoint{2.956788in}{1.286815in}}%
\pgfpathclose%
\pgfusepath{fill}%
\end{pgfscope}%
\begin{pgfscope}%
\pgfpathrectangle{\pgfqpoint{0.804646in}{0.600000in}}{\pgfqpoint{2.573292in}{2.070576in}}%
\pgfusepath{clip}%
\pgfsetbuttcap%
\pgfsetmiterjoin%
\definecolor{currentfill}{rgb}{0.754268,0.565033,0.211761}%
\pgfsetfillcolor{currentfill}%
\pgfsetlinewidth{0.000000pt}%
\definecolor{currentstroke}{rgb}{0.000000,0.000000,0.000000}%
\pgfsetstrokecolor{currentstroke}%
\pgfsetstrokeopacity{0.000000}%
\pgfsetdash{}{0pt}%
\pgfpathmoveto{\pgfqpoint{2.967730in}{1.864265in}}%
\pgfpathlineto{\pgfqpoint{2.976483in}{1.864265in}}%
\pgfpathlineto{\pgfqpoint{2.976483in}{1.867525in}}%
\pgfpathlineto{\pgfqpoint{2.967730in}{1.867525in}}%
\pgfpathlineto{\pgfqpoint{2.967730in}{1.864265in}}%
\pgfpathclose%
\pgfusepath{fill}%
\end{pgfscope}%
\begin{pgfscope}%
\pgfpathrectangle{\pgfqpoint{0.804646in}{0.600000in}}{\pgfqpoint{2.573292in}{2.070576in}}%
\pgfusepath{clip}%
\pgfsetbuttcap%
\pgfsetmiterjoin%
\definecolor{currentfill}{rgb}{0.754268,0.565033,0.211761}%
\pgfsetfillcolor{currentfill}%
\pgfsetlinewidth{0.000000pt}%
\definecolor{currentstroke}{rgb}{0.000000,0.000000,0.000000}%
\pgfsetstrokecolor{currentstroke}%
\pgfsetstrokeopacity{0.000000}%
\pgfsetdash{}{0pt}%
\pgfpathmoveto{\pgfqpoint{2.978672in}{1.239020in}}%
\pgfpathlineto{\pgfqpoint{2.987425in}{1.239020in}}%
\pgfpathlineto{\pgfqpoint{2.987425in}{1.231605in}}%
\pgfpathlineto{\pgfqpoint{2.978672in}{1.231605in}}%
\pgfpathlineto{\pgfqpoint{2.978672in}{1.239020in}}%
\pgfpathclose%
\pgfusepath{fill}%
\end{pgfscope}%
\begin{pgfscope}%
\pgfpathrectangle{\pgfqpoint{0.804646in}{0.600000in}}{\pgfqpoint{2.573292in}{2.070576in}}%
\pgfusepath{clip}%
\pgfsetbuttcap%
\pgfsetmiterjoin%
\definecolor{currentfill}{rgb}{0.754268,0.565033,0.211761}%
\pgfsetfillcolor{currentfill}%
\pgfsetlinewidth{0.000000pt}%
\definecolor{currentstroke}{rgb}{0.000000,0.000000,0.000000}%
\pgfsetstrokecolor{currentstroke}%
\pgfsetstrokeopacity{0.000000}%
\pgfsetdash{}{0pt}%
\pgfpathmoveto{\pgfqpoint{2.989614in}{1.286095in}}%
\pgfpathlineto{\pgfqpoint{2.998367in}{1.286095in}}%
\pgfpathlineto{\pgfqpoint{2.998367in}{1.243700in}}%
\pgfpathlineto{\pgfqpoint{2.989614in}{1.243700in}}%
\pgfpathlineto{\pgfqpoint{2.989614in}{1.286095in}}%
\pgfpathclose%
\pgfusepath{fill}%
\end{pgfscope}%
\begin{pgfscope}%
\pgfpathrectangle{\pgfqpoint{0.804646in}{0.600000in}}{\pgfqpoint{2.573292in}{2.070576in}}%
\pgfusepath{clip}%
\pgfsetbuttcap%
\pgfsetmiterjoin%
\definecolor{currentfill}{rgb}{0.754268,0.565033,0.211761}%
\pgfsetfillcolor{currentfill}%
\pgfsetlinewidth{0.000000pt}%
\definecolor{currentstroke}{rgb}{0.000000,0.000000,0.000000}%
\pgfsetstrokecolor{currentstroke}%
\pgfsetstrokeopacity{0.000000}%
\pgfsetdash{}{0pt}%
\pgfpathmoveto{\pgfqpoint{3.000555in}{1.236249in}}%
\pgfpathlineto{\pgfqpoint{3.009309in}{1.236249in}}%
\pgfpathlineto{\pgfqpoint{3.009309in}{1.203860in}}%
\pgfpathlineto{\pgfqpoint{3.000555in}{1.203860in}}%
\pgfpathlineto{\pgfqpoint{3.000555in}{1.236249in}}%
\pgfpathclose%
\pgfusepath{fill}%
\end{pgfscope}%
\begin{pgfscope}%
\pgfpathrectangle{\pgfqpoint{0.804646in}{0.600000in}}{\pgfqpoint{2.573292in}{2.070576in}}%
\pgfusepath{clip}%
\pgfsetbuttcap%
\pgfsetmiterjoin%
\definecolor{currentfill}{rgb}{0.754268,0.565033,0.211761}%
\pgfsetfillcolor{currentfill}%
\pgfsetlinewidth{0.000000pt}%
\definecolor{currentstroke}{rgb}{0.000000,0.000000,0.000000}%
\pgfsetstrokecolor{currentstroke}%
\pgfsetstrokeopacity{0.000000}%
\pgfsetdash{}{0pt}%
\pgfpathmoveto{\pgfqpoint{3.011497in}{1.237534in}}%
\pgfpathlineto{\pgfqpoint{3.020251in}{1.237534in}}%
\pgfpathlineto{\pgfqpoint{3.020251in}{1.229681in}}%
\pgfpathlineto{\pgfqpoint{3.011497in}{1.229681in}}%
\pgfpathlineto{\pgfqpoint{3.011497in}{1.237534in}}%
\pgfpathclose%
\pgfusepath{fill}%
\end{pgfscope}%
\begin{pgfscope}%
\pgfpathrectangle{\pgfqpoint{0.804646in}{0.600000in}}{\pgfqpoint{2.573292in}{2.070576in}}%
\pgfusepath{clip}%
\pgfsetbuttcap%
\pgfsetmiterjoin%
\definecolor{currentfill}{rgb}{0.754268,0.565033,0.211761}%
\pgfsetfillcolor{currentfill}%
\pgfsetlinewidth{0.000000pt}%
\definecolor{currentstroke}{rgb}{0.000000,0.000000,0.000000}%
\pgfsetstrokecolor{currentstroke}%
\pgfsetstrokeopacity{0.000000}%
\pgfsetdash{}{0pt}%
\pgfpathmoveto{\pgfqpoint{3.022439in}{1.242077in}}%
\pgfpathlineto{\pgfqpoint{3.031192in}{1.242077in}}%
\pgfpathlineto{\pgfqpoint{3.031192in}{1.228395in}}%
\pgfpathlineto{\pgfqpoint{3.022439in}{1.228395in}}%
\pgfpathlineto{\pgfqpoint{3.022439in}{1.242077in}}%
\pgfpathclose%
\pgfusepath{fill}%
\end{pgfscope}%
\begin{pgfscope}%
\pgfpathrectangle{\pgfqpoint{0.804646in}{0.600000in}}{\pgfqpoint{2.573292in}{2.070576in}}%
\pgfusepath{clip}%
\pgfsetbuttcap%
\pgfsetmiterjoin%
\definecolor{currentfill}{rgb}{0.754268,0.565033,0.211761}%
\pgfsetfillcolor{currentfill}%
\pgfsetlinewidth{0.000000pt}%
\definecolor{currentstroke}{rgb}{0.000000,0.000000,0.000000}%
\pgfsetstrokecolor{currentstroke}%
\pgfsetstrokeopacity{0.000000}%
\pgfsetdash{}{0pt}%
\pgfpathmoveto{\pgfqpoint{3.033381in}{1.211386in}}%
\pgfpathlineto{\pgfqpoint{3.042134in}{1.211386in}}%
\pgfpathlineto{\pgfqpoint{3.042134in}{1.204423in}}%
\pgfpathlineto{\pgfqpoint{3.033381in}{1.204423in}}%
\pgfpathlineto{\pgfqpoint{3.033381in}{1.211386in}}%
\pgfpathclose%
\pgfusepath{fill}%
\end{pgfscope}%
\begin{pgfscope}%
\pgfpathrectangle{\pgfqpoint{0.804646in}{0.600000in}}{\pgfqpoint{2.573292in}{2.070576in}}%
\pgfusepath{clip}%
\pgfsetbuttcap%
\pgfsetmiterjoin%
\definecolor{currentfill}{rgb}{0.754268,0.565033,0.211761}%
\pgfsetfillcolor{currentfill}%
\pgfsetlinewidth{0.000000pt}%
\definecolor{currentstroke}{rgb}{0.000000,0.000000,0.000000}%
\pgfsetstrokecolor{currentstroke}%
\pgfsetstrokeopacity{0.000000}%
\pgfsetdash{}{0pt}%
\pgfpathmoveto{\pgfqpoint{3.044323in}{1.914480in}}%
\pgfpathlineto{\pgfqpoint{3.053076in}{1.914480in}}%
\pgfpathlineto{\pgfqpoint{3.053076in}{1.945329in}}%
\pgfpathlineto{\pgfqpoint{3.044323in}{1.945329in}}%
\pgfpathlineto{\pgfqpoint{3.044323in}{1.914480in}}%
\pgfpathclose%
\pgfusepath{fill}%
\end{pgfscope}%
\begin{pgfscope}%
\pgfpathrectangle{\pgfqpoint{0.804646in}{0.600000in}}{\pgfqpoint{2.573292in}{2.070576in}}%
\pgfusepath{clip}%
\pgfsetbuttcap%
\pgfsetmiterjoin%
\definecolor{currentfill}{rgb}{0.754268,0.565033,0.211761}%
\pgfsetfillcolor{currentfill}%
\pgfsetlinewidth{0.000000pt}%
\definecolor{currentstroke}{rgb}{0.000000,0.000000,0.000000}%
\pgfsetstrokecolor{currentstroke}%
\pgfsetstrokeopacity{0.000000}%
\pgfsetdash{}{0pt}%
\pgfpathmoveto{\pgfqpoint{3.055264in}{1.859742in}}%
\pgfpathlineto{\pgfqpoint{3.064018in}{1.859742in}}%
\pgfpathlineto{\pgfqpoint{3.064018in}{1.870671in}}%
\pgfpathlineto{\pgfqpoint{3.055264in}{1.870671in}}%
\pgfpathlineto{\pgfqpoint{3.055264in}{1.859742in}}%
\pgfpathclose%
\pgfusepath{fill}%
\end{pgfscope}%
\begin{pgfscope}%
\pgfpathrectangle{\pgfqpoint{0.804646in}{0.600000in}}{\pgfqpoint{2.573292in}{2.070576in}}%
\pgfusepath{clip}%
\pgfsetbuttcap%
\pgfsetmiterjoin%
\definecolor{currentfill}{rgb}{0.754268,0.565033,0.211761}%
\pgfsetfillcolor{currentfill}%
\pgfsetlinewidth{0.000000pt}%
\definecolor{currentstroke}{rgb}{0.000000,0.000000,0.000000}%
\pgfsetstrokecolor{currentstroke}%
\pgfsetstrokeopacity{0.000000}%
\pgfsetdash{}{0pt}%
\pgfpathmoveto{\pgfqpoint{3.066206in}{1.892030in}}%
\pgfpathlineto{\pgfqpoint{3.074960in}{1.892030in}}%
\pgfpathlineto{\pgfqpoint{3.074960in}{1.917095in}}%
\pgfpathlineto{\pgfqpoint{3.066206in}{1.917095in}}%
\pgfpathlineto{\pgfqpoint{3.066206in}{1.892030in}}%
\pgfpathclose%
\pgfusepath{fill}%
\end{pgfscope}%
\begin{pgfscope}%
\pgfpathrectangle{\pgfqpoint{0.804646in}{0.600000in}}{\pgfqpoint{2.573292in}{2.070576in}}%
\pgfusepath{clip}%
\pgfsetbuttcap%
\pgfsetmiterjoin%
\definecolor{currentfill}{rgb}{0.754268,0.565033,0.211761}%
\pgfsetfillcolor{currentfill}%
\pgfsetlinewidth{0.000000pt}%
\definecolor{currentstroke}{rgb}{0.000000,0.000000,0.000000}%
\pgfsetstrokecolor{currentstroke}%
\pgfsetstrokeopacity{0.000000}%
\pgfsetdash{}{0pt}%
\pgfpathmoveto{\pgfqpoint{3.077148in}{1.093077in}}%
\pgfpathlineto{\pgfqpoint{3.085901in}{1.093077in}}%
\pgfpathlineto{\pgfqpoint{3.085901in}{1.087298in}}%
\pgfpathlineto{\pgfqpoint{3.077148in}{1.087298in}}%
\pgfpathlineto{\pgfqpoint{3.077148in}{1.093077in}}%
\pgfpathclose%
\pgfusepath{fill}%
\end{pgfscope}%
\begin{pgfscope}%
\pgfpathrectangle{\pgfqpoint{0.804646in}{0.600000in}}{\pgfqpoint{2.573292in}{2.070576in}}%
\pgfusepath{clip}%
\pgfsetbuttcap%
\pgfsetmiterjoin%
\definecolor{currentfill}{rgb}{0.754268,0.565033,0.211761}%
\pgfsetfillcolor{currentfill}%
\pgfsetlinewidth{0.000000pt}%
\definecolor{currentstroke}{rgb}{0.000000,0.000000,0.000000}%
\pgfsetstrokecolor{currentstroke}%
\pgfsetstrokeopacity{0.000000}%
\pgfsetdash{}{0pt}%
\pgfpathmoveto{\pgfqpoint{3.088090in}{2.006791in}}%
\pgfpathlineto{\pgfqpoint{3.096843in}{2.006791in}}%
\pgfpathlineto{\pgfqpoint{3.096843in}{2.038442in}}%
\pgfpathlineto{\pgfqpoint{3.088090in}{2.038442in}}%
\pgfpathlineto{\pgfqpoint{3.088090in}{2.006791in}}%
\pgfpathclose%
\pgfusepath{fill}%
\end{pgfscope}%
\begin{pgfscope}%
\pgfpathrectangle{\pgfqpoint{0.804646in}{0.600000in}}{\pgfqpoint{2.573292in}{2.070576in}}%
\pgfusepath{clip}%
\pgfsetbuttcap%
\pgfsetmiterjoin%
\definecolor{currentfill}{rgb}{0.754268,0.565033,0.211761}%
\pgfsetfillcolor{currentfill}%
\pgfsetlinewidth{0.000000pt}%
\definecolor{currentstroke}{rgb}{0.000000,0.000000,0.000000}%
\pgfsetstrokecolor{currentstroke}%
\pgfsetstrokeopacity{0.000000}%
\pgfsetdash{}{0pt}%
\pgfpathmoveto{\pgfqpoint{3.099032in}{1.950376in}}%
\pgfpathlineto{\pgfqpoint{3.107785in}{1.950376in}}%
\pgfpathlineto{\pgfqpoint{3.107785in}{1.974128in}}%
\pgfpathlineto{\pgfqpoint{3.099032in}{1.974128in}}%
\pgfpathlineto{\pgfqpoint{3.099032in}{1.950376in}}%
\pgfpathclose%
\pgfusepath{fill}%
\end{pgfscope}%
\begin{pgfscope}%
\pgfpathrectangle{\pgfqpoint{0.804646in}{0.600000in}}{\pgfqpoint{2.573292in}{2.070576in}}%
\pgfusepath{clip}%
\pgfsetbuttcap%
\pgfsetmiterjoin%
\definecolor{currentfill}{rgb}{0.754268,0.565033,0.211761}%
\pgfsetfillcolor{currentfill}%
\pgfsetlinewidth{0.000000pt}%
\definecolor{currentstroke}{rgb}{0.000000,0.000000,0.000000}%
\pgfsetstrokecolor{currentstroke}%
\pgfsetstrokeopacity{0.000000}%
\pgfsetdash{}{0pt}%
\pgfpathmoveto{\pgfqpoint{3.109973in}{1.993549in}}%
\pgfpathlineto{\pgfqpoint{3.118727in}{1.993549in}}%
\pgfpathlineto{\pgfqpoint{3.118727in}{2.028177in}}%
\pgfpathlineto{\pgfqpoint{3.109973in}{2.028177in}}%
\pgfpathlineto{\pgfqpoint{3.109973in}{1.993549in}}%
\pgfpathclose%
\pgfusepath{fill}%
\end{pgfscope}%
\begin{pgfscope}%
\pgfpathrectangle{\pgfqpoint{0.804646in}{0.600000in}}{\pgfqpoint{2.573292in}{2.070576in}}%
\pgfusepath{clip}%
\pgfsetbuttcap%
\pgfsetmiterjoin%
\definecolor{currentfill}{rgb}{0.754268,0.565033,0.211761}%
\pgfsetfillcolor{currentfill}%
\pgfsetlinewidth{0.000000pt}%
\definecolor{currentstroke}{rgb}{0.000000,0.000000,0.000000}%
\pgfsetstrokecolor{currentstroke}%
\pgfsetstrokeopacity{0.000000}%
\pgfsetdash{}{0pt}%
\pgfpathmoveto{\pgfqpoint{3.120915in}{1.975502in}}%
\pgfpathlineto{\pgfqpoint{3.129669in}{1.975502in}}%
\pgfpathlineto{\pgfqpoint{3.129669in}{1.987262in}}%
\pgfpathlineto{\pgfqpoint{3.120915in}{1.987262in}}%
\pgfpathlineto{\pgfqpoint{3.120915in}{1.975502in}}%
\pgfpathclose%
\pgfusepath{fill}%
\end{pgfscope}%
\begin{pgfscope}%
\pgfpathrectangle{\pgfqpoint{0.804646in}{0.600000in}}{\pgfqpoint{2.573292in}{2.070576in}}%
\pgfusepath{clip}%
\pgfsetbuttcap%
\pgfsetmiterjoin%
\definecolor{currentfill}{rgb}{0.754268,0.565033,0.211761}%
\pgfsetfillcolor{currentfill}%
\pgfsetlinewidth{0.000000pt}%
\definecolor{currentstroke}{rgb}{0.000000,0.000000,0.000000}%
\pgfsetstrokecolor{currentstroke}%
\pgfsetstrokeopacity{0.000000}%
\pgfsetdash{}{0pt}%
\pgfpathmoveto{\pgfqpoint{3.131857in}{1.966033in}}%
\pgfpathlineto{\pgfqpoint{3.140610in}{1.966033in}}%
\pgfpathlineto{\pgfqpoint{3.140610in}{2.007704in}}%
\pgfpathlineto{\pgfqpoint{3.131857in}{2.007704in}}%
\pgfpathlineto{\pgfqpoint{3.131857in}{1.966033in}}%
\pgfpathclose%
\pgfusepath{fill}%
\end{pgfscope}%
\begin{pgfscope}%
\pgfpathrectangle{\pgfqpoint{0.804646in}{0.600000in}}{\pgfqpoint{2.573292in}{2.070576in}}%
\pgfusepath{clip}%
\pgfsetbuttcap%
\pgfsetmiterjoin%
\definecolor{currentfill}{rgb}{0.754268,0.565033,0.211761}%
\pgfsetfillcolor{currentfill}%
\pgfsetlinewidth{0.000000pt}%
\definecolor{currentstroke}{rgb}{0.000000,0.000000,0.000000}%
\pgfsetstrokecolor{currentstroke}%
\pgfsetstrokeopacity{0.000000}%
\pgfsetdash{}{0pt}%
\pgfpathmoveto{\pgfqpoint{3.142799in}{2.006038in}}%
\pgfpathlineto{\pgfqpoint{3.151552in}{2.006038in}}%
\pgfpathlineto{\pgfqpoint{3.151552in}{2.059492in}}%
\pgfpathlineto{\pgfqpoint{3.142799in}{2.059492in}}%
\pgfpathlineto{\pgfqpoint{3.142799in}{2.006038in}}%
\pgfpathclose%
\pgfusepath{fill}%
\end{pgfscope}%
\begin{pgfscope}%
\pgfpathrectangle{\pgfqpoint{0.804646in}{0.600000in}}{\pgfqpoint{2.573292in}{2.070576in}}%
\pgfusepath{clip}%
\pgfsetbuttcap%
\pgfsetmiterjoin%
\definecolor{currentfill}{rgb}{0.754268,0.565033,0.211761}%
\pgfsetfillcolor{currentfill}%
\pgfsetlinewidth{0.000000pt}%
\definecolor{currentstroke}{rgb}{0.000000,0.000000,0.000000}%
\pgfsetstrokecolor{currentstroke}%
\pgfsetstrokeopacity{0.000000}%
\pgfsetdash{}{0pt}%
\pgfpathmoveto{\pgfqpoint{3.153741in}{1.985216in}}%
\pgfpathlineto{\pgfqpoint{3.162494in}{1.985216in}}%
\pgfpathlineto{\pgfqpoint{3.162494in}{2.006923in}}%
\pgfpathlineto{\pgfqpoint{3.153741in}{2.006923in}}%
\pgfpathlineto{\pgfqpoint{3.153741in}{1.985216in}}%
\pgfpathclose%
\pgfusepath{fill}%
\end{pgfscope}%
\begin{pgfscope}%
\pgfpathrectangle{\pgfqpoint{0.804646in}{0.600000in}}{\pgfqpoint{2.573292in}{2.070576in}}%
\pgfusepath{clip}%
\pgfsetbuttcap%
\pgfsetmiterjoin%
\definecolor{currentfill}{rgb}{0.754268,0.565033,0.211761}%
\pgfsetfillcolor{currentfill}%
\pgfsetlinewidth{0.000000pt}%
\definecolor{currentstroke}{rgb}{0.000000,0.000000,0.000000}%
\pgfsetstrokecolor{currentstroke}%
\pgfsetstrokeopacity{0.000000}%
\pgfsetdash{}{0pt}%
\pgfpathmoveto{\pgfqpoint{3.164682in}{1.953977in}}%
\pgfpathlineto{\pgfqpoint{3.173436in}{1.953977in}}%
\pgfpathlineto{\pgfqpoint{3.173436in}{1.976799in}}%
\pgfpathlineto{\pgfqpoint{3.164682in}{1.976799in}}%
\pgfpathlineto{\pgfqpoint{3.164682in}{1.953977in}}%
\pgfpathclose%
\pgfusepath{fill}%
\end{pgfscope}%
\begin{pgfscope}%
\pgfpathrectangle{\pgfqpoint{0.804646in}{0.600000in}}{\pgfqpoint{2.573292in}{2.070576in}}%
\pgfusepath{clip}%
\pgfsetbuttcap%
\pgfsetmiterjoin%
\definecolor{currentfill}{rgb}{0.754268,0.565033,0.211761}%
\pgfsetfillcolor{currentfill}%
\pgfsetlinewidth{0.000000pt}%
\definecolor{currentstroke}{rgb}{0.000000,0.000000,0.000000}%
\pgfsetstrokecolor{currentstroke}%
\pgfsetstrokeopacity{0.000000}%
\pgfsetdash{}{0pt}%
\pgfpathmoveto{\pgfqpoint{3.175624in}{1.973189in}}%
\pgfpathlineto{\pgfqpoint{3.184378in}{1.973189in}}%
\pgfpathlineto{\pgfqpoint{3.184378in}{1.982271in}}%
\pgfpathlineto{\pgfqpoint{3.175624in}{1.982271in}}%
\pgfpathlineto{\pgfqpoint{3.175624in}{1.973189in}}%
\pgfpathclose%
\pgfusepath{fill}%
\end{pgfscope}%
\begin{pgfscope}%
\pgfpathrectangle{\pgfqpoint{0.804646in}{0.600000in}}{\pgfqpoint{2.573292in}{2.070576in}}%
\pgfusepath{clip}%
\pgfsetbuttcap%
\pgfsetmiterjoin%
\definecolor{currentfill}{rgb}{0.754268,0.565033,0.211761}%
\pgfsetfillcolor{currentfill}%
\pgfsetlinewidth{0.000000pt}%
\definecolor{currentstroke}{rgb}{0.000000,0.000000,0.000000}%
\pgfsetstrokecolor{currentstroke}%
\pgfsetstrokeopacity{0.000000}%
\pgfsetdash{}{0pt}%
\pgfpathmoveto{\pgfqpoint{3.186566in}{1.962810in}}%
\pgfpathlineto{\pgfqpoint{3.195319in}{1.962810in}}%
\pgfpathlineto{\pgfqpoint{3.195319in}{1.999728in}}%
\pgfpathlineto{\pgfqpoint{3.186566in}{1.999728in}}%
\pgfpathlineto{\pgfqpoint{3.186566in}{1.962810in}}%
\pgfpathclose%
\pgfusepath{fill}%
\end{pgfscope}%
\begin{pgfscope}%
\pgfpathrectangle{\pgfqpoint{0.804646in}{0.600000in}}{\pgfqpoint{2.573292in}{2.070576in}}%
\pgfusepath{clip}%
\pgfsetbuttcap%
\pgfsetmiterjoin%
\definecolor{currentfill}{rgb}{0.754268,0.565033,0.211761}%
\pgfsetfillcolor{currentfill}%
\pgfsetlinewidth{0.000000pt}%
\definecolor{currentstroke}{rgb}{0.000000,0.000000,0.000000}%
\pgfsetstrokecolor{currentstroke}%
\pgfsetstrokeopacity{0.000000}%
\pgfsetdash{}{0pt}%
\pgfpathmoveto{\pgfqpoint{3.197508in}{1.994396in}}%
\pgfpathlineto{\pgfqpoint{3.206261in}{1.994396in}}%
\pgfpathlineto{\pgfqpoint{3.206261in}{2.034197in}}%
\pgfpathlineto{\pgfqpoint{3.197508in}{2.034197in}}%
\pgfpathlineto{\pgfqpoint{3.197508in}{1.994396in}}%
\pgfpathclose%
\pgfusepath{fill}%
\end{pgfscope}%
\begin{pgfscope}%
\pgfpathrectangle{\pgfqpoint{0.804646in}{0.600000in}}{\pgfqpoint{2.573292in}{2.070576in}}%
\pgfusepath{clip}%
\pgfsetbuttcap%
\pgfsetmiterjoin%
\definecolor{currentfill}{rgb}{0.754268,0.565033,0.211761}%
\pgfsetfillcolor{currentfill}%
\pgfsetlinewidth{0.000000pt}%
\definecolor{currentstroke}{rgb}{0.000000,0.000000,0.000000}%
\pgfsetstrokecolor{currentstroke}%
\pgfsetstrokeopacity{0.000000}%
\pgfsetdash{}{0pt}%
\pgfpathmoveto{\pgfqpoint{3.208450in}{2.024407in}}%
\pgfpathlineto{\pgfqpoint{3.217203in}{2.024407in}}%
\pgfpathlineto{\pgfqpoint{3.217203in}{2.066838in}}%
\pgfpathlineto{\pgfqpoint{3.208450in}{2.066838in}}%
\pgfpathlineto{\pgfqpoint{3.208450in}{2.024407in}}%
\pgfpathclose%
\pgfusepath{fill}%
\end{pgfscope}%
\begin{pgfscope}%
\pgfpathrectangle{\pgfqpoint{0.804646in}{0.600000in}}{\pgfqpoint{2.573292in}{2.070576in}}%
\pgfusepath{clip}%
\pgfsetbuttcap%
\pgfsetmiterjoin%
\definecolor{currentfill}{rgb}{0.754268,0.565033,0.211761}%
\pgfsetfillcolor{currentfill}%
\pgfsetlinewidth{0.000000pt}%
\definecolor{currentstroke}{rgb}{0.000000,0.000000,0.000000}%
\pgfsetstrokecolor{currentstroke}%
\pgfsetstrokeopacity{0.000000}%
\pgfsetdash{}{0pt}%
\pgfpathmoveto{\pgfqpoint{3.219391in}{2.093302in}}%
\pgfpathlineto{\pgfqpoint{3.228145in}{2.093302in}}%
\pgfpathlineto{\pgfqpoint{3.228145in}{2.181627in}}%
\pgfpathlineto{\pgfqpoint{3.219391in}{2.181627in}}%
\pgfpathlineto{\pgfqpoint{3.219391in}{2.093302in}}%
\pgfpathclose%
\pgfusepath{fill}%
\end{pgfscope}%
\begin{pgfscope}%
\pgfpathrectangle{\pgfqpoint{0.804646in}{0.600000in}}{\pgfqpoint{2.573292in}{2.070576in}}%
\pgfusepath{clip}%
\pgfsetbuttcap%
\pgfsetmiterjoin%
\definecolor{currentfill}{rgb}{0.754268,0.565033,0.211761}%
\pgfsetfillcolor{currentfill}%
\pgfsetlinewidth{0.000000pt}%
\definecolor{currentstroke}{rgb}{0.000000,0.000000,0.000000}%
\pgfsetstrokecolor{currentstroke}%
\pgfsetstrokeopacity{0.000000}%
\pgfsetdash{}{0pt}%
\pgfpathmoveto{\pgfqpoint{3.230333in}{2.087018in}}%
\pgfpathlineto{\pgfqpoint{3.239087in}{2.087018in}}%
\pgfpathlineto{\pgfqpoint{3.239087in}{2.171684in}}%
\pgfpathlineto{\pgfqpoint{3.230333in}{2.171684in}}%
\pgfpathlineto{\pgfqpoint{3.230333in}{2.087018in}}%
\pgfpathclose%
\pgfusepath{fill}%
\end{pgfscope}%
\begin{pgfscope}%
\pgfpathrectangle{\pgfqpoint{0.804646in}{0.600000in}}{\pgfqpoint{2.573292in}{2.070576in}}%
\pgfusepath{clip}%
\pgfsetbuttcap%
\pgfsetmiterjoin%
\definecolor{currentfill}{rgb}{0.754268,0.565033,0.211761}%
\pgfsetfillcolor{currentfill}%
\pgfsetlinewidth{0.000000pt}%
\definecolor{currentstroke}{rgb}{0.000000,0.000000,0.000000}%
\pgfsetstrokecolor{currentstroke}%
\pgfsetstrokeopacity{0.000000}%
\pgfsetdash{}{0pt}%
\pgfpathmoveto{\pgfqpoint{3.241275in}{2.137110in}}%
\pgfpathlineto{\pgfqpoint{3.250028in}{2.137110in}}%
\pgfpathlineto{\pgfqpoint{3.250028in}{2.194423in}}%
\pgfpathlineto{\pgfqpoint{3.241275in}{2.194423in}}%
\pgfpathlineto{\pgfqpoint{3.241275in}{2.137110in}}%
\pgfpathclose%
\pgfusepath{fill}%
\end{pgfscope}%
\begin{pgfscope}%
\pgfpathrectangle{\pgfqpoint{0.804646in}{0.600000in}}{\pgfqpoint{2.573292in}{2.070576in}}%
\pgfusepath{clip}%
\pgfsetbuttcap%
\pgfsetmiterjoin%
\definecolor{currentfill}{rgb}{0.754268,0.565033,0.211761}%
\pgfsetfillcolor{currentfill}%
\pgfsetlinewidth{0.000000pt}%
\definecolor{currentstroke}{rgb}{0.000000,0.000000,0.000000}%
\pgfsetstrokecolor{currentstroke}%
\pgfsetstrokeopacity{0.000000}%
\pgfsetdash{}{0pt}%
\pgfpathmoveto{\pgfqpoint{3.252217in}{2.168003in}}%
\pgfpathlineto{\pgfqpoint{3.260970in}{2.168003in}}%
\pgfpathlineto{\pgfqpoint{3.260970in}{2.237434in}}%
\pgfpathlineto{\pgfqpoint{3.252217in}{2.237434in}}%
\pgfpathlineto{\pgfqpoint{3.252217in}{2.168003in}}%
\pgfpathclose%
\pgfusepath{fill}%
\end{pgfscope}%
\begin{pgfscope}%
\pgfsetbuttcap%
\pgfsetroundjoin%
\definecolor{currentfill}{rgb}{0.000000,0.000000,0.000000}%
\pgfsetfillcolor{currentfill}%
\pgfsetlinewidth{0.803000pt}%
\definecolor{currentstroke}{rgb}{0.000000,0.000000,0.000000}%
\pgfsetstrokecolor{currentstroke}%
\pgfsetdash{}{0pt}%
\pgfsys@defobject{currentmarker}{\pgfqpoint{0.000000in}{-0.048611in}}{\pgfqpoint{0.000000in}{0.000000in}}{%
\pgfpathmoveto{\pgfqpoint{0.000000in}{0.000000in}}%
\pgfpathlineto{\pgfqpoint{0.000000in}{-0.048611in}}%
\pgfusepath{stroke,fill}%
}%
\begin{pgfscope}%
\pgfsys@transformshift{1.451197in}{0.600000in}%
\pgfsys@useobject{currentmarker}{}%
\end{pgfscope}%
\end{pgfscope}%
\begin{pgfscope}%
\definecolor{textcolor}{rgb}{0.000000,0.000000,0.000000}%
\pgfsetstrokecolor{textcolor}%
\pgfsetfillcolor{textcolor}%
\pgftext[x=1.451197in,y=0.502778in,,top]{\color{textcolor}{\rmfamily\fontsize{10.000000}{12.000000}\selectfont\catcode`\^=\active\def^{\ifmmode\sp\else\^{}\fi}\catcode`\%=\active\def%{\%}1978}}%
\end{pgfscope}%
\begin{pgfscope}%
\pgfsetbuttcap%
\pgfsetroundjoin%
\definecolor{currentfill}{rgb}{0.000000,0.000000,0.000000}%
\pgfsetfillcolor{currentfill}%
\pgfsetlinewidth{0.803000pt}%
\definecolor{currentstroke}{rgb}{0.000000,0.000000,0.000000}%
\pgfsetstrokecolor{currentstroke}%
\pgfsetdash{}{0pt}%
\pgfsys@defobject{currentmarker}{\pgfqpoint{0.000000in}{-0.048611in}}{\pgfqpoint{0.000000in}{0.000000in}}{%
\pgfpathmoveto{\pgfqpoint{0.000000in}{0.000000in}}%
\pgfpathlineto{\pgfqpoint{0.000000in}{-0.048611in}}%
\pgfusepath{stroke,fill}%
}%
\begin{pgfscope}%
\pgfsys@transformshift{1.998287in}{0.600000in}%
\pgfsys@useobject{currentmarker}{}%
\end{pgfscope}%
\end{pgfscope}%
\begin{pgfscope}%
\definecolor{textcolor}{rgb}{0.000000,0.000000,0.000000}%
\pgfsetstrokecolor{textcolor}%
\pgfsetfillcolor{textcolor}%
\pgftext[x=1.998287in,y=0.502778in,,top]{\color{textcolor}{\rmfamily\fontsize{10.000000}{12.000000}\selectfont\catcode`\^=\active\def^{\ifmmode\sp\else\^{}\fi}\catcode`\%=\active\def%{\%}1991}}%
\end{pgfscope}%
\begin{pgfscope}%
\pgfsetbuttcap%
\pgfsetroundjoin%
\definecolor{currentfill}{rgb}{0.000000,0.000000,0.000000}%
\pgfsetfillcolor{currentfill}%
\pgfsetlinewidth{0.803000pt}%
\definecolor{currentstroke}{rgb}{0.000000,0.000000,0.000000}%
\pgfsetstrokecolor{currentstroke}%
\pgfsetdash{}{0pt}%
\pgfsys@defobject{currentmarker}{\pgfqpoint{0.000000in}{-0.048611in}}{\pgfqpoint{0.000000in}{0.000000in}}{%
\pgfpathmoveto{\pgfqpoint{0.000000in}{0.000000in}}%
\pgfpathlineto{\pgfqpoint{0.000000in}{-0.048611in}}%
\pgfusepath{stroke,fill}%
}%
\begin{pgfscope}%
\pgfsys@transformshift{2.545377in}{0.600000in}%
\pgfsys@useobject{currentmarker}{}%
\end{pgfscope}%
\end{pgfscope}%
\begin{pgfscope}%
\definecolor{textcolor}{rgb}{0.000000,0.000000,0.000000}%
\pgfsetstrokecolor{textcolor}%
\pgfsetfillcolor{textcolor}%
\pgftext[x=2.545377in,y=0.502778in,,top]{\color{textcolor}{\rmfamily\fontsize{10.000000}{12.000000}\selectfont\catcode`\^=\active\def^{\ifmmode\sp\else\^{}\fi}\catcode`\%=\active\def%{\%}2003}}%
\end{pgfscope}%
\begin{pgfscope}%
\pgfsetbuttcap%
\pgfsetroundjoin%
\definecolor{currentfill}{rgb}{0.000000,0.000000,0.000000}%
\pgfsetfillcolor{currentfill}%
\pgfsetlinewidth{0.803000pt}%
\definecolor{currentstroke}{rgb}{0.000000,0.000000,0.000000}%
\pgfsetstrokecolor{currentstroke}%
\pgfsetdash{}{0pt}%
\pgfsys@defobject{currentmarker}{\pgfqpoint{0.000000in}{-0.048611in}}{\pgfqpoint{0.000000in}{0.000000in}}{%
\pgfpathmoveto{\pgfqpoint{0.000000in}{0.000000in}}%
\pgfpathlineto{\pgfqpoint{0.000000in}{-0.048611in}}%
\pgfusepath{stroke,fill}%
}%
\begin{pgfscope}%
\pgfsys@transformshift{3.092467in}{0.600000in}%
\pgfsys@useobject{currentmarker}{}%
\end{pgfscope}%
\end{pgfscope}%
\begin{pgfscope}%
\definecolor{textcolor}{rgb}{0.000000,0.000000,0.000000}%
\pgfsetstrokecolor{textcolor}%
\pgfsetfillcolor{textcolor}%
\pgftext[x=3.092467in,y=0.502778in,,top]{\color{textcolor}{\rmfamily\fontsize{10.000000}{12.000000}\selectfont\catcode`\^=\active\def^{\ifmmode\sp\else\^{}\fi}\catcode`\%=\active\def%{\%}2016}}%
\end{pgfscope}%
\begin{pgfscope}%
\pgfsetbuttcap%
\pgfsetroundjoin%
\definecolor{currentfill}{rgb}{0.000000,0.000000,0.000000}%
\pgfsetfillcolor{currentfill}%
\pgfsetlinewidth{0.803000pt}%
\definecolor{currentstroke}{rgb}{0.000000,0.000000,0.000000}%
\pgfsetstrokecolor{currentstroke}%
\pgfsetdash{}{0pt}%
\pgfsys@defobject{currentmarker}{\pgfqpoint{-0.048611in}{0.000000in}}{\pgfqpoint{-0.000000in}{0.000000in}}{%
\pgfpathmoveto{\pgfqpoint{-0.000000in}{0.000000in}}%
\pgfpathlineto{\pgfqpoint{-0.048611in}{0.000000in}}%
\pgfusepath{stroke,fill}%
}%
\begin{pgfscope}%
\pgfsys@transformshift{0.804646in}{1.086242in}%
\pgfsys@useobject{currentmarker}{}%
\end{pgfscope}%
\end{pgfscope}%
\begin{pgfscope}%
\definecolor{textcolor}{rgb}{0.000000,0.000000,0.000000}%
\pgfsetstrokecolor{textcolor}%
\pgfsetfillcolor{textcolor}%
\pgftext[x=0.529954in, y=1.033480in, left, base]{\color{textcolor}{\rmfamily\fontsize{10.000000}{12.000000}\selectfont\catcode`\^=\active\def^{\ifmmode\sp\else\^{}\fi}\catcode`\%=\active\def%{\%}$\mathdefault{\ensuremath{-}5}$}}%
\end{pgfscope}%
\begin{pgfscope}%
\pgfsetbuttcap%
\pgfsetroundjoin%
\definecolor{currentfill}{rgb}{0.000000,0.000000,0.000000}%
\pgfsetfillcolor{currentfill}%
\pgfsetlinewidth{0.803000pt}%
\definecolor{currentstroke}{rgb}{0.000000,0.000000,0.000000}%
\pgfsetstrokecolor{currentstroke}%
\pgfsetdash{}{0pt}%
\pgfsys@defobject{currentmarker}{\pgfqpoint{-0.048611in}{0.000000in}}{\pgfqpoint{-0.000000in}{0.000000in}}{%
\pgfpathmoveto{\pgfqpoint{-0.000000in}{0.000000in}}%
\pgfpathlineto{\pgfqpoint{-0.048611in}{0.000000in}}%
\pgfusepath{stroke,fill}%
}%
\begin{pgfscope}%
\pgfsys@transformshift{0.804646in}{1.613090in}%
\pgfsys@useobject{currentmarker}{}%
\end{pgfscope}%
\end{pgfscope}%
\begin{pgfscope}%
\definecolor{textcolor}{rgb}{0.000000,0.000000,0.000000}%
\pgfsetstrokecolor{textcolor}%
\pgfsetfillcolor{textcolor}%
\pgftext[x=0.637979in, y=1.560328in, left, base]{\color{textcolor}{\rmfamily\fontsize{10.000000}{12.000000}\selectfont\catcode`\^=\active\def^{\ifmmode\sp\else\^{}\fi}\catcode`\%=\active\def%{\%}$\mathdefault{0}$}}%
\end{pgfscope}%
\begin{pgfscope}%
\pgfsetbuttcap%
\pgfsetroundjoin%
\definecolor{currentfill}{rgb}{0.000000,0.000000,0.000000}%
\pgfsetfillcolor{currentfill}%
\pgfsetlinewidth{0.803000pt}%
\definecolor{currentstroke}{rgb}{0.000000,0.000000,0.000000}%
\pgfsetstrokecolor{currentstroke}%
\pgfsetdash{}{0pt}%
\pgfsys@defobject{currentmarker}{\pgfqpoint{-0.048611in}{0.000000in}}{\pgfqpoint{-0.000000in}{0.000000in}}{%
\pgfpathmoveto{\pgfqpoint{-0.000000in}{0.000000in}}%
\pgfpathlineto{\pgfqpoint{-0.048611in}{0.000000in}}%
\pgfusepath{stroke,fill}%
}%
\begin{pgfscope}%
\pgfsys@transformshift{0.804646in}{2.139938in}%
\pgfsys@useobject{currentmarker}{}%
\end{pgfscope}%
\end{pgfscope}%
\begin{pgfscope}%
\definecolor{textcolor}{rgb}{0.000000,0.000000,0.000000}%
\pgfsetstrokecolor{textcolor}%
\pgfsetfillcolor{textcolor}%
\pgftext[x=0.637979in, y=2.087176in, left, base]{\color{textcolor}{\rmfamily\fontsize{10.000000}{12.000000}\selectfont\catcode`\^=\active\def^{\ifmmode\sp\else\^{}\fi}\catcode`\%=\active\def%{\%}$\mathdefault{5}$}}%
\end{pgfscope}%
\begin{pgfscope}%
\pgfsetbuttcap%
\pgfsetroundjoin%
\definecolor{currentfill}{rgb}{0.000000,0.000000,0.000000}%
\pgfsetfillcolor{currentfill}%
\pgfsetlinewidth{0.803000pt}%
\definecolor{currentstroke}{rgb}{0.000000,0.000000,0.000000}%
\pgfsetstrokecolor{currentstroke}%
\pgfsetdash{}{0pt}%
\pgfsys@defobject{currentmarker}{\pgfqpoint{-0.048611in}{0.000000in}}{\pgfqpoint{-0.000000in}{0.000000in}}{%
\pgfpathmoveto{\pgfqpoint{-0.000000in}{0.000000in}}%
\pgfpathlineto{\pgfqpoint{-0.048611in}{0.000000in}}%
\pgfusepath{stroke,fill}%
}%
\begin{pgfscope}%
\pgfsys@transformshift{0.804646in}{2.666786in}%
\pgfsys@useobject{currentmarker}{}%
\end{pgfscope}%
\end{pgfscope}%
\begin{pgfscope}%
\definecolor{textcolor}{rgb}{0.000000,0.000000,0.000000}%
\pgfsetstrokecolor{textcolor}%
\pgfsetfillcolor{textcolor}%
\pgftext[x=0.568534in, y=2.614024in, left, base]{\color{textcolor}{\rmfamily\fontsize{10.000000}{12.000000}\selectfont\catcode`\^=\active\def^{\ifmmode\sp\else\^{}\fi}\catcode`\%=\active\def%{\%}$\mathdefault{10}$}}%
\end{pgfscope}%
\begin{pgfscope}%
\pgfpathrectangle{\pgfqpoint{0.804646in}{0.600000in}}{\pgfqpoint{2.573292in}{2.070576in}}%
\pgfusepath{clip}%
\pgfsetrectcap%
\pgfsetroundjoin%
\pgfsetlinewidth{1.003750pt}%
\definecolor{currentstroke}{rgb}{0.000000,0.000000,0.000000}%
\pgfsetstrokecolor{currentstroke}%
\pgfsetdash{}{0pt}%
\pgfpathmoveto{\pgfqpoint{0.804646in}{1.613090in}}%
\pgfpathlineto{\pgfqpoint{3.377938in}{1.613090in}}%
\pgfusepath{stroke}%
\end{pgfscope}%
\begin{pgfscope}%
\pgfpathrectangle{\pgfqpoint{0.804646in}{0.600000in}}{\pgfqpoint{2.573292in}{2.070576in}}%
\pgfusepath{clip}%
\pgfsetrectcap%
\pgfsetroundjoin%
\pgfsetlinewidth{1.505625pt}%
\definecolor{currentstroke}{rgb}{0.000000,0.000000,0.000000}%
\pgfsetstrokecolor{currentstroke}%
\pgfsetdash{}{0pt}%
\pgfpathmoveto{\pgfqpoint{0.925990in}{1.539220in}}%
\pgfpathlineto{\pgfqpoint{0.936932in}{1.536491in}}%
\pgfpathlineto{\pgfqpoint{0.947874in}{1.536250in}}%
\pgfpathlineto{\pgfqpoint{0.958816in}{1.540095in}}%
\pgfpathlineto{\pgfqpoint{0.969758in}{1.522122in}}%
\pgfpathlineto{\pgfqpoint{0.980699in}{1.499423in}}%
\pgfpathlineto{\pgfqpoint{0.991641in}{1.544707in}}%
\pgfpathlineto{\pgfqpoint{1.002583in}{1.561586in}}%
\pgfpathlineto{\pgfqpoint{1.013525in}{1.606387in}}%
\pgfpathlineto{\pgfqpoint{1.024467in}{1.586818in}}%
\pgfpathlineto{\pgfqpoint{1.035408in}{1.594167in}}%
\pgfpathlineto{\pgfqpoint{1.046350in}{1.587786in}}%
\pgfpathlineto{\pgfqpoint{1.057292in}{1.568703in}}%
\pgfpathlineto{\pgfqpoint{1.068234in}{1.580922in}}%
\pgfpathlineto{\pgfqpoint{1.079176in}{1.566163in}}%
\pgfpathlineto{\pgfqpoint{1.090117in}{1.538649in}}%
\pgfpathlineto{\pgfqpoint{1.101059in}{1.545677in}}%
\pgfpathlineto{\pgfqpoint{1.112001in}{1.527111in}}%
\pgfpathlineto{\pgfqpoint{1.122943in}{1.535140in}}%
\pgfpathlineto{\pgfqpoint{1.133885in}{1.528086in}}%
\pgfpathlineto{\pgfqpoint{1.144826in}{1.517941in}}%
\pgfpathlineto{\pgfqpoint{1.155768in}{1.536307in}}%
\pgfpathlineto{\pgfqpoint{1.166710in}{1.532711in}}%
\pgfpathlineto{\pgfqpoint{1.177652in}{1.570818in}}%
\pgfpathlineto{\pgfqpoint{1.188594in}{1.585361in}}%
\pgfpathlineto{\pgfqpoint{1.199535in}{1.614772in}}%
\pgfpathlineto{\pgfqpoint{1.210477in}{1.635057in}}%
\pgfpathlineto{\pgfqpoint{1.221419in}{1.636381in}}%
\pgfpathlineto{\pgfqpoint{1.232361in}{1.622665in}}%
\pgfpathlineto{\pgfqpoint{1.243303in}{1.592845in}}%
\pgfpathlineto{\pgfqpoint{1.254244in}{1.572947in}}%
\pgfpathlineto{\pgfqpoint{1.265186in}{1.569139in}}%
\pgfpathlineto{\pgfqpoint{1.276128in}{1.549854in}}%
\pgfpathlineto{\pgfqpoint{1.287070in}{1.511286in}}%
\pgfpathlineto{\pgfqpoint{1.298012in}{1.476993in}}%
\pgfpathlineto{\pgfqpoint{1.308953in}{1.470534in}}%
\pgfpathlineto{\pgfqpoint{1.319895in}{1.518469in}}%
\pgfpathlineto{\pgfqpoint{1.330837in}{1.506317in}}%
\pgfpathlineto{\pgfqpoint{1.341779in}{1.570832in}}%
\pgfpathlineto{\pgfqpoint{1.352721in}{1.568245in}}%
\pgfpathlineto{\pgfqpoint{1.363662in}{1.569976in}}%
\pgfpathlineto{\pgfqpoint{1.374604in}{1.563379in}}%
\pgfpathlineto{\pgfqpoint{1.385546in}{1.585129in}}%
\pgfpathlineto{\pgfqpoint{1.396488in}{1.597103in}}%
\pgfpathlineto{\pgfqpoint{1.407430in}{1.613431in}}%
\pgfpathlineto{\pgfqpoint{1.418371in}{1.597950in}}%
\pgfpathlineto{\pgfqpoint{1.429313in}{1.575475in}}%
\pgfpathlineto{\pgfqpoint{1.440255in}{1.634481in}}%
\pgfpathlineto{\pgfqpoint{1.451197in}{1.625489in}}%
\pgfpathlineto{\pgfqpoint{1.462139in}{1.622971in}}%
\pgfpathlineto{\pgfqpoint{1.473080in}{1.617317in}}%
\pgfpathlineto{\pgfqpoint{1.484022in}{1.595313in}}%
\pgfpathlineto{\pgfqpoint{1.494964in}{1.597108in}}%
\pgfpathlineto{\pgfqpoint{1.505906in}{1.616781in}}%
\pgfpathlineto{\pgfqpoint{1.516848in}{1.628108in}}%
\pgfpathlineto{\pgfqpoint{1.527789in}{1.541825in}}%
\pgfpathlineto{\pgfqpoint{1.538731in}{1.541053in}}%
\pgfpathlineto{\pgfqpoint{1.549673in}{1.554909in}}%
\pgfpathlineto{\pgfqpoint{1.560615in}{1.542281in}}%
\pgfpathlineto{\pgfqpoint{1.571557in}{1.518936in}}%
\pgfpathlineto{\pgfqpoint{1.582498in}{1.500612in}}%
\pgfpathlineto{\pgfqpoint{1.593440in}{1.463298in}}%
\pgfpathlineto{\pgfqpoint{1.604382in}{1.451632in}}%
\pgfpathlineto{\pgfqpoint{1.615324in}{1.436172in}}%
\pgfpathlineto{\pgfqpoint{1.626266in}{1.437575in}}%
\pgfpathlineto{\pgfqpoint{1.648149in}{1.476175in}}%
\pgfpathlineto{\pgfqpoint{1.659091in}{1.500238in}}%
\pgfpathlineto{\pgfqpoint{1.670033in}{1.542763in}}%
\pgfpathlineto{\pgfqpoint{1.680974in}{1.562297in}}%
\pgfpathlineto{\pgfqpoint{1.691916in}{1.584611in}}%
\pgfpathlineto{\pgfqpoint{1.702858in}{1.584096in}}%
\pgfpathlineto{\pgfqpoint{1.713800in}{1.580213in}}%
\pgfpathlineto{\pgfqpoint{1.724742in}{1.578078in}}%
\pgfpathlineto{\pgfqpoint{1.735683in}{1.622396in}}%
\pgfpathlineto{\pgfqpoint{1.746625in}{1.627018in}}%
\pgfpathlineto{\pgfqpoint{1.757567in}{1.645668in}}%
\pgfpathlineto{\pgfqpoint{1.768509in}{1.648473in}}%
\pgfpathlineto{\pgfqpoint{1.779451in}{1.683058in}}%
\pgfpathlineto{\pgfqpoint{1.790392in}{1.661384in}}%
\pgfpathlineto{\pgfqpoint{1.812276in}{1.684342in}}%
\pgfpathlineto{\pgfqpoint{1.823218in}{1.650481in}}%
\pgfpathlineto{\pgfqpoint{1.834160in}{1.704068in}}%
\pgfpathlineto{\pgfqpoint{1.845101in}{1.711908in}}%
\pgfpathlineto{\pgfqpoint{1.856043in}{1.703677in}}%
\pgfpathlineto{\pgfqpoint{1.866985in}{1.729211in}}%
\pgfpathlineto{\pgfqpoint{1.877927in}{1.724616in}}%
\pgfpathlineto{\pgfqpoint{1.899810in}{1.747831in}}%
\pgfpathlineto{\pgfqpoint{1.910752in}{1.754396in}}%
\pgfpathlineto{\pgfqpoint{1.921694in}{1.751356in}}%
\pgfpathlineto{\pgfqpoint{1.932636in}{1.742795in}}%
\pgfpathlineto{\pgfqpoint{1.943578in}{1.737343in}}%
\pgfpathlineto{\pgfqpoint{1.954519in}{1.758407in}}%
\pgfpathlineto{\pgfqpoint{1.965461in}{1.735228in}}%
\pgfpathlineto{\pgfqpoint{1.987345in}{1.715822in}}%
\pgfpathlineto{\pgfqpoint{1.998287in}{1.647342in}}%
\pgfpathlineto{\pgfqpoint{2.009228in}{1.650936in}}%
\pgfpathlineto{\pgfqpoint{2.020170in}{1.646648in}}%
\pgfpathlineto{\pgfqpoint{2.031112in}{1.614011in}}%
\pgfpathlineto{\pgfqpoint{2.042054in}{1.653396in}}%
\pgfpathlineto{\pgfqpoint{2.052996in}{1.645182in}}%
\pgfpathlineto{\pgfqpoint{2.063937in}{1.655064in}}%
\pgfpathlineto{\pgfqpoint{2.074879in}{1.662879in}}%
\pgfpathlineto{\pgfqpoint{2.085821in}{1.660142in}}%
\pgfpathlineto{\pgfqpoint{2.096763in}{1.654585in}}%
\pgfpathlineto{\pgfqpoint{2.107705in}{1.661023in}}%
\pgfpathlineto{\pgfqpoint{2.118646in}{1.661084in}}%
\pgfpathlineto{\pgfqpoint{2.129588in}{1.657302in}}%
\pgfpathlineto{\pgfqpoint{2.140530in}{1.672109in}}%
\pgfpathlineto{\pgfqpoint{2.151472in}{1.662440in}}%
\pgfpathlineto{\pgfqpoint{2.162414in}{1.676641in}}%
\pgfpathlineto{\pgfqpoint{2.173355in}{1.661919in}}%
\pgfpathlineto{\pgfqpoint{2.184297in}{1.649153in}}%
\pgfpathlineto{\pgfqpoint{2.195239in}{1.656710in}}%
\pgfpathlineto{\pgfqpoint{2.206181in}{1.644293in}}%
\pgfpathlineto{\pgfqpoint{2.217123in}{1.648403in}}%
\pgfpathlineto{\pgfqpoint{2.228064in}{1.678918in}}%
\pgfpathlineto{\pgfqpoint{2.239006in}{1.670076in}}%
\pgfpathlineto{\pgfqpoint{2.249948in}{1.671258in}}%
\pgfpathlineto{\pgfqpoint{2.260890in}{1.674776in}}%
\pgfpathlineto{\pgfqpoint{2.271832in}{1.671634in}}%
\pgfpathlineto{\pgfqpoint{2.293715in}{1.690786in}}%
\pgfpathlineto{\pgfqpoint{2.304657in}{1.690092in}}%
\pgfpathlineto{\pgfqpoint{2.315599in}{1.708126in}}%
\pgfpathlineto{\pgfqpoint{2.326541in}{1.720931in}}%
\pgfpathlineto{\pgfqpoint{2.337482in}{1.739313in}}%
\pgfpathlineto{\pgfqpoint{2.348424in}{1.742310in}}%
\pgfpathlineto{\pgfqpoint{2.359366in}{1.765634in}}%
\pgfpathlineto{\pgfqpoint{2.370308in}{1.771246in}}%
\pgfpathlineto{\pgfqpoint{2.392191in}{1.800495in}}%
\pgfpathlineto{\pgfqpoint{2.403133in}{1.817943in}}%
\pgfpathlineto{\pgfqpoint{2.414075in}{1.816593in}}%
\pgfpathlineto{\pgfqpoint{2.425017in}{1.829387in}}%
\pgfpathlineto{\pgfqpoint{2.435959in}{1.827180in}}%
\pgfpathlineto{\pgfqpoint{2.446900in}{1.797661in}}%
\pgfpathlineto{\pgfqpoint{2.457842in}{1.795760in}}%
\pgfpathlineto{\pgfqpoint{2.468784in}{1.814514in}}%
\pgfpathlineto{\pgfqpoint{2.479726in}{1.775789in}}%
\pgfpathlineto{\pgfqpoint{2.490668in}{1.793446in}}%
\pgfpathlineto{\pgfqpoint{2.501609in}{1.787072in}}%
\pgfpathlineto{\pgfqpoint{2.512551in}{1.767969in}}%
\pgfpathlineto{\pgfqpoint{2.523493in}{1.772716in}}%
\pgfpathlineto{\pgfqpoint{2.534435in}{1.776156in}}%
\pgfpathlineto{\pgfqpoint{2.545377in}{1.782592in}}%
\pgfpathlineto{\pgfqpoint{2.556318in}{1.783055in}}%
\pgfpathlineto{\pgfqpoint{2.567260in}{1.791327in}}%
\pgfpathlineto{\pgfqpoint{2.578202in}{1.783331in}}%
\pgfpathlineto{\pgfqpoint{2.589144in}{1.788099in}}%
\pgfpathlineto{\pgfqpoint{2.600086in}{1.797635in}}%
\pgfpathlineto{\pgfqpoint{2.611027in}{1.787993in}}%
\pgfpathlineto{\pgfqpoint{2.621969in}{1.805152in}}%
\pgfpathlineto{\pgfqpoint{2.632911in}{1.813456in}}%
\pgfpathlineto{\pgfqpoint{2.643853in}{1.795948in}}%
\pgfpathlineto{\pgfqpoint{2.654795in}{1.796258in}}%
\pgfpathlineto{\pgfqpoint{2.665736in}{1.788182in}}%
\pgfpathlineto{\pgfqpoint{2.676678in}{1.792476in}}%
\pgfpathlineto{\pgfqpoint{2.687620in}{1.775936in}}%
\pgfpathlineto{\pgfqpoint{2.698562in}{1.771248in}}%
\pgfpathlineto{\pgfqpoint{2.709504in}{1.762040in}}%
\pgfpathlineto{\pgfqpoint{2.720445in}{1.756272in}}%
\pgfpathlineto{\pgfqpoint{2.731387in}{1.759862in}}%
\pgfpathlineto{\pgfqpoint{2.742329in}{1.760164in}}%
\pgfpathlineto{\pgfqpoint{2.753271in}{1.747131in}}%
\pgfpathlineto{\pgfqpoint{2.764213in}{1.736887in}}%
\pgfpathlineto{\pgfqpoint{2.786096in}{1.603541in}}%
\pgfpathlineto{\pgfqpoint{2.797038in}{1.568513in}}%
\pgfpathlineto{\pgfqpoint{2.807980in}{1.575976in}}%
\pgfpathlineto{\pgfqpoint{2.818922in}{1.571121in}}%
\pgfpathlineto{\pgfqpoint{2.829863in}{1.564923in}}%
\pgfpathlineto{\pgfqpoint{2.840805in}{1.565868in}}%
\pgfpathlineto{\pgfqpoint{2.862689in}{1.555460in}}%
\pgfpathlineto{\pgfqpoint{2.873631in}{1.555136in}}%
\pgfpathlineto{\pgfqpoint{2.884572in}{1.553439in}}%
\pgfpathlineto{\pgfqpoint{2.895514in}{1.537775in}}%
\pgfpathlineto{\pgfqpoint{2.906456in}{1.532200in}}%
\pgfpathlineto{\pgfqpoint{2.917398in}{1.520638in}}%
\pgfpathlineto{\pgfqpoint{2.928340in}{1.520410in}}%
\pgfpathlineto{\pgfqpoint{2.939281in}{1.486290in}}%
\pgfpathlineto{\pgfqpoint{2.950223in}{1.481947in}}%
\pgfpathlineto{\pgfqpoint{2.961165in}{1.479850in}}%
\pgfpathlineto{\pgfqpoint{2.972107in}{1.462654in}}%
\pgfpathlineto{\pgfqpoint{2.983049in}{1.451791in}}%
\pgfpathlineto{\pgfqpoint{2.993990in}{1.446811in}}%
\pgfpathlineto{\pgfqpoint{3.004932in}{1.436216in}}%
\pgfpathlineto{\pgfqpoint{3.015874in}{1.444078in}}%
\pgfpathlineto{\pgfqpoint{3.026816in}{1.445443in}}%
\pgfpathlineto{\pgfqpoint{3.037758in}{1.457932in}}%
\pgfpathlineto{\pgfqpoint{3.048699in}{1.440832in}}%
\pgfpathlineto{\pgfqpoint{3.059641in}{1.438110in}}%
\pgfpathlineto{\pgfqpoint{3.070583in}{1.445135in}}%
\pgfpathlineto{\pgfqpoint{3.081525in}{1.431634in}}%
\pgfpathlineto{\pgfqpoint{3.092467in}{1.432476in}}%
\pgfpathlineto{\pgfqpoint{3.103408in}{1.424067in}}%
\pgfpathlineto{\pgfqpoint{3.114350in}{1.428230in}}%
\pgfpathlineto{\pgfqpoint{3.125292in}{1.418661in}}%
\pgfpathlineto{\pgfqpoint{3.136234in}{1.426823in}}%
\pgfpathlineto{\pgfqpoint{3.147176in}{1.423233in}}%
\pgfpathlineto{\pgfqpoint{3.158117in}{1.409809in}}%
\pgfpathlineto{\pgfqpoint{3.169059in}{1.411149in}}%
\pgfpathlineto{\pgfqpoint{3.180001in}{1.426441in}}%
\pgfpathlineto{\pgfqpoint{3.190943in}{1.416457in}}%
\pgfpathlineto{\pgfqpoint{3.201885in}{1.412845in}}%
\pgfpathlineto{\pgfqpoint{3.212826in}{1.403625in}}%
\pgfpathlineto{\pgfqpoint{3.223768in}{1.378823in}}%
\pgfpathlineto{\pgfqpoint{3.234710in}{1.386593in}}%
\pgfpathlineto{\pgfqpoint{3.245652in}{1.398251in}}%
\pgfpathlineto{\pgfqpoint{3.256594in}{1.393857in}}%
\pgfpathlineto{\pgfqpoint{3.256594in}{1.393857in}}%
\pgfusepath{stroke}%
\end{pgfscope}%
\begin{pgfscope}%
\pgfsetrectcap%
\pgfsetmiterjoin%
\pgfsetlinewidth{0.803000pt}%
\definecolor{currentstroke}{rgb}{0.000000,0.000000,0.000000}%
\pgfsetstrokecolor{currentstroke}%
\pgfsetdash{}{0pt}%
\pgfpathmoveto{\pgfqpoint{0.804646in}{0.600000in}}%
\pgfpathlineto{\pgfqpoint{0.804646in}{2.670576in}}%
\pgfusepath{stroke}%
\end{pgfscope}%
\begin{pgfscope}%
\pgfsetrectcap%
\pgfsetmiterjoin%
\pgfsetlinewidth{0.803000pt}%
\definecolor{currentstroke}{rgb}{0.000000,0.000000,0.000000}%
\pgfsetstrokecolor{currentstroke}%
\pgfsetdash{}{0pt}%
\pgfpathmoveto{\pgfqpoint{3.377938in}{0.600000in}}%
\pgfpathlineto{\pgfqpoint{3.377938in}{2.670576in}}%
\pgfusepath{stroke}%
\end{pgfscope}%
\begin{pgfscope}%
\pgfsetrectcap%
\pgfsetmiterjoin%
\pgfsetlinewidth{0.803000pt}%
\definecolor{currentstroke}{rgb}{0.000000,0.000000,0.000000}%
\pgfsetstrokecolor{currentstroke}%
\pgfsetdash{}{0pt}%
\pgfpathmoveto{\pgfqpoint{0.804646in}{0.600000in}}%
\pgfpathlineto{\pgfqpoint{3.377938in}{0.600000in}}%
\pgfusepath{stroke}%
\end{pgfscope}%
\begin{pgfscope}%
\pgfsetrectcap%
\pgfsetmiterjoin%
\pgfsetlinewidth{0.803000pt}%
\definecolor{currentstroke}{rgb}{0.000000,0.000000,0.000000}%
\pgfsetstrokecolor{currentstroke}%
\pgfsetdash{}{0pt}%
\pgfpathmoveto{\pgfqpoint{0.804646in}{2.670576in}}%
\pgfpathlineto{\pgfqpoint{3.377938in}{2.670576in}}%
\pgfusepath{stroke}%
\end{pgfscope}%
\begin{pgfscope}%
\definecolor{textcolor}{rgb}{0.000000,0.000000,0.000000}%
\pgfsetstrokecolor{textcolor}%
\pgfsetfillcolor{textcolor}%
\pgftext[x=2.091292in,y=2.753910in,,base]{\color{textcolor}{\rmfamily\fontsize{10.000000}{12.000000}\selectfont\catcode`\^=\active\def^{\ifmmode\sp\else\^{}\fi}\catcode`\%=\active\def%{\%}Consumption}}%
\end{pgfscope}%
\begin{pgfscope}%
\pgfsetbuttcap%
\pgfsetmiterjoin%
\definecolor{currentfill}{rgb}{1.000000,1.000000,1.000000}%
\pgfsetfillcolor{currentfill}%
\pgfsetlinewidth{0.000000pt}%
\definecolor{currentstroke}{rgb}{0.000000,0.000000,0.000000}%
\pgfsetstrokecolor{currentstroke}%
\pgfsetstrokeopacity{0.000000}%
\pgfsetdash{}{0pt}%
\pgfpathmoveto{\pgfqpoint{3.776708in}{0.600000in}}%
\pgfpathlineto{\pgfqpoint{6.350000in}{0.600000in}}%
\pgfpathlineto{\pgfqpoint{6.350000in}{2.670576in}}%
\pgfpathlineto{\pgfqpoint{3.776708in}{2.670576in}}%
\pgfpathlineto{\pgfqpoint{3.776708in}{0.600000in}}%
\pgfpathclose%
\pgfusepath{fill}%
\end{pgfscope}%
\begin{pgfscope}%
\pgfpathrectangle{\pgfqpoint{3.776708in}{0.600000in}}{\pgfqpoint{2.573292in}{2.070576in}}%
\pgfusepath{clip}%
\pgfsetbuttcap%
\pgfsetmiterjoin%
\definecolor{currentfill}{rgb}{0.066899,0.263188,0.377594}%
\pgfsetfillcolor{currentfill}%
\pgfsetlinewidth{0.000000pt}%
\definecolor{currentstroke}{rgb}{0.000000,0.000000,0.000000}%
\pgfsetstrokecolor{currentstroke}%
\pgfsetstrokeopacity{0.000000}%
\pgfsetdash{}{0pt}%
\pgfpathmoveto{\pgfqpoint{3.893676in}{1.609196in}}%
\pgfpathlineto{\pgfqpoint{3.902429in}{1.609196in}}%
\pgfpathlineto{\pgfqpoint{3.902429in}{1.614315in}}%
\pgfpathlineto{\pgfqpoint{3.893676in}{1.614315in}}%
\pgfpathlineto{\pgfqpoint{3.893676in}{1.609196in}}%
\pgfpathclose%
\pgfusepath{fill}%
\end{pgfscope}%
\begin{pgfscope}%
\pgfpathrectangle{\pgfqpoint{3.776708in}{0.600000in}}{\pgfqpoint{2.573292in}{2.070576in}}%
\pgfusepath{clip}%
\pgfsetbuttcap%
\pgfsetmiterjoin%
\definecolor{currentfill}{rgb}{0.066899,0.263188,0.377594}%
\pgfsetfillcolor{currentfill}%
\pgfsetlinewidth{0.000000pt}%
\definecolor{currentstroke}{rgb}{0.000000,0.000000,0.000000}%
\pgfsetstrokecolor{currentstroke}%
\pgfsetstrokeopacity{0.000000}%
\pgfsetdash{}{0pt}%
\pgfpathmoveto{\pgfqpoint{3.904617in}{1.609196in}}%
\pgfpathlineto{\pgfqpoint{3.913371in}{1.609196in}}%
\pgfpathlineto{\pgfqpoint{3.913371in}{1.612638in}}%
\pgfpathlineto{\pgfqpoint{3.904617in}{1.612638in}}%
\pgfpathlineto{\pgfqpoint{3.904617in}{1.609196in}}%
\pgfpathclose%
\pgfusepath{fill}%
\end{pgfscope}%
\begin{pgfscope}%
\pgfpathrectangle{\pgfqpoint{3.776708in}{0.600000in}}{\pgfqpoint{2.573292in}{2.070576in}}%
\pgfusepath{clip}%
\pgfsetbuttcap%
\pgfsetmiterjoin%
\definecolor{currentfill}{rgb}{0.066899,0.263188,0.377594}%
\pgfsetfillcolor{currentfill}%
\pgfsetlinewidth{0.000000pt}%
\definecolor{currentstroke}{rgb}{0.000000,0.000000,0.000000}%
\pgfsetstrokecolor{currentstroke}%
\pgfsetstrokeopacity{0.000000}%
\pgfsetdash{}{0pt}%
\pgfpathmoveto{\pgfqpoint{3.915559in}{1.609196in}}%
\pgfpathlineto{\pgfqpoint{3.924313in}{1.609196in}}%
\pgfpathlineto{\pgfqpoint{3.924313in}{1.609787in}}%
\pgfpathlineto{\pgfqpoint{3.915559in}{1.609787in}}%
\pgfpathlineto{\pgfqpoint{3.915559in}{1.609196in}}%
\pgfpathclose%
\pgfusepath{fill}%
\end{pgfscope}%
\begin{pgfscope}%
\pgfpathrectangle{\pgfqpoint{3.776708in}{0.600000in}}{\pgfqpoint{2.573292in}{2.070576in}}%
\pgfusepath{clip}%
\pgfsetbuttcap%
\pgfsetmiterjoin%
\definecolor{currentfill}{rgb}{0.066899,0.263188,0.377594}%
\pgfsetfillcolor{currentfill}%
\pgfsetlinewidth{0.000000pt}%
\definecolor{currentstroke}{rgb}{0.000000,0.000000,0.000000}%
\pgfsetstrokecolor{currentstroke}%
\pgfsetstrokeopacity{0.000000}%
\pgfsetdash{}{0pt}%
\pgfpathmoveto{\pgfqpoint{3.926501in}{1.609196in}}%
\pgfpathlineto{\pgfqpoint{3.935254in}{1.609196in}}%
\pgfpathlineto{\pgfqpoint{3.935254in}{1.604210in}}%
\pgfpathlineto{\pgfqpoint{3.926501in}{1.604210in}}%
\pgfpathlineto{\pgfqpoint{3.926501in}{1.609196in}}%
\pgfpathclose%
\pgfusepath{fill}%
\end{pgfscope}%
\begin{pgfscope}%
\pgfpathrectangle{\pgfqpoint{3.776708in}{0.600000in}}{\pgfqpoint{2.573292in}{2.070576in}}%
\pgfusepath{clip}%
\pgfsetbuttcap%
\pgfsetmiterjoin%
\definecolor{currentfill}{rgb}{0.066899,0.263188,0.377594}%
\pgfsetfillcolor{currentfill}%
\pgfsetlinewidth{0.000000pt}%
\definecolor{currentstroke}{rgb}{0.000000,0.000000,0.000000}%
\pgfsetstrokecolor{currentstroke}%
\pgfsetstrokeopacity{0.000000}%
\pgfsetdash{}{0pt}%
\pgfpathmoveto{\pgfqpoint{3.937443in}{1.609196in}}%
\pgfpathlineto{\pgfqpoint{3.946196in}{1.609196in}}%
\pgfpathlineto{\pgfqpoint{3.946196in}{1.605750in}}%
\pgfpathlineto{\pgfqpoint{3.937443in}{1.605750in}}%
\pgfpathlineto{\pgfqpoint{3.937443in}{1.609196in}}%
\pgfpathclose%
\pgfusepath{fill}%
\end{pgfscope}%
\begin{pgfscope}%
\pgfpathrectangle{\pgfqpoint{3.776708in}{0.600000in}}{\pgfqpoint{2.573292in}{2.070576in}}%
\pgfusepath{clip}%
\pgfsetbuttcap%
\pgfsetmiterjoin%
\definecolor{currentfill}{rgb}{0.066899,0.263188,0.377594}%
\pgfsetfillcolor{currentfill}%
\pgfsetlinewidth{0.000000pt}%
\definecolor{currentstroke}{rgb}{0.000000,0.000000,0.000000}%
\pgfsetstrokecolor{currentstroke}%
\pgfsetstrokeopacity{0.000000}%
\pgfsetdash{}{0pt}%
\pgfpathmoveto{\pgfqpoint{3.948385in}{1.609196in}}%
\pgfpathlineto{\pgfqpoint{3.957138in}{1.609196in}}%
\pgfpathlineto{\pgfqpoint{3.957138in}{1.607970in}}%
\pgfpathlineto{\pgfqpoint{3.948385in}{1.607970in}}%
\pgfpathlineto{\pgfqpoint{3.948385in}{1.609196in}}%
\pgfpathclose%
\pgfusepath{fill}%
\end{pgfscope}%
\begin{pgfscope}%
\pgfpathrectangle{\pgfqpoint{3.776708in}{0.600000in}}{\pgfqpoint{2.573292in}{2.070576in}}%
\pgfusepath{clip}%
\pgfsetbuttcap%
\pgfsetmiterjoin%
\definecolor{currentfill}{rgb}{0.066899,0.263188,0.377594}%
\pgfsetfillcolor{currentfill}%
\pgfsetlinewidth{0.000000pt}%
\definecolor{currentstroke}{rgb}{0.000000,0.000000,0.000000}%
\pgfsetstrokecolor{currentstroke}%
\pgfsetstrokeopacity{0.000000}%
\pgfsetdash{}{0pt}%
\pgfpathmoveto{\pgfqpoint{3.959326in}{1.609196in}}%
\pgfpathlineto{\pgfqpoint{3.968080in}{1.609196in}}%
\pgfpathlineto{\pgfqpoint{3.968080in}{1.608764in}}%
\pgfpathlineto{\pgfqpoint{3.959326in}{1.608764in}}%
\pgfpathlineto{\pgfqpoint{3.959326in}{1.609196in}}%
\pgfpathclose%
\pgfusepath{fill}%
\end{pgfscope}%
\begin{pgfscope}%
\pgfpathrectangle{\pgfqpoint{3.776708in}{0.600000in}}{\pgfqpoint{2.573292in}{2.070576in}}%
\pgfusepath{clip}%
\pgfsetbuttcap%
\pgfsetmiterjoin%
\definecolor{currentfill}{rgb}{0.066899,0.263188,0.377594}%
\pgfsetfillcolor{currentfill}%
\pgfsetlinewidth{0.000000pt}%
\definecolor{currentstroke}{rgb}{0.000000,0.000000,0.000000}%
\pgfsetstrokecolor{currentstroke}%
\pgfsetstrokeopacity{0.000000}%
\pgfsetdash{}{0pt}%
\pgfpathmoveto{\pgfqpoint{3.970268in}{1.609196in}}%
\pgfpathlineto{\pgfqpoint{3.979022in}{1.609196in}}%
\pgfpathlineto{\pgfqpoint{3.979022in}{1.608098in}}%
\pgfpathlineto{\pgfqpoint{3.970268in}{1.608098in}}%
\pgfpathlineto{\pgfqpoint{3.970268in}{1.609196in}}%
\pgfpathclose%
\pgfusepath{fill}%
\end{pgfscope}%
\begin{pgfscope}%
\pgfpathrectangle{\pgfqpoint{3.776708in}{0.600000in}}{\pgfqpoint{2.573292in}{2.070576in}}%
\pgfusepath{clip}%
\pgfsetbuttcap%
\pgfsetmiterjoin%
\definecolor{currentfill}{rgb}{0.066899,0.263188,0.377594}%
\pgfsetfillcolor{currentfill}%
\pgfsetlinewidth{0.000000pt}%
\definecolor{currentstroke}{rgb}{0.000000,0.000000,0.000000}%
\pgfsetstrokecolor{currentstroke}%
\pgfsetstrokeopacity{0.000000}%
\pgfsetdash{}{0pt}%
\pgfpathmoveto{\pgfqpoint{3.981210in}{1.609196in}}%
\pgfpathlineto{\pgfqpoint{3.989963in}{1.609196in}}%
\pgfpathlineto{\pgfqpoint{3.989963in}{1.603896in}}%
\pgfpathlineto{\pgfqpoint{3.981210in}{1.603896in}}%
\pgfpathlineto{\pgfqpoint{3.981210in}{1.609196in}}%
\pgfpathclose%
\pgfusepath{fill}%
\end{pgfscope}%
\begin{pgfscope}%
\pgfpathrectangle{\pgfqpoint{3.776708in}{0.600000in}}{\pgfqpoint{2.573292in}{2.070576in}}%
\pgfusepath{clip}%
\pgfsetbuttcap%
\pgfsetmiterjoin%
\definecolor{currentfill}{rgb}{0.066899,0.263188,0.377594}%
\pgfsetfillcolor{currentfill}%
\pgfsetlinewidth{0.000000pt}%
\definecolor{currentstroke}{rgb}{0.000000,0.000000,0.000000}%
\pgfsetstrokecolor{currentstroke}%
\pgfsetstrokeopacity{0.000000}%
\pgfsetdash{}{0pt}%
\pgfpathmoveto{\pgfqpoint{3.992152in}{1.609196in}}%
\pgfpathlineto{\pgfqpoint{4.000905in}{1.609196in}}%
\pgfpathlineto{\pgfqpoint{4.000905in}{1.603350in}}%
\pgfpathlineto{\pgfqpoint{3.992152in}{1.603350in}}%
\pgfpathlineto{\pgfqpoint{3.992152in}{1.609196in}}%
\pgfpathclose%
\pgfusepath{fill}%
\end{pgfscope}%
\begin{pgfscope}%
\pgfpathrectangle{\pgfqpoint{3.776708in}{0.600000in}}{\pgfqpoint{2.573292in}{2.070576in}}%
\pgfusepath{clip}%
\pgfsetbuttcap%
\pgfsetmiterjoin%
\definecolor{currentfill}{rgb}{0.066899,0.263188,0.377594}%
\pgfsetfillcolor{currentfill}%
\pgfsetlinewidth{0.000000pt}%
\definecolor{currentstroke}{rgb}{0.000000,0.000000,0.000000}%
\pgfsetstrokecolor{currentstroke}%
\pgfsetstrokeopacity{0.000000}%
\pgfsetdash{}{0pt}%
\pgfpathmoveto{\pgfqpoint{4.003094in}{1.609196in}}%
\pgfpathlineto{\pgfqpoint{4.011847in}{1.609196in}}%
\pgfpathlineto{\pgfqpoint{4.011847in}{1.601644in}}%
\pgfpathlineto{\pgfqpoint{4.003094in}{1.601644in}}%
\pgfpathlineto{\pgfqpoint{4.003094in}{1.609196in}}%
\pgfpathclose%
\pgfusepath{fill}%
\end{pgfscope}%
\begin{pgfscope}%
\pgfpathrectangle{\pgfqpoint{3.776708in}{0.600000in}}{\pgfqpoint{2.573292in}{2.070576in}}%
\pgfusepath{clip}%
\pgfsetbuttcap%
\pgfsetmiterjoin%
\definecolor{currentfill}{rgb}{0.066899,0.263188,0.377594}%
\pgfsetfillcolor{currentfill}%
\pgfsetlinewidth{0.000000pt}%
\definecolor{currentstroke}{rgb}{0.000000,0.000000,0.000000}%
\pgfsetstrokecolor{currentstroke}%
\pgfsetstrokeopacity{0.000000}%
\pgfsetdash{}{0pt}%
\pgfpathmoveto{\pgfqpoint{4.014035in}{1.609196in}}%
\pgfpathlineto{\pgfqpoint{4.022789in}{1.609196in}}%
\pgfpathlineto{\pgfqpoint{4.022789in}{1.601779in}}%
\pgfpathlineto{\pgfqpoint{4.014035in}{1.601779in}}%
\pgfpathlineto{\pgfqpoint{4.014035in}{1.609196in}}%
\pgfpathclose%
\pgfusepath{fill}%
\end{pgfscope}%
\begin{pgfscope}%
\pgfpathrectangle{\pgfqpoint{3.776708in}{0.600000in}}{\pgfqpoint{2.573292in}{2.070576in}}%
\pgfusepath{clip}%
\pgfsetbuttcap%
\pgfsetmiterjoin%
\definecolor{currentfill}{rgb}{0.066899,0.263188,0.377594}%
\pgfsetfillcolor{currentfill}%
\pgfsetlinewidth{0.000000pt}%
\definecolor{currentstroke}{rgb}{0.000000,0.000000,0.000000}%
\pgfsetstrokecolor{currentstroke}%
\pgfsetstrokeopacity{0.000000}%
\pgfsetdash{}{0pt}%
\pgfpathmoveto{\pgfqpoint{4.024977in}{1.609196in}}%
\pgfpathlineto{\pgfqpoint{4.033731in}{1.609196in}}%
\pgfpathlineto{\pgfqpoint{4.033731in}{1.601033in}}%
\pgfpathlineto{\pgfqpoint{4.024977in}{1.601033in}}%
\pgfpathlineto{\pgfqpoint{4.024977in}{1.609196in}}%
\pgfpathclose%
\pgfusepath{fill}%
\end{pgfscope}%
\begin{pgfscope}%
\pgfpathrectangle{\pgfqpoint{3.776708in}{0.600000in}}{\pgfqpoint{2.573292in}{2.070576in}}%
\pgfusepath{clip}%
\pgfsetbuttcap%
\pgfsetmiterjoin%
\definecolor{currentfill}{rgb}{0.066899,0.263188,0.377594}%
\pgfsetfillcolor{currentfill}%
\pgfsetlinewidth{0.000000pt}%
\definecolor{currentstroke}{rgb}{0.000000,0.000000,0.000000}%
\pgfsetstrokecolor{currentstroke}%
\pgfsetstrokeopacity{0.000000}%
\pgfsetdash{}{0pt}%
\pgfpathmoveto{\pgfqpoint{4.035919in}{1.609196in}}%
\pgfpathlineto{\pgfqpoint{4.044672in}{1.609196in}}%
\pgfpathlineto{\pgfqpoint{4.044672in}{1.597757in}}%
\pgfpathlineto{\pgfqpoint{4.035919in}{1.597757in}}%
\pgfpathlineto{\pgfqpoint{4.035919in}{1.609196in}}%
\pgfpathclose%
\pgfusepath{fill}%
\end{pgfscope}%
\begin{pgfscope}%
\pgfpathrectangle{\pgfqpoint{3.776708in}{0.600000in}}{\pgfqpoint{2.573292in}{2.070576in}}%
\pgfusepath{clip}%
\pgfsetbuttcap%
\pgfsetmiterjoin%
\definecolor{currentfill}{rgb}{0.066899,0.263188,0.377594}%
\pgfsetfillcolor{currentfill}%
\pgfsetlinewidth{0.000000pt}%
\definecolor{currentstroke}{rgb}{0.000000,0.000000,0.000000}%
\pgfsetstrokecolor{currentstroke}%
\pgfsetstrokeopacity{0.000000}%
\pgfsetdash{}{0pt}%
\pgfpathmoveto{\pgfqpoint{4.046861in}{1.609196in}}%
\pgfpathlineto{\pgfqpoint{4.055614in}{1.609196in}}%
\pgfpathlineto{\pgfqpoint{4.055614in}{1.590826in}}%
\pgfpathlineto{\pgfqpoint{4.046861in}{1.590826in}}%
\pgfpathlineto{\pgfqpoint{4.046861in}{1.609196in}}%
\pgfpathclose%
\pgfusepath{fill}%
\end{pgfscope}%
\begin{pgfscope}%
\pgfpathrectangle{\pgfqpoint{3.776708in}{0.600000in}}{\pgfqpoint{2.573292in}{2.070576in}}%
\pgfusepath{clip}%
\pgfsetbuttcap%
\pgfsetmiterjoin%
\definecolor{currentfill}{rgb}{0.066899,0.263188,0.377594}%
\pgfsetfillcolor{currentfill}%
\pgfsetlinewidth{0.000000pt}%
\definecolor{currentstroke}{rgb}{0.000000,0.000000,0.000000}%
\pgfsetstrokecolor{currentstroke}%
\pgfsetstrokeopacity{0.000000}%
\pgfsetdash{}{0pt}%
\pgfpathmoveto{\pgfqpoint{4.057803in}{1.609196in}}%
\pgfpathlineto{\pgfqpoint{4.066556in}{1.609196in}}%
\pgfpathlineto{\pgfqpoint{4.066556in}{1.585187in}}%
\pgfpathlineto{\pgfqpoint{4.057803in}{1.585187in}}%
\pgfpathlineto{\pgfqpoint{4.057803in}{1.609196in}}%
\pgfpathclose%
\pgfusepath{fill}%
\end{pgfscope}%
\begin{pgfscope}%
\pgfpathrectangle{\pgfqpoint{3.776708in}{0.600000in}}{\pgfqpoint{2.573292in}{2.070576in}}%
\pgfusepath{clip}%
\pgfsetbuttcap%
\pgfsetmiterjoin%
\definecolor{currentfill}{rgb}{0.066899,0.263188,0.377594}%
\pgfsetfillcolor{currentfill}%
\pgfsetlinewidth{0.000000pt}%
\definecolor{currentstroke}{rgb}{0.000000,0.000000,0.000000}%
\pgfsetstrokecolor{currentstroke}%
\pgfsetstrokeopacity{0.000000}%
\pgfsetdash{}{0pt}%
\pgfpathmoveto{\pgfqpoint{4.068744in}{1.609196in}}%
\pgfpathlineto{\pgfqpoint{4.077498in}{1.609196in}}%
\pgfpathlineto{\pgfqpoint{4.077498in}{1.578195in}}%
\pgfpathlineto{\pgfqpoint{4.068744in}{1.578195in}}%
\pgfpathlineto{\pgfqpoint{4.068744in}{1.609196in}}%
\pgfpathclose%
\pgfusepath{fill}%
\end{pgfscope}%
\begin{pgfscope}%
\pgfpathrectangle{\pgfqpoint{3.776708in}{0.600000in}}{\pgfqpoint{2.573292in}{2.070576in}}%
\pgfusepath{clip}%
\pgfsetbuttcap%
\pgfsetmiterjoin%
\definecolor{currentfill}{rgb}{0.066899,0.263188,0.377594}%
\pgfsetfillcolor{currentfill}%
\pgfsetlinewidth{0.000000pt}%
\definecolor{currentstroke}{rgb}{0.000000,0.000000,0.000000}%
\pgfsetstrokecolor{currentstroke}%
\pgfsetstrokeopacity{0.000000}%
\pgfsetdash{}{0pt}%
\pgfpathmoveto{\pgfqpoint{4.079686in}{1.609196in}}%
\pgfpathlineto{\pgfqpoint{4.088440in}{1.609196in}}%
\pgfpathlineto{\pgfqpoint{4.088440in}{1.570939in}}%
\pgfpathlineto{\pgfqpoint{4.079686in}{1.570939in}}%
\pgfpathlineto{\pgfqpoint{4.079686in}{1.609196in}}%
\pgfpathclose%
\pgfusepath{fill}%
\end{pgfscope}%
\begin{pgfscope}%
\pgfpathrectangle{\pgfqpoint{3.776708in}{0.600000in}}{\pgfqpoint{2.573292in}{2.070576in}}%
\pgfusepath{clip}%
\pgfsetbuttcap%
\pgfsetmiterjoin%
\definecolor{currentfill}{rgb}{0.066899,0.263188,0.377594}%
\pgfsetfillcolor{currentfill}%
\pgfsetlinewidth{0.000000pt}%
\definecolor{currentstroke}{rgb}{0.000000,0.000000,0.000000}%
\pgfsetstrokecolor{currentstroke}%
\pgfsetstrokeopacity{0.000000}%
\pgfsetdash{}{0pt}%
\pgfpathmoveto{\pgfqpoint{4.090628in}{1.609196in}}%
\pgfpathlineto{\pgfqpoint{4.099381in}{1.609196in}}%
\pgfpathlineto{\pgfqpoint{4.099381in}{1.571912in}}%
\pgfpathlineto{\pgfqpoint{4.090628in}{1.571912in}}%
\pgfpathlineto{\pgfqpoint{4.090628in}{1.609196in}}%
\pgfpathclose%
\pgfusepath{fill}%
\end{pgfscope}%
\begin{pgfscope}%
\pgfpathrectangle{\pgfqpoint{3.776708in}{0.600000in}}{\pgfqpoint{2.573292in}{2.070576in}}%
\pgfusepath{clip}%
\pgfsetbuttcap%
\pgfsetmiterjoin%
\definecolor{currentfill}{rgb}{0.066899,0.263188,0.377594}%
\pgfsetfillcolor{currentfill}%
\pgfsetlinewidth{0.000000pt}%
\definecolor{currentstroke}{rgb}{0.000000,0.000000,0.000000}%
\pgfsetstrokecolor{currentstroke}%
\pgfsetstrokeopacity{0.000000}%
\pgfsetdash{}{0pt}%
\pgfpathmoveto{\pgfqpoint{4.101570in}{1.609196in}}%
\pgfpathlineto{\pgfqpoint{4.110323in}{1.609196in}}%
\pgfpathlineto{\pgfqpoint{4.110323in}{1.570270in}}%
\pgfpathlineto{\pgfqpoint{4.101570in}{1.570270in}}%
\pgfpathlineto{\pgfqpoint{4.101570in}{1.609196in}}%
\pgfpathclose%
\pgfusepath{fill}%
\end{pgfscope}%
\begin{pgfscope}%
\pgfpathrectangle{\pgfqpoint{3.776708in}{0.600000in}}{\pgfqpoint{2.573292in}{2.070576in}}%
\pgfusepath{clip}%
\pgfsetbuttcap%
\pgfsetmiterjoin%
\definecolor{currentfill}{rgb}{0.066899,0.263188,0.377594}%
\pgfsetfillcolor{currentfill}%
\pgfsetlinewidth{0.000000pt}%
\definecolor{currentstroke}{rgb}{0.000000,0.000000,0.000000}%
\pgfsetstrokecolor{currentstroke}%
\pgfsetstrokeopacity{0.000000}%
\pgfsetdash{}{0pt}%
\pgfpathmoveto{\pgfqpoint{4.112512in}{1.609196in}}%
\pgfpathlineto{\pgfqpoint{4.121265in}{1.609196in}}%
\pgfpathlineto{\pgfqpoint{4.121265in}{1.567112in}}%
\pgfpathlineto{\pgfqpoint{4.112512in}{1.567112in}}%
\pgfpathlineto{\pgfqpoint{4.112512in}{1.609196in}}%
\pgfpathclose%
\pgfusepath{fill}%
\end{pgfscope}%
\begin{pgfscope}%
\pgfpathrectangle{\pgfqpoint{3.776708in}{0.600000in}}{\pgfqpoint{2.573292in}{2.070576in}}%
\pgfusepath{clip}%
\pgfsetbuttcap%
\pgfsetmiterjoin%
\definecolor{currentfill}{rgb}{0.066899,0.263188,0.377594}%
\pgfsetfillcolor{currentfill}%
\pgfsetlinewidth{0.000000pt}%
\definecolor{currentstroke}{rgb}{0.000000,0.000000,0.000000}%
\pgfsetstrokecolor{currentstroke}%
\pgfsetstrokeopacity{0.000000}%
\pgfsetdash{}{0pt}%
\pgfpathmoveto{\pgfqpoint{4.123453in}{1.609196in}}%
\pgfpathlineto{\pgfqpoint{4.132207in}{1.609196in}}%
\pgfpathlineto{\pgfqpoint{4.132207in}{1.562657in}}%
\pgfpathlineto{\pgfqpoint{4.123453in}{1.562657in}}%
\pgfpathlineto{\pgfqpoint{4.123453in}{1.609196in}}%
\pgfpathclose%
\pgfusepath{fill}%
\end{pgfscope}%
\begin{pgfscope}%
\pgfpathrectangle{\pgfqpoint{3.776708in}{0.600000in}}{\pgfqpoint{2.573292in}{2.070576in}}%
\pgfusepath{clip}%
\pgfsetbuttcap%
\pgfsetmiterjoin%
\definecolor{currentfill}{rgb}{0.066899,0.263188,0.377594}%
\pgfsetfillcolor{currentfill}%
\pgfsetlinewidth{0.000000pt}%
\definecolor{currentstroke}{rgb}{0.000000,0.000000,0.000000}%
\pgfsetstrokecolor{currentstroke}%
\pgfsetstrokeopacity{0.000000}%
\pgfsetdash{}{0pt}%
\pgfpathmoveto{\pgfqpoint{4.134395in}{1.609196in}}%
\pgfpathlineto{\pgfqpoint{4.143149in}{1.609196in}}%
\pgfpathlineto{\pgfqpoint{4.143149in}{1.570642in}}%
\pgfpathlineto{\pgfqpoint{4.134395in}{1.570642in}}%
\pgfpathlineto{\pgfqpoint{4.134395in}{1.609196in}}%
\pgfpathclose%
\pgfusepath{fill}%
\end{pgfscope}%
\begin{pgfscope}%
\pgfpathrectangle{\pgfqpoint{3.776708in}{0.600000in}}{\pgfqpoint{2.573292in}{2.070576in}}%
\pgfusepath{clip}%
\pgfsetbuttcap%
\pgfsetmiterjoin%
\definecolor{currentfill}{rgb}{0.066899,0.263188,0.377594}%
\pgfsetfillcolor{currentfill}%
\pgfsetlinewidth{0.000000pt}%
\definecolor{currentstroke}{rgb}{0.000000,0.000000,0.000000}%
\pgfsetstrokecolor{currentstroke}%
\pgfsetstrokeopacity{0.000000}%
\pgfsetdash{}{0pt}%
\pgfpathmoveto{\pgfqpoint{4.145337in}{1.609196in}}%
\pgfpathlineto{\pgfqpoint{4.154090in}{1.609196in}}%
\pgfpathlineto{\pgfqpoint{4.154090in}{1.569645in}}%
\pgfpathlineto{\pgfqpoint{4.145337in}{1.569645in}}%
\pgfpathlineto{\pgfqpoint{4.145337in}{1.609196in}}%
\pgfpathclose%
\pgfusepath{fill}%
\end{pgfscope}%
\begin{pgfscope}%
\pgfpathrectangle{\pgfqpoint{3.776708in}{0.600000in}}{\pgfqpoint{2.573292in}{2.070576in}}%
\pgfusepath{clip}%
\pgfsetbuttcap%
\pgfsetmiterjoin%
\definecolor{currentfill}{rgb}{0.066899,0.263188,0.377594}%
\pgfsetfillcolor{currentfill}%
\pgfsetlinewidth{0.000000pt}%
\definecolor{currentstroke}{rgb}{0.000000,0.000000,0.000000}%
\pgfsetstrokecolor{currentstroke}%
\pgfsetstrokeopacity{0.000000}%
\pgfsetdash{}{0pt}%
\pgfpathmoveto{\pgfqpoint{4.156279in}{1.609196in}}%
\pgfpathlineto{\pgfqpoint{4.165032in}{1.609196in}}%
\pgfpathlineto{\pgfqpoint{4.165032in}{1.567532in}}%
\pgfpathlineto{\pgfqpoint{4.156279in}{1.567532in}}%
\pgfpathlineto{\pgfqpoint{4.156279in}{1.609196in}}%
\pgfpathclose%
\pgfusepath{fill}%
\end{pgfscope}%
\begin{pgfscope}%
\pgfpathrectangle{\pgfqpoint{3.776708in}{0.600000in}}{\pgfqpoint{2.573292in}{2.070576in}}%
\pgfusepath{clip}%
\pgfsetbuttcap%
\pgfsetmiterjoin%
\definecolor{currentfill}{rgb}{0.066899,0.263188,0.377594}%
\pgfsetfillcolor{currentfill}%
\pgfsetlinewidth{0.000000pt}%
\definecolor{currentstroke}{rgb}{0.000000,0.000000,0.000000}%
\pgfsetstrokecolor{currentstroke}%
\pgfsetstrokeopacity{0.000000}%
\pgfsetdash{}{0pt}%
\pgfpathmoveto{\pgfqpoint{4.167221in}{1.609196in}}%
\pgfpathlineto{\pgfqpoint{4.175974in}{1.609196in}}%
\pgfpathlineto{\pgfqpoint{4.175974in}{1.568497in}}%
\pgfpathlineto{\pgfqpoint{4.167221in}{1.568497in}}%
\pgfpathlineto{\pgfqpoint{4.167221in}{1.609196in}}%
\pgfpathclose%
\pgfusepath{fill}%
\end{pgfscope}%
\begin{pgfscope}%
\pgfpathrectangle{\pgfqpoint{3.776708in}{0.600000in}}{\pgfqpoint{2.573292in}{2.070576in}}%
\pgfusepath{clip}%
\pgfsetbuttcap%
\pgfsetmiterjoin%
\definecolor{currentfill}{rgb}{0.066899,0.263188,0.377594}%
\pgfsetfillcolor{currentfill}%
\pgfsetlinewidth{0.000000pt}%
\definecolor{currentstroke}{rgb}{0.000000,0.000000,0.000000}%
\pgfsetstrokecolor{currentstroke}%
\pgfsetstrokeopacity{0.000000}%
\pgfsetdash{}{0pt}%
\pgfpathmoveto{\pgfqpoint{4.178162in}{1.609196in}}%
\pgfpathlineto{\pgfqpoint{4.186916in}{1.609196in}}%
\pgfpathlineto{\pgfqpoint{4.186916in}{1.569369in}}%
\pgfpathlineto{\pgfqpoint{4.178162in}{1.569369in}}%
\pgfpathlineto{\pgfqpoint{4.178162in}{1.609196in}}%
\pgfpathclose%
\pgfusepath{fill}%
\end{pgfscope}%
\begin{pgfscope}%
\pgfpathrectangle{\pgfqpoint{3.776708in}{0.600000in}}{\pgfqpoint{2.573292in}{2.070576in}}%
\pgfusepath{clip}%
\pgfsetbuttcap%
\pgfsetmiterjoin%
\definecolor{currentfill}{rgb}{0.066899,0.263188,0.377594}%
\pgfsetfillcolor{currentfill}%
\pgfsetlinewidth{0.000000pt}%
\definecolor{currentstroke}{rgb}{0.000000,0.000000,0.000000}%
\pgfsetstrokecolor{currentstroke}%
\pgfsetstrokeopacity{0.000000}%
\pgfsetdash{}{0pt}%
\pgfpathmoveto{\pgfqpoint{4.189104in}{1.609196in}}%
\pgfpathlineto{\pgfqpoint{4.197858in}{1.609196in}}%
\pgfpathlineto{\pgfqpoint{4.197858in}{1.570665in}}%
\pgfpathlineto{\pgfqpoint{4.189104in}{1.570665in}}%
\pgfpathlineto{\pgfqpoint{4.189104in}{1.609196in}}%
\pgfpathclose%
\pgfusepath{fill}%
\end{pgfscope}%
\begin{pgfscope}%
\pgfpathrectangle{\pgfqpoint{3.776708in}{0.600000in}}{\pgfqpoint{2.573292in}{2.070576in}}%
\pgfusepath{clip}%
\pgfsetbuttcap%
\pgfsetmiterjoin%
\definecolor{currentfill}{rgb}{0.066899,0.263188,0.377594}%
\pgfsetfillcolor{currentfill}%
\pgfsetlinewidth{0.000000pt}%
\definecolor{currentstroke}{rgb}{0.000000,0.000000,0.000000}%
\pgfsetstrokecolor{currentstroke}%
\pgfsetstrokeopacity{0.000000}%
\pgfsetdash{}{0pt}%
\pgfpathmoveto{\pgfqpoint{4.200046in}{1.609196in}}%
\pgfpathlineto{\pgfqpoint{4.208799in}{1.609196in}}%
\pgfpathlineto{\pgfqpoint{4.208799in}{1.569061in}}%
\pgfpathlineto{\pgfqpoint{4.200046in}{1.569061in}}%
\pgfpathlineto{\pgfqpoint{4.200046in}{1.609196in}}%
\pgfpathclose%
\pgfusepath{fill}%
\end{pgfscope}%
\begin{pgfscope}%
\pgfpathrectangle{\pgfqpoint{3.776708in}{0.600000in}}{\pgfqpoint{2.573292in}{2.070576in}}%
\pgfusepath{clip}%
\pgfsetbuttcap%
\pgfsetmiterjoin%
\definecolor{currentfill}{rgb}{0.066899,0.263188,0.377594}%
\pgfsetfillcolor{currentfill}%
\pgfsetlinewidth{0.000000pt}%
\definecolor{currentstroke}{rgb}{0.000000,0.000000,0.000000}%
\pgfsetstrokecolor{currentstroke}%
\pgfsetstrokeopacity{0.000000}%
\pgfsetdash{}{0pt}%
\pgfpathmoveto{\pgfqpoint{4.210988in}{1.609196in}}%
\pgfpathlineto{\pgfqpoint{4.219741in}{1.609196in}}%
\pgfpathlineto{\pgfqpoint{4.219741in}{1.570307in}}%
\pgfpathlineto{\pgfqpoint{4.210988in}{1.570307in}}%
\pgfpathlineto{\pgfqpoint{4.210988in}{1.609196in}}%
\pgfpathclose%
\pgfusepath{fill}%
\end{pgfscope}%
\begin{pgfscope}%
\pgfpathrectangle{\pgfqpoint{3.776708in}{0.600000in}}{\pgfqpoint{2.573292in}{2.070576in}}%
\pgfusepath{clip}%
\pgfsetbuttcap%
\pgfsetmiterjoin%
\definecolor{currentfill}{rgb}{0.066899,0.263188,0.377594}%
\pgfsetfillcolor{currentfill}%
\pgfsetlinewidth{0.000000pt}%
\definecolor{currentstroke}{rgb}{0.000000,0.000000,0.000000}%
\pgfsetstrokecolor{currentstroke}%
\pgfsetstrokeopacity{0.000000}%
\pgfsetdash{}{0pt}%
\pgfpathmoveto{\pgfqpoint{4.221930in}{1.609196in}}%
\pgfpathlineto{\pgfqpoint{4.230683in}{1.609196in}}%
\pgfpathlineto{\pgfqpoint{4.230683in}{1.569944in}}%
\pgfpathlineto{\pgfqpoint{4.221930in}{1.569944in}}%
\pgfpathlineto{\pgfqpoint{4.221930in}{1.609196in}}%
\pgfpathclose%
\pgfusepath{fill}%
\end{pgfscope}%
\begin{pgfscope}%
\pgfpathrectangle{\pgfqpoint{3.776708in}{0.600000in}}{\pgfqpoint{2.573292in}{2.070576in}}%
\pgfusepath{clip}%
\pgfsetbuttcap%
\pgfsetmiterjoin%
\definecolor{currentfill}{rgb}{0.066899,0.263188,0.377594}%
\pgfsetfillcolor{currentfill}%
\pgfsetlinewidth{0.000000pt}%
\definecolor{currentstroke}{rgb}{0.000000,0.000000,0.000000}%
\pgfsetstrokecolor{currentstroke}%
\pgfsetstrokeopacity{0.000000}%
\pgfsetdash{}{0pt}%
\pgfpathmoveto{\pgfqpoint{4.232871in}{1.609196in}}%
\pgfpathlineto{\pgfqpoint{4.241625in}{1.609196in}}%
\pgfpathlineto{\pgfqpoint{4.241625in}{1.569251in}}%
\pgfpathlineto{\pgfqpoint{4.232871in}{1.569251in}}%
\pgfpathlineto{\pgfqpoint{4.232871in}{1.609196in}}%
\pgfpathclose%
\pgfusepath{fill}%
\end{pgfscope}%
\begin{pgfscope}%
\pgfpathrectangle{\pgfqpoint{3.776708in}{0.600000in}}{\pgfqpoint{2.573292in}{2.070576in}}%
\pgfusepath{clip}%
\pgfsetbuttcap%
\pgfsetmiterjoin%
\definecolor{currentfill}{rgb}{0.066899,0.263188,0.377594}%
\pgfsetfillcolor{currentfill}%
\pgfsetlinewidth{0.000000pt}%
\definecolor{currentstroke}{rgb}{0.000000,0.000000,0.000000}%
\pgfsetstrokecolor{currentstroke}%
\pgfsetstrokeopacity{0.000000}%
\pgfsetdash{}{0pt}%
\pgfpathmoveto{\pgfqpoint{4.243813in}{1.609196in}}%
\pgfpathlineto{\pgfqpoint{4.252567in}{1.609196in}}%
\pgfpathlineto{\pgfqpoint{4.252567in}{1.569658in}}%
\pgfpathlineto{\pgfqpoint{4.243813in}{1.569658in}}%
\pgfpathlineto{\pgfqpoint{4.243813in}{1.609196in}}%
\pgfpathclose%
\pgfusepath{fill}%
\end{pgfscope}%
\begin{pgfscope}%
\pgfpathrectangle{\pgfqpoint{3.776708in}{0.600000in}}{\pgfqpoint{2.573292in}{2.070576in}}%
\pgfusepath{clip}%
\pgfsetbuttcap%
\pgfsetmiterjoin%
\definecolor{currentfill}{rgb}{0.066899,0.263188,0.377594}%
\pgfsetfillcolor{currentfill}%
\pgfsetlinewidth{0.000000pt}%
\definecolor{currentstroke}{rgb}{0.000000,0.000000,0.000000}%
\pgfsetstrokecolor{currentstroke}%
\pgfsetstrokeopacity{0.000000}%
\pgfsetdash{}{0pt}%
\pgfpathmoveto{\pgfqpoint{4.254755in}{1.609196in}}%
\pgfpathlineto{\pgfqpoint{4.263508in}{1.609196in}}%
\pgfpathlineto{\pgfqpoint{4.263508in}{1.570404in}}%
\pgfpathlineto{\pgfqpoint{4.254755in}{1.570404in}}%
\pgfpathlineto{\pgfqpoint{4.254755in}{1.609196in}}%
\pgfpathclose%
\pgfusepath{fill}%
\end{pgfscope}%
\begin{pgfscope}%
\pgfpathrectangle{\pgfqpoint{3.776708in}{0.600000in}}{\pgfqpoint{2.573292in}{2.070576in}}%
\pgfusepath{clip}%
\pgfsetbuttcap%
\pgfsetmiterjoin%
\definecolor{currentfill}{rgb}{0.066899,0.263188,0.377594}%
\pgfsetfillcolor{currentfill}%
\pgfsetlinewidth{0.000000pt}%
\definecolor{currentstroke}{rgb}{0.000000,0.000000,0.000000}%
\pgfsetstrokecolor{currentstroke}%
\pgfsetstrokeopacity{0.000000}%
\pgfsetdash{}{0pt}%
\pgfpathmoveto{\pgfqpoint{4.265697in}{1.609196in}}%
\pgfpathlineto{\pgfqpoint{4.274450in}{1.609196in}}%
\pgfpathlineto{\pgfqpoint{4.274450in}{1.563869in}}%
\pgfpathlineto{\pgfqpoint{4.265697in}{1.563869in}}%
\pgfpathlineto{\pgfqpoint{4.265697in}{1.609196in}}%
\pgfpathclose%
\pgfusepath{fill}%
\end{pgfscope}%
\begin{pgfscope}%
\pgfpathrectangle{\pgfqpoint{3.776708in}{0.600000in}}{\pgfqpoint{2.573292in}{2.070576in}}%
\pgfusepath{clip}%
\pgfsetbuttcap%
\pgfsetmiterjoin%
\definecolor{currentfill}{rgb}{0.066899,0.263188,0.377594}%
\pgfsetfillcolor{currentfill}%
\pgfsetlinewidth{0.000000pt}%
\definecolor{currentstroke}{rgb}{0.000000,0.000000,0.000000}%
\pgfsetstrokecolor{currentstroke}%
\pgfsetstrokeopacity{0.000000}%
\pgfsetdash{}{0pt}%
\pgfpathmoveto{\pgfqpoint{4.276639in}{1.609196in}}%
\pgfpathlineto{\pgfqpoint{4.285392in}{1.609196in}}%
\pgfpathlineto{\pgfqpoint{4.285392in}{1.557315in}}%
\pgfpathlineto{\pgfqpoint{4.276639in}{1.557315in}}%
\pgfpathlineto{\pgfqpoint{4.276639in}{1.609196in}}%
\pgfpathclose%
\pgfusepath{fill}%
\end{pgfscope}%
\begin{pgfscope}%
\pgfpathrectangle{\pgfqpoint{3.776708in}{0.600000in}}{\pgfqpoint{2.573292in}{2.070576in}}%
\pgfusepath{clip}%
\pgfsetbuttcap%
\pgfsetmiterjoin%
\definecolor{currentfill}{rgb}{0.066899,0.263188,0.377594}%
\pgfsetfillcolor{currentfill}%
\pgfsetlinewidth{0.000000pt}%
\definecolor{currentstroke}{rgb}{0.000000,0.000000,0.000000}%
\pgfsetstrokecolor{currentstroke}%
\pgfsetstrokeopacity{0.000000}%
\pgfsetdash{}{0pt}%
\pgfpathmoveto{\pgfqpoint{4.287580in}{1.609196in}}%
\pgfpathlineto{\pgfqpoint{4.296334in}{1.609196in}}%
\pgfpathlineto{\pgfqpoint{4.296334in}{1.550276in}}%
\pgfpathlineto{\pgfqpoint{4.287580in}{1.550276in}}%
\pgfpathlineto{\pgfqpoint{4.287580in}{1.609196in}}%
\pgfpathclose%
\pgfusepath{fill}%
\end{pgfscope}%
\begin{pgfscope}%
\pgfpathrectangle{\pgfqpoint{3.776708in}{0.600000in}}{\pgfqpoint{2.573292in}{2.070576in}}%
\pgfusepath{clip}%
\pgfsetbuttcap%
\pgfsetmiterjoin%
\definecolor{currentfill}{rgb}{0.066899,0.263188,0.377594}%
\pgfsetfillcolor{currentfill}%
\pgfsetlinewidth{0.000000pt}%
\definecolor{currentstroke}{rgb}{0.000000,0.000000,0.000000}%
\pgfsetstrokecolor{currentstroke}%
\pgfsetstrokeopacity{0.000000}%
\pgfsetdash{}{0pt}%
\pgfpathmoveto{\pgfqpoint{4.298522in}{1.609196in}}%
\pgfpathlineto{\pgfqpoint{4.307276in}{1.609196in}}%
\pgfpathlineto{\pgfqpoint{4.307276in}{1.548095in}}%
\pgfpathlineto{\pgfqpoint{4.298522in}{1.548095in}}%
\pgfpathlineto{\pgfqpoint{4.298522in}{1.609196in}}%
\pgfpathclose%
\pgfusepath{fill}%
\end{pgfscope}%
\begin{pgfscope}%
\pgfpathrectangle{\pgfqpoint{3.776708in}{0.600000in}}{\pgfqpoint{2.573292in}{2.070576in}}%
\pgfusepath{clip}%
\pgfsetbuttcap%
\pgfsetmiterjoin%
\definecolor{currentfill}{rgb}{0.066899,0.263188,0.377594}%
\pgfsetfillcolor{currentfill}%
\pgfsetlinewidth{0.000000pt}%
\definecolor{currentstroke}{rgb}{0.000000,0.000000,0.000000}%
\pgfsetstrokecolor{currentstroke}%
\pgfsetstrokeopacity{0.000000}%
\pgfsetdash{}{0pt}%
\pgfpathmoveto{\pgfqpoint{4.309464in}{1.609196in}}%
\pgfpathlineto{\pgfqpoint{4.318217in}{1.609196in}}%
\pgfpathlineto{\pgfqpoint{4.318217in}{1.544782in}}%
\pgfpathlineto{\pgfqpoint{4.309464in}{1.544782in}}%
\pgfpathlineto{\pgfqpoint{4.309464in}{1.609196in}}%
\pgfpathclose%
\pgfusepath{fill}%
\end{pgfscope}%
\begin{pgfscope}%
\pgfpathrectangle{\pgfqpoint{3.776708in}{0.600000in}}{\pgfqpoint{2.573292in}{2.070576in}}%
\pgfusepath{clip}%
\pgfsetbuttcap%
\pgfsetmiterjoin%
\definecolor{currentfill}{rgb}{0.066899,0.263188,0.377594}%
\pgfsetfillcolor{currentfill}%
\pgfsetlinewidth{0.000000pt}%
\definecolor{currentstroke}{rgb}{0.000000,0.000000,0.000000}%
\pgfsetstrokecolor{currentstroke}%
\pgfsetstrokeopacity{0.000000}%
\pgfsetdash{}{0pt}%
\pgfpathmoveto{\pgfqpoint{4.320406in}{1.609196in}}%
\pgfpathlineto{\pgfqpoint{4.329159in}{1.609196in}}%
\pgfpathlineto{\pgfqpoint{4.329159in}{1.541583in}}%
\pgfpathlineto{\pgfqpoint{4.320406in}{1.541583in}}%
\pgfpathlineto{\pgfqpoint{4.320406in}{1.609196in}}%
\pgfpathclose%
\pgfusepath{fill}%
\end{pgfscope}%
\begin{pgfscope}%
\pgfpathrectangle{\pgfqpoint{3.776708in}{0.600000in}}{\pgfqpoint{2.573292in}{2.070576in}}%
\pgfusepath{clip}%
\pgfsetbuttcap%
\pgfsetmiterjoin%
\definecolor{currentfill}{rgb}{0.066899,0.263188,0.377594}%
\pgfsetfillcolor{currentfill}%
\pgfsetlinewidth{0.000000pt}%
\definecolor{currentstroke}{rgb}{0.000000,0.000000,0.000000}%
\pgfsetstrokecolor{currentstroke}%
\pgfsetstrokeopacity{0.000000}%
\pgfsetdash{}{0pt}%
\pgfpathmoveto{\pgfqpoint{4.331348in}{1.609196in}}%
\pgfpathlineto{\pgfqpoint{4.340101in}{1.609196in}}%
\pgfpathlineto{\pgfqpoint{4.340101in}{1.538411in}}%
\pgfpathlineto{\pgfqpoint{4.331348in}{1.538411in}}%
\pgfpathlineto{\pgfqpoint{4.331348in}{1.609196in}}%
\pgfpathclose%
\pgfusepath{fill}%
\end{pgfscope}%
\begin{pgfscope}%
\pgfpathrectangle{\pgfqpoint{3.776708in}{0.600000in}}{\pgfqpoint{2.573292in}{2.070576in}}%
\pgfusepath{clip}%
\pgfsetbuttcap%
\pgfsetmiterjoin%
\definecolor{currentfill}{rgb}{0.066899,0.263188,0.377594}%
\pgfsetfillcolor{currentfill}%
\pgfsetlinewidth{0.000000pt}%
\definecolor{currentstroke}{rgb}{0.000000,0.000000,0.000000}%
\pgfsetstrokecolor{currentstroke}%
\pgfsetstrokeopacity{0.000000}%
\pgfsetdash{}{0pt}%
\pgfpathmoveto{\pgfqpoint{4.342289in}{1.609196in}}%
\pgfpathlineto{\pgfqpoint{4.351043in}{1.609196in}}%
\pgfpathlineto{\pgfqpoint{4.351043in}{1.539114in}}%
\pgfpathlineto{\pgfqpoint{4.342289in}{1.539114in}}%
\pgfpathlineto{\pgfqpoint{4.342289in}{1.609196in}}%
\pgfpathclose%
\pgfusepath{fill}%
\end{pgfscope}%
\begin{pgfscope}%
\pgfpathrectangle{\pgfqpoint{3.776708in}{0.600000in}}{\pgfqpoint{2.573292in}{2.070576in}}%
\pgfusepath{clip}%
\pgfsetbuttcap%
\pgfsetmiterjoin%
\definecolor{currentfill}{rgb}{0.066899,0.263188,0.377594}%
\pgfsetfillcolor{currentfill}%
\pgfsetlinewidth{0.000000pt}%
\definecolor{currentstroke}{rgb}{0.000000,0.000000,0.000000}%
\pgfsetstrokecolor{currentstroke}%
\pgfsetstrokeopacity{0.000000}%
\pgfsetdash{}{0pt}%
\pgfpathmoveto{\pgfqpoint{4.353231in}{1.609196in}}%
\pgfpathlineto{\pgfqpoint{4.361985in}{1.609196in}}%
\pgfpathlineto{\pgfqpoint{4.361985in}{1.539783in}}%
\pgfpathlineto{\pgfqpoint{4.353231in}{1.539783in}}%
\pgfpathlineto{\pgfqpoint{4.353231in}{1.609196in}}%
\pgfpathclose%
\pgfusepath{fill}%
\end{pgfscope}%
\begin{pgfscope}%
\pgfpathrectangle{\pgfqpoint{3.776708in}{0.600000in}}{\pgfqpoint{2.573292in}{2.070576in}}%
\pgfusepath{clip}%
\pgfsetbuttcap%
\pgfsetmiterjoin%
\definecolor{currentfill}{rgb}{0.066899,0.263188,0.377594}%
\pgfsetfillcolor{currentfill}%
\pgfsetlinewidth{0.000000pt}%
\definecolor{currentstroke}{rgb}{0.000000,0.000000,0.000000}%
\pgfsetstrokecolor{currentstroke}%
\pgfsetstrokeopacity{0.000000}%
\pgfsetdash{}{0pt}%
\pgfpathmoveto{\pgfqpoint{4.364173in}{1.609196in}}%
\pgfpathlineto{\pgfqpoint{4.372926in}{1.609196in}}%
\pgfpathlineto{\pgfqpoint{4.372926in}{1.542506in}}%
\pgfpathlineto{\pgfqpoint{4.364173in}{1.542506in}}%
\pgfpathlineto{\pgfqpoint{4.364173in}{1.609196in}}%
\pgfpathclose%
\pgfusepath{fill}%
\end{pgfscope}%
\begin{pgfscope}%
\pgfpathrectangle{\pgfqpoint{3.776708in}{0.600000in}}{\pgfqpoint{2.573292in}{2.070576in}}%
\pgfusepath{clip}%
\pgfsetbuttcap%
\pgfsetmiterjoin%
\definecolor{currentfill}{rgb}{0.066899,0.263188,0.377594}%
\pgfsetfillcolor{currentfill}%
\pgfsetlinewidth{0.000000pt}%
\definecolor{currentstroke}{rgb}{0.000000,0.000000,0.000000}%
\pgfsetstrokecolor{currentstroke}%
\pgfsetstrokeopacity{0.000000}%
\pgfsetdash{}{0pt}%
\pgfpathmoveto{\pgfqpoint{4.375115in}{1.609196in}}%
\pgfpathlineto{\pgfqpoint{4.383868in}{1.609196in}}%
\pgfpathlineto{\pgfqpoint{4.383868in}{1.545446in}}%
\pgfpathlineto{\pgfqpoint{4.375115in}{1.545446in}}%
\pgfpathlineto{\pgfqpoint{4.375115in}{1.609196in}}%
\pgfpathclose%
\pgfusepath{fill}%
\end{pgfscope}%
\begin{pgfscope}%
\pgfpathrectangle{\pgfqpoint{3.776708in}{0.600000in}}{\pgfqpoint{2.573292in}{2.070576in}}%
\pgfusepath{clip}%
\pgfsetbuttcap%
\pgfsetmiterjoin%
\definecolor{currentfill}{rgb}{0.066899,0.263188,0.377594}%
\pgfsetfillcolor{currentfill}%
\pgfsetlinewidth{0.000000pt}%
\definecolor{currentstroke}{rgb}{0.000000,0.000000,0.000000}%
\pgfsetstrokecolor{currentstroke}%
\pgfsetstrokeopacity{0.000000}%
\pgfsetdash{}{0pt}%
\pgfpathmoveto{\pgfqpoint{4.386057in}{1.609196in}}%
\pgfpathlineto{\pgfqpoint{4.394810in}{1.609196in}}%
\pgfpathlineto{\pgfqpoint{4.394810in}{1.550014in}}%
\pgfpathlineto{\pgfqpoint{4.386057in}{1.550014in}}%
\pgfpathlineto{\pgfqpoint{4.386057in}{1.609196in}}%
\pgfpathclose%
\pgfusepath{fill}%
\end{pgfscope}%
\begin{pgfscope}%
\pgfpathrectangle{\pgfqpoint{3.776708in}{0.600000in}}{\pgfqpoint{2.573292in}{2.070576in}}%
\pgfusepath{clip}%
\pgfsetbuttcap%
\pgfsetmiterjoin%
\definecolor{currentfill}{rgb}{0.066899,0.263188,0.377594}%
\pgfsetfillcolor{currentfill}%
\pgfsetlinewidth{0.000000pt}%
\definecolor{currentstroke}{rgb}{0.000000,0.000000,0.000000}%
\pgfsetstrokecolor{currentstroke}%
\pgfsetstrokeopacity{0.000000}%
\pgfsetdash{}{0pt}%
\pgfpathmoveto{\pgfqpoint{4.396998in}{1.609196in}}%
\pgfpathlineto{\pgfqpoint{4.405752in}{1.609196in}}%
\pgfpathlineto{\pgfqpoint{4.405752in}{1.554298in}}%
\pgfpathlineto{\pgfqpoint{4.396998in}{1.554298in}}%
\pgfpathlineto{\pgfqpoint{4.396998in}{1.609196in}}%
\pgfpathclose%
\pgfusepath{fill}%
\end{pgfscope}%
\begin{pgfscope}%
\pgfpathrectangle{\pgfqpoint{3.776708in}{0.600000in}}{\pgfqpoint{2.573292in}{2.070576in}}%
\pgfusepath{clip}%
\pgfsetbuttcap%
\pgfsetmiterjoin%
\definecolor{currentfill}{rgb}{0.066899,0.263188,0.377594}%
\pgfsetfillcolor{currentfill}%
\pgfsetlinewidth{0.000000pt}%
\definecolor{currentstroke}{rgb}{0.000000,0.000000,0.000000}%
\pgfsetstrokecolor{currentstroke}%
\pgfsetstrokeopacity{0.000000}%
\pgfsetdash{}{0pt}%
\pgfpathmoveto{\pgfqpoint{4.407940in}{1.609196in}}%
\pgfpathlineto{\pgfqpoint{4.416694in}{1.609196in}}%
\pgfpathlineto{\pgfqpoint{4.416694in}{1.563504in}}%
\pgfpathlineto{\pgfqpoint{4.407940in}{1.563504in}}%
\pgfpathlineto{\pgfqpoint{4.407940in}{1.609196in}}%
\pgfpathclose%
\pgfusepath{fill}%
\end{pgfscope}%
\begin{pgfscope}%
\pgfpathrectangle{\pgfqpoint{3.776708in}{0.600000in}}{\pgfqpoint{2.573292in}{2.070576in}}%
\pgfusepath{clip}%
\pgfsetbuttcap%
\pgfsetmiterjoin%
\definecolor{currentfill}{rgb}{0.066899,0.263188,0.377594}%
\pgfsetfillcolor{currentfill}%
\pgfsetlinewidth{0.000000pt}%
\definecolor{currentstroke}{rgb}{0.000000,0.000000,0.000000}%
\pgfsetstrokecolor{currentstroke}%
\pgfsetstrokeopacity{0.000000}%
\pgfsetdash{}{0pt}%
\pgfpathmoveto{\pgfqpoint{4.418882in}{1.609196in}}%
\pgfpathlineto{\pgfqpoint{4.427635in}{1.609196in}}%
\pgfpathlineto{\pgfqpoint{4.427635in}{1.571360in}}%
\pgfpathlineto{\pgfqpoint{4.418882in}{1.571360in}}%
\pgfpathlineto{\pgfqpoint{4.418882in}{1.609196in}}%
\pgfpathclose%
\pgfusepath{fill}%
\end{pgfscope}%
\begin{pgfscope}%
\pgfpathrectangle{\pgfqpoint{3.776708in}{0.600000in}}{\pgfqpoint{2.573292in}{2.070576in}}%
\pgfusepath{clip}%
\pgfsetbuttcap%
\pgfsetmiterjoin%
\definecolor{currentfill}{rgb}{0.066899,0.263188,0.377594}%
\pgfsetfillcolor{currentfill}%
\pgfsetlinewidth{0.000000pt}%
\definecolor{currentstroke}{rgb}{0.000000,0.000000,0.000000}%
\pgfsetstrokecolor{currentstroke}%
\pgfsetstrokeopacity{0.000000}%
\pgfsetdash{}{0pt}%
\pgfpathmoveto{\pgfqpoint{4.429824in}{1.609196in}}%
\pgfpathlineto{\pgfqpoint{4.438577in}{1.609196in}}%
\pgfpathlineto{\pgfqpoint{4.438577in}{1.580604in}}%
\pgfpathlineto{\pgfqpoint{4.429824in}{1.580604in}}%
\pgfpathlineto{\pgfqpoint{4.429824in}{1.609196in}}%
\pgfpathclose%
\pgfusepath{fill}%
\end{pgfscope}%
\begin{pgfscope}%
\pgfpathrectangle{\pgfqpoint{3.776708in}{0.600000in}}{\pgfqpoint{2.573292in}{2.070576in}}%
\pgfusepath{clip}%
\pgfsetbuttcap%
\pgfsetmiterjoin%
\definecolor{currentfill}{rgb}{0.066899,0.263188,0.377594}%
\pgfsetfillcolor{currentfill}%
\pgfsetlinewidth{0.000000pt}%
\definecolor{currentstroke}{rgb}{0.000000,0.000000,0.000000}%
\pgfsetstrokecolor{currentstroke}%
\pgfsetstrokeopacity{0.000000}%
\pgfsetdash{}{0pt}%
\pgfpathmoveto{\pgfqpoint{4.440766in}{1.609196in}}%
\pgfpathlineto{\pgfqpoint{4.449519in}{1.609196in}}%
\pgfpathlineto{\pgfqpoint{4.449519in}{1.586667in}}%
\pgfpathlineto{\pgfqpoint{4.440766in}{1.586667in}}%
\pgfpathlineto{\pgfqpoint{4.440766in}{1.609196in}}%
\pgfpathclose%
\pgfusepath{fill}%
\end{pgfscope}%
\begin{pgfscope}%
\pgfpathrectangle{\pgfqpoint{3.776708in}{0.600000in}}{\pgfqpoint{2.573292in}{2.070576in}}%
\pgfusepath{clip}%
\pgfsetbuttcap%
\pgfsetmiterjoin%
\definecolor{currentfill}{rgb}{0.066899,0.263188,0.377594}%
\pgfsetfillcolor{currentfill}%
\pgfsetlinewidth{0.000000pt}%
\definecolor{currentstroke}{rgb}{0.000000,0.000000,0.000000}%
\pgfsetstrokecolor{currentstroke}%
\pgfsetstrokeopacity{0.000000}%
\pgfsetdash{}{0pt}%
\pgfpathmoveto{\pgfqpoint{4.451707in}{1.609196in}}%
\pgfpathlineto{\pgfqpoint{4.460461in}{1.609196in}}%
\pgfpathlineto{\pgfqpoint{4.460461in}{1.593226in}}%
\pgfpathlineto{\pgfqpoint{4.451707in}{1.593226in}}%
\pgfpathlineto{\pgfqpoint{4.451707in}{1.609196in}}%
\pgfpathclose%
\pgfusepath{fill}%
\end{pgfscope}%
\begin{pgfscope}%
\pgfpathrectangle{\pgfqpoint{3.776708in}{0.600000in}}{\pgfqpoint{2.573292in}{2.070576in}}%
\pgfusepath{clip}%
\pgfsetbuttcap%
\pgfsetmiterjoin%
\definecolor{currentfill}{rgb}{0.066899,0.263188,0.377594}%
\pgfsetfillcolor{currentfill}%
\pgfsetlinewidth{0.000000pt}%
\definecolor{currentstroke}{rgb}{0.000000,0.000000,0.000000}%
\pgfsetstrokecolor{currentstroke}%
\pgfsetstrokeopacity{0.000000}%
\pgfsetdash{}{0pt}%
\pgfpathmoveto{\pgfqpoint{4.462649in}{1.609196in}}%
\pgfpathlineto{\pgfqpoint{4.471403in}{1.609196in}}%
\pgfpathlineto{\pgfqpoint{4.471403in}{1.602463in}}%
\pgfpathlineto{\pgfqpoint{4.462649in}{1.602463in}}%
\pgfpathlineto{\pgfqpoint{4.462649in}{1.609196in}}%
\pgfpathclose%
\pgfusepath{fill}%
\end{pgfscope}%
\begin{pgfscope}%
\pgfpathrectangle{\pgfqpoint{3.776708in}{0.600000in}}{\pgfqpoint{2.573292in}{2.070576in}}%
\pgfusepath{clip}%
\pgfsetbuttcap%
\pgfsetmiterjoin%
\definecolor{currentfill}{rgb}{0.066899,0.263188,0.377594}%
\pgfsetfillcolor{currentfill}%
\pgfsetlinewidth{0.000000pt}%
\definecolor{currentstroke}{rgb}{0.000000,0.000000,0.000000}%
\pgfsetstrokecolor{currentstroke}%
\pgfsetstrokeopacity{0.000000}%
\pgfsetdash{}{0pt}%
\pgfpathmoveto{\pgfqpoint{4.473591in}{1.609196in}}%
\pgfpathlineto{\pgfqpoint{4.482344in}{1.609196in}}%
\pgfpathlineto{\pgfqpoint{4.482344in}{1.601184in}}%
\pgfpathlineto{\pgfqpoint{4.473591in}{1.601184in}}%
\pgfpathlineto{\pgfqpoint{4.473591in}{1.609196in}}%
\pgfpathclose%
\pgfusepath{fill}%
\end{pgfscope}%
\begin{pgfscope}%
\pgfpathrectangle{\pgfqpoint{3.776708in}{0.600000in}}{\pgfqpoint{2.573292in}{2.070576in}}%
\pgfusepath{clip}%
\pgfsetbuttcap%
\pgfsetmiterjoin%
\definecolor{currentfill}{rgb}{0.066899,0.263188,0.377594}%
\pgfsetfillcolor{currentfill}%
\pgfsetlinewidth{0.000000pt}%
\definecolor{currentstroke}{rgb}{0.000000,0.000000,0.000000}%
\pgfsetstrokecolor{currentstroke}%
\pgfsetstrokeopacity{0.000000}%
\pgfsetdash{}{0pt}%
\pgfpathmoveto{\pgfqpoint{4.484533in}{1.609196in}}%
\pgfpathlineto{\pgfqpoint{4.493286in}{1.609196in}}%
\pgfpathlineto{\pgfqpoint{4.493286in}{1.598017in}}%
\pgfpathlineto{\pgfqpoint{4.484533in}{1.598017in}}%
\pgfpathlineto{\pgfqpoint{4.484533in}{1.609196in}}%
\pgfpathclose%
\pgfusepath{fill}%
\end{pgfscope}%
\begin{pgfscope}%
\pgfpathrectangle{\pgfqpoint{3.776708in}{0.600000in}}{\pgfqpoint{2.573292in}{2.070576in}}%
\pgfusepath{clip}%
\pgfsetbuttcap%
\pgfsetmiterjoin%
\definecolor{currentfill}{rgb}{0.066899,0.263188,0.377594}%
\pgfsetfillcolor{currentfill}%
\pgfsetlinewidth{0.000000pt}%
\definecolor{currentstroke}{rgb}{0.000000,0.000000,0.000000}%
\pgfsetstrokecolor{currentstroke}%
\pgfsetstrokeopacity{0.000000}%
\pgfsetdash{}{0pt}%
\pgfpathmoveto{\pgfqpoint{4.495475in}{1.609196in}}%
\pgfpathlineto{\pgfqpoint{4.504228in}{1.609196in}}%
\pgfpathlineto{\pgfqpoint{4.504228in}{1.592277in}}%
\pgfpathlineto{\pgfqpoint{4.495475in}{1.592277in}}%
\pgfpathlineto{\pgfqpoint{4.495475in}{1.609196in}}%
\pgfpathclose%
\pgfusepath{fill}%
\end{pgfscope}%
\begin{pgfscope}%
\pgfpathrectangle{\pgfqpoint{3.776708in}{0.600000in}}{\pgfqpoint{2.573292in}{2.070576in}}%
\pgfusepath{clip}%
\pgfsetbuttcap%
\pgfsetmiterjoin%
\definecolor{currentfill}{rgb}{0.066899,0.263188,0.377594}%
\pgfsetfillcolor{currentfill}%
\pgfsetlinewidth{0.000000pt}%
\definecolor{currentstroke}{rgb}{0.000000,0.000000,0.000000}%
\pgfsetstrokecolor{currentstroke}%
\pgfsetstrokeopacity{0.000000}%
\pgfsetdash{}{0pt}%
\pgfpathmoveto{\pgfqpoint{4.506416in}{1.609196in}}%
\pgfpathlineto{\pgfqpoint{4.515170in}{1.609196in}}%
\pgfpathlineto{\pgfqpoint{4.515170in}{1.593854in}}%
\pgfpathlineto{\pgfqpoint{4.506416in}{1.593854in}}%
\pgfpathlineto{\pgfqpoint{4.506416in}{1.609196in}}%
\pgfpathclose%
\pgfusepath{fill}%
\end{pgfscope}%
\begin{pgfscope}%
\pgfpathrectangle{\pgfqpoint{3.776708in}{0.600000in}}{\pgfqpoint{2.573292in}{2.070576in}}%
\pgfusepath{clip}%
\pgfsetbuttcap%
\pgfsetmiterjoin%
\definecolor{currentfill}{rgb}{0.066899,0.263188,0.377594}%
\pgfsetfillcolor{currentfill}%
\pgfsetlinewidth{0.000000pt}%
\definecolor{currentstroke}{rgb}{0.000000,0.000000,0.000000}%
\pgfsetstrokecolor{currentstroke}%
\pgfsetstrokeopacity{0.000000}%
\pgfsetdash{}{0pt}%
\pgfpathmoveto{\pgfqpoint{4.517358in}{1.609196in}}%
\pgfpathlineto{\pgfqpoint{4.526112in}{1.609196in}}%
\pgfpathlineto{\pgfqpoint{4.526112in}{1.595584in}}%
\pgfpathlineto{\pgfqpoint{4.517358in}{1.595584in}}%
\pgfpathlineto{\pgfqpoint{4.517358in}{1.609196in}}%
\pgfpathclose%
\pgfusepath{fill}%
\end{pgfscope}%
\begin{pgfscope}%
\pgfpathrectangle{\pgfqpoint{3.776708in}{0.600000in}}{\pgfqpoint{2.573292in}{2.070576in}}%
\pgfusepath{clip}%
\pgfsetbuttcap%
\pgfsetmiterjoin%
\definecolor{currentfill}{rgb}{0.066899,0.263188,0.377594}%
\pgfsetfillcolor{currentfill}%
\pgfsetlinewidth{0.000000pt}%
\definecolor{currentstroke}{rgb}{0.000000,0.000000,0.000000}%
\pgfsetstrokecolor{currentstroke}%
\pgfsetstrokeopacity{0.000000}%
\pgfsetdash{}{0pt}%
\pgfpathmoveto{\pgfqpoint{4.528300in}{1.609196in}}%
\pgfpathlineto{\pgfqpoint{4.537053in}{1.609196in}}%
\pgfpathlineto{\pgfqpoint{4.537053in}{1.591604in}}%
\pgfpathlineto{\pgfqpoint{4.528300in}{1.591604in}}%
\pgfpathlineto{\pgfqpoint{4.528300in}{1.609196in}}%
\pgfpathclose%
\pgfusepath{fill}%
\end{pgfscope}%
\begin{pgfscope}%
\pgfpathrectangle{\pgfqpoint{3.776708in}{0.600000in}}{\pgfqpoint{2.573292in}{2.070576in}}%
\pgfusepath{clip}%
\pgfsetbuttcap%
\pgfsetmiterjoin%
\definecolor{currentfill}{rgb}{0.066899,0.263188,0.377594}%
\pgfsetfillcolor{currentfill}%
\pgfsetlinewidth{0.000000pt}%
\definecolor{currentstroke}{rgb}{0.000000,0.000000,0.000000}%
\pgfsetstrokecolor{currentstroke}%
\pgfsetstrokeopacity{0.000000}%
\pgfsetdash{}{0pt}%
\pgfpathmoveto{\pgfqpoint{4.539242in}{1.609196in}}%
\pgfpathlineto{\pgfqpoint{4.547995in}{1.609196in}}%
\pgfpathlineto{\pgfqpoint{4.547995in}{1.587656in}}%
\pgfpathlineto{\pgfqpoint{4.539242in}{1.587656in}}%
\pgfpathlineto{\pgfqpoint{4.539242in}{1.609196in}}%
\pgfpathclose%
\pgfusepath{fill}%
\end{pgfscope}%
\begin{pgfscope}%
\pgfpathrectangle{\pgfqpoint{3.776708in}{0.600000in}}{\pgfqpoint{2.573292in}{2.070576in}}%
\pgfusepath{clip}%
\pgfsetbuttcap%
\pgfsetmiterjoin%
\definecolor{currentfill}{rgb}{0.066899,0.263188,0.377594}%
\pgfsetfillcolor{currentfill}%
\pgfsetlinewidth{0.000000pt}%
\definecolor{currentstroke}{rgb}{0.000000,0.000000,0.000000}%
\pgfsetstrokecolor{currentstroke}%
\pgfsetstrokeopacity{0.000000}%
\pgfsetdash{}{0pt}%
\pgfpathmoveto{\pgfqpoint{4.550183in}{1.609196in}}%
\pgfpathlineto{\pgfqpoint{4.558937in}{1.609196in}}%
\pgfpathlineto{\pgfqpoint{4.558937in}{1.580438in}}%
\pgfpathlineto{\pgfqpoint{4.550183in}{1.580438in}}%
\pgfpathlineto{\pgfqpoint{4.550183in}{1.609196in}}%
\pgfpathclose%
\pgfusepath{fill}%
\end{pgfscope}%
\begin{pgfscope}%
\pgfpathrectangle{\pgfqpoint{3.776708in}{0.600000in}}{\pgfqpoint{2.573292in}{2.070576in}}%
\pgfusepath{clip}%
\pgfsetbuttcap%
\pgfsetmiterjoin%
\definecolor{currentfill}{rgb}{0.066899,0.263188,0.377594}%
\pgfsetfillcolor{currentfill}%
\pgfsetlinewidth{0.000000pt}%
\definecolor{currentstroke}{rgb}{0.000000,0.000000,0.000000}%
\pgfsetstrokecolor{currentstroke}%
\pgfsetstrokeopacity{0.000000}%
\pgfsetdash{}{0pt}%
\pgfpathmoveto{\pgfqpoint{4.561125in}{1.609196in}}%
\pgfpathlineto{\pgfqpoint{4.569879in}{1.609196in}}%
\pgfpathlineto{\pgfqpoint{4.569879in}{1.577748in}}%
\pgfpathlineto{\pgfqpoint{4.561125in}{1.577748in}}%
\pgfpathlineto{\pgfqpoint{4.561125in}{1.609196in}}%
\pgfpathclose%
\pgfusepath{fill}%
\end{pgfscope}%
\begin{pgfscope}%
\pgfpathrectangle{\pgfqpoint{3.776708in}{0.600000in}}{\pgfqpoint{2.573292in}{2.070576in}}%
\pgfusepath{clip}%
\pgfsetbuttcap%
\pgfsetmiterjoin%
\definecolor{currentfill}{rgb}{0.066899,0.263188,0.377594}%
\pgfsetfillcolor{currentfill}%
\pgfsetlinewidth{0.000000pt}%
\definecolor{currentstroke}{rgb}{0.000000,0.000000,0.000000}%
\pgfsetstrokecolor{currentstroke}%
\pgfsetstrokeopacity{0.000000}%
\pgfsetdash{}{0pt}%
\pgfpathmoveto{\pgfqpoint{4.572067in}{1.609196in}}%
\pgfpathlineto{\pgfqpoint{4.580821in}{1.609196in}}%
\pgfpathlineto{\pgfqpoint{4.580821in}{1.571933in}}%
\pgfpathlineto{\pgfqpoint{4.572067in}{1.571933in}}%
\pgfpathlineto{\pgfqpoint{4.572067in}{1.609196in}}%
\pgfpathclose%
\pgfusepath{fill}%
\end{pgfscope}%
\begin{pgfscope}%
\pgfpathrectangle{\pgfqpoint{3.776708in}{0.600000in}}{\pgfqpoint{2.573292in}{2.070576in}}%
\pgfusepath{clip}%
\pgfsetbuttcap%
\pgfsetmiterjoin%
\definecolor{currentfill}{rgb}{0.066899,0.263188,0.377594}%
\pgfsetfillcolor{currentfill}%
\pgfsetlinewidth{0.000000pt}%
\definecolor{currentstroke}{rgb}{0.000000,0.000000,0.000000}%
\pgfsetstrokecolor{currentstroke}%
\pgfsetstrokeopacity{0.000000}%
\pgfsetdash{}{0pt}%
\pgfpathmoveto{\pgfqpoint{4.583009in}{1.609196in}}%
\pgfpathlineto{\pgfqpoint{4.591762in}{1.609196in}}%
\pgfpathlineto{\pgfqpoint{4.591762in}{1.562177in}}%
\pgfpathlineto{\pgfqpoint{4.583009in}{1.562177in}}%
\pgfpathlineto{\pgfqpoint{4.583009in}{1.609196in}}%
\pgfpathclose%
\pgfusepath{fill}%
\end{pgfscope}%
\begin{pgfscope}%
\pgfpathrectangle{\pgfqpoint{3.776708in}{0.600000in}}{\pgfqpoint{2.573292in}{2.070576in}}%
\pgfusepath{clip}%
\pgfsetbuttcap%
\pgfsetmiterjoin%
\definecolor{currentfill}{rgb}{0.066899,0.263188,0.377594}%
\pgfsetfillcolor{currentfill}%
\pgfsetlinewidth{0.000000pt}%
\definecolor{currentstroke}{rgb}{0.000000,0.000000,0.000000}%
\pgfsetstrokecolor{currentstroke}%
\pgfsetstrokeopacity{0.000000}%
\pgfsetdash{}{0pt}%
\pgfpathmoveto{\pgfqpoint{4.593951in}{1.609196in}}%
\pgfpathlineto{\pgfqpoint{4.602704in}{1.609196in}}%
\pgfpathlineto{\pgfqpoint{4.602704in}{1.556860in}}%
\pgfpathlineto{\pgfqpoint{4.593951in}{1.556860in}}%
\pgfpathlineto{\pgfqpoint{4.593951in}{1.609196in}}%
\pgfpathclose%
\pgfusepath{fill}%
\end{pgfscope}%
\begin{pgfscope}%
\pgfpathrectangle{\pgfqpoint{3.776708in}{0.600000in}}{\pgfqpoint{2.573292in}{2.070576in}}%
\pgfusepath{clip}%
\pgfsetbuttcap%
\pgfsetmiterjoin%
\definecolor{currentfill}{rgb}{0.066899,0.263188,0.377594}%
\pgfsetfillcolor{currentfill}%
\pgfsetlinewidth{0.000000pt}%
\definecolor{currentstroke}{rgb}{0.000000,0.000000,0.000000}%
\pgfsetstrokecolor{currentstroke}%
\pgfsetstrokeopacity{0.000000}%
\pgfsetdash{}{0pt}%
\pgfpathmoveto{\pgfqpoint{4.604892in}{1.609196in}}%
\pgfpathlineto{\pgfqpoint{4.613646in}{1.609196in}}%
\pgfpathlineto{\pgfqpoint{4.613646in}{1.547654in}}%
\pgfpathlineto{\pgfqpoint{4.604892in}{1.547654in}}%
\pgfpathlineto{\pgfqpoint{4.604892in}{1.609196in}}%
\pgfpathclose%
\pgfusepath{fill}%
\end{pgfscope}%
\begin{pgfscope}%
\pgfpathrectangle{\pgfqpoint{3.776708in}{0.600000in}}{\pgfqpoint{2.573292in}{2.070576in}}%
\pgfusepath{clip}%
\pgfsetbuttcap%
\pgfsetmiterjoin%
\definecolor{currentfill}{rgb}{0.066899,0.263188,0.377594}%
\pgfsetfillcolor{currentfill}%
\pgfsetlinewidth{0.000000pt}%
\definecolor{currentstroke}{rgb}{0.000000,0.000000,0.000000}%
\pgfsetstrokecolor{currentstroke}%
\pgfsetstrokeopacity{0.000000}%
\pgfsetdash{}{0pt}%
\pgfpathmoveto{\pgfqpoint{4.615834in}{1.609196in}}%
\pgfpathlineto{\pgfqpoint{4.624588in}{1.609196in}}%
\pgfpathlineto{\pgfqpoint{4.624588in}{1.541289in}}%
\pgfpathlineto{\pgfqpoint{4.615834in}{1.541289in}}%
\pgfpathlineto{\pgfqpoint{4.615834in}{1.609196in}}%
\pgfpathclose%
\pgfusepath{fill}%
\end{pgfscope}%
\begin{pgfscope}%
\pgfpathrectangle{\pgfqpoint{3.776708in}{0.600000in}}{\pgfqpoint{2.573292in}{2.070576in}}%
\pgfusepath{clip}%
\pgfsetbuttcap%
\pgfsetmiterjoin%
\definecolor{currentfill}{rgb}{0.066899,0.263188,0.377594}%
\pgfsetfillcolor{currentfill}%
\pgfsetlinewidth{0.000000pt}%
\definecolor{currentstroke}{rgb}{0.000000,0.000000,0.000000}%
\pgfsetstrokecolor{currentstroke}%
\pgfsetstrokeopacity{0.000000}%
\pgfsetdash{}{0pt}%
\pgfpathmoveto{\pgfqpoint{4.626776in}{1.609196in}}%
\pgfpathlineto{\pgfqpoint{4.635530in}{1.609196in}}%
\pgfpathlineto{\pgfqpoint{4.635530in}{1.536245in}}%
\pgfpathlineto{\pgfqpoint{4.626776in}{1.536245in}}%
\pgfpathlineto{\pgfqpoint{4.626776in}{1.609196in}}%
\pgfpathclose%
\pgfusepath{fill}%
\end{pgfscope}%
\begin{pgfscope}%
\pgfpathrectangle{\pgfqpoint{3.776708in}{0.600000in}}{\pgfqpoint{2.573292in}{2.070576in}}%
\pgfusepath{clip}%
\pgfsetbuttcap%
\pgfsetmiterjoin%
\definecolor{currentfill}{rgb}{0.066899,0.263188,0.377594}%
\pgfsetfillcolor{currentfill}%
\pgfsetlinewidth{0.000000pt}%
\definecolor{currentstroke}{rgb}{0.000000,0.000000,0.000000}%
\pgfsetstrokecolor{currentstroke}%
\pgfsetstrokeopacity{0.000000}%
\pgfsetdash{}{0pt}%
\pgfpathmoveto{\pgfqpoint{4.637718in}{1.609196in}}%
\pgfpathlineto{\pgfqpoint{4.646471in}{1.609196in}}%
\pgfpathlineto{\pgfqpoint{4.646471in}{1.533329in}}%
\pgfpathlineto{\pgfqpoint{4.637718in}{1.533329in}}%
\pgfpathlineto{\pgfqpoint{4.637718in}{1.609196in}}%
\pgfpathclose%
\pgfusepath{fill}%
\end{pgfscope}%
\begin{pgfscope}%
\pgfpathrectangle{\pgfqpoint{3.776708in}{0.600000in}}{\pgfqpoint{2.573292in}{2.070576in}}%
\pgfusepath{clip}%
\pgfsetbuttcap%
\pgfsetmiterjoin%
\definecolor{currentfill}{rgb}{0.066899,0.263188,0.377594}%
\pgfsetfillcolor{currentfill}%
\pgfsetlinewidth{0.000000pt}%
\definecolor{currentstroke}{rgb}{0.000000,0.000000,0.000000}%
\pgfsetstrokecolor{currentstroke}%
\pgfsetstrokeopacity{0.000000}%
\pgfsetdash{}{0pt}%
\pgfpathmoveto{\pgfqpoint{4.648660in}{1.609196in}}%
\pgfpathlineto{\pgfqpoint{4.657413in}{1.609196in}}%
\pgfpathlineto{\pgfqpoint{4.657413in}{1.531793in}}%
\pgfpathlineto{\pgfqpoint{4.648660in}{1.531793in}}%
\pgfpathlineto{\pgfqpoint{4.648660in}{1.609196in}}%
\pgfpathclose%
\pgfusepath{fill}%
\end{pgfscope}%
\begin{pgfscope}%
\pgfpathrectangle{\pgfqpoint{3.776708in}{0.600000in}}{\pgfqpoint{2.573292in}{2.070576in}}%
\pgfusepath{clip}%
\pgfsetbuttcap%
\pgfsetmiterjoin%
\definecolor{currentfill}{rgb}{0.066899,0.263188,0.377594}%
\pgfsetfillcolor{currentfill}%
\pgfsetlinewidth{0.000000pt}%
\definecolor{currentstroke}{rgb}{0.000000,0.000000,0.000000}%
\pgfsetstrokecolor{currentstroke}%
\pgfsetstrokeopacity{0.000000}%
\pgfsetdash{}{0pt}%
\pgfpathmoveto{\pgfqpoint{4.659601in}{1.609196in}}%
\pgfpathlineto{\pgfqpoint{4.668355in}{1.609196in}}%
\pgfpathlineto{\pgfqpoint{4.668355in}{1.533087in}}%
\pgfpathlineto{\pgfqpoint{4.659601in}{1.533087in}}%
\pgfpathlineto{\pgfqpoint{4.659601in}{1.609196in}}%
\pgfpathclose%
\pgfusepath{fill}%
\end{pgfscope}%
\begin{pgfscope}%
\pgfpathrectangle{\pgfqpoint{3.776708in}{0.600000in}}{\pgfqpoint{2.573292in}{2.070576in}}%
\pgfusepath{clip}%
\pgfsetbuttcap%
\pgfsetmiterjoin%
\definecolor{currentfill}{rgb}{0.066899,0.263188,0.377594}%
\pgfsetfillcolor{currentfill}%
\pgfsetlinewidth{0.000000pt}%
\definecolor{currentstroke}{rgb}{0.000000,0.000000,0.000000}%
\pgfsetstrokecolor{currentstroke}%
\pgfsetstrokeopacity{0.000000}%
\pgfsetdash{}{0pt}%
\pgfpathmoveto{\pgfqpoint{4.670543in}{1.609196in}}%
\pgfpathlineto{\pgfqpoint{4.679297in}{1.609196in}}%
\pgfpathlineto{\pgfqpoint{4.679297in}{1.535824in}}%
\pgfpathlineto{\pgfqpoint{4.670543in}{1.535824in}}%
\pgfpathlineto{\pgfqpoint{4.670543in}{1.609196in}}%
\pgfpathclose%
\pgfusepath{fill}%
\end{pgfscope}%
\begin{pgfscope}%
\pgfpathrectangle{\pgfqpoint{3.776708in}{0.600000in}}{\pgfqpoint{2.573292in}{2.070576in}}%
\pgfusepath{clip}%
\pgfsetbuttcap%
\pgfsetmiterjoin%
\definecolor{currentfill}{rgb}{0.066899,0.263188,0.377594}%
\pgfsetfillcolor{currentfill}%
\pgfsetlinewidth{0.000000pt}%
\definecolor{currentstroke}{rgb}{0.000000,0.000000,0.000000}%
\pgfsetstrokecolor{currentstroke}%
\pgfsetstrokeopacity{0.000000}%
\pgfsetdash{}{0pt}%
\pgfpathmoveto{\pgfqpoint{4.681485in}{1.609196in}}%
\pgfpathlineto{\pgfqpoint{4.690239in}{1.609196in}}%
\pgfpathlineto{\pgfqpoint{4.690239in}{1.537798in}}%
\pgfpathlineto{\pgfqpoint{4.681485in}{1.537798in}}%
\pgfpathlineto{\pgfqpoint{4.681485in}{1.609196in}}%
\pgfpathclose%
\pgfusepath{fill}%
\end{pgfscope}%
\begin{pgfscope}%
\pgfpathrectangle{\pgfqpoint{3.776708in}{0.600000in}}{\pgfqpoint{2.573292in}{2.070576in}}%
\pgfusepath{clip}%
\pgfsetbuttcap%
\pgfsetmiterjoin%
\definecolor{currentfill}{rgb}{0.066899,0.263188,0.377594}%
\pgfsetfillcolor{currentfill}%
\pgfsetlinewidth{0.000000pt}%
\definecolor{currentstroke}{rgb}{0.000000,0.000000,0.000000}%
\pgfsetstrokecolor{currentstroke}%
\pgfsetstrokeopacity{0.000000}%
\pgfsetdash{}{0pt}%
\pgfpathmoveto{\pgfqpoint{4.692427in}{1.609196in}}%
\pgfpathlineto{\pgfqpoint{4.701180in}{1.609196in}}%
\pgfpathlineto{\pgfqpoint{4.701180in}{1.541176in}}%
\pgfpathlineto{\pgfqpoint{4.692427in}{1.541176in}}%
\pgfpathlineto{\pgfqpoint{4.692427in}{1.609196in}}%
\pgfpathclose%
\pgfusepath{fill}%
\end{pgfscope}%
\begin{pgfscope}%
\pgfpathrectangle{\pgfqpoint{3.776708in}{0.600000in}}{\pgfqpoint{2.573292in}{2.070576in}}%
\pgfusepath{clip}%
\pgfsetbuttcap%
\pgfsetmiterjoin%
\definecolor{currentfill}{rgb}{0.066899,0.263188,0.377594}%
\pgfsetfillcolor{currentfill}%
\pgfsetlinewidth{0.000000pt}%
\definecolor{currentstroke}{rgb}{0.000000,0.000000,0.000000}%
\pgfsetstrokecolor{currentstroke}%
\pgfsetstrokeopacity{0.000000}%
\pgfsetdash{}{0pt}%
\pgfpathmoveto{\pgfqpoint{4.703369in}{1.609196in}}%
\pgfpathlineto{\pgfqpoint{4.712122in}{1.609196in}}%
\pgfpathlineto{\pgfqpoint{4.712122in}{1.545449in}}%
\pgfpathlineto{\pgfqpoint{4.703369in}{1.545449in}}%
\pgfpathlineto{\pgfqpoint{4.703369in}{1.609196in}}%
\pgfpathclose%
\pgfusepath{fill}%
\end{pgfscope}%
\begin{pgfscope}%
\pgfpathrectangle{\pgfqpoint{3.776708in}{0.600000in}}{\pgfqpoint{2.573292in}{2.070576in}}%
\pgfusepath{clip}%
\pgfsetbuttcap%
\pgfsetmiterjoin%
\definecolor{currentfill}{rgb}{0.066899,0.263188,0.377594}%
\pgfsetfillcolor{currentfill}%
\pgfsetlinewidth{0.000000pt}%
\definecolor{currentstroke}{rgb}{0.000000,0.000000,0.000000}%
\pgfsetstrokecolor{currentstroke}%
\pgfsetstrokeopacity{0.000000}%
\pgfsetdash{}{0pt}%
\pgfpathmoveto{\pgfqpoint{4.714310in}{1.609196in}}%
\pgfpathlineto{\pgfqpoint{4.723064in}{1.609196in}}%
\pgfpathlineto{\pgfqpoint{4.723064in}{1.550454in}}%
\pgfpathlineto{\pgfqpoint{4.714310in}{1.550454in}}%
\pgfpathlineto{\pgfqpoint{4.714310in}{1.609196in}}%
\pgfpathclose%
\pgfusepath{fill}%
\end{pgfscope}%
\begin{pgfscope}%
\pgfpathrectangle{\pgfqpoint{3.776708in}{0.600000in}}{\pgfqpoint{2.573292in}{2.070576in}}%
\pgfusepath{clip}%
\pgfsetbuttcap%
\pgfsetmiterjoin%
\definecolor{currentfill}{rgb}{0.066899,0.263188,0.377594}%
\pgfsetfillcolor{currentfill}%
\pgfsetlinewidth{0.000000pt}%
\definecolor{currentstroke}{rgb}{0.000000,0.000000,0.000000}%
\pgfsetstrokecolor{currentstroke}%
\pgfsetstrokeopacity{0.000000}%
\pgfsetdash{}{0pt}%
\pgfpathmoveto{\pgfqpoint{4.725252in}{1.609196in}}%
\pgfpathlineto{\pgfqpoint{4.734006in}{1.609196in}}%
\pgfpathlineto{\pgfqpoint{4.734006in}{1.553773in}}%
\pgfpathlineto{\pgfqpoint{4.725252in}{1.553773in}}%
\pgfpathlineto{\pgfqpoint{4.725252in}{1.609196in}}%
\pgfpathclose%
\pgfusepath{fill}%
\end{pgfscope}%
\begin{pgfscope}%
\pgfpathrectangle{\pgfqpoint{3.776708in}{0.600000in}}{\pgfqpoint{2.573292in}{2.070576in}}%
\pgfusepath{clip}%
\pgfsetbuttcap%
\pgfsetmiterjoin%
\definecolor{currentfill}{rgb}{0.066899,0.263188,0.377594}%
\pgfsetfillcolor{currentfill}%
\pgfsetlinewidth{0.000000pt}%
\definecolor{currentstroke}{rgb}{0.000000,0.000000,0.000000}%
\pgfsetstrokecolor{currentstroke}%
\pgfsetstrokeopacity{0.000000}%
\pgfsetdash{}{0pt}%
\pgfpathmoveto{\pgfqpoint{4.736194in}{1.609196in}}%
\pgfpathlineto{\pgfqpoint{4.744948in}{1.609196in}}%
\pgfpathlineto{\pgfqpoint{4.744948in}{1.557758in}}%
\pgfpathlineto{\pgfqpoint{4.736194in}{1.557758in}}%
\pgfpathlineto{\pgfqpoint{4.736194in}{1.609196in}}%
\pgfpathclose%
\pgfusepath{fill}%
\end{pgfscope}%
\begin{pgfscope}%
\pgfpathrectangle{\pgfqpoint{3.776708in}{0.600000in}}{\pgfqpoint{2.573292in}{2.070576in}}%
\pgfusepath{clip}%
\pgfsetbuttcap%
\pgfsetmiterjoin%
\definecolor{currentfill}{rgb}{0.066899,0.263188,0.377594}%
\pgfsetfillcolor{currentfill}%
\pgfsetlinewidth{0.000000pt}%
\definecolor{currentstroke}{rgb}{0.000000,0.000000,0.000000}%
\pgfsetstrokecolor{currentstroke}%
\pgfsetstrokeopacity{0.000000}%
\pgfsetdash{}{0pt}%
\pgfpathmoveto{\pgfqpoint{4.747136in}{1.609196in}}%
\pgfpathlineto{\pgfqpoint{4.755889in}{1.609196in}}%
\pgfpathlineto{\pgfqpoint{4.755889in}{1.557655in}}%
\pgfpathlineto{\pgfqpoint{4.747136in}{1.557655in}}%
\pgfpathlineto{\pgfqpoint{4.747136in}{1.609196in}}%
\pgfpathclose%
\pgfusepath{fill}%
\end{pgfscope}%
\begin{pgfscope}%
\pgfpathrectangle{\pgfqpoint{3.776708in}{0.600000in}}{\pgfqpoint{2.573292in}{2.070576in}}%
\pgfusepath{clip}%
\pgfsetbuttcap%
\pgfsetmiterjoin%
\definecolor{currentfill}{rgb}{0.066899,0.263188,0.377594}%
\pgfsetfillcolor{currentfill}%
\pgfsetlinewidth{0.000000pt}%
\definecolor{currentstroke}{rgb}{0.000000,0.000000,0.000000}%
\pgfsetstrokecolor{currentstroke}%
\pgfsetstrokeopacity{0.000000}%
\pgfsetdash{}{0pt}%
\pgfpathmoveto{\pgfqpoint{4.758078in}{1.609196in}}%
\pgfpathlineto{\pgfqpoint{4.766831in}{1.609196in}}%
\pgfpathlineto{\pgfqpoint{4.766831in}{1.558429in}}%
\pgfpathlineto{\pgfqpoint{4.758078in}{1.558429in}}%
\pgfpathlineto{\pgfqpoint{4.758078in}{1.609196in}}%
\pgfpathclose%
\pgfusepath{fill}%
\end{pgfscope}%
\begin{pgfscope}%
\pgfpathrectangle{\pgfqpoint{3.776708in}{0.600000in}}{\pgfqpoint{2.573292in}{2.070576in}}%
\pgfusepath{clip}%
\pgfsetbuttcap%
\pgfsetmiterjoin%
\definecolor{currentfill}{rgb}{0.066899,0.263188,0.377594}%
\pgfsetfillcolor{currentfill}%
\pgfsetlinewidth{0.000000pt}%
\definecolor{currentstroke}{rgb}{0.000000,0.000000,0.000000}%
\pgfsetstrokecolor{currentstroke}%
\pgfsetstrokeopacity{0.000000}%
\pgfsetdash{}{0pt}%
\pgfpathmoveto{\pgfqpoint{4.769019in}{1.609196in}}%
\pgfpathlineto{\pgfqpoint{4.777773in}{1.609196in}}%
\pgfpathlineto{\pgfqpoint{4.777773in}{1.559746in}}%
\pgfpathlineto{\pgfqpoint{4.769019in}{1.559746in}}%
\pgfpathlineto{\pgfqpoint{4.769019in}{1.609196in}}%
\pgfpathclose%
\pgfusepath{fill}%
\end{pgfscope}%
\begin{pgfscope}%
\pgfpathrectangle{\pgfqpoint{3.776708in}{0.600000in}}{\pgfqpoint{2.573292in}{2.070576in}}%
\pgfusepath{clip}%
\pgfsetbuttcap%
\pgfsetmiterjoin%
\definecolor{currentfill}{rgb}{0.066899,0.263188,0.377594}%
\pgfsetfillcolor{currentfill}%
\pgfsetlinewidth{0.000000pt}%
\definecolor{currentstroke}{rgb}{0.000000,0.000000,0.000000}%
\pgfsetstrokecolor{currentstroke}%
\pgfsetstrokeopacity{0.000000}%
\pgfsetdash{}{0pt}%
\pgfpathmoveto{\pgfqpoint{4.779961in}{1.609196in}}%
\pgfpathlineto{\pgfqpoint{4.788715in}{1.609196in}}%
\pgfpathlineto{\pgfqpoint{4.788715in}{1.562972in}}%
\pgfpathlineto{\pgfqpoint{4.779961in}{1.562972in}}%
\pgfpathlineto{\pgfqpoint{4.779961in}{1.609196in}}%
\pgfpathclose%
\pgfusepath{fill}%
\end{pgfscope}%
\begin{pgfscope}%
\pgfpathrectangle{\pgfqpoint{3.776708in}{0.600000in}}{\pgfqpoint{2.573292in}{2.070576in}}%
\pgfusepath{clip}%
\pgfsetbuttcap%
\pgfsetmiterjoin%
\definecolor{currentfill}{rgb}{0.066899,0.263188,0.377594}%
\pgfsetfillcolor{currentfill}%
\pgfsetlinewidth{0.000000pt}%
\definecolor{currentstroke}{rgb}{0.000000,0.000000,0.000000}%
\pgfsetstrokecolor{currentstroke}%
\pgfsetstrokeopacity{0.000000}%
\pgfsetdash{}{0pt}%
\pgfpathmoveto{\pgfqpoint{4.790903in}{1.609196in}}%
\pgfpathlineto{\pgfqpoint{4.799657in}{1.609196in}}%
\pgfpathlineto{\pgfqpoint{4.799657in}{1.569702in}}%
\pgfpathlineto{\pgfqpoint{4.790903in}{1.569702in}}%
\pgfpathlineto{\pgfqpoint{4.790903in}{1.609196in}}%
\pgfpathclose%
\pgfusepath{fill}%
\end{pgfscope}%
\begin{pgfscope}%
\pgfpathrectangle{\pgfqpoint{3.776708in}{0.600000in}}{\pgfqpoint{2.573292in}{2.070576in}}%
\pgfusepath{clip}%
\pgfsetbuttcap%
\pgfsetmiterjoin%
\definecolor{currentfill}{rgb}{0.066899,0.263188,0.377594}%
\pgfsetfillcolor{currentfill}%
\pgfsetlinewidth{0.000000pt}%
\definecolor{currentstroke}{rgb}{0.000000,0.000000,0.000000}%
\pgfsetstrokecolor{currentstroke}%
\pgfsetstrokeopacity{0.000000}%
\pgfsetdash{}{0pt}%
\pgfpathmoveto{\pgfqpoint{4.801845in}{1.609196in}}%
\pgfpathlineto{\pgfqpoint{4.810598in}{1.609196in}}%
\pgfpathlineto{\pgfqpoint{4.810598in}{1.573438in}}%
\pgfpathlineto{\pgfqpoint{4.801845in}{1.573438in}}%
\pgfpathlineto{\pgfqpoint{4.801845in}{1.609196in}}%
\pgfpathclose%
\pgfusepath{fill}%
\end{pgfscope}%
\begin{pgfscope}%
\pgfpathrectangle{\pgfqpoint{3.776708in}{0.600000in}}{\pgfqpoint{2.573292in}{2.070576in}}%
\pgfusepath{clip}%
\pgfsetbuttcap%
\pgfsetmiterjoin%
\definecolor{currentfill}{rgb}{0.066899,0.263188,0.377594}%
\pgfsetfillcolor{currentfill}%
\pgfsetlinewidth{0.000000pt}%
\definecolor{currentstroke}{rgb}{0.000000,0.000000,0.000000}%
\pgfsetstrokecolor{currentstroke}%
\pgfsetstrokeopacity{0.000000}%
\pgfsetdash{}{0pt}%
\pgfpathmoveto{\pgfqpoint{4.812787in}{1.609196in}}%
\pgfpathlineto{\pgfqpoint{4.821540in}{1.609196in}}%
\pgfpathlineto{\pgfqpoint{4.821540in}{1.576240in}}%
\pgfpathlineto{\pgfqpoint{4.812787in}{1.576240in}}%
\pgfpathlineto{\pgfqpoint{4.812787in}{1.609196in}}%
\pgfpathclose%
\pgfusepath{fill}%
\end{pgfscope}%
\begin{pgfscope}%
\pgfpathrectangle{\pgfqpoint{3.776708in}{0.600000in}}{\pgfqpoint{2.573292in}{2.070576in}}%
\pgfusepath{clip}%
\pgfsetbuttcap%
\pgfsetmiterjoin%
\definecolor{currentfill}{rgb}{0.066899,0.263188,0.377594}%
\pgfsetfillcolor{currentfill}%
\pgfsetlinewidth{0.000000pt}%
\definecolor{currentstroke}{rgb}{0.000000,0.000000,0.000000}%
\pgfsetstrokecolor{currentstroke}%
\pgfsetstrokeopacity{0.000000}%
\pgfsetdash{}{0pt}%
\pgfpathmoveto{\pgfqpoint{4.823728in}{1.609196in}}%
\pgfpathlineto{\pgfqpoint{4.832482in}{1.609196in}}%
\pgfpathlineto{\pgfqpoint{4.832482in}{1.583277in}}%
\pgfpathlineto{\pgfqpoint{4.823728in}{1.583277in}}%
\pgfpathlineto{\pgfqpoint{4.823728in}{1.609196in}}%
\pgfpathclose%
\pgfusepath{fill}%
\end{pgfscope}%
\begin{pgfscope}%
\pgfpathrectangle{\pgfqpoint{3.776708in}{0.600000in}}{\pgfqpoint{2.573292in}{2.070576in}}%
\pgfusepath{clip}%
\pgfsetbuttcap%
\pgfsetmiterjoin%
\definecolor{currentfill}{rgb}{0.066899,0.263188,0.377594}%
\pgfsetfillcolor{currentfill}%
\pgfsetlinewidth{0.000000pt}%
\definecolor{currentstroke}{rgb}{0.000000,0.000000,0.000000}%
\pgfsetstrokecolor{currentstroke}%
\pgfsetstrokeopacity{0.000000}%
\pgfsetdash{}{0pt}%
\pgfpathmoveto{\pgfqpoint{4.834670in}{1.609196in}}%
\pgfpathlineto{\pgfqpoint{4.843424in}{1.609196in}}%
\pgfpathlineto{\pgfqpoint{4.843424in}{1.586731in}}%
\pgfpathlineto{\pgfqpoint{4.834670in}{1.586731in}}%
\pgfpathlineto{\pgfqpoint{4.834670in}{1.609196in}}%
\pgfpathclose%
\pgfusepath{fill}%
\end{pgfscope}%
\begin{pgfscope}%
\pgfpathrectangle{\pgfqpoint{3.776708in}{0.600000in}}{\pgfqpoint{2.573292in}{2.070576in}}%
\pgfusepath{clip}%
\pgfsetbuttcap%
\pgfsetmiterjoin%
\definecolor{currentfill}{rgb}{0.066899,0.263188,0.377594}%
\pgfsetfillcolor{currentfill}%
\pgfsetlinewidth{0.000000pt}%
\definecolor{currentstroke}{rgb}{0.000000,0.000000,0.000000}%
\pgfsetstrokecolor{currentstroke}%
\pgfsetstrokeopacity{0.000000}%
\pgfsetdash{}{0pt}%
\pgfpathmoveto{\pgfqpoint{4.845612in}{1.609196in}}%
\pgfpathlineto{\pgfqpoint{4.854366in}{1.609196in}}%
\pgfpathlineto{\pgfqpoint{4.854366in}{1.593550in}}%
\pgfpathlineto{\pgfqpoint{4.845612in}{1.593550in}}%
\pgfpathlineto{\pgfqpoint{4.845612in}{1.609196in}}%
\pgfpathclose%
\pgfusepath{fill}%
\end{pgfscope}%
\begin{pgfscope}%
\pgfpathrectangle{\pgfqpoint{3.776708in}{0.600000in}}{\pgfqpoint{2.573292in}{2.070576in}}%
\pgfusepath{clip}%
\pgfsetbuttcap%
\pgfsetmiterjoin%
\definecolor{currentfill}{rgb}{0.066899,0.263188,0.377594}%
\pgfsetfillcolor{currentfill}%
\pgfsetlinewidth{0.000000pt}%
\definecolor{currentstroke}{rgb}{0.000000,0.000000,0.000000}%
\pgfsetstrokecolor{currentstroke}%
\pgfsetstrokeopacity{0.000000}%
\pgfsetdash{}{0pt}%
\pgfpathmoveto{\pgfqpoint{4.856554in}{1.609196in}}%
\pgfpathlineto{\pgfqpoint{4.865307in}{1.609196in}}%
\pgfpathlineto{\pgfqpoint{4.865307in}{1.598915in}}%
\pgfpathlineto{\pgfqpoint{4.856554in}{1.598915in}}%
\pgfpathlineto{\pgfqpoint{4.856554in}{1.609196in}}%
\pgfpathclose%
\pgfusepath{fill}%
\end{pgfscope}%
\begin{pgfscope}%
\pgfpathrectangle{\pgfqpoint{3.776708in}{0.600000in}}{\pgfqpoint{2.573292in}{2.070576in}}%
\pgfusepath{clip}%
\pgfsetbuttcap%
\pgfsetmiterjoin%
\definecolor{currentfill}{rgb}{0.066899,0.263188,0.377594}%
\pgfsetfillcolor{currentfill}%
\pgfsetlinewidth{0.000000pt}%
\definecolor{currentstroke}{rgb}{0.000000,0.000000,0.000000}%
\pgfsetstrokecolor{currentstroke}%
\pgfsetstrokeopacity{0.000000}%
\pgfsetdash{}{0pt}%
\pgfpathmoveto{\pgfqpoint{4.867496in}{1.609196in}}%
\pgfpathlineto{\pgfqpoint{4.876249in}{1.609196in}}%
\pgfpathlineto{\pgfqpoint{4.876249in}{1.604175in}}%
\pgfpathlineto{\pgfqpoint{4.867496in}{1.604175in}}%
\pgfpathlineto{\pgfqpoint{4.867496in}{1.609196in}}%
\pgfpathclose%
\pgfusepath{fill}%
\end{pgfscope}%
\begin{pgfscope}%
\pgfpathrectangle{\pgfqpoint{3.776708in}{0.600000in}}{\pgfqpoint{2.573292in}{2.070576in}}%
\pgfusepath{clip}%
\pgfsetbuttcap%
\pgfsetmiterjoin%
\definecolor{currentfill}{rgb}{0.066899,0.263188,0.377594}%
\pgfsetfillcolor{currentfill}%
\pgfsetlinewidth{0.000000pt}%
\definecolor{currentstroke}{rgb}{0.000000,0.000000,0.000000}%
\pgfsetstrokecolor{currentstroke}%
\pgfsetstrokeopacity{0.000000}%
\pgfsetdash{}{0pt}%
\pgfpathmoveto{\pgfqpoint{4.878437in}{1.609196in}}%
\pgfpathlineto{\pgfqpoint{4.887191in}{1.609196in}}%
\pgfpathlineto{\pgfqpoint{4.887191in}{1.608819in}}%
\pgfpathlineto{\pgfqpoint{4.878437in}{1.608819in}}%
\pgfpathlineto{\pgfqpoint{4.878437in}{1.609196in}}%
\pgfpathclose%
\pgfusepath{fill}%
\end{pgfscope}%
\begin{pgfscope}%
\pgfpathrectangle{\pgfqpoint{3.776708in}{0.600000in}}{\pgfqpoint{2.573292in}{2.070576in}}%
\pgfusepath{clip}%
\pgfsetbuttcap%
\pgfsetmiterjoin%
\definecolor{currentfill}{rgb}{0.066899,0.263188,0.377594}%
\pgfsetfillcolor{currentfill}%
\pgfsetlinewidth{0.000000pt}%
\definecolor{currentstroke}{rgb}{0.000000,0.000000,0.000000}%
\pgfsetstrokecolor{currentstroke}%
\pgfsetstrokeopacity{0.000000}%
\pgfsetdash{}{0pt}%
\pgfpathmoveto{\pgfqpoint{4.889379in}{1.609196in}}%
\pgfpathlineto{\pgfqpoint{4.898133in}{1.609196in}}%
\pgfpathlineto{\pgfqpoint{4.898133in}{1.613020in}}%
\pgfpathlineto{\pgfqpoint{4.889379in}{1.613020in}}%
\pgfpathlineto{\pgfqpoint{4.889379in}{1.609196in}}%
\pgfpathclose%
\pgfusepath{fill}%
\end{pgfscope}%
\begin{pgfscope}%
\pgfpathrectangle{\pgfqpoint{3.776708in}{0.600000in}}{\pgfqpoint{2.573292in}{2.070576in}}%
\pgfusepath{clip}%
\pgfsetbuttcap%
\pgfsetmiterjoin%
\definecolor{currentfill}{rgb}{0.066899,0.263188,0.377594}%
\pgfsetfillcolor{currentfill}%
\pgfsetlinewidth{0.000000pt}%
\definecolor{currentstroke}{rgb}{0.000000,0.000000,0.000000}%
\pgfsetstrokecolor{currentstroke}%
\pgfsetstrokeopacity{0.000000}%
\pgfsetdash{}{0pt}%
\pgfpathmoveto{\pgfqpoint{4.900321in}{1.609196in}}%
\pgfpathlineto{\pgfqpoint{4.909075in}{1.609196in}}%
\pgfpathlineto{\pgfqpoint{4.909075in}{1.616469in}}%
\pgfpathlineto{\pgfqpoint{4.900321in}{1.616469in}}%
\pgfpathlineto{\pgfqpoint{4.900321in}{1.609196in}}%
\pgfpathclose%
\pgfusepath{fill}%
\end{pgfscope}%
\begin{pgfscope}%
\pgfpathrectangle{\pgfqpoint{3.776708in}{0.600000in}}{\pgfqpoint{2.573292in}{2.070576in}}%
\pgfusepath{clip}%
\pgfsetbuttcap%
\pgfsetmiterjoin%
\definecolor{currentfill}{rgb}{0.066899,0.263188,0.377594}%
\pgfsetfillcolor{currentfill}%
\pgfsetlinewidth{0.000000pt}%
\definecolor{currentstroke}{rgb}{0.000000,0.000000,0.000000}%
\pgfsetstrokecolor{currentstroke}%
\pgfsetstrokeopacity{0.000000}%
\pgfsetdash{}{0pt}%
\pgfpathmoveto{\pgfqpoint{4.911263in}{1.609196in}}%
\pgfpathlineto{\pgfqpoint{4.920016in}{1.609196in}}%
\pgfpathlineto{\pgfqpoint{4.920016in}{1.620128in}}%
\pgfpathlineto{\pgfqpoint{4.911263in}{1.620128in}}%
\pgfpathlineto{\pgfqpoint{4.911263in}{1.609196in}}%
\pgfpathclose%
\pgfusepath{fill}%
\end{pgfscope}%
\begin{pgfscope}%
\pgfpathrectangle{\pgfqpoint{3.776708in}{0.600000in}}{\pgfqpoint{2.573292in}{2.070576in}}%
\pgfusepath{clip}%
\pgfsetbuttcap%
\pgfsetmiterjoin%
\definecolor{currentfill}{rgb}{0.066899,0.263188,0.377594}%
\pgfsetfillcolor{currentfill}%
\pgfsetlinewidth{0.000000pt}%
\definecolor{currentstroke}{rgb}{0.000000,0.000000,0.000000}%
\pgfsetstrokecolor{currentstroke}%
\pgfsetstrokeopacity{0.000000}%
\pgfsetdash{}{0pt}%
\pgfpathmoveto{\pgfqpoint{4.922205in}{1.609196in}}%
\pgfpathlineto{\pgfqpoint{4.930958in}{1.609196in}}%
\pgfpathlineto{\pgfqpoint{4.930958in}{1.624685in}}%
\pgfpathlineto{\pgfqpoint{4.922205in}{1.624685in}}%
\pgfpathlineto{\pgfqpoint{4.922205in}{1.609196in}}%
\pgfpathclose%
\pgfusepath{fill}%
\end{pgfscope}%
\begin{pgfscope}%
\pgfpathrectangle{\pgfqpoint{3.776708in}{0.600000in}}{\pgfqpoint{2.573292in}{2.070576in}}%
\pgfusepath{clip}%
\pgfsetbuttcap%
\pgfsetmiterjoin%
\definecolor{currentfill}{rgb}{0.066899,0.263188,0.377594}%
\pgfsetfillcolor{currentfill}%
\pgfsetlinewidth{0.000000pt}%
\definecolor{currentstroke}{rgb}{0.000000,0.000000,0.000000}%
\pgfsetstrokecolor{currentstroke}%
\pgfsetstrokeopacity{0.000000}%
\pgfsetdash{}{0pt}%
\pgfpathmoveto{\pgfqpoint{4.933146in}{1.609196in}}%
\pgfpathlineto{\pgfqpoint{4.941900in}{1.609196in}}%
\pgfpathlineto{\pgfqpoint{4.941900in}{1.629001in}}%
\pgfpathlineto{\pgfqpoint{4.933146in}{1.629001in}}%
\pgfpathlineto{\pgfqpoint{4.933146in}{1.609196in}}%
\pgfpathclose%
\pgfusepath{fill}%
\end{pgfscope}%
\begin{pgfscope}%
\pgfpathrectangle{\pgfqpoint{3.776708in}{0.600000in}}{\pgfqpoint{2.573292in}{2.070576in}}%
\pgfusepath{clip}%
\pgfsetbuttcap%
\pgfsetmiterjoin%
\definecolor{currentfill}{rgb}{0.066899,0.263188,0.377594}%
\pgfsetfillcolor{currentfill}%
\pgfsetlinewidth{0.000000pt}%
\definecolor{currentstroke}{rgb}{0.000000,0.000000,0.000000}%
\pgfsetstrokecolor{currentstroke}%
\pgfsetstrokeopacity{0.000000}%
\pgfsetdash{}{0pt}%
\pgfpathmoveto{\pgfqpoint{4.944088in}{1.609196in}}%
\pgfpathlineto{\pgfqpoint{4.952842in}{1.609196in}}%
\pgfpathlineto{\pgfqpoint{4.952842in}{1.630721in}}%
\pgfpathlineto{\pgfqpoint{4.944088in}{1.630721in}}%
\pgfpathlineto{\pgfqpoint{4.944088in}{1.609196in}}%
\pgfpathclose%
\pgfusepath{fill}%
\end{pgfscope}%
\begin{pgfscope}%
\pgfpathrectangle{\pgfqpoint{3.776708in}{0.600000in}}{\pgfqpoint{2.573292in}{2.070576in}}%
\pgfusepath{clip}%
\pgfsetbuttcap%
\pgfsetmiterjoin%
\definecolor{currentfill}{rgb}{0.066899,0.263188,0.377594}%
\pgfsetfillcolor{currentfill}%
\pgfsetlinewidth{0.000000pt}%
\definecolor{currentstroke}{rgb}{0.000000,0.000000,0.000000}%
\pgfsetstrokecolor{currentstroke}%
\pgfsetstrokeopacity{0.000000}%
\pgfsetdash{}{0pt}%
\pgfpathmoveto{\pgfqpoint{4.955030in}{1.609196in}}%
\pgfpathlineto{\pgfqpoint{4.963783in}{1.609196in}}%
\pgfpathlineto{\pgfqpoint{4.963783in}{1.629178in}}%
\pgfpathlineto{\pgfqpoint{4.955030in}{1.629178in}}%
\pgfpathlineto{\pgfqpoint{4.955030in}{1.609196in}}%
\pgfpathclose%
\pgfusepath{fill}%
\end{pgfscope}%
\begin{pgfscope}%
\pgfpathrectangle{\pgfqpoint{3.776708in}{0.600000in}}{\pgfqpoint{2.573292in}{2.070576in}}%
\pgfusepath{clip}%
\pgfsetbuttcap%
\pgfsetmiterjoin%
\definecolor{currentfill}{rgb}{0.066899,0.263188,0.377594}%
\pgfsetfillcolor{currentfill}%
\pgfsetlinewidth{0.000000pt}%
\definecolor{currentstroke}{rgb}{0.000000,0.000000,0.000000}%
\pgfsetstrokecolor{currentstroke}%
\pgfsetstrokeopacity{0.000000}%
\pgfsetdash{}{0pt}%
\pgfpathmoveto{\pgfqpoint{4.965972in}{1.609196in}}%
\pgfpathlineto{\pgfqpoint{4.974725in}{1.609196in}}%
\pgfpathlineto{\pgfqpoint{4.974725in}{1.630679in}}%
\pgfpathlineto{\pgfqpoint{4.965972in}{1.630679in}}%
\pgfpathlineto{\pgfqpoint{4.965972in}{1.609196in}}%
\pgfpathclose%
\pgfusepath{fill}%
\end{pgfscope}%
\begin{pgfscope}%
\pgfpathrectangle{\pgfqpoint{3.776708in}{0.600000in}}{\pgfqpoint{2.573292in}{2.070576in}}%
\pgfusepath{clip}%
\pgfsetbuttcap%
\pgfsetmiterjoin%
\definecolor{currentfill}{rgb}{0.066899,0.263188,0.377594}%
\pgfsetfillcolor{currentfill}%
\pgfsetlinewidth{0.000000pt}%
\definecolor{currentstroke}{rgb}{0.000000,0.000000,0.000000}%
\pgfsetstrokecolor{currentstroke}%
\pgfsetstrokeopacity{0.000000}%
\pgfsetdash{}{0pt}%
\pgfpathmoveto{\pgfqpoint{4.976914in}{1.609196in}}%
\pgfpathlineto{\pgfqpoint{4.985667in}{1.609196in}}%
\pgfpathlineto{\pgfqpoint{4.985667in}{1.631213in}}%
\pgfpathlineto{\pgfqpoint{4.976914in}{1.631213in}}%
\pgfpathlineto{\pgfqpoint{4.976914in}{1.609196in}}%
\pgfpathclose%
\pgfusepath{fill}%
\end{pgfscope}%
\begin{pgfscope}%
\pgfpathrectangle{\pgfqpoint{3.776708in}{0.600000in}}{\pgfqpoint{2.573292in}{2.070576in}}%
\pgfusepath{clip}%
\pgfsetbuttcap%
\pgfsetmiterjoin%
\definecolor{currentfill}{rgb}{0.066899,0.263188,0.377594}%
\pgfsetfillcolor{currentfill}%
\pgfsetlinewidth{0.000000pt}%
\definecolor{currentstroke}{rgb}{0.000000,0.000000,0.000000}%
\pgfsetstrokecolor{currentstroke}%
\pgfsetstrokeopacity{0.000000}%
\pgfsetdash{}{0pt}%
\pgfpathmoveto{\pgfqpoint{4.987855in}{1.609196in}}%
\pgfpathlineto{\pgfqpoint{4.996609in}{1.609196in}}%
\pgfpathlineto{\pgfqpoint{4.996609in}{1.631212in}}%
\pgfpathlineto{\pgfqpoint{4.987855in}{1.631212in}}%
\pgfpathlineto{\pgfqpoint{4.987855in}{1.609196in}}%
\pgfpathclose%
\pgfusepath{fill}%
\end{pgfscope}%
\begin{pgfscope}%
\pgfpathrectangle{\pgfqpoint{3.776708in}{0.600000in}}{\pgfqpoint{2.573292in}{2.070576in}}%
\pgfusepath{clip}%
\pgfsetbuttcap%
\pgfsetmiterjoin%
\definecolor{currentfill}{rgb}{0.066899,0.263188,0.377594}%
\pgfsetfillcolor{currentfill}%
\pgfsetlinewidth{0.000000pt}%
\definecolor{currentstroke}{rgb}{0.000000,0.000000,0.000000}%
\pgfsetstrokecolor{currentstroke}%
\pgfsetstrokeopacity{0.000000}%
\pgfsetdash{}{0pt}%
\pgfpathmoveto{\pgfqpoint{4.998797in}{1.609196in}}%
\pgfpathlineto{\pgfqpoint{5.007551in}{1.609196in}}%
\pgfpathlineto{\pgfqpoint{5.007551in}{1.631439in}}%
\pgfpathlineto{\pgfqpoint{4.998797in}{1.631439in}}%
\pgfpathlineto{\pgfqpoint{4.998797in}{1.609196in}}%
\pgfpathclose%
\pgfusepath{fill}%
\end{pgfscope}%
\begin{pgfscope}%
\pgfpathrectangle{\pgfqpoint{3.776708in}{0.600000in}}{\pgfqpoint{2.573292in}{2.070576in}}%
\pgfusepath{clip}%
\pgfsetbuttcap%
\pgfsetmiterjoin%
\definecolor{currentfill}{rgb}{0.066899,0.263188,0.377594}%
\pgfsetfillcolor{currentfill}%
\pgfsetlinewidth{0.000000pt}%
\definecolor{currentstroke}{rgb}{0.000000,0.000000,0.000000}%
\pgfsetstrokecolor{currentstroke}%
\pgfsetstrokeopacity{0.000000}%
\pgfsetdash{}{0pt}%
\pgfpathmoveto{\pgfqpoint{5.009739in}{1.609196in}}%
\pgfpathlineto{\pgfqpoint{5.018492in}{1.609196in}}%
\pgfpathlineto{\pgfqpoint{5.018492in}{1.628463in}}%
\pgfpathlineto{\pgfqpoint{5.009739in}{1.628463in}}%
\pgfpathlineto{\pgfqpoint{5.009739in}{1.609196in}}%
\pgfpathclose%
\pgfusepath{fill}%
\end{pgfscope}%
\begin{pgfscope}%
\pgfpathrectangle{\pgfqpoint{3.776708in}{0.600000in}}{\pgfqpoint{2.573292in}{2.070576in}}%
\pgfusepath{clip}%
\pgfsetbuttcap%
\pgfsetmiterjoin%
\definecolor{currentfill}{rgb}{0.066899,0.263188,0.377594}%
\pgfsetfillcolor{currentfill}%
\pgfsetlinewidth{0.000000pt}%
\definecolor{currentstroke}{rgb}{0.000000,0.000000,0.000000}%
\pgfsetstrokecolor{currentstroke}%
\pgfsetstrokeopacity{0.000000}%
\pgfsetdash{}{0pt}%
\pgfpathmoveto{\pgfqpoint{5.020681in}{1.609196in}}%
\pgfpathlineto{\pgfqpoint{5.029434in}{1.609196in}}%
\pgfpathlineto{\pgfqpoint{5.029434in}{1.628208in}}%
\pgfpathlineto{\pgfqpoint{5.020681in}{1.628208in}}%
\pgfpathlineto{\pgfqpoint{5.020681in}{1.609196in}}%
\pgfpathclose%
\pgfusepath{fill}%
\end{pgfscope}%
\begin{pgfscope}%
\pgfpathrectangle{\pgfqpoint{3.776708in}{0.600000in}}{\pgfqpoint{2.573292in}{2.070576in}}%
\pgfusepath{clip}%
\pgfsetbuttcap%
\pgfsetmiterjoin%
\definecolor{currentfill}{rgb}{0.066899,0.263188,0.377594}%
\pgfsetfillcolor{currentfill}%
\pgfsetlinewidth{0.000000pt}%
\definecolor{currentstroke}{rgb}{0.000000,0.000000,0.000000}%
\pgfsetstrokecolor{currentstroke}%
\pgfsetstrokeopacity{0.000000}%
\pgfsetdash{}{0pt}%
\pgfpathmoveto{\pgfqpoint{5.031623in}{1.609196in}}%
\pgfpathlineto{\pgfqpoint{5.040376in}{1.609196in}}%
\pgfpathlineto{\pgfqpoint{5.040376in}{1.626799in}}%
\pgfpathlineto{\pgfqpoint{5.031623in}{1.626799in}}%
\pgfpathlineto{\pgfqpoint{5.031623in}{1.609196in}}%
\pgfpathclose%
\pgfusepath{fill}%
\end{pgfscope}%
\begin{pgfscope}%
\pgfpathrectangle{\pgfqpoint{3.776708in}{0.600000in}}{\pgfqpoint{2.573292in}{2.070576in}}%
\pgfusepath{clip}%
\pgfsetbuttcap%
\pgfsetmiterjoin%
\definecolor{currentfill}{rgb}{0.066899,0.263188,0.377594}%
\pgfsetfillcolor{currentfill}%
\pgfsetlinewidth{0.000000pt}%
\definecolor{currentstroke}{rgb}{0.000000,0.000000,0.000000}%
\pgfsetstrokecolor{currentstroke}%
\pgfsetstrokeopacity{0.000000}%
\pgfsetdash{}{0pt}%
\pgfpathmoveto{\pgfqpoint{5.042564in}{1.609196in}}%
\pgfpathlineto{\pgfqpoint{5.051318in}{1.609196in}}%
\pgfpathlineto{\pgfqpoint{5.051318in}{1.626691in}}%
\pgfpathlineto{\pgfqpoint{5.042564in}{1.626691in}}%
\pgfpathlineto{\pgfqpoint{5.042564in}{1.609196in}}%
\pgfpathclose%
\pgfusepath{fill}%
\end{pgfscope}%
\begin{pgfscope}%
\pgfpathrectangle{\pgfqpoint{3.776708in}{0.600000in}}{\pgfqpoint{2.573292in}{2.070576in}}%
\pgfusepath{clip}%
\pgfsetbuttcap%
\pgfsetmiterjoin%
\definecolor{currentfill}{rgb}{0.066899,0.263188,0.377594}%
\pgfsetfillcolor{currentfill}%
\pgfsetlinewidth{0.000000pt}%
\definecolor{currentstroke}{rgb}{0.000000,0.000000,0.000000}%
\pgfsetstrokecolor{currentstroke}%
\pgfsetstrokeopacity{0.000000}%
\pgfsetdash{}{0pt}%
\pgfpathmoveto{\pgfqpoint{5.053506in}{1.609196in}}%
\pgfpathlineto{\pgfqpoint{5.062260in}{1.609196in}}%
\pgfpathlineto{\pgfqpoint{5.062260in}{1.626699in}}%
\pgfpathlineto{\pgfqpoint{5.053506in}{1.626699in}}%
\pgfpathlineto{\pgfqpoint{5.053506in}{1.609196in}}%
\pgfpathclose%
\pgfusepath{fill}%
\end{pgfscope}%
\begin{pgfscope}%
\pgfpathrectangle{\pgfqpoint{3.776708in}{0.600000in}}{\pgfqpoint{2.573292in}{2.070576in}}%
\pgfusepath{clip}%
\pgfsetbuttcap%
\pgfsetmiterjoin%
\definecolor{currentfill}{rgb}{0.066899,0.263188,0.377594}%
\pgfsetfillcolor{currentfill}%
\pgfsetlinewidth{0.000000pt}%
\definecolor{currentstroke}{rgb}{0.000000,0.000000,0.000000}%
\pgfsetstrokecolor{currentstroke}%
\pgfsetstrokeopacity{0.000000}%
\pgfsetdash{}{0pt}%
\pgfpathmoveto{\pgfqpoint{5.064448in}{1.609196in}}%
\pgfpathlineto{\pgfqpoint{5.073201in}{1.609196in}}%
\pgfpathlineto{\pgfqpoint{5.073201in}{1.628191in}}%
\pgfpathlineto{\pgfqpoint{5.064448in}{1.628191in}}%
\pgfpathlineto{\pgfqpoint{5.064448in}{1.609196in}}%
\pgfpathclose%
\pgfusepath{fill}%
\end{pgfscope}%
\begin{pgfscope}%
\pgfpathrectangle{\pgfqpoint{3.776708in}{0.600000in}}{\pgfqpoint{2.573292in}{2.070576in}}%
\pgfusepath{clip}%
\pgfsetbuttcap%
\pgfsetmiterjoin%
\definecolor{currentfill}{rgb}{0.066899,0.263188,0.377594}%
\pgfsetfillcolor{currentfill}%
\pgfsetlinewidth{0.000000pt}%
\definecolor{currentstroke}{rgb}{0.000000,0.000000,0.000000}%
\pgfsetstrokecolor{currentstroke}%
\pgfsetstrokeopacity{0.000000}%
\pgfsetdash{}{0pt}%
\pgfpathmoveto{\pgfqpoint{5.075390in}{1.609196in}}%
\pgfpathlineto{\pgfqpoint{5.084143in}{1.609196in}}%
\pgfpathlineto{\pgfqpoint{5.084143in}{1.628622in}}%
\pgfpathlineto{\pgfqpoint{5.075390in}{1.628622in}}%
\pgfpathlineto{\pgfqpoint{5.075390in}{1.609196in}}%
\pgfpathclose%
\pgfusepath{fill}%
\end{pgfscope}%
\begin{pgfscope}%
\pgfpathrectangle{\pgfqpoint{3.776708in}{0.600000in}}{\pgfqpoint{2.573292in}{2.070576in}}%
\pgfusepath{clip}%
\pgfsetbuttcap%
\pgfsetmiterjoin%
\definecolor{currentfill}{rgb}{0.066899,0.263188,0.377594}%
\pgfsetfillcolor{currentfill}%
\pgfsetlinewidth{0.000000pt}%
\definecolor{currentstroke}{rgb}{0.000000,0.000000,0.000000}%
\pgfsetstrokecolor{currentstroke}%
\pgfsetstrokeopacity{0.000000}%
\pgfsetdash{}{0pt}%
\pgfpathmoveto{\pgfqpoint{5.086332in}{1.609196in}}%
\pgfpathlineto{\pgfqpoint{5.095085in}{1.609196in}}%
\pgfpathlineto{\pgfqpoint{5.095085in}{1.631312in}}%
\pgfpathlineto{\pgfqpoint{5.086332in}{1.631312in}}%
\pgfpathlineto{\pgfqpoint{5.086332in}{1.609196in}}%
\pgfpathclose%
\pgfusepath{fill}%
\end{pgfscope}%
\begin{pgfscope}%
\pgfpathrectangle{\pgfqpoint{3.776708in}{0.600000in}}{\pgfqpoint{2.573292in}{2.070576in}}%
\pgfusepath{clip}%
\pgfsetbuttcap%
\pgfsetmiterjoin%
\definecolor{currentfill}{rgb}{0.066899,0.263188,0.377594}%
\pgfsetfillcolor{currentfill}%
\pgfsetlinewidth{0.000000pt}%
\definecolor{currentstroke}{rgb}{0.000000,0.000000,0.000000}%
\pgfsetstrokecolor{currentstroke}%
\pgfsetstrokeopacity{0.000000}%
\pgfsetdash{}{0pt}%
\pgfpathmoveto{\pgfqpoint{5.097273in}{1.609196in}}%
\pgfpathlineto{\pgfqpoint{5.106027in}{1.609196in}}%
\pgfpathlineto{\pgfqpoint{5.106027in}{1.634183in}}%
\pgfpathlineto{\pgfqpoint{5.097273in}{1.634183in}}%
\pgfpathlineto{\pgfqpoint{5.097273in}{1.609196in}}%
\pgfpathclose%
\pgfusepath{fill}%
\end{pgfscope}%
\begin{pgfscope}%
\pgfpathrectangle{\pgfqpoint{3.776708in}{0.600000in}}{\pgfqpoint{2.573292in}{2.070576in}}%
\pgfusepath{clip}%
\pgfsetbuttcap%
\pgfsetmiterjoin%
\definecolor{currentfill}{rgb}{0.066899,0.263188,0.377594}%
\pgfsetfillcolor{currentfill}%
\pgfsetlinewidth{0.000000pt}%
\definecolor{currentstroke}{rgb}{0.000000,0.000000,0.000000}%
\pgfsetstrokecolor{currentstroke}%
\pgfsetstrokeopacity{0.000000}%
\pgfsetdash{}{0pt}%
\pgfpathmoveto{\pgfqpoint{5.108215in}{1.609196in}}%
\pgfpathlineto{\pgfqpoint{5.116969in}{1.609196in}}%
\pgfpathlineto{\pgfqpoint{5.116969in}{1.637561in}}%
\pgfpathlineto{\pgfqpoint{5.108215in}{1.637561in}}%
\pgfpathlineto{\pgfqpoint{5.108215in}{1.609196in}}%
\pgfpathclose%
\pgfusepath{fill}%
\end{pgfscope}%
\begin{pgfscope}%
\pgfpathrectangle{\pgfqpoint{3.776708in}{0.600000in}}{\pgfqpoint{2.573292in}{2.070576in}}%
\pgfusepath{clip}%
\pgfsetbuttcap%
\pgfsetmiterjoin%
\definecolor{currentfill}{rgb}{0.066899,0.263188,0.377594}%
\pgfsetfillcolor{currentfill}%
\pgfsetlinewidth{0.000000pt}%
\definecolor{currentstroke}{rgb}{0.000000,0.000000,0.000000}%
\pgfsetstrokecolor{currentstroke}%
\pgfsetstrokeopacity{0.000000}%
\pgfsetdash{}{0pt}%
\pgfpathmoveto{\pgfqpoint{5.119157in}{1.609196in}}%
\pgfpathlineto{\pgfqpoint{5.127910in}{1.609196in}}%
\pgfpathlineto{\pgfqpoint{5.127910in}{1.642014in}}%
\pgfpathlineto{\pgfqpoint{5.119157in}{1.642014in}}%
\pgfpathlineto{\pgfqpoint{5.119157in}{1.609196in}}%
\pgfpathclose%
\pgfusepath{fill}%
\end{pgfscope}%
\begin{pgfscope}%
\pgfpathrectangle{\pgfqpoint{3.776708in}{0.600000in}}{\pgfqpoint{2.573292in}{2.070576in}}%
\pgfusepath{clip}%
\pgfsetbuttcap%
\pgfsetmiterjoin%
\definecolor{currentfill}{rgb}{0.066899,0.263188,0.377594}%
\pgfsetfillcolor{currentfill}%
\pgfsetlinewidth{0.000000pt}%
\definecolor{currentstroke}{rgb}{0.000000,0.000000,0.000000}%
\pgfsetstrokecolor{currentstroke}%
\pgfsetstrokeopacity{0.000000}%
\pgfsetdash{}{0pt}%
\pgfpathmoveto{\pgfqpoint{5.130099in}{1.609196in}}%
\pgfpathlineto{\pgfqpoint{5.138852in}{1.609196in}}%
\pgfpathlineto{\pgfqpoint{5.138852in}{1.645294in}}%
\pgfpathlineto{\pgfqpoint{5.130099in}{1.645294in}}%
\pgfpathlineto{\pgfqpoint{5.130099in}{1.609196in}}%
\pgfpathclose%
\pgfusepath{fill}%
\end{pgfscope}%
\begin{pgfscope}%
\pgfpathrectangle{\pgfqpoint{3.776708in}{0.600000in}}{\pgfqpoint{2.573292in}{2.070576in}}%
\pgfusepath{clip}%
\pgfsetbuttcap%
\pgfsetmiterjoin%
\definecolor{currentfill}{rgb}{0.066899,0.263188,0.377594}%
\pgfsetfillcolor{currentfill}%
\pgfsetlinewidth{0.000000pt}%
\definecolor{currentstroke}{rgb}{0.000000,0.000000,0.000000}%
\pgfsetstrokecolor{currentstroke}%
\pgfsetstrokeopacity{0.000000}%
\pgfsetdash{}{0pt}%
\pgfpathmoveto{\pgfqpoint{5.141041in}{1.609196in}}%
\pgfpathlineto{\pgfqpoint{5.149794in}{1.609196in}}%
\pgfpathlineto{\pgfqpoint{5.149794in}{1.647527in}}%
\pgfpathlineto{\pgfqpoint{5.141041in}{1.647527in}}%
\pgfpathlineto{\pgfqpoint{5.141041in}{1.609196in}}%
\pgfpathclose%
\pgfusepath{fill}%
\end{pgfscope}%
\begin{pgfscope}%
\pgfpathrectangle{\pgfqpoint{3.776708in}{0.600000in}}{\pgfqpoint{2.573292in}{2.070576in}}%
\pgfusepath{clip}%
\pgfsetbuttcap%
\pgfsetmiterjoin%
\definecolor{currentfill}{rgb}{0.066899,0.263188,0.377594}%
\pgfsetfillcolor{currentfill}%
\pgfsetlinewidth{0.000000pt}%
\definecolor{currentstroke}{rgb}{0.000000,0.000000,0.000000}%
\pgfsetstrokecolor{currentstroke}%
\pgfsetstrokeopacity{0.000000}%
\pgfsetdash{}{0pt}%
\pgfpathmoveto{\pgfqpoint{5.151982in}{1.609196in}}%
\pgfpathlineto{\pgfqpoint{5.160736in}{1.609196in}}%
\pgfpathlineto{\pgfqpoint{5.160736in}{1.648946in}}%
\pgfpathlineto{\pgfqpoint{5.151982in}{1.648946in}}%
\pgfpathlineto{\pgfqpoint{5.151982in}{1.609196in}}%
\pgfpathclose%
\pgfusepath{fill}%
\end{pgfscope}%
\begin{pgfscope}%
\pgfpathrectangle{\pgfqpoint{3.776708in}{0.600000in}}{\pgfqpoint{2.573292in}{2.070576in}}%
\pgfusepath{clip}%
\pgfsetbuttcap%
\pgfsetmiterjoin%
\definecolor{currentfill}{rgb}{0.066899,0.263188,0.377594}%
\pgfsetfillcolor{currentfill}%
\pgfsetlinewidth{0.000000pt}%
\definecolor{currentstroke}{rgb}{0.000000,0.000000,0.000000}%
\pgfsetstrokecolor{currentstroke}%
\pgfsetstrokeopacity{0.000000}%
\pgfsetdash{}{0pt}%
\pgfpathmoveto{\pgfqpoint{5.162924in}{1.609196in}}%
\pgfpathlineto{\pgfqpoint{5.171678in}{1.609196in}}%
\pgfpathlineto{\pgfqpoint{5.171678in}{1.651851in}}%
\pgfpathlineto{\pgfqpoint{5.162924in}{1.651851in}}%
\pgfpathlineto{\pgfqpoint{5.162924in}{1.609196in}}%
\pgfpathclose%
\pgfusepath{fill}%
\end{pgfscope}%
\begin{pgfscope}%
\pgfpathrectangle{\pgfqpoint{3.776708in}{0.600000in}}{\pgfqpoint{2.573292in}{2.070576in}}%
\pgfusepath{clip}%
\pgfsetbuttcap%
\pgfsetmiterjoin%
\definecolor{currentfill}{rgb}{0.066899,0.263188,0.377594}%
\pgfsetfillcolor{currentfill}%
\pgfsetlinewidth{0.000000pt}%
\definecolor{currentstroke}{rgb}{0.000000,0.000000,0.000000}%
\pgfsetstrokecolor{currentstroke}%
\pgfsetstrokeopacity{0.000000}%
\pgfsetdash{}{0pt}%
\pgfpathmoveto{\pgfqpoint{5.173866in}{1.609196in}}%
\pgfpathlineto{\pgfqpoint{5.182619in}{1.609196in}}%
\pgfpathlineto{\pgfqpoint{5.182619in}{1.654412in}}%
\pgfpathlineto{\pgfqpoint{5.173866in}{1.654412in}}%
\pgfpathlineto{\pgfqpoint{5.173866in}{1.609196in}}%
\pgfpathclose%
\pgfusepath{fill}%
\end{pgfscope}%
\begin{pgfscope}%
\pgfpathrectangle{\pgfqpoint{3.776708in}{0.600000in}}{\pgfqpoint{2.573292in}{2.070576in}}%
\pgfusepath{clip}%
\pgfsetbuttcap%
\pgfsetmiterjoin%
\definecolor{currentfill}{rgb}{0.066899,0.263188,0.377594}%
\pgfsetfillcolor{currentfill}%
\pgfsetlinewidth{0.000000pt}%
\definecolor{currentstroke}{rgb}{0.000000,0.000000,0.000000}%
\pgfsetstrokecolor{currentstroke}%
\pgfsetstrokeopacity{0.000000}%
\pgfsetdash{}{0pt}%
\pgfpathmoveto{\pgfqpoint{5.184808in}{1.609196in}}%
\pgfpathlineto{\pgfqpoint{5.193561in}{1.609196in}}%
\pgfpathlineto{\pgfqpoint{5.193561in}{1.657945in}}%
\pgfpathlineto{\pgfqpoint{5.184808in}{1.657945in}}%
\pgfpathlineto{\pgfqpoint{5.184808in}{1.609196in}}%
\pgfpathclose%
\pgfusepath{fill}%
\end{pgfscope}%
\begin{pgfscope}%
\pgfpathrectangle{\pgfqpoint{3.776708in}{0.600000in}}{\pgfqpoint{2.573292in}{2.070576in}}%
\pgfusepath{clip}%
\pgfsetbuttcap%
\pgfsetmiterjoin%
\definecolor{currentfill}{rgb}{0.066899,0.263188,0.377594}%
\pgfsetfillcolor{currentfill}%
\pgfsetlinewidth{0.000000pt}%
\definecolor{currentstroke}{rgb}{0.000000,0.000000,0.000000}%
\pgfsetstrokecolor{currentstroke}%
\pgfsetstrokeopacity{0.000000}%
\pgfsetdash{}{0pt}%
\pgfpathmoveto{\pgfqpoint{5.195750in}{1.609196in}}%
\pgfpathlineto{\pgfqpoint{5.204503in}{1.609196in}}%
\pgfpathlineto{\pgfqpoint{5.204503in}{1.660511in}}%
\pgfpathlineto{\pgfqpoint{5.195750in}{1.660511in}}%
\pgfpathlineto{\pgfqpoint{5.195750in}{1.609196in}}%
\pgfpathclose%
\pgfusepath{fill}%
\end{pgfscope}%
\begin{pgfscope}%
\pgfpathrectangle{\pgfqpoint{3.776708in}{0.600000in}}{\pgfqpoint{2.573292in}{2.070576in}}%
\pgfusepath{clip}%
\pgfsetbuttcap%
\pgfsetmiterjoin%
\definecolor{currentfill}{rgb}{0.066899,0.263188,0.377594}%
\pgfsetfillcolor{currentfill}%
\pgfsetlinewidth{0.000000pt}%
\definecolor{currentstroke}{rgb}{0.000000,0.000000,0.000000}%
\pgfsetstrokecolor{currentstroke}%
\pgfsetstrokeopacity{0.000000}%
\pgfsetdash{}{0pt}%
\pgfpathmoveto{\pgfqpoint{5.206691in}{1.609196in}}%
\pgfpathlineto{\pgfqpoint{5.215445in}{1.609196in}}%
\pgfpathlineto{\pgfqpoint{5.215445in}{1.664475in}}%
\pgfpathlineto{\pgfqpoint{5.206691in}{1.664475in}}%
\pgfpathlineto{\pgfqpoint{5.206691in}{1.609196in}}%
\pgfpathclose%
\pgfusepath{fill}%
\end{pgfscope}%
\begin{pgfscope}%
\pgfpathrectangle{\pgfqpoint{3.776708in}{0.600000in}}{\pgfqpoint{2.573292in}{2.070576in}}%
\pgfusepath{clip}%
\pgfsetbuttcap%
\pgfsetmiterjoin%
\definecolor{currentfill}{rgb}{0.066899,0.263188,0.377594}%
\pgfsetfillcolor{currentfill}%
\pgfsetlinewidth{0.000000pt}%
\definecolor{currentstroke}{rgb}{0.000000,0.000000,0.000000}%
\pgfsetstrokecolor{currentstroke}%
\pgfsetstrokeopacity{0.000000}%
\pgfsetdash{}{0pt}%
\pgfpathmoveto{\pgfqpoint{5.217633in}{1.609196in}}%
\pgfpathlineto{\pgfqpoint{5.226387in}{1.609196in}}%
\pgfpathlineto{\pgfqpoint{5.226387in}{1.670588in}}%
\pgfpathlineto{\pgfqpoint{5.217633in}{1.670588in}}%
\pgfpathlineto{\pgfqpoint{5.217633in}{1.609196in}}%
\pgfpathclose%
\pgfusepath{fill}%
\end{pgfscope}%
\begin{pgfscope}%
\pgfpathrectangle{\pgfqpoint{3.776708in}{0.600000in}}{\pgfqpoint{2.573292in}{2.070576in}}%
\pgfusepath{clip}%
\pgfsetbuttcap%
\pgfsetmiterjoin%
\definecolor{currentfill}{rgb}{0.066899,0.263188,0.377594}%
\pgfsetfillcolor{currentfill}%
\pgfsetlinewidth{0.000000pt}%
\definecolor{currentstroke}{rgb}{0.000000,0.000000,0.000000}%
\pgfsetstrokecolor{currentstroke}%
\pgfsetstrokeopacity{0.000000}%
\pgfsetdash{}{0pt}%
\pgfpathmoveto{\pgfqpoint{5.228575in}{1.609196in}}%
\pgfpathlineto{\pgfqpoint{5.237328in}{1.609196in}}%
\pgfpathlineto{\pgfqpoint{5.237328in}{1.677953in}}%
\pgfpathlineto{\pgfqpoint{5.228575in}{1.677953in}}%
\pgfpathlineto{\pgfqpoint{5.228575in}{1.609196in}}%
\pgfpathclose%
\pgfusepath{fill}%
\end{pgfscope}%
\begin{pgfscope}%
\pgfpathrectangle{\pgfqpoint{3.776708in}{0.600000in}}{\pgfqpoint{2.573292in}{2.070576in}}%
\pgfusepath{clip}%
\pgfsetbuttcap%
\pgfsetmiterjoin%
\definecolor{currentfill}{rgb}{0.066899,0.263188,0.377594}%
\pgfsetfillcolor{currentfill}%
\pgfsetlinewidth{0.000000pt}%
\definecolor{currentstroke}{rgb}{0.000000,0.000000,0.000000}%
\pgfsetstrokecolor{currentstroke}%
\pgfsetstrokeopacity{0.000000}%
\pgfsetdash{}{0pt}%
\pgfpathmoveto{\pgfqpoint{5.239517in}{1.609196in}}%
\pgfpathlineto{\pgfqpoint{5.248270in}{1.609196in}}%
\pgfpathlineto{\pgfqpoint{5.248270in}{1.683448in}}%
\pgfpathlineto{\pgfqpoint{5.239517in}{1.683448in}}%
\pgfpathlineto{\pgfqpoint{5.239517in}{1.609196in}}%
\pgfpathclose%
\pgfusepath{fill}%
\end{pgfscope}%
\begin{pgfscope}%
\pgfpathrectangle{\pgfqpoint{3.776708in}{0.600000in}}{\pgfqpoint{2.573292in}{2.070576in}}%
\pgfusepath{clip}%
\pgfsetbuttcap%
\pgfsetmiterjoin%
\definecolor{currentfill}{rgb}{0.066899,0.263188,0.377594}%
\pgfsetfillcolor{currentfill}%
\pgfsetlinewidth{0.000000pt}%
\definecolor{currentstroke}{rgb}{0.000000,0.000000,0.000000}%
\pgfsetstrokecolor{currentstroke}%
\pgfsetstrokeopacity{0.000000}%
\pgfsetdash{}{0pt}%
\pgfpathmoveto{\pgfqpoint{5.250459in}{1.609196in}}%
\pgfpathlineto{\pgfqpoint{5.259212in}{1.609196in}}%
\pgfpathlineto{\pgfqpoint{5.259212in}{1.690384in}}%
\pgfpathlineto{\pgfqpoint{5.250459in}{1.690384in}}%
\pgfpathlineto{\pgfqpoint{5.250459in}{1.609196in}}%
\pgfpathclose%
\pgfusepath{fill}%
\end{pgfscope}%
\begin{pgfscope}%
\pgfpathrectangle{\pgfqpoint{3.776708in}{0.600000in}}{\pgfqpoint{2.573292in}{2.070576in}}%
\pgfusepath{clip}%
\pgfsetbuttcap%
\pgfsetmiterjoin%
\definecolor{currentfill}{rgb}{0.066899,0.263188,0.377594}%
\pgfsetfillcolor{currentfill}%
\pgfsetlinewidth{0.000000pt}%
\definecolor{currentstroke}{rgb}{0.000000,0.000000,0.000000}%
\pgfsetstrokecolor{currentstroke}%
\pgfsetstrokeopacity{0.000000}%
\pgfsetdash{}{0pt}%
\pgfpathmoveto{\pgfqpoint{5.261400in}{1.609196in}}%
\pgfpathlineto{\pgfqpoint{5.270154in}{1.609196in}}%
\pgfpathlineto{\pgfqpoint{5.270154in}{1.697364in}}%
\pgfpathlineto{\pgfqpoint{5.261400in}{1.697364in}}%
\pgfpathlineto{\pgfqpoint{5.261400in}{1.609196in}}%
\pgfpathclose%
\pgfusepath{fill}%
\end{pgfscope}%
\begin{pgfscope}%
\pgfpathrectangle{\pgfqpoint{3.776708in}{0.600000in}}{\pgfqpoint{2.573292in}{2.070576in}}%
\pgfusepath{clip}%
\pgfsetbuttcap%
\pgfsetmiterjoin%
\definecolor{currentfill}{rgb}{0.066899,0.263188,0.377594}%
\pgfsetfillcolor{currentfill}%
\pgfsetlinewidth{0.000000pt}%
\definecolor{currentstroke}{rgb}{0.000000,0.000000,0.000000}%
\pgfsetstrokecolor{currentstroke}%
\pgfsetstrokeopacity{0.000000}%
\pgfsetdash{}{0pt}%
\pgfpathmoveto{\pgfqpoint{5.272342in}{1.609196in}}%
\pgfpathlineto{\pgfqpoint{5.281096in}{1.609196in}}%
\pgfpathlineto{\pgfqpoint{5.281096in}{1.703424in}}%
\pgfpathlineto{\pgfqpoint{5.272342in}{1.703424in}}%
\pgfpathlineto{\pgfqpoint{5.272342in}{1.609196in}}%
\pgfpathclose%
\pgfusepath{fill}%
\end{pgfscope}%
\begin{pgfscope}%
\pgfpathrectangle{\pgfqpoint{3.776708in}{0.600000in}}{\pgfqpoint{2.573292in}{2.070576in}}%
\pgfusepath{clip}%
\pgfsetbuttcap%
\pgfsetmiterjoin%
\definecolor{currentfill}{rgb}{0.066899,0.263188,0.377594}%
\pgfsetfillcolor{currentfill}%
\pgfsetlinewidth{0.000000pt}%
\definecolor{currentstroke}{rgb}{0.000000,0.000000,0.000000}%
\pgfsetstrokecolor{currentstroke}%
\pgfsetstrokeopacity{0.000000}%
\pgfsetdash{}{0pt}%
\pgfpathmoveto{\pgfqpoint{5.283284in}{1.609196in}}%
\pgfpathlineto{\pgfqpoint{5.292037in}{1.609196in}}%
\pgfpathlineto{\pgfqpoint{5.292037in}{1.708754in}}%
\pgfpathlineto{\pgfqpoint{5.283284in}{1.708754in}}%
\pgfpathlineto{\pgfqpoint{5.283284in}{1.609196in}}%
\pgfpathclose%
\pgfusepath{fill}%
\end{pgfscope}%
\begin{pgfscope}%
\pgfpathrectangle{\pgfqpoint{3.776708in}{0.600000in}}{\pgfqpoint{2.573292in}{2.070576in}}%
\pgfusepath{clip}%
\pgfsetbuttcap%
\pgfsetmiterjoin%
\definecolor{currentfill}{rgb}{0.066899,0.263188,0.377594}%
\pgfsetfillcolor{currentfill}%
\pgfsetlinewidth{0.000000pt}%
\definecolor{currentstroke}{rgb}{0.000000,0.000000,0.000000}%
\pgfsetstrokecolor{currentstroke}%
\pgfsetstrokeopacity{0.000000}%
\pgfsetdash{}{0pt}%
\pgfpathmoveto{\pgfqpoint{5.294226in}{1.609196in}}%
\pgfpathlineto{\pgfqpoint{5.302979in}{1.609196in}}%
\pgfpathlineto{\pgfqpoint{5.302979in}{1.712145in}}%
\pgfpathlineto{\pgfqpoint{5.294226in}{1.712145in}}%
\pgfpathlineto{\pgfqpoint{5.294226in}{1.609196in}}%
\pgfpathclose%
\pgfusepath{fill}%
\end{pgfscope}%
\begin{pgfscope}%
\pgfpathrectangle{\pgfqpoint{3.776708in}{0.600000in}}{\pgfqpoint{2.573292in}{2.070576in}}%
\pgfusepath{clip}%
\pgfsetbuttcap%
\pgfsetmiterjoin%
\definecolor{currentfill}{rgb}{0.066899,0.263188,0.377594}%
\pgfsetfillcolor{currentfill}%
\pgfsetlinewidth{0.000000pt}%
\definecolor{currentstroke}{rgb}{0.000000,0.000000,0.000000}%
\pgfsetstrokecolor{currentstroke}%
\pgfsetstrokeopacity{0.000000}%
\pgfsetdash{}{0pt}%
\pgfpathmoveto{\pgfqpoint{5.305168in}{1.609196in}}%
\pgfpathlineto{\pgfqpoint{5.313921in}{1.609196in}}%
\pgfpathlineto{\pgfqpoint{5.313921in}{1.718409in}}%
\pgfpathlineto{\pgfqpoint{5.305168in}{1.718409in}}%
\pgfpathlineto{\pgfqpoint{5.305168in}{1.609196in}}%
\pgfpathclose%
\pgfusepath{fill}%
\end{pgfscope}%
\begin{pgfscope}%
\pgfpathrectangle{\pgfqpoint{3.776708in}{0.600000in}}{\pgfqpoint{2.573292in}{2.070576in}}%
\pgfusepath{clip}%
\pgfsetbuttcap%
\pgfsetmiterjoin%
\definecolor{currentfill}{rgb}{0.066899,0.263188,0.377594}%
\pgfsetfillcolor{currentfill}%
\pgfsetlinewidth{0.000000pt}%
\definecolor{currentstroke}{rgb}{0.000000,0.000000,0.000000}%
\pgfsetstrokecolor{currentstroke}%
\pgfsetstrokeopacity{0.000000}%
\pgfsetdash{}{0pt}%
\pgfpathmoveto{\pgfqpoint{5.316109in}{1.609196in}}%
\pgfpathlineto{\pgfqpoint{5.324863in}{1.609196in}}%
\pgfpathlineto{\pgfqpoint{5.324863in}{1.723949in}}%
\pgfpathlineto{\pgfqpoint{5.316109in}{1.723949in}}%
\pgfpathlineto{\pgfqpoint{5.316109in}{1.609196in}}%
\pgfpathclose%
\pgfusepath{fill}%
\end{pgfscope}%
\begin{pgfscope}%
\pgfpathrectangle{\pgfqpoint{3.776708in}{0.600000in}}{\pgfqpoint{2.573292in}{2.070576in}}%
\pgfusepath{clip}%
\pgfsetbuttcap%
\pgfsetmiterjoin%
\definecolor{currentfill}{rgb}{0.066899,0.263188,0.377594}%
\pgfsetfillcolor{currentfill}%
\pgfsetlinewidth{0.000000pt}%
\definecolor{currentstroke}{rgb}{0.000000,0.000000,0.000000}%
\pgfsetstrokecolor{currentstroke}%
\pgfsetstrokeopacity{0.000000}%
\pgfsetdash{}{0pt}%
\pgfpathmoveto{\pgfqpoint{5.327051in}{1.609196in}}%
\pgfpathlineto{\pgfqpoint{5.335805in}{1.609196in}}%
\pgfpathlineto{\pgfqpoint{5.335805in}{1.727505in}}%
\pgfpathlineto{\pgfqpoint{5.327051in}{1.727505in}}%
\pgfpathlineto{\pgfqpoint{5.327051in}{1.609196in}}%
\pgfpathclose%
\pgfusepath{fill}%
\end{pgfscope}%
\begin{pgfscope}%
\pgfpathrectangle{\pgfqpoint{3.776708in}{0.600000in}}{\pgfqpoint{2.573292in}{2.070576in}}%
\pgfusepath{clip}%
\pgfsetbuttcap%
\pgfsetmiterjoin%
\definecolor{currentfill}{rgb}{0.066899,0.263188,0.377594}%
\pgfsetfillcolor{currentfill}%
\pgfsetlinewidth{0.000000pt}%
\definecolor{currentstroke}{rgb}{0.000000,0.000000,0.000000}%
\pgfsetstrokecolor{currentstroke}%
\pgfsetstrokeopacity{0.000000}%
\pgfsetdash{}{0pt}%
\pgfpathmoveto{\pgfqpoint{5.337993in}{1.609196in}}%
\pgfpathlineto{\pgfqpoint{5.346746in}{1.609196in}}%
\pgfpathlineto{\pgfqpoint{5.346746in}{1.731708in}}%
\pgfpathlineto{\pgfqpoint{5.337993in}{1.731708in}}%
\pgfpathlineto{\pgfqpoint{5.337993in}{1.609196in}}%
\pgfpathclose%
\pgfusepath{fill}%
\end{pgfscope}%
\begin{pgfscope}%
\pgfpathrectangle{\pgfqpoint{3.776708in}{0.600000in}}{\pgfqpoint{2.573292in}{2.070576in}}%
\pgfusepath{clip}%
\pgfsetbuttcap%
\pgfsetmiterjoin%
\definecolor{currentfill}{rgb}{0.066899,0.263188,0.377594}%
\pgfsetfillcolor{currentfill}%
\pgfsetlinewidth{0.000000pt}%
\definecolor{currentstroke}{rgb}{0.000000,0.000000,0.000000}%
\pgfsetstrokecolor{currentstroke}%
\pgfsetstrokeopacity{0.000000}%
\pgfsetdash{}{0pt}%
\pgfpathmoveto{\pgfqpoint{5.348935in}{1.609196in}}%
\pgfpathlineto{\pgfqpoint{5.357688in}{1.609196in}}%
\pgfpathlineto{\pgfqpoint{5.357688in}{1.736333in}}%
\pgfpathlineto{\pgfqpoint{5.348935in}{1.736333in}}%
\pgfpathlineto{\pgfqpoint{5.348935in}{1.609196in}}%
\pgfpathclose%
\pgfusepath{fill}%
\end{pgfscope}%
\begin{pgfscope}%
\pgfpathrectangle{\pgfqpoint{3.776708in}{0.600000in}}{\pgfqpoint{2.573292in}{2.070576in}}%
\pgfusepath{clip}%
\pgfsetbuttcap%
\pgfsetmiterjoin%
\definecolor{currentfill}{rgb}{0.066899,0.263188,0.377594}%
\pgfsetfillcolor{currentfill}%
\pgfsetlinewidth{0.000000pt}%
\definecolor{currentstroke}{rgb}{0.000000,0.000000,0.000000}%
\pgfsetstrokecolor{currentstroke}%
\pgfsetstrokeopacity{0.000000}%
\pgfsetdash{}{0pt}%
\pgfpathmoveto{\pgfqpoint{5.359877in}{1.609196in}}%
\pgfpathlineto{\pgfqpoint{5.368630in}{1.609196in}}%
\pgfpathlineto{\pgfqpoint{5.368630in}{1.738517in}}%
\pgfpathlineto{\pgfqpoint{5.359877in}{1.738517in}}%
\pgfpathlineto{\pgfqpoint{5.359877in}{1.609196in}}%
\pgfpathclose%
\pgfusepath{fill}%
\end{pgfscope}%
\begin{pgfscope}%
\pgfpathrectangle{\pgfqpoint{3.776708in}{0.600000in}}{\pgfqpoint{2.573292in}{2.070576in}}%
\pgfusepath{clip}%
\pgfsetbuttcap%
\pgfsetmiterjoin%
\definecolor{currentfill}{rgb}{0.066899,0.263188,0.377594}%
\pgfsetfillcolor{currentfill}%
\pgfsetlinewidth{0.000000pt}%
\definecolor{currentstroke}{rgb}{0.000000,0.000000,0.000000}%
\pgfsetstrokecolor{currentstroke}%
\pgfsetstrokeopacity{0.000000}%
\pgfsetdash{}{0pt}%
\pgfpathmoveto{\pgfqpoint{5.370818in}{1.609196in}}%
\pgfpathlineto{\pgfqpoint{5.379572in}{1.609196in}}%
\pgfpathlineto{\pgfqpoint{5.379572in}{1.739853in}}%
\pgfpathlineto{\pgfqpoint{5.370818in}{1.739853in}}%
\pgfpathlineto{\pgfqpoint{5.370818in}{1.609196in}}%
\pgfpathclose%
\pgfusepath{fill}%
\end{pgfscope}%
\begin{pgfscope}%
\pgfpathrectangle{\pgfqpoint{3.776708in}{0.600000in}}{\pgfqpoint{2.573292in}{2.070576in}}%
\pgfusepath{clip}%
\pgfsetbuttcap%
\pgfsetmiterjoin%
\definecolor{currentfill}{rgb}{0.066899,0.263188,0.377594}%
\pgfsetfillcolor{currentfill}%
\pgfsetlinewidth{0.000000pt}%
\definecolor{currentstroke}{rgb}{0.000000,0.000000,0.000000}%
\pgfsetstrokecolor{currentstroke}%
\pgfsetstrokeopacity{0.000000}%
\pgfsetdash{}{0pt}%
\pgfpathmoveto{\pgfqpoint{5.381760in}{1.609196in}}%
\pgfpathlineto{\pgfqpoint{5.390514in}{1.609196in}}%
\pgfpathlineto{\pgfqpoint{5.390514in}{1.739120in}}%
\pgfpathlineto{\pgfqpoint{5.381760in}{1.739120in}}%
\pgfpathlineto{\pgfqpoint{5.381760in}{1.609196in}}%
\pgfpathclose%
\pgfusepath{fill}%
\end{pgfscope}%
\begin{pgfscope}%
\pgfpathrectangle{\pgfqpoint{3.776708in}{0.600000in}}{\pgfqpoint{2.573292in}{2.070576in}}%
\pgfusepath{clip}%
\pgfsetbuttcap%
\pgfsetmiterjoin%
\definecolor{currentfill}{rgb}{0.066899,0.263188,0.377594}%
\pgfsetfillcolor{currentfill}%
\pgfsetlinewidth{0.000000pt}%
\definecolor{currentstroke}{rgb}{0.000000,0.000000,0.000000}%
\pgfsetstrokecolor{currentstroke}%
\pgfsetstrokeopacity{0.000000}%
\pgfsetdash{}{0pt}%
\pgfpathmoveto{\pgfqpoint{5.392702in}{1.609196in}}%
\pgfpathlineto{\pgfqpoint{5.401455in}{1.609196in}}%
\pgfpathlineto{\pgfqpoint{5.401455in}{1.736386in}}%
\pgfpathlineto{\pgfqpoint{5.392702in}{1.736386in}}%
\pgfpathlineto{\pgfqpoint{5.392702in}{1.609196in}}%
\pgfpathclose%
\pgfusepath{fill}%
\end{pgfscope}%
\begin{pgfscope}%
\pgfpathrectangle{\pgfqpoint{3.776708in}{0.600000in}}{\pgfqpoint{2.573292in}{2.070576in}}%
\pgfusepath{clip}%
\pgfsetbuttcap%
\pgfsetmiterjoin%
\definecolor{currentfill}{rgb}{0.066899,0.263188,0.377594}%
\pgfsetfillcolor{currentfill}%
\pgfsetlinewidth{0.000000pt}%
\definecolor{currentstroke}{rgb}{0.000000,0.000000,0.000000}%
\pgfsetstrokecolor{currentstroke}%
\pgfsetstrokeopacity{0.000000}%
\pgfsetdash{}{0pt}%
\pgfpathmoveto{\pgfqpoint{5.403644in}{1.609196in}}%
\pgfpathlineto{\pgfqpoint{5.412397in}{1.609196in}}%
\pgfpathlineto{\pgfqpoint{5.412397in}{1.733194in}}%
\pgfpathlineto{\pgfqpoint{5.403644in}{1.733194in}}%
\pgfpathlineto{\pgfqpoint{5.403644in}{1.609196in}}%
\pgfpathclose%
\pgfusepath{fill}%
\end{pgfscope}%
\begin{pgfscope}%
\pgfpathrectangle{\pgfqpoint{3.776708in}{0.600000in}}{\pgfqpoint{2.573292in}{2.070576in}}%
\pgfusepath{clip}%
\pgfsetbuttcap%
\pgfsetmiterjoin%
\definecolor{currentfill}{rgb}{0.066899,0.263188,0.377594}%
\pgfsetfillcolor{currentfill}%
\pgfsetlinewidth{0.000000pt}%
\definecolor{currentstroke}{rgb}{0.000000,0.000000,0.000000}%
\pgfsetstrokecolor{currentstroke}%
\pgfsetstrokeopacity{0.000000}%
\pgfsetdash{}{0pt}%
\pgfpathmoveto{\pgfqpoint{5.414586in}{1.609196in}}%
\pgfpathlineto{\pgfqpoint{5.423339in}{1.609196in}}%
\pgfpathlineto{\pgfqpoint{5.423339in}{1.731762in}}%
\pgfpathlineto{\pgfqpoint{5.414586in}{1.731762in}}%
\pgfpathlineto{\pgfqpoint{5.414586in}{1.609196in}}%
\pgfpathclose%
\pgfusepath{fill}%
\end{pgfscope}%
\begin{pgfscope}%
\pgfpathrectangle{\pgfqpoint{3.776708in}{0.600000in}}{\pgfqpoint{2.573292in}{2.070576in}}%
\pgfusepath{clip}%
\pgfsetbuttcap%
\pgfsetmiterjoin%
\definecolor{currentfill}{rgb}{0.066899,0.263188,0.377594}%
\pgfsetfillcolor{currentfill}%
\pgfsetlinewidth{0.000000pt}%
\definecolor{currentstroke}{rgb}{0.000000,0.000000,0.000000}%
\pgfsetstrokecolor{currentstroke}%
\pgfsetstrokeopacity{0.000000}%
\pgfsetdash{}{0pt}%
\pgfpathmoveto{\pgfqpoint{5.425527in}{1.609196in}}%
\pgfpathlineto{\pgfqpoint{5.434281in}{1.609196in}}%
\pgfpathlineto{\pgfqpoint{5.434281in}{1.726860in}}%
\pgfpathlineto{\pgfqpoint{5.425527in}{1.726860in}}%
\pgfpathlineto{\pgfqpoint{5.425527in}{1.609196in}}%
\pgfpathclose%
\pgfusepath{fill}%
\end{pgfscope}%
\begin{pgfscope}%
\pgfpathrectangle{\pgfqpoint{3.776708in}{0.600000in}}{\pgfqpoint{2.573292in}{2.070576in}}%
\pgfusepath{clip}%
\pgfsetbuttcap%
\pgfsetmiterjoin%
\definecolor{currentfill}{rgb}{0.066899,0.263188,0.377594}%
\pgfsetfillcolor{currentfill}%
\pgfsetlinewidth{0.000000pt}%
\definecolor{currentstroke}{rgb}{0.000000,0.000000,0.000000}%
\pgfsetstrokecolor{currentstroke}%
\pgfsetstrokeopacity{0.000000}%
\pgfsetdash{}{0pt}%
\pgfpathmoveto{\pgfqpoint{5.436469in}{1.609196in}}%
\pgfpathlineto{\pgfqpoint{5.445223in}{1.609196in}}%
\pgfpathlineto{\pgfqpoint{5.445223in}{1.719898in}}%
\pgfpathlineto{\pgfqpoint{5.436469in}{1.719898in}}%
\pgfpathlineto{\pgfqpoint{5.436469in}{1.609196in}}%
\pgfpathclose%
\pgfusepath{fill}%
\end{pgfscope}%
\begin{pgfscope}%
\pgfpathrectangle{\pgfqpoint{3.776708in}{0.600000in}}{\pgfqpoint{2.573292in}{2.070576in}}%
\pgfusepath{clip}%
\pgfsetbuttcap%
\pgfsetmiterjoin%
\definecolor{currentfill}{rgb}{0.066899,0.263188,0.377594}%
\pgfsetfillcolor{currentfill}%
\pgfsetlinewidth{0.000000pt}%
\definecolor{currentstroke}{rgb}{0.000000,0.000000,0.000000}%
\pgfsetstrokecolor{currentstroke}%
\pgfsetstrokeopacity{0.000000}%
\pgfsetdash{}{0pt}%
\pgfpathmoveto{\pgfqpoint{5.447411in}{1.609196in}}%
\pgfpathlineto{\pgfqpoint{5.456164in}{1.609196in}}%
\pgfpathlineto{\pgfqpoint{5.456164in}{1.715168in}}%
\pgfpathlineto{\pgfqpoint{5.447411in}{1.715168in}}%
\pgfpathlineto{\pgfqpoint{5.447411in}{1.609196in}}%
\pgfpathclose%
\pgfusepath{fill}%
\end{pgfscope}%
\begin{pgfscope}%
\pgfpathrectangle{\pgfqpoint{3.776708in}{0.600000in}}{\pgfqpoint{2.573292in}{2.070576in}}%
\pgfusepath{clip}%
\pgfsetbuttcap%
\pgfsetmiterjoin%
\definecolor{currentfill}{rgb}{0.066899,0.263188,0.377594}%
\pgfsetfillcolor{currentfill}%
\pgfsetlinewidth{0.000000pt}%
\definecolor{currentstroke}{rgb}{0.000000,0.000000,0.000000}%
\pgfsetstrokecolor{currentstroke}%
\pgfsetstrokeopacity{0.000000}%
\pgfsetdash{}{0pt}%
\pgfpathmoveto{\pgfqpoint{5.458353in}{1.609196in}}%
\pgfpathlineto{\pgfqpoint{5.467106in}{1.609196in}}%
\pgfpathlineto{\pgfqpoint{5.467106in}{1.709292in}}%
\pgfpathlineto{\pgfqpoint{5.458353in}{1.709292in}}%
\pgfpathlineto{\pgfqpoint{5.458353in}{1.609196in}}%
\pgfpathclose%
\pgfusepath{fill}%
\end{pgfscope}%
\begin{pgfscope}%
\pgfpathrectangle{\pgfqpoint{3.776708in}{0.600000in}}{\pgfqpoint{2.573292in}{2.070576in}}%
\pgfusepath{clip}%
\pgfsetbuttcap%
\pgfsetmiterjoin%
\definecolor{currentfill}{rgb}{0.066899,0.263188,0.377594}%
\pgfsetfillcolor{currentfill}%
\pgfsetlinewidth{0.000000pt}%
\definecolor{currentstroke}{rgb}{0.000000,0.000000,0.000000}%
\pgfsetstrokecolor{currentstroke}%
\pgfsetstrokeopacity{0.000000}%
\pgfsetdash{}{0pt}%
\pgfpathmoveto{\pgfqpoint{5.469295in}{1.609196in}}%
\pgfpathlineto{\pgfqpoint{5.478048in}{1.609196in}}%
\pgfpathlineto{\pgfqpoint{5.478048in}{1.704078in}}%
\pgfpathlineto{\pgfqpoint{5.469295in}{1.704078in}}%
\pgfpathlineto{\pgfqpoint{5.469295in}{1.609196in}}%
\pgfpathclose%
\pgfusepath{fill}%
\end{pgfscope}%
\begin{pgfscope}%
\pgfpathrectangle{\pgfqpoint{3.776708in}{0.600000in}}{\pgfqpoint{2.573292in}{2.070576in}}%
\pgfusepath{clip}%
\pgfsetbuttcap%
\pgfsetmiterjoin%
\definecolor{currentfill}{rgb}{0.066899,0.263188,0.377594}%
\pgfsetfillcolor{currentfill}%
\pgfsetlinewidth{0.000000pt}%
\definecolor{currentstroke}{rgb}{0.000000,0.000000,0.000000}%
\pgfsetstrokecolor{currentstroke}%
\pgfsetstrokeopacity{0.000000}%
\pgfsetdash{}{0pt}%
\pgfpathmoveto{\pgfqpoint{5.480236in}{1.609196in}}%
\pgfpathlineto{\pgfqpoint{5.488990in}{1.609196in}}%
\pgfpathlineto{\pgfqpoint{5.488990in}{1.700426in}}%
\pgfpathlineto{\pgfqpoint{5.480236in}{1.700426in}}%
\pgfpathlineto{\pgfqpoint{5.480236in}{1.609196in}}%
\pgfpathclose%
\pgfusepath{fill}%
\end{pgfscope}%
\begin{pgfscope}%
\pgfpathrectangle{\pgfqpoint{3.776708in}{0.600000in}}{\pgfqpoint{2.573292in}{2.070576in}}%
\pgfusepath{clip}%
\pgfsetbuttcap%
\pgfsetmiterjoin%
\definecolor{currentfill}{rgb}{0.066899,0.263188,0.377594}%
\pgfsetfillcolor{currentfill}%
\pgfsetlinewidth{0.000000pt}%
\definecolor{currentstroke}{rgb}{0.000000,0.000000,0.000000}%
\pgfsetstrokecolor{currentstroke}%
\pgfsetstrokeopacity{0.000000}%
\pgfsetdash{}{0pt}%
\pgfpathmoveto{\pgfqpoint{5.491178in}{1.609196in}}%
\pgfpathlineto{\pgfqpoint{5.499932in}{1.609196in}}%
\pgfpathlineto{\pgfqpoint{5.499932in}{1.694448in}}%
\pgfpathlineto{\pgfqpoint{5.491178in}{1.694448in}}%
\pgfpathlineto{\pgfqpoint{5.491178in}{1.609196in}}%
\pgfpathclose%
\pgfusepath{fill}%
\end{pgfscope}%
\begin{pgfscope}%
\pgfpathrectangle{\pgfqpoint{3.776708in}{0.600000in}}{\pgfqpoint{2.573292in}{2.070576in}}%
\pgfusepath{clip}%
\pgfsetbuttcap%
\pgfsetmiterjoin%
\definecolor{currentfill}{rgb}{0.066899,0.263188,0.377594}%
\pgfsetfillcolor{currentfill}%
\pgfsetlinewidth{0.000000pt}%
\definecolor{currentstroke}{rgb}{0.000000,0.000000,0.000000}%
\pgfsetstrokecolor{currentstroke}%
\pgfsetstrokeopacity{0.000000}%
\pgfsetdash{}{0pt}%
\pgfpathmoveto{\pgfqpoint{5.502120in}{1.609196in}}%
\pgfpathlineto{\pgfqpoint{5.510873in}{1.609196in}}%
\pgfpathlineto{\pgfqpoint{5.510873in}{1.686971in}}%
\pgfpathlineto{\pgfqpoint{5.502120in}{1.686971in}}%
\pgfpathlineto{\pgfqpoint{5.502120in}{1.609196in}}%
\pgfpathclose%
\pgfusepath{fill}%
\end{pgfscope}%
\begin{pgfscope}%
\pgfpathrectangle{\pgfqpoint{3.776708in}{0.600000in}}{\pgfqpoint{2.573292in}{2.070576in}}%
\pgfusepath{clip}%
\pgfsetbuttcap%
\pgfsetmiterjoin%
\definecolor{currentfill}{rgb}{0.066899,0.263188,0.377594}%
\pgfsetfillcolor{currentfill}%
\pgfsetlinewidth{0.000000pt}%
\definecolor{currentstroke}{rgb}{0.000000,0.000000,0.000000}%
\pgfsetstrokecolor{currentstroke}%
\pgfsetstrokeopacity{0.000000}%
\pgfsetdash{}{0pt}%
\pgfpathmoveto{\pgfqpoint{5.513062in}{1.609196in}}%
\pgfpathlineto{\pgfqpoint{5.521815in}{1.609196in}}%
\pgfpathlineto{\pgfqpoint{5.521815in}{1.681952in}}%
\pgfpathlineto{\pgfqpoint{5.513062in}{1.681952in}}%
\pgfpathlineto{\pgfqpoint{5.513062in}{1.609196in}}%
\pgfpathclose%
\pgfusepath{fill}%
\end{pgfscope}%
\begin{pgfscope}%
\pgfpathrectangle{\pgfqpoint{3.776708in}{0.600000in}}{\pgfqpoint{2.573292in}{2.070576in}}%
\pgfusepath{clip}%
\pgfsetbuttcap%
\pgfsetmiterjoin%
\definecolor{currentfill}{rgb}{0.066899,0.263188,0.377594}%
\pgfsetfillcolor{currentfill}%
\pgfsetlinewidth{0.000000pt}%
\definecolor{currentstroke}{rgb}{0.000000,0.000000,0.000000}%
\pgfsetstrokecolor{currentstroke}%
\pgfsetstrokeopacity{0.000000}%
\pgfsetdash{}{0pt}%
\pgfpathmoveto{\pgfqpoint{5.524004in}{1.609196in}}%
\pgfpathlineto{\pgfqpoint{5.532757in}{1.609196in}}%
\pgfpathlineto{\pgfqpoint{5.532757in}{1.676364in}}%
\pgfpathlineto{\pgfqpoint{5.524004in}{1.676364in}}%
\pgfpathlineto{\pgfqpoint{5.524004in}{1.609196in}}%
\pgfpathclose%
\pgfusepath{fill}%
\end{pgfscope}%
\begin{pgfscope}%
\pgfpathrectangle{\pgfqpoint{3.776708in}{0.600000in}}{\pgfqpoint{2.573292in}{2.070576in}}%
\pgfusepath{clip}%
\pgfsetbuttcap%
\pgfsetmiterjoin%
\definecolor{currentfill}{rgb}{0.066899,0.263188,0.377594}%
\pgfsetfillcolor{currentfill}%
\pgfsetlinewidth{0.000000pt}%
\definecolor{currentstroke}{rgb}{0.000000,0.000000,0.000000}%
\pgfsetstrokecolor{currentstroke}%
\pgfsetstrokeopacity{0.000000}%
\pgfsetdash{}{0pt}%
\pgfpathmoveto{\pgfqpoint{5.534945in}{1.609196in}}%
\pgfpathlineto{\pgfqpoint{5.543699in}{1.609196in}}%
\pgfpathlineto{\pgfqpoint{5.543699in}{1.672037in}}%
\pgfpathlineto{\pgfqpoint{5.534945in}{1.672037in}}%
\pgfpathlineto{\pgfqpoint{5.534945in}{1.609196in}}%
\pgfpathclose%
\pgfusepath{fill}%
\end{pgfscope}%
\begin{pgfscope}%
\pgfpathrectangle{\pgfqpoint{3.776708in}{0.600000in}}{\pgfqpoint{2.573292in}{2.070576in}}%
\pgfusepath{clip}%
\pgfsetbuttcap%
\pgfsetmiterjoin%
\definecolor{currentfill}{rgb}{0.066899,0.263188,0.377594}%
\pgfsetfillcolor{currentfill}%
\pgfsetlinewidth{0.000000pt}%
\definecolor{currentstroke}{rgb}{0.000000,0.000000,0.000000}%
\pgfsetstrokecolor{currentstroke}%
\pgfsetstrokeopacity{0.000000}%
\pgfsetdash{}{0pt}%
\pgfpathmoveto{\pgfqpoint{5.545887in}{1.609196in}}%
\pgfpathlineto{\pgfqpoint{5.554641in}{1.609196in}}%
\pgfpathlineto{\pgfqpoint{5.554641in}{1.668337in}}%
\pgfpathlineto{\pgfqpoint{5.545887in}{1.668337in}}%
\pgfpathlineto{\pgfqpoint{5.545887in}{1.609196in}}%
\pgfpathclose%
\pgfusepath{fill}%
\end{pgfscope}%
\begin{pgfscope}%
\pgfpathrectangle{\pgfqpoint{3.776708in}{0.600000in}}{\pgfqpoint{2.573292in}{2.070576in}}%
\pgfusepath{clip}%
\pgfsetbuttcap%
\pgfsetmiterjoin%
\definecolor{currentfill}{rgb}{0.066899,0.263188,0.377594}%
\pgfsetfillcolor{currentfill}%
\pgfsetlinewidth{0.000000pt}%
\definecolor{currentstroke}{rgb}{0.000000,0.000000,0.000000}%
\pgfsetstrokecolor{currentstroke}%
\pgfsetstrokeopacity{0.000000}%
\pgfsetdash{}{0pt}%
\pgfpathmoveto{\pgfqpoint{5.556829in}{1.609196in}}%
\pgfpathlineto{\pgfqpoint{5.565582in}{1.609196in}}%
\pgfpathlineto{\pgfqpoint{5.565582in}{1.664539in}}%
\pgfpathlineto{\pgfqpoint{5.556829in}{1.664539in}}%
\pgfpathlineto{\pgfqpoint{5.556829in}{1.609196in}}%
\pgfpathclose%
\pgfusepath{fill}%
\end{pgfscope}%
\begin{pgfscope}%
\pgfpathrectangle{\pgfqpoint{3.776708in}{0.600000in}}{\pgfqpoint{2.573292in}{2.070576in}}%
\pgfusepath{clip}%
\pgfsetbuttcap%
\pgfsetmiterjoin%
\definecolor{currentfill}{rgb}{0.066899,0.263188,0.377594}%
\pgfsetfillcolor{currentfill}%
\pgfsetlinewidth{0.000000pt}%
\definecolor{currentstroke}{rgb}{0.000000,0.000000,0.000000}%
\pgfsetstrokecolor{currentstroke}%
\pgfsetstrokeopacity{0.000000}%
\pgfsetdash{}{0pt}%
\pgfpathmoveto{\pgfqpoint{5.567771in}{1.609196in}}%
\pgfpathlineto{\pgfqpoint{5.576524in}{1.609196in}}%
\pgfpathlineto{\pgfqpoint{5.576524in}{1.661381in}}%
\pgfpathlineto{\pgfqpoint{5.567771in}{1.661381in}}%
\pgfpathlineto{\pgfqpoint{5.567771in}{1.609196in}}%
\pgfpathclose%
\pgfusepath{fill}%
\end{pgfscope}%
\begin{pgfscope}%
\pgfpathrectangle{\pgfqpoint{3.776708in}{0.600000in}}{\pgfqpoint{2.573292in}{2.070576in}}%
\pgfusepath{clip}%
\pgfsetbuttcap%
\pgfsetmiterjoin%
\definecolor{currentfill}{rgb}{0.066899,0.263188,0.377594}%
\pgfsetfillcolor{currentfill}%
\pgfsetlinewidth{0.000000pt}%
\definecolor{currentstroke}{rgb}{0.000000,0.000000,0.000000}%
\pgfsetstrokecolor{currentstroke}%
\pgfsetstrokeopacity{0.000000}%
\pgfsetdash{}{0pt}%
\pgfpathmoveto{\pgfqpoint{5.578713in}{1.609196in}}%
\pgfpathlineto{\pgfqpoint{5.587466in}{1.609196in}}%
\pgfpathlineto{\pgfqpoint{5.587466in}{1.658074in}}%
\pgfpathlineto{\pgfqpoint{5.578713in}{1.658074in}}%
\pgfpathlineto{\pgfqpoint{5.578713in}{1.609196in}}%
\pgfpathclose%
\pgfusepath{fill}%
\end{pgfscope}%
\begin{pgfscope}%
\pgfpathrectangle{\pgfqpoint{3.776708in}{0.600000in}}{\pgfqpoint{2.573292in}{2.070576in}}%
\pgfusepath{clip}%
\pgfsetbuttcap%
\pgfsetmiterjoin%
\definecolor{currentfill}{rgb}{0.066899,0.263188,0.377594}%
\pgfsetfillcolor{currentfill}%
\pgfsetlinewidth{0.000000pt}%
\definecolor{currentstroke}{rgb}{0.000000,0.000000,0.000000}%
\pgfsetstrokecolor{currentstroke}%
\pgfsetstrokeopacity{0.000000}%
\pgfsetdash{}{0pt}%
\pgfpathmoveto{\pgfqpoint{5.589654in}{1.609196in}}%
\pgfpathlineto{\pgfqpoint{5.598408in}{1.609196in}}%
\pgfpathlineto{\pgfqpoint{5.598408in}{1.652962in}}%
\pgfpathlineto{\pgfqpoint{5.589654in}{1.652962in}}%
\pgfpathlineto{\pgfqpoint{5.589654in}{1.609196in}}%
\pgfpathclose%
\pgfusepath{fill}%
\end{pgfscope}%
\begin{pgfscope}%
\pgfpathrectangle{\pgfqpoint{3.776708in}{0.600000in}}{\pgfqpoint{2.573292in}{2.070576in}}%
\pgfusepath{clip}%
\pgfsetbuttcap%
\pgfsetmiterjoin%
\definecolor{currentfill}{rgb}{0.066899,0.263188,0.377594}%
\pgfsetfillcolor{currentfill}%
\pgfsetlinewidth{0.000000pt}%
\definecolor{currentstroke}{rgb}{0.000000,0.000000,0.000000}%
\pgfsetstrokecolor{currentstroke}%
\pgfsetstrokeopacity{0.000000}%
\pgfsetdash{}{0pt}%
\pgfpathmoveto{\pgfqpoint{5.600596in}{1.609196in}}%
\pgfpathlineto{\pgfqpoint{5.609350in}{1.609196in}}%
\pgfpathlineto{\pgfqpoint{5.609350in}{1.647900in}}%
\pgfpathlineto{\pgfqpoint{5.600596in}{1.647900in}}%
\pgfpathlineto{\pgfqpoint{5.600596in}{1.609196in}}%
\pgfpathclose%
\pgfusepath{fill}%
\end{pgfscope}%
\begin{pgfscope}%
\pgfpathrectangle{\pgfqpoint{3.776708in}{0.600000in}}{\pgfqpoint{2.573292in}{2.070576in}}%
\pgfusepath{clip}%
\pgfsetbuttcap%
\pgfsetmiterjoin%
\definecolor{currentfill}{rgb}{0.066899,0.263188,0.377594}%
\pgfsetfillcolor{currentfill}%
\pgfsetlinewidth{0.000000pt}%
\definecolor{currentstroke}{rgb}{0.000000,0.000000,0.000000}%
\pgfsetstrokecolor{currentstroke}%
\pgfsetstrokeopacity{0.000000}%
\pgfsetdash{}{0pt}%
\pgfpathmoveto{\pgfqpoint{5.611538in}{1.609196in}}%
\pgfpathlineto{\pgfqpoint{5.620291in}{1.609196in}}%
\pgfpathlineto{\pgfqpoint{5.620291in}{1.643952in}}%
\pgfpathlineto{\pgfqpoint{5.611538in}{1.643952in}}%
\pgfpathlineto{\pgfqpoint{5.611538in}{1.609196in}}%
\pgfpathclose%
\pgfusepath{fill}%
\end{pgfscope}%
\begin{pgfscope}%
\pgfpathrectangle{\pgfqpoint{3.776708in}{0.600000in}}{\pgfqpoint{2.573292in}{2.070576in}}%
\pgfusepath{clip}%
\pgfsetbuttcap%
\pgfsetmiterjoin%
\definecolor{currentfill}{rgb}{0.066899,0.263188,0.377594}%
\pgfsetfillcolor{currentfill}%
\pgfsetlinewidth{0.000000pt}%
\definecolor{currentstroke}{rgb}{0.000000,0.000000,0.000000}%
\pgfsetstrokecolor{currentstroke}%
\pgfsetstrokeopacity{0.000000}%
\pgfsetdash{}{0pt}%
\pgfpathmoveto{\pgfqpoint{5.622480in}{1.609196in}}%
\pgfpathlineto{\pgfqpoint{5.631233in}{1.609196in}}%
\pgfpathlineto{\pgfqpoint{5.631233in}{1.641091in}}%
\pgfpathlineto{\pgfqpoint{5.622480in}{1.641091in}}%
\pgfpathlineto{\pgfqpoint{5.622480in}{1.609196in}}%
\pgfpathclose%
\pgfusepath{fill}%
\end{pgfscope}%
\begin{pgfscope}%
\pgfpathrectangle{\pgfqpoint{3.776708in}{0.600000in}}{\pgfqpoint{2.573292in}{2.070576in}}%
\pgfusepath{clip}%
\pgfsetbuttcap%
\pgfsetmiterjoin%
\definecolor{currentfill}{rgb}{0.066899,0.263188,0.377594}%
\pgfsetfillcolor{currentfill}%
\pgfsetlinewidth{0.000000pt}%
\definecolor{currentstroke}{rgb}{0.000000,0.000000,0.000000}%
\pgfsetstrokecolor{currentstroke}%
\pgfsetstrokeopacity{0.000000}%
\pgfsetdash{}{0pt}%
\pgfpathmoveto{\pgfqpoint{5.633422in}{1.609196in}}%
\pgfpathlineto{\pgfqpoint{5.642175in}{1.609196in}}%
\pgfpathlineto{\pgfqpoint{5.642175in}{1.639431in}}%
\pgfpathlineto{\pgfqpoint{5.633422in}{1.639431in}}%
\pgfpathlineto{\pgfqpoint{5.633422in}{1.609196in}}%
\pgfpathclose%
\pgfusepath{fill}%
\end{pgfscope}%
\begin{pgfscope}%
\pgfpathrectangle{\pgfqpoint{3.776708in}{0.600000in}}{\pgfqpoint{2.573292in}{2.070576in}}%
\pgfusepath{clip}%
\pgfsetbuttcap%
\pgfsetmiterjoin%
\definecolor{currentfill}{rgb}{0.066899,0.263188,0.377594}%
\pgfsetfillcolor{currentfill}%
\pgfsetlinewidth{0.000000pt}%
\definecolor{currentstroke}{rgb}{0.000000,0.000000,0.000000}%
\pgfsetstrokecolor{currentstroke}%
\pgfsetstrokeopacity{0.000000}%
\pgfsetdash{}{0pt}%
\pgfpathmoveto{\pgfqpoint{5.644363in}{1.609196in}}%
\pgfpathlineto{\pgfqpoint{5.653117in}{1.609196in}}%
\pgfpathlineto{\pgfqpoint{5.653117in}{1.634522in}}%
\pgfpathlineto{\pgfqpoint{5.644363in}{1.634522in}}%
\pgfpathlineto{\pgfqpoint{5.644363in}{1.609196in}}%
\pgfpathclose%
\pgfusepath{fill}%
\end{pgfscope}%
\begin{pgfscope}%
\pgfpathrectangle{\pgfqpoint{3.776708in}{0.600000in}}{\pgfqpoint{2.573292in}{2.070576in}}%
\pgfusepath{clip}%
\pgfsetbuttcap%
\pgfsetmiterjoin%
\definecolor{currentfill}{rgb}{0.066899,0.263188,0.377594}%
\pgfsetfillcolor{currentfill}%
\pgfsetlinewidth{0.000000pt}%
\definecolor{currentstroke}{rgb}{0.000000,0.000000,0.000000}%
\pgfsetstrokecolor{currentstroke}%
\pgfsetstrokeopacity{0.000000}%
\pgfsetdash{}{0pt}%
\pgfpathmoveto{\pgfqpoint{5.655305in}{1.609196in}}%
\pgfpathlineto{\pgfqpoint{5.664059in}{1.609196in}}%
\pgfpathlineto{\pgfqpoint{5.664059in}{1.631068in}}%
\pgfpathlineto{\pgfqpoint{5.655305in}{1.631068in}}%
\pgfpathlineto{\pgfqpoint{5.655305in}{1.609196in}}%
\pgfpathclose%
\pgfusepath{fill}%
\end{pgfscope}%
\begin{pgfscope}%
\pgfpathrectangle{\pgfqpoint{3.776708in}{0.600000in}}{\pgfqpoint{2.573292in}{2.070576in}}%
\pgfusepath{clip}%
\pgfsetbuttcap%
\pgfsetmiterjoin%
\definecolor{currentfill}{rgb}{0.066899,0.263188,0.377594}%
\pgfsetfillcolor{currentfill}%
\pgfsetlinewidth{0.000000pt}%
\definecolor{currentstroke}{rgb}{0.000000,0.000000,0.000000}%
\pgfsetstrokecolor{currentstroke}%
\pgfsetstrokeopacity{0.000000}%
\pgfsetdash{}{0pt}%
\pgfpathmoveto{\pgfqpoint{5.666247in}{1.609196in}}%
\pgfpathlineto{\pgfqpoint{5.675000in}{1.609196in}}%
\pgfpathlineto{\pgfqpoint{5.675000in}{1.628842in}}%
\pgfpathlineto{\pgfqpoint{5.666247in}{1.628842in}}%
\pgfpathlineto{\pgfqpoint{5.666247in}{1.609196in}}%
\pgfpathclose%
\pgfusepath{fill}%
\end{pgfscope}%
\begin{pgfscope}%
\pgfpathrectangle{\pgfqpoint{3.776708in}{0.600000in}}{\pgfqpoint{2.573292in}{2.070576in}}%
\pgfusepath{clip}%
\pgfsetbuttcap%
\pgfsetmiterjoin%
\definecolor{currentfill}{rgb}{0.066899,0.263188,0.377594}%
\pgfsetfillcolor{currentfill}%
\pgfsetlinewidth{0.000000pt}%
\definecolor{currentstroke}{rgb}{0.000000,0.000000,0.000000}%
\pgfsetstrokecolor{currentstroke}%
\pgfsetstrokeopacity{0.000000}%
\pgfsetdash{}{0pt}%
\pgfpathmoveto{\pgfqpoint{5.677189in}{1.609196in}}%
\pgfpathlineto{\pgfqpoint{5.685942in}{1.609196in}}%
\pgfpathlineto{\pgfqpoint{5.685942in}{1.626812in}}%
\pgfpathlineto{\pgfqpoint{5.677189in}{1.626812in}}%
\pgfpathlineto{\pgfqpoint{5.677189in}{1.609196in}}%
\pgfpathclose%
\pgfusepath{fill}%
\end{pgfscope}%
\begin{pgfscope}%
\pgfpathrectangle{\pgfqpoint{3.776708in}{0.600000in}}{\pgfqpoint{2.573292in}{2.070576in}}%
\pgfusepath{clip}%
\pgfsetbuttcap%
\pgfsetmiterjoin%
\definecolor{currentfill}{rgb}{0.066899,0.263188,0.377594}%
\pgfsetfillcolor{currentfill}%
\pgfsetlinewidth{0.000000pt}%
\definecolor{currentstroke}{rgb}{0.000000,0.000000,0.000000}%
\pgfsetstrokecolor{currentstroke}%
\pgfsetstrokeopacity{0.000000}%
\pgfsetdash{}{0pt}%
\pgfpathmoveto{\pgfqpoint{5.688131in}{1.609196in}}%
\pgfpathlineto{\pgfqpoint{5.696884in}{1.609196in}}%
\pgfpathlineto{\pgfqpoint{5.696884in}{1.624217in}}%
\pgfpathlineto{\pgfqpoint{5.688131in}{1.624217in}}%
\pgfpathlineto{\pgfqpoint{5.688131in}{1.609196in}}%
\pgfpathclose%
\pgfusepath{fill}%
\end{pgfscope}%
\begin{pgfscope}%
\pgfpathrectangle{\pgfqpoint{3.776708in}{0.600000in}}{\pgfqpoint{2.573292in}{2.070576in}}%
\pgfusepath{clip}%
\pgfsetbuttcap%
\pgfsetmiterjoin%
\definecolor{currentfill}{rgb}{0.066899,0.263188,0.377594}%
\pgfsetfillcolor{currentfill}%
\pgfsetlinewidth{0.000000pt}%
\definecolor{currentstroke}{rgb}{0.000000,0.000000,0.000000}%
\pgfsetstrokecolor{currentstroke}%
\pgfsetstrokeopacity{0.000000}%
\pgfsetdash{}{0pt}%
\pgfpathmoveto{\pgfqpoint{5.699072in}{1.609196in}}%
\pgfpathlineto{\pgfqpoint{5.707826in}{1.609196in}}%
\pgfpathlineto{\pgfqpoint{5.707826in}{1.619650in}}%
\pgfpathlineto{\pgfqpoint{5.699072in}{1.619650in}}%
\pgfpathlineto{\pgfqpoint{5.699072in}{1.609196in}}%
\pgfpathclose%
\pgfusepath{fill}%
\end{pgfscope}%
\begin{pgfscope}%
\pgfpathrectangle{\pgfqpoint{3.776708in}{0.600000in}}{\pgfqpoint{2.573292in}{2.070576in}}%
\pgfusepath{clip}%
\pgfsetbuttcap%
\pgfsetmiterjoin%
\definecolor{currentfill}{rgb}{0.066899,0.263188,0.377594}%
\pgfsetfillcolor{currentfill}%
\pgfsetlinewidth{0.000000pt}%
\definecolor{currentstroke}{rgb}{0.000000,0.000000,0.000000}%
\pgfsetstrokecolor{currentstroke}%
\pgfsetstrokeopacity{0.000000}%
\pgfsetdash{}{0pt}%
\pgfpathmoveto{\pgfqpoint{5.710014in}{1.609196in}}%
\pgfpathlineto{\pgfqpoint{5.718768in}{1.609196in}}%
\pgfpathlineto{\pgfqpoint{5.718768in}{1.614631in}}%
\pgfpathlineto{\pgfqpoint{5.710014in}{1.614631in}}%
\pgfpathlineto{\pgfqpoint{5.710014in}{1.609196in}}%
\pgfpathclose%
\pgfusepath{fill}%
\end{pgfscope}%
\begin{pgfscope}%
\pgfpathrectangle{\pgfqpoint{3.776708in}{0.600000in}}{\pgfqpoint{2.573292in}{2.070576in}}%
\pgfusepath{clip}%
\pgfsetbuttcap%
\pgfsetmiterjoin%
\definecolor{currentfill}{rgb}{0.066899,0.263188,0.377594}%
\pgfsetfillcolor{currentfill}%
\pgfsetlinewidth{0.000000pt}%
\definecolor{currentstroke}{rgb}{0.000000,0.000000,0.000000}%
\pgfsetstrokecolor{currentstroke}%
\pgfsetstrokeopacity{0.000000}%
\pgfsetdash{}{0pt}%
\pgfpathmoveto{\pgfqpoint{5.720956in}{1.609196in}}%
\pgfpathlineto{\pgfqpoint{5.729709in}{1.609196in}}%
\pgfpathlineto{\pgfqpoint{5.729709in}{1.612010in}}%
\pgfpathlineto{\pgfqpoint{5.720956in}{1.612010in}}%
\pgfpathlineto{\pgfqpoint{5.720956in}{1.609196in}}%
\pgfpathclose%
\pgfusepath{fill}%
\end{pgfscope}%
\begin{pgfscope}%
\pgfpathrectangle{\pgfqpoint{3.776708in}{0.600000in}}{\pgfqpoint{2.573292in}{2.070576in}}%
\pgfusepath{clip}%
\pgfsetbuttcap%
\pgfsetmiterjoin%
\definecolor{currentfill}{rgb}{0.066899,0.263188,0.377594}%
\pgfsetfillcolor{currentfill}%
\pgfsetlinewidth{0.000000pt}%
\definecolor{currentstroke}{rgb}{0.000000,0.000000,0.000000}%
\pgfsetstrokecolor{currentstroke}%
\pgfsetstrokeopacity{0.000000}%
\pgfsetdash{}{0pt}%
\pgfpathmoveto{\pgfqpoint{5.731898in}{1.609196in}}%
\pgfpathlineto{\pgfqpoint{5.740651in}{1.609196in}}%
\pgfpathlineto{\pgfqpoint{5.740651in}{1.609234in}}%
\pgfpathlineto{\pgfqpoint{5.731898in}{1.609234in}}%
\pgfpathlineto{\pgfqpoint{5.731898in}{1.609196in}}%
\pgfpathclose%
\pgfusepath{fill}%
\end{pgfscope}%
\begin{pgfscope}%
\pgfpathrectangle{\pgfqpoint{3.776708in}{0.600000in}}{\pgfqpoint{2.573292in}{2.070576in}}%
\pgfusepath{clip}%
\pgfsetbuttcap%
\pgfsetmiterjoin%
\definecolor{currentfill}{rgb}{0.066899,0.263188,0.377594}%
\pgfsetfillcolor{currentfill}%
\pgfsetlinewidth{0.000000pt}%
\definecolor{currentstroke}{rgb}{0.000000,0.000000,0.000000}%
\pgfsetstrokecolor{currentstroke}%
\pgfsetstrokeopacity{0.000000}%
\pgfsetdash{}{0pt}%
\pgfpathmoveto{\pgfqpoint{5.742840in}{1.609196in}}%
\pgfpathlineto{\pgfqpoint{5.751593in}{1.609196in}}%
\pgfpathlineto{\pgfqpoint{5.751593in}{1.603004in}}%
\pgfpathlineto{\pgfqpoint{5.742840in}{1.603004in}}%
\pgfpathlineto{\pgfqpoint{5.742840in}{1.609196in}}%
\pgfpathclose%
\pgfusepath{fill}%
\end{pgfscope}%
\begin{pgfscope}%
\pgfpathrectangle{\pgfqpoint{3.776708in}{0.600000in}}{\pgfqpoint{2.573292in}{2.070576in}}%
\pgfusepath{clip}%
\pgfsetbuttcap%
\pgfsetmiterjoin%
\definecolor{currentfill}{rgb}{0.066899,0.263188,0.377594}%
\pgfsetfillcolor{currentfill}%
\pgfsetlinewidth{0.000000pt}%
\definecolor{currentstroke}{rgb}{0.000000,0.000000,0.000000}%
\pgfsetstrokecolor{currentstroke}%
\pgfsetstrokeopacity{0.000000}%
\pgfsetdash{}{0pt}%
\pgfpathmoveto{\pgfqpoint{5.753781in}{1.609196in}}%
\pgfpathlineto{\pgfqpoint{5.762535in}{1.609196in}}%
\pgfpathlineto{\pgfqpoint{5.762535in}{1.594748in}}%
\pgfpathlineto{\pgfqpoint{5.753781in}{1.594748in}}%
\pgfpathlineto{\pgfqpoint{5.753781in}{1.609196in}}%
\pgfpathclose%
\pgfusepath{fill}%
\end{pgfscope}%
\begin{pgfscope}%
\pgfpathrectangle{\pgfqpoint{3.776708in}{0.600000in}}{\pgfqpoint{2.573292in}{2.070576in}}%
\pgfusepath{clip}%
\pgfsetbuttcap%
\pgfsetmiterjoin%
\definecolor{currentfill}{rgb}{0.066899,0.263188,0.377594}%
\pgfsetfillcolor{currentfill}%
\pgfsetlinewidth{0.000000pt}%
\definecolor{currentstroke}{rgb}{0.000000,0.000000,0.000000}%
\pgfsetstrokecolor{currentstroke}%
\pgfsetstrokeopacity{0.000000}%
\pgfsetdash{}{0pt}%
\pgfpathmoveto{\pgfqpoint{5.764723in}{1.609196in}}%
\pgfpathlineto{\pgfqpoint{5.773477in}{1.609196in}}%
\pgfpathlineto{\pgfqpoint{5.773477in}{1.583784in}}%
\pgfpathlineto{\pgfqpoint{5.764723in}{1.583784in}}%
\pgfpathlineto{\pgfqpoint{5.764723in}{1.609196in}}%
\pgfpathclose%
\pgfusepath{fill}%
\end{pgfscope}%
\begin{pgfscope}%
\pgfpathrectangle{\pgfqpoint{3.776708in}{0.600000in}}{\pgfqpoint{2.573292in}{2.070576in}}%
\pgfusepath{clip}%
\pgfsetbuttcap%
\pgfsetmiterjoin%
\definecolor{currentfill}{rgb}{0.066899,0.263188,0.377594}%
\pgfsetfillcolor{currentfill}%
\pgfsetlinewidth{0.000000pt}%
\definecolor{currentstroke}{rgb}{0.000000,0.000000,0.000000}%
\pgfsetstrokecolor{currentstroke}%
\pgfsetstrokeopacity{0.000000}%
\pgfsetdash{}{0pt}%
\pgfpathmoveto{\pgfqpoint{5.775665in}{1.609196in}}%
\pgfpathlineto{\pgfqpoint{5.784418in}{1.609196in}}%
\pgfpathlineto{\pgfqpoint{5.784418in}{1.569692in}}%
\pgfpathlineto{\pgfqpoint{5.775665in}{1.569692in}}%
\pgfpathlineto{\pgfqpoint{5.775665in}{1.609196in}}%
\pgfpathclose%
\pgfusepath{fill}%
\end{pgfscope}%
\begin{pgfscope}%
\pgfpathrectangle{\pgfqpoint{3.776708in}{0.600000in}}{\pgfqpoint{2.573292in}{2.070576in}}%
\pgfusepath{clip}%
\pgfsetbuttcap%
\pgfsetmiterjoin%
\definecolor{currentfill}{rgb}{0.066899,0.263188,0.377594}%
\pgfsetfillcolor{currentfill}%
\pgfsetlinewidth{0.000000pt}%
\definecolor{currentstroke}{rgb}{0.000000,0.000000,0.000000}%
\pgfsetstrokecolor{currentstroke}%
\pgfsetstrokeopacity{0.000000}%
\pgfsetdash{}{0pt}%
\pgfpathmoveto{\pgfqpoint{5.786607in}{1.609196in}}%
\pgfpathlineto{\pgfqpoint{5.795360in}{1.609196in}}%
\pgfpathlineto{\pgfqpoint{5.795360in}{1.558623in}}%
\pgfpathlineto{\pgfqpoint{5.786607in}{1.558623in}}%
\pgfpathlineto{\pgfqpoint{5.786607in}{1.609196in}}%
\pgfpathclose%
\pgfusepath{fill}%
\end{pgfscope}%
\begin{pgfscope}%
\pgfpathrectangle{\pgfqpoint{3.776708in}{0.600000in}}{\pgfqpoint{2.573292in}{2.070576in}}%
\pgfusepath{clip}%
\pgfsetbuttcap%
\pgfsetmiterjoin%
\definecolor{currentfill}{rgb}{0.066899,0.263188,0.377594}%
\pgfsetfillcolor{currentfill}%
\pgfsetlinewidth{0.000000pt}%
\definecolor{currentstroke}{rgb}{0.000000,0.000000,0.000000}%
\pgfsetstrokecolor{currentstroke}%
\pgfsetstrokeopacity{0.000000}%
\pgfsetdash{}{0pt}%
\pgfpathmoveto{\pgfqpoint{5.797549in}{1.609196in}}%
\pgfpathlineto{\pgfqpoint{5.806302in}{1.609196in}}%
\pgfpathlineto{\pgfqpoint{5.806302in}{1.548656in}}%
\pgfpathlineto{\pgfqpoint{5.797549in}{1.548656in}}%
\pgfpathlineto{\pgfqpoint{5.797549in}{1.609196in}}%
\pgfpathclose%
\pgfusepath{fill}%
\end{pgfscope}%
\begin{pgfscope}%
\pgfpathrectangle{\pgfqpoint{3.776708in}{0.600000in}}{\pgfqpoint{2.573292in}{2.070576in}}%
\pgfusepath{clip}%
\pgfsetbuttcap%
\pgfsetmiterjoin%
\definecolor{currentfill}{rgb}{0.066899,0.263188,0.377594}%
\pgfsetfillcolor{currentfill}%
\pgfsetlinewidth{0.000000pt}%
\definecolor{currentstroke}{rgb}{0.000000,0.000000,0.000000}%
\pgfsetstrokecolor{currentstroke}%
\pgfsetstrokeopacity{0.000000}%
\pgfsetdash{}{0pt}%
\pgfpathmoveto{\pgfqpoint{5.808490in}{1.609196in}}%
\pgfpathlineto{\pgfqpoint{5.817244in}{1.609196in}}%
\pgfpathlineto{\pgfqpoint{5.817244in}{1.539838in}}%
\pgfpathlineto{\pgfqpoint{5.808490in}{1.539838in}}%
\pgfpathlineto{\pgfqpoint{5.808490in}{1.609196in}}%
\pgfpathclose%
\pgfusepath{fill}%
\end{pgfscope}%
\begin{pgfscope}%
\pgfpathrectangle{\pgfqpoint{3.776708in}{0.600000in}}{\pgfqpoint{2.573292in}{2.070576in}}%
\pgfusepath{clip}%
\pgfsetbuttcap%
\pgfsetmiterjoin%
\definecolor{currentfill}{rgb}{0.066899,0.263188,0.377594}%
\pgfsetfillcolor{currentfill}%
\pgfsetlinewidth{0.000000pt}%
\definecolor{currentstroke}{rgb}{0.000000,0.000000,0.000000}%
\pgfsetstrokecolor{currentstroke}%
\pgfsetstrokeopacity{0.000000}%
\pgfsetdash{}{0pt}%
\pgfpathmoveto{\pgfqpoint{5.819432in}{1.609196in}}%
\pgfpathlineto{\pgfqpoint{5.828186in}{1.609196in}}%
\pgfpathlineto{\pgfqpoint{5.828186in}{1.531718in}}%
\pgfpathlineto{\pgfqpoint{5.819432in}{1.531718in}}%
\pgfpathlineto{\pgfqpoint{5.819432in}{1.609196in}}%
\pgfpathclose%
\pgfusepath{fill}%
\end{pgfscope}%
\begin{pgfscope}%
\pgfpathrectangle{\pgfqpoint{3.776708in}{0.600000in}}{\pgfqpoint{2.573292in}{2.070576in}}%
\pgfusepath{clip}%
\pgfsetbuttcap%
\pgfsetmiterjoin%
\definecolor{currentfill}{rgb}{0.066899,0.263188,0.377594}%
\pgfsetfillcolor{currentfill}%
\pgfsetlinewidth{0.000000pt}%
\definecolor{currentstroke}{rgb}{0.000000,0.000000,0.000000}%
\pgfsetstrokecolor{currentstroke}%
\pgfsetstrokeopacity{0.000000}%
\pgfsetdash{}{0pt}%
\pgfpathmoveto{\pgfqpoint{5.830374in}{1.609196in}}%
\pgfpathlineto{\pgfqpoint{5.839127in}{1.609196in}}%
\pgfpathlineto{\pgfqpoint{5.839127in}{1.526790in}}%
\pgfpathlineto{\pgfqpoint{5.830374in}{1.526790in}}%
\pgfpathlineto{\pgfqpoint{5.830374in}{1.609196in}}%
\pgfpathclose%
\pgfusepath{fill}%
\end{pgfscope}%
\begin{pgfscope}%
\pgfpathrectangle{\pgfqpoint{3.776708in}{0.600000in}}{\pgfqpoint{2.573292in}{2.070576in}}%
\pgfusepath{clip}%
\pgfsetbuttcap%
\pgfsetmiterjoin%
\definecolor{currentfill}{rgb}{0.066899,0.263188,0.377594}%
\pgfsetfillcolor{currentfill}%
\pgfsetlinewidth{0.000000pt}%
\definecolor{currentstroke}{rgb}{0.000000,0.000000,0.000000}%
\pgfsetstrokecolor{currentstroke}%
\pgfsetstrokeopacity{0.000000}%
\pgfsetdash{}{0pt}%
\pgfpathmoveto{\pgfqpoint{5.841316in}{1.609196in}}%
\pgfpathlineto{\pgfqpoint{5.850069in}{1.609196in}}%
\pgfpathlineto{\pgfqpoint{5.850069in}{1.520032in}}%
\pgfpathlineto{\pgfqpoint{5.841316in}{1.520032in}}%
\pgfpathlineto{\pgfqpoint{5.841316in}{1.609196in}}%
\pgfpathclose%
\pgfusepath{fill}%
\end{pgfscope}%
\begin{pgfscope}%
\pgfpathrectangle{\pgfqpoint{3.776708in}{0.600000in}}{\pgfqpoint{2.573292in}{2.070576in}}%
\pgfusepath{clip}%
\pgfsetbuttcap%
\pgfsetmiterjoin%
\definecolor{currentfill}{rgb}{0.066899,0.263188,0.377594}%
\pgfsetfillcolor{currentfill}%
\pgfsetlinewidth{0.000000pt}%
\definecolor{currentstroke}{rgb}{0.000000,0.000000,0.000000}%
\pgfsetstrokecolor{currentstroke}%
\pgfsetstrokeopacity{0.000000}%
\pgfsetdash{}{0pt}%
\pgfpathmoveto{\pgfqpoint{5.852258in}{1.609196in}}%
\pgfpathlineto{\pgfqpoint{5.861011in}{1.609196in}}%
\pgfpathlineto{\pgfqpoint{5.861011in}{1.513669in}}%
\pgfpathlineto{\pgfqpoint{5.852258in}{1.513669in}}%
\pgfpathlineto{\pgfqpoint{5.852258in}{1.609196in}}%
\pgfpathclose%
\pgfusepath{fill}%
\end{pgfscope}%
\begin{pgfscope}%
\pgfpathrectangle{\pgfqpoint{3.776708in}{0.600000in}}{\pgfqpoint{2.573292in}{2.070576in}}%
\pgfusepath{clip}%
\pgfsetbuttcap%
\pgfsetmiterjoin%
\definecolor{currentfill}{rgb}{0.066899,0.263188,0.377594}%
\pgfsetfillcolor{currentfill}%
\pgfsetlinewidth{0.000000pt}%
\definecolor{currentstroke}{rgb}{0.000000,0.000000,0.000000}%
\pgfsetstrokecolor{currentstroke}%
\pgfsetstrokeopacity{0.000000}%
\pgfsetdash{}{0pt}%
\pgfpathmoveto{\pgfqpoint{5.863199in}{1.609196in}}%
\pgfpathlineto{\pgfqpoint{5.871953in}{1.609196in}}%
\pgfpathlineto{\pgfqpoint{5.871953in}{1.509660in}}%
\pgfpathlineto{\pgfqpoint{5.863199in}{1.509660in}}%
\pgfpathlineto{\pgfqpoint{5.863199in}{1.609196in}}%
\pgfpathclose%
\pgfusepath{fill}%
\end{pgfscope}%
\begin{pgfscope}%
\pgfpathrectangle{\pgfqpoint{3.776708in}{0.600000in}}{\pgfqpoint{2.573292in}{2.070576in}}%
\pgfusepath{clip}%
\pgfsetbuttcap%
\pgfsetmiterjoin%
\definecolor{currentfill}{rgb}{0.066899,0.263188,0.377594}%
\pgfsetfillcolor{currentfill}%
\pgfsetlinewidth{0.000000pt}%
\definecolor{currentstroke}{rgb}{0.000000,0.000000,0.000000}%
\pgfsetstrokecolor{currentstroke}%
\pgfsetstrokeopacity{0.000000}%
\pgfsetdash{}{0pt}%
\pgfpathmoveto{\pgfqpoint{5.874141in}{1.609196in}}%
\pgfpathlineto{\pgfqpoint{5.882895in}{1.609196in}}%
\pgfpathlineto{\pgfqpoint{5.882895in}{1.506209in}}%
\pgfpathlineto{\pgfqpoint{5.874141in}{1.506209in}}%
\pgfpathlineto{\pgfqpoint{5.874141in}{1.609196in}}%
\pgfpathclose%
\pgfusepath{fill}%
\end{pgfscope}%
\begin{pgfscope}%
\pgfpathrectangle{\pgfqpoint{3.776708in}{0.600000in}}{\pgfqpoint{2.573292in}{2.070576in}}%
\pgfusepath{clip}%
\pgfsetbuttcap%
\pgfsetmiterjoin%
\definecolor{currentfill}{rgb}{0.066899,0.263188,0.377594}%
\pgfsetfillcolor{currentfill}%
\pgfsetlinewidth{0.000000pt}%
\definecolor{currentstroke}{rgb}{0.000000,0.000000,0.000000}%
\pgfsetstrokecolor{currentstroke}%
\pgfsetstrokeopacity{0.000000}%
\pgfsetdash{}{0pt}%
\pgfpathmoveto{\pgfqpoint{5.885083in}{1.609196in}}%
\pgfpathlineto{\pgfqpoint{5.893836in}{1.609196in}}%
\pgfpathlineto{\pgfqpoint{5.893836in}{1.505150in}}%
\pgfpathlineto{\pgfqpoint{5.885083in}{1.505150in}}%
\pgfpathlineto{\pgfqpoint{5.885083in}{1.609196in}}%
\pgfpathclose%
\pgfusepath{fill}%
\end{pgfscope}%
\begin{pgfscope}%
\pgfpathrectangle{\pgfqpoint{3.776708in}{0.600000in}}{\pgfqpoint{2.573292in}{2.070576in}}%
\pgfusepath{clip}%
\pgfsetbuttcap%
\pgfsetmiterjoin%
\definecolor{currentfill}{rgb}{0.066899,0.263188,0.377594}%
\pgfsetfillcolor{currentfill}%
\pgfsetlinewidth{0.000000pt}%
\definecolor{currentstroke}{rgb}{0.000000,0.000000,0.000000}%
\pgfsetstrokecolor{currentstroke}%
\pgfsetstrokeopacity{0.000000}%
\pgfsetdash{}{0pt}%
\pgfpathmoveto{\pgfqpoint{5.896025in}{1.609196in}}%
\pgfpathlineto{\pgfqpoint{5.904778in}{1.609196in}}%
\pgfpathlineto{\pgfqpoint{5.904778in}{1.503887in}}%
\pgfpathlineto{\pgfqpoint{5.896025in}{1.503887in}}%
\pgfpathlineto{\pgfqpoint{5.896025in}{1.609196in}}%
\pgfpathclose%
\pgfusepath{fill}%
\end{pgfscope}%
\begin{pgfscope}%
\pgfpathrectangle{\pgfqpoint{3.776708in}{0.600000in}}{\pgfqpoint{2.573292in}{2.070576in}}%
\pgfusepath{clip}%
\pgfsetbuttcap%
\pgfsetmiterjoin%
\definecolor{currentfill}{rgb}{0.066899,0.263188,0.377594}%
\pgfsetfillcolor{currentfill}%
\pgfsetlinewidth{0.000000pt}%
\definecolor{currentstroke}{rgb}{0.000000,0.000000,0.000000}%
\pgfsetstrokecolor{currentstroke}%
\pgfsetstrokeopacity{0.000000}%
\pgfsetdash{}{0pt}%
\pgfpathmoveto{\pgfqpoint{5.906967in}{1.609196in}}%
\pgfpathlineto{\pgfqpoint{5.915720in}{1.609196in}}%
\pgfpathlineto{\pgfqpoint{5.915720in}{1.504897in}}%
\pgfpathlineto{\pgfqpoint{5.906967in}{1.504897in}}%
\pgfpathlineto{\pgfqpoint{5.906967in}{1.609196in}}%
\pgfpathclose%
\pgfusepath{fill}%
\end{pgfscope}%
\begin{pgfscope}%
\pgfpathrectangle{\pgfqpoint{3.776708in}{0.600000in}}{\pgfqpoint{2.573292in}{2.070576in}}%
\pgfusepath{clip}%
\pgfsetbuttcap%
\pgfsetmiterjoin%
\definecolor{currentfill}{rgb}{0.066899,0.263188,0.377594}%
\pgfsetfillcolor{currentfill}%
\pgfsetlinewidth{0.000000pt}%
\definecolor{currentstroke}{rgb}{0.000000,0.000000,0.000000}%
\pgfsetstrokecolor{currentstroke}%
\pgfsetstrokeopacity{0.000000}%
\pgfsetdash{}{0pt}%
\pgfpathmoveto{\pgfqpoint{5.917908in}{1.609196in}}%
\pgfpathlineto{\pgfqpoint{5.926662in}{1.609196in}}%
\pgfpathlineto{\pgfqpoint{5.926662in}{1.505862in}}%
\pgfpathlineto{\pgfqpoint{5.917908in}{1.505862in}}%
\pgfpathlineto{\pgfqpoint{5.917908in}{1.609196in}}%
\pgfpathclose%
\pgfusepath{fill}%
\end{pgfscope}%
\begin{pgfscope}%
\pgfpathrectangle{\pgfqpoint{3.776708in}{0.600000in}}{\pgfqpoint{2.573292in}{2.070576in}}%
\pgfusepath{clip}%
\pgfsetbuttcap%
\pgfsetmiterjoin%
\definecolor{currentfill}{rgb}{0.066899,0.263188,0.377594}%
\pgfsetfillcolor{currentfill}%
\pgfsetlinewidth{0.000000pt}%
\definecolor{currentstroke}{rgb}{0.000000,0.000000,0.000000}%
\pgfsetstrokecolor{currentstroke}%
\pgfsetstrokeopacity{0.000000}%
\pgfsetdash{}{0pt}%
\pgfpathmoveto{\pgfqpoint{5.928850in}{1.609196in}}%
\pgfpathlineto{\pgfqpoint{5.937604in}{1.609196in}}%
\pgfpathlineto{\pgfqpoint{5.937604in}{1.509252in}}%
\pgfpathlineto{\pgfqpoint{5.928850in}{1.509252in}}%
\pgfpathlineto{\pgfqpoint{5.928850in}{1.609196in}}%
\pgfpathclose%
\pgfusepath{fill}%
\end{pgfscope}%
\begin{pgfscope}%
\pgfpathrectangle{\pgfqpoint{3.776708in}{0.600000in}}{\pgfqpoint{2.573292in}{2.070576in}}%
\pgfusepath{clip}%
\pgfsetbuttcap%
\pgfsetmiterjoin%
\definecolor{currentfill}{rgb}{0.066899,0.263188,0.377594}%
\pgfsetfillcolor{currentfill}%
\pgfsetlinewidth{0.000000pt}%
\definecolor{currentstroke}{rgb}{0.000000,0.000000,0.000000}%
\pgfsetstrokecolor{currentstroke}%
\pgfsetstrokeopacity{0.000000}%
\pgfsetdash{}{0pt}%
\pgfpathmoveto{\pgfqpoint{5.939792in}{1.609196in}}%
\pgfpathlineto{\pgfqpoint{5.948545in}{1.609196in}}%
\pgfpathlineto{\pgfqpoint{5.948545in}{1.511834in}}%
\pgfpathlineto{\pgfqpoint{5.939792in}{1.511834in}}%
\pgfpathlineto{\pgfqpoint{5.939792in}{1.609196in}}%
\pgfpathclose%
\pgfusepath{fill}%
\end{pgfscope}%
\begin{pgfscope}%
\pgfpathrectangle{\pgfqpoint{3.776708in}{0.600000in}}{\pgfqpoint{2.573292in}{2.070576in}}%
\pgfusepath{clip}%
\pgfsetbuttcap%
\pgfsetmiterjoin%
\definecolor{currentfill}{rgb}{0.066899,0.263188,0.377594}%
\pgfsetfillcolor{currentfill}%
\pgfsetlinewidth{0.000000pt}%
\definecolor{currentstroke}{rgb}{0.000000,0.000000,0.000000}%
\pgfsetstrokecolor{currentstroke}%
\pgfsetstrokeopacity{0.000000}%
\pgfsetdash{}{0pt}%
\pgfpathmoveto{\pgfqpoint{5.950734in}{1.609196in}}%
\pgfpathlineto{\pgfqpoint{5.959487in}{1.609196in}}%
\pgfpathlineto{\pgfqpoint{5.959487in}{1.516279in}}%
\pgfpathlineto{\pgfqpoint{5.950734in}{1.516279in}}%
\pgfpathlineto{\pgfqpoint{5.950734in}{1.609196in}}%
\pgfpathclose%
\pgfusepath{fill}%
\end{pgfscope}%
\begin{pgfscope}%
\pgfpathrectangle{\pgfqpoint{3.776708in}{0.600000in}}{\pgfqpoint{2.573292in}{2.070576in}}%
\pgfusepath{clip}%
\pgfsetbuttcap%
\pgfsetmiterjoin%
\definecolor{currentfill}{rgb}{0.066899,0.263188,0.377594}%
\pgfsetfillcolor{currentfill}%
\pgfsetlinewidth{0.000000pt}%
\definecolor{currentstroke}{rgb}{0.000000,0.000000,0.000000}%
\pgfsetstrokecolor{currentstroke}%
\pgfsetstrokeopacity{0.000000}%
\pgfsetdash{}{0pt}%
\pgfpathmoveto{\pgfqpoint{5.961676in}{1.609196in}}%
\pgfpathlineto{\pgfqpoint{5.970429in}{1.609196in}}%
\pgfpathlineto{\pgfqpoint{5.970429in}{1.521732in}}%
\pgfpathlineto{\pgfqpoint{5.961676in}{1.521732in}}%
\pgfpathlineto{\pgfqpoint{5.961676in}{1.609196in}}%
\pgfpathclose%
\pgfusepath{fill}%
\end{pgfscope}%
\begin{pgfscope}%
\pgfpathrectangle{\pgfqpoint{3.776708in}{0.600000in}}{\pgfqpoint{2.573292in}{2.070576in}}%
\pgfusepath{clip}%
\pgfsetbuttcap%
\pgfsetmiterjoin%
\definecolor{currentfill}{rgb}{0.066899,0.263188,0.377594}%
\pgfsetfillcolor{currentfill}%
\pgfsetlinewidth{0.000000pt}%
\definecolor{currentstroke}{rgb}{0.000000,0.000000,0.000000}%
\pgfsetstrokecolor{currentstroke}%
\pgfsetstrokeopacity{0.000000}%
\pgfsetdash{}{0pt}%
\pgfpathmoveto{\pgfqpoint{5.972617in}{1.609196in}}%
\pgfpathlineto{\pgfqpoint{5.981371in}{1.609196in}}%
\pgfpathlineto{\pgfqpoint{5.981371in}{1.526935in}}%
\pgfpathlineto{\pgfqpoint{5.972617in}{1.526935in}}%
\pgfpathlineto{\pgfqpoint{5.972617in}{1.609196in}}%
\pgfpathclose%
\pgfusepath{fill}%
\end{pgfscope}%
\begin{pgfscope}%
\pgfpathrectangle{\pgfqpoint{3.776708in}{0.600000in}}{\pgfqpoint{2.573292in}{2.070576in}}%
\pgfusepath{clip}%
\pgfsetbuttcap%
\pgfsetmiterjoin%
\definecolor{currentfill}{rgb}{0.066899,0.263188,0.377594}%
\pgfsetfillcolor{currentfill}%
\pgfsetlinewidth{0.000000pt}%
\definecolor{currentstroke}{rgb}{0.000000,0.000000,0.000000}%
\pgfsetstrokecolor{currentstroke}%
\pgfsetstrokeopacity{0.000000}%
\pgfsetdash{}{0pt}%
\pgfpathmoveto{\pgfqpoint{5.983559in}{1.609196in}}%
\pgfpathlineto{\pgfqpoint{5.992313in}{1.609196in}}%
\pgfpathlineto{\pgfqpoint{5.992313in}{1.532992in}}%
\pgfpathlineto{\pgfqpoint{5.983559in}{1.532992in}}%
\pgfpathlineto{\pgfqpoint{5.983559in}{1.609196in}}%
\pgfpathclose%
\pgfusepath{fill}%
\end{pgfscope}%
\begin{pgfscope}%
\pgfpathrectangle{\pgfqpoint{3.776708in}{0.600000in}}{\pgfqpoint{2.573292in}{2.070576in}}%
\pgfusepath{clip}%
\pgfsetbuttcap%
\pgfsetmiterjoin%
\definecolor{currentfill}{rgb}{0.066899,0.263188,0.377594}%
\pgfsetfillcolor{currentfill}%
\pgfsetlinewidth{0.000000pt}%
\definecolor{currentstroke}{rgb}{0.000000,0.000000,0.000000}%
\pgfsetstrokecolor{currentstroke}%
\pgfsetstrokeopacity{0.000000}%
\pgfsetdash{}{0pt}%
\pgfpathmoveto{\pgfqpoint{5.994501in}{1.609196in}}%
\pgfpathlineto{\pgfqpoint{6.003254in}{1.609196in}}%
\pgfpathlineto{\pgfqpoint{6.003254in}{1.539475in}}%
\pgfpathlineto{\pgfqpoint{5.994501in}{1.539475in}}%
\pgfpathlineto{\pgfqpoint{5.994501in}{1.609196in}}%
\pgfpathclose%
\pgfusepath{fill}%
\end{pgfscope}%
\begin{pgfscope}%
\pgfpathrectangle{\pgfqpoint{3.776708in}{0.600000in}}{\pgfqpoint{2.573292in}{2.070576in}}%
\pgfusepath{clip}%
\pgfsetbuttcap%
\pgfsetmiterjoin%
\definecolor{currentfill}{rgb}{0.066899,0.263188,0.377594}%
\pgfsetfillcolor{currentfill}%
\pgfsetlinewidth{0.000000pt}%
\definecolor{currentstroke}{rgb}{0.000000,0.000000,0.000000}%
\pgfsetstrokecolor{currentstroke}%
\pgfsetstrokeopacity{0.000000}%
\pgfsetdash{}{0pt}%
\pgfpathmoveto{\pgfqpoint{6.005443in}{1.609196in}}%
\pgfpathlineto{\pgfqpoint{6.014196in}{1.609196in}}%
\pgfpathlineto{\pgfqpoint{6.014196in}{1.544856in}}%
\pgfpathlineto{\pgfqpoint{6.005443in}{1.544856in}}%
\pgfpathlineto{\pgfqpoint{6.005443in}{1.609196in}}%
\pgfpathclose%
\pgfusepath{fill}%
\end{pgfscope}%
\begin{pgfscope}%
\pgfpathrectangle{\pgfqpoint{3.776708in}{0.600000in}}{\pgfqpoint{2.573292in}{2.070576in}}%
\pgfusepath{clip}%
\pgfsetbuttcap%
\pgfsetmiterjoin%
\definecolor{currentfill}{rgb}{0.066899,0.263188,0.377594}%
\pgfsetfillcolor{currentfill}%
\pgfsetlinewidth{0.000000pt}%
\definecolor{currentstroke}{rgb}{0.000000,0.000000,0.000000}%
\pgfsetstrokecolor{currentstroke}%
\pgfsetstrokeopacity{0.000000}%
\pgfsetdash{}{0pt}%
\pgfpathmoveto{\pgfqpoint{6.016385in}{1.609196in}}%
\pgfpathlineto{\pgfqpoint{6.025138in}{1.609196in}}%
\pgfpathlineto{\pgfqpoint{6.025138in}{1.551295in}}%
\pgfpathlineto{\pgfqpoint{6.016385in}{1.551295in}}%
\pgfpathlineto{\pgfqpoint{6.016385in}{1.609196in}}%
\pgfpathclose%
\pgfusepath{fill}%
\end{pgfscope}%
\begin{pgfscope}%
\pgfpathrectangle{\pgfqpoint{3.776708in}{0.600000in}}{\pgfqpoint{2.573292in}{2.070576in}}%
\pgfusepath{clip}%
\pgfsetbuttcap%
\pgfsetmiterjoin%
\definecolor{currentfill}{rgb}{0.066899,0.263188,0.377594}%
\pgfsetfillcolor{currentfill}%
\pgfsetlinewidth{0.000000pt}%
\definecolor{currentstroke}{rgb}{0.000000,0.000000,0.000000}%
\pgfsetstrokecolor{currentstroke}%
\pgfsetstrokeopacity{0.000000}%
\pgfsetdash{}{0pt}%
\pgfpathmoveto{\pgfqpoint{6.027326in}{1.609196in}}%
\pgfpathlineto{\pgfqpoint{6.036080in}{1.609196in}}%
\pgfpathlineto{\pgfqpoint{6.036080in}{1.559615in}}%
\pgfpathlineto{\pgfqpoint{6.027326in}{1.559615in}}%
\pgfpathlineto{\pgfqpoint{6.027326in}{1.609196in}}%
\pgfpathclose%
\pgfusepath{fill}%
\end{pgfscope}%
\begin{pgfscope}%
\pgfpathrectangle{\pgfqpoint{3.776708in}{0.600000in}}{\pgfqpoint{2.573292in}{2.070576in}}%
\pgfusepath{clip}%
\pgfsetbuttcap%
\pgfsetmiterjoin%
\definecolor{currentfill}{rgb}{0.066899,0.263188,0.377594}%
\pgfsetfillcolor{currentfill}%
\pgfsetlinewidth{0.000000pt}%
\definecolor{currentstroke}{rgb}{0.000000,0.000000,0.000000}%
\pgfsetstrokecolor{currentstroke}%
\pgfsetstrokeopacity{0.000000}%
\pgfsetdash{}{0pt}%
\pgfpathmoveto{\pgfqpoint{6.038268in}{1.609196in}}%
\pgfpathlineto{\pgfqpoint{6.047022in}{1.609196in}}%
\pgfpathlineto{\pgfqpoint{6.047022in}{1.565303in}}%
\pgfpathlineto{\pgfqpoint{6.038268in}{1.565303in}}%
\pgfpathlineto{\pgfqpoint{6.038268in}{1.609196in}}%
\pgfpathclose%
\pgfusepath{fill}%
\end{pgfscope}%
\begin{pgfscope}%
\pgfpathrectangle{\pgfqpoint{3.776708in}{0.600000in}}{\pgfqpoint{2.573292in}{2.070576in}}%
\pgfusepath{clip}%
\pgfsetbuttcap%
\pgfsetmiterjoin%
\definecolor{currentfill}{rgb}{0.066899,0.263188,0.377594}%
\pgfsetfillcolor{currentfill}%
\pgfsetlinewidth{0.000000pt}%
\definecolor{currentstroke}{rgb}{0.000000,0.000000,0.000000}%
\pgfsetstrokecolor{currentstroke}%
\pgfsetstrokeopacity{0.000000}%
\pgfsetdash{}{0pt}%
\pgfpathmoveto{\pgfqpoint{6.049210in}{1.609196in}}%
\pgfpathlineto{\pgfqpoint{6.057963in}{1.609196in}}%
\pgfpathlineto{\pgfqpoint{6.057963in}{1.572901in}}%
\pgfpathlineto{\pgfqpoint{6.049210in}{1.572901in}}%
\pgfpathlineto{\pgfqpoint{6.049210in}{1.609196in}}%
\pgfpathclose%
\pgfusepath{fill}%
\end{pgfscope}%
\begin{pgfscope}%
\pgfpathrectangle{\pgfqpoint{3.776708in}{0.600000in}}{\pgfqpoint{2.573292in}{2.070576in}}%
\pgfusepath{clip}%
\pgfsetbuttcap%
\pgfsetmiterjoin%
\definecolor{currentfill}{rgb}{0.066899,0.263188,0.377594}%
\pgfsetfillcolor{currentfill}%
\pgfsetlinewidth{0.000000pt}%
\definecolor{currentstroke}{rgb}{0.000000,0.000000,0.000000}%
\pgfsetstrokecolor{currentstroke}%
\pgfsetstrokeopacity{0.000000}%
\pgfsetdash{}{0pt}%
\pgfpathmoveto{\pgfqpoint{6.060152in}{1.609196in}}%
\pgfpathlineto{\pgfqpoint{6.068905in}{1.609196in}}%
\pgfpathlineto{\pgfqpoint{6.068905in}{1.578572in}}%
\pgfpathlineto{\pgfqpoint{6.060152in}{1.578572in}}%
\pgfpathlineto{\pgfqpoint{6.060152in}{1.609196in}}%
\pgfpathclose%
\pgfusepath{fill}%
\end{pgfscope}%
\begin{pgfscope}%
\pgfpathrectangle{\pgfqpoint{3.776708in}{0.600000in}}{\pgfqpoint{2.573292in}{2.070576in}}%
\pgfusepath{clip}%
\pgfsetbuttcap%
\pgfsetmiterjoin%
\definecolor{currentfill}{rgb}{0.066899,0.263188,0.377594}%
\pgfsetfillcolor{currentfill}%
\pgfsetlinewidth{0.000000pt}%
\definecolor{currentstroke}{rgb}{0.000000,0.000000,0.000000}%
\pgfsetstrokecolor{currentstroke}%
\pgfsetstrokeopacity{0.000000}%
\pgfsetdash{}{0pt}%
\pgfpathmoveto{\pgfqpoint{6.071094in}{1.609196in}}%
\pgfpathlineto{\pgfqpoint{6.079847in}{1.609196in}}%
\pgfpathlineto{\pgfqpoint{6.079847in}{1.586559in}}%
\pgfpathlineto{\pgfqpoint{6.071094in}{1.586559in}}%
\pgfpathlineto{\pgfqpoint{6.071094in}{1.609196in}}%
\pgfpathclose%
\pgfusepath{fill}%
\end{pgfscope}%
\begin{pgfscope}%
\pgfpathrectangle{\pgfqpoint{3.776708in}{0.600000in}}{\pgfqpoint{2.573292in}{2.070576in}}%
\pgfusepath{clip}%
\pgfsetbuttcap%
\pgfsetmiterjoin%
\definecolor{currentfill}{rgb}{0.066899,0.263188,0.377594}%
\pgfsetfillcolor{currentfill}%
\pgfsetlinewidth{0.000000pt}%
\definecolor{currentstroke}{rgb}{0.000000,0.000000,0.000000}%
\pgfsetstrokecolor{currentstroke}%
\pgfsetstrokeopacity{0.000000}%
\pgfsetdash{}{0pt}%
\pgfpathmoveto{\pgfqpoint{6.082035in}{1.609196in}}%
\pgfpathlineto{\pgfqpoint{6.090789in}{1.609196in}}%
\pgfpathlineto{\pgfqpoint{6.090789in}{1.592613in}}%
\pgfpathlineto{\pgfqpoint{6.082035in}{1.592613in}}%
\pgfpathlineto{\pgfqpoint{6.082035in}{1.609196in}}%
\pgfpathclose%
\pgfusepath{fill}%
\end{pgfscope}%
\begin{pgfscope}%
\pgfpathrectangle{\pgfqpoint{3.776708in}{0.600000in}}{\pgfqpoint{2.573292in}{2.070576in}}%
\pgfusepath{clip}%
\pgfsetbuttcap%
\pgfsetmiterjoin%
\definecolor{currentfill}{rgb}{0.066899,0.263188,0.377594}%
\pgfsetfillcolor{currentfill}%
\pgfsetlinewidth{0.000000pt}%
\definecolor{currentstroke}{rgb}{0.000000,0.000000,0.000000}%
\pgfsetstrokecolor{currentstroke}%
\pgfsetstrokeopacity{0.000000}%
\pgfsetdash{}{0pt}%
\pgfpathmoveto{\pgfqpoint{6.092977in}{1.609196in}}%
\pgfpathlineto{\pgfqpoint{6.101731in}{1.609196in}}%
\pgfpathlineto{\pgfqpoint{6.101731in}{1.599631in}}%
\pgfpathlineto{\pgfqpoint{6.092977in}{1.599631in}}%
\pgfpathlineto{\pgfqpoint{6.092977in}{1.609196in}}%
\pgfpathclose%
\pgfusepath{fill}%
\end{pgfscope}%
\begin{pgfscope}%
\pgfpathrectangle{\pgfqpoint{3.776708in}{0.600000in}}{\pgfqpoint{2.573292in}{2.070576in}}%
\pgfusepath{clip}%
\pgfsetbuttcap%
\pgfsetmiterjoin%
\definecolor{currentfill}{rgb}{0.066899,0.263188,0.377594}%
\pgfsetfillcolor{currentfill}%
\pgfsetlinewidth{0.000000pt}%
\definecolor{currentstroke}{rgb}{0.000000,0.000000,0.000000}%
\pgfsetstrokecolor{currentstroke}%
\pgfsetstrokeopacity{0.000000}%
\pgfsetdash{}{0pt}%
\pgfpathmoveto{\pgfqpoint{6.103919in}{1.609196in}}%
\pgfpathlineto{\pgfqpoint{6.112672in}{1.609196in}}%
\pgfpathlineto{\pgfqpoint{6.112672in}{1.604411in}}%
\pgfpathlineto{\pgfqpoint{6.103919in}{1.604411in}}%
\pgfpathlineto{\pgfqpoint{6.103919in}{1.609196in}}%
\pgfpathclose%
\pgfusepath{fill}%
\end{pgfscope}%
\begin{pgfscope}%
\pgfpathrectangle{\pgfqpoint{3.776708in}{0.600000in}}{\pgfqpoint{2.573292in}{2.070576in}}%
\pgfusepath{clip}%
\pgfsetbuttcap%
\pgfsetmiterjoin%
\definecolor{currentfill}{rgb}{0.066899,0.263188,0.377594}%
\pgfsetfillcolor{currentfill}%
\pgfsetlinewidth{0.000000pt}%
\definecolor{currentstroke}{rgb}{0.000000,0.000000,0.000000}%
\pgfsetstrokecolor{currentstroke}%
\pgfsetstrokeopacity{0.000000}%
\pgfsetdash{}{0pt}%
\pgfpathmoveto{\pgfqpoint{6.114861in}{1.609196in}}%
\pgfpathlineto{\pgfqpoint{6.123614in}{1.609196in}}%
\pgfpathlineto{\pgfqpoint{6.123614in}{1.609314in}}%
\pgfpathlineto{\pgfqpoint{6.114861in}{1.609314in}}%
\pgfpathlineto{\pgfqpoint{6.114861in}{1.609196in}}%
\pgfpathclose%
\pgfusepath{fill}%
\end{pgfscope}%
\begin{pgfscope}%
\pgfpathrectangle{\pgfqpoint{3.776708in}{0.600000in}}{\pgfqpoint{2.573292in}{2.070576in}}%
\pgfusepath{clip}%
\pgfsetbuttcap%
\pgfsetmiterjoin%
\definecolor{currentfill}{rgb}{0.066899,0.263188,0.377594}%
\pgfsetfillcolor{currentfill}%
\pgfsetlinewidth{0.000000pt}%
\definecolor{currentstroke}{rgb}{0.000000,0.000000,0.000000}%
\pgfsetstrokecolor{currentstroke}%
\pgfsetstrokeopacity{0.000000}%
\pgfsetdash{}{0pt}%
\pgfpathmoveto{\pgfqpoint{6.125803in}{1.609196in}}%
\pgfpathlineto{\pgfqpoint{6.134556in}{1.609196in}}%
\pgfpathlineto{\pgfqpoint{6.134556in}{1.616040in}}%
\pgfpathlineto{\pgfqpoint{6.125803in}{1.616040in}}%
\pgfpathlineto{\pgfqpoint{6.125803in}{1.609196in}}%
\pgfpathclose%
\pgfusepath{fill}%
\end{pgfscope}%
\begin{pgfscope}%
\pgfpathrectangle{\pgfqpoint{3.776708in}{0.600000in}}{\pgfqpoint{2.573292in}{2.070576in}}%
\pgfusepath{clip}%
\pgfsetbuttcap%
\pgfsetmiterjoin%
\definecolor{currentfill}{rgb}{0.066899,0.263188,0.377594}%
\pgfsetfillcolor{currentfill}%
\pgfsetlinewidth{0.000000pt}%
\definecolor{currentstroke}{rgb}{0.000000,0.000000,0.000000}%
\pgfsetstrokecolor{currentstroke}%
\pgfsetstrokeopacity{0.000000}%
\pgfsetdash{}{0pt}%
\pgfpathmoveto{\pgfqpoint{6.136744in}{1.609196in}}%
\pgfpathlineto{\pgfqpoint{6.145498in}{1.609196in}}%
\pgfpathlineto{\pgfqpoint{6.145498in}{1.622954in}}%
\pgfpathlineto{\pgfqpoint{6.136744in}{1.622954in}}%
\pgfpathlineto{\pgfqpoint{6.136744in}{1.609196in}}%
\pgfpathclose%
\pgfusepath{fill}%
\end{pgfscope}%
\begin{pgfscope}%
\pgfpathrectangle{\pgfqpoint{3.776708in}{0.600000in}}{\pgfqpoint{2.573292in}{2.070576in}}%
\pgfusepath{clip}%
\pgfsetbuttcap%
\pgfsetmiterjoin%
\definecolor{currentfill}{rgb}{0.066899,0.263188,0.377594}%
\pgfsetfillcolor{currentfill}%
\pgfsetlinewidth{0.000000pt}%
\definecolor{currentstroke}{rgb}{0.000000,0.000000,0.000000}%
\pgfsetstrokecolor{currentstroke}%
\pgfsetstrokeopacity{0.000000}%
\pgfsetdash{}{0pt}%
\pgfpathmoveto{\pgfqpoint{6.147686in}{1.609196in}}%
\pgfpathlineto{\pgfqpoint{6.156440in}{1.609196in}}%
\pgfpathlineto{\pgfqpoint{6.156440in}{1.630631in}}%
\pgfpathlineto{\pgfqpoint{6.147686in}{1.630631in}}%
\pgfpathlineto{\pgfqpoint{6.147686in}{1.609196in}}%
\pgfpathclose%
\pgfusepath{fill}%
\end{pgfscope}%
\begin{pgfscope}%
\pgfpathrectangle{\pgfqpoint{3.776708in}{0.600000in}}{\pgfqpoint{2.573292in}{2.070576in}}%
\pgfusepath{clip}%
\pgfsetbuttcap%
\pgfsetmiterjoin%
\definecolor{currentfill}{rgb}{0.066899,0.263188,0.377594}%
\pgfsetfillcolor{currentfill}%
\pgfsetlinewidth{0.000000pt}%
\definecolor{currentstroke}{rgb}{0.000000,0.000000,0.000000}%
\pgfsetstrokecolor{currentstroke}%
\pgfsetstrokeopacity{0.000000}%
\pgfsetdash{}{0pt}%
\pgfpathmoveto{\pgfqpoint{6.158628in}{1.609196in}}%
\pgfpathlineto{\pgfqpoint{6.167381in}{1.609196in}}%
\pgfpathlineto{\pgfqpoint{6.167381in}{1.637837in}}%
\pgfpathlineto{\pgfqpoint{6.158628in}{1.637837in}}%
\pgfpathlineto{\pgfqpoint{6.158628in}{1.609196in}}%
\pgfpathclose%
\pgfusepath{fill}%
\end{pgfscope}%
\begin{pgfscope}%
\pgfpathrectangle{\pgfqpoint{3.776708in}{0.600000in}}{\pgfqpoint{2.573292in}{2.070576in}}%
\pgfusepath{clip}%
\pgfsetbuttcap%
\pgfsetmiterjoin%
\definecolor{currentfill}{rgb}{0.066899,0.263188,0.377594}%
\pgfsetfillcolor{currentfill}%
\pgfsetlinewidth{0.000000pt}%
\definecolor{currentstroke}{rgb}{0.000000,0.000000,0.000000}%
\pgfsetstrokecolor{currentstroke}%
\pgfsetstrokeopacity{0.000000}%
\pgfsetdash{}{0pt}%
\pgfpathmoveto{\pgfqpoint{6.169570in}{1.609196in}}%
\pgfpathlineto{\pgfqpoint{6.178323in}{1.609196in}}%
\pgfpathlineto{\pgfqpoint{6.178323in}{1.644369in}}%
\pgfpathlineto{\pgfqpoint{6.169570in}{1.644369in}}%
\pgfpathlineto{\pgfqpoint{6.169570in}{1.609196in}}%
\pgfpathclose%
\pgfusepath{fill}%
\end{pgfscope}%
\begin{pgfscope}%
\pgfpathrectangle{\pgfqpoint{3.776708in}{0.600000in}}{\pgfqpoint{2.573292in}{2.070576in}}%
\pgfusepath{clip}%
\pgfsetbuttcap%
\pgfsetmiterjoin%
\definecolor{currentfill}{rgb}{0.066899,0.263188,0.377594}%
\pgfsetfillcolor{currentfill}%
\pgfsetlinewidth{0.000000pt}%
\definecolor{currentstroke}{rgb}{0.000000,0.000000,0.000000}%
\pgfsetstrokecolor{currentstroke}%
\pgfsetstrokeopacity{0.000000}%
\pgfsetdash{}{0pt}%
\pgfpathmoveto{\pgfqpoint{6.180512in}{1.609196in}}%
\pgfpathlineto{\pgfqpoint{6.189265in}{1.609196in}}%
\pgfpathlineto{\pgfqpoint{6.189265in}{1.651200in}}%
\pgfpathlineto{\pgfqpoint{6.180512in}{1.651200in}}%
\pgfpathlineto{\pgfqpoint{6.180512in}{1.609196in}}%
\pgfpathclose%
\pgfusepath{fill}%
\end{pgfscope}%
\begin{pgfscope}%
\pgfpathrectangle{\pgfqpoint{3.776708in}{0.600000in}}{\pgfqpoint{2.573292in}{2.070576in}}%
\pgfusepath{clip}%
\pgfsetbuttcap%
\pgfsetmiterjoin%
\definecolor{currentfill}{rgb}{0.066899,0.263188,0.377594}%
\pgfsetfillcolor{currentfill}%
\pgfsetlinewidth{0.000000pt}%
\definecolor{currentstroke}{rgb}{0.000000,0.000000,0.000000}%
\pgfsetstrokecolor{currentstroke}%
\pgfsetstrokeopacity{0.000000}%
\pgfsetdash{}{0pt}%
\pgfpathmoveto{\pgfqpoint{6.191453in}{1.609196in}}%
\pgfpathlineto{\pgfqpoint{6.200207in}{1.609196in}}%
\pgfpathlineto{\pgfqpoint{6.200207in}{1.658757in}}%
\pgfpathlineto{\pgfqpoint{6.191453in}{1.658757in}}%
\pgfpathlineto{\pgfqpoint{6.191453in}{1.609196in}}%
\pgfpathclose%
\pgfusepath{fill}%
\end{pgfscope}%
\begin{pgfscope}%
\pgfpathrectangle{\pgfqpoint{3.776708in}{0.600000in}}{\pgfqpoint{2.573292in}{2.070576in}}%
\pgfusepath{clip}%
\pgfsetbuttcap%
\pgfsetmiterjoin%
\definecolor{currentfill}{rgb}{0.066899,0.263188,0.377594}%
\pgfsetfillcolor{currentfill}%
\pgfsetlinewidth{0.000000pt}%
\definecolor{currentstroke}{rgb}{0.000000,0.000000,0.000000}%
\pgfsetstrokecolor{currentstroke}%
\pgfsetstrokeopacity{0.000000}%
\pgfsetdash{}{0pt}%
\pgfpathmoveto{\pgfqpoint{6.202395in}{1.609196in}}%
\pgfpathlineto{\pgfqpoint{6.211149in}{1.609196in}}%
\pgfpathlineto{\pgfqpoint{6.211149in}{1.665721in}}%
\pgfpathlineto{\pgfqpoint{6.202395in}{1.665721in}}%
\pgfpathlineto{\pgfqpoint{6.202395in}{1.609196in}}%
\pgfpathclose%
\pgfusepath{fill}%
\end{pgfscope}%
\begin{pgfscope}%
\pgfpathrectangle{\pgfqpoint{3.776708in}{0.600000in}}{\pgfqpoint{2.573292in}{2.070576in}}%
\pgfusepath{clip}%
\pgfsetbuttcap%
\pgfsetmiterjoin%
\definecolor{currentfill}{rgb}{0.066899,0.263188,0.377594}%
\pgfsetfillcolor{currentfill}%
\pgfsetlinewidth{0.000000pt}%
\definecolor{currentstroke}{rgb}{0.000000,0.000000,0.000000}%
\pgfsetstrokecolor{currentstroke}%
\pgfsetstrokeopacity{0.000000}%
\pgfsetdash{}{0pt}%
\pgfpathmoveto{\pgfqpoint{6.213337in}{1.609196in}}%
\pgfpathlineto{\pgfqpoint{6.222090in}{1.609196in}}%
\pgfpathlineto{\pgfqpoint{6.222090in}{1.672770in}}%
\pgfpathlineto{\pgfqpoint{6.213337in}{1.672770in}}%
\pgfpathlineto{\pgfqpoint{6.213337in}{1.609196in}}%
\pgfpathclose%
\pgfusepath{fill}%
\end{pgfscope}%
\begin{pgfscope}%
\pgfpathrectangle{\pgfqpoint{3.776708in}{0.600000in}}{\pgfqpoint{2.573292in}{2.070576in}}%
\pgfusepath{clip}%
\pgfsetbuttcap%
\pgfsetmiterjoin%
\definecolor{currentfill}{rgb}{0.066899,0.263188,0.377594}%
\pgfsetfillcolor{currentfill}%
\pgfsetlinewidth{0.000000pt}%
\definecolor{currentstroke}{rgb}{0.000000,0.000000,0.000000}%
\pgfsetstrokecolor{currentstroke}%
\pgfsetstrokeopacity{0.000000}%
\pgfsetdash{}{0pt}%
\pgfpathmoveto{\pgfqpoint{6.224279in}{1.609196in}}%
\pgfpathlineto{\pgfqpoint{6.233032in}{1.609196in}}%
\pgfpathlineto{\pgfqpoint{6.233032in}{1.679365in}}%
\pgfpathlineto{\pgfqpoint{6.224279in}{1.679365in}}%
\pgfpathlineto{\pgfqpoint{6.224279in}{1.609196in}}%
\pgfpathclose%
\pgfusepath{fill}%
\end{pgfscope}%
\begin{pgfscope}%
\pgfpathrectangle{\pgfqpoint{3.776708in}{0.600000in}}{\pgfqpoint{2.573292in}{2.070576in}}%
\pgfusepath{clip}%
\pgfsetbuttcap%
\pgfsetmiterjoin%
\definecolor{currentfill}{rgb}{0.133298,0.375282,0.379395}%
\pgfsetfillcolor{currentfill}%
\pgfsetlinewidth{0.000000pt}%
\definecolor{currentstroke}{rgb}{0.000000,0.000000,0.000000}%
\pgfsetstrokecolor{currentstroke}%
\pgfsetstrokeopacity{0.000000}%
\pgfsetdash{}{0pt}%
\pgfpathmoveto{\pgfqpoint{3.893676in}{1.614315in}}%
\pgfpathlineto{\pgfqpoint{3.902429in}{1.614315in}}%
\pgfpathlineto{\pgfqpoint{3.902429in}{1.651251in}}%
\pgfpathlineto{\pgfqpoint{3.893676in}{1.651251in}}%
\pgfpathlineto{\pgfqpoint{3.893676in}{1.614315in}}%
\pgfpathclose%
\pgfusepath{fill}%
\end{pgfscope}%
\begin{pgfscope}%
\pgfpathrectangle{\pgfqpoint{3.776708in}{0.600000in}}{\pgfqpoint{2.573292in}{2.070576in}}%
\pgfusepath{clip}%
\pgfsetbuttcap%
\pgfsetmiterjoin%
\definecolor{currentfill}{rgb}{0.133298,0.375282,0.379395}%
\pgfsetfillcolor{currentfill}%
\pgfsetlinewidth{0.000000pt}%
\definecolor{currentstroke}{rgb}{0.000000,0.000000,0.000000}%
\pgfsetstrokecolor{currentstroke}%
\pgfsetstrokeopacity{0.000000}%
\pgfsetdash{}{0pt}%
\pgfpathmoveto{\pgfqpoint{3.904617in}{1.612638in}}%
\pgfpathlineto{\pgfqpoint{3.913371in}{1.612638in}}%
\pgfpathlineto{\pgfqpoint{3.913371in}{1.655386in}}%
\pgfpathlineto{\pgfqpoint{3.904617in}{1.655386in}}%
\pgfpathlineto{\pgfqpoint{3.904617in}{1.612638in}}%
\pgfpathclose%
\pgfusepath{fill}%
\end{pgfscope}%
\begin{pgfscope}%
\pgfpathrectangle{\pgfqpoint{3.776708in}{0.600000in}}{\pgfqpoint{2.573292in}{2.070576in}}%
\pgfusepath{clip}%
\pgfsetbuttcap%
\pgfsetmiterjoin%
\definecolor{currentfill}{rgb}{0.133298,0.375282,0.379395}%
\pgfsetfillcolor{currentfill}%
\pgfsetlinewidth{0.000000pt}%
\definecolor{currentstroke}{rgb}{0.000000,0.000000,0.000000}%
\pgfsetstrokecolor{currentstroke}%
\pgfsetstrokeopacity{0.000000}%
\pgfsetdash{}{0pt}%
\pgfpathmoveto{\pgfqpoint{3.915559in}{1.609787in}}%
\pgfpathlineto{\pgfqpoint{3.924313in}{1.609787in}}%
\pgfpathlineto{\pgfqpoint{3.924313in}{1.654428in}}%
\pgfpathlineto{\pgfqpoint{3.915559in}{1.654428in}}%
\pgfpathlineto{\pgfqpoint{3.915559in}{1.609787in}}%
\pgfpathclose%
\pgfusepath{fill}%
\end{pgfscope}%
\begin{pgfscope}%
\pgfpathrectangle{\pgfqpoint{3.776708in}{0.600000in}}{\pgfqpoint{2.573292in}{2.070576in}}%
\pgfusepath{clip}%
\pgfsetbuttcap%
\pgfsetmiterjoin%
\definecolor{currentfill}{rgb}{0.133298,0.375282,0.379395}%
\pgfsetfillcolor{currentfill}%
\pgfsetlinewidth{0.000000pt}%
\definecolor{currentstroke}{rgb}{0.000000,0.000000,0.000000}%
\pgfsetstrokecolor{currentstroke}%
\pgfsetstrokeopacity{0.000000}%
\pgfsetdash{}{0pt}%
\pgfpathmoveto{\pgfqpoint{3.926501in}{1.609196in}}%
\pgfpathlineto{\pgfqpoint{3.935254in}{1.609196in}}%
\pgfpathlineto{\pgfqpoint{3.935254in}{1.659911in}}%
\pgfpathlineto{\pgfqpoint{3.926501in}{1.659911in}}%
\pgfpathlineto{\pgfqpoint{3.926501in}{1.609196in}}%
\pgfpathclose%
\pgfusepath{fill}%
\end{pgfscope}%
\begin{pgfscope}%
\pgfpathrectangle{\pgfqpoint{3.776708in}{0.600000in}}{\pgfqpoint{2.573292in}{2.070576in}}%
\pgfusepath{clip}%
\pgfsetbuttcap%
\pgfsetmiterjoin%
\definecolor{currentfill}{rgb}{0.133298,0.375282,0.379395}%
\pgfsetfillcolor{currentfill}%
\pgfsetlinewidth{0.000000pt}%
\definecolor{currentstroke}{rgb}{0.000000,0.000000,0.000000}%
\pgfsetstrokecolor{currentstroke}%
\pgfsetstrokeopacity{0.000000}%
\pgfsetdash{}{0pt}%
\pgfpathmoveto{\pgfqpoint{3.937443in}{1.609196in}}%
\pgfpathlineto{\pgfqpoint{3.946196in}{1.609196in}}%
\pgfpathlineto{\pgfqpoint{3.946196in}{1.676666in}}%
\pgfpathlineto{\pgfqpoint{3.937443in}{1.676666in}}%
\pgfpathlineto{\pgfqpoint{3.937443in}{1.609196in}}%
\pgfpathclose%
\pgfusepath{fill}%
\end{pgfscope}%
\begin{pgfscope}%
\pgfpathrectangle{\pgfqpoint{3.776708in}{0.600000in}}{\pgfqpoint{2.573292in}{2.070576in}}%
\pgfusepath{clip}%
\pgfsetbuttcap%
\pgfsetmiterjoin%
\definecolor{currentfill}{rgb}{0.133298,0.375282,0.379395}%
\pgfsetfillcolor{currentfill}%
\pgfsetlinewidth{0.000000pt}%
\definecolor{currentstroke}{rgb}{0.000000,0.000000,0.000000}%
\pgfsetstrokecolor{currentstroke}%
\pgfsetstrokeopacity{0.000000}%
\pgfsetdash{}{0pt}%
\pgfpathmoveto{\pgfqpoint{3.948385in}{1.609196in}}%
\pgfpathlineto{\pgfqpoint{3.957138in}{1.609196in}}%
\pgfpathlineto{\pgfqpoint{3.957138in}{1.695221in}}%
\pgfpathlineto{\pgfqpoint{3.948385in}{1.695221in}}%
\pgfpathlineto{\pgfqpoint{3.948385in}{1.609196in}}%
\pgfpathclose%
\pgfusepath{fill}%
\end{pgfscope}%
\begin{pgfscope}%
\pgfpathrectangle{\pgfqpoint{3.776708in}{0.600000in}}{\pgfqpoint{2.573292in}{2.070576in}}%
\pgfusepath{clip}%
\pgfsetbuttcap%
\pgfsetmiterjoin%
\definecolor{currentfill}{rgb}{0.133298,0.375282,0.379395}%
\pgfsetfillcolor{currentfill}%
\pgfsetlinewidth{0.000000pt}%
\definecolor{currentstroke}{rgb}{0.000000,0.000000,0.000000}%
\pgfsetstrokecolor{currentstroke}%
\pgfsetstrokeopacity{0.000000}%
\pgfsetdash{}{0pt}%
\pgfpathmoveto{\pgfqpoint{3.959326in}{1.609196in}}%
\pgfpathlineto{\pgfqpoint{3.968080in}{1.609196in}}%
\pgfpathlineto{\pgfqpoint{3.968080in}{1.709995in}}%
\pgfpathlineto{\pgfqpoint{3.959326in}{1.709995in}}%
\pgfpathlineto{\pgfqpoint{3.959326in}{1.609196in}}%
\pgfpathclose%
\pgfusepath{fill}%
\end{pgfscope}%
\begin{pgfscope}%
\pgfpathrectangle{\pgfqpoint{3.776708in}{0.600000in}}{\pgfqpoint{2.573292in}{2.070576in}}%
\pgfusepath{clip}%
\pgfsetbuttcap%
\pgfsetmiterjoin%
\definecolor{currentfill}{rgb}{0.133298,0.375282,0.379395}%
\pgfsetfillcolor{currentfill}%
\pgfsetlinewidth{0.000000pt}%
\definecolor{currentstroke}{rgb}{0.000000,0.000000,0.000000}%
\pgfsetstrokecolor{currentstroke}%
\pgfsetstrokeopacity{0.000000}%
\pgfsetdash{}{0pt}%
\pgfpathmoveto{\pgfqpoint{3.970268in}{1.609196in}}%
\pgfpathlineto{\pgfqpoint{3.979022in}{1.609196in}}%
\pgfpathlineto{\pgfqpoint{3.979022in}{1.722918in}}%
\pgfpathlineto{\pgfqpoint{3.970268in}{1.722918in}}%
\pgfpathlineto{\pgfqpoint{3.970268in}{1.609196in}}%
\pgfpathclose%
\pgfusepath{fill}%
\end{pgfscope}%
\begin{pgfscope}%
\pgfpathrectangle{\pgfqpoint{3.776708in}{0.600000in}}{\pgfqpoint{2.573292in}{2.070576in}}%
\pgfusepath{clip}%
\pgfsetbuttcap%
\pgfsetmiterjoin%
\definecolor{currentfill}{rgb}{0.133298,0.375282,0.379395}%
\pgfsetfillcolor{currentfill}%
\pgfsetlinewidth{0.000000pt}%
\definecolor{currentstroke}{rgb}{0.000000,0.000000,0.000000}%
\pgfsetstrokecolor{currentstroke}%
\pgfsetstrokeopacity{0.000000}%
\pgfsetdash{}{0pt}%
\pgfpathmoveto{\pgfqpoint{3.981210in}{1.609196in}}%
\pgfpathlineto{\pgfqpoint{3.989963in}{1.609196in}}%
\pgfpathlineto{\pgfqpoint{3.989963in}{1.738785in}}%
\pgfpathlineto{\pgfqpoint{3.981210in}{1.738785in}}%
\pgfpathlineto{\pgfqpoint{3.981210in}{1.609196in}}%
\pgfpathclose%
\pgfusepath{fill}%
\end{pgfscope}%
\begin{pgfscope}%
\pgfpathrectangle{\pgfqpoint{3.776708in}{0.600000in}}{\pgfqpoint{2.573292in}{2.070576in}}%
\pgfusepath{clip}%
\pgfsetbuttcap%
\pgfsetmiterjoin%
\definecolor{currentfill}{rgb}{0.133298,0.375282,0.379395}%
\pgfsetfillcolor{currentfill}%
\pgfsetlinewidth{0.000000pt}%
\definecolor{currentstroke}{rgb}{0.000000,0.000000,0.000000}%
\pgfsetstrokecolor{currentstroke}%
\pgfsetstrokeopacity{0.000000}%
\pgfsetdash{}{0pt}%
\pgfpathmoveto{\pgfqpoint{3.992152in}{1.609196in}}%
\pgfpathlineto{\pgfqpoint{4.000905in}{1.609196in}}%
\pgfpathlineto{\pgfqpoint{4.000905in}{1.758897in}}%
\pgfpathlineto{\pgfqpoint{3.992152in}{1.758897in}}%
\pgfpathlineto{\pgfqpoint{3.992152in}{1.609196in}}%
\pgfpathclose%
\pgfusepath{fill}%
\end{pgfscope}%
\begin{pgfscope}%
\pgfpathrectangle{\pgfqpoint{3.776708in}{0.600000in}}{\pgfqpoint{2.573292in}{2.070576in}}%
\pgfusepath{clip}%
\pgfsetbuttcap%
\pgfsetmiterjoin%
\definecolor{currentfill}{rgb}{0.133298,0.375282,0.379395}%
\pgfsetfillcolor{currentfill}%
\pgfsetlinewidth{0.000000pt}%
\definecolor{currentstroke}{rgb}{0.000000,0.000000,0.000000}%
\pgfsetstrokecolor{currentstroke}%
\pgfsetstrokeopacity{0.000000}%
\pgfsetdash{}{0pt}%
\pgfpathmoveto{\pgfqpoint{4.003094in}{1.609196in}}%
\pgfpathlineto{\pgfqpoint{4.011847in}{1.609196in}}%
\pgfpathlineto{\pgfqpoint{4.011847in}{1.773087in}}%
\pgfpathlineto{\pgfqpoint{4.003094in}{1.773087in}}%
\pgfpathlineto{\pgfqpoint{4.003094in}{1.609196in}}%
\pgfpathclose%
\pgfusepath{fill}%
\end{pgfscope}%
\begin{pgfscope}%
\pgfpathrectangle{\pgfqpoint{3.776708in}{0.600000in}}{\pgfqpoint{2.573292in}{2.070576in}}%
\pgfusepath{clip}%
\pgfsetbuttcap%
\pgfsetmiterjoin%
\definecolor{currentfill}{rgb}{0.133298,0.375282,0.379395}%
\pgfsetfillcolor{currentfill}%
\pgfsetlinewidth{0.000000pt}%
\definecolor{currentstroke}{rgb}{0.000000,0.000000,0.000000}%
\pgfsetstrokecolor{currentstroke}%
\pgfsetstrokeopacity{0.000000}%
\pgfsetdash{}{0pt}%
\pgfpathmoveto{\pgfqpoint{4.014035in}{1.609196in}}%
\pgfpathlineto{\pgfqpoint{4.022789in}{1.609196in}}%
\pgfpathlineto{\pgfqpoint{4.022789in}{1.786930in}}%
\pgfpathlineto{\pgfqpoint{4.014035in}{1.786930in}}%
\pgfpathlineto{\pgfqpoint{4.014035in}{1.609196in}}%
\pgfpathclose%
\pgfusepath{fill}%
\end{pgfscope}%
\begin{pgfscope}%
\pgfpathrectangle{\pgfqpoint{3.776708in}{0.600000in}}{\pgfqpoint{2.573292in}{2.070576in}}%
\pgfusepath{clip}%
\pgfsetbuttcap%
\pgfsetmiterjoin%
\definecolor{currentfill}{rgb}{0.133298,0.375282,0.379395}%
\pgfsetfillcolor{currentfill}%
\pgfsetlinewidth{0.000000pt}%
\definecolor{currentstroke}{rgb}{0.000000,0.000000,0.000000}%
\pgfsetstrokecolor{currentstroke}%
\pgfsetstrokeopacity{0.000000}%
\pgfsetdash{}{0pt}%
\pgfpathmoveto{\pgfqpoint{4.024977in}{1.609196in}}%
\pgfpathlineto{\pgfqpoint{4.033731in}{1.609196in}}%
\pgfpathlineto{\pgfqpoint{4.033731in}{1.804878in}}%
\pgfpathlineto{\pgfqpoint{4.024977in}{1.804878in}}%
\pgfpathlineto{\pgfqpoint{4.024977in}{1.609196in}}%
\pgfpathclose%
\pgfusepath{fill}%
\end{pgfscope}%
\begin{pgfscope}%
\pgfpathrectangle{\pgfqpoint{3.776708in}{0.600000in}}{\pgfqpoint{2.573292in}{2.070576in}}%
\pgfusepath{clip}%
\pgfsetbuttcap%
\pgfsetmiterjoin%
\definecolor{currentfill}{rgb}{0.133298,0.375282,0.379395}%
\pgfsetfillcolor{currentfill}%
\pgfsetlinewidth{0.000000pt}%
\definecolor{currentstroke}{rgb}{0.000000,0.000000,0.000000}%
\pgfsetstrokecolor{currentstroke}%
\pgfsetstrokeopacity{0.000000}%
\pgfsetdash{}{0pt}%
\pgfpathmoveto{\pgfqpoint{4.035919in}{1.609196in}}%
\pgfpathlineto{\pgfqpoint{4.044672in}{1.609196in}}%
\pgfpathlineto{\pgfqpoint{4.044672in}{1.817628in}}%
\pgfpathlineto{\pgfqpoint{4.035919in}{1.817628in}}%
\pgfpathlineto{\pgfqpoint{4.035919in}{1.609196in}}%
\pgfpathclose%
\pgfusepath{fill}%
\end{pgfscope}%
\begin{pgfscope}%
\pgfpathrectangle{\pgfqpoint{3.776708in}{0.600000in}}{\pgfqpoint{2.573292in}{2.070576in}}%
\pgfusepath{clip}%
\pgfsetbuttcap%
\pgfsetmiterjoin%
\definecolor{currentfill}{rgb}{0.133298,0.375282,0.379395}%
\pgfsetfillcolor{currentfill}%
\pgfsetlinewidth{0.000000pt}%
\definecolor{currentstroke}{rgb}{0.000000,0.000000,0.000000}%
\pgfsetstrokecolor{currentstroke}%
\pgfsetstrokeopacity{0.000000}%
\pgfsetdash{}{0pt}%
\pgfpathmoveto{\pgfqpoint{4.046861in}{1.609196in}}%
\pgfpathlineto{\pgfqpoint{4.055614in}{1.609196in}}%
\pgfpathlineto{\pgfqpoint{4.055614in}{1.832648in}}%
\pgfpathlineto{\pgfqpoint{4.046861in}{1.832648in}}%
\pgfpathlineto{\pgfqpoint{4.046861in}{1.609196in}}%
\pgfpathclose%
\pgfusepath{fill}%
\end{pgfscope}%
\begin{pgfscope}%
\pgfpathrectangle{\pgfqpoint{3.776708in}{0.600000in}}{\pgfqpoint{2.573292in}{2.070576in}}%
\pgfusepath{clip}%
\pgfsetbuttcap%
\pgfsetmiterjoin%
\definecolor{currentfill}{rgb}{0.133298,0.375282,0.379395}%
\pgfsetfillcolor{currentfill}%
\pgfsetlinewidth{0.000000pt}%
\definecolor{currentstroke}{rgb}{0.000000,0.000000,0.000000}%
\pgfsetstrokecolor{currentstroke}%
\pgfsetstrokeopacity{0.000000}%
\pgfsetdash{}{0pt}%
\pgfpathmoveto{\pgfqpoint{4.057803in}{1.609196in}}%
\pgfpathlineto{\pgfqpoint{4.066556in}{1.609196in}}%
\pgfpathlineto{\pgfqpoint{4.066556in}{1.843014in}}%
\pgfpathlineto{\pgfqpoint{4.057803in}{1.843014in}}%
\pgfpathlineto{\pgfqpoint{4.057803in}{1.609196in}}%
\pgfpathclose%
\pgfusepath{fill}%
\end{pgfscope}%
\begin{pgfscope}%
\pgfpathrectangle{\pgfqpoint{3.776708in}{0.600000in}}{\pgfqpoint{2.573292in}{2.070576in}}%
\pgfusepath{clip}%
\pgfsetbuttcap%
\pgfsetmiterjoin%
\definecolor{currentfill}{rgb}{0.133298,0.375282,0.379395}%
\pgfsetfillcolor{currentfill}%
\pgfsetlinewidth{0.000000pt}%
\definecolor{currentstroke}{rgb}{0.000000,0.000000,0.000000}%
\pgfsetstrokecolor{currentstroke}%
\pgfsetstrokeopacity{0.000000}%
\pgfsetdash{}{0pt}%
\pgfpathmoveto{\pgfqpoint{4.068744in}{1.609196in}}%
\pgfpathlineto{\pgfqpoint{4.077498in}{1.609196in}}%
\pgfpathlineto{\pgfqpoint{4.077498in}{1.850170in}}%
\pgfpathlineto{\pgfqpoint{4.068744in}{1.850170in}}%
\pgfpathlineto{\pgfqpoint{4.068744in}{1.609196in}}%
\pgfpathclose%
\pgfusepath{fill}%
\end{pgfscope}%
\begin{pgfscope}%
\pgfpathrectangle{\pgfqpoint{3.776708in}{0.600000in}}{\pgfqpoint{2.573292in}{2.070576in}}%
\pgfusepath{clip}%
\pgfsetbuttcap%
\pgfsetmiterjoin%
\definecolor{currentfill}{rgb}{0.133298,0.375282,0.379395}%
\pgfsetfillcolor{currentfill}%
\pgfsetlinewidth{0.000000pt}%
\definecolor{currentstroke}{rgb}{0.000000,0.000000,0.000000}%
\pgfsetstrokecolor{currentstroke}%
\pgfsetstrokeopacity{0.000000}%
\pgfsetdash{}{0pt}%
\pgfpathmoveto{\pgfqpoint{4.079686in}{1.609196in}}%
\pgfpathlineto{\pgfqpoint{4.088440in}{1.609196in}}%
\pgfpathlineto{\pgfqpoint{4.088440in}{1.868437in}}%
\pgfpathlineto{\pgfqpoint{4.079686in}{1.868437in}}%
\pgfpathlineto{\pgfqpoint{4.079686in}{1.609196in}}%
\pgfpathclose%
\pgfusepath{fill}%
\end{pgfscope}%
\begin{pgfscope}%
\pgfpathrectangle{\pgfqpoint{3.776708in}{0.600000in}}{\pgfqpoint{2.573292in}{2.070576in}}%
\pgfusepath{clip}%
\pgfsetbuttcap%
\pgfsetmiterjoin%
\definecolor{currentfill}{rgb}{0.133298,0.375282,0.379395}%
\pgfsetfillcolor{currentfill}%
\pgfsetlinewidth{0.000000pt}%
\definecolor{currentstroke}{rgb}{0.000000,0.000000,0.000000}%
\pgfsetstrokecolor{currentstroke}%
\pgfsetstrokeopacity{0.000000}%
\pgfsetdash{}{0pt}%
\pgfpathmoveto{\pgfqpoint{4.090628in}{1.609196in}}%
\pgfpathlineto{\pgfqpoint{4.099381in}{1.609196in}}%
\pgfpathlineto{\pgfqpoint{4.099381in}{1.888504in}}%
\pgfpathlineto{\pgfqpoint{4.090628in}{1.888504in}}%
\pgfpathlineto{\pgfqpoint{4.090628in}{1.609196in}}%
\pgfpathclose%
\pgfusepath{fill}%
\end{pgfscope}%
\begin{pgfscope}%
\pgfpathrectangle{\pgfqpoint{3.776708in}{0.600000in}}{\pgfqpoint{2.573292in}{2.070576in}}%
\pgfusepath{clip}%
\pgfsetbuttcap%
\pgfsetmiterjoin%
\definecolor{currentfill}{rgb}{0.133298,0.375282,0.379395}%
\pgfsetfillcolor{currentfill}%
\pgfsetlinewidth{0.000000pt}%
\definecolor{currentstroke}{rgb}{0.000000,0.000000,0.000000}%
\pgfsetstrokecolor{currentstroke}%
\pgfsetstrokeopacity{0.000000}%
\pgfsetdash{}{0pt}%
\pgfpathmoveto{\pgfqpoint{4.101570in}{1.609196in}}%
\pgfpathlineto{\pgfqpoint{4.110323in}{1.609196in}}%
\pgfpathlineto{\pgfqpoint{4.110323in}{1.910838in}}%
\pgfpathlineto{\pgfqpoint{4.101570in}{1.910838in}}%
\pgfpathlineto{\pgfqpoint{4.101570in}{1.609196in}}%
\pgfpathclose%
\pgfusepath{fill}%
\end{pgfscope}%
\begin{pgfscope}%
\pgfpathrectangle{\pgfqpoint{3.776708in}{0.600000in}}{\pgfqpoint{2.573292in}{2.070576in}}%
\pgfusepath{clip}%
\pgfsetbuttcap%
\pgfsetmiterjoin%
\definecolor{currentfill}{rgb}{0.133298,0.375282,0.379395}%
\pgfsetfillcolor{currentfill}%
\pgfsetlinewidth{0.000000pt}%
\definecolor{currentstroke}{rgb}{0.000000,0.000000,0.000000}%
\pgfsetstrokecolor{currentstroke}%
\pgfsetstrokeopacity{0.000000}%
\pgfsetdash{}{0pt}%
\pgfpathmoveto{\pgfqpoint{4.112512in}{1.609196in}}%
\pgfpathlineto{\pgfqpoint{4.121265in}{1.609196in}}%
\pgfpathlineto{\pgfqpoint{4.121265in}{1.928472in}}%
\pgfpathlineto{\pgfqpoint{4.112512in}{1.928472in}}%
\pgfpathlineto{\pgfqpoint{4.112512in}{1.609196in}}%
\pgfpathclose%
\pgfusepath{fill}%
\end{pgfscope}%
\begin{pgfscope}%
\pgfpathrectangle{\pgfqpoint{3.776708in}{0.600000in}}{\pgfqpoint{2.573292in}{2.070576in}}%
\pgfusepath{clip}%
\pgfsetbuttcap%
\pgfsetmiterjoin%
\definecolor{currentfill}{rgb}{0.133298,0.375282,0.379395}%
\pgfsetfillcolor{currentfill}%
\pgfsetlinewidth{0.000000pt}%
\definecolor{currentstroke}{rgb}{0.000000,0.000000,0.000000}%
\pgfsetstrokecolor{currentstroke}%
\pgfsetstrokeopacity{0.000000}%
\pgfsetdash{}{0pt}%
\pgfpathmoveto{\pgfqpoint{4.123453in}{1.609196in}}%
\pgfpathlineto{\pgfqpoint{4.132207in}{1.609196in}}%
\pgfpathlineto{\pgfqpoint{4.132207in}{1.943333in}}%
\pgfpathlineto{\pgfqpoint{4.123453in}{1.943333in}}%
\pgfpathlineto{\pgfqpoint{4.123453in}{1.609196in}}%
\pgfpathclose%
\pgfusepath{fill}%
\end{pgfscope}%
\begin{pgfscope}%
\pgfpathrectangle{\pgfqpoint{3.776708in}{0.600000in}}{\pgfqpoint{2.573292in}{2.070576in}}%
\pgfusepath{clip}%
\pgfsetbuttcap%
\pgfsetmiterjoin%
\definecolor{currentfill}{rgb}{0.133298,0.375282,0.379395}%
\pgfsetfillcolor{currentfill}%
\pgfsetlinewidth{0.000000pt}%
\definecolor{currentstroke}{rgb}{0.000000,0.000000,0.000000}%
\pgfsetstrokecolor{currentstroke}%
\pgfsetstrokeopacity{0.000000}%
\pgfsetdash{}{0pt}%
\pgfpathmoveto{\pgfqpoint{4.134395in}{1.609196in}}%
\pgfpathlineto{\pgfqpoint{4.143149in}{1.609196in}}%
\pgfpathlineto{\pgfqpoint{4.143149in}{1.969953in}}%
\pgfpathlineto{\pgfqpoint{4.134395in}{1.969953in}}%
\pgfpathlineto{\pgfqpoint{4.134395in}{1.609196in}}%
\pgfpathclose%
\pgfusepath{fill}%
\end{pgfscope}%
\begin{pgfscope}%
\pgfpathrectangle{\pgfqpoint{3.776708in}{0.600000in}}{\pgfqpoint{2.573292in}{2.070576in}}%
\pgfusepath{clip}%
\pgfsetbuttcap%
\pgfsetmiterjoin%
\definecolor{currentfill}{rgb}{0.133298,0.375282,0.379395}%
\pgfsetfillcolor{currentfill}%
\pgfsetlinewidth{0.000000pt}%
\definecolor{currentstroke}{rgb}{0.000000,0.000000,0.000000}%
\pgfsetstrokecolor{currentstroke}%
\pgfsetstrokeopacity{0.000000}%
\pgfsetdash{}{0pt}%
\pgfpathmoveto{\pgfqpoint{4.145337in}{1.609196in}}%
\pgfpathlineto{\pgfqpoint{4.154090in}{1.609196in}}%
\pgfpathlineto{\pgfqpoint{4.154090in}{1.994726in}}%
\pgfpathlineto{\pgfqpoint{4.145337in}{1.994726in}}%
\pgfpathlineto{\pgfqpoint{4.145337in}{1.609196in}}%
\pgfpathclose%
\pgfusepath{fill}%
\end{pgfscope}%
\begin{pgfscope}%
\pgfpathrectangle{\pgfqpoint{3.776708in}{0.600000in}}{\pgfqpoint{2.573292in}{2.070576in}}%
\pgfusepath{clip}%
\pgfsetbuttcap%
\pgfsetmiterjoin%
\definecolor{currentfill}{rgb}{0.133298,0.375282,0.379395}%
\pgfsetfillcolor{currentfill}%
\pgfsetlinewidth{0.000000pt}%
\definecolor{currentstroke}{rgb}{0.000000,0.000000,0.000000}%
\pgfsetstrokecolor{currentstroke}%
\pgfsetstrokeopacity{0.000000}%
\pgfsetdash{}{0pt}%
\pgfpathmoveto{\pgfqpoint{4.156279in}{1.609196in}}%
\pgfpathlineto{\pgfqpoint{4.165032in}{1.609196in}}%
\pgfpathlineto{\pgfqpoint{4.165032in}{2.027471in}}%
\pgfpathlineto{\pgfqpoint{4.156279in}{2.027471in}}%
\pgfpathlineto{\pgfqpoint{4.156279in}{1.609196in}}%
\pgfpathclose%
\pgfusepath{fill}%
\end{pgfscope}%
\begin{pgfscope}%
\pgfpathrectangle{\pgfqpoint{3.776708in}{0.600000in}}{\pgfqpoint{2.573292in}{2.070576in}}%
\pgfusepath{clip}%
\pgfsetbuttcap%
\pgfsetmiterjoin%
\definecolor{currentfill}{rgb}{0.133298,0.375282,0.379395}%
\pgfsetfillcolor{currentfill}%
\pgfsetlinewidth{0.000000pt}%
\definecolor{currentstroke}{rgb}{0.000000,0.000000,0.000000}%
\pgfsetstrokecolor{currentstroke}%
\pgfsetstrokeopacity{0.000000}%
\pgfsetdash{}{0pt}%
\pgfpathmoveto{\pgfqpoint{4.167221in}{1.609196in}}%
\pgfpathlineto{\pgfqpoint{4.175974in}{1.609196in}}%
\pgfpathlineto{\pgfqpoint{4.175974in}{2.061880in}}%
\pgfpathlineto{\pgfqpoint{4.167221in}{2.061880in}}%
\pgfpathlineto{\pgfqpoint{4.167221in}{1.609196in}}%
\pgfpathclose%
\pgfusepath{fill}%
\end{pgfscope}%
\begin{pgfscope}%
\pgfpathrectangle{\pgfqpoint{3.776708in}{0.600000in}}{\pgfqpoint{2.573292in}{2.070576in}}%
\pgfusepath{clip}%
\pgfsetbuttcap%
\pgfsetmiterjoin%
\definecolor{currentfill}{rgb}{0.133298,0.375282,0.379395}%
\pgfsetfillcolor{currentfill}%
\pgfsetlinewidth{0.000000pt}%
\definecolor{currentstroke}{rgb}{0.000000,0.000000,0.000000}%
\pgfsetstrokecolor{currentstroke}%
\pgfsetstrokeopacity{0.000000}%
\pgfsetdash{}{0pt}%
\pgfpathmoveto{\pgfqpoint{4.178162in}{1.609196in}}%
\pgfpathlineto{\pgfqpoint{4.186916in}{1.609196in}}%
\pgfpathlineto{\pgfqpoint{4.186916in}{2.093036in}}%
\pgfpathlineto{\pgfqpoint{4.178162in}{2.093036in}}%
\pgfpathlineto{\pgfqpoint{4.178162in}{1.609196in}}%
\pgfpathclose%
\pgfusepath{fill}%
\end{pgfscope}%
\begin{pgfscope}%
\pgfpathrectangle{\pgfqpoint{3.776708in}{0.600000in}}{\pgfqpoint{2.573292in}{2.070576in}}%
\pgfusepath{clip}%
\pgfsetbuttcap%
\pgfsetmiterjoin%
\definecolor{currentfill}{rgb}{0.133298,0.375282,0.379395}%
\pgfsetfillcolor{currentfill}%
\pgfsetlinewidth{0.000000pt}%
\definecolor{currentstroke}{rgb}{0.000000,0.000000,0.000000}%
\pgfsetstrokecolor{currentstroke}%
\pgfsetstrokeopacity{0.000000}%
\pgfsetdash{}{0pt}%
\pgfpathmoveto{\pgfqpoint{4.189104in}{1.609196in}}%
\pgfpathlineto{\pgfqpoint{4.197858in}{1.609196in}}%
\pgfpathlineto{\pgfqpoint{4.197858in}{2.124148in}}%
\pgfpathlineto{\pgfqpoint{4.189104in}{2.124148in}}%
\pgfpathlineto{\pgfqpoint{4.189104in}{1.609196in}}%
\pgfpathclose%
\pgfusepath{fill}%
\end{pgfscope}%
\begin{pgfscope}%
\pgfpathrectangle{\pgfqpoint{3.776708in}{0.600000in}}{\pgfqpoint{2.573292in}{2.070576in}}%
\pgfusepath{clip}%
\pgfsetbuttcap%
\pgfsetmiterjoin%
\definecolor{currentfill}{rgb}{0.133298,0.375282,0.379395}%
\pgfsetfillcolor{currentfill}%
\pgfsetlinewidth{0.000000pt}%
\definecolor{currentstroke}{rgb}{0.000000,0.000000,0.000000}%
\pgfsetstrokecolor{currentstroke}%
\pgfsetstrokeopacity{0.000000}%
\pgfsetdash{}{0pt}%
\pgfpathmoveto{\pgfqpoint{4.200046in}{1.609196in}}%
\pgfpathlineto{\pgfqpoint{4.208799in}{1.609196in}}%
\pgfpathlineto{\pgfqpoint{4.208799in}{2.151830in}}%
\pgfpathlineto{\pgfqpoint{4.200046in}{2.151830in}}%
\pgfpathlineto{\pgfqpoint{4.200046in}{1.609196in}}%
\pgfpathclose%
\pgfusepath{fill}%
\end{pgfscope}%
\begin{pgfscope}%
\pgfpathrectangle{\pgfqpoint{3.776708in}{0.600000in}}{\pgfqpoint{2.573292in}{2.070576in}}%
\pgfusepath{clip}%
\pgfsetbuttcap%
\pgfsetmiterjoin%
\definecolor{currentfill}{rgb}{0.133298,0.375282,0.379395}%
\pgfsetfillcolor{currentfill}%
\pgfsetlinewidth{0.000000pt}%
\definecolor{currentstroke}{rgb}{0.000000,0.000000,0.000000}%
\pgfsetstrokecolor{currentstroke}%
\pgfsetstrokeopacity{0.000000}%
\pgfsetdash{}{0pt}%
\pgfpathmoveto{\pgfqpoint{4.210988in}{1.609196in}}%
\pgfpathlineto{\pgfqpoint{4.219741in}{1.609196in}}%
\pgfpathlineto{\pgfqpoint{4.219741in}{2.172906in}}%
\pgfpathlineto{\pgfqpoint{4.210988in}{2.172906in}}%
\pgfpathlineto{\pgfqpoint{4.210988in}{1.609196in}}%
\pgfpathclose%
\pgfusepath{fill}%
\end{pgfscope}%
\begin{pgfscope}%
\pgfpathrectangle{\pgfqpoint{3.776708in}{0.600000in}}{\pgfqpoint{2.573292in}{2.070576in}}%
\pgfusepath{clip}%
\pgfsetbuttcap%
\pgfsetmiterjoin%
\definecolor{currentfill}{rgb}{0.133298,0.375282,0.379395}%
\pgfsetfillcolor{currentfill}%
\pgfsetlinewidth{0.000000pt}%
\definecolor{currentstroke}{rgb}{0.000000,0.000000,0.000000}%
\pgfsetstrokecolor{currentstroke}%
\pgfsetstrokeopacity{0.000000}%
\pgfsetdash{}{0pt}%
\pgfpathmoveto{\pgfqpoint{4.221930in}{1.609196in}}%
\pgfpathlineto{\pgfqpoint{4.230683in}{1.609196in}}%
\pgfpathlineto{\pgfqpoint{4.230683in}{2.186849in}}%
\pgfpathlineto{\pgfqpoint{4.221930in}{2.186849in}}%
\pgfpathlineto{\pgfqpoint{4.221930in}{1.609196in}}%
\pgfpathclose%
\pgfusepath{fill}%
\end{pgfscope}%
\begin{pgfscope}%
\pgfpathrectangle{\pgfqpoint{3.776708in}{0.600000in}}{\pgfqpoint{2.573292in}{2.070576in}}%
\pgfusepath{clip}%
\pgfsetbuttcap%
\pgfsetmiterjoin%
\definecolor{currentfill}{rgb}{0.133298,0.375282,0.379395}%
\pgfsetfillcolor{currentfill}%
\pgfsetlinewidth{0.000000pt}%
\definecolor{currentstroke}{rgb}{0.000000,0.000000,0.000000}%
\pgfsetstrokecolor{currentstroke}%
\pgfsetstrokeopacity{0.000000}%
\pgfsetdash{}{0pt}%
\pgfpathmoveto{\pgfqpoint{4.232871in}{1.609196in}}%
\pgfpathlineto{\pgfqpoint{4.241625in}{1.609196in}}%
\pgfpathlineto{\pgfqpoint{4.241625in}{2.199192in}}%
\pgfpathlineto{\pgfqpoint{4.232871in}{2.199192in}}%
\pgfpathlineto{\pgfqpoint{4.232871in}{1.609196in}}%
\pgfpathclose%
\pgfusepath{fill}%
\end{pgfscope}%
\begin{pgfscope}%
\pgfpathrectangle{\pgfqpoint{3.776708in}{0.600000in}}{\pgfqpoint{2.573292in}{2.070576in}}%
\pgfusepath{clip}%
\pgfsetbuttcap%
\pgfsetmiterjoin%
\definecolor{currentfill}{rgb}{0.133298,0.375282,0.379395}%
\pgfsetfillcolor{currentfill}%
\pgfsetlinewidth{0.000000pt}%
\definecolor{currentstroke}{rgb}{0.000000,0.000000,0.000000}%
\pgfsetstrokecolor{currentstroke}%
\pgfsetstrokeopacity{0.000000}%
\pgfsetdash{}{0pt}%
\pgfpathmoveto{\pgfqpoint{4.243813in}{1.609196in}}%
\pgfpathlineto{\pgfqpoint{4.252567in}{1.609196in}}%
\pgfpathlineto{\pgfqpoint{4.252567in}{2.207370in}}%
\pgfpathlineto{\pgfqpoint{4.243813in}{2.207370in}}%
\pgfpathlineto{\pgfqpoint{4.243813in}{1.609196in}}%
\pgfpathclose%
\pgfusepath{fill}%
\end{pgfscope}%
\begin{pgfscope}%
\pgfpathrectangle{\pgfqpoint{3.776708in}{0.600000in}}{\pgfqpoint{2.573292in}{2.070576in}}%
\pgfusepath{clip}%
\pgfsetbuttcap%
\pgfsetmiterjoin%
\definecolor{currentfill}{rgb}{0.133298,0.375282,0.379395}%
\pgfsetfillcolor{currentfill}%
\pgfsetlinewidth{0.000000pt}%
\definecolor{currentstroke}{rgb}{0.000000,0.000000,0.000000}%
\pgfsetstrokecolor{currentstroke}%
\pgfsetstrokeopacity{0.000000}%
\pgfsetdash{}{0pt}%
\pgfpathmoveto{\pgfqpoint{4.254755in}{1.609196in}}%
\pgfpathlineto{\pgfqpoint{4.263508in}{1.609196in}}%
\pgfpathlineto{\pgfqpoint{4.263508in}{2.215878in}}%
\pgfpathlineto{\pgfqpoint{4.254755in}{2.215878in}}%
\pgfpathlineto{\pgfqpoint{4.254755in}{1.609196in}}%
\pgfpathclose%
\pgfusepath{fill}%
\end{pgfscope}%
\begin{pgfscope}%
\pgfpathrectangle{\pgfqpoint{3.776708in}{0.600000in}}{\pgfqpoint{2.573292in}{2.070576in}}%
\pgfusepath{clip}%
\pgfsetbuttcap%
\pgfsetmiterjoin%
\definecolor{currentfill}{rgb}{0.133298,0.375282,0.379395}%
\pgfsetfillcolor{currentfill}%
\pgfsetlinewidth{0.000000pt}%
\definecolor{currentstroke}{rgb}{0.000000,0.000000,0.000000}%
\pgfsetstrokecolor{currentstroke}%
\pgfsetstrokeopacity{0.000000}%
\pgfsetdash{}{0pt}%
\pgfpathmoveto{\pgfqpoint{4.265697in}{1.609196in}}%
\pgfpathlineto{\pgfqpoint{4.274450in}{1.609196in}}%
\pgfpathlineto{\pgfqpoint{4.274450in}{2.218492in}}%
\pgfpathlineto{\pgfqpoint{4.265697in}{2.218492in}}%
\pgfpathlineto{\pgfqpoint{4.265697in}{1.609196in}}%
\pgfpathclose%
\pgfusepath{fill}%
\end{pgfscope}%
\begin{pgfscope}%
\pgfpathrectangle{\pgfqpoint{3.776708in}{0.600000in}}{\pgfqpoint{2.573292in}{2.070576in}}%
\pgfusepath{clip}%
\pgfsetbuttcap%
\pgfsetmiterjoin%
\definecolor{currentfill}{rgb}{0.133298,0.375282,0.379395}%
\pgfsetfillcolor{currentfill}%
\pgfsetlinewidth{0.000000pt}%
\definecolor{currentstroke}{rgb}{0.000000,0.000000,0.000000}%
\pgfsetstrokecolor{currentstroke}%
\pgfsetstrokeopacity{0.000000}%
\pgfsetdash{}{0pt}%
\pgfpathmoveto{\pgfqpoint{4.276639in}{1.609196in}}%
\pgfpathlineto{\pgfqpoint{4.285392in}{1.609196in}}%
\pgfpathlineto{\pgfqpoint{4.285392in}{2.208160in}}%
\pgfpathlineto{\pgfqpoint{4.276639in}{2.208160in}}%
\pgfpathlineto{\pgfqpoint{4.276639in}{1.609196in}}%
\pgfpathclose%
\pgfusepath{fill}%
\end{pgfscope}%
\begin{pgfscope}%
\pgfpathrectangle{\pgfqpoint{3.776708in}{0.600000in}}{\pgfqpoint{2.573292in}{2.070576in}}%
\pgfusepath{clip}%
\pgfsetbuttcap%
\pgfsetmiterjoin%
\definecolor{currentfill}{rgb}{0.133298,0.375282,0.379395}%
\pgfsetfillcolor{currentfill}%
\pgfsetlinewidth{0.000000pt}%
\definecolor{currentstroke}{rgb}{0.000000,0.000000,0.000000}%
\pgfsetstrokecolor{currentstroke}%
\pgfsetstrokeopacity{0.000000}%
\pgfsetdash{}{0pt}%
\pgfpathmoveto{\pgfqpoint{4.287580in}{1.609196in}}%
\pgfpathlineto{\pgfqpoint{4.296334in}{1.609196in}}%
\pgfpathlineto{\pgfqpoint{4.296334in}{2.198952in}}%
\pgfpathlineto{\pgfqpoint{4.287580in}{2.198952in}}%
\pgfpathlineto{\pgfqpoint{4.287580in}{1.609196in}}%
\pgfpathclose%
\pgfusepath{fill}%
\end{pgfscope}%
\begin{pgfscope}%
\pgfpathrectangle{\pgfqpoint{3.776708in}{0.600000in}}{\pgfqpoint{2.573292in}{2.070576in}}%
\pgfusepath{clip}%
\pgfsetbuttcap%
\pgfsetmiterjoin%
\definecolor{currentfill}{rgb}{0.133298,0.375282,0.379395}%
\pgfsetfillcolor{currentfill}%
\pgfsetlinewidth{0.000000pt}%
\definecolor{currentstroke}{rgb}{0.000000,0.000000,0.000000}%
\pgfsetstrokecolor{currentstroke}%
\pgfsetstrokeopacity{0.000000}%
\pgfsetdash{}{0pt}%
\pgfpathmoveto{\pgfqpoint{4.298522in}{1.609196in}}%
\pgfpathlineto{\pgfqpoint{4.307276in}{1.609196in}}%
\pgfpathlineto{\pgfqpoint{4.307276in}{2.186221in}}%
\pgfpathlineto{\pgfqpoint{4.298522in}{2.186221in}}%
\pgfpathlineto{\pgfqpoint{4.298522in}{1.609196in}}%
\pgfpathclose%
\pgfusepath{fill}%
\end{pgfscope}%
\begin{pgfscope}%
\pgfpathrectangle{\pgfqpoint{3.776708in}{0.600000in}}{\pgfqpoint{2.573292in}{2.070576in}}%
\pgfusepath{clip}%
\pgfsetbuttcap%
\pgfsetmiterjoin%
\definecolor{currentfill}{rgb}{0.133298,0.375282,0.379395}%
\pgfsetfillcolor{currentfill}%
\pgfsetlinewidth{0.000000pt}%
\definecolor{currentstroke}{rgb}{0.000000,0.000000,0.000000}%
\pgfsetstrokecolor{currentstroke}%
\pgfsetstrokeopacity{0.000000}%
\pgfsetdash{}{0pt}%
\pgfpathmoveto{\pgfqpoint{4.309464in}{1.609196in}}%
\pgfpathlineto{\pgfqpoint{4.318217in}{1.609196in}}%
\pgfpathlineto{\pgfqpoint{4.318217in}{2.167588in}}%
\pgfpathlineto{\pgfqpoint{4.309464in}{2.167588in}}%
\pgfpathlineto{\pgfqpoint{4.309464in}{1.609196in}}%
\pgfpathclose%
\pgfusepath{fill}%
\end{pgfscope}%
\begin{pgfscope}%
\pgfpathrectangle{\pgfqpoint{3.776708in}{0.600000in}}{\pgfqpoint{2.573292in}{2.070576in}}%
\pgfusepath{clip}%
\pgfsetbuttcap%
\pgfsetmiterjoin%
\definecolor{currentfill}{rgb}{0.133298,0.375282,0.379395}%
\pgfsetfillcolor{currentfill}%
\pgfsetlinewidth{0.000000pt}%
\definecolor{currentstroke}{rgb}{0.000000,0.000000,0.000000}%
\pgfsetstrokecolor{currentstroke}%
\pgfsetstrokeopacity{0.000000}%
\pgfsetdash{}{0pt}%
\pgfpathmoveto{\pgfqpoint{4.320406in}{1.609196in}}%
\pgfpathlineto{\pgfqpoint{4.329159in}{1.609196in}}%
\pgfpathlineto{\pgfqpoint{4.329159in}{2.153235in}}%
\pgfpathlineto{\pgfqpoint{4.320406in}{2.153235in}}%
\pgfpathlineto{\pgfqpoint{4.320406in}{1.609196in}}%
\pgfpathclose%
\pgfusepath{fill}%
\end{pgfscope}%
\begin{pgfscope}%
\pgfpathrectangle{\pgfqpoint{3.776708in}{0.600000in}}{\pgfqpoint{2.573292in}{2.070576in}}%
\pgfusepath{clip}%
\pgfsetbuttcap%
\pgfsetmiterjoin%
\definecolor{currentfill}{rgb}{0.133298,0.375282,0.379395}%
\pgfsetfillcolor{currentfill}%
\pgfsetlinewidth{0.000000pt}%
\definecolor{currentstroke}{rgb}{0.000000,0.000000,0.000000}%
\pgfsetstrokecolor{currentstroke}%
\pgfsetstrokeopacity{0.000000}%
\pgfsetdash{}{0pt}%
\pgfpathmoveto{\pgfqpoint{4.331348in}{1.609196in}}%
\pgfpathlineto{\pgfqpoint{4.340101in}{1.609196in}}%
\pgfpathlineto{\pgfqpoint{4.340101in}{2.145067in}}%
\pgfpathlineto{\pgfqpoint{4.331348in}{2.145067in}}%
\pgfpathlineto{\pgfqpoint{4.331348in}{1.609196in}}%
\pgfpathclose%
\pgfusepath{fill}%
\end{pgfscope}%
\begin{pgfscope}%
\pgfpathrectangle{\pgfqpoint{3.776708in}{0.600000in}}{\pgfqpoint{2.573292in}{2.070576in}}%
\pgfusepath{clip}%
\pgfsetbuttcap%
\pgfsetmiterjoin%
\definecolor{currentfill}{rgb}{0.133298,0.375282,0.379395}%
\pgfsetfillcolor{currentfill}%
\pgfsetlinewidth{0.000000pt}%
\definecolor{currentstroke}{rgb}{0.000000,0.000000,0.000000}%
\pgfsetstrokecolor{currentstroke}%
\pgfsetstrokeopacity{0.000000}%
\pgfsetdash{}{0pt}%
\pgfpathmoveto{\pgfqpoint{4.342289in}{1.609196in}}%
\pgfpathlineto{\pgfqpoint{4.351043in}{1.609196in}}%
\pgfpathlineto{\pgfqpoint{4.351043in}{2.144673in}}%
\pgfpathlineto{\pgfqpoint{4.342289in}{2.144673in}}%
\pgfpathlineto{\pgfqpoint{4.342289in}{1.609196in}}%
\pgfpathclose%
\pgfusepath{fill}%
\end{pgfscope}%
\begin{pgfscope}%
\pgfpathrectangle{\pgfqpoint{3.776708in}{0.600000in}}{\pgfqpoint{2.573292in}{2.070576in}}%
\pgfusepath{clip}%
\pgfsetbuttcap%
\pgfsetmiterjoin%
\definecolor{currentfill}{rgb}{0.133298,0.375282,0.379395}%
\pgfsetfillcolor{currentfill}%
\pgfsetlinewidth{0.000000pt}%
\definecolor{currentstroke}{rgb}{0.000000,0.000000,0.000000}%
\pgfsetstrokecolor{currentstroke}%
\pgfsetstrokeopacity{0.000000}%
\pgfsetdash{}{0pt}%
\pgfpathmoveto{\pgfqpoint{4.353231in}{1.609196in}}%
\pgfpathlineto{\pgfqpoint{4.361985in}{1.609196in}}%
\pgfpathlineto{\pgfqpoint{4.361985in}{2.142505in}}%
\pgfpathlineto{\pgfqpoint{4.353231in}{2.142505in}}%
\pgfpathlineto{\pgfqpoint{4.353231in}{1.609196in}}%
\pgfpathclose%
\pgfusepath{fill}%
\end{pgfscope}%
\begin{pgfscope}%
\pgfpathrectangle{\pgfqpoint{3.776708in}{0.600000in}}{\pgfqpoint{2.573292in}{2.070576in}}%
\pgfusepath{clip}%
\pgfsetbuttcap%
\pgfsetmiterjoin%
\definecolor{currentfill}{rgb}{0.133298,0.375282,0.379395}%
\pgfsetfillcolor{currentfill}%
\pgfsetlinewidth{0.000000pt}%
\definecolor{currentstroke}{rgb}{0.000000,0.000000,0.000000}%
\pgfsetstrokecolor{currentstroke}%
\pgfsetstrokeopacity{0.000000}%
\pgfsetdash{}{0pt}%
\pgfpathmoveto{\pgfqpoint{4.364173in}{1.609196in}}%
\pgfpathlineto{\pgfqpoint{4.372926in}{1.609196in}}%
\pgfpathlineto{\pgfqpoint{4.372926in}{2.138064in}}%
\pgfpathlineto{\pgfqpoint{4.364173in}{2.138064in}}%
\pgfpathlineto{\pgfqpoint{4.364173in}{1.609196in}}%
\pgfpathclose%
\pgfusepath{fill}%
\end{pgfscope}%
\begin{pgfscope}%
\pgfpathrectangle{\pgfqpoint{3.776708in}{0.600000in}}{\pgfqpoint{2.573292in}{2.070576in}}%
\pgfusepath{clip}%
\pgfsetbuttcap%
\pgfsetmiterjoin%
\definecolor{currentfill}{rgb}{0.133298,0.375282,0.379395}%
\pgfsetfillcolor{currentfill}%
\pgfsetlinewidth{0.000000pt}%
\definecolor{currentstroke}{rgb}{0.000000,0.000000,0.000000}%
\pgfsetstrokecolor{currentstroke}%
\pgfsetstrokeopacity{0.000000}%
\pgfsetdash{}{0pt}%
\pgfpathmoveto{\pgfqpoint{4.375115in}{1.609196in}}%
\pgfpathlineto{\pgfqpoint{4.383868in}{1.609196in}}%
\pgfpathlineto{\pgfqpoint{4.383868in}{2.134869in}}%
\pgfpathlineto{\pgfqpoint{4.375115in}{2.134869in}}%
\pgfpathlineto{\pgfqpoint{4.375115in}{1.609196in}}%
\pgfpathclose%
\pgfusepath{fill}%
\end{pgfscope}%
\begin{pgfscope}%
\pgfpathrectangle{\pgfqpoint{3.776708in}{0.600000in}}{\pgfqpoint{2.573292in}{2.070576in}}%
\pgfusepath{clip}%
\pgfsetbuttcap%
\pgfsetmiterjoin%
\definecolor{currentfill}{rgb}{0.133298,0.375282,0.379395}%
\pgfsetfillcolor{currentfill}%
\pgfsetlinewidth{0.000000pt}%
\definecolor{currentstroke}{rgb}{0.000000,0.000000,0.000000}%
\pgfsetstrokecolor{currentstroke}%
\pgfsetstrokeopacity{0.000000}%
\pgfsetdash{}{0pt}%
\pgfpathmoveto{\pgfqpoint{4.386057in}{1.609196in}}%
\pgfpathlineto{\pgfqpoint{4.394810in}{1.609196in}}%
\pgfpathlineto{\pgfqpoint{4.394810in}{2.143042in}}%
\pgfpathlineto{\pgfqpoint{4.386057in}{2.143042in}}%
\pgfpathlineto{\pgfqpoint{4.386057in}{1.609196in}}%
\pgfpathclose%
\pgfusepath{fill}%
\end{pgfscope}%
\begin{pgfscope}%
\pgfpathrectangle{\pgfqpoint{3.776708in}{0.600000in}}{\pgfqpoint{2.573292in}{2.070576in}}%
\pgfusepath{clip}%
\pgfsetbuttcap%
\pgfsetmiterjoin%
\definecolor{currentfill}{rgb}{0.133298,0.375282,0.379395}%
\pgfsetfillcolor{currentfill}%
\pgfsetlinewidth{0.000000pt}%
\definecolor{currentstroke}{rgb}{0.000000,0.000000,0.000000}%
\pgfsetstrokecolor{currentstroke}%
\pgfsetstrokeopacity{0.000000}%
\pgfsetdash{}{0pt}%
\pgfpathmoveto{\pgfqpoint{4.396998in}{1.609196in}}%
\pgfpathlineto{\pgfqpoint{4.405752in}{1.609196in}}%
\pgfpathlineto{\pgfqpoint{4.405752in}{2.137254in}}%
\pgfpathlineto{\pgfqpoint{4.396998in}{2.137254in}}%
\pgfpathlineto{\pgfqpoint{4.396998in}{1.609196in}}%
\pgfpathclose%
\pgfusepath{fill}%
\end{pgfscope}%
\begin{pgfscope}%
\pgfpathrectangle{\pgfqpoint{3.776708in}{0.600000in}}{\pgfqpoint{2.573292in}{2.070576in}}%
\pgfusepath{clip}%
\pgfsetbuttcap%
\pgfsetmiterjoin%
\definecolor{currentfill}{rgb}{0.133298,0.375282,0.379395}%
\pgfsetfillcolor{currentfill}%
\pgfsetlinewidth{0.000000pt}%
\definecolor{currentstroke}{rgb}{0.000000,0.000000,0.000000}%
\pgfsetstrokecolor{currentstroke}%
\pgfsetstrokeopacity{0.000000}%
\pgfsetdash{}{0pt}%
\pgfpathmoveto{\pgfqpoint{4.407940in}{1.609196in}}%
\pgfpathlineto{\pgfqpoint{4.416694in}{1.609196in}}%
\pgfpathlineto{\pgfqpoint{4.416694in}{2.137953in}}%
\pgfpathlineto{\pgfqpoint{4.407940in}{2.137953in}}%
\pgfpathlineto{\pgfqpoint{4.407940in}{1.609196in}}%
\pgfpathclose%
\pgfusepath{fill}%
\end{pgfscope}%
\begin{pgfscope}%
\pgfpathrectangle{\pgfqpoint{3.776708in}{0.600000in}}{\pgfqpoint{2.573292in}{2.070576in}}%
\pgfusepath{clip}%
\pgfsetbuttcap%
\pgfsetmiterjoin%
\definecolor{currentfill}{rgb}{0.133298,0.375282,0.379395}%
\pgfsetfillcolor{currentfill}%
\pgfsetlinewidth{0.000000pt}%
\definecolor{currentstroke}{rgb}{0.000000,0.000000,0.000000}%
\pgfsetstrokecolor{currentstroke}%
\pgfsetstrokeopacity{0.000000}%
\pgfsetdash{}{0pt}%
\pgfpathmoveto{\pgfqpoint{4.418882in}{1.609196in}}%
\pgfpathlineto{\pgfqpoint{4.427635in}{1.609196in}}%
\pgfpathlineto{\pgfqpoint{4.427635in}{2.137107in}}%
\pgfpathlineto{\pgfqpoint{4.418882in}{2.137107in}}%
\pgfpathlineto{\pgfqpoint{4.418882in}{1.609196in}}%
\pgfpathclose%
\pgfusepath{fill}%
\end{pgfscope}%
\begin{pgfscope}%
\pgfpathrectangle{\pgfqpoint{3.776708in}{0.600000in}}{\pgfqpoint{2.573292in}{2.070576in}}%
\pgfusepath{clip}%
\pgfsetbuttcap%
\pgfsetmiterjoin%
\definecolor{currentfill}{rgb}{0.133298,0.375282,0.379395}%
\pgfsetfillcolor{currentfill}%
\pgfsetlinewidth{0.000000pt}%
\definecolor{currentstroke}{rgb}{0.000000,0.000000,0.000000}%
\pgfsetstrokecolor{currentstroke}%
\pgfsetstrokeopacity{0.000000}%
\pgfsetdash{}{0pt}%
\pgfpathmoveto{\pgfqpoint{4.429824in}{1.609196in}}%
\pgfpathlineto{\pgfqpoint{4.438577in}{1.609196in}}%
\pgfpathlineto{\pgfqpoint{4.438577in}{2.137780in}}%
\pgfpathlineto{\pgfqpoint{4.429824in}{2.137780in}}%
\pgfpathlineto{\pgfqpoint{4.429824in}{1.609196in}}%
\pgfpathclose%
\pgfusepath{fill}%
\end{pgfscope}%
\begin{pgfscope}%
\pgfpathrectangle{\pgfqpoint{3.776708in}{0.600000in}}{\pgfqpoint{2.573292in}{2.070576in}}%
\pgfusepath{clip}%
\pgfsetbuttcap%
\pgfsetmiterjoin%
\definecolor{currentfill}{rgb}{0.133298,0.375282,0.379395}%
\pgfsetfillcolor{currentfill}%
\pgfsetlinewidth{0.000000pt}%
\definecolor{currentstroke}{rgb}{0.000000,0.000000,0.000000}%
\pgfsetstrokecolor{currentstroke}%
\pgfsetstrokeopacity{0.000000}%
\pgfsetdash{}{0pt}%
\pgfpathmoveto{\pgfqpoint{4.440766in}{1.609196in}}%
\pgfpathlineto{\pgfqpoint{4.449519in}{1.609196in}}%
\pgfpathlineto{\pgfqpoint{4.449519in}{2.131460in}}%
\pgfpathlineto{\pgfqpoint{4.440766in}{2.131460in}}%
\pgfpathlineto{\pgfqpoint{4.440766in}{1.609196in}}%
\pgfpathclose%
\pgfusepath{fill}%
\end{pgfscope}%
\begin{pgfscope}%
\pgfpathrectangle{\pgfqpoint{3.776708in}{0.600000in}}{\pgfqpoint{2.573292in}{2.070576in}}%
\pgfusepath{clip}%
\pgfsetbuttcap%
\pgfsetmiterjoin%
\definecolor{currentfill}{rgb}{0.133298,0.375282,0.379395}%
\pgfsetfillcolor{currentfill}%
\pgfsetlinewidth{0.000000pt}%
\definecolor{currentstroke}{rgb}{0.000000,0.000000,0.000000}%
\pgfsetstrokecolor{currentstroke}%
\pgfsetstrokeopacity{0.000000}%
\pgfsetdash{}{0pt}%
\pgfpathmoveto{\pgfqpoint{4.451707in}{1.609196in}}%
\pgfpathlineto{\pgfqpoint{4.460461in}{1.609196in}}%
\pgfpathlineto{\pgfqpoint{4.460461in}{2.130108in}}%
\pgfpathlineto{\pgfqpoint{4.451707in}{2.130108in}}%
\pgfpathlineto{\pgfqpoint{4.451707in}{1.609196in}}%
\pgfpathclose%
\pgfusepath{fill}%
\end{pgfscope}%
\begin{pgfscope}%
\pgfpathrectangle{\pgfqpoint{3.776708in}{0.600000in}}{\pgfqpoint{2.573292in}{2.070576in}}%
\pgfusepath{clip}%
\pgfsetbuttcap%
\pgfsetmiterjoin%
\definecolor{currentfill}{rgb}{0.133298,0.375282,0.379395}%
\pgfsetfillcolor{currentfill}%
\pgfsetlinewidth{0.000000pt}%
\definecolor{currentstroke}{rgb}{0.000000,0.000000,0.000000}%
\pgfsetstrokecolor{currentstroke}%
\pgfsetstrokeopacity{0.000000}%
\pgfsetdash{}{0pt}%
\pgfpathmoveto{\pgfqpoint{4.462649in}{1.609196in}}%
\pgfpathlineto{\pgfqpoint{4.471403in}{1.609196in}}%
\pgfpathlineto{\pgfqpoint{4.471403in}{2.117696in}}%
\pgfpathlineto{\pgfqpoint{4.462649in}{2.117696in}}%
\pgfpathlineto{\pgfqpoint{4.462649in}{1.609196in}}%
\pgfpathclose%
\pgfusepath{fill}%
\end{pgfscope}%
\begin{pgfscope}%
\pgfpathrectangle{\pgfqpoint{3.776708in}{0.600000in}}{\pgfqpoint{2.573292in}{2.070576in}}%
\pgfusepath{clip}%
\pgfsetbuttcap%
\pgfsetmiterjoin%
\definecolor{currentfill}{rgb}{0.133298,0.375282,0.379395}%
\pgfsetfillcolor{currentfill}%
\pgfsetlinewidth{0.000000pt}%
\definecolor{currentstroke}{rgb}{0.000000,0.000000,0.000000}%
\pgfsetstrokecolor{currentstroke}%
\pgfsetstrokeopacity{0.000000}%
\pgfsetdash{}{0pt}%
\pgfpathmoveto{\pgfqpoint{4.473591in}{1.609196in}}%
\pgfpathlineto{\pgfqpoint{4.482344in}{1.609196in}}%
\pgfpathlineto{\pgfqpoint{4.482344in}{2.097641in}}%
\pgfpathlineto{\pgfqpoint{4.473591in}{2.097641in}}%
\pgfpathlineto{\pgfqpoint{4.473591in}{1.609196in}}%
\pgfpathclose%
\pgfusepath{fill}%
\end{pgfscope}%
\begin{pgfscope}%
\pgfpathrectangle{\pgfqpoint{3.776708in}{0.600000in}}{\pgfqpoint{2.573292in}{2.070576in}}%
\pgfusepath{clip}%
\pgfsetbuttcap%
\pgfsetmiterjoin%
\definecolor{currentfill}{rgb}{0.133298,0.375282,0.379395}%
\pgfsetfillcolor{currentfill}%
\pgfsetlinewidth{0.000000pt}%
\definecolor{currentstroke}{rgb}{0.000000,0.000000,0.000000}%
\pgfsetstrokecolor{currentstroke}%
\pgfsetstrokeopacity{0.000000}%
\pgfsetdash{}{0pt}%
\pgfpathmoveto{\pgfqpoint{4.484533in}{1.609196in}}%
\pgfpathlineto{\pgfqpoint{4.493286in}{1.609196in}}%
\pgfpathlineto{\pgfqpoint{4.493286in}{2.076490in}}%
\pgfpathlineto{\pgfqpoint{4.484533in}{2.076490in}}%
\pgfpathlineto{\pgfqpoint{4.484533in}{1.609196in}}%
\pgfpathclose%
\pgfusepath{fill}%
\end{pgfscope}%
\begin{pgfscope}%
\pgfpathrectangle{\pgfqpoint{3.776708in}{0.600000in}}{\pgfqpoint{2.573292in}{2.070576in}}%
\pgfusepath{clip}%
\pgfsetbuttcap%
\pgfsetmiterjoin%
\definecolor{currentfill}{rgb}{0.133298,0.375282,0.379395}%
\pgfsetfillcolor{currentfill}%
\pgfsetlinewidth{0.000000pt}%
\definecolor{currentstroke}{rgb}{0.000000,0.000000,0.000000}%
\pgfsetstrokecolor{currentstroke}%
\pgfsetstrokeopacity{0.000000}%
\pgfsetdash{}{0pt}%
\pgfpathmoveto{\pgfqpoint{4.495475in}{1.609196in}}%
\pgfpathlineto{\pgfqpoint{4.504228in}{1.609196in}}%
\pgfpathlineto{\pgfqpoint{4.504228in}{2.051679in}}%
\pgfpathlineto{\pgfqpoint{4.495475in}{2.051679in}}%
\pgfpathlineto{\pgfqpoint{4.495475in}{1.609196in}}%
\pgfpathclose%
\pgfusepath{fill}%
\end{pgfscope}%
\begin{pgfscope}%
\pgfpathrectangle{\pgfqpoint{3.776708in}{0.600000in}}{\pgfqpoint{2.573292in}{2.070576in}}%
\pgfusepath{clip}%
\pgfsetbuttcap%
\pgfsetmiterjoin%
\definecolor{currentfill}{rgb}{0.133298,0.375282,0.379395}%
\pgfsetfillcolor{currentfill}%
\pgfsetlinewidth{0.000000pt}%
\definecolor{currentstroke}{rgb}{0.000000,0.000000,0.000000}%
\pgfsetstrokecolor{currentstroke}%
\pgfsetstrokeopacity{0.000000}%
\pgfsetdash{}{0pt}%
\pgfpathmoveto{\pgfqpoint{4.506416in}{1.609196in}}%
\pgfpathlineto{\pgfqpoint{4.515170in}{1.609196in}}%
\pgfpathlineto{\pgfqpoint{4.515170in}{2.034530in}}%
\pgfpathlineto{\pgfqpoint{4.506416in}{2.034530in}}%
\pgfpathlineto{\pgfqpoint{4.506416in}{1.609196in}}%
\pgfpathclose%
\pgfusepath{fill}%
\end{pgfscope}%
\begin{pgfscope}%
\pgfpathrectangle{\pgfqpoint{3.776708in}{0.600000in}}{\pgfqpoint{2.573292in}{2.070576in}}%
\pgfusepath{clip}%
\pgfsetbuttcap%
\pgfsetmiterjoin%
\definecolor{currentfill}{rgb}{0.133298,0.375282,0.379395}%
\pgfsetfillcolor{currentfill}%
\pgfsetlinewidth{0.000000pt}%
\definecolor{currentstroke}{rgb}{0.000000,0.000000,0.000000}%
\pgfsetstrokecolor{currentstroke}%
\pgfsetstrokeopacity{0.000000}%
\pgfsetdash{}{0pt}%
\pgfpathmoveto{\pgfqpoint{4.517358in}{1.609196in}}%
\pgfpathlineto{\pgfqpoint{4.526112in}{1.609196in}}%
\pgfpathlineto{\pgfqpoint{4.526112in}{2.020154in}}%
\pgfpathlineto{\pgfqpoint{4.517358in}{2.020154in}}%
\pgfpathlineto{\pgfqpoint{4.517358in}{1.609196in}}%
\pgfpathclose%
\pgfusepath{fill}%
\end{pgfscope}%
\begin{pgfscope}%
\pgfpathrectangle{\pgfqpoint{3.776708in}{0.600000in}}{\pgfqpoint{2.573292in}{2.070576in}}%
\pgfusepath{clip}%
\pgfsetbuttcap%
\pgfsetmiterjoin%
\definecolor{currentfill}{rgb}{0.133298,0.375282,0.379395}%
\pgfsetfillcolor{currentfill}%
\pgfsetlinewidth{0.000000pt}%
\definecolor{currentstroke}{rgb}{0.000000,0.000000,0.000000}%
\pgfsetstrokecolor{currentstroke}%
\pgfsetstrokeopacity{0.000000}%
\pgfsetdash{}{0pt}%
\pgfpathmoveto{\pgfqpoint{4.528300in}{1.609196in}}%
\pgfpathlineto{\pgfqpoint{4.537053in}{1.609196in}}%
\pgfpathlineto{\pgfqpoint{4.537053in}{1.994423in}}%
\pgfpathlineto{\pgfqpoint{4.528300in}{1.994423in}}%
\pgfpathlineto{\pgfqpoint{4.528300in}{1.609196in}}%
\pgfpathclose%
\pgfusepath{fill}%
\end{pgfscope}%
\begin{pgfscope}%
\pgfpathrectangle{\pgfqpoint{3.776708in}{0.600000in}}{\pgfqpoint{2.573292in}{2.070576in}}%
\pgfusepath{clip}%
\pgfsetbuttcap%
\pgfsetmiterjoin%
\definecolor{currentfill}{rgb}{0.133298,0.375282,0.379395}%
\pgfsetfillcolor{currentfill}%
\pgfsetlinewidth{0.000000pt}%
\definecolor{currentstroke}{rgb}{0.000000,0.000000,0.000000}%
\pgfsetstrokecolor{currentstroke}%
\pgfsetstrokeopacity{0.000000}%
\pgfsetdash{}{0pt}%
\pgfpathmoveto{\pgfqpoint{4.539242in}{1.609196in}}%
\pgfpathlineto{\pgfqpoint{4.547995in}{1.609196in}}%
\pgfpathlineto{\pgfqpoint{4.547995in}{1.957632in}}%
\pgfpathlineto{\pgfqpoint{4.539242in}{1.957632in}}%
\pgfpathlineto{\pgfqpoint{4.539242in}{1.609196in}}%
\pgfpathclose%
\pgfusepath{fill}%
\end{pgfscope}%
\begin{pgfscope}%
\pgfpathrectangle{\pgfqpoint{3.776708in}{0.600000in}}{\pgfqpoint{2.573292in}{2.070576in}}%
\pgfusepath{clip}%
\pgfsetbuttcap%
\pgfsetmiterjoin%
\definecolor{currentfill}{rgb}{0.133298,0.375282,0.379395}%
\pgfsetfillcolor{currentfill}%
\pgfsetlinewidth{0.000000pt}%
\definecolor{currentstroke}{rgb}{0.000000,0.000000,0.000000}%
\pgfsetstrokecolor{currentstroke}%
\pgfsetstrokeopacity{0.000000}%
\pgfsetdash{}{0pt}%
\pgfpathmoveto{\pgfqpoint{4.550183in}{1.609196in}}%
\pgfpathlineto{\pgfqpoint{4.558937in}{1.609196in}}%
\pgfpathlineto{\pgfqpoint{4.558937in}{1.920819in}}%
\pgfpathlineto{\pgfqpoint{4.550183in}{1.920819in}}%
\pgfpathlineto{\pgfqpoint{4.550183in}{1.609196in}}%
\pgfpathclose%
\pgfusepath{fill}%
\end{pgfscope}%
\begin{pgfscope}%
\pgfpathrectangle{\pgfqpoint{3.776708in}{0.600000in}}{\pgfqpoint{2.573292in}{2.070576in}}%
\pgfusepath{clip}%
\pgfsetbuttcap%
\pgfsetmiterjoin%
\definecolor{currentfill}{rgb}{0.133298,0.375282,0.379395}%
\pgfsetfillcolor{currentfill}%
\pgfsetlinewidth{0.000000pt}%
\definecolor{currentstroke}{rgb}{0.000000,0.000000,0.000000}%
\pgfsetstrokecolor{currentstroke}%
\pgfsetstrokeopacity{0.000000}%
\pgfsetdash{}{0pt}%
\pgfpathmoveto{\pgfqpoint{4.561125in}{1.609196in}}%
\pgfpathlineto{\pgfqpoint{4.569879in}{1.609196in}}%
\pgfpathlineto{\pgfqpoint{4.569879in}{1.885162in}}%
\pgfpathlineto{\pgfqpoint{4.561125in}{1.885162in}}%
\pgfpathlineto{\pgfqpoint{4.561125in}{1.609196in}}%
\pgfpathclose%
\pgfusepath{fill}%
\end{pgfscope}%
\begin{pgfscope}%
\pgfpathrectangle{\pgfqpoint{3.776708in}{0.600000in}}{\pgfqpoint{2.573292in}{2.070576in}}%
\pgfusepath{clip}%
\pgfsetbuttcap%
\pgfsetmiterjoin%
\definecolor{currentfill}{rgb}{0.133298,0.375282,0.379395}%
\pgfsetfillcolor{currentfill}%
\pgfsetlinewidth{0.000000pt}%
\definecolor{currentstroke}{rgb}{0.000000,0.000000,0.000000}%
\pgfsetstrokecolor{currentstroke}%
\pgfsetstrokeopacity{0.000000}%
\pgfsetdash{}{0pt}%
\pgfpathmoveto{\pgfqpoint{4.572067in}{1.609196in}}%
\pgfpathlineto{\pgfqpoint{4.580821in}{1.609196in}}%
\pgfpathlineto{\pgfqpoint{4.580821in}{1.853247in}}%
\pgfpathlineto{\pgfqpoint{4.572067in}{1.853247in}}%
\pgfpathlineto{\pgfqpoint{4.572067in}{1.609196in}}%
\pgfpathclose%
\pgfusepath{fill}%
\end{pgfscope}%
\begin{pgfscope}%
\pgfpathrectangle{\pgfqpoint{3.776708in}{0.600000in}}{\pgfqpoint{2.573292in}{2.070576in}}%
\pgfusepath{clip}%
\pgfsetbuttcap%
\pgfsetmiterjoin%
\definecolor{currentfill}{rgb}{0.133298,0.375282,0.379395}%
\pgfsetfillcolor{currentfill}%
\pgfsetlinewidth{0.000000pt}%
\definecolor{currentstroke}{rgb}{0.000000,0.000000,0.000000}%
\pgfsetstrokecolor{currentstroke}%
\pgfsetstrokeopacity{0.000000}%
\pgfsetdash{}{0pt}%
\pgfpathmoveto{\pgfqpoint{4.583009in}{1.609196in}}%
\pgfpathlineto{\pgfqpoint{4.591762in}{1.609196in}}%
\pgfpathlineto{\pgfqpoint{4.591762in}{1.820957in}}%
\pgfpathlineto{\pgfqpoint{4.583009in}{1.820957in}}%
\pgfpathlineto{\pgfqpoint{4.583009in}{1.609196in}}%
\pgfpathclose%
\pgfusepath{fill}%
\end{pgfscope}%
\begin{pgfscope}%
\pgfpathrectangle{\pgfqpoint{3.776708in}{0.600000in}}{\pgfqpoint{2.573292in}{2.070576in}}%
\pgfusepath{clip}%
\pgfsetbuttcap%
\pgfsetmiterjoin%
\definecolor{currentfill}{rgb}{0.133298,0.375282,0.379395}%
\pgfsetfillcolor{currentfill}%
\pgfsetlinewidth{0.000000pt}%
\definecolor{currentstroke}{rgb}{0.000000,0.000000,0.000000}%
\pgfsetstrokecolor{currentstroke}%
\pgfsetstrokeopacity{0.000000}%
\pgfsetdash{}{0pt}%
\pgfpathmoveto{\pgfqpoint{4.593951in}{1.609196in}}%
\pgfpathlineto{\pgfqpoint{4.602704in}{1.609196in}}%
\pgfpathlineto{\pgfqpoint{4.602704in}{1.791671in}}%
\pgfpathlineto{\pgfqpoint{4.593951in}{1.791671in}}%
\pgfpathlineto{\pgfqpoint{4.593951in}{1.609196in}}%
\pgfpathclose%
\pgfusepath{fill}%
\end{pgfscope}%
\begin{pgfscope}%
\pgfpathrectangle{\pgfqpoint{3.776708in}{0.600000in}}{\pgfqpoint{2.573292in}{2.070576in}}%
\pgfusepath{clip}%
\pgfsetbuttcap%
\pgfsetmiterjoin%
\definecolor{currentfill}{rgb}{0.133298,0.375282,0.379395}%
\pgfsetfillcolor{currentfill}%
\pgfsetlinewidth{0.000000pt}%
\definecolor{currentstroke}{rgb}{0.000000,0.000000,0.000000}%
\pgfsetstrokecolor{currentstroke}%
\pgfsetstrokeopacity{0.000000}%
\pgfsetdash{}{0pt}%
\pgfpathmoveto{\pgfqpoint{4.604892in}{1.609196in}}%
\pgfpathlineto{\pgfqpoint{4.613646in}{1.609196in}}%
\pgfpathlineto{\pgfqpoint{4.613646in}{1.764991in}}%
\pgfpathlineto{\pgfqpoint{4.604892in}{1.764991in}}%
\pgfpathlineto{\pgfqpoint{4.604892in}{1.609196in}}%
\pgfpathclose%
\pgfusepath{fill}%
\end{pgfscope}%
\begin{pgfscope}%
\pgfpathrectangle{\pgfqpoint{3.776708in}{0.600000in}}{\pgfqpoint{2.573292in}{2.070576in}}%
\pgfusepath{clip}%
\pgfsetbuttcap%
\pgfsetmiterjoin%
\definecolor{currentfill}{rgb}{0.133298,0.375282,0.379395}%
\pgfsetfillcolor{currentfill}%
\pgfsetlinewidth{0.000000pt}%
\definecolor{currentstroke}{rgb}{0.000000,0.000000,0.000000}%
\pgfsetstrokecolor{currentstroke}%
\pgfsetstrokeopacity{0.000000}%
\pgfsetdash{}{0pt}%
\pgfpathmoveto{\pgfqpoint{4.615834in}{1.609196in}}%
\pgfpathlineto{\pgfqpoint{4.624588in}{1.609196in}}%
\pgfpathlineto{\pgfqpoint{4.624588in}{1.736697in}}%
\pgfpathlineto{\pgfqpoint{4.615834in}{1.736697in}}%
\pgfpathlineto{\pgfqpoint{4.615834in}{1.609196in}}%
\pgfpathclose%
\pgfusepath{fill}%
\end{pgfscope}%
\begin{pgfscope}%
\pgfpathrectangle{\pgfqpoint{3.776708in}{0.600000in}}{\pgfqpoint{2.573292in}{2.070576in}}%
\pgfusepath{clip}%
\pgfsetbuttcap%
\pgfsetmiterjoin%
\definecolor{currentfill}{rgb}{0.133298,0.375282,0.379395}%
\pgfsetfillcolor{currentfill}%
\pgfsetlinewidth{0.000000pt}%
\definecolor{currentstroke}{rgb}{0.000000,0.000000,0.000000}%
\pgfsetstrokecolor{currentstroke}%
\pgfsetstrokeopacity{0.000000}%
\pgfsetdash{}{0pt}%
\pgfpathmoveto{\pgfqpoint{4.626776in}{1.609196in}}%
\pgfpathlineto{\pgfqpoint{4.635530in}{1.609196in}}%
\pgfpathlineto{\pgfqpoint{4.635530in}{1.713901in}}%
\pgfpathlineto{\pgfqpoint{4.626776in}{1.713901in}}%
\pgfpathlineto{\pgfqpoint{4.626776in}{1.609196in}}%
\pgfpathclose%
\pgfusepath{fill}%
\end{pgfscope}%
\begin{pgfscope}%
\pgfpathrectangle{\pgfqpoint{3.776708in}{0.600000in}}{\pgfqpoint{2.573292in}{2.070576in}}%
\pgfusepath{clip}%
\pgfsetbuttcap%
\pgfsetmiterjoin%
\definecolor{currentfill}{rgb}{0.133298,0.375282,0.379395}%
\pgfsetfillcolor{currentfill}%
\pgfsetlinewidth{0.000000pt}%
\definecolor{currentstroke}{rgb}{0.000000,0.000000,0.000000}%
\pgfsetstrokecolor{currentstroke}%
\pgfsetstrokeopacity{0.000000}%
\pgfsetdash{}{0pt}%
\pgfpathmoveto{\pgfqpoint{4.637718in}{1.609196in}}%
\pgfpathlineto{\pgfqpoint{4.646471in}{1.609196in}}%
\pgfpathlineto{\pgfqpoint{4.646471in}{1.693678in}}%
\pgfpathlineto{\pgfqpoint{4.637718in}{1.693678in}}%
\pgfpathlineto{\pgfqpoint{4.637718in}{1.609196in}}%
\pgfpathclose%
\pgfusepath{fill}%
\end{pgfscope}%
\begin{pgfscope}%
\pgfpathrectangle{\pgfqpoint{3.776708in}{0.600000in}}{\pgfqpoint{2.573292in}{2.070576in}}%
\pgfusepath{clip}%
\pgfsetbuttcap%
\pgfsetmiterjoin%
\definecolor{currentfill}{rgb}{0.133298,0.375282,0.379395}%
\pgfsetfillcolor{currentfill}%
\pgfsetlinewidth{0.000000pt}%
\definecolor{currentstroke}{rgb}{0.000000,0.000000,0.000000}%
\pgfsetstrokecolor{currentstroke}%
\pgfsetstrokeopacity{0.000000}%
\pgfsetdash{}{0pt}%
\pgfpathmoveto{\pgfqpoint{4.648660in}{1.609196in}}%
\pgfpathlineto{\pgfqpoint{4.657413in}{1.609196in}}%
\pgfpathlineto{\pgfqpoint{4.657413in}{1.668263in}}%
\pgfpathlineto{\pgfqpoint{4.648660in}{1.668263in}}%
\pgfpathlineto{\pgfqpoint{4.648660in}{1.609196in}}%
\pgfpathclose%
\pgfusepath{fill}%
\end{pgfscope}%
\begin{pgfscope}%
\pgfpathrectangle{\pgfqpoint{3.776708in}{0.600000in}}{\pgfqpoint{2.573292in}{2.070576in}}%
\pgfusepath{clip}%
\pgfsetbuttcap%
\pgfsetmiterjoin%
\definecolor{currentfill}{rgb}{0.133298,0.375282,0.379395}%
\pgfsetfillcolor{currentfill}%
\pgfsetlinewidth{0.000000pt}%
\definecolor{currentstroke}{rgb}{0.000000,0.000000,0.000000}%
\pgfsetstrokecolor{currentstroke}%
\pgfsetstrokeopacity{0.000000}%
\pgfsetdash{}{0pt}%
\pgfpathmoveto{\pgfqpoint{4.659601in}{1.609196in}}%
\pgfpathlineto{\pgfqpoint{4.668355in}{1.609196in}}%
\pgfpathlineto{\pgfqpoint{4.668355in}{1.648504in}}%
\pgfpathlineto{\pgfqpoint{4.659601in}{1.648504in}}%
\pgfpathlineto{\pgfqpoint{4.659601in}{1.609196in}}%
\pgfpathclose%
\pgfusepath{fill}%
\end{pgfscope}%
\begin{pgfscope}%
\pgfpathrectangle{\pgfqpoint{3.776708in}{0.600000in}}{\pgfqpoint{2.573292in}{2.070576in}}%
\pgfusepath{clip}%
\pgfsetbuttcap%
\pgfsetmiterjoin%
\definecolor{currentfill}{rgb}{0.133298,0.375282,0.379395}%
\pgfsetfillcolor{currentfill}%
\pgfsetlinewidth{0.000000pt}%
\definecolor{currentstroke}{rgb}{0.000000,0.000000,0.000000}%
\pgfsetstrokecolor{currentstroke}%
\pgfsetstrokeopacity{0.000000}%
\pgfsetdash{}{0pt}%
\pgfpathmoveto{\pgfqpoint{4.670543in}{1.609196in}}%
\pgfpathlineto{\pgfqpoint{4.679297in}{1.609196in}}%
\pgfpathlineto{\pgfqpoint{4.679297in}{1.624297in}}%
\pgfpathlineto{\pgfqpoint{4.670543in}{1.624297in}}%
\pgfpathlineto{\pgfqpoint{4.670543in}{1.609196in}}%
\pgfpathclose%
\pgfusepath{fill}%
\end{pgfscope}%
\begin{pgfscope}%
\pgfpathrectangle{\pgfqpoint{3.776708in}{0.600000in}}{\pgfqpoint{2.573292in}{2.070576in}}%
\pgfusepath{clip}%
\pgfsetbuttcap%
\pgfsetmiterjoin%
\definecolor{currentfill}{rgb}{0.133298,0.375282,0.379395}%
\pgfsetfillcolor{currentfill}%
\pgfsetlinewidth{0.000000pt}%
\definecolor{currentstroke}{rgb}{0.000000,0.000000,0.000000}%
\pgfsetstrokecolor{currentstroke}%
\pgfsetstrokeopacity{0.000000}%
\pgfsetdash{}{0pt}%
\pgfpathmoveto{\pgfqpoint{4.681485in}{1.537798in}}%
\pgfpathlineto{\pgfqpoint{4.690239in}{1.537798in}}%
\pgfpathlineto{\pgfqpoint{4.690239in}{1.527900in}}%
\pgfpathlineto{\pgfqpoint{4.681485in}{1.527900in}}%
\pgfpathlineto{\pgfqpoint{4.681485in}{1.537798in}}%
\pgfpathclose%
\pgfusepath{fill}%
\end{pgfscope}%
\begin{pgfscope}%
\pgfpathrectangle{\pgfqpoint{3.776708in}{0.600000in}}{\pgfqpoint{2.573292in}{2.070576in}}%
\pgfusepath{clip}%
\pgfsetbuttcap%
\pgfsetmiterjoin%
\definecolor{currentfill}{rgb}{0.133298,0.375282,0.379395}%
\pgfsetfillcolor{currentfill}%
\pgfsetlinewidth{0.000000pt}%
\definecolor{currentstroke}{rgb}{0.000000,0.000000,0.000000}%
\pgfsetstrokecolor{currentstroke}%
\pgfsetstrokeopacity{0.000000}%
\pgfsetdash{}{0pt}%
\pgfpathmoveto{\pgfqpoint{4.692427in}{1.541176in}}%
\pgfpathlineto{\pgfqpoint{4.701180in}{1.541176in}}%
\pgfpathlineto{\pgfqpoint{4.701180in}{1.505682in}}%
\pgfpathlineto{\pgfqpoint{4.692427in}{1.505682in}}%
\pgfpathlineto{\pgfqpoint{4.692427in}{1.541176in}}%
\pgfpathclose%
\pgfusepath{fill}%
\end{pgfscope}%
\begin{pgfscope}%
\pgfpathrectangle{\pgfqpoint{3.776708in}{0.600000in}}{\pgfqpoint{2.573292in}{2.070576in}}%
\pgfusepath{clip}%
\pgfsetbuttcap%
\pgfsetmiterjoin%
\definecolor{currentfill}{rgb}{0.133298,0.375282,0.379395}%
\pgfsetfillcolor{currentfill}%
\pgfsetlinewidth{0.000000pt}%
\definecolor{currentstroke}{rgb}{0.000000,0.000000,0.000000}%
\pgfsetstrokecolor{currentstroke}%
\pgfsetstrokeopacity{0.000000}%
\pgfsetdash{}{0pt}%
\pgfpathmoveto{\pgfqpoint{4.703369in}{1.545449in}}%
\pgfpathlineto{\pgfqpoint{4.712122in}{1.545449in}}%
\pgfpathlineto{\pgfqpoint{4.712122in}{1.486478in}}%
\pgfpathlineto{\pgfqpoint{4.703369in}{1.486478in}}%
\pgfpathlineto{\pgfqpoint{4.703369in}{1.545449in}}%
\pgfpathclose%
\pgfusepath{fill}%
\end{pgfscope}%
\begin{pgfscope}%
\pgfpathrectangle{\pgfqpoint{3.776708in}{0.600000in}}{\pgfqpoint{2.573292in}{2.070576in}}%
\pgfusepath{clip}%
\pgfsetbuttcap%
\pgfsetmiterjoin%
\definecolor{currentfill}{rgb}{0.133298,0.375282,0.379395}%
\pgfsetfillcolor{currentfill}%
\pgfsetlinewidth{0.000000pt}%
\definecolor{currentstroke}{rgb}{0.000000,0.000000,0.000000}%
\pgfsetstrokecolor{currentstroke}%
\pgfsetstrokeopacity{0.000000}%
\pgfsetdash{}{0pt}%
\pgfpathmoveto{\pgfqpoint{4.714310in}{1.550454in}}%
\pgfpathlineto{\pgfqpoint{4.723064in}{1.550454in}}%
\pgfpathlineto{\pgfqpoint{4.723064in}{1.464129in}}%
\pgfpathlineto{\pgfqpoint{4.714310in}{1.464129in}}%
\pgfpathlineto{\pgfqpoint{4.714310in}{1.550454in}}%
\pgfpathclose%
\pgfusepath{fill}%
\end{pgfscope}%
\begin{pgfscope}%
\pgfpathrectangle{\pgfqpoint{3.776708in}{0.600000in}}{\pgfqpoint{2.573292in}{2.070576in}}%
\pgfusepath{clip}%
\pgfsetbuttcap%
\pgfsetmiterjoin%
\definecolor{currentfill}{rgb}{0.133298,0.375282,0.379395}%
\pgfsetfillcolor{currentfill}%
\pgfsetlinewidth{0.000000pt}%
\definecolor{currentstroke}{rgb}{0.000000,0.000000,0.000000}%
\pgfsetstrokecolor{currentstroke}%
\pgfsetstrokeopacity{0.000000}%
\pgfsetdash{}{0pt}%
\pgfpathmoveto{\pgfqpoint{4.725252in}{1.553773in}}%
\pgfpathlineto{\pgfqpoint{4.734006in}{1.553773in}}%
\pgfpathlineto{\pgfqpoint{4.734006in}{1.441974in}}%
\pgfpathlineto{\pgfqpoint{4.725252in}{1.441974in}}%
\pgfpathlineto{\pgfqpoint{4.725252in}{1.553773in}}%
\pgfpathclose%
\pgfusepath{fill}%
\end{pgfscope}%
\begin{pgfscope}%
\pgfpathrectangle{\pgfqpoint{3.776708in}{0.600000in}}{\pgfqpoint{2.573292in}{2.070576in}}%
\pgfusepath{clip}%
\pgfsetbuttcap%
\pgfsetmiterjoin%
\definecolor{currentfill}{rgb}{0.133298,0.375282,0.379395}%
\pgfsetfillcolor{currentfill}%
\pgfsetlinewidth{0.000000pt}%
\definecolor{currentstroke}{rgb}{0.000000,0.000000,0.000000}%
\pgfsetstrokecolor{currentstroke}%
\pgfsetstrokeopacity{0.000000}%
\pgfsetdash{}{0pt}%
\pgfpathmoveto{\pgfqpoint{4.736194in}{1.557758in}}%
\pgfpathlineto{\pgfqpoint{4.744948in}{1.557758in}}%
\pgfpathlineto{\pgfqpoint{4.744948in}{1.422829in}}%
\pgfpathlineto{\pgfqpoint{4.736194in}{1.422829in}}%
\pgfpathlineto{\pgfqpoint{4.736194in}{1.557758in}}%
\pgfpathclose%
\pgfusepath{fill}%
\end{pgfscope}%
\begin{pgfscope}%
\pgfpathrectangle{\pgfqpoint{3.776708in}{0.600000in}}{\pgfqpoint{2.573292in}{2.070576in}}%
\pgfusepath{clip}%
\pgfsetbuttcap%
\pgfsetmiterjoin%
\definecolor{currentfill}{rgb}{0.133298,0.375282,0.379395}%
\pgfsetfillcolor{currentfill}%
\pgfsetlinewidth{0.000000pt}%
\definecolor{currentstroke}{rgb}{0.000000,0.000000,0.000000}%
\pgfsetstrokecolor{currentstroke}%
\pgfsetstrokeopacity{0.000000}%
\pgfsetdash{}{0pt}%
\pgfpathmoveto{\pgfqpoint{4.747136in}{1.557655in}}%
\pgfpathlineto{\pgfqpoint{4.755889in}{1.557655in}}%
\pgfpathlineto{\pgfqpoint{4.755889in}{1.396505in}}%
\pgfpathlineto{\pgfqpoint{4.747136in}{1.396505in}}%
\pgfpathlineto{\pgfqpoint{4.747136in}{1.557655in}}%
\pgfpathclose%
\pgfusepath{fill}%
\end{pgfscope}%
\begin{pgfscope}%
\pgfpathrectangle{\pgfqpoint{3.776708in}{0.600000in}}{\pgfqpoint{2.573292in}{2.070576in}}%
\pgfusepath{clip}%
\pgfsetbuttcap%
\pgfsetmiterjoin%
\definecolor{currentfill}{rgb}{0.133298,0.375282,0.379395}%
\pgfsetfillcolor{currentfill}%
\pgfsetlinewidth{0.000000pt}%
\definecolor{currentstroke}{rgb}{0.000000,0.000000,0.000000}%
\pgfsetstrokecolor{currentstroke}%
\pgfsetstrokeopacity{0.000000}%
\pgfsetdash{}{0pt}%
\pgfpathmoveto{\pgfqpoint{4.758078in}{1.558429in}}%
\pgfpathlineto{\pgfqpoint{4.766831in}{1.558429in}}%
\pgfpathlineto{\pgfqpoint{4.766831in}{1.373863in}}%
\pgfpathlineto{\pgfqpoint{4.758078in}{1.373863in}}%
\pgfpathlineto{\pgfqpoint{4.758078in}{1.558429in}}%
\pgfpathclose%
\pgfusepath{fill}%
\end{pgfscope}%
\begin{pgfscope}%
\pgfpathrectangle{\pgfqpoint{3.776708in}{0.600000in}}{\pgfqpoint{2.573292in}{2.070576in}}%
\pgfusepath{clip}%
\pgfsetbuttcap%
\pgfsetmiterjoin%
\definecolor{currentfill}{rgb}{0.133298,0.375282,0.379395}%
\pgfsetfillcolor{currentfill}%
\pgfsetlinewidth{0.000000pt}%
\definecolor{currentstroke}{rgb}{0.000000,0.000000,0.000000}%
\pgfsetstrokecolor{currentstroke}%
\pgfsetstrokeopacity{0.000000}%
\pgfsetdash{}{0pt}%
\pgfpathmoveto{\pgfqpoint{4.769019in}{1.559746in}}%
\pgfpathlineto{\pgfqpoint{4.777773in}{1.559746in}}%
\pgfpathlineto{\pgfqpoint{4.777773in}{1.353038in}}%
\pgfpathlineto{\pgfqpoint{4.769019in}{1.353038in}}%
\pgfpathlineto{\pgfqpoint{4.769019in}{1.559746in}}%
\pgfpathclose%
\pgfusepath{fill}%
\end{pgfscope}%
\begin{pgfscope}%
\pgfpathrectangle{\pgfqpoint{3.776708in}{0.600000in}}{\pgfqpoint{2.573292in}{2.070576in}}%
\pgfusepath{clip}%
\pgfsetbuttcap%
\pgfsetmiterjoin%
\definecolor{currentfill}{rgb}{0.133298,0.375282,0.379395}%
\pgfsetfillcolor{currentfill}%
\pgfsetlinewidth{0.000000pt}%
\definecolor{currentstroke}{rgb}{0.000000,0.000000,0.000000}%
\pgfsetstrokecolor{currentstroke}%
\pgfsetstrokeopacity{0.000000}%
\pgfsetdash{}{0pt}%
\pgfpathmoveto{\pgfqpoint{4.779961in}{1.562972in}}%
\pgfpathlineto{\pgfqpoint{4.788715in}{1.562972in}}%
\pgfpathlineto{\pgfqpoint{4.788715in}{1.337089in}}%
\pgfpathlineto{\pgfqpoint{4.779961in}{1.337089in}}%
\pgfpathlineto{\pgfqpoint{4.779961in}{1.562972in}}%
\pgfpathclose%
\pgfusepath{fill}%
\end{pgfscope}%
\begin{pgfscope}%
\pgfpathrectangle{\pgfqpoint{3.776708in}{0.600000in}}{\pgfqpoint{2.573292in}{2.070576in}}%
\pgfusepath{clip}%
\pgfsetbuttcap%
\pgfsetmiterjoin%
\definecolor{currentfill}{rgb}{0.133298,0.375282,0.379395}%
\pgfsetfillcolor{currentfill}%
\pgfsetlinewidth{0.000000pt}%
\definecolor{currentstroke}{rgb}{0.000000,0.000000,0.000000}%
\pgfsetstrokecolor{currentstroke}%
\pgfsetstrokeopacity{0.000000}%
\pgfsetdash{}{0pt}%
\pgfpathmoveto{\pgfqpoint{4.790903in}{1.569702in}}%
\pgfpathlineto{\pgfqpoint{4.799657in}{1.569702in}}%
\pgfpathlineto{\pgfqpoint{4.799657in}{1.323511in}}%
\pgfpathlineto{\pgfqpoint{4.790903in}{1.323511in}}%
\pgfpathlineto{\pgfqpoint{4.790903in}{1.569702in}}%
\pgfpathclose%
\pgfusepath{fill}%
\end{pgfscope}%
\begin{pgfscope}%
\pgfpathrectangle{\pgfqpoint{3.776708in}{0.600000in}}{\pgfqpoint{2.573292in}{2.070576in}}%
\pgfusepath{clip}%
\pgfsetbuttcap%
\pgfsetmiterjoin%
\definecolor{currentfill}{rgb}{0.133298,0.375282,0.379395}%
\pgfsetfillcolor{currentfill}%
\pgfsetlinewidth{0.000000pt}%
\definecolor{currentstroke}{rgb}{0.000000,0.000000,0.000000}%
\pgfsetstrokecolor{currentstroke}%
\pgfsetstrokeopacity{0.000000}%
\pgfsetdash{}{0pt}%
\pgfpathmoveto{\pgfqpoint{4.801845in}{1.573438in}}%
\pgfpathlineto{\pgfqpoint{4.810598in}{1.573438in}}%
\pgfpathlineto{\pgfqpoint{4.810598in}{1.308235in}}%
\pgfpathlineto{\pgfqpoint{4.801845in}{1.308235in}}%
\pgfpathlineto{\pgfqpoint{4.801845in}{1.573438in}}%
\pgfpathclose%
\pgfusepath{fill}%
\end{pgfscope}%
\begin{pgfscope}%
\pgfpathrectangle{\pgfqpoint{3.776708in}{0.600000in}}{\pgfqpoint{2.573292in}{2.070576in}}%
\pgfusepath{clip}%
\pgfsetbuttcap%
\pgfsetmiterjoin%
\definecolor{currentfill}{rgb}{0.133298,0.375282,0.379395}%
\pgfsetfillcolor{currentfill}%
\pgfsetlinewidth{0.000000pt}%
\definecolor{currentstroke}{rgb}{0.000000,0.000000,0.000000}%
\pgfsetstrokecolor{currentstroke}%
\pgfsetstrokeopacity{0.000000}%
\pgfsetdash{}{0pt}%
\pgfpathmoveto{\pgfqpoint{4.812787in}{1.576240in}}%
\pgfpathlineto{\pgfqpoint{4.821540in}{1.576240in}}%
\pgfpathlineto{\pgfqpoint{4.821540in}{1.291266in}}%
\pgfpathlineto{\pgfqpoint{4.812787in}{1.291266in}}%
\pgfpathlineto{\pgfqpoint{4.812787in}{1.576240in}}%
\pgfpathclose%
\pgfusepath{fill}%
\end{pgfscope}%
\begin{pgfscope}%
\pgfpathrectangle{\pgfqpoint{3.776708in}{0.600000in}}{\pgfqpoint{2.573292in}{2.070576in}}%
\pgfusepath{clip}%
\pgfsetbuttcap%
\pgfsetmiterjoin%
\definecolor{currentfill}{rgb}{0.133298,0.375282,0.379395}%
\pgfsetfillcolor{currentfill}%
\pgfsetlinewidth{0.000000pt}%
\definecolor{currentstroke}{rgb}{0.000000,0.000000,0.000000}%
\pgfsetstrokecolor{currentstroke}%
\pgfsetstrokeopacity{0.000000}%
\pgfsetdash{}{0pt}%
\pgfpathmoveto{\pgfqpoint{4.823728in}{1.583277in}}%
\pgfpathlineto{\pgfqpoint{4.832482in}{1.583277in}}%
\pgfpathlineto{\pgfqpoint{4.832482in}{1.277058in}}%
\pgfpathlineto{\pgfqpoint{4.823728in}{1.277058in}}%
\pgfpathlineto{\pgfqpoint{4.823728in}{1.583277in}}%
\pgfpathclose%
\pgfusepath{fill}%
\end{pgfscope}%
\begin{pgfscope}%
\pgfpathrectangle{\pgfqpoint{3.776708in}{0.600000in}}{\pgfqpoint{2.573292in}{2.070576in}}%
\pgfusepath{clip}%
\pgfsetbuttcap%
\pgfsetmiterjoin%
\definecolor{currentfill}{rgb}{0.133298,0.375282,0.379395}%
\pgfsetfillcolor{currentfill}%
\pgfsetlinewidth{0.000000pt}%
\definecolor{currentstroke}{rgb}{0.000000,0.000000,0.000000}%
\pgfsetstrokecolor{currentstroke}%
\pgfsetstrokeopacity{0.000000}%
\pgfsetdash{}{0pt}%
\pgfpathmoveto{\pgfqpoint{4.834670in}{1.586731in}}%
\pgfpathlineto{\pgfqpoint{4.843424in}{1.586731in}}%
\pgfpathlineto{\pgfqpoint{4.843424in}{1.260627in}}%
\pgfpathlineto{\pgfqpoint{4.834670in}{1.260627in}}%
\pgfpathlineto{\pgfqpoint{4.834670in}{1.586731in}}%
\pgfpathclose%
\pgfusepath{fill}%
\end{pgfscope}%
\begin{pgfscope}%
\pgfpathrectangle{\pgfqpoint{3.776708in}{0.600000in}}{\pgfqpoint{2.573292in}{2.070576in}}%
\pgfusepath{clip}%
\pgfsetbuttcap%
\pgfsetmiterjoin%
\definecolor{currentfill}{rgb}{0.133298,0.375282,0.379395}%
\pgfsetfillcolor{currentfill}%
\pgfsetlinewidth{0.000000pt}%
\definecolor{currentstroke}{rgb}{0.000000,0.000000,0.000000}%
\pgfsetstrokecolor{currentstroke}%
\pgfsetstrokeopacity{0.000000}%
\pgfsetdash{}{0pt}%
\pgfpathmoveto{\pgfqpoint{4.845612in}{1.593550in}}%
\pgfpathlineto{\pgfqpoint{4.854366in}{1.593550in}}%
\pgfpathlineto{\pgfqpoint{4.854366in}{1.249403in}}%
\pgfpathlineto{\pgfqpoint{4.845612in}{1.249403in}}%
\pgfpathlineto{\pgfqpoint{4.845612in}{1.593550in}}%
\pgfpathclose%
\pgfusepath{fill}%
\end{pgfscope}%
\begin{pgfscope}%
\pgfpathrectangle{\pgfqpoint{3.776708in}{0.600000in}}{\pgfqpoint{2.573292in}{2.070576in}}%
\pgfusepath{clip}%
\pgfsetbuttcap%
\pgfsetmiterjoin%
\definecolor{currentfill}{rgb}{0.133298,0.375282,0.379395}%
\pgfsetfillcolor{currentfill}%
\pgfsetlinewidth{0.000000pt}%
\definecolor{currentstroke}{rgb}{0.000000,0.000000,0.000000}%
\pgfsetstrokecolor{currentstroke}%
\pgfsetstrokeopacity{0.000000}%
\pgfsetdash{}{0pt}%
\pgfpathmoveto{\pgfqpoint{4.856554in}{1.598915in}}%
\pgfpathlineto{\pgfqpoint{4.865307in}{1.598915in}}%
\pgfpathlineto{\pgfqpoint{4.865307in}{1.236258in}}%
\pgfpathlineto{\pgfqpoint{4.856554in}{1.236258in}}%
\pgfpathlineto{\pgfqpoint{4.856554in}{1.598915in}}%
\pgfpathclose%
\pgfusepath{fill}%
\end{pgfscope}%
\begin{pgfscope}%
\pgfpathrectangle{\pgfqpoint{3.776708in}{0.600000in}}{\pgfqpoint{2.573292in}{2.070576in}}%
\pgfusepath{clip}%
\pgfsetbuttcap%
\pgfsetmiterjoin%
\definecolor{currentfill}{rgb}{0.133298,0.375282,0.379395}%
\pgfsetfillcolor{currentfill}%
\pgfsetlinewidth{0.000000pt}%
\definecolor{currentstroke}{rgb}{0.000000,0.000000,0.000000}%
\pgfsetstrokecolor{currentstroke}%
\pgfsetstrokeopacity{0.000000}%
\pgfsetdash{}{0pt}%
\pgfpathmoveto{\pgfqpoint{4.867496in}{1.604175in}}%
\pgfpathlineto{\pgfqpoint{4.876249in}{1.604175in}}%
\pgfpathlineto{\pgfqpoint{4.876249in}{1.216892in}}%
\pgfpathlineto{\pgfqpoint{4.867496in}{1.216892in}}%
\pgfpathlineto{\pgfqpoint{4.867496in}{1.604175in}}%
\pgfpathclose%
\pgfusepath{fill}%
\end{pgfscope}%
\begin{pgfscope}%
\pgfpathrectangle{\pgfqpoint{3.776708in}{0.600000in}}{\pgfqpoint{2.573292in}{2.070576in}}%
\pgfusepath{clip}%
\pgfsetbuttcap%
\pgfsetmiterjoin%
\definecolor{currentfill}{rgb}{0.133298,0.375282,0.379395}%
\pgfsetfillcolor{currentfill}%
\pgfsetlinewidth{0.000000pt}%
\definecolor{currentstroke}{rgb}{0.000000,0.000000,0.000000}%
\pgfsetstrokecolor{currentstroke}%
\pgfsetstrokeopacity{0.000000}%
\pgfsetdash{}{0pt}%
\pgfpathmoveto{\pgfqpoint{4.878437in}{1.608819in}}%
\pgfpathlineto{\pgfqpoint{4.887191in}{1.608819in}}%
\pgfpathlineto{\pgfqpoint{4.887191in}{1.198969in}}%
\pgfpathlineto{\pgfqpoint{4.878437in}{1.198969in}}%
\pgfpathlineto{\pgfqpoint{4.878437in}{1.608819in}}%
\pgfpathclose%
\pgfusepath{fill}%
\end{pgfscope}%
\begin{pgfscope}%
\pgfpathrectangle{\pgfqpoint{3.776708in}{0.600000in}}{\pgfqpoint{2.573292in}{2.070576in}}%
\pgfusepath{clip}%
\pgfsetbuttcap%
\pgfsetmiterjoin%
\definecolor{currentfill}{rgb}{0.133298,0.375282,0.379395}%
\pgfsetfillcolor{currentfill}%
\pgfsetlinewidth{0.000000pt}%
\definecolor{currentstroke}{rgb}{0.000000,0.000000,0.000000}%
\pgfsetstrokecolor{currentstroke}%
\pgfsetstrokeopacity{0.000000}%
\pgfsetdash{}{0pt}%
\pgfpathmoveto{\pgfqpoint{4.889379in}{1.609196in}}%
\pgfpathlineto{\pgfqpoint{4.898133in}{1.609196in}}%
\pgfpathlineto{\pgfqpoint{4.898133in}{1.175137in}}%
\pgfpathlineto{\pgfqpoint{4.889379in}{1.175137in}}%
\pgfpathlineto{\pgfqpoint{4.889379in}{1.609196in}}%
\pgfpathclose%
\pgfusepath{fill}%
\end{pgfscope}%
\begin{pgfscope}%
\pgfpathrectangle{\pgfqpoint{3.776708in}{0.600000in}}{\pgfqpoint{2.573292in}{2.070576in}}%
\pgfusepath{clip}%
\pgfsetbuttcap%
\pgfsetmiterjoin%
\definecolor{currentfill}{rgb}{0.133298,0.375282,0.379395}%
\pgfsetfillcolor{currentfill}%
\pgfsetlinewidth{0.000000pt}%
\definecolor{currentstroke}{rgb}{0.000000,0.000000,0.000000}%
\pgfsetstrokecolor{currentstroke}%
\pgfsetstrokeopacity{0.000000}%
\pgfsetdash{}{0pt}%
\pgfpathmoveto{\pgfqpoint{4.900321in}{1.609196in}}%
\pgfpathlineto{\pgfqpoint{4.909075in}{1.609196in}}%
\pgfpathlineto{\pgfqpoint{4.909075in}{1.147418in}}%
\pgfpathlineto{\pgfqpoint{4.900321in}{1.147418in}}%
\pgfpathlineto{\pgfqpoint{4.900321in}{1.609196in}}%
\pgfpathclose%
\pgfusepath{fill}%
\end{pgfscope}%
\begin{pgfscope}%
\pgfpathrectangle{\pgfqpoint{3.776708in}{0.600000in}}{\pgfqpoint{2.573292in}{2.070576in}}%
\pgfusepath{clip}%
\pgfsetbuttcap%
\pgfsetmiterjoin%
\definecolor{currentfill}{rgb}{0.133298,0.375282,0.379395}%
\pgfsetfillcolor{currentfill}%
\pgfsetlinewidth{0.000000pt}%
\definecolor{currentstroke}{rgb}{0.000000,0.000000,0.000000}%
\pgfsetstrokecolor{currentstroke}%
\pgfsetstrokeopacity{0.000000}%
\pgfsetdash{}{0pt}%
\pgfpathmoveto{\pgfqpoint{4.911263in}{1.609196in}}%
\pgfpathlineto{\pgfqpoint{4.920016in}{1.609196in}}%
\pgfpathlineto{\pgfqpoint{4.920016in}{1.122612in}}%
\pgfpathlineto{\pgfqpoint{4.911263in}{1.122612in}}%
\pgfpathlineto{\pgfqpoint{4.911263in}{1.609196in}}%
\pgfpathclose%
\pgfusepath{fill}%
\end{pgfscope}%
\begin{pgfscope}%
\pgfpathrectangle{\pgfqpoint{3.776708in}{0.600000in}}{\pgfqpoint{2.573292in}{2.070576in}}%
\pgfusepath{clip}%
\pgfsetbuttcap%
\pgfsetmiterjoin%
\definecolor{currentfill}{rgb}{0.133298,0.375282,0.379395}%
\pgfsetfillcolor{currentfill}%
\pgfsetlinewidth{0.000000pt}%
\definecolor{currentstroke}{rgb}{0.000000,0.000000,0.000000}%
\pgfsetstrokecolor{currentstroke}%
\pgfsetstrokeopacity{0.000000}%
\pgfsetdash{}{0pt}%
\pgfpathmoveto{\pgfqpoint{4.922205in}{1.609196in}}%
\pgfpathlineto{\pgfqpoint{4.930958in}{1.609196in}}%
\pgfpathlineto{\pgfqpoint{4.930958in}{1.104879in}}%
\pgfpathlineto{\pgfqpoint{4.922205in}{1.104879in}}%
\pgfpathlineto{\pgfqpoint{4.922205in}{1.609196in}}%
\pgfpathclose%
\pgfusepath{fill}%
\end{pgfscope}%
\begin{pgfscope}%
\pgfpathrectangle{\pgfqpoint{3.776708in}{0.600000in}}{\pgfqpoint{2.573292in}{2.070576in}}%
\pgfusepath{clip}%
\pgfsetbuttcap%
\pgfsetmiterjoin%
\definecolor{currentfill}{rgb}{0.133298,0.375282,0.379395}%
\pgfsetfillcolor{currentfill}%
\pgfsetlinewidth{0.000000pt}%
\definecolor{currentstroke}{rgb}{0.000000,0.000000,0.000000}%
\pgfsetstrokecolor{currentstroke}%
\pgfsetstrokeopacity{0.000000}%
\pgfsetdash{}{0pt}%
\pgfpathmoveto{\pgfqpoint{4.933146in}{1.609196in}}%
\pgfpathlineto{\pgfqpoint{4.941900in}{1.609196in}}%
\pgfpathlineto{\pgfqpoint{4.941900in}{1.087366in}}%
\pgfpathlineto{\pgfqpoint{4.933146in}{1.087366in}}%
\pgfpathlineto{\pgfqpoint{4.933146in}{1.609196in}}%
\pgfpathclose%
\pgfusepath{fill}%
\end{pgfscope}%
\begin{pgfscope}%
\pgfpathrectangle{\pgfqpoint{3.776708in}{0.600000in}}{\pgfqpoint{2.573292in}{2.070576in}}%
\pgfusepath{clip}%
\pgfsetbuttcap%
\pgfsetmiterjoin%
\definecolor{currentfill}{rgb}{0.133298,0.375282,0.379395}%
\pgfsetfillcolor{currentfill}%
\pgfsetlinewidth{0.000000pt}%
\definecolor{currentstroke}{rgb}{0.000000,0.000000,0.000000}%
\pgfsetstrokecolor{currentstroke}%
\pgfsetstrokeopacity{0.000000}%
\pgfsetdash{}{0pt}%
\pgfpathmoveto{\pgfqpoint{4.944088in}{1.609196in}}%
\pgfpathlineto{\pgfqpoint{4.952842in}{1.609196in}}%
\pgfpathlineto{\pgfqpoint{4.952842in}{1.066231in}}%
\pgfpathlineto{\pgfqpoint{4.944088in}{1.066231in}}%
\pgfpathlineto{\pgfqpoint{4.944088in}{1.609196in}}%
\pgfpathclose%
\pgfusepath{fill}%
\end{pgfscope}%
\begin{pgfscope}%
\pgfpathrectangle{\pgfqpoint{3.776708in}{0.600000in}}{\pgfqpoint{2.573292in}{2.070576in}}%
\pgfusepath{clip}%
\pgfsetbuttcap%
\pgfsetmiterjoin%
\definecolor{currentfill}{rgb}{0.133298,0.375282,0.379395}%
\pgfsetfillcolor{currentfill}%
\pgfsetlinewidth{0.000000pt}%
\definecolor{currentstroke}{rgb}{0.000000,0.000000,0.000000}%
\pgfsetstrokecolor{currentstroke}%
\pgfsetstrokeopacity{0.000000}%
\pgfsetdash{}{0pt}%
\pgfpathmoveto{\pgfqpoint{4.955030in}{1.609196in}}%
\pgfpathlineto{\pgfqpoint{4.963783in}{1.609196in}}%
\pgfpathlineto{\pgfqpoint{4.963783in}{1.045913in}}%
\pgfpathlineto{\pgfqpoint{4.955030in}{1.045913in}}%
\pgfpathlineto{\pgfqpoint{4.955030in}{1.609196in}}%
\pgfpathclose%
\pgfusepath{fill}%
\end{pgfscope}%
\begin{pgfscope}%
\pgfpathrectangle{\pgfqpoint{3.776708in}{0.600000in}}{\pgfqpoint{2.573292in}{2.070576in}}%
\pgfusepath{clip}%
\pgfsetbuttcap%
\pgfsetmiterjoin%
\definecolor{currentfill}{rgb}{0.133298,0.375282,0.379395}%
\pgfsetfillcolor{currentfill}%
\pgfsetlinewidth{0.000000pt}%
\definecolor{currentstroke}{rgb}{0.000000,0.000000,0.000000}%
\pgfsetstrokecolor{currentstroke}%
\pgfsetstrokeopacity{0.000000}%
\pgfsetdash{}{0pt}%
\pgfpathmoveto{\pgfqpoint{4.965972in}{1.609196in}}%
\pgfpathlineto{\pgfqpoint{4.974725in}{1.609196in}}%
\pgfpathlineto{\pgfqpoint{4.974725in}{1.029817in}}%
\pgfpathlineto{\pgfqpoint{4.965972in}{1.029817in}}%
\pgfpathlineto{\pgfqpoint{4.965972in}{1.609196in}}%
\pgfpathclose%
\pgfusepath{fill}%
\end{pgfscope}%
\begin{pgfscope}%
\pgfpathrectangle{\pgfqpoint{3.776708in}{0.600000in}}{\pgfqpoint{2.573292in}{2.070576in}}%
\pgfusepath{clip}%
\pgfsetbuttcap%
\pgfsetmiterjoin%
\definecolor{currentfill}{rgb}{0.133298,0.375282,0.379395}%
\pgfsetfillcolor{currentfill}%
\pgfsetlinewidth{0.000000pt}%
\definecolor{currentstroke}{rgb}{0.000000,0.000000,0.000000}%
\pgfsetstrokecolor{currentstroke}%
\pgfsetstrokeopacity{0.000000}%
\pgfsetdash{}{0pt}%
\pgfpathmoveto{\pgfqpoint{4.976914in}{1.609196in}}%
\pgfpathlineto{\pgfqpoint{4.985667in}{1.609196in}}%
\pgfpathlineto{\pgfqpoint{4.985667in}{1.010644in}}%
\pgfpathlineto{\pgfqpoint{4.976914in}{1.010644in}}%
\pgfpathlineto{\pgfqpoint{4.976914in}{1.609196in}}%
\pgfpathclose%
\pgfusepath{fill}%
\end{pgfscope}%
\begin{pgfscope}%
\pgfpathrectangle{\pgfqpoint{3.776708in}{0.600000in}}{\pgfqpoint{2.573292in}{2.070576in}}%
\pgfusepath{clip}%
\pgfsetbuttcap%
\pgfsetmiterjoin%
\definecolor{currentfill}{rgb}{0.133298,0.375282,0.379395}%
\pgfsetfillcolor{currentfill}%
\pgfsetlinewidth{0.000000pt}%
\definecolor{currentstroke}{rgb}{0.000000,0.000000,0.000000}%
\pgfsetstrokecolor{currentstroke}%
\pgfsetstrokeopacity{0.000000}%
\pgfsetdash{}{0pt}%
\pgfpathmoveto{\pgfqpoint{4.987855in}{1.609196in}}%
\pgfpathlineto{\pgfqpoint{4.996609in}{1.609196in}}%
\pgfpathlineto{\pgfqpoint{4.996609in}{0.994779in}}%
\pgfpathlineto{\pgfqpoint{4.987855in}{0.994779in}}%
\pgfpathlineto{\pgfqpoint{4.987855in}{1.609196in}}%
\pgfpathclose%
\pgfusepath{fill}%
\end{pgfscope}%
\begin{pgfscope}%
\pgfpathrectangle{\pgfqpoint{3.776708in}{0.600000in}}{\pgfqpoint{2.573292in}{2.070576in}}%
\pgfusepath{clip}%
\pgfsetbuttcap%
\pgfsetmiterjoin%
\definecolor{currentfill}{rgb}{0.133298,0.375282,0.379395}%
\pgfsetfillcolor{currentfill}%
\pgfsetlinewidth{0.000000pt}%
\definecolor{currentstroke}{rgb}{0.000000,0.000000,0.000000}%
\pgfsetstrokecolor{currentstroke}%
\pgfsetstrokeopacity{0.000000}%
\pgfsetdash{}{0pt}%
\pgfpathmoveto{\pgfqpoint{4.998797in}{1.609196in}}%
\pgfpathlineto{\pgfqpoint{5.007551in}{1.609196in}}%
\pgfpathlineto{\pgfqpoint{5.007551in}{0.977409in}}%
\pgfpathlineto{\pgfqpoint{4.998797in}{0.977409in}}%
\pgfpathlineto{\pgfqpoint{4.998797in}{1.609196in}}%
\pgfpathclose%
\pgfusepath{fill}%
\end{pgfscope}%
\begin{pgfscope}%
\pgfpathrectangle{\pgfqpoint{3.776708in}{0.600000in}}{\pgfqpoint{2.573292in}{2.070576in}}%
\pgfusepath{clip}%
\pgfsetbuttcap%
\pgfsetmiterjoin%
\definecolor{currentfill}{rgb}{0.133298,0.375282,0.379395}%
\pgfsetfillcolor{currentfill}%
\pgfsetlinewidth{0.000000pt}%
\definecolor{currentstroke}{rgb}{0.000000,0.000000,0.000000}%
\pgfsetstrokecolor{currentstroke}%
\pgfsetstrokeopacity{0.000000}%
\pgfsetdash{}{0pt}%
\pgfpathmoveto{\pgfqpoint{5.009739in}{1.609196in}}%
\pgfpathlineto{\pgfqpoint{5.018492in}{1.609196in}}%
\pgfpathlineto{\pgfqpoint{5.018492in}{0.958015in}}%
\pgfpathlineto{\pgfqpoint{5.009739in}{0.958015in}}%
\pgfpathlineto{\pgfqpoint{5.009739in}{1.609196in}}%
\pgfpathclose%
\pgfusepath{fill}%
\end{pgfscope}%
\begin{pgfscope}%
\pgfpathrectangle{\pgfqpoint{3.776708in}{0.600000in}}{\pgfqpoint{2.573292in}{2.070576in}}%
\pgfusepath{clip}%
\pgfsetbuttcap%
\pgfsetmiterjoin%
\definecolor{currentfill}{rgb}{0.133298,0.375282,0.379395}%
\pgfsetfillcolor{currentfill}%
\pgfsetlinewidth{0.000000pt}%
\definecolor{currentstroke}{rgb}{0.000000,0.000000,0.000000}%
\pgfsetstrokecolor{currentstroke}%
\pgfsetstrokeopacity{0.000000}%
\pgfsetdash{}{0pt}%
\pgfpathmoveto{\pgfqpoint{5.020681in}{1.609196in}}%
\pgfpathlineto{\pgfqpoint{5.029434in}{1.609196in}}%
\pgfpathlineto{\pgfqpoint{5.029434in}{0.944201in}}%
\pgfpathlineto{\pgfqpoint{5.020681in}{0.944201in}}%
\pgfpathlineto{\pgfqpoint{5.020681in}{1.609196in}}%
\pgfpathclose%
\pgfusepath{fill}%
\end{pgfscope}%
\begin{pgfscope}%
\pgfpathrectangle{\pgfqpoint{3.776708in}{0.600000in}}{\pgfqpoint{2.573292in}{2.070576in}}%
\pgfusepath{clip}%
\pgfsetbuttcap%
\pgfsetmiterjoin%
\definecolor{currentfill}{rgb}{0.133298,0.375282,0.379395}%
\pgfsetfillcolor{currentfill}%
\pgfsetlinewidth{0.000000pt}%
\definecolor{currentstroke}{rgb}{0.000000,0.000000,0.000000}%
\pgfsetstrokecolor{currentstroke}%
\pgfsetstrokeopacity{0.000000}%
\pgfsetdash{}{0pt}%
\pgfpathmoveto{\pgfqpoint{5.031623in}{1.609196in}}%
\pgfpathlineto{\pgfqpoint{5.040376in}{1.609196in}}%
\pgfpathlineto{\pgfqpoint{5.040376in}{0.929592in}}%
\pgfpathlineto{\pgfqpoint{5.031623in}{0.929592in}}%
\pgfpathlineto{\pgfqpoint{5.031623in}{1.609196in}}%
\pgfpathclose%
\pgfusepath{fill}%
\end{pgfscope}%
\begin{pgfscope}%
\pgfpathrectangle{\pgfqpoint{3.776708in}{0.600000in}}{\pgfqpoint{2.573292in}{2.070576in}}%
\pgfusepath{clip}%
\pgfsetbuttcap%
\pgfsetmiterjoin%
\definecolor{currentfill}{rgb}{0.133298,0.375282,0.379395}%
\pgfsetfillcolor{currentfill}%
\pgfsetlinewidth{0.000000pt}%
\definecolor{currentstroke}{rgb}{0.000000,0.000000,0.000000}%
\pgfsetstrokecolor{currentstroke}%
\pgfsetstrokeopacity{0.000000}%
\pgfsetdash{}{0pt}%
\pgfpathmoveto{\pgfqpoint{5.042564in}{1.609196in}}%
\pgfpathlineto{\pgfqpoint{5.051318in}{1.609196in}}%
\pgfpathlineto{\pgfqpoint{5.051318in}{0.918365in}}%
\pgfpathlineto{\pgfqpoint{5.042564in}{0.918365in}}%
\pgfpathlineto{\pgfqpoint{5.042564in}{1.609196in}}%
\pgfpathclose%
\pgfusepath{fill}%
\end{pgfscope}%
\begin{pgfscope}%
\pgfpathrectangle{\pgfqpoint{3.776708in}{0.600000in}}{\pgfqpoint{2.573292in}{2.070576in}}%
\pgfusepath{clip}%
\pgfsetbuttcap%
\pgfsetmiterjoin%
\definecolor{currentfill}{rgb}{0.133298,0.375282,0.379395}%
\pgfsetfillcolor{currentfill}%
\pgfsetlinewidth{0.000000pt}%
\definecolor{currentstroke}{rgb}{0.000000,0.000000,0.000000}%
\pgfsetstrokecolor{currentstroke}%
\pgfsetstrokeopacity{0.000000}%
\pgfsetdash{}{0pt}%
\pgfpathmoveto{\pgfqpoint{5.053506in}{1.609196in}}%
\pgfpathlineto{\pgfqpoint{5.062260in}{1.609196in}}%
\pgfpathlineto{\pgfqpoint{5.062260in}{0.904029in}}%
\pgfpathlineto{\pgfqpoint{5.053506in}{0.904029in}}%
\pgfpathlineto{\pgfqpoint{5.053506in}{1.609196in}}%
\pgfpathclose%
\pgfusepath{fill}%
\end{pgfscope}%
\begin{pgfscope}%
\pgfpathrectangle{\pgfqpoint{3.776708in}{0.600000in}}{\pgfqpoint{2.573292in}{2.070576in}}%
\pgfusepath{clip}%
\pgfsetbuttcap%
\pgfsetmiterjoin%
\definecolor{currentfill}{rgb}{0.133298,0.375282,0.379395}%
\pgfsetfillcolor{currentfill}%
\pgfsetlinewidth{0.000000pt}%
\definecolor{currentstroke}{rgb}{0.000000,0.000000,0.000000}%
\pgfsetstrokecolor{currentstroke}%
\pgfsetstrokeopacity{0.000000}%
\pgfsetdash{}{0pt}%
\pgfpathmoveto{\pgfqpoint{5.064448in}{1.609196in}}%
\pgfpathlineto{\pgfqpoint{5.073201in}{1.609196in}}%
\pgfpathlineto{\pgfqpoint{5.073201in}{0.891737in}}%
\pgfpathlineto{\pgfqpoint{5.064448in}{0.891737in}}%
\pgfpathlineto{\pgfqpoint{5.064448in}{1.609196in}}%
\pgfpathclose%
\pgfusepath{fill}%
\end{pgfscope}%
\begin{pgfscope}%
\pgfpathrectangle{\pgfqpoint{3.776708in}{0.600000in}}{\pgfqpoint{2.573292in}{2.070576in}}%
\pgfusepath{clip}%
\pgfsetbuttcap%
\pgfsetmiterjoin%
\definecolor{currentfill}{rgb}{0.133298,0.375282,0.379395}%
\pgfsetfillcolor{currentfill}%
\pgfsetlinewidth{0.000000pt}%
\definecolor{currentstroke}{rgb}{0.000000,0.000000,0.000000}%
\pgfsetstrokecolor{currentstroke}%
\pgfsetstrokeopacity{0.000000}%
\pgfsetdash{}{0pt}%
\pgfpathmoveto{\pgfqpoint{5.075390in}{1.609196in}}%
\pgfpathlineto{\pgfqpoint{5.084143in}{1.609196in}}%
\pgfpathlineto{\pgfqpoint{5.084143in}{0.877451in}}%
\pgfpathlineto{\pgfqpoint{5.075390in}{0.877451in}}%
\pgfpathlineto{\pgfqpoint{5.075390in}{1.609196in}}%
\pgfpathclose%
\pgfusepath{fill}%
\end{pgfscope}%
\begin{pgfscope}%
\pgfpathrectangle{\pgfqpoint{3.776708in}{0.600000in}}{\pgfqpoint{2.573292in}{2.070576in}}%
\pgfusepath{clip}%
\pgfsetbuttcap%
\pgfsetmiterjoin%
\definecolor{currentfill}{rgb}{0.133298,0.375282,0.379395}%
\pgfsetfillcolor{currentfill}%
\pgfsetlinewidth{0.000000pt}%
\definecolor{currentstroke}{rgb}{0.000000,0.000000,0.000000}%
\pgfsetstrokecolor{currentstroke}%
\pgfsetstrokeopacity{0.000000}%
\pgfsetdash{}{0pt}%
\pgfpathmoveto{\pgfqpoint{5.086332in}{1.609196in}}%
\pgfpathlineto{\pgfqpoint{5.095085in}{1.609196in}}%
\pgfpathlineto{\pgfqpoint{5.095085in}{0.862917in}}%
\pgfpathlineto{\pgfqpoint{5.086332in}{0.862917in}}%
\pgfpathlineto{\pgfqpoint{5.086332in}{1.609196in}}%
\pgfpathclose%
\pgfusepath{fill}%
\end{pgfscope}%
\begin{pgfscope}%
\pgfpathrectangle{\pgfqpoint{3.776708in}{0.600000in}}{\pgfqpoint{2.573292in}{2.070576in}}%
\pgfusepath{clip}%
\pgfsetbuttcap%
\pgfsetmiterjoin%
\definecolor{currentfill}{rgb}{0.133298,0.375282,0.379395}%
\pgfsetfillcolor{currentfill}%
\pgfsetlinewidth{0.000000pt}%
\definecolor{currentstroke}{rgb}{0.000000,0.000000,0.000000}%
\pgfsetstrokecolor{currentstroke}%
\pgfsetstrokeopacity{0.000000}%
\pgfsetdash{}{0pt}%
\pgfpathmoveto{\pgfqpoint{5.097273in}{1.609196in}}%
\pgfpathlineto{\pgfqpoint{5.106027in}{1.609196in}}%
\pgfpathlineto{\pgfqpoint{5.106027in}{0.847365in}}%
\pgfpathlineto{\pgfqpoint{5.097273in}{0.847365in}}%
\pgfpathlineto{\pgfqpoint{5.097273in}{1.609196in}}%
\pgfpathclose%
\pgfusepath{fill}%
\end{pgfscope}%
\begin{pgfscope}%
\pgfpathrectangle{\pgfqpoint{3.776708in}{0.600000in}}{\pgfqpoint{2.573292in}{2.070576in}}%
\pgfusepath{clip}%
\pgfsetbuttcap%
\pgfsetmiterjoin%
\definecolor{currentfill}{rgb}{0.133298,0.375282,0.379395}%
\pgfsetfillcolor{currentfill}%
\pgfsetlinewidth{0.000000pt}%
\definecolor{currentstroke}{rgb}{0.000000,0.000000,0.000000}%
\pgfsetstrokecolor{currentstroke}%
\pgfsetstrokeopacity{0.000000}%
\pgfsetdash{}{0pt}%
\pgfpathmoveto{\pgfqpoint{5.108215in}{1.609196in}}%
\pgfpathlineto{\pgfqpoint{5.116969in}{1.609196in}}%
\pgfpathlineto{\pgfqpoint{5.116969in}{0.832516in}}%
\pgfpathlineto{\pgfqpoint{5.108215in}{0.832516in}}%
\pgfpathlineto{\pgfqpoint{5.108215in}{1.609196in}}%
\pgfpathclose%
\pgfusepath{fill}%
\end{pgfscope}%
\begin{pgfscope}%
\pgfpathrectangle{\pgfqpoint{3.776708in}{0.600000in}}{\pgfqpoint{2.573292in}{2.070576in}}%
\pgfusepath{clip}%
\pgfsetbuttcap%
\pgfsetmiterjoin%
\definecolor{currentfill}{rgb}{0.133298,0.375282,0.379395}%
\pgfsetfillcolor{currentfill}%
\pgfsetlinewidth{0.000000pt}%
\definecolor{currentstroke}{rgb}{0.000000,0.000000,0.000000}%
\pgfsetstrokecolor{currentstroke}%
\pgfsetstrokeopacity{0.000000}%
\pgfsetdash{}{0pt}%
\pgfpathmoveto{\pgfqpoint{5.119157in}{1.609196in}}%
\pgfpathlineto{\pgfqpoint{5.127910in}{1.609196in}}%
\pgfpathlineto{\pgfqpoint{5.127910in}{0.816969in}}%
\pgfpathlineto{\pgfqpoint{5.119157in}{0.816969in}}%
\pgfpathlineto{\pgfqpoint{5.119157in}{1.609196in}}%
\pgfpathclose%
\pgfusepath{fill}%
\end{pgfscope}%
\begin{pgfscope}%
\pgfpathrectangle{\pgfqpoint{3.776708in}{0.600000in}}{\pgfqpoint{2.573292in}{2.070576in}}%
\pgfusepath{clip}%
\pgfsetbuttcap%
\pgfsetmiterjoin%
\definecolor{currentfill}{rgb}{0.133298,0.375282,0.379395}%
\pgfsetfillcolor{currentfill}%
\pgfsetlinewidth{0.000000pt}%
\definecolor{currentstroke}{rgb}{0.000000,0.000000,0.000000}%
\pgfsetstrokecolor{currentstroke}%
\pgfsetstrokeopacity{0.000000}%
\pgfsetdash{}{0pt}%
\pgfpathmoveto{\pgfqpoint{5.130099in}{1.609196in}}%
\pgfpathlineto{\pgfqpoint{5.138852in}{1.609196in}}%
\pgfpathlineto{\pgfqpoint{5.138852in}{0.801056in}}%
\pgfpathlineto{\pgfqpoint{5.130099in}{0.801056in}}%
\pgfpathlineto{\pgfqpoint{5.130099in}{1.609196in}}%
\pgfpathclose%
\pgfusepath{fill}%
\end{pgfscope}%
\begin{pgfscope}%
\pgfpathrectangle{\pgfqpoint{3.776708in}{0.600000in}}{\pgfqpoint{2.573292in}{2.070576in}}%
\pgfusepath{clip}%
\pgfsetbuttcap%
\pgfsetmiterjoin%
\definecolor{currentfill}{rgb}{0.133298,0.375282,0.379395}%
\pgfsetfillcolor{currentfill}%
\pgfsetlinewidth{0.000000pt}%
\definecolor{currentstroke}{rgb}{0.000000,0.000000,0.000000}%
\pgfsetstrokecolor{currentstroke}%
\pgfsetstrokeopacity{0.000000}%
\pgfsetdash{}{0pt}%
\pgfpathmoveto{\pgfqpoint{5.141041in}{1.609196in}}%
\pgfpathlineto{\pgfqpoint{5.149794in}{1.609196in}}%
\pgfpathlineto{\pgfqpoint{5.149794in}{0.784114in}}%
\pgfpathlineto{\pgfqpoint{5.141041in}{0.784114in}}%
\pgfpathlineto{\pgfqpoint{5.141041in}{1.609196in}}%
\pgfpathclose%
\pgfusepath{fill}%
\end{pgfscope}%
\begin{pgfscope}%
\pgfpathrectangle{\pgfqpoint{3.776708in}{0.600000in}}{\pgfqpoint{2.573292in}{2.070576in}}%
\pgfusepath{clip}%
\pgfsetbuttcap%
\pgfsetmiterjoin%
\definecolor{currentfill}{rgb}{0.133298,0.375282,0.379395}%
\pgfsetfillcolor{currentfill}%
\pgfsetlinewidth{0.000000pt}%
\definecolor{currentstroke}{rgb}{0.000000,0.000000,0.000000}%
\pgfsetstrokecolor{currentstroke}%
\pgfsetstrokeopacity{0.000000}%
\pgfsetdash{}{0pt}%
\pgfpathmoveto{\pgfqpoint{5.151982in}{1.609196in}}%
\pgfpathlineto{\pgfqpoint{5.160736in}{1.609196in}}%
\pgfpathlineto{\pgfqpoint{5.160736in}{0.765725in}}%
\pgfpathlineto{\pgfqpoint{5.151982in}{0.765725in}}%
\pgfpathlineto{\pgfqpoint{5.151982in}{1.609196in}}%
\pgfpathclose%
\pgfusepath{fill}%
\end{pgfscope}%
\begin{pgfscope}%
\pgfpathrectangle{\pgfqpoint{3.776708in}{0.600000in}}{\pgfqpoint{2.573292in}{2.070576in}}%
\pgfusepath{clip}%
\pgfsetbuttcap%
\pgfsetmiterjoin%
\definecolor{currentfill}{rgb}{0.133298,0.375282,0.379395}%
\pgfsetfillcolor{currentfill}%
\pgfsetlinewidth{0.000000pt}%
\definecolor{currentstroke}{rgb}{0.000000,0.000000,0.000000}%
\pgfsetstrokecolor{currentstroke}%
\pgfsetstrokeopacity{0.000000}%
\pgfsetdash{}{0pt}%
\pgfpathmoveto{\pgfqpoint{5.162924in}{1.609196in}}%
\pgfpathlineto{\pgfqpoint{5.171678in}{1.609196in}}%
\pgfpathlineto{\pgfqpoint{5.171678in}{0.747989in}}%
\pgfpathlineto{\pgfqpoint{5.162924in}{0.747989in}}%
\pgfpathlineto{\pgfqpoint{5.162924in}{1.609196in}}%
\pgfpathclose%
\pgfusepath{fill}%
\end{pgfscope}%
\begin{pgfscope}%
\pgfpathrectangle{\pgfqpoint{3.776708in}{0.600000in}}{\pgfqpoint{2.573292in}{2.070576in}}%
\pgfusepath{clip}%
\pgfsetbuttcap%
\pgfsetmiterjoin%
\definecolor{currentfill}{rgb}{0.133298,0.375282,0.379395}%
\pgfsetfillcolor{currentfill}%
\pgfsetlinewidth{0.000000pt}%
\definecolor{currentstroke}{rgb}{0.000000,0.000000,0.000000}%
\pgfsetstrokecolor{currentstroke}%
\pgfsetstrokeopacity{0.000000}%
\pgfsetdash{}{0pt}%
\pgfpathmoveto{\pgfqpoint{5.173866in}{1.609196in}}%
\pgfpathlineto{\pgfqpoint{5.182619in}{1.609196in}}%
\pgfpathlineto{\pgfqpoint{5.182619in}{0.733356in}}%
\pgfpathlineto{\pgfqpoint{5.173866in}{0.733356in}}%
\pgfpathlineto{\pgfqpoint{5.173866in}{1.609196in}}%
\pgfpathclose%
\pgfusepath{fill}%
\end{pgfscope}%
\begin{pgfscope}%
\pgfpathrectangle{\pgfqpoint{3.776708in}{0.600000in}}{\pgfqpoint{2.573292in}{2.070576in}}%
\pgfusepath{clip}%
\pgfsetbuttcap%
\pgfsetmiterjoin%
\definecolor{currentfill}{rgb}{0.133298,0.375282,0.379395}%
\pgfsetfillcolor{currentfill}%
\pgfsetlinewidth{0.000000pt}%
\definecolor{currentstroke}{rgb}{0.000000,0.000000,0.000000}%
\pgfsetstrokecolor{currentstroke}%
\pgfsetstrokeopacity{0.000000}%
\pgfsetdash{}{0pt}%
\pgfpathmoveto{\pgfqpoint{5.184808in}{1.609196in}}%
\pgfpathlineto{\pgfqpoint{5.193561in}{1.609196in}}%
\pgfpathlineto{\pgfqpoint{5.193561in}{0.722080in}}%
\pgfpathlineto{\pgfqpoint{5.184808in}{0.722080in}}%
\pgfpathlineto{\pgfqpoint{5.184808in}{1.609196in}}%
\pgfpathclose%
\pgfusepath{fill}%
\end{pgfscope}%
\begin{pgfscope}%
\pgfpathrectangle{\pgfqpoint{3.776708in}{0.600000in}}{\pgfqpoint{2.573292in}{2.070576in}}%
\pgfusepath{clip}%
\pgfsetbuttcap%
\pgfsetmiterjoin%
\definecolor{currentfill}{rgb}{0.133298,0.375282,0.379395}%
\pgfsetfillcolor{currentfill}%
\pgfsetlinewidth{0.000000pt}%
\definecolor{currentstroke}{rgb}{0.000000,0.000000,0.000000}%
\pgfsetstrokecolor{currentstroke}%
\pgfsetstrokeopacity{0.000000}%
\pgfsetdash{}{0pt}%
\pgfpathmoveto{\pgfqpoint{5.195750in}{1.609196in}}%
\pgfpathlineto{\pgfqpoint{5.204503in}{1.609196in}}%
\pgfpathlineto{\pgfqpoint{5.204503in}{0.712100in}}%
\pgfpathlineto{\pgfqpoint{5.195750in}{0.712100in}}%
\pgfpathlineto{\pgfqpoint{5.195750in}{1.609196in}}%
\pgfpathclose%
\pgfusepath{fill}%
\end{pgfscope}%
\begin{pgfscope}%
\pgfpathrectangle{\pgfqpoint{3.776708in}{0.600000in}}{\pgfqpoint{2.573292in}{2.070576in}}%
\pgfusepath{clip}%
\pgfsetbuttcap%
\pgfsetmiterjoin%
\definecolor{currentfill}{rgb}{0.133298,0.375282,0.379395}%
\pgfsetfillcolor{currentfill}%
\pgfsetlinewidth{0.000000pt}%
\definecolor{currentstroke}{rgb}{0.000000,0.000000,0.000000}%
\pgfsetstrokecolor{currentstroke}%
\pgfsetstrokeopacity{0.000000}%
\pgfsetdash{}{0pt}%
\pgfpathmoveto{\pgfqpoint{5.206691in}{1.609196in}}%
\pgfpathlineto{\pgfqpoint{5.215445in}{1.609196in}}%
\pgfpathlineto{\pgfqpoint{5.215445in}{0.703274in}}%
\pgfpathlineto{\pgfqpoint{5.206691in}{0.703274in}}%
\pgfpathlineto{\pgfqpoint{5.206691in}{1.609196in}}%
\pgfpathclose%
\pgfusepath{fill}%
\end{pgfscope}%
\begin{pgfscope}%
\pgfpathrectangle{\pgfqpoint{3.776708in}{0.600000in}}{\pgfqpoint{2.573292in}{2.070576in}}%
\pgfusepath{clip}%
\pgfsetbuttcap%
\pgfsetmiterjoin%
\definecolor{currentfill}{rgb}{0.133298,0.375282,0.379395}%
\pgfsetfillcolor{currentfill}%
\pgfsetlinewidth{0.000000pt}%
\definecolor{currentstroke}{rgb}{0.000000,0.000000,0.000000}%
\pgfsetstrokecolor{currentstroke}%
\pgfsetstrokeopacity{0.000000}%
\pgfsetdash{}{0pt}%
\pgfpathmoveto{\pgfqpoint{5.217633in}{1.609196in}}%
\pgfpathlineto{\pgfqpoint{5.226387in}{1.609196in}}%
\pgfpathlineto{\pgfqpoint{5.226387in}{0.700089in}}%
\pgfpathlineto{\pgfqpoint{5.217633in}{0.700089in}}%
\pgfpathlineto{\pgfqpoint{5.217633in}{1.609196in}}%
\pgfpathclose%
\pgfusepath{fill}%
\end{pgfscope}%
\begin{pgfscope}%
\pgfpathrectangle{\pgfqpoint{3.776708in}{0.600000in}}{\pgfqpoint{2.573292in}{2.070576in}}%
\pgfusepath{clip}%
\pgfsetbuttcap%
\pgfsetmiterjoin%
\definecolor{currentfill}{rgb}{0.133298,0.375282,0.379395}%
\pgfsetfillcolor{currentfill}%
\pgfsetlinewidth{0.000000pt}%
\definecolor{currentstroke}{rgb}{0.000000,0.000000,0.000000}%
\pgfsetstrokecolor{currentstroke}%
\pgfsetstrokeopacity{0.000000}%
\pgfsetdash{}{0pt}%
\pgfpathmoveto{\pgfqpoint{5.228575in}{1.609196in}}%
\pgfpathlineto{\pgfqpoint{5.237328in}{1.609196in}}%
\pgfpathlineto{\pgfqpoint{5.237328in}{0.698498in}}%
\pgfpathlineto{\pgfqpoint{5.228575in}{0.698498in}}%
\pgfpathlineto{\pgfqpoint{5.228575in}{1.609196in}}%
\pgfpathclose%
\pgfusepath{fill}%
\end{pgfscope}%
\begin{pgfscope}%
\pgfpathrectangle{\pgfqpoint{3.776708in}{0.600000in}}{\pgfqpoint{2.573292in}{2.070576in}}%
\pgfusepath{clip}%
\pgfsetbuttcap%
\pgfsetmiterjoin%
\definecolor{currentfill}{rgb}{0.133298,0.375282,0.379395}%
\pgfsetfillcolor{currentfill}%
\pgfsetlinewidth{0.000000pt}%
\definecolor{currentstroke}{rgb}{0.000000,0.000000,0.000000}%
\pgfsetstrokecolor{currentstroke}%
\pgfsetstrokeopacity{0.000000}%
\pgfsetdash{}{0pt}%
\pgfpathmoveto{\pgfqpoint{5.239517in}{1.609196in}}%
\pgfpathlineto{\pgfqpoint{5.248270in}{1.609196in}}%
\pgfpathlineto{\pgfqpoint{5.248270in}{0.694117in}}%
\pgfpathlineto{\pgfqpoint{5.239517in}{0.694117in}}%
\pgfpathlineto{\pgfqpoint{5.239517in}{1.609196in}}%
\pgfpathclose%
\pgfusepath{fill}%
\end{pgfscope}%
\begin{pgfscope}%
\pgfpathrectangle{\pgfqpoint{3.776708in}{0.600000in}}{\pgfqpoint{2.573292in}{2.070576in}}%
\pgfusepath{clip}%
\pgfsetbuttcap%
\pgfsetmiterjoin%
\definecolor{currentfill}{rgb}{0.133298,0.375282,0.379395}%
\pgfsetfillcolor{currentfill}%
\pgfsetlinewidth{0.000000pt}%
\definecolor{currentstroke}{rgb}{0.000000,0.000000,0.000000}%
\pgfsetstrokecolor{currentstroke}%
\pgfsetstrokeopacity{0.000000}%
\pgfsetdash{}{0pt}%
\pgfpathmoveto{\pgfqpoint{5.250459in}{1.609196in}}%
\pgfpathlineto{\pgfqpoint{5.259212in}{1.609196in}}%
\pgfpathlineto{\pgfqpoint{5.259212in}{0.695887in}}%
\pgfpathlineto{\pgfqpoint{5.250459in}{0.695887in}}%
\pgfpathlineto{\pgfqpoint{5.250459in}{1.609196in}}%
\pgfpathclose%
\pgfusepath{fill}%
\end{pgfscope}%
\begin{pgfscope}%
\pgfpathrectangle{\pgfqpoint{3.776708in}{0.600000in}}{\pgfqpoint{2.573292in}{2.070576in}}%
\pgfusepath{clip}%
\pgfsetbuttcap%
\pgfsetmiterjoin%
\definecolor{currentfill}{rgb}{0.133298,0.375282,0.379395}%
\pgfsetfillcolor{currentfill}%
\pgfsetlinewidth{0.000000pt}%
\definecolor{currentstroke}{rgb}{0.000000,0.000000,0.000000}%
\pgfsetstrokecolor{currentstroke}%
\pgfsetstrokeopacity{0.000000}%
\pgfsetdash{}{0pt}%
\pgfpathmoveto{\pgfqpoint{5.261400in}{1.609196in}}%
\pgfpathlineto{\pgfqpoint{5.270154in}{1.609196in}}%
\pgfpathlineto{\pgfqpoint{5.270154in}{0.700069in}}%
\pgfpathlineto{\pgfqpoint{5.261400in}{0.700069in}}%
\pgfpathlineto{\pgfqpoint{5.261400in}{1.609196in}}%
\pgfpathclose%
\pgfusepath{fill}%
\end{pgfscope}%
\begin{pgfscope}%
\pgfpathrectangle{\pgfqpoint{3.776708in}{0.600000in}}{\pgfqpoint{2.573292in}{2.070576in}}%
\pgfusepath{clip}%
\pgfsetbuttcap%
\pgfsetmiterjoin%
\definecolor{currentfill}{rgb}{0.133298,0.375282,0.379395}%
\pgfsetfillcolor{currentfill}%
\pgfsetlinewidth{0.000000pt}%
\definecolor{currentstroke}{rgb}{0.000000,0.000000,0.000000}%
\pgfsetstrokecolor{currentstroke}%
\pgfsetstrokeopacity{0.000000}%
\pgfsetdash{}{0pt}%
\pgfpathmoveto{\pgfqpoint{5.272342in}{1.609196in}}%
\pgfpathlineto{\pgfqpoint{5.281096in}{1.609196in}}%
\pgfpathlineto{\pgfqpoint{5.281096in}{0.706866in}}%
\pgfpathlineto{\pgfqpoint{5.272342in}{0.706866in}}%
\pgfpathlineto{\pgfqpoint{5.272342in}{1.609196in}}%
\pgfpathclose%
\pgfusepath{fill}%
\end{pgfscope}%
\begin{pgfscope}%
\pgfpathrectangle{\pgfqpoint{3.776708in}{0.600000in}}{\pgfqpoint{2.573292in}{2.070576in}}%
\pgfusepath{clip}%
\pgfsetbuttcap%
\pgfsetmiterjoin%
\definecolor{currentfill}{rgb}{0.133298,0.375282,0.379395}%
\pgfsetfillcolor{currentfill}%
\pgfsetlinewidth{0.000000pt}%
\definecolor{currentstroke}{rgb}{0.000000,0.000000,0.000000}%
\pgfsetstrokecolor{currentstroke}%
\pgfsetstrokeopacity{0.000000}%
\pgfsetdash{}{0pt}%
\pgfpathmoveto{\pgfqpoint{5.283284in}{1.609196in}}%
\pgfpathlineto{\pgfqpoint{5.292037in}{1.609196in}}%
\pgfpathlineto{\pgfqpoint{5.292037in}{0.719340in}}%
\pgfpathlineto{\pgfqpoint{5.283284in}{0.719340in}}%
\pgfpathlineto{\pgfqpoint{5.283284in}{1.609196in}}%
\pgfpathclose%
\pgfusepath{fill}%
\end{pgfscope}%
\begin{pgfscope}%
\pgfpathrectangle{\pgfqpoint{3.776708in}{0.600000in}}{\pgfqpoint{2.573292in}{2.070576in}}%
\pgfusepath{clip}%
\pgfsetbuttcap%
\pgfsetmiterjoin%
\definecolor{currentfill}{rgb}{0.133298,0.375282,0.379395}%
\pgfsetfillcolor{currentfill}%
\pgfsetlinewidth{0.000000pt}%
\definecolor{currentstroke}{rgb}{0.000000,0.000000,0.000000}%
\pgfsetstrokecolor{currentstroke}%
\pgfsetstrokeopacity{0.000000}%
\pgfsetdash{}{0pt}%
\pgfpathmoveto{\pgfqpoint{5.294226in}{1.609196in}}%
\pgfpathlineto{\pgfqpoint{5.302979in}{1.609196in}}%
\pgfpathlineto{\pgfqpoint{5.302979in}{0.737132in}}%
\pgfpathlineto{\pgfqpoint{5.294226in}{0.737132in}}%
\pgfpathlineto{\pgfqpoint{5.294226in}{1.609196in}}%
\pgfpathclose%
\pgfusepath{fill}%
\end{pgfscope}%
\begin{pgfscope}%
\pgfpathrectangle{\pgfqpoint{3.776708in}{0.600000in}}{\pgfqpoint{2.573292in}{2.070576in}}%
\pgfusepath{clip}%
\pgfsetbuttcap%
\pgfsetmiterjoin%
\definecolor{currentfill}{rgb}{0.133298,0.375282,0.379395}%
\pgfsetfillcolor{currentfill}%
\pgfsetlinewidth{0.000000pt}%
\definecolor{currentstroke}{rgb}{0.000000,0.000000,0.000000}%
\pgfsetstrokecolor{currentstroke}%
\pgfsetstrokeopacity{0.000000}%
\pgfsetdash{}{0pt}%
\pgfpathmoveto{\pgfqpoint{5.305168in}{1.609196in}}%
\pgfpathlineto{\pgfqpoint{5.313921in}{1.609196in}}%
\pgfpathlineto{\pgfqpoint{5.313921in}{0.754561in}}%
\pgfpathlineto{\pgfqpoint{5.305168in}{0.754561in}}%
\pgfpathlineto{\pgfqpoint{5.305168in}{1.609196in}}%
\pgfpathclose%
\pgfusepath{fill}%
\end{pgfscope}%
\begin{pgfscope}%
\pgfpathrectangle{\pgfqpoint{3.776708in}{0.600000in}}{\pgfqpoint{2.573292in}{2.070576in}}%
\pgfusepath{clip}%
\pgfsetbuttcap%
\pgfsetmiterjoin%
\definecolor{currentfill}{rgb}{0.133298,0.375282,0.379395}%
\pgfsetfillcolor{currentfill}%
\pgfsetlinewidth{0.000000pt}%
\definecolor{currentstroke}{rgb}{0.000000,0.000000,0.000000}%
\pgfsetstrokecolor{currentstroke}%
\pgfsetstrokeopacity{0.000000}%
\pgfsetdash{}{0pt}%
\pgfpathmoveto{\pgfqpoint{5.316109in}{1.609196in}}%
\pgfpathlineto{\pgfqpoint{5.324863in}{1.609196in}}%
\pgfpathlineto{\pgfqpoint{5.324863in}{0.776628in}}%
\pgfpathlineto{\pgfqpoint{5.316109in}{0.776628in}}%
\pgfpathlineto{\pgfqpoint{5.316109in}{1.609196in}}%
\pgfpathclose%
\pgfusepath{fill}%
\end{pgfscope}%
\begin{pgfscope}%
\pgfpathrectangle{\pgfqpoint{3.776708in}{0.600000in}}{\pgfqpoint{2.573292in}{2.070576in}}%
\pgfusepath{clip}%
\pgfsetbuttcap%
\pgfsetmiterjoin%
\definecolor{currentfill}{rgb}{0.133298,0.375282,0.379395}%
\pgfsetfillcolor{currentfill}%
\pgfsetlinewidth{0.000000pt}%
\definecolor{currentstroke}{rgb}{0.000000,0.000000,0.000000}%
\pgfsetstrokecolor{currentstroke}%
\pgfsetstrokeopacity{0.000000}%
\pgfsetdash{}{0pt}%
\pgfpathmoveto{\pgfqpoint{5.327051in}{1.609196in}}%
\pgfpathlineto{\pgfqpoint{5.335805in}{1.609196in}}%
\pgfpathlineto{\pgfqpoint{5.335805in}{0.801073in}}%
\pgfpathlineto{\pgfqpoint{5.327051in}{0.801073in}}%
\pgfpathlineto{\pgfqpoint{5.327051in}{1.609196in}}%
\pgfpathclose%
\pgfusepath{fill}%
\end{pgfscope}%
\begin{pgfscope}%
\pgfpathrectangle{\pgfqpoint{3.776708in}{0.600000in}}{\pgfqpoint{2.573292in}{2.070576in}}%
\pgfusepath{clip}%
\pgfsetbuttcap%
\pgfsetmiterjoin%
\definecolor{currentfill}{rgb}{0.133298,0.375282,0.379395}%
\pgfsetfillcolor{currentfill}%
\pgfsetlinewidth{0.000000pt}%
\definecolor{currentstroke}{rgb}{0.000000,0.000000,0.000000}%
\pgfsetstrokecolor{currentstroke}%
\pgfsetstrokeopacity{0.000000}%
\pgfsetdash{}{0pt}%
\pgfpathmoveto{\pgfqpoint{5.337993in}{1.609196in}}%
\pgfpathlineto{\pgfqpoint{5.346746in}{1.609196in}}%
\pgfpathlineto{\pgfqpoint{5.346746in}{0.825999in}}%
\pgfpathlineto{\pgfqpoint{5.337993in}{0.825999in}}%
\pgfpathlineto{\pgfqpoint{5.337993in}{1.609196in}}%
\pgfpathclose%
\pgfusepath{fill}%
\end{pgfscope}%
\begin{pgfscope}%
\pgfpathrectangle{\pgfqpoint{3.776708in}{0.600000in}}{\pgfqpoint{2.573292in}{2.070576in}}%
\pgfusepath{clip}%
\pgfsetbuttcap%
\pgfsetmiterjoin%
\definecolor{currentfill}{rgb}{0.133298,0.375282,0.379395}%
\pgfsetfillcolor{currentfill}%
\pgfsetlinewidth{0.000000pt}%
\definecolor{currentstroke}{rgb}{0.000000,0.000000,0.000000}%
\pgfsetstrokecolor{currentstroke}%
\pgfsetstrokeopacity{0.000000}%
\pgfsetdash{}{0pt}%
\pgfpathmoveto{\pgfqpoint{5.348935in}{1.609196in}}%
\pgfpathlineto{\pgfqpoint{5.357688in}{1.609196in}}%
\pgfpathlineto{\pgfqpoint{5.357688in}{0.855951in}}%
\pgfpathlineto{\pgfqpoint{5.348935in}{0.855951in}}%
\pgfpathlineto{\pgfqpoint{5.348935in}{1.609196in}}%
\pgfpathclose%
\pgfusepath{fill}%
\end{pgfscope}%
\begin{pgfscope}%
\pgfpathrectangle{\pgfqpoint{3.776708in}{0.600000in}}{\pgfqpoint{2.573292in}{2.070576in}}%
\pgfusepath{clip}%
\pgfsetbuttcap%
\pgfsetmiterjoin%
\definecolor{currentfill}{rgb}{0.133298,0.375282,0.379395}%
\pgfsetfillcolor{currentfill}%
\pgfsetlinewidth{0.000000pt}%
\definecolor{currentstroke}{rgb}{0.000000,0.000000,0.000000}%
\pgfsetstrokecolor{currentstroke}%
\pgfsetstrokeopacity{0.000000}%
\pgfsetdash{}{0pt}%
\pgfpathmoveto{\pgfqpoint{5.359877in}{1.609196in}}%
\pgfpathlineto{\pgfqpoint{5.368630in}{1.609196in}}%
\pgfpathlineto{\pgfqpoint{5.368630in}{0.885200in}}%
\pgfpathlineto{\pgfqpoint{5.359877in}{0.885200in}}%
\pgfpathlineto{\pgfqpoint{5.359877in}{1.609196in}}%
\pgfpathclose%
\pgfusepath{fill}%
\end{pgfscope}%
\begin{pgfscope}%
\pgfpathrectangle{\pgfqpoint{3.776708in}{0.600000in}}{\pgfqpoint{2.573292in}{2.070576in}}%
\pgfusepath{clip}%
\pgfsetbuttcap%
\pgfsetmiterjoin%
\definecolor{currentfill}{rgb}{0.133298,0.375282,0.379395}%
\pgfsetfillcolor{currentfill}%
\pgfsetlinewidth{0.000000pt}%
\definecolor{currentstroke}{rgb}{0.000000,0.000000,0.000000}%
\pgfsetstrokecolor{currentstroke}%
\pgfsetstrokeopacity{0.000000}%
\pgfsetdash{}{0pt}%
\pgfpathmoveto{\pgfqpoint{5.370818in}{1.609196in}}%
\pgfpathlineto{\pgfqpoint{5.379572in}{1.609196in}}%
\pgfpathlineto{\pgfqpoint{5.379572in}{0.910602in}}%
\pgfpathlineto{\pgfqpoint{5.370818in}{0.910602in}}%
\pgfpathlineto{\pgfqpoint{5.370818in}{1.609196in}}%
\pgfpathclose%
\pgfusepath{fill}%
\end{pgfscope}%
\begin{pgfscope}%
\pgfpathrectangle{\pgfqpoint{3.776708in}{0.600000in}}{\pgfqpoint{2.573292in}{2.070576in}}%
\pgfusepath{clip}%
\pgfsetbuttcap%
\pgfsetmiterjoin%
\definecolor{currentfill}{rgb}{0.133298,0.375282,0.379395}%
\pgfsetfillcolor{currentfill}%
\pgfsetlinewidth{0.000000pt}%
\definecolor{currentstroke}{rgb}{0.000000,0.000000,0.000000}%
\pgfsetstrokecolor{currentstroke}%
\pgfsetstrokeopacity{0.000000}%
\pgfsetdash{}{0pt}%
\pgfpathmoveto{\pgfqpoint{5.381760in}{1.609196in}}%
\pgfpathlineto{\pgfqpoint{5.390514in}{1.609196in}}%
\pgfpathlineto{\pgfqpoint{5.390514in}{0.935717in}}%
\pgfpathlineto{\pgfqpoint{5.381760in}{0.935717in}}%
\pgfpathlineto{\pgfqpoint{5.381760in}{1.609196in}}%
\pgfpathclose%
\pgfusepath{fill}%
\end{pgfscope}%
\begin{pgfscope}%
\pgfpathrectangle{\pgfqpoint{3.776708in}{0.600000in}}{\pgfqpoint{2.573292in}{2.070576in}}%
\pgfusepath{clip}%
\pgfsetbuttcap%
\pgfsetmiterjoin%
\definecolor{currentfill}{rgb}{0.133298,0.375282,0.379395}%
\pgfsetfillcolor{currentfill}%
\pgfsetlinewidth{0.000000pt}%
\definecolor{currentstroke}{rgb}{0.000000,0.000000,0.000000}%
\pgfsetstrokecolor{currentstroke}%
\pgfsetstrokeopacity{0.000000}%
\pgfsetdash{}{0pt}%
\pgfpathmoveto{\pgfqpoint{5.392702in}{1.609196in}}%
\pgfpathlineto{\pgfqpoint{5.401455in}{1.609196in}}%
\pgfpathlineto{\pgfqpoint{5.401455in}{0.960430in}}%
\pgfpathlineto{\pgfqpoint{5.392702in}{0.960430in}}%
\pgfpathlineto{\pgfqpoint{5.392702in}{1.609196in}}%
\pgfpathclose%
\pgfusepath{fill}%
\end{pgfscope}%
\begin{pgfscope}%
\pgfpathrectangle{\pgfqpoint{3.776708in}{0.600000in}}{\pgfqpoint{2.573292in}{2.070576in}}%
\pgfusepath{clip}%
\pgfsetbuttcap%
\pgfsetmiterjoin%
\definecolor{currentfill}{rgb}{0.133298,0.375282,0.379395}%
\pgfsetfillcolor{currentfill}%
\pgfsetlinewidth{0.000000pt}%
\definecolor{currentstroke}{rgb}{0.000000,0.000000,0.000000}%
\pgfsetstrokecolor{currentstroke}%
\pgfsetstrokeopacity{0.000000}%
\pgfsetdash{}{0pt}%
\pgfpathmoveto{\pgfqpoint{5.403644in}{1.609196in}}%
\pgfpathlineto{\pgfqpoint{5.412397in}{1.609196in}}%
\pgfpathlineto{\pgfqpoint{5.412397in}{0.989894in}}%
\pgfpathlineto{\pgfqpoint{5.403644in}{0.989894in}}%
\pgfpathlineto{\pgfqpoint{5.403644in}{1.609196in}}%
\pgfpathclose%
\pgfusepath{fill}%
\end{pgfscope}%
\begin{pgfscope}%
\pgfpathrectangle{\pgfqpoint{3.776708in}{0.600000in}}{\pgfqpoint{2.573292in}{2.070576in}}%
\pgfusepath{clip}%
\pgfsetbuttcap%
\pgfsetmiterjoin%
\definecolor{currentfill}{rgb}{0.133298,0.375282,0.379395}%
\pgfsetfillcolor{currentfill}%
\pgfsetlinewidth{0.000000pt}%
\definecolor{currentstroke}{rgb}{0.000000,0.000000,0.000000}%
\pgfsetstrokecolor{currentstroke}%
\pgfsetstrokeopacity{0.000000}%
\pgfsetdash{}{0pt}%
\pgfpathmoveto{\pgfqpoint{5.414586in}{1.609196in}}%
\pgfpathlineto{\pgfqpoint{5.423339in}{1.609196in}}%
\pgfpathlineto{\pgfqpoint{5.423339in}{1.018013in}}%
\pgfpathlineto{\pgfqpoint{5.414586in}{1.018013in}}%
\pgfpathlineto{\pgfqpoint{5.414586in}{1.609196in}}%
\pgfpathclose%
\pgfusepath{fill}%
\end{pgfscope}%
\begin{pgfscope}%
\pgfpathrectangle{\pgfqpoint{3.776708in}{0.600000in}}{\pgfqpoint{2.573292in}{2.070576in}}%
\pgfusepath{clip}%
\pgfsetbuttcap%
\pgfsetmiterjoin%
\definecolor{currentfill}{rgb}{0.133298,0.375282,0.379395}%
\pgfsetfillcolor{currentfill}%
\pgfsetlinewidth{0.000000pt}%
\definecolor{currentstroke}{rgb}{0.000000,0.000000,0.000000}%
\pgfsetstrokecolor{currentstroke}%
\pgfsetstrokeopacity{0.000000}%
\pgfsetdash{}{0pt}%
\pgfpathmoveto{\pgfqpoint{5.425527in}{1.609196in}}%
\pgfpathlineto{\pgfqpoint{5.434281in}{1.609196in}}%
\pgfpathlineto{\pgfqpoint{5.434281in}{1.048547in}}%
\pgfpathlineto{\pgfqpoint{5.425527in}{1.048547in}}%
\pgfpathlineto{\pgfqpoint{5.425527in}{1.609196in}}%
\pgfpathclose%
\pgfusepath{fill}%
\end{pgfscope}%
\begin{pgfscope}%
\pgfpathrectangle{\pgfqpoint{3.776708in}{0.600000in}}{\pgfqpoint{2.573292in}{2.070576in}}%
\pgfusepath{clip}%
\pgfsetbuttcap%
\pgfsetmiterjoin%
\definecolor{currentfill}{rgb}{0.133298,0.375282,0.379395}%
\pgfsetfillcolor{currentfill}%
\pgfsetlinewidth{0.000000pt}%
\definecolor{currentstroke}{rgb}{0.000000,0.000000,0.000000}%
\pgfsetstrokecolor{currentstroke}%
\pgfsetstrokeopacity{0.000000}%
\pgfsetdash{}{0pt}%
\pgfpathmoveto{\pgfqpoint{5.436469in}{1.609196in}}%
\pgfpathlineto{\pgfqpoint{5.445223in}{1.609196in}}%
\pgfpathlineto{\pgfqpoint{5.445223in}{1.081062in}}%
\pgfpathlineto{\pgfqpoint{5.436469in}{1.081062in}}%
\pgfpathlineto{\pgfqpoint{5.436469in}{1.609196in}}%
\pgfpathclose%
\pgfusepath{fill}%
\end{pgfscope}%
\begin{pgfscope}%
\pgfpathrectangle{\pgfqpoint{3.776708in}{0.600000in}}{\pgfqpoint{2.573292in}{2.070576in}}%
\pgfusepath{clip}%
\pgfsetbuttcap%
\pgfsetmiterjoin%
\definecolor{currentfill}{rgb}{0.133298,0.375282,0.379395}%
\pgfsetfillcolor{currentfill}%
\pgfsetlinewidth{0.000000pt}%
\definecolor{currentstroke}{rgb}{0.000000,0.000000,0.000000}%
\pgfsetstrokecolor{currentstroke}%
\pgfsetstrokeopacity{0.000000}%
\pgfsetdash{}{0pt}%
\pgfpathmoveto{\pgfqpoint{5.447411in}{1.609196in}}%
\pgfpathlineto{\pgfqpoint{5.456164in}{1.609196in}}%
\pgfpathlineto{\pgfqpoint{5.456164in}{1.115304in}}%
\pgfpathlineto{\pgfqpoint{5.447411in}{1.115304in}}%
\pgfpathlineto{\pgfqpoint{5.447411in}{1.609196in}}%
\pgfpathclose%
\pgfusepath{fill}%
\end{pgfscope}%
\begin{pgfscope}%
\pgfpathrectangle{\pgfqpoint{3.776708in}{0.600000in}}{\pgfqpoint{2.573292in}{2.070576in}}%
\pgfusepath{clip}%
\pgfsetbuttcap%
\pgfsetmiterjoin%
\definecolor{currentfill}{rgb}{0.133298,0.375282,0.379395}%
\pgfsetfillcolor{currentfill}%
\pgfsetlinewidth{0.000000pt}%
\definecolor{currentstroke}{rgb}{0.000000,0.000000,0.000000}%
\pgfsetstrokecolor{currentstroke}%
\pgfsetstrokeopacity{0.000000}%
\pgfsetdash{}{0pt}%
\pgfpathmoveto{\pgfqpoint{5.458353in}{1.609196in}}%
\pgfpathlineto{\pgfqpoint{5.467106in}{1.609196in}}%
\pgfpathlineto{\pgfqpoint{5.467106in}{1.150574in}}%
\pgfpathlineto{\pgfqpoint{5.458353in}{1.150574in}}%
\pgfpathlineto{\pgfqpoint{5.458353in}{1.609196in}}%
\pgfpathclose%
\pgfusepath{fill}%
\end{pgfscope}%
\begin{pgfscope}%
\pgfpathrectangle{\pgfqpoint{3.776708in}{0.600000in}}{\pgfqpoint{2.573292in}{2.070576in}}%
\pgfusepath{clip}%
\pgfsetbuttcap%
\pgfsetmiterjoin%
\definecolor{currentfill}{rgb}{0.133298,0.375282,0.379395}%
\pgfsetfillcolor{currentfill}%
\pgfsetlinewidth{0.000000pt}%
\definecolor{currentstroke}{rgb}{0.000000,0.000000,0.000000}%
\pgfsetstrokecolor{currentstroke}%
\pgfsetstrokeopacity{0.000000}%
\pgfsetdash{}{0pt}%
\pgfpathmoveto{\pgfqpoint{5.469295in}{1.609196in}}%
\pgfpathlineto{\pgfqpoint{5.478048in}{1.609196in}}%
\pgfpathlineto{\pgfqpoint{5.478048in}{1.185898in}}%
\pgfpathlineto{\pgfqpoint{5.469295in}{1.185898in}}%
\pgfpathlineto{\pgfqpoint{5.469295in}{1.609196in}}%
\pgfpathclose%
\pgfusepath{fill}%
\end{pgfscope}%
\begin{pgfscope}%
\pgfpathrectangle{\pgfqpoint{3.776708in}{0.600000in}}{\pgfqpoint{2.573292in}{2.070576in}}%
\pgfusepath{clip}%
\pgfsetbuttcap%
\pgfsetmiterjoin%
\definecolor{currentfill}{rgb}{0.133298,0.375282,0.379395}%
\pgfsetfillcolor{currentfill}%
\pgfsetlinewidth{0.000000pt}%
\definecolor{currentstroke}{rgb}{0.000000,0.000000,0.000000}%
\pgfsetstrokecolor{currentstroke}%
\pgfsetstrokeopacity{0.000000}%
\pgfsetdash{}{0pt}%
\pgfpathmoveto{\pgfqpoint{5.480236in}{1.609196in}}%
\pgfpathlineto{\pgfqpoint{5.488990in}{1.609196in}}%
\pgfpathlineto{\pgfqpoint{5.488990in}{1.222955in}}%
\pgfpathlineto{\pgfqpoint{5.480236in}{1.222955in}}%
\pgfpathlineto{\pgfqpoint{5.480236in}{1.609196in}}%
\pgfpathclose%
\pgfusepath{fill}%
\end{pgfscope}%
\begin{pgfscope}%
\pgfpathrectangle{\pgfqpoint{3.776708in}{0.600000in}}{\pgfqpoint{2.573292in}{2.070576in}}%
\pgfusepath{clip}%
\pgfsetbuttcap%
\pgfsetmiterjoin%
\definecolor{currentfill}{rgb}{0.133298,0.375282,0.379395}%
\pgfsetfillcolor{currentfill}%
\pgfsetlinewidth{0.000000pt}%
\definecolor{currentstroke}{rgb}{0.000000,0.000000,0.000000}%
\pgfsetstrokecolor{currentstroke}%
\pgfsetstrokeopacity{0.000000}%
\pgfsetdash{}{0pt}%
\pgfpathmoveto{\pgfqpoint{5.491178in}{1.609196in}}%
\pgfpathlineto{\pgfqpoint{5.499932in}{1.609196in}}%
\pgfpathlineto{\pgfqpoint{5.499932in}{1.257900in}}%
\pgfpathlineto{\pgfqpoint{5.491178in}{1.257900in}}%
\pgfpathlineto{\pgfqpoint{5.491178in}{1.609196in}}%
\pgfpathclose%
\pgfusepath{fill}%
\end{pgfscope}%
\begin{pgfscope}%
\pgfpathrectangle{\pgfqpoint{3.776708in}{0.600000in}}{\pgfqpoint{2.573292in}{2.070576in}}%
\pgfusepath{clip}%
\pgfsetbuttcap%
\pgfsetmiterjoin%
\definecolor{currentfill}{rgb}{0.133298,0.375282,0.379395}%
\pgfsetfillcolor{currentfill}%
\pgfsetlinewidth{0.000000pt}%
\definecolor{currentstroke}{rgb}{0.000000,0.000000,0.000000}%
\pgfsetstrokecolor{currentstroke}%
\pgfsetstrokeopacity{0.000000}%
\pgfsetdash{}{0pt}%
\pgfpathmoveto{\pgfqpoint{5.502120in}{1.609196in}}%
\pgfpathlineto{\pgfqpoint{5.510873in}{1.609196in}}%
\pgfpathlineto{\pgfqpoint{5.510873in}{1.291853in}}%
\pgfpathlineto{\pgfqpoint{5.502120in}{1.291853in}}%
\pgfpathlineto{\pgfqpoint{5.502120in}{1.609196in}}%
\pgfpathclose%
\pgfusepath{fill}%
\end{pgfscope}%
\begin{pgfscope}%
\pgfpathrectangle{\pgfqpoint{3.776708in}{0.600000in}}{\pgfqpoint{2.573292in}{2.070576in}}%
\pgfusepath{clip}%
\pgfsetbuttcap%
\pgfsetmiterjoin%
\definecolor{currentfill}{rgb}{0.133298,0.375282,0.379395}%
\pgfsetfillcolor{currentfill}%
\pgfsetlinewidth{0.000000pt}%
\definecolor{currentstroke}{rgb}{0.000000,0.000000,0.000000}%
\pgfsetstrokecolor{currentstroke}%
\pgfsetstrokeopacity{0.000000}%
\pgfsetdash{}{0pt}%
\pgfpathmoveto{\pgfqpoint{5.513062in}{1.609196in}}%
\pgfpathlineto{\pgfqpoint{5.521815in}{1.609196in}}%
\pgfpathlineto{\pgfqpoint{5.521815in}{1.325348in}}%
\pgfpathlineto{\pgfqpoint{5.513062in}{1.325348in}}%
\pgfpathlineto{\pgfqpoint{5.513062in}{1.609196in}}%
\pgfpathclose%
\pgfusepath{fill}%
\end{pgfscope}%
\begin{pgfscope}%
\pgfpathrectangle{\pgfqpoint{3.776708in}{0.600000in}}{\pgfqpoint{2.573292in}{2.070576in}}%
\pgfusepath{clip}%
\pgfsetbuttcap%
\pgfsetmiterjoin%
\definecolor{currentfill}{rgb}{0.133298,0.375282,0.379395}%
\pgfsetfillcolor{currentfill}%
\pgfsetlinewidth{0.000000pt}%
\definecolor{currentstroke}{rgb}{0.000000,0.000000,0.000000}%
\pgfsetstrokecolor{currentstroke}%
\pgfsetstrokeopacity{0.000000}%
\pgfsetdash{}{0pt}%
\pgfpathmoveto{\pgfqpoint{5.524004in}{1.609196in}}%
\pgfpathlineto{\pgfqpoint{5.532757in}{1.609196in}}%
\pgfpathlineto{\pgfqpoint{5.532757in}{1.358365in}}%
\pgfpathlineto{\pgfqpoint{5.524004in}{1.358365in}}%
\pgfpathlineto{\pgfqpoint{5.524004in}{1.609196in}}%
\pgfpathclose%
\pgfusepath{fill}%
\end{pgfscope}%
\begin{pgfscope}%
\pgfpathrectangle{\pgfqpoint{3.776708in}{0.600000in}}{\pgfqpoint{2.573292in}{2.070576in}}%
\pgfusepath{clip}%
\pgfsetbuttcap%
\pgfsetmiterjoin%
\definecolor{currentfill}{rgb}{0.133298,0.375282,0.379395}%
\pgfsetfillcolor{currentfill}%
\pgfsetlinewidth{0.000000pt}%
\definecolor{currentstroke}{rgb}{0.000000,0.000000,0.000000}%
\pgfsetstrokecolor{currentstroke}%
\pgfsetstrokeopacity{0.000000}%
\pgfsetdash{}{0pt}%
\pgfpathmoveto{\pgfqpoint{5.534945in}{1.609196in}}%
\pgfpathlineto{\pgfqpoint{5.543699in}{1.609196in}}%
\pgfpathlineto{\pgfqpoint{5.543699in}{1.385576in}}%
\pgfpathlineto{\pgfqpoint{5.534945in}{1.385576in}}%
\pgfpathlineto{\pgfqpoint{5.534945in}{1.609196in}}%
\pgfpathclose%
\pgfusepath{fill}%
\end{pgfscope}%
\begin{pgfscope}%
\pgfpathrectangle{\pgfqpoint{3.776708in}{0.600000in}}{\pgfqpoint{2.573292in}{2.070576in}}%
\pgfusepath{clip}%
\pgfsetbuttcap%
\pgfsetmiterjoin%
\definecolor{currentfill}{rgb}{0.133298,0.375282,0.379395}%
\pgfsetfillcolor{currentfill}%
\pgfsetlinewidth{0.000000pt}%
\definecolor{currentstroke}{rgb}{0.000000,0.000000,0.000000}%
\pgfsetstrokecolor{currentstroke}%
\pgfsetstrokeopacity{0.000000}%
\pgfsetdash{}{0pt}%
\pgfpathmoveto{\pgfqpoint{5.545887in}{1.609196in}}%
\pgfpathlineto{\pgfqpoint{5.554641in}{1.609196in}}%
\pgfpathlineto{\pgfqpoint{5.554641in}{1.410323in}}%
\pgfpathlineto{\pgfqpoint{5.545887in}{1.410323in}}%
\pgfpathlineto{\pgfqpoint{5.545887in}{1.609196in}}%
\pgfpathclose%
\pgfusepath{fill}%
\end{pgfscope}%
\begin{pgfscope}%
\pgfpathrectangle{\pgfqpoint{3.776708in}{0.600000in}}{\pgfqpoint{2.573292in}{2.070576in}}%
\pgfusepath{clip}%
\pgfsetbuttcap%
\pgfsetmiterjoin%
\definecolor{currentfill}{rgb}{0.133298,0.375282,0.379395}%
\pgfsetfillcolor{currentfill}%
\pgfsetlinewidth{0.000000pt}%
\definecolor{currentstroke}{rgb}{0.000000,0.000000,0.000000}%
\pgfsetstrokecolor{currentstroke}%
\pgfsetstrokeopacity{0.000000}%
\pgfsetdash{}{0pt}%
\pgfpathmoveto{\pgfqpoint{5.556829in}{1.609196in}}%
\pgfpathlineto{\pgfqpoint{5.565582in}{1.609196in}}%
\pgfpathlineto{\pgfqpoint{5.565582in}{1.427853in}}%
\pgfpathlineto{\pgfqpoint{5.556829in}{1.427853in}}%
\pgfpathlineto{\pgfqpoint{5.556829in}{1.609196in}}%
\pgfpathclose%
\pgfusepath{fill}%
\end{pgfscope}%
\begin{pgfscope}%
\pgfpathrectangle{\pgfqpoint{3.776708in}{0.600000in}}{\pgfqpoint{2.573292in}{2.070576in}}%
\pgfusepath{clip}%
\pgfsetbuttcap%
\pgfsetmiterjoin%
\definecolor{currentfill}{rgb}{0.133298,0.375282,0.379395}%
\pgfsetfillcolor{currentfill}%
\pgfsetlinewidth{0.000000pt}%
\definecolor{currentstroke}{rgb}{0.000000,0.000000,0.000000}%
\pgfsetstrokecolor{currentstroke}%
\pgfsetstrokeopacity{0.000000}%
\pgfsetdash{}{0pt}%
\pgfpathmoveto{\pgfqpoint{5.567771in}{1.609196in}}%
\pgfpathlineto{\pgfqpoint{5.576524in}{1.609196in}}%
\pgfpathlineto{\pgfqpoint{5.576524in}{1.445159in}}%
\pgfpathlineto{\pgfqpoint{5.567771in}{1.445159in}}%
\pgfpathlineto{\pgfqpoint{5.567771in}{1.609196in}}%
\pgfpathclose%
\pgfusepath{fill}%
\end{pgfscope}%
\begin{pgfscope}%
\pgfpathrectangle{\pgfqpoint{3.776708in}{0.600000in}}{\pgfqpoint{2.573292in}{2.070576in}}%
\pgfusepath{clip}%
\pgfsetbuttcap%
\pgfsetmiterjoin%
\definecolor{currentfill}{rgb}{0.133298,0.375282,0.379395}%
\pgfsetfillcolor{currentfill}%
\pgfsetlinewidth{0.000000pt}%
\definecolor{currentstroke}{rgb}{0.000000,0.000000,0.000000}%
\pgfsetstrokecolor{currentstroke}%
\pgfsetstrokeopacity{0.000000}%
\pgfsetdash{}{0pt}%
\pgfpathmoveto{\pgfqpoint{5.578713in}{1.609196in}}%
\pgfpathlineto{\pgfqpoint{5.587466in}{1.609196in}}%
\pgfpathlineto{\pgfqpoint{5.587466in}{1.459246in}}%
\pgfpathlineto{\pgfqpoint{5.578713in}{1.459246in}}%
\pgfpathlineto{\pgfqpoint{5.578713in}{1.609196in}}%
\pgfpathclose%
\pgfusepath{fill}%
\end{pgfscope}%
\begin{pgfscope}%
\pgfpathrectangle{\pgfqpoint{3.776708in}{0.600000in}}{\pgfqpoint{2.573292in}{2.070576in}}%
\pgfusepath{clip}%
\pgfsetbuttcap%
\pgfsetmiterjoin%
\definecolor{currentfill}{rgb}{0.133298,0.375282,0.379395}%
\pgfsetfillcolor{currentfill}%
\pgfsetlinewidth{0.000000pt}%
\definecolor{currentstroke}{rgb}{0.000000,0.000000,0.000000}%
\pgfsetstrokecolor{currentstroke}%
\pgfsetstrokeopacity{0.000000}%
\pgfsetdash{}{0pt}%
\pgfpathmoveto{\pgfqpoint{5.589654in}{1.609196in}}%
\pgfpathlineto{\pgfqpoint{5.598408in}{1.609196in}}%
\pgfpathlineto{\pgfqpoint{5.598408in}{1.466890in}}%
\pgfpathlineto{\pgfqpoint{5.589654in}{1.466890in}}%
\pgfpathlineto{\pgfqpoint{5.589654in}{1.609196in}}%
\pgfpathclose%
\pgfusepath{fill}%
\end{pgfscope}%
\begin{pgfscope}%
\pgfpathrectangle{\pgfqpoint{3.776708in}{0.600000in}}{\pgfqpoint{2.573292in}{2.070576in}}%
\pgfusepath{clip}%
\pgfsetbuttcap%
\pgfsetmiterjoin%
\definecolor{currentfill}{rgb}{0.133298,0.375282,0.379395}%
\pgfsetfillcolor{currentfill}%
\pgfsetlinewidth{0.000000pt}%
\definecolor{currentstroke}{rgb}{0.000000,0.000000,0.000000}%
\pgfsetstrokecolor{currentstroke}%
\pgfsetstrokeopacity{0.000000}%
\pgfsetdash{}{0pt}%
\pgfpathmoveto{\pgfqpoint{5.600596in}{1.609196in}}%
\pgfpathlineto{\pgfqpoint{5.609350in}{1.609196in}}%
\pgfpathlineto{\pgfqpoint{5.609350in}{1.474494in}}%
\pgfpathlineto{\pgfqpoint{5.600596in}{1.474494in}}%
\pgfpathlineto{\pgfqpoint{5.600596in}{1.609196in}}%
\pgfpathclose%
\pgfusepath{fill}%
\end{pgfscope}%
\begin{pgfscope}%
\pgfpathrectangle{\pgfqpoint{3.776708in}{0.600000in}}{\pgfqpoint{2.573292in}{2.070576in}}%
\pgfusepath{clip}%
\pgfsetbuttcap%
\pgfsetmiterjoin%
\definecolor{currentfill}{rgb}{0.133298,0.375282,0.379395}%
\pgfsetfillcolor{currentfill}%
\pgfsetlinewidth{0.000000pt}%
\definecolor{currentstroke}{rgb}{0.000000,0.000000,0.000000}%
\pgfsetstrokecolor{currentstroke}%
\pgfsetstrokeopacity{0.000000}%
\pgfsetdash{}{0pt}%
\pgfpathmoveto{\pgfqpoint{5.611538in}{1.609196in}}%
\pgfpathlineto{\pgfqpoint{5.620291in}{1.609196in}}%
\pgfpathlineto{\pgfqpoint{5.620291in}{1.479360in}}%
\pgfpathlineto{\pgfqpoint{5.611538in}{1.479360in}}%
\pgfpathlineto{\pgfqpoint{5.611538in}{1.609196in}}%
\pgfpathclose%
\pgfusepath{fill}%
\end{pgfscope}%
\begin{pgfscope}%
\pgfpathrectangle{\pgfqpoint{3.776708in}{0.600000in}}{\pgfqpoint{2.573292in}{2.070576in}}%
\pgfusepath{clip}%
\pgfsetbuttcap%
\pgfsetmiterjoin%
\definecolor{currentfill}{rgb}{0.133298,0.375282,0.379395}%
\pgfsetfillcolor{currentfill}%
\pgfsetlinewidth{0.000000pt}%
\definecolor{currentstroke}{rgb}{0.000000,0.000000,0.000000}%
\pgfsetstrokecolor{currentstroke}%
\pgfsetstrokeopacity{0.000000}%
\pgfsetdash{}{0pt}%
\pgfpathmoveto{\pgfqpoint{5.622480in}{1.609196in}}%
\pgfpathlineto{\pgfqpoint{5.631233in}{1.609196in}}%
\pgfpathlineto{\pgfqpoint{5.631233in}{1.480076in}}%
\pgfpathlineto{\pgfqpoint{5.622480in}{1.480076in}}%
\pgfpathlineto{\pgfqpoint{5.622480in}{1.609196in}}%
\pgfpathclose%
\pgfusepath{fill}%
\end{pgfscope}%
\begin{pgfscope}%
\pgfpathrectangle{\pgfqpoint{3.776708in}{0.600000in}}{\pgfqpoint{2.573292in}{2.070576in}}%
\pgfusepath{clip}%
\pgfsetbuttcap%
\pgfsetmiterjoin%
\definecolor{currentfill}{rgb}{0.133298,0.375282,0.379395}%
\pgfsetfillcolor{currentfill}%
\pgfsetlinewidth{0.000000pt}%
\definecolor{currentstroke}{rgb}{0.000000,0.000000,0.000000}%
\pgfsetstrokecolor{currentstroke}%
\pgfsetstrokeopacity{0.000000}%
\pgfsetdash{}{0pt}%
\pgfpathmoveto{\pgfqpoint{5.633422in}{1.609196in}}%
\pgfpathlineto{\pgfqpoint{5.642175in}{1.609196in}}%
\pgfpathlineto{\pgfqpoint{5.642175in}{1.480487in}}%
\pgfpathlineto{\pgfqpoint{5.633422in}{1.480487in}}%
\pgfpathlineto{\pgfqpoint{5.633422in}{1.609196in}}%
\pgfpathclose%
\pgfusepath{fill}%
\end{pgfscope}%
\begin{pgfscope}%
\pgfpathrectangle{\pgfqpoint{3.776708in}{0.600000in}}{\pgfqpoint{2.573292in}{2.070576in}}%
\pgfusepath{clip}%
\pgfsetbuttcap%
\pgfsetmiterjoin%
\definecolor{currentfill}{rgb}{0.133298,0.375282,0.379395}%
\pgfsetfillcolor{currentfill}%
\pgfsetlinewidth{0.000000pt}%
\definecolor{currentstroke}{rgb}{0.000000,0.000000,0.000000}%
\pgfsetstrokecolor{currentstroke}%
\pgfsetstrokeopacity{0.000000}%
\pgfsetdash{}{0pt}%
\pgfpathmoveto{\pgfqpoint{5.644363in}{1.609196in}}%
\pgfpathlineto{\pgfqpoint{5.653117in}{1.609196in}}%
\pgfpathlineto{\pgfqpoint{5.653117in}{1.480219in}}%
\pgfpathlineto{\pgfqpoint{5.644363in}{1.480219in}}%
\pgfpathlineto{\pgfqpoint{5.644363in}{1.609196in}}%
\pgfpathclose%
\pgfusepath{fill}%
\end{pgfscope}%
\begin{pgfscope}%
\pgfpathrectangle{\pgfqpoint{3.776708in}{0.600000in}}{\pgfqpoint{2.573292in}{2.070576in}}%
\pgfusepath{clip}%
\pgfsetbuttcap%
\pgfsetmiterjoin%
\definecolor{currentfill}{rgb}{0.133298,0.375282,0.379395}%
\pgfsetfillcolor{currentfill}%
\pgfsetlinewidth{0.000000pt}%
\definecolor{currentstroke}{rgb}{0.000000,0.000000,0.000000}%
\pgfsetstrokecolor{currentstroke}%
\pgfsetstrokeopacity{0.000000}%
\pgfsetdash{}{0pt}%
\pgfpathmoveto{\pgfqpoint{5.655305in}{1.609196in}}%
\pgfpathlineto{\pgfqpoint{5.664059in}{1.609196in}}%
\pgfpathlineto{\pgfqpoint{5.664059in}{1.476681in}}%
\pgfpathlineto{\pgfqpoint{5.655305in}{1.476681in}}%
\pgfpathlineto{\pgfqpoint{5.655305in}{1.609196in}}%
\pgfpathclose%
\pgfusepath{fill}%
\end{pgfscope}%
\begin{pgfscope}%
\pgfpathrectangle{\pgfqpoint{3.776708in}{0.600000in}}{\pgfqpoint{2.573292in}{2.070576in}}%
\pgfusepath{clip}%
\pgfsetbuttcap%
\pgfsetmiterjoin%
\definecolor{currentfill}{rgb}{0.133298,0.375282,0.379395}%
\pgfsetfillcolor{currentfill}%
\pgfsetlinewidth{0.000000pt}%
\definecolor{currentstroke}{rgb}{0.000000,0.000000,0.000000}%
\pgfsetstrokecolor{currentstroke}%
\pgfsetstrokeopacity{0.000000}%
\pgfsetdash{}{0pt}%
\pgfpathmoveto{\pgfqpoint{5.666247in}{1.609196in}}%
\pgfpathlineto{\pgfqpoint{5.675000in}{1.609196in}}%
\pgfpathlineto{\pgfqpoint{5.675000in}{1.483344in}}%
\pgfpathlineto{\pgfqpoint{5.666247in}{1.483344in}}%
\pgfpathlineto{\pgfqpoint{5.666247in}{1.609196in}}%
\pgfpathclose%
\pgfusepath{fill}%
\end{pgfscope}%
\begin{pgfscope}%
\pgfpathrectangle{\pgfqpoint{3.776708in}{0.600000in}}{\pgfqpoint{2.573292in}{2.070576in}}%
\pgfusepath{clip}%
\pgfsetbuttcap%
\pgfsetmiterjoin%
\definecolor{currentfill}{rgb}{0.133298,0.375282,0.379395}%
\pgfsetfillcolor{currentfill}%
\pgfsetlinewidth{0.000000pt}%
\definecolor{currentstroke}{rgb}{0.000000,0.000000,0.000000}%
\pgfsetstrokecolor{currentstroke}%
\pgfsetstrokeopacity{0.000000}%
\pgfsetdash{}{0pt}%
\pgfpathmoveto{\pgfqpoint{5.677189in}{1.609196in}}%
\pgfpathlineto{\pgfqpoint{5.685942in}{1.609196in}}%
\pgfpathlineto{\pgfqpoint{5.685942in}{1.484567in}}%
\pgfpathlineto{\pgfqpoint{5.677189in}{1.484567in}}%
\pgfpathlineto{\pgfqpoint{5.677189in}{1.609196in}}%
\pgfpathclose%
\pgfusepath{fill}%
\end{pgfscope}%
\begin{pgfscope}%
\pgfpathrectangle{\pgfqpoint{3.776708in}{0.600000in}}{\pgfqpoint{2.573292in}{2.070576in}}%
\pgfusepath{clip}%
\pgfsetbuttcap%
\pgfsetmiterjoin%
\definecolor{currentfill}{rgb}{0.133298,0.375282,0.379395}%
\pgfsetfillcolor{currentfill}%
\pgfsetlinewidth{0.000000pt}%
\definecolor{currentstroke}{rgb}{0.000000,0.000000,0.000000}%
\pgfsetstrokecolor{currentstroke}%
\pgfsetstrokeopacity{0.000000}%
\pgfsetdash{}{0pt}%
\pgfpathmoveto{\pgfqpoint{5.688131in}{1.609196in}}%
\pgfpathlineto{\pgfqpoint{5.696884in}{1.609196in}}%
\pgfpathlineto{\pgfqpoint{5.696884in}{1.486779in}}%
\pgfpathlineto{\pgfqpoint{5.688131in}{1.486779in}}%
\pgfpathlineto{\pgfqpoint{5.688131in}{1.609196in}}%
\pgfpathclose%
\pgfusepath{fill}%
\end{pgfscope}%
\begin{pgfscope}%
\pgfpathrectangle{\pgfqpoint{3.776708in}{0.600000in}}{\pgfqpoint{2.573292in}{2.070576in}}%
\pgfusepath{clip}%
\pgfsetbuttcap%
\pgfsetmiterjoin%
\definecolor{currentfill}{rgb}{0.133298,0.375282,0.379395}%
\pgfsetfillcolor{currentfill}%
\pgfsetlinewidth{0.000000pt}%
\definecolor{currentstroke}{rgb}{0.000000,0.000000,0.000000}%
\pgfsetstrokecolor{currentstroke}%
\pgfsetstrokeopacity{0.000000}%
\pgfsetdash{}{0pt}%
\pgfpathmoveto{\pgfqpoint{5.699072in}{1.609196in}}%
\pgfpathlineto{\pgfqpoint{5.707826in}{1.609196in}}%
\pgfpathlineto{\pgfqpoint{5.707826in}{1.491434in}}%
\pgfpathlineto{\pgfqpoint{5.699072in}{1.491434in}}%
\pgfpathlineto{\pgfqpoint{5.699072in}{1.609196in}}%
\pgfpathclose%
\pgfusepath{fill}%
\end{pgfscope}%
\begin{pgfscope}%
\pgfpathrectangle{\pgfqpoint{3.776708in}{0.600000in}}{\pgfqpoint{2.573292in}{2.070576in}}%
\pgfusepath{clip}%
\pgfsetbuttcap%
\pgfsetmiterjoin%
\definecolor{currentfill}{rgb}{0.133298,0.375282,0.379395}%
\pgfsetfillcolor{currentfill}%
\pgfsetlinewidth{0.000000pt}%
\definecolor{currentstroke}{rgb}{0.000000,0.000000,0.000000}%
\pgfsetstrokecolor{currentstroke}%
\pgfsetstrokeopacity{0.000000}%
\pgfsetdash{}{0pt}%
\pgfpathmoveto{\pgfqpoint{5.710014in}{1.609196in}}%
\pgfpathlineto{\pgfqpoint{5.718768in}{1.609196in}}%
\pgfpathlineto{\pgfqpoint{5.718768in}{1.498041in}}%
\pgfpathlineto{\pgfqpoint{5.710014in}{1.498041in}}%
\pgfpathlineto{\pgfqpoint{5.710014in}{1.609196in}}%
\pgfpathclose%
\pgfusepath{fill}%
\end{pgfscope}%
\begin{pgfscope}%
\pgfpathrectangle{\pgfqpoint{3.776708in}{0.600000in}}{\pgfqpoint{2.573292in}{2.070576in}}%
\pgfusepath{clip}%
\pgfsetbuttcap%
\pgfsetmiterjoin%
\definecolor{currentfill}{rgb}{0.133298,0.375282,0.379395}%
\pgfsetfillcolor{currentfill}%
\pgfsetlinewidth{0.000000pt}%
\definecolor{currentstroke}{rgb}{0.000000,0.000000,0.000000}%
\pgfsetstrokecolor{currentstroke}%
\pgfsetstrokeopacity{0.000000}%
\pgfsetdash{}{0pt}%
\pgfpathmoveto{\pgfqpoint{5.720956in}{1.609196in}}%
\pgfpathlineto{\pgfqpoint{5.729709in}{1.609196in}}%
\pgfpathlineto{\pgfqpoint{5.729709in}{1.510011in}}%
\pgfpathlineto{\pgfqpoint{5.720956in}{1.510011in}}%
\pgfpathlineto{\pgfqpoint{5.720956in}{1.609196in}}%
\pgfpathclose%
\pgfusepath{fill}%
\end{pgfscope}%
\begin{pgfscope}%
\pgfpathrectangle{\pgfqpoint{3.776708in}{0.600000in}}{\pgfqpoint{2.573292in}{2.070576in}}%
\pgfusepath{clip}%
\pgfsetbuttcap%
\pgfsetmiterjoin%
\definecolor{currentfill}{rgb}{0.133298,0.375282,0.379395}%
\pgfsetfillcolor{currentfill}%
\pgfsetlinewidth{0.000000pt}%
\definecolor{currentstroke}{rgb}{0.000000,0.000000,0.000000}%
\pgfsetstrokecolor{currentstroke}%
\pgfsetstrokeopacity{0.000000}%
\pgfsetdash{}{0pt}%
\pgfpathmoveto{\pgfqpoint{5.731898in}{1.609196in}}%
\pgfpathlineto{\pgfqpoint{5.740651in}{1.609196in}}%
\pgfpathlineto{\pgfqpoint{5.740651in}{1.532156in}}%
\pgfpathlineto{\pgfqpoint{5.731898in}{1.532156in}}%
\pgfpathlineto{\pgfqpoint{5.731898in}{1.609196in}}%
\pgfpathclose%
\pgfusepath{fill}%
\end{pgfscope}%
\begin{pgfscope}%
\pgfpathrectangle{\pgfqpoint{3.776708in}{0.600000in}}{\pgfqpoint{2.573292in}{2.070576in}}%
\pgfusepath{clip}%
\pgfsetbuttcap%
\pgfsetmiterjoin%
\definecolor{currentfill}{rgb}{0.133298,0.375282,0.379395}%
\pgfsetfillcolor{currentfill}%
\pgfsetlinewidth{0.000000pt}%
\definecolor{currentstroke}{rgb}{0.000000,0.000000,0.000000}%
\pgfsetstrokecolor{currentstroke}%
\pgfsetstrokeopacity{0.000000}%
\pgfsetdash{}{0pt}%
\pgfpathmoveto{\pgfqpoint{5.742840in}{1.603004in}}%
\pgfpathlineto{\pgfqpoint{5.751593in}{1.603004in}}%
\pgfpathlineto{\pgfqpoint{5.751593in}{1.545400in}}%
\pgfpathlineto{\pgfqpoint{5.742840in}{1.545400in}}%
\pgfpathlineto{\pgfqpoint{5.742840in}{1.603004in}}%
\pgfpathclose%
\pgfusepath{fill}%
\end{pgfscope}%
\begin{pgfscope}%
\pgfpathrectangle{\pgfqpoint{3.776708in}{0.600000in}}{\pgfqpoint{2.573292in}{2.070576in}}%
\pgfusepath{clip}%
\pgfsetbuttcap%
\pgfsetmiterjoin%
\definecolor{currentfill}{rgb}{0.133298,0.375282,0.379395}%
\pgfsetfillcolor{currentfill}%
\pgfsetlinewidth{0.000000pt}%
\definecolor{currentstroke}{rgb}{0.000000,0.000000,0.000000}%
\pgfsetstrokecolor{currentstroke}%
\pgfsetstrokeopacity{0.000000}%
\pgfsetdash{}{0pt}%
\pgfpathmoveto{\pgfqpoint{5.753781in}{1.594748in}}%
\pgfpathlineto{\pgfqpoint{5.762535in}{1.594748in}}%
\pgfpathlineto{\pgfqpoint{5.762535in}{1.556350in}}%
\pgfpathlineto{\pgfqpoint{5.753781in}{1.556350in}}%
\pgfpathlineto{\pgfqpoint{5.753781in}{1.594748in}}%
\pgfpathclose%
\pgfusepath{fill}%
\end{pgfscope}%
\begin{pgfscope}%
\pgfpathrectangle{\pgfqpoint{3.776708in}{0.600000in}}{\pgfqpoint{2.573292in}{2.070576in}}%
\pgfusepath{clip}%
\pgfsetbuttcap%
\pgfsetmiterjoin%
\definecolor{currentfill}{rgb}{0.133298,0.375282,0.379395}%
\pgfsetfillcolor{currentfill}%
\pgfsetlinewidth{0.000000pt}%
\definecolor{currentstroke}{rgb}{0.000000,0.000000,0.000000}%
\pgfsetstrokecolor{currentstroke}%
\pgfsetstrokeopacity{0.000000}%
\pgfsetdash{}{0pt}%
\pgfpathmoveto{\pgfqpoint{5.764723in}{1.583784in}}%
\pgfpathlineto{\pgfqpoint{5.773477in}{1.583784in}}%
\pgfpathlineto{\pgfqpoint{5.773477in}{1.569202in}}%
\pgfpathlineto{\pgfqpoint{5.764723in}{1.569202in}}%
\pgfpathlineto{\pgfqpoint{5.764723in}{1.583784in}}%
\pgfpathclose%
\pgfusepath{fill}%
\end{pgfscope}%
\begin{pgfscope}%
\pgfpathrectangle{\pgfqpoint{3.776708in}{0.600000in}}{\pgfqpoint{2.573292in}{2.070576in}}%
\pgfusepath{clip}%
\pgfsetbuttcap%
\pgfsetmiterjoin%
\definecolor{currentfill}{rgb}{0.133298,0.375282,0.379395}%
\pgfsetfillcolor{currentfill}%
\pgfsetlinewidth{0.000000pt}%
\definecolor{currentstroke}{rgb}{0.000000,0.000000,0.000000}%
\pgfsetstrokecolor{currentstroke}%
\pgfsetstrokeopacity{0.000000}%
\pgfsetdash{}{0pt}%
\pgfpathmoveto{\pgfqpoint{5.775665in}{1.609196in}}%
\pgfpathlineto{\pgfqpoint{5.784418in}{1.609196in}}%
\pgfpathlineto{\pgfqpoint{5.784418in}{1.625566in}}%
\pgfpathlineto{\pgfqpoint{5.775665in}{1.625566in}}%
\pgfpathlineto{\pgfqpoint{5.775665in}{1.609196in}}%
\pgfpathclose%
\pgfusepath{fill}%
\end{pgfscope}%
\begin{pgfscope}%
\pgfpathrectangle{\pgfqpoint{3.776708in}{0.600000in}}{\pgfqpoint{2.573292in}{2.070576in}}%
\pgfusepath{clip}%
\pgfsetbuttcap%
\pgfsetmiterjoin%
\definecolor{currentfill}{rgb}{0.133298,0.375282,0.379395}%
\pgfsetfillcolor{currentfill}%
\pgfsetlinewidth{0.000000pt}%
\definecolor{currentstroke}{rgb}{0.000000,0.000000,0.000000}%
\pgfsetstrokecolor{currentstroke}%
\pgfsetstrokeopacity{0.000000}%
\pgfsetdash{}{0pt}%
\pgfpathmoveto{\pgfqpoint{5.786607in}{1.609196in}}%
\pgfpathlineto{\pgfqpoint{5.795360in}{1.609196in}}%
\pgfpathlineto{\pgfqpoint{5.795360in}{1.658450in}}%
\pgfpathlineto{\pgfqpoint{5.786607in}{1.658450in}}%
\pgfpathlineto{\pgfqpoint{5.786607in}{1.609196in}}%
\pgfpathclose%
\pgfusepath{fill}%
\end{pgfscope}%
\begin{pgfscope}%
\pgfpathrectangle{\pgfqpoint{3.776708in}{0.600000in}}{\pgfqpoint{2.573292in}{2.070576in}}%
\pgfusepath{clip}%
\pgfsetbuttcap%
\pgfsetmiterjoin%
\definecolor{currentfill}{rgb}{0.133298,0.375282,0.379395}%
\pgfsetfillcolor{currentfill}%
\pgfsetlinewidth{0.000000pt}%
\definecolor{currentstroke}{rgb}{0.000000,0.000000,0.000000}%
\pgfsetstrokecolor{currentstroke}%
\pgfsetstrokeopacity{0.000000}%
\pgfsetdash{}{0pt}%
\pgfpathmoveto{\pgfqpoint{5.797549in}{1.609196in}}%
\pgfpathlineto{\pgfqpoint{5.806302in}{1.609196in}}%
\pgfpathlineto{\pgfqpoint{5.806302in}{1.691844in}}%
\pgfpathlineto{\pgfqpoint{5.797549in}{1.691844in}}%
\pgfpathlineto{\pgfqpoint{5.797549in}{1.609196in}}%
\pgfpathclose%
\pgfusepath{fill}%
\end{pgfscope}%
\begin{pgfscope}%
\pgfpathrectangle{\pgfqpoint{3.776708in}{0.600000in}}{\pgfqpoint{2.573292in}{2.070576in}}%
\pgfusepath{clip}%
\pgfsetbuttcap%
\pgfsetmiterjoin%
\definecolor{currentfill}{rgb}{0.133298,0.375282,0.379395}%
\pgfsetfillcolor{currentfill}%
\pgfsetlinewidth{0.000000pt}%
\definecolor{currentstroke}{rgb}{0.000000,0.000000,0.000000}%
\pgfsetstrokecolor{currentstroke}%
\pgfsetstrokeopacity{0.000000}%
\pgfsetdash{}{0pt}%
\pgfpathmoveto{\pgfqpoint{5.808490in}{1.609196in}}%
\pgfpathlineto{\pgfqpoint{5.817244in}{1.609196in}}%
\pgfpathlineto{\pgfqpoint{5.817244in}{1.727075in}}%
\pgfpathlineto{\pgfqpoint{5.808490in}{1.727075in}}%
\pgfpathlineto{\pgfqpoint{5.808490in}{1.609196in}}%
\pgfpathclose%
\pgfusepath{fill}%
\end{pgfscope}%
\begin{pgfscope}%
\pgfpathrectangle{\pgfqpoint{3.776708in}{0.600000in}}{\pgfqpoint{2.573292in}{2.070576in}}%
\pgfusepath{clip}%
\pgfsetbuttcap%
\pgfsetmiterjoin%
\definecolor{currentfill}{rgb}{0.133298,0.375282,0.379395}%
\pgfsetfillcolor{currentfill}%
\pgfsetlinewidth{0.000000pt}%
\definecolor{currentstroke}{rgb}{0.000000,0.000000,0.000000}%
\pgfsetstrokecolor{currentstroke}%
\pgfsetstrokeopacity{0.000000}%
\pgfsetdash{}{0pt}%
\pgfpathmoveto{\pgfqpoint{5.819432in}{1.609196in}}%
\pgfpathlineto{\pgfqpoint{5.828186in}{1.609196in}}%
\pgfpathlineto{\pgfqpoint{5.828186in}{1.756177in}}%
\pgfpathlineto{\pgfqpoint{5.819432in}{1.756177in}}%
\pgfpathlineto{\pgfqpoint{5.819432in}{1.609196in}}%
\pgfpathclose%
\pgfusepath{fill}%
\end{pgfscope}%
\begin{pgfscope}%
\pgfpathrectangle{\pgfqpoint{3.776708in}{0.600000in}}{\pgfqpoint{2.573292in}{2.070576in}}%
\pgfusepath{clip}%
\pgfsetbuttcap%
\pgfsetmiterjoin%
\definecolor{currentfill}{rgb}{0.133298,0.375282,0.379395}%
\pgfsetfillcolor{currentfill}%
\pgfsetlinewidth{0.000000pt}%
\definecolor{currentstroke}{rgb}{0.000000,0.000000,0.000000}%
\pgfsetstrokecolor{currentstroke}%
\pgfsetstrokeopacity{0.000000}%
\pgfsetdash{}{0pt}%
\pgfpathmoveto{\pgfqpoint{5.830374in}{1.609196in}}%
\pgfpathlineto{\pgfqpoint{5.839127in}{1.609196in}}%
\pgfpathlineto{\pgfqpoint{5.839127in}{1.787168in}}%
\pgfpathlineto{\pgfqpoint{5.830374in}{1.787168in}}%
\pgfpathlineto{\pgfqpoint{5.830374in}{1.609196in}}%
\pgfpathclose%
\pgfusepath{fill}%
\end{pgfscope}%
\begin{pgfscope}%
\pgfpathrectangle{\pgfqpoint{3.776708in}{0.600000in}}{\pgfqpoint{2.573292in}{2.070576in}}%
\pgfusepath{clip}%
\pgfsetbuttcap%
\pgfsetmiterjoin%
\definecolor{currentfill}{rgb}{0.133298,0.375282,0.379395}%
\pgfsetfillcolor{currentfill}%
\pgfsetlinewidth{0.000000pt}%
\definecolor{currentstroke}{rgb}{0.000000,0.000000,0.000000}%
\pgfsetstrokecolor{currentstroke}%
\pgfsetstrokeopacity{0.000000}%
\pgfsetdash{}{0pt}%
\pgfpathmoveto{\pgfqpoint{5.841316in}{1.609196in}}%
\pgfpathlineto{\pgfqpoint{5.850069in}{1.609196in}}%
\pgfpathlineto{\pgfqpoint{5.850069in}{1.815235in}}%
\pgfpathlineto{\pgfqpoint{5.841316in}{1.815235in}}%
\pgfpathlineto{\pgfqpoint{5.841316in}{1.609196in}}%
\pgfpathclose%
\pgfusepath{fill}%
\end{pgfscope}%
\begin{pgfscope}%
\pgfpathrectangle{\pgfqpoint{3.776708in}{0.600000in}}{\pgfqpoint{2.573292in}{2.070576in}}%
\pgfusepath{clip}%
\pgfsetbuttcap%
\pgfsetmiterjoin%
\definecolor{currentfill}{rgb}{0.133298,0.375282,0.379395}%
\pgfsetfillcolor{currentfill}%
\pgfsetlinewidth{0.000000pt}%
\definecolor{currentstroke}{rgb}{0.000000,0.000000,0.000000}%
\pgfsetstrokecolor{currentstroke}%
\pgfsetstrokeopacity{0.000000}%
\pgfsetdash{}{0pt}%
\pgfpathmoveto{\pgfqpoint{5.852258in}{1.609196in}}%
\pgfpathlineto{\pgfqpoint{5.861011in}{1.609196in}}%
\pgfpathlineto{\pgfqpoint{5.861011in}{1.841997in}}%
\pgfpathlineto{\pgfqpoint{5.852258in}{1.841997in}}%
\pgfpathlineto{\pgfqpoint{5.852258in}{1.609196in}}%
\pgfpathclose%
\pgfusepath{fill}%
\end{pgfscope}%
\begin{pgfscope}%
\pgfpathrectangle{\pgfqpoint{3.776708in}{0.600000in}}{\pgfqpoint{2.573292in}{2.070576in}}%
\pgfusepath{clip}%
\pgfsetbuttcap%
\pgfsetmiterjoin%
\definecolor{currentfill}{rgb}{0.133298,0.375282,0.379395}%
\pgfsetfillcolor{currentfill}%
\pgfsetlinewidth{0.000000pt}%
\definecolor{currentstroke}{rgb}{0.000000,0.000000,0.000000}%
\pgfsetstrokecolor{currentstroke}%
\pgfsetstrokeopacity{0.000000}%
\pgfsetdash{}{0pt}%
\pgfpathmoveto{\pgfqpoint{5.863199in}{1.609196in}}%
\pgfpathlineto{\pgfqpoint{5.871953in}{1.609196in}}%
\pgfpathlineto{\pgfqpoint{5.871953in}{1.863774in}}%
\pgfpathlineto{\pgfqpoint{5.863199in}{1.863774in}}%
\pgfpathlineto{\pgfqpoint{5.863199in}{1.609196in}}%
\pgfpathclose%
\pgfusepath{fill}%
\end{pgfscope}%
\begin{pgfscope}%
\pgfpathrectangle{\pgfqpoint{3.776708in}{0.600000in}}{\pgfqpoint{2.573292in}{2.070576in}}%
\pgfusepath{clip}%
\pgfsetbuttcap%
\pgfsetmiterjoin%
\definecolor{currentfill}{rgb}{0.133298,0.375282,0.379395}%
\pgfsetfillcolor{currentfill}%
\pgfsetlinewidth{0.000000pt}%
\definecolor{currentstroke}{rgb}{0.000000,0.000000,0.000000}%
\pgfsetstrokecolor{currentstroke}%
\pgfsetstrokeopacity{0.000000}%
\pgfsetdash{}{0pt}%
\pgfpathmoveto{\pgfqpoint{5.874141in}{1.609196in}}%
\pgfpathlineto{\pgfqpoint{5.882895in}{1.609196in}}%
\pgfpathlineto{\pgfqpoint{5.882895in}{1.877332in}}%
\pgfpathlineto{\pgfqpoint{5.874141in}{1.877332in}}%
\pgfpathlineto{\pgfqpoint{5.874141in}{1.609196in}}%
\pgfpathclose%
\pgfusepath{fill}%
\end{pgfscope}%
\begin{pgfscope}%
\pgfpathrectangle{\pgfqpoint{3.776708in}{0.600000in}}{\pgfqpoint{2.573292in}{2.070576in}}%
\pgfusepath{clip}%
\pgfsetbuttcap%
\pgfsetmiterjoin%
\definecolor{currentfill}{rgb}{0.133298,0.375282,0.379395}%
\pgfsetfillcolor{currentfill}%
\pgfsetlinewidth{0.000000pt}%
\definecolor{currentstroke}{rgb}{0.000000,0.000000,0.000000}%
\pgfsetstrokecolor{currentstroke}%
\pgfsetstrokeopacity{0.000000}%
\pgfsetdash{}{0pt}%
\pgfpathmoveto{\pgfqpoint{5.885083in}{1.609196in}}%
\pgfpathlineto{\pgfqpoint{5.893836in}{1.609196in}}%
\pgfpathlineto{\pgfqpoint{5.893836in}{1.896193in}}%
\pgfpathlineto{\pgfqpoint{5.885083in}{1.896193in}}%
\pgfpathlineto{\pgfqpoint{5.885083in}{1.609196in}}%
\pgfpathclose%
\pgfusepath{fill}%
\end{pgfscope}%
\begin{pgfscope}%
\pgfpathrectangle{\pgfqpoint{3.776708in}{0.600000in}}{\pgfqpoint{2.573292in}{2.070576in}}%
\pgfusepath{clip}%
\pgfsetbuttcap%
\pgfsetmiterjoin%
\definecolor{currentfill}{rgb}{0.133298,0.375282,0.379395}%
\pgfsetfillcolor{currentfill}%
\pgfsetlinewidth{0.000000pt}%
\definecolor{currentstroke}{rgb}{0.000000,0.000000,0.000000}%
\pgfsetstrokecolor{currentstroke}%
\pgfsetstrokeopacity{0.000000}%
\pgfsetdash{}{0pt}%
\pgfpathmoveto{\pgfqpoint{5.896025in}{1.609196in}}%
\pgfpathlineto{\pgfqpoint{5.904778in}{1.609196in}}%
\pgfpathlineto{\pgfqpoint{5.904778in}{1.907418in}}%
\pgfpathlineto{\pgfqpoint{5.896025in}{1.907418in}}%
\pgfpathlineto{\pgfqpoint{5.896025in}{1.609196in}}%
\pgfpathclose%
\pgfusepath{fill}%
\end{pgfscope}%
\begin{pgfscope}%
\pgfpathrectangle{\pgfqpoint{3.776708in}{0.600000in}}{\pgfqpoint{2.573292in}{2.070576in}}%
\pgfusepath{clip}%
\pgfsetbuttcap%
\pgfsetmiterjoin%
\definecolor{currentfill}{rgb}{0.133298,0.375282,0.379395}%
\pgfsetfillcolor{currentfill}%
\pgfsetlinewidth{0.000000pt}%
\definecolor{currentstroke}{rgb}{0.000000,0.000000,0.000000}%
\pgfsetstrokecolor{currentstroke}%
\pgfsetstrokeopacity{0.000000}%
\pgfsetdash{}{0pt}%
\pgfpathmoveto{\pgfqpoint{5.906967in}{1.609196in}}%
\pgfpathlineto{\pgfqpoint{5.915720in}{1.609196in}}%
\pgfpathlineto{\pgfqpoint{5.915720in}{1.918280in}}%
\pgfpathlineto{\pgfqpoint{5.906967in}{1.918280in}}%
\pgfpathlineto{\pgfqpoint{5.906967in}{1.609196in}}%
\pgfpathclose%
\pgfusepath{fill}%
\end{pgfscope}%
\begin{pgfscope}%
\pgfpathrectangle{\pgfqpoint{3.776708in}{0.600000in}}{\pgfqpoint{2.573292in}{2.070576in}}%
\pgfusepath{clip}%
\pgfsetbuttcap%
\pgfsetmiterjoin%
\definecolor{currentfill}{rgb}{0.133298,0.375282,0.379395}%
\pgfsetfillcolor{currentfill}%
\pgfsetlinewidth{0.000000pt}%
\definecolor{currentstroke}{rgb}{0.000000,0.000000,0.000000}%
\pgfsetstrokecolor{currentstroke}%
\pgfsetstrokeopacity{0.000000}%
\pgfsetdash{}{0pt}%
\pgfpathmoveto{\pgfqpoint{5.917908in}{1.609196in}}%
\pgfpathlineto{\pgfqpoint{5.926662in}{1.609196in}}%
\pgfpathlineto{\pgfqpoint{5.926662in}{1.925531in}}%
\pgfpathlineto{\pgfqpoint{5.917908in}{1.925531in}}%
\pgfpathlineto{\pgfqpoint{5.917908in}{1.609196in}}%
\pgfpathclose%
\pgfusepath{fill}%
\end{pgfscope}%
\begin{pgfscope}%
\pgfpathrectangle{\pgfqpoint{3.776708in}{0.600000in}}{\pgfqpoint{2.573292in}{2.070576in}}%
\pgfusepath{clip}%
\pgfsetbuttcap%
\pgfsetmiterjoin%
\definecolor{currentfill}{rgb}{0.133298,0.375282,0.379395}%
\pgfsetfillcolor{currentfill}%
\pgfsetlinewidth{0.000000pt}%
\definecolor{currentstroke}{rgb}{0.000000,0.000000,0.000000}%
\pgfsetstrokecolor{currentstroke}%
\pgfsetstrokeopacity{0.000000}%
\pgfsetdash{}{0pt}%
\pgfpathmoveto{\pgfqpoint{5.928850in}{1.609196in}}%
\pgfpathlineto{\pgfqpoint{5.937604in}{1.609196in}}%
\pgfpathlineto{\pgfqpoint{5.937604in}{1.928597in}}%
\pgfpathlineto{\pgfqpoint{5.928850in}{1.928597in}}%
\pgfpathlineto{\pgfqpoint{5.928850in}{1.609196in}}%
\pgfpathclose%
\pgfusepath{fill}%
\end{pgfscope}%
\begin{pgfscope}%
\pgfpathrectangle{\pgfqpoint{3.776708in}{0.600000in}}{\pgfqpoint{2.573292in}{2.070576in}}%
\pgfusepath{clip}%
\pgfsetbuttcap%
\pgfsetmiterjoin%
\definecolor{currentfill}{rgb}{0.133298,0.375282,0.379395}%
\pgfsetfillcolor{currentfill}%
\pgfsetlinewidth{0.000000pt}%
\definecolor{currentstroke}{rgb}{0.000000,0.000000,0.000000}%
\pgfsetstrokecolor{currentstroke}%
\pgfsetstrokeopacity{0.000000}%
\pgfsetdash{}{0pt}%
\pgfpathmoveto{\pgfqpoint{5.939792in}{1.609196in}}%
\pgfpathlineto{\pgfqpoint{5.948545in}{1.609196in}}%
\pgfpathlineto{\pgfqpoint{5.948545in}{1.929353in}}%
\pgfpathlineto{\pgfqpoint{5.939792in}{1.929353in}}%
\pgfpathlineto{\pgfqpoint{5.939792in}{1.609196in}}%
\pgfpathclose%
\pgfusepath{fill}%
\end{pgfscope}%
\begin{pgfscope}%
\pgfpathrectangle{\pgfqpoint{3.776708in}{0.600000in}}{\pgfqpoint{2.573292in}{2.070576in}}%
\pgfusepath{clip}%
\pgfsetbuttcap%
\pgfsetmiterjoin%
\definecolor{currentfill}{rgb}{0.133298,0.375282,0.379395}%
\pgfsetfillcolor{currentfill}%
\pgfsetlinewidth{0.000000pt}%
\definecolor{currentstroke}{rgb}{0.000000,0.000000,0.000000}%
\pgfsetstrokecolor{currentstroke}%
\pgfsetstrokeopacity{0.000000}%
\pgfsetdash{}{0pt}%
\pgfpathmoveto{\pgfqpoint{5.950734in}{1.609196in}}%
\pgfpathlineto{\pgfqpoint{5.959487in}{1.609196in}}%
\pgfpathlineto{\pgfqpoint{5.959487in}{1.934272in}}%
\pgfpathlineto{\pgfqpoint{5.950734in}{1.934272in}}%
\pgfpathlineto{\pgfqpoint{5.950734in}{1.609196in}}%
\pgfpathclose%
\pgfusepath{fill}%
\end{pgfscope}%
\begin{pgfscope}%
\pgfpathrectangle{\pgfqpoint{3.776708in}{0.600000in}}{\pgfqpoint{2.573292in}{2.070576in}}%
\pgfusepath{clip}%
\pgfsetbuttcap%
\pgfsetmiterjoin%
\definecolor{currentfill}{rgb}{0.133298,0.375282,0.379395}%
\pgfsetfillcolor{currentfill}%
\pgfsetlinewidth{0.000000pt}%
\definecolor{currentstroke}{rgb}{0.000000,0.000000,0.000000}%
\pgfsetstrokecolor{currentstroke}%
\pgfsetstrokeopacity{0.000000}%
\pgfsetdash{}{0pt}%
\pgfpathmoveto{\pgfqpoint{5.961676in}{1.609196in}}%
\pgfpathlineto{\pgfqpoint{5.970429in}{1.609196in}}%
\pgfpathlineto{\pgfqpoint{5.970429in}{1.940037in}}%
\pgfpathlineto{\pgfqpoint{5.961676in}{1.940037in}}%
\pgfpathlineto{\pgfqpoint{5.961676in}{1.609196in}}%
\pgfpathclose%
\pgfusepath{fill}%
\end{pgfscope}%
\begin{pgfscope}%
\pgfpathrectangle{\pgfqpoint{3.776708in}{0.600000in}}{\pgfqpoint{2.573292in}{2.070576in}}%
\pgfusepath{clip}%
\pgfsetbuttcap%
\pgfsetmiterjoin%
\definecolor{currentfill}{rgb}{0.133298,0.375282,0.379395}%
\pgfsetfillcolor{currentfill}%
\pgfsetlinewidth{0.000000pt}%
\definecolor{currentstroke}{rgb}{0.000000,0.000000,0.000000}%
\pgfsetstrokecolor{currentstroke}%
\pgfsetstrokeopacity{0.000000}%
\pgfsetdash{}{0pt}%
\pgfpathmoveto{\pgfqpoint{5.972617in}{1.609196in}}%
\pgfpathlineto{\pgfqpoint{5.981371in}{1.609196in}}%
\pgfpathlineto{\pgfqpoint{5.981371in}{1.943025in}}%
\pgfpathlineto{\pgfqpoint{5.972617in}{1.943025in}}%
\pgfpathlineto{\pgfqpoint{5.972617in}{1.609196in}}%
\pgfpathclose%
\pgfusepath{fill}%
\end{pgfscope}%
\begin{pgfscope}%
\pgfpathrectangle{\pgfqpoint{3.776708in}{0.600000in}}{\pgfqpoint{2.573292in}{2.070576in}}%
\pgfusepath{clip}%
\pgfsetbuttcap%
\pgfsetmiterjoin%
\definecolor{currentfill}{rgb}{0.133298,0.375282,0.379395}%
\pgfsetfillcolor{currentfill}%
\pgfsetlinewidth{0.000000pt}%
\definecolor{currentstroke}{rgb}{0.000000,0.000000,0.000000}%
\pgfsetstrokecolor{currentstroke}%
\pgfsetstrokeopacity{0.000000}%
\pgfsetdash{}{0pt}%
\pgfpathmoveto{\pgfqpoint{5.983559in}{1.609196in}}%
\pgfpathlineto{\pgfqpoint{5.992313in}{1.609196in}}%
\pgfpathlineto{\pgfqpoint{5.992313in}{1.947227in}}%
\pgfpathlineto{\pgfqpoint{5.983559in}{1.947227in}}%
\pgfpathlineto{\pgfqpoint{5.983559in}{1.609196in}}%
\pgfpathclose%
\pgfusepath{fill}%
\end{pgfscope}%
\begin{pgfscope}%
\pgfpathrectangle{\pgfqpoint{3.776708in}{0.600000in}}{\pgfqpoint{2.573292in}{2.070576in}}%
\pgfusepath{clip}%
\pgfsetbuttcap%
\pgfsetmiterjoin%
\definecolor{currentfill}{rgb}{0.133298,0.375282,0.379395}%
\pgfsetfillcolor{currentfill}%
\pgfsetlinewidth{0.000000pt}%
\definecolor{currentstroke}{rgb}{0.000000,0.000000,0.000000}%
\pgfsetstrokecolor{currentstroke}%
\pgfsetstrokeopacity{0.000000}%
\pgfsetdash{}{0pt}%
\pgfpathmoveto{\pgfqpoint{5.994501in}{1.609196in}}%
\pgfpathlineto{\pgfqpoint{6.003254in}{1.609196in}}%
\pgfpathlineto{\pgfqpoint{6.003254in}{1.948863in}}%
\pgfpathlineto{\pgfqpoint{5.994501in}{1.948863in}}%
\pgfpathlineto{\pgfqpoint{5.994501in}{1.609196in}}%
\pgfpathclose%
\pgfusepath{fill}%
\end{pgfscope}%
\begin{pgfscope}%
\pgfpathrectangle{\pgfqpoint{3.776708in}{0.600000in}}{\pgfqpoint{2.573292in}{2.070576in}}%
\pgfusepath{clip}%
\pgfsetbuttcap%
\pgfsetmiterjoin%
\definecolor{currentfill}{rgb}{0.133298,0.375282,0.379395}%
\pgfsetfillcolor{currentfill}%
\pgfsetlinewidth{0.000000pt}%
\definecolor{currentstroke}{rgb}{0.000000,0.000000,0.000000}%
\pgfsetstrokecolor{currentstroke}%
\pgfsetstrokeopacity{0.000000}%
\pgfsetdash{}{0pt}%
\pgfpathmoveto{\pgfqpoint{6.005443in}{1.609196in}}%
\pgfpathlineto{\pgfqpoint{6.014196in}{1.609196in}}%
\pgfpathlineto{\pgfqpoint{6.014196in}{1.946587in}}%
\pgfpathlineto{\pgfqpoint{6.005443in}{1.946587in}}%
\pgfpathlineto{\pgfqpoint{6.005443in}{1.609196in}}%
\pgfpathclose%
\pgfusepath{fill}%
\end{pgfscope}%
\begin{pgfscope}%
\pgfpathrectangle{\pgfqpoint{3.776708in}{0.600000in}}{\pgfqpoint{2.573292in}{2.070576in}}%
\pgfusepath{clip}%
\pgfsetbuttcap%
\pgfsetmiterjoin%
\definecolor{currentfill}{rgb}{0.133298,0.375282,0.379395}%
\pgfsetfillcolor{currentfill}%
\pgfsetlinewidth{0.000000pt}%
\definecolor{currentstroke}{rgb}{0.000000,0.000000,0.000000}%
\pgfsetstrokecolor{currentstroke}%
\pgfsetstrokeopacity{0.000000}%
\pgfsetdash{}{0pt}%
\pgfpathmoveto{\pgfqpoint{6.016385in}{1.609196in}}%
\pgfpathlineto{\pgfqpoint{6.025138in}{1.609196in}}%
\pgfpathlineto{\pgfqpoint{6.025138in}{1.943097in}}%
\pgfpathlineto{\pgfqpoint{6.016385in}{1.943097in}}%
\pgfpathlineto{\pgfqpoint{6.016385in}{1.609196in}}%
\pgfpathclose%
\pgfusepath{fill}%
\end{pgfscope}%
\begin{pgfscope}%
\pgfpathrectangle{\pgfqpoint{3.776708in}{0.600000in}}{\pgfqpoint{2.573292in}{2.070576in}}%
\pgfusepath{clip}%
\pgfsetbuttcap%
\pgfsetmiterjoin%
\definecolor{currentfill}{rgb}{0.133298,0.375282,0.379395}%
\pgfsetfillcolor{currentfill}%
\pgfsetlinewidth{0.000000pt}%
\definecolor{currentstroke}{rgb}{0.000000,0.000000,0.000000}%
\pgfsetstrokecolor{currentstroke}%
\pgfsetstrokeopacity{0.000000}%
\pgfsetdash{}{0pt}%
\pgfpathmoveto{\pgfqpoint{6.027326in}{1.609196in}}%
\pgfpathlineto{\pgfqpoint{6.036080in}{1.609196in}}%
\pgfpathlineto{\pgfqpoint{6.036080in}{1.949672in}}%
\pgfpathlineto{\pgfqpoint{6.027326in}{1.949672in}}%
\pgfpathlineto{\pgfqpoint{6.027326in}{1.609196in}}%
\pgfpathclose%
\pgfusepath{fill}%
\end{pgfscope}%
\begin{pgfscope}%
\pgfpathrectangle{\pgfqpoint{3.776708in}{0.600000in}}{\pgfqpoint{2.573292in}{2.070576in}}%
\pgfusepath{clip}%
\pgfsetbuttcap%
\pgfsetmiterjoin%
\definecolor{currentfill}{rgb}{0.133298,0.375282,0.379395}%
\pgfsetfillcolor{currentfill}%
\pgfsetlinewidth{0.000000pt}%
\definecolor{currentstroke}{rgb}{0.000000,0.000000,0.000000}%
\pgfsetstrokecolor{currentstroke}%
\pgfsetstrokeopacity{0.000000}%
\pgfsetdash{}{0pt}%
\pgfpathmoveto{\pgfqpoint{6.038268in}{1.609196in}}%
\pgfpathlineto{\pgfqpoint{6.047022in}{1.609196in}}%
\pgfpathlineto{\pgfqpoint{6.047022in}{1.951262in}}%
\pgfpathlineto{\pgfqpoint{6.038268in}{1.951262in}}%
\pgfpathlineto{\pgfqpoint{6.038268in}{1.609196in}}%
\pgfpathclose%
\pgfusepath{fill}%
\end{pgfscope}%
\begin{pgfscope}%
\pgfpathrectangle{\pgfqpoint{3.776708in}{0.600000in}}{\pgfqpoint{2.573292in}{2.070576in}}%
\pgfusepath{clip}%
\pgfsetbuttcap%
\pgfsetmiterjoin%
\definecolor{currentfill}{rgb}{0.133298,0.375282,0.379395}%
\pgfsetfillcolor{currentfill}%
\pgfsetlinewidth{0.000000pt}%
\definecolor{currentstroke}{rgb}{0.000000,0.000000,0.000000}%
\pgfsetstrokecolor{currentstroke}%
\pgfsetstrokeopacity{0.000000}%
\pgfsetdash{}{0pt}%
\pgfpathmoveto{\pgfqpoint{6.049210in}{1.609196in}}%
\pgfpathlineto{\pgfqpoint{6.057963in}{1.609196in}}%
\pgfpathlineto{\pgfqpoint{6.057963in}{1.953463in}}%
\pgfpathlineto{\pgfqpoint{6.049210in}{1.953463in}}%
\pgfpathlineto{\pgfqpoint{6.049210in}{1.609196in}}%
\pgfpathclose%
\pgfusepath{fill}%
\end{pgfscope}%
\begin{pgfscope}%
\pgfpathrectangle{\pgfqpoint{3.776708in}{0.600000in}}{\pgfqpoint{2.573292in}{2.070576in}}%
\pgfusepath{clip}%
\pgfsetbuttcap%
\pgfsetmiterjoin%
\definecolor{currentfill}{rgb}{0.133298,0.375282,0.379395}%
\pgfsetfillcolor{currentfill}%
\pgfsetlinewidth{0.000000pt}%
\definecolor{currentstroke}{rgb}{0.000000,0.000000,0.000000}%
\pgfsetstrokecolor{currentstroke}%
\pgfsetstrokeopacity{0.000000}%
\pgfsetdash{}{0pt}%
\pgfpathmoveto{\pgfqpoint{6.060152in}{1.609196in}}%
\pgfpathlineto{\pgfqpoint{6.068905in}{1.609196in}}%
\pgfpathlineto{\pgfqpoint{6.068905in}{1.957345in}}%
\pgfpathlineto{\pgfqpoint{6.060152in}{1.957345in}}%
\pgfpathlineto{\pgfqpoint{6.060152in}{1.609196in}}%
\pgfpathclose%
\pgfusepath{fill}%
\end{pgfscope}%
\begin{pgfscope}%
\pgfpathrectangle{\pgfqpoint{3.776708in}{0.600000in}}{\pgfqpoint{2.573292in}{2.070576in}}%
\pgfusepath{clip}%
\pgfsetbuttcap%
\pgfsetmiterjoin%
\definecolor{currentfill}{rgb}{0.133298,0.375282,0.379395}%
\pgfsetfillcolor{currentfill}%
\pgfsetlinewidth{0.000000pt}%
\definecolor{currentstroke}{rgb}{0.000000,0.000000,0.000000}%
\pgfsetstrokecolor{currentstroke}%
\pgfsetstrokeopacity{0.000000}%
\pgfsetdash{}{0pt}%
\pgfpathmoveto{\pgfqpoint{6.071094in}{1.609196in}}%
\pgfpathlineto{\pgfqpoint{6.079847in}{1.609196in}}%
\pgfpathlineto{\pgfqpoint{6.079847in}{1.972364in}}%
\pgfpathlineto{\pgfqpoint{6.071094in}{1.972364in}}%
\pgfpathlineto{\pgfqpoint{6.071094in}{1.609196in}}%
\pgfpathclose%
\pgfusepath{fill}%
\end{pgfscope}%
\begin{pgfscope}%
\pgfpathrectangle{\pgfqpoint{3.776708in}{0.600000in}}{\pgfqpoint{2.573292in}{2.070576in}}%
\pgfusepath{clip}%
\pgfsetbuttcap%
\pgfsetmiterjoin%
\definecolor{currentfill}{rgb}{0.133298,0.375282,0.379395}%
\pgfsetfillcolor{currentfill}%
\pgfsetlinewidth{0.000000pt}%
\definecolor{currentstroke}{rgb}{0.000000,0.000000,0.000000}%
\pgfsetstrokecolor{currentstroke}%
\pgfsetstrokeopacity{0.000000}%
\pgfsetdash{}{0pt}%
\pgfpathmoveto{\pgfqpoint{6.082035in}{1.609196in}}%
\pgfpathlineto{\pgfqpoint{6.090789in}{1.609196in}}%
\pgfpathlineto{\pgfqpoint{6.090789in}{1.980842in}}%
\pgfpathlineto{\pgfqpoint{6.082035in}{1.980842in}}%
\pgfpathlineto{\pgfqpoint{6.082035in}{1.609196in}}%
\pgfpathclose%
\pgfusepath{fill}%
\end{pgfscope}%
\begin{pgfscope}%
\pgfpathrectangle{\pgfqpoint{3.776708in}{0.600000in}}{\pgfqpoint{2.573292in}{2.070576in}}%
\pgfusepath{clip}%
\pgfsetbuttcap%
\pgfsetmiterjoin%
\definecolor{currentfill}{rgb}{0.133298,0.375282,0.379395}%
\pgfsetfillcolor{currentfill}%
\pgfsetlinewidth{0.000000pt}%
\definecolor{currentstroke}{rgb}{0.000000,0.000000,0.000000}%
\pgfsetstrokecolor{currentstroke}%
\pgfsetstrokeopacity{0.000000}%
\pgfsetdash{}{0pt}%
\pgfpathmoveto{\pgfqpoint{6.092977in}{1.609196in}}%
\pgfpathlineto{\pgfqpoint{6.101731in}{1.609196in}}%
\pgfpathlineto{\pgfqpoint{6.101731in}{1.993283in}}%
\pgfpathlineto{\pgfqpoint{6.092977in}{1.993283in}}%
\pgfpathlineto{\pgfqpoint{6.092977in}{1.609196in}}%
\pgfpathclose%
\pgfusepath{fill}%
\end{pgfscope}%
\begin{pgfscope}%
\pgfpathrectangle{\pgfqpoint{3.776708in}{0.600000in}}{\pgfqpoint{2.573292in}{2.070576in}}%
\pgfusepath{clip}%
\pgfsetbuttcap%
\pgfsetmiterjoin%
\definecolor{currentfill}{rgb}{0.133298,0.375282,0.379395}%
\pgfsetfillcolor{currentfill}%
\pgfsetlinewidth{0.000000pt}%
\definecolor{currentstroke}{rgb}{0.000000,0.000000,0.000000}%
\pgfsetstrokecolor{currentstroke}%
\pgfsetstrokeopacity{0.000000}%
\pgfsetdash{}{0pt}%
\pgfpathmoveto{\pgfqpoint{6.103919in}{1.609196in}}%
\pgfpathlineto{\pgfqpoint{6.112672in}{1.609196in}}%
\pgfpathlineto{\pgfqpoint{6.112672in}{2.002785in}}%
\pgfpathlineto{\pgfqpoint{6.103919in}{2.002785in}}%
\pgfpathlineto{\pgfqpoint{6.103919in}{1.609196in}}%
\pgfpathclose%
\pgfusepath{fill}%
\end{pgfscope}%
\begin{pgfscope}%
\pgfpathrectangle{\pgfqpoint{3.776708in}{0.600000in}}{\pgfqpoint{2.573292in}{2.070576in}}%
\pgfusepath{clip}%
\pgfsetbuttcap%
\pgfsetmiterjoin%
\definecolor{currentfill}{rgb}{0.133298,0.375282,0.379395}%
\pgfsetfillcolor{currentfill}%
\pgfsetlinewidth{0.000000pt}%
\definecolor{currentstroke}{rgb}{0.000000,0.000000,0.000000}%
\pgfsetstrokecolor{currentstroke}%
\pgfsetstrokeopacity{0.000000}%
\pgfsetdash{}{0pt}%
\pgfpathmoveto{\pgfqpoint{6.114861in}{1.609314in}}%
\pgfpathlineto{\pgfqpoint{6.123614in}{1.609314in}}%
\pgfpathlineto{\pgfqpoint{6.123614in}{2.007522in}}%
\pgfpathlineto{\pgfqpoint{6.114861in}{2.007522in}}%
\pgfpathlineto{\pgfqpoint{6.114861in}{1.609314in}}%
\pgfpathclose%
\pgfusepath{fill}%
\end{pgfscope}%
\begin{pgfscope}%
\pgfpathrectangle{\pgfqpoint{3.776708in}{0.600000in}}{\pgfqpoint{2.573292in}{2.070576in}}%
\pgfusepath{clip}%
\pgfsetbuttcap%
\pgfsetmiterjoin%
\definecolor{currentfill}{rgb}{0.133298,0.375282,0.379395}%
\pgfsetfillcolor{currentfill}%
\pgfsetlinewidth{0.000000pt}%
\definecolor{currentstroke}{rgb}{0.000000,0.000000,0.000000}%
\pgfsetstrokecolor{currentstroke}%
\pgfsetstrokeopacity{0.000000}%
\pgfsetdash{}{0pt}%
\pgfpathmoveto{\pgfqpoint{6.125803in}{1.616040in}}%
\pgfpathlineto{\pgfqpoint{6.134556in}{1.616040in}}%
\pgfpathlineto{\pgfqpoint{6.134556in}{2.023679in}}%
\pgfpathlineto{\pgfqpoint{6.125803in}{2.023679in}}%
\pgfpathlineto{\pgfqpoint{6.125803in}{1.616040in}}%
\pgfpathclose%
\pgfusepath{fill}%
\end{pgfscope}%
\begin{pgfscope}%
\pgfpathrectangle{\pgfqpoint{3.776708in}{0.600000in}}{\pgfqpoint{2.573292in}{2.070576in}}%
\pgfusepath{clip}%
\pgfsetbuttcap%
\pgfsetmiterjoin%
\definecolor{currentfill}{rgb}{0.133298,0.375282,0.379395}%
\pgfsetfillcolor{currentfill}%
\pgfsetlinewidth{0.000000pt}%
\definecolor{currentstroke}{rgb}{0.000000,0.000000,0.000000}%
\pgfsetstrokecolor{currentstroke}%
\pgfsetstrokeopacity{0.000000}%
\pgfsetdash{}{0pt}%
\pgfpathmoveto{\pgfqpoint{6.136744in}{1.622954in}}%
\pgfpathlineto{\pgfqpoint{6.145498in}{1.622954in}}%
\pgfpathlineto{\pgfqpoint{6.145498in}{2.039491in}}%
\pgfpathlineto{\pgfqpoint{6.136744in}{2.039491in}}%
\pgfpathlineto{\pgfqpoint{6.136744in}{1.622954in}}%
\pgfpathclose%
\pgfusepath{fill}%
\end{pgfscope}%
\begin{pgfscope}%
\pgfpathrectangle{\pgfqpoint{3.776708in}{0.600000in}}{\pgfqpoint{2.573292in}{2.070576in}}%
\pgfusepath{clip}%
\pgfsetbuttcap%
\pgfsetmiterjoin%
\definecolor{currentfill}{rgb}{0.133298,0.375282,0.379395}%
\pgfsetfillcolor{currentfill}%
\pgfsetlinewidth{0.000000pt}%
\definecolor{currentstroke}{rgb}{0.000000,0.000000,0.000000}%
\pgfsetstrokecolor{currentstroke}%
\pgfsetstrokeopacity{0.000000}%
\pgfsetdash{}{0pt}%
\pgfpathmoveto{\pgfqpoint{6.147686in}{1.630631in}}%
\pgfpathlineto{\pgfqpoint{6.156440in}{1.630631in}}%
\pgfpathlineto{\pgfqpoint{6.156440in}{2.054565in}}%
\pgfpathlineto{\pgfqpoint{6.147686in}{2.054565in}}%
\pgfpathlineto{\pgfqpoint{6.147686in}{1.630631in}}%
\pgfpathclose%
\pgfusepath{fill}%
\end{pgfscope}%
\begin{pgfscope}%
\pgfpathrectangle{\pgfqpoint{3.776708in}{0.600000in}}{\pgfqpoint{2.573292in}{2.070576in}}%
\pgfusepath{clip}%
\pgfsetbuttcap%
\pgfsetmiterjoin%
\definecolor{currentfill}{rgb}{0.133298,0.375282,0.379395}%
\pgfsetfillcolor{currentfill}%
\pgfsetlinewidth{0.000000pt}%
\definecolor{currentstroke}{rgb}{0.000000,0.000000,0.000000}%
\pgfsetstrokecolor{currentstroke}%
\pgfsetstrokeopacity{0.000000}%
\pgfsetdash{}{0pt}%
\pgfpathmoveto{\pgfqpoint{6.158628in}{1.637837in}}%
\pgfpathlineto{\pgfqpoint{6.167381in}{1.637837in}}%
\pgfpathlineto{\pgfqpoint{6.167381in}{2.068040in}}%
\pgfpathlineto{\pgfqpoint{6.158628in}{2.068040in}}%
\pgfpathlineto{\pgfqpoint{6.158628in}{1.637837in}}%
\pgfpathclose%
\pgfusepath{fill}%
\end{pgfscope}%
\begin{pgfscope}%
\pgfpathrectangle{\pgfqpoint{3.776708in}{0.600000in}}{\pgfqpoint{2.573292in}{2.070576in}}%
\pgfusepath{clip}%
\pgfsetbuttcap%
\pgfsetmiterjoin%
\definecolor{currentfill}{rgb}{0.133298,0.375282,0.379395}%
\pgfsetfillcolor{currentfill}%
\pgfsetlinewidth{0.000000pt}%
\definecolor{currentstroke}{rgb}{0.000000,0.000000,0.000000}%
\pgfsetstrokecolor{currentstroke}%
\pgfsetstrokeopacity{0.000000}%
\pgfsetdash{}{0pt}%
\pgfpathmoveto{\pgfqpoint{6.169570in}{1.644369in}}%
\pgfpathlineto{\pgfqpoint{6.178323in}{1.644369in}}%
\pgfpathlineto{\pgfqpoint{6.178323in}{2.076509in}}%
\pgfpathlineto{\pgfqpoint{6.169570in}{2.076509in}}%
\pgfpathlineto{\pgfqpoint{6.169570in}{1.644369in}}%
\pgfpathclose%
\pgfusepath{fill}%
\end{pgfscope}%
\begin{pgfscope}%
\pgfpathrectangle{\pgfqpoint{3.776708in}{0.600000in}}{\pgfqpoint{2.573292in}{2.070576in}}%
\pgfusepath{clip}%
\pgfsetbuttcap%
\pgfsetmiterjoin%
\definecolor{currentfill}{rgb}{0.133298,0.375282,0.379395}%
\pgfsetfillcolor{currentfill}%
\pgfsetlinewidth{0.000000pt}%
\definecolor{currentstroke}{rgb}{0.000000,0.000000,0.000000}%
\pgfsetstrokecolor{currentstroke}%
\pgfsetstrokeopacity{0.000000}%
\pgfsetdash{}{0pt}%
\pgfpathmoveto{\pgfqpoint{6.180512in}{1.651200in}}%
\pgfpathlineto{\pgfqpoint{6.189265in}{1.651200in}}%
\pgfpathlineto{\pgfqpoint{6.189265in}{2.086150in}}%
\pgfpathlineto{\pgfqpoint{6.180512in}{2.086150in}}%
\pgfpathlineto{\pgfqpoint{6.180512in}{1.651200in}}%
\pgfpathclose%
\pgfusepath{fill}%
\end{pgfscope}%
\begin{pgfscope}%
\pgfpathrectangle{\pgfqpoint{3.776708in}{0.600000in}}{\pgfqpoint{2.573292in}{2.070576in}}%
\pgfusepath{clip}%
\pgfsetbuttcap%
\pgfsetmiterjoin%
\definecolor{currentfill}{rgb}{0.133298,0.375282,0.379395}%
\pgfsetfillcolor{currentfill}%
\pgfsetlinewidth{0.000000pt}%
\definecolor{currentstroke}{rgb}{0.000000,0.000000,0.000000}%
\pgfsetstrokecolor{currentstroke}%
\pgfsetstrokeopacity{0.000000}%
\pgfsetdash{}{0pt}%
\pgfpathmoveto{\pgfqpoint{6.191453in}{1.658757in}}%
\pgfpathlineto{\pgfqpoint{6.200207in}{1.658757in}}%
\pgfpathlineto{\pgfqpoint{6.200207in}{2.095758in}}%
\pgfpathlineto{\pgfqpoint{6.191453in}{2.095758in}}%
\pgfpathlineto{\pgfqpoint{6.191453in}{1.658757in}}%
\pgfpathclose%
\pgfusepath{fill}%
\end{pgfscope}%
\begin{pgfscope}%
\pgfpathrectangle{\pgfqpoint{3.776708in}{0.600000in}}{\pgfqpoint{2.573292in}{2.070576in}}%
\pgfusepath{clip}%
\pgfsetbuttcap%
\pgfsetmiterjoin%
\definecolor{currentfill}{rgb}{0.133298,0.375282,0.379395}%
\pgfsetfillcolor{currentfill}%
\pgfsetlinewidth{0.000000pt}%
\definecolor{currentstroke}{rgb}{0.000000,0.000000,0.000000}%
\pgfsetstrokecolor{currentstroke}%
\pgfsetstrokeopacity{0.000000}%
\pgfsetdash{}{0pt}%
\pgfpathmoveto{\pgfqpoint{6.202395in}{1.665721in}}%
\pgfpathlineto{\pgfqpoint{6.211149in}{1.665721in}}%
\pgfpathlineto{\pgfqpoint{6.211149in}{2.110884in}}%
\pgfpathlineto{\pgfqpoint{6.202395in}{2.110884in}}%
\pgfpathlineto{\pgfqpoint{6.202395in}{1.665721in}}%
\pgfpathclose%
\pgfusepath{fill}%
\end{pgfscope}%
\begin{pgfscope}%
\pgfpathrectangle{\pgfqpoint{3.776708in}{0.600000in}}{\pgfqpoint{2.573292in}{2.070576in}}%
\pgfusepath{clip}%
\pgfsetbuttcap%
\pgfsetmiterjoin%
\definecolor{currentfill}{rgb}{0.133298,0.375282,0.379395}%
\pgfsetfillcolor{currentfill}%
\pgfsetlinewidth{0.000000pt}%
\definecolor{currentstroke}{rgb}{0.000000,0.000000,0.000000}%
\pgfsetstrokecolor{currentstroke}%
\pgfsetstrokeopacity{0.000000}%
\pgfsetdash{}{0pt}%
\pgfpathmoveto{\pgfqpoint{6.213337in}{1.672770in}}%
\pgfpathlineto{\pgfqpoint{6.222090in}{1.672770in}}%
\pgfpathlineto{\pgfqpoint{6.222090in}{2.126757in}}%
\pgfpathlineto{\pgfqpoint{6.213337in}{2.126757in}}%
\pgfpathlineto{\pgfqpoint{6.213337in}{1.672770in}}%
\pgfpathclose%
\pgfusepath{fill}%
\end{pgfscope}%
\begin{pgfscope}%
\pgfpathrectangle{\pgfqpoint{3.776708in}{0.600000in}}{\pgfqpoint{2.573292in}{2.070576in}}%
\pgfusepath{clip}%
\pgfsetbuttcap%
\pgfsetmiterjoin%
\definecolor{currentfill}{rgb}{0.133298,0.375282,0.379395}%
\pgfsetfillcolor{currentfill}%
\pgfsetlinewidth{0.000000pt}%
\definecolor{currentstroke}{rgb}{0.000000,0.000000,0.000000}%
\pgfsetstrokecolor{currentstroke}%
\pgfsetstrokeopacity{0.000000}%
\pgfsetdash{}{0pt}%
\pgfpathmoveto{\pgfqpoint{6.224279in}{1.679365in}}%
\pgfpathlineto{\pgfqpoint{6.233032in}{1.679365in}}%
\pgfpathlineto{\pgfqpoint{6.233032in}{2.145343in}}%
\pgfpathlineto{\pgfqpoint{6.224279in}{2.145343in}}%
\pgfpathlineto{\pgfqpoint{6.224279in}{1.679365in}}%
\pgfpathclose%
\pgfusepath{fill}%
\end{pgfscope}%
\begin{pgfscope}%
\pgfpathrectangle{\pgfqpoint{3.776708in}{0.600000in}}{\pgfqpoint{2.573292in}{2.070576in}}%
\pgfusepath{clip}%
\pgfsetbuttcap%
\pgfsetmiterjoin%
\definecolor{currentfill}{rgb}{0.302379,0.450282,0.300122}%
\pgfsetfillcolor{currentfill}%
\pgfsetlinewidth{0.000000pt}%
\definecolor{currentstroke}{rgb}{0.000000,0.000000,0.000000}%
\pgfsetstrokecolor{currentstroke}%
\pgfsetstrokeopacity{0.000000}%
\pgfsetdash{}{0pt}%
\pgfpathmoveto{\pgfqpoint{3.893676in}{1.609196in}}%
\pgfpathlineto{\pgfqpoint{3.902429in}{1.609196in}}%
\pgfpathlineto{\pgfqpoint{3.902429in}{1.593157in}}%
\pgfpathlineto{\pgfqpoint{3.893676in}{1.593157in}}%
\pgfpathlineto{\pgfqpoint{3.893676in}{1.609196in}}%
\pgfpathclose%
\pgfusepath{fill}%
\end{pgfscope}%
\begin{pgfscope}%
\pgfpathrectangle{\pgfqpoint{3.776708in}{0.600000in}}{\pgfqpoint{2.573292in}{2.070576in}}%
\pgfusepath{clip}%
\pgfsetbuttcap%
\pgfsetmiterjoin%
\definecolor{currentfill}{rgb}{0.302379,0.450282,0.300122}%
\pgfsetfillcolor{currentfill}%
\pgfsetlinewidth{0.000000pt}%
\definecolor{currentstroke}{rgb}{0.000000,0.000000,0.000000}%
\pgfsetstrokecolor{currentstroke}%
\pgfsetstrokeopacity{0.000000}%
\pgfsetdash{}{0pt}%
\pgfpathmoveto{\pgfqpoint{3.904617in}{1.609196in}}%
\pgfpathlineto{\pgfqpoint{3.913371in}{1.609196in}}%
\pgfpathlineto{\pgfqpoint{3.913371in}{1.593139in}}%
\pgfpathlineto{\pgfqpoint{3.904617in}{1.593139in}}%
\pgfpathlineto{\pgfqpoint{3.904617in}{1.609196in}}%
\pgfpathclose%
\pgfusepath{fill}%
\end{pgfscope}%
\begin{pgfscope}%
\pgfpathrectangle{\pgfqpoint{3.776708in}{0.600000in}}{\pgfqpoint{2.573292in}{2.070576in}}%
\pgfusepath{clip}%
\pgfsetbuttcap%
\pgfsetmiterjoin%
\definecolor{currentfill}{rgb}{0.302379,0.450282,0.300122}%
\pgfsetfillcolor{currentfill}%
\pgfsetlinewidth{0.000000pt}%
\definecolor{currentstroke}{rgb}{0.000000,0.000000,0.000000}%
\pgfsetstrokecolor{currentstroke}%
\pgfsetstrokeopacity{0.000000}%
\pgfsetdash{}{0pt}%
\pgfpathmoveto{\pgfqpoint{3.915559in}{1.609196in}}%
\pgfpathlineto{\pgfqpoint{3.924313in}{1.609196in}}%
\pgfpathlineto{\pgfqpoint{3.924313in}{1.600116in}}%
\pgfpathlineto{\pgfqpoint{3.915559in}{1.600116in}}%
\pgfpathlineto{\pgfqpoint{3.915559in}{1.609196in}}%
\pgfpathclose%
\pgfusepath{fill}%
\end{pgfscope}%
\begin{pgfscope}%
\pgfpathrectangle{\pgfqpoint{3.776708in}{0.600000in}}{\pgfqpoint{2.573292in}{2.070576in}}%
\pgfusepath{clip}%
\pgfsetbuttcap%
\pgfsetmiterjoin%
\definecolor{currentfill}{rgb}{0.302379,0.450282,0.300122}%
\pgfsetfillcolor{currentfill}%
\pgfsetlinewidth{0.000000pt}%
\definecolor{currentstroke}{rgb}{0.000000,0.000000,0.000000}%
\pgfsetstrokecolor{currentstroke}%
\pgfsetstrokeopacity{0.000000}%
\pgfsetdash{}{0pt}%
\pgfpathmoveto{\pgfqpoint{3.926501in}{1.604210in}}%
\pgfpathlineto{\pgfqpoint{3.935254in}{1.604210in}}%
\pgfpathlineto{\pgfqpoint{3.935254in}{1.596144in}}%
\pgfpathlineto{\pgfqpoint{3.926501in}{1.596144in}}%
\pgfpathlineto{\pgfqpoint{3.926501in}{1.604210in}}%
\pgfpathclose%
\pgfusepath{fill}%
\end{pgfscope}%
\begin{pgfscope}%
\pgfpathrectangle{\pgfqpoint{3.776708in}{0.600000in}}{\pgfqpoint{2.573292in}{2.070576in}}%
\pgfusepath{clip}%
\pgfsetbuttcap%
\pgfsetmiterjoin%
\definecolor{currentfill}{rgb}{0.302379,0.450282,0.300122}%
\pgfsetfillcolor{currentfill}%
\pgfsetlinewidth{0.000000pt}%
\definecolor{currentstroke}{rgb}{0.000000,0.000000,0.000000}%
\pgfsetstrokecolor{currentstroke}%
\pgfsetstrokeopacity{0.000000}%
\pgfsetdash{}{0pt}%
\pgfpathmoveto{\pgfqpoint{3.937443in}{1.605750in}}%
\pgfpathlineto{\pgfqpoint{3.946196in}{1.605750in}}%
\pgfpathlineto{\pgfqpoint{3.946196in}{1.584454in}}%
\pgfpathlineto{\pgfqpoint{3.937443in}{1.584454in}}%
\pgfpathlineto{\pgfqpoint{3.937443in}{1.605750in}}%
\pgfpathclose%
\pgfusepath{fill}%
\end{pgfscope}%
\begin{pgfscope}%
\pgfpathrectangle{\pgfqpoint{3.776708in}{0.600000in}}{\pgfqpoint{2.573292in}{2.070576in}}%
\pgfusepath{clip}%
\pgfsetbuttcap%
\pgfsetmiterjoin%
\definecolor{currentfill}{rgb}{0.302379,0.450282,0.300122}%
\pgfsetfillcolor{currentfill}%
\pgfsetlinewidth{0.000000pt}%
\definecolor{currentstroke}{rgb}{0.000000,0.000000,0.000000}%
\pgfsetstrokecolor{currentstroke}%
\pgfsetstrokeopacity{0.000000}%
\pgfsetdash{}{0pt}%
\pgfpathmoveto{\pgfqpoint{3.948385in}{1.607970in}}%
\pgfpathlineto{\pgfqpoint{3.957138in}{1.607970in}}%
\pgfpathlineto{\pgfqpoint{3.957138in}{1.570779in}}%
\pgfpathlineto{\pgfqpoint{3.948385in}{1.570779in}}%
\pgfpathlineto{\pgfqpoint{3.948385in}{1.607970in}}%
\pgfpathclose%
\pgfusepath{fill}%
\end{pgfscope}%
\begin{pgfscope}%
\pgfpathrectangle{\pgfqpoint{3.776708in}{0.600000in}}{\pgfqpoint{2.573292in}{2.070576in}}%
\pgfusepath{clip}%
\pgfsetbuttcap%
\pgfsetmiterjoin%
\definecolor{currentfill}{rgb}{0.302379,0.450282,0.300122}%
\pgfsetfillcolor{currentfill}%
\pgfsetlinewidth{0.000000pt}%
\definecolor{currentstroke}{rgb}{0.000000,0.000000,0.000000}%
\pgfsetstrokecolor{currentstroke}%
\pgfsetstrokeopacity{0.000000}%
\pgfsetdash{}{0pt}%
\pgfpathmoveto{\pgfqpoint{3.959326in}{1.608764in}}%
\pgfpathlineto{\pgfqpoint{3.968080in}{1.608764in}}%
\pgfpathlineto{\pgfqpoint{3.968080in}{1.558446in}}%
\pgfpathlineto{\pgfqpoint{3.959326in}{1.558446in}}%
\pgfpathlineto{\pgfqpoint{3.959326in}{1.608764in}}%
\pgfpathclose%
\pgfusepath{fill}%
\end{pgfscope}%
\begin{pgfscope}%
\pgfpathrectangle{\pgfqpoint{3.776708in}{0.600000in}}{\pgfqpoint{2.573292in}{2.070576in}}%
\pgfusepath{clip}%
\pgfsetbuttcap%
\pgfsetmiterjoin%
\definecolor{currentfill}{rgb}{0.302379,0.450282,0.300122}%
\pgfsetfillcolor{currentfill}%
\pgfsetlinewidth{0.000000pt}%
\definecolor{currentstroke}{rgb}{0.000000,0.000000,0.000000}%
\pgfsetstrokecolor{currentstroke}%
\pgfsetstrokeopacity{0.000000}%
\pgfsetdash{}{0pt}%
\pgfpathmoveto{\pgfqpoint{3.970268in}{1.608098in}}%
\pgfpathlineto{\pgfqpoint{3.979022in}{1.608098in}}%
\pgfpathlineto{\pgfqpoint{3.979022in}{1.551881in}}%
\pgfpathlineto{\pgfqpoint{3.970268in}{1.551881in}}%
\pgfpathlineto{\pgfqpoint{3.970268in}{1.608098in}}%
\pgfpathclose%
\pgfusepath{fill}%
\end{pgfscope}%
\begin{pgfscope}%
\pgfpathrectangle{\pgfqpoint{3.776708in}{0.600000in}}{\pgfqpoint{2.573292in}{2.070576in}}%
\pgfusepath{clip}%
\pgfsetbuttcap%
\pgfsetmiterjoin%
\definecolor{currentfill}{rgb}{0.302379,0.450282,0.300122}%
\pgfsetfillcolor{currentfill}%
\pgfsetlinewidth{0.000000pt}%
\definecolor{currentstroke}{rgb}{0.000000,0.000000,0.000000}%
\pgfsetstrokecolor{currentstroke}%
\pgfsetstrokeopacity{0.000000}%
\pgfsetdash{}{0pt}%
\pgfpathmoveto{\pgfqpoint{3.981210in}{1.603896in}}%
\pgfpathlineto{\pgfqpoint{3.989963in}{1.603896in}}%
\pgfpathlineto{\pgfqpoint{3.989963in}{1.548918in}}%
\pgfpathlineto{\pgfqpoint{3.981210in}{1.548918in}}%
\pgfpathlineto{\pgfqpoint{3.981210in}{1.603896in}}%
\pgfpathclose%
\pgfusepath{fill}%
\end{pgfscope}%
\begin{pgfscope}%
\pgfpathrectangle{\pgfqpoint{3.776708in}{0.600000in}}{\pgfqpoint{2.573292in}{2.070576in}}%
\pgfusepath{clip}%
\pgfsetbuttcap%
\pgfsetmiterjoin%
\definecolor{currentfill}{rgb}{0.302379,0.450282,0.300122}%
\pgfsetfillcolor{currentfill}%
\pgfsetlinewidth{0.000000pt}%
\definecolor{currentstroke}{rgb}{0.000000,0.000000,0.000000}%
\pgfsetstrokecolor{currentstroke}%
\pgfsetstrokeopacity{0.000000}%
\pgfsetdash{}{0pt}%
\pgfpathmoveto{\pgfqpoint{3.992152in}{1.603350in}}%
\pgfpathlineto{\pgfqpoint{4.000905in}{1.603350in}}%
\pgfpathlineto{\pgfqpoint{4.000905in}{1.542720in}}%
\pgfpathlineto{\pgfqpoint{3.992152in}{1.542720in}}%
\pgfpathlineto{\pgfqpoint{3.992152in}{1.603350in}}%
\pgfpathclose%
\pgfusepath{fill}%
\end{pgfscope}%
\begin{pgfscope}%
\pgfpathrectangle{\pgfqpoint{3.776708in}{0.600000in}}{\pgfqpoint{2.573292in}{2.070576in}}%
\pgfusepath{clip}%
\pgfsetbuttcap%
\pgfsetmiterjoin%
\definecolor{currentfill}{rgb}{0.302379,0.450282,0.300122}%
\pgfsetfillcolor{currentfill}%
\pgfsetlinewidth{0.000000pt}%
\definecolor{currentstroke}{rgb}{0.000000,0.000000,0.000000}%
\pgfsetstrokecolor{currentstroke}%
\pgfsetstrokeopacity{0.000000}%
\pgfsetdash{}{0pt}%
\pgfpathmoveto{\pgfqpoint{4.003094in}{1.601644in}}%
\pgfpathlineto{\pgfqpoint{4.011847in}{1.601644in}}%
\pgfpathlineto{\pgfqpoint{4.011847in}{1.542700in}}%
\pgfpathlineto{\pgfqpoint{4.003094in}{1.542700in}}%
\pgfpathlineto{\pgfqpoint{4.003094in}{1.601644in}}%
\pgfpathclose%
\pgfusepath{fill}%
\end{pgfscope}%
\begin{pgfscope}%
\pgfpathrectangle{\pgfqpoint{3.776708in}{0.600000in}}{\pgfqpoint{2.573292in}{2.070576in}}%
\pgfusepath{clip}%
\pgfsetbuttcap%
\pgfsetmiterjoin%
\definecolor{currentfill}{rgb}{0.302379,0.450282,0.300122}%
\pgfsetfillcolor{currentfill}%
\pgfsetlinewidth{0.000000pt}%
\definecolor{currentstroke}{rgb}{0.000000,0.000000,0.000000}%
\pgfsetstrokecolor{currentstroke}%
\pgfsetstrokeopacity{0.000000}%
\pgfsetdash{}{0pt}%
\pgfpathmoveto{\pgfqpoint{4.014035in}{1.601779in}}%
\pgfpathlineto{\pgfqpoint{4.022789in}{1.601779in}}%
\pgfpathlineto{\pgfqpoint{4.022789in}{1.543333in}}%
\pgfpathlineto{\pgfqpoint{4.014035in}{1.543333in}}%
\pgfpathlineto{\pgfqpoint{4.014035in}{1.601779in}}%
\pgfpathclose%
\pgfusepath{fill}%
\end{pgfscope}%
\begin{pgfscope}%
\pgfpathrectangle{\pgfqpoint{3.776708in}{0.600000in}}{\pgfqpoint{2.573292in}{2.070576in}}%
\pgfusepath{clip}%
\pgfsetbuttcap%
\pgfsetmiterjoin%
\definecolor{currentfill}{rgb}{0.302379,0.450282,0.300122}%
\pgfsetfillcolor{currentfill}%
\pgfsetlinewidth{0.000000pt}%
\definecolor{currentstroke}{rgb}{0.000000,0.000000,0.000000}%
\pgfsetstrokecolor{currentstroke}%
\pgfsetstrokeopacity{0.000000}%
\pgfsetdash{}{0pt}%
\pgfpathmoveto{\pgfqpoint{4.024977in}{1.601033in}}%
\pgfpathlineto{\pgfqpoint{4.033731in}{1.601033in}}%
\pgfpathlineto{\pgfqpoint{4.033731in}{1.548131in}}%
\pgfpathlineto{\pgfqpoint{4.024977in}{1.548131in}}%
\pgfpathlineto{\pgfqpoint{4.024977in}{1.601033in}}%
\pgfpathclose%
\pgfusepath{fill}%
\end{pgfscope}%
\begin{pgfscope}%
\pgfpathrectangle{\pgfqpoint{3.776708in}{0.600000in}}{\pgfqpoint{2.573292in}{2.070576in}}%
\pgfusepath{clip}%
\pgfsetbuttcap%
\pgfsetmiterjoin%
\definecolor{currentfill}{rgb}{0.302379,0.450282,0.300122}%
\pgfsetfillcolor{currentfill}%
\pgfsetlinewidth{0.000000pt}%
\definecolor{currentstroke}{rgb}{0.000000,0.000000,0.000000}%
\pgfsetstrokecolor{currentstroke}%
\pgfsetstrokeopacity{0.000000}%
\pgfsetdash{}{0pt}%
\pgfpathmoveto{\pgfqpoint{4.035919in}{1.597757in}}%
\pgfpathlineto{\pgfqpoint{4.044672in}{1.597757in}}%
\pgfpathlineto{\pgfqpoint{4.044672in}{1.557673in}}%
\pgfpathlineto{\pgfqpoint{4.035919in}{1.557673in}}%
\pgfpathlineto{\pgfqpoint{4.035919in}{1.597757in}}%
\pgfpathclose%
\pgfusepath{fill}%
\end{pgfscope}%
\begin{pgfscope}%
\pgfpathrectangle{\pgfqpoint{3.776708in}{0.600000in}}{\pgfqpoint{2.573292in}{2.070576in}}%
\pgfusepath{clip}%
\pgfsetbuttcap%
\pgfsetmiterjoin%
\definecolor{currentfill}{rgb}{0.302379,0.450282,0.300122}%
\pgfsetfillcolor{currentfill}%
\pgfsetlinewidth{0.000000pt}%
\definecolor{currentstroke}{rgb}{0.000000,0.000000,0.000000}%
\pgfsetstrokecolor{currentstroke}%
\pgfsetstrokeopacity{0.000000}%
\pgfsetdash{}{0pt}%
\pgfpathmoveto{\pgfqpoint{4.046861in}{1.590826in}}%
\pgfpathlineto{\pgfqpoint{4.055614in}{1.590826in}}%
\pgfpathlineto{\pgfqpoint{4.055614in}{1.563629in}}%
\pgfpathlineto{\pgfqpoint{4.046861in}{1.563629in}}%
\pgfpathlineto{\pgfqpoint{4.046861in}{1.590826in}}%
\pgfpathclose%
\pgfusepath{fill}%
\end{pgfscope}%
\begin{pgfscope}%
\pgfpathrectangle{\pgfqpoint{3.776708in}{0.600000in}}{\pgfqpoint{2.573292in}{2.070576in}}%
\pgfusepath{clip}%
\pgfsetbuttcap%
\pgfsetmiterjoin%
\definecolor{currentfill}{rgb}{0.302379,0.450282,0.300122}%
\pgfsetfillcolor{currentfill}%
\pgfsetlinewidth{0.000000pt}%
\definecolor{currentstroke}{rgb}{0.000000,0.000000,0.000000}%
\pgfsetstrokecolor{currentstroke}%
\pgfsetstrokeopacity{0.000000}%
\pgfsetdash{}{0pt}%
\pgfpathmoveto{\pgfqpoint{4.057803in}{1.585187in}}%
\pgfpathlineto{\pgfqpoint{4.066556in}{1.585187in}}%
\pgfpathlineto{\pgfqpoint{4.066556in}{1.568877in}}%
\pgfpathlineto{\pgfqpoint{4.057803in}{1.568877in}}%
\pgfpathlineto{\pgfqpoint{4.057803in}{1.585187in}}%
\pgfpathclose%
\pgfusepath{fill}%
\end{pgfscope}%
\begin{pgfscope}%
\pgfpathrectangle{\pgfqpoint{3.776708in}{0.600000in}}{\pgfqpoint{2.573292in}{2.070576in}}%
\pgfusepath{clip}%
\pgfsetbuttcap%
\pgfsetmiterjoin%
\definecolor{currentfill}{rgb}{0.302379,0.450282,0.300122}%
\pgfsetfillcolor{currentfill}%
\pgfsetlinewidth{0.000000pt}%
\definecolor{currentstroke}{rgb}{0.000000,0.000000,0.000000}%
\pgfsetstrokecolor{currentstroke}%
\pgfsetstrokeopacity{0.000000}%
\pgfsetdash{}{0pt}%
\pgfpathmoveto{\pgfqpoint{4.068744in}{1.578195in}}%
\pgfpathlineto{\pgfqpoint{4.077498in}{1.578195in}}%
\pgfpathlineto{\pgfqpoint{4.077498in}{1.575643in}}%
\pgfpathlineto{\pgfqpoint{4.068744in}{1.575643in}}%
\pgfpathlineto{\pgfqpoint{4.068744in}{1.578195in}}%
\pgfpathclose%
\pgfusepath{fill}%
\end{pgfscope}%
\begin{pgfscope}%
\pgfpathrectangle{\pgfqpoint{3.776708in}{0.600000in}}{\pgfqpoint{2.573292in}{2.070576in}}%
\pgfusepath{clip}%
\pgfsetbuttcap%
\pgfsetmiterjoin%
\definecolor{currentfill}{rgb}{0.302379,0.450282,0.300122}%
\pgfsetfillcolor{currentfill}%
\pgfsetlinewidth{0.000000pt}%
\definecolor{currentstroke}{rgb}{0.000000,0.000000,0.000000}%
\pgfsetstrokecolor{currentstroke}%
\pgfsetstrokeopacity{0.000000}%
\pgfsetdash{}{0pt}%
\pgfpathmoveto{\pgfqpoint{4.079686in}{1.570939in}}%
\pgfpathlineto{\pgfqpoint{4.088440in}{1.570939in}}%
\pgfpathlineto{\pgfqpoint{4.088440in}{1.566432in}}%
\pgfpathlineto{\pgfqpoint{4.079686in}{1.566432in}}%
\pgfpathlineto{\pgfqpoint{4.079686in}{1.570939in}}%
\pgfpathclose%
\pgfusepath{fill}%
\end{pgfscope}%
\begin{pgfscope}%
\pgfpathrectangle{\pgfqpoint{3.776708in}{0.600000in}}{\pgfqpoint{2.573292in}{2.070576in}}%
\pgfusepath{clip}%
\pgfsetbuttcap%
\pgfsetmiterjoin%
\definecolor{currentfill}{rgb}{0.302379,0.450282,0.300122}%
\pgfsetfillcolor{currentfill}%
\pgfsetlinewidth{0.000000pt}%
\definecolor{currentstroke}{rgb}{0.000000,0.000000,0.000000}%
\pgfsetstrokecolor{currentstroke}%
\pgfsetstrokeopacity{0.000000}%
\pgfsetdash{}{0pt}%
\pgfpathmoveto{\pgfqpoint{4.090628in}{1.571912in}}%
\pgfpathlineto{\pgfqpoint{4.099381in}{1.571912in}}%
\pgfpathlineto{\pgfqpoint{4.099381in}{1.553081in}}%
\pgfpathlineto{\pgfqpoint{4.090628in}{1.553081in}}%
\pgfpathlineto{\pgfqpoint{4.090628in}{1.571912in}}%
\pgfpathclose%
\pgfusepath{fill}%
\end{pgfscope}%
\begin{pgfscope}%
\pgfpathrectangle{\pgfqpoint{3.776708in}{0.600000in}}{\pgfqpoint{2.573292in}{2.070576in}}%
\pgfusepath{clip}%
\pgfsetbuttcap%
\pgfsetmiterjoin%
\definecolor{currentfill}{rgb}{0.302379,0.450282,0.300122}%
\pgfsetfillcolor{currentfill}%
\pgfsetlinewidth{0.000000pt}%
\definecolor{currentstroke}{rgb}{0.000000,0.000000,0.000000}%
\pgfsetstrokecolor{currentstroke}%
\pgfsetstrokeopacity{0.000000}%
\pgfsetdash{}{0pt}%
\pgfpathmoveto{\pgfqpoint{4.101570in}{1.570270in}}%
\pgfpathlineto{\pgfqpoint{4.110323in}{1.570270in}}%
\pgfpathlineto{\pgfqpoint{4.110323in}{1.531900in}}%
\pgfpathlineto{\pgfqpoint{4.101570in}{1.531900in}}%
\pgfpathlineto{\pgfqpoint{4.101570in}{1.570270in}}%
\pgfpathclose%
\pgfusepath{fill}%
\end{pgfscope}%
\begin{pgfscope}%
\pgfpathrectangle{\pgfqpoint{3.776708in}{0.600000in}}{\pgfqpoint{2.573292in}{2.070576in}}%
\pgfusepath{clip}%
\pgfsetbuttcap%
\pgfsetmiterjoin%
\definecolor{currentfill}{rgb}{0.302379,0.450282,0.300122}%
\pgfsetfillcolor{currentfill}%
\pgfsetlinewidth{0.000000pt}%
\definecolor{currentstroke}{rgb}{0.000000,0.000000,0.000000}%
\pgfsetstrokecolor{currentstroke}%
\pgfsetstrokeopacity{0.000000}%
\pgfsetdash{}{0pt}%
\pgfpathmoveto{\pgfqpoint{4.112512in}{1.567112in}}%
\pgfpathlineto{\pgfqpoint{4.121265in}{1.567112in}}%
\pgfpathlineto{\pgfqpoint{4.121265in}{1.517198in}}%
\pgfpathlineto{\pgfqpoint{4.112512in}{1.517198in}}%
\pgfpathlineto{\pgfqpoint{4.112512in}{1.567112in}}%
\pgfpathclose%
\pgfusepath{fill}%
\end{pgfscope}%
\begin{pgfscope}%
\pgfpathrectangle{\pgfqpoint{3.776708in}{0.600000in}}{\pgfqpoint{2.573292in}{2.070576in}}%
\pgfusepath{clip}%
\pgfsetbuttcap%
\pgfsetmiterjoin%
\definecolor{currentfill}{rgb}{0.302379,0.450282,0.300122}%
\pgfsetfillcolor{currentfill}%
\pgfsetlinewidth{0.000000pt}%
\definecolor{currentstroke}{rgb}{0.000000,0.000000,0.000000}%
\pgfsetstrokecolor{currentstroke}%
\pgfsetstrokeopacity{0.000000}%
\pgfsetdash{}{0pt}%
\pgfpathmoveto{\pgfqpoint{4.123453in}{1.562657in}}%
\pgfpathlineto{\pgfqpoint{4.132207in}{1.562657in}}%
\pgfpathlineto{\pgfqpoint{4.132207in}{1.509661in}}%
\pgfpathlineto{\pgfqpoint{4.123453in}{1.509661in}}%
\pgfpathlineto{\pgfqpoint{4.123453in}{1.562657in}}%
\pgfpathclose%
\pgfusepath{fill}%
\end{pgfscope}%
\begin{pgfscope}%
\pgfpathrectangle{\pgfqpoint{3.776708in}{0.600000in}}{\pgfqpoint{2.573292in}{2.070576in}}%
\pgfusepath{clip}%
\pgfsetbuttcap%
\pgfsetmiterjoin%
\definecolor{currentfill}{rgb}{0.302379,0.450282,0.300122}%
\pgfsetfillcolor{currentfill}%
\pgfsetlinewidth{0.000000pt}%
\definecolor{currentstroke}{rgb}{0.000000,0.000000,0.000000}%
\pgfsetstrokecolor{currentstroke}%
\pgfsetstrokeopacity{0.000000}%
\pgfsetdash{}{0pt}%
\pgfpathmoveto{\pgfqpoint{4.134395in}{1.570642in}}%
\pgfpathlineto{\pgfqpoint{4.143149in}{1.570642in}}%
\pgfpathlineto{\pgfqpoint{4.143149in}{1.496613in}}%
\pgfpathlineto{\pgfqpoint{4.134395in}{1.496613in}}%
\pgfpathlineto{\pgfqpoint{4.134395in}{1.570642in}}%
\pgfpathclose%
\pgfusepath{fill}%
\end{pgfscope}%
\begin{pgfscope}%
\pgfpathrectangle{\pgfqpoint{3.776708in}{0.600000in}}{\pgfqpoint{2.573292in}{2.070576in}}%
\pgfusepath{clip}%
\pgfsetbuttcap%
\pgfsetmiterjoin%
\definecolor{currentfill}{rgb}{0.302379,0.450282,0.300122}%
\pgfsetfillcolor{currentfill}%
\pgfsetlinewidth{0.000000pt}%
\definecolor{currentstroke}{rgb}{0.000000,0.000000,0.000000}%
\pgfsetstrokecolor{currentstroke}%
\pgfsetstrokeopacity{0.000000}%
\pgfsetdash{}{0pt}%
\pgfpathmoveto{\pgfqpoint{4.145337in}{1.569645in}}%
\pgfpathlineto{\pgfqpoint{4.154090in}{1.569645in}}%
\pgfpathlineto{\pgfqpoint{4.154090in}{1.485077in}}%
\pgfpathlineto{\pgfqpoint{4.145337in}{1.485077in}}%
\pgfpathlineto{\pgfqpoint{4.145337in}{1.569645in}}%
\pgfpathclose%
\pgfusepath{fill}%
\end{pgfscope}%
\begin{pgfscope}%
\pgfpathrectangle{\pgfqpoint{3.776708in}{0.600000in}}{\pgfqpoint{2.573292in}{2.070576in}}%
\pgfusepath{clip}%
\pgfsetbuttcap%
\pgfsetmiterjoin%
\definecolor{currentfill}{rgb}{0.302379,0.450282,0.300122}%
\pgfsetfillcolor{currentfill}%
\pgfsetlinewidth{0.000000pt}%
\definecolor{currentstroke}{rgb}{0.000000,0.000000,0.000000}%
\pgfsetstrokecolor{currentstroke}%
\pgfsetstrokeopacity{0.000000}%
\pgfsetdash{}{0pt}%
\pgfpathmoveto{\pgfqpoint{4.156279in}{1.567532in}}%
\pgfpathlineto{\pgfqpoint{4.165032in}{1.567532in}}%
\pgfpathlineto{\pgfqpoint{4.165032in}{1.471065in}}%
\pgfpathlineto{\pgfqpoint{4.156279in}{1.471065in}}%
\pgfpathlineto{\pgfqpoint{4.156279in}{1.567532in}}%
\pgfpathclose%
\pgfusepath{fill}%
\end{pgfscope}%
\begin{pgfscope}%
\pgfpathrectangle{\pgfqpoint{3.776708in}{0.600000in}}{\pgfqpoint{2.573292in}{2.070576in}}%
\pgfusepath{clip}%
\pgfsetbuttcap%
\pgfsetmiterjoin%
\definecolor{currentfill}{rgb}{0.302379,0.450282,0.300122}%
\pgfsetfillcolor{currentfill}%
\pgfsetlinewidth{0.000000pt}%
\definecolor{currentstroke}{rgb}{0.000000,0.000000,0.000000}%
\pgfsetstrokecolor{currentstroke}%
\pgfsetstrokeopacity{0.000000}%
\pgfsetdash{}{0pt}%
\pgfpathmoveto{\pgfqpoint{4.167221in}{1.568497in}}%
\pgfpathlineto{\pgfqpoint{4.175974in}{1.568497in}}%
\pgfpathlineto{\pgfqpoint{4.175974in}{1.457203in}}%
\pgfpathlineto{\pgfqpoint{4.167221in}{1.457203in}}%
\pgfpathlineto{\pgfqpoint{4.167221in}{1.568497in}}%
\pgfpathclose%
\pgfusepath{fill}%
\end{pgfscope}%
\begin{pgfscope}%
\pgfpathrectangle{\pgfqpoint{3.776708in}{0.600000in}}{\pgfqpoint{2.573292in}{2.070576in}}%
\pgfusepath{clip}%
\pgfsetbuttcap%
\pgfsetmiterjoin%
\definecolor{currentfill}{rgb}{0.302379,0.450282,0.300122}%
\pgfsetfillcolor{currentfill}%
\pgfsetlinewidth{0.000000pt}%
\definecolor{currentstroke}{rgb}{0.000000,0.000000,0.000000}%
\pgfsetstrokecolor{currentstroke}%
\pgfsetstrokeopacity{0.000000}%
\pgfsetdash{}{0pt}%
\pgfpathmoveto{\pgfqpoint{4.178162in}{1.569369in}}%
\pgfpathlineto{\pgfqpoint{4.186916in}{1.569369in}}%
\pgfpathlineto{\pgfqpoint{4.186916in}{1.450589in}}%
\pgfpathlineto{\pgfqpoint{4.178162in}{1.450589in}}%
\pgfpathlineto{\pgfqpoint{4.178162in}{1.569369in}}%
\pgfpathclose%
\pgfusepath{fill}%
\end{pgfscope}%
\begin{pgfscope}%
\pgfpathrectangle{\pgfqpoint{3.776708in}{0.600000in}}{\pgfqpoint{2.573292in}{2.070576in}}%
\pgfusepath{clip}%
\pgfsetbuttcap%
\pgfsetmiterjoin%
\definecolor{currentfill}{rgb}{0.302379,0.450282,0.300122}%
\pgfsetfillcolor{currentfill}%
\pgfsetlinewidth{0.000000pt}%
\definecolor{currentstroke}{rgb}{0.000000,0.000000,0.000000}%
\pgfsetstrokecolor{currentstroke}%
\pgfsetstrokeopacity{0.000000}%
\pgfsetdash{}{0pt}%
\pgfpathmoveto{\pgfqpoint{4.189104in}{1.570665in}}%
\pgfpathlineto{\pgfqpoint{4.197858in}{1.570665in}}%
\pgfpathlineto{\pgfqpoint{4.197858in}{1.445010in}}%
\pgfpathlineto{\pgfqpoint{4.189104in}{1.445010in}}%
\pgfpathlineto{\pgfqpoint{4.189104in}{1.570665in}}%
\pgfpathclose%
\pgfusepath{fill}%
\end{pgfscope}%
\begin{pgfscope}%
\pgfpathrectangle{\pgfqpoint{3.776708in}{0.600000in}}{\pgfqpoint{2.573292in}{2.070576in}}%
\pgfusepath{clip}%
\pgfsetbuttcap%
\pgfsetmiterjoin%
\definecolor{currentfill}{rgb}{0.302379,0.450282,0.300122}%
\pgfsetfillcolor{currentfill}%
\pgfsetlinewidth{0.000000pt}%
\definecolor{currentstroke}{rgb}{0.000000,0.000000,0.000000}%
\pgfsetstrokecolor{currentstroke}%
\pgfsetstrokeopacity{0.000000}%
\pgfsetdash{}{0pt}%
\pgfpathmoveto{\pgfqpoint{4.200046in}{1.569061in}}%
\pgfpathlineto{\pgfqpoint{4.208799in}{1.569061in}}%
\pgfpathlineto{\pgfqpoint{4.208799in}{1.441753in}}%
\pgfpathlineto{\pgfqpoint{4.200046in}{1.441753in}}%
\pgfpathlineto{\pgfqpoint{4.200046in}{1.569061in}}%
\pgfpathclose%
\pgfusepath{fill}%
\end{pgfscope}%
\begin{pgfscope}%
\pgfpathrectangle{\pgfqpoint{3.776708in}{0.600000in}}{\pgfqpoint{2.573292in}{2.070576in}}%
\pgfusepath{clip}%
\pgfsetbuttcap%
\pgfsetmiterjoin%
\definecolor{currentfill}{rgb}{0.302379,0.450282,0.300122}%
\pgfsetfillcolor{currentfill}%
\pgfsetlinewidth{0.000000pt}%
\definecolor{currentstroke}{rgb}{0.000000,0.000000,0.000000}%
\pgfsetstrokecolor{currentstroke}%
\pgfsetstrokeopacity{0.000000}%
\pgfsetdash{}{0pt}%
\pgfpathmoveto{\pgfqpoint{4.210988in}{1.570307in}}%
\pgfpathlineto{\pgfqpoint{4.219741in}{1.570307in}}%
\pgfpathlineto{\pgfqpoint{4.219741in}{1.456043in}}%
\pgfpathlineto{\pgfqpoint{4.210988in}{1.456043in}}%
\pgfpathlineto{\pgfqpoint{4.210988in}{1.570307in}}%
\pgfpathclose%
\pgfusepath{fill}%
\end{pgfscope}%
\begin{pgfscope}%
\pgfpathrectangle{\pgfqpoint{3.776708in}{0.600000in}}{\pgfqpoint{2.573292in}{2.070576in}}%
\pgfusepath{clip}%
\pgfsetbuttcap%
\pgfsetmiterjoin%
\definecolor{currentfill}{rgb}{0.302379,0.450282,0.300122}%
\pgfsetfillcolor{currentfill}%
\pgfsetlinewidth{0.000000pt}%
\definecolor{currentstroke}{rgb}{0.000000,0.000000,0.000000}%
\pgfsetstrokecolor{currentstroke}%
\pgfsetstrokeopacity{0.000000}%
\pgfsetdash{}{0pt}%
\pgfpathmoveto{\pgfqpoint{4.221930in}{1.569944in}}%
\pgfpathlineto{\pgfqpoint{4.230683in}{1.569944in}}%
\pgfpathlineto{\pgfqpoint{4.230683in}{1.467761in}}%
\pgfpathlineto{\pgfqpoint{4.221930in}{1.467761in}}%
\pgfpathlineto{\pgfqpoint{4.221930in}{1.569944in}}%
\pgfpathclose%
\pgfusepath{fill}%
\end{pgfscope}%
\begin{pgfscope}%
\pgfpathrectangle{\pgfqpoint{3.776708in}{0.600000in}}{\pgfqpoint{2.573292in}{2.070576in}}%
\pgfusepath{clip}%
\pgfsetbuttcap%
\pgfsetmiterjoin%
\definecolor{currentfill}{rgb}{0.302379,0.450282,0.300122}%
\pgfsetfillcolor{currentfill}%
\pgfsetlinewidth{0.000000pt}%
\definecolor{currentstroke}{rgb}{0.000000,0.000000,0.000000}%
\pgfsetstrokecolor{currentstroke}%
\pgfsetstrokeopacity{0.000000}%
\pgfsetdash{}{0pt}%
\pgfpathmoveto{\pgfqpoint{4.232871in}{1.569251in}}%
\pgfpathlineto{\pgfqpoint{4.241625in}{1.569251in}}%
\pgfpathlineto{\pgfqpoint{4.241625in}{1.471474in}}%
\pgfpathlineto{\pgfqpoint{4.232871in}{1.471474in}}%
\pgfpathlineto{\pgfqpoint{4.232871in}{1.569251in}}%
\pgfpathclose%
\pgfusepath{fill}%
\end{pgfscope}%
\begin{pgfscope}%
\pgfpathrectangle{\pgfqpoint{3.776708in}{0.600000in}}{\pgfqpoint{2.573292in}{2.070576in}}%
\pgfusepath{clip}%
\pgfsetbuttcap%
\pgfsetmiterjoin%
\definecolor{currentfill}{rgb}{0.302379,0.450282,0.300122}%
\pgfsetfillcolor{currentfill}%
\pgfsetlinewidth{0.000000pt}%
\definecolor{currentstroke}{rgb}{0.000000,0.000000,0.000000}%
\pgfsetstrokecolor{currentstroke}%
\pgfsetstrokeopacity{0.000000}%
\pgfsetdash{}{0pt}%
\pgfpathmoveto{\pgfqpoint{4.243813in}{1.569658in}}%
\pgfpathlineto{\pgfqpoint{4.252567in}{1.569658in}}%
\pgfpathlineto{\pgfqpoint{4.252567in}{1.474100in}}%
\pgfpathlineto{\pgfqpoint{4.243813in}{1.474100in}}%
\pgfpathlineto{\pgfqpoint{4.243813in}{1.569658in}}%
\pgfpathclose%
\pgfusepath{fill}%
\end{pgfscope}%
\begin{pgfscope}%
\pgfpathrectangle{\pgfqpoint{3.776708in}{0.600000in}}{\pgfqpoint{2.573292in}{2.070576in}}%
\pgfusepath{clip}%
\pgfsetbuttcap%
\pgfsetmiterjoin%
\definecolor{currentfill}{rgb}{0.302379,0.450282,0.300122}%
\pgfsetfillcolor{currentfill}%
\pgfsetlinewidth{0.000000pt}%
\definecolor{currentstroke}{rgb}{0.000000,0.000000,0.000000}%
\pgfsetstrokecolor{currentstroke}%
\pgfsetstrokeopacity{0.000000}%
\pgfsetdash{}{0pt}%
\pgfpathmoveto{\pgfqpoint{4.254755in}{1.570404in}}%
\pgfpathlineto{\pgfqpoint{4.263508in}{1.570404in}}%
\pgfpathlineto{\pgfqpoint{4.263508in}{1.483465in}}%
\pgfpathlineto{\pgfqpoint{4.254755in}{1.483465in}}%
\pgfpathlineto{\pgfqpoint{4.254755in}{1.570404in}}%
\pgfpathclose%
\pgfusepath{fill}%
\end{pgfscope}%
\begin{pgfscope}%
\pgfpathrectangle{\pgfqpoint{3.776708in}{0.600000in}}{\pgfqpoint{2.573292in}{2.070576in}}%
\pgfusepath{clip}%
\pgfsetbuttcap%
\pgfsetmiterjoin%
\definecolor{currentfill}{rgb}{0.302379,0.450282,0.300122}%
\pgfsetfillcolor{currentfill}%
\pgfsetlinewidth{0.000000pt}%
\definecolor{currentstroke}{rgb}{0.000000,0.000000,0.000000}%
\pgfsetstrokecolor{currentstroke}%
\pgfsetstrokeopacity{0.000000}%
\pgfsetdash{}{0pt}%
\pgfpathmoveto{\pgfqpoint{4.265697in}{1.563869in}}%
\pgfpathlineto{\pgfqpoint{4.274450in}{1.563869in}}%
\pgfpathlineto{\pgfqpoint{4.274450in}{1.479511in}}%
\pgfpathlineto{\pgfqpoint{4.265697in}{1.479511in}}%
\pgfpathlineto{\pgfqpoint{4.265697in}{1.563869in}}%
\pgfpathclose%
\pgfusepath{fill}%
\end{pgfscope}%
\begin{pgfscope}%
\pgfpathrectangle{\pgfqpoint{3.776708in}{0.600000in}}{\pgfqpoint{2.573292in}{2.070576in}}%
\pgfusepath{clip}%
\pgfsetbuttcap%
\pgfsetmiterjoin%
\definecolor{currentfill}{rgb}{0.302379,0.450282,0.300122}%
\pgfsetfillcolor{currentfill}%
\pgfsetlinewidth{0.000000pt}%
\definecolor{currentstroke}{rgb}{0.000000,0.000000,0.000000}%
\pgfsetstrokecolor{currentstroke}%
\pgfsetstrokeopacity{0.000000}%
\pgfsetdash{}{0pt}%
\pgfpathmoveto{\pgfqpoint{4.276639in}{1.557315in}}%
\pgfpathlineto{\pgfqpoint{4.285392in}{1.557315in}}%
\pgfpathlineto{\pgfqpoint{4.285392in}{1.472560in}}%
\pgfpathlineto{\pgfqpoint{4.276639in}{1.472560in}}%
\pgfpathlineto{\pgfqpoint{4.276639in}{1.557315in}}%
\pgfpathclose%
\pgfusepath{fill}%
\end{pgfscope}%
\begin{pgfscope}%
\pgfpathrectangle{\pgfqpoint{3.776708in}{0.600000in}}{\pgfqpoint{2.573292in}{2.070576in}}%
\pgfusepath{clip}%
\pgfsetbuttcap%
\pgfsetmiterjoin%
\definecolor{currentfill}{rgb}{0.302379,0.450282,0.300122}%
\pgfsetfillcolor{currentfill}%
\pgfsetlinewidth{0.000000pt}%
\definecolor{currentstroke}{rgb}{0.000000,0.000000,0.000000}%
\pgfsetstrokecolor{currentstroke}%
\pgfsetstrokeopacity{0.000000}%
\pgfsetdash{}{0pt}%
\pgfpathmoveto{\pgfqpoint{4.287580in}{1.550276in}}%
\pgfpathlineto{\pgfqpoint{4.296334in}{1.550276in}}%
\pgfpathlineto{\pgfqpoint{4.296334in}{1.454564in}}%
\pgfpathlineto{\pgfqpoint{4.287580in}{1.454564in}}%
\pgfpathlineto{\pgfqpoint{4.287580in}{1.550276in}}%
\pgfpathclose%
\pgfusepath{fill}%
\end{pgfscope}%
\begin{pgfscope}%
\pgfpathrectangle{\pgfqpoint{3.776708in}{0.600000in}}{\pgfqpoint{2.573292in}{2.070576in}}%
\pgfusepath{clip}%
\pgfsetbuttcap%
\pgfsetmiterjoin%
\definecolor{currentfill}{rgb}{0.302379,0.450282,0.300122}%
\pgfsetfillcolor{currentfill}%
\pgfsetlinewidth{0.000000pt}%
\definecolor{currentstroke}{rgb}{0.000000,0.000000,0.000000}%
\pgfsetstrokecolor{currentstroke}%
\pgfsetstrokeopacity{0.000000}%
\pgfsetdash{}{0pt}%
\pgfpathmoveto{\pgfqpoint{4.298522in}{1.548095in}}%
\pgfpathlineto{\pgfqpoint{4.307276in}{1.548095in}}%
\pgfpathlineto{\pgfqpoint{4.307276in}{1.445838in}}%
\pgfpathlineto{\pgfqpoint{4.298522in}{1.445838in}}%
\pgfpathlineto{\pgfqpoint{4.298522in}{1.548095in}}%
\pgfpathclose%
\pgfusepath{fill}%
\end{pgfscope}%
\begin{pgfscope}%
\pgfpathrectangle{\pgfqpoint{3.776708in}{0.600000in}}{\pgfqpoint{2.573292in}{2.070576in}}%
\pgfusepath{clip}%
\pgfsetbuttcap%
\pgfsetmiterjoin%
\definecolor{currentfill}{rgb}{0.302379,0.450282,0.300122}%
\pgfsetfillcolor{currentfill}%
\pgfsetlinewidth{0.000000pt}%
\definecolor{currentstroke}{rgb}{0.000000,0.000000,0.000000}%
\pgfsetstrokecolor{currentstroke}%
\pgfsetstrokeopacity{0.000000}%
\pgfsetdash{}{0pt}%
\pgfpathmoveto{\pgfqpoint{4.309464in}{1.544782in}}%
\pgfpathlineto{\pgfqpoint{4.318217in}{1.544782in}}%
\pgfpathlineto{\pgfqpoint{4.318217in}{1.443942in}}%
\pgfpathlineto{\pgfqpoint{4.309464in}{1.443942in}}%
\pgfpathlineto{\pgfqpoint{4.309464in}{1.544782in}}%
\pgfpathclose%
\pgfusepath{fill}%
\end{pgfscope}%
\begin{pgfscope}%
\pgfpathrectangle{\pgfqpoint{3.776708in}{0.600000in}}{\pgfqpoint{2.573292in}{2.070576in}}%
\pgfusepath{clip}%
\pgfsetbuttcap%
\pgfsetmiterjoin%
\definecolor{currentfill}{rgb}{0.302379,0.450282,0.300122}%
\pgfsetfillcolor{currentfill}%
\pgfsetlinewidth{0.000000pt}%
\definecolor{currentstroke}{rgb}{0.000000,0.000000,0.000000}%
\pgfsetstrokecolor{currentstroke}%
\pgfsetstrokeopacity{0.000000}%
\pgfsetdash{}{0pt}%
\pgfpathmoveto{\pgfqpoint{4.320406in}{1.541583in}}%
\pgfpathlineto{\pgfqpoint{4.329159in}{1.541583in}}%
\pgfpathlineto{\pgfqpoint{4.329159in}{1.437903in}}%
\pgfpathlineto{\pgfqpoint{4.320406in}{1.437903in}}%
\pgfpathlineto{\pgfqpoint{4.320406in}{1.541583in}}%
\pgfpathclose%
\pgfusepath{fill}%
\end{pgfscope}%
\begin{pgfscope}%
\pgfpathrectangle{\pgfqpoint{3.776708in}{0.600000in}}{\pgfqpoint{2.573292in}{2.070576in}}%
\pgfusepath{clip}%
\pgfsetbuttcap%
\pgfsetmiterjoin%
\definecolor{currentfill}{rgb}{0.302379,0.450282,0.300122}%
\pgfsetfillcolor{currentfill}%
\pgfsetlinewidth{0.000000pt}%
\definecolor{currentstroke}{rgb}{0.000000,0.000000,0.000000}%
\pgfsetstrokecolor{currentstroke}%
\pgfsetstrokeopacity{0.000000}%
\pgfsetdash{}{0pt}%
\pgfpathmoveto{\pgfqpoint{4.331348in}{1.538411in}}%
\pgfpathlineto{\pgfqpoint{4.340101in}{1.538411in}}%
\pgfpathlineto{\pgfqpoint{4.340101in}{1.427592in}}%
\pgfpathlineto{\pgfqpoint{4.331348in}{1.427592in}}%
\pgfpathlineto{\pgfqpoint{4.331348in}{1.538411in}}%
\pgfpathclose%
\pgfusepath{fill}%
\end{pgfscope}%
\begin{pgfscope}%
\pgfpathrectangle{\pgfqpoint{3.776708in}{0.600000in}}{\pgfqpoint{2.573292in}{2.070576in}}%
\pgfusepath{clip}%
\pgfsetbuttcap%
\pgfsetmiterjoin%
\definecolor{currentfill}{rgb}{0.302379,0.450282,0.300122}%
\pgfsetfillcolor{currentfill}%
\pgfsetlinewidth{0.000000pt}%
\definecolor{currentstroke}{rgb}{0.000000,0.000000,0.000000}%
\pgfsetstrokecolor{currentstroke}%
\pgfsetstrokeopacity{0.000000}%
\pgfsetdash{}{0pt}%
\pgfpathmoveto{\pgfqpoint{4.342289in}{1.539114in}}%
\pgfpathlineto{\pgfqpoint{4.351043in}{1.539114in}}%
\pgfpathlineto{\pgfqpoint{4.351043in}{1.414221in}}%
\pgfpathlineto{\pgfqpoint{4.342289in}{1.414221in}}%
\pgfpathlineto{\pgfqpoint{4.342289in}{1.539114in}}%
\pgfpathclose%
\pgfusepath{fill}%
\end{pgfscope}%
\begin{pgfscope}%
\pgfpathrectangle{\pgfqpoint{3.776708in}{0.600000in}}{\pgfqpoint{2.573292in}{2.070576in}}%
\pgfusepath{clip}%
\pgfsetbuttcap%
\pgfsetmiterjoin%
\definecolor{currentfill}{rgb}{0.302379,0.450282,0.300122}%
\pgfsetfillcolor{currentfill}%
\pgfsetlinewidth{0.000000pt}%
\definecolor{currentstroke}{rgb}{0.000000,0.000000,0.000000}%
\pgfsetstrokecolor{currentstroke}%
\pgfsetstrokeopacity{0.000000}%
\pgfsetdash{}{0pt}%
\pgfpathmoveto{\pgfqpoint{4.353231in}{1.539783in}}%
\pgfpathlineto{\pgfqpoint{4.361985in}{1.539783in}}%
\pgfpathlineto{\pgfqpoint{4.361985in}{1.402583in}}%
\pgfpathlineto{\pgfqpoint{4.353231in}{1.402583in}}%
\pgfpathlineto{\pgfqpoint{4.353231in}{1.539783in}}%
\pgfpathclose%
\pgfusepath{fill}%
\end{pgfscope}%
\begin{pgfscope}%
\pgfpathrectangle{\pgfqpoint{3.776708in}{0.600000in}}{\pgfqpoint{2.573292in}{2.070576in}}%
\pgfusepath{clip}%
\pgfsetbuttcap%
\pgfsetmiterjoin%
\definecolor{currentfill}{rgb}{0.302379,0.450282,0.300122}%
\pgfsetfillcolor{currentfill}%
\pgfsetlinewidth{0.000000pt}%
\definecolor{currentstroke}{rgb}{0.000000,0.000000,0.000000}%
\pgfsetstrokecolor{currentstroke}%
\pgfsetstrokeopacity{0.000000}%
\pgfsetdash{}{0pt}%
\pgfpathmoveto{\pgfqpoint{4.364173in}{1.542506in}}%
\pgfpathlineto{\pgfqpoint{4.372926in}{1.542506in}}%
\pgfpathlineto{\pgfqpoint{4.372926in}{1.395256in}}%
\pgfpathlineto{\pgfqpoint{4.364173in}{1.395256in}}%
\pgfpathlineto{\pgfqpoint{4.364173in}{1.542506in}}%
\pgfpathclose%
\pgfusepath{fill}%
\end{pgfscope}%
\begin{pgfscope}%
\pgfpathrectangle{\pgfqpoint{3.776708in}{0.600000in}}{\pgfqpoint{2.573292in}{2.070576in}}%
\pgfusepath{clip}%
\pgfsetbuttcap%
\pgfsetmiterjoin%
\definecolor{currentfill}{rgb}{0.302379,0.450282,0.300122}%
\pgfsetfillcolor{currentfill}%
\pgfsetlinewidth{0.000000pt}%
\definecolor{currentstroke}{rgb}{0.000000,0.000000,0.000000}%
\pgfsetstrokecolor{currentstroke}%
\pgfsetstrokeopacity{0.000000}%
\pgfsetdash{}{0pt}%
\pgfpathmoveto{\pgfqpoint{4.375115in}{1.545446in}}%
\pgfpathlineto{\pgfqpoint{4.383868in}{1.545446in}}%
\pgfpathlineto{\pgfqpoint{4.383868in}{1.393238in}}%
\pgfpathlineto{\pgfqpoint{4.375115in}{1.393238in}}%
\pgfpathlineto{\pgfqpoint{4.375115in}{1.545446in}}%
\pgfpathclose%
\pgfusepath{fill}%
\end{pgfscope}%
\begin{pgfscope}%
\pgfpathrectangle{\pgfqpoint{3.776708in}{0.600000in}}{\pgfqpoint{2.573292in}{2.070576in}}%
\pgfusepath{clip}%
\pgfsetbuttcap%
\pgfsetmiterjoin%
\definecolor{currentfill}{rgb}{0.302379,0.450282,0.300122}%
\pgfsetfillcolor{currentfill}%
\pgfsetlinewidth{0.000000pt}%
\definecolor{currentstroke}{rgb}{0.000000,0.000000,0.000000}%
\pgfsetstrokecolor{currentstroke}%
\pgfsetstrokeopacity{0.000000}%
\pgfsetdash{}{0pt}%
\pgfpathmoveto{\pgfqpoint{4.386057in}{1.550014in}}%
\pgfpathlineto{\pgfqpoint{4.394810in}{1.550014in}}%
\pgfpathlineto{\pgfqpoint{4.394810in}{1.382370in}}%
\pgfpathlineto{\pgfqpoint{4.386057in}{1.382370in}}%
\pgfpathlineto{\pgfqpoint{4.386057in}{1.550014in}}%
\pgfpathclose%
\pgfusepath{fill}%
\end{pgfscope}%
\begin{pgfscope}%
\pgfpathrectangle{\pgfqpoint{3.776708in}{0.600000in}}{\pgfqpoint{2.573292in}{2.070576in}}%
\pgfusepath{clip}%
\pgfsetbuttcap%
\pgfsetmiterjoin%
\definecolor{currentfill}{rgb}{0.302379,0.450282,0.300122}%
\pgfsetfillcolor{currentfill}%
\pgfsetlinewidth{0.000000pt}%
\definecolor{currentstroke}{rgb}{0.000000,0.000000,0.000000}%
\pgfsetstrokecolor{currentstroke}%
\pgfsetstrokeopacity{0.000000}%
\pgfsetdash{}{0pt}%
\pgfpathmoveto{\pgfqpoint{4.396998in}{1.554298in}}%
\pgfpathlineto{\pgfqpoint{4.405752in}{1.554298in}}%
\pgfpathlineto{\pgfqpoint{4.405752in}{1.389016in}}%
\pgfpathlineto{\pgfqpoint{4.396998in}{1.389016in}}%
\pgfpathlineto{\pgfqpoint{4.396998in}{1.554298in}}%
\pgfpathclose%
\pgfusepath{fill}%
\end{pgfscope}%
\begin{pgfscope}%
\pgfpathrectangle{\pgfqpoint{3.776708in}{0.600000in}}{\pgfqpoint{2.573292in}{2.070576in}}%
\pgfusepath{clip}%
\pgfsetbuttcap%
\pgfsetmiterjoin%
\definecolor{currentfill}{rgb}{0.302379,0.450282,0.300122}%
\pgfsetfillcolor{currentfill}%
\pgfsetlinewidth{0.000000pt}%
\definecolor{currentstroke}{rgb}{0.000000,0.000000,0.000000}%
\pgfsetstrokecolor{currentstroke}%
\pgfsetstrokeopacity{0.000000}%
\pgfsetdash{}{0pt}%
\pgfpathmoveto{\pgfqpoint{4.407940in}{1.563504in}}%
\pgfpathlineto{\pgfqpoint{4.416694in}{1.563504in}}%
\pgfpathlineto{\pgfqpoint{4.416694in}{1.394253in}}%
\pgfpathlineto{\pgfqpoint{4.407940in}{1.394253in}}%
\pgfpathlineto{\pgfqpoint{4.407940in}{1.563504in}}%
\pgfpathclose%
\pgfusepath{fill}%
\end{pgfscope}%
\begin{pgfscope}%
\pgfpathrectangle{\pgfqpoint{3.776708in}{0.600000in}}{\pgfqpoint{2.573292in}{2.070576in}}%
\pgfusepath{clip}%
\pgfsetbuttcap%
\pgfsetmiterjoin%
\definecolor{currentfill}{rgb}{0.302379,0.450282,0.300122}%
\pgfsetfillcolor{currentfill}%
\pgfsetlinewidth{0.000000pt}%
\definecolor{currentstroke}{rgb}{0.000000,0.000000,0.000000}%
\pgfsetstrokecolor{currentstroke}%
\pgfsetstrokeopacity{0.000000}%
\pgfsetdash{}{0pt}%
\pgfpathmoveto{\pgfqpoint{4.418882in}{1.571360in}}%
\pgfpathlineto{\pgfqpoint{4.427635in}{1.571360in}}%
\pgfpathlineto{\pgfqpoint{4.427635in}{1.404544in}}%
\pgfpathlineto{\pgfqpoint{4.418882in}{1.404544in}}%
\pgfpathlineto{\pgfqpoint{4.418882in}{1.571360in}}%
\pgfpathclose%
\pgfusepath{fill}%
\end{pgfscope}%
\begin{pgfscope}%
\pgfpathrectangle{\pgfqpoint{3.776708in}{0.600000in}}{\pgfqpoint{2.573292in}{2.070576in}}%
\pgfusepath{clip}%
\pgfsetbuttcap%
\pgfsetmiterjoin%
\definecolor{currentfill}{rgb}{0.302379,0.450282,0.300122}%
\pgfsetfillcolor{currentfill}%
\pgfsetlinewidth{0.000000pt}%
\definecolor{currentstroke}{rgb}{0.000000,0.000000,0.000000}%
\pgfsetstrokecolor{currentstroke}%
\pgfsetstrokeopacity{0.000000}%
\pgfsetdash{}{0pt}%
\pgfpathmoveto{\pgfqpoint{4.429824in}{1.580604in}}%
\pgfpathlineto{\pgfqpoint{4.438577in}{1.580604in}}%
\pgfpathlineto{\pgfqpoint{4.438577in}{1.416801in}}%
\pgfpathlineto{\pgfqpoint{4.429824in}{1.416801in}}%
\pgfpathlineto{\pgfqpoint{4.429824in}{1.580604in}}%
\pgfpathclose%
\pgfusepath{fill}%
\end{pgfscope}%
\begin{pgfscope}%
\pgfpathrectangle{\pgfqpoint{3.776708in}{0.600000in}}{\pgfqpoint{2.573292in}{2.070576in}}%
\pgfusepath{clip}%
\pgfsetbuttcap%
\pgfsetmiterjoin%
\definecolor{currentfill}{rgb}{0.302379,0.450282,0.300122}%
\pgfsetfillcolor{currentfill}%
\pgfsetlinewidth{0.000000pt}%
\definecolor{currentstroke}{rgb}{0.000000,0.000000,0.000000}%
\pgfsetstrokecolor{currentstroke}%
\pgfsetstrokeopacity{0.000000}%
\pgfsetdash{}{0pt}%
\pgfpathmoveto{\pgfqpoint{4.440766in}{1.586667in}}%
\pgfpathlineto{\pgfqpoint{4.449519in}{1.586667in}}%
\pgfpathlineto{\pgfqpoint{4.449519in}{1.437287in}}%
\pgfpathlineto{\pgfqpoint{4.440766in}{1.437287in}}%
\pgfpathlineto{\pgfqpoint{4.440766in}{1.586667in}}%
\pgfpathclose%
\pgfusepath{fill}%
\end{pgfscope}%
\begin{pgfscope}%
\pgfpathrectangle{\pgfqpoint{3.776708in}{0.600000in}}{\pgfqpoint{2.573292in}{2.070576in}}%
\pgfusepath{clip}%
\pgfsetbuttcap%
\pgfsetmiterjoin%
\definecolor{currentfill}{rgb}{0.302379,0.450282,0.300122}%
\pgfsetfillcolor{currentfill}%
\pgfsetlinewidth{0.000000pt}%
\definecolor{currentstroke}{rgb}{0.000000,0.000000,0.000000}%
\pgfsetstrokecolor{currentstroke}%
\pgfsetstrokeopacity{0.000000}%
\pgfsetdash{}{0pt}%
\pgfpathmoveto{\pgfqpoint{4.451707in}{1.593226in}}%
\pgfpathlineto{\pgfqpoint{4.460461in}{1.593226in}}%
\pgfpathlineto{\pgfqpoint{4.460461in}{1.453238in}}%
\pgfpathlineto{\pgfqpoint{4.451707in}{1.453238in}}%
\pgfpathlineto{\pgfqpoint{4.451707in}{1.593226in}}%
\pgfpathclose%
\pgfusepath{fill}%
\end{pgfscope}%
\begin{pgfscope}%
\pgfpathrectangle{\pgfqpoint{3.776708in}{0.600000in}}{\pgfqpoint{2.573292in}{2.070576in}}%
\pgfusepath{clip}%
\pgfsetbuttcap%
\pgfsetmiterjoin%
\definecolor{currentfill}{rgb}{0.302379,0.450282,0.300122}%
\pgfsetfillcolor{currentfill}%
\pgfsetlinewidth{0.000000pt}%
\definecolor{currentstroke}{rgb}{0.000000,0.000000,0.000000}%
\pgfsetstrokecolor{currentstroke}%
\pgfsetstrokeopacity{0.000000}%
\pgfsetdash{}{0pt}%
\pgfpathmoveto{\pgfqpoint{4.462649in}{1.602463in}}%
\pgfpathlineto{\pgfqpoint{4.471403in}{1.602463in}}%
\pgfpathlineto{\pgfqpoint{4.471403in}{1.475696in}}%
\pgfpathlineto{\pgfqpoint{4.462649in}{1.475696in}}%
\pgfpathlineto{\pgfqpoint{4.462649in}{1.602463in}}%
\pgfpathclose%
\pgfusepath{fill}%
\end{pgfscope}%
\begin{pgfscope}%
\pgfpathrectangle{\pgfqpoint{3.776708in}{0.600000in}}{\pgfqpoint{2.573292in}{2.070576in}}%
\pgfusepath{clip}%
\pgfsetbuttcap%
\pgfsetmiterjoin%
\definecolor{currentfill}{rgb}{0.302379,0.450282,0.300122}%
\pgfsetfillcolor{currentfill}%
\pgfsetlinewidth{0.000000pt}%
\definecolor{currentstroke}{rgb}{0.000000,0.000000,0.000000}%
\pgfsetstrokecolor{currentstroke}%
\pgfsetstrokeopacity{0.000000}%
\pgfsetdash{}{0pt}%
\pgfpathmoveto{\pgfqpoint{4.473591in}{1.601184in}}%
\pgfpathlineto{\pgfqpoint{4.482344in}{1.601184in}}%
\pgfpathlineto{\pgfqpoint{4.482344in}{1.502021in}}%
\pgfpathlineto{\pgfqpoint{4.473591in}{1.502021in}}%
\pgfpathlineto{\pgfqpoint{4.473591in}{1.601184in}}%
\pgfpathclose%
\pgfusepath{fill}%
\end{pgfscope}%
\begin{pgfscope}%
\pgfpathrectangle{\pgfqpoint{3.776708in}{0.600000in}}{\pgfqpoint{2.573292in}{2.070576in}}%
\pgfusepath{clip}%
\pgfsetbuttcap%
\pgfsetmiterjoin%
\definecolor{currentfill}{rgb}{0.302379,0.450282,0.300122}%
\pgfsetfillcolor{currentfill}%
\pgfsetlinewidth{0.000000pt}%
\definecolor{currentstroke}{rgb}{0.000000,0.000000,0.000000}%
\pgfsetstrokecolor{currentstroke}%
\pgfsetstrokeopacity{0.000000}%
\pgfsetdash{}{0pt}%
\pgfpathmoveto{\pgfqpoint{4.484533in}{1.598017in}}%
\pgfpathlineto{\pgfqpoint{4.493286in}{1.598017in}}%
\pgfpathlineto{\pgfqpoint{4.493286in}{1.537177in}}%
\pgfpathlineto{\pgfqpoint{4.484533in}{1.537177in}}%
\pgfpathlineto{\pgfqpoint{4.484533in}{1.598017in}}%
\pgfpathclose%
\pgfusepath{fill}%
\end{pgfscope}%
\begin{pgfscope}%
\pgfpathrectangle{\pgfqpoint{3.776708in}{0.600000in}}{\pgfqpoint{2.573292in}{2.070576in}}%
\pgfusepath{clip}%
\pgfsetbuttcap%
\pgfsetmiterjoin%
\definecolor{currentfill}{rgb}{0.302379,0.450282,0.300122}%
\pgfsetfillcolor{currentfill}%
\pgfsetlinewidth{0.000000pt}%
\definecolor{currentstroke}{rgb}{0.000000,0.000000,0.000000}%
\pgfsetstrokecolor{currentstroke}%
\pgfsetstrokeopacity{0.000000}%
\pgfsetdash{}{0pt}%
\pgfpathmoveto{\pgfqpoint{4.495475in}{1.592277in}}%
\pgfpathlineto{\pgfqpoint{4.504228in}{1.592277in}}%
\pgfpathlineto{\pgfqpoint{4.504228in}{1.569673in}}%
\pgfpathlineto{\pgfqpoint{4.495475in}{1.569673in}}%
\pgfpathlineto{\pgfqpoint{4.495475in}{1.592277in}}%
\pgfpathclose%
\pgfusepath{fill}%
\end{pgfscope}%
\begin{pgfscope}%
\pgfpathrectangle{\pgfqpoint{3.776708in}{0.600000in}}{\pgfqpoint{2.573292in}{2.070576in}}%
\pgfusepath{clip}%
\pgfsetbuttcap%
\pgfsetmiterjoin%
\definecolor{currentfill}{rgb}{0.302379,0.450282,0.300122}%
\pgfsetfillcolor{currentfill}%
\pgfsetlinewidth{0.000000pt}%
\definecolor{currentstroke}{rgb}{0.000000,0.000000,0.000000}%
\pgfsetstrokecolor{currentstroke}%
\pgfsetstrokeopacity{0.000000}%
\pgfsetdash{}{0pt}%
\pgfpathmoveto{\pgfqpoint{4.506416in}{2.034530in}}%
\pgfpathlineto{\pgfqpoint{4.515170in}{2.034530in}}%
\pgfpathlineto{\pgfqpoint{4.515170in}{2.052481in}}%
\pgfpathlineto{\pgfqpoint{4.506416in}{2.052481in}}%
\pgfpathlineto{\pgfqpoint{4.506416in}{2.034530in}}%
\pgfpathclose%
\pgfusepath{fill}%
\end{pgfscope}%
\begin{pgfscope}%
\pgfpathrectangle{\pgfqpoint{3.776708in}{0.600000in}}{\pgfqpoint{2.573292in}{2.070576in}}%
\pgfusepath{clip}%
\pgfsetbuttcap%
\pgfsetmiterjoin%
\definecolor{currentfill}{rgb}{0.302379,0.450282,0.300122}%
\pgfsetfillcolor{currentfill}%
\pgfsetlinewidth{0.000000pt}%
\definecolor{currentstroke}{rgb}{0.000000,0.000000,0.000000}%
\pgfsetstrokecolor{currentstroke}%
\pgfsetstrokeopacity{0.000000}%
\pgfsetdash{}{0pt}%
\pgfpathmoveto{\pgfqpoint{4.517358in}{2.020154in}}%
\pgfpathlineto{\pgfqpoint{4.526112in}{2.020154in}}%
\pgfpathlineto{\pgfqpoint{4.526112in}{2.037203in}}%
\pgfpathlineto{\pgfqpoint{4.517358in}{2.037203in}}%
\pgfpathlineto{\pgfqpoint{4.517358in}{2.020154in}}%
\pgfpathclose%
\pgfusepath{fill}%
\end{pgfscope}%
\begin{pgfscope}%
\pgfpathrectangle{\pgfqpoint{3.776708in}{0.600000in}}{\pgfqpoint{2.573292in}{2.070576in}}%
\pgfusepath{clip}%
\pgfsetbuttcap%
\pgfsetmiterjoin%
\definecolor{currentfill}{rgb}{0.302379,0.450282,0.300122}%
\pgfsetfillcolor{currentfill}%
\pgfsetlinewidth{0.000000pt}%
\definecolor{currentstroke}{rgb}{0.000000,0.000000,0.000000}%
\pgfsetstrokecolor{currentstroke}%
\pgfsetstrokeopacity{0.000000}%
\pgfsetdash{}{0pt}%
\pgfpathmoveto{\pgfqpoint{4.528300in}{1.994423in}}%
\pgfpathlineto{\pgfqpoint{4.537053in}{1.994423in}}%
\pgfpathlineto{\pgfqpoint{4.537053in}{2.040760in}}%
\pgfpathlineto{\pgfqpoint{4.528300in}{2.040760in}}%
\pgfpathlineto{\pgfqpoint{4.528300in}{1.994423in}}%
\pgfpathclose%
\pgfusepath{fill}%
\end{pgfscope}%
\begin{pgfscope}%
\pgfpathrectangle{\pgfqpoint{3.776708in}{0.600000in}}{\pgfqpoint{2.573292in}{2.070576in}}%
\pgfusepath{clip}%
\pgfsetbuttcap%
\pgfsetmiterjoin%
\definecolor{currentfill}{rgb}{0.302379,0.450282,0.300122}%
\pgfsetfillcolor{currentfill}%
\pgfsetlinewidth{0.000000pt}%
\definecolor{currentstroke}{rgb}{0.000000,0.000000,0.000000}%
\pgfsetstrokecolor{currentstroke}%
\pgfsetstrokeopacity{0.000000}%
\pgfsetdash{}{0pt}%
\pgfpathmoveto{\pgfqpoint{4.539242in}{1.957632in}}%
\pgfpathlineto{\pgfqpoint{4.547995in}{1.957632in}}%
\pgfpathlineto{\pgfqpoint{4.547995in}{2.066454in}}%
\pgfpathlineto{\pgfqpoint{4.539242in}{2.066454in}}%
\pgfpathlineto{\pgfqpoint{4.539242in}{1.957632in}}%
\pgfpathclose%
\pgfusepath{fill}%
\end{pgfscope}%
\begin{pgfscope}%
\pgfpathrectangle{\pgfqpoint{3.776708in}{0.600000in}}{\pgfqpoint{2.573292in}{2.070576in}}%
\pgfusepath{clip}%
\pgfsetbuttcap%
\pgfsetmiterjoin%
\definecolor{currentfill}{rgb}{0.302379,0.450282,0.300122}%
\pgfsetfillcolor{currentfill}%
\pgfsetlinewidth{0.000000pt}%
\definecolor{currentstroke}{rgb}{0.000000,0.000000,0.000000}%
\pgfsetstrokecolor{currentstroke}%
\pgfsetstrokeopacity{0.000000}%
\pgfsetdash{}{0pt}%
\pgfpathmoveto{\pgfqpoint{4.550183in}{1.920819in}}%
\pgfpathlineto{\pgfqpoint{4.558937in}{1.920819in}}%
\pgfpathlineto{\pgfqpoint{4.558937in}{2.081413in}}%
\pgfpathlineto{\pgfqpoint{4.550183in}{2.081413in}}%
\pgfpathlineto{\pgfqpoint{4.550183in}{1.920819in}}%
\pgfpathclose%
\pgfusepath{fill}%
\end{pgfscope}%
\begin{pgfscope}%
\pgfpathrectangle{\pgfqpoint{3.776708in}{0.600000in}}{\pgfqpoint{2.573292in}{2.070576in}}%
\pgfusepath{clip}%
\pgfsetbuttcap%
\pgfsetmiterjoin%
\definecolor{currentfill}{rgb}{0.302379,0.450282,0.300122}%
\pgfsetfillcolor{currentfill}%
\pgfsetlinewidth{0.000000pt}%
\definecolor{currentstroke}{rgb}{0.000000,0.000000,0.000000}%
\pgfsetstrokecolor{currentstroke}%
\pgfsetstrokeopacity{0.000000}%
\pgfsetdash{}{0pt}%
\pgfpathmoveto{\pgfqpoint{4.561125in}{1.885162in}}%
\pgfpathlineto{\pgfqpoint{4.569879in}{1.885162in}}%
\pgfpathlineto{\pgfqpoint{4.569879in}{2.105016in}}%
\pgfpathlineto{\pgfqpoint{4.561125in}{2.105016in}}%
\pgfpathlineto{\pgfqpoint{4.561125in}{1.885162in}}%
\pgfpathclose%
\pgfusepath{fill}%
\end{pgfscope}%
\begin{pgfscope}%
\pgfpathrectangle{\pgfqpoint{3.776708in}{0.600000in}}{\pgfqpoint{2.573292in}{2.070576in}}%
\pgfusepath{clip}%
\pgfsetbuttcap%
\pgfsetmiterjoin%
\definecolor{currentfill}{rgb}{0.302379,0.450282,0.300122}%
\pgfsetfillcolor{currentfill}%
\pgfsetlinewidth{0.000000pt}%
\definecolor{currentstroke}{rgb}{0.000000,0.000000,0.000000}%
\pgfsetstrokecolor{currentstroke}%
\pgfsetstrokeopacity{0.000000}%
\pgfsetdash{}{0pt}%
\pgfpathmoveto{\pgfqpoint{4.572067in}{1.853247in}}%
\pgfpathlineto{\pgfqpoint{4.580821in}{1.853247in}}%
\pgfpathlineto{\pgfqpoint{4.580821in}{2.112539in}}%
\pgfpathlineto{\pgfqpoint{4.572067in}{2.112539in}}%
\pgfpathlineto{\pgfqpoint{4.572067in}{1.853247in}}%
\pgfpathclose%
\pgfusepath{fill}%
\end{pgfscope}%
\begin{pgfscope}%
\pgfpathrectangle{\pgfqpoint{3.776708in}{0.600000in}}{\pgfqpoint{2.573292in}{2.070576in}}%
\pgfusepath{clip}%
\pgfsetbuttcap%
\pgfsetmiterjoin%
\definecolor{currentfill}{rgb}{0.302379,0.450282,0.300122}%
\pgfsetfillcolor{currentfill}%
\pgfsetlinewidth{0.000000pt}%
\definecolor{currentstroke}{rgb}{0.000000,0.000000,0.000000}%
\pgfsetstrokecolor{currentstroke}%
\pgfsetstrokeopacity{0.000000}%
\pgfsetdash{}{0pt}%
\pgfpathmoveto{\pgfqpoint{4.583009in}{1.820957in}}%
\pgfpathlineto{\pgfqpoint{4.591762in}{1.820957in}}%
\pgfpathlineto{\pgfqpoint{4.591762in}{2.114793in}}%
\pgfpathlineto{\pgfqpoint{4.583009in}{2.114793in}}%
\pgfpathlineto{\pgfqpoint{4.583009in}{1.820957in}}%
\pgfpathclose%
\pgfusepath{fill}%
\end{pgfscope}%
\begin{pgfscope}%
\pgfpathrectangle{\pgfqpoint{3.776708in}{0.600000in}}{\pgfqpoint{2.573292in}{2.070576in}}%
\pgfusepath{clip}%
\pgfsetbuttcap%
\pgfsetmiterjoin%
\definecolor{currentfill}{rgb}{0.302379,0.450282,0.300122}%
\pgfsetfillcolor{currentfill}%
\pgfsetlinewidth{0.000000pt}%
\definecolor{currentstroke}{rgb}{0.000000,0.000000,0.000000}%
\pgfsetstrokecolor{currentstroke}%
\pgfsetstrokeopacity{0.000000}%
\pgfsetdash{}{0pt}%
\pgfpathmoveto{\pgfqpoint{4.593951in}{1.791671in}}%
\pgfpathlineto{\pgfqpoint{4.602704in}{1.791671in}}%
\pgfpathlineto{\pgfqpoint{4.602704in}{2.118748in}}%
\pgfpathlineto{\pgfqpoint{4.593951in}{2.118748in}}%
\pgfpathlineto{\pgfqpoint{4.593951in}{1.791671in}}%
\pgfpathclose%
\pgfusepath{fill}%
\end{pgfscope}%
\begin{pgfscope}%
\pgfpathrectangle{\pgfqpoint{3.776708in}{0.600000in}}{\pgfqpoint{2.573292in}{2.070576in}}%
\pgfusepath{clip}%
\pgfsetbuttcap%
\pgfsetmiterjoin%
\definecolor{currentfill}{rgb}{0.302379,0.450282,0.300122}%
\pgfsetfillcolor{currentfill}%
\pgfsetlinewidth{0.000000pt}%
\definecolor{currentstroke}{rgb}{0.000000,0.000000,0.000000}%
\pgfsetstrokecolor{currentstroke}%
\pgfsetstrokeopacity{0.000000}%
\pgfsetdash{}{0pt}%
\pgfpathmoveto{\pgfqpoint{4.604892in}{1.764991in}}%
\pgfpathlineto{\pgfqpoint{4.613646in}{1.764991in}}%
\pgfpathlineto{\pgfqpoint{4.613646in}{2.114798in}}%
\pgfpathlineto{\pgfqpoint{4.604892in}{2.114798in}}%
\pgfpathlineto{\pgfqpoint{4.604892in}{1.764991in}}%
\pgfpathclose%
\pgfusepath{fill}%
\end{pgfscope}%
\begin{pgfscope}%
\pgfpathrectangle{\pgfqpoint{3.776708in}{0.600000in}}{\pgfqpoint{2.573292in}{2.070576in}}%
\pgfusepath{clip}%
\pgfsetbuttcap%
\pgfsetmiterjoin%
\definecolor{currentfill}{rgb}{0.302379,0.450282,0.300122}%
\pgfsetfillcolor{currentfill}%
\pgfsetlinewidth{0.000000pt}%
\definecolor{currentstroke}{rgb}{0.000000,0.000000,0.000000}%
\pgfsetstrokecolor{currentstroke}%
\pgfsetstrokeopacity{0.000000}%
\pgfsetdash{}{0pt}%
\pgfpathmoveto{\pgfqpoint{4.615834in}{1.736697in}}%
\pgfpathlineto{\pgfqpoint{4.624588in}{1.736697in}}%
\pgfpathlineto{\pgfqpoint{4.624588in}{2.097089in}}%
\pgfpathlineto{\pgfqpoint{4.615834in}{2.097089in}}%
\pgfpathlineto{\pgfqpoint{4.615834in}{1.736697in}}%
\pgfpathclose%
\pgfusepath{fill}%
\end{pgfscope}%
\begin{pgfscope}%
\pgfpathrectangle{\pgfqpoint{3.776708in}{0.600000in}}{\pgfqpoint{2.573292in}{2.070576in}}%
\pgfusepath{clip}%
\pgfsetbuttcap%
\pgfsetmiterjoin%
\definecolor{currentfill}{rgb}{0.302379,0.450282,0.300122}%
\pgfsetfillcolor{currentfill}%
\pgfsetlinewidth{0.000000pt}%
\definecolor{currentstroke}{rgb}{0.000000,0.000000,0.000000}%
\pgfsetstrokecolor{currentstroke}%
\pgfsetstrokeopacity{0.000000}%
\pgfsetdash{}{0pt}%
\pgfpathmoveto{\pgfqpoint{4.626776in}{1.713901in}}%
\pgfpathlineto{\pgfqpoint{4.635530in}{1.713901in}}%
\pgfpathlineto{\pgfqpoint{4.635530in}{2.080254in}}%
\pgfpathlineto{\pgfqpoint{4.626776in}{2.080254in}}%
\pgfpathlineto{\pgfqpoint{4.626776in}{1.713901in}}%
\pgfpathclose%
\pgfusepath{fill}%
\end{pgfscope}%
\begin{pgfscope}%
\pgfpathrectangle{\pgfqpoint{3.776708in}{0.600000in}}{\pgfqpoint{2.573292in}{2.070576in}}%
\pgfusepath{clip}%
\pgfsetbuttcap%
\pgfsetmiterjoin%
\definecolor{currentfill}{rgb}{0.302379,0.450282,0.300122}%
\pgfsetfillcolor{currentfill}%
\pgfsetlinewidth{0.000000pt}%
\definecolor{currentstroke}{rgb}{0.000000,0.000000,0.000000}%
\pgfsetstrokecolor{currentstroke}%
\pgfsetstrokeopacity{0.000000}%
\pgfsetdash{}{0pt}%
\pgfpathmoveto{\pgfqpoint{4.637718in}{1.693678in}}%
\pgfpathlineto{\pgfqpoint{4.646471in}{1.693678in}}%
\pgfpathlineto{\pgfqpoint{4.646471in}{2.062162in}}%
\pgfpathlineto{\pgfqpoint{4.637718in}{2.062162in}}%
\pgfpathlineto{\pgfqpoint{4.637718in}{1.693678in}}%
\pgfpathclose%
\pgfusepath{fill}%
\end{pgfscope}%
\begin{pgfscope}%
\pgfpathrectangle{\pgfqpoint{3.776708in}{0.600000in}}{\pgfqpoint{2.573292in}{2.070576in}}%
\pgfusepath{clip}%
\pgfsetbuttcap%
\pgfsetmiterjoin%
\definecolor{currentfill}{rgb}{0.302379,0.450282,0.300122}%
\pgfsetfillcolor{currentfill}%
\pgfsetlinewidth{0.000000pt}%
\definecolor{currentstroke}{rgb}{0.000000,0.000000,0.000000}%
\pgfsetstrokecolor{currentstroke}%
\pgfsetstrokeopacity{0.000000}%
\pgfsetdash{}{0pt}%
\pgfpathmoveto{\pgfqpoint{4.648660in}{1.668263in}}%
\pgfpathlineto{\pgfqpoint{4.657413in}{1.668263in}}%
\pgfpathlineto{\pgfqpoint{4.657413in}{2.047308in}}%
\pgfpathlineto{\pgfqpoint{4.648660in}{2.047308in}}%
\pgfpathlineto{\pgfqpoint{4.648660in}{1.668263in}}%
\pgfpathclose%
\pgfusepath{fill}%
\end{pgfscope}%
\begin{pgfscope}%
\pgfpathrectangle{\pgfqpoint{3.776708in}{0.600000in}}{\pgfqpoint{2.573292in}{2.070576in}}%
\pgfusepath{clip}%
\pgfsetbuttcap%
\pgfsetmiterjoin%
\definecolor{currentfill}{rgb}{0.302379,0.450282,0.300122}%
\pgfsetfillcolor{currentfill}%
\pgfsetlinewidth{0.000000pt}%
\definecolor{currentstroke}{rgb}{0.000000,0.000000,0.000000}%
\pgfsetstrokecolor{currentstroke}%
\pgfsetstrokeopacity{0.000000}%
\pgfsetdash{}{0pt}%
\pgfpathmoveto{\pgfqpoint{4.659601in}{1.648504in}}%
\pgfpathlineto{\pgfqpoint{4.668355in}{1.648504in}}%
\pgfpathlineto{\pgfqpoint{4.668355in}{2.034183in}}%
\pgfpathlineto{\pgfqpoint{4.659601in}{2.034183in}}%
\pgfpathlineto{\pgfqpoint{4.659601in}{1.648504in}}%
\pgfpathclose%
\pgfusepath{fill}%
\end{pgfscope}%
\begin{pgfscope}%
\pgfpathrectangle{\pgfqpoint{3.776708in}{0.600000in}}{\pgfqpoint{2.573292in}{2.070576in}}%
\pgfusepath{clip}%
\pgfsetbuttcap%
\pgfsetmiterjoin%
\definecolor{currentfill}{rgb}{0.302379,0.450282,0.300122}%
\pgfsetfillcolor{currentfill}%
\pgfsetlinewidth{0.000000pt}%
\definecolor{currentstroke}{rgb}{0.000000,0.000000,0.000000}%
\pgfsetstrokecolor{currentstroke}%
\pgfsetstrokeopacity{0.000000}%
\pgfsetdash{}{0pt}%
\pgfpathmoveto{\pgfqpoint{4.670543in}{1.624297in}}%
\pgfpathlineto{\pgfqpoint{4.679297in}{1.624297in}}%
\pgfpathlineto{\pgfqpoint{4.679297in}{2.021203in}}%
\pgfpathlineto{\pgfqpoint{4.670543in}{2.021203in}}%
\pgfpathlineto{\pgfqpoint{4.670543in}{1.624297in}}%
\pgfpathclose%
\pgfusepath{fill}%
\end{pgfscope}%
\begin{pgfscope}%
\pgfpathrectangle{\pgfqpoint{3.776708in}{0.600000in}}{\pgfqpoint{2.573292in}{2.070576in}}%
\pgfusepath{clip}%
\pgfsetbuttcap%
\pgfsetmiterjoin%
\definecolor{currentfill}{rgb}{0.302379,0.450282,0.300122}%
\pgfsetfillcolor{currentfill}%
\pgfsetlinewidth{0.000000pt}%
\definecolor{currentstroke}{rgb}{0.000000,0.000000,0.000000}%
\pgfsetstrokecolor{currentstroke}%
\pgfsetstrokeopacity{0.000000}%
\pgfsetdash{}{0pt}%
\pgfpathmoveto{\pgfqpoint{4.681485in}{1.609196in}}%
\pgfpathlineto{\pgfqpoint{4.690239in}{1.609196in}}%
\pgfpathlineto{\pgfqpoint{4.690239in}{2.021834in}}%
\pgfpathlineto{\pgfqpoint{4.681485in}{2.021834in}}%
\pgfpathlineto{\pgfqpoint{4.681485in}{1.609196in}}%
\pgfpathclose%
\pgfusepath{fill}%
\end{pgfscope}%
\begin{pgfscope}%
\pgfpathrectangle{\pgfqpoint{3.776708in}{0.600000in}}{\pgfqpoint{2.573292in}{2.070576in}}%
\pgfusepath{clip}%
\pgfsetbuttcap%
\pgfsetmiterjoin%
\definecolor{currentfill}{rgb}{0.302379,0.450282,0.300122}%
\pgfsetfillcolor{currentfill}%
\pgfsetlinewidth{0.000000pt}%
\definecolor{currentstroke}{rgb}{0.000000,0.000000,0.000000}%
\pgfsetstrokecolor{currentstroke}%
\pgfsetstrokeopacity{0.000000}%
\pgfsetdash{}{0pt}%
\pgfpathmoveto{\pgfqpoint{4.692427in}{1.609196in}}%
\pgfpathlineto{\pgfqpoint{4.701180in}{1.609196in}}%
\pgfpathlineto{\pgfqpoint{4.701180in}{2.041676in}}%
\pgfpathlineto{\pgfqpoint{4.692427in}{2.041676in}}%
\pgfpathlineto{\pgfqpoint{4.692427in}{1.609196in}}%
\pgfpathclose%
\pgfusepath{fill}%
\end{pgfscope}%
\begin{pgfscope}%
\pgfpathrectangle{\pgfqpoint{3.776708in}{0.600000in}}{\pgfqpoint{2.573292in}{2.070576in}}%
\pgfusepath{clip}%
\pgfsetbuttcap%
\pgfsetmiterjoin%
\definecolor{currentfill}{rgb}{0.302379,0.450282,0.300122}%
\pgfsetfillcolor{currentfill}%
\pgfsetlinewidth{0.000000pt}%
\definecolor{currentstroke}{rgb}{0.000000,0.000000,0.000000}%
\pgfsetstrokecolor{currentstroke}%
\pgfsetstrokeopacity{0.000000}%
\pgfsetdash{}{0pt}%
\pgfpathmoveto{\pgfqpoint{4.703369in}{1.609196in}}%
\pgfpathlineto{\pgfqpoint{4.712122in}{1.609196in}}%
\pgfpathlineto{\pgfqpoint{4.712122in}{2.048226in}}%
\pgfpathlineto{\pgfqpoint{4.703369in}{2.048226in}}%
\pgfpathlineto{\pgfqpoint{4.703369in}{1.609196in}}%
\pgfpathclose%
\pgfusepath{fill}%
\end{pgfscope}%
\begin{pgfscope}%
\pgfpathrectangle{\pgfqpoint{3.776708in}{0.600000in}}{\pgfqpoint{2.573292in}{2.070576in}}%
\pgfusepath{clip}%
\pgfsetbuttcap%
\pgfsetmiterjoin%
\definecolor{currentfill}{rgb}{0.302379,0.450282,0.300122}%
\pgfsetfillcolor{currentfill}%
\pgfsetlinewidth{0.000000pt}%
\definecolor{currentstroke}{rgb}{0.000000,0.000000,0.000000}%
\pgfsetstrokecolor{currentstroke}%
\pgfsetstrokeopacity{0.000000}%
\pgfsetdash{}{0pt}%
\pgfpathmoveto{\pgfqpoint{4.714310in}{1.609196in}}%
\pgfpathlineto{\pgfqpoint{4.723064in}{1.609196in}}%
\pgfpathlineto{\pgfqpoint{4.723064in}{2.052234in}}%
\pgfpathlineto{\pgfqpoint{4.714310in}{2.052234in}}%
\pgfpathlineto{\pgfqpoint{4.714310in}{1.609196in}}%
\pgfpathclose%
\pgfusepath{fill}%
\end{pgfscope}%
\begin{pgfscope}%
\pgfpathrectangle{\pgfqpoint{3.776708in}{0.600000in}}{\pgfqpoint{2.573292in}{2.070576in}}%
\pgfusepath{clip}%
\pgfsetbuttcap%
\pgfsetmiterjoin%
\definecolor{currentfill}{rgb}{0.302379,0.450282,0.300122}%
\pgfsetfillcolor{currentfill}%
\pgfsetlinewidth{0.000000pt}%
\definecolor{currentstroke}{rgb}{0.000000,0.000000,0.000000}%
\pgfsetstrokecolor{currentstroke}%
\pgfsetstrokeopacity{0.000000}%
\pgfsetdash{}{0pt}%
\pgfpathmoveto{\pgfqpoint{4.725252in}{1.609196in}}%
\pgfpathlineto{\pgfqpoint{4.734006in}{1.609196in}}%
\pgfpathlineto{\pgfqpoint{4.734006in}{2.056481in}}%
\pgfpathlineto{\pgfqpoint{4.725252in}{2.056481in}}%
\pgfpathlineto{\pgfqpoint{4.725252in}{1.609196in}}%
\pgfpathclose%
\pgfusepath{fill}%
\end{pgfscope}%
\begin{pgfscope}%
\pgfpathrectangle{\pgfqpoint{3.776708in}{0.600000in}}{\pgfqpoint{2.573292in}{2.070576in}}%
\pgfusepath{clip}%
\pgfsetbuttcap%
\pgfsetmiterjoin%
\definecolor{currentfill}{rgb}{0.302379,0.450282,0.300122}%
\pgfsetfillcolor{currentfill}%
\pgfsetlinewidth{0.000000pt}%
\definecolor{currentstroke}{rgb}{0.000000,0.000000,0.000000}%
\pgfsetstrokecolor{currentstroke}%
\pgfsetstrokeopacity{0.000000}%
\pgfsetdash{}{0pt}%
\pgfpathmoveto{\pgfqpoint{4.736194in}{1.609196in}}%
\pgfpathlineto{\pgfqpoint{4.744948in}{1.609196in}}%
\pgfpathlineto{\pgfqpoint{4.744948in}{2.059297in}}%
\pgfpathlineto{\pgfqpoint{4.736194in}{2.059297in}}%
\pgfpathlineto{\pgfqpoint{4.736194in}{1.609196in}}%
\pgfpathclose%
\pgfusepath{fill}%
\end{pgfscope}%
\begin{pgfscope}%
\pgfpathrectangle{\pgfqpoint{3.776708in}{0.600000in}}{\pgfqpoint{2.573292in}{2.070576in}}%
\pgfusepath{clip}%
\pgfsetbuttcap%
\pgfsetmiterjoin%
\definecolor{currentfill}{rgb}{0.302379,0.450282,0.300122}%
\pgfsetfillcolor{currentfill}%
\pgfsetlinewidth{0.000000pt}%
\definecolor{currentstroke}{rgb}{0.000000,0.000000,0.000000}%
\pgfsetstrokecolor{currentstroke}%
\pgfsetstrokeopacity{0.000000}%
\pgfsetdash{}{0pt}%
\pgfpathmoveto{\pgfqpoint{4.747136in}{1.609196in}}%
\pgfpathlineto{\pgfqpoint{4.755889in}{1.609196in}}%
\pgfpathlineto{\pgfqpoint{4.755889in}{2.064724in}}%
\pgfpathlineto{\pgfqpoint{4.747136in}{2.064724in}}%
\pgfpathlineto{\pgfqpoint{4.747136in}{1.609196in}}%
\pgfpathclose%
\pgfusepath{fill}%
\end{pgfscope}%
\begin{pgfscope}%
\pgfpathrectangle{\pgfqpoint{3.776708in}{0.600000in}}{\pgfqpoint{2.573292in}{2.070576in}}%
\pgfusepath{clip}%
\pgfsetbuttcap%
\pgfsetmiterjoin%
\definecolor{currentfill}{rgb}{0.302379,0.450282,0.300122}%
\pgfsetfillcolor{currentfill}%
\pgfsetlinewidth{0.000000pt}%
\definecolor{currentstroke}{rgb}{0.000000,0.000000,0.000000}%
\pgfsetstrokecolor{currentstroke}%
\pgfsetstrokeopacity{0.000000}%
\pgfsetdash{}{0pt}%
\pgfpathmoveto{\pgfqpoint{4.758078in}{1.609196in}}%
\pgfpathlineto{\pgfqpoint{4.766831in}{1.609196in}}%
\pgfpathlineto{\pgfqpoint{4.766831in}{2.070628in}}%
\pgfpathlineto{\pgfqpoint{4.758078in}{2.070628in}}%
\pgfpathlineto{\pgfqpoint{4.758078in}{1.609196in}}%
\pgfpathclose%
\pgfusepath{fill}%
\end{pgfscope}%
\begin{pgfscope}%
\pgfpathrectangle{\pgfqpoint{3.776708in}{0.600000in}}{\pgfqpoint{2.573292in}{2.070576in}}%
\pgfusepath{clip}%
\pgfsetbuttcap%
\pgfsetmiterjoin%
\definecolor{currentfill}{rgb}{0.302379,0.450282,0.300122}%
\pgfsetfillcolor{currentfill}%
\pgfsetlinewidth{0.000000pt}%
\definecolor{currentstroke}{rgb}{0.000000,0.000000,0.000000}%
\pgfsetstrokecolor{currentstroke}%
\pgfsetstrokeopacity{0.000000}%
\pgfsetdash{}{0pt}%
\pgfpathmoveto{\pgfqpoint{4.769019in}{1.609196in}}%
\pgfpathlineto{\pgfqpoint{4.777773in}{1.609196in}}%
\pgfpathlineto{\pgfqpoint{4.777773in}{2.069824in}}%
\pgfpathlineto{\pgfqpoint{4.769019in}{2.069824in}}%
\pgfpathlineto{\pgfqpoint{4.769019in}{1.609196in}}%
\pgfpathclose%
\pgfusepath{fill}%
\end{pgfscope}%
\begin{pgfscope}%
\pgfpathrectangle{\pgfqpoint{3.776708in}{0.600000in}}{\pgfqpoint{2.573292in}{2.070576in}}%
\pgfusepath{clip}%
\pgfsetbuttcap%
\pgfsetmiterjoin%
\definecolor{currentfill}{rgb}{0.302379,0.450282,0.300122}%
\pgfsetfillcolor{currentfill}%
\pgfsetlinewidth{0.000000pt}%
\definecolor{currentstroke}{rgb}{0.000000,0.000000,0.000000}%
\pgfsetstrokecolor{currentstroke}%
\pgfsetstrokeopacity{0.000000}%
\pgfsetdash{}{0pt}%
\pgfpathmoveto{\pgfqpoint{4.779961in}{1.609196in}}%
\pgfpathlineto{\pgfqpoint{4.788715in}{1.609196in}}%
\pgfpathlineto{\pgfqpoint{4.788715in}{2.063144in}}%
\pgfpathlineto{\pgfqpoint{4.779961in}{2.063144in}}%
\pgfpathlineto{\pgfqpoint{4.779961in}{1.609196in}}%
\pgfpathclose%
\pgfusepath{fill}%
\end{pgfscope}%
\begin{pgfscope}%
\pgfpathrectangle{\pgfqpoint{3.776708in}{0.600000in}}{\pgfqpoint{2.573292in}{2.070576in}}%
\pgfusepath{clip}%
\pgfsetbuttcap%
\pgfsetmiterjoin%
\definecolor{currentfill}{rgb}{0.302379,0.450282,0.300122}%
\pgfsetfillcolor{currentfill}%
\pgfsetlinewidth{0.000000pt}%
\definecolor{currentstroke}{rgb}{0.000000,0.000000,0.000000}%
\pgfsetstrokecolor{currentstroke}%
\pgfsetstrokeopacity{0.000000}%
\pgfsetdash{}{0pt}%
\pgfpathmoveto{\pgfqpoint{4.790903in}{1.609196in}}%
\pgfpathlineto{\pgfqpoint{4.799657in}{1.609196in}}%
\pgfpathlineto{\pgfqpoint{4.799657in}{2.052978in}}%
\pgfpathlineto{\pgfqpoint{4.790903in}{2.052978in}}%
\pgfpathlineto{\pgfqpoint{4.790903in}{1.609196in}}%
\pgfpathclose%
\pgfusepath{fill}%
\end{pgfscope}%
\begin{pgfscope}%
\pgfpathrectangle{\pgfqpoint{3.776708in}{0.600000in}}{\pgfqpoint{2.573292in}{2.070576in}}%
\pgfusepath{clip}%
\pgfsetbuttcap%
\pgfsetmiterjoin%
\definecolor{currentfill}{rgb}{0.302379,0.450282,0.300122}%
\pgfsetfillcolor{currentfill}%
\pgfsetlinewidth{0.000000pt}%
\definecolor{currentstroke}{rgb}{0.000000,0.000000,0.000000}%
\pgfsetstrokecolor{currentstroke}%
\pgfsetstrokeopacity{0.000000}%
\pgfsetdash{}{0pt}%
\pgfpathmoveto{\pgfqpoint{4.801845in}{1.609196in}}%
\pgfpathlineto{\pgfqpoint{4.810598in}{1.609196in}}%
\pgfpathlineto{\pgfqpoint{4.810598in}{2.045440in}}%
\pgfpathlineto{\pgfqpoint{4.801845in}{2.045440in}}%
\pgfpathlineto{\pgfqpoint{4.801845in}{1.609196in}}%
\pgfpathclose%
\pgfusepath{fill}%
\end{pgfscope}%
\begin{pgfscope}%
\pgfpathrectangle{\pgfqpoint{3.776708in}{0.600000in}}{\pgfqpoint{2.573292in}{2.070576in}}%
\pgfusepath{clip}%
\pgfsetbuttcap%
\pgfsetmiterjoin%
\definecolor{currentfill}{rgb}{0.302379,0.450282,0.300122}%
\pgfsetfillcolor{currentfill}%
\pgfsetlinewidth{0.000000pt}%
\definecolor{currentstroke}{rgb}{0.000000,0.000000,0.000000}%
\pgfsetstrokecolor{currentstroke}%
\pgfsetstrokeopacity{0.000000}%
\pgfsetdash{}{0pt}%
\pgfpathmoveto{\pgfqpoint{4.812787in}{1.609196in}}%
\pgfpathlineto{\pgfqpoint{4.821540in}{1.609196in}}%
\pgfpathlineto{\pgfqpoint{4.821540in}{2.038075in}}%
\pgfpathlineto{\pgfqpoint{4.812787in}{2.038075in}}%
\pgfpathlineto{\pgfqpoint{4.812787in}{1.609196in}}%
\pgfpathclose%
\pgfusepath{fill}%
\end{pgfscope}%
\begin{pgfscope}%
\pgfpathrectangle{\pgfqpoint{3.776708in}{0.600000in}}{\pgfqpoint{2.573292in}{2.070576in}}%
\pgfusepath{clip}%
\pgfsetbuttcap%
\pgfsetmiterjoin%
\definecolor{currentfill}{rgb}{0.302379,0.450282,0.300122}%
\pgfsetfillcolor{currentfill}%
\pgfsetlinewidth{0.000000pt}%
\definecolor{currentstroke}{rgb}{0.000000,0.000000,0.000000}%
\pgfsetstrokecolor{currentstroke}%
\pgfsetstrokeopacity{0.000000}%
\pgfsetdash{}{0pt}%
\pgfpathmoveto{\pgfqpoint{4.823728in}{1.609196in}}%
\pgfpathlineto{\pgfqpoint{4.832482in}{1.609196in}}%
\pgfpathlineto{\pgfqpoint{4.832482in}{2.032216in}}%
\pgfpathlineto{\pgfqpoint{4.823728in}{2.032216in}}%
\pgfpathlineto{\pgfqpoint{4.823728in}{1.609196in}}%
\pgfpathclose%
\pgfusepath{fill}%
\end{pgfscope}%
\begin{pgfscope}%
\pgfpathrectangle{\pgfqpoint{3.776708in}{0.600000in}}{\pgfqpoint{2.573292in}{2.070576in}}%
\pgfusepath{clip}%
\pgfsetbuttcap%
\pgfsetmiterjoin%
\definecolor{currentfill}{rgb}{0.302379,0.450282,0.300122}%
\pgfsetfillcolor{currentfill}%
\pgfsetlinewidth{0.000000pt}%
\definecolor{currentstroke}{rgb}{0.000000,0.000000,0.000000}%
\pgfsetstrokecolor{currentstroke}%
\pgfsetstrokeopacity{0.000000}%
\pgfsetdash{}{0pt}%
\pgfpathmoveto{\pgfqpoint{4.834670in}{1.609196in}}%
\pgfpathlineto{\pgfqpoint{4.843424in}{1.609196in}}%
\pgfpathlineto{\pgfqpoint{4.843424in}{2.029998in}}%
\pgfpathlineto{\pgfqpoint{4.834670in}{2.029998in}}%
\pgfpathlineto{\pgfqpoint{4.834670in}{1.609196in}}%
\pgfpathclose%
\pgfusepath{fill}%
\end{pgfscope}%
\begin{pgfscope}%
\pgfpathrectangle{\pgfqpoint{3.776708in}{0.600000in}}{\pgfqpoint{2.573292in}{2.070576in}}%
\pgfusepath{clip}%
\pgfsetbuttcap%
\pgfsetmiterjoin%
\definecolor{currentfill}{rgb}{0.302379,0.450282,0.300122}%
\pgfsetfillcolor{currentfill}%
\pgfsetlinewidth{0.000000pt}%
\definecolor{currentstroke}{rgb}{0.000000,0.000000,0.000000}%
\pgfsetstrokecolor{currentstroke}%
\pgfsetstrokeopacity{0.000000}%
\pgfsetdash{}{0pt}%
\pgfpathmoveto{\pgfqpoint{4.845612in}{1.609196in}}%
\pgfpathlineto{\pgfqpoint{4.854366in}{1.609196in}}%
\pgfpathlineto{\pgfqpoint{4.854366in}{2.022791in}}%
\pgfpathlineto{\pgfqpoint{4.845612in}{2.022791in}}%
\pgfpathlineto{\pgfqpoint{4.845612in}{1.609196in}}%
\pgfpathclose%
\pgfusepath{fill}%
\end{pgfscope}%
\begin{pgfscope}%
\pgfpathrectangle{\pgfqpoint{3.776708in}{0.600000in}}{\pgfqpoint{2.573292in}{2.070576in}}%
\pgfusepath{clip}%
\pgfsetbuttcap%
\pgfsetmiterjoin%
\definecolor{currentfill}{rgb}{0.302379,0.450282,0.300122}%
\pgfsetfillcolor{currentfill}%
\pgfsetlinewidth{0.000000pt}%
\definecolor{currentstroke}{rgb}{0.000000,0.000000,0.000000}%
\pgfsetstrokecolor{currentstroke}%
\pgfsetstrokeopacity{0.000000}%
\pgfsetdash{}{0pt}%
\pgfpathmoveto{\pgfqpoint{4.856554in}{1.609196in}}%
\pgfpathlineto{\pgfqpoint{4.865307in}{1.609196in}}%
\pgfpathlineto{\pgfqpoint{4.865307in}{2.014374in}}%
\pgfpathlineto{\pgfqpoint{4.856554in}{2.014374in}}%
\pgfpathlineto{\pgfqpoint{4.856554in}{1.609196in}}%
\pgfpathclose%
\pgfusepath{fill}%
\end{pgfscope}%
\begin{pgfscope}%
\pgfpathrectangle{\pgfqpoint{3.776708in}{0.600000in}}{\pgfqpoint{2.573292in}{2.070576in}}%
\pgfusepath{clip}%
\pgfsetbuttcap%
\pgfsetmiterjoin%
\definecolor{currentfill}{rgb}{0.302379,0.450282,0.300122}%
\pgfsetfillcolor{currentfill}%
\pgfsetlinewidth{0.000000pt}%
\definecolor{currentstroke}{rgb}{0.000000,0.000000,0.000000}%
\pgfsetstrokecolor{currentstroke}%
\pgfsetstrokeopacity{0.000000}%
\pgfsetdash{}{0pt}%
\pgfpathmoveto{\pgfqpoint{4.867496in}{1.609196in}}%
\pgfpathlineto{\pgfqpoint{4.876249in}{1.609196in}}%
\pgfpathlineto{\pgfqpoint{4.876249in}{2.017129in}}%
\pgfpathlineto{\pgfqpoint{4.867496in}{2.017129in}}%
\pgfpathlineto{\pgfqpoint{4.867496in}{1.609196in}}%
\pgfpathclose%
\pgfusepath{fill}%
\end{pgfscope}%
\begin{pgfscope}%
\pgfpathrectangle{\pgfqpoint{3.776708in}{0.600000in}}{\pgfqpoint{2.573292in}{2.070576in}}%
\pgfusepath{clip}%
\pgfsetbuttcap%
\pgfsetmiterjoin%
\definecolor{currentfill}{rgb}{0.302379,0.450282,0.300122}%
\pgfsetfillcolor{currentfill}%
\pgfsetlinewidth{0.000000pt}%
\definecolor{currentstroke}{rgb}{0.000000,0.000000,0.000000}%
\pgfsetstrokecolor{currentstroke}%
\pgfsetstrokeopacity{0.000000}%
\pgfsetdash{}{0pt}%
\pgfpathmoveto{\pgfqpoint{4.878437in}{1.609196in}}%
\pgfpathlineto{\pgfqpoint{4.887191in}{1.609196in}}%
\pgfpathlineto{\pgfqpoint{4.887191in}{2.021216in}}%
\pgfpathlineto{\pgfqpoint{4.878437in}{2.021216in}}%
\pgfpathlineto{\pgfqpoint{4.878437in}{1.609196in}}%
\pgfpathclose%
\pgfusepath{fill}%
\end{pgfscope}%
\begin{pgfscope}%
\pgfpathrectangle{\pgfqpoint{3.776708in}{0.600000in}}{\pgfqpoint{2.573292in}{2.070576in}}%
\pgfusepath{clip}%
\pgfsetbuttcap%
\pgfsetmiterjoin%
\definecolor{currentfill}{rgb}{0.302379,0.450282,0.300122}%
\pgfsetfillcolor{currentfill}%
\pgfsetlinewidth{0.000000pt}%
\definecolor{currentstroke}{rgb}{0.000000,0.000000,0.000000}%
\pgfsetstrokecolor{currentstroke}%
\pgfsetstrokeopacity{0.000000}%
\pgfsetdash{}{0pt}%
\pgfpathmoveto{\pgfqpoint{4.889379in}{1.613020in}}%
\pgfpathlineto{\pgfqpoint{4.898133in}{1.613020in}}%
\pgfpathlineto{\pgfqpoint{4.898133in}{2.033920in}}%
\pgfpathlineto{\pgfqpoint{4.889379in}{2.033920in}}%
\pgfpathlineto{\pgfqpoint{4.889379in}{1.613020in}}%
\pgfpathclose%
\pgfusepath{fill}%
\end{pgfscope}%
\begin{pgfscope}%
\pgfpathrectangle{\pgfqpoint{3.776708in}{0.600000in}}{\pgfqpoint{2.573292in}{2.070576in}}%
\pgfusepath{clip}%
\pgfsetbuttcap%
\pgfsetmiterjoin%
\definecolor{currentfill}{rgb}{0.302379,0.450282,0.300122}%
\pgfsetfillcolor{currentfill}%
\pgfsetlinewidth{0.000000pt}%
\definecolor{currentstroke}{rgb}{0.000000,0.000000,0.000000}%
\pgfsetstrokecolor{currentstroke}%
\pgfsetstrokeopacity{0.000000}%
\pgfsetdash{}{0pt}%
\pgfpathmoveto{\pgfqpoint{4.900321in}{1.616469in}}%
\pgfpathlineto{\pgfqpoint{4.909075in}{1.616469in}}%
\pgfpathlineto{\pgfqpoint{4.909075in}{2.054115in}}%
\pgfpathlineto{\pgfqpoint{4.900321in}{2.054115in}}%
\pgfpathlineto{\pgfqpoint{4.900321in}{1.616469in}}%
\pgfpathclose%
\pgfusepath{fill}%
\end{pgfscope}%
\begin{pgfscope}%
\pgfpathrectangle{\pgfqpoint{3.776708in}{0.600000in}}{\pgfqpoint{2.573292in}{2.070576in}}%
\pgfusepath{clip}%
\pgfsetbuttcap%
\pgfsetmiterjoin%
\definecolor{currentfill}{rgb}{0.302379,0.450282,0.300122}%
\pgfsetfillcolor{currentfill}%
\pgfsetlinewidth{0.000000pt}%
\definecolor{currentstroke}{rgb}{0.000000,0.000000,0.000000}%
\pgfsetstrokecolor{currentstroke}%
\pgfsetstrokeopacity{0.000000}%
\pgfsetdash{}{0pt}%
\pgfpathmoveto{\pgfqpoint{4.911263in}{1.620128in}}%
\pgfpathlineto{\pgfqpoint{4.920016in}{1.620128in}}%
\pgfpathlineto{\pgfqpoint{4.920016in}{2.069913in}}%
\pgfpathlineto{\pgfqpoint{4.911263in}{2.069913in}}%
\pgfpathlineto{\pgfqpoint{4.911263in}{1.620128in}}%
\pgfpathclose%
\pgfusepath{fill}%
\end{pgfscope}%
\begin{pgfscope}%
\pgfpathrectangle{\pgfqpoint{3.776708in}{0.600000in}}{\pgfqpoint{2.573292in}{2.070576in}}%
\pgfusepath{clip}%
\pgfsetbuttcap%
\pgfsetmiterjoin%
\definecolor{currentfill}{rgb}{0.302379,0.450282,0.300122}%
\pgfsetfillcolor{currentfill}%
\pgfsetlinewidth{0.000000pt}%
\definecolor{currentstroke}{rgb}{0.000000,0.000000,0.000000}%
\pgfsetstrokecolor{currentstroke}%
\pgfsetstrokeopacity{0.000000}%
\pgfsetdash{}{0pt}%
\pgfpathmoveto{\pgfqpoint{4.922205in}{1.624685in}}%
\pgfpathlineto{\pgfqpoint{4.930958in}{1.624685in}}%
\pgfpathlineto{\pgfqpoint{4.930958in}{2.076875in}}%
\pgfpathlineto{\pgfqpoint{4.922205in}{2.076875in}}%
\pgfpathlineto{\pgfqpoint{4.922205in}{1.624685in}}%
\pgfpathclose%
\pgfusepath{fill}%
\end{pgfscope}%
\begin{pgfscope}%
\pgfpathrectangle{\pgfqpoint{3.776708in}{0.600000in}}{\pgfqpoint{2.573292in}{2.070576in}}%
\pgfusepath{clip}%
\pgfsetbuttcap%
\pgfsetmiterjoin%
\definecolor{currentfill}{rgb}{0.302379,0.450282,0.300122}%
\pgfsetfillcolor{currentfill}%
\pgfsetlinewidth{0.000000pt}%
\definecolor{currentstroke}{rgb}{0.000000,0.000000,0.000000}%
\pgfsetstrokecolor{currentstroke}%
\pgfsetstrokeopacity{0.000000}%
\pgfsetdash{}{0pt}%
\pgfpathmoveto{\pgfqpoint{4.933146in}{1.629001in}}%
\pgfpathlineto{\pgfqpoint{4.941900in}{1.629001in}}%
\pgfpathlineto{\pgfqpoint{4.941900in}{2.080229in}}%
\pgfpathlineto{\pgfqpoint{4.933146in}{2.080229in}}%
\pgfpathlineto{\pgfqpoint{4.933146in}{1.629001in}}%
\pgfpathclose%
\pgfusepath{fill}%
\end{pgfscope}%
\begin{pgfscope}%
\pgfpathrectangle{\pgfqpoint{3.776708in}{0.600000in}}{\pgfqpoint{2.573292in}{2.070576in}}%
\pgfusepath{clip}%
\pgfsetbuttcap%
\pgfsetmiterjoin%
\definecolor{currentfill}{rgb}{0.302379,0.450282,0.300122}%
\pgfsetfillcolor{currentfill}%
\pgfsetlinewidth{0.000000pt}%
\definecolor{currentstroke}{rgb}{0.000000,0.000000,0.000000}%
\pgfsetstrokecolor{currentstroke}%
\pgfsetstrokeopacity{0.000000}%
\pgfsetdash{}{0pt}%
\pgfpathmoveto{\pgfqpoint{4.944088in}{1.630721in}}%
\pgfpathlineto{\pgfqpoint{4.952842in}{1.630721in}}%
\pgfpathlineto{\pgfqpoint{4.952842in}{2.085609in}}%
\pgfpathlineto{\pgfqpoint{4.944088in}{2.085609in}}%
\pgfpathlineto{\pgfqpoint{4.944088in}{1.630721in}}%
\pgfpathclose%
\pgfusepath{fill}%
\end{pgfscope}%
\begin{pgfscope}%
\pgfpathrectangle{\pgfqpoint{3.776708in}{0.600000in}}{\pgfqpoint{2.573292in}{2.070576in}}%
\pgfusepath{clip}%
\pgfsetbuttcap%
\pgfsetmiterjoin%
\definecolor{currentfill}{rgb}{0.302379,0.450282,0.300122}%
\pgfsetfillcolor{currentfill}%
\pgfsetlinewidth{0.000000pt}%
\definecolor{currentstroke}{rgb}{0.000000,0.000000,0.000000}%
\pgfsetstrokecolor{currentstroke}%
\pgfsetstrokeopacity{0.000000}%
\pgfsetdash{}{0pt}%
\pgfpathmoveto{\pgfqpoint{4.955030in}{1.629178in}}%
\pgfpathlineto{\pgfqpoint{4.963783in}{1.629178in}}%
\pgfpathlineto{\pgfqpoint{4.963783in}{2.088528in}}%
\pgfpathlineto{\pgfqpoint{4.955030in}{2.088528in}}%
\pgfpathlineto{\pgfqpoint{4.955030in}{1.629178in}}%
\pgfpathclose%
\pgfusepath{fill}%
\end{pgfscope}%
\begin{pgfscope}%
\pgfpathrectangle{\pgfqpoint{3.776708in}{0.600000in}}{\pgfqpoint{2.573292in}{2.070576in}}%
\pgfusepath{clip}%
\pgfsetbuttcap%
\pgfsetmiterjoin%
\definecolor{currentfill}{rgb}{0.302379,0.450282,0.300122}%
\pgfsetfillcolor{currentfill}%
\pgfsetlinewidth{0.000000pt}%
\definecolor{currentstroke}{rgb}{0.000000,0.000000,0.000000}%
\pgfsetstrokecolor{currentstroke}%
\pgfsetstrokeopacity{0.000000}%
\pgfsetdash{}{0pt}%
\pgfpathmoveto{\pgfqpoint{4.965972in}{1.630679in}}%
\pgfpathlineto{\pgfqpoint{4.974725in}{1.630679in}}%
\pgfpathlineto{\pgfqpoint{4.974725in}{2.087955in}}%
\pgfpathlineto{\pgfqpoint{4.965972in}{2.087955in}}%
\pgfpathlineto{\pgfqpoint{4.965972in}{1.630679in}}%
\pgfpathclose%
\pgfusepath{fill}%
\end{pgfscope}%
\begin{pgfscope}%
\pgfpathrectangle{\pgfqpoint{3.776708in}{0.600000in}}{\pgfqpoint{2.573292in}{2.070576in}}%
\pgfusepath{clip}%
\pgfsetbuttcap%
\pgfsetmiterjoin%
\definecolor{currentfill}{rgb}{0.302379,0.450282,0.300122}%
\pgfsetfillcolor{currentfill}%
\pgfsetlinewidth{0.000000pt}%
\definecolor{currentstroke}{rgb}{0.000000,0.000000,0.000000}%
\pgfsetstrokecolor{currentstroke}%
\pgfsetstrokeopacity{0.000000}%
\pgfsetdash{}{0pt}%
\pgfpathmoveto{\pgfqpoint{4.976914in}{1.631213in}}%
\pgfpathlineto{\pgfqpoint{4.985667in}{1.631213in}}%
\pgfpathlineto{\pgfqpoint{4.985667in}{2.083569in}}%
\pgfpathlineto{\pgfqpoint{4.976914in}{2.083569in}}%
\pgfpathlineto{\pgfqpoint{4.976914in}{1.631213in}}%
\pgfpathclose%
\pgfusepath{fill}%
\end{pgfscope}%
\begin{pgfscope}%
\pgfpathrectangle{\pgfqpoint{3.776708in}{0.600000in}}{\pgfqpoint{2.573292in}{2.070576in}}%
\pgfusepath{clip}%
\pgfsetbuttcap%
\pgfsetmiterjoin%
\definecolor{currentfill}{rgb}{0.302379,0.450282,0.300122}%
\pgfsetfillcolor{currentfill}%
\pgfsetlinewidth{0.000000pt}%
\definecolor{currentstroke}{rgb}{0.000000,0.000000,0.000000}%
\pgfsetstrokecolor{currentstroke}%
\pgfsetstrokeopacity{0.000000}%
\pgfsetdash{}{0pt}%
\pgfpathmoveto{\pgfqpoint{4.987855in}{1.631212in}}%
\pgfpathlineto{\pgfqpoint{4.996609in}{1.631212in}}%
\pgfpathlineto{\pgfqpoint{4.996609in}{2.073181in}}%
\pgfpathlineto{\pgfqpoint{4.987855in}{2.073181in}}%
\pgfpathlineto{\pgfqpoint{4.987855in}{1.631212in}}%
\pgfpathclose%
\pgfusepath{fill}%
\end{pgfscope}%
\begin{pgfscope}%
\pgfpathrectangle{\pgfqpoint{3.776708in}{0.600000in}}{\pgfqpoint{2.573292in}{2.070576in}}%
\pgfusepath{clip}%
\pgfsetbuttcap%
\pgfsetmiterjoin%
\definecolor{currentfill}{rgb}{0.302379,0.450282,0.300122}%
\pgfsetfillcolor{currentfill}%
\pgfsetlinewidth{0.000000pt}%
\definecolor{currentstroke}{rgb}{0.000000,0.000000,0.000000}%
\pgfsetstrokecolor{currentstroke}%
\pgfsetstrokeopacity{0.000000}%
\pgfsetdash{}{0pt}%
\pgfpathmoveto{\pgfqpoint{4.998797in}{1.631439in}}%
\pgfpathlineto{\pgfqpoint{5.007551in}{1.631439in}}%
\pgfpathlineto{\pgfqpoint{5.007551in}{2.065582in}}%
\pgfpathlineto{\pgfqpoint{4.998797in}{2.065582in}}%
\pgfpathlineto{\pgfqpoint{4.998797in}{1.631439in}}%
\pgfpathclose%
\pgfusepath{fill}%
\end{pgfscope}%
\begin{pgfscope}%
\pgfpathrectangle{\pgfqpoint{3.776708in}{0.600000in}}{\pgfqpoint{2.573292in}{2.070576in}}%
\pgfusepath{clip}%
\pgfsetbuttcap%
\pgfsetmiterjoin%
\definecolor{currentfill}{rgb}{0.302379,0.450282,0.300122}%
\pgfsetfillcolor{currentfill}%
\pgfsetlinewidth{0.000000pt}%
\definecolor{currentstroke}{rgb}{0.000000,0.000000,0.000000}%
\pgfsetstrokecolor{currentstroke}%
\pgfsetstrokeopacity{0.000000}%
\pgfsetdash{}{0pt}%
\pgfpathmoveto{\pgfqpoint{5.009739in}{1.628463in}}%
\pgfpathlineto{\pgfqpoint{5.018492in}{1.628463in}}%
\pgfpathlineto{\pgfqpoint{5.018492in}{2.054645in}}%
\pgfpathlineto{\pgfqpoint{5.009739in}{2.054645in}}%
\pgfpathlineto{\pgfqpoint{5.009739in}{1.628463in}}%
\pgfpathclose%
\pgfusepath{fill}%
\end{pgfscope}%
\begin{pgfscope}%
\pgfpathrectangle{\pgfqpoint{3.776708in}{0.600000in}}{\pgfqpoint{2.573292in}{2.070576in}}%
\pgfusepath{clip}%
\pgfsetbuttcap%
\pgfsetmiterjoin%
\definecolor{currentfill}{rgb}{0.302379,0.450282,0.300122}%
\pgfsetfillcolor{currentfill}%
\pgfsetlinewidth{0.000000pt}%
\definecolor{currentstroke}{rgb}{0.000000,0.000000,0.000000}%
\pgfsetstrokecolor{currentstroke}%
\pgfsetstrokeopacity{0.000000}%
\pgfsetdash{}{0pt}%
\pgfpathmoveto{\pgfqpoint{5.020681in}{1.628208in}}%
\pgfpathlineto{\pgfqpoint{5.029434in}{1.628208in}}%
\pgfpathlineto{\pgfqpoint{5.029434in}{2.036665in}}%
\pgfpathlineto{\pgfqpoint{5.020681in}{2.036665in}}%
\pgfpathlineto{\pgfqpoint{5.020681in}{1.628208in}}%
\pgfpathclose%
\pgfusepath{fill}%
\end{pgfscope}%
\begin{pgfscope}%
\pgfpathrectangle{\pgfqpoint{3.776708in}{0.600000in}}{\pgfqpoint{2.573292in}{2.070576in}}%
\pgfusepath{clip}%
\pgfsetbuttcap%
\pgfsetmiterjoin%
\definecolor{currentfill}{rgb}{0.302379,0.450282,0.300122}%
\pgfsetfillcolor{currentfill}%
\pgfsetlinewidth{0.000000pt}%
\definecolor{currentstroke}{rgb}{0.000000,0.000000,0.000000}%
\pgfsetstrokecolor{currentstroke}%
\pgfsetstrokeopacity{0.000000}%
\pgfsetdash{}{0pt}%
\pgfpathmoveto{\pgfqpoint{5.031623in}{1.626799in}}%
\pgfpathlineto{\pgfqpoint{5.040376in}{1.626799in}}%
\pgfpathlineto{\pgfqpoint{5.040376in}{2.018518in}}%
\pgfpathlineto{\pgfqpoint{5.031623in}{2.018518in}}%
\pgfpathlineto{\pgfqpoint{5.031623in}{1.626799in}}%
\pgfpathclose%
\pgfusepath{fill}%
\end{pgfscope}%
\begin{pgfscope}%
\pgfpathrectangle{\pgfqpoint{3.776708in}{0.600000in}}{\pgfqpoint{2.573292in}{2.070576in}}%
\pgfusepath{clip}%
\pgfsetbuttcap%
\pgfsetmiterjoin%
\definecolor{currentfill}{rgb}{0.302379,0.450282,0.300122}%
\pgfsetfillcolor{currentfill}%
\pgfsetlinewidth{0.000000pt}%
\definecolor{currentstroke}{rgb}{0.000000,0.000000,0.000000}%
\pgfsetstrokecolor{currentstroke}%
\pgfsetstrokeopacity{0.000000}%
\pgfsetdash{}{0pt}%
\pgfpathmoveto{\pgfqpoint{5.042564in}{1.626691in}}%
\pgfpathlineto{\pgfqpoint{5.051318in}{1.626691in}}%
\pgfpathlineto{\pgfqpoint{5.051318in}{1.996655in}}%
\pgfpathlineto{\pgfqpoint{5.042564in}{1.996655in}}%
\pgfpathlineto{\pgfqpoint{5.042564in}{1.626691in}}%
\pgfpathclose%
\pgfusepath{fill}%
\end{pgfscope}%
\begin{pgfscope}%
\pgfpathrectangle{\pgfqpoint{3.776708in}{0.600000in}}{\pgfqpoint{2.573292in}{2.070576in}}%
\pgfusepath{clip}%
\pgfsetbuttcap%
\pgfsetmiterjoin%
\definecolor{currentfill}{rgb}{0.302379,0.450282,0.300122}%
\pgfsetfillcolor{currentfill}%
\pgfsetlinewidth{0.000000pt}%
\definecolor{currentstroke}{rgb}{0.000000,0.000000,0.000000}%
\pgfsetstrokecolor{currentstroke}%
\pgfsetstrokeopacity{0.000000}%
\pgfsetdash{}{0pt}%
\pgfpathmoveto{\pgfqpoint{5.053506in}{1.626699in}}%
\pgfpathlineto{\pgfqpoint{5.062260in}{1.626699in}}%
\pgfpathlineto{\pgfqpoint{5.062260in}{1.977227in}}%
\pgfpathlineto{\pgfqpoint{5.053506in}{1.977227in}}%
\pgfpathlineto{\pgfqpoint{5.053506in}{1.626699in}}%
\pgfpathclose%
\pgfusepath{fill}%
\end{pgfscope}%
\begin{pgfscope}%
\pgfpathrectangle{\pgfqpoint{3.776708in}{0.600000in}}{\pgfqpoint{2.573292in}{2.070576in}}%
\pgfusepath{clip}%
\pgfsetbuttcap%
\pgfsetmiterjoin%
\definecolor{currentfill}{rgb}{0.302379,0.450282,0.300122}%
\pgfsetfillcolor{currentfill}%
\pgfsetlinewidth{0.000000pt}%
\definecolor{currentstroke}{rgb}{0.000000,0.000000,0.000000}%
\pgfsetstrokecolor{currentstroke}%
\pgfsetstrokeopacity{0.000000}%
\pgfsetdash{}{0pt}%
\pgfpathmoveto{\pgfqpoint{5.064448in}{1.628191in}}%
\pgfpathlineto{\pgfqpoint{5.073201in}{1.628191in}}%
\pgfpathlineto{\pgfqpoint{5.073201in}{1.959040in}}%
\pgfpathlineto{\pgfqpoint{5.064448in}{1.959040in}}%
\pgfpathlineto{\pgfqpoint{5.064448in}{1.628191in}}%
\pgfpathclose%
\pgfusepath{fill}%
\end{pgfscope}%
\begin{pgfscope}%
\pgfpathrectangle{\pgfqpoint{3.776708in}{0.600000in}}{\pgfqpoint{2.573292in}{2.070576in}}%
\pgfusepath{clip}%
\pgfsetbuttcap%
\pgfsetmiterjoin%
\definecolor{currentfill}{rgb}{0.302379,0.450282,0.300122}%
\pgfsetfillcolor{currentfill}%
\pgfsetlinewidth{0.000000pt}%
\definecolor{currentstroke}{rgb}{0.000000,0.000000,0.000000}%
\pgfsetstrokecolor{currentstroke}%
\pgfsetstrokeopacity{0.000000}%
\pgfsetdash{}{0pt}%
\pgfpathmoveto{\pgfqpoint{5.075390in}{1.628622in}}%
\pgfpathlineto{\pgfqpoint{5.084143in}{1.628622in}}%
\pgfpathlineto{\pgfqpoint{5.084143in}{1.941103in}}%
\pgfpathlineto{\pgfqpoint{5.075390in}{1.941103in}}%
\pgfpathlineto{\pgfqpoint{5.075390in}{1.628622in}}%
\pgfpathclose%
\pgfusepath{fill}%
\end{pgfscope}%
\begin{pgfscope}%
\pgfpathrectangle{\pgfqpoint{3.776708in}{0.600000in}}{\pgfqpoint{2.573292in}{2.070576in}}%
\pgfusepath{clip}%
\pgfsetbuttcap%
\pgfsetmiterjoin%
\definecolor{currentfill}{rgb}{0.302379,0.450282,0.300122}%
\pgfsetfillcolor{currentfill}%
\pgfsetlinewidth{0.000000pt}%
\definecolor{currentstroke}{rgb}{0.000000,0.000000,0.000000}%
\pgfsetstrokecolor{currentstroke}%
\pgfsetstrokeopacity{0.000000}%
\pgfsetdash{}{0pt}%
\pgfpathmoveto{\pgfqpoint{5.086332in}{1.631312in}}%
\pgfpathlineto{\pgfqpoint{5.095085in}{1.631312in}}%
\pgfpathlineto{\pgfqpoint{5.095085in}{1.927349in}}%
\pgfpathlineto{\pgfqpoint{5.086332in}{1.927349in}}%
\pgfpathlineto{\pgfqpoint{5.086332in}{1.631312in}}%
\pgfpathclose%
\pgfusepath{fill}%
\end{pgfscope}%
\begin{pgfscope}%
\pgfpathrectangle{\pgfqpoint{3.776708in}{0.600000in}}{\pgfqpoint{2.573292in}{2.070576in}}%
\pgfusepath{clip}%
\pgfsetbuttcap%
\pgfsetmiterjoin%
\definecolor{currentfill}{rgb}{0.302379,0.450282,0.300122}%
\pgfsetfillcolor{currentfill}%
\pgfsetlinewidth{0.000000pt}%
\definecolor{currentstroke}{rgb}{0.000000,0.000000,0.000000}%
\pgfsetstrokecolor{currentstroke}%
\pgfsetstrokeopacity{0.000000}%
\pgfsetdash{}{0pt}%
\pgfpathmoveto{\pgfqpoint{5.097273in}{1.634183in}}%
\pgfpathlineto{\pgfqpoint{5.106027in}{1.634183in}}%
\pgfpathlineto{\pgfqpoint{5.106027in}{1.915714in}}%
\pgfpathlineto{\pgfqpoint{5.097273in}{1.915714in}}%
\pgfpathlineto{\pgfqpoint{5.097273in}{1.634183in}}%
\pgfpathclose%
\pgfusepath{fill}%
\end{pgfscope}%
\begin{pgfscope}%
\pgfpathrectangle{\pgfqpoint{3.776708in}{0.600000in}}{\pgfqpoint{2.573292in}{2.070576in}}%
\pgfusepath{clip}%
\pgfsetbuttcap%
\pgfsetmiterjoin%
\definecolor{currentfill}{rgb}{0.302379,0.450282,0.300122}%
\pgfsetfillcolor{currentfill}%
\pgfsetlinewidth{0.000000pt}%
\definecolor{currentstroke}{rgb}{0.000000,0.000000,0.000000}%
\pgfsetstrokecolor{currentstroke}%
\pgfsetstrokeopacity{0.000000}%
\pgfsetdash{}{0pt}%
\pgfpathmoveto{\pgfqpoint{5.108215in}{1.637561in}}%
\pgfpathlineto{\pgfqpoint{5.116969in}{1.637561in}}%
\pgfpathlineto{\pgfqpoint{5.116969in}{1.906607in}}%
\pgfpathlineto{\pgfqpoint{5.108215in}{1.906607in}}%
\pgfpathlineto{\pgfqpoint{5.108215in}{1.637561in}}%
\pgfpathclose%
\pgfusepath{fill}%
\end{pgfscope}%
\begin{pgfscope}%
\pgfpathrectangle{\pgfqpoint{3.776708in}{0.600000in}}{\pgfqpoint{2.573292in}{2.070576in}}%
\pgfusepath{clip}%
\pgfsetbuttcap%
\pgfsetmiterjoin%
\definecolor{currentfill}{rgb}{0.302379,0.450282,0.300122}%
\pgfsetfillcolor{currentfill}%
\pgfsetlinewidth{0.000000pt}%
\definecolor{currentstroke}{rgb}{0.000000,0.000000,0.000000}%
\pgfsetstrokecolor{currentstroke}%
\pgfsetstrokeopacity{0.000000}%
\pgfsetdash{}{0pt}%
\pgfpathmoveto{\pgfqpoint{5.119157in}{1.642014in}}%
\pgfpathlineto{\pgfqpoint{5.127910in}{1.642014in}}%
\pgfpathlineto{\pgfqpoint{5.127910in}{1.901250in}}%
\pgfpathlineto{\pgfqpoint{5.119157in}{1.901250in}}%
\pgfpathlineto{\pgfqpoint{5.119157in}{1.642014in}}%
\pgfpathclose%
\pgfusepath{fill}%
\end{pgfscope}%
\begin{pgfscope}%
\pgfpathrectangle{\pgfqpoint{3.776708in}{0.600000in}}{\pgfqpoint{2.573292in}{2.070576in}}%
\pgfusepath{clip}%
\pgfsetbuttcap%
\pgfsetmiterjoin%
\definecolor{currentfill}{rgb}{0.302379,0.450282,0.300122}%
\pgfsetfillcolor{currentfill}%
\pgfsetlinewidth{0.000000pt}%
\definecolor{currentstroke}{rgb}{0.000000,0.000000,0.000000}%
\pgfsetstrokecolor{currentstroke}%
\pgfsetstrokeopacity{0.000000}%
\pgfsetdash{}{0pt}%
\pgfpathmoveto{\pgfqpoint{5.130099in}{1.645294in}}%
\pgfpathlineto{\pgfqpoint{5.138852in}{1.645294in}}%
\pgfpathlineto{\pgfqpoint{5.138852in}{1.900037in}}%
\pgfpathlineto{\pgfqpoint{5.130099in}{1.900037in}}%
\pgfpathlineto{\pgfqpoint{5.130099in}{1.645294in}}%
\pgfpathclose%
\pgfusepath{fill}%
\end{pgfscope}%
\begin{pgfscope}%
\pgfpathrectangle{\pgfqpoint{3.776708in}{0.600000in}}{\pgfqpoint{2.573292in}{2.070576in}}%
\pgfusepath{clip}%
\pgfsetbuttcap%
\pgfsetmiterjoin%
\definecolor{currentfill}{rgb}{0.302379,0.450282,0.300122}%
\pgfsetfillcolor{currentfill}%
\pgfsetlinewidth{0.000000pt}%
\definecolor{currentstroke}{rgb}{0.000000,0.000000,0.000000}%
\pgfsetstrokecolor{currentstroke}%
\pgfsetstrokeopacity{0.000000}%
\pgfsetdash{}{0pt}%
\pgfpathmoveto{\pgfqpoint{5.141041in}{1.647527in}}%
\pgfpathlineto{\pgfqpoint{5.149794in}{1.647527in}}%
\pgfpathlineto{\pgfqpoint{5.149794in}{1.901980in}}%
\pgfpathlineto{\pgfqpoint{5.141041in}{1.901980in}}%
\pgfpathlineto{\pgfqpoint{5.141041in}{1.647527in}}%
\pgfpathclose%
\pgfusepath{fill}%
\end{pgfscope}%
\begin{pgfscope}%
\pgfpathrectangle{\pgfqpoint{3.776708in}{0.600000in}}{\pgfqpoint{2.573292in}{2.070576in}}%
\pgfusepath{clip}%
\pgfsetbuttcap%
\pgfsetmiterjoin%
\definecolor{currentfill}{rgb}{0.302379,0.450282,0.300122}%
\pgfsetfillcolor{currentfill}%
\pgfsetlinewidth{0.000000pt}%
\definecolor{currentstroke}{rgb}{0.000000,0.000000,0.000000}%
\pgfsetstrokecolor{currentstroke}%
\pgfsetstrokeopacity{0.000000}%
\pgfsetdash{}{0pt}%
\pgfpathmoveto{\pgfqpoint{5.151982in}{1.648946in}}%
\pgfpathlineto{\pgfqpoint{5.160736in}{1.648946in}}%
\pgfpathlineto{\pgfqpoint{5.160736in}{1.908586in}}%
\pgfpathlineto{\pgfqpoint{5.151982in}{1.908586in}}%
\pgfpathlineto{\pgfqpoint{5.151982in}{1.648946in}}%
\pgfpathclose%
\pgfusepath{fill}%
\end{pgfscope}%
\begin{pgfscope}%
\pgfpathrectangle{\pgfqpoint{3.776708in}{0.600000in}}{\pgfqpoint{2.573292in}{2.070576in}}%
\pgfusepath{clip}%
\pgfsetbuttcap%
\pgfsetmiterjoin%
\definecolor{currentfill}{rgb}{0.302379,0.450282,0.300122}%
\pgfsetfillcolor{currentfill}%
\pgfsetlinewidth{0.000000pt}%
\definecolor{currentstroke}{rgb}{0.000000,0.000000,0.000000}%
\pgfsetstrokecolor{currentstroke}%
\pgfsetstrokeopacity{0.000000}%
\pgfsetdash{}{0pt}%
\pgfpathmoveto{\pgfqpoint{5.162924in}{1.651851in}}%
\pgfpathlineto{\pgfqpoint{5.171678in}{1.651851in}}%
\pgfpathlineto{\pgfqpoint{5.171678in}{1.917842in}}%
\pgfpathlineto{\pgfqpoint{5.162924in}{1.917842in}}%
\pgfpathlineto{\pgfqpoint{5.162924in}{1.651851in}}%
\pgfpathclose%
\pgfusepath{fill}%
\end{pgfscope}%
\begin{pgfscope}%
\pgfpathrectangle{\pgfqpoint{3.776708in}{0.600000in}}{\pgfqpoint{2.573292in}{2.070576in}}%
\pgfusepath{clip}%
\pgfsetbuttcap%
\pgfsetmiterjoin%
\definecolor{currentfill}{rgb}{0.302379,0.450282,0.300122}%
\pgfsetfillcolor{currentfill}%
\pgfsetlinewidth{0.000000pt}%
\definecolor{currentstroke}{rgb}{0.000000,0.000000,0.000000}%
\pgfsetstrokecolor{currentstroke}%
\pgfsetstrokeopacity{0.000000}%
\pgfsetdash{}{0pt}%
\pgfpathmoveto{\pgfqpoint{5.173866in}{1.654412in}}%
\pgfpathlineto{\pgfqpoint{5.182619in}{1.654412in}}%
\pgfpathlineto{\pgfqpoint{5.182619in}{1.925488in}}%
\pgfpathlineto{\pgfqpoint{5.173866in}{1.925488in}}%
\pgfpathlineto{\pgfqpoint{5.173866in}{1.654412in}}%
\pgfpathclose%
\pgfusepath{fill}%
\end{pgfscope}%
\begin{pgfscope}%
\pgfpathrectangle{\pgfqpoint{3.776708in}{0.600000in}}{\pgfqpoint{2.573292in}{2.070576in}}%
\pgfusepath{clip}%
\pgfsetbuttcap%
\pgfsetmiterjoin%
\definecolor{currentfill}{rgb}{0.302379,0.450282,0.300122}%
\pgfsetfillcolor{currentfill}%
\pgfsetlinewidth{0.000000pt}%
\definecolor{currentstroke}{rgb}{0.000000,0.000000,0.000000}%
\pgfsetstrokecolor{currentstroke}%
\pgfsetstrokeopacity{0.000000}%
\pgfsetdash{}{0pt}%
\pgfpathmoveto{\pgfqpoint{5.184808in}{1.657945in}}%
\pgfpathlineto{\pgfqpoint{5.193561in}{1.657945in}}%
\pgfpathlineto{\pgfqpoint{5.193561in}{1.932313in}}%
\pgfpathlineto{\pgfqpoint{5.184808in}{1.932313in}}%
\pgfpathlineto{\pgfqpoint{5.184808in}{1.657945in}}%
\pgfpathclose%
\pgfusepath{fill}%
\end{pgfscope}%
\begin{pgfscope}%
\pgfpathrectangle{\pgfqpoint{3.776708in}{0.600000in}}{\pgfqpoint{2.573292in}{2.070576in}}%
\pgfusepath{clip}%
\pgfsetbuttcap%
\pgfsetmiterjoin%
\definecolor{currentfill}{rgb}{0.302379,0.450282,0.300122}%
\pgfsetfillcolor{currentfill}%
\pgfsetlinewidth{0.000000pt}%
\definecolor{currentstroke}{rgb}{0.000000,0.000000,0.000000}%
\pgfsetstrokecolor{currentstroke}%
\pgfsetstrokeopacity{0.000000}%
\pgfsetdash{}{0pt}%
\pgfpathmoveto{\pgfqpoint{5.195750in}{1.660511in}}%
\pgfpathlineto{\pgfqpoint{5.204503in}{1.660511in}}%
\pgfpathlineto{\pgfqpoint{5.204503in}{1.937917in}}%
\pgfpathlineto{\pgfqpoint{5.195750in}{1.937917in}}%
\pgfpathlineto{\pgfqpoint{5.195750in}{1.660511in}}%
\pgfpathclose%
\pgfusepath{fill}%
\end{pgfscope}%
\begin{pgfscope}%
\pgfpathrectangle{\pgfqpoint{3.776708in}{0.600000in}}{\pgfqpoint{2.573292in}{2.070576in}}%
\pgfusepath{clip}%
\pgfsetbuttcap%
\pgfsetmiterjoin%
\definecolor{currentfill}{rgb}{0.302379,0.450282,0.300122}%
\pgfsetfillcolor{currentfill}%
\pgfsetlinewidth{0.000000pt}%
\definecolor{currentstroke}{rgb}{0.000000,0.000000,0.000000}%
\pgfsetstrokecolor{currentstroke}%
\pgfsetstrokeopacity{0.000000}%
\pgfsetdash{}{0pt}%
\pgfpathmoveto{\pgfqpoint{5.206691in}{1.664475in}}%
\pgfpathlineto{\pgfqpoint{5.215445in}{1.664475in}}%
\pgfpathlineto{\pgfqpoint{5.215445in}{1.944394in}}%
\pgfpathlineto{\pgfqpoint{5.206691in}{1.944394in}}%
\pgfpathlineto{\pgfqpoint{5.206691in}{1.664475in}}%
\pgfpathclose%
\pgfusepath{fill}%
\end{pgfscope}%
\begin{pgfscope}%
\pgfpathrectangle{\pgfqpoint{3.776708in}{0.600000in}}{\pgfqpoint{2.573292in}{2.070576in}}%
\pgfusepath{clip}%
\pgfsetbuttcap%
\pgfsetmiterjoin%
\definecolor{currentfill}{rgb}{0.302379,0.450282,0.300122}%
\pgfsetfillcolor{currentfill}%
\pgfsetlinewidth{0.000000pt}%
\definecolor{currentstroke}{rgb}{0.000000,0.000000,0.000000}%
\pgfsetstrokecolor{currentstroke}%
\pgfsetstrokeopacity{0.000000}%
\pgfsetdash{}{0pt}%
\pgfpathmoveto{\pgfqpoint{5.217633in}{1.670588in}}%
\pgfpathlineto{\pgfqpoint{5.226387in}{1.670588in}}%
\pgfpathlineto{\pgfqpoint{5.226387in}{1.949442in}}%
\pgfpathlineto{\pgfqpoint{5.217633in}{1.949442in}}%
\pgfpathlineto{\pgfqpoint{5.217633in}{1.670588in}}%
\pgfpathclose%
\pgfusepath{fill}%
\end{pgfscope}%
\begin{pgfscope}%
\pgfpathrectangle{\pgfqpoint{3.776708in}{0.600000in}}{\pgfqpoint{2.573292in}{2.070576in}}%
\pgfusepath{clip}%
\pgfsetbuttcap%
\pgfsetmiterjoin%
\definecolor{currentfill}{rgb}{0.302379,0.450282,0.300122}%
\pgfsetfillcolor{currentfill}%
\pgfsetlinewidth{0.000000pt}%
\definecolor{currentstroke}{rgb}{0.000000,0.000000,0.000000}%
\pgfsetstrokecolor{currentstroke}%
\pgfsetstrokeopacity{0.000000}%
\pgfsetdash{}{0pt}%
\pgfpathmoveto{\pgfqpoint{5.228575in}{1.677953in}}%
\pgfpathlineto{\pgfqpoint{5.237328in}{1.677953in}}%
\pgfpathlineto{\pgfqpoint{5.237328in}{1.953261in}}%
\pgfpathlineto{\pgfqpoint{5.228575in}{1.953261in}}%
\pgfpathlineto{\pgfqpoint{5.228575in}{1.677953in}}%
\pgfpathclose%
\pgfusepath{fill}%
\end{pgfscope}%
\begin{pgfscope}%
\pgfpathrectangle{\pgfqpoint{3.776708in}{0.600000in}}{\pgfqpoint{2.573292in}{2.070576in}}%
\pgfusepath{clip}%
\pgfsetbuttcap%
\pgfsetmiterjoin%
\definecolor{currentfill}{rgb}{0.302379,0.450282,0.300122}%
\pgfsetfillcolor{currentfill}%
\pgfsetlinewidth{0.000000pt}%
\definecolor{currentstroke}{rgb}{0.000000,0.000000,0.000000}%
\pgfsetstrokecolor{currentstroke}%
\pgfsetstrokeopacity{0.000000}%
\pgfsetdash{}{0pt}%
\pgfpathmoveto{\pgfqpoint{5.239517in}{1.683448in}}%
\pgfpathlineto{\pgfqpoint{5.248270in}{1.683448in}}%
\pgfpathlineto{\pgfqpoint{5.248270in}{1.963092in}}%
\pgfpathlineto{\pgfqpoint{5.239517in}{1.963092in}}%
\pgfpathlineto{\pgfqpoint{5.239517in}{1.683448in}}%
\pgfpathclose%
\pgfusepath{fill}%
\end{pgfscope}%
\begin{pgfscope}%
\pgfpathrectangle{\pgfqpoint{3.776708in}{0.600000in}}{\pgfqpoint{2.573292in}{2.070576in}}%
\pgfusepath{clip}%
\pgfsetbuttcap%
\pgfsetmiterjoin%
\definecolor{currentfill}{rgb}{0.302379,0.450282,0.300122}%
\pgfsetfillcolor{currentfill}%
\pgfsetlinewidth{0.000000pt}%
\definecolor{currentstroke}{rgb}{0.000000,0.000000,0.000000}%
\pgfsetstrokecolor{currentstroke}%
\pgfsetstrokeopacity{0.000000}%
\pgfsetdash{}{0pt}%
\pgfpathmoveto{\pgfqpoint{5.250459in}{1.690384in}}%
\pgfpathlineto{\pgfqpoint{5.259212in}{1.690384in}}%
\pgfpathlineto{\pgfqpoint{5.259212in}{1.970649in}}%
\pgfpathlineto{\pgfqpoint{5.250459in}{1.970649in}}%
\pgfpathlineto{\pgfqpoint{5.250459in}{1.690384in}}%
\pgfpathclose%
\pgfusepath{fill}%
\end{pgfscope}%
\begin{pgfscope}%
\pgfpathrectangle{\pgfqpoint{3.776708in}{0.600000in}}{\pgfqpoint{2.573292in}{2.070576in}}%
\pgfusepath{clip}%
\pgfsetbuttcap%
\pgfsetmiterjoin%
\definecolor{currentfill}{rgb}{0.302379,0.450282,0.300122}%
\pgfsetfillcolor{currentfill}%
\pgfsetlinewidth{0.000000pt}%
\definecolor{currentstroke}{rgb}{0.000000,0.000000,0.000000}%
\pgfsetstrokecolor{currentstroke}%
\pgfsetstrokeopacity{0.000000}%
\pgfsetdash{}{0pt}%
\pgfpathmoveto{\pgfqpoint{5.261400in}{1.697364in}}%
\pgfpathlineto{\pgfqpoint{5.270154in}{1.697364in}}%
\pgfpathlineto{\pgfqpoint{5.270154in}{1.978882in}}%
\pgfpathlineto{\pgfqpoint{5.261400in}{1.978882in}}%
\pgfpathlineto{\pgfqpoint{5.261400in}{1.697364in}}%
\pgfpathclose%
\pgfusepath{fill}%
\end{pgfscope}%
\begin{pgfscope}%
\pgfpathrectangle{\pgfqpoint{3.776708in}{0.600000in}}{\pgfqpoint{2.573292in}{2.070576in}}%
\pgfusepath{clip}%
\pgfsetbuttcap%
\pgfsetmiterjoin%
\definecolor{currentfill}{rgb}{0.302379,0.450282,0.300122}%
\pgfsetfillcolor{currentfill}%
\pgfsetlinewidth{0.000000pt}%
\definecolor{currentstroke}{rgb}{0.000000,0.000000,0.000000}%
\pgfsetstrokecolor{currentstroke}%
\pgfsetstrokeopacity{0.000000}%
\pgfsetdash{}{0pt}%
\pgfpathmoveto{\pgfqpoint{5.272342in}{1.703424in}}%
\pgfpathlineto{\pgfqpoint{5.281096in}{1.703424in}}%
\pgfpathlineto{\pgfqpoint{5.281096in}{1.988813in}}%
\pgfpathlineto{\pgfqpoint{5.272342in}{1.988813in}}%
\pgfpathlineto{\pgfqpoint{5.272342in}{1.703424in}}%
\pgfpathclose%
\pgfusepath{fill}%
\end{pgfscope}%
\begin{pgfscope}%
\pgfpathrectangle{\pgfqpoint{3.776708in}{0.600000in}}{\pgfqpoint{2.573292in}{2.070576in}}%
\pgfusepath{clip}%
\pgfsetbuttcap%
\pgfsetmiterjoin%
\definecolor{currentfill}{rgb}{0.302379,0.450282,0.300122}%
\pgfsetfillcolor{currentfill}%
\pgfsetlinewidth{0.000000pt}%
\definecolor{currentstroke}{rgb}{0.000000,0.000000,0.000000}%
\pgfsetstrokecolor{currentstroke}%
\pgfsetstrokeopacity{0.000000}%
\pgfsetdash{}{0pt}%
\pgfpathmoveto{\pgfqpoint{5.283284in}{1.708754in}}%
\pgfpathlineto{\pgfqpoint{5.292037in}{1.708754in}}%
\pgfpathlineto{\pgfqpoint{5.292037in}{1.995608in}}%
\pgfpathlineto{\pgfqpoint{5.283284in}{1.995608in}}%
\pgfpathlineto{\pgfqpoint{5.283284in}{1.708754in}}%
\pgfpathclose%
\pgfusepath{fill}%
\end{pgfscope}%
\begin{pgfscope}%
\pgfpathrectangle{\pgfqpoint{3.776708in}{0.600000in}}{\pgfqpoint{2.573292in}{2.070576in}}%
\pgfusepath{clip}%
\pgfsetbuttcap%
\pgfsetmiterjoin%
\definecolor{currentfill}{rgb}{0.302379,0.450282,0.300122}%
\pgfsetfillcolor{currentfill}%
\pgfsetlinewidth{0.000000pt}%
\definecolor{currentstroke}{rgb}{0.000000,0.000000,0.000000}%
\pgfsetstrokecolor{currentstroke}%
\pgfsetstrokeopacity{0.000000}%
\pgfsetdash{}{0pt}%
\pgfpathmoveto{\pgfqpoint{5.294226in}{1.712145in}}%
\pgfpathlineto{\pgfqpoint{5.302979in}{1.712145in}}%
\pgfpathlineto{\pgfqpoint{5.302979in}{1.997168in}}%
\pgfpathlineto{\pgfqpoint{5.294226in}{1.997168in}}%
\pgfpathlineto{\pgfqpoint{5.294226in}{1.712145in}}%
\pgfpathclose%
\pgfusepath{fill}%
\end{pgfscope}%
\begin{pgfscope}%
\pgfpathrectangle{\pgfqpoint{3.776708in}{0.600000in}}{\pgfqpoint{2.573292in}{2.070576in}}%
\pgfusepath{clip}%
\pgfsetbuttcap%
\pgfsetmiterjoin%
\definecolor{currentfill}{rgb}{0.302379,0.450282,0.300122}%
\pgfsetfillcolor{currentfill}%
\pgfsetlinewidth{0.000000pt}%
\definecolor{currentstroke}{rgb}{0.000000,0.000000,0.000000}%
\pgfsetstrokecolor{currentstroke}%
\pgfsetstrokeopacity{0.000000}%
\pgfsetdash{}{0pt}%
\pgfpathmoveto{\pgfqpoint{5.305168in}{1.718409in}}%
\pgfpathlineto{\pgfqpoint{5.313921in}{1.718409in}}%
\pgfpathlineto{\pgfqpoint{5.313921in}{2.002661in}}%
\pgfpathlineto{\pgfqpoint{5.305168in}{2.002661in}}%
\pgfpathlineto{\pgfqpoint{5.305168in}{1.718409in}}%
\pgfpathclose%
\pgfusepath{fill}%
\end{pgfscope}%
\begin{pgfscope}%
\pgfpathrectangle{\pgfqpoint{3.776708in}{0.600000in}}{\pgfqpoint{2.573292in}{2.070576in}}%
\pgfusepath{clip}%
\pgfsetbuttcap%
\pgfsetmiterjoin%
\definecolor{currentfill}{rgb}{0.302379,0.450282,0.300122}%
\pgfsetfillcolor{currentfill}%
\pgfsetlinewidth{0.000000pt}%
\definecolor{currentstroke}{rgb}{0.000000,0.000000,0.000000}%
\pgfsetstrokecolor{currentstroke}%
\pgfsetstrokeopacity{0.000000}%
\pgfsetdash{}{0pt}%
\pgfpathmoveto{\pgfqpoint{5.316109in}{1.723949in}}%
\pgfpathlineto{\pgfqpoint{5.324863in}{1.723949in}}%
\pgfpathlineto{\pgfqpoint{5.324863in}{2.003324in}}%
\pgfpathlineto{\pgfqpoint{5.316109in}{2.003324in}}%
\pgfpathlineto{\pgfqpoint{5.316109in}{1.723949in}}%
\pgfpathclose%
\pgfusepath{fill}%
\end{pgfscope}%
\begin{pgfscope}%
\pgfpathrectangle{\pgfqpoint{3.776708in}{0.600000in}}{\pgfqpoint{2.573292in}{2.070576in}}%
\pgfusepath{clip}%
\pgfsetbuttcap%
\pgfsetmiterjoin%
\definecolor{currentfill}{rgb}{0.302379,0.450282,0.300122}%
\pgfsetfillcolor{currentfill}%
\pgfsetlinewidth{0.000000pt}%
\definecolor{currentstroke}{rgb}{0.000000,0.000000,0.000000}%
\pgfsetstrokecolor{currentstroke}%
\pgfsetstrokeopacity{0.000000}%
\pgfsetdash{}{0pt}%
\pgfpathmoveto{\pgfqpoint{5.327051in}{1.727505in}}%
\pgfpathlineto{\pgfqpoint{5.335805in}{1.727505in}}%
\pgfpathlineto{\pgfqpoint{5.335805in}{1.999346in}}%
\pgfpathlineto{\pgfqpoint{5.327051in}{1.999346in}}%
\pgfpathlineto{\pgfqpoint{5.327051in}{1.727505in}}%
\pgfpathclose%
\pgfusepath{fill}%
\end{pgfscope}%
\begin{pgfscope}%
\pgfpathrectangle{\pgfqpoint{3.776708in}{0.600000in}}{\pgfqpoint{2.573292in}{2.070576in}}%
\pgfusepath{clip}%
\pgfsetbuttcap%
\pgfsetmiterjoin%
\definecolor{currentfill}{rgb}{0.302379,0.450282,0.300122}%
\pgfsetfillcolor{currentfill}%
\pgfsetlinewidth{0.000000pt}%
\definecolor{currentstroke}{rgb}{0.000000,0.000000,0.000000}%
\pgfsetstrokecolor{currentstroke}%
\pgfsetstrokeopacity{0.000000}%
\pgfsetdash{}{0pt}%
\pgfpathmoveto{\pgfqpoint{5.337993in}{1.731708in}}%
\pgfpathlineto{\pgfqpoint{5.346746in}{1.731708in}}%
\pgfpathlineto{\pgfqpoint{5.346746in}{1.997285in}}%
\pgfpathlineto{\pgfqpoint{5.337993in}{1.997285in}}%
\pgfpathlineto{\pgfqpoint{5.337993in}{1.731708in}}%
\pgfpathclose%
\pgfusepath{fill}%
\end{pgfscope}%
\begin{pgfscope}%
\pgfpathrectangle{\pgfqpoint{3.776708in}{0.600000in}}{\pgfqpoint{2.573292in}{2.070576in}}%
\pgfusepath{clip}%
\pgfsetbuttcap%
\pgfsetmiterjoin%
\definecolor{currentfill}{rgb}{0.302379,0.450282,0.300122}%
\pgfsetfillcolor{currentfill}%
\pgfsetlinewidth{0.000000pt}%
\definecolor{currentstroke}{rgb}{0.000000,0.000000,0.000000}%
\pgfsetstrokecolor{currentstroke}%
\pgfsetstrokeopacity{0.000000}%
\pgfsetdash{}{0pt}%
\pgfpathmoveto{\pgfqpoint{5.348935in}{1.736333in}}%
\pgfpathlineto{\pgfqpoint{5.357688in}{1.736333in}}%
\pgfpathlineto{\pgfqpoint{5.357688in}{1.993775in}}%
\pgfpathlineto{\pgfqpoint{5.348935in}{1.993775in}}%
\pgfpathlineto{\pgfqpoint{5.348935in}{1.736333in}}%
\pgfpathclose%
\pgfusepath{fill}%
\end{pgfscope}%
\begin{pgfscope}%
\pgfpathrectangle{\pgfqpoint{3.776708in}{0.600000in}}{\pgfqpoint{2.573292in}{2.070576in}}%
\pgfusepath{clip}%
\pgfsetbuttcap%
\pgfsetmiterjoin%
\definecolor{currentfill}{rgb}{0.302379,0.450282,0.300122}%
\pgfsetfillcolor{currentfill}%
\pgfsetlinewidth{0.000000pt}%
\definecolor{currentstroke}{rgb}{0.000000,0.000000,0.000000}%
\pgfsetstrokecolor{currentstroke}%
\pgfsetstrokeopacity{0.000000}%
\pgfsetdash{}{0pt}%
\pgfpathmoveto{\pgfqpoint{5.359877in}{1.738517in}}%
\pgfpathlineto{\pgfqpoint{5.368630in}{1.738517in}}%
\pgfpathlineto{\pgfqpoint{5.368630in}{1.988048in}}%
\pgfpathlineto{\pgfqpoint{5.359877in}{1.988048in}}%
\pgfpathlineto{\pgfqpoint{5.359877in}{1.738517in}}%
\pgfpathclose%
\pgfusepath{fill}%
\end{pgfscope}%
\begin{pgfscope}%
\pgfpathrectangle{\pgfqpoint{3.776708in}{0.600000in}}{\pgfqpoint{2.573292in}{2.070576in}}%
\pgfusepath{clip}%
\pgfsetbuttcap%
\pgfsetmiterjoin%
\definecolor{currentfill}{rgb}{0.302379,0.450282,0.300122}%
\pgfsetfillcolor{currentfill}%
\pgfsetlinewidth{0.000000pt}%
\definecolor{currentstroke}{rgb}{0.000000,0.000000,0.000000}%
\pgfsetstrokecolor{currentstroke}%
\pgfsetstrokeopacity{0.000000}%
\pgfsetdash{}{0pt}%
\pgfpathmoveto{\pgfqpoint{5.370818in}{1.739853in}}%
\pgfpathlineto{\pgfqpoint{5.379572in}{1.739853in}}%
\pgfpathlineto{\pgfqpoint{5.379572in}{1.985176in}}%
\pgfpathlineto{\pgfqpoint{5.370818in}{1.985176in}}%
\pgfpathlineto{\pgfqpoint{5.370818in}{1.739853in}}%
\pgfpathclose%
\pgfusepath{fill}%
\end{pgfscope}%
\begin{pgfscope}%
\pgfpathrectangle{\pgfqpoint{3.776708in}{0.600000in}}{\pgfqpoint{2.573292in}{2.070576in}}%
\pgfusepath{clip}%
\pgfsetbuttcap%
\pgfsetmiterjoin%
\definecolor{currentfill}{rgb}{0.302379,0.450282,0.300122}%
\pgfsetfillcolor{currentfill}%
\pgfsetlinewidth{0.000000pt}%
\definecolor{currentstroke}{rgb}{0.000000,0.000000,0.000000}%
\pgfsetstrokecolor{currentstroke}%
\pgfsetstrokeopacity{0.000000}%
\pgfsetdash{}{0pt}%
\pgfpathmoveto{\pgfqpoint{5.381760in}{1.739120in}}%
\pgfpathlineto{\pgfqpoint{5.390514in}{1.739120in}}%
\pgfpathlineto{\pgfqpoint{5.390514in}{1.984319in}}%
\pgfpathlineto{\pgfqpoint{5.381760in}{1.984319in}}%
\pgfpathlineto{\pgfqpoint{5.381760in}{1.739120in}}%
\pgfpathclose%
\pgfusepath{fill}%
\end{pgfscope}%
\begin{pgfscope}%
\pgfpathrectangle{\pgfqpoint{3.776708in}{0.600000in}}{\pgfqpoint{2.573292in}{2.070576in}}%
\pgfusepath{clip}%
\pgfsetbuttcap%
\pgfsetmiterjoin%
\definecolor{currentfill}{rgb}{0.302379,0.450282,0.300122}%
\pgfsetfillcolor{currentfill}%
\pgfsetlinewidth{0.000000pt}%
\definecolor{currentstroke}{rgb}{0.000000,0.000000,0.000000}%
\pgfsetstrokecolor{currentstroke}%
\pgfsetstrokeopacity{0.000000}%
\pgfsetdash{}{0pt}%
\pgfpathmoveto{\pgfqpoint{5.392702in}{1.736386in}}%
\pgfpathlineto{\pgfqpoint{5.401455in}{1.736386in}}%
\pgfpathlineto{\pgfqpoint{5.401455in}{1.984353in}}%
\pgfpathlineto{\pgfqpoint{5.392702in}{1.984353in}}%
\pgfpathlineto{\pgfqpoint{5.392702in}{1.736386in}}%
\pgfpathclose%
\pgfusepath{fill}%
\end{pgfscope}%
\begin{pgfscope}%
\pgfpathrectangle{\pgfqpoint{3.776708in}{0.600000in}}{\pgfqpoint{2.573292in}{2.070576in}}%
\pgfusepath{clip}%
\pgfsetbuttcap%
\pgfsetmiterjoin%
\definecolor{currentfill}{rgb}{0.302379,0.450282,0.300122}%
\pgfsetfillcolor{currentfill}%
\pgfsetlinewidth{0.000000pt}%
\definecolor{currentstroke}{rgb}{0.000000,0.000000,0.000000}%
\pgfsetstrokecolor{currentstroke}%
\pgfsetstrokeopacity{0.000000}%
\pgfsetdash{}{0pt}%
\pgfpathmoveto{\pgfqpoint{5.403644in}{1.733194in}}%
\pgfpathlineto{\pgfqpoint{5.412397in}{1.733194in}}%
\pgfpathlineto{\pgfqpoint{5.412397in}{1.979928in}}%
\pgfpathlineto{\pgfqpoint{5.403644in}{1.979928in}}%
\pgfpathlineto{\pgfqpoint{5.403644in}{1.733194in}}%
\pgfpathclose%
\pgfusepath{fill}%
\end{pgfscope}%
\begin{pgfscope}%
\pgfpathrectangle{\pgfqpoint{3.776708in}{0.600000in}}{\pgfqpoint{2.573292in}{2.070576in}}%
\pgfusepath{clip}%
\pgfsetbuttcap%
\pgfsetmiterjoin%
\definecolor{currentfill}{rgb}{0.302379,0.450282,0.300122}%
\pgfsetfillcolor{currentfill}%
\pgfsetlinewidth{0.000000pt}%
\definecolor{currentstroke}{rgb}{0.000000,0.000000,0.000000}%
\pgfsetstrokecolor{currentstroke}%
\pgfsetstrokeopacity{0.000000}%
\pgfsetdash{}{0pt}%
\pgfpathmoveto{\pgfqpoint{5.414586in}{1.731762in}}%
\pgfpathlineto{\pgfqpoint{5.423339in}{1.731762in}}%
\pgfpathlineto{\pgfqpoint{5.423339in}{1.973566in}}%
\pgfpathlineto{\pgfqpoint{5.414586in}{1.973566in}}%
\pgfpathlineto{\pgfqpoint{5.414586in}{1.731762in}}%
\pgfpathclose%
\pgfusepath{fill}%
\end{pgfscope}%
\begin{pgfscope}%
\pgfpathrectangle{\pgfqpoint{3.776708in}{0.600000in}}{\pgfqpoint{2.573292in}{2.070576in}}%
\pgfusepath{clip}%
\pgfsetbuttcap%
\pgfsetmiterjoin%
\definecolor{currentfill}{rgb}{0.302379,0.450282,0.300122}%
\pgfsetfillcolor{currentfill}%
\pgfsetlinewidth{0.000000pt}%
\definecolor{currentstroke}{rgb}{0.000000,0.000000,0.000000}%
\pgfsetstrokecolor{currentstroke}%
\pgfsetstrokeopacity{0.000000}%
\pgfsetdash{}{0pt}%
\pgfpathmoveto{\pgfqpoint{5.425527in}{1.726860in}}%
\pgfpathlineto{\pgfqpoint{5.434281in}{1.726860in}}%
\pgfpathlineto{\pgfqpoint{5.434281in}{1.959786in}}%
\pgfpathlineto{\pgfqpoint{5.425527in}{1.959786in}}%
\pgfpathlineto{\pgfqpoint{5.425527in}{1.726860in}}%
\pgfpathclose%
\pgfusepath{fill}%
\end{pgfscope}%
\begin{pgfscope}%
\pgfpathrectangle{\pgfqpoint{3.776708in}{0.600000in}}{\pgfqpoint{2.573292in}{2.070576in}}%
\pgfusepath{clip}%
\pgfsetbuttcap%
\pgfsetmiterjoin%
\definecolor{currentfill}{rgb}{0.302379,0.450282,0.300122}%
\pgfsetfillcolor{currentfill}%
\pgfsetlinewidth{0.000000pt}%
\definecolor{currentstroke}{rgb}{0.000000,0.000000,0.000000}%
\pgfsetstrokecolor{currentstroke}%
\pgfsetstrokeopacity{0.000000}%
\pgfsetdash{}{0pt}%
\pgfpathmoveto{\pgfqpoint{5.436469in}{1.719898in}}%
\pgfpathlineto{\pgfqpoint{5.445223in}{1.719898in}}%
\pgfpathlineto{\pgfqpoint{5.445223in}{1.938271in}}%
\pgfpathlineto{\pgfqpoint{5.436469in}{1.938271in}}%
\pgfpathlineto{\pgfqpoint{5.436469in}{1.719898in}}%
\pgfpathclose%
\pgfusepath{fill}%
\end{pgfscope}%
\begin{pgfscope}%
\pgfpathrectangle{\pgfqpoint{3.776708in}{0.600000in}}{\pgfqpoint{2.573292in}{2.070576in}}%
\pgfusepath{clip}%
\pgfsetbuttcap%
\pgfsetmiterjoin%
\definecolor{currentfill}{rgb}{0.302379,0.450282,0.300122}%
\pgfsetfillcolor{currentfill}%
\pgfsetlinewidth{0.000000pt}%
\definecolor{currentstroke}{rgb}{0.000000,0.000000,0.000000}%
\pgfsetstrokecolor{currentstroke}%
\pgfsetstrokeopacity{0.000000}%
\pgfsetdash{}{0pt}%
\pgfpathmoveto{\pgfqpoint{5.447411in}{1.715168in}}%
\pgfpathlineto{\pgfqpoint{5.456164in}{1.715168in}}%
\pgfpathlineto{\pgfqpoint{5.456164in}{1.912266in}}%
\pgfpathlineto{\pgfqpoint{5.447411in}{1.912266in}}%
\pgfpathlineto{\pgfqpoint{5.447411in}{1.715168in}}%
\pgfpathclose%
\pgfusepath{fill}%
\end{pgfscope}%
\begin{pgfscope}%
\pgfpathrectangle{\pgfqpoint{3.776708in}{0.600000in}}{\pgfqpoint{2.573292in}{2.070576in}}%
\pgfusepath{clip}%
\pgfsetbuttcap%
\pgfsetmiterjoin%
\definecolor{currentfill}{rgb}{0.302379,0.450282,0.300122}%
\pgfsetfillcolor{currentfill}%
\pgfsetlinewidth{0.000000pt}%
\definecolor{currentstroke}{rgb}{0.000000,0.000000,0.000000}%
\pgfsetstrokecolor{currentstroke}%
\pgfsetstrokeopacity{0.000000}%
\pgfsetdash{}{0pt}%
\pgfpathmoveto{\pgfqpoint{5.458353in}{1.709292in}}%
\pgfpathlineto{\pgfqpoint{5.467106in}{1.709292in}}%
\pgfpathlineto{\pgfqpoint{5.467106in}{1.881998in}}%
\pgfpathlineto{\pgfqpoint{5.458353in}{1.881998in}}%
\pgfpathlineto{\pgfqpoint{5.458353in}{1.709292in}}%
\pgfpathclose%
\pgfusepath{fill}%
\end{pgfscope}%
\begin{pgfscope}%
\pgfpathrectangle{\pgfqpoint{3.776708in}{0.600000in}}{\pgfqpoint{2.573292in}{2.070576in}}%
\pgfusepath{clip}%
\pgfsetbuttcap%
\pgfsetmiterjoin%
\definecolor{currentfill}{rgb}{0.302379,0.450282,0.300122}%
\pgfsetfillcolor{currentfill}%
\pgfsetlinewidth{0.000000pt}%
\definecolor{currentstroke}{rgb}{0.000000,0.000000,0.000000}%
\pgfsetstrokecolor{currentstroke}%
\pgfsetstrokeopacity{0.000000}%
\pgfsetdash{}{0pt}%
\pgfpathmoveto{\pgfqpoint{5.469295in}{1.704078in}}%
\pgfpathlineto{\pgfqpoint{5.478048in}{1.704078in}}%
\pgfpathlineto{\pgfqpoint{5.478048in}{1.851084in}}%
\pgfpathlineto{\pgfqpoint{5.469295in}{1.851084in}}%
\pgfpathlineto{\pgfqpoint{5.469295in}{1.704078in}}%
\pgfpathclose%
\pgfusepath{fill}%
\end{pgfscope}%
\begin{pgfscope}%
\pgfpathrectangle{\pgfqpoint{3.776708in}{0.600000in}}{\pgfqpoint{2.573292in}{2.070576in}}%
\pgfusepath{clip}%
\pgfsetbuttcap%
\pgfsetmiterjoin%
\definecolor{currentfill}{rgb}{0.302379,0.450282,0.300122}%
\pgfsetfillcolor{currentfill}%
\pgfsetlinewidth{0.000000pt}%
\definecolor{currentstroke}{rgb}{0.000000,0.000000,0.000000}%
\pgfsetstrokecolor{currentstroke}%
\pgfsetstrokeopacity{0.000000}%
\pgfsetdash{}{0pt}%
\pgfpathmoveto{\pgfqpoint{5.480236in}{1.700426in}}%
\pgfpathlineto{\pgfqpoint{5.488990in}{1.700426in}}%
\pgfpathlineto{\pgfqpoint{5.488990in}{1.820553in}}%
\pgfpathlineto{\pgfqpoint{5.480236in}{1.820553in}}%
\pgfpathlineto{\pgfqpoint{5.480236in}{1.700426in}}%
\pgfpathclose%
\pgfusepath{fill}%
\end{pgfscope}%
\begin{pgfscope}%
\pgfpathrectangle{\pgfqpoint{3.776708in}{0.600000in}}{\pgfqpoint{2.573292in}{2.070576in}}%
\pgfusepath{clip}%
\pgfsetbuttcap%
\pgfsetmiterjoin%
\definecolor{currentfill}{rgb}{0.302379,0.450282,0.300122}%
\pgfsetfillcolor{currentfill}%
\pgfsetlinewidth{0.000000pt}%
\definecolor{currentstroke}{rgb}{0.000000,0.000000,0.000000}%
\pgfsetstrokecolor{currentstroke}%
\pgfsetstrokeopacity{0.000000}%
\pgfsetdash{}{0pt}%
\pgfpathmoveto{\pgfqpoint{5.491178in}{1.694448in}}%
\pgfpathlineto{\pgfqpoint{5.499932in}{1.694448in}}%
\pgfpathlineto{\pgfqpoint{5.499932in}{1.790230in}}%
\pgfpathlineto{\pgfqpoint{5.491178in}{1.790230in}}%
\pgfpathlineto{\pgfqpoint{5.491178in}{1.694448in}}%
\pgfpathclose%
\pgfusepath{fill}%
\end{pgfscope}%
\begin{pgfscope}%
\pgfpathrectangle{\pgfqpoint{3.776708in}{0.600000in}}{\pgfqpoint{2.573292in}{2.070576in}}%
\pgfusepath{clip}%
\pgfsetbuttcap%
\pgfsetmiterjoin%
\definecolor{currentfill}{rgb}{0.302379,0.450282,0.300122}%
\pgfsetfillcolor{currentfill}%
\pgfsetlinewidth{0.000000pt}%
\definecolor{currentstroke}{rgb}{0.000000,0.000000,0.000000}%
\pgfsetstrokecolor{currentstroke}%
\pgfsetstrokeopacity{0.000000}%
\pgfsetdash{}{0pt}%
\pgfpathmoveto{\pgfqpoint{5.502120in}{1.686971in}}%
\pgfpathlineto{\pgfqpoint{5.510873in}{1.686971in}}%
\pgfpathlineto{\pgfqpoint{5.510873in}{1.759890in}}%
\pgfpathlineto{\pgfqpoint{5.502120in}{1.759890in}}%
\pgfpathlineto{\pgfqpoint{5.502120in}{1.686971in}}%
\pgfpathclose%
\pgfusepath{fill}%
\end{pgfscope}%
\begin{pgfscope}%
\pgfpathrectangle{\pgfqpoint{3.776708in}{0.600000in}}{\pgfqpoint{2.573292in}{2.070576in}}%
\pgfusepath{clip}%
\pgfsetbuttcap%
\pgfsetmiterjoin%
\definecolor{currentfill}{rgb}{0.302379,0.450282,0.300122}%
\pgfsetfillcolor{currentfill}%
\pgfsetlinewidth{0.000000pt}%
\definecolor{currentstroke}{rgb}{0.000000,0.000000,0.000000}%
\pgfsetstrokecolor{currentstroke}%
\pgfsetstrokeopacity{0.000000}%
\pgfsetdash{}{0pt}%
\pgfpathmoveto{\pgfqpoint{5.513062in}{1.681952in}}%
\pgfpathlineto{\pgfqpoint{5.521815in}{1.681952in}}%
\pgfpathlineto{\pgfqpoint{5.521815in}{1.729284in}}%
\pgfpathlineto{\pgfqpoint{5.513062in}{1.729284in}}%
\pgfpathlineto{\pgfqpoint{5.513062in}{1.681952in}}%
\pgfpathclose%
\pgfusepath{fill}%
\end{pgfscope}%
\begin{pgfscope}%
\pgfpathrectangle{\pgfqpoint{3.776708in}{0.600000in}}{\pgfqpoint{2.573292in}{2.070576in}}%
\pgfusepath{clip}%
\pgfsetbuttcap%
\pgfsetmiterjoin%
\definecolor{currentfill}{rgb}{0.302379,0.450282,0.300122}%
\pgfsetfillcolor{currentfill}%
\pgfsetlinewidth{0.000000pt}%
\definecolor{currentstroke}{rgb}{0.000000,0.000000,0.000000}%
\pgfsetstrokecolor{currentstroke}%
\pgfsetstrokeopacity{0.000000}%
\pgfsetdash{}{0pt}%
\pgfpathmoveto{\pgfqpoint{5.524004in}{1.676364in}}%
\pgfpathlineto{\pgfqpoint{5.532757in}{1.676364in}}%
\pgfpathlineto{\pgfqpoint{5.532757in}{1.698552in}}%
\pgfpathlineto{\pgfqpoint{5.524004in}{1.698552in}}%
\pgfpathlineto{\pgfqpoint{5.524004in}{1.676364in}}%
\pgfpathclose%
\pgfusepath{fill}%
\end{pgfscope}%
\begin{pgfscope}%
\pgfpathrectangle{\pgfqpoint{3.776708in}{0.600000in}}{\pgfqpoint{2.573292in}{2.070576in}}%
\pgfusepath{clip}%
\pgfsetbuttcap%
\pgfsetmiterjoin%
\definecolor{currentfill}{rgb}{0.302379,0.450282,0.300122}%
\pgfsetfillcolor{currentfill}%
\pgfsetlinewidth{0.000000pt}%
\definecolor{currentstroke}{rgb}{0.000000,0.000000,0.000000}%
\pgfsetstrokecolor{currentstroke}%
\pgfsetstrokeopacity{0.000000}%
\pgfsetdash{}{0pt}%
\pgfpathmoveto{\pgfqpoint{5.534945in}{1.385576in}}%
\pgfpathlineto{\pgfqpoint{5.543699in}{1.385576in}}%
\pgfpathlineto{\pgfqpoint{5.543699in}{1.382388in}}%
\pgfpathlineto{\pgfqpoint{5.534945in}{1.382388in}}%
\pgfpathlineto{\pgfqpoint{5.534945in}{1.385576in}}%
\pgfpathclose%
\pgfusepath{fill}%
\end{pgfscope}%
\begin{pgfscope}%
\pgfpathrectangle{\pgfqpoint{3.776708in}{0.600000in}}{\pgfqpoint{2.573292in}{2.070576in}}%
\pgfusepath{clip}%
\pgfsetbuttcap%
\pgfsetmiterjoin%
\definecolor{currentfill}{rgb}{0.302379,0.450282,0.300122}%
\pgfsetfillcolor{currentfill}%
\pgfsetlinewidth{0.000000pt}%
\definecolor{currentstroke}{rgb}{0.000000,0.000000,0.000000}%
\pgfsetstrokecolor{currentstroke}%
\pgfsetstrokeopacity{0.000000}%
\pgfsetdash{}{0pt}%
\pgfpathmoveto{\pgfqpoint{5.545887in}{1.410323in}}%
\pgfpathlineto{\pgfqpoint{5.554641in}{1.410323in}}%
\pgfpathlineto{\pgfqpoint{5.554641in}{1.382000in}}%
\pgfpathlineto{\pgfqpoint{5.545887in}{1.382000in}}%
\pgfpathlineto{\pgfqpoint{5.545887in}{1.410323in}}%
\pgfpathclose%
\pgfusepath{fill}%
\end{pgfscope}%
\begin{pgfscope}%
\pgfpathrectangle{\pgfqpoint{3.776708in}{0.600000in}}{\pgfqpoint{2.573292in}{2.070576in}}%
\pgfusepath{clip}%
\pgfsetbuttcap%
\pgfsetmiterjoin%
\definecolor{currentfill}{rgb}{0.302379,0.450282,0.300122}%
\pgfsetfillcolor{currentfill}%
\pgfsetlinewidth{0.000000pt}%
\definecolor{currentstroke}{rgb}{0.000000,0.000000,0.000000}%
\pgfsetstrokecolor{currentstroke}%
\pgfsetstrokeopacity{0.000000}%
\pgfsetdash{}{0pt}%
\pgfpathmoveto{\pgfqpoint{5.556829in}{1.427853in}}%
\pgfpathlineto{\pgfqpoint{5.565582in}{1.427853in}}%
\pgfpathlineto{\pgfqpoint{5.565582in}{1.379518in}}%
\pgfpathlineto{\pgfqpoint{5.556829in}{1.379518in}}%
\pgfpathlineto{\pgfqpoint{5.556829in}{1.427853in}}%
\pgfpathclose%
\pgfusepath{fill}%
\end{pgfscope}%
\begin{pgfscope}%
\pgfpathrectangle{\pgfqpoint{3.776708in}{0.600000in}}{\pgfqpoint{2.573292in}{2.070576in}}%
\pgfusepath{clip}%
\pgfsetbuttcap%
\pgfsetmiterjoin%
\definecolor{currentfill}{rgb}{0.302379,0.450282,0.300122}%
\pgfsetfillcolor{currentfill}%
\pgfsetlinewidth{0.000000pt}%
\definecolor{currentstroke}{rgb}{0.000000,0.000000,0.000000}%
\pgfsetstrokecolor{currentstroke}%
\pgfsetstrokeopacity{0.000000}%
\pgfsetdash{}{0pt}%
\pgfpathmoveto{\pgfqpoint{5.567771in}{1.445159in}}%
\pgfpathlineto{\pgfqpoint{5.576524in}{1.445159in}}%
\pgfpathlineto{\pgfqpoint{5.576524in}{1.377594in}}%
\pgfpathlineto{\pgfqpoint{5.567771in}{1.377594in}}%
\pgfpathlineto{\pgfqpoint{5.567771in}{1.445159in}}%
\pgfpathclose%
\pgfusepath{fill}%
\end{pgfscope}%
\begin{pgfscope}%
\pgfpathrectangle{\pgfqpoint{3.776708in}{0.600000in}}{\pgfqpoint{2.573292in}{2.070576in}}%
\pgfusepath{clip}%
\pgfsetbuttcap%
\pgfsetmiterjoin%
\definecolor{currentfill}{rgb}{0.302379,0.450282,0.300122}%
\pgfsetfillcolor{currentfill}%
\pgfsetlinewidth{0.000000pt}%
\definecolor{currentstroke}{rgb}{0.000000,0.000000,0.000000}%
\pgfsetstrokecolor{currentstroke}%
\pgfsetstrokeopacity{0.000000}%
\pgfsetdash{}{0pt}%
\pgfpathmoveto{\pgfqpoint{5.578713in}{1.459246in}}%
\pgfpathlineto{\pgfqpoint{5.587466in}{1.459246in}}%
\pgfpathlineto{\pgfqpoint{5.587466in}{1.377101in}}%
\pgfpathlineto{\pgfqpoint{5.578713in}{1.377101in}}%
\pgfpathlineto{\pgfqpoint{5.578713in}{1.459246in}}%
\pgfpathclose%
\pgfusepath{fill}%
\end{pgfscope}%
\begin{pgfscope}%
\pgfpathrectangle{\pgfqpoint{3.776708in}{0.600000in}}{\pgfqpoint{2.573292in}{2.070576in}}%
\pgfusepath{clip}%
\pgfsetbuttcap%
\pgfsetmiterjoin%
\definecolor{currentfill}{rgb}{0.302379,0.450282,0.300122}%
\pgfsetfillcolor{currentfill}%
\pgfsetlinewidth{0.000000pt}%
\definecolor{currentstroke}{rgb}{0.000000,0.000000,0.000000}%
\pgfsetstrokecolor{currentstroke}%
\pgfsetstrokeopacity{0.000000}%
\pgfsetdash{}{0pt}%
\pgfpathmoveto{\pgfqpoint{5.589654in}{1.466890in}}%
\pgfpathlineto{\pgfqpoint{5.598408in}{1.466890in}}%
\pgfpathlineto{\pgfqpoint{5.598408in}{1.376128in}}%
\pgfpathlineto{\pgfqpoint{5.589654in}{1.376128in}}%
\pgfpathlineto{\pgfqpoint{5.589654in}{1.466890in}}%
\pgfpathclose%
\pgfusepath{fill}%
\end{pgfscope}%
\begin{pgfscope}%
\pgfpathrectangle{\pgfqpoint{3.776708in}{0.600000in}}{\pgfqpoint{2.573292in}{2.070576in}}%
\pgfusepath{clip}%
\pgfsetbuttcap%
\pgfsetmiterjoin%
\definecolor{currentfill}{rgb}{0.302379,0.450282,0.300122}%
\pgfsetfillcolor{currentfill}%
\pgfsetlinewidth{0.000000pt}%
\definecolor{currentstroke}{rgb}{0.000000,0.000000,0.000000}%
\pgfsetstrokecolor{currentstroke}%
\pgfsetstrokeopacity{0.000000}%
\pgfsetdash{}{0pt}%
\pgfpathmoveto{\pgfqpoint{5.600596in}{1.474494in}}%
\pgfpathlineto{\pgfqpoint{5.609350in}{1.474494in}}%
\pgfpathlineto{\pgfqpoint{5.609350in}{1.375689in}}%
\pgfpathlineto{\pgfqpoint{5.600596in}{1.375689in}}%
\pgfpathlineto{\pgfqpoint{5.600596in}{1.474494in}}%
\pgfpathclose%
\pgfusepath{fill}%
\end{pgfscope}%
\begin{pgfscope}%
\pgfpathrectangle{\pgfqpoint{3.776708in}{0.600000in}}{\pgfqpoint{2.573292in}{2.070576in}}%
\pgfusepath{clip}%
\pgfsetbuttcap%
\pgfsetmiterjoin%
\definecolor{currentfill}{rgb}{0.302379,0.450282,0.300122}%
\pgfsetfillcolor{currentfill}%
\pgfsetlinewidth{0.000000pt}%
\definecolor{currentstroke}{rgb}{0.000000,0.000000,0.000000}%
\pgfsetstrokecolor{currentstroke}%
\pgfsetstrokeopacity{0.000000}%
\pgfsetdash{}{0pt}%
\pgfpathmoveto{\pgfqpoint{5.611538in}{1.479360in}}%
\pgfpathlineto{\pgfqpoint{5.620291in}{1.479360in}}%
\pgfpathlineto{\pgfqpoint{5.620291in}{1.377516in}}%
\pgfpathlineto{\pgfqpoint{5.611538in}{1.377516in}}%
\pgfpathlineto{\pgfqpoint{5.611538in}{1.479360in}}%
\pgfpathclose%
\pgfusepath{fill}%
\end{pgfscope}%
\begin{pgfscope}%
\pgfpathrectangle{\pgfqpoint{3.776708in}{0.600000in}}{\pgfqpoint{2.573292in}{2.070576in}}%
\pgfusepath{clip}%
\pgfsetbuttcap%
\pgfsetmiterjoin%
\definecolor{currentfill}{rgb}{0.302379,0.450282,0.300122}%
\pgfsetfillcolor{currentfill}%
\pgfsetlinewidth{0.000000pt}%
\definecolor{currentstroke}{rgb}{0.000000,0.000000,0.000000}%
\pgfsetstrokecolor{currentstroke}%
\pgfsetstrokeopacity{0.000000}%
\pgfsetdash{}{0pt}%
\pgfpathmoveto{\pgfqpoint{5.622480in}{1.480076in}}%
\pgfpathlineto{\pgfqpoint{5.631233in}{1.480076in}}%
\pgfpathlineto{\pgfqpoint{5.631233in}{1.380940in}}%
\pgfpathlineto{\pgfqpoint{5.622480in}{1.380940in}}%
\pgfpathlineto{\pgfqpoint{5.622480in}{1.480076in}}%
\pgfpathclose%
\pgfusepath{fill}%
\end{pgfscope}%
\begin{pgfscope}%
\pgfpathrectangle{\pgfqpoint{3.776708in}{0.600000in}}{\pgfqpoint{2.573292in}{2.070576in}}%
\pgfusepath{clip}%
\pgfsetbuttcap%
\pgfsetmiterjoin%
\definecolor{currentfill}{rgb}{0.302379,0.450282,0.300122}%
\pgfsetfillcolor{currentfill}%
\pgfsetlinewidth{0.000000pt}%
\definecolor{currentstroke}{rgb}{0.000000,0.000000,0.000000}%
\pgfsetstrokecolor{currentstroke}%
\pgfsetstrokeopacity{0.000000}%
\pgfsetdash{}{0pt}%
\pgfpathmoveto{\pgfqpoint{5.633422in}{1.480487in}}%
\pgfpathlineto{\pgfqpoint{5.642175in}{1.480487in}}%
\pgfpathlineto{\pgfqpoint{5.642175in}{1.383861in}}%
\pgfpathlineto{\pgfqpoint{5.633422in}{1.383861in}}%
\pgfpathlineto{\pgfqpoint{5.633422in}{1.480487in}}%
\pgfpathclose%
\pgfusepath{fill}%
\end{pgfscope}%
\begin{pgfscope}%
\pgfpathrectangle{\pgfqpoint{3.776708in}{0.600000in}}{\pgfqpoint{2.573292in}{2.070576in}}%
\pgfusepath{clip}%
\pgfsetbuttcap%
\pgfsetmiterjoin%
\definecolor{currentfill}{rgb}{0.302379,0.450282,0.300122}%
\pgfsetfillcolor{currentfill}%
\pgfsetlinewidth{0.000000pt}%
\definecolor{currentstroke}{rgb}{0.000000,0.000000,0.000000}%
\pgfsetstrokecolor{currentstroke}%
\pgfsetstrokeopacity{0.000000}%
\pgfsetdash{}{0pt}%
\pgfpathmoveto{\pgfqpoint{5.644363in}{1.480219in}}%
\pgfpathlineto{\pgfqpoint{5.653117in}{1.480219in}}%
\pgfpathlineto{\pgfqpoint{5.653117in}{1.392228in}}%
\pgfpathlineto{\pgfqpoint{5.644363in}{1.392228in}}%
\pgfpathlineto{\pgfqpoint{5.644363in}{1.480219in}}%
\pgfpathclose%
\pgfusepath{fill}%
\end{pgfscope}%
\begin{pgfscope}%
\pgfpathrectangle{\pgfqpoint{3.776708in}{0.600000in}}{\pgfqpoint{2.573292in}{2.070576in}}%
\pgfusepath{clip}%
\pgfsetbuttcap%
\pgfsetmiterjoin%
\definecolor{currentfill}{rgb}{0.302379,0.450282,0.300122}%
\pgfsetfillcolor{currentfill}%
\pgfsetlinewidth{0.000000pt}%
\definecolor{currentstroke}{rgb}{0.000000,0.000000,0.000000}%
\pgfsetstrokecolor{currentstroke}%
\pgfsetstrokeopacity{0.000000}%
\pgfsetdash{}{0pt}%
\pgfpathmoveto{\pgfqpoint{5.655305in}{1.476681in}}%
\pgfpathlineto{\pgfqpoint{5.664059in}{1.476681in}}%
\pgfpathlineto{\pgfqpoint{5.664059in}{1.404245in}}%
\pgfpathlineto{\pgfqpoint{5.655305in}{1.404245in}}%
\pgfpathlineto{\pgfqpoint{5.655305in}{1.476681in}}%
\pgfpathclose%
\pgfusepath{fill}%
\end{pgfscope}%
\begin{pgfscope}%
\pgfpathrectangle{\pgfqpoint{3.776708in}{0.600000in}}{\pgfqpoint{2.573292in}{2.070576in}}%
\pgfusepath{clip}%
\pgfsetbuttcap%
\pgfsetmiterjoin%
\definecolor{currentfill}{rgb}{0.302379,0.450282,0.300122}%
\pgfsetfillcolor{currentfill}%
\pgfsetlinewidth{0.000000pt}%
\definecolor{currentstroke}{rgb}{0.000000,0.000000,0.000000}%
\pgfsetstrokecolor{currentstroke}%
\pgfsetstrokeopacity{0.000000}%
\pgfsetdash{}{0pt}%
\pgfpathmoveto{\pgfqpoint{5.666247in}{1.483344in}}%
\pgfpathlineto{\pgfqpoint{5.675000in}{1.483344in}}%
\pgfpathlineto{\pgfqpoint{5.675000in}{1.415318in}}%
\pgfpathlineto{\pgfqpoint{5.666247in}{1.415318in}}%
\pgfpathlineto{\pgfqpoint{5.666247in}{1.483344in}}%
\pgfpathclose%
\pgfusepath{fill}%
\end{pgfscope}%
\begin{pgfscope}%
\pgfpathrectangle{\pgfqpoint{3.776708in}{0.600000in}}{\pgfqpoint{2.573292in}{2.070576in}}%
\pgfusepath{clip}%
\pgfsetbuttcap%
\pgfsetmiterjoin%
\definecolor{currentfill}{rgb}{0.302379,0.450282,0.300122}%
\pgfsetfillcolor{currentfill}%
\pgfsetlinewidth{0.000000pt}%
\definecolor{currentstroke}{rgb}{0.000000,0.000000,0.000000}%
\pgfsetstrokecolor{currentstroke}%
\pgfsetstrokeopacity{0.000000}%
\pgfsetdash{}{0pt}%
\pgfpathmoveto{\pgfqpoint{5.677189in}{1.484567in}}%
\pgfpathlineto{\pgfqpoint{5.685942in}{1.484567in}}%
\pgfpathlineto{\pgfqpoint{5.685942in}{1.425064in}}%
\pgfpathlineto{\pgfqpoint{5.677189in}{1.425064in}}%
\pgfpathlineto{\pgfqpoint{5.677189in}{1.484567in}}%
\pgfpathclose%
\pgfusepath{fill}%
\end{pgfscope}%
\begin{pgfscope}%
\pgfpathrectangle{\pgfqpoint{3.776708in}{0.600000in}}{\pgfqpoint{2.573292in}{2.070576in}}%
\pgfusepath{clip}%
\pgfsetbuttcap%
\pgfsetmiterjoin%
\definecolor{currentfill}{rgb}{0.302379,0.450282,0.300122}%
\pgfsetfillcolor{currentfill}%
\pgfsetlinewidth{0.000000pt}%
\definecolor{currentstroke}{rgb}{0.000000,0.000000,0.000000}%
\pgfsetstrokecolor{currentstroke}%
\pgfsetstrokeopacity{0.000000}%
\pgfsetdash{}{0pt}%
\pgfpathmoveto{\pgfqpoint{5.688131in}{1.486779in}}%
\pgfpathlineto{\pgfqpoint{5.696884in}{1.486779in}}%
\pgfpathlineto{\pgfqpoint{5.696884in}{1.437364in}}%
\pgfpathlineto{\pgfqpoint{5.688131in}{1.437364in}}%
\pgfpathlineto{\pgfqpoint{5.688131in}{1.486779in}}%
\pgfpathclose%
\pgfusepath{fill}%
\end{pgfscope}%
\begin{pgfscope}%
\pgfpathrectangle{\pgfqpoint{3.776708in}{0.600000in}}{\pgfqpoint{2.573292in}{2.070576in}}%
\pgfusepath{clip}%
\pgfsetbuttcap%
\pgfsetmiterjoin%
\definecolor{currentfill}{rgb}{0.302379,0.450282,0.300122}%
\pgfsetfillcolor{currentfill}%
\pgfsetlinewidth{0.000000pt}%
\definecolor{currentstroke}{rgb}{0.000000,0.000000,0.000000}%
\pgfsetstrokecolor{currentstroke}%
\pgfsetstrokeopacity{0.000000}%
\pgfsetdash{}{0pt}%
\pgfpathmoveto{\pgfqpoint{5.699072in}{1.491434in}}%
\pgfpathlineto{\pgfqpoint{5.707826in}{1.491434in}}%
\pgfpathlineto{\pgfqpoint{5.707826in}{1.453163in}}%
\pgfpathlineto{\pgfqpoint{5.699072in}{1.453163in}}%
\pgfpathlineto{\pgfqpoint{5.699072in}{1.491434in}}%
\pgfpathclose%
\pgfusepath{fill}%
\end{pgfscope}%
\begin{pgfscope}%
\pgfpathrectangle{\pgfqpoint{3.776708in}{0.600000in}}{\pgfqpoint{2.573292in}{2.070576in}}%
\pgfusepath{clip}%
\pgfsetbuttcap%
\pgfsetmiterjoin%
\definecolor{currentfill}{rgb}{0.302379,0.450282,0.300122}%
\pgfsetfillcolor{currentfill}%
\pgfsetlinewidth{0.000000pt}%
\definecolor{currentstroke}{rgb}{0.000000,0.000000,0.000000}%
\pgfsetstrokecolor{currentstroke}%
\pgfsetstrokeopacity{0.000000}%
\pgfsetdash{}{0pt}%
\pgfpathmoveto{\pgfqpoint{5.710014in}{1.498041in}}%
\pgfpathlineto{\pgfqpoint{5.718768in}{1.498041in}}%
\pgfpathlineto{\pgfqpoint{5.718768in}{1.466340in}}%
\pgfpathlineto{\pgfqpoint{5.710014in}{1.466340in}}%
\pgfpathlineto{\pgfqpoint{5.710014in}{1.498041in}}%
\pgfpathclose%
\pgfusepath{fill}%
\end{pgfscope}%
\begin{pgfscope}%
\pgfpathrectangle{\pgfqpoint{3.776708in}{0.600000in}}{\pgfqpoint{2.573292in}{2.070576in}}%
\pgfusepath{clip}%
\pgfsetbuttcap%
\pgfsetmiterjoin%
\definecolor{currentfill}{rgb}{0.302379,0.450282,0.300122}%
\pgfsetfillcolor{currentfill}%
\pgfsetlinewidth{0.000000pt}%
\definecolor{currentstroke}{rgb}{0.000000,0.000000,0.000000}%
\pgfsetstrokecolor{currentstroke}%
\pgfsetstrokeopacity{0.000000}%
\pgfsetdash{}{0pt}%
\pgfpathmoveto{\pgfqpoint{5.720956in}{1.510011in}}%
\pgfpathlineto{\pgfqpoint{5.729709in}{1.510011in}}%
\pgfpathlineto{\pgfqpoint{5.729709in}{1.475267in}}%
\pgfpathlineto{\pgfqpoint{5.720956in}{1.475267in}}%
\pgfpathlineto{\pgfqpoint{5.720956in}{1.510011in}}%
\pgfpathclose%
\pgfusepath{fill}%
\end{pgfscope}%
\begin{pgfscope}%
\pgfpathrectangle{\pgfqpoint{3.776708in}{0.600000in}}{\pgfqpoint{2.573292in}{2.070576in}}%
\pgfusepath{clip}%
\pgfsetbuttcap%
\pgfsetmiterjoin%
\definecolor{currentfill}{rgb}{0.302379,0.450282,0.300122}%
\pgfsetfillcolor{currentfill}%
\pgfsetlinewidth{0.000000pt}%
\definecolor{currentstroke}{rgb}{0.000000,0.000000,0.000000}%
\pgfsetstrokecolor{currentstroke}%
\pgfsetstrokeopacity{0.000000}%
\pgfsetdash{}{0pt}%
\pgfpathmoveto{\pgfqpoint{5.731898in}{1.532156in}}%
\pgfpathlineto{\pgfqpoint{5.740651in}{1.532156in}}%
\pgfpathlineto{\pgfqpoint{5.740651in}{1.480394in}}%
\pgfpathlineto{\pgfqpoint{5.731898in}{1.480394in}}%
\pgfpathlineto{\pgfqpoint{5.731898in}{1.532156in}}%
\pgfpathclose%
\pgfusepath{fill}%
\end{pgfscope}%
\begin{pgfscope}%
\pgfpathrectangle{\pgfqpoint{3.776708in}{0.600000in}}{\pgfqpoint{2.573292in}{2.070576in}}%
\pgfusepath{clip}%
\pgfsetbuttcap%
\pgfsetmiterjoin%
\definecolor{currentfill}{rgb}{0.302379,0.450282,0.300122}%
\pgfsetfillcolor{currentfill}%
\pgfsetlinewidth{0.000000pt}%
\definecolor{currentstroke}{rgb}{0.000000,0.000000,0.000000}%
\pgfsetstrokecolor{currentstroke}%
\pgfsetstrokeopacity{0.000000}%
\pgfsetdash{}{0pt}%
\pgfpathmoveto{\pgfqpoint{5.742840in}{1.545400in}}%
\pgfpathlineto{\pgfqpoint{5.751593in}{1.545400in}}%
\pgfpathlineto{\pgfqpoint{5.751593in}{1.482100in}}%
\pgfpathlineto{\pgfqpoint{5.742840in}{1.482100in}}%
\pgfpathlineto{\pgfqpoint{5.742840in}{1.545400in}}%
\pgfpathclose%
\pgfusepath{fill}%
\end{pgfscope}%
\begin{pgfscope}%
\pgfpathrectangle{\pgfqpoint{3.776708in}{0.600000in}}{\pgfqpoint{2.573292in}{2.070576in}}%
\pgfusepath{clip}%
\pgfsetbuttcap%
\pgfsetmiterjoin%
\definecolor{currentfill}{rgb}{0.302379,0.450282,0.300122}%
\pgfsetfillcolor{currentfill}%
\pgfsetlinewidth{0.000000pt}%
\definecolor{currentstroke}{rgb}{0.000000,0.000000,0.000000}%
\pgfsetstrokecolor{currentstroke}%
\pgfsetstrokeopacity{0.000000}%
\pgfsetdash{}{0pt}%
\pgfpathmoveto{\pgfqpoint{5.753781in}{1.556350in}}%
\pgfpathlineto{\pgfqpoint{5.762535in}{1.556350in}}%
\pgfpathlineto{\pgfqpoint{5.762535in}{1.480588in}}%
\pgfpathlineto{\pgfqpoint{5.753781in}{1.480588in}}%
\pgfpathlineto{\pgfqpoint{5.753781in}{1.556350in}}%
\pgfpathclose%
\pgfusepath{fill}%
\end{pgfscope}%
\begin{pgfscope}%
\pgfpathrectangle{\pgfqpoint{3.776708in}{0.600000in}}{\pgfqpoint{2.573292in}{2.070576in}}%
\pgfusepath{clip}%
\pgfsetbuttcap%
\pgfsetmiterjoin%
\definecolor{currentfill}{rgb}{0.302379,0.450282,0.300122}%
\pgfsetfillcolor{currentfill}%
\pgfsetlinewidth{0.000000pt}%
\definecolor{currentstroke}{rgb}{0.000000,0.000000,0.000000}%
\pgfsetstrokecolor{currentstroke}%
\pgfsetstrokeopacity{0.000000}%
\pgfsetdash{}{0pt}%
\pgfpathmoveto{\pgfqpoint{5.764723in}{1.569202in}}%
\pgfpathlineto{\pgfqpoint{5.773477in}{1.569202in}}%
\pgfpathlineto{\pgfqpoint{5.773477in}{1.478621in}}%
\pgfpathlineto{\pgfqpoint{5.764723in}{1.478621in}}%
\pgfpathlineto{\pgfqpoint{5.764723in}{1.569202in}}%
\pgfpathclose%
\pgfusepath{fill}%
\end{pgfscope}%
\begin{pgfscope}%
\pgfpathrectangle{\pgfqpoint{3.776708in}{0.600000in}}{\pgfqpoint{2.573292in}{2.070576in}}%
\pgfusepath{clip}%
\pgfsetbuttcap%
\pgfsetmiterjoin%
\definecolor{currentfill}{rgb}{0.302379,0.450282,0.300122}%
\pgfsetfillcolor{currentfill}%
\pgfsetlinewidth{0.000000pt}%
\definecolor{currentstroke}{rgb}{0.000000,0.000000,0.000000}%
\pgfsetstrokecolor{currentstroke}%
\pgfsetstrokeopacity{0.000000}%
\pgfsetdash{}{0pt}%
\pgfpathmoveto{\pgfqpoint{5.775665in}{1.569692in}}%
\pgfpathlineto{\pgfqpoint{5.784418in}{1.569692in}}%
\pgfpathlineto{\pgfqpoint{5.784418in}{1.460370in}}%
\pgfpathlineto{\pgfqpoint{5.775665in}{1.460370in}}%
\pgfpathlineto{\pgfqpoint{5.775665in}{1.569692in}}%
\pgfpathclose%
\pgfusepath{fill}%
\end{pgfscope}%
\begin{pgfscope}%
\pgfpathrectangle{\pgfqpoint{3.776708in}{0.600000in}}{\pgfqpoint{2.573292in}{2.070576in}}%
\pgfusepath{clip}%
\pgfsetbuttcap%
\pgfsetmiterjoin%
\definecolor{currentfill}{rgb}{0.302379,0.450282,0.300122}%
\pgfsetfillcolor{currentfill}%
\pgfsetlinewidth{0.000000pt}%
\definecolor{currentstroke}{rgb}{0.000000,0.000000,0.000000}%
\pgfsetstrokecolor{currentstroke}%
\pgfsetstrokeopacity{0.000000}%
\pgfsetdash{}{0pt}%
\pgfpathmoveto{\pgfqpoint{5.786607in}{1.558623in}}%
\pgfpathlineto{\pgfqpoint{5.795360in}{1.558623in}}%
\pgfpathlineto{\pgfqpoint{5.795360in}{1.427572in}}%
\pgfpathlineto{\pgfqpoint{5.786607in}{1.427572in}}%
\pgfpathlineto{\pgfqpoint{5.786607in}{1.558623in}}%
\pgfpathclose%
\pgfusepath{fill}%
\end{pgfscope}%
\begin{pgfscope}%
\pgfpathrectangle{\pgfqpoint{3.776708in}{0.600000in}}{\pgfqpoint{2.573292in}{2.070576in}}%
\pgfusepath{clip}%
\pgfsetbuttcap%
\pgfsetmiterjoin%
\definecolor{currentfill}{rgb}{0.302379,0.450282,0.300122}%
\pgfsetfillcolor{currentfill}%
\pgfsetlinewidth{0.000000pt}%
\definecolor{currentstroke}{rgb}{0.000000,0.000000,0.000000}%
\pgfsetstrokecolor{currentstroke}%
\pgfsetstrokeopacity{0.000000}%
\pgfsetdash{}{0pt}%
\pgfpathmoveto{\pgfqpoint{5.797549in}{1.548656in}}%
\pgfpathlineto{\pgfqpoint{5.806302in}{1.548656in}}%
\pgfpathlineto{\pgfqpoint{5.806302in}{1.396764in}}%
\pgfpathlineto{\pgfqpoint{5.797549in}{1.396764in}}%
\pgfpathlineto{\pgfqpoint{5.797549in}{1.548656in}}%
\pgfpathclose%
\pgfusepath{fill}%
\end{pgfscope}%
\begin{pgfscope}%
\pgfpathrectangle{\pgfqpoint{3.776708in}{0.600000in}}{\pgfqpoint{2.573292in}{2.070576in}}%
\pgfusepath{clip}%
\pgfsetbuttcap%
\pgfsetmiterjoin%
\definecolor{currentfill}{rgb}{0.302379,0.450282,0.300122}%
\pgfsetfillcolor{currentfill}%
\pgfsetlinewidth{0.000000pt}%
\definecolor{currentstroke}{rgb}{0.000000,0.000000,0.000000}%
\pgfsetstrokecolor{currentstroke}%
\pgfsetstrokeopacity{0.000000}%
\pgfsetdash{}{0pt}%
\pgfpathmoveto{\pgfqpoint{5.808490in}{1.539838in}}%
\pgfpathlineto{\pgfqpoint{5.817244in}{1.539838in}}%
\pgfpathlineto{\pgfqpoint{5.817244in}{1.365164in}}%
\pgfpathlineto{\pgfqpoint{5.808490in}{1.365164in}}%
\pgfpathlineto{\pgfqpoint{5.808490in}{1.539838in}}%
\pgfpathclose%
\pgfusepath{fill}%
\end{pgfscope}%
\begin{pgfscope}%
\pgfpathrectangle{\pgfqpoint{3.776708in}{0.600000in}}{\pgfqpoint{2.573292in}{2.070576in}}%
\pgfusepath{clip}%
\pgfsetbuttcap%
\pgfsetmiterjoin%
\definecolor{currentfill}{rgb}{0.302379,0.450282,0.300122}%
\pgfsetfillcolor{currentfill}%
\pgfsetlinewidth{0.000000pt}%
\definecolor{currentstroke}{rgb}{0.000000,0.000000,0.000000}%
\pgfsetstrokecolor{currentstroke}%
\pgfsetstrokeopacity{0.000000}%
\pgfsetdash{}{0pt}%
\pgfpathmoveto{\pgfqpoint{5.819432in}{1.531718in}}%
\pgfpathlineto{\pgfqpoint{5.828186in}{1.531718in}}%
\pgfpathlineto{\pgfqpoint{5.828186in}{1.339198in}}%
\pgfpathlineto{\pgfqpoint{5.819432in}{1.339198in}}%
\pgfpathlineto{\pgfqpoint{5.819432in}{1.531718in}}%
\pgfpathclose%
\pgfusepath{fill}%
\end{pgfscope}%
\begin{pgfscope}%
\pgfpathrectangle{\pgfqpoint{3.776708in}{0.600000in}}{\pgfqpoint{2.573292in}{2.070576in}}%
\pgfusepath{clip}%
\pgfsetbuttcap%
\pgfsetmiterjoin%
\definecolor{currentfill}{rgb}{0.302379,0.450282,0.300122}%
\pgfsetfillcolor{currentfill}%
\pgfsetlinewidth{0.000000pt}%
\definecolor{currentstroke}{rgb}{0.000000,0.000000,0.000000}%
\pgfsetstrokecolor{currentstroke}%
\pgfsetstrokeopacity{0.000000}%
\pgfsetdash{}{0pt}%
\pgfpathmoveto{\pgfqpoint{5.830374in}{1.526790in}}%
\pgfpathlineto{\pgfqpoint{5.839127in}{1.526790in}}%
\pgfpathlineto{\pgfqpoint{5.839127in}{1.311872in}}%
\pgfpathlineto{\pgfqpoint{5.830374in}{1.311872in}}%
\pgfpathlineto{\pgfqpoint{5.830374in}{1.526790in}}%
\pgfpathclose%
\pgfusepath{fill}%
\end{pgfscope}%
\begin{pgfscope}%
\pgfpathrectangle{\pgfqpoint{3.776708in}{0.600000in}}{\pgfqpoint{2.573292in}{2.070576in}}%
\pgfusepath{clip}%
\pgfsetbuttcap%
\pgfsetmiterjoin%
\definecolor{currentfill}{rgb}{0.302379,0.450282,0.300122}%
\pgfsetfillcolor{currentfill}%
\pgfsetlinewidth{0.000000pt}%
\definecolor{currentstroke}{rgb}{0.000000,0.000000,0.000000}%
\pgfsetstrokecolor{currentstroke}%
\pgfsetstrokeopacity{0.000000}%
\pgfsetdash{}{0pt}%
\pgfpathmoveto{\pgfqpoint{5.841316in}{1.520032in}}%
\pgfpathlineto{\pgfqpoint{5.850069in}{1.520032in}}%
\pgfpathlineto{\pgfqpoint{5.850069in}{1.285098in}}%
\pgfpathlineto{\pgfqpoint{5.841316in}{1.285098in}}%
\pgfpathlineto{\pgfqpoint{5.841316in}{1.520032in}}%
\pgfpathclose%
\pgfusepath{fill}%
\end{pgfscope}%
\begin{pgfscope}%
\pgfpathrectangle{\pgfqpoint{3.776708in}{0.600000in}}{\pgfqpoint{2.573292in}{2.070576in}}%
\pgfusepath{clip}%
\pgfsetbuttcap%
\pgfsetmiterjoin%
\definecolor{currentfill}{rgb}{0.302379,0.450282,0.300122}%
\pgfsetfillcolor{currentfill}%
\pgfsetlinewidth{0.000000pt}%
\definecolor{currentstroke}{rgb}{0.000000,0.000000,0.000000}%
\pgfsetstrokecolor{currentstroke}%
\pgfsetstrokeopacity{0.000000}%
\pgfsetdash{}{0pt}%
\pgfpathmoveto{\pgfqpoint{5.852258in}{1.513669in}}%
\pgfpathlineto{\pgfqpoint{5.861011in}{1.513669in}}%
\pgfpathlineto{\pgfqpoint{5.861011in}{1.257904in}}%
\pgfpathlineto{\pgfqpoint{5.852258in}{1.257904in}}%
\pgfpathlineto{\pgfqpoint{5.852258in}{1.513669in}}%
\pgfpathclose%
\pgfusepath{fill}%
\end{pgfscope}%
\begin{pgfscope}%
\pgfpathrectangle{\pgfqpoint{3.776708in}{0.600000in}}{\pgfqpoint{2.573292in}{2.070576in}}%
\pgfusepath{clip}%
\pgfsetbuttcap%
\pgfsetmiterjoin%
\definecolor{currentfill}{rgb}{0.302379,0.450282,0.300122}%
\pgfsetfillcolor{currentfill}%
\pgfsetlinewidth{0.000000pt}%
\definecolor{currentstroke}{rgb}{0.000000,0.000000,0.000000}%
\pgfsetstrokecolor{currentstroke}%
\pgfsetstrokeopacity{0.000000}%
\pgfsetdash{}{0pt}%
\pgfpathmoveto{\pgfqpoint{5.863199in}{1.509660in}}%
\pgfpathlineto{\pgfqpoint{5.871953in}{1.509660in}}%
\pgfpathlineto{\pgfqpoint{5.871953in}{1.234263in}}%
\pgfpathlineto{\pgfqpoint{5.863199in}{1.234263in}}%
\pgfpathlineto{\pgfqpoint{5.863199in}{1.509660in}}%
\pgfpathclose%
\pgfusepath{fill}%
\end{pgfscope}%
\begin{pgfscope}%
\pgfpathrectangle{\pgfqpoint{3.776708in}{0.600000in}}{\pgfqpoint{2.573292in}{2.070576in}}%
\pgfusepath{clip}%
\pgfsetbuttcap%
\pgfsetmiterjoin%
\definecolor{currentfill}{rgb}{0.302379,0.450282,0.300122}%
\pgfsetfillcolor{currentfill}%
\pgfsetlinewidth{0.000000pt}%
\definecolor{currentstroke}{rgb}{0.000000,0.000000,0.000000}%
\pgfsetstrokecolor{currentstroke}%
\pgfsetstrokeopacity{0.000000}%
\pgfsetdash{}{0pt}%
\pgfpathmoveto{\pgfqpoint{5.874141in}{1.506209in}}%
\pgfpathlineto{\pgfqpoint{5.882895in}{1.506209in}}%
\pgfpathlineto{\pgfqpoint{5.882895in}{1.220406in}}%
\pgfpathlineto{\pgfqpoint{5.874141in}{1.220406in}}%
\pgfpathlineto{\pgfqpoint{5.874141in}{1.506209in}}%
\pgfpathclose%
\pgfusepath{fill}%
\end{pgfscope}%
\begin{pgfscope}%
\pgfpathrectangle{\pgfqpoint{3.776708in}{0.600000in}}{\pgfqpoint{2.573292in}{2.070576in}}%
\pgfusepath{clip}%
\pgfsetbuttcap%
\pgfsetmiterjoin%
\definecolor{currentfill}{rgb}{0.302379,0.450282,0.300122}%
\pgfsetfillcolor{currentfill}%
\pgfsetlinewidth{0.000000pt}%
\definecolor{currentstroke}{rgb}{0.000000,0.000000,0.000000}%
\pgfsetstrokecolor{currentstroke}%
\pgfsetstrokeopacity{0.000000}%
\pgfsetdash{}{0pt}%
\pgfpathmoveto{\pgfqpoint{5.885083in}{1.505150in}}%
\pgfpathlineto{\pgfqpoint{5.893836in}{1.505150in}}%
\pgfpathlineto{\pgfqpoint{5.893836in}{1.201964in}}%
\pgfpathlineto{\pgfqpoint{5.885083in}{1.201964in}}%
\pgfpathlineto{\pgfqpoint{5.885083in}{1.505150in}}%
\pgfpathclose%
\pgfusepath{fill}%
\end{pgfscope}%
\begin{pgfscope}%
\pgfpathrectangle{\pgfqpoint{3.776708in}{0.600000in}}{\pgfqpoint{2.573292in}{2.070576in}}%
\pgfusepath{clip}%
\pgfsetbuttcap%
\pgfsetmiterjoin%
\definecolor{currentfill}{rgb}{0.302379,0.450282,0.300122}%
\pgfsetfillcolor{currentfill}%
\pgfsetlinewidth{0.000000pt}%
\definecolor{currentstroke}{rgb}{0.000000,0.000000,0.000000}%
\pgfsetstrokecolor{currentstroke}%
\pgfsetstrokeopacity{0.000000}%
\pgfsetdash{}{0pt}%
\pgfpathmoveto{\pgfqpoint{5.896025in}{1.503887in}}%
\pgfpathlineto{\pgfqpoint{5.904778in}{1.503887in}}%
\pgfpathlineto{\pgfqpoint{5.904778in}{1.187582in}}%
\pgfpathlineto{\pgfqpoint{5.896025in}{1.187582in}}%
\pgfpathlineto{\pgfqpoint{5.896025in}{1.503887in}}%
\pgfpathclose%
\pgfusepath{fill}%
\end{pgfscope}%
\begin{pgfscope}%
\pgfpathrectangle{\pgfqpoint{3.776708in}{0.600000in}}{\pgfqpoint{2.573292in}{2.070576in}}%
\pgfusepath{clip}%
\pgfsetbuttcap%
\pgfsetmiterjoin%
\definecolor{currentfill}{rgb}{0.302379,0.450282,0.300122}%
\pgfsetfillcolor{currentfill}%
\pgfsetlinewidth{0.000000pt}%
\definecolor{currentstroke}{rgb}{0.000000,0.000000,0.000000}%
\pgfsetstrokecolor{currentstroke}%
\pgfsetstrokeopacity{0.000000}%
\pgfsetdash{}{0pt}%
\pgfpathmoveto{\pgfqpoint{5.906967in}{1.504897in}}%
\pgfpathlineto{\pgfqpoint{5.915720in}{1.504897in}}%
\pgfpathlineto{\pgfqpoint{5.915720in}{1.174266in}}%
\pgfpathlineto{\pgfqpoint{5.906967in}{1.174266in}}%
\pgfpathlineto{\pgfqpoint{5.906967in}{1.504897in}}%
\pgfpathclose%
\pgfusepath{fill}%
\end{pgfscope}%
\begin{pgfscope}%
\pgfpathrectangle{\pgfqpoint{3.776708in}{0.600000in}}{\pgfqpoint{2.573292in}{2.070576in}}%
\pgfusepath{clip}%
\pgfsetbuttcap%
\pgfsetmiterjoin%
\definecolor{currentfill}{rgb}{0.302379,0.450282,0.300122}%
\pgfsetfillcolor{currentfill}%
\pgfsetlinewidth{0.000000pt}%
\definecolor{currentstroke}{rgb}{0.000000,0.000000,0.000000}%
\pgfsetstrokecolor{currentstroke}%
\pgfsetstrokeopacity{0.000000}%
\pgfsetdash{}{0pt}%
\pgfpathmoveto{\pgfqpoint{5.917908in}{1.505862in}}%
\pgfpathlineto{\pgfqpoint{5.926662in}{1.505862in}}%
\pgfpathlineto{\pgfqpoint{5.926662in}{1.162732in}}%
\pgfpathlineto{\pgfqpoint{5.917908in}{1.162732in}}%
\pgfpathlineto{\pgfqpoint{5.917908in}{1.505862in}}%
\pgfpathclose%
\pgfusepath{fill}%
\end{pgfscope}%
\begin{pgfscope}%
\pgfpathrectangle{\pgfqpoint{3.776708in}{0.600000in}}{\pgfqpoint{2.573292in}{2.070576in}}%
\pgfusepath{clip}%
\pgfsetbuttcap%
\pgfsetmiterjoin%
\definecolor{currentfill}{rgb}{0.302379,0.450282,0.300122}%
\pgfsetfillcolor{currentfill}%
\pgfsetlinewidth{0.000000pt}%
\definecolor{currentstroke}{rgb}{0.000000,0.000000,0.000000}%
\pgfsetstrokecolor{currentstroke}%
\pgfsetstrokeopacity{0.000000}%
\pgfsetdash{}{0pt}%
\pgfpathmoveto{\pgfqpoint{5.928850in}{1.509252in}}%
\pgfpathlineto{\pgfqpoint{5.937604in}{1.509252in}}%
\pgfpathlineto{\pgfqpoint{5.937604in}{1.155648in}}%
\pgfpathlineto{\pgfqpoint{5.928850in}{1.155648in}}%
\pgfpathlineto{\pgfqpoint{5.928850in}{1.509252in}}%
\pgfpathclose%
\pgfusepath{fill}%
\end{pgfscope}%
\begin{pgfscope}%
\pgfpathrectangle{\pgfqpoint{3.776708in}{0.600000in}}{\pgfqpoint{2.573292in}{2.070576in}}%
\pgfusepath{clip}%
\pgfsetbuttcap%
\pgfsetmiterjoin%
\definecolor{currentfill}{rgb}{0.302379,0.450282,0.300122}%
\pgfsetfillcolor{currentfill}%
\pgfsetlinewidth{0.000000pt}%
\definecolor{currentstroke}{rgb}{0.000000,0.000000,0.000000}%
\pgfsetstrokecolor{currentstroke}%
\pgfsetstrokeopacity{0.000000}%
\pgfsetdash{}{0pt}%
\pgfpathmoveto{\pgfqpoint{5.939792in}{1.511834in}}%
\pgfpathlineto{\pgfqpoint{5.948545in}{1.511834in}}%
\pgfpathlineto{\pgfqpoint{5.948545in}{1.151443in}}%
\pgfpathlineto{\pgfqpoint{5.939792in}{1.151443in}}%
\pgfpathlineto{\pgfqpoint{5.939792in}{1.511834in}}%
\pgfpathclose%
\pgfusepath{fill}%
\end{pgfscope}%
\begin{pgfscope}%
\pgfpathrectangle{\pgfqpoint{3.776708in}{0.600000in}}{\pgfqpoint{2.573292in}{2.070576in}}%
\pgfusepath{clip}%
\pgfsetbuttcap%
\pgfsetmiterjoin%
\definecolor{currentfill}{rgb}{0.302379,0.450282,0.300122}%
\pgfsetfillcolor{currentfill}%
\pgfsetlinewidth{0.000000pt}%
\definecolor{currentstroke}{rgb}{0.000000,0.000000,0.000000}%
\pgfsetstrokecolor{currentstroke}%
\pgfsetstrokeopacity{0.000000}%
\pgfsetdash{}{0pt}%
\pgfpathmoveto{\pgfqpoint{5.950734in}{1.516279in}}%
\pgfpathlineto{\pgfqpoint{5.959487in}{1.516279in}}%
\pgfpathlineto{\pgfqpoint{5.959487in}{1.144668in}}%
\pgfpathlineto{\pgfqpoint{5.950734in}{1.144668in}}%
\pgfpathlineto{\pgfqpoint{5.950734in}{1.516279in}}%
\pgfpathclose%
\pgfusepath{fill}%
\end{pgfscope}%
\begin{pgfscope}%
\pgfpathrectangle{\pgfqpoint{3.776708in}{0.600000in}}{\pgfqpoint{2.573292in}{2.070576in}}%
\pgfusepath{clip}%
\pgfsetbuttcap%
\pgfsetmiterjoin%
\definecolor{currentfill}{rgb}{0.302379,0.450282,0.300122}%
\pgfsetfillcolor{currentfill}%
\pgfsetlinewidth{0.000000pt}%
\definecolor{currentstroke}{rgb}{0.000000,0.000000,0.000000}%
\pgfsetstrokecolor{currentstroke}%
\pgfsetstrokeopacity{0.000000}%
\pgfsetdash{}{0pt}%
\pgfpathmoveto{\pgfqpoint{5.961676in}{1.521732in}}%
\pgfpathlineto{\pgfqpoint{5.970429in}{1.521732in}}%
\pgfpathlineto{\pgfqpoint{5.970429in}{1.138392in}}%
\pgfpathlineto{\pgfqpoint{5.961676in}{1.138392in}}%
\pgfpathlineto{\pgfqpoint{5.961676in}{1.521732in}}%
\pgfpathclose%
\pgfusepath{fill}%
\end{pgfscope}%
\begin{pgfscope}%
\pgfpathrectangle{\pgfqpoint{3.776708in}{0.600000in}}{\pgfqpoint{2.573292in}{2.070576in}}%
\pgfusepath{clip}%
\pgfsetbuttcap%
\pgfsetmiterjoin%
\definecolor{currentfill}{rgb}{0.302379,0.450282,0.300122}%
\pgfsetfillcolor{currentfill}%
\pgfsetlinewidth{0.000000pt}%
\definecolor{currentstroke}{rgb}{0.000000,0.000000,0.000000}%
\pgfsetstrokecolor{currentstroke}%
\pgfsetstrokeopacity{0.000000}%
\pgfsetdash{}{0pt}%
\pgfpathmoveto{\pgfqpoint{5.972617in}{1.526935in}}%
\pgfpathlineto{\pgfqpoint{5.981371in}{1.526935in}}%
\pgfpathlineto{\pgfqpoint{5.981371in}{1.134405in}}%
\pgfpathlineto{\pgfqpoint{5.972617in}{1.134405in}}%
\pgfpathlineto{\pgfqpoint{5.972617in}{1.526935in}}%
\pgfpathclose%
\pgfusepath{fill}%
\end{pgfscope}%
\begin{pgfscope}%
\pgfpathrectangle{\pgfqpoint{3.776708in}{0.600000in}}{\pgfqpoint{2.573292in}{2.070576in}}%
\pgfusepath{clip}%
\pgfsetbuttcap%
\pgfsetmiterjoin%
\definecolor{currentfill}{rgb}{0.302379,0.450282,0.300122}%
\pgfsetfillcolor{currentfill}%
\pgfsetlinewidth{0.000000pt}%
\definecolor{currentstroke}{rgb}{0.000000,0.000000,0.000000}%
\pgfsetstrokecolor{currentstroke}%
\pgfsetstrokeopacity{0.000000}%
\pgfsetdash{}{0pt}%
\pgfpathmoveto{\pgfqpoint{5.983559in}{1.532992in}}%
\pgfpathlineto{\pgfqpoint{5.992313in}{1.532992in}}%
\pgfpathlineto{\pgfqpoint{5.992313in}{1.128971in}}%
\pgfpathlineto{\pgfqpoint{5.983559in}{1.128971in}}%
\pgfpathlineto{\pgfqpoint{5.983559in}{1.532992in}}%
\pgfpathclose%
\pgfusepath{fill}%
\end{pgfscope}%
\begin{pgfscope}%
\pgfpathrectangle{\pgfqpoint{3.776708in}{0.600000in}}{\pgfqpoint{2.573292in}{2.070576in}}%
\pgfusepath{clip}%
\pgfsetbuttcap%
\pgfsetmiterjoin%
\definecolor{currentfill}{rgb}{0.302379,0.450282,0.300122}%
\pgfsetfillcolor{currentfill}%
\pgfsetlinewidth{0.000000pt}%
\definecolor{currentstroke}{rgb}{0.000000,0.000000,0.000000}%
\pgfsetstrokecolor{currentstroke}%
\pgfsetstrokeopacity{0.000000}%
\pgfsetdash{}{0pt}%
\pgfpathmoveto{\pgfqpoint{5.994501in}{1.539475in}}%
\pgfpathlineto{\pgfqpoint{6.003254in}{1.539475in}}%
\pgfpathlineto{\pgfqpoint{6.003254in}{1.127487in}}%
\pgfpathlineto{\pgfqpoint{5.994501in}{1.127487in}}%
\pgfpathlineto{\pgfqpoint{5.994501in}{1.539475in}}%
\pgfpathclose%
\pgfusepath{fill}%
\end{pgfscope}%
\begin{pgfscope}%
\pgfpathrectangle{\pgfqpoint{3.776708in}{0.600000in}}{\pgfqpoint{2.573292in}{2.070576in}}%
\pgfusepath{clip}%
\pgfsetbuttcap%
\pgfsetmiterjoin%
\definecolor{currentfill}{rgb}{0.302379,0.450282,0.300122}%
\pgfsetfillcolor{currentfill}%
\pgfsetlinewidth{0.000000pt}%
\definecolor{currentstroke}{rgb}{0.000000,0.000000,0.000000}%
\pgfsetstrokecolor{currentstroke}%
\pgfsetstrokeopacity{0.000000}%
\pgfsetdash{}{0pt}%
\pgfpathmoveto{\pgfqpoint{6.005443in}{1.544856in}}%
\pgfpathlineto{\pgfqpoint{6.014196in}{1.544856in}}%
\pgfpathlineto{\pgfqpoint{6.014196in}{1.130385in}}%
\pgfpathlineto{\pgfqpoint{6.005443in}{1.130385in}}%
\pgfpathlineto{\pgfqpoint{6.005443in}{1.544856in}}%
\pgfpathclose%
\pgfusepath{fill}%
\end{pgfscope}%
\begin{pgfscope}%
\pgfpathrectangle{\pgfqpoint{3.776708in}{0.600000in}}{\pgfqpoint{2.573292in}{2.070576in}}%
\pgfusepath{clip}%
\pgfsetbuttcap%
\pgfsetmiterjoin%
\definecolor{currentfill}{rgb}{0.302379,0.450282,0.300122}%
\pgfsetfillcolor{currentfill}%
\pgfsetlinewidth{0.000000pt}%
\definecolor{currentstroke}{rgb}{0.000000,0.000000,0.000000}%
\pgfsetstrokecolor{currentstroke}%
\pgfsetstrokeopacity{0.000000}%
\pgfsetdash{}{0pt}%
\pgfpathmoveto{\pgfqpoint{6.016385in}{1.551295in}}%
\pgfpathlineto{\pgfqpoint{6.025138in}{1.551295in}}%
\pgfpathlineto{\pgfqpoint{6.025138in}{1.136910in}}%
\pgfpathlineto{\pgfqpoint{6.016385in}{1.136910in}}%
\pgfpathlineto{\pgfqpoint{6.016385in}{1.551295in}}%
\pgfpathclose%
\pgfusepath{fill}%
\end{pgfscope}%
\begin{pgfscope}%
\pgfpathrectangle{\pgfqpoint{3.776708in}{0.600000in}}{\pgfqpoint{2.573292in}{2.070576in}}%
\pgfusepath{clip}%
\pgfsetbuttcap%
\pgfsetmiterjoin%
\definecolor{currentfill}{rgb}{0.302379,0.450282,0.300122}%
\pgfsetfillcolor{currentfill}%
\pgfsetlinewidth{0.000000pt}%
\definecolor{currentstroke}{rgb}{0.000000,0.000000,0.000000}%
\pgfsetstrokecolor{currentstroke}%
\pgfsetstrokeopacity{0.000000}%
\pgfsetdash{}{0pt}%
\pgfpathmoveto{\pgfqpoint{6.027326in}{1.559615in}}%
\pgfpathlineto{\pgfqpoint{6.036080in}{1.559615in}}%
\pgfpathlineto{\pgfqpoint{6.036080in}{1.134409in}}%
\pgfpathlineto{\pgfqpoint{6.027326in}{1.134409in}}%
\pgfpathlineto{\pgfqpoint{6.027326in}{1.559615in}}%
\pgfpathclose%
\pgfusepath{fill}%
\end{pgfscope}%
\begin{pgfscope}%
\pgfpathrectangle{\pgfqpoint{3.776708in}{0.600000in}}{\pgfqpoint{2.573292in}{2.070576in}}%
\pgfusepath{clip}%
\pgfsetbuttcap%
\pgfsetmiterjoin%
\definecolor{currentfill}{rgb}{0.302379,0.450282,0.300122}%
\pgfsetfillcolor{currentfill}%
\pgfsetlinewidth{0.000000pt}%
\definecolor{currentstroke}{rgb}{0.000000,0.000000,0.000000}%
\pgfsetstrokecolor{currentstroke}%
\pgfsetstrokeopacity{0.000000}%
\pgfsetdash{}{0pt}%
\pgfpathmoveto{\pgfqpoint{6.038268in}{1.565303in}}%
\pgfpathlineto{\pgfqpoint{6.047022in}{1.565303in}}%
\pgfpathlineto{\pgfqpoint{6.047022in}{1.135299in}}%
\pgfpathlineto{\pgfqpoint{6.038268in}{1.135299in}}%
\pgfpathlineto{\pgfqpoint{6.038268in}{1.565303in}}%
\pgfpathclose%
\pgfusepath{fill}%
\end{pgfscope}%
\begin{pgfscope}%
\pgfpathrectangle{\pgfqpoint{3.776708in}{0.600000in}}{\pgfqpoint{2.573292in}{2.070576in}}%
\pgfusepath{clip}%
\pgfsetbuttcap%
\pgfsetmiterjoin%
\definecolor{currentfill}{rgb}{0.302379,0.450282,0.300122}%
\pgfsetfillcolor{currentfill}%
\pgfsetlinewidth{0.000000pt}%
\definecolor{currentstroke}{rgb}{0.000000,0.000000,0.000000}%
\pgfsetstrokecolor{currentstroke}%
\pgfsetstrokeopacity{0.000000}%
\pgfsetdash{}{0pt}%
\pgfpathmoveto{\pgfqpoint{6.049210in}{1.572901in}}%
\pgfpathlineto{\pgfqpoint{6.057963in}{1.572901in}}%
\pgfpathlineto{\pgfqpoint{6.057963in}{1.141382in}}%
\pgfpathlineto{\pgfqpoint{6.049210in}{1.141382in}}%
\pgfpathlineto{\pgfqpoint{6.049210in}{1.572901in}}%
\pgfpathclose%
\pgfusepath{fill}%
\end{pgfscope}%
\begin{pgfscope}%
\pgfpathrectangle{\pgfqpoint{3.776708in}{0.600000in}}{\pgfqpoint{2.573292in}{2.070576in}}%
\pgfusepath{clip}%
\pgfsetbuttcap%
\pgfsetmiterjoin%
\definecolor{currentfill}{rgb}{0.302379,0.450282,0.300122}%
\pgfsetfillcolor{currentfill}%
\pgfsetlinewidth{0.000000pt}%
\definecolor{currentstroke}{rgb}{0.000000,0.000000,0.000000}%
\pgfsetstrokecolor{currentstroke}%
\pgfsetstrokeopacity{0.000000}%
\pgfsetdash{}{0pt}%
\pgfpathmoveto{\pgfqpoint{6.060152in}{1.578572in}}%
\pgfpathlineto{\pgfqpoint{6.068905in}{1.578572in}}%
\pgfpathlineto{\pgfqpoint{6.068905in}{1.147216in}}%
\pgfpathlineto{\pgfqpoint{6.060152in}{1.147216in}}%
\pgfpathlineto{\pgfqpoint{6.060152in}{1.578572in}}%
\pgfpathclose%
\pgfusepath{fill}%
\end{pgfscope}%
\begin{pgfscope}%
\pgfpathrectangle{\pgfqpoint{3.776708in}{0.600000in}}{\pgfqpoint{2.573292in}{2.070576in}}%
\pgfusepath{clip}%
\pgfsetbuttcap%
\pgfsetmiterjoin%
\definecolor{currentfill}{rgb}{0.302379,0.450282,0.300122}%
\pgfsetfillcolor{currentfill}%
\pgfsetlinewidth{0.000000pt}%
\definecolor{currentstroke}{rgb}{0.000000,0.000000,0.000000}%
\pgfsetstrokecolor{currentstroke}%
\pgfsetstrokeopacity{0.000000}%
\pgfsetdash{}{0pt}%
\pgfpathmoveto{\pgfqpoint{6.071094in}{1.586559in}}%
\pgfpathlineto{\pgfqpoint{6.079847in}{1.586559in}}%
\pgfpathlineto{\pgfqpoint{6.079847in}{1.142977in}}%
\pgfpathlineto{\pgfqpoint{6.071094in}{1.142977in}}%
\pgfpathlineto{\pgfqpoint{6.071094in}{1.586559in}}%
\pgfpathclose%
\pgfusepath{fill}%
\end{pgfscope}%
\begin{pgfscope}%
\pgfpathrectangle{\pgfqpoint{3.776708in}{0.600000in}}{\pgfqpoint{2.573292in}{2.070576in}}%
\pgfusepath{clip}%
\pgfsetbuttcap%
\pgfsetmiterjoin%
\definecolor{currentfill}{rgb}{0.302379,0.450282,0.300122}%
\pgfsetfillcolor{currentfill}%
\pgfsetlinewidth{0.000000pt}%
\definecolor{currentstroke}{rgb}{0.000000,0.000000,0.000000}%
\pgfsetstrokecolor{currentstroke}%
\pgfsetstrokeopacity{0.000000}%
\pgfsetdash{}{0pt}%
\pgfpathmoveto{\pgfqpoint{6.082035in}{1.592613in}}%
\pgfpathlineto{\pgfqpoint{6.090789in}{1.592613in}}%
\pgfpathlineto{\pgfqpoint{6.090789in}{1.144969in}}%
\pgfpathlineto{\pgfqpoint{6.082035in}{1.144969in}}%
\pgfpathlineto{\pgfqpoint{6.082035in}{1.592613in}}%
\pgfpathclose%
\pgfusepath{fill}%
\end{pgfscope}%
\begin{pgfscope}%
\pgfpathrectangle{\pgfqpoint{3.776708in}{0.600000in}}{\pgfqpoint{2.573292in}{2.070576in}}%
\pgfusepath{clip}%
\pgfsetbuttcap%
\pgfsetmiterjoin%
\definecolor{currentfill}{rgb}{0.302379,0.450282,0.300122}%
\pgfsetfillcolor{currentfill}%
\pgfsetlinewidth{0.000000pt}%
\definecolor{currentstroke}{rgb}{0.000000,0.000000,0.000000}%
\pgfsetstrokecolor{currentstroke}%
\pgfsetstrokeopacity{0.000000}%
\pgfsetdash{}{0pt}%
\pgfpathmoveto{\pgfqpoint{6.092977in}{1.599631in}}%
\pgfpathlineto{\pgfqpoint{6.101731in}{1.599631in}}%
\pgfpathlineto{\pgfqpoint{6.101731in}{1.144062in}}%
\pgfpathlineto{\pgfqpoint{6.092977in}{1.144062in}}%
\pgfpathlineto{\pgfqpoint{6.092977in}{1.599631in}}%
\pgfpathclose%
\pgfusepath{fill}%
\end{pgfscope}%
\begin{pgfscope}%
\pgfpathrectangle{\pgfqpoint{3.776708in}{0.600000in}}{\pgfqpoint{2.573292in}{2.070576in}}%
\pgfusepath{clip}%
\pgfsetbuttcap%
\pgfsetmiterjoin%
\definecolor{currentfill}{rgb}{0.302379,0.450282,0.300122}%
\pgfsetfillcolor{currentfill}%
\pgfsetlinewidth{0.000000pt}%
\definecolor{currentstroke}{rgb}{0.000000,0.000000,0.000000}%
\pgfsetstrokecolor{currentstroke}%
\pgfsetstrokeopacity{0.000000}%
\pgfsetdash{}{0pt}%
\pgfpathmoveto{\pgfqpoint{6.103919in}{1.604411in}}%
\pgfpathlineto{\pgfqpoint{6.112672in}{1.604411in}}%
\pgfpathlineto{\pgfqpoint{6.112672in}{1.142047in}}%
\pgfpathlineto{\pgfqpoint{6.103919in}{1.142047in}}%
\pgfpathlineto{\pgfqpoint{6.103919in}{1.604411in}}%
\pgfpathclose%
\pgfusepath{fill}%
\end{pgfscope}%
\begin{pgfscope}%
\pgfpathrectangle{\pgfqpoint{3.776708in}{0.600000in}}{\pgfqpoint{2.573292in}{2.070576in}}%
\pgfusepath{clip}%
\pgfsetbuttcap%
\pgfsetmiterjoin%
\definecolor{currentfill}{rgb}{0.302379,0.450282,0.300122}%
\pgfsetfillcolor{currentfill}%
\pgfsetlinewidth{0.000000pt}%
\definecolor{currentstroke}{rgb}{0.000000,0.000000,0.000000}%
\pgfsetstrokecolor{currentstroke}%
\pgfsetstrokeopacity{0.000000}%
\pgfsetdash{}{0pt}%
\pgfpathmoveto{\pgfqpoint{6.114861in}{1.609196in}}%
\pgfpathlineto{\pgfqpoint{6.123614in}{1.609196in}}%
\pgfpathlineto{\pgfqpoint{6.123614in}{1.145986in}}%
\pgfpathlineto{\pgfqpoint{6.114861in}{1.145986in}}%
\pgfpathlineto{\pgfqpoint{6.114861in}{1.609196in}}%
\pgfpathclose%
\pgfusepath{fill}%
\end{pgfscope}%
\begin{pgfscope}%
\pgfpathrectangle{\pgfqpoint{3.776708in}{0.600000in}}{\pgfqpoint{2.573292in}{2.070576in}}%
\pgfusepath{clip}%
\pgfsetbuttcap%
\pgfsetmiterjoin%
\definecolor{currentfill}{rgb}{0.302379,0.450282,0.300122}%
\pgfsetfillcolor{currentfill}%
\pgfsetlinewidth{0.000000pt}%
\definecolor{currentstroke}{rgb}{0.000000,0.000000,0.000000}%
\pgfsetstrokecolor{currentstroke}%
\pgfsetstrokeopacity{0.000000}%
\pgfsetdash{}{0pt}%
\pgfpathmoveto{\pgfqpoint{6.125803in}{1.609196in}}%
\pgfpathlineto{\pgfqpoint{6.134556in}{1.609196in}}%
\pgfpathlineto{\pgfqpoint{6.134556in}{1.142121in}}%
\pgfpathlineto{\pgfqpoint{6.125803in}{1.142121in}}%
\pgfpathlineto{\pgfqpoint{6.125803in}{1.609196in}}%
\pgfpathclose%
\pgfusepath{fill}%
\end{pgfscope}%
\begin{pgfscope}%
\pgfpathrectangle{\pgfqpoint{3.776708in}{0.600000in}}{\pgfqpoint{2.573292in}{2.070576in}}%
\pgfusepath{clip}%
\pgfsetbuttcap%
\pgfsetmiterjoin%
\definecolor{currentfill}{rgb}{0.302379,0.450282,0.300122}%
\pgfsetfillcolor{currentfill}%
\pgfsetlinewidth{0.000000pt}%
\definecolor{currentstroke}{rgb}{0.000000,0.000000,0.000000}%
\pgfsetstrokecolor{currentstroke}%
\pgfsetstrokeopacity{0.000000}%
\pgfsetdash{}{0pt}%
\pgfpathmoveto{\pgfqpoint{6.136744in}{1.609196in}}%
\pgfpathlineto{\pgfqpoint{6.145498in}{1.609196in}}%
\pgfpathlineto{\pgfqpoint{6.145498in}{1.138503in}}%
\pgfpathlineto{\pgfqpoint{6.136744in}{1.138503in}}%
\pgfpathlineto{\pgfqpoint{6.136744in}{1.609196in}}%
\pgfpathclose%
\pgfusepath{fill}%
\end{pgfscope}%
\begin{pgfscope}%
\pgfpathrectangle{\pgfqpoint{3.776708in}{0.600000in}}{\pgfqpoint{2.573292in}{2.070576in}}%
\pgfusepath{clip}%
\pgfsetbuttcap%
\pgfsetmiterjoin%
\definecolor{currentfill}{rgb}{0.302379,0.450282,0.300122}%
\pgfsetfillcolor{currentfill}%
\pgfsetlinewidth{0.000000pt}%
\definecolor{currentstroke}{rgb}{0.000000,0.000000,0.000000}%
\pgfsetstrokecolor{currentstroke}%
\pgfsetstrokeopacity{0.000000}%
\pgfsetdash{}{0pt}%
\pgfpathmoveto{\pgfqpoint{6.147686in}{1.609196in}}%
\pgfpathlineto{\pgfqpoint{6.156440in}{1.609196in}}%
\pgfpathlineto{\pgfqpoint{6.156440in}{1.134733in}}%
\pgfpathlineto{\pgfqpoint{6.147686in}{1.134733in}}%
\pgfpathlineto{\pgfqpoint{6.147686in}{1.609196in}}%
\pgfpathclose%
\pgfusepath{fill}%
\end{pgfscope}%
\begin{pgfscope}%
\pgfpathrectangle{\pgfqpoint{3.776708in}{0.600000in}}{\pgfqpoint{2.573292in}{2.070576in}}%
\pgfusepath{clip}%
\pgfsetbuttcap%
\pgfsetmiterjoin%
\definecolor{currentfill}{rgb}{0.302379,0.450282,0.300122}%
\pgfsetfillcolor{currentfill}%
\pgfsetlinewidth{0.000000pt}%
\definecolor{currentstroke}{rgb}{0.000000,0.000000,0.000000}%
\pgfsetstrokecolor{currentstroke}%
\pgfsetstrokeopacity{0.000000}%
\pgfsetdash{}{0pt}%
\pgfpathmoveto{\pgfqpoint{6.158628in}{1.609196in}}%
\pgfpathlineto{\pgfqpoint{6.167381in}{1.609196in}}%
\pgfpathlineto{\pgfqpoint{6.167381in}{1.132101in}}%
\pgfpathlineto{\pgfqpoint{6.158628in}{1.132101in}}%
\pgfpathlineto{\pgfqpoint{6.158628in}{1.609196in}}%
\pgfpathclose%
\pgfusepath{fill}%
\end{pgfscope}%
\begin{pgfscope}%
\pgfpathrectangle{\pgfqpoint{3.776708in}{0.600000in}}{\pgfqpoint{2.573292in}{2.070576in}}%
\pgfusepath{clip}%
\pgfsetbuttcap%
\pgfsetmiterjoin%
\definecolor{currentfill}{rgb}{0.302379,0.450282,0.300122}%
\pgfsetfillcolor{currentfill}%
\pgfsetlinewidth{0.000000pt}%
\definecolor{currentstroke}{rgb}{0.000000,0.000000,0.000000}%
\pgfsetstrokecolor{currentstroke}%
\pgfsetstrokeopacity{0.000000}%
\pgfsetdash{}{0pt}%
\pgfpathmoveto{\pgfqpoint{6.169570in}{1.609196in}}%
\pgfpathlineto{\pgfqpoint{6.178323in}{1.609196in}}%
\pgfpathlineto{\pgfqpoint{6.178323in}{1.135881in}}%
\pgfpathlineto{\pgfqpoint{6.169570in}{1.135881in}}%
\pgfpathlineto{\pgfqpoint{6.169570in}{1.609196in}}%
\pgfpathclose%
\pgfusepath{fill}%
\end{pgfscope}%
\begin{pgfscope}%
\pgfpathrectangle{\pgfqpoint{3.776708in}{0.600000in}}{\pgfqpoint{2.573292in}{2.070576in}}%
\pgfusepath{clip}%
\pgfsetbuttcap%
\pgfsetmiterjoin%
\definecolor{currentfill}{rgb}{0.302379,0.450282,0.300122}%
\pgfsetfillcolor{currentfill}%
\pgfsetlinewidth{0.000000pt}%
\definecolor{currentstroke}{rgb}{0.000000,0.000000,0.000000}%
\pgfsetstrokecolor{currentstroke}%
\pgfsetstrokeopacity{0.000000}%
\pgfsetdash{}{0pt}%
\pgfpathmoveto{\pgfqpoint{6.180512in}{1.609196in}}%
\pgfpathlineto{\pgfqpoint{6.189265in}{1.609196in}}%
\pgfpathlineto{\pgfqpoint{6.189265in}{1.140310in}}%
\pgfpathlineto{\pgfqpoint{6.180512in}{1.140310in}}%
\pgfpathlineto{\pgfqpoint{6.180512in}{1.609196in}}%
\pgfpathclose%
\pgfusepath{fill}%
\end{pgfscope}%
\begin{pgfscope}%
\pgfpathrectangle{\pgfqpoint{3.776708in}{0.600000in}}{\pgfqpoint{2.573292in}{2.070576in}}%
\pgfusepath{clip}%
\pgfsetbuttcap%
\pgfsetmiterjoin%
\definecolor{currentfill}{rgb}{0.302379,0.450282,0.300122}%
\pgfsetfillcolor{currentfill}%
\pgfsetlinewidth{0.000000pt}%
\definecolor{currentstroke}{rgb}{0.000000,0.000000,0.000000}%
\pgfsetstrokecolor{currentstroke}%
\pgfsetstrokeopacity{0.000000}%
\pgfsetdash{}{0pt}%
\pgfpathmoveto{\pgfqpoint{6.191453in}{1.609196in}}%
\pgfpathlineto{\pgfqpoint{6.200207in}{1.609196in}}%
\pgfpathlineto{\pgfqpoint{6.200207in}{1.147919in}}%
\pgfpathlineto{\pgfqpoint{6.191453in}{1.147919in}}%
\pgfpathlineto{\pgfqpoint{6.191453in}{1.609196in}}%
\pgfpathclose%
\pgfusepath{fill}%
\end{pgfscope}%
\begin{pgfscope}%
\pgfpathrectangle{\pgfqpoint{3.776708in}{0.600000in}}{\pgfqpoint{2.573292in}{2.070576in}}%
\pgfusepath{clip}%
\pgfsetbuttcap%
\pgfsetmiterjoin%
\definecolor{currentfill}{rgb}{0.302379,0.450282,0.300122}%
\pgfsetfillcolor{currentfill}%
\pgfsetlinewidth{0.000000pt}%
\definecolor{currentstroke}{rgb}{0.000000,0.000000,0.000000}%
\pgfsetstrokecolor{currentstroke}%
\pgfsetstrokeopacity{0.000000}%
\pgfsetdash{}{0pt}%
\pgfpathmoveto{\pgfqpoint{6.202395in}{1.609196in}}%
\pgfpathlineto{\pgfqpoint{6.211149in}{1.609196in}}%
\pgfpathlineto{\pgfqpoint{6.211149in}{1.152333in}}%
\pgfpathlineto{\pgfqpoint{6.202395in}{1.152333in}}%
\pgfpathlineto{\pgfqpoint{6.202395in}{1.609196in}}%
\pgfpathclose%
\pgfusepath{fill}%
\end{pgfscope}%
\begin{pgfscope}%
\pgfpathrectangle{\pgfqpoint{3.776708in}{0.600000in}}{\pgfqpoint{2.573292in}{2.070576in}}%
\pgfusepath{clip}%
\pgfsetbuttcap%
\pgfsetmiterjoin%
\definecolor{currentfill}{rgb}{0.302379,0.450282,0.300122}%
\pgfsetfillcolor{currentfill}%
\pgfsetlinewidth{0.000000pt}%
\definecolor{currentstroke}{rgb}{0.000000,0.000000,0.000000}%
\pgfsetstrokecolor{currentstroke}%
\pgfsetstrokeopacity{0.000000}%
\pgfsetdash{}{0pt}%
\pgfpathmoveto{\pgfqpoint{6.213337in}{1.609196in}}%
\pgfpathlineto{\pgfqpoint{6.222090in}{1.609196in}}%
\pgfpathlineto{\pgfqpoint{6.222090in}{1.159262in}}%
\pgfpathlineto{\pgfqpoint{6.213337in}{1.159262in}}%
\pgfpathlineto{\pgfqpoint{6.213337in}{1.609196in}}%
\pgfpathclose%
\pgfusepath{fill}%
\end{pgfscope}%
\begin{pgfscope}%
\pgfpathrectangle{\pgfqpoint{3.776708in}{0.600000in}}{\pgfqpoint{2.573292in}{2.070576in}}%
\pgfusepath{clip}%
\pgfsetbuttcap%
\pgfsetmiterjoin%
\definecolor{currentfill}{rgb}{0.302379,0.450282,0.300122}%
\pgfsetfillcolor{currentfill}%
\pgfsetlinewidth{0.000000pt}%
\definecolor{currentstroke}{rgb}{0.000000,0.000000,0.000000}%
\pgfsetstrokecolor{currentstroke}%
\pgfsetstrokeopacity{0.000000}%
\pgfsetdash{}{0pt}%
\pgfpathmoveto{\pgfqpoint{6.224279in}{1.609196in}}%
\pgfpathlineto{\pgfqpoint{6.233032in}{1.609196in}}%
\pgfpathlineto{\pgfqpoint{6.233032in}{1.164543in}}%
\pgfpathlineto{\pgfqpoint{6.224279in}{1.164543in}}%
\pgfpathlineto{\pgfqpoint{6.224279in}{1.609196in}}%
\pgfpathclose%
\pgfusepath{fill}%
\end{pgfscope}%
\begin{pgfscope}%
\pgfpathrectangle{\pgfqpoint{3.776708in}{0.600000in}}{\pgfqpoint{2.573292in}{2.070576in}}%
\pgfusepath{clip}%
\pgfsetbuttcap%
\pgfsetmiterjoin%
\definecolor{currentfill}{rgb}{0.511253,0.510898,0.193296}%
\pgfsetfillcolor{currentfill}%
\pgfsetlinewidth{0.000000pt}%
\definecolor{currentstroke}{rgb}{0.000000,0.000000,0.000000}%
\pgfsetstrokecolor{currentstroke}%
\pgfsetstrokeopacity{0.000000}%
\pgfsetdash{}{0pt}%
\pgfpathmoveto{\pgfqpoint{3.893676in}{1.651251in}}%
\pgfpathlineto{\pgfqpoint{3.902429in}{1.651251in}}%
\pgfpathlineto{\pgfqpoint{3.902429in}{1.683487in}}%
\pgfpathlineto{\pgfqpoint{3.893676in}{1.683487in}}%
\pgfpathlineto{\pgfqpoint{3.893676in}{1.651251in}}%
\pgfpathclose%
\pgfusepath{fill}%
\end{pgfscope}%
\begin{pgfscope}%
\pgfpathrectangle{\pgfqpoint{3.776708in}{0.600000in}}{\pgfqpoint{2.573292in}{2.070576in}}%
\pgfusepath{clip}%
\pgfsetbuttcap%
\pgfsetmiterjoin%
\definecolor{currentfill}{rgb}{0.511253,0.510898,0.193296}%
\pgfsetfillcolor{currentfill}%
\pgfsetlinewidth{0.000000pt}%
\definecolor{currentstroke}{rgb}{0.000000,0.000000,0.000000}%
\pgfsetstrokecolor{currentstroke}%
\pgfsetstrokeopacity{0.000000}%
\pgfsetdash{}{0pt}%
\pgfpathmoveto{\pgfqpoint{3.904617in}{1.655386in}}%
\pgfpathlineto{\pgfqpoint{3.913371in}{1.655386in}}%
\pgfpathlineto{\pgfqpoint{3.913371in}{1.687387in}}%
\pgfpathlineto{\pgfqpoint{3.904617in}{1.687387in}}%
\pgfpathlineto{\pgfqpoint{3.904617in}{1.655386in}}%
\pgfpathclose%
\pgfusepath{fill}%
\end{pgfscope}%
\begin{pgfscope}%
\pgfpathrectangle{\pgfqpoint{3.776708in}{0.600000in}}{\pgfqpoint{2.573292in}{2.070576in}}%
\pgfusepath{clip}%
\pgfsetbuttcap%
\pgfsetmiterjoin%
\definecolor{currentfill}{rgb}{0.511253,0.510898,0.193296}%
\pgfsetfillcolor{currentfill}%
\pgfsetlinewidth{0.000000pt}%
\definecolor{currentstroke}{rgb}{0.000000,0.000000,0.000000}%
\pgfsetstrokecolor{currentstroke}%
\pgfsetstrokeopacity{0.000000}%
\pgfsetdash{}{0pt}%
\pgfpathmoveto{\pgfqpoint{3.915559in}{1.654428in}}%
\pgfpathlineto{\pgfqpoint{3.924313in}{1.654428in}}%
\pgfpathlineto{\pgfqpoint{3.924313in}{1.690905in}}%
\pgfpathlineto{\pgfqpoint{3.915559in}{1.690905in}}%
\pgfpathlineto{\pgfqpoint{3.915559in}{1.654428in}}%
\pgfpathclose%
\pgfusepath{fill}%
\end{pgfscope}%
\begin{pgfscope}%
\pgfpathrectangle{\pgfqpoint{3.776708in}{0.600000in}}{\pgfqpoint{2.573292in}{2.070576in}}%
\pgfusepath{clip}%
\pgfsetbuttcap%
\pgfsetmiterjoin%
\definecolor{currentfill}{rgb}{0.511253,0.510898,0.193296}%
\pgfsetfillcolor{currentfill}%
\pgfsetlinewidth{0.000000pt}%
\definecolor{currentstroke}{rgb}{0.000000,0.000000,0.000000}%
\pgfsetstrokecolor{currentstroke}%
\pgfsetstrokeopacity{0.000000}%
\pgfsetdash{}{0pt}%
\pgfpathmoveto{\pgfqpoint{3.926501in}{1.659911in}}%
\pgfpathlineto{\pgfqpoint{3.935254in}{1.659911in}}%
\pgfpathlineto{\pgfqpoint{3.935254in}{1.694430in}}%
\pgfpathlineto{\pgfqpoint{3.926501in}{1.694430in}}%
\pgfpathlineto{\pgfqpoint{3.926501in}{1.659911in}}%
\pgfpathclose%
\pgfusepath{fill}%
\end{pgfscope}%
\begin{pgfscope}%
\pgfpathrectangle{\pgfqpoint{3.776708in}{0.600000in}}{\pgfqpoint{2.573292in}{2.070576in}}%
\pgfusepath{clip}%
\pgfsetbuttcap%
\pgfsetmiterjoin%
\definecolor{currentfill}{rgb}{0.511253,0.510898,0.193296}%
\pgfsetfillcolor{currentfill}%
\pgfsetlinewidth{0.000000pt}%
\definecolor{currentstroke}{rgb}{0.000000,0.000000,0.000000}%
\pgfsetstrokecolor{currentstroke}%
\pgfsetstrokeopacity{0.000000}%
\pgfsetdash{}{0pt}%
\pgfpathmoveto{\pgfqpoint{3.937443in}{1.676666in}}%
\pgfpathlineto{\pgfqpoint{3.946196in}{1.676666in}}%
\pgfpathlineto{\pgfqpoint{3.946196in}{1.711254in}}%
\pgfpathlineto{\pgfqpoint{3.937443in}{1.711254in}}%
\pgfpathlineto{\pgfqpoint{3.937443in}{1.676666in}}%
\pgfpathclose%
\pgfusepath{fill}%
\end{pgfscope}%
\begin{pgfscope}%
\pgfpathrectangle{\pgfqpoint{3.776708in}{0.600000in}}{\pgfqpoint{2.573292in}{2.070576in}}%
\pgfusepath{clip}%
\pgfsetbuttcap%
\pgfsetmiterjoin%
\definecolor{currentfill}{rgb}{0.511253,0.510898,0.193296}%
\pgfsetfillcolor{currentfill}%
\pgfsetlinewidth{0.000000pt}%
\definecolor{currentstroke}{rgb}{0.000000,0.000000,0.000000}%
\pgfsetstrokecolor{currentstroke}%
\pgfsetstrokeopacity{0.000000}%
\pgfsetdash{}{0pt}%
\pgfpathmoveto{\pgfqpoint{3.948385in}{1.695221in}}%
\pgfpathlineto{\pgfqpoint{3.957138in}{1.695221in}}%
\pgfpathlineto{\pgfqpoint{3.957138in}{1.722391in}}%
\pgfpathlineto{\pgfqpoint{3.948385in}{1.722391in}}%
\pgfpathlineto{\pgfqpoint{3.948385in}{1.695221in}}%
\pgfpathclose%
\pgfusepath{fill}%
\end{pgfscope}%
\begin{pgfscope}%
\pgfpathrectangle{\pgfqpoint{3.776708in}{0.600000in}}{\pgfqpoint{2.573292in}{2.070576in}}%
\pgfusepath{clip}%
\pgfsetbuttcap%
\pgfsetmiterjoin%
\definecolor{currentfill}{rgb}{0.511253,0.510898,0.193296}%
\pgfsetfillcolor{currentfill}%
\pgfsetlinewidth{0.000000pt}%
\definecolor{currentstroke}{rgb}{0.000000,0.000000,0.000000}%
\pgfsetstrokecolor{currentstroke}%
\pgfsetstrokeopacity{0.000000}%
\pgfsetdash{}{0pt}%
\pgfpathmoveto{\pgfqpoint{3.959326in}{1.709995in}}%
\pgfpathlineto{\pgfqpoint{3.968080in}{1.709995in}}%
\pgfpathlineto{\pgfqpoint{3.968080in}{1.736117in}}%
\pgfpathlineto{\pgfqpoint{3.959326in}{1.736117in}}%
\pgfpathlineto{\pgfqpoint{3.959326in}{1.709995in}}%
\pgfpathclose%
\pgfusepath{fill}%
\end{pgfscope}%
\begin{pgfscope}%
\pgfpathrectangle{\pgfqpoint{3.776708in}{0.600000in}}{\pgfqpoint{2.573292in}{2.070576in}}%
\pgfusepath{clip}%
\pgfsetbuttcap%
\pgfsetmiterjoin%
\definecolor{currentfill}{rgb}{0.511253,0.510898,0.193296}%
\pgfsetfillcolor{currentfill}%
\pgfsetlinewidth{0.000000pt}%
\definecolor{currentstroke}{rgb}{0.000000,0.000000,0.000000}%
\pgfsetstrokecolor{currentstroke}%
\pgfsetstrokeopacity{0.000000}%
\pgfsetdash{}{0pt}%
\pgfpathmoveto{\pgfqpoint{3.970268in}{1.722918in}}%
\pgfpathlineto{\pgfqpoint{3.979022in}{1.722918in}}%
\pgfpathlineto{\pgfqpoint{3.979022in}{1.746093in}}%
\pgfpathlineto{\pgfqpoint{3.970268in}{1.746093in}}%
\pgfpathlineto{\pgfqpoint{3.970268in}{1.722918in}}%
\pgfpathclose%
\pgfusepath{fill}%
\end{pgfscope}%
\begin{pgfscope}%
\pgfpathrectangle{\pgfqpoint{3.776708in}{0.600000in}}{\pgfqpoint{2.573292in}{2.070576in}}%
\pgfusepath{clip}%
\pgfsetbuttcap%
\pgfsetmiterjoin%
\definecolor{currentfill}{rgb}{0.511253,0.510898,0.193296}%
\pgfsetfillcolor{currentfill}%
\pgfsetlinewidth{0.000000pt}%
\definecolor{currentstroke}{rgb}{0.000000,0.000000,0.000000}%
\pgfsetstrokecolor{currentstroke}%
\pgfsetstrokeopacity{0.000000}%
\pgfsetdash{}{0pt}%
\pgfpathmoveto{\pgfqpoint{3.981210in}{1.738785in}}%
\pgfpathlineto{\pgfqpoint{3.989963in}{1.738785in}}%
\pgfpathlineto{\pgfqpoint{3.989963in}{1.762190in}}%
\pgfpathlineto{\pgfqpoint{3.981210in}{1.762190in}}%
\pgfpathlineto{\pgfqpoint{3.981210in}{1.738785in}}%
\pgfpathclose%
\pgfusepath{fill}%
\end{pgfscope}%
\begin{pgfscope}%
\pgfpathrectangle{\pgfqpoint{3.776708in}{0.600000in}}{\pgfqpoint{2.573292in}{2.070576in}}%
\pgfusepath{clip}%
\pgfsetbuttcap%
\pgfsetmiterjoin%
\definecolor{currentfill}{rgb}{0.511253,0.510898,0.193296}%
\pgfsetfillcolor{currentfill}%
\pgfsetlinewidth{0.000000pt}%
\definecolor{currentstroke}{rgb}{0.000000,0.000000,0.000000}%
\pgfsetstrokecolor{currentstroke}%
\pgfsetstrokeopacity{0.000000}%
\pgfsetdash{}{0pt}%
\pgfpathmoveto{\pgfqpoint{3.992152in}{1.758897in}}%
\pgfpathlineto{\pgfqpoint{4.000905in}{1.758897in}}%
\pgfpathlineto{\pgfqpoint{4.000905in}{1.775280in}}%
\pgfpathlineto{\pgfqpoint{3.992152in}{1.775280in}}%
\pgfpathlineto{\pgfqpoint{3.992152in}{1.758897in}}%
\pgfpathclose%
\pgfusepath{fill}%
\end{pgfscope}%
\begin{pgfscope}%
\pgfpathrectangle{\pgfqpoint{3.776708in}{0.600000in}}{\pgfqpoint{2.573292in}{2.070576in}}%
\pgfusepath{clip}%
\pgfsetbuttcap%
\pgfsetmiterjoin%
\definecolor{currentfill}{rgb}{0.511253,0.510898,0.193296}%
\pgfsetfillcolor{currentfill}%
\pgfsetlinewidth{0.000000pt}%
\definecolor{currentstroke}{rgb}{0.000000,0.000000,0.000000}%
\pgfsetstrokecolor{currentstroke}%
\pgfsetstrokeopacity{0.000000}%
\pgfsetdash{}{0pt}%
\pgfpathmoveto{\pgfqpoint{4.003094in}{1.773087in}}%
\pgfpathlineto{\pgfqpoint{4.011847in}{1.773087in}}%
\pgfpathlineto{\pgfqpoint{4.011847in}{1.785285in}}%
\pgfpathlineto{\pgfqpoint{4.003094in}{1.785285in}}%
\pgfpathlineto{\pgfqpoint{4.003094in}{1.773087in}}%
\pgfpathclose%
\pgfusepath{fill}%
\end{pgfscope}%
\begin{pgfscope}%
\pgfpathrectangle{\pgfqpoint{3.776708in}{0.600000in}}{\pgfqpoint{2.573292in}{2.070576in}}%
\pgfusepath{clip}%
\pgfsetbuttcap%
\pgfsetmiterjoin%
\definecolor{currentfill}{rgb}{0.511253,0.510898,0.193296}%
\pgfsetfillcolor{currentfill}%
\pgfsetlinewidth{0.000000pt}%
\definecolor{currentstroke}{rgb}{0.000000,0.000000,0.000000}%
\pgfsetstrokecolor{currentstroke}%
\pgfsetstrokeopacity{0.000000}%
\pgfsetdash{}{0pt}%
\pgfpathmoveto{\pgfqpoint{4.014035in}{1.786930in}}%
\pgfpathlineto{\pgfqpoint{4.022789in}{1.786930in}}%
\pgfpathlineto{\pgfqpoint{4.022789in}{1.794785in}}%
\pgfpathlineto{\pgfqpoint{4.014035in}{1.794785in}}%
\pgfpathlineto{\pgfqpoint{4.014035in}{1.786930in}}%
\pgfpathclose%
\pgfusepath{fill}%
\end{pgfscope}%
\begin{pgfscope}%
\pgfpathrectangle{\pgfqpoint{3.776708in}{0.600000in}}{\pgfqpoint{2.573292in}{2.070576in}}%
\pgfusepath{clip}%
\pgfsetbuttcap%
\pgfsetmiterjoin%
\definecolor{currentfill}{rgb}{0.511253,0.510898,0.193296}%
\pgfsetfillcolor{currentfill}%
\pgfsetlinewidth{0.000000pt}%
\definecolor{currentstroke}{rgb}{0.000000,0.000000,0.000000}%
\pgfsetstrokecolor{currentstroke}%
\pgfsetstrokeopacity{0.000000}%
\pgfsetdash{}{0pt}%
\pgfpathmoveto{\pgfqpoint{4.024977in}{1.548131in}}%
\pgfpathlineto{\pgfqpoint{4.033731in}{1.548131in}}%
\pgfpathlineto{\pgfqpoint{4.033731in}{1.547086in}}%
\pgfpathlineto{\pgfqpoint{4.024977in}{1.547086in}}%
\pgfpathlineto{\pgfqpoint{4.024977in}{1.548131in}}%
\pgfpathclose%
\pgfusepath{fill}%
\end{pgfscope}%
\begin{pgfscope}%
\pgfpathrectangle{\pgfqpoint{3.776708in}{0.600000in}}{\pgfqpoint{2.573292in}{2.070576in}}%
\pgfusepath{clip}%
\pgfsetbuttcap%
\pgfsetmiterjoin%
\definecolor{currentfill}{rgb}{0.511253,0.510898,0.193296}%
\pgfsetfillcolor{currentfill}%
\pgfsetlinewidth{0.000000pt}%
\definecolor{currentstroke}{rgb}{0.000000,0.000000,0.000000}%
\pgfsetstrokecolor{currentstroke}%
\pgfsetstrokeopacity{0.000000}%
\pgfsetdash{}{0pt}%
\pgfpathmoveto{\pgfqpoint{4.035919in}{1.557673in}}%
\pgfpathlineto{\pgfqpoint{4.044672in}{1.557673in}}%
\pgfpathlineto{\pgfqpoint{4.044672in}{1.557070in}}%
\pgfpathlineto{\pgfqpoint{4.035919in}{1.557070in}}%
\pgfpathlineto{\pgfqpoint{4.035919in}{1.557673in}}%
\pgfpathclose%
\pgfusepath{fill}%
\end{pgfscope}%
\begin{pgfscope}%
\pgfpathrectangle{\pgfqpoint{3.776708in}{0.600000in}}{\pgfqpoint{2.573292in}{2.070576in}}%
\pgfusepath{clip}%
\pgfsetbuttcap%
\pgfsetmiterjoin%
\definecolor{currentfill}{rgb}{0.511253,0.510898,0.193296}%
\pgfsetfillcolor{currentfill}%
\pgfsetlinewidth{0.000000pt}%
\definecolor{currentstroke}{rgb}{0.000000,0.000000,0.000000}%
\pgfsetstrokecolor{currentstroke}%
\pgfsetstrokeopacity{0.000000}%
\pgfsetdash{}{0pt}%
\pgfpathmoveto{\pgfqpoint{4.046861in}{1.563629in}}%
\pgfpathlineto{\pgfqpoint{4.055614in}{1.563629in}}%
\pgfpathlineto{\pgfqpoint{4.055614in}{1.553860in}}%
\pgfpathlineto{\pgfqpoint{4.046861in}{1.553860in}}%
\pgfpathlineto{\pgfqpoint{4.046861in}{1.563629in}}%
\pgfpathclose%
\pgfusepath{fill}%
\end{pgfscope}%
\begin{pgfscope}%
\pgfpathrectangle{\pgfqpoint{3.776708in}{0.600000in}}{\pgfqpoint{2.573292in}{2.070576in}}%
\pgfusepath{clip}%
\pgfsetbuttcap%
\pgfsetmiterjoin%
\definecolor{currentfill}{rgb}{0.511253,0.510898,0.193296}%
\pgfsetfillcolor{currentfill}%
\pgfsetlinewidth{0.000000pt}%
\definecolor{currentstroke}{rgb}{0.000000,0.000000,0.000000}%
\pgfsetstrokecolor{currentstroke}%
\pgfsetstrokeopacity{0.000000}%
\pgfsetdash{}{0pt}%
\pgfpathmoveto{\pgfqpoint{4.057803in}{1.568877in}}%
\pgfpathlineto{\pgfqpoint{4.066556in}{1.568877in}}%
\pgfpathlineto{\pgfqpoint{4.066556in}{1.553290in}}%
\pgfpathlineto{\pgfqpoint{4.057803in}{1.553290in}}%
\pgfpathlineto{\pgfqpoint{4.057803in}{1.568877in}}%
\pgfpathclose%
\pgfusepath{fill}%
\end{pgfscope}%
\begin{pgfscope}%
\pgfpathrectangle{\pgfqpoint{3.776708in}{0.600000in}}{\pgfqpoint{2.573292in}{2.070576in}}%
\pgfusepath{clip}%
\pgfsetbuttcap%
\pgfsetmiterjoin%
\definecolor{currentfill}{rgb}{0.511253,0.510898,0.193296}%
\pgfsetfillcolor{currentfill}%
\pgfsetlinewidth{0.000000pt}%
\definecolor{currentstroke}{rgb}{0.000000,0.000000,0.000000}%
\pgfsetstrokecolor{currentstroke}%
\pgfsetstrokeopacity{0.000000}%
\pgfsetdash{}{0pt}%
\pgfpathmoveto{\pgfqpoint{4.068744in}{1.575643in}}%
\pgfpathlineto{\pgfqpoint{4.077498in}{1.575643in}}%
\pgfpathlineto{\pgfqpoint{4.077498in}{1.565624in}}%
\pgfpathlineto{\pgfqpoint{4.068744in}{1.565624in}}%
\pgfpathlineto{\pgfqpoint{4.068744in}{1.575643in}}%
\pgfpathclose%
\pgfusepath{fill}%
\end{pgfscope}%
\begin{pgfscope}%
\pgfpathrectangle{\pgfqpoint{3.776708in}{0.600000in}}{\pgfqpoint{2.573292in}{2.070576in}}%
\pgfusepath{clip}%
\pgfsetbuttcap%
\pgfsetmiterjoin%
\definecolor{currentfill}{rgb}{0.511253,0.510898,0.193296}%
\pgfsetfillcolor{currentfill}%
\pgfsetlinewidth{0.000000pt}%
\definecolor{currentstroke}{rgb}{0.000000,0.000000,0.000000}%
\pgfsetstrokecolor{currentstroke}%
\pgfsetstrokeopacity{0.000000}%
\pgfsetdash{}{0pt}%
\pgfpathmoveto{\pgfqpoint{4.079686in}{1.566432in}}%
\pgfpathlineto{\pgfqpoint{4.088440in}{1.566432in}}%
\pgfpathlineto{\pgfqpoint{4.088440in}{1.552931in}}%
\pgfpathlineto{\pgfqpoint{4.079686in}{1.552931in}}%
\pgfpathlineto{\pgfqpoint{4.079686in}{1.566432in}}%
\pgfpathclose%
\pgfusepath{fill}%
\end{pgfscope}%
\begin{pgfscope}%
\pgfpathrectangle{\pgfqpoint{3.776708in}{0.600000in}}{\pgfqpoint{2.573292in}{2.070576in}}%
\pgfusepath{clip}%
\pgfsetbuttcap%
\pgfsetmiterjoin%
\definecolor{currentfill}{rgb}{0.511253,0.510898,0.193296}%
\pgfsetfillcolor{currentfill}%
\pgfsetlinewidth{0.000000pt}%
\definecolor{currentstroke}{rgb}{0.000000,0.000000,0.000000}%
\pgfsetstrokecolor{currentstroke}%
\pgfsetstrokeopacity{0.000000}%
\pgfsetdash{}{0pt}%
\pgfpathmoveto{\pgfqpoint{4.090628in}{1.553081in}}%
\pgfpathlineto{\pgfqpoint{4.099381in}{1.553081in}}%
\pgfpathlineto{\pgfqpoint{4.099381in}{1.535919in}}%
\pgfpathlineto{\pgfqpoint{4.090628in}{1.535919in}}%
\pgfpathlineto{\pgfqpoint{4.090628in}{1.553081in}}%
\pgfpathclose%
\pgfusepath{fill}%
\end{pgfscope}%
\begin{pgfscope}%
\pgfpathrectangle{\pgfqpoint{3.776708in}{0.600000in}}{\pgfqpoint{2.573292in}{2.070576in}}%
\pgfusepath{clip}%
\pgfsetbuttcap%
\pgfsetmiterjoin%
\definecolor{currentfill}{rgb}{0.511253,0.510898,0.193296}%
\pgfsetfillcolor{currentfill}%
\pgfsetlinewidth{0.000000pt}%
\definecolor{currentstroke}{rgb}{0.000000,0.000000,0.000000}%
\pgfsetstrokecolor{currentstroke}%
\pgfsetstrokeopacity{0.000000}%
\pgfsetdash{}{0pt}%
\pgfpathmoveto{\pgfqpoint{4.101570in}{1.531900in}}%
\pgfpathlineto{\pgfqpoint{4.110323in}{1.531900in}}%
\pgfpathlineto{\pgfqpoint{4.110323in}{1.509510in}}%
\pgfpathlineto{\pgfqpoint{4.101570in}{1.509510in}}%
\pgfpathlineto{\pgfqpoint{4.101570in}{1.531900in}}%
\pgfpathclose%
\pgfusepath{fill}%
\end{pgfscope}%
\begin{pgfscope}%
\pgfpathrectangle{\pgfqpoint{3.776708in}{0.600000in}}{\pgfqpoint{2.573292in}{2.070576in}}%
\pgfusepath{clip}%
\pgfsetbuttcap%
\pgfsetmiterjoin%
\definecolor{currentfill}{rgb}{0.511253,0.510898,0.193296}%
\pgfsetfillcolor{currentfill}%
\pgfsetlinewidth{0.000000pt}%
\definecolor{currentstroke}{rgb}{0.000000,0.000000,0.000000}%
\pgfsetstrokecolor{currentstroke}%
\pgfsetstrokeopacity{0.000000}%
\pgfsetdash{}{0pt}%
\pgfpathmoveto{\pgfqpoint{4.112512in}{1.517198in}}%
\pgfpathlineto{\pgfqpoint{4.121265in}{1.517198in}}%
\pgfpathlineto{\pgfqpoint{4.121265in}{1.497026in}}%
\pgfpathlineto{\pgfqpoint{4.112512in}{1.497026in}}%
\pgfpathlineto{\pgfqpoint{4.112512in}{1.517198in}}%
\pgfpathclose%
\pgfusepath{fill}%
\end{pgfscope}%
\begin{pgfscope}%
\pgfpathrectangle{\pgfqpoint{3.776708in}{0.600000in}}{\pgfqpoint{2.573292in}{2.070576in}}%
\pgfusepath{clip}%
\pgfsetbuttcap%
\pgfsetmiterjoin%
\definecolor{currentfill}{rgb}{0.511253,0.510898,0.193296}%
\pgfsetfillcolor{currentfill}%
\pgfsetlinewidth{0.000000pt}%
\definecolor{currentstroke}{rgb}{0.000000,0.000000,0.000000}%
\pgfsetstrokecolor{currentstroke}%
\pgfsetstrokeopacity{0.000000}%
\pgfsetdash{}{0pt}%
\pgfpathmoveto{\pgfqpoint{4.123453in}{1.509661in}}%
\pgfpathlineto{\pgfqpoint{4.132207in}{1.509661in}}%
\pgfpathlineto{\pgfqpoint{4.132207in}{1.494019in}}%
\pgfpathlineto{\pgfqpoint{4.123453in}{1.494019in}}%
\pgfpathlineto{\pgfqpoint{4.123453in}{1.509661in}}%
\pgfpathclose%
\pgfusepath{fill}%
\end{pgfscope}%
\begin{pgfscope}%
\pgfpathrectangle{\pgfqpoint{3.776708in}{0.600000in}}{\pgfqpoint{2.573292in}{2.070576in}}%
\pgfusepath{clip}%
\pgfsetbuttcap%
\pgfsetmiterjoin%
\definecolor{currentfill}{rgb}{0.511253,0.510898,0.193296}%
\pgfsetfillcolor{currentfill}%
\pgfsetlinewidth{0.000000pt}%
\definecolor{currentstroke}{rgb}{0.000000,0.000000,0.000000}%
\pgfsetstrokecolor{currentstroke}%
\pgfsetstrokeopacity{0.000000}%
\pgfsetdash{}{0pt}%
\pgfpathmoveto{\pgfqpoint{4.134395in}{1.496613in}}%
\pgfpathlineto{\pgfqpoint{4.143149in}{1.496613in}}%
\pgfpathlineto{\pgfqpoint{4.143149in}{1.481495in}}%
\pgfpathlineto{\pgfqpoint{4.134395in}{1.481495in}}%
\pgfpathlineto{\pgfqpoint{4.134395in}{1.496613in}}%
\pgfpathclose%
\pgfusepath{fill}%
\end{pgfscope}%
\begin{pgfscope}%
\pgfpathrectangle{\pgfqpoint{3.776708in}{0.600000in}}{\pgfqpoint{2.573292in}{2.070576in}}%
\pgfusepath{clip}%
\pgfsetbuttcap%
\pgfsetmiterjoin%
\definecolor{currentfill}{rgb}{0.511253,0.510898,0.193296}%
\pgfsetfillcolor{currentfill}%
\pgfsetlinewidth{0.000000pt}%
\definecolor{currentstroke}{rgb}{0.000000,0.000000,0.000000}%
\pgfsetstrokecolor{currentstroke}%
\pgfsetstrokeopacity{0.000000}%
\pgfsetdash{}{0pt}%
\pgfpathmoveto{\pgfqpoint{4.145337in}{1.485077in}}%
\pgfpathlineto{\pgfqpoint{4.154090in}{1.485077in}}%
\pgfpathlineto{\pgfqpoint{4.154090in}{1.471984in}}%
\pgfpathlineto{\pgfqpoint{4.145337in}{1.471984in}}%
\pgfpathlineto{\pgfqpoint{4.145337in}{1.485077in}}%
\pgfpathclose%
\pgfusepath{fill}%
\end{pgfscope}%
\begin{pgfscope}%
\pgfpathrectangle{\pgfqpoint{3.776708in}{0.600000in}}{\pgfqpoint{2.573292in}{2.070576in}}%
\pgfusepath{clip}%
\pgfsetbuttcap%
\pgfsetmiterjoin%
\definecolor{currentfill}{rgb}{0.511253,0.510898,0.193296}%
\pgfsetfillcolor{currentfill}%
\pgfsetlinewidth{0.000000pt}%
\definecolor{currentstroke}{rgb}{0.000000,0.000000,0.000000}%
\pgfsetstrokecolor{currentstroke}%
\pgfsetstrokeopacity{0.000000}%
\pgfsetdash{}{0pt}%
\pgfpathmoveto{\pgfqpoint{4.156279in}{1.471065in}}%
\pgfpathlineto{\pgfqpoint{4.165032in}{1.471065in}}%
\pgfpathlineto{\pgfqpoint{4.165032in}{1.451818in}}%
\pgfpathlineto{\pgfqpoint{4.156279in}{1.451818in}}%
\pgfpathlineto{\pgfqpoint{4.156279in}{1.471065in}}%
\pgfpathclose%
\pgfusepath{fill}%
\end{pgfscope}%
\begin{pgfscope}%
\pgfpathrectangle{\pgfqpoint{3.776708in}{0.600000in}}{\pgfqpoint{2.573292in}{2.070576in}}%
\pgfusepath{clip}%
\pgfsetbuttcap%
\pgfsetmiterjoin%
\definecolor{currentfill}{rgb}{0.511253,0.510898,0.193296}%
\pgfsetfillcolor{currentfill}%
\pgfsetlinewidth{0.000000pt}%
\definecolor{currentstroke}{rgb}{0.000000,0.000000,0.000000}%
\pgfsetstrokecolor{currentstroke}%
\pgfsetstrokeopacity{0.000000}%
\pgfsetdash{}{0pt}%
\pgfpathmoveto{\pgfqpoint{4.167221in}{1.457203in}}%
\pgfpathlineto{\pgfqpoint{4.175974in}{1.457203in}}%
\pgfpathlineto{\pgfqpoint{4.175974in}{1.434389in}}%
\pgfpathlineto{\pgfqpoint{4.167221in}{1.434389in}}%
\pgfpathlineto{\pgfqpoint{4.167221in}{1.457203in}}%
\pgfpathclose%
\pgfusepath{fill}%
\end{pgfscope}%
\begin{pgfscope}%
\pgfpathrectangle{\pgfqpoint{3.776708in}{0.600000in}}{\pgfqpoint{2.573292in}{2.070576in}}%
\pgfusepath{clip}%
\pgfsetbuttcap%
\pgfsetmiterjoin%
\definecolor{currentfill}{rgb}{0.511253,0.510898,0.193296}%
\pgfsetfillcolor{currentfill}%
\pgfsetlinewidth{0.000000pt}%
\definecolor{currentstroke}{rgb}{0.000000,0.000000,0.000000}%
\pgfsetstrokecolor{currentstroke}%
\pgfsetstrokeopacity{0.000000}%
\pgfsetdash{}{0pt}%
\pgfpathmoveto{\pgfqpoint{4.178162in}{1.450589in}}%
\pgfpathlineto{\pgfqpoint{4.186916in}{1.450589in}}%
\pgfpathlineto{\pgfqpoint{4.186916in}{1.421997in}}%
\pgfpathlineto{\pgfqpoint{4.178162in}{1.421997in}}%
\pgfpathlineto{\pgfqpoint{4.178162in}{1.450589in}}%
\pgfpathclose%
\pgfusepath{fill}%
\end{pgfscope}%
\begin{pgfscope}%
\pgfpathrectangle{\pgfqpoint{3.776708in}{0.600000in}}{\pgfqpoint{2.573292in}{2.070576in}}%
\pgfusepath{clip}%
\pgfsetbuttcap%
\pgfsetmiterjoin%
\definecolor{currentfill}{rgb}{0.511253,0.510898,0.193296}%
\pgfsetfillcolor{currentfill}%
\pgfsetlinewidth{0.000000pt}%
\definecolor{currentstroke}{rgb}{0.000000,0.000000,0.000000}%
\pgfsetstrokecolor{currentstroke}%
\pgfsetstrokeopacity{0.000000}%
\pgfsetdash{}{0pt}%
\pgfpathmoveto{\pgfqpoint{4.189104in}{1.445010in}}%
\pgfpathlineto{\pgfqpoint{4.197858in}{1.445010in}}%
\pgfpathlineto{\pgfqpoint{4.197858in}{1.402528in}}%
\pgfpathlineto{\pgfqpoint{4.189104in}{1.402528in}}%
\pgfpathlineto{\pgfqpoint{4.189104in}{1.445010in}}%
\pgfpathclose%
\pgfusepath{fill}%
\end{pgfscope}%
\begin{pgfscope}%
\pgfpathrectangle{\pgfqpoint{3.776708in}{0.600000in}}{\pgfqpoint{2.573292in}{2.070576in}}%
\pgfusepath{clip}%
\pgfsetbuttcap%
\pgfsetmiterjoin%
\definecolor{currentfill}{rgb}{0.511253,0.510898,0.193296}%
\pgfsetfillcolor{currentfill}%
\pgfsetlinewidth{0.000000pt}%
\definecolor{currentstroke}{rgb}{0.000000,0.000000,0.000000}%
\pgfsetstrokecolor{currentstroke}%
\pgfsetstrokeopacity{0.000000}%
\pgfsetdash{}{0pt}%
\pgfpathmoveto{\pgfqpoint{4.200046in}{1.441753in}}%
\pgfpathlineto{\pgfqpoint{4.208799in}{1.441753in}}%
\pgfpathlineto{\pgfqpoint{4.208799in}{1.384695in}}%
\pgfpathlineto{\pgfqpoint{4.200046in}{1.384695in}}%
\pgfpathlineto{\pgfqpoint{4.200046in}{1.441753in}}%
\pgfpathclose%
\pgfusepath{fill}%
\end{pgfscope}%
\begin{pgfscope}%
\pgfpathrectangle{\pgfqpoint{3.776708in}{0.600000in}}{\pgfqpoint{2.573292in}{2.070576in}}%
\pgfusepath{clip}%
\pgfsetbuttcap%
\pgfsetmiterjoin%
\definecolor{currentfill}{rgb}{0.511253,0.510898,0.193296}%
\pgfsetfillcolor{currentfill}%
\pgfsetlinewidth{0.000000pt}%
\definecolor{currentstroke}{rgb}{0.000000,0.000000,0.000000}%
\pgfsetstrokecolor{currentstroke}%
\pgfsetstrokeopacity{0.000000}%
\pgfsetdash{}{0pt}%
\pgfpathmoveto{\pgfqpoint{4.210988in}{1.456043in}}%
\pgfpathlineto{\pgfqpoint{4.219741in}{1.456043in}}%
\pgfpathlineto{\pgfqpoint{4.219741in}{1.380814in}}%
\pgfpathlineto{\pgfqpoint{4.210988in}{1.380814in}}%
\pgfpathlineto{\pgfqpoint{4.210988in}{1.456043in}}%
\pgfpathclose%
\pgfusepath{fill}%
\end{pgfscope}%
\begin{pgfscope}%
\pgfpathrectangle{\pgfqpoint{3.776708in}{0.600000in}}{\pgfqpoint{2.573292in}{2.070576in}}%
\pgfusepath{clip}%
\pgfsetbuttcap%
\pgfsetmiterjoin%
\definecolor{currentfill}{rgb}{0.511253,0.510898,0.193296}%
\pgfsetfillcolor{currentfill}%
\pgfsetlinewidth{0.000000pt}%
\definecolor{currentstroke}{rgb}{0.000000,0.000000,0.000000}%
\pgfsetstrokecolor{currentstroke}%
\pgfsetstrokeopacity{0.000000}%
\pgfsetdash{}{0pt}%
\pgfpathmoveto{\pgfqpoint{4.221930in}{1.467761in}}%
\pgfpathlineto{\pgfqpoint{4.230683in}{1.467761in}}%
\pgfpathlineto{\pgfqpoint{4.230683in}{1.378390in}}%
\pgfpathlineto{\pgfqpoint{4.221930in}{1.378390in}}%
\pgfpathlineto{\pgfqpoint{4.221930in}{1.467761in}}%
\pgfpathclose%
\pgfusepath{fill}%
\end{pgfscope}%
\begin{pgfscope}%
\pgfpathrectangle{\pgfqpoint{3.776708in}{0.600000in}}{\pgfqpoint{2.573292in}{2.070576in}}%
\pgfusepath{clip}%
\pgfsetbuttcap%
\pgfsetmiterjoin%
\definecolor{currentfill}{rgb}{0.511253,0.510898,0.193296}%
\pgfsetfillcolor{currentfill}%
\pgfsetlinewidth{0.000000pt}%
\definecolor{currentstroke}{rgb}{0.000000,0.000000,0.000000}%
\pgfsetstrokecolor{currentstroke}%
\pgfsetstrokeopacity{0.000000}%
\pgfsetdash{}{0pt}%
\pgfpathmoveto{\pgfqpoint{4.232871in}{1.471474in}}%
\pgfpathlineto{\pgfqpoint{4.241625in}{1.471474in}}%
\pgfpathlineto{\pgfqpoint{4.241625in}{1.362963in}}%
\pgfpathlineto{\pgfqpoint{4.232871in}{1.362963in}}%
\pgfpathlineto{\pgfqpoint{4.232871in}{1.471474in}}%
\pgfpathclose%
\pgfusepath{fill}%
\end{pgfscope}%
\begin{pgfscope}%
\pgfpathrectangle{\pgfqpoint{3.776708in}{0.600000in}}{\pgfqpoint{2.573292in}{2.070576in}}%
\pgfusepath{clip}%
\pgfsetbuttcap%
\pgfsetmiterjoin%
\definecolor{currentfill}{rgb}{0.511253,0.510898,0.193296}%
\pgfsetfillcolor{currentfill}%
\pgfsetlinewidth{0.000000pt}%
\definecolor{currentstroke}{rgb}{0.000000,0.000000,0.000000}%
\pgfsetstrokecolor{currentstroke}%
\pgfsetstrokeopacity{0.000000}%
\pgfsetdash{}{0pt}%
\pgfpathmoveto{\pgfqpoint{4.243813in}{1.474100in}}%
\pgfpathlineto{\pgfqpoint{4.252567in}{1.474100in}}%
\pgfpathlineto{\pgfqpoint{4.252567in}{1.354497in}}%
\pgfpathlineto{\pgfqpoint{4.243813in}{1.354497in}}%
\pgfpathlineto{\pgfqpoint{4.243813in}{1.474100in}}%
\pgfpathclose%
\pgfusepath{fill}%
\end{pgfscope}%
\begin{pgfscope}%
\pgfpathrectangle{\pgfqpoint{3.776708in}{0.600000in}}{\pgfqpoint{2.573292in}{2.070576in}}%
\pgfusepath{clip}%
\pgfsetbuttcap%
\pgfsetmiterjoin%
\definecolor{currentfill}{rgb}{0.511253,0.510898,0.193296}%
\pgfsetfillcolor{currentfill}%
\pgfsetlinewidth{0.000000pt}%
\definecolor{currentstroke}{rgb}{0.000000,0.000000,0.000000}%
\pgfsetstrokecolor{currentstroke}%
\pgfsetstrokeopacity{0.000000}%
\pgfsetdash{}{0pt}%
\pgfpathmoveto{\pgfqpoint{4.254755in}{1.483465in}}%
\pgfpathlineto{\pgfqpoint{4.263508in}{1.483465in}}%
\pgfpathlineto{\pgfqpoint{4.263508in}{1.346888in}}%
\pgfpathlineto{\pgfqpoint{4.254755in}{1.346888in}}%
\pgfpathlineto{\pgfqpoint{4.254755in}{1.483465in}}%
\pgfpathclose%
\pgfusepath{fill}%
\end{pgfscope}%
\begin{pgfscope}%
\pgfpathrectangle{\pgfqpoint{3.776708in}{0.600000in}}{\pgfqpoint{2.573292in}{2.070576in}}%
\pgfusepath{clip}%
\pgfsetbuttcap%
\pgfsetmiterjoin%
\definecolor{currentfill}{rgb}{0.511253,0.510898,0.193296}%
\pgfsetfillcolor{currentfill}%
\pgfsetlinewidth{0.000000pt}%
\definecolor{currentstroke}{rgb}{0.000000,0.000000,0.000000}%
\pgfsetstrokecolor{currentstroke}%
\pgfsetstrokeopacity{0.000000}%
\pgfsetdash{}{0pt}%
\pgfpathmoveto{\pgfqpoint{4.265697in}{1.479511in}}%
\pgfpathlineto{\pgfqpoint{4.274450in}{1.479511in}}%
\pgfpathlineto{\pgfqpoint{4.274450in}{1.329941in}}%
\pgfpathlineto{\pgfqpoint{4.265697in}{1.329941in}}%
\pgfpathlineto{\pgfqpoint{4.265697in}{1.479511in}}%
\pgfpathclose%
\pgfusepath{fill}%
\end{pgfscope}%
\begin{pgfscope}%
\pgfpathrectangle{\pgfqpoint{3.776708in}{0.600000in}}{\pgfqpoint{2.573292in}{2.070576in}}%
\pgfusepath{clip}%
\pgfsetbuttcap%
\pgfsetmiterjoin%
\definecolor{currentfill}{rgb}{0.511253,0.510898,0.193296}%
\pgfsetfillcolor{currentfill}%
\pgfsetlinewidth{0.000000pt}%
\definecolor{currentstroke}{rgb}{0.000000,0.000000,0.000000}%
\pgfsetstrokecolor{currentstroke}%
\pgfsetstrokeopacity{0.000000}%
\pgfsetdash{}{0pt}%
\pgfpathmoveto{\pgfqpoint{4.276639in}{1.472560in}}%
\pgfpathlineto{\pgfqpoint{4.285392in}{1.472560in}}%
\pgfpathlineto{\pgfqpoint{4.285392in}{1.327035in}}%
\pgfpathlineto{\pgfqpoint{4.276639in}{1.327035in}}%
\pgfpathlineto{\pgfqpoint{4.276639in}{1.472560in}}%
\pgfpathclose%
\pgfusepath{fill}%
\end{pgfscope}%
\begin{pgfscope}%
\pgfpathrectangle{\pgfqpoint{3.776708in}{0.600000in}}{\pgfqpoint{2.573292in}{2.070576in}}%
\pgfusepath{clip}%
\pgfsetbuttcap%
\pgfsetmiterjoin%
\definecolor{currentfill}{rgb}{0.511253,0.510898,0.193296}%
\pgfsetfillcolor{currentfill}%
\pgfsetlinewidth{0.000000pt}%
\definecolor{currentstroke}{rgb}{0.000000,0.000000,0.000000}%
\pgfsetstrokecolor{currentstroke}%
\pgfsetstrokeopacity{0.000000}%
\pgfsetdash{}{0pt}%
\pgfpathmoveto{\pgfqpoint{4.287580in}{1.454564in}}%
\pgfpathlineto{\pgfqpoint{4.296334in}{1.454564in}}%
\pgfpathlineto{\pgfqpoint{4.296334in}{1.312655in}}%
\pgfpathlineto{\pgfqpoint{4.287580in}{1.312655in}}%
\pgfpathlineto{\pgfqpoint{4.287580in}{1.454564in}}%
\pgfpathclose%
\pgfusepath{fill}%
\end{pgfscope}%
\begin{pgfscope}%
\pgfpathrectangle{\pgfqpoint{3.776708in}{0.600000in}}{\pgfqpoint{2.573292in}{2.070576in}}%
\pgfusepath{clip}%
\pgfsetbuttcap%
\pgfsetmiterjoin%
\definecolor{currentfill}{rgb}{0.511253,0.510898,0.193296}%
\pgfsetfillcolor{currentfill}%
\pgfsetlinewidth{0.000000pt}%
\definecolor{currentstroke}{rgb}{0.000000,0.000000,0.000000}%
\pgfsetstrokecolor{currentstroke}%
\pgfsetstrokeopacity{0.000000}%
\pgfsetdash{}{0pt}%
\pgfpathmoveto{\pgfqpoint{4.298522in}{1.445838in}}%
\pgfpathlineto{\pgfqpoint{4.307276in}{1.445838in}}%
\pgfpathlineto{\pgfqpoint{4.307276in}{1.307691in}}%
\pgfpathlineto{\pgfqpoint{4.298522in}{1.307691in}}%
\pgfpathlineto{\pgfqpoint{4.298522in}{1.445838in}}%
\pgfpathclose%
\pgfusepath{fill}%
\end{pgfscope}%
\begin{pgfscope}%
\pgfpathrectangle{\pgfqpoint{3.776708in}{0.600000in}}{\pgfqpoint{2.573292in}{2.070576in}}%
\pgfusepath{clip}%
\pgfsetbuttcap%
\pgfsetmiterjoin%
\definecolor{currentfill}{rgb}{0.511253,0.510898,0.193296}%
\pgfsetfillcolor{currentfill}%
\pgfsetlinewidth{0.000000pt}%
\definecolor{currentstroke}{rgb}{0.000000,0.000000,0.000000}%
\pgfsetstrokecolor{currentstroke}%
\pgfsetstrokeopacity{0.000000}%
\pgfsetdash{}{0pt}%
\pgfpathmoveto{\pgfqpoint{4.309464in}{1.443942in}}%
\pgfpathlineto{\pgfqpoint{4.318217in}{1.443942in}}%
\pgfpathlineto{\pgfqpoint{4.318217in}{1.318908in}}%
\pgfpathlineto{\pgfqpoint{4.309464in}{1.318908in}}%
\pgfpathlineto{\pgfqpoint{4.309464in}{1.443942in}}%
\pgfpathclose%
\pgfusepath{fill}%
\end{pgfscope}%
\begin{pgfscope}%
\pgfpathrectangle{\pgfqpoint{3.776708in}{0.600000in}}{\pgfqpoint{2.573292in}{2.070576in}}%
\pgfusepath{clip}%
\pgfsetbuttcap%
\pgfsetmiterjoin%
\definecolor{currentfill}{rgb}{0.511253,0.510898,0.193296}%
\pgfsetfillcolor{currentfill}%
\pgfsetlinewidth{0.000000pt}%
\definecolor{currentstroke}{rgb}{0.000000,0.000000,0.000000}%
\pgfsetstrokecolor{currentstroke}%
\pgfsetstrokeopacity{0.000000}%
\pgfsetdash{}{0pt}%
\pgfpathmoveto{\pgfqpoint{4.320406in}{1.437903in}}%
\pgfpathlineto{\pgfqpoint{4.329159in}{1.437903in}}%
\pgfpathlineto{\pgfqpoint{4.329159in}{1.324950in}}%
\pgfpathlineto{\pgfqpoint{4.320406in}{1.324950in}}%
\pgfpathlineto{\pgfqpoint{4.320406in}{1.437903in}}%
\pgfpathclose%
\pgfusepath{fill}%
\end{pgfscope}%
\begin{pgfscope}%
\pgfpathrectangle{\pgfqpoint{3.776708in}{0.600000in}}{\pgfqpoint{2.573292in}{2.070576in}}%
\pgfusepath{clip}%
\pgfsetbuttcap%
\pgfsetmiterjoin%
\definecolor{currentfill}{rgb}{0.511253,0.510898,0.193296}%
\pgfsetfillcolor{currentfill}%
\pgfsetlinewidth{0.000000pt}%
\definecolor{currentstroke}{rgb}{0.000000,0.000000,0.000000}%
\pgfsetstrokecolor{currentstroke}%
\pgfsetstrokeopacity{0.000000}%
\pgfsetdash{}{0pt}%
\pgfpathmoveto{\pgfqpoint{4.331348in}{1.427592in}}%
\pgfpathlineto{\pgfqpoint{4.340101in}{1.427592in}}%
\pgfpathlineto{\pgfqpoint{4.340101in}{1.320763in}}%
\pgfpathlineto{\pgfqpoint{4.331348in}{1.320763in}}%
\pgfpathlineto{\pgfqpoint{4.331348in}{1.427592in}}%
\pgfpathclose%
\pgfusepath{fill}%
\end{pgfscope}%
\begin{pgfscope}%
\pgfpathrectangle{\pgfqpoint{3.776708in}{0.600000in}}{\pgfqpoint{2.573292in}{2.070576in}}%
\pgfusepath{clip}%
\pgfsetbuttcap%
\pgfsetmiterjoin%
\definecolor{currentfill}{rgb}{0.511253,0.510898,0.193296}%
\pgfsetfillcolor{currentfill}%
\pgfsetlinewidth{0.000000pt}%
\definecolor{currentstroke}{rgb}{0.000000,0.000000,0.000000}%
\pgfsetstrokecolor{currentstroke}%
\pgfsetstrokeopacity{0.000000}%
\pgfsetdash{}{0pt}%
\pgfpathmoveto{\pgfqpoint{4.342289in}{1.414221in}}%
\pgfpathlineto{\pgfqpoint{4.351043in}{1.414221in}}%
\pgfpathlineto{\pgfqpoint{4.351043in}{1.307070in}}%
\pgfpathlineto{\pgfqpoint{4.342289in}{1.307070in}}%
\pgfpathlineto{\pgfqpoint{4.342289in}{1.414221in}}%
\pgfpathclose%
\pgfusepath{fill}%
\end{pgfscope}%
\begin{pgfscope}%
\pgfpathrectangle{\pgfqpoint{3.776708in}{0.600000in}}{\pgfqpoint{2.573292in}{2.070576in}}%
\pgfusepath{clip}%
\pgfsetbuttcap%
\pgfsetmiterjoin%
\definecolor{currentfill}{rgb}{0.511253,0.510898,0.193296}%
\pgfsetfillcolor{currentfill}%
\pgfsetlinewidth{0.000000pt}%
\definecolor{currentstroke}{rgb}{0.000000,0.000000,0.000000}%
\pgfsetstrokecolor{currentstroke}%
\pgfsetstrokeopacity{0.000000}%
\pgfsetdash{}{0pt}%
\pgfpathmoveto{\pgfqpoint{4.353231in}{1.402583in}}%
\pgfpathlineto{\pgfqpoint{4.361985in}{1.402583in}}%
\pgfpathlineto{\pgfqpoint{4.361985in}{1.297310in}}%
\pgfpathlineto{\pgfqpoint{4.353231in}{1.297310in}}%
\pgfpathlineto{\pgfqpoint{4.353231in}{1.402583in}}%
\pgfpathclose%
\pgfusepath{fill}%
\end{pgfscope}%
\begin{pgfscope}%
\pgfpathrectangle{\pgfqpoint{3.776708in}{0.600000in}}{\pgfqpoint{2.573292in}{2.070576in}}%
\pgfusepath{clip}%
\pgfsetbuttcap%
\pgfsetmiterjoin%
\definecolor{currentfill}{rgb}{0.511253,0.510898,0.193296}%
\pgfsetfillcolor{currentfill}%
\pgfsetlinewidth{0.000000pt}%
\definecolor{currentstroke}{rgb}{0.000000,0.000000,0.000000}%
\pgfsetstrokecolor{currentstroke}%
\pgfsetstrokeopacity{0.000000}%
\pgfsetdash{}{0pt}%
\pgfpathmoveto{\pgfqpoint{4.364173in}{1.395256in}}%
\pgfpathlineto{\pgfqpoint{4.372926in}{1.395256in}}%
\pgfpathlineto{\pgfqpoint{4.372926in}{1.291703in}}%
\pgfpathlineto{\pgfqpoint{4.364173in}{1.291703in}}%
\pgfpathlineto{\pgfqpoint{4.364173in}{1.395256in}}%
\pgfpathclose%
\pgfusepath{fill}%
\end{pgfscope}%
\begin{pgfscope}%
\pgfpathrectangle{\pgfqpoint{3.776708in}{0.600000in}}{\pgfqpoint{2.573292in}{2.070576in}}%
\pgfusepath{clip}%
\pgfsetbuttcap%
\pgfsetmiterjoin%
\definecolor{currentfill}{rgb}{0.511253,0.510898,0.193296}%
\pgfsetfillcolor{currentfill}%
\pgfsetlinewidth{0.000000pt}%
\definecolor{currentstroke}{rgb}{0.000000,0.000000,0.000000}%
\pgfsetstrokecolor{currentstroke}%
\pgfsetstrokeopacity{0.000000}%
\pgfsetdash{}{0pt}%
\pgfpathmoveto{\pgfqpoint{4.375115in}{1.393238in}}%
\pgfpathlineto{\pgfqpoint{4.383868in}{1.393238in}}%
\pgfpathlineto{\pgfqpoint{4.383868in}{1.295633in}}%
\pgfpathlineto{\pgfqpoint{4.375115in}{1.295633in}}%
\pgfpathlineto{\pgfqpoint{4.375115in}{1.393238in}}%
\pgfpathclose%
\pgfusepath{fill}%
\end{pgfscope}%
\begin{pgfscope}%
\pgfpathrectangle{\pgfqpoint{3.776708in}{0.600000in}}{\pgfqpoint{2.573292in}{2.070576in}}%
\pgfusepath{clip}%
\pgfsetbuttcap%
\pgfsetmiterjoin%
\definecolor{currentfill}{rgb}{0.511253,0.510898,0.193296}%
\pgfsetfillcolor{currentfill}%
\pgfsetlinewidth{0.000000pt}%
\definecolor{currentstroke}{rgb}{0.000000,0.000000,0.000000}%
\pgfsetstrokecolor{currentstroke}%
\pgfsetstrokeopacity{0.000000}%
\pgfsetdash{}{0pt}%
\pgfpathmoveto{\pgfqpoint{4.386057in}{1.382370in}}%
\pgfpathlineto{\pgfqpoint{4.394810in}{1.382370in}}%
\pgfpathlineto{\pgfqpoint{4.394810in}{1.273727in}}%
\pgfpathlineto{\pgfqpoint{4.386057in}{1.273727in}}%
\pgfpathlineto{\pgfqpoint{4.386057in}{1.382370in}}%
\pgfpathclose%
\pgfusepath{fill}%
\end{pgfscope}%
\begin{pgfscope}%
\pgfpathrectangle{\pgfqpoint{3.776708in}{0.600000in}}{\pgfqpoint{2.573292in}{2.070576in}}%
\pgfusepath{clip}%
\pgfsetbuttcap%
\pgfsetmiterjoin%
\definecolor{currentfill}{rgb}{0.511253,0.510898,0.193296}%
\pgfsetfillcolor{currentfill}%
\pgfsetlinewidth{0.000000pt}%
\definecolor{currentstroke}{rgb}{0.000000,0.000000,0.000000}%
\pgfsetstrokecolor{currentstroke}%
\pgfsetstrokeopacity{0.000000}%
\pgfsetdash{}{0pt}%
\pgfpathmoveto{\pgfqpoint{4.396998in}{1.389016in}}%
\pgfpathlineto{\pgfqpoint{4.405752in}{1.389016in}}%
\pgfpathlineto{\pgfqpoint{4.405752in}{1.282668in}}%
\pgfpathlineto{\pgfqpoint{4.396998in}{1.282668in}}%
\pgfpathlineto{\pgfqpoint{4.396998in}{1.389016in}}%
\pgfpathclose%
\pgfusepath{fill}%
\end{pgfscope}%
\begin{pgfscope}%
\pgfpathrectangle{\pgfqpoint{3.776708in}{0.600000in}}{\pgfqpoint{2.573292in}{2.070576in}}%
\pgfusepath{clip}%
\pgfsetbuttcap%
\pgfsetmiterjoin%
\definecolor{currentfill}{rgb}{0.511253,0.510898,0.193296}%
\pgfsetfillcolor{currentfill}%
\pgfsetlinewidth{0.000000pt}%
\definecolor{currentstroke}{rgb}{0.000000,0.000000,0.000000}%
\pgfsetstrokecolor{currentstroke}%
\pgfsetstrokeopacity{0.000000}%
\pgfsetdash{}{0pt}%
\pgfpathmoveto{\pgfqpoint{4.407940in}{1.394253in}}%
\pgfpathlineto{\pgfqpoint{4.416694in}{1.394253in}}%
\pgfpathlineto{\pgfqpoint{4.416694in}{1.281707in}}%
\pgfpathlineto{\pgfqpoint{4.407940in}{1.281707in}}%
\pgfpathlineto{\pgfqpoint{4.407940in}{1.394253in}}%
\pgfpathclose%
\pgfusepath{fill}%
\end{pgfscope}%
\begin{pgfscope}%
\pgfpathrectangle{\pgfqpoint{3.776708in}{0.600000in}}{\pgfqpoint{2.573292in}{2.070576in}}%
\pgfusepath{clip}%
\pgfsetbuttcap%
\pgfsetmiterjoin%
\definecolor{currentfill}{rgb}{0.511253,0.510898,0.193296}%
\pgfsetfillcolor{currentfill}%
\pgfsetlinewidth{0.000000pt}%
\definecolor{currentstroke}{rgb}{0.000000,0.000000,0.000000}%
\pgfsetstrokecolor{currentstroke}%
\pgfsetstrokeopacity{0.000000}%
\pgfsetdash{}{0pt}%
\pgfpathmoveto{\pgfqpoint{4.418882in}{1.404544in}}%
\pgfpathlineto{\pgfqpoint{4.427635in}{1.404544in}}%
\pgfpathlineto{\pgfqpoint{4.427635in}{1.287423in}}%
\pgfpathlineto{\pgfqpoint{4.418882in}{1.287423in}}%
\pgfpathlineto{\pgfqpoint{4.418882in}{1.404544in}}%
\pgfpathclose%
\pgfusepath{fill}%
\end{pgfscope}%
\begin{pgfscope}%
\pgfpathrectangle{\pgfqpoint{3.776708in}{0.600000in}}{\pgfqpoint{2.573292in}{2.070576in}}%
\pgfusepath{clip}%
\pgfsetbuttcap%
\pgfsetmiterjoin%
\definecolor{currentfill}{rgb}{0.511253,0.510898,0.193296}%
\pgfsetfillcolor{currentfill}%
\pgfsetlinewidth{0.000000pt}%
\definecolor{currentstroke}{rgb}{0.000000,0.000000,0.000000}%
\pgfsetstrokecolor{currentstroke}%
\pgfsetstrokeopacity{0.000000}%
\pgfsetdash{}{0pt}%
\pgfpathmoveto{\pgfqpoint{4.429824in}{1.416801in}}%
\pgfpathlineto{\pgfqpoint{4.438577in}{1.416801in}}%
\pgfpathlineto{\pgfqpoint{4.438577in}{1.289685in}}%
\pgfpathlineto{\pgfqpoint{4.429824in}{1.289685in}}%
\pgfpathlineto{\pgfqpoint{4.429824in}{1.416801in}}%
\pgfpathclose%
\pgfusepath{fill}%
\end{pgfscope}%
\begin{pgfscope}%
\pgfpathrectangle{\pgfqpoint{3.776708in}{0.600000in}}{\pgfqpoint{2.573292in}{2.070576in}}%
\pgfusepath{clip}%
\pgfsetbuttcap%
\pgfsetmiterjoin%
\definecolor{currentfill}{rgb}{0.511253,0.510898,0.193296}%
\pgfsetfillcolor{currentfill}%
\pgfsetlinewidth{0.000000pt}%
\definecolor{currentstroke}{rgb}{0.000000,0.000000,0.000000}%
\pgfsetstrokecolor{currentstroke}%
\pgfsetstrokeopacity{0.000000}%
\pgfsetdash{}{0pt}%
\pgfpathmoveto{\pgfqpoint{4.440766in}{1.437287in}}%
\pgfpathlineto{\pgfqpoint{4.449519in}{1.437287in}}%
\pgfpathlineto{\pgfqpoint{4.449519in}{1.302714in}}%
\pgfpathlineto{\pgfqpoint{4.440766in}{1.302714in}}%
\pgfpathlineto{\pgfqpoint{4.440766in}{1.437287in}}%
\pgfpathclose%
\pgfusepath{fill}%
\end{pgfscope}%
\begin{pgfscope}%
\pgfpathrectangle{\pgfqpoint{3.776708in}{0.600000in}}{\pgfqpoint{2.573292in}{2.070576in}}%
\pgfusepath{clip}%
\pgfsetbuttcap%
\pgfsetmiterjoin%
\definecolor{currentfill}{rgb}{0.511253,0.510898,0.193296}%
\pgfsetfillcolor{currentfill}%
\pgfsetlinewidth{0.000000pt}%
\definecolor{currentstroke}{rgb}{0.000000,0.000000,0.000000}%
\pgfsetstrokecolor{currentstroke}%
\pgfsetstrokeopacity{0.000000}%
\pgfsetdash{}{0pt}%
\pgfpathmoveto{\pgfqpoint{4.451707in}{1.453238in}}%
\pgfpathlineto{\pgfqpoint{4.460461in}{1.453238in}}%
\pgfpathlineto{\pgfqpoint{4.460461in}{1.297137in}}%
\pgfpathlineto{\pgfqpoint{4.451707in}{1.297137in}}%
\pgfpathlineto{\pgfqpoint{4.451707in}{1.453238in}}%
\pgfpathclose%
\pgfusepath{fill}%
\end{pgfscope}%
\begin{pgfscope}%
\pgfpathrectangle{\pgfqpoint{3.776708in}{0.600000in}}{\pgfqpoint{2.573292in}{2.070576in}}%
\pgfusepath{clip}%
\pgfsetbuttcap%
\pgfsetmiterjoin%
\definecolor{currentfill}{rgb}{0.511253,0.510898,0.193296}%
\pgfsetfillcolor{currentfill}%
\pgfsetlinewidth{0.000000pt}%
\definecolor{currentstroke}{rgb}{0.000000,0.000000,0.000000}%
\pgfsetstrokecolor{currentstroke}%
\pgfsetstrokeopacity{0.000000}%
\pgfsetdash{}{0pt}%
\pgfpathmoveto{\pgfqpoint{4.462649in}{1.475696in}}%
\pgfpathlineto{\pgfqpoint{4.471403in}{1.475696in}}%
\pgfpathlineto{\pgfqpoint{4.471403in}{1.314509in}}%
\pgfpathlineto{\pgfqpoint{4.462649in}{1.314509in}}%
\pgfpathlineto{\pgfqpoint{4.462649in}{1.475696in}}%
\pgfpathclose%
\pgfusepath{fill}%
\end{pgfscope}%
\begin{pgfscope}%
\pgfpathrectangle{\pgfqpoint{3.776708in}{0.600000in}}{\pgfqpoint{2.573292in}{2.070576in}}%
\pgfusepath{clip}%
\pgfsetbuttcap%
\pgfsetmiterjoin%
\definecolor{currentfill}{rgb}{0.511253,0.510898,0.193296}%
\pgfsetfillcolor{currentfill}%
\pgfsetlinewidth{0.000000pt}%
\definecolor{currentstroke}{rgb}{0.000000,0.000000,0.000000}%
\pgfsetstrokecolor{currentstroke}%
\pgfsetstrokeopacity{0.000000}%
\pgfsetdash{}{0pt}%
\pgfpathmoveto{\pgfqpoint{4.473591in}{1.502021in}}%
\pgfpathlineto{\pgfqpoint{4.482344in}{1.502021in}}%
\pgfpathlineto{\pgfqpoint{4.482344in}{1.340250in}}%
\pgfpathlineto{\pgfqpoint{4.473591in}{1.340250in}}%
\pgfpathlineto{\pgfqpoint{4.473591in}{1.502021in}}%
\pgfpathclose%
\pgfusepath{fill}%
\end{pgfscope}%
\begin{pgfscope}%
\pgfpathrectangle{\pgfqpoint{3.776708in}{0.600000in}}{\pgfqpoint{2.573292in}{2.070576in}}%
\pgfusepath{clip}%
\pgfsetbuttcap%
\pgfsetmiterjoin%
\definecolor{currentfill}{rgb}{0.511253,0.510898,0.193296}%
\pgfsetfillcolor{currentfill}%
\pgfsetlinewidth{0.000000pt}%
\definecolor{currentstroke}{rgb}{0.000000,0.000000,0.000000}%
\pgfsetstrokecolor{currentstroke}%
\pgfsetstrokeopacity{0.000000}%
\pgfsetdash{}{0pt}%
\pgfpathmoveto{\pgfqpoint{4.484533in}{1.537177in}}%
\pgfpathlineto{\pgfqpoint{4.493286in}{1.537177in}}%
\pgfpathlineto{\pgfqpoint{4.493286in}{1.371343in}}%
\pgfpathlineto{\pgfqpoint{4.484533in}{1.371343in}}%
\pgfpathlineto{\pgfqpoint{4.484533in}{1.537177in}}%
\pgfpathclose%
\pgfusepath{fill}%
\end{pgfscope}%
\begin{pgfscope}%
\pgfpathrectangle{\pgfqpoint{3.776708in}{0.600000in}}{\pgfqpoint{2.573292in}{2.070576in}}%
\pgfusepath{clip}%
\pgfsetbuttcap%
\pgfsetmiterjoin%
\definecolor{currentfill}{rgb}{0.511253,0.510898,0.193296}%
\pgfsetfillcolor{currentfill}%
\pgfsetlinewidth{0.000000pt}%
\definecolor{currentstroke}{rgb}{0.000000,0.000000,0.000000}%
\pgfsetstrokecolor{currentstroke}%
\pgfsetstrokeopacity{0.000000}%
\pgfsetdash{}{0pt}%
\pgfpathmoveto{\pgfqpoint{4.495475in}{1.569673in}}%
\pgfpathlineto{\pgfqpoint{4.504228in}{1.569673in}}%
\pgfpathlineto{\pgfqpoint{4.504228in}{1.404178in}}%
\pgfpathlineto{\pgfqpoint{4.495475in}{1.404178in}}%
\pgfpathlineto{\pgfqpoint{4.495475in}{1.569673in}}%
\pgfpathclose%
\pgfusepath{fill}%
\end{pgfscope}%
\begin{pgfscope}%
\pgfpathrectangle{\pgfqpoint{3.776708in}{0.600000in}}{\pgfqpoint{2.573292in}{2.070576in}}%
\pgfusepath{clip}%
\pgfsetbuttcap%
\pgfsetmiterjoin%
\definecolor{currentfill}{rgb}{0.511253,0.510898,0.193296}%
\pgfsetfillcolor{currentfill}%
\pgfsetlinewidth{0.000000pt}%
\definecolor{currentstroke}{rgb}{0.000000,0.000000,0.000000}%
\pgfsetstrokecolor{currentstroke}%
\pgfsetstrokeopacity{0.000000}%
\pgfsetdash{}{0pt}%
\pgfpathmoveto{\pgfqpoint{4.506416in}{1.593854in}}%
\pgfpathlineto{\pgfqpoint{4.515170in}{1.593854in}}%
\pgfpathlineto{\pgfqpoint{4.515170in}{1.419491in}}%
\pgfpathlineto{\pgfqpoint{4.506416in}{1.419491in}}%
\pgfpathlineto{\pgfqpoint{4.506416in}{1.593854in}}%
\pgfpathclose%
\pgfusepath{fill}%
\end{pgfscope}%
\begin{pgfscope}%
\pgfpathrectangle{\pgfqpoint{3.776708in}{0.600000in}}{\pgfqpoint{2.573292in}{2.070576in}}%
\pgfusepath{clip}%
\pgfsetbuttcap%
\pgfsetmiterjoin%
\definecolor{currentfill}{rgb}{0.511253,0.510898,0.193296}%
\pgfsetfillcolor{currentfill}%
\pgfsetlinewidth{0.000000pt}%
\definecolor{currentstroke}{rgb}{0.000000,0.000000,0.000000}%
\pgfsetstrokecolor{currentstroke}%
\pgfsetstrokeopacity{0.000000}%
\pgfsetdash{}{0pt}%
\pgfpathmoveto{\pgfqpoint{4.517358in}{1.595584in}}%
\pgfpathlineto{\pgfqpoint{4.526112in}{1.595584in}}%
\pgfpathlineto{\pgfqpoint{4.526112in}{1.418360in}}%
\pgfpathlineto{\pgfqpoint{4.517358in}{1.418360in}}%
\pgfpathlineto{\pgfqpoint{4.517358in}{1.595584in}}%
\pgfpathclose%
\pgfusepath{fill}%
\end{pgfscope}%
\begin{pgfscope}%
\pgfpathrectangle{\pgfqpoint{3.776708in}{0.600000in}}{\pgfqpoint{2.573292in}{2.070576in}}%
\pgfusepath{clip}%
\pgfsetbuttcap%
\pgfsetmiterjoin%
\definecolor{currentfill}{rgb}{0.511253,0.510898,0.193296}%
\pgfsetfillcolor{currentfill}%
\pgfsetlinewidth{0.000000pt}%
\definecolor{currentstroke}{rgb}{0.000000,0.000000,0.000000}%
\pgfsetstrokecolor{currentstroke}%
\pgfsetstrokeopacity{0.000000}%
\pgfsetdash{}{0pt}%
\pgfpathmoveto{\pgfqpoint{4.528300in}{1.591604in}}%
\pgfpathlineto{\pgfqpoint{4.537053in}{1.591604in}}%
\pgfpathlineto{\pgfqpoint{4.537053in}{1.413920in}}%
\pgfpathlineto{\pgfqpoint{4.528300in}{1.413920in}}%
\pgfpathlineto{\pgfqpoint{4.528300in}{1.591604in}}%
\pgfpathclose%
\pgfusepath{fill}%
\end{pgfscope}%
\begin{pgfscope}%
\pgfpathrectangle{\pgfqpoint{3.776708in}{0.600000in}}{\pgfqpoint{2.573292in}{2.070576in}}%
\pgfusepath{clip}%
\pgfsetbuttcap%
\pgfsetmiterjoin%
\definecolor{currentfill}{rgb}{0.511253,0.510898,0.193296}%
\pgfsetfillcolor{currentfill}%
\pgfsetlinewidth{0.000000pt}%
\definecolor{currentstroke}{rgb}{0.000000,0.000000,0.000000}%
\pgfsetstrokecolor{currentstroke}%
\pgfsetstrokeopacity{0.000000}%
\pgfsetdash{}{0pt}%
\pgfpathmoveto{\pgfqpoint{4.539242in}{1.587656in}}%
\pgfpathlineto{\pgfqpoint{4.547995in}{1.587656in}}%
\pgfpathlineto{\pgfqpoint{4.547995in}{1.408323in}}%
\pgfpathlineto{\pgfqpoint{4.539242in}{1.408323in}}%
\pgfpathlineto{\pgfqpoint{4.539242in}{1.587656in}}%
\pgfpathclose%
\pgfusepath{fill}%
\end{pgfscope}%
\begin{pgfscope}%
\pgfpathrectangle{\pgfqpoint{3.776708in}{0.600000in}}{\pgfqpoint{2.573292in}{2.070576in}}%
\pgfusepath{clip}%
\pgfsetbuttcap%
\pgfsetmiterjoin%
\definecolor{currentfill}{rgb}{0.511253,0.510898,0.193296}%
\pgfsetfillcolor{currentfill}%
\pgfsetlinewidth{0.000000pt}%
\definecolor{currentstroke}{rgb}{0.000000,0.000000,0.000000}%
\pgfsetstrokecolor{currentstroke}%
\pgfsetstrokeopacity{0.000000}%
\pgfsetdash{}{0pt}%
\pgfpathmoveto{\pgfqpoint{4.550183in}{1.580438in}}%
\pgfpathlineto{\pgfqpoint{4.558937in}{1.580438in}}%
\pgfpathlineto{\pgfqpoint{4.558937in}{1.404137in}}%
\pgfpathlineto{\pgfqpoint{4.550183in}{1.404137in}}%
\pgfpathlineto{\pgfqpoint{4.550183in}{1.580438in}}%
\pgfpathclose%
\pgfusepath{fill}%
\end{pgfscope}%
\begin{pgfscope}%
\pgfpathrectangle{\pgfqpoint{3.776708in}{0.600000in}}{\pgfqpoint{2.573292in}{2.070576in}}%
\pgfusepath{clip}%
\pgfsetbuttcap%
\pgfsetmiterjoin%
\definecolor{currentfill}{rgb}{0.511253,0.510898,0.193296}%
\pgfsetfillcolor{currentfill}%
\pgfsetlinewidth{0.000000pt}%
\definecolor{currentstroke}{rgb}{0.000000,0.000000,0.000000}%
\pgfsetstrokecolor{currentstroke}%
\pgfsetstrokeopacity{0.000000}%
\pgfsetdash{}{0pt}%
\pgfpathmoveto{\pgfqpoint{4.561125in}{1.577748in}}%
\pgfpathlineto{\pgfqpoint{4.569879in}{1.577748in}}%
\pgfpathlineto{\pgfqpoint{4.569879in}{1.402510in}}%
\pgfpathlineto{\pgfqpoint{4.561125in}{1.402510in}}%
\pgfpathlineto{\pgfqpoint{4.561125in}{1.577748in}}%
\pgfpathclose%
\pgfusepath{fill}%
\end{pgfscope}%
\begin{pgfscope}%
\pgfpathrectangle{\pgfqpoint{3.776708in}{0.600000in}}{\pgfqpoint{2.573292in}{2.070576in}}%
\pgfusepath{clip}%
\pgfsetbuttcap%
\pgfsetmiterjoin%
\definecolor{currentfill}{rgb}{0.511253,0.510898,0.193296}%
\pgfsetfillcolor{currentfill}%
\pgfsetlinewidth{0.000000pt}%
\definecolor{currentstroke}{rgb}{0.000000,0.000000,0.000000}%
\pgfsetstrokecolor{currentstroke}%
\pgfsetstrokeopacity{0.000000}%
\pgfsetdash{}{0pt}%
\pgfpathmoveto{\pgfqpoint{4.572067in}{1.571933in}}%
\pgfpathlineto{\pgfqpoint{4.580821in}{1.571933in}}%
\pgfpathlineto{\pgfqpoint{4.580821in}{1.406236in}}%
\pgfpathlineto{\pgfqpoint{4.572067in}{1.406236in}}%
\pgfpathlineto{\pgfqpoint{4.572067in}{1.571933in}}%
\pgfpathclose%
\pgfusepath{fill}%
\end{pgfscope}%
\begin{pgfscope}%
\pgfpathrectangle{\pgfqpoint{3.776708in}{0.600000in}}{\pgfqpoint{2.573292in}{2.070576in}}%
\pgfusepath{clip}%
\pgfsetbuttcap%
\pgfsetmiterjoin%
\definecolor{currentfill}{rgb}{0.511253,0.510898,0.193296}%
\pgfsetfillcolor{currentfill}%
\pgfsetlinewidth{0.000000pt}%
\definecolor{currentstroke}{rgb}{0.000000,0.000000,0.000000}%
\pgfsetstrokecolor{currentstroke}%
\pgfsetstrokeopacity{0.000000}%
\pgfsetdash{}{0pt}%
\pgfpathmoveto{\pgfqpoint{4.583009in}{1.562177in}}%
\pgfpathlineto{\pgfqpoint{4.591762in}{1.562177in}}%
\pgfpathlineto{\pgfqpoint{4.591762in}{1.400320in}}%
\pgfpathlineto{\pgfqpoint{4.583009in}{1.400320in}}%
\pgfpathlineto{\pgfqpoint{4.583009in}{1.562177in}}%
\pgfpathclose%
\pgfusepath{fill}%
\end{pgfscope}%
\begin{pgfscope}%
\pgfpathrectangle{\pgfqpoint{3.776708in}{0.600000in}}{\pgfqpoint{2.573292in}{2.070576in}}%
\pgfusepath{clip}%
\pgfsetbuttcap%
\pgfsetmiterjoin%
\definecolor{currentfill}{rgb}{0.511253,0.510898,0.193296}%
\pgfsetfillcolor{currentfill}%
\pgfsetlinewidth{0.000000pt}%
\definecolor{currentstroke}{rgb}{0.000000,0.000000,0.000000}%
\pgfsetstrokecolor{currentstroke}%
\pgfsetstrokeopacity{0.000000}%
\pgfsetdash{}{0pt}%
\pgfpathmoveto{\pgfqpoint{4.593951in}{1.556860in}}%
\pgfpathlineto{\pgfqpoint{4.602704in}{1.556860in}}%
\pgfpathlineto{\pgfqpoint{4.602704in}{1.405917in}}%
\pgfpathlineto{\pgfqpoint{4.593951in}{1.405917in}}%
\pgfpathlineto{\pgfqpoint{4.593951in}{1.556860in}}%
\pgfpathclose%
\pgfusepath{fill}%
\end{pgfscope}%
\begin{pgfscope}%
\pgfpathrectangle{\pgfqpoint{3.776708in}{0.600000in}}{\pgfqpoint{2.573292in}{2.070576in}}%
\pgfusepath{clip}%
\pgfsetbuttcap%
\pgfsetmiterjoin%
\definecolor{currentfill}{rgb}{0.511253,0.510898,0.193296}%
\pgfsetfillcolor{currentfill}%
\pgfsetlinewidth{0.000000pt}%
\definecolor{currentstroke}{rgb}{0.000000,0.000000,0.000000}%
\pgfsetstrokecolor{currentstroke}%
\pgfsetstrokeopacity{0.000000}%
\pgfsetdash{}{0pt}%
\pgfpathmoveto{\pgfqpoint{4.604892in}{1.547654in}}%
\pgfpathlineto{\pgfqpoint{4.613646in}{1.547654in}}%
\pgfpathlineto{\pgfqpoint{4.613646in}{1.406544in}}%
\pgfpathlineto{\pgfqpoint{4.604892in}{1.406544in}}%
\pgfpathlineto{\pgfqpoint{4.604892in}{1.547654in}}%
\pgfpathclose%
\pgfusepath{fill}%
\end{pgfscope}%
\begin{pgfscope}%
\pgfpathrectangle{\pgfqpoint{3.776708in}{0.600000in}}{\pgfqpoint{2.573292in}{2.070576in}}%
\pgfusepath{clip}%
\pgfsetbuttcap%
\pgfsetmiterjoin%
\definecolor{currentfill}{rgb}{0.511253,0.510898,0.193296}%
\pgfsetfillcolor{currentfill}%
\pgfsetlinewidth{0.000000pt}%
\definecolor{currentstroke}{rgb}{0.000000,0.000000,0.000000}%
\pgfsetstrokecolor{currentstroke}%
\pgfsetstrokeopacity{0.000000}%
\pgfsetdash{}{0pt}%
\pgfpathmoveto{\pgfqpoint{4.615834in}{1.541289in}}%
\pgfpathlineto{\pgfqpoint{4.624588in}{1.541289in}}%
\pgfpathlineto{\pgfqpoint{4.624588in}{1.418203in}}%
\pgfpathlineto{\pgfqpoint{4.615834in}{1.418203in}}%
\pgfpathlineto{\pgfqpoint{4.615834in}{1.541289in}}%
\pgfpathclose%
\pgfusepath{fill}%
\end{pgfscope}%
\begin{pgfscope}%
\pgfpathrectangle{\pgfqpoint{3.776708in}{0.600000in}}{\pgfqpoint{2.573292in}{2.070576in}}%
\pgfusepath{clip}%
\pgfsetbuttcap%
\pgfsetmiterjoin%
\definecolor{currentfill}{rgb}{0.511253,0.510898,0.193296}%
\pgfsetfillcolor{currentfill}%
\pgfsetlinewidth{0.000000pt}%
\definecolor{currentstroke}{rgb}{0.000000,0.000000,0.000000}%
\pgfsetstrokecolor{currentstroke}%
\pgfsetstrokeopacity{0.000000}%
\pgfsetdash{}{0pt}%
\pgfpathmoveto{\pgfqpoint{4.626776in}{1.536245in}}%
\pgfpathlineto{\pgfqpoint{4.635530in}{1.536245in}}%
\pgfpathlineto{\pgfqpoint{4.635530in}{1.428266in}}%
\pgfpathlineto{\pgfqpoint{4.626776in}{1.428266in}}%
\pgfpathlineto{\pgfqpoint{4.626776in}{1.536245in}}%
\pgfpathclose%
\pgfusepath{fill}%
\end{pgfscope}%
\begin{pgfscope}%
\pgfpathrectangle{\pgfqpoint{3.776708in}{0.600000in}}{\pgfqpoint{2.573292in}{2.070576in}}%
\pgfusepath{clip}%
\pgfsetbuttcap%
\pgfsetmiterjoin%
\definecolor{currentfill}{rgb}{0.511253,0.510898,0.193296}%
\pgfsetfillcolor{currentfill}%
\pgfsetlinewidth{0.000000pt}%
\definecolor{currentstroke}{rgb}{0.000000,0.000000,0.000000}%
\pgfsetstrokecolor{currentstroke}%
\pgfsetstrokeopacity{0.000000}%
\pgfsetdash{}{0pt}%
\pgfpathmoveto{\pgfqpoint{4.637718in}{1.533329in}}%
\pgfpathlineto{\pgfqpoint{4.646471in}{1.533329in}}%
\pgfpathlineto{\pgfqpoint{4.646471in}{1.435518in}}%
\pgfpathlineto{\pgfqpoint{4.637718in}{1.435518in}}%
\pgfpathlineto{\pgfqpoint{4.637718in}{1.533329in}}%
\pgfpathclose%
\pgfusepath{fill}%
\end{pgfscope}%
\begin{pgfscope}%
\pgfpathrectangle{\pgfqpoint{3.776708in}{0.600000in}}{\pgfqpoint{2.573292in}{2.070576in}}%
\pgfusepath{clip}%
\pgfsetbuttcap%
\pgfsetmiterjoin%
\definecolor{currentfill}{rgb}{0.511253,0.510898,0.193296}%
\pgfsetfillcolor{currentfill}%
\pgfsetlinewidth{0.000000pt}%
\definecolor{currentstroke}{rgb}{0.000000,0.000000,0.000000}%
\pgfsetstrokecolor{currentstroke}%
\pgfsetstrokeopacity{0.000000}%
\pgfsetdash{}{0pt}%
\pgfpathmoveto{\pgfqpoint{4.648660in}{1.531793in}}%
\pgfpathlineto{\pgfqpoint{4.657413in}{1.531793in}}%
\pgfpathlineto{\pgfqpoint{4.657413in}{1.445060in}}%
\pgfpathlineto{\pgfqpoint{4.648660in}{1.445060in}}%
\pgfpathlineto{\pgfqpoint{4.648660in}{1.531793in}}%
\pgfpathclose%
\pgfusepath{fill}%
\end{pgfscope}%
\begin{pgfscope}%
\pgfpathrectangle{\pgfqpoint{3.776708in}{0.600000in}}{\pgfqpoint{2.573292in}{2.070576in}}%
\pgfusepath{clip}%
\pgfsetbuttcap%
\pgfsetmiterjoin%
\definecolor{currentfill}{rgb}{0.511253,0.510898,0.193296}%
\pgfsetfillcolor{currentfill}%
\pgfsetlinewidth{0.000000pt}%
\definecolor{currentstroke}{rgb}{0.000000,0.000000,0.000000}%
\pgfsetstrokecolor{currentstroke}%
\pgfsetstrokeopacity{0.000000}%
\pgfsetdash{}{0pt}%
\pgfpathmoveto{\pgfqpoint{4.659601in}{1.533087in}}%
\pgfpathlineto{\pgfqpoint{4.668355in}{1.533087in}}%
\pgfpathlineto{\pgfqpoint{4.668355in}{1.450601in}}%
\pgfpathlineto{\pgfqpoint{4.659601in}{1.450601in}}%
\pgfpathlineto{\pgfqpoint{4.659601in}{1.533087in}}%
\pgfpathclose%
\pgfusepath{fill}%
\end{pgfscope}%
\begin{pgfscope}%
\pgfpathrectangle{\pgfqpoint{3.776708in}{0.600000in}}{\pgfqpoint{2.573292in}{2.070576in}}%
\pgfusepath{clip}%
\pgfsetbuttcap%
\pgfsetmiterjoin%
\definecolor{currentfill}{rgb}{0.511253,0.510898,0.193296}%
\pgfsetfillcolor{currentfill}%
\pgfsetlinewidth{0.000000pt}%
\definecolor{currentstroke}{rgb}{0.000000,0.000000,0.000000}%
\pgfsetstrokecolor{currentstroke}%
\pgfsetstrokeopacity{0.000000}%
\pgfsetdash{}{0pt}%
\pgfpathmoveto{\pgfqpoint{4.670543in}{1.535824in}}%
\pgfpathlineto{\pgfqpoint{4.679297in}{1.535824in}}%
\pgfpathlineto{\pgfqpoint{4.679297in}{1.458635in}}%
\pgfpathlineto{\pgfqpoint{4.670543in}{1.458635in}}%
\pgfpathlineto{\pgfqpoint{4.670543in}{1.535824in}}%
\pgfpathclose%
\pgfusepath{fill}%
\end{pgfscope}%
\begin{pgfscope}%
\pgfpathrectangle{\pgfqpoint{3.776708in}{0.600000in}}{\pgfqpoint{2.573292in}{2.070576in}}%
\pgfusepath{clip}%
\pgfsetbuttcap%
\pgfsetmiterjoin%
\definecolor{currentfill}{rgb}{0.511253,0.510898,0.193296}%
\pgfsetfillcolor{currentfill}%
\pgfsetlinewidth{0.000000pt}%
\definecolor{currentstroke}{rgb}{0.000000,0.000000,0.000000}%
\pgfsetstrokecolor{currentstroke}%
\pgfsetstrokeopacity{0.000000}%
\pgfsetdash{}{0pt}%
\pgfpathmoveto{\pgfqpoint{4.681485in}{1.527900in}}%
\pgfpathlineto{\pgfqpoint{4.690239in}{1.527900in}}%
\pgfpathlineto{\pgfqpoint{4.690239in}{1.454787in}}%
\pgfpathlineto{\pgfqpoint{4.681485in}{1.454787in}}%
\pgfpathlineto{\pgfqpoint{4.681485in}{1.527900in}}%
\pgfpathclose%
\pgfusepath{fill}%
\end{pgfscope}%
\begin{pgfscope}%
\pgfpathrectangle{\pgfqpoint{3.776708in}{0.600000in}}{\pgfqpoint{2.573292in}{2.070576in}}%
\pgfusepath{clip}%
\pgfsetbuttcap%
\pgfsetmiterjoin%
\definecolor{currentfill}{rgb}{0.511253,0.510898,0.193296}%
\pgfsetfillcolor{currentfill}%
\pgfsetlinewidth{0.000000pt}%
\definecolor{currentstroke}{rgb}{0.000000,0.000000,0.000000}%
\pgfsetstrokecolor{currentstroke}%
\pgfsetstrokeopacity{0.000000}%
\pgfsetdash{}{0pt}%
\pgfpathmoveto{\pgfqpoint{4.692427in}{1.505682in}}%
\pgfpathlineto{\pgfqpoint{4.701180in}{1.505682in}}%
\pgfpathlineto{\pgfqpoint{4.701180in}{1.436143in}}%
\pgfpathlineto{\pgfqpoint{4.692427in}{1.436143in}}%
\pgfpathlineto{\pgfqpoint{4.692427in}{1.505682in}}%
\pgfpathclose%
\pgfusepath{fill}%
\end{pgfscope}%
\begin{pgfscope}%
\pgfpathrectangle{\pgfqpoint{3.776708in}{0.600000in}}{\pgfqpoint{2.573292in}{2.070576in}}%
\pgfusepath{clip}%
\pgfsetbuttcap%
\pgfsetmiterjoin%
\definecolor{currentfill}{rgb}{0.511253,0.510898,0.193296}%
\pgfsetfillcolor{currentfill}%
\pgfsetlinewidth{0.000000pt}%
\definecolor{currentstroke}{rgb}{0.000000,0.000000,0.000000}%
\pgfsetstrokecolor{currentstroke}%
\pgfsetstrokeopacity{0.000000}%
\pgfsetdash{}{0pt}%
\pgfpathmoveto{\pgfqpoint{4.703369in}{1.486478in}}%
\pgfpathlineto{\pgfqpoint{4.712122in}{1.486478in}}%
\pgfpathlineto{\pgfqpoint{4.712122in}{1.417621in}}%
\pgfpathlineto{\pgfqpoint{4.703369in}{1.417621in}}%
\pgfpathlineto{\pgfqpoint{4.703369in}{1.486478in}}%
\pgfpathclose%
\pgfusepath{fill}%
\end{pgfscope}%
\begin{pgfscope}%
\pgfpathrectangle{\pgfqpoint{3.776708in}{0.600000in}}{\pgfqpoint{2.573292in}{2.070576in}}%
\pgfusepath{clip}%
\pgfsetbuttcap%
\pgfsetmiterjoin%
\definecolor{currentfill}{rgb}{0.511253,0.510898,0.193296}%
\pgfsetfillcolor{currentfill}%
\pgfsetlinewidth{0.000000pt}%
\definecolor{currentstroke}{rgb}{0.000000,0.000000,0.000000}%
\pgfsetstrokecolor{currentstroke}%
\pgfsetstrokeopacity{0.000000}%
\pgfsetdash{}{0pt}%
\pgfpathmoveto{\pgfqpoint{4.714310in}{1.464129in}}%
\pgfpathlineto{\pgfqpoint{4.723064in}{1.464129in}}%
\pgfpathlineto{\pgfqpoint{4.723064in}{1.405070in}}%
\pgfpathlineto{\pgfqpoint{4.714310in}{1.405070in}}%
\pgfpathlineto{\pgfqpoint{4.714310in}{1.464129in}}%
\pgfpathclose%
\pgfusepath{fill}%
\end{pgfscope}%
\begin{pgfscope}%
\pgfpathrectangle{\pgfqpoint{3.776708in}{0.600000in}}{\pgfqpoint{2.573292in}{2.070576in}}%
\pgfusepath{clip}%
\pgfsetbuttcap%
\pgfsetmiterjoin%
\definecolor{currentfill}{rgb}{0.511253,0.510898,0.193296}%
\pgfsetfillcolor{currentfill}%
\pgfsetlinewidth{0.000000pt}%
\definecolor{currentstroke}{rgb}{0.000000,0.000000,0.000000}%
\pgfsetstrokecolor{currentstroke}%
\pgfsetstrokeopacity{0.000000}%
\pgfsetdash{}{0pt}%
\pgfpathmoveto{\pgfqpoint{4.725252in}{1.441974in}}%
\pgfpathlineto{\pgfqpoint{4.734006in}{1.441974in}}%
\pgfpathlineto{\pgfqpoint{4.734006in}{1.388708in}}%
\pgfpathlineto{\pgfqpoint{4.725252in}{1.388708in}}%
\pgfpathlineto{\pgfqpoint{4.725252in}{1.441974in}}%
\pgfpathclose%
\pgfusepath{fill}%
\end{pgfscope}%
\begin{pgfscope}%
\pgfpathrectangle{\pgfqpoint{3.776708in}{0.600000in}}{\pgfqpoint{2.573292in}{2.070576in}}%
\pgfusepath{clip}%
\pgfsetbuttcap%
\pgfsetmiterjoin%
\definecolor{currentfill}{rgb}{0.511253,0.510898,0.193296}%
\pgfsetfillcolor{currentfill}%
\pgfsetlinewidth{0.000000pt}%
\definecolor{currentstroke}{rgb}{0.000000,0.000000,0.000000}%
\pgfsetstrokecolor{currentstroke}%
\pgfsetstrokeopacity{0.000000}%
\pgfsetdash{}{0pt}%
\pgfpathmoveto{\pgfqpoint{4.736194in}{1.422829in}}%
\pgfpathlineto{\pgfqpoint{4.744948in}{1.422829in}}%
\pgfpathlineto{\pgfqpoint{4.744948in}{1.379497in}}%
\pgfpathlineto{\pgfqpoint{4.736194in}{1.379497in}}%
\pgfpathlineto{\pgfqpoint{4.736194in}{1.422829in}}%
\pgfpathclose%
\pgfusepath{fill}%
\end{pgfscope}%
\begin{pgfscope}%
\pgfpathrectangle{\pgfqpoint{3.776708in}{0.600000in}}{\pgfqpoint{2.573292in}{2.070576in}}%
\pgfusepath{clip}%
\pgfsetbuttcap%
\pgfsetmiterjoin%
\definecolor{currentfill}{rgb}{0.511253,0.510898,0.193296}%
\pgfsetfillcolor{currentfill}%
\pgfsetlinewidth{0.000000pt}%
\definecolor{currentstroke}{rgb}{0.000000,0.000000,0.000000}%
\pgfsetstrokecolor{currentstroke}%
\pgfsetstrokeopacity{0.000000}%
\pgfsetdash{}{0pt}%
\pgfpathmoveto{\pgfqpoint{4.747136in}{1.396505in}}%
\pgfpathlineto{\pgfqpoint{4.755889in}{1.396505in}}%
\pgfpathlineto{\pgfqpoint{4.755889in}{1.362904in}}%
\pgfpathlineto{\pgfqpoint{4.747136in}{1.362904in}}%
\pgfpathlineto{\pgfqpoint{4.747136in}{1.396505in}}%
\pgfpathclose%
\pgfusepath{fill}%
\end{pgfscope}%
\begin{pgfscope}%
\pgfpathrectangle{\pgfqpoint{3.776708in}{0.600000in}}{\pgfqpoint{2.573292in}{2.070576in}}%
\pgfusepath{clip}%
\pgfsetbuttcap%
\pgfsetmiterjoin%
\definecolor{currentfill}{rgb}{0.511253,0.510898,0.193296}%
\pgfsetfillcolor{currentfill}%
\pgfsetlinewidth{0.000000pt}%
\definecolor{currentstroke}{rgb}{0.000000,0.000000,0.000000}%
\pgfsetstrokecolor{currentstroke}%
\pgfsetstrokeopacity{0.000000}%
\pgfsetdash{}{0pt}%
\pgfpathmoveto{\pgfqpoint{4.758078in}{1.373863in}}%
\pgfpathlineto{\pgfqpoint{4.766831in}{1.373863in}}%
\pgfpathlineto{\pgfqpoint{4.766831in}{1.347682in}}%
\pgfpathlineto{\pgfqpoint{4.758078in}{1.347682in}}%
\pgfpathlineto{\pgfqpoint{4.758078in}{1.373863in}}%
\pgfpathclose%
\pgfusepath{fill}%
\end{pgfscope}%
\begin{pgfscope}%
\pgfpathrectangle{\pgfqpoint{3.776708in}{0.600000in}}{\pgfqpoint{2.573292in}{2.070576in}}%
\pgfusepath{clip}%
\pgfsetbuttcap%
\pgfsetmiterjoin%
\definecolor{currentfill}{rgb}{0.511253,0.510898,0.193296}%
\pgfsetfillcolor{currentfill}%
\pgfsetlinewidth{0.000000pt}%
\definecolor{currentstroke}{rgb}{0.000000,0.000000,0.000000}%
\pgfsetstrokecolor{currentstroke}%
\pgfsetstrokeopacity{0.000000}%
\pgfsetdash{}{0pt}%
\pgfpathmoveto{\pgfqpoint{4.769019in}{1.353038in}}%
\pgfpathlineto{\pgfqpoint{4.777773in}{1.353038in}}%
\pgfpathlineto{\pgfqpoint{4.777773in}{1.333791in}}%
\pgfpathlineto{\pgfqpoint{4.769019in}{1.333791in}}%
\pgfpathlineto{\pgfqpoint{4.769019in}{1.353038in}}%
\pgfpathclose%
\pgfusepath{fill}%
\end{pgfscope}%
\begin{pgfscope}%
\pgfpathrectangle{\pgfqpoint{3.776708in}{0.600000in}}{\pgfqpoint{2.573292in}{2.070576in}}%
\pgfusepath{clip}%
\pgfsetbuttcap%
\pgfsetmiterjoin%
\definecolor{currentfill}{rgb}{0.511253,0.510898,0.193296}%
\pgfsetfillcolor{currentfill}%
\pgfsetlinewidth{0.000000pt}%
\definecolor{currentstroke}{rgb}{0.000000,0.000000,0.000000}%
\pgfsetstrokecolor{currentstroke}%
\pgfsetstrokeopacity{0.000000}%
\pgfsetdash{}{0pt}%
\pgfpathmoveto{\pgfqpoint{4.779961in}{1.337089in}}%
\pgfpathlineto{\pgfqpoint{4.788715in}{1.337089in}}%
\pgfpathlineto{\pgfqpoint{4.788715in}{1.325206in}}%
\pgfpathlineto{\pgfqpoint{4.779961in}{1.325206in}}%
\pgfpathlineto{\pgfqpoint{4.779961in}{1.337089in}}%
\pgfpathclose%
\pgfusepath{fill}%
\end{pgfscope}%
\begin{pgfscope}%
\pgfpathrectangle{\pgfqpoint{3.776708in}{0.600000in}}{\pgfqpoint{2.573292in}{2.070576in}}%
\pgfusepath{clip}%
\pgfsetbuttcap%
\pgfsetmiterjoin%
\definecolor{currentfill}{rgb}{0.511253,0.510898,0.193296}%
\pgfsetfillcolor{currentfill}%
\pgfsetlinewidth{0.000000pt}%
\definecolor{currentstroke}{rgb}{0.000000,0.000000,0.000000}%
\pgfsetstrokecolor{currentstroke}%
\pgfsetstrokeopacity{0.000000}%
\pgfsetdash{}{0pt}%
\pgfpathmoveto{\pgfqpoint{4.790903in}{1.323511in}}%
\pgfpathlineto{\pgfqpoint{4.799657in}{1.323511in}}%
\pgfpathlineto{\pgfqpoint{4.799657in}{1.314617in}}%
\pgfpathlineto{\pgfqpoint{4.790903in}{1.314617in}}%
\pgfpathlineto{\pgfqpoint{4.790903in}{1.323511in}}%
\pgfpathclose%
\pgfusepath{fill}%
\end{pgfscope}%
\begin{pgfscope}%
\pgfpathrectangle{\pgfqpoint{3.776708in}{0.600000in}}{\pgfqpoint{2.573292in}{2.070576in}}%
\pgfusepath{clip}%
\pgfsetbuttcap%
\pgfsetmiterjoin%
\definecolor{currentfill}{rgb}{0.511253,0.510898,0.193296}%
\pgfsetfillcolor{currentfill}%
\pgfsetlinewidth{0.000000pt}%
\definecolor{currentstroke}{rgb}{0.000000,0.000000,0.000000}%
\pgfsetstrokecolor{currentstroke}%
\pgfsetstrokeopacity{0.000000}%
\pgfsetdash{}{0pt}%
\pgfpathmoveto{\pgfqpoint{4.801845in}{1.308235in}}%
\pgfpathlineto{\pgfqpoint{4.810598in}{1.308235in}}%
\pgfpathlineto{\pgfqpoint{4.810598in}{1.304450in}}%
\pgfpathlineto{\pgfqpoint{4.801845in}{1.304450in}}%
\pgfpathlineto{\pgfqpoint{4.801845in}{1.308235in}}%
\pgfpathclose%
\pgfusepath{fill}%
\end{pgfscope}%
\begin{pgfscope}%
\pgfpathrectangle{\pgfqpoint{3.776708in}{0.600000in}}{\pgfqpoint{2.573292in}{2.070576in}}%
\pgfusepath{clip}%
\pgfsetbuttcap%
\pgfsetmiterjoin%
\definecolor{currentfill}{rgb}{0.511253,0.510898,0.193296}%
\pgfsetfillcolor{currentfill}%
\pgfsetlinewidth{0.000000pt}%
\definecolor{currentstroke}{rgb}{0.000000,0.000000,0.000000}%
\pgfsetstrokecolor{currentstroke}%
\pgfsetstrokeopacity{0.000000}%
\pgfsetdash{}{0pt}%
\pgfpathmoveto{\pgfqpoint{4.812787in}{1.291266in}}%
\pgfpathlineto{\pgfqpoint{4.821540in}{1.291266in}}%
\pgfpathlineto{\pgfqpoint{4.821540in}{1.289406in}}%
\pgfpathlineto{\pgfqpoint{4.812787in}{1.289406in}}%
\pgfpathlineto{\pgfqpoint{4.812787in}{1.291266in}}%
\pgfpathclose%
\pgfusepath{fill}%
\end{pgfscope}%
\begin{pgfscope}%
\pgfpathrectangle{\pgfqpoint{3.776708in}{0.600000in}}{\pgfqpoint{2.573292in}{2.070576in}}%
\pgfusepath{clip}%
\pgfsetbuttcap%
\pgfsetmiterjoin%
\definecolor{currentfill}{rgb}{0.511253,0.510898,0.193296}%
\pgfsetfillcolor{currentfill}%
\pgfsetlinewidth{0.000000pt}%
\definecolor{currentstroke}{rgb}{0.000000,0.000000,0.000000}%
\pgfsetstrokecolor{currentstroke}%
\pgfsetstrokeopacity{0.000000}%
\pgfsetdash{}{0pt}%
\pgfpathmoveto{\pgfqpoint{4.823728in}{2.032216in}}%
\pgfpathlineto{\pgfqpoint{4.832482in}{2.032216in}}%
\pgfpathlineto{\pgfqpoint{4.832482in}{2.036492in}}%
\pgfpathlineto{\pgfqpoint{4.823728in}{2.036492in}}%
\pgfpathlineto{\pgfqpoint{4.823728in}{2.032216in}}%
\pgfpathclose%
\pgfusepath{fill}%
\end{pgfscope}%
\begin{pgfscope}%
\pgfpathrectangle{\pgfqpoint{3.776708in}{0.600000in}}{\pgfqpoint{2.573292in}{2.070576in}}%
\pgfusepath{clip}%
\pgfsetbuttcap%
\pgfsetmiterjoin%
\definecolor{currentfill}{rgb}{0.511253,0.510898,0.193296}%
\pgfsetfillcolor{currentfill}%
\pgfsetlinewidth{0.000000pt}%
\definecolor{currentstroke}{rgb}{0.000000,0.000000,0.000000}%
\pgfsetstrokecolor{currentstroke}%
\pgfsetstrokeopacity{0.000000}%
\pgfsetdash{}{0pt}%
\pgfpathmoveto{\pgfqpoint{4.834670in}{2.029998in}}%
\pgfpathlineto{\pgfqpoint{4.843424in}{2.029998in}}%
\pgfpathlineto{\pgfqpoint{4.843424in}{2.037574in}}%
\pgfpathlineto{\pgfqpoint{4.834670in}{2.037574in}}%
\pgfpathlineto{\pgfqpoint{4.834670in}{2.029998in}}%
\pgfpathclose%
\pgfusepath{fill}%
\end{pgfscope}%
\begin{pgfscope}%
\pgfpathrectangle{\pgfqpoint{3.776708in}{0.600000in}}{\pgfqpoint{2.573292in}{2.070576in}}%
\pgfusepath{clip}%
\pgfsetbuttcap%
\pgfsetmiterjoin%
\definecolor{currentfill}{rgb}{0.511253,0.510898,0.193296}%
\pgfsetfillcolor{currentfill}%
\pgfsetlinewidth{0.000000pt}%
\definecolor{currentstroke}{rgb}{0.000000,0.000000,0.000000}%
\pgfsetstrokecolor{currentstroke}%
\pgfsetstrokeopacity{0.000000}%
\pgfsetdash{}{0pt}%
\pgfpathmoveto{\pgfqpoint{4.845612in}{2.022791in}}%
\pgfpathlineto{\pgfqpoint{4.854366in}{2.022791in}}%
\pgfpathlineto{\pgfqpoint{4.854366in}{2.029496in}}%
\pgfpathlineto{\pgfqpoint{4.845612in}{2.029496in}}%
\pgfpathlineto{\pgfqpoint{4.845612in}{2.022791in}}%
\pgfpathclose%
\pgfusepath{fill}%
\end{pgfscope}%
\begin{pgfscope}%
\pgfpathrectangle{\pgfqpoint{3.776708in}{0.600000in}}{\pgfqpoint{2.573292in}{2.070576in}}%
\pgfusepath{clip}%
\pgfsetbuttcap%
\pgfsetmiterjoin%
\definecolor{currentfill}{rgb}{0.511253,0.510898,0.193296}%
\pgfsetfillcolor{currentfill}%
\pgfsetlinewidth{0.000000pt}%
\definecolor{currentstroke}{rgb}{0.000000,0.000000,0.000000}%
\pgfsetstrokecolor{currentstroke}%
\pgfsetstrokeopacity{0.000000}%
\pgfsetdash{}{0pt}%
\pgfpathmoveto{\pgfqpoint{4.856554in}{2.014374in}}%
\pgfpathlineto{\pgfqpoint{4.865307in}{2.014374in}}%
\pgfpathlineto{\pgfqpoint{4.865307in}{2.020382in}}%
\pgfpathlineto{\pgfqpoint{4.856554in}{2.020382in}}%
\pgfpathlineto{\pgfqpoint{4.856554in}{2.014374in}}%
\pgfpathclose%
\pgfusepath{fill}%
\end{pgfscope}%
\begin{pgfscope}%
\pgfpathrectangle{\pgfqpoint{3.776708in}{0.600000in}}{\pgfqpoint{2.573292in}{2.070576in}}%
\pgfusepath{clip}%
\pgfsetbuttcap%
\pgfsetmiterjoin%
\definecolor{currentfill}{rgb}{0.511253,0.510898,0.193296}%
\pgfsetfillcolor{currentfill}%
\pgfsetlinewidth{0.000000pt}%
\definecolor{currentstroke}{rgb}{0.000000,0.000000,0.000000}%
\pgfsetstrokecolor{currentstroke}%
\pgfsetstrokeopacity{0.000000}%
\pgfsetdash{}{0pt}%
\pgfpathmoveto{\pgfqpoint{4.867496in}{2.017129in}}%
\pgfpathlineto{\pgfqpoint{4.876249in}{2.017129in}}%
\pgfpathlineto{\pgfqpoint{4.876249in}{2.029793in}}%
\pgfpathlineto{\pgfqpoint{4.867496in}{2.029793in}}%
\pgfpathlineto{\pgfqpoint{4.867496in}{2.017129in}}%
\pgfpathclose%
\pgfusepath{fill}%
\end{pgfscope}%
\begin{pgfscope}%
\pgfpathrectangle{\pgfqpoint{3.776708in}{0.600000in}}{\pgfqpoint{2.573292in}{2.070576in}}%
\pgfusepath{clip}%
\pgfsetbuttcap%
\pgfsetmiterjoin%
\definecolor{currentfill}{rgb}{0.511253,0.510898,0.193296}%
\pgfsetfillcolor{currentfill}%
\pgfsetlinewidth{0.000000pt}%
\definecolor{currentstroke}{rgb}{0.000000,0.000000,0.000000}%
\pgfsetstrokecolor{currentstroke}%
\pgfsetstrokeopacity{0.000000}%
\pgfsetdash{}{0pt}%
\pgfpathmoveto{\pgfqpoint{4.878437in}{2.021216in}}%
\pgfpathlineto{\pgfqpoint{4.887191in}{2.021216in}}%
\pgfpathlineto{\pgfqpoint{4.887191in}{2.034929in}}%
\pgfpathlineto{\pgfqpoint{4.878437in}{2.034929in}}%
\pgfpathlineto{\pgfqpoint{4.878437in}{2.021216in}}%
\pgfpathclose%
\pgfusepath{fill}%
\end{pgfscope}%
\begin{pgfscope}%
\pgfpathrectangle{\pgfqpoint{3.776708in}{0.600000in}}{\pgfqpoint{2.573292in}{2.070576in}}%
\pgfusepath{clip}%
\pgfsetbuttcap%
\pgfsetmiterjoin%
\definecolor{currentfill}{rgb}{0.511253,0.510898,0.193296}%
\pgfsetfillcolor{currentfill}%
\pgfsetlinewidth{0.000000pt}%
\definecolor{currentstroke}{rgb}{0.000000,0.000000,0.000000}%
\pgfsetstrokecolor{currentstroke}%
\pgfsetstrokeopacity{0.000000}%
\pgfsetdash{}{0pt}%
\pgfpathmoveto{\pgfqpoint{4.889379in}{2.033920in}}%
\pgfpathlineto{\pgfqpoint{4.898133in}{2.033920in}}%
\pgfpathlineto{\pgfqpoint{4.898133in}{2.049097in}}%
\pgfpathlineto{\pgfqpoint{4.889379in}{2.049097in}}%
\pgfpathlineto{\pgfqpoint{4.889379in}{2.033920in}}%
\pgfpathclose%
\pgfusepath{fill}%
\end{pgfscope}%
\begin{pgfscope}%
\pgfpathrectangle{\pgfqpoint{3.776708in}{0.600000in}}{\pgfqpoint{2.573292in}{2.070576in}}%
\pgfusepath{clip}%
\pgfsetbuttcap%
\pgfsetmiterjoin%
\definecolor{currentfill}{rgb}{0.511253,0.510898,0.193296}%
\pgfsetfillcolor{currentfill}%
\pgfsetlinewidth{0.000000pt}%
\definecolor{currentstroke}{rgb}{0.000000,0.000000,0.000000}%
\pgfsetstrokecolor{currentstroke}%
\pgfsetstrokeopacity{0.000000}%
\pgfsetdash{}{0pt}%
\pgfpathmoveto{\pgfqpoint{4.900321in}{2.054115in}}%
\pgfpathlineto{\pgfqpoint{4.909075in}{2.054115in}}%
\pgfpathlineto{\pgfqpoint{4.909075in}{2.072653in}}%
\pgfpathlineto{\pgfqpoint{4.900321in}{2.072653in}}%
\pgfpathlineto{\pgfqpoint{4.900321in}{2.054115in}}%
\pgfpathclose%
\pgfusepath{fill}%
\end{pgfscope}%
\begin{pgfscope}%
\pgfpathrectangle{\pgfqpoint{3.776708in}{0.600000in}}{\pgfqpoint{2.573292in}{2.070576in}}%
\pgfusepath{clip}%
\pgfsetbuttcap%
\pgfsetmiterjoin%
\definecolor{currentfill}{rgb}{0.511253,0.510898,0.193296}%
\pgfsetfillcolor{currentfill}%
\pgfsetlinewidth{0.000000pt}%
\definecolor{currentstroke}{rgb}{0.000000,0.000000,0.000000}%
\pgfsetstrokecolor{currentstroke}%
\pgfsetstrokeopacity{0.000000}%
\pgfsetdash{}{0pt}%
\pgfpathmoveto{\pgfqpoint{4.911263in}{2.069913in}}%
\pgfpathlineto{\pgfqpoint{4.920016in}{2.069913in}}%
\pgfpathlineto{\pgfqpoint{4.920016in}{2.094202in}}%
\pgfpathlineto{\pgfqpoint{4.911263in}{2.094202in}}%
\pgfpathlineto{\pgfqpoint{4.911263in}{2.069913in}}%
\pgfpathclose%
\pgfusepath{fill}%
\end{pgfscope}%
\begin{pgfscope}%
\pgfpathrectangle{\pgfqpoint{3.776708in}{0.600000in}}{\pgfqpoint{2.573292in}{2.070576in}}%
\pgfusepath{clip}%
\pgfsetbuttcap%
\pgfsetmiterjoin%
\definecolor{currentfill}{rgb}{0.511253,0.510898,0.193296}%
\pgfsetfillcolor{currentfill}%
\pgfsetlinewidth{0.000000pt}%
\definecolor{currentstroke}{rgb}{0.000000,0.000000,0.000000}%
\pgfsetstrokecolor{currentstroke}%
\pgfsetstrokeopacity{0.000000}%
\pgfsetdash{}{0pt}%
\pgfpathmoveto{\pgfqpoint{4.922205in}{2.076875in}}%
\pgfpathlineto{\pgfqpoint{4.930958in}{2.076875in}}%
\pgfpathlineto{\pgfqpoint{4.930958in}{2.101744in}}%
\pgfpathlineto{\pgfqpoint{4.922205in}{2.101744in}}%
\pgfpathlineto{\pgfqpoint{4.922205in}{2.076875in}}%
\pgfpathclose%
\pgfusepath{fill}%
\end{pgfscope}%
\begin{pgfscope}%
\pgfpathrectangle{\pgfqpoint{3.776708in}{0.600000in}}{\pgfqpoint{2.573292in}{2.070576in}}%
\pgfusepath{clip}%
\pgfsetbuttcap%
\pgfsetmiterjoin%
\definecolor{currentfill}{rgb}{0.511253,0.510898,0.193296}%
\pgfsetfillcolor{currentfill}%
\pgfsetlinewidth{0.000000pt}%
\definecolor{currentstroke}{rgb}{0.000000,0.000000,0.000000}%
\pgfsetstrokecolor{currentstroke}%
\pgfsetstrokeopacity{0.000000}%
\pgfsetdash{}{0pt}%
\pgfpathmoveto{\pgfqpoint{4.933146in}{2.080229in}}%
\pgfpathlineto{\pgfqpoint{4.941900in}{2.080229in}}%
\pgfpathlineto{\pgfqpoint{4.941900in}{2.104889in}}%
\pgfpathlineto{\pgfqpoint{4.933146in}{2.104889in}}%
\pgfpathlineto{\pgfqpoint{4.933146in}{2.080229in}}%
\pgfpathclose%
\pgfusepath{fill}%
\end{pgfscope}%
\begin{pgfscope}%
\pgfpathrectangle{\pgfqpoint{3.776708in}{0.600000in}}{\pgfqpoint{2.573292in}{2.070576in}}%
\pgfusepath{clip}%
\pgfsetbuttcap%
\pgfsetmiterjoin%
\definecolor{currentfill}{rgb}{0.511253,0.510898,0.193296}%
\pgfsetfillcolor{currentfill}%
\pgfsetlinewidth{0.000000pt}%
\definecolor{currentstroke}{rgb}{0.000000,0.000000,0.000000}%
\pgfsetstrokecolor{currentstroke}%
\pgfsetstrokeopacity{0.000000}%
\pgfsetdash{}{0pt}%
\pgfpathmoveto{\pgfqpoint{4.944088in}{2.085609in}}%
\pgfpathlineto{\pgfqpoint{4.952842in}{2.085609in}}%
\pgfpathlineto{\pgfqpoint{4.952842in}{2.115013in}}%
\pgfpathlineto{\pgfqpoint{4.944088in}{2.115013in}}%
\pgfpathlineto{\pgfqpoint{4.944088in}{2.085609in}}%
\pgfpathclose%
\pgfusepath{fill}%
\end{pgfscope}%
\begin{pgfscope}%
\pgfpathrectangle{\pgfqpoint{3.776708in}{0.600000in}}{\pgfqpoint{2.573292in}{2.070576in}}%
\pgfusepath{clip}%
\pgfsetbuttcap%
\pgfsetmiterjoin%
\definecolor{currentfill}{rgb}{0.511253,0.510898,0.193296}%
\pgfsetfillcolor{currentfill}%
\pgfsetlinewidth{0.000000pt}%
\definecolor{currentstroke}{rgb}{0.000000,0.000000,0.000000}%
\pgfsetstrokecolor{currentstroke}%
\pgfsetstrokeopacity{0.000000}%
\pgfsetdash{}{0pt}%
\pgfpathmoveto{\pgfqpoint{4.955030in}{2.088528in}}%
\pgfpathlineto{\pgfqpoint{4.963783in}{2.088528in}}%
\pgfpathlineto{\pgfqpoint{4.963783in}{2.126009in}}%
\pgfpathlineto{\pgfqpoint{4.955030in}{2.126009in}}%
\pgfpathlineto{\pgfqpoint{4.955030in}{2.088528in}}%
\pgfpathclose%
\pgfusepath{fill}%
\end{pgfscope}%
\begin{pgfscope}%
\pgfpathrectangle{\pgfqpoint{3.776708in}{0.600000in}}{\pgfqpoint{2.573292in}{2.070576in}}%
\pgfusepath{clip}%
\pgfsetbuttcap%
\pgfsetmiterjoin%
\definecolor{currentfill}{rgb}{0.511253,0.510898,0.193296}%
\pgfsetfillcolor{currentfill}%
\pgfsetlinewidth{0.000000pt}%
\definecolor{currentstroke}{rgb}{0.000000,0.000000,0.000000}%
\pgfsetstrokecolor{currentstroke}%
\pgfsetstrokeopacity{0.000000}%
\pgfsetdash{}{0pt}%
\pgfpathmoveto{\pgfqpoint{4.965972in}{2.087955in}}%
\pgfpathlineto{\pgfqpoint{4.974725in}{2.087955in}}%
\pgfpathlineto{\pgfqpoint{4.974725in}{2.124836in}}%
\pgfpathlineto{\pgfqpoint{4.965972in}{2.124836in}}%
\pgfpathlineto{\pgfqpoint{4.965972in}{2.087955in}}%
\pgfpathclose%
\pgfusepath{fill}%
\end{pgfscope}%
\begin{pgfscope}%
\pgfpathrectangle{\pgfqpoint{3.776708in}{0.600000in}}{\pgfqpoint{2.573292in}{2.070576in}}%
\pgfusepath{clip}%
\pgfsetbuttcap%
\pgfsetmiterjoin%
\definecolor{currentfill}{rgb}{0.511253,0.510898,0.193296}%
\pgfsetfillcolor{currentfill}%
\pgfsetlinewidth{0.000000pt}%
\definecolor{currentstroke}{rgb}{0.000000,0.000000,0.000000}%
\pgfsetstrokecolor{currentstroke}%
\pgfsetstrokeopacity{0.000000}%
\pgfsetdash{}{0pt}%
\pgfpathmoveto{\pgfqpoint{4.976914in}{2.083569in}}%
\pgfpathlineto{\pgfqpoint{4.985667in}{2.083569in}}%
\pgfpathlineto{\pgfqpoint{4.985667in}{2.125220in}}%
\pgfpathlineto{\pgfqpoint{4.976914in}{2.125220in}}%
\pgfpathlineto{\pgfqpoint{4.976914in}{2.083569in}}%
\pgfpathclose%
\pgfusepath{fill}%
\end{pgfscope}%
\begin{pgfscope}%
\pgfpathrectangle{\pgfqpoint{3.776708in}{0.600000in}}{\pgfqpoint{2.573292in}{2.070576in}}%
\pgfusepath{clip}%
\pgfsetbuttcap%
\pgfsetmiterjoin%
\definecolor{currentfill}{rgb}{0.511253,0.510898,0.193296}%
\pgfsetfillcolor{currentfill}%
\pgfsetlinewidth{0.000000pt}%
\definecolor{currentstroke}{rgb}{0.000000,0.000000,0.000000}%
\pgfsetstrokecolor{currentstroke}%
\pgfsetstrokeopacity{0.000000}%
\pgfsetdash{}{0pt}%
\pgfpathmoveto{\pgfqpoint{4.987855in}{2.073181in}}%
\pgfpathlineto{\pgfqpoint{4.996609in}{2.073181in}}%
\pgfpathlineto{\pgfqpoint{4.996609in}{2.119994in}}%
\pgfpathlineto{\pgfqpoint{4.987855in}{2.119994in}}%
\pgfpathlineto{\pgfqpoint{4.987855in}{2.073181in}}%
\pgfpathclose%
\pgfusepath{fill}%
\end{pgfscope}%
\begin{pgfscope}%
\pgfpathrectangle{\pgfqpoint{3.776708in}{0.600000in}}{\pgfqpoint{2.573292in}{2.070576in}}%
\pgfusepath{clip}%
\pgfsetbuttcap%
\pgfsetmiterjoin%
\definecolor{currentfill}{rgb}{0.511253,0.510898,0.193296}%
\pgfsetfillcolor{currentfill}%
\pgfsetlinewidth{0.000000pt}%
\definecolor{currentstroke}{rgb}{0.000000,0.000000,0.000000}%
\pgfsetstrokecolor{currentstroke}%
\pgfsetstrokeopacity{0.000000}%
\pgfsetdash{}{0pt}%
\pgfpathmoveto{\pgfqpoint{4.998797in}{2.065582in}}%
\pgfpathlineto{\pgfqpoint{5.007551in}{2.065582in}}%
\pgfpathlineto{\pgfqpoint{5.007551in}{2.116754in}}%
\pgfpathlineto{\pgfqpoint{4.998797in}{2.116754in}}%
\pgfpathlineto{\pgfqpoint{4.998797in}{2.065582in}}%
\pgfpathclose%
\pgfusepath{fill}%
\end{pgfscope}%
\begin{pgfscope}%
\pgfpathrectangle{\pgfqpoint{3.776708in}{0.600000in}}{\pgfqpoint{2.573292in}{2.070576in}}%
\pgfusepath{clip}%
\pgfsetbuttcap%
\pgfsetmiterjoin%
\definecolor{currentfill}{rgb}{0.511253,0.510898,0.193296}%
\pgfsetfillcolor{currentfill}%
\pgfsetlinewidth{0.000000pt}%
\definecolor{currentstroke}{rgb}{0.000000,0.000000,0.000000}%
\pgfsetstrokecolor{currentstroke}%
\pgfsetstrokeopacity{0.000000}%
\pgfsetdash{}{0pt}%
\pgfpathmoveto{\pgfqpoint{5.009739in}{2.054645in}}%
\pgfpathlineto{\pgfqpoint{5.018492in}{2.054645in}}%
\pgfpathlineto{\pgfqpoint{5.018492in}{2.114279in}}%
\pgfpathlineto{\pgfqpoint{5.009739in}{2.114279in}}%
\pgfpathlineto{\pgfqpoint{5.009739in}{2.054645in}}%
\pgfpathclose%
\pgfusepath{fill}%
\end{pgfscope}%
\begin{pgfscope}%
\pgfpathrectangle{\pgfqpoint{3.776708in}{0.600000in}}{\pgfqpoint{2.573292in}{2.070576in}}%
\pgfusepath{clip}%
\pgfsetbuttcap%
\pgfsetmiterjoin%
\definecolor{currentfill}{rgb}{0.511253,0.510898,0.193296}%
\pgfsetfillcolor{currentfill}%
\pgfsetlinewidth{0.000000pt}%
\definecolor{currentstroke}{rgb}{0.000000,0.000000,0.000000}%
\pgfsetstrokecolor{currentstroke}%
\pgfsetstrokeopacity{0.000000}%
\pgfsetdash{}{0pt}%
\pgfpathmoveto{\pgfqpoint{5.020681in}{2.036665in}}%
\pgfpathlineto{\pgfqpoint{5.029434in}{2.036665in}}%
\pgfpathlineto{\pgfqpoint{5.029434in}{2.101203in}}%
\pgfpathlineto{\pgfqpoint{5.020681in}{2.101203in}}%
\pgfpathlineto{\pgfqpoint{5.020681in}{2.036665in}}%
\pgfpathclose%
\pgfusepath{fill}%
\end{pgfscope}%
\begin{pgfscope}%
\pgfpathrectangle{\pgfqpoint{3.776708in}{0.600000in}}{\pgfqpoint{2.573292in}{2.070576in}}%
\pgfusepath{clip}%
\pgfsetbuttcap%
\pgfsetmiterjoin%
\definecolor{currentfill}{rgb}{0.511253,0.510898,0.193296}%
\pgfsetfillcolor{currentfill}%
\pgfsetlinewidth{0.000000pt}%
\definecolor{currentstroke}{rgb}{0.000000,0.000000,0.000000}%
\pgfsetstrokecolor{currentstroke}%
\pgfsetstrokeopacity{0.000000}%
\pgfsetdash{}{0pt}%
\pgfpathmoveto{\pgfqpoint{5.031623in}{2.018518in}}%
\pgfpathlineto{\pgfqpoint{5.040376in}{2.018518in}}%
\pgfpathlineto{\pgfqpoint{5.040376in}{2.088388in}}%
\pgfpathlineto{\pgfqpoint{5.031623in}{2.088388in}}%
\pgfpathlineto{\pgfqpoint{5.031623in}{2.018518in}}%
\pgfpathclose%
\pgfusepath{fill}%
\end{pgfscope}%
\begin{pgfscope}%
\pgfpathrectangle{\pgfqpoint{3.776708in}{0.600000in}}{\pgfqpoint{2.573292in}{2.070576in}}%
\pgfusepath{clip}%
\pgfsetbuttcap%
\pgfsetmiterjoin%
\definecolor{currentfill}{rgb}{0.511253,0.510898,0.193296}%
\pgfsetfillcolor{currentfill}%
\pgfsetlinewidth{0.000000pt}%
\definecolor{currentstroke}{rgb}{0.000000,0.000000,0.000000}%
\pgfsetstrokecolor{currentstroke}%
\pgfsetstrokeopacity{0.000000}%
\pgfsetdash{}{0pt}%
\pgfpathmoveto{\pgfqpoint{5.042564in}{1.996655in}}%
\pgfpathlineto{\pgfqpoint{5.051318in}{1.996655in}}%
\pgfpathlineto{\pgfqpoint{5.051318in}{2.069998in}}%
\pgfpathlineto{\pgfqpoint{5.042564in}{2.069998in}}%
\pgfpathlineto{\pgfqpoint{5.042564in}{1.996655in}}%
\pgfpathclose%
\pgfusepath{fill}%
\end{pgfscope}%
\begin{pgfscope}%
\pgfpathrectangle{\pgfqpoint{3.776708in}{0.600000in}}{\pgfqpoint{2.573292in}{2.070576in}}%
\pgfusepath{clip}%
\pgfsetbuttcap%
\pgfsetmiterjoin%
\definecolor{currentfill}{rgb}{0.511253,0.510898,0.193296}%
\pgfsetfillcolor{currentfill}%
\pgfsetlinewidth{0.000000pt}%
\definecolor{currentstroke}{rgb}{0.000000,0.000000,0.000000}%
\pgfsetstrokecolor{currentstroke}%
\pgfsetstrokeopacity{0.000000}%
\pgfsetdash{}{0pt}%
\pgfpathmoveto{\pgfqpoint{5.053506in}{1.977227in}}%
\pgfpathlineto{\pgfqpoint{5.062260in}{1.977227in}}%
\pgfpathlineto{\pgfqpoint{5.062260in}{2.056116in}}%
\pgfpathlineto{\pgfqpoint{5.053506in}{2.056116in}}%
\pgfpathlineto{\pgfqpoint{5.053506in}{1.977227in}}%
\pgfpathclose%
\pgfusepath{fill}%
\end{pgfscope}%
\begin{pgfscope}%
\pgfpathrectangle{\pgfqpoint{3.776708in}{0.600000in}}{\pgfqpoint{2.573292in}{2.070576in}}%
\pgfusepath{clip}%
\pgfsetbuttcap%
\pgfsetmiterjoin%
\definecolor{currentfill}{rgb}{0.511253,0.510898,0.193296}%
\pgfsetfillcolor{currentfill}%
\pgfsetlinewidth{0.000000pt}%
\definecolor{currentstroke}{rgb}{0.000000,0.000000,0.000000}%
\pgfsetstrokecolor{currentstroke}%
\pgfsetstrokeopacity{0.000000}%
\pgfsetdash{}{0pt}%
\pgfpathmoveto{\pgfqpoint{5.064448in}{1.959040in}}%
\pgfpathlineto{\pgfqpoint{5.073201in}{1.959040in}}%
\pgfpathlineto{\pgfqpoint{5.073201in}{2.042065in}}%
\pgfpathlineto{\pgfqpoint{5.064448in}{2.042065in}}%
\pgfpathlineto{\pgfqpoint{5.064448in}{1.959040in}}%
\pgfpathclose%
\pgfusepath{fill}%
\end{pgfscope}%
\begin{pgfscope}%
\pgfpathrectangle{\pgfqpoint{3.776708in}{0.600000in}}{\pgfqpoint{2.573292in}{2.070576in}}%
\pgfusepath{clip}%
\pgfsetbuttcap%
\pgfsetmiterjoin%
\definecolor{currentfill}{rgb}{0.511253,0.510898,0.193296}%
\pgfsetfillcolor{currentfill}%
\pgfsetlinewidth{0.000000pt}%
\definecolor{currentstroke}{rgb}{0.000000,0.000000,0.000000}%
\pgfsetstrokecolor{currentstroke}%
\pgfsetstrokeopacity{0.000000}%
\pgfsetdash{}{0pt}%
\pgfpathmoveto{\pgfqpoint{5.075390in}{1.941103in}}%
\pgfpathlineto{\pgfqpoint{5.084143in}{1.941103in}}%
\pgfpathlineto{\pgfqpoint{5.084143in}{2.027982in}}%
\pgfpathlineto{\pgfqpoint{5.075390in}{2.027982in}}%
\pgfpathlineto{\pgfqpoint{5.075390in}{1.941103in}}%
\pgfpathclose%
\pgfusepath{fill}%
\end{pgfscope}%
\begin{pgfscope}%
\pgfpathrectangle{\pgfqpoint{3.776708in}{0.600000in}}{\pgfqpoint{2.573292in}{2.070576in}}%
\pgfusepath{clip}%
\pgfsetbuttcap%
\pgfsetmiterjoin%
\definecolor{currentfill}{rgb}{0.511253,0.510898,0.193296}%
\pgfsetfillcolor{currentfill}%
\pgfsetlinewidth{0.000000pt}%
\definecolor{currentstroke}{rgb}{0.000000,0.000000,0.000000}%
\pgfsetstrokecolor{currentstroke}%
\pgfsetstrokeopacity{0.000000}%
\pgfsetdash{}{0pt}%
\pgfpathmoveto{\pgfqpoint{5.086332in}{1.927349in}}%
\pgfpathlineto{\pgfqpoint{5.095085in}{1.927349in}}%
\pgfpathlineto{\pgfqpoint{5.095085in}{2.019991in}}%
\pgfpathlineto{\pgfqpoint{5.086332in}{2.019991in}}%
\pgfpathlineto{\pgfqpoint{5.086332in}{1.927349in}}%
\pgfpathclose%
\pgfusepath{fill}%
\end{pgfscope}%
\begin{pgfscope}%
\pgfpathrectangle{\pgfqpoint{3.776708in}{0.600000in}}{\pgfqpoint{2.573292in}{2.070576in}}%
\pgfusepath{clip}%
\pgfsetbuttcap%
\pgfsetmiterjoin%
\definecolor{currentfill}{rgb}{0.511253,0.510898,0.193296}%
\pgfsetfillcolor{currentfill}%
\pgfsetlinewidth{0.000000pt}%
\definecolor{currentstroke}{rgb}{0.000000,0.000000,0.000000}%
\pgfsetstrokecolor{currentstroke}%
\pgfsetstrokeopacity{0.000000}%
\pgfsetdash{}{0pt}%
\pgfpathmoveto{\pgfqpoint{5.097273in}{1.915714in}}%
\pgfpathlineto{\pgfqpoint{5.106027in}{1.915714in}}%
\pgfpathlineto{\pgfqpoint{5.106027in}{2.012674in}}%
\pgfpathlineto{\pgfqpoint{5.097273in}{2.012674in}}%
\pgfpathlineto{\pgfqpoint{5.097273in}{1.915714in}}%
\pgfpathclose%
\pgfusepath{fill}%
\end{pgfscope}%
\begin{pgfscope}%
\pgfpathrectangle{\pgfqpoint{3.776708in}{0.600000in}}{\pgfqpoint{2.573292in}{2.070576in}}%
\pgfusepath{clip}%
\pgfsetbuttcap%
\pgfsetmiterjoin%
\definecolor{currentfill}{rgb}{0.511253,0.510898,0.193296}%
\pgfsetfillcolor{currentfill}%
\pgfsetlinewidth{0.000000pt}%
\definecolor{currentstroke}{rgb}{0.000000,0.000000,0.000000}%
\pgfsetstrokecolor{currentstroke}%
\pgfsetstrokeopacity{0.000000}%
\pgfsetdash{}{0pt}%
\pgfpathmoveto{\pgfqpoint{5.108215in}{1.906607in}}%
\pgfpathlineto{\pgfqpoint{5.116969in}{1.906607in}}%
\pgfpathlineto{\pgfqpoint{5.116969in}{2.006648in}}%
\pgfpathlineto{\pgfqpoint{5.108215in}{2.006648in}}%
\pgfpathlineto{\pgfqpoint{5.108215in}{1.906607in}}%
\pgfpathclose%
\pgfusepath{fill}%
\end{pgfscope}%
\begin{pgfscope}%
\pgfpathrectangle{\pgfqpoint{3.776708in}{0.600000in}}{\pgfqpoint{2.573292in}{2.070576in}}%
\pgfusepath{clip}%
\pgfsetbuttcap%
\pgfsetmiterjoin%
\definecolor{currentfill}{rgb}{0.511253,0.510898,0.193296}%
\pgfsetfillcolor{currentfill}%
\pgfsetlinewidth{0.000000pt}%
\definecolor{currentstroke}{rgb}{0.000000,0.000000,0.000000}%
\pgfsetstrokecolor{currentstroke}%
\pgfsetstrokeopacity{0.000000}%
\pgfsetdash{}{0pt}%
\pgfpathmoveto{\pgfqpoint{5.119157in}{1.901250in}}%
\pgfpathlineto{\pgfqpoint{5.127910in}{1.901250in}}%
\pgfpathlineto{\pgfqpoint{5.127910in}{2.002851in}}%
\pgfpathlineto{\pgfqpoint{5.119157in}{2.002851in}}%
\pgfpathlineto{\pgfqpoint{5.119157in}{1.901250in}}%
\pgfpathclose%
\pgfusepath{fill}%
\end{pgfscope}%
\begin{pgfscope}%
\pgfpathrectangle{\pgfqpoint{3.776708in}{0.600000in}}{\pgfqpoint{2.573292in}{2.070576in}}%
\pgfusepath{clip}%
\pgfsetbuttcap%
\pgfsetmiterjoin%
\definecolor{currentfill}{rgb}{0.511253,0.510898,0.193296}%
\pgfsetfillcolor{currentfill}%
\pgfsetlinewidth{0.000000pt}%
\definecolor{currentstroke}{rgb}{0.000000,0.000000,0.000000}%
\pgfsetstrokecolor{currentstroke}%
\pgfsetstrokeopacity{0.000000}%
\pgfsetdash{}{0pt}%
\pgfpathmoveto{\pgfqpoint{5.130099in}{1.900037in}}%
\pgfpathlineto{\pgfqpoint{5.138852in}{1.900037in}}%
\pgfpathlineto{\pgfqpoint{5.138852in}{2.005634in}}%
\pgfpathlineto{\pgfqpoint{5.130099in}{2.005634in}}%
\pgfpathlineto{\pgfqpoint{5.130099in}{1.900037in}}%
\pgfpathclose%
\pgfusepath{fill}%
\end{pgfscope}%
\begin{pgfscope}%
\pgfpathrectangle{\pgfqpoint{3.776708in}{0.600000in}}{\pgfqpoint{2.573292in}{2.070576in}}%
\pgfusepath{clip}%
\pgfsetbuttcap%
\pgfsetmiterjoin%
\definecolor{currentfill}{rgb}{0.511253,0.510898,0.193296}%
\pgfsetfillcolor{currentfill}%
\pgfsetlinewidth{0.000000pt}%
\definecolor{currentstroke}{rgb}{0.000000,0.000000,0.000000}%
\pgfsetstrokecolor{currentstroke}%
\pgfsetstrokeopacity{0.000000}%
\pgfsetdash{}{0pt}%
\pgfpathmoveto{\pgfqpoint{5.141041in}{1.901980in}}%
\pgfpathlineto{\pgfqpoint{5.149794in}{1.901980in}}%
\pgfpathlineto{\pgfqpoint{5.149794in}{2.009418in}}%
\pgfpathlineto{\pgfqpoint{5.141041in}{2.009418in}}%
\pgfpathlineto{\pgfqpoint{5.141041in}{1.901980in}}%
\pgfpathclose%
\pgfusepath{fill}%
\end{pgfscope}%
\begin{pgfscope}%
\pgfpathrectangle{\pgfqpoint{3.776708in}{0.600000in}}{\pgfqpoint{2.573292in}{2.070576in}}%
\pgfusepath{clip}%
\pgfsetbuttcap%
\pgfsetmiterjoin%
\definecolor{currentfill}{rgb}{0.511253,0.510898,0.193296}%
\pgfsetfillcolor{currentfill}%
\pgfsetlinewidth{0.000000pt}%
\definecolor{currentstroke}{rgb}{0.000000,0.000000,0.000000}%
\pgfsetstrokecolor{currentstroke}%
\pgfsetstrokeopacity{0.000000}%
\pgfsetdash{}{0pt}%
\pgfpathmoveto{\pgfqpoint{5.151982in}{1.908586in}}%
\pgfpathlineto{\pgfqpoint{5.160736in}{1.908586in}}%
\pgfpathlineto{\pgfqpoint{5.160736in}{2.020559in}}%
\pgfpathlineto{\pgfqpoint{5.151982in}{2.020559in}}%
\pgfpathlineto{\pgfqpoint{5.151982in}{1.908586in}}%
\pgfpathclose%
\pgfusepath{fill}%
\end{pgfscope}%
\begin{pgfscope}%
\pgfpathrectangle{\pgfqpoint{3.776708in}{0.600000in}}{\pgfqpoint{2.573292in}{2.070576in}}%
\pgfusepath{clip}%
\pgfsetbuttcap%
\pgfsetmiterjoin%
\definecolor{currentfill}{rgb}{0.511253,0.510898,0.193296}%
\pgfsetfillcolor{currentfill}%
\pgfsetlinewidth{0.000000pt}%
\definecolor{currentstroke}{rgb}{0.000000,0.000000,0.000000}%
\pgfsetstrokecolor{currentstroke}%
\pgfsetstrokeopacity{0.000000}%
\pgfsetdash{}{0pt}%
\pgfpathmoveto{\pgfqpoint{5.162924in}{1.917842in}}%
\pgfpathlineto{\pgfqpoint{5.171678in}{1.917842in}}%
\pgfpathlineto{\pgfqpoint{5.171678in}{2.033854in}}%
\pgfpathlineto{\pgfqpoint{5.162924in}{2.033854in}}%
\pgfpathlineto{\pgfqpoint{5.162924in}{1.917842in}}%
\pgfpathclose%
\pgfusepath{fill}%
\end{pgfscope}%
\begin{pgfscope}%
\pgfpathrectangle{\pgfqpoint{3.776708in}{0.600000in}}{\pgfqpoint{2.573292in}{2.070576in}}%
\pgfusepath{clip}%
\pgfsetbuttcap%
\pgfsetmiterjoin%
\definecolor{currentfill}{rgb}{0.511253,0.510898,0.193296}%
\pgfsetfillcolor{currentfill}%
\pgfsetlinewidth{0.000000pt}%
\definecolor{currentstroke}{rgb}{0.000000,0.000000,0.000000}%
\pgfsetstrokecolor{currentstroke}%
\pgfsetstrokeopacity{0.000000}%
\pgfsetdash{}{0pt}%
\pgfpathmoveto{\pgfqpoint{5.173866in}{1.925488in}}%
\pgfpathlineto{\pgfqpoint{5.182619in}{1.925488in}}%
\pgfpathlineto{\pgfqpoint{5.182619in}{2.042490in}}%
\pgfpathlineto{\pgfqpoint{5.173866in}{2.042490in}}%
\pgfpathlineto{\pgfqpoint{5.173866in}{1.925488in}}%
\pgfpathclose%
\pgfusepath{fill}%
\end{pgfscope}%
\begin{pgfscope}%
\pgfpathrectangle{\pgfqpoint{3.776708in}{0.600000in}}{\pgfqpoint{2.573292in}{2.070576in}}%
\pgfusepath{clip}%
\pgfsetbuttcap%
\pgfsetmiterjoin%
\definecolor{currentfill}{rgb}{0.511253,0.510898,0.193296}%
\pgfsetfillcolor{currentfill}%
\pgfsetlinewidth{0.000000pt}%
\definecolor{currentstroke}{rgb}{0.000000,0.000000,0.000000}%
\pgfsetstrokecolor{currentstroke}%
\pgfsetstrokeopacity{0.000000}%
\pgfsetdash{}{0pt}%
\pgfpathmoveto{\pgfqpoint{5.184808in}{1.932313in}}%
\pgfpathlineto{\pgfqpoint{5.193561in}{1.932313in}}%
\pgfpathlineto{\pgfqpoint{5.193561in}{2.053877in}}%
\pgfpathlineto{\pgfqpoint{5.184808in}{2.053877in}}%
\pgfpathlineto{\pgfqpoint{5.184808in}{1.932313in}}%
\pgfpathclose%
\pgfusepath{fill}%
\end{pgfscope}%
\begin{pgfscope}%
\pgfpathrectangle{\pgfqpoint{3.776708in}{0.600000in}}{\pgfqpoint{2.573292in}{2.070576in}}%
\pgfusepath{clip}%
\pgfsetbuttcap%
\pgfsetmiterjoin%
\definecolor{currentfill}{rgb}{0.511253,0.510898,0.193296}%
\pgfsetfillcolor{currentfill}%
\pgfsetlinewidth{0.000000pt}%
\definecolor{currentstroke}{rgb}{0.000000,0.000000,0.000000}%
\pgfsetstrokecolor{currentstroke}%
\pgfsetstrokeopacity{0.000000}%
\pgfsetdash{}{0pt}%
\pgfpathmoveto{\pgfqpoint{5.195750in}{1.937917in}}%
\pgfpathlineto{\pgfqpoint{5.204503in}{1.937917in}}%
\pgfpathlineto{\pgfqpoint{5.204503in}{2.061739in}}%
\pgfpathlineto{\pgfqpoint{5.195750in}{2.061739in}}%
\pgfpathlineto{\pgfqpoint{5.195750in}{1.937917in}}%
\pgfpathclose%
\pgfusepath{fill}%
\end{pgfscope}%
\begin{pgfscope}%
\pgfpathrectangle{\pgfqpoint{3.776708in}{0.600000in}}{\pgfqpoint{2.573292in}{2.070576in}}%
\pgfusepath{clip}%
\pgfsetbuttcap%
\pgfsetmiterjoin%
\definecolor{currentfill}{rgb}{0.511253,0.510898,0.193296}%
\pgfsetfillcolor{currentfill}%
\pgfsetlinewidth{0.000000pt}%
\definecolor{currentstroke}{rgb}{0.000000,0.000000,0.000000}%
\pgfsetstrokecolor{currentstroke}%
\pgfsetstrokeopacity{0.000000}%
\pgfsetdash{}{0pt}%
\pgfpathmoveto{\pgfqpoint{5.206691in}{1.944394in}}%
\pgfpathlineto{\pgfqpoint{5.215445in}{1.944394in}}%
\pgfpathlineto{\pgfqpoint{5.215445in}{2.071857in}}%
\pgfpathlineto{\pgfqpoint{5.206691in}{2.071857in}}%
\pgfpathlineto{\pgfqpoint{5.206691in}{1.944394in}}%
\pgfpathclose%
\pgfusepath{fill}%
\end{pgfscope}%
\begin{pgfscope}%
\pgfpathrectangle{\pgfqpoint{3.776708in}{0.600000in}}{\pgfqpoint{2.573292in}{2.070576in}}%
\pgfusepath{clip}%
\pgfsetbuttcap%
\pgfsetmiterjoin%
\definecolor{currentfill}{rgb}{0.511253,0.510898,0.193296}%
\pgfsetfillcolor{currentfill}%
\pgfsetlinewidth{0.000000pt}%
\definecolor{currentstroke}{rgb}{0.000000,0.000000,0.000000}%
\pgfsetstrokecolor{currentstroke}%
\pgfsetstrokeopacity{0.000000}%
\pgfsetdash{}{0pt}%
\pgfpathmoveto{\pgfqpoint{5.217633in}{1.949442in}}%
\pgfpathlineto{\pgfqpoint{5.226387in}{1.949442in}}%
\pgfpathlineto{\pgfqpoint{5.226387in}{2.076228in}}%
\pgfpathlineto{\pgfqpoint{5.217633in}{2.076228in}}%
\pgfpathlineto{\pgfqpoint{5.217633in}{1.949442in}}%
\pgfpathclose%
\pgfusepath{fill}%
\end{pgfscope}%
\begin{pgfscope}%
\pgfpathrectangle{\pgfqpoint{3.776708in}{0.600000in}}{\pgfqpoint{2.573292in}{2.070576in}}%
\pgfusepath{clip}%
\pgfsetbuttcap%
\pgfsetmiterjoin%
\definecolor{currentfill}{rgb}{0.511253,0.510898,0.193296}%
\pgfsetfillcolor{currentfill}%
\pgfsetlinewidth{0.000000pt}%
\definecolor{currentstroke}{rgb}{0.000000,0.000000,0.000000}%
\pgfsetstrokecolor{currentstroke}%
\pgfsetstrokeopacity{0.000000}%
\pgfsetdash{}{0pt}%
\pgfpathmoveto{\pgfqpoint{5.228575in}{1.953261in}}%
\pgfpathlineto{\pgfqpoint{5.237328in}{1.953261in}}%
\pgfpathlineto{\pgfqpoint{5.237328in}{2.080283in}}%
\pgfpathlineto{\pgfqpoint{5.228575in}{2.080283in}}%
\pgfpathlineto{\pgfqpoint{5.228575in}{1.953261in}}%
\pgfpathclose%
\pgfusepath{fill}%
\end{pgfscope}%
\begin{pgfscope}%
\pgfpathrectangle{\pgfqpoint{3.776708in}{0.600000in}}{\pgfqpoint{2.573292in}{2.070576in}}%
\pgfusepath{clip}%
\pgfsetbuttcap%
\pgfsetmiterjoin%
\definecolor{currentfill}{rgb}{0.511253,0.510898,0.193296}%
\pgfsetfillcolor{currentfill}%
\pgfsetlinewidth{0.000000pt}%
\definecolor{currentstroke}{rgb}{0.000000,0.000000,0.000000}%
\pgfsetstrokecolor{currentstroke}%
\pgfsetstrokeopacity{0.000000}%
\pgfsetdash{}{0pt}%
\pgfpathmoveto{\pgfqpoint{5.239517in}{1.963092in}}%
\pgfpathlineto{\pgfqpoint{5.248270in}{1.963092in}}%
\pgfpathlineto{\pgfqpoint{5.248270in}{2.090849in}}%
\pgfpathlineto{\pgfqpoint{5.239517in}{2.090849in}}%
\pgfpathlineto{\pgfqpoint{5.239517in}{1.963092in}}%
\pgfpathclose%
\pgfusepath{fill}%
\end{pgfscope}%
\begin{pgfscope}%
\pgfpathrectangle{\pgfqpoint{3.776708in}{0.600000in}}{\pgfqpoint{2.573292in}{2.070576in}}%
\pgfusepath{clip}%
\pgfsetbuttcap%
\pgfsetmiterjoin%
\definecolor{currentfill}{rgb}{0.511253,0.510898,0.193296}%
\pgfsetfillcolor{currentfill}%
\pgfsetlinewidth{0.000000pt}%
\definecolor{currentstroke}{rgb}{0.000000,0.000000,0.000000}%
\pgfsetstrokecolor{currentstroke}%
\pgfsetstrokeopacity{0.000000}%
\pgfsetdash{}{0pt}%
\pgfpathmoveto{\pgfqpoint{5.250459in}{1.970649in}}%
\pgfpathlineto{\pgfqpoint{5.259212in}{1.970649in}}%
\pgfpathlineto{\pgfqpoint{5.259212in}{2.096721in}}%
\pgfpathlineto{\pgfqpoint{5.250459in}{2.096721in}}%
\pgfpathlineto{\pgfqpoint{5.250459in}{1.970649in}}%
\pgfpathclose%
\pgfusepath{fill}%
\end{pgfscope}%
\begin{pgfscope}%
\pgfpathrectangle{\pgfqpoint{3.776708in}{0.600000in}}{\pgfqpoint{2.573292in}{2.070576in}}%
\pgfusepath{clip}%
\pgfsetbuttcap%
\pgfsetmiterjoin%
\definecolor{currentfill}{rgb}{0.511253,0.510898,0.193296}%
\pgfsetfillcolor{currentfill}%
\pgfsetlinewidth{0.000000pt}%
\definecolor{currentstroke}{rgb}{0.000000,0.000000,0.000000}%
\pgfsetstrokecolor{currentstroke}%
\pgfsetstrokeopacity{0.000000}%
\pgfsetdash{}{0pt}%
\pgfpathmoveto{\pgfqpoint{5.261400in}{1.978882in}}%
\pgfpathlineto{\pgfqpoint{5.270154in}{1.978882in}}%
\pgfpathlineto{\pgfqpoint{5.270154in}{2.108014in}}%
\pgfpathlineto{\pgfqpoint{5.261400in}{2.108014in}}%
\pgfpathlineto{\pgfqpoint{5.261400in}{1.978882in}}%
\pgfpathclose%
\pgfusepath{fill}%
\end{pgfscope}%
\begin{pgfscope}%
\pgfpathrectangle{\pgfqpoint{3.776708in}{0.600000in}}{\pgfqpoint{2.573292in}{2.070576in}}%
\pgfusepath{clip}%
\pgfsetbuttcap%
\pgfsetmiterjoin%
\definecolor{currentfill}{rgb}{0.511253,0.510898,0.193296}%
\pgfsetfillcolor{currentfill}%
\pgfsetlinewidth{0.000000pt}%
\definecolor{currentstroke}{rgb}{0.000000,0.000000,0.000000}%
\pgfsetstrokecolor{currentstroke}%
\pgfsetstrokeopacity{0.000000}%
\pgfsetdash{}{0pt}%
\pgfpathmoveto{\pgfqpoint{5.272342in}{1.988813in}}%
\pgfpathlineto{\pgfqpoint{5.281096in}{1.988813in}}%
\pgfpathlineto{\pgfqpoint{5.281096in}{2.119966in}}%
\pgfpathlineto{\pgfqpoint{5.272342in}{2.119966in}}%
\pgfpathlineto{\pgfqpoint{5.272342in}{1.988813in}}%
\pgfpathclose%
\pgfusepath{fill}%
\end{pgfscope}%
\begin{pgfscope}%
\pgfpathrectangle{\pgfqpoint{3.776708in}{0.600000in}}{\pgfqpoint{2.573292in}{2.070576in}}%
\pgfusepath{clip}%
\pgfsetbuttcap%
\pgfsetmiterjoin%
\definecolor{currentfill}{rgb}{0.511253,0.510898,0.193296}%
\pgfsetfillcolor{currentfill}%
\pgfsetlinewidth{0.000000pt}%
\definecolor{currentstroke}{rgb}{0.000000,0.000000,0.000000}%
\pgfsetstrokecolor{currentstroke}%
\pgfsetstrokeopacity{0.000000}%
\pgfsetdash{}{0pt}%
\pgfpathmoveto{\pgfqpoint{5.283284in}{1.995608in}}%
\pgfpathlineto{\pgfqpoint{5.292037in}{1.995608in}}%
\pgfpathlineto{\pgfqpoint{5.292037in}{2.126303in}}%
\pgfpathlineto{\pgfqpoint{5.283284in}{2.126303in}}%
\pgfpathlineto{\pgfqpoint{5.283284in}{1.995608in}}%
\pgfpathclose%
\pgfusepath{fill}%
\end{pgfscope}%
\begin{pgfscope}%
\pgfpathrectangle{\pgfqpoint{3.776708in}{0.600000in}}{\pgfqpoint{2.573292in}{2.070576in}}%
\pgfusepath{clip}%
\pgfsetbuttcap%
\pgfsetmiterjoin%
\definecolor{currentfill}{rgb}{0.511253,0.510898,0.193296}%
\pgfsetfillcolor{currentfill}%
\pgfsetlinewidth{0.000000pt}%
\definecolor{currentstroke}{rgb}{0.000000,0.000000,0.000000}%
\pgfsetstrokecolor{currentstroke}%
\pgfsetstrokeopacity{0.000000}%
\pgfsetdash{}{0pt}%
\pgfpathmoveto{\pgfqpoint{5.294226in}{1.997168in}}%
\pgfpathlineto{\pgfqpoint{5.302979in}{1.997168in}}%
\pgfpathlineto{\pgfqpoint{5.302979in}{2.120832in}}%
\pgfpathlineto{\pgfqpoint{5.294226in}{2.120832in}}%
\pgfpathlineto{\pgfqpoint{5.294226in}{1.997168in}}%
\pgfpathclose%
\pgfusepath{fill}%
\end{pgfscope}%
\begin{pgfscope}%
\pgfpathrectangle{\pgfqpoint{3.776708in}{0.600000in}}{\pgfqpoint{2.573292in}{2.070576in}}%
\pgfusepath{clip}%
\pgfsetbuttcap%
\pgfsetmiterjoin%
\definecolor{currentfill}{rgb}{0.511253,0.510898,0.193296}%
\pgfsetfillcolor{currentfill}%
\pgfsetlinewidth{0.000000pt}%
\definecolor{currentstroke}{rgb}{0.000000,0.000000,0.000000}%
\pgfsetstrokecolor{currentstroke}%
\pgfsetstrokeopacity{0.000000}%
\pgfsetdash{}{0pt}%
\pgfpathmoveto{\pgfqpoint{5.305168in}{2.002661in}}%
\pgfpathlineto{\pgfqpoint{5.313921in}{2.002661in}}%
\pgfpathlineto{\pgfqpoint{5.313921in}{2.125696in}}%
\pgfpathlineto{\pgfqpoint{5.305168in}{2.125696in}}%
\pgfpathlineto{\pgfqpoint{5.305168in}{2.002661in}}%
\pgfpathclose%
\pgfusepath{fill}%
\end{pgfscope}%
\begin{pgfscope}%
\pgfpathrectangle{\pgfqpoint{3.776708in}{0.600000in}}{\pgfqpoint{2.573292in}{2.070576in}}%
\pgfusepath{clip}%
\pgfsetbuttcap%
\pgfsetmiterjoin%
\definecolor{currentfill}{rgb}{0.511253,0.510898,0.193296}%
\pgfsetfillcolor{currentfill}%
\pgfsetlinewidth{0.000000pt}%
\definecolor{currentstroke}{rgb}{0.000000,0.000000,0.000000}%
\pgfsetstrokecolor{currentstroke}%
\pgfsetstrokeopacity{0.000000}%
\pgfsetdash{}{0pt}%
\pgfpathmoveto{\pgfqpoint{5.316109in}{2.003324in}}%
\pgfpathlineto{\pgfqpoint{5.324863in}{2.003324in}}%
\pgfpathlineto{\pgfqpoint{5.324863in}{2.122971in}}%
\pgfpathlineto{\pgfqpoint{5.316109in}{2.122971in}}%
\pgfpathlineto{\pgfqpoint{5.316109in}{2.003324in}}%
\pgfpathclose%
\pgfusepath{fill}%
\end{pgfscope}%
\begin{pgfscope}%
\pgfpathrectangle{\pgfqpoint{3.776708in}{0.600000in}}{\pgfqpoint{2.573292in}{2.070576in}}%
\pgfusepath{clip}%
\pgfsetbuttcap%
\pgfsetmiterjoin%
\definecolor{currentfill}{rgb}{0.511253,0.510898,0.193296}%
\pgfsetfillcolor{currentfill}%
\pgfsetlinewidth{0.000000pt}%
\definecolor{currentstroke}{rgb}{0.000000,0.000000,0.000000}%
\pgfsetstrokecolor{currentstroke}%
\pgfsetstrokeopacity{0.000000}%
\pgfsetdash{}{0pt}%
\pgfpathmoveto{\pgfqpoint{5.327051in}{1.999346in}}%
\pgfpathlineto{\pgfqpoint{5.335805in}{1.999346in}}%
\pgfpathlineto{\pgfqpoint{5.335805in}{2.114199in}}%
\pgfpathlineto{\pgfqpoint{5.327051in}{2.114199in}}%
\pgfpathlineto{\pgfqpoint{5.327051in}{1.999346in}}%
\pgfpathclose%
\pgfusepath{fill}%
\end{pgfscope}%
\begin{pgfscope}%
\pgfpathrectangle{\pgfqpoint{3.776708in}{0.600000in}}{\pgfqpoint{2.573292in}{2.070576in}}%
\pgfusepath{clip}%
\pgfsetbuttcap%
\pgfsetmiterjoin%
\definecolor{currentfill}{rgb}{0.511253,0.510898,0.193296}%
\pgfsetfillcolor{currentfill}%
\pgfsetlinewidth{0.000000pt}%
\definecolor{currentstroke}{rgb}{0.000000,0.000000,0.000000}%
\pgfsetstrokecolor{currentstroke}%
\pgfsetstrokeopacity{0.000000}%
\pgfsetdash{}{0pt}%
\pgfpathmoveto{\pgfqpoint{5.337993in}{1.997285in}}%
\pgfpathlineto{\pgfqpoint{5.346746in}{1.997285in}}%
\pgfpathlineto{\pgfqpoint{5.346746in}{2.108698in}}%
\pgfpathlineto{\pgfqpoint{5.337993in}{2.108698in}}%
\pgfpathlineto{\pgfqpoint{5.337993in}{1.997285in}}%
\pgfpathclose%
\pgfusepath{fill}%
\end{pgfscope}%
\begin{pgfscope}%
\pgfpathrectangle{\pgfqpoint{3.776708in}{0.600000in}}{\pgfqpoint{2.573292in}{2.070576in}}%
\pgfusepath{clip}%
\pgfsetbuttcap%
\pgfsetmiterjoin%
\definecolor{currentfill}{rgb}{0.511253,0.510898,0.193296}%
\pgfsetfillcolor{currentfill}%
\pgfsetlinewidth{0.000000pt}%
\definecolor{currentstroke}{rgb}{0.000000,0.000000,0.000000}%
\pgfsetstrokecolor{currentstroke}%
\pgfsetstrokeopacity{0.000000}%
\pgfsetdash{}{0pt}%
\pgfpathmoveto{\pgfqpoint{5.348935in}{1.993775in}}%
\pgfpathlineto{\pgfqpoint{5.357688in}{1.993775in}}%
\pgfpathlineto{\pgfqpoint{5.357688in}{2.099290in}}%
\pgfpathlineto{\pgfqpoint{5.348935in}{2.099290in}}%
\pgfpathlineto{\pgfqpoint{5.348935in}{1.993775in}}%
\pgfpathclose%
\pgfusepath{fill}%
\end{pgfscope}%
\begin{pgfscope}%
\pgfpathrectangle{\pgfqpoint{3.776708in}{0.600000in}}{\pgfqpoint{2.573292in}{2.070576in}}%
\pgfusepath{clip}%
\pgfsetbuttcap%
\pgfsetmiterjoin%
\definecolor{currentfill}{rgb}{0.511253,0.510898,0.193296}%
\pgfsetfillcolor{currentfill}%
\pgfsetlinewidth{0.000000pt}%
\definecolor{currentstroke}{rgb}{0.000000,0.000000,0.000000}%
\pgfsetstrokecolor{currentstroke}%
\pgfsetstrokeopacity{0.000000}%
\pgfsetdash{}{0pt}%
\pgfpathmoveto{\pgfqpoint{5.359877in}{1.988048in}}%
\pgfpathlineto{\pgfqpoint{5.368630in}{1.988048in}}%
\pgfpathlineto{\pgfqpoint{5.368630in}{2.084858in}}%
\pgfpathlineto{\pgfqpoint{5.359877in}{2.084858in}}%
\pgfpathlineto{\pgfqpoint{5.359877in}{1.988048in}}%
\pgfpathclose%
\pgfusepath{fill}%
\end{pgfscope}%
\begin{pgfscope}%
\pgfpathrectangle{\pgfqpoint{3.776708in}{0.600000in}}{\pgfqpoint{2.573292in}{2.070576in}}%
\pgfusepath{clip}%
\pgfsetbuttcap%
\pgfsetmiterjoin%
\definecolor{currentfill}{rgb}{0.511253,0.510898,0.193296}%
\pgfsetfillcolor{currentfill}%
\pgfsetlinewidth{0.000000pt}%
\definecolor{currentstroke}{rgb}{0.000000,0.000000,0.000000}%
\pgfsetstrokecolor{currentstroke}%
\pgfsetstrokeopacity{0.000000}%
\pgfsetdash{}{0pt}%
\pgfpathmoveto{\pgfqpoint{5.370818in}{1.985176in}}%
\pgfpathlineto{\pgfqpoint{5.379572in}{1.985176in}}%
\pgfpathlineto{\pgfqpoint{5.379572in}{2.074670in}}%
\pgfpathlineto{\pgfqpoint{5.370818in}{2.074670in}}%
\pgfpathlineto{\pgfqpoint{5.370818in}{1.985176in}}%
\pgfpathclose%
\pgfusepath{fill}%
\end{pgfscope}%
\begin{pgfscope}%
\pgfpathrectangle{\pgfqpoint{3.776708in}{0.600000in}}{\pgfqpoint{2.573292in}{2.070576in}}%
\pgfusepath{clip}%
\pgfsetbuttcap%
\pgfsetmiterjoin%
\definecolor{currentfill}{rgb}{0.511253,0.510898,0.193296}%
\pgfsetfillcolor{currentfill}%
\pgfsetlinewidth{0.000000pt}%
\definecolor{currentstroke}{rgb}{0.000000,0.000000,0.000000}%
\pgfsetstrokecolor{currentstroke}%
\pgfsetstrokeopacity{0.000000}%
\pgfsetdash{}{0pt}%
\pgfpathmoveto{\pgfqpoint{5.381760in}{1.984319in}}%
\pgfpathlineto{\pgfqpoint{5.390514in}{1.984319in}}%
\pgfpathlineto{\pgfqpoint{5.390514in}{2.068820in}}%
\pgfpathlineto{\pgfqpoint{5.381760in}{2.068820in}}%
\pgfpathlineto{\pgfqpoint{5.381760in}{1.984319in}}%
\pgfpathclose%
\pgfusepath{fill}%
\end{pgfscope}%
\begin{pgfscope}%
\pgfpathrectangle{\pgfqpoint{3.776708in}{0.600000in}}{\pgfqpoint{2.573292in}{2.070576in}}%
\pgfusepath{clip}%
\pgfsetbuttcap%
\pgfsetmiterjoin%
\definecolor{currentfill}{rgb}{0.511253,0.510898,0.193296}%
\pgfsetfillcolor{currentfill}%
\pgfsetlinewidth{0.000000pt}%
\definecolor{currentstroke}{rgb}{0.000000,0.000000,0.000000}%
\pgfsetstrokecolor{currentstroke}%
\pgfsetstrokeopacity{0.000000}%
\pgfsetdash{}{0pt}%
\pgfpathmoveto{\pgfqpoint{5.392702in}{1.984353in}}%
\pgfpathlineto{\pgfqpoint{5.401455in}{1.984353in}}%
\pgfpathlineto{\pgfqpoint{5.401455in}{2.066336in}}%
\pgfpathlineto{\pgfqpoint{5.392702in}{2.066336in}}%
\pgfpathlineto{\pgfqpoint{5.392702in}{1.984353in}}%
\pgfpathclose%
\pgfusepath{fill}%
\end{pgfscope}%
\begin{pgfscope}%
\pgfpathrectangle{\pgfqpoint{3.776708in}{0.600000in}}{\pgfqpoint{2.573292in}{2.070576in}}%
\pgfusepath{clip}%
\pgfsetbuttcap%
\pgfsetmiterjoin%
\definecolor{currentfill}{rgb}{0.511253,0.510898,0.193296}%
\pgfsetfillcolor{currentfill}%
\pgfsetlinewidth{0.000000pt}%
\definecolor{currentstroke}{rgb}{0.000000,0.000000,0.000000}%
\pgfsetstrokecolor{currentstroke}%
\pgfsetstrokeopacity{0.000000}%
\pgfsetdash{}{0pt}%
\pgfpathmoveto{\pgfqpoint{5.403644in}{1.979928in}}%
\pgfpathlineto{\pgfqpoint{5.412397in}{1.979928in}}%
\pgfpathlineto{\pgfqpoint{5.412397in}{2.058271in}}%
\pgfpathlineto{\pgfqpoint{5.403644in}{2.058271in}}%
\pgfpathlineto{\pgfqpoint{5.403644in}{1.979928in}}%
\pgfpathclose%
\pgfusepath{fill}%
\end{pgfscope}%
\begin{pgfscope}%
\pgfpathrectangle{\pgfqpoint{3.776708in}{0.600000in}}{\pgfqpoint{2.573292in}{2.070576in}}%
\pgfusepath{clip}%
\pgfsetbuttcap%
\pgfsetmiterjoin%
\definecolor{currentfill}{rgb}{0.511253,0.510898,0.193296}%
\pgfsetfillcolor{currentfill}%
\pgfsetlinewidth{0.000000pt}%
\definecolor{currentstroke}{rgb}{0.000000,0.000000,0.000000}%
\pgfsetstrokecolor{currentstroke}%
\pgfsetstrokeopacity{0.000000}%
\pgfsetdash{}{0pt}%
\pgfpathmoveto{\pgfqpoint{5.414586in}{1.973566in}}%
\pgfpathlineto{\pgfqpoint{5.423339in}{1.973566in}}%
\pgfpathlineto{\pgfqpoint{5.423339in}{2.046898in}}%
\pgfpathlineto{\pgfqpoint{5.414586in}{2.046898in}}%
\pgfpathlineto{\pgfqpoint{5.414586in}{1.973566in}}%
\pgfpathclose%
\pgfusepath{fill}%
\end{pgfscope}%
\begin{pgfscope}%
\pgfpathrectangle{\pgfqpoint{3.776708in}{0.600000in}}{\pgfqpoint{2.573292in}{2.070576in}}%
\pgfusepath{clip}%
\pgfsetbuttcap%
\pgfsetmiterjoin%
\definecolor{currentfill}{rgb}{0.511253,0.510898,0.193296}%
\pgfsetfillcolor{currentfill}%
\pgfsetlinewidth{0.000000pt}%
\definecolor{currentstroke}{rgb}{0.000000,0.000000,0.000000}%
\pgfsetstrokecolor{currentstroke}%
\pgfsetstrokeopacity{0.000000}%
\pgfsetdash{}{0pt}%
\pgfpathmoveto{\pgfqpoint{5.425527in}{1.959786in}}%
\pgfpathlineto{\pgfqpoint{5.434281in}{1.959786in}}%
\pgfpathlineto{\pgfqpoint{5.434281in}{2.031737in}}%
\pgfpathlineto{\pgfqpoint{5.425527in}{2.031737in}}%
\pgfpathlineto{\pgfqpoint{5.425527in}{1.959786in}}%
\pgfpathclose%
\pgfusepath{fill}%
\end{pgfscope}%
\begin{pgfscope}%
\pgfpathrectangle{\pgfqpoint{3.776708in}{0.600000in}}{\pgfqpoint{2.573292in}{2.070576in}}%
\pgfusepath{clip}%
\pgfsetbuttcap%
\pgfsetmiterjoin%
\definecolor{currentfill}{rgb}{0.511253,0.510898,0.193296}%
\pgfsetfillcolor{currentfill}%
\pgfsetlinewidth{0.000000pt}%
\definecolor{currentstroke}{rgb}{0.000000,0.000000,0.000000}%
\pgfsetstrokecolor{currentstroke}%
\pgfsetstrokeopacity{0.000000}%
\pgfsetdash{}{0pt}%
\pgfpathmoveto{\pgfqpoint{5.436469in}{1.938271in}}%
\pgfpathlineto{\pgfqpoint{5.445223in}{1.938271in}}%
\pgfpathlineto{\pgfqpoint{5.445223in}{2.012810in}}%
\pgfpathlineto{\pgfqpoint{5.436469in}{2.012810in}}%
\pgfpathlineto{\pgfqpoint{5.436469in}{1.938271in}}%
\pgfpathclose%
\pgfusepath{fill}%
\end{pgfscope}%
\begin{pgfscope}%
\pgfpathrectangle{\pgfqpoint{3.776708in}{0.600000in}}{\pgfqpoint{2.573292in}{2.070576in}}%
\pgfusepath{clip}%
\pgfsetbuttcap%
\pgfsetmiterjoin%
\definecolor{currentfill}{rgb}{0.511253,0.510898,0.193296}%
\pgfsetfillcolor{currentfill}%
\pgfsetlinewidth{0.000000pt}%
\definecolor{currentstroke}{rgb}{0.000000,0.000000,0.000000}%
\pgfsetstrokecolor{currentstroke}%
\pgfsetstrokeopacity{0.000000}%
\pgfsetdash{}{0pt}%
\pgfpathmoveto{\pgfqpoint{5.447411in}{1.912266in}}%
\pgfpathlineto{\pgfqpoint{5.456164in}{1.912266in}}%
\pgfpathlineto{\pgfqpoint{5.456164in}{1.985311in}}%
\pgfpathlineto{\pgfqpoint{5.447411in}{1.985311in}}%
\pgfpathlineto{\pgfqpoint{5.447411in}{1.912266in}}%
\pgfpathclose%
\pgfusepath{fill}%
\end{pgfscope}%
\begin{pgfscope}%
\pgfpathrectangle{\pgfqpoint{3.776708in}{0.600000in}}{\pgfqpoint{2.573292in}{2.070576in}}%
\pgfusepath{clip}%
\pgfsetbuttcap%
\pgfsetmiterjoin%
\definecolor{currentfill}{rgb}{0.511253,0.510898,0.193296}%
\pgfsetfillcolor{currentfill}%
\pgfsetlinewidth{0.000000pt}%
\definecolor{currentstroke}{rgb}{0.000000,0.000000,0.000000}%
\pgfsetstrokecolor{currentstroke}%
\pgfsetstrokeopacity{0.000000}%
\pgfsetdash{}{0pt}%
\pgfpathmoveto{\pgfqpoint{5.458353in}{1.881998in}}%
\pgfpathlineto{\pgfqpoint{5.467106in}{1.881998in}}%
\pgfpathlineto{\pgfqpoint{5.467106in}{1.954401in}}%
\pgfpathlineto{\pgfqpoint{5.458353in}{1.954401in}}%
\pgfpathlineto{\pgfqpoint{5.458353in}{1.881998in}}%
\pgfpathclose%
\pgfusepath{fill}%
\end{pgfscope}%
\begin{pgfscope}%
\pgfpathrectangle{\pgfqpoint{3.776708in}{0.600000in}}{\pgfqpoint{2.573292in}{2.070576in}}%
\pgfusepath{clip}%
\pgfsetbuttcap%
\pgfsetmiterjoin%
\definecolor{currentfill}{rgb}{0.511253,0.510898,0.193296}%
\pgfsetfillcolor{currentfill}%
\pgfsetlinewidth{0.000000pt}%
\definecolor{currentstroke}{rgb}{0.000000,0.000000,0.000000}%
\pgfsetstrokecolor{currentstroke}%
\pgfsetstrokeopacity{0.000000}%
\pgfsetdash{}{0pt}%
\pgfpathmoveto{\pgfqpoint{5.469295in}{1.851084in}}%
\pgfpathlineto{\pgfqpoint{5.478048in}{1.851084in}}%
\pgfpathlineto{\pgfqpoint{5.478048in}{1.923184in}}%
\pgfpathlineto{\pgfqpoint{5.469295in}{1.923184in}}%
\pgfpathlineto{\pgfqpoint{5.469295in}{1.851084in}}%
\pgfpathclose%
\pgfusepath{fill}%
\end{pgfscope}%
\begin{pgfscope}%
\pgfpathrectangle{\pgfqpoint{3.776708in}{0.600000in}}{\pgfqpoint{2.573292in}{2.070576in}}%
\pgfusepath{clip}%
\pgfsetbuttcap%
\pgfsetmiterjoin%
\definecolor{currentfill}{rgb}{0.511253,0.510898,0.193296}%
\pgfsetfillcolor{currentfill}%
\pgfsetlinewidth{0.000000pt}%
\definecolor{currentstroke}{rgb}{0.000000,0.000000,0.000000}%
\pgfsetstrokecolor{currentstroke}%
\pgfsetstrokeopacity{0.000000}%
\pgfsetdash{}{0pt}%
\pgfpathmoveto{\pgfqpoint{5.480236in}{1.820553in}}%
\pgfpathlineto{\pgfqpoint{5.488990in}{1.820553in}}%
\pgfpathlineto{\pgfqpoint{5.488990in}{1.891728in}}%
\pgfpathlineto{\pgfqpoint{5.480236in}{1.891728in}}%
\pgfpathlineto{\pgfqpoint{5.480236in}{1.820553in}}%
\pgfpathclose%
\pgfusepath{fill}%
\end{pgfscope}%
\begin{pgfscope}%
\pgfpathrectangle{\pgfqpoint{3.776708in}{0.600000in}}{\pgfqpoint{2.573292in}{2.070576in}}%
\pgfusepath{clip}%
\pgfsetbuttcap%
\pgfsetmiterjoin%
\definecolor{currentfill}{rgb}{0.511253,0.510898,0.193296}%
\pgfsetfillcolor{currentfill}%
\pgfsetlinewidth{0.000000pt}%
\definecolor{currentstroke}{rgb}{0.000000,0.000000,0.000000}%
\pgfsetstrokecolor{currentstroke}%
\pgfsetstrokeopacity{0.000000}%
\pgfsetdash{}{0pt}%
\pgfpathmoveto{\pgfqpoint{5.491178in}{1.790230in}}%
\pgfpathlineto{\pgfqpoint{5.499932in}{1.790230in}}%
\pgfpathlineto{\pgfqpoint{5.499932in}{1.858961in}}%
\pgfpathlineto{\pgfqpoint{5.491178in}{1.858961in}}%
\pgfpathlineto{\pgfqpoint{5.491178in}{1.790230in}}%
\pgfpathclose%
\pgfusepath{fill}%
\end{pgfscope}%
\begin{pgfscope}%
\pgfpathrectangle{\pgfqpoint{3.776708in}{0.600000in}}{\pgfqpoint{2.573292in}{2.070576in}}%
\pgfusepath{clip}%
\pgfsetbuttcap%
\pgfsetmiterjoin%
\definecolor{currentfill}{rgb}{0.511253,0.510898,0.193296}%
\pgfsetfillcolor{currentfill}%
\pgfsetlinewidth{0.000000pt}%
\definecolor{currentstroke}{rgb}{0.000000,0.000000,0.000000}%
\pgfsetstrokecolor{currentstroke}%
\pgfsetstrokeopacity{0.000000}%
\pgfsetdash{}{0pt}%
\pgfpathmoveto{\pgfqpoint{5.502120in}{1.759890in}}%
\pgfpathlineto{\pgfqpoint{5.510873in}{1.759890in}}%
\pgfpathlineto{\pgfqpoint{5.510873in}{1.828411in}}%
\pgfpathlineto{\pgfqpoint{5.502120in}{1.828411in}}%
\pgfpathlineto{\pgfqpoint{5.502120in}{1.759890in}}%
\pgfpathclose%
\pgfusepath{fill}%
\end{pgfscope}%
\begin{pgfscope}%
\pgfpathrectangle{\pgfqpoint{3.776708in}{0.600000in}}{\pgfqpoint{2.573292in}{2.070576in}}%
\pgfusepath{clip}%
\pgfsetbuttcap%
\pgfsetmiterjoin%
\definecolor{currentfill}{rgb}{0.511253,0.510898,0.193296}%
\pgfsetfillcolor{currentfill}%
\pgfsetlinewidth{0.000000pt}%
\definecolor{currentstroke}{rgb}{0.000000,0.000000,0.000000}%
\pgfsetstrokecolor{currentstroke}%
\pgfsetstrokeopacity{0.000000}%
\pgfsetdash{}{0pt}%
\pgfpathmoveto{\pgfqpoint{5.513062in}{1.729284in}}%
\pgfpathlineto{\pgfqpoint{5.521815in}{1.729284in}}%
\pgfpathlineto{\pgfqpoint{5.521815in}{1.794597in}}%
\pgfpathlineto{\pgfqpoint{5.513062in}{1.794597in}}%
\pgfpathlineto{\pgfqpoint{5.513062in}{1.729284in}}%
\pgfpathclose%
\pgfusepath{fill}%
\end{pgfscope}%
\begin{pgfscope}%
\pgfpathrectangle{\pgfqpoint{3.776708in}{0.600000in}}{\pgfqpoint{2.573292in}{2.070576in}}%
\pgfusepath{clip}%
\pgfsetbuttcap%
\pgfsetmiterjoin%
\definecolor{currentfill}{rgb}{0.511253,0.510898,0.193296}%
\pgfsetfillcolor{currentfill}%
\pgfsetlinewidth{0.000000pt}%
\definecolor{currentstroke}{rgb}{0.000000,0.000000,0.000000}%
\pgfsetstrokecolor{currentstroke}%
\pgfsetstrokeopacity{0.000000}%
\pgfsetdash{}{0pt}%
\pgfpathmoveto{\pgfqpoint{5.524004in}{1.698552in}}%
\pgfpathlineto{\pgfqpoint{5.532757in}{1.698552in}}%
\pgfpathlineto{\pgfqpoint{5.532757in}{1.756210in}}%
\pgfpathlineto{\pgfqpoint{5.524004in}{1.756210in}}%
\pgfpathlineto{\pgfqpoint{5.524004in}{1.698552in}}%
\pgfpathclose%
\pgfusepath{fill}%
\end{pgfscope}%
\begin{pgfscope}%
\pgfpathrectangle{\pgfqpoint{3.776708in}{0.600000in}}{\pgfqpoint{2.573292in}{2.070576in}}%
\pgfusepath{clip}%
\pgfsetbuttcap%
\pgfsetmiterjoin%
\definecolor{currentfill}{rgb}{0.511253,0.510898,0.193296}%
\pgfsetfillcolor{currentfill}%
\pgfsetlinewidth{0.000000pt}%
\definecolor{currentstroke}{rgb}{0.000000,0.000000,0.000000}%
\pgfsetstrokecolor{currentstroke}%
\pgfsetstrokeopacity{0.000000}%
\pgfsetdash{}{0pt}%
\pgfpathmoveto{\pgfqpoint{5.534945in}{1.672037in}}%
\pgfpathlineto{\pgfqpoint{5.543699in}{1.672037in}}%
\pgfpathlineto{\pgfqpoint{5.543699in}{1.724566in}}%
\pgfpathlineto{\pgfqpoint{5.534945in}{1.724566in}}%
\pgfpathlineto{\pgfqpoint{5.534945in}{1.672037in}}%
\pgfpathclose%
\pgfusepath{fill}%
\end{pgfscope}%
\begin{pgfscope}%
\pgfpathrectangle{\pgfqpoint{3.776708in}{0.600000in}}{\pgfqpoint{2.573292in}{2.070576in}}%
\pgfusepath{clip}%
\pgfsetbuttcap%
\pgfsetmiterjoin%
\definecolor{currentfill}{rgb}{0.511253,0.510898,0.193296}%
\pgfsetfillcolor{currentfill}%
\pgfsetlinewidth{0.000000pt}%
\definecolor{currentstroke}{rgb}{0.000000,0.000000,0.000000}%
\pgfsetstrokecolor{currentstroke}%
\pgfsetstrokeopacity{0.000000}%
\pgfsetdash{}{0pt}%
\pgfpathmoveto{\pgfqpoint{5.545887in}{1.668337in}}%
\pgfpathlineto{\pgfqpoint{5.554641in}{1.668337in}}%
\pgfpathlineto{\pgfqpoint{5.554641in}{1.713949in}}%
\pgfpathlineto{\pgfqpoint{5.545887in}{1.713949in}}%
\pgfpathlineto{\pgfqpoint{5.545887in}{1.668337in}}%
\pgfpathclose%
\pgfusepath{fill}%
\end{pgfscope}%
\begin{pgfscope}%
\pgfpathrectangle{\pgfqpoint{3.776708in}{0.600000in}}{\pgfqpoint{2.573292in}{2.070576in}}%
\pgfusepath{clip}%
\pgfsetbuttcap%
\pgfsetmiterjoin%
\definecolor{currentfill}{rgb}{0.511253,0.510898,0.193296}%
\pgfsetfillcolor{currentfill}%
\pgfsetlinewidth{0.000000pt}%
\definecolor{currentstroke}{rgb}{0.000000,0.000000,0.000000}%
\pgfsetstrokecolor{currentstroke}%
\pgfsetstrokeopacity{0.000000}%
\pgfsetdash{}{0pt}%
\pgfpathmoveto{\pgfqpoint{5.556829in}{1.664539in}}%
\pgfpathlineto{\pgfqpoint{5.565582in}{1.664539in}}%
\pgfpathlineto{\pgfqpoint{5.565582in}{1.707415in}}%
\pgfpathlineto{\pgfqpoint{5.556829in}{1.707415in}}%
\pgfpathlineto{\pgfqpoint{5.556829in}{1.664539in}}%
\pgfpathclose%
\pgfusepath{fill}%
\end{pgfscope}%
\begin{pgfscope}%
\pgfpathrectangle{\pgfqpoint{3.776708in}{0.600000in}}{\pgfqpoint{2.573292in}{2.070576in}}%
\pgfusepath{clip}%
\pgfsetbuttcap%
\pgfsetmiterjoin%
\definecolor{currentfill}{rgb}{0.511253,0.510898,0.193296}%
\pgfsetfillcolor{currentfill}%
\pgfsetlinewidth{0.000000pt}%
\definecolor{currentstroke}{rgb}{0.000000,0.000000,0.000000}%
\pgfsetstrokecolor{currentstroke}%
\pgfsetstrokeopacity{0.000000}%
\pgfsetdash{}{0pt}%
\pgfpathmoveto{\pgfqpoint{5.567771in}{1.661381in}}%
\pgfpathlineto{\pgfqpoint{5.576524in}{1.661381in}}%
\pgfpathlineto{\pgfqpoint{5.576524in}{1.702075in}}%
\pgfpathlineto{\pgfqpoint{5.567771in}{1.702075in}}%
\pgfpathlineto{\pgfqpoint{5.567771in}{1.661381in}}%
\pgfpathclose%
\pgfusepath{fill}%
\end{pgfscope}%
\begin{pgfscope}%
\pgfpathrectangle{\pgfqpoint{3.776708in}{0.600000in}}{\pgfqpoint{2.573292in}{2.070576in}}%
\pgfusepath{clip}%
\pgfsetbuttcap%
\pgfsetmiterjoin%
\definecolor{currentfill}{rgb}{0.511253,0.510898,0.193296}%
\pgfsetfillcolor{currentfill}%
\pgfsetlinewidth{0.000000pt}%
\definecolor{currentstroke}{rgb}{0.000000,0.000000,0.000000}%
\pgfsetstrokecolor{currentstroke}%
\pgfsetstrokeopacity{0.000000}%
\pgfsetdash{}{0pt}%
\pgfpathmoveto{\pgfqpoint{5.578713in}{1.658074in}}%
\pgfpathlineto{\pgfqpoint{5.587466in}{1.658074in}}%
\pgfpathlineto{\pgfqpoint{5.587466in}{1.693748in}}%
\pgfpathlineto{\pgfqpoint{5.578713in}{1.693748in}}%
\pgfpathlineto{\pgfqpoint{5.578713in}{1.658074in}}%
\pgfpathclose%
\pgfusepath{fill}%
\end{pgfscope}%
\begin{pgfscope}%
\pgfpathrectangle{\pgfqpoint{3.776708in}{0.600000in}}{\pgfqpoint{2.573292in}{2.070576in}}%
\pgfusepath{clip}%
\pgfsetbuttcap%
\pgfsetmiterjoin%
\definecolor{currentfill}{rgb}{0.511253,0.510898,0.193296}%
\pgfsetfillcolor{currentfill}%
\pgfsetlinewidth{0.000000pt}%
\definecolor{currentstroke}{rgb}{0.000000,0.000000,0.000000}%
\pgfsetstrokecolor{currentstroke}%
\pgfsetstrokeopacity{0.000000}%
\pgfsetdash{}{0pt}%
\pgfpathmoveto{\pgfqpoint{5.589654in}{1.652962in}}%
\pgfpathlineto{\pgfqpoint{5.598408in}{1.652962in}}%
\pgfpathlineto{\pgfqpoint{5.598408in}{1.686593in}}%
\pgfpathlineto{\pgfqpoint{5.589654in}{1.686593in}}%
\pgfpathlineto{\pgfqpoint{5.589654in}{1.652962in}}%
\pgfpathclose%
\pgfusepath{fill}%
\end{pgfscope}%
\begin{pgfscope}%
\pgfpathrectangle{\pgfqpoint{3.776708in}{0.600000in}}{\pgfqpoint{2.573292in}{2.070576in}}%
\pgfusepath{clip}%
\pgfsetbuttcap%
\pgfsetmiterjoin%
\definecolor{currentfill}{rgb}{0.511253,0.510898,0.193296}%
\pgfsetfillcolor{currentfill}%
\pgfsetlinewidth{0.000000pt}%
\definecolor{currentstroke}{rgb}{0.000000,0.000000,0.000000}%
\pgfsetstrokecolor{currentstroke}%
\pgfsetstrokeopacity{0.000000}%
\pgfsetdash{}{0pt}%
\pgfpathmoveto{\pgfqpoint{5.600596in}{1.647900in}}%
\pgfpathlineto{\pgfqpoint{5.609350in}{1.647900in}}%
\pgfpathlineto{\pgfqpoint{5.609350in}{1.678916in}}%
\pgfpathlineto{\pgfqpoint{5.600596in}{1.678916in}}%
\pgfpathlineto{\pgfqpoint{5.600596in}{1.647900in}}%
\pgfpathclose%
\pgfusepath{fill}%
\end{pgfscope}%
\begin{pgfscope}%
\pgfpathrectangle{\pgfqpoint{3.776708in}{0.600000in}}{\pgfqpoint{2.573292in}{2.070576in}}%
\pgfusepath{clip}%
\pgfsetbuttcap%
\pgfsetmiterjoin%
\definecolor{currentfill}{rgb}{0.511253,0.510898,0.193296}%
\pgfsetfillcolor{currentfill}%
\pgfsetlinewidth{0.000000pt}%
\definecolor{currentstroke}{rgb}{0.000000,0.000000,0.000000}%
\pgfsetstrokecolor{currentstroke}%
\pgfsetstrokeopacity{0.000000}%
\pgfsetdash{}{0pt}%
\pgfpathmoveto{\pgfqpoint{5.611538in}{1.643952in}}%
\pgfpathlineto{\pgfqpoint{5.620291in}{1.643952in}}%
\pgfpathlineto{\pgfqpoint{5.620291in}{1.674279in}}%
\pgfpathlineto{\pgfqpoint{5.611538in}{1.674279in}}%
\pgfpathlineto{\pgfqpoint{5.611538in}{1.643952in}}%
\pgfpathclose%
\pgfusepath{fill}%
\end{pgfscope}%
\begin{pgfscope}%
\pgfpathrectangle{\pgfqpoint{3.776708in}{0.600000in}}{\pgfqpoint{2.573292in}{2.070576in}}%
\pgfusepath{clip}%
\pgfsetbuttcap%
\pgfsetmiterjoin%
\definecolor{currentfill}{rgb}{0.511253,0.510898,0.193296}%
\pgfsetfillcolor{currentfill}%
\pgfsetlinewidth{0.000000pt}%
\definecolor{currentstroke}{rgb}{0.000000,0.000000,0.000000}%
\pgfsetstrokecolor{currentstroke}%
\pgfsetstrokeopacity{0.000000}%
\pgfsetdash{}{0pt}%
\pgfpathmoveto{\pgfqpoint{5.622480in}{1.641091in}}%
\pgfpathlineto{\pgfqpoint{5.631233in}{1.641091in}}%
\pgfpathlineto{\pgfqpoint{5.631233in}{1.671208in}}%
\pgfpathlineto{\pgfqpoint{5.622480in}{1.671208in}}%
\pgfpathlineto{\pgfqpoint{5.622480in}{1.641091in}}%
\pgfpathclose%
\pgfusepath{fill}%
\end{pgfscope}%
\begin{pgfscope}%
\pgfpathrectangle{\pgfqpoint{3.776708in}{0.600000in}}{\pgfqpoint{2.573292in}{2.070576in}}%
\pgfusepath{clip}%
\pgfsetbuttcap%
\pgfsetmiterjoin%
\definecolor{currentfill}{rgb}{0.511253,0.510898,0.193296}%
\pgfsetfillcolor{currentfill}%
\pgfsetlinewidth{0.000000pt}%
\definecolor{currentstroke}{rgb}{0.000000,0.000000,0.000000}%
\pgfsetstrokecolor{currentstroke}%
\pgfsetstrokeopacity{0.000000}%
\pgfsetdash{}{0pt}%
\pgfpathmoveto{\pgfqpoint{5.633422in}{1.639431in}}%
\pgfpathlineto{\pgfqpoint{5.642175in}{1.639431in}}%
\pgfpathlineto{\pgfqpoint{5.642175in}{1.669522in}}%
\pgfpathlineto{\pgfqpoint{5.633422in}{1.669522in}}%
\pgfpathlineto{\pgfqpoint{5.633422in}{1.639431in}}%
\pgfpathclose%
\pgfusepath{fill}%
\end{pgfscope}%
\begin{pgfscope}%
\pgfpathrectangle{\pgfqpoint{3.776708in}{0.600000in}}{\pgfqpoint{2.573292in}{2.070576in}}%
\pgfusepath{clip}%
\pgfsetbuttcap%
\pgfsetmiterjoin%
\definecolor{currentfill}{rgb}{0.511253,0.510898,0.193296}%
\pgfsetfillcolor{currentfill}%
\pgfsetlinewidth{0.000000pt}%
\definecolor{currentstroke}{rgb}{0.000000,0.000000,0.000000}%
\pgfsetstrokecolor{currentstroke}%
\pgfsetstrokeopacity{0.000000}%
\pgfsetdash{}{0pt}%
\pgfpathmoveto{\pgfqpoint{5.644363in}{1.634522in}}%
\pgfpathlineto{\pgfqpoint{5.653117in}{1.634522in}}%
\pgfpathlineto{\pgfqpoint{5.653117in}{1.664850in}}%
\pgfpathlineto{\pgfqpoint{5.644363in}{1.664850in}}%
\pgfpathlineto{\pgfqpoint{5.644363in}{1.634522in}}%
\pgfpathclose%
\pgfusepath{fill}%
\end{pgfscope}%
\begin{pgfscope}%
\pgfpathrectangle{\pgfqpoint{3.776708in}{0.600000in}}{\pgfqpoint{2.573292in}{2.070576in}}%
\pgfusepath{clip}%
\pgfsetbuttcap%
\pgfsetmiterjoin%
\definecolor{currentfill}{rgb}{0.511253,0.510898,0.193296}%
\pgfsetfillcolor{currentfill}%
\pgfsetlinewidth{0.000000pt}%
\definecolor{currentstroke}{rgb}{0.000000,0.000000,0.000000}%
\pgfsetstrokecolor{currentstroke}%
\pgfsetstrokeopacity{0.000000}%
\pgfsetdash{}{0pt}%
\pgfpathmoveto{\pgfqpoint{5.655305in}{1.631068in}}%
\pgfpathlineto{\pgfqpoint{5.664059in}{1.631068in}}%
\pgfpathlineto{\pgfqpoint{5.664059in}{1.666871in}}%
\pgfpathlineto{\pgfqpoint{5.655305in}{1.666871in}}%
\pgfpathlineto{\pgfqpoint{5.655305in}{1.631068in}}%
\pgfpathclose%
\pgfusepath{fill}%
\end{pgfscope}%
\begin{pgfscope}%
\pgfpathrectangle{\pgfqpoint{3.776708in}{0.600000in}}{\pgfqpoint{2.573292in}{2.070576in}}%
\pgfusepath{clip}%
\pgfsetbuttcap%
\pgfsetmiterjoin%
\definecolor{currentfill}{rgb}{0.511253,0.510898,0.193296}%
\pgfsetfillcolor{currentfill}%
\pgfsetlinewidth{0.000000pt}%
\definecolor{currentstroke}{rgb}{0.000000,0.000000,0.000000}%
\pgfsetstrokecolor{currentstroke}%
\pgfsetstrokeopacity{0.000000}%
\pgfsetdash{}{0pt}%
\pgfpathmoveto{\pgfqpoint{5.666247in}{1.628842in}}%
\pgfpathlineto{\pgfqpoint{5.675000in}{1.628842in}}%
\pgfpathlineto{\pgfqpoint{5.675000in}{1.660248in}}%
\pgfpathlineto{\pgfqpoint{5.666247in}{1.660248in}}%
\pgfpathlineto{\pgfqpoint{5.666247in}{1.628842in}}%
\pgfpathclose%
\pgfusepath{fill}%
\end{pgfscope}%
\begin{pgfscope}%
\pgfpathrectangle{\pgfqpoint{3.776708in}{0.600000in}}{\pgfqpoint{2.573292in}{2.070576in}}%
\pgfusepath{clip}%
\pgfsetbuttcap%
\pgfsetmiterjoin%
\definecolor{currentfill}{rgb}{0.511253,0.510898,0.193296}%
\pgfsetfillcolor{currentfill}%
\pgfsetlinewidth{0.000000pt}%
\definecolor{currentstroke}{rgb}{0.000000,0.000000,0.000000}%
\pgfsetstrokecolor{currentstroke}%
\pgfsetstrokeopacity{0.000000}%
\pgfsetdash{}{0pt}%
\pgfpathmoveto{\pgfqpoint{5.677189in}{1.626812in}}%
\pgfpathlineto{\pgfqpoint{5.685942in}{1.626812in}}%
\pgfpathlineto{\pgfqpoint{5.685942in}{1.657128in}}%
\pgfpathlineto{\pgfqpoint{5.677189in}{1.657128in}}%
\pgfpathlineto{\pgfqpoint{5.677189in}{1.626812in}}%
\pgfpathclose%
\pgfusepath{fill}%
\end{pgfscope}%
\begin{pgfscope}%
\pgfpathrectangle{\pgfqpoint{3.776708in}{0.600000in}}{\pgfqpoint{2.573292in}{2.070576in}}%
\pgfusepath{clip}%
\pgfsetbuttcap%
\pgfsetmiterjoin%
\definecolor{currentfill}{rgb}{0.511253,0.510898,0.193296}%
\pgfsetfillcolor{currentfill}%
\pgfsetlinewidth{0.000000pt}%
\definecolor{currentstroke}{rgb}{0.000000,0.000000,0.000000}%
\pgfsetstrokecolor{currentstroke}%
\pgfsetstrokeopacity{0.000000}%
\pgfsetdash{}{0pt}%
\pgfpathmoveto{\pgfqpoint{5.688131in}{1.624217in}}%
\pgfpathlineto{\pgfqpoint{5.696884in}{1.624217in}}%
\pgfpathlineto{\pgfqpoint{5.696884in}{1.656995in}}%
\pgfpathlineto{\pgfqpoint{5.688131in}{1.656995in}}%
\pgfpathlineto{\pgfqpoint{5.688131in}{1.624217in}}%
\pgfpathclose%
\pgfusepath{fill}%
\end{pgfscope}%
\begin{pgfscope}%
\pgfpathrectangle{\pgfqpoint{3.776708in}{0.600000in}}{\pgfqpoint{2.573292in}{2.070576in}}%
\pgfusepath{clip}%
\pgfsetbuttcap%
\pgfsetmiterjoin%
\definecolor{currentfill}{rgb}{0.511253,0.510898,0.193296}%
\pgfsetfillcolor{currentfill}%
\pgfsetlinewidth{0.000000pt}%
\definecolor{currentstroke}{rgb}{0.000000,0.000000,0.000000}%
\pgfsetstrokecolor{currentstroke}%
\pgfsetstrokeopacity{0.000000}%
\pgfsetdash{}{0pt}%
\pgfpathmoveto{\pgfqpoint{5.699072in}{1.619650in}}%
\pgfpathlineto{\pgfqpoint{5.707826in}{1.619650in}}%
\pgfpathlineto{\pgfqpoint{5.707826in}{1.654179in}}%
\pgfpathlineto{\pgfqpoint{5.699072in}{1.654179in}}%
\pgfpathlineto{\pgfqpoint{5.699072in}{1.619650in}}%
\pgfpathclose%
\pgfusepath{fill}%
\end{pgfscope}%
\begin{pgfscope}%
\pgfpathrectangle{\pgfqpoint{3.776708in}{0.600000in}}{\pgfqpoint{2.573292in}{2.070576in}}%
\pgfusepath{clip}%
\pgfsetbuttcap%
\pgfsetmiterjoin%
\definecolor{currentfill}{rgb}{0.511253,0.510898,0.193296}%
\pgfsetfillcolor{currentfill}%
\pgfsetlinewidth{0.000000pt}%
\definecolor{currentstroke}{rgb}{0.000000,0.000000,0.000000}%
\pgfsetstrokecolor{currentstroke}%
\pgfsetstrokeopacity{0.000000}%
\pgfsetdash{}{0pt}%
\pgfpathmoveto{\pgfqpoint{5.710014in}{1.614631in}}%
\pgfpathlineto{\pgfqpoint{5.718768in}{1.614631in}}%
\pgfpathlineto{\pgfqpoint{5.718768in}{1.653626in}}%
\pgfpathlineto{\pgfqpoint{5.710014in}{1.653626in}}%
\pgfpathlineto{\pgfqpoint{5.710014in}{1.614631in}}%
\pgfpathclose%
\pgfusepath{fill}%
\end{pgfscope}%
\begin{pgfscope}%
\pgfpathrectangle{\pgfqpoint{3.776708in}{0.600000in}}{\pgfqpoint{2.573292in}{2.070576in}}%
\pgfusepath{clip}%
\pgfsetbuttcap%
\pgfsetmiterjoin%
\definecolor{currentfill}{rgb}{0.511253,0.510898,0.193296}%
\pgfsetfillcolor{currentfill}%
\pgfsetlinewidth{0.000000pt}%
\definecolor{currentstroke}{rgb}{0.000000,0.000000,0.000000}%
\pgfsetstrokecolor{currentstroke}%
\pgfsetstrokeopacity{0.000000}%
\pgfsetdash{}{0pt}%
\pgfpathmoveto{\pgfqpoint{5.720956in}{1.612010in}}%
\pgfpathlineto{\pgfqpoint{5.729709in}{1.612010in}}%
\pgfpathlineto{\pgfqpoint{5.729709in}{1.651147in}}%
\pgfpathlineto{\pgfqpoint{5.720956in}{1.651147in}}%
\pgfpathlineto{\pgfqpoint{5.720956in}{1.612010in}}%
\pgfpathclose%
\pgfusepath{fill}%
\end{pgfscope}%
\begin{pgfscope}%
\pgfpathrectangle{\pgfqpoint{3.776708in}{0.600000in}}{\pgfqpoint{2.573292in}{2.070576in}}%
\pgfusepath{clip}%
\pgfsetbuttcap%
\pgfsetmiterjoin%
\definecolor{currentfill}{rgb}{0.511253,0.510898,0.193296}%
\pgfsetfillcolor{currentfill}%
\pgfsetlinewidth{0.000000pt}%
\definecolor{currentstroke}{rgb}{0.000000,0.000000,0.000000}%
\pgfsetstrokecolor{currentstroke}%
\pgfsetstrokeopacity{0.000000}%
\pgfsetdash{}{0pt}%
\pgfpathmoveto{\pgfqpoint{5.731898in}{1.609234in}}%
\pgfpathlineto{\pgfqpoint{5.740651in}{1.609234in}}%
\pgfpathlineto{\pgfqpoint{5.740651in}{1.650857in}}%
\pgfpathlineto{\pgfqpoint{5.731898in}{1.650857in}}%
\pgfpathlineto{\pgfqpoint{5.731898in}{1.609234in}}%
\pgfpathclose%
\pgfusepath{fill}%
\end{pgfscope}%
\begin{pgfscope}%
\pgfpathrectangle{\pgfqpoint{3.776708in}{0.600000in}}{\pgfqpoint{2.573292in}{2.070576in}}%
\pgfusepath{clip}%
\pgfsetbuttcap%
\pgfsetmiterjoin%
\definecolor{currentfill}{rgb}{0.511253,0.510898,0.193296}%
\pgfsetfillcolor{currentfill}%
\pgfsetlinewidth{0.000000pt}%
\definecolor{currentstroke}{rgb}{0.000000,0.000000,0.000000}%
\pgfsetstrokecolor{currentstroke}%
\pgfsetstrokeopacity{0.000000}%
\pgfsetdash{}{0pt}%
\pgfpathmoveto{\pgfqpoint{5.742840in}{1.609196in}}%
\pgfpathlineto{\pgfqpoint{5.751593in}{1.609196in}}%
\pgfpathlineto{\pgfqpoint{5.751593in}{1.660875in}}%
\pgfpathlineto{\pgfqpoint{5.742840in}{1.660875in}}%
\pgfpathlineto{\pgfqpoint{5.742840in}{1.609196in}}%
\pgfpathclose%
\pgfusepath{fill}%
\end{pgfscope}%
\begin{pgfscope}%
\pgfpathrectangle{\pgfqpoint{3.776708in}{0.600000in}}{\pgfqpoint{2.573292in}{2.070576in}}%
\pgfusepath{clip}%
\pgfsetbuttcap%
\pgfsetmiterjoin%
\definecolor{currentfill}{rgb}{0.511253,0.510898,0.193296}%
\pgfsetfillcolor{currentfill}%
\pgfsetlinewidth{0.000000pt}%
\definecolor{currentstroke}{rgb}{0.000000,0.000000,0.000000}%
\pgfsetstrokecolor{currentstroke}%
\pgfsetstrokeopacity{0.000000}%
\pgfsetdash{}{0pt}%
\pgfpathmoveto{\pgfqpoint{5.753781in}{1.609196in}}%
\pgfpathlineto{\pgfqpoint{5.762535in}{1.609196in}}%
\pgfpathlineto{\pgfqpoint{5.762535in}{1.672179in}}%
\pgfpathlineto{\pgfqpoint{5.753781in}{1.672179in}}%
\pgfpathlineto{\pgfqpoint{5.753781in}{1.609196in}}%
\pgfpathclose%
\pgfusepath{fill}%
\end{pgfscope}%
\begin{pgfscope}%
\pgfpathrectangle{\pgfqpoint{3.776708in}{0.600000in}}{\pgfqpoint{2.573292in}{2.070576in}}%
\pgfusepath{clip}%
\pgfsetbuttcap%
\pgfsetmiterjoin%
\definecolor{currentfill}{rgb}{0.511253,0.510898,0.193296}%
\pgfsetfillcolor{currentfill}%
\pgfsetlinewidth{0.000000pt}%
\definecolor{currentstroke}{rgb}{0.000000,0.000000,0.000000}%
\pgfsetstrokecolor{currentstroke}%
\pgfsetstrokeopacity{0.000000}%
\pgfsetdash{}{0pt}%
\pgfpathmoveto{\pgfqpoint{5.764723in}{1.609196in}}%
\pgfpathlineto{\pgfqpoint{5.773477in}{1.609196in}}%
\pgfpathlineto{\pgfqpoint{5.773477in}{1.683656in}}%
\pgfpathlineto{\pgfqpoint{5.764723in}{1.683656in}}%
\pgfpathlineto{\pgfqpoint{5.764723in}{1.609196in}}%
\pgfpathclose%
\pgfusepath{fill}%
\end{pgfscope}%
\begin{pgfscope}%
\pgfpathrectangle{\pgfqpoint{3.776708in}{0.600000in}}{\pgfqpoint{2.573292in}{2.070576in}}%
\pgfusepath{clip}%
\pgfsetbuttcap%
\pgfsetmiterjoin%
\definecolor{currentfill}{rgb}{0.511253,0.510898,0.193296}%
\pgfsetfillcolor{currentfill}%
\pgfsetlinewidth{0.000000pt}%
\definecolor{currentstroke}{rgb}{0.000000,0.000000,0.000000}%
\pgfsetstrokecolor{currentstroke}%
\pgfsetstrokeopacity{0.000000}%
\pgfsetdash{}{0pt}%
\pgfpathmoveto{\pgfqpoint{5.775665in}{1.625566in}}%
\pgfpathlineto{\pgfqpoint{5.784418in}{1.625566in}}%
\pgfpathlineto{\pgfqpoint{5.784418in}{1.704292in}}%
\pgfpathlineto{\pgfqpoint{5.775665in}{1.704292in}}%
\pgfpathlineto{\pgfqpoint{5.775665in}{1.625566in}}%
\pgfpathclose%
\pgfusepath{fill}%
\end{pgfscope}%
\begin{pgfscope}%
\pgfpathrectangle{\pgfqpoint{3.776708in}{0.600000in}}{\pgfqpoint{2.573292in}{2.070576in}}%
\pgfusepath{clip}%
\pgfsetbuttcap%
\pgfsetmiterjoin%
\definecolor{currentfill}{rgb}{0.511253,0.510898,0.193296}%
\pgfsetfillcolor{currentfill}%
\pgfsetlinewidth{0.000000pt}%
\definecolor{currentstroke}{rgb}{0.000000,0.000000,0.000000}%
\pgfsetstrokecolor{currentstroke}%
\pgfsetstrokeopacity{0.000000}%
\pgfsetdash{}{0pt}%
\pgfpathmoveto{\pgfqpoint{5.786607in}{1.658450in}}%
\pgfpathlineto{\pgfqpoint{5.795360in}{1.658450in}}%
\pgfpathlineto{\pgfqpoint{5.795360in}{1.740526in}}%
\pgfpathlineto{\pgfqpoint{5.786607in}{1.740526in}}%
\pgfpathlineto{\pgfqpoint{5.786607in}{1.658450in}}%
\pgfpathclose%
\pgfusepath{fill}%
\end{pgfscope}%
\begin{pgfscope}%
\pgfpathrectangle{\pgfqpoint{3.776708in}{0.600000in}}{\pgfqpoint{2.573292in}{2.070576in}}%
\pgfusepath{clip}%
\pgfsetbuttcap%
\pgfsetmiterjoin%
\definecolor{currentfill}{rgb}{0.511253,0.510898,0.193296}%
\pgfsetfillcolor{currentfill}%
\pgfsetlinewidth{0.000000pt}%
\definecolor{currentstroke}{rgb}{0.000000,0.000000,0.000000}%
\pgfsetstrokecolor{currentstroke}%
\pgfsetstrokeopacity{0.000000}%
\pgfsetdash{}{0pt}%
\pgfpathmoveto{\pgfqpoint{5.797549in}{1.691844in}}%
\pgfpathlineto{\pgfqpoint{5.806302in}{1.691844in}}%
\pgfpathlineto{\pgfqpoint{5.806302in}{1.777068in}}%
\pgfpathlineto{\pgfqpoint{5.797549in}{1.777068in}}%
\pgfpathlineto{\pgfqpoint{5.797549in}{1.691844in}}%
\pgfpathclose%
\pgfusepath{fill}%
\end{pgfscope}%
\begin{pgfscope}%
\pgfpathrectangle{\pgfqpoint{3.776708in}{0.600000in}}{\pgfqpoint{2.573292in}{2.070576in}}%
\pgfusepath{clip}%
\pgfsetbuttcap%
\pgfsetmiterjoin%
\definecolor{currentfill}{rgb}{0.511253,0.510898,0.193296}%
\pgfsetfillcolor{currentfill}%
\pgfsetlinewidth{0.000000pt}%
\definecolor{currentstroke}{rgb}{0.000000,0.000000,0.000000}%
\pgfsetstrokecolor{currentstroke}%
\pgfsetstrokeopacity{0.000000}%
\pgfsetdash{}{0pt}%
\pgfpathmoveto{\pgfqpoint{5.808490in}{1.727075in}}%
\pgfpathlineto{\pgfqpoint{5.817244in}{1.727075in}}%
\pgfpathlineto{\pgfqpoint{5.817244in}{1.808837in}}%
\pgfpathlineto{\pgfqpoint{5.808490in}{1.808837in}}%
\pgfpathlineto{\pgfqpoint{5.808490in}{1.727075in}}%
\pgfpathclose%
\pgfusepath{fill}%
\end{pgfscope}%
\begin{pgfscope}%
\pgfpathrectangle{\pgfqpoint{3.776708in}{0.600000in}}{\pgfqpoint{2.573292in}{2.070576in}}%
\pgfusepath{clip}%
\pgfsetbuttcap%
\pgfsetmiterjoin%
\definecolor{currentfill}{rgb}{0.511253,0.510898,0.193296}%
\pgfsetfillcolor{currentfill}%
\pgfsetlinewidth{0.000000pt}%
\definecolor{currentstroke}{rgb}{0.000000,0.000000,0.000000}%
\pgfsetstrokecolor{currentstroke}%
\pgfsetstrokeopacity{0.000000}%
\pgfsetdash{}{0pt}%
\pgfpathmoveto{\pgfqpoint{5.819432in}{1.756177in}}%
\pgfpathlineto{\pgfqpoint{5.828186in}{1.756177in}}%
\pgfpathlineto{\pgfqpoint{5.828186in}{1.836858in}}%
\pgfpathlineto{\pgfqpoint{5.819432in}{1.836858in}}%
\pgfpathlineto{\pgfqpoint{5.819432in}{1.756177in}}%
\pgfpathclose%
\pgfusepath{fill}%
\end{pgfscope}%
\begin{pgfscope}%
\pgfpathrectangle{\pgfqpoint{3.776708in}{0.600000in}}{\pgfqpoint{2.573292in}{2.070576in}}%
\pgfusepath{clip}%
\pgfsetbuttcap%
\pgfsetmiterjoin%
\definecolor{currentfill}{rgb}{0.511253,0.510898,0.193296}%
\pgfsetfillcolor{currentfill}%
\pgfsetlinewidth{0.000000pt}%
\definecolor{currentstroke}{rgb}{0.000000,0.000000,0.000000}%
\pgfsetstrokecolor{currentstroke}%
\pgfsetstrokeopacity{0.000000}%
\pgfsetdash{}{0pt}%
\pgfpathmoveto{\pgfqpoint{5.830374in}{1.787168in}}%
\pgfpathlineto{\pgfqpoint{5.839127in}{1.787168in}}%
\pgfpathlineto{\pgfqpoint{5.839127in}{1.866069in}}%
\pgfpathlineto{\pgfqpoint{5.830374in}{1.866069in}}%
\pgfpathlineto{\pgfqpoint{5.830374in}{1.787168in}}%
\pgfpathclose%
\pgfusepath{fill}%
\end{pgfscope}%
\begin{pgfscope}%
\pgfpathrectangle{\pgfqpoint{3.776708in}{0.600000in}}{\pgfqpoint{2.573292in}{2.070576in}}%
\pgfusepath{clip}%
\pgfsetbuttcap%
\pgfsetmiterjoin%
\definecolor{currentfill}{rgb}{0.511253,0.510898,0.193296}%
\pgfsetfillcolor{currentfill}%
\pgfsetlinewidth{0.000000pt}%
\definecolor{currentstroke}{rgb}{0.000000,0.000000,0.000000}%
\pgfsetstrokecolor{currentstroke}%
\pgfsetstrokeopacity{0.000000}%
\pgfsetdash{}{0pt}%
\pgfpathmoveto{\pgfqpoint{5.841316in}{1.815235in}}%
\pgfpathlineto{\pgfqpoint{5.850069in}{1.815235in}}%
\pgfpathlineto{\pgfqpoint{5.850069in}{1.892429in}}%
\pgfpathlineto{\pgfqpoint{5.841316in}{1.892429in}}%
\pgfpathlineto{\pgfqpoint{5.841316in}{1.815235in}}%
\pgfpathclose%
\pgfusepath{fill}%
\end{pgfscope}%
\begin{pgfscope}%
\pgfpathrectangle{\pgfqpoint{3.776708in}{0.600000in}}{\pgfqpoint{2.573292in}{2.070576in}}%
\pgfusepath{clip}%
\pgfsetbuttcap%
\pgfsetmiterjoin%
\definecolor{currentfill}{rgb}{0.511253,0.510898,0.193296}%
\pgfsetfillcolor{currentfill}%
\pgfsetlinewidth{0.000000pt}%
\definecolor{currentstroke}{rgb}{0.000000,0.000000,0.000000}%
\pgfsetstrokecolor{currentstroke}%
\pgfsetstrokeopacity{0.000000}%
\pgfsetdash{}{0pt}%
\pgfpathmoveto{\pgfqpoint{5.852258in}{1.841997in}}%
\pgfpathlineto{\pgfqpoint{5.861011in}{1.841997in}}%
\pgfpathlineto{\pgfqpoint{5.861011in}{1.910056in}}%
\pgfpathlineto{\pgfqpoint{5.852258in}{1.910056in}}%
\pgfpathlineto{\pgfqpoint{5.852258in}{1.841997in}}%
\pgfpathclose%
\pgfusepath{fill}%
\end{pgfscope}%
\begin{pgfscope}%
\pgfpathrectangle{\pgfqpoint{3.776708in}{0.600000in}}{\pgfqpoint{2.573292in}{2.070576in}}%
\pgfusepath{clip}%
\pgfsetbuttcap%
\pgfsetmiterjoin%
\definecolor{currentfill}{rgb}{0.511253,0.510898,0.193296}%
\pgfsetfillcolor{currentfill}%
\pgfsetlinewidth{0.000000pt}%
\definecolor{currentstroke}{rgb}{0.000000,0.000000,0.000000}%
\pgfsetstrokecolor{currentstroke}%
\pgfsetstrokeopacity{0.000000}%
\pgfsetdash{}{0pt}%
\pgfpathmoveto{\pgfqpoint{5.863199in}{1.863774in}}%
\pgfpathlineto{\pgfqpoint{5.871953in}{1.863774in}}%
\pgfpathlineto{\pgfqpoint{5.871953in}{1.930085in}}%
\pgfpathlineto{\pgfqpoint{5.863199in}{1.930085in}}%
\pgfpathlineto{\pgfqpoint{5.863199in}{1.863774in}}%
\pgfpathclose%
\pgfusepath{fill}%
\end{pgfscope}%
\begin{pgfscope}%
\pgfpathrectangle{\pgfqpoint{3.776708in}{0.600000in}}{\pgfqpoint{2.573292in}{2.070576in}}%
\pgfusepath{clip}%
\pgfsetbuttcap%
\pgfsetmiterjoin%
\definecolor{currentfill}{rgb}{0.511253,0.510898,0.193296}%
\pgfsetfillcolor{currentfill}%
\pgfsetlinewidth{0.000000pt}%
\definecolor{currentstroke}{rgb}{0.000000,0.000000,0.000000}%
\pgfsetstrokecolor{currentstroke}%
\pgfsetstrokeopacity{0.000000}%
\pgfsetdash{}{0pt}%
\pgfpathmoveto{\pgfqpoint{5.874141in}{1.877332in}}%
\pgfpathlineto{\pgfqpoint{5.882895in}{1.877332in}}%
\pgfpathlineto{\pgfqpoint{5.882895in}{1.947692in}}%
\pgfpathlineto{\pgfqpoint{5.874141in}{1.947692in}}%
\pgfpathlineto{\pgfqpoint{5.874141in}{1.877332in}}%
\pgfpathclose%
\pgfusepath{fill}%
\end{pgfscope}%
\begin{pgfscope}%
\pgfpathrectangle{\pgfqpoint{3.776708in}{0.600000in}}{\pgfqpoint{2.573292in}{2.070576in}}%
\pgfusepath{clip}%
\pgfsetbuttcap%
\pgfsetmiterjoin%
\definecolor{currentfill}{rgb}{0.511253,0.510898,0.193296}%
\pgfsetfillcolor{currentfill}%
\pgfsetlinewidth{0.000000pt}%
\definecolor{currentstroke}{rgb}{0.000000,0.000000,0.000000}%
\pgfsetstrokecolor{currentstroke}%
\pgfsetstrokeopacity{0.000000}%
\pgfsetdash{}{0pt}%
\pgfpathmoveto{\pgfqpoint{5.885083in}{1.896193in}}%
\pgfpathlineto{\pgfqpoint{5.893836in}{1.896193in}}%
\pgfpathlineto{\pgfqpoint{5.893836in}{1.959122in}}%
\pgfpathlineto{\pgfqpoint{5.885083in}{1.959122in}}%
\pgfpathlineto{\pgfqpoint{5.885083in}{1.896193in}}%
\pgfpathclose%
\pgfusepath{fill}%
\end{pgfscope}%
\begin{pgfscope}%
\pgfpathrectangle{\pgfqpoint{3.776708in}{0.600000in}}{\pgfqpoint{2.573292in}{2.070576in}}%
\pgfusepath{clip}%
\pgfsetbuttcap%
\pgfsetmiterjoin%
\definecolor{currentfill}{rgb}{0.511253,0.510898,0.193296}%
\pgfsetfillcolor{currentfill}%
\pgfsetlinewidth{0.000000pt}%
\definecolor{currentstroke}{rgb}{0.000000,0.000000,0.000000}%
\pgfsetstrokecolor{currentstroke}%
\pgfsetstrokeopacity{0.000000}%
\pgfsetdash{}{0pt}%
\pgfpathmoveto{\pgfqpoint{5.896025in}{1.907418in}}%
\pgfpathlineto{\pgfqpoint{5.904778in}{1.907418in}}%
\pgfpathlineto{\pgfqpoint{5.904778in}{1.968661in}}%
\pgfpathlineto{\pgfqpoint{5.896025in}{1.968661in}}%
\pgfpathlineto{\pgfqpoint{5.896025in}{1.907418in}}%
\pgfpathclose%
\pgfusepath{fill}%
\end{pgfscope}%
\begin{pgfscope}%
\pgfpathrectangle{\pgfqpoint{3.776708in}{0.600000in}}{\pgfqpoint{2.573292in}{2.070576in}}%
\pgfusepath{clip}%
\pgfsetbuttcap%
\pgfsetmiterjoin%
\definecolor{currentfill}{rgb}{0.511253,0.510898,0.193296}%
\pgfsetfillcolor{currentfill}%
\pgfsetlinewidth{0.000000pt}%
\definecolor{currentstroke}{rgb}{0.000000,0.000000,0.000000}%
\pgfsetstrokecolor{currentstroke}%
\pgfsetstrokeopacity{0.000000}%
\pgfsetdash{}{0pt}%
\pgfpathmoveto{\pgfqpoint{5.906967in}{1.918280in}}%
\pgfpathlineto{\pgfqpoint{5.915720in}{1.918280in}}%
\pgfpathlineto{\pgfqpoint{5.915720in}{1.973143in}}%
\pgfpathlineto{\pgfqpoint{5.906967in}{1.973143in}}%
\pgfpathlineto{\pgfqpoint{5.906967in}{1.918280in}}%
\pgfpathclose%
\pgfusepath{fill}%
\end{pgfscope}%
\begin{pgfscope}%
\pgfpathrectangle{\pgfqpoint{3.776708in}{0.600000in}}{\pgfqpoint{2.573292in}{2.070576in}}%
\pgfusepath{clip}%
\pgfsetbuttcap%
\pgfsetmiterjoin%
\definecolor{currentfill}{rgb}{0.511253,0.510898,0.193296}%
\pgfsetfillcolor{currentfill}%
\pgfsetlinewidth{0.000000pt}%
\definecolor{currentstroke}{rgb}{0.000000,0.000000,0.000000}%
\pgfsetstrokecolor{currentstroke}%
\pgfsetstrokeopacity{0.000000}%
\pgfsetdash{}{0pt}%
\pgfpathmoveto{\pgfqpoint{5.917908in}{1.925531in}}%
\pgfpathlineto{\pgfqpoint{5.926662in}{1.925531in}}%
\pgfpathlineto{\pgfqpoint{5.926662in}{1.976428in}}%
\pgfpathlineto{\pgfqpoint{5.917908in}{1.976428in}}%
\pgfpathlineto{\pgfqpoint{5.917908in}{1.925531in}}%
\pgfpathclose%
\pgfusepath{fill}%
\end{pgfscope}%
\begin{pgfscope}%
\pgfpathrectangle{\pgfqpoint{3.776708in}{0.600000in}}{\pgfqpoint{2.573292in}{2.070576in}}%
\pgfusepath{clip}%
\pgfsetbuttcap%
\pgfsetmiterjoin%
\definecolor{currentfill}{rgb}{0.511253,0.510898,0.193296}%
\pgfsetfillcolor{currentfill}%
\pgfsetlinewidth{0.000000pt}%
\definecolor{currentstroke}{rgb}{0.000000,0.000000,0.000000}%
\pgfsetstrokecolor{currentstroke}%
\pgfsetstrokeopacity{0.000000}%
\pgfsetdash{}{0pt}%
\pgfpathmoveto{\pgfqpoint{5.928850in}{1.928597in}}%
\pgfpathlineto{\pgfqpoint{5.937604in}{1.928597in}}%
\pgfpathlineto{\pgfqpoint{5.937604in}{1.980451in}}%
\pgfpathlineto{\pgfqpoint{5.928850in}{1.980451in}}%
\pgfpathlineto{\pgfqpoint{5.928850in}{1.928597in}}%
\pgfpathclose%
\pgfusepath{fill}%
\end{pgfscope}%
\begin{pgfscope}%
\pgfpathrectangle{\pgfqpoint{3.776708in}{0.600000in}}{\pgfqpoint{2.573292in}{2.070576in}}%
\pgfusepath{clip}%
\pgfsetbuttcap%
\pgfsetmiterjoin%
\definecolor{currentfill}{rgb}{0.511253,0.510898,0.193296}%
\pgfsetfillcolor{currentfill}%
\pgfsetlinewidth{0.000000pt}%
\definecolor{currentstroke}{rgb}{0.000000,0.000000,0.000000}%
\pgfsetstrokecolor{currentstroke}%
\pgfsetstrokeopacity{0.000000}%
\pgfsetdash{}{0pt}%
\pgfpathmoveto{\pgfqpoint{5.939792in}{1.929353in}}%
\pgfpathlineto{\pgfqpoint{5.948545in}{1.929353in}}%
\pgfpathlineto{\pgfqpoint{5.948545in}{1.980467in}}%
\pgfpathlineto{\pgfqpoint{5.939792in}{1.980467in}}%
\pgfpathlineto{\pgfqpoint{5.939792in}{1.929353in}}%
\pgfpathclose%
\pgfusepath{fill}%
\end{pgfscope}%
\begin{pgfscope}%
\pgfpathrectangle{\pgfqpoint{3.776708in}{0.600000in}}{\pgfqpoint{2.573292in}{2.070576in}}%
\pgfusepath{clip}%
\pgfsetbuttcap%
\pgfsetmiterjoin%
\definecolor{currentfill}{rgb}{0.511253,0.510898,0.193296}%
\pgfsetfillcolor{currentfill}%
\pgfsetlinewidth{0.000000pt}%
\definecolor{currentstroke}{rgb}{0.000000,0.000000,0.000000}%
\pgfsetstrokecolor{currentstroke}%
\pgfsetstrokeopacity{0.000000}%
\pgfsetdash{}{0pt}%
\pgfpathmoveto{\pgfqpoint{5.950734in}{1.934272in}}%
\pgfpathlineto{\pgfqpoint{5.959487in}{1.934272in}}%
\pgfpathlineto{\pgfqpoint{5.959487in}{1.979121in}}%
\pgfpathlineto{\pgfqpoint{5.950734in}{1.979121in}}%
\pgfpathlineto{\pgfqpoint{5.950734in}{1.934272in}}%
\pgfpathclose%
\pgfusepath{fill}%
\end{pgfscope}%
\begin{pgfscope}%
\pgfpathrectangle{\pgfqpoint{3.776708in}{0.600000in}}{\pgfqpoint{2.573292in}{2.070576in}}%
\pgfusepath{clip}%
\pgfsetbuttcap%
\pgfsetmiterjoin%
\definecolor{currentfill}{rgb}{0.511253,0.510898,0.193296}%
\pgfsetfillcolor{currentfill}%
\pgfsetlinewidth{0.000000pt}%
\definecolor{currentstroke}{rgb}{0.000000,0.000000,0.000000}%
\pgfsetstrokecolor{currentstroke}%
\pgfsetstrokeopacity{0.000000}%
\pgfsetdash{}{0pt}%
\pgfpathmoveto{\pgfqpoint{5.961676in}{1.940037in}}%
\pgfpathlineto{\pgfqpoint{5.970429in}{1.940037in}}%
\pgfpathlineto{\pgfqpoint{5.970429in}{1.980276in}}%
\pgfpathlineto{\pgfqpoint{5.961676in}{1.980276in}}%
\pgfpathlineto{\pgfqpoint{5.961676in}{1.940037in}}%
\pgfpathclose%
\pgfusepath{fill}%
\end{pgfscope}%
\begin{pgfscope}%
\pgfpathrectangle{\pgfqpoint{3.776708in}{0.600000in}}{\pgfqpoint{2.573292in}{2.070576in}}%
\pgfusepath{clip}%
\pgfsetbuttcap%
\pgfsetmiterjoin%
\definecolor{currentfill}{rgb}{0.511253,0.510898,0.193296}%
\pgfsetfillcolor{currentfill}%
\pgfsetlinewidth{0.000000pt}%
\definecolor{currentstroke}{rgb}{0.000000,0.000000,0.000000}%
\pgfsetstrokecolor{currentstroke}%
\pgfsetstrokeopacity{0.000000}%
\pgfsetdash{}{0pt}%
\pgfpathmoveto{\pgfqpoint{5.972617in}{1.943025in}}%
\pgfpathlineto{\pgfqpoint{5.981371in}{1.943025in}}%
\pgfpathlineto{\pgfqpoint{5.981371in}{1.983382in}}%
\pgfpathlineto{\pgfqpoint{5.972617in}{1.983382in}}%
\pgfpathlineto{\pgfqpoint{5.972617in}{1.943025in}}%
\pgfpathclose%
\pgfusepath{fill}%
\end{pgfscope}%
\begin{pgfscope}%
\pgfpathrectangle{\pgfqpoint{3.776708in}{0.600000in}}{\pgfqpoint{2.573292in}{2.070576in}}%
\pgfusepath{clip}%
\pgfsetbuttcap%
\pgfsetmiterjoin%
\definecolor{currentfill}{rgb}{0.511253,0.510898,0.193296}%
\pgfsetfillcolor{currentfill}%
\pgfsetlinewidth{0.000000pt}%
\definecolor{currentstroke}{rgb}{0.000000,0.000000,0.000000}%
\pgfsetstrokecolor{currentstroke}%
\pgfsetstrokeopacity{0.000000}%
\pgfsetdash{}{0pt}%
\pgfpathmoveto{\pgfqpoint{5.983559in}{1.947227in}}%
\pgfpathlineto{\pgfqpoint{5.992313in}{1.947227in}}%
\pgfpathlineto{\pgfqpoint{5.992313in}{1.981129in}}%
\pgfpathlineto{\pgfqpoint{5.983559in}{1.981129in}}%
\pgfpathlineto{\pgfqpoint{5.983559in}{1.947227in}}%
\pgfpathclose%
\pgfusepath{fill}%
\end{pgfscope}%
\begin{pgfscope}%
\pgfpathrectangle{\pgfqpoint{3.776708in}{0.600000in}}{\pgfqpoint{2.573292in}{2.070576in}}%
\pgfusepath{clip}%
\pgfsetbuttcap%
\pgfsetmiterjoin%
\definecolor{currentfill}{rgb}{0.511253,0.510898,0.193296}%
\pgfsetfillcolor{currentfill}%
\pgfsetlinewidth{0.000000pt}%
\definecolor{currentstroke}{rgb}{0.000000,0.000000,0.000000}%
\pgfsetstrokecolor{currentstroke}%
\pgfsetstrokeopacity{0.000000}%
\pgfsetdash{}{0pt}%
\pgfpathmoveto{\pgfqpoint{5.994501in}{1.948863in}}%
\pgfpathlineto{\pgfqpoint{6.003254in}{1.948863in}}%
\pgfpathlineto{\pgfqpoint{6.003254in}{1.978236in}}%
\pgfpathlineto{\pgfqpoint{5.994501in}{1.978236in}}%
\pgfpathlineto{\pgfqpoint{5.994501in}{1.948863in}}%
\pgfpathclose%
\pgfusepath{fill}%
\end{pgfscope}%
\begin{pgfscope}%
\pgfpathrectangle{\pgfqpoint{3.776708in}{0.600000in}}{\pgfqpoint{2.573292in}{2.070576in}}%
\pgfusepath{clip}%
\pgfsetbuttcap%
\pgfsetmiterjoin%
\definecolor{currentfill}{rgb}{0.511253,0.510898,0.193296}%
\pgfsetfillcolor{currentfill}%
\pgfsetlinewidth{0.000000pt}%
\definecolor{currentstroke}{rgb}{0.000000,0.000000,0.000000}%
\pgfsetstrokecolor{currentstroke}%
\pgfsetstrokeopacity{0.000000}%
\pgfsetdash{}{0pt}%
\pgfpathmoveto{\pgfqpoint{6.005443in}{1.946587in}}%
\pgfpathlineto{\pgfqpoint{6.014196in}{1.946587in}}%
\pgfpathlineto{\pgfqpoint{6.014196in}{1.977570in}}%
\pgfpathlineto{\pgfqpoint{6.005443in}{1.977570in}}%
\pgfpathlineto{\pgfqpoint{6.005443in}{1.946587in}}%
\pgfpathclose%
\pgfusepath{fill}%
\end{pgfscope}%
\begin{pgfscope}%
\pgfpathrectangle{\pgfqpoint{3.776708in}{0.600000in}}{\pgfqpoint{2.573292in}{2.070576in}}%
\pgfusepath{clip}%
\pgfsetbuttcap%
\pgfsetmiterjoin%
\definecolor{currentfill}{rgb}{0.511253,0.510898,0.193296}%
\pgfsetfillcolor{currentfill}%
\pgfsetlinewidth{0.000000pt}%
\definecolor{currentstroke}{rgb}{0.000000,0.000000,0.000000}%
\pgfsetstrokecolor{currentstroke}%
\pgfsetstrokeopacity{0.000000}%
\pgfsetdash{}{0pt}%
\pgfpathmoveto{\pgfqpoint{6.016385in}{1.943097in}}%
\pgfpathlineto{\pgfqpoint{6.025138in}{1.943097in}}%
\pgfpathlineto{\pgfqpoint{6.025138in}{1.975209in}}%
\pgfpathlineto{\pgfqpoint{6.016385in}{1.975209in}}%
\pgfpathlineto{\pgfqpoint{6.016385in}{1.943097in}}%
\pgfpathclose%
\pgfusepath{fill}%
\end{pgfscope}%
\begin{pgfscope}%
\pgfpathrectangle{\pgfqpoint{3.776708in}{0.600000in}}{\pgfqpoint{2.573292in}{2.070576in}}%
\pgfusepath{clip}%
\pgfsetbuttcap%
\pgfsetmiterjoin%
\definecolor{currentfill}{rgb}{0.511253,0.510898,0.193296}%
\pgfsetfillcolor{currentfill}%
\pgfsetlinewidth{0.000000pt}%
\definecolor{currentstroke}{rgb}{0.000000,0.000000,0.000000}%
\pgfsetstrokecolor{currentstroke}%
\pgfsetstrokeopacity{0.000000}%
\pgfsetdash{}{0pt}%
\pgfpathmoveto{\pgfqpoint{6.027326in}{1.949672in}}%
\pgfpathlineto{\pgfqpoint{6.036080in}{1.949672in}}%
\pgfpathlineto{\pgfqpoint{6.036080in}{1.972868in}}%
\pgfpathlineto{\pgfqpoint{6.027326in}{1.972868in}}%
\pgfpathlineto{\pgfqpoint{6.027326in}{1.949672in}}%
\pgfpathclose%
\pgfusepath{fill}%
\end{pgfscope}%
\begin{pgfscope}%
\pgfpathrectangle{\pgfqpoint{3.776708in}{0.600000in}}{\pgfqpoint{2.573292in}{2.070576in}}%
\pgfusepath{clip}%
\pgfsetbuttcap%
\pgfsetmiterjoin%
\definecolor{currentfill}{rgb}{0.511253,0.510898,0.193296}%
\pgfsetfillcolor{currentfill}%
\pgfsetlinewidth{0.000000pt}%
\definecolor{currentstroke}{rgb}{0.000000,0.000000,0.000000}%
\pgfsetstrokecolor{currentstroke}%
\pgfsetstrokeopacity{0.000000}%
\pgfsetdash{}{0pt}%
\pgfpathmoveto{\pgfqpoint{6.038268in}{1.951262in}}%
\pgfpathlineto{\pgfqpoint{6.047022in}{1.951262in}}%
\pgfpathlineto{\pgfqpoint{6.047022in}{1.971539in}}%
\pgfpathlineto{\pgfqpoint{6.038268in}{1.971539in}}%
\pgfpathlineto{\pgfqpoint{6.038268in}{1.951262in}}%
\pgfpathclose%
\pgfusepath{fill}%
\end{pgfscope}%
\begin{pgfscope}%
\pgfpathrectangle{\pgfqpoint{3.776708in}{0.600000in}}{\pgfqpoint{2.573292in}{2.070576in}}%
\pgfusepath{clip}%
\pgfsetbuttcap%
\pgfsetmiterjoin%
\definecolor{currentfill}{rgb}{0.511253,0.510898,0.193296}%
\pgfsetfillcolor{currentfill}%
\pgfsetlinewidth{0.000000pt}%
\definecolor{currentstroke}{rgb}{0.000000,0.000000,0.000000}%
\pgfsetstrokecolor{currentstroke}%
\pgfsetstrokeopacity{0.000000}%
\pgfsetdash{}{0pt}%
\pgfpathmoveto{\pgfqpoint{6.049210in}{1.953463in}}%
\pgfpathlineto{\pgfqpoint{6.057963in}{1.953463in}}%
\pgfpathlineto{\pgfqpoint{6.057963in}{1.979791in}}%
\pgfpathlineto{\pgfqpoint{6.049210in}{1.979791in}}%
\pgfpathlineto{\pgfqpoint{6.049210in}{1.953463in}}%
\pgfpathclose%
\pgfusepath{fill}%
\end{pgfscope}%
\begin{pgfscope}%
\pgfpathrectangle{\pgfqpoint{3.776708in}{0.600000in}}{\pgfqpoint{2.573292in}{2.070576in}}%
\pgfusepath{clip}%
\pgfsetbuttcap%
\pgfsetmiterjoin%
\definecolor{currentfill}{rgb}{0.511253,0.510898,0.193296}%
\pgfsetfillcolor{currentfill}%
\pgfsetlinewidth{0.000000pt}%
\definecolor{currentstroke}{rgb}{0.000000,0.000000,0.000000}%
\pgfsetstrokecolor{currentstroke}%
\pgfsetstrokeopacity{0.000000}%
\pgfsetdash{}{0pt}%
\pgfpathmoveto{\pgfqpoint{6.060152in}{1.957345in}}%
\pgfpathlineto{\pgfqpoint{6.068905in}{1.957345in}}%
\pgfpathlineto{\pgfqpoint{6.068905in}{1.987154in}}%
\pgfpathlineto{\pgfqpoint{6.060152in}{1.987154in}}%
\pgfpathlineto{\pgfqpoint{6.060152in}{1.957345in}}%
\pgfpathclose%
\pgfusepath{fill}%
\end{pgfscope}%
\begin{pgfscope}%
\pgfpathrectangle{\pgfqpoint{3.776708in}{0.600000in}}{\pgfqpoint{2.573292in}{2.070576in}}%
\pgfusepath{clip}%
\pgfsetbuttcap%
\pgfsetmiterjoin%
\definecolor{currentfill}{rgb}{0.511253,0.510898,0.193296}%
\pgfsetfillcolor{currentfill}%
\pgfsetlinewidth{0.000000pt}%
\definecolor{currentstroke}{rgb}{0.000000,0.000000,0.000000}%
\pgfsetstrokecolor{currentstroke}%
\pgfsetstrokeopacity{0.000000}%
\pgfsetdash{}{0pt}%
\pgfpathmoveto{\pgfqpoint{6.071094in}{1.972364in}}%
\pgfpathlineto{\pgfqpoint{6.079847in}{1.972364in}}%
\pgfpathlineto{\pgfqpoint{6.079847in}{1.992652in}}%
\pgfpathlineto{\pgfqpoint{6.071094in}{1.992652in}}%
\pgfpathlineto{\pgfqpoint{6.071094in}{1.972364in}}%
\pgfpathclose%
\pgfusepath{fill}%
\end{pgfscope}%
\begin{pgfscope}%
\pgfpathrectangle{\pgfqpoint{3.776708in}{0.600000in}}{\pgfqpoint{2.573292in}{2.070576in}}%
\pgfusepath{clip}%
\pgfsetbuttcap%
\pgfsetmiterjoin%
\definecolor{currentfill}{rgb}{0.511253,0.510898,0.193296}%
\pgfsetfillcolor{currentfill}%
\pgfsetlinewidth{0.000000pt}%
\definecolor{currentstroke}{rgb}{0.000000,0.000000,0.000000}%
\pgfsetstrokecolor{currentstroke}%
\pgfsetstrokeopacity{0.000000}%
\pgfsetdash{}{0pt}%
\pgfpathmoveto{\pgfqpoint{6.082035in}{1.980842in}}%
\pgfpathlineto{\pgfqpoint{6.090789in}{1.980842in}}%
\pgfpathlineto{\pgfqpoint{6.090789in}{1.998369in}}%
\pgfpathlineto{\pgfqpoint{6.082035in}{1.998369in}}%
\pgfpathlineto{\pgfqpoint{6.082035in}{1.980842in}}%
\pgfpathclose%
\pgfusepath{fill}%
\end{pgfscope}%
\begin{pgfscope}%
\pgfpathrectangle{\pgfqpoint{3.776708in}{0.600000in}}{\pgfqpoint{2.573292in}{2.070576in}}%
\pgfusepath{clip}%
\pgfsetbuttcap%
\pgfsetmiterjoin%
\definecolor{currentfill}{rgb}{0.511253,0.510898,0.193296}%
\pgfsetfillcolor{currentfill}%
\pgfsetlinewidth{0.000000pt}%
\definecolor{currentstroke}{rgb}{0.000000,0.000000,0.000000}%
\pgfsetstrokecolor{currentstroke}%
\pgfsetstrokeopacity{0.000000}%
\pgfsetdash{}{0pt}%
\pgfpathmoveto{\pgfqpoint{6.092977in}{1.993283in}}%
\pgfpathlineto{\pgfqpoint{6.101731in}{1.993283in}}%
\pgfpathlineto{\pgfqpoint{6.101731in}{2.005265in}}%
\pgfpathlineto{\pgfqpoint{6.092977in}{2.005265in}}%
\pgfpathlineto{\pgfqpoint{6.092977in}{1.993283in}}%
\pgfpathclose%
\pgfusepath{fill}%
\end{pgfscope}%
\begin{pgfscope}%
\pgfpathrectangle{\pgfqpoint{3.776708in}{0.600000in}}{\pgfqpoint{2.573292in}{2.070576in}}%
\pgfusepath{clip}%
\pgfsetbuttcap%
\pgfsetmiterjoin%
\definecolor{currentfill}{rgb}{0.511253,0.510898,0.193296}%
\pgfsetfillcolor{currentfill}%
\pgfsetlinewidth{0.000000pt}%
\definecolor{currentstroke}{rgb}{0.000000,0.000000,0.000000}%
\pgfsetstrokecolor{currentstroke}%
\pgfsetstrokeopacity{0.000000}%
\pgfsetdash{}{0pt}%
\pgfpathmoveto{\pgfqpoint{6.103919in}{2.002785in}}%
\pgfpathlineto{\pgfqpoint{6.112672in}{2.002785in}}%
\pgfpathlineto{\pgfqpoint{6.112672in}{2.007011in}}%
\pgfpathlineto{\pgfqpoint{6.103919in}{2.007011in}}%
\pgfpathlineto{\pgfqpoint{6.103919in}{2.002785in}}%
\pgfpathclose%
\pgfusepath{fill}%
\end{pgfscope}%
\begin{pgfscope}%
\pgfpathrectangle{\pgfqpoint{3.776708in}{0.600000in}}{\pgfqpoint{2.573292in}{2.070576in}}%
\pgfusepath{clip}%
\pgfsetbuttcap%
\pgfsetmiterjoin%
\definecolor{currentfill}{rgb}{0.511253,0.510898,0.193296}%
\pgfsetfillcolor{currentfill}%
\pgfsetlinewidth{0.000000pt}%
\definecolor{currentstroke}{rgb}{0.000000,0.000000,0.000000}%
\pgfsetstrokecolor{currentstroke}%
\pgfsetstrokeopacity{0.000000}%
\pgfsetdash{}{0pt}%
\pgfpathmoveto{\pgfqpoint{6.114861in}{2.007522in}}%
\pgfpathlineto{\pgfqpoint{6.123614in}{2.007522in}}%
\pgfpathlineto{\pgfqpoint{6.123614in}{2.009813in}}%
\pgfpathlineto{\pgfqpoint{6.114861in}{2.009813in}}%
\pgfpathlineto{\pgfqpoint{6.114861in}{2.007522in}}%
\pgfpathclose%
\pgfusepath{fill}%
\end{pgfscope}%
\begin{pgfscope}%
\pgfpathrectangle{\pgfqpoint{3.776708in}{0.600000in}}{\pgfqpoint{2.573292in}{2.070576in}}%
\pgfusepath{clip}%
\pgfsetbuttcap%
\pgfsetmiterjoin%
\definecolor{currentfill}{rgb}{0.511253,0.510898,0.193296}%
\pgfsetfillcolor{currentfill}%
\pgfsetlinewidth{0.000000pt}%
\definecolor{currentstroke}{rgb}{0.000000,0.000000,0.000000}%
\pgfsetstrokecolor{currentstroke}%
\pgfsetstrokeopacity{0.000000}%
\pgfsetdash{}{0pt}%
\pgfpathmoveto{\pgfqpoint{6.125803in}{1.142121in}}%
\pgfpathlineto{\pgfqpoint{6.134556in}{1.142121in}}%
\pgfpathlineto{\pgfqpoint{6.134556in}{1.139304in}}%
\pgfpathlineto{\pgfqpoint{6.125803in}{1.139304in}}%
\pgfpathlineto{\pgfqpoint{6.125803in}{1.142121in}}%
\pgfpathclose%
\pgfusepath{fill}%
\end{pgfscope}%
\begin{pgfscope}%
\pgfpathrectangle{\pgfqpoint{3.776708in}{0.600000in}}{\pgfqpoint{2.573292in}{2.070576in}}%
\pgfusepath{clip}%
\pgfsetbuttcap%
\pgfsetmiterjoin%
\definecolor{currentfill}{rgb}{0.511253,0.510898,0.193296}%
\pgfsetfillcolor{currentfill}%
\pgfsetlinewidth{0.000000pt}%
\definecolor{currentstroke}{rgb}{0.000000,0.000000,0.000000}%
\pgfsetstrokecolor{currentstroke}%
\pgfsetstrokeopacity{0.000000}%
\pgfsetdash{}{0pt}%
\pgfpathmoveto{\pgfqpoint{6.136744in}{1.138503in}}%
\pgfpathlineto{\pgfqpoint{6.145498in}{1.138503in}}%
\pgfpathlineto{\pgfqpoint{6.145498in}{1.126424in}}%
\pgfpathlineto{\pgfqpoint{6.136744in}{1.126424in}}%
\pgfpathlineto{\pgfqpoint{6.136744in}{1.138503in}}%
\pgfpathclose%
\pgfusepath{fill}%
\end{pgfscope}%
\begin{pgfscope}%
\pgfpathrectangle{\pgfqpoint{3.776708in}{0.600000in}}{\pgfqpoint{2.573292in}{2.070576in}}%
\pgfusepath{clip}%
\pgfsetbuttcap%
\pgfsetmiterjoin%
\definecolor{currentfill}{rgb}{0.511253,0.510898,0.193296}%
\pgfsetfillcolor{currentfill}%
\pgfsetlinewidth{0.000000pt}%
\definecolor{currentstroke}{rgb}{0.000000,0.000000,0.000000}%
\pgfsetstrokecolor{currentstroke}%
\pgfsetstrokeopacity{0.000000}%
\pgfsetdash{}{0pt}%
\pgfpathmoveto{\pgfqpoint{6.147686in}{1.134733in}}%
\pgfpathlineto{\pgfqpoint{6.156440in}{1.134733in}}%
\pgfpathlineto{\pgfqpoint{6.156440in}{1.119378in}}%
\pgfpathlineto{\pgfqpoint{6.147686in}{1.119378in}}%
\pgfpathlineto{\pgfqpoint{6.147686in}{1.134733in}}%
\pgfpathclose%
\pgfusepath{fill}%
\end{pgfscope}%
\begin{pgfscope}%
\pgfpathrectangle{\pgfqpoint{3.776708in}{0.600000in}}{\pgfqpoint{2.573292in}{2.070576in}}%
\pgfusepath{clip}%
\pgfsetbuttcap%
\pgfsetmiterjoin%
\definecolor{currentfill}{rgb}{0.511253,0.510898,0.193296}%
\pgfsetfillcolor{currentfill}%
\pgfsetlinewidth{0.000000pt}%
\definecolor{currentstroke}{rgb}{0.000000,0.000000,0.000000}%
\pgfsetstrokecolor{currentstroke}%
\pgfsetstrokeopacity{0.000000}%
\pgfsetdash{}{0pt}%
\pgfpathmoveto{\pgfqpoint{6.158628in}{1.132101in}}%
\pgfpathlineto{\pgfqpoint{6.167381in}{1.132101in}}%
\pgfpathlineto{\pgfqpoint{6.167381in}{1.108718in}}%
\pgfpathlineto{\pgfqpoint{6.158628in}{1.108718in}}%
\pgfpathlineto{\pgfqpoint{6.158628in}{1.132101in}}%
\pgfpathclose%
\pgfusepath{fill}%
\end{pgfscope}%
\begin{pgfscope}%
\pgfpathrectangle{\pgfqpoint{3.776708in}{0.600000in}}{\pgfqpoint{2.573292in}{2.070576in}}%
\pgfusepath{clip}%
\pgfsetbuttcap%
\pgfsetmiterjoin%
\definecolor{currentfill}{rgb}{0.511253,0.510898,0.193296}%
\pgfsetfillcolor{currentfill}%
\pgfsetlinewidth{0.000000pt}%
\definecolor{currentstroke}{rgb}{0.000000,0.000000,0.000000}%
\pgfsetstrokecolor{currentstroke}%
\pgfsetstrokeopacity{0.000000}%
\pgfsetdash{}{0pt}%
\pgfpathmoveto{\pgfqpoint{6.169570in}{1.135881in}}%
\pgfpathlineto{\pgfqpoint{6.178323in}{1.135881in}}%
\pgfpathlineto{\pgfqpoint{6.178323in}{1.109903in}}%
\pgfpathlineto{\pgfqpoint{6.169570in}{1.109903in}}%
\pgfpathlineto{\pgfqpoint{6.169570in}{1.135881in}}%
\pgfpathclose%
\pgfusepath{fill}%
\end{pgfscope}%
\begin{pgfscope}%
\pgfpathrectangle{\pgfqpoint{3.776708in}{0.600000in}}{\pgfqpoint{2.573292in}{2.070576in}}%
\pgfusepath{clip}%
\pgfsetbuttcap%
\pgfsetmiterjoin%
\definecolor{currentfill}{rgb}{0.511253,0.510898,0.193296}%
\pgfsetfillcolor{currentfill}%
\pgfsetlinewidth{0.000000pt}%
\definecolor{currentstroke}{rgb}{0.000000,0.000000,0.000000}%
\pgfsetstrokecolor{currentstroke}%
\pgfsetstrokeopacity{0.000000}%
\pgfsetdash{}{0pt}%
\pgfpathmoveto{\pgfqpoint{6.180512in}{1.140310in}}%
\pgfpathlineto{\pgfqpoint{6.189265in}{1.140310in}}%
\pgfpathlineto{\pgfqpoint{6.189265in}{1.113747in}}%
\pgfpathlineto{\pgfqpoint{6.180512in}{1.113747in}}%
\pgfpathlineto{\pgfqpoint{6.180512in}{1.140310in}}%
\pgfpathclose%
\pgfusepath{fill}%
\end{pgfscope}%
\begin{pgfscope}%
\pgfpathrectangle{\pgfqpoint{3.776708in}{0.600000in}}{\pgfqpoint{2.573292in}{2.070576in}}%
\pgfusepath{clip}%
\pgfsetbuttcap%
\pgfsetmiterjoin%
\definecolor{currentfill}{rgb}{0.511253,0.510898,0.193296}%
\pgfsetfillcolor{currentfill}%
\pgfsetlinewidth{0.000000pt}%
\definecolor{currentstroke}{rgb}{0.000000,0.000000,0.000000}%
\pgfsetstrokecolor{currentstroke}%
\pgfsetstrokeopacity{0.000000}%
\pgfsetdash{}{0pt}%
\pgfpathmoveto{\pgfqpoint{6.191453in}{1.147919in}}%
\pgfpathlineto{\pgfqpoint{6.200207in}{1.147919in}}%
\pgfpathlineto{\pgfqpoint{6.200207in}{1.120754in}}%
\pgfpathlineto{\pgfqpoint{6.191453in}{1.120754in}}%
\pgfpathlineto{\pgfqpoint{6.191453in}{1.147919in}}%
\pgfpathclose%
\pgfusepath{fill}%
\end{pgfscope}%
\begin{pgfscope}%
\pgfpathrectangle{\pgfqpoint{3.776708in}{0.600000in}}{\pgfqpoint{2.573292in}{2.070576in}}%
\pgfusepath{clip}%
\pgfsetbuttcap%
\pgfsetmiterjoin%
\definecolor{currentfill}{rgb}{0.511253,0.510898,0.193296}%
\pgfsetfillcolor{currentfill}%
\pgfsetlinewidth{0.000000pt}%
\definecolor{currentstroke}{rgb}{0.000000,0.000000,0.000000}%
\pgfsetstrokecolor{currentstroke}%
\pgfsetstrokeopacity{0.000000}%
\pgfsetdash{}{0pt}%
\pgfpathmoveto{\pgfqpoint{6.202395in}{1.152333in}}%
\pgfpathlineto{\pgfqpoint{6.211149in}{1.152333in}}%
\pgfpathlineto{\pgfqpoint{6.211149in}{1.118516in}}%
\pgfpathlineto{\pgfqpoint{6.202395in}{1.118516in}}%
\pgfpathlineto{\pgfqpoint{6.202395in}{1.152333in}}%
\pgfpathclose%
\pgfusepath{fill}%
\end{pgfscope}%
\begin{pgfscope}%
\pgfpathrectangle{\pgfqpoint{3.776708in}{0.600000in}}{\pgfqpoint{2.573292in}{2.070576in}}%
\pgfusepath{clip}%
\pgfsetbuttcap%
\pgfsetmiterjoin%
\definecolor{currentfill}{rgb}{0.511253,0.510898,0.193296}%
\pgfsetfillcolor{currentfill}%
\pgfsetlinewidth{0.000000pt}%
\definecolor{currentstroke}{rgb}{0.000000,0.000000,0.000000}%
\pgfsetstrokecolor{currentstroke}%
\pgfsetstrokeopacity{0.000000}%
\pgfsetdash{}{0pt}%
\pgfpathmoveto{\pgfqpoint{6.213337in}{1.159262in}}%
\pgfpathlineto{\pgfqpoint{6.222090in}{1.159262in}}%
\pgfpathlineto{\pgfqpoint{6.222090in}{1.125321in}}%
\pgfpathlineto{\pgfqpoint{6.213337in}{1.125321in}}%
\pgfpathlineto{\pgfqpoint{6.213337in}{1.159262in}}%
\pgfpathclose%
\pgfusepath{fill}%
\end{pgfscope}%
\begin{pgfscope}%
\pgfpathrectangle{\pgfqpoint{3.776708in}{0.600000in}}{\pgfqpoint{2.573292in}{2.070576in}}%
\pgfusepath{clip}%
\pgfsetbuttcap%
\pgfsetmiterjoin%
\definecolor{currentfill}{rgb}{0.511253,0.510898,0.193296}%
\pgfsetfillcolor{currentfill}%
\pgfsetlinewidth{0.000000pt}%
\definecolor{currentstroke}{rgb}{0.000000,0.000000,0.000000}%
\pgfsetstrokecolor{currentstroke}%
\pgfsetstrokeopacity{0.000000}%
\pgfsetdash{}{0pt}%
\pgfpathmoveto{\pgfqpoint{6.224279in}{1.164543in}}%
\pgfpathlineto{\pgfqpoint{6.233032in}{1.164543in}}%
\pgfpathlineto{\pgfqpoint{6.233032in}{1.127807in}}%
\pgfpathlineto{\pgfqpoint{6.224279in}{1.127807in}}%
\pgfpathlineto{\pgfqpoint{6.224279in}{1.164543in}}%
\pgfpathclose%
\pgfusepath{fill}%
\end{pgfscope}%
\begin{pgfscope}%
\pgfpathrectangle{\pgfqpoint{3.776708in}{0.600000in}}{\pgfqpoint{2.573292in}{2.070576in}}%
\pgfusepath{clip}%
\pgfsetbuttcap%
\pgfsetmiterjoin%
\definecolor{currentfill}{rgb}{0.754268,0.565033,0.211761}%
\pgfsetfillcolor{currentfill}%
\pgfsetlinewidth{0.000000pt}%
\definecolor{currentstroke}{rgb}{0.000000,0.000000,0.000000}%
\pgfsetstrokecolor{currentstroke}%
\pgfsetstrokeopacity{0.000000}%
\pgfsetdash{}{0pt}%
\pgfpathmoveto{\pgfqpoint{3.893676in}{1.683487in}}%
\pgfpathlineto{\pgfqpoint{3.902429in}{1.683487in}}%
\pgfpathlineto{\pgfqpoint{3.902429in}{1.770971in}}%
\pgfpathlineto{\pgfqpoint{3.893676in}{1.770971in}}%
\pgfpathlineto{\pgfqpoint{3.893676in}{1.683487in}}%
\pgfpathclose%
\pgfusepath{fill}%
\end{pgfscope}%
\begin{pgfscope}%
\pgfpathrectangle{\pgfqpoint{3.776708in}{0.600000in}}{\pgfqpoint{2.573292in}{2.070576in}}%
\pgfusepath{clip}%
\pgfsetbuttcap%
\pgfsetmiterjoin%
\definecolor{currentfill}{rgb}{0.754268,0.565033,0.211761}%
\pgfsetfillcolor{currentfill}%
\pgfsetlinewidth{0.000000pt}%
\definecolor{currentstroke}{rgb}{0.000000,0.000000,0.000000}%
\pgfsetstrokecolor{currentstroke}%
\pgfsetstrokeopacity{0.000000}%
\pgfsetdash{}{0pt}%
\pgfpathmoveto{\pgfqpoint{3.904617in}{1.687387in}}%
\pgfpathlineto{\pgfqpoint{3.913371in}{1.687387in}}%
\pgfpathlineto{\pgfqpoint{3.913371in}{1.767038in}}%
\pgfpathlineto{\pgfqpoint{3.904617in}{1.767038in}}%
\pgfpathlineto{\pgfqpoint{3.904617in}{1.687387in}}%
\pgfpathclose%
\pgfusepath{fill}%
\end{pgfscope}%
\begin{pgfscope}%
\pgfpathrectangle{\pgfqpoint{3.776708in}{0.600000in}}{\pgfqpoint{2.573292in}{2.070576in}}%
\pgfusepath{clip}%
\pgfsetbuttcap%
\pgfsetmiterjoin%
\definecolor{currentfill}{rgb}{0.754268,0.565033,0.211761}%
\pgfsetfillcolor{currentfill}%
\pgfsetlinewidth{0.000000pt}%
\definecolor{currentstroke}{rgb}{0.000000,0.000000,0.000000}%
\pgfsetstrokecolor{currentstroke}%
\pgfsetstrokeopacity{0.000000}%
\pgfsetdash{}{0pt}%
\pgfpathmoveto{\pgfqpoint{3.915559in}{1.690905in}}%
\pgfpathlineto{\pgfqpoint{3.924313in}{1.690905in}}%
\pgfpathlineto{\pgfqpoint{3.924313in}{1.754048in}}%
\pgfpathlineto{\pgfqpoint{3.915559in}{1.754048in}}%
\pgfpathlineto{\pgfqpoint{3.915559in}{1.690905in}}%
\pgfpathclose%
\pgfusepath{fill}%
\end{pgfscope}%
\begin{pgfscope}%
\pgfpathrectangle{\pgfqpoint{3.776708in}{0.600000in}}{\pgfqpoint{2.573292in}{2.070576in}}%
\pgfusepath{clip}%
\pgfsetbuttcap%
\pgfsetmiterjoin%
\definecolor{currentfill}{rgb}{0.754268,0.565033,0.211761}%
\pgfsetfillcolor{currentfill}%
\pgfsetlinewidth{0.000000pt}%
\definecolor{currentstroke}{rgb}{0.000000,0.000000,0.000000}%
\pgfsetstrokecolor{currentstroke}%
\pgfsetstrokeopacity{0.000000}%
\pgfsetdash{}{0pt}%
\pgfpathmoveto{\pgfqpoint{3.926501in}{1.694430in}}%
\pgfpathlineto{\pgfqpoint{3.935254in}{1.694430in}}%
\pgfpathlineto{\pgfqpoint{3.935254in}{1.737201in}}%
\pgfpathlineto{\pgfqpoint{3.926501in}{1.737201in}}%
\pgfpathlineto{\pgfqpoint{3.926501in}{1.694430in}}%
\pgfpathclose%
\pgfusepath{fill}%
\end{pgfscope}%
\begin{pgfscope}%
\pgfpathrectangle{\pgfqpoint{3.776708in}{0.600000in}}{\pgfqpoint{2.573292in}{2.070576in}}%
\pgfusepath{clip}%
\pgfsetbuttcap%
\pgfsetmiterjoin%
\definecolor{currentfill}{rgb}{0.754268,0.565033,0.211761}%
\pgfsetfillcolor{currentfill}%
\pgfsetlinewidth{0.000000pt}%
\definecolor{currentstroke}{rgb}{0.000000,0.000000,0.000000}%
\pgfsetstrokecolor{currentstroke}%
\pgfsetstrokeopacity{0.000000}%
\pgfsetdash{}{0pt}%
\pgfpathmoveto{\pgfqpoint{3.937443in}{1.711254in}}%
\pgfpathlineto{\pgfqpoint{3.946196in}{1.711254in}}%
\pgfpathlineto{\pgfqpoint{3.946196in}{1.757117in}}%
\pgfpathlineto{\pgfqpoint{3.937443in}{1.757117in}}%
\pgfpathlineto{\pgfqpoint{3.937443in}{1.711254in}}%
\pgfpathclose%
\pgfusepath{fill}%
\end{pgfscope}%
\begin{pgfscope}%
\pgfpathrectangle{\pgfqpoint{3.776708in}{0.600000in}}{\pgfqpoint{2.573292in}{2.070576in}}%
\pgfusepath{clip}%
\pgfsetbuttcap%
\pgfsetmiterjoin%
\definecolor{currentfill}{rgb}{0.754268,0.565033,0.211761}%
\pgfsetfillcolor{currentfill}%
\pgfsetlinewidth{0.000000pt}%
\definecolor{currentstroke}{rgb}{0.000000,0.000000,0.000000}%
\pgfsetstrokecolor{currentstroke}%
\pgfsetstrokeopacity{0.000000}%
\pgfsetdash{}{0pt}%
\pgfpathmoveto{\pgfqpoint{3.948385in}{1.722391in}}%
\pgfpathlineto{\pgfqpoint{3.957138in}{1.722391in}}%
\pgfpathlineto{\pgfqpoint{3.957138in}{1.771392in}}%
\pgfpathlineto{\pgfqpoint{3.948385in}{1.771392in}}%
\pgfpathlineto{\pgfqpoint{3.948385in}{1.722391in}}%
\pgfpathclose%
\pgfusepath{fill}%
\end{pgfscope}%
\begin{pgfscope}%
\pgfpathrectangle{\pgfqpoint{3.776708in}{0.600000in}}{\pgfqpoint{2.573292in}{2.070576in}}%
\pgfusepath{clip}%
\pgfsetbuttcap%
\pgfsetmiterjoin%
\definecolor{currentfill}{rgb}{0.754268,0.565033,0.211761}%
\pgfsetfillcolor{currentfill}%
\pgfsetlinewidth{0.000000pt}%
\definecolor{currentstroke}{rgb}{0.000000,0.000000,0.000000}%
\pgfsetstrokecolor{currentstroke}%
\pgfsetstrokeopacity{0.000000}%
\pgfsetdash{}{0pt}%
\pgfpathmoveto{\pgfqpoint{3.959326in}{1.736117in}}%
\pgfpathlineto{\pgfqpoint{3.968080in}{1.736117in}}%
\pgfpathlineto{\pgfqpoint{3.968080in}{1.778866in}}%
\pgfpathlineto{\pgfqpoint{3.959326in}{1.778866in}}%
\pgfpathlineto{\pgfqpoint{3.959326in}{1.736117in}}%
\pgfpathclose%
\pgfusepath{fill}%
\end{pgfscope}%
\begin{pgfscope}%
\pgfpathrectangle{\pgfqpoint{3.776708in}{0.600000in}}{\pgfqpoint{2.573292in}{2.070576in}}%
\pgfusepath{clip}%
\pgfsetbuttcap%
\pgfsetmiterjoin%
\definecolor{currentfill}{rgb}{0.754268,0.565033,0.211761}%
\pgfsetfillcolor{currentfill}%
\pgfsetlinewidth{0.000000pt}%
\definecolor{currentstroke}{rgb}{0.000000,0.000000,0.000000}%
\pgfsetstrokecolor{currentstroke}%
\pgfsetstrokeopacity{0.000000}%
\pgfsetdash{}{0pt}%
\pgfpathmoveto{\pgfqpoint{3.970268in}{1.746093in}}%
\pgfpathlineto{\pgfqpoint{3.979022in}{1.746093in}}%
\pgfpathlineto{\pgfqpoint{3.979022in}{1.768360in}}%
\pgfpathlineto{\pgfqpoint{3.970268in}{1.768360in}}%
\pgfpathlineto{\pgfqpoint{3.970268in}{1.746093in}}%
\pgfpathclose%
\pgfusepath{fill}%
\end{pgfscope}%
\begin{pgfscope}%
\pgfpathrectangle{\pgfqpoint{3.776708in}{0.600000in}}{\pgfqpoint{2.573292in}{2.070576in}}%
\pgfusepath{clip}%
\pgfsetbuttcap%
\pgfsetmiterjoin%
\definecolor{currentfill}{rgb}{0.754268,0.565033,0.211761}%
\pgfsetfillcolor{currentfill}%
\pgfsetlinewidth{0.000000pt}%
\definecolor{currentstroke}{rgb}{0.000000,0.000000,0.000000}%
\pgfsetstrokecolor{currentstroke}%
\pgfsetstrokeopacity{0.000000}%
\pgfsetdash{}{0pt}%
\pgfpathmoveto{\pgfqpoint{3.981210in}{1.762190in}}%
\pgfpathlineto{\pgfqpoint{3.989963in}{1.762190in}}%
\pgfpathlineto{\pgfqpoint{3.989963in}{1.773268in}}%
\pgfpathlineto{\pgfqpoint{3.981210in}{1.773268in}}%
\pgfpathlineto{\pgfqpoint{3.981210in}{1.762190in}}%
\pgfpathclose%
\pgfusepath{fill}%
\end{pgfscope}%
\begin{pgfscope}%
\pgfpathrectangle{\pgfqpoint{3.776708in}{0.600000in}}{\pgfqpoint{2.573292in}{2.070576in}}%
\pgfusepath{clip}%
\pgfsetbuttcap%
\pgfsetmiterjoin%
\definecolor{currentfill}{rgb}{0.754268,0.565033,0.211761}%
\pgfsetfillcolor{currentfill}%
\pgfsetlinewidth{0.000000pt}%
\definecolor{currentstroke}{rgb}{0.000000,0.000000,0.000000}%
\pgfsetstrokecolor{currentstroke}%
\pgfsetstrokeopacity{0.000000}%
\pgfsetdash{}{0pt}%
\pgfpathmoveto{\pgfqpoint{3.992152in}{1.542720in}}%
\pgfpathlineto{\pgfqpoint{4.000905in}{1.542720in}}%
\pgfpathlineto{\pgfqpoint{4.000905in}{1.538035in}}%
\pgfpathlineto{\pgfqpoint{3.992152in}{1.538035in}}%
\pgfpathlineto{\pgfqpoint{3.992152in}{1.542720in}}%
\pgfpathclose%
\pgfusepath{fill}%
\end{pgfscope}%
\begin{pgfscope}%
\pgfpathrectangle{\pgfqpoint{3.776708in}{0.600000in}}{\pgfqpoint{2.573292in}{2.070576in}}%
\pgfusepath{clip}%
\pgfsetbuttcap%
\pgfsetmiterjoin%
\definecolor{currentfill}{rgb}{0.754268,0.565033,0.211761}%
\pgfsetfillcolor{currentfill}%
\pgfsetlinewidth{0.000000pt}%
\definecolor{currentstroke}{rgb}{0.000000,0.000000,0.000000}%
\pgfsetstrokecolor{currentstroke}%
\pgfsetstrokeopacity{0.000000}%
\pgfsetdash{}{0pt}%
\pgfpathmoveto{\pgfqpoint{4.003094in}{1.542700in}}%
\pgfpathlineto{\pgfqpoint{4.011847in}{1.542700in}}%
\pgfpathlineto{\pgfqpoint{4.011847in}{1.518879in}}%
\pgfpathlineto{\pgfqpoint{4.003094in}{1.518879in}}%
\pgfpathlineto{\pgfqpoint{4.003094in}{1.542700in}}%
\pgfpathclose%
\pgfusepath{fill}%
\end{pgfscope}%
\begin{pgfscope}%
\pgfpathrectangle{\pgfqpoint{3.776708in}{0.600000in}}{\pgfqpoint{2.573292in}{2.070576in}}%
\pgfusepath{clip}%
\pgfsetbuttcap%
\pgfsetmiterjoin%
\definecolor{currentfill}{rgb}{0.754268,0.565033,0.211761}%
\pgfsetfillcolor{currentfill}%
\pgfsetlinewidth{0.000000pt}%
\definecolor{currentstroke}{rgb}{0.000000,0.000000,0.000000}%
\pgfsetstrokecolor{currentstroke}%
\pgfsetstrokeopacity{0.000000}%
\pgfsetdash{}{0pt}%
\pgfpathmoveto{\pgfqpoint{4.014035in}{1.543333in}}%
\pgfpathlineto{\pgfqpoint{4.022789in}{1.543333in}}%
\pgfpathlineto{\pgfqpoint{4.022789in}{1.487916in}}%
\pgfpathlineto{\pgfqpoint{4.014035in}{1.487916in}}%
\pgfpathlineto{\pgfqpoint{4.014035in}{1.543333in}}%
\pgfpathclose%
\pgfusepath{fill}%
\end{pgfscope}%
\begin{pgfscope}%
\pgfpathrectangle{\pgfqpoint{3.776708in}{0.600000in}}{\pgfqpoint{2.573292in}{2.070576in}}%
\pgfusepath{clip}%
\pgfsetbuttcap%
\pgfsetmiterjoin%
\definecolor{currentfill}{rgb}{0.754268,0.565033,0.211761}%
\pgfsetfillcolor{currentfill}%
\pgfsetlinewidth{0.000000pt}%
\definecolor{currentstroke}{rgb}{0.000000,0.000000,0.000000}%
\pgfsetstrokecolor{currentstroke}%
\pgfsetstrokeopacity{0.000000}%
\pgfsetdash{}{0pt}%
\pgfpathmoveto{\pgfqpoint{4.024977in}{1.547086in}}%
\pgfpathlineto{\pgfqpoint{4.033731in}{1.547086in}}%
\pgfpathlineto{\pgfqpoint{4.033731in}{1.475258in}}%
\pgfpathlineto{\pgfqpoint{4.024977in}{1.475258in}}%
\pgfpathlineto{\pgfqpoint{4.024977in}{1.547086in}}%
\pgfpathclose%
\pgfusepath{fill}%
\end{pgfscope}%
\begin{pgfscope}%
\pgfpathrectangle{\pgfqpoint{3.776708in}{0.600000in}}{\pgfqpoint{2.573292in}{2.070576in}}%
\pgfusepath{clip}%
\pgfsetbuttcap%
\pgfsetmiterjoin%
\definecolor{currentfill}{rgb}{0.754268,0.565033,0.211761}%
\pgfsetfillcolor{currentfill}%
\pgfsetlinewidth{0.000000pt}%
\definecolor{currentstroke}{rgb}{0.000000,0.000000,0.000000}%
\pgfsetstrokecolor{currentstroke}%
\pgfsetstrokeopacity{0.000000}%
\pgfsetdash{}{0pt}%
\pgfpathmoveto{\pgfqpoint{4.035919in}{1.557070in}}%
\pgfpathlineto{\pgfqpoint{4.044672in}{1.557070in}}%
\pgfpathlineto{\pgfqpoint{4.044672in}{1.459735in}}%
\pgfpathlineto{\pgfqpoint{4.035919in}{1.459735in}}%
\pgfpathlineto{\pgfqpoint{4.035919in}{1.557070in}}%
\pgfpathclose%
\pgfusepath{fill}%
\end{pgfscope}%
\begin{pgfscope}%
\pgfpathrectangle{\pgfqpoint{3.776708in}{0.600000in}}{\pgfqpoint{2.573292in}{2.070576in}}%
\pgfusepath{clip}%
\pgfsetbuttcap%
\pgfsetmiterjoin%
\definecolor{currentfill}{rgb}{0.754268,0.565033,0.211761}%
\pgfsetfillcolor{currentfill}%
\pgfsetlinewidth{0.000000pt}%
\definecolor{currentstroke}{rgb}{0.000000,0.000000,0.000000}%
\pgfsetstrokecolor{currentstroke}%
\pgfsetstrokeopacity{0.000000}%
\pgfsetdash{}{0pt}%
\pgfpathmoveto{\pgfqpoint{4.046861in}{1.553860in}}%
\pgfpathlineto{\pgfqpoint{4.055614in}{1.553860in}}%
\pgfpathlineto{\pgfqpoint{4.055614in}{1.436464in}}%
\pgfpathlineto{\pgfqpoint{4.046861in}{1.436464in}}%
\pgfpathlineto{\pgfqpoint{4.046861in}{1.553860in}}%
\pgfpathclose%
\pgfusepath{fill}%
\end{pgfscope}%
\begin{pgfscope}%
\pgfpathrectangle{\pgfqpoint{3.776708in}{0.600000in}}{\pgfqpoint{2.573292in}{2.070576in}}%
\pgfusepath{clip}%
\pgfsetbuttcap%
\pgfsetmiterjoin%
\definecolor{currentfill}{rgb}{0.754268,0.565033,0.211761}%
\pgfsetfillcolor{currentfill}%
\pgfsetlinewidth{0.000000pt}%
\definecolor{currentstroke}{rgb}{0.000000,0.000000,0.000000}%
\pgfsetstrokecolor{currentstroke}%
\pgfsetstrokeopacity{0.000000}%
\pgfsetdash{}{0pt}%
\pgfpathmoveto{\pgfqpoint{4.057803in}{1.553290in}}%
\pgfpathlineto{\pgfqpoint{4.066556in}{1.553290in}}%
\pgfpathlineto{\pgfqpoint{4.066556in}{1.410317in}}%
\pgfpathlineto{\pgfqpoint{4.057803in}{1.410317in}}%
\pgfpathlineto{\pgfqpoint{4.057803in}{1.553290in}}%
\pgfpathclose%
\pgfusepath{fill}%
\end{pgfscope}%
\begin{pgfscope}%
\pgfpathrectangle{\pgfqpoint{3.776708in}{0.600000in}}{\pgfqpoint{2.573292in}{2.070576in}}%
\pgfusepath{clip}%
\pgfsetbuttcap%
\pgfsetmiterjoin%
\definecolor{currentfill}{rgb}{0.754268,0.565033,0.211761}%
\pgfsetfillcolor{currentfill}%
\pgfsetlinewidth{0.000000pt}%
\definecolor{currentstroke}{rgb}{0.000000,0.000000,0.000000}%
\pgfsetstrokecolor{currentstroke}%
\pgfsetstrokeopacity{0.000000}%
\pgfsetdash{}{0pt}%
\pgfpathmoveto{\pgfqpoint{4.068744in}{1.565624in}}%
\pgfpathlineto{\pgfqpoint{4.077498in}{1.565624in}}%
\pgfpathlineto{\pgfqpoint{4.077498in}{1.403653in}}%
\pgfpathlineto{\pgfqpoint{4.068744in}{1.403653in}}%
\pgfpathlineto{\pgfqpoint{4.068744in}{1.565624in}}%
\pgfpathclose%
\pgfusepath{fill}%
\end{pgfscope}%
\begin{pgfscope}%
\pgfpathrectangle{\pgfqpoint{3.776708in}{0.600000in}}{\pgfqpoint{2.573292in}{2.070576in}}%
\pgfusepath{clip}%
\pgfsetbuttcap%
\pgfsetmiterjoin%
\definecolor{currentfill}{rgb}{0.754268,0.565033,0.211761}%
\pgfsetfillcolor{currentfill}%
\pgfsetlinewidth{0.000000pt}%
\definecolor{currentstroke}{rgb}{0.000000,0.000000,0.000000}%
\pgfsetstrokecolor{currentstroke}%
\pgfsetstrokeopacity{0.000000}%
\pgfsetdash{}{0pt}%
\pgfpathmoveto{\pgfqpoint{4.079686in}{1.552931in}}%
\pgfpathlineto{\pgfqpoint{4.088440in}{1.552931in}}%
\pgfpathlineto{\pgfqpoint{4.088440in}{1.385448in}}%
\pgfpathlineto{\pgfqpoint{4.079686in}{1.385448in}}%
\pgfpathlineto{\pgfqpoint{4.079686in}{1.552931in}}%
\pgfpathclose%
\pgfusepath{fill}%
\end{pgfscope}%
\begin{pgfscope}%
\pgfpathrectangle{\pgfqpoint{3.776708in}{0.600000in}}{\pgfqpoint{2.573292in}{2.070576in}}%
\pgfusepath{clip}%
\pgfsetbuttcap%
\pgfsetmiterjoin%
\definecolor{currentfill}{rgb}{0.754268,0.565033,0.211761}%
\pgfsetfillcolor{currentfill}%
\pgfsetlinewidth{0.000000pt}%
\definecolor{currentstroke}{rgb}{0.000000,0.000000,0.000000}%
\pgfsetstrokecolor{currentstroke}%
\pgfsetstrokeopacity{0.000000}%
\pgfsetdash{}{0pt}%
\pgfpathmoveto{\pgfqpoint{4.090628in}{1.535919in}}%
\pgfpathlineto{\pgfqpoint{4.099381in}{1.535919in}}%
\pgfpathlineto{\pgfqpoint{4.099381in}{1.354461in}}%
\pgfpathlineto{\pgfqpoint{4.090628in}{1.354461in}}%
\pgfpathlineto{\pgfqpoint{4.090628in}{1.535919in}}%
\pgfpathclose%
\pgfusepath{fill}%
\end{pgfscope}%
\begin{pgfscope}%
\pgfpathrectangle{\pgfqpoint{3.776708in}{0.600000in}}{\pgfqpoint{2.573292in}{2.070576in}}%
\pgfusepath{clip}%
\pgfsetbuttcap%
\pgfsetmiterjoin%
\definecolor{currentfill}{rgb}{0.754268,0.565033,0.211761}%
\pgfsetfillcolor{currentfill}%
\pgfsetlinewidth{0.000000pt}%
\definecolor{currentstroke}{rgb}{0.000000,0.000000,0.000000}%
\pgfsetstrokecolor{currentstroke}%
\pgfsetstrokeopacity{0.000000}%
\pgfsetdash{}{0pt}%
\pgfpathmoveto{\pgfqpoint{4.101570in}{1.509510in}}%
\pgfpathlineto{\pgfqpoint{4.110323in}{1.509510in}}%
\pgfpathlineto{\pgfqpoint{4.110323in}{1.326198in}}%
\pgfpathlineto{\pgfqpoint{4.101570in}{1.326198in}}%
\pgfpathlineto{\pgfqpoint{4.101570in}{1.509510in}}%
\pgfpathclose%
\pgfusepath{fill}%
\end{pgfscope}%
\begin{pgfscope}%
\pgfpathrectangle{\pgfqpoint{3.776708in}{0.600000in}}{\pgfqpoint{2.573292in}{2.070576in}}%
\pgfusepath{clip}%
\pgfsetbuttcap%
\pgfsetmiterjoin%
\definecolor{currentfill}{rgb}{0.754268,0.565033,0.211761}%
\pgfsetfillcolor{currentfill}%
\pgfsetlinewidth{0.000000pt}%
\definecolor{currentstroke}{rgb}{0.000000,0.000000,0.000000}%
\pgfsetstrokecolor{currentstroke}%
\pgfsetstrokeopacity{0.000000}%
\pgfsetdash{}{0pt}%
\pgfpathmoveto{\pgfqpoint{4.112512in}{1.497026in}}%
\pgfpathlineto{\pgfqpoint{4.121265in}{1.497026in}}%
\pgfpathlineto{\pgfqpoint{4.121265in}{1.315135in}}%
\pgfpathlineto{\pgfqpoint{4.112512in}{1.315135in}}%
\pgfpathlineto{\pgfqpoint{4.112512in}{1.497026in}}%
\pgfpathclose%
\pgfusepath{fill}%
\end{pgfscope}%
\begin{pgfscope}%
\pgfpathrectangle{\pgfqpoint{3.776708in}{0.600000in}}{\pgfqpoint{2.573292in}{2.070576in}}%
\pgfusepath{clip}%
\pgfsetbuttcap%
\pgfsetmiterjoin%
\definecolor{currentfill}{rgb}{0.754268,0.565033,0.211761}%
\pgfsetfillcolor{currentfill}%
\pgfsetlinewidth{0.000000pt}%
\definecolor{currentstroke}{rgb}{0.000000,0.000000,0.000000}%
\pgfsetstrokecolor{currentstroke}%
\pgfsetstrokeopacity{0.000000}%
\pgfsetdash{}{0pt}%
\pgfpathmoveto{\pgfqpoint{4.123453in}{1.494019in}}%
\pgfpathlineto{\pgfqpoint{4.132207in}{1.494019in}}%
\pgfpathlineto{\pgfqpoint{4.132207in}{1.303366in}}%
\pgfpathlineto{\pgfqpoint{4.123453in}{1.303366in}}%
\pgfpathlineto{\pgfqpoint{4.123453in}{1.494019in}}%
\pgfpathclose%
\pgfusepath{fill}%
\end{pgfscope}%
\begin{pgfscope}%
\pgfpathrectangle{\pgfqpoint{3.776708in}{0.600000in}}{\pgfqpoint{2.573292in}{2.070576in}}%
\pgfusepath{clip}%
\pgfsetbuttcap%
\pgfsetmiterjoin%
\definecolor{currentfill}{rgb}{0.754268,0.565033,0.211761}%
\pgfsetfillcolor{currentfill}%
\pgfsetlinewidth{0.000000pt}%
\definecolor{currentstroke}{rgb}{0.000000,0.000000,0.000000}%
\pgfsetstrokecolor{currentstroke}%
\pgfsetstrokeopacity{0.000000}%
\pgfsetdash{}{0pt}%
\pgfpathmoveto{\pgfqpoint{4.134395in}{1.481495in}}%
\pgfpathlineto{\pgfqpoint{4.143149in}{1.481495in}}%
\pgfpathlineto{\pgfqpoint{4.143149in}{1.266652in}}%
\pgfpathlineto{\pgfqpoint{4.134395in}{1.266652in}}%
\pgfpathlineto{\pgfqpoint{4.134395in}{1.481495in}}%
\pgfpathclose%
\pgfusepath{fill}%
\end{pgfscope}%
\begin{pgfscope}%
\pgfpathrectangle{\pgfqpoint{3.776708in}{0.600000in}}{\pgfqpoint{2.573292in}{2.070576in}}%
\pgfusepath{clip}%
\pgfsetbuttcap%
\pgfsetmiterjoin%
\definecolor{currentfill}{rgb}{0.754268,0.565033,0.211761}%
\pgfsetfillcolor{currentfill}%
\pgfsetlinewidth{0.000000pt}%
\definecolor{currentstroke}{rgb}{0.000000,0.000000,0.000000}%
\pgfsetstrokecolor{currentstroke}%
\pgfsetstrokeopacity{0.000000}%
\pgfsetdash{}{0pt}%
\pgfpathmoveto{\pgfqpoint{4.145337in}{1.471984in}}%
\pgfpathlineto{\pgfqpoint{4.154090in}{1.471984in}}%
\pgfpathlineto{\pgfqpoint{4.154090in}{1.231563in}}%
\pgfpathlineto{\pgfqpoint{4.145337in}{1.231563in}}%
\pgfpathlineto{\pgfqpoint{4.145337in}{1.471984in}}%
\pgfpathclose%
\pgfusepath{fill}%
\end{pgfscope}%
\begin{pgfscope}%
\pgfpathrectangle{\pgfqpoint{3.776708in}{0.600000in}}{\pgfqpoint{2.573292in}{2.070576in}}%
\pgfusepath{clip}%
\pgfsetbuttcap%
\pgfsetmiterjoin%
\definecolor{currentfill}{rgb}{0.754268,0.565033,0.211761}%
\pgfsetfillcolor{currentfill}%
\pgfsetlinewidth{0.000000pt}%
\definecolor{currentstroke}{rgb}{0.000000,0.000000,0.000000}%
\pgfsetstrokecolor{currentstroke}%
\pgfsetstrokeopacity{0.000000}%
\pgfsetdash{}{0pt}%
\pgfpathmoveto{\pgfqpoint{4.156279in}{1.451818in}}%
\pgfpathlineto{\pgfqpoint{4.165032in}{1.451818in}}%
\pgfpathlineto{\pgfqpoint{4.165032in}{1.196371in}}%
\pgfpathlineto{\pgfqpoint{4.156279in}{1.196371in}}%
\pgfpathlineto{\pgfqpoint{4.156279in}{1.451818in}}%
\pgfpathclose%
\pgfusepath{fill}%
\end{pgfscope}%
\begin{pgfscope}%
\pgfpathrectangle{\pgfqpoint{3.776708in}{0.600000in}}{\pgfqpoint{2.573292in}{2.070576in}}%
\pgfusepath{clip}%
\pgfsetbuttcap%
\pgfsetmiterjoin%
\definecolor{currentfill}{rgb}{0.754268,0.565033,0.211761}%
\pgfsetfillcolor{currentfill}%
\pgfsetlinewidth{0.000000pt}%
\definecolor{currentstroke}{rgb}{0.000000,0.000000,0.000000}%
\pgfsetstrokecolor{currentstroke}%
\pgfsetstrokeopacity{0.000000}%
\pgfsetdash{}{0pt}%
\pgfpathmoveto{\pgfqpoint{4.167221in}{1.434389in}}%
\pgfpathlineto{\pgfqpoint{4.175974in}{1.434389in}}%
\pgfpathlineto{\pgfqpoint{4.175974in}{1.165391in}}%
\pgfpathlineto{\pgfqpoint{4.167221in}{1.165391in}}%
\pgfpathlineto{\pgfqpoint{4.167221in}{1.434389in}}%
\pgfpathclose%
\pgfusepath{fill}%
\end{pgfscope}%
\begin{pgfscope}%
\pgfpathrectangle{\pgfqpoint{3.776708in}{0.600000in}}{\pgfqpoint{2.573292in}{2.070576in}}%
\pgfusepath{clip}%
\pgfsetbuttcap%
\pgfsetmiterjoin%
\definecolor{currentfill}{rgb}{0.754268,0.565033,0.211761}%
\pgfsetfillcolor{currentfill}%
\pgfsetlinewidth{0.000000pt}%
\definecolor{currentstroke}{rgb}{0.000000,0.000000,0.000000}%
\pgfsetstrokecolor{currentstroke}%
\pgfsetstrokeopacity{0.000000}%
\pgfsetdash{}{0pt}%
\pgfpathmoveto{\pgfqpoint{4.178162in}{1.421997in}}%
\pgfpathlineto{\pgfqpoint{4.186916in}{1.421997in}}%
\pgfpathlineto{\pgfqpoint{4.186916in}{1.133310in}}%
\pgfpathlineto{\pgfqpoint{4.178162in}{1.133310in}}%
\pgfpathlineto{\pgfqpoint{4.178162in}{1.421997in}}%
\pgfpathclose%
\pgfusepath{fill}%
\end{pgfscope}%
\begin{pgfscope}%
\pgfpathrectangle{\pgfqpoint{3.776708in}{0.600000in}}{\pgfqpoint{2.573292in}{2.070576in}}%
\pgfusepath{clip}%
\pgfsetbuttcap%
\pgfsetmiterjoin%
\definecolor{currentfill}{rgb}{0.754268,0.565033,0.211761}%
\pgfsetfillcolor{currentfill}%
\pgfsetlinewidth{0.000000pt}%
\definecolor{currentstroke}{rgb}{0.000000,0.000000,0.000000}%
\pgfsetstrokecolor{currentstroke}%
\pgfsetstrokeopacity{0.000000}%
\pgfsetdash{}{0pt}%
\pgfpathmoveto{\pgfqpoint{4.189104in}{1.402528in}}%
\pgfpathlineto{\pgfqpoint{4.197858in}{1.402528in}}%
\pgfpathlineto{\pgfqpoint{4.197858in}{1.087193in}}%
\pgfpathlineto{\pgfqpoint{4.189104in}{1.087193in}}%
\pgfpathlineto{\pgfqpoint{4.189104in}{1.402528in}}%
\pgfpathclose%
\pgfusepath{fill}%
\end{pgfscope}%
\begin{pgfscope}%
\pgfpathrectangle{\pgfqpoint{3.776708in}{0.600000in}}{\pgfqpoint{2.573292in}{2.070576in}}%
\pgfusepath{clip}%
\pgfsetbuttcap%
\pgfsetmiterjoin%
\definecolor{currentfill}{rgb}{0.754268,0.565033,0.211761}%
\pgfsetfillcolor{currentfill}%
\pgfsetlinewidth{0.000000pt}%
\definecolor{currentstroke}{rgb}{0.000000,0.000000,0.000000}%
\pgfsetstrokecolor{currentstroke}%
\pgfsetstrokeopacity{0.000000}%
\pgfsetdash{}{0pt}%
\pgfpathmoveto{\pgfqpoint{4.200046in}{1.384695in}}%
\pgfpathlineto{\pgfqpoint{4.208799in}{1.384695in}}%
\pgfpathlineto{\pgfqpoint{4.208799in}{1.047321in}}%
\pgfpathlineto{\pgfqpoint{4.200046in}{1.047321in}}%
\pgfpathlineto{\pgfqpoint{4.200046in}{1.384695in}}%
\pgfpathclose%
\pgfusepath{fill}%
\end{pgfscope}%
\begin{pgfscope}%
\pgfpathrectangle{\pgfqpoint{3.776708in}{0.600000in}}{\pgfqpoint{2.573292in}{2.070576in}}%
\pgfusepath{clip}%
\pgfsetbuttcap%
\pgfsetmiterjoin%
\definecolor{currentfill}{rgb}{0.754268,0.565033,0.211761}%
\pgfsetfillcolor{currentfill}%
\pgfsetlinewidth{0.000000pt}%
\definecolor{currentstroke}{rgb}{0.000000,0.000000,0.000000}%
\pgfsetstrokecolor{currentstroke}%
\pgfsetstrokeopacity{0.000000}%
\pgfsetdash{}{0pt}%
\pgfpathmoveto{\pgfqpoint{4.210988in}{1.380814in}}%
\pgfpathlineto{\pgfqpoint{4.219741in}{1.380814in}}%
\pgfpathlineto{\pgfqpoint{4.219741in}{1.019094in}}%
\pgfpathlineto{\pgfqpoint{4.210988in}{1.019094in}}%
\pgfpathlineto{\pgfqpoint{4.210988in}{1.380814in}}%
\pgfpathclose%
\pgfusepath{fill}%
\end{pgfscope}%
\begin{pgfscope}%
\pgfpathrectangle{\pgfqpoint{3.776708in}{0.600000in}}{\pgfqpoint{2.573292in}{2.070576in}}%
\pgfusepath{clip}%
\pgfsetbuttcap%
\pgfsetmiterjoin%
\definecolor{currentfill}{rgb}{0.754268,0.565033,0.211761}%
\pgfsetfillcolor{currentfill}%
\pgfsetlinewidth{0.000000pt}%
\definecolor{currentstroke}{rgb}{0.000000,0.000000,0.000000}%
\pgfsetstrokecolor{currentstroke}%
\pgfsetstrokeopacity{0.000000}%
\pgfsetdash{}{0pt}%
\pgfpathmoveto{\pgfqpoint{4.221930in}{1.378390in}}%
\pgfpathlineto{\pgfqpoint{4.230683in}{1.378390in}}%
\pgfpathlineto{\pgfqpoint{4.230683in}{0.994629in}}%
\pgfpathlineto{\pgfqpoint{4.221930in}{0.994629in}}%
\pgfpathlineto{\pgfqpoint{4.221930in}{1.378390in}}%
\pgfpathclose%
\pgfusepath{fill}%
\end{pgfscope}%
\begin{pgfscope}%
\pgfpathrectangle{\pgfqpoint{3.776708in}{0.600000in}}{\pgfqpoint{2.573292in}{2.070576in}}%
\pgfusepath{clip}%
\pgfsetbuttcap%
\pgfsetmiterjoin%
\definecolor{currentfill}{rgb}{0.754268,0.565033,0.211761}%
\pgfsetfillcolor{currentfill}%
\pgfsetlinewidth{0.000000pt}%
\definecolor{currentstroke}{rgb}{0.000000,0.000000,0.000000}%
\pgfsetstrokecolor{currentstroke}%
\pgfsetstrokeopacity{0.000000}%
\pgfsetdash{}{0pt}%
\pgfpathmoveto{\pgfqpoint{4.232871in}{1.362963in}}%
\pgfpathlineto{\pgfqpoint{4.241625in}{1.362963in}}%
\pgfpathlineto{\pgfqpoint{4.241625in}{0.965320in}}%
\pgfpathlineto{\pgfqpoint{4.232871in}{0.965320in}}%
\pgfpathlineto{\pgfqpoint{4.232871in}{1.362963in}}%
\pgfpathclose%
\pgfusepath{fill}%
\end{pgfscope}%
\begin{pgfscope}%
\pgfpathrectangle{\pgfqpoint{3.776708in}{0.600000in}}{\pgfqpoint{2.573292in}{2.070576in}}%
\pgfusepath{clip}%
\pgfsetbuttcap%
\pgfsetmiterjoin%
\definecolor{currentfill}{rgb}{0.754268,0.565033,0.211761}%
\pgfsetfillcolor{currentfill}%
\pgfsetlinewidth{0.000000pt}%
\definecolor{currentstroke}{rgb}{0.000000,0.000000,0.000000}%
\pgfsetstrokecolor{currentstroke}%
\pgfsetstrokeopacity{0.000000}%
\pgfsetdash{}{0pt}%
\pgfpathmoveto{\pgfqpoint{4.243813in}{1.354497in}}%
\pgfpathlineto{\pgfqpoint{4.252567in}{1.354497in}}%
\pgfpathlineto{\pgfqpoint{4.252567in}{0.944116in}}%
\pgfpathlineto{\pgfqpoint{4.243813in}{0.944116in}}%
\pgfpathlineto{\pgfqpoint{4.243813in}{1.354497in}}%
\pgfpathclose%
\pgfusepath{fill}%
\end{pgfscope}%
\begin{pgfscope}%
\pgfpathrectangle{\pgfqpoint{3.776708in}{0.600000in}}{\pgfqpoint{2.573292in}{2.070576in}}%
\pgfusepath{clip}%
\pgfsetbuttcap%
\pgfsetmiterjoin%
\definecolor{currentfill}{rgb}{0.754268,0.565033,0.211761}%
\pgfsetfillcolor{currentfill}%
\pgfsetlinewidth{0.000000pt}%
\definecolor{currentstroke}{rgb}{0.000000,0.000000,0.000000}%
\pgfsetstrokecolor{currentstroke}%
\pgfsetstrokeopacity{0.000000}%
\pgfsetdash{}{0pt}%
\pgfpathmoveto{\pgfqpoint{4.254755in}{1.346888in}}%
\pgfpathlineto{\pgfqpoint{4.263508in}{1.346888in}}%
\pgfpathlineto{\pgfqpoint{4.263508in}{0.926204in}}%
\pgfpathlineto{\pgfqpoint{4.254755in}{0.926204in}}%
\pgfpathlineto{\pgfqpoint{4.254755in}{1.346888in}}%
\pgfpathclose%
\pgfusepath{fill}%
\end{pgfscope}%
\begin{pgfscope}%
\pgfpathrectangle{\pgfqpoint{3.776708in}{0.600000in}}{\pgfqpoint{2.573292in}{2.070576in}}%
\pgfusepath{clip}%
\pgfsetbuttcap%
\pgfsetmiterjoin%
\definecolor{currentfill}{rgb}{0.754268,0.565033,0.211761}%
\pgfsetfillcolor{currentfill}%
\pgfsetlinewidth{0.000000pt}%
\definecolor{currentstroke}{rgb}{0.000000,0.000000,0.000000}%
\pgfsetstrokecolor{currentstroke}%
\pgfsetstrokeopacity{0.000000}%
\pgfsetdash{}{0pt}%
\pgfpathmoveto{\pgfqpoint{4.265697in}{1.329941in}}%
\pgfpathlineto{\pgfqpoint{4.274450in}{1.329941in}}%
\pgfpathlineto{\pgfqpoint{4.274450in}{0.923011in}}%
\pgfpathlineto{\pgfqpoint{4.265697in}{0.923011in}}%
\pgfpathlineto{\pgfqpoint{4.265697in}{1.329941in}}%
\pgfpathclose%
\pgfusepath{fill}%
\end{pgfscope}%
\begin{pgfscope}%
\pgfpathrectangle{\pgfqpoint{3.776708in}{0.600000in}}{\pgfqpoint{2.573292in}{2.070576in}}%
\pgfusepath{clip}%
\pgfsetbuttcap%
\pgfsetmiterjoin%
\definecolor{currentfill}{rgb}{0.754268,0.565033,0.211761}%
\pgfsetfillcolor{currentfill}%
\pgfsetlinewidth{0.000000pt}%
\definecolor{currentstroke}{rgb}{0.000000,0.000000,0.000000}%
\pgfsetstrokecolor{currentstroke}%
\pgfsetstrokeopacity{0.000000}%
\pgfsetdash{}{0pt}%
\pgfpathmoveto{\pgfqpoint{4.276639in}{1.327035in}}%
\pgfpathlineto{\pgfqpoint{4.285392in}{1.327035in}}%
\pgfpathlineto{\pgfqpoint{4.285392in}{0.942080in}}%
\pgfpathlineto{\pgfqpoint{4.276639in}{0.942080in}}%
\pgfpathlineto{\pgfqpoint{4.276639in}{1.327035in}}%
\pgfpathclose%
\pgfusepath{fill}%
\end{pgfscope}%
\begin{pgfscope}%
\pgfpathrectangle{\pgfqpoint{3.776708in}{0.600000in}}{\pgfqpoint{2.573292in}{2.070576in}}%
\pgfusepath{clip}%
\pgfsetbuttcap%
\pgfsetmiterjoin%
\definecolor{currentfill}{rgb}{0.754268,0.565033,0.211761}%
\pgfsetfillcolor{currentfill}%
\pgfsetlinewidth{0.000000pt}%
\definecolor{currentstroke}{rgb}{0.000000,0.000000,0.000000}%
\pgfsetstrokecolor{currentstroke}%
\pgfsetstrokeopacity{0.000000}%
\pgfsetdash{}{0pt}%
\pgfpathmoveto{\pgfqpoint{4.287580in}{1.312655in}}%
\pgfpathlineto{\pgfqpoint{4.296334in}{1.312655in}}%
\pgfpathlineto{\pgfqpoint{4.296334in}{0.954556in}}%
\pgfpathlineto{\pgfqpoint{4.287580in}{0.954556in}}%
\pgfpathlineto{\pgfqpoint{4.287580in}{1.312655in}}%
\pgfpathclose%
\pgfusepath{fill}%
\end{pgfscope}%
\begin{pgfscope}%
\pgfpathrectangle{\pgfqpoint{3.776708in}{0.600000in}}{\pgfqpoint{2.573292in}{2.070576in}}%
\pgfusepath{clip}%
\pgfsetbuttcap%
\pgfsetmiterjoin%
\definecolor{currentfill}{rgb}{0.754268,0.565033,0.211761}%
\pgfsetfillcolor{currentfill}%
\pgfsetlinewidth{0.000000pt}%
\definecolor{currentstroke}{rgb}{0.000000,0.000000,0.000000}%
\pgfsetstrokecolor{currentstroke}%
\pgfsetstrokeopacity{0.000000}%
\pgfsetdash{}{0pt}%
\pgfpathmoveto{\pgfqpoint{4.298522in}{1.307691in}}%
\pgfpathlineto{\pgfqpoint{4.307276in}{1.307691in}}%
\pgfpathlineto{\pgfqpoint{4.307276in}{0.972608in}}%
\pgfpathlineto{\pgfqpoint{4.298522in}{0.972608in}}%
\pgfpathlineto{\pgfqpoint{4.298522in}{1.307691in}}%
\pgfpathclose%
\pgfusepath{fill}%
\end{pgfscope}%
\begin{pgfscope}%
\pgfpathrectangle{\pgfqpoint{3.776708in}{0.600000in}}{\pgfqpoint{2.573292in}{2.070576in}}%
\pgfusepath{clip}%
\pgfsetbuttcap%
\pgfsetmiterjoin%
\definecolor{currentfill}{rgb}{0.754268,0.565033,0.211761}%
\pgfsetfillcolor{currentfill}%
\pgfsetlinewidth{0.000000pt}%
\definecolor{currentstroke}{rgb}{0.000000,0.000000,0.000000}%
\pgfsetstrokecolor{currentstroke}%
\pgfsetstrokeopacity{0.000000}%
\pgfsetdash{}{0pt}%
\pgfpathmoveto{\pgfqpoint{4.309464in}{1.318908in}}%
\pgfpathlineto{\pgfqpoint{4.318217in}{1.318908in}}%
\pgfpathlineto{\pgfqpoint{4.318217in}{0.999677in}}%
\pgfpathlineto{\pgfqpoint{4.309464in}{0.999677in}}%
\pgfpathlineto{\pgfqpoint{4.309464in}{1.318908in}}%
\pgfpathclose%
\pgfusepath{fill}%
\end{pgfscope}%
\begin{pgfscope}%
\pgfpathrectangle{\pgfqpoint{3.776708in}{0.600000in}}{\pgfqpoint{2.573292in}{2.070576in}}%
\pgfusepath{clip}%
\pgfsetbuttcap%
\pgfsetmiterjoin%
\definecolor{currentfill}{rgb}{0.754268,0.565033,0.211761}%
\pgfsetfillcolor{currentfill}%
\pgfsetlinewidth{0.000000pt}%
\definecolor{currentstroke}{rgb}{0.000000,0.000000,0.000000}%
\pgfsetstrokecolor{currentstroke}%
\pgfsetstrokeopacity{0.000000}%
\pgfsetdash{}{0pt}%
\pgfpathmoveto{\pgfqpoint{4.320406in}{1.324950in}}%
\pgfpathlineto{\pgfqpoint{4.329159in}{1.324950in}}%
\pgfpathlineto{\pgfqpoint{4.329159in}{1.019003in}}%
\pgfpathlineto{\pgfqpoint{4.320406in}{1.019003in}}%
\pgfpathlineto{\pgfqpoint{4.320406in}{1.324950in}}%
\pgfpathclose%
\pgfusepath{fill}%
\end{pgfscope}%
\begin{pgfscope}%
\pgfpathrectangle{\pgfqpoint{3.776708in}{0.600000in}}{\pgfqpoint{2.573292in}{2.070576in}}%
\pgfusepath{clip}%
\pgfsetbuttcap%
\pgfsetmiterjoin%
\definecolor{currentfill}{rgb}{0.754268,0.565033,0.211761}%
\pgfsetfillcolor{currentfill}%
\pgfsetlinewidth{0.000000pt}%
\definecolor{currentstroke}{rgb}{0.000000,0.000000,0.000000}%
\pgfsetstrokecolor{currentstroke}%
\pgfsetstrokeopacity{0.000000}%
\pgfsetdash{}{0pt}%
\pgfpathmoveto{\pgfqpoint{4.331348in}{1.320763in}}%
\pgfpathlineto{\pgfqpoint{4.340101in}{1.320763in}}%
\pgfpathlineto{\pgfqpoint{4.340101in}{1.025701in}}%
\pgfpathlineto{\pgfqpoint{4.331348in}{1.025701in}}%
\pgfpathlineto{\pgfqpoint{4.331348in}{1.320763in}}%
\pgfpathclose%
\pgfusepath{fill}%
\end{pgfscope}%
\begin{pgfscope}%
\pgfpathrectangle{\pgfqpoint{3.776708in}{0.600000in}}{\pgfqpoint{2.573292in}{2.070576in}}%
\pgfusepath{clip}%
\pgfsetbuttcap%
\pgfsetmiterjoin%
\definecolor{currentfill}{rgb}{0.754268,0.565033,0.211761}%
\pgfsetfillcolor{currentfill}%
\pgfsetlinewidth{0.000000pt}%
\definecolor{currentstroke}{rgb}{0.000000,0.000000,0.000000}%
\pgfsetstrokecolor{currentstroke}%
\pgfsetstrokeopacity{0.000000}%
\pgfsetdash{}{0pt}%
\pgfpathmoveto{\pgfqpoint{4.342289in}{1.307070in}}%
\pgfpathlineto{\pgfqpoint{4.351043in}{1.307070in}}%
\pgfpathlineto{\pgfqpoint{4.351043in}{1.025234in}}%
\pgfpathlineto{\pgfqpoint{4.342289in}{1.025234in}}%
\pgfpathlineto{\pgfqpoint{4.342289in}{1.307070in}}%
\pgfpathclose%
\pgfusepath{fill}%
\end{pgfscope}%
\begin{pgfscope}%
\pgfpathrectangle{\pgfqpoint{3.776708in}{0.600000in}}{\pgfqpoint{2.573292in}{2.070576in}}%
\pgfusepath{clip}%
\pgfsetbuttcap%
\pgfsetmiterjoin%
\definecolor{currentfill}{rgb}{0.754268,0.565033,0.211761}%
\pgfsetfillcolor{currentfill}%
\pgfsetlinewidth{0.000000pt}%
\definecolor{currentstroke}{rgb}{0.000000,0.000000,0.000000}%
\pgfsetstrokecolor{currentstroke}%
\pgfsetstrokeopacity{0.000000}%
\pgfsetdash{}{0pt}%
\pgfpathmoveto{\pgfqpoint{4.353231in}{1.297310in}}%
\pgfpathlineto{\pgfqpoint{4.361985in}{1.297310in}}%
\pgfpathlineto{\pgfqpoint{4.361985in}{1.024894in}}%
\pgfpathlineto{\pgfqpoint{4.353231in}{1.024894in}}%
\pgfpathlineto{\pgfqpoint{4.353231in}{1.297310in}}%
\pgfpathclose%
\pgfusepath{fill}%
\end{pgfscope}%
\begin{pgfscope}%
\pgfpathrectangle{\pgfqpoint{3.776708in}{0.600000in}}{\pgfqpoint{2.573292in}{2.070576in}}%
\pgfusepath{clip}%
\pgfsetbuttcap%
\pgfsetmiterjoin%
\definecolor{currentfill}{rgb}{0.754268,0.565033,0.211761}%
\pgfsetfillcolor{currentfill}%
\pgfsetlinewidth{0.000000pt}%
\definecolor{currentstroke}{rgb}{0.000000,0.000000,0.000000}%
\pgfsetstrokecolor{currentstroke}%
\pgfsetstrokeopacity{0.000000}%
\pgfsetdash{}{0pt}%
\pgfpathmoveto{\pgfqpoint{4.364173in}{1.291703in}}%
\pgfpathlineto{\pgfqpoint{4.372926in}{1.291703in}}%
\pgfpathlineto{\pgfqpoint{4.372926in}{1.020019in}}%
\pgfpathlineto{\pgfqpoint{4.364173in}{1.020019in}}%
\pgfpathlineto{\pgfqpoint{4.364173in}{1.291703in}}%
\pgfpathclose%
\pgfusepath{fill}%
\end{pgfscope}%
\begin{pgfscope}%
\pgfpathrectangle{\pgfqpoint{3.776708in}{0.600000in}}{\pgfqpoint{2.573292in}{2.070576in}}%
\pgfusepath{clip}%
\pgfsetbuttcap%
\pgfsetmiterjoin%
\definecolor{currentfill}{rgb}{0.754268,0.565033,0.211761}%
\pgfsetfillcolor{currentfill}%
\pgfsetlinewidth{0.000000pt}%
\definecolor{currentstroke}{rgb}{0.000000,0.000000,0.000000}%
\pgfsetstrokecolor{currentstroke}%
\pgfsetstrokeopacity{0.000000}%
\pgfsetdash{}{0pt}%
\pgfpathmoveto{\pgfqpoint{4.375115in}{1.295633in}}%
\pgfpathlineto{\pgfqpoint{4.383868in}{1.295633in}}%
\pgfpathlineto{\pgfqpoint{4.383868in}{1.028282in}}%
\pgfpathlineto{\pgfqpoint{4.375115in}{1.028282in}}%
\pgfpathlineto{\pgfqpoint{4.375115in}{1.295633in}}%
\pgfpathclose%
\pgfusepath{fill}%
\end{pgfscope}%
\begin{pgfscope}%
\pgfpathrectangle{\pgfqpoint{3.776708in}{0.600000in}}{\pgfqpoint{2.573292in}{2.070576in}}%
\pgfusepath{clip}%
\pgfsetbuttcap%
\pgfsetmiterjoin%
\definecolor{currentfill}{rgb}{0.754268,0.565033,0.211761}%
\pgfsetfillcolor{currentfill}%
\pgfsetlinewidth{0.000000pt}%
\definecolor{currentstroke}{rgb}{0.000000,0.000000,0.000000}%
\pgfsetstrokecolor{currentstroke}%
\pgfsetstrokeopacity{0.000000}%
\pgfsetdash{}{0pt}%
\pgfpathmoveto{\pgfqpoint{4.386057in}{1.273727in}}%
\pgfpathlineto{\pgfqpoint{4.394810in}{1.273727in}}%
\pgfpathlineto{\pgfqpoint{4.394810in}{1.016937in}}%
\pgfpathlineto{\pgfqpoint{4.386057in}{1.016937in}}%
\pgfpathlineto{\pgfqpoint{4.386057in}{1.273727in}}%
\pgfpathclose%
\pgfusepath{fill}%
\end{pgfscope}%
\begin{pgfscope}%
\pgfpathrectangle{\pgfqpoint{3.776708in}{0.600000in}}{\pgfqpoint{2.573292in}{2.070576in}}%
\pgfusepath{clip}%
\pgfsetbuttcap%
\pgfsetmiterjoin%
\definecolor{currentfill}{rgb}{0.754268,0.565033,0.211761}%
\pgfsetfillcolor{currentfill}%
\pgfsetlinewidth{0.000000pt}%
\definecolor{currentstroke}{rgb}{0.000000,0.000000,0.000000}%
\pgfsetstrokecolor{currentstroke}%
\pgfsetstrokeopacity{0.000000}%
\pgfsetdash{}{0pt}%
\pgfpathmoveto{\pgfqpoint{4.396998in}{1.282668in}}%
\pgfpathlineto{\pgfqpoint{4.405752in}{1.282668in}}%
\pgfpathlineto{\pgfqpoint{4.405752in}{1.022104in}}%
\pgfpathlineto{\pgfqpoint{4.396998in}{1.022104in}}%
\pgfpathlineto{\pgfqpoint{4.396998in}{1.282668in}}%
\pgfpathclose%
\pgfusepath{fill}%
\end{pgfscope}%
\begin{pgfscope}%
\pgfpathrectangle{\pgfqpoint{3.776708in}{0.600000in}}{\pgfqpoint{2.573292in}{2.070576in}}%
\pgfusepath{clip}%
\pgfsetbuttcap%
\pgfsetmiterjoin%
\definecolor{currentfill}{rgb}{0.754268,0.565033,0.211761}%
\pgfsetfillcolor{currentfill}%
\pgfsetlinewidth{0.000000pt}%
\definecolor{currentstroke}{rgb}{0.000000,0.000000,0.000000}%
\pgfsetstrokecolor{currentstroke}%
\pgfsetstrokeopacity{0.000000}%
\pgfsetdash{}{0pt}%
\pgfpathmoveto{\pgfqpoint{4.407940in}{1.281707in}}%
\pgfpathlineto{\pgfqpoint{4.416694in}{1.281707in}}%
\pgfpathlineto{\pgfqpoint{4.416694in}{1.013127in}}%
\pgfpathlineto{\pgfqpoint{4.407940in}{1.013127in}}%
\pgfpathlineto{\pgfqpoint{4.407940in}{1.281707in}}%
\pgfpathclose%
\pgfusepath{fill}%
\end{pgfscope}%
\begin{pgfscope}%
\pgfpathrectangle{\pgfqpoint{3.776708in}{0.600000in}}{\pgfqpoint{2.573292in}{2.070576in}}%
\pgfusepath{clip}%
\pgfsetbuttcap%
\pgfsetmiterjoin%
\definecolor{currentfill}{rgb}{0.754268,0.565033,0.211761}%
\pgfsetfillcolor{currentfill}%
\pgfsetlinewidth{0.000000pt}%
\definecolor{currentstroke}{rgb}{0.000000,0.000000,0.000000}%
\pgfsetstrokecolor{currentstroke}%
\pgfsetstrokeopacity{0.000000}%
\pgfsetdash{}{0pt}%
\pgfpathmoveto{\pgfqpoint{4.418882in}{1.287423in}}%
\pgfpathlineto{\pgfqpoint{4.427635in}{1.287423in}}%
\pgfpathlineto{\pgfqpoint{4.427635in}{1.013751in}}%
\pgfpathlineto{\pgfqpoint{4.418882in}{1.013751in}}%
\pgfpathlineto{\pgfqpoint{4.418882in}{1.287423in}}%
\pgfpathclose%
\pgfusepath{fill}%
\end{pgfscope}%
\begin{pgfscope}%
\pgfpathrectangle{\pgfqpoint{3.776708in}{0.600000in}}{\pgfqpoint{2.573292in}{2.070576in}}%
\pgfusepath{clip}%
\pgfsetbuttcap%
\pgfsetmiterjoin%
\definecolor{currentfill}{rgb}{0.754268,0.565033,0.211761}%
\pgfsetfillcolor{currentfill}%
\pgfsetlinewidth{0.000000pt}%
\definecolor{currentstroke}{rgb}{0.000000,0.000000,0.000000}%
\pgfsetstrokecolor{currentstroke}%
\pgfsetstrokeopacity{0.000000}%
\pgfsetdash{}{0pt}%
\pgfpathmoveto{\pgfqpoint{4.429824in}{1.289685in}}%
\pgfpathlineto{\pgfqpoint{4.438577in}{1.289685in}}%
\pgfpathlineto{\pgfqpoint{4.438577in}{1.007618in}}%
\pgfpathlineto{\pgfqpoint{4.429824in}{1.007618in}}%
\pgfpathlineto{\pgfqpoint{4.429824in}{1.289685in}}%
\pgfpathclose%
\pgfusepath{fill}%
\end{pgfscope}%
\begin{pgfscope}%
\pgfpathrectangle{\pgfqpoint{3.776708in}{0.600000in}}{\pgfqpoint{2.573292in}{2.070576in}}%
\pgfusepath{clip}%
\pgfsetbuttcap%
\pgfsetmiterjoin%
\definecolor{currentfill}{rgb}{0.754268,0.565033,0.211761}%
\pgfsetfillcolor{currentfill}%
\pgfsetlinewidth{0.000000pt}%
\definecolor{currentstroke}{rgb}{0.000000,0.000000,0.000000}%
\pgfsetstrokecolor{currentstroke}%
\pgfsetstrokeopacity{0.000000}%
\pgfsetdash{}{0pt}%
\pgfpathmoveto{\pgfqpoint{4.440766in}{1.302714in}}%
\pgfpathlineto{\pgfqpoint{4.449519in}{1.302714in}}%
\pgfpathlineto{\pgfqpoint{4.449519in}{1.003366in}}%
\pgfpathlineto{\pgfqpoint{4.440766in}{1.003366in}}%
\pgfpathlineto{\pgfqpoint{4.440766in}{1.302714in}}%
\pgfpathclose%
\pgfusepath{fill}%
\end{pgfscope}%
\begin{pgfscope}%
\pgfpathrectangle{\pgfqpoint{3.776708in}{0.600000in}}{\pgfqpoint{2.573292in}{2.070576in}}%
\pgfusepath{clip}%
\pgfsetbuttcap%
\pgfsetmiterjoin%
\definecolor{currentfill}{rgb}{0.754268,0.565033,0.211761}%
\pgfsetfillcolor{currentfill}%
\pgfsetlinewidth{0.000000pt}%
\definecolor{currentstroke}{rgb}{0.000000,0.000000,0.000000}%
\pgfsetstrokecolor{currentstroke}%
\pgfsetstrokeopacity{0.000000}%
\pgfsetdash{}{0pt}%
\pgfpathmoveto{\pgfqpoint{4.451707in}{1.297137in}}%
\pgfpathlineto{\pgfqpoint{4.460461in}{1.297137in}}%
\pgfpathlineto{\pgfqpoint{4.460461in}{0.991505in}}%
\pgfpathlineto{\pgfqpoint{4.451707in}{0.991505in}}%
\pgfpathlineto{\pgfqpoint{4.451707in}{1.297137in}}%
\pgfpathclose%
\pgfusepath{fill}%
\end{pgfscope}%
\begin{pgfscope}%
\pgfpathrectangle{\pgfqpoint{3.776708in}{0.600000in}}{\pgfqpoint{2.573292in}{2.070576in}}%
\pgfusepath{clip}%
\pgfsetbuttcap%
\pgfsetmiterjoin%
\definecolor{currentfill}{rgb}{0.754268,0.565033,0.211761}%
\pgfsetfillcolor{currentfill}%
\pgfsetlinewidth{0.000000pt}%
\definecolor{currentstroke}{rgb}{0.000000,0.000000,0.000000}%
\pgfsetstrokecolor{currentstroke}%
\pgfsetstrokeopacity{0.000000}%
\pgfsetdash{}{0pt}%
\pgfpathmoveto{\pgfqpoint{4.462649in}{1.314509in}}%
\pgfpathlineto{\pgfqpoint{4.471403in}{1.314509in}}%
\pgfpathlineto{\pgfqpoint{4.471403in}{0.999764in}}%
\pgfpathlineto{\pgfqpoint{4.462649in}{0.999764in}}%
\pgfpathlineto{\pgfqpoint{4.462649in}{1.314509in}}%
\pgfpathclose%
\pgfusepath{fill}%
\end{pgfscope}%
\begin{pgfscope}%
\pgfpathrectangle{\pgfqpoint{3.776708in}{0.600000in}}{\pgfqpoint{2.573292in}{2.070576in}}%
\pgfusepath{clip}%
\pgfsetbuttcap%
\pgfsetmiterjoin%
\definecolor{currentfill}{rgb}{0.754268,0.565033,0.211761}%
\pgfsetfillcolor{currentfill}%
\pgfsetlinewidth{0.000000pt}%
\definecolor{currentstroke}{rgb}{0.000000,0.000000,0.000000}%
\pgfsetstrokecolor{currentstroke}%
\pgfsetstrokeopacity{0.000000}%
\pgfsetdash{}{0pt}%
\pgfpathmoveto{\pgfqpoint{4.473591in}{1.340250in}}%
\pgfpathlineto{\pgfqpoint{4.482344in}{1.340250in}}%
\pgfpathlineto{\pgfqpoint{4.482344in}{1.015027in}}%
\pgfpathlineto{\pgfqpoint{4.473591in}{1.015027in}}%
\pgfpathlineto{\pgfqpoint{4.473591in}{1.340250in}}%
\pgfpathclose%
\pgfusepath{fill}%
\end{pgfscope}%
\begin{pgfscope}%
\pgfpathrectangle{\pgfqpoint{3.776708in}{0.600000in}}{\pgfqpoint{2.573292in}{2.070576in}}%
\pgfusepath{clip}%
\pgfsetbuttcap%
\pgfsetmiterjoin%
\definecolor{currentfill}{rgb}{0.754268,0.565033,0.211761}%
\pgfsetfillcolor{currentfill}%
\pgfsetlinewidth{0.000000pt}%
\definecolor{currentstroke}{rgb}{0.000000,0.000000,0.000000}%
\pgfsetstrokecolor{currentstroke}%
\pgfsetstrokeopacity{0.000000}%
\pgfsetdash{}{0pt}%
\pgfpathmoveto{\pgfqpoint{4.484533in}{1.371343in}}%
\pgfpathlineto{\pgfqpoint{4.493286in}{1.371343in}}%
\pgfpathlineto{\pgfqpoint{4.493286in}{1.029952in}}%
\pgfpathlineto{\pgfqpoint{4.484533in}{1.029952in}}%
\pgfpathlineto{\pgfqpoint{4.484533in}{1.371343in}}%
\pgfpathclose%
\pgfusepath{fill}%
\end{pgfscope}%
\begin{pgfscope}%
\pgfpathrectangle{\pgfqpoint{3.776708in}{0.600000in}}{\pgfqpoint{2.573292in}{2.070576in}}%
\pgfusepath{clip}%
\pgfsetbuttcap%
\pgfsetmiterjoin%
\definecolor{currentfill}{rgb}{0.754268,0.565033,0.211761}%
\pgfsetfillcolor{currentfill}%
\pgfsetlinewidth{0.000000pt}%
\definecolor{currentstroke}{rgb}{0.000000,0.000000,0.000000}%
\pgfsetstrokecolor{currentstroke}%
\pgfsetstrokeopacity{0.000000}%
\pgfsetdash{}{0pt}%
\pgfpathmoveto{\pgfqpoint{4.495475in}{1.404178in}}%
\pgfpathlineto{\pgfqpoint{4.504228in}{1.404178in}}%
\pgfpathlineto{\pgfqpoint{4.504228in}{1.044800in}}%
\pgfpathlineto{\pgfqpoint{4.495475in}{1.044800in}}%
\pgfpathlineto{\pgfqpoint{4.495475in}{1.404178in}}%
\pgfpathclose%
\pgfusepath{fill}%
\end{pgfscope}%
\begin{pgfscope}%
\pgfpathrectangle{\pgfqpoint{3.776708in}{0.600000in}}{\pgfqpoint{2.573292in}{2.070576in}}%
\pgfusepath{clip}%
\pgfsetbuttcap%
\pgfsetmiterjoin%
\definecolor{currentfill}{rgb}{0.754268,0.565033,0.211761}%
\pgfsetfillcolor{currentfill}%
\pgfsetlinewidth{0.000000pt}%
\definecolor{currentstroke}{rgb}{0.000000,0.000000,0.000000}%
\pgfsetstrokecolor{currentstroke}%
\pgfsetstrokeopacity{0.000000}%
\pgfsetdash{}{0pt}%
\pgfpathmoveto{\pgfqpoint{4.506416in}{1.419491in}}%
\pgfpathlineto{\pgfqpoint{4.515170in}{1.419491in}}%
\pgfpathlineto{\pgfqpoint{4.515170in}{1.043085in}}%
\pgfpathlineto{\pgfqpoint{4.506416in}{1.043085in}}%
\pgfpathlineto{\pgfqpoint{4.506416in}{1.419491in}}%
\pgfpathclose%
\pgfusepath{fill}%
\end{pgfscope}%
\begin{pgfscope}%
\pgfpathrectangle{\pgfqpoint{3.776708in}{0.600000in}}{\pgfqpoint{2.573292in}{2.070576in}}%
\pgfusepath{clip}%
\pgfsetbuttcap%
\pgfsetmiterjoin%
\definecolor{currentfill}{rgb}{0.754268,0.565033,0.211761}%
\pgfsetfillcolor{currentfill}%
\pgfsetlinewidth{0.000000pt}%
\definecolor{currentstroke}{rgb}{0.000000,0.000000,0.000000}%
\pgfsetstrokecolor{currentstroke}%
\pgfsetstrokeopacity{0.000000}%
\pgfsetdash{}{0pt}%
\pgfpathmoveto{\pgfqpoint{4.517358in}{1.418360in}}%
\pgfpathlineto{\pgfqpoint{4.526112in}{1.418360in}}%
\pgfpathlineto{\pgfqpoint{4.526112in}{1.051196in}}%
\pgfpathlineto{\pgfqpoint{4.517358in}{1.051196in}}%
\pgfpathlineto{\pgfqpoint{4.517358in}{1.418360in}}%
\pgfpathclose%
\pgfusepath{fill}%
\end{pgfscope}%
\begin{pgfscope}%
\pgfpathrectangle{\pgfqpoint{3.776708in}{0.600000in}}{\pgfqpoint{2.573292in}{2.070576in}}%
\pgfusepath{clip}%
\pgfsetbuttcap%
\pgfsetmiterjoin%
\definecolor{currentfill}{rgb}{0.754268,0.565033,0.211761}%
\pgfsetfillcolor{currentfill}%
\pgfsetlinewidth{0.000000pt}%
\definecolor{currentstroke}{rgb}{0.000000,0.000000,0.000000}%
\pgfsetstrokecolor{currentstroke}%
\pgfsetstrokeopacity{0.000000}%
\pgfsetdash{}{0pt}%
\pgfpathmoveto{\pgfqpoint{4.528300in}{1.413920in}}%
\pgfpathlineto{\pgfqpoint{4.537053in}{1.413920in}}%
\pgfpathlineto{\pgfqpoint{4.537053in}{1.046473in}}%
\pgfpathlineto{\pgfqpoint{4.528300in}{1.046473in}}%
\pgfpathlineto{\pgfqpoint{4.528300in}{1.413920in}}%
\pgfpathclose%
\pgfusepath{fill}%
\end{pgfscope}%
\begin{pgfscope}%
\pgfpathrectangle{\pgfqpoint{3.776708in}{0.600000in}}{\pgfqpoint{2.573292in}{2.070576in}}%
\pgfusepath{clip}%
\pgfsetbuttcap%
\pgfsetmiterjoin%
\definecolor{currentfill}{rgb}{0.754268,0.565033,0.211761}%
\pgfsetfillcolor{currentfill}%
\pgfsetlinewidth{0.000000pt}%
\definecolor{currentstroke}{rgb}{0.000000,0.000000,0.000000}%
\pgfsetstrokecolor{currentstroke}%
\pgfsetstrokeopacity{0.000000}%
\pgfsetdash{}{0pt}%
\pgfpathmoveto{\pgfqpoint{4.539242in}{1.408323in}}%
\pgfpathlineto{\pgfqpoint{4.547995in}{1.408323in}}%
\pgfpathlineto{\pgfqpoint{4.547995in}{1.008380in}}%
\pgfpathlineto{\pgfqpoint{4.539242in}{1.008380in}}%
\pgfpathlineto{\pgfqpoint{4.539242in}{1.408323in}}%
\pgfpathclose%
\pgfusepath{fill}%
\end{pgfscope}%
\begin{pgfscope}%
\pgfpathrectangle{\pgfqpoint{3.776708in}{0.600000in}}{\pgfqpoint{2.573292in}{2.070576in}}%
\pgfusepath{clip}%
\pgfsetbuttcap%
\pgfsetmiterjoin%
\definecolor{currentfill}{rgb}{0.754268,0.565033,0.211761}%
\pgfsetfillcolor{currentfill}%
\pgfsetlinewidth{0.000000pt}%
\definecolor{currentstroke}{rgb}{0.000000,0.000000,0.000000}%
\pgfsetstrokecolor{currentstroke}%
\pgfsetstrokeopacity{0.000000}%
\pgfsetdash{}{0pt}%
\pgfpathmoveto{\pgfqpoint{4.550183in}{1.404137in}}%
\pgfpathlineto{\pgfqpoint{4.558937in}{1.404137in}}%
\pgfpathlineto{\pgfqpoint{4.558937in}{0.991181in}}%
\pgfpathlineto{\pgfqpoint{4.550183in}{0.991181in}}%
\pgfpathlineto{\pgfqpoint{4.550183in}{1.404137in}}%
\pgfpathclose%
\pgfusepath{fill}%
\end{pgfscope}%
\begin{pgfscope}%
\pgfpathrectangle{\pgfqpoint{3.776708in}{0.600000in}}{\pgfqpoint{2.573292in}{2.070576in}}%
\pgfusepath{clip}%
\pgfsetbuttcap%
\pgfsetmiterjoin%
\definecolor{currentfill}{rgb}{0.754268,0.565033,0.211761}%
\pgfsetfillcolor{currentfill}%
\pgfsetlinewidth{0.000000pt}%
\definecolor{currentstroke}{rgb}{0.000000,0.000000,0.000000}%
\pgfsetstrokecolor{currentstroke}%
\pgfsetstrokeopacity{0.000000}%
\pgfsetdash{}{0pt}%
\pgfpathmoveto{\pgfqpoint{4.561125in}{1.402510in}}%
\pgfpathlineto{\pgfqpoint{4.569879in}{1.402510in}}%
\pgfpathlineto{\pgfqpoint{4.569879in}{0.967587in}}%
\pgfpathlineto{\pgfqpoint{4.561125in}{0.967587in}}%
\pgfpathlineto{\pgfqpoint{4.561125in}{1.402510in}}%
\pgfpathclose%
\pgfusepath{fill}%
\end{pgfscope}%
\begin{pgfscope}%
\pgfpathrectangle{\pgfqpoint{3.776708in}{0.600000in}}{\pgfqpoint{2.573292in}{2.070576in}}%
\pgfusepath{clip}%
\pgfsetbuttcap%
\pgfsetmiterjoin%
\definecolor{currentfill}{rgb}{0.754268,0.565033,0.211761}%
\pgfsetfillcolor{currentfill}%
\pgfsetlinewidth{0.000000pt}%
\definecolor{currentstroke}{rgb}{0.000000,0.000000,0.000000}%
\pgfsetstrokecolor{currentstroke}%
\pgfsetstrokeopacity{0.000000}%
\pgfsetdash{}{0pt}%
\pgfpathmoveto{\pgfqpoint{4.572067in}{1.406236in}}%
\pgfpathlineto{\pgfqpoint{4.580821in}{1.406236in}}%
\pgfpathlineto{\pgfqpoint{4.580821in}{0.962436in}}%
\pgfpathlineto{\pgfqpoint{4.572067in}{0.962436in}}%
\pgfpathlineto{\pgfqpoint{4.572067in}{1.406236in}}%
\pgfpathclose%
\pgfusepath{fill}%
\end{pgfscope}%
\begin{pgfscope}%
\pgfpathrectangle{\pgfqpoint{3.776708in}{0.600000in}}{\pgfqpoint{2.573292in}{2.070576in}}%
\pgfusepath{clip}%
\pgfsetbuttcap%
\pgfsetmiterjoin%
\definecolor{currentfill}{rgb}{0.754268,0.565033,0.211761}%
\pgfsetfillcolor{currentfill}%
\pgfsetlinewidth{0.000000pt}%
\definecolor{currentstroke}{rgb}{0.000000,0.000000,0.000000}%
\pgfsetstrokecolor{currentstroke}%
\pgfsetstrokeopacity{0.000000}%
\pgfsetdash{}{0pt}%
\pgfpathmoveto{\pgfqpoint{4.583009in}{1.400320in}}%
\pgfpathlineto{\pgfqpoint{4.591762in}{1.400320in}}%
\pgfpathlineto{\pgfqpoint{4.591762in}{0.956081in}}%
\pgfpathlineto{\pgfqpoint{4.583009in}{0.956081in}}%
\pgfpathlineto{\pgfqpoint{4.583009in}{1.400320in}}%
\pgfpathclose%
\pgfusepath{fill}%
\end{pgfscope}%
\begin{pgfscope}%
\pgfpathrectangle{\pgfqpoint{3.776708in}{0.600000in}}{\pgfqpoint{2.573292in}{2.070576in}}%
\pgfusepath{clip}%
\pgfsetbuttcap%
\pgfsetmiterjoin%
\definecolor{currentfill}{rgb}{0.754268,0.565033,0.211761}%
\pgfsetfillcolor{currentfill}%
\pgfsetlinewidth{0.000000pt}%
\definecolor{currentstroke}{rgb}{0.000000,0.000000,0.000000}%
\pgfsetstrokecolor{currentstroke}%
\pgfsetstrokeopacity{0.000000}%
\pgfsetdash{}{0pt}%
\pgfpathmoveto{\pgfqpoint{4.593951in}{1.405917in}}%
\pgfpathlineto{\pgfqpoint{4.602704in}{1.405917in}}%
\pgfpathlineto{\pgfqpoint{4.602704in}{0.966407in}}%
\pgfpathlineto{\pgfqpoint{4.593951in}{0.966407in}}%
\pgfpathlineto{\pgfqpoint{4.593951in}{1.405917in}}%
\pgfpathclose%
\pgfusepath{fill}%
\end{pgfscope}%
\begin{pgfscope}%
\pgfpathrectangle{\pgfqpoint{3.776708in}{0.600000in}}{\pgfqpoint{2.573292in}{2.070576in}}%
\pgfusepath{clip}%
\pgfsetbuttcap%
\pgfsetmiterjoin%
\definecolor{currentfill}{rgb}{0.754268,0.565033,0.211761}%
\pgfsetfillcolor{currentfill}%
\pgfsetlinewidth{0.000000pt}%
\definecolor{currentstroke}{rgb}{0.000000,0.000000,0.000000}%
\pgfsetstrokecolor{currentstroke}%
\pgfsetstrokeopacity{0.000000}%
\pgfsetdash{}{0pt}%
\pgfpathmoveto{\pgfqpoint{4.604892in}{1.406544in}}%
\pgfpathlineto{\pgfqpoint{4.613646in}{1.406544in}}%
\pgfpathlineto{\pgfqpoint{4.613646in}{0.981915in}}%
\pgfpathlineto{\pgfqpoint{4.604892in}{0.981915in}}%
\pgfpathlineto{\pgfqpoint{4.604892in}{1.406544in}}%
\pgfpathclose%
\pgfusepath{fill}%
\end{pgfscope}%
\begin{pgfscope}%
\pgfpathrectangle{\pgfqpoint{3.776708in}{0.600000in}}{\pgfqpoint{2.573292in}{2.070576in}}%
\pgfusepath{clip}%
\pgfsetbuttcap%
\pgfsetmiterjoin%
\definecolor{currentfill}{rgb}{0.754268,0.565033,0.211761}%
\pgfsetfillcolor{currentfill}%
\pgfsetlinewidth{0.000000pt}%
\definecolor{currentstroke}{rgb}{0.000000,0.000000,0.000000}%
\pgfsetstrokecolor{currentstroke}%
\pgfsetstrokeopacity{0.000000}%
\pgfsetdash{}{0pt}%
\pgfpathmoveto{\pgfqpoint{4.615834in}{1.418203in}}%
\pgfpathlineto{\pgfqpoint{4.624588in}{1.418203in}}%
\pgfpathlineto{\pgfqpoint{4.624588in}{1.008783in}}%
\pgfpathlineto{\pgfqpoint{4.615834in}{1.008783in}}%
\pgfpathlineto{\pgfqpoint{4.615834in}{1.418203in}}%
\pgfpathclose%
\pgfusepath{fill}%
\end{pgfscope}%
\begin{pgfscope}%
\pgfpathrectangle{\pgfqpoint{3.776708in}{0.600000in}}{\pgfqpoint{2.573292in}{2.070576in}}%
\pgfusepath{clip}%
\pgfsetbuttcap%
\pgfsetmiterjoin%
\definecolor{currentfill}{rgb}{0.754268,0.565033,0.211761}%
\pgfsetfillcolor{currentfill}%
\pgfsetlinewidth{0.000000pt}%
\definecolor{currentstroke}{rgb}{0.000000,0.000000,0.000000}%
\pgfsetstrokecolor{currentstroke}%
\pgfsetstrokeopacity{0.000000}%
\pgfsetdash{}{0pt}%
\pgfpathmoveto{\pgfqpoint{4.626776in}{1.428266in}}%
\pgfpathlineto{\pgfqpoint{4.635530in}{1.428266in}}%
\pgfpathlineto{\pgfqpoint{4.635530in}{1.044583in}}%
\pgfpathlineto{\pgfqpoint{4.626776in}{1.044583in}}%
\pgfpathlineto{\pgfqpoint{4.626776in}{1.428266in}}%
\pgfpathclose%
\pgfusepath{fill}%
\end{pgfscope}%
\begin{pgfscope}%
\pgfpathrectangle{\pgfqpoint{3.776708in}{0.600000in}}{\pgfqpoint{2.573292in}{2.070576in}}%
\pgfusepath{clip}%
\pgfsetbuttcap%
\pgfsetmiterjoin%
\definecolor{currentfill}{rgb}{0.754268,0.565033,0.211761}%
\pgfsetfillcolor{currentfill}%
\pgfsetlinewidth{0.000000pt}%
\definecolor{currentstroke}{rgb}{0.000000,0.000000,0.000000}%
\pgfsetstrokecolor{currentstroke}%
\pgfsetstrokeopacity{0.000000}%
\pgfsetdash{}{0pt}%
\pgfpathmoveto{\pgfqpoint{4.637718in}{1.435518in}}%
\pgfpathlineto{\pgfqpoint{4.646471in}{1.435518in}}%
\pgfpathlineto{\pgfqpoint{4.646471in}{1.072128in}}%
\pgfpathlineto{\pgfqpoint{4.637718in}{1.072128in}}%
\pgfpathlineto{\pgfqpoint{4.637718in}{1.435518in}}%
\pgfpathclose%
\pgfusepath{fill}%
\end{pgfscope}%
\begin{pgfscope}%
\pgfpathrectangle{\pgfqpoint{3.776708in}{0.600000in}}{\pgfqpoint{2.573292in}{2.070576in}}%
\pgfusepath{clip}%
\pgfsetbuttcap%
\pgfsetmiterjoin%
\definecolor{currentfill}{rgb}{0.754268,0.565033,0.211761}%
\pgfsetfillcolor{currentfill}%
\pgfsetlinewidth{0.000000pt}%
\definecolor{currentstroke}{rgb}{0.000000,0.000000,0.000000}%
\pgfsetstrokecolor{currentstroke}%
\pgfsetstrokeopacity{0.000000}%
\pgfsetdash{}{0pt}%
\pgfpathmoveto{\pgfqpoint{4.648660in}{1.445060in}}%
\pgfpathlineto{\pgfqpoint{4.657413in}{1.445060in}}%
\pgfpathlineto{\pgfqpoint{4.657413in}{1.088690in}}%
\pgfpathlineto{\pgfqpoint{4.648660in}{1.088690in}}%
\pgfpathlineto{\pgfqpoint{4.648660in}{1.445060in}}%
\pgfpathclose%
\pgfusepath{fill}%
\end{pgfscope}%
\begin{pgfscope}%
\pgfpathrectangle{\pgfqpoint{3.776708in}{0.600000in}}{\pgfqpoint{2.573292in}{2.070576in}}%
\pgfusepath{clip}%
\pgfsetbuttcap%
\pgfsetmiterjoin%
\definecolor{currentfill}{rgb}{0.754268,0.565033,0.211761}%
\pgfsetfillcolor{currentfill}%
\pgfsetlinewidth{0.000000pt}%
\definecolor{currentstroke}{rgb}{0.000000,0.000000,0.000000}%
\pgfsetstrokecolor{currentstroke}%
\pgfsetstrokeopacity{0.000000}%
\pgfsetdash{}{0pt}%
\pgfpathmoveto{\pgfqpoint{4.659601in}{1.450601in}}%
\pgfpathlineto{\pgfqpoint{4.668355in}{1.450601in}}%
\pgfpathlineto{\pgfqpoint{4.668355in}{1.108846in}}%
\pgfpathlineto{\pgfqpoint{4.659601in}{1.108846in}}%
\pgfpathlineto{\pgfqpoint{4.659601in}{1.450601in}}%
\pgfpathclose%
\pgfusepath{fill}%
\end{pgfscope}%
\begin{pgfscope}%
\pgfpathrectangle{\pgfqpoint{3.776708in}{0.600000in}}{\pgfqpoint{2.573292in}{2.070576in}}%
\pgfusepath{clip}%
\pgfsetbuttcap%
\pgfsetmiterjoin%
\definecolor{currentfill}{rgb}{0.754268,0.565033,0.211761}%
\pgfsetfillcolor{currentfill}%
\pgfsetlinewidth{0.000000pt}%
\definecolor{currentstroke}{rgb}{0.000000,0.000000,0.000000}%
\pgfsetstrokecolor{currentstroke}%
\pgfsetstrokeopacity{0.000000}%
\pgfsetdash{}{0pt}%
\pgfpathmoveto{\pgfqpoint{4.670543in}{1.458635in}}%
\pgfpathlineto{\pgfqpoint{4.679297in}{1.458635in}}%
\pgfpathlineto{\pgfqpoint{4.679297in}{1.127672in}}%
\pgfpathlineto{\pgfqpoint{4.670543in}{1.127672in}}%
\pgfpathlineto{\pgfqpoint{4.670543in}{1.458635in}}%
\pgfpathclose%
\pgfusepath{fill}%
\end{pgfscope}%
\begin{pgfscope}%
\pgfpathrectangle{\pgfqpoint{3.776708in}{0.600000in}}{\pgfqpoint{2.573292in}{2.070576in}}%
\pgfusepath{clip}%
\pgfsetbuttcap%
\pgfsetmiterjoin%
\definecolor{currentfill}{rgb}{0.754268,0.565033,0.211761}%
\pgfsetfillcolor{currentfill}%
\pgfsetlinewidth{0.000000pt}%
\definecolor{currentstroke}{rgb}{0.000000,0.000000,0.000000}%
\pgfsetstrokecolor{currentstroke}%
\pgfsetstrokeopacity{0.000000}%
\pgfsetdash{}{0pt}%
\pgfpathmoveto{\pgfqpoint{4.681485in}{1.454787in}}%
\pgfpathlineto{\pgfqpoint{4.690239in}{1.454787in}}%
\pgfpathlineto{\pgfqpoint{4.690239in}{1.135375in}}%
\pgfpathlineto{\pgfqpoint{4.681485in}{1.135375in}}%
\pgfpathlineto{\pgfqpoint{4.681485in}{1.454787in}}%
\pgfpathclose%
\pgfusepath{fill}%
\end{pgfscope}%
\begin{pgfscope}%
\pgfpathrectangle{\pgfqpoint{3.776708in}{0.600000in}}{\pgfqpoint{2.573292in}{2.070576in}}%
\pgfusepath{clip}%
\pgfsetbuttcap%
\pgfsetmiterjoin%
\definecolor{currentfill}{rgb}{0.754268,0.565033,0.211761}%
\pgfsetfillcolor{currentfill}%
\pgfsetlinewidth{0.000000pt}%
\definecolor{currentstroke}{rgb}{0.000000,0.000000,0.000000}%
\pgfsetstrokecolor{currentstroke}%
\pgfsetstrokeopacity{0.000000}%
\pgfsetdash{}{0pt}%
\pgfpathmoveto{\pgfqpoint{4.692427in}{1.436143in}}%
\pgfpathlineto{\pgfqpoint{4.701180in}{1.436143in}}%
\pgfpathlineto{\pgfqpoint{4.701180in}{1.132916in}}%
\pgfpathlineto{\pgfqpoint{4.692427in}{1.132916in}}%
\pgfpathlineto{\pgfqpoint{4.692427in}{1.436143in}}%
\pgfpathclose%
\pgfusepath{fill}%
\end{pgfscope}%
\begin{pgfscope}%
\pgfpathrectangle{\pgfqpoint{3.776708in}{0.600000in}}{\pgfqpoint{2.573292in}{2.070576in}}%
\pgfusepath{clip}%
\pgfsetbuttcap%
\pgfsetmiterjoin%
\definecolor{currentfill}{rgb}{0.754268,0.565033,0.211761}%
\pgfsetfillcolor{currentfill}%
\pgfsetlinewidth{0.000000pt}%
\definecolor{currentstroke}{rgb}{0.000000,0.000000,0.000000}%
\pgfsetstrokecolor{currentstroke}%
\pgfsetstrokeopacity{0.000000}%
\pgfsetdash{}{0pt}%
\pgfpathmoveto{\pgfqpoint{4.703369in}{1.417621in}}%
\pgfpathlineto{\pgfqpoint{4.712122in}{1.417621in}}%
\pgfpathlineto{\pgfqpoint{4.712122in}{1.129316in}}%
\pgfpathlineto{\pgfqpoint{4.703369in}{1.129316in}}%
\pgfpathlineto{\pgfqpoint{4.703369in}{1.417621in}}%
\pgfpathclose%
\pgfusepath{fill}%
\end{pgfscope}%
\begin{pgfscope}%
\pgfpathrectangle{\pgfqpoint{3.776708in}{0.600000in}}{\pgfqpoint{2.573292in}{2.070576in}}%
\pgfusepath{clip}%
\pgfsetbuttcap%
\pgfsetmiterjoin%
\definecolor{currentfill}{rgb}{0.754268,0.565033,0.211761}%
\pgfsetfillcolor{currentfill}%
\pgfsetlinewidth{0.000000pt}%
\definecolor{currentstroke}{rgb}{0.000000,0.000000,0.000000}%
\pgfsetstrokecolor{currentstroke}%
\pgfsetstrokeopacity{0.000000}%
\pgfsetdash{}{0pt}%
\pgfpathmoveto{\pgfqpoint{4.714310in}{1.405070in}}%
\pgfpathlineto{\pgfqpoint{4.723064in}{1.405070in}}%
\pgfpathlineto{\pgfqpoint{4.723064in}{1.134054in}}%
\pgfpathlineto{\pgfqpoint{4.714310in}{1.134054in}}%
\pgfpathlineto{\pgfqpoint{4.714310in}{1.405070in}}%
\pgfpathclose%
\pgfusepath{fill}%
\end{pgfscope}%
\begin{pgfscope}%
\pgfpathrectangle{\pgfqpoint{3.776708in}{0.600000in}}{\pgfqpoint{2.573292in}{2.070576in}}%
\pgfusepath{clip}%
\pgfsetbuttcap%
\pgfsetmiterjoin%
\definecolor{currentfill}{rgb}{0.754268,0.565033,0.211761}%
\pgfsetfillcolor{currentfill}%
\pgfsetlinewidth{0.000000pt}%
\definecolor{currentstroke}{rgb}{0.000000,0.000000,0.000000}%
\pgfsetstrokecolor{currentstroke}%
\pgfsetstrokeopacity{0.000000}%
\pgfsetdash{}{0pt}%
\pgfpathmoveto{\pgfqpoint{4.725252in}{1.388708in}}%
\pgfpathlineto{\pgfqpoint{4.734006in}{1.388708in}}%
\pgfpathlineto{\pgfqpoint{4.734006in}{1.133713in}}%
\pgfpathlineto{\pgfqpoint{4.725252in}{1.133713in}}%
\pgfpathlineto{\pgfqpoint{4.725252in}{1.388708in}}%
\pgfpathclose%
\pgfusepath{fill}%
\end{pgfscope}%
\begin{pgfscope}%
\pgfpathrectangle{\pgfqpoint{3.776708in}{0.600000in}}{\pgfqpoint{2.573292in}{2.070576in}}%
\pgfusepath{clip}%
\pgfsetbuttcap%
\pgfsetmiterjoin%
\definecolor{currentfill}{rgb}{0.754268,0.565033,0.211761}%
\pgfsetfillcolor{currentfill}%
\pgfsetlinewidth{0.000000pt}%
\definecolor{currentstroke}{rgb}{0.000000,0.000000,0.000000}%
\pgfsetstrokecolor{currentstroke}%
\pgfsetstrokeopacity{0.000000}%
\pgfsetdash{}{0pt}%
\pgfpathmoveto{\pgfqpoint{4.736194in}{1.379497in}}%
\pgfpathlineto{\pgfqpoint{4.744948in}{1.379497in}}%
\pgfpathlineto{\pgfqpoint{4.744948in}{1.153631in}}%
\pgfpathlineto{\pgfqpoint{4.736194in}{1.153631in}}%
\pgfpathlineto{\pgfqpoint{4.736194in}{1.379497in}}%
\pgfpathclose%
\pgfusepath{fill}%
\end{pgfscope}%
\begin{pgfscope}%
\pgfpathrectangle{\pgfqpoint{3.776708in}{0.600000in}}{\pgfqpoint{2.573292in}{2.070576in}}%
\pgfusepath{clip}%
\pgfsetbuttcap%
\pgfsetmiterjoin%
\definecolor{currentfill}{rgb}{0.754268,0.565033,0.211761}%
\pgfsetfillcolor{currentfill}%
\pgfsetlinewidth{0.000000pt}%
\definecolor{currentstroke}{rgb}{0.000000,0.000000,0.000000}%
\pgfsetstrokecolor{currentstroke}%
\pgfsetstrokeopacity{0.000000}%
\pgfsetdash{}{0pt}%
\pgfpathmoveto{\pgfqpoint{4.747136in}{1.362904in}}%
\pgfpathlineto{\pgfqpoint{4.755889in}{1.362904in}}%
\pgfpathlineto{\pgfqpoint{4.755889in}{1.149792in}}%
\pgfpathlineto{\pgfqpoint{4.747136in}{1.149792in}}%
\pgfpathlineto{\pgfqpoint{4.747136in}{1.362904in}}%
\pgfpathclose%
\pgfusepath{fill}%
\end{pgfscope}%
\begin{pgfscope}%
\pgfpathrectangle{\pgfqpoint{3.776708in}{0.600000in}}{\pgfqpoint{2.573292in}{2.070576in}}%
\pgfusepath{clip}%
\pgfsetbuttcap%
\pgfsetmiterjoin%
\definecolor{currentfill}{rgb}{0.754268,0.565033,0.211761}%
\pgfsetfillcolor{currentfill}%
\pgfsetlinewidth{0.000000pt}%
\definecolor{currentstroke}{rgb}{0.000000,0.000000,0.000000}%
\pgfsetstrokecolor{currentstroke}%
\pgfsetstrokeopacity{0.000000}%
\pgfsetdash{}{0pt}%
\pgfpathmoveto{\pgfqpoint{4.758078in}{1.347682in}}%
\pgfpathlineto{\pgfqpoint{4.766831in}{1.347682in}}%
\pgfpathlineto{\pgfqpoint{4.766831in}{1.153494in}}%
\pgfpathlineto{\pgfqpoint{4.758078in}{1.153494in}}%
\pgfpathlineto{\pgfqpoint{4.758078in}{1.347682in}}%
\pgfpathclose%
\pgfusepath{fill}%
\end{pgfscope}%
\begin{pgfscope}%
\pgfpathrectangle{\pgfqpoint{3.776708in}{0.600000in}}{\pgfqpoint{2.573292in}{2.070576in}}%
\pgfusepath{clip}%
\pgfsetbuttcap%
\pgfsetmiterjoin%
\definecolor{currentfill}{rgb}{0.754268,0.565033,0.211761}%
\pgfsetfillcolor{currentfill}%
\pgfsetlinewidth{0.000000pt}%
\definecolor{currentstroke}{rgb}{0.000000,0.000000,0.000000}%
\pgfsetstrokecolor{currentstroke}%
\pgfsetstrokeopacity{0.000000}%
\pgfsetdash{}{0pt}%
\pgfpathmoveto{\pgfqpoint{4.769019in}{1.333791in}}%
\pgfpathlineto{\pgfqpoint{4.777773in}{1.333791in}}%
\pgfpathlineto{\pgfqpoint{4.777773in}{1.161431in}}%
\pgfpathlineto{\pgfqpoint{4.769019in}{1.161431in}}%
\pgfpathlineto{\pgfqpoint{4.769019in}{1.333791in}}%
\pgfpathclose%
\pgfusepath{fill}%
\end{pgfscope}%
\begin{pgfscope}%
\pgfpathrectangle{\pgfqpoint{3.776708in}{0.600000in}}{\pgfqpoint{2.573292in}{2.070576in}}%
\pgfusepath{clip}%
\pgfsetbuttcap%
\pgfsetmiterjoin%
\definecolor{currentfill}{rgb}{0.754268,0.565033,0.211761}%
\pgfsetfillcolor{currentfill}%
\pgfsetlinewidth{0.000000pt}%
\definecolor{currentstroke}{rgb}{0.000000,0.000000,0.000000}%
\pgfsetstrokecolor{currentstroke}%
\pgfsetstrokeopacity{0.000000}%
\pgfsetdash{}{0pt}%
\pgfpathmoveto{\pgfqpoint{4.779961in}{1.325206in}}%
\pgfpathlineto{\pgfqpoint{4.788715in}{1.325206in}}%
\pgfpathlineto{\pgfqpoint{4.788715in}{1.179255in}}%
\pgfpathlineto{\pgfqpoint{4.779961in}{1.179255in}}%
\pgfpathlineto{\pgfqpoint{4.779961in}{1.325206in}}%
\pgfpathclose%
\pgfusepath{fill}%
\end{pgfscope}%
\begin{pgfscope}%
\pgfpathrectangle{\pgfqpoint{3.776708in}{0.600000in}}{\pgfqpoint{2.573292in}{2.070576in}}%
\pgfusepath{clip}%
\pgfsetbuttcap%
\pgfsetmiterjoin%
\definecolor{currentfill}{rgb}{0.754268,0.565033,0.211761}%
\pgfsetfillcolor{currentfill}%
\pgfsetlinewidth{0.000000pt}%
\definecolor{currentstroke}{rgb}{0.000000,0.000000,0.000000}%
\pgfsetstrokecolor{currentstroke}%
\pgfsetstrokeopacity{0.000000}%
\pgfsetdash{}{0pt}%
\pgfpathmoveto{\pgfqpoint{4.790903in}{1.314617in}}%
\pgfpathlineto{\pgfqpoint{4.799657in}{1.314617in}}%
\pgfpathlineto{\pgfqpoint{4.799657in}{1.187186in}}%
\pgfpathlineto{\pgfqpoint{4.790903in}{1.187186in}}%
\pgfpathlineto{\pgfqpoint{4.790903in}{1.314617in}}%
\pgfpathclose%
\pgfusepath{fill}%
\end{pgfscope}%
\begin{pgfscope}%
\pgfpathrectangle{\pgfqpoint{3.776708in}{0.600000in}}{\pgfqpoint{2.573292in}{2.070576in}}%
\pgfusepath{clip}%
\pgfsetbuttcap%
\pgfsetmiterjoin%
\definecolor{currentfill}{rgb}{0.754268,0.565033,0.211761}%
\pgfsetfillcolor{currentfill}%
\pgfsetlinewidth{0.000000pt}%
\definecolor{currentstroke}{rgb}{0.000000,0.000000,0.000000}%
\pgfsetstrokecolor{currentstroke}%
\pgfsetstrokeopacity{0.000000}%
\pgfsetdash{}{0pt}%
\pgfpathmoveto{\pgfqpoint{4.801845in}{1.304450in}}%
\pgfpathlineto{\pgfqpoint{4.810598in}{1.304450in}}%
\pgfpathlineto{\pgfqpoint{4.810598in}{1.198686in}}%
\pgfpathlineto{\pgfqpoint{4.801845in}{1.198686in}}%
\pgfpathlineto{\pgfqpoint{4.801845in}{1.304450in}}%
\pgfpathclose%
\pgfusepath{fill}%
\end{pgfscope}%
\begin{pgfscope}%
\pgfpathrectangle{\pgfqpoint{3.776708in}{0.600000in}}{\pgfqpoint{2.573292in}{2.070576in}}%
\pgfusepath{clip}%
\pgfsetbuttcap%
\pgfsetmiterjoin%
\definecolor{currentfill}{rgb}{0.754268,0.565033,0.211761}%
\pgfsetfillcolor{currentfill}%
\pgfsetlinewidth{0.000000pt}%
\definecolor{currentstroke}{rgb}{0.000000,0.000000,0.000000}%
\pgfsetstrokecolor{currentstroke}%
\pgfsetstrokeopacity{0.000000}%
\pgfsetdash{}{0pt}%
\pgfpathmoveto{\pgfqpoint{4.812787in}{1.289406in}}%
\pgfpathlineto{\pgfqpoint{4.821540in}{1.289406in}}%
\pgfpathlineto{\pgfqpoint{4.821540in}{1.204689in}}%
\pgfpathlineto{\pgfqpoint{4.812787in}{1.204689in}}%
\pgfpathlineto{\pgfqpoint{4.812787in}{1.289406in}}%
\pgfpathclose%
\pgfusepath{fill}%
\end{pgfscope}%
\begin{pgfscope}%
\pgfpathrectangle{\pgfqpoint{3.776708in}{0.600000in}}{\pgfqpoint{2.573292in}{2.070576in}}%
\pgfusepath{clip}%
\pgfsetbuttcap%
\pgfsetmiterjoin%
\definecolor{currentfill}{rgb}{0.754268,0.565033,0.211761}%
\pgfsetfillcolor{currentfill}%
\pgfsetlinewidth{0.000000pt}%
\definecolor{currentstroke}{rgb}{0.000000,0.000000,0.000000}%
\pgfsetstrokecolor{currentstroke}%
\pgfsetstrokeopacity{0.000000}%
\pgfsetdash{}{0pt}%
\pgfpathmoveto{\pgfqpoint{4.823728in}{1.277058in}}%
\pgfpathlineto{\pgfqpoint{4.832482in}{1.277058in}}%
\pgfpathlineto{\pgfqpoint{4.832482in}{1.212646in}}%
\pgfpathlineto{\pgfqpoint{4.823728in}{1.212646in}}%
\pgfpathlineto{\pgfqpoint{4.823728in}{1.277058in}}%
\pgfpathclose%
\pgfusepath{fill}%
\end{pgfscope}%
\begin{pgfscope}%
\pgfpathrectangle{\pgfqpoint{3.776708in}{0.600000in}}{\pgfqpoint{2.573292in}{2.070576in}}%
\pgfusepath{clip}%
\pgfsetbuttcap%
\pgfsetmiterjoin%
\definecolor{currentfill}{rgb}{0.754268,0.565033,0.211761}%
\pgfsetfillcolor{currentfill}%
\pgfsetlinewidth{0.000000pt}%
\definecolor{currentstroke}{rgb}{0.000000,0.000000,0.000000}%
\pgfsetstrokecolor{currentstroke}%
\pgfsetstrokeopacity{0.000000}%
\pgfsetdash{}{0pt}%
\pgfpathmoveto{\pgfqpoint{4.834670in}{1.260627in}}%
\pgfpathlineto{\pgfqpoint{4.843424in}{1.260627in}}%
\pgfpathlineto{\pgfqpoint{4.843424in}{1.212696in}}%
\pgfpathlineto{\pgfqpoint{4.834670in}{1.212696in}}%
\pgfpathlineto{\pgfqpoint{4.834670in}{1.260627in}}%
\pgfpathclose%
\pgfusepath{fill}%
\end{pgfscope}%
\begin{pgfscope}%
\pgfpathrectangle{\pgfqpoint{3.776708in}{0.600000in}}{\pgfqpoint{2.573292in}{2.070576in}}%
\pgfusepath{clip}%
\pgfsetbuttcap%
\pgfsetmiterjoin%
\definecolor{currentfill}{rgb}{0.754268,0.565033,0.211761}%
\pgfsetfillcolor{currentfill}%
\pgfsetlinewidth{0.000000pt}%
\definecolor{currentstroke}{rgb}{0.000000,0.000000,0.000000}%
\pgfsetstrokecolor{currentstroke}%
\pgfsetstrokeopacity{0.000000}%
\pgfsetdash{}{0pt}%
\pgfpathmoveto{\pgfqpoint{4.845612in}{1.249403in}}%
\pgfpathlineto{\pgfqpoint{4.854366in}{1.249403in}}%
\pgfpathlineto{\pgfqpoint{4.854366in}{1.221646in}}%
\pgfpathlineto{\pgfqpoint{4.845612in}{1.221646in}}%
\pgfpathlineto{\pgfqpoint{4.845612in}{1.249403in}}%
\pgfpathclose%
\pgfusepath{fill}%
\end{pgfscope}%
\begin{pgfscope}%
\pgfpathrectangle{\pgfqpoint{3.776708in}{0.600000in}}{\pgfqpoint{2.573292in}{2.070576in}}%
\pgfusepath{clip}%
\pgfsetbuttcap%
\pgfsetmiterjoin%
\definecolor{currentfill}{rgb}{0.754268,0.565033,0.211761}%
\pgfsetfillcolor{currentfill}%
\pgfsetlinewidth{0.000000pt}%
\definecolor{currentstroke}{rgb}{0.000000,0.000000,0.000000}%
\pgfsetstrokecolor{currentstroke}%
\pgfsetstrokeopacity{0.000000}%
\pgfsetdash{}{0pt}%
\pgfpathmoveto{\pgfqpoint{4.856554in}{1.236258in}}%
\pgfpathlineto{\pgfqpoint{4.865307in}{1.236258in}}%
\pgfpathlineto{\pgfqpoint{4.865307in}{1.228974in}}%
\pgfpathlineto{\pgfqpoint{4.856554in}{1.228974in}}%
\pgfpathlineto{\pgfqpoint{4.856554in}{1.236258in}}%
\pgfpathclose%
\pgfusepath{fill}%
\end{pgfscope}%
\begin{pgfscope}%
\pgfpathrectangle{\pgfqpoint{3.776708in}{0.600000in}}{\pgfqpoint{2.573292in}{2.070576in}}%
\pgfusepath{clip}%
\pgfsetbuttcap%
\pgfsetmiterjoin%
\definecolor{currentfill}{rgb}{0.754268,0.565033,0.211761}%
\pgfsetfillcolor{currentfill}%
\pgfsetlinewidth{0.000000pt}%
\definecolor{currentstroke}{rgb}{0.000000,0.000000,0.000000}%
\pgfsetstrokecolor{currentstroke}%
\pgfsetstrokeopacity{0.000000}%
\pgfsetdash{}{0pt}%
\pgfpathmoveto{\pgfqpoint{4.867496in}{2.029793in}}%
\pgfpathlineto{\pgfqpoint{4.876249in}{2.029793in}}%
\pgfpathlineto{\pgfqpoint{4.876249in}{2.037034in}}%
\pgfpathlineto{\pgfqpoint{4.867496in}{2.037034in}}%
\pgfpathlineto{\pgfqpoint{4.867496in}{2.029793in}}%
\pgfpathclose%
\pgfusepath{fill}%
\end{pgfscope}%
\begin{pgfscope}%
\pgfpathrectangle{\pgfqpoint{3.776708in}{0.600000in}}{\pgfqpoint{2.573292in}{2.070576in}}%
\pgfusepath{clip}%
\pgfsetbuttcap%
\pgfsetmiterjoin%
\definecolor{currentfill}{rgb}{0.754268,0.565033,0.211761}%
\pgfsetfillcolor{currentfill}%
\pgfsetlinewidth{0.000000pt}%
\definecolor{currentstroke}{rgb}{0.000000,0.000000,0.000000}%
\pgfsetstrokecolor{currentstroke}%
\pgfsetstrokeopacity{0.000000}%
\pgfsetdash{}{0pt}%
\pgfpathmoveto{\pgfqpoint{4.878437in}{2.034929in}}%
\pgfpathlineto{\pgfqpoint{4.887191in}{2.034929in}}%
\pgfpathlineto{\pgfqpoint{4.887191in}{2.053930in}}%
\pgfpathlineto{\pgfqpoint{4.878437in}{2.053930in}}%
\pgfpathlineto{\pgfqpoint{4.878437in}{2.034929in}}%
\pgfpathclose%
\pgfusepath{fill}%
\end{pgfscope}%
\begin{pgfscope}%
\pgfpathrectangle{\pgfqpoint{3.776708in}{0.600000in}}{\pgfqpoint{2.573292in}{2.070576in}}%
\pgfusepath{clip}%
\pgfsetbuttcap%
\pgfsetmiterjoin%
\definecolor{currentfill}{rgb}{0.754268,0.565033,0.211761}%
\pgfsetfillcolor{currentfill}%
\pgfsetlinewidth{0.000000pt}%
\definecolor{currentstroke}{rgb}{0.000000,0.000000,0.000000}%
\pgfsetstrokecolor{currentstroke}%
\pgfsetstrokeopacity{0.000000}%
\pgfsetdash{}{0pt}%
\pgfpathmoveto{\pgfqpoint{4.889379in}{2.049097in}}%
\pgfpathlineto{\pgfqpoint{4.898133in}{2.049097in}}%
\pgfpathlineto{\pgfqpoint{4.898133in}{2.076856in}}%
\pgfpathlineto{\pgfqpoint{4.889379in}{2.076856in}}%
\pgfpathlineto{\pgfqpoint{4.889379in}{2.049097in}}%
\pgfpathclose%
\pgfusepath{fill}%
\end{pgfscope}%
\begin{pgfscope}%
\pgfpathrectangle{\pgfqpoint{3.776708in}{0.600000in}}{\pgfqpoint{2.573292in}{2.070576in}}%
\pgfusepath{clip}%
\pgfsetbuttcap%
\pgfsetmiterjoin%
\definecolor{currentfill}{rgb}{0.754268,0.565033,0.211761}%
\pgfsetfillcolor{currentfill}%
\pgfsetlinewidth{0.000000pt}%
\definecolor{currentstroke}{rgb}{0.000000,0.000000,0.000000}%
\pgfsetstrokecolor{currentstroke}%
\pgfsetstrokeopacity{0.000000}%
\pgfsetdash{}{0pt}%
\pgfpathmoveto{\pgfqpoint{4.900321in}{2.072653in}}%
\pgfpathlineto{\pgfqpoint{4.909075in}{2.072653in}}%
\pgfpathlineto{\pgfqpoint{4.909075in}{2.104563in}}%
\pgfpathlineto{\pgfqpoint{4.900321in}{2.104563in}}%
\pgfpathlineto{\pgfqpoint{4.900321in}{2.072653in}}%
\pgfpathclose%
\pgfusepath{fill}%
\end{pgfscope}%
\begin{pgfscope}%
\pgfpathrectangle{\pgfqpoint{3.776708in}{0.600000in}}{\pgfqpoint{2.573292in}{2.070576in}}%
\pgfusepath{clip}%
\pgfsetbuttcap%
\pgfsetmiterjoin%
\definecolor{currentfill}{rgb}{0.754268,0.565033,0.211761}%
\pgfsetfillcolor{currentfill}%
\pgfsetlinewidth{0.000000pt}%
\definecolor{currentstroke}{rgb}{0.000000,0.000000,0.000000}%
\pgfsetstrokecolor{currentstroke}%
\pgfsetstrokeopacity{0.000000}%
\pgfsetdash{}{0pt}%
\pgfpathmoveto{\pgfqpoint{4.911263in}{2.094202in}}%
\pgfpathlineto{\pgfqpoint{4.920016in}{2.094202in}}%
\pgfpathlineto{\pgfqpoint{4.920016in}{2.135449in}}%
\pgfpathlineto{\pgfqpoint{4.911263in}{2.135449in}}%
\pgfpathlineto{\pgfqpoint{4.911263in}{2.094202in}}%
\pgfpathclose%
\pgfusepath{fill}%
\end{pgfscope}%
\begin{pgfscope}%
\pgfpathrectangle{\pgfqpoint{3.776708in}{0.600000in}}{\pgfqpoint{2.573292in}{2.070576in}}%
\pgfusepath{clip}%
\pgfsetbuttcap%
\pgfsetmiterjoin%
\definecolor{currentfill}{rgb}{0.754268,0.565033,0.211761}%
\pgfsetfillcolor{currentfill}%
\pgfsetlinewidth{0.000000pt}%
\definecolor{currentstroke}{rgb}{0.000000,0.000000,0.000000}%
\pgfsetstrokecolor{currentstroke}%
\pgfsetstrokeopacity{0.000000}%
\pgfsetdash{}{0pt}%
\pgfpathmoveto{\pgfqpoint{4.922205in}{2.101744in}}%
\pgfpathlineto{\pgfqpoint{4.930958in}{2.101744in}}%
\pgfpathlineto{\pgfqpoint{4.930958in}{2.157809in}}%
\pgfpathlineto{\pgfqpoint{4.922205in}{2.157809in}}%
\pgfpathlineto{\pgfqpoint{4.922205in}{2.101744in}}%
\pgfpathclose%
\pgfusepath{fill}%
\end{pgfscope}%
\begin{pgfscope}%
\pgfpathrectangle{\pgfqpoint{3.776708in}{0.600000in}}{\pgfqpoint{2.573292in}{2.070576in}}%
\pgfusepath{clip}%
\pgfsetbuttcap%
\pgfsetmiterjoin%
\definecolor{currentfill}{rgb}{0.754268,0.565033,0.211761}%
\pgfsetfillcolor{currentfill}%
\pgfsetlinewidth{0.000000pt}%
\definecolor{currentstroke}{rgb}{0.000000,0.000000,0.000000}%
\pgfsetstrokecolor{currentstroke}%
\pgfsetstrokeopacity{0.000000}%
\pgfsetdash{}{0pt}%
\pgfpathmoveto{\pgfqpoint{4.933146in}{2.104889in}}%
\pgfpathlineto{\pgfqpoint{4.941900in}{2.104889in}}%
\pgfpathlineto{\pgfqpoint{4.941900in}{2.177880in}}%
\pgfpathlineto{\pgfqpoint{4.933146in}{2.177880in}}%
\pgfpathlineto{\pgfqpoint{4.933146in}{2.104889in}}%
\pgfpathclose%
\pgfusepath{fill}%
\end{pgfscope}%
\begin{pgfscope}%
\pgfpathrectangle{\pgfqpoint{3.776708in}{0.600000in}}{\pgfqpoint{2.573292in}{2.070576in}}%
\pgfusepath{clip}%
\pgfsetbuttcap%
\pgfsetmiterjoin%
\definecolor{currentfill}{rgb}{0.754268,0.565033,0.211761}%
\pgfsetfillcolor{currentfill}%
\pgfsetlinewidth{0.000000pt}%
\definecolor{currentstroke}{rgb}{0.000000,0.000000,0.000000}%
\pgfsetstrokecolor{currentstroke}%
\pgfsetstrokeopacity{0.000000}%
\pgfsetdash{}{0pt}%
\pgfpathmoveto{\pgfqpoint{4.944088in}{2.115013in}}%
\pgfpathlineto{\pgfqpoint{4.952842in}{2.115013in}}%
\pgfpathlineto{\pgfqpoint{4.952842in}{2.201798in}}%
\pgfpathlineto{\pgfqpoint{4.944088in}{2.201798in}}%
\pgfpathlineto{\pgfqpoint{4.944088in}{2.115013in}}%
\pgfpathclose%
\pgfusepath{fill}%
\end{pgfscope}%
\begin{pgfscope}%
\pgfpathrectangle{\pgfqpoint{3.776708in}{0.600000in}}{\pgfqpoint{2.573292in}{2.070576in}}%
\pgfusepath{clip}%
\pgfsetbuttcap%
\pgfsetmiterjoin%
\definecolor{currentfill}{rgb}{0.754268,0.565033,0.211761}%
\pgfsetfillcolor{currentfill}%
\pgfsetlinewidth{0.000000pt}%
\definecolor{currentstroke}{rgb}{0.000000,0.000000,0.000000}%
\pgfsetstrokecolor{currentstroke}%
\pgfsetstrokeopacity{0.000000}%
\pgfsetdash{}{0pt}%
\pgfpathmoveto{\pgfqpoint{4.955030in}{2.126009in}}%
\pgfpathlineto{\pgfqpoint{4.963783in}{2.126009in}}%
\pgfpathlineto{\pgfqpoint{4.963783in}{2.231073in}}%
\pgfpathlineto{\pgfqpoint{4.955030in}{2.231073in}}%
\pgfpathlineto{\pgfqpoint{4.955030in}{2.126009in}}%
\pgfpathclose%
\pgfusepath{fill}%
\end{pgfscope}%
\begin{pgfscope}%
\pgfpathrectangle{\pgfqpoint{3.776708in}{0.600000in}}{\pgfqpoint{2.573292in}{2.070576in}}%
\pgfusepath{clip}%
\pgfsetbuttcap%
\pgfsetmiterjoin%
\definecolor{currentfill}{rgb}{0.754268,0.565033,0.211761}%
\pgfsetfillcolor{currentfill}%
\pgfsetlinewidth{0.000000pt}%
\definecolor{currentstroke}{rgb}{0.000000,0.000000,0.000000}%
\pgfsetstrokecolor{currentstroke}%
\pgfsetstrokeopacity{0.000000}%
\pgfsetdash{}{0pt}%
\pgfpathmoveto{\pgfqpoint{4.965972in}{2.124836in}}%
\pgfpathlineto{\pgfqpoint{4.974725in}{2.124836in}}%
\pgfpathlineto{\pgfqpoint{4.974725in}{2.250078in}}%
\pgfpathlineto{\pgfqpoint{4.965972in}{2.250078in}}%
\pgfpathlineto{\pgfqpoint{4.965972in}{2.124836in}}%
\pgfpathclose%
\pgfusepath{fill}%
\end{pgfscope}%
\begin{pgfscope}%
\pgfpathrectangle{\pgfqpoint{3.776708in}{0.600000in}}{\pgfqpoint{2.573292in}{2.070576in}}%
\pgfusepath{clip}%
\pgfsetbuttcap%
\pgfsetmiterjoin%
\definecolor{currentfill}{rgb}{0.754268,0.565033,0.211761}%
\pgfsetfillcolor{currentfill}%
\pgfsetlinewidth{0.000000pt}%
\definecolor{currentstroke}{rgb}{0.000000,0.000000,0.000000}%
\pgfsetstrokecolor{currentstroke}%
\pgfsetstrokeopacity{0.000000}%
\pgfsetdash{}{0pt}%
\pgfpathmoveto{\pgfqpoint{4.976914in}{2.125220in}}%
\pgfpathlineto{\pgfqpoint{4.985667in}{2.125220in}}%
\pgfpathlineto{\pgfqpoint{4.985667in}{2.269235in}}%
\pgfpathlineto{\pgfqpoint{4.976914in}{2.269235in}}%
\pgfpathlineto{\pgfqpoint{4.976914in}{2.125220in}}%
\pgfpathclose%
\pgfusepath{fill}%
\end{pgfscope}%
\begin{pgfscope}%
\pgfpathrectangle{\pgfqpoint{3.776708in}{0.600000in}}{\pgfqpoint{2.573292in}{2.070576in}}%
\pgfusepath{clip}%
\pgfsetbuttcap%
\pgfsetmiterjoin%
\definecolor{currentfill}{rgb}{0.754268,0.565033,0.211761}%
\pgfsetfillcolor{currentfill}%
\pgfsetlinewidth{0.000000pt}%
\definecolor{currentstroke}{rgb}{0.000000,0.000000,0.000000}%
\pgfsetstrokecolor{currentstroke}%
\pgfsetstrokeopacity{0.000000}%
\pgfsetdash{}{0pt}%
\pgfpathmoveto{\pgfqpoint{4.987855in}{2.119994in}}%
\pgfpathlineto{\pgfqpoint{4.996609in}{2.119994in}}%
\pgfpathlineto{\pgfqpoint{4.996609in}{2.291769in}}%
\pgfpathlineto{\pgfqpoint{4.987855in}{2.291769in}}%
\pgfpathlineto{\pgfqpoint{4.987855in}{2.119994in}}%
\pgfpathclose%
\pgfusepath{fill}%
\end{pgfscope}%
\begin{pgfscope}%
\pgfpathrectangle{\pgfqpoint{3.776708in}{0.600000in}}{\pgfqpoint{2.573292in}{2.070576in}}%
\pgfusepath{clip}%
\pgfsetbuttcap%
\pgfsetmiterjoin%
\definecolor{currentfill}{rgb}{0.754268,0.565033,0.211761}%
\pgfsetfillcolor{currentfill}%
\pgfsetlinewidth{0.000000pt}%
\definecolor{currentstroke}{rgb}{0.000000,0.000000,0.000000}%
\pgfsetstrokecolor{currentstroke}%
\pgfsetstrokeopacity{0.000000}%
\pgfsetdash{}{0pt}%
\pgfpathmoveto{\pgfqpoint{4.998797in}{2.116754in}}%
\pgfpathlineto{\pgfqpoint{5.007551in}{2.116754in}}%
\pgfpathlineto{\pgfqpoint{5.007551in}{2.317210in}}%
\pgfpathlineto{\pgfqpoint{4.998797in}{2.317210in}}%
\pgfpathlineto{\pgfqpoint{4.998797in}{2.116754in}}%
\pgfpathclose%
\pgfusepath{fill}%
\end{pgfscope}%
\begin{pgfscope}%
\pgfpathrectangle{\pgfqpoint{3.776708in}{0.600000in}}{\pgfqpoint{2.573292in}{2.070576in}}%
\pgfusepath{clip}%
\pgfsetbuttcap%
\pgfsetmiterjoin%
\definecolor{currentfill}{rgb}{0.754268,0.565033,0.211761}%
\pgfsetfillcolor{currentfill}%
\pgfsetlinewidth{0.000000pt}%
\definecolor{currentstroke}{rgb}{0.000000,0.000000,0.000000}%
\pgfsetstrokecolor{currentstroke}%
\pgfsetstrokeopacity{0.000000}%
\pgfsetdash{}{0pt}%
\pgfpathmoveto{\pgfqpoint{5.009739in}{2.114279in}}%
\pgfpathlineto{\pgfqpoint{5.018492in}{2.114279in}}%
\pgfpathlineto{\pgfqpoint{5.018492in}{2.338351in}}%
\pgfpathlineto{\pgfqpoint{5.009739in}{2.338351in}}%
\pgfpathlineto{\pgfqpoint{5.009739in}{2.114279in}}%
\pgfpathclose%
\pgfusepath{fill}%
\end{pgfscope}%
\begin{pgfscope}%
\pgfpathrectangle{\pgfqpoint{3.776708in}{0.600000in}}{\pgfqpoint{2.573292in}{2.070576in}}%
\pgfusepath{clip}%
\pgfsetbuttcap%
\pgfsetmiterjoin%
\definecolor{currentfill}{rgb}{0.754268,0.565033,0.211761}%
\pgfsetfillcolor{currentfill}%
\pgfsetlinewidth{0.000000pt}%
\definecolor{currentstroke}{rgb}{0.000000,0.000000,0.000000}%
\pgfsetstrokecolor{currentstroke}%
\pgfsetstrokeopacity{0.000000}%
\pgfsetdash{}{0pt}%
\pgfpathmoveto{\pgfqpoint{5.020681in}{2.101203in}}%
\pgfpathlineto{\pgfqpoint{5.029434in}{2.101203in}}%
\pgfpathlineto{\pgfqpoint{5.029434in}{2.355416in}}%
\pgfpathlineto{\pgfqpoint{5.020681in}{2.355416in}}%
\pgfpathlineto{\pgfqpoint{5.020681in}{2.101203in}}%
\pgfpathclose%
\pgfusepath{fill}%
\end{pgfscope}%
\begin{pgfscope}%
\pgfpathrectangle{\pgfqpoint{3.776708in}{0.600000in}}{\pgfqpoint{2.573292in}{2.070576in}}%
\pgfusepath{clip}%
\pgfsetbuttcap%
\pgfsetmiterjoin%
\definecolor{currentfill}{rgb}{0.754268,0.565033,0.211761}%
\pgfsetfillcolor{currentfill}%
\pgfsetlinewidth{0.000000pt}%
\definecolor{currentstroke}{rgb}{0.000000,0.000000,0.000000}%
\pgfsetstrokecolor{currentstroke}%
\pgfsetstrokeopacity{0.000000}%
\pgfsetdash{}{0pt}%
\pgfpathmoveto{\pgfqpoint{5.031623in}{2.088388in}}%
\pgfpathlineto{\pgfqpoint{5.040376in}{2.088388in}}%
\pgfpathlineto{\pgfqpoint{5.040376in}{2.370470in}}%
\pgfpathlineto{\pgfqpoint{5.031623in}{2.370470in}}%
\pgfpathlineto{\pgfqpoint{5.031623in}{2.088388in}}%
\pgfpathclose%
\pgfusepath{fill}%
\end{pgfscope}%
\begin{pgfscope}%
\pgfpathrectangle{\pgfqpoint{3.776708in}{0.600000in}}{\pgfqpoint{2.573292in}{2.070576in}}%
\pgfusepath{clip}%
\pgfsetbuttcap%
\pgfsetmiterjoin%
\definecolor{currentfill}{rgb}{0.754268,0.565033,0.211761}%
\pgfsetfillcolor{currentfill}%
\pgfsetlinewidth{0.000000pt}%
\definecolor{currentstroke}{rgb}{0.000000,0.000000,0.000000}%
\pgfsetstrokecolor{currentstroke}%
\pgfsetstrokeopacity{0.000000}%
\pgfsetdash{}{0pt}%
\pgfpathmoveto{\pgfqpoint{5.042564in}{2.069998in}}%
\pgfpathlineto{\pgfqpoint{5.051318in}{2.069998in}}%
\pgfpathlineto{\pgfqpoint{5.051318in}{2.384774in}}%
\pgfpathlineto{\pgfqpoint{5.042564in}{2.384774in}}%
\pgfpathlineto{\pgfqpoint{5.042564in}{2.069998in}}%
\pgfpathclose%
\pgfusepath{fill}%
\end{pgfscope}%
\begin{pgfscope}%
\pgfpathrectangle{\pgfqpoint{3.776708in}{0.600000in}}{\pgfqpoint{2.573292in}{2.070576in}}%
\pgfusepath{clip}%
\pgfsetbuttcap%
\pgfsetmiterjoin%
\definecolor{currentfill}{rgb}{0.754268,0.565033,0.211761}%
\pgfsetfillcolor{currentfill}%
\pgfsetlinewidth{0.000000pt}%
\definecolor{currentstroke}{rgb}{0.000000,0.000000,0.000000}%
\pgfsetstrokecolor{currentstroke}%
\pgfsetstrokeopacity{0.000000}%
\pgfsetdash{}{0pt}%
\pgfpathmoveto{\pgfqpoint{5.053506in}{2.056116in}}%
\pgfpathlineto{\pgfqpoint{5.062260in}{2.056116in}}%
\pgfpathlineto{\pgfqpoint{5.062260in}{2.395978in}}%
\pgfpathlineto{\pgfqpoint{5.053506in}{2.395978in}}%
\pgfpathlineto{\pgfqpoint{5.053506in}{2.056116in}}%
\pgfpathclose%
\pgfusepath{fill}%
\end{pgfscope}%
\begin{pgfscope}%
\pgfpathrectangle{\pgfqpoint{3.776708in}{0.600000in}}{\pgfqpoint{2.573292in}{2.070576in}}%
\pgfusepath{clip}%
\pgfsetbuttcap%
\pgfsetmiterjoin%
\definecolor{currentfill}{rgb}{0.754268,0.565033,0.211761}%
\pgfsetfillcolor{currentfill}%
\pgfsetlinewidth{0.000000pt}%
\definecolor{currentstroke}{rgb}{0.000000,0.000000,0.000000}%
\pgfsetstrokecolor{currentstroke}%
\pgfsetstrokeopacity{0.000000}%
\pgfsetdash{}{0pt}%
\pgfpathmoveto{\pgfqpoint{5.064448in}{2.042065in}}%
\pgfpathlineto{\pgfqpoint{5.073201in}{2.042065in}}%
\pgfpathlineto{\pgfqpoint{5.073201in}{2.412663in}}%
\pgfpathlineto{\pgfqpoint{5.064448in}{2.412663in}}%
\pgfpathlineto{\pgfqpoint{5.064448in}{2.042065in}}%
\pgfpathclose%
\pgfusepath{fill}%
\end{pgfscope}%
\begin{pgfscope}%
\pgfpathrectangle{\pgfqpoint{3.776708in}{0.600000in}}{\pgfqpoint{2.573292in}{2.070576in}}%
\pgfusepath{clip}%
\pgfsetbuttcap%
\pgfsetmiterjoin%
\definecolor{currentfill}{rgb}{0.754268,0.565033,0.211761}%
\pgfsetfillcolor{currentfill}%
\pgfsetlinewidth{0.000000pt}%
\definecolor{currentstroke}{rgb}{0.000000,0.000000,0.000000}%
\pgfsetstrokecolor{currentstroke}%
\pgfsetstrokeopacity{0.000000}%
\pgfsetdash{}{0pt}%
\pgfpathmoveto{\pgfqpoint{5.075390in}{2.027982in}}%
\pgfpathlineto{\pgfqpoint{5.084143in}{2.027982in}}%
\pgfpathlineto{\pgfqpoint{5.084143in}{2.423950in}}%
\pgfpathlineto{\pgfqpoint{5.075390in}{2.423950in}}%
\pgfpathlineto{\pgfqpoint{5.075390in}{2.027982in}}%
\pgfpathclose%
\pgfusepath{fill}%
\end{pgfscope}%
\begin{pgfscope}%
\pgfpathrectangle{\pgfqpoint{3.776708in}{0.600000in}}{\pgfqpoint{2.573292in}{2.070576in}}%
\pgfusepath{clip}%
\pgfsetbuttcap%
\pgfsetmiterjoin%
\definecolor{currentfill}{rgb}{0.754268,0.565033,0.211761}%
\pgfsetfillcolor{currentfill}%
\pgfsetlinewidth{0.000000pt}%
\definecolor{currentstroke}{rgb}{0.000000,0.000000,0.000000}%
\pgfsetstrokecolor{currentstroke}%
\pgfsetstrokeopacity{0.000000}%
\pgfsetdash{}{0pt}%
\pgfpathmoveto{\pgfqpoint{5.086332in}{2.019991in}}%
\pgfpathlineto{\pgfqpoint{5.095085in}{2.019991in}}%
\pgfpathlineto{\pgfqpoint{5.095085in}{2.442622in}}%
\pgfpathlineto{\pgfqpoint{5.086332in}{2.442622in}}%
\pgfpathlineto{\pgfqpoint{5.086332in}{2.019991in}}%
\pgfpathclose%
\pgfusepath{fill}%
\end{pgfscope}%
\begin{pgfscope}%
\pgfpathrectangle{\pgfqpoint{3.776708in}{0.600000in}}{\pgfqpoint{2.573292in}{2.070576in}}%
\pgfusepath{clip}%
\pgfsetbuttcap%
\pgfsetmiterjoin%
\definecolor{currentfill}{rgb}{0.754268,0.565033,0.211761}%
\pgfsetfillcolor{currentfill}%
\pgfsetlinewidth{0.000000pt}%
\definecolor{currentstroke}{rgb}{0.000000,0.000000,0.000000}%
\pgfsetstrokecolor{currentstroke}%
\pgfsetstrokeopacity{0.000000}%
\pgfsetdash{}{0pt}%
\pgfpathmoveto{\pgfqpoint{5.097273in}{2.012674in}}%
\pgfpathlineto{\pgfqpoint{5.106027in}{2.012674in}}%
\pgfpathlineto{\pgfqpoint{5.106027in}{2.453669in}}%
\pgfpathlineto{\pgfqpoint{5.097273in}{2.453669in}}%
\pgfpathlineto{\pgfqpoint{5.097273in}{2.012674in}}%
\pgfpathclose%
\pgfusepath{fill}%
\end{pgfscope}%
\begin{pgfscope}%
\pgfpathrectangle{\pgfqpoint{3.776708in}{0.600000in}}{\pgfqpoint{2.573292in}{2.070576in}}%
\pgfusepath{clip}%
\pgfsetbuttcap%
\pgfsetmiterjoin%
\definecolor{currentfill}{rgb}{0.754268,0.565033,0.211761}%
\pgfsetfillcolor{currentfill}%
\pgfsetlinewidth{0.000000pt}%
\definecolor{currentstroke}{rgb}{0.000000,0.000000,0.000000}%
\pgfsetstrokecolor{currentstroke}%
\pgfsetstrokeopacity{0.000000}%
\pgfsetdash{}{0pt}%
\pgfpathmoveto{\pgfqpoint{5.108215in}{2.006648in}}%
\pgfpathlineto{\pgfqpoint{5.116969in}{2.006648in}}%
\pgfpathlineto{\pgfqpoint{5.116969in}{2.467090in}}%
\pgfpathlineto{\pgfqpoint{5.108215in}{2.467090in}}%
\pgfpathlineto{\pgfqpoint{5.108215in}{2.006648in}}%
\pgfpathclose%
\pgfusepath{fill}%
\end{pgfscope}%
\begin{pgfscope}%
\pgfpathrectangle{\pgfqpoint{3.776708in}{0.600000in}}{\pgfqpoint{2.573292in}{2.070576in}}%
\pgfusepath{clip}%
\pgfsetbuttcap%
\pgfsetmiterjoin%
\definecolor{currentfill}{rgb}{0.754268,0.565033,0.211761}%
\pgfsetfillcolor{currentfill}%
\pgfsetlinewidth{0.000000pt}%
\definecolor{currentstroke}{rgb}{0.000000,0.000000,0.000000}%
\pgfsetstrokecolor{currentstroke}%
\pgfsetstrokeopacity{0.000000}%
\pgfsetdash{}{0pt}%
\pgfpathmoveto{\pgfqpoint{5.119157in}{2.002851in}}%
\pgfpathlineto{\pgfqpoint{5.127910in}{2.002851in}}%
\pgfpathlineto{\pgfqpoint{5.127910in}{2.478097in}}%
\pgfpathlineto{\pgfqpoint{5.119157in}{2.478097in}}%
\pgfpathlineto{\pgfqpoint{5.119157in}{2.002851in}}%
\pgfpathclose%
\pgfusepath{fill}%
\end{pgfscope}%
\begin{pgfscope}%
\pgfpathrectangle{\pgfqpoint{3.776708in}{0.600000in}}{\pgfqpoint{2.573292in}{2.070576in}}%
\pgfusepath{clip}%
\pgfsetbuttcap%
\pgfsetmiterjoin%
\definecolor{currentfill}{rgb}{0.754268,0.565033,0.211761}%
\pgfsetfillcolor{currentfill}%
\pgfsetlinewidth{0.000000pt}%
\definecolor{currentstroke}{rgb}{0.000000,0.000000,0.000000}%
\pgfsetstrokecolor{currentstroke}%
\pgfsetstrokeopacity{0.000000}%
\pgfsetdash{}{0pt}%
\pgfpathmoveto{\pgfqpoint{5.130099in}{2.005634in}}%
\pgfpathlineto{\pgfqpoint{5.138852in}{2.005634in}}%
\pgfpathlineto{\pgfqpoint{5.138852in}{2.495737in}}%
\pgfpathlineto{\pgfqpoint{5.130099in}{2.495737in}}%
\pgfpathlineto{\pgfqpoint{5.130099in}{2.005634in}}%
\pgfpathclose%
\pgfusepath{fill}%
\end{pgfscope}%
\begin{pgfscope}%
\pgfpathrectangle{\pgfqpoint{3.776708in}{0.600000in}}{\pgfqpoint{2.573292in}{2.070576in}}%
\pgfusepath{clip}%
\pgfsetbuttcap%
\pgfsetmiterjoin%
\definecolor{currentfill}{rgb}{0.754268,0.565033,0.211761}%
\pgfsetfillcolor{currentfill}%
\pgfsetlinewidth{0.000000pt}%
\definecolor{currentstroke}{rgb}{0.000000,0.000000,0.000000}%
\pgfsetstrokecolor{currentstroke}%
\pgfsetstrokeopacity{0.000000}%
\pgfsetdash{}{0pt}%
\pgfpathmoveto{\pgfqpoint{5.141041in}{2.009418in}}%
\pgfpathlineto{\pgfqpoint{5.149794in}{2.009418in}}%
\pgfpathlineto{\pgfqpoint{5.149794in}{2.510037in}}%
\pgfpathlineto{\pgfqpoint{5.141041in}{2.510037in}}%
\pgfpathlineto{\pgfqpoint{5.141041in}{2.009418in}}%
\pgfpathclose%
\pgfusepath{fill}%
\end{pgfscope}%
\begin{pgfscope}%
\pgfpathrectangle{\pgfqpoint{3.776708in}{0.600000in}}{\pgfqpoint{2.573292in}{2.070576in}}%
\pgfusepath{clip}%
\pgfsetbuttcap%
\pgfsetmiterjoin%
\definecolor{currentfill}{rgb}{0.754268,0.565033,0.211761}%
\pgfsetfillcolor{currentfill}%
\pgfsetlinewidth{0.000000pt}%
\definecolor{currentstroke}{rgb}{0.000000,0.000000,0.000000}%
\pgfsetstrokecolor{currentstroke}%
\pgfsetstrokeopacity{0.000000}%
\pgfsetdash{}{0pt}%
\pgfpathmoveto{\pgfqpoint{5.151982in}{2.020559in}}%
\pgfpathlineto{\pgfqpoint{5.160736in}{2.020559in}}%
\pgfpathlineto{\pgfqpoint{5.160736in}{2.528328in}}%
\pgfpathlineto{\pgfqpoint{5.151982in}{2.528328in}}%
\pgfpathlineto{\pgfqpoint{5.151982in}{2.020559in}}%
\pgfpathclose%
\pgfusepath{fill}%
\end{pgfscope}%
\begin{pgfscope}%
\pgfpathrectangle{\pgfqpoint{3.776708in}{0.600000in}}{\pgfqpoint{2.573292in}{2.070576in}}%
\pgfusepath{clip}%
\pgfsetbuttcap%
\pgfsetmiterjoin%
\definecolor{currentfill}{rgb}{0.754268,0.565033,0.211761}%
\pgfsetfillcolor{currentfill}%
\pgfsetlinewidth{0.000000pt}%
\definecolor{currentstroke}{rgb}{0.000000,0.000000,0.000000}%
\pgfsetstrokecolor{currentstroke}%
\pgfsetstrokeopacity{0.000000}%
\pgfsetdash{}{0pt}%
\pgfpathmoveto{\pgfqpoint{5.162924in}{2.033854in}}%
\pgfpathlineto{\pgfqpoint{5.171678in}{2.033854in}}%
\pgfpathlineto{\pgfqpoint{5.171678in}{2.539253in}}%
\pgfpathlineto{\pgfqpoint{5.162924in}{2.539253in}}%
\pgfpathlineto{\pgfqpoint{5.162924in}{2.033854in}}%
\pgfpathclose%
\pgfusepath{fill}%
\end{pgfscope}%
\begin{pgfscope}%
\pgfpathrectangle{\pgfqpoint{3.776708in}{0.600000in}}{\pgfqpoint{2.573292in}{2.070576in}}%
\pgfusepath{clip}%
\pgfsetbuttcap%
\pgfsetmiterjoin%
\definecolor{currentfill}{rgb}{0.754268,0.565033,0.211761}%
\pgfsetfillcolor{currentfill}%
\pgfsetlinewidth{0.000000pt}%
\definecolor{currentstroke}{rgb}{0.000000,0.000000,0.000000}%
\pgfsetstrokecolor{currentstroke}%
\pgfsetstrokeopacity{0.000000}%
\pgfsetdash{}{0pt}%
\pgfpathmoveto{\pgfqpoint{5.173866in}{2.042490in}}%
\pgfpathlineto{\pgfqpoint{5.182619in}{2.042490in}}%
\pgfpathlineto{\pgfqpoint{5.182619in}{2.546234in}}%
\pgfpathlineto{\pgfqpoint{5.173866in}{2.546234in}}%
\pgfpathlineto{\pgfqpoint{5.173866in}{2.042490in}}%
\pgfpathclose%
\pgfusepath{fill}%
\end{pgfscope}%
\begin{pgfscope}%
\pgfpathrectangle{\pgfqpoint{3.776708in}{0.600000in}}{\pgfqpoint{2.573292in}{2.070576in}}%
\pgfusepath{clip}%
\pgfsetbuttcap%
\pgfsetmiterjoin%
\definecolor{currentfill}{rgb}{0.754268,0.565033,0.211761}%
\pgfsetfillcolor{currentfill}%
\pgfsetlinewidth{0.000000pt}%
\definecolor{currentstroke}{rgb}{0.000000,0.000000,0.000000}%
\pgfsetstrokecolor{currentstroke}%
\pgfsetstrokeopacity{0.000000}%
\pgfsetdash{}{0pt}%
\pgfpathmoveto{\pgfqpoint{5.184808in}{2.053877in}}%
\pgfpathlineto{\pgfqpoint{5.193561in}{2.053877in}}%
\pgfpathlineto{\pgfqpoint{5.193561in}{2.561310in}}%
\pgfpathlineto{\pgfqpoint{5.184808in}{2.561310in}}%
\pgfpathlineto{\pgfqpoint{5.184808in}{2.053877in}}%
\pgfpathclose%
\pgfusepath{fill}%
\end{pgfscope}%
\begin{pgfscope}%
\pgfpathrectangle{\pgfqpoint{3.776708in}{0.600000in}}{\pgfqpoint{2.573292in}{2.070576in}}%
\pgfusepath{clip}%
\pgfsetbuttcap%
\pgfsetmiterjoin%
\definecolor{currentfill}{rgb}{0.754268,0.565033,0.211761}%
\pgfsetfillcolor{currentfill}%
\pgfsetlinewidth{0.000000pt}%
\definecolor{currentstroke}{rgb}{0.000000,0.000000,0.000000}%
\pgfsetstrokecolor{currentstroke}%
\pgfsetstrokeopacity{0.000000}%
\pgfsetdash{}{0pt}%
\pgfpathmoveto{\pgfqpoint{5.195750in}{2.061739in}}%
\pgfpathlineto{\pgfqpoint{5.204503in}{2.061739in}}%
\pgfpathlineto{\pgfqpoint{5.204503in}{2.566946in}}%
\pgfpathlineto{\pgfqpoint{5.195750in}{2.566946in}}%
\pgfpathlineto{\pgfqpoint{5.195750in}{2.061739in}}%
\pgfpathclose%
\pgfusepath{fill}%
\end{pgfscope}%
\begin{pgfscope}%
\pgfpathrectangle{\pgfqpoint{3.776708in}{0.600000in}}{\pgfqpoint{2.573292in}{2.070576in}}%
\pgfusepath{clip}%
\pgfsetbuttcap%
\pgfsetmiterjoin%
\definecolor{currentfill}{rgb}{0.754268,0.565033,0.211761}%
\pgfsetfillcolor{currentfill}%
\pgfsetlinewidth{0.000000pt}%
\definecolor{currentstroke}{rgb}{0.000000,0.000000,0.000000}%
\pgfsetstrokecolor{currentstroke}%
\pgfsetstrokeopacity{0.000000}%
\pgfsetdash{}{0pt}%
\pgfpathmoveto{\pgfqpoint{5.206691in}{2.071857in}}%
\pgfpathlineto{\pgfqpoint{5.215445in}{2.071857in}}%
\pgfpathlineto{\pgfqpoint{5.215445in}{2.573521in}}%
\pgfpathlineto{\pgfqpoint{5.206691in}{2.573521in}}%
\pgfpathlineto{\pgfqpoint{5.206691in}{2.071857in}}%
\pgfpathclose%
\pgfusepath{fill}%
\end{pgfscope}%
\begin{pgfscope}%
\pgfpathrectangle{\pgfqpoint{3.776708in}{0.600000in}}{\pgfqpoint{2.573292in}{2.070576in}}%
\pgfusepath{clip}%
\pgfsetbuttcap%
\pgfsetmiterjoin%
\definecolor{currentfill}{rgb}{0.754268,0.565033,0.211761}%
\pgfsetfillcolor{currentfill}%
\pgfsetlinewidth{0.000000pt}%
\definecolor{currentstroke}{rgb}{0.000000,0.000000,0.000000}%
\pgfsetstrokecolor{currentstroke}%
\pgfsetstrokeopacity{0.000000}%
\pgfsetdash{}{0pt}%
\pgfpathmoveto{\pgfqpoint{5.217633in}{2.076228in}}%
\pgfpathlineto{\pgfqpoint{5.226387in}{2.076228in}}%
\pgfpathlineto{\pgfqpoint{5.226387in}{2.576459in}}%
\pgfpathlineto{\pgfqpoint{5.217633in}{2.576459in}}%
\pgfpathlineto{\pgfqpoint{5.217633in}{2.076228in}}%
\pgfpathclose%
\pgfusepath{fill}%
\end{pgfscope}%
\begin{pgfscope}%
\pgfpathrectangle{\pgfqpoint{3.776708in}{0.600000in}}{\pgfqpoint{2.573292in}{2.070576in}}%
\pgfusepath{clip}%
\pgfsetbuttcap%
\pgfsetmiterjoin%
\definecolor{currentfill}{rgb}{0.754268,0.565033,0.211761}%
\pgfsetfillcolor{currentfill}%
\pgfsetlinewidth{0.000000pt}%
\definecolor{currentstroke}{rgb}{0.000000,0.000000,0.000000}%
\pgfsetstrokecolor{currentstroke}%
\pgfsetstrokeopacity{0.000000}%
\pgfsetdash{}{0pt}%
\pgfpathmoveto{\pgfqpoint{5.228575in}{2.080283in}}%
\pgfpathlineto{\pgfqpoint{5.237328in}{2.080283in}}%
\pgfpathlineto{\pgfqpoint{5.237328in}{2.573947in}}%
\pgfpathlineto{\pgfqpoint{5.228575in}{2.573947in}}%
\pgfpathlineto{\pgfqpoint{5.228575in}{2.080283in}}%
\pgfpathclose%
\pgfusepath{fill}%
\end{pgfscope}%
\begin{pgfscope}%
\pgfpathrectangle{\pgfqpoint{3.776708in}{0.600000in}}{\pgfqpoint{2.573292in}{2.070576in}}%
\pgfusepath{clip}%
\pgfsetbuttcap%
\pgfsetmiterjoin%
\definecolor{currentfill}{rgb}{0.754268,0.565033,0.211761}%
\pgfsetfillcolor{currentfill}%
\pgfsetlinewidth{0.000000pt}%
\definecolor{currentstroke}{rgb}{0.000000,0.000000,0.000000}%
\pgfsetstrokecolor{currentstroke}%
\pgfsetstrokeopacity{0.000000}%
\pgfsetdash{}{0pt}%
\pgfpathmoveto{\pgfqpoint{5.239517in}{2.090849in}}%
\pgfpathlineto{\pgfqpoint{5.248270in}{2.090849in}}%
\pgfpathlineto{\pgfqpoint{5.248270in}{2.570471in}}%
\pgfpathlineto{\pgfqpoint{5.239517in}{2.570471in}}%
\pgfpathlineto{\pgfqpoint{5.239517in}{2.090849in}}%
\pgfpathclose%
\pgfusepath{fill}%
\end{pgfscope}%
\begin{pgfscope}%
\pgfpathrectangle{\pgfqpoint{3.776708in}{0.600000in}}{\pgfqpoint{2.573292in}{2.070576in}}%
\pgfusepath{clip}%
\pgfsetbuttcap%
\pgfsetmiterjoin%
\definecolor{currentfill}{rgb}{0.754268,0.565033,0.211761}%
\pgfsetfillcolor{currentfill}%
\pgfsetlinewidth{0.000000pt}%
\definecolor{currentstroke}{rgb}{0.000000,0.000000,0.000000}%
\pgfsetstrokecolor{currentstroke}%
\pgfsetstrokeopacity{0.000000}%
\pgfsetdash{}{0pt}%
\pgfpathmoveto{\pgfqpoint{5.250459in}{2.096721in}}%
\pgfpathlineto{\pgfqpoint{5.259212in}{2.096721in}}%
\pgfpathlineto{\pgfqpoint{5.259212in}{2.563255in}}%
\pgfpathlineto{\pgfqpoint{5.250459in}{2.563255in}}%
\pgfpathlineto{\pgfqpoint{5.250459in}{2.096721in}}%
\pgfpathclose%
\pgfusepath{fill}%
\end{pgfscope}%
\begin{pgfscope}%
\pgfpathrectangle{\pgfqpoint{3.776708in}{0.600000in}}{\pgfqpoint{2.573292in}{2.070576in}}%
\pgfusepath{clip}%
\pgfsetbuttcap%
\pgfsetmiterjoin%
\definecolor{currentfill}{rgb}{0.754268,0.565033,0.211761}%
\pgfsetfillcolor{currentfill}%
\pgfsetlinewidth{0.000000pt}%
\definecolor{currentstroke}{rgb}{0.000000,0.000000,0.000000}%
\pgfsetstrokecolor{currentstroke}%
\pgfsetstrokeopacity{0.000000}%
\pgfsetdash{}{0pt}%
\pgfpathmoveto{\pgfqpoint{5.261400in}{2.108014in}}%
\pgfpathlineto{\pgfqpoint{5.270154in}{2.108014in}}%
\pgfpathlineto{\pgfqpoint{5.270154in}{2.558686in}}%
\pgfpathlineto{\pgfqpoint{5.261400in}{2.558686in}}%
\pgfpathlineto{\pgfqpoint{5.261400in}{2.108014in}}%
\pgfpathclose%
\pgfusepath{fill}%
\end{pgfscope}%
\begin{pgfscope}%
\pgfpathrectangle{\pgfqpoint{3.776708in}{0.600000in}}{\pgfqpoint{2.573292in}{2.070576in}}%
\pgfusepath{clip}%
\pgfsetbuttcap%
\pgfsetmiterjoin%
\definecolor{currentfill}{rgb}{0.754268,0.565033,0.211761}%
\pgfsetfillcolor{currentfill}%
\pgfsetlinewidth{0.000000pt}%
\definecolor{currentstroke}{rgb}{0.000000,0.000000,0.000000}%
\pgfsetstrokecolor{currentstroke}%
\pgfsetstrokeopacity{0.000000}%
\pgfsetdash{}{0pt}%
\pgfpathmoveto{\pgfqpoint{5.272342in}{2.119966in}}%
\pgfpathlineto{\pgfqpoint{5.281096in}{2.119966in}}%
\pgfpathlineto{\pgfqpoint{5.281096in}{2.548292in}}%
\pgfpathlineto{\pgfqpoint{5.272342in}{2.548292in}}%
\pgfpathlineto{\pgfqpoint{5.272342in}{2.119966in}}%
\pgfpathclose%
\pgfusepath{fill}%
\end{pgfscope}%
\begin{pgfscope}%
\pgfpathrectangle{\pgfqpoint{3.776708in}{0.600000in}}{\pgfqpoint{2.573292in}{2.070576in}}%
\pgfusepath{clip}%
\pgfsetbuttcap%
\pgfsetmiterjoin%
\definecolor{currentfill}{rgb}{0.754268,0.565033,0.211761}%
\pgfsetfillcolor{currentfill}%
\pgfsetlinewidth{0.000000pt}%
\definecolor{currentstroke}{rgb}{0.000000,0.000000,0.000000}%
\pgfsetstrokecolor{currentstroke}%
\pgfsetstrokeopacity{0.000000}%
\pgfsetdash{}{0pt}%
\pgfpathmoveto{\pgfqpoint{5.283284in}{2.126303in}}%
\pgfpathlineto{\pgfqpoint{5.292037in}{2.126303in}}%
\pgfpathlineto{\pgfqpoint{5.292037in}{2.528777in}}%
\pgfpathlineto{\pgfqpoint{5.283284in}{2.528777in}}%
\pgfpathlineto{\pgfqpoint{5.283284in}{2.126303in}}%
\pgfpathclose%
\pgfusepath{fill}%
\end{pgfscope}%
\begin{pgfscope}%
\pgfpathrectangle{\pgfqpoint{3.776708in}{0.600000in}}{\pgfqpoint{2.573292in}{2.070576in}}%
\pgfusepath{clip}%
\pgfsetbuttcap%
\pgfsetmiterjoin%
\definecolor{currentfill}{rgb}{0.754268,0.565033,0.211761}%
\pgfsetfillcolor{currentfill}%
\pgfsetlinewidth{0.000000pt}%
\definecolor{currentstroke}{rgb}{0.000000,0.000000,0.000000}%
\pgfsetstrokecolor{currentstroke}%
\pgfsetstrokeopacity{0.000000}%
\pgfsetdash{}{0pt}%
\pgfpathmoveto{\pgfqpoint{5.294226in}{2.120832in}}%
\pgfpathlineto{\pgfqpoint{5.302979in}{2.120832in}}%
\pgfpathlineto{\pgfqpoint{5.302979in}{2.500400in}}%
\pgfpathlineto{\pgfqpoint{5.294226in}{2.500400in}}%
\pgfpathlineto{\pgfqpoint{5.294226in}{2.120832in}}%
\pgfpathclose%
\pgfusepath{fill}%
\end{pgfscope}%
\begin{pgfscope}%
\pgfpathrectangle{\pgfqpoint{3.776708in}{0.600000in}}{\pgfqpoint{2.573292in}{2.070576in}}%
\pgfusepath{clip}%
\pgfsetbuttcap%
\pgfsetmiterjoin%
\definecolor{currentfill}{rgb}{0.754268,0.565033,0.211761}%
\pgfsetfillcolor{currentfill}%
\pgfsetlinewidth{0.000000pt}%
\definecolor{currentstroke}{rgb}{0.000000,0.000000,0.000000}%
\pgfsetstrokecolor{currentstroke}%
\pgfsetstrokeopacity{0.000000}%
\pgfsetdash{}{0pt}%
\pgfpathmoveto{\pgfqpoint{5.305168in}{2.125696in}}%
\pgfpathlineto{\pgfqpoint{5.313921in}{2.125696in}}%
\pgfpathlineto{\pgfqpoint{5.313921in}{2.482654in}}%
\pgfpathlineto{\pgfqpoint{5.305168in}{2.482654in}}%
\pgfpathlineto{\pgfqpoint{5.305168in}{2.125696in}}%
\pgfpathclose%
\pgfusepath{fill}%
\end{pgfscope}%
\begin{pgfscope}%
\pgfpathrectangle{\pgfqpoint{3.776708in}{0.600000in}}{\pgfqpoint{2.573292in}{2.070576in}}%
\pgfusepath{clip}%
\pgfsetbuttcap%
\pgfsetmiterjoin%
\definecolor{currentfill}{rgb}{0.754268,0.565033,0.211761}%
\pgfsetfillcolor{currentfill}%
\pgfsetlinewidth{0.000000pt}%
\definecolor{currentstroke}{rgb}{0.000000,0.000000,0.000000}%
\pgfsetstrokecolor{currentstroke}%
\pgfsetstrokeopacity{0.000000}%
\pgfsetdash{}{0pt}%
\pgfpathmoveto{\pgfqpoint{5.316109in}{2.122971in}}%
\pgfpathlineto{\pgfqpoint{5.324863in}{2.122971in}}%
\pgfpathlineto{\pgfqpoint{5.324863in}{2.455799in}}%
\pgfpathlineto{\pgfqpoint{5.316109in}{2.455799in}}%
\pgfpathlineto{\pgfqpoint{5.316109in}{2.122971in}}%
\pgfpathclose%
\pgfusepath{fill}%
\end{pgfscope}%
\begin{pgfscope}%
\pgfpathrectangle{\pgfqpoint{3.776708in}{0.600000in}}{\pgfqpoint{2.573292in}{2.070576in}}%
\pgfusepath{clip}%
\pgfsetbuttcap%
\pgfsetmiterjoin%
\definecolor{currentfill}{rgb}{0.754268,0.565033,0.211761}%
\pgfsetfillcolor{currentfill}%
\pgfsetlinewidth{0.000000pt}%
\definecolor{currentstroke}{rgb}{0.000000,0.000000,0.000000}%
\pgfsetstrokecolor{currentstroke}%
\pgfsetstrokeopacity{0.000000}%
\pgfsetdash{}{0pt}%
\pgfpathmoveto{\pgfqpoint{5.327051in}{2.114199in}}%
\pgfpathlineto{\pgfqpoint{5.335805in}{2.114199in}}%
\pgfpathlineto{\pgfqpoint{5.335805in}{2.422056in}}%
\pgfpathlineto{\pgfqpoint{5.327051in}{2.422056in}}%
\pgfpathlineto{\pgfqpoint{5.327051in}{2.114199in}}%
\pgfpathclose%
\pgfusepath{fill}%
\end{pgfscope}%
\begin{pgfscope}%
\pgfpathrectangle{\pgfqpoint{3.776708in}{0.600000in}}{\pgfqpoint{2.573292in}{2.070576in}}%
\pgfusepath{clip}%
\pgfsetbuttcap%
\pgfsetmiterjoin%
\definecolor{currentfill}{rgb}{0.754268,0.565033,0.211761}%
\pgfsetfillcolor{currentfill}%
\pgfsetlinewidth{0.000000pt}%
\definecolor{currentstroke}{rgb}{0.000000,0.000000,0.000000}%
\pgfsetstrokecolor{currentstroke}%
\pgfsetstrokeopacity{0.000000}%
\pgfsetdash{}{0pt}%
\pgfpathmoveto{\pgfqpoint{5.337993in}{2.108698in}}%
\pgfpathlineto{\pgfqpoint{5.346746in}{2.108698in}}%
\pgfpathlineto{\pgfqpoint{5.346746in}{2.390279in}}%
\pgfpathlineto{\pgfqpoint{5.337993in}{2.390279in}}%
\pgfpathlineto{\pgfqpoint{5.337993in}{2.108698in}}%
\pgfpathclose%
\pgfusepath{fill}%
\end{pgfscope}%
\begin{pgfscope}%
\pgfpathrectangle{\pgfqpoint{3.776708in}{0.600000in}}{\pgfqpoint{2.573292in}{2.070576in}}%
\pgfusepath{clip}%
\pgfsetbuttcap%
\pgfsetmiterjoin%
\definecolor{currentfill}{rgb}{0.754268,0.565033,0.211761}%
\pgfsetfillcolor{currentfill}%
\pgfsetlinewidth{0.000000pt}%
\definecolor{currentstroke}{rgb}{0.000000,0.000000,0.000000}%
\pgfsetstrokecolor{currentstroke}%
\pgfsetstrokeopacity{0.000000}%
\pgfsetdash{}{0pt}%
\pgfpathmoveto{\pgfqpoint{5.348935in}{2.099290in}}%
\pgfpathlineto{\pgfqpoint{5.357688in}{2.099290in}}%
\pgfpathlineto{\pgfqpoint{5.357688in}{2.361195in}}%
\pgfpathlineto{\pgfqpoint{5.348935in}{2.361195in}}%
\pgfpathlineto{\pgfqpoint{5.348935in}{2.099290in}}%
\pgfpathclose%
\pgfusepath{fill}%
\end{pgfscope}%
\begin{pgfscope}%
\pgfpathrectangle{\pgfqpoint{3.776708in}{0.600000in}}{\pgfqpoint{2.573292in}{2.070576in}}%
\pgfusepath{clip}%
\pgfsetbuttcap%
\pgfsetmiterjoin%
\definecolor{currentfill}{rgb}{0.754268,0.565033,0.211761}%
\pgfsetfillcolor{currentfill}%
\pgfsetlinewidth{0.000000pt}%
\definecolor{currentstroke}{rgb}{0.000000,0.000000,0.000000}%
\pgfsetstrokecolor{currentstroke}%
\pgfsetstrokeopacity{0.000000}%
\pgfsetdash{}{0pt}%
\pgfpathmoveto{\pgfqpoint{5.359877in}{2.084858in}}%
\pgfpathlineto{\pgfqpoint{5.368630in}{2.084858in}}%
\pgfpathlineto{\pgfqpoint{5.368630in}{2.322053in}}%
\pgfpathlineto{\pgfqpoint{5.359877in}{2.322053in}}%
\pgfpathlineto{\pgfqpoint{5.359877in}{2.084858in}}%
\pgfpathclose%
\pgfusepath{fill}%
\end{pgfscope}%
\begin{pgfscope}%
\pgfpathrectangle{\pgfqpoint{3.776708in}{0.600000in}}{\pgfqpoint{2.573292in}{2.070576in}}%
\pgfusepath{clip}%
\pgfsetbuttcap%
\pgfsetmiterjoin%
\definecolor{currentfill}{rgb}{0.754268,0.565033,0.211761}%
\pgfsetfillcolor{currentfill}%
\pgfsetlinewidth{0.000000pt}%
\definecolor{currentstroke}{rgb}{0.000000,0.000000,0.000000}%
\pgfsetstrokecolor{currentstroke}%
\pgfsetstrokeopacity{0.000000}%
\pgfsetdash{}{0pt}%
\pgfpathmoveto{\pgfqpoint{5.370818in}{2.074670in}}%
\pgfpathlineto{\pgfqpoint{5.379572in}{2.074670in}}%
\pgfpathlineto{\pgfqpoint{5.379572in}{2.279843in}}%
\pgfpathlineto{\pgfqpoint{5.370818in}{2.279843in}}%
\pgfpathlineto{\pgfqpoint{5.370818in}{2.074670in}}%
\pgfpathclose%
\pgfusepath{fill}%
\end{pgfscope}%
\begin{pgfscope}%
\pgfpathrectangle{\pgfqpoint{3.776708in}{0.600000in}}{\pgfqpoint{2.573292in}{2.070576in}}%
\pgfusepath{clip}%
\pgfsetbuttcap%
\pgfsetmiterjoin%
\definecolor{currentfill}{rgb}{0.754268,0.565033,0.211761}%
\pgfsetfillcolor{currentfill}%
\pgfsetlinewidth{0.000000pt}%
\definecolor{currentstroke}{rgb}{0.000000,0.000000,0.000000}%
\pgfsetstrokecolor{currentstroke}%
\pgfsetstrokeopacity{0.000000}%
\pgfsetdash{}{0pt}%
\pgfpathmoveto{\pgfqpoint{5.381760in}{2.068820in}}%
\pgfpathlineto{\pgfqpoint{5.390514in}{2.068820in}}%
\pgfpathlineto{\pgfqpoint{5.390514in}{2.244456in}}%
\pgfpathlineto{\pgfqpoint{5.381760in}{2.244456in}}%
\pgfpathlineto{\pgfqpoint{5.381760in}{2.068820in}}%
\pgfpathclose%
\pgfusepath{fill}%
\end{pgfscope}%
\begin{pgfscope}%
\pgfpathrectangle{\pgfqpoint{3.776708in}{0.600000in}}{\pgfqpoint{2.573292in}{2.070576in}}%
\pgfusepath{clip}%
\pgfsetbuttcap%
\pgfsetmiterjoin%
\definecolor{currentfill}{rgb}{0.754268,0.565033,0.211761}%
\pgfsetfillcolor{currentfill}%
\pgfsetlinewidth{0.000000pt}%
\definecolor{currentstroke}{rgb}{0.000000,0.000000,0.000000}%
\pgfsetstrokecolor{currentstroke}%
\pgfsetstrokeopacity{0.000000}%
\pgfsetdash{}{0pt}%
\pgfpathmoveto{\pgfqpoint{5.392702in}{2.066336in}}%
\pgfpathlineto{\pgfqpoint{5.401455in}{2.066336in}}%
\pgfpathlineto{\pgfqpoint{5.401455in}{2.209608in}}%
\pgfpathlineto{\pgfqpoint{5.392702in}{2.209608in}}%
\pgfpathlineto{\pgfqpoint{5.392702in}{2.066336in}}%
\pgfpathclose%
\pgfusepath{fill}%
\end{pgfscope}%
\begin{pgfscope}%
\pgfpathrectangle{\pgfqpoint{3.776708in}{0.600000in}}{\pgfqpoint{2.573292in}{2.070576in}}%
\pgfusepath{clip}%
\pgfsetbuttcap%
\pgfsetmiterjoin%
\definecolor{currentfill}{rgb}{0.754268,0.565033,0.211761}%
\pgfsetfillcolor{currentfill}%
\pgfsetlinewidth{0.000000pt}%
\definecolor{currentstroke}{rgb}{0.000000,0.000000,0.000000}%
\pgfsetstrokecolor{currentstroke}%
\pgfsetstrokeopacity{0.000000}%
\pgfsetdash{}{0pt}%
\pgfpathmoveto{\pgfqpoint{5.403644in}{2.058271in}}%
\pgfpathlineto{\pgfqpoint{5.412397in}{2.058271in}}%
\pgfpathlineto{\pgfqpoint{5.412397in}{2.180172in}}%
\pgfpathlineto{\pgfqpoint{5.403644in}{2.180172in}}%
\pgfpathlineto{\pgfqpoint{5.403644in}{2.058271in}}%
\pgfpathclose%
\pgfusepath{fill}%
\end{pgfscope}%
\begin{pgfscope}%
\pgfpathrectangle{\pgfqpoint{3.776708in}{0.600000in}}{\pgfqpoint{2.573292in}{2.070576in}}%
\pgfusepath{clip}%
\pgfsetbuttcap%
\pgfsetmiterjoin%
\definecolor{currentfill}{rgb}{0.754268,0.565033,0.211761}%
\pgfsetfillcolor{currentfill}%
\pgfsetlinewidth{0.000000pt}%
\definecolor{currentstroke}{rgb}{0.000000,0.000000,0.000000}%
\pgfsetstrokecolor{currentstroke}%
\pgfsetstrokeopacity{0.000000}%
\pgfsetdash{}{0pt}%
\pgfpathmoveto{\pgfqpoint{5.414586in}{2.046898in}}%
\pgfpathlineto{\pgfqpoint{5.423339in}{2.046898in}}%
\pgfpathlineto{\pgfqpoint{5.423339in}{2.138900in}}%
\pgfpathlineto{\pgfqpoint{5.414586in}{2.138900in}}%
\pgfpathlineto{\pgfqpoint{5.414586in}{2.046898in}}%
\pgfpathclose%
\pgfusepath{fill}%
\end{pgfscope}%
\begin{pgfscope}%
\pgfpathrectangle{\pgfqpoint{3.776708in}{0.600000in}}{\pgfqpoint{2.573292in}{2.070576in}}%
\pgfusepath{clip}%
\pgfsetbuttcap%
\pgfsetmiterjoin%
\definecolor{currentfill}{rgb}{0.754268,0.565033,0.211761}%
\pgfsetfillcolor{currentfill}%
\pgfsetlinewidth{0.000000pt}%
\definecolor{currentstroke}{rgb}{0.000000,0.000000,0.000000}%
\pgfsetstrokecolor{currentstroke}%
\pgfsetstrokeopacity{0.000000}%
\pgfsetdash{}{0pt}%
\pgfpathmoveto{\pgfqpoint{5.425527in}{2.031737in}}%
\pgfpathlineto{\pgfqpoint{5.434281in}{2.031737in}}%
\pgfpathlineto{\pgfqpoint{5.434281in}{2.106923in}}%
\pgfpathlineto{\pgfqpoint{5.425527in}{2.106923in}}%
\pgfpathlineto{\pgfqpoint{5.425527in}{2.031737in}}%
\pgfpathclose%
\pgfusepath{fill}%
\end{pgfscope}%
\begin{pgfscope}%
\pgfpathrectangle{\pgfqpoint{3.776708in}{0.600000in}}{\pgfqpoint{2.573292in}{2.070576in}}%
\pgfusepath{clip}%
\pgfsetbuttcap%
\pgfsetmiterjoin%
\definecolor{currentfill}{rgb}{0.754268,0.565033,0.211761}%
\pgfsetfillcolor{currentfill}%
\pgfsetlinewidth{0.000000pt}%
\definecolor{currentstroke}{rgb}{0.000000,0.000000,0.000000}%
\pgfsetstrokecolor{currentstroke}%
\pgfsetstrokeopacity{0.000000}%
\pgfsetdash{}{0pt}%
\pgfpathmoveto{\pgfqpoint{5.436469in}{2.012810in}}%
\pgfpathlineto{\pgfqpoint{5.445223in}{2.012810in}}%
\pgfpathlineto{\pgfqpoint{5.445223in}{2.077765in}}%
\pgfpathlineto{\pgfqpoint{5.436469in}{2.077765in}}%
\pgfpathlineto{\pgfqpoint{5.436469in}{2.012810in}}%
\pgfpathclose%
\pgfusepath{fill}%
\end{pgfscope}%
\begin{pgfscope}%
\pgfpathrectangle{\pgfqpoint{3.776708in}{0.600000in}}{\pgfqpoint{2.573292in}{2.070576in}}%
\pgfusepath{clip}%
\pgfsetbuttcap%
\pgfsetmiterjoin%
\definecolor{currentfill}{rgb}{0.754268,0.565033,0.211761}%
\pgfsetfillcolor{currentfill}%
\pgfsetlinewidth{0.000000pt}%
\definecolor{currentstroke}{rgb}{0.000000,0.000000,0.000000}%
\pgfsetstrokecolor{currentstroke}%
\pgfsetstrokeopacity{0.000000}%
\pgfsetdash{}{0pt}%
\pgfpathmoveto{\pgfqpoint{5.447411in}{1.985311in}}%
\pgfpathlineto{\pgfqpoint{5.456164in}{1.985311in}}%
\pgfpathlineto{\pgfqpoint{5.456164in}{2.040873in}}%
\pgfpathlineto{\pgfqpoint{5.447411in}{2.040873in}}%
\pgfpathlineto{\pgfqpoint{5.447411in}{1.985311in}}%
\pgfpathclose%
\pgfusepath{fill}%
\end{pgfscope}%
\begin{pgfscope}%
\pgfpathrectangle{\pgfqpoint{3.776708in}{0.600000in}}{\pgfqpoint{2.573292in}{2.070576in}}%
\pgfusepath{clip}%
\pgfsetbuttcap%
\pgfsetmiterjoin%
\definecolor{currentfill}{rgb}{0.754268,0.565033,0.211761}%
\pgfsetfillcolor{currentfill}%
\pgfsetlinewidth{0.000000pt}%
\definecolor{currentstroke}{rgb}{0.000000,0.000000,0.000000}%
\pgfsetstrokecolor{currentstroke}%
\pgfsetstrokeopacity{0.000000}%
\pgfsetdash{}{0pt}%
\pgfpathmoveto{\pgfqpoint{5.458353in}{1.954401in}}%
\pgfpathlineto{\pgfqpoint{5.467106in}{1.954401in}}%
\pgfpathlineto{\pgfqpoint{5.467106in}{2.007455in}}%
\pgfpathlineto{\pgfqpoint{5.458353in}{2.007455in}}%
\pgfpathlineto{\pgfqpoint{5.458353in}{1.954401in}}%
\pgfpathclose%
\pgfusepath{fill}%
\end{pgfscope}%
\begin{pgfscope}%
\pgfpathrectangle{\pgfqpoint{3.776708in}{0.600000in}}{\pgfqpoint{2.573292in}{2.070576in}}%
\pgfusepath{clip}%
\pgfsetbuttcap%
\pgfsetmiterjoin%
\definecolor{currentfill}{rgb}{0.754268,0.565033,0.211761}%
\pgfsetfillcolor{currentfill}%
\pgfsetlinewidth{0.000000pt}%
\definecolor{currentstroke}{rgb}{0.000000,0.000000,0.000000}%
\pgfsetstrokecolor{currentstroke}%
\pgfsetstrokeopacity{0.000000}%
\pgfsetdash{}{0pt}%
\pgfpathmoveto{\pgfqpoint{5.469295in}{1.923184in}}%
\pgfpathlineto{\pgfqpoint{5.478048in}{1.923184in}}%
\pgfpathlineto{\pgfqpoint{5.478048in}{1.971521in}}%
\pgfpathlineto{\pgfqpoint{5.469295in}{1.971521in}}%
\pgfpathlineto{\pgfqpoint{5.469295in}{1.923184in}}%
\pgfpathclose%
\pgfusepath{fill}%
\end{pgfscope}%
\begin{pgfscope}%
\pgfpathrectangle{\pgfqpoint{3.776708in}{0.600000in}}{\pgfqpoint{2.573292in}{2.070576in}}%
\pgfusepath{clip}%
\pgfsetbuttcap%
\pgfsetmiterjoin%
\definecolor{currentfill}{rgb}{0.754268,0.565033,0.211761}%
\pgfsetfillcolor{currentfill}%
\pgfsetlinewidth{0.000000pt}%
\definecolor{currentstroke}{rgb}{0.000000,0.000000,0.000000}%
\pgfsetstrokecolor{currentstroke}%
\pgfsetstrokeopacity{0.000000}%
\pgfsetdash{}{0pt}%
\pgfpathmoveto{\pgfqpoint{5.480236in}{1.891728in}}%
\pgfpathlineto{\pgfqpoint{5.488990in}{1.891728in}}%
\pgfpathlineto{\pgfqpoint{5.488990in}{1.938990in}}%
\pgfpathlineto{\pgfqpoint{5.480236in}{1.938990in}}%
\pgfpathlineto{\pgfqpoint{5.480236in}{1.891728in}}%
\pgfpathclose%
\pgfusepath{fill}%
\end{pgfscope}%
\begin{pgfscope}%
\pgfpathrectangle{\pgfqpoint{3.776708in}{0.600000in}}{\pgfqpoint{2.573292in}{2.070576in}}%
\pgfusepath{clip}%
\pgfsetbuttcap%
\pgfsetmiterjoin%
\definecolor{currentfill}{rgb}{0.754268,0.565033,0.211761}%
\pgfsetfillcolor{currentfill}%
\pgfsetlinewidth{0.000000pt}%
\definecolor{currentstroke}{rgb}{0.000000,0.000000,0.000000}%
\pgfsetstrokecolor{currentstroke}%
\pgfsetstrokeopacity{0.000000}%
\pgfsetdash{}{0pt}%
\pgfpathmoveto{\pgfqpoint{5.491178in}{1.858961in}}%
\pgfpathlineto{\pgfqpoint{5.499932in}{1.858961in}}%
\pgfpathlineto{\pgfqpoint{5.499932in}{1.899676in}}%
\pgfpathlineto{\pgfqpoint{5.491178in}{1.899676in}}%
\pgfpathlineto{\pgfqpoint{5.491178in}{1.858961in}}%
\pgfpathclose%
\pgfusepath{fill}%
\end{pgfscope}%
\begin{pgfscope}%
\pgfpathrectangle{\pgfqpoint{3.776708in}{0.600000in}}{\pgfqpoint{2.573292in}{2.070576in}}%
\pgfusepath{clip}%
\pgfsetbuttcap%
\pgfsetmiterjoin%
\definecolor{currentfill}{rgb}{0.754268,0.565033,0.211761}%
\pgfsetfillcolor{currentfill}%
\pgfsetlinewidth{0.000000pt}%
\definecolor{currentstroke}{rgb}{0.000000,0.000000,0.000000}%
\pgfsetstrokecolor{currentstroke}%
\pgfsetstrokeopacity{0.000000}%
\pgfsetdash{}{0pt}%
\pgfpathmoveto{\pgfqpoint{5.502120in}{1.828411in}}%
\pgfpathlineto{\pgfqpoint{5.510873in}{1.828411in}}%
\pgfpathlineto{\pgfqpoint{5.510873in}{1.873444in}}%
\pgfpathlineto{\pgfqpoint{5.502120in}{1.873444in}}%
\pgfpathlineto{\pgfqpoint{5.502120in}{1.828411in}}%
\pgfpathclose%
\pgfusepath{fill}%
\end{pgfscope}%
\begin{pgfscope}%
\pgfpathrectangle{\pgfqpoint{3.776708in}{0.600000in}}{\pgfqpoint{2.573292in}{2.070576in}}%
\pgfusepath{clip}%
\pgfsetbuttcap%
\pgfsetmiterjoin%
\definecolor{currentfill}{rgb}{0.754268,0.565033,0.211761}%
\pgfsetfillcolor{currentfill}%
\pgfsetlinewidth{0.000000pt}%
\definecolor{currentstroke}{rgb}{0.000000,0.000000,0.000000}%
\pgfsetstrokecolor{currentstroke}%
\pgfsetstrokeopacity{0.000000}%
\pgfsetdash{}{0pt}%
\pgfpathmoveto{\pgfqpoint{5.513062in}{1.794597in}}%
\pgfpathlineto{\pgfqpoint{5.521815in}{1.794597in}}%
\pgfpathlineto{\pgfqpoint{5.521815in}{1.839082in}}%
\pgfpathlineto{\pgfqpoint{5.513062in}{1.839082in}}%
\pgfpathlineto{\pgfqpoint{5.513062in}{1.794597in}}%
\pgfpathclose%
\pgfusepath{fill}%
\end{pgfscope}%
\begin{pgfscope}%
\pgfpathrectangle{\pgfqpoint{3.776708in}{0.600000in}}{\pgfqpoint{2.573292in}{2.070576in}}%
\pgfusepath{clip}%
\pgfsetbuttcap%
\pgfsetmiterjoin%
\definecolor{currentfill}{rgb}{0.754268,0.565033,0.211761}%
\pgfsetfillcolor{currentfill}%
\pgfsetlinewidth{0.000000pt}%
\definecolor{currentstroke}{rgb}{0.000000,0.000000,0.000000}%
\pgfsetstrokecolor{currentstroke}%
\pgfsetstrokeopacity{0.000000}%
\pgfsetdash{}{0pt}%
\pgfpathmoveto{\pgfqpoint{5.524004in}{1.756210in}}%
\pgfpathlineto{\pgfqpoint{5.532757in}{1.756210in}}%
\pgfpathlineto{\pgfqpoint{5.532757in}{1.811968in}}%
\pgfpathlineto{\pgfqpoint{5.524004in}{1.811968in}}%
\pgfpathlineto{\pgfqpoint{5.524004in}{1.756210in}}%
\pgfpathclose%
\pgfusepath{fill}%
\end{pgfscope}%
\begin{pgfscope}%
\pgfpathrectangle{\pgfqpoint{3.776708in}{0.600000in}}{\pgfqpoint{2.573292in}{2.070576in}}%
\pgfusepath{clip}%
\pgfsetbuttcap%
\pgfsetmiterjoin%
\definecolor{currentfill}{rgb}{0.754268,0.565033,0.211761}%
\pgfsetfillcolor{currentfill}%
\pgfsetlinewidth{0.000000pt}%
\definecolor{currentstroke}{rgb}{0.000000,0.000000,0.000000}%
\pgfsetstrokecolor{currentstroke}%
\pgfsetstrokeopacity{0.000000}%
\pgfsetdash{}{0pt}%
\pgfpathmoveto{\pgfqpoint{5.534945in}{1.724566in}}%
\pgfpathlineto{\pgfqpoint{5.543699in}{1.724566in}}%
\pgfpathlineto{\pgfqpoint{5.543699in}{1.787581in}}%
\pgfpathlineto{\pgfqpoint{5.534945in}{1.787581in}}%
\pgfpathlineto{\pgfqpoint{5.534945in}{1.724566in}}%
\pgfpathclose%
\pgfusepath{fill}%
\end{pgfscope}%
\begin{pgfscope}%
\pgfpathrectangle{\pgfqpoint{3.776708in}{0.600000in}}{\pgfqpoint{2.573292in}{2.070576in}}%
\pgfusepath{clip}%
\pgfsetbuttcap%
\pgfsetmiterjoin%
\definecolor{currentfill}{rgb}{0.754268,0.565033,0.211761}%
\pgfsetfillcolor{currentfill}%
\pgfsetlinewidth{0.000000pt}%
\definecolor{currentstroke}{rgb}{0.000000,0.000000,0.000000}%
\pgfsetstrokecolor{currentstroke}%
\pgfsetstrokeopacity{0.000000}%
\pgfsetdash{}{0pt}%
\pgfpathmoveto{\pgfqpoint{5.545887in}{1.713949in}}%
\pgfpathlineto{\pgfqpoint{5.554641in}{1.713949in}}%
\pgfpathlineto{\pgfqpoint{5.554641in}{1.787631in}}%
\pgfpathlineto{\pgfqpoint{5.545887in}{1.787631in}}%
\pgfpathlineto{\pgfqpoint{5.545887in}{1.713949in}}%
\pgfpathclose%
\pgfusepath{fill}%
\end{pgfscope}%
\begin{pgfscope}%
\pgfpathrectangle{\pgfqpoint{3.776708in}{0.600000in}}{\pgfqpoint{2.573292in}{2.070576in}}%
\pgfusepath{clip}%
\pgfsetbuttcap%
\pgfsetmiterjoin%
\definecolor{currentfill}{rgb}{0.754268,0.565033,0.211761}%
\pgfsetfillcolor{currentfill}%
\pgfsetlinewidth{0.000000pt}%
\definecolor{currentstroke}{rgb}{0.000000,0.000000,0.000000}%
\pgfsetstrokecolor{currentstroke}%
\pgfsetstrokeopacity{0.000000}%
\pgfsetdash{}{0pt}%
\pgfpathmoveto{\pgfqpoint{5.556829in}{1.707415in}}%
\pgfpathlineto{\pgfqpoint{5.565582in}{1.707415in}}%
\pgfpathlineto{\pgfqpoint{5.565582in}{1.787719in}}%
\pgfpathlineto{\pgfqpoint{5.556829in}{1.787719in}}%
\pgfpathlineto{\pgfqpoint{5.556829in}{1.707415in}}%
\pgfpathclose%
\pgfusepath{fill}%
\end{pgfscope}%
\begin{pgfscope}%
\pgfpathrectangle{\pgfqpoint{3.776708in}{0.600000in}}{\pgfqpoint{2.573292in}{2.070576in}}%
\pgfusepath{clip}%
\pgfsetbuttcap%
\pgfsetmiterjoin%
\definecolor{currentfill}{rgb}{0.754268,0.565033,0.211761}%
\pgfsetfillcolor{currentfill}%
\pgfsetlinewidth{0.000000pt}%
\definecolor{currentstroke}{rgb}{0.000000,0.000000,0.000000}%
\pgfsetstrokecolor{currentstroke}%
\pgfsetstrokeopacity{0.000000}%
\pgfsetdash{}{0pt}%
\pgfpathmoveto{\pgfqpoint{5.567771in}{1.702075in}}%
\pgfpathlineto{\pgfqpoint{5.576524in}{1.702075in}}%
\pgfpathlineto{\pgfqpoint{5.576524in}{1.793666in}}%
\pgfpathlineto{\pgfqpoint{5.567771in}{1.793666in}}%
\pgfpathlineto{\pgfqpoint{5.567771in}{1.702075in}}%
\pgfpathclose%
\pgfusepath{fill}%
\end{pgfscope}%
\begin{pgfscope}%
\pgfpathrectangle{\pgfqpoint{3.776708in}{0.600000in}}{\pgfqpoint{2.573292in}{2.070576in}}%
\pgfusepath{clip}%
\pgfsetbuttcap%
\pgfsetmiterjoin%
\definecolor{currentfill}{rgb}{0.754268,0.565033,0.211761}%
\pgfsetfillcolor{currentfill}%
\pgfsetlinewidth{0.000000pt}%
\definecolor{currentstroke}{rgb}{0.000000,0.000000,0.000000}%
\pgfsetstrokecolor{currentstroke}%
\pgfsetstrokeopacity{0.000000}%
\pgfsetdash{}{0pt}%
\pgfpathmoveto{\pgfqpoint{5.578713in}{1.693748in}}%
\pgfpathlineto{\pgfqpoint{5.587466in}{1.693748in}}%
\pgfpathlineto{\pgfqpoint{5.587466in}{1.795542in}}%
\pgfpathlineto{\pgfqpoint{5.578713in}{1.795542in}}%
\pgfpathlineto{\pgfqpoint{5.578713in}{1.693748in}}%
\pgfpathclose%
\pgfusepath{fill}%
\end{pgfscope}%
\begin{pgfscope}%
\pgfpathrectangle{\pgfqpoint{3.776708in}{0.600000in}}{\pgfqpoint{2.573292in}{2.070576in}}%
\pgfusepath{clip}%
\pgfsetbuttcap%
\pgfsetmiterjoin%
\definecolor{currentfill}{rgb}{0.754268,0.565033,0.211761}%
\pgfsetfillcolor{currentfill}%
\pgfsetlinewidth{0.000000pt}%
\definecolor{currentstroke}{rgb}{0.000000,0.000000,0.000000}%
\pgfsetstrokecolor{currentstroke}%
\pgfsetstrokeopacity{0.000000}%
\pgfsetdash{}{0pt}%
\pgfpathmoveto{\pgfqpoint{5.589654in}{1.686593in}}%
\pgfpathlineto{\pgfqpoint{5.598408in}{1.686593in}}%
\pgfpathlineto{\pgfqpoint{5.598408in}{1.790635in}}%
\pgfpathlineto{\pgfqpoint{5.589654in}{1.790635in}}%
\pgfpathlineto{\pgfqpoint{5.589654in}{1.686593in}}%
\pgfpathclose%
\pgfusepath{fill}%
\end{pgfscope}%
\begin{pgfscope}%
\pgfpathrectangle{\pgfqpoint{3.776708in}{0.600000in}}{\pgfqpoint{2.573292in}{2.070576in}}%
\pgfusepath{clip}%
\pgfsetbuttcap%
\pgfsetmiterjoin%
\definecolor{currentfill}{rgb}{0.754268,0.565033,0.211761}%
\pgfsetfillcolor{currentfill}%
\pgfsetlinewidth{0.000000pt}%
\definecolor{currentstroke}{rgb}{0.000000,0.000000,0.000000}%
\pgfsetstrokecolor{currentstroke}%
\pgfsetstrokeopacity{0.000000}%
\pgfsetdash{}{0pt}%
\pgfpathmoveto{\pgfqpoint{5.600596in}{1.678916in}}%
\pgfpathlineto{\pgfqpoint{5.609350in}{1.678916in}}%
\pgfpathlineto{\pgfqpoint{5.609350in}{1.786291in}}%
\pgfpathlineto{\pgfqpoint{5.600596in}{1.786291in}}%
\pgfpathlineto{\pgfqpoint{5.600596in}{1.678916in}}%
\pgfpathclose%
\pgfusepath{fill}%
\end{pgfscope}%
\begin{pgfscope}%
\pgfpathrectangle{\pgfqpoint{3.776708in}{0.600000in}}{\pgfqpoint{2.573292in}{2.070576in}}%
\pgfusepath{clip}%
\pgfsetbuttcap%
\pgfsetmiterjoin%
\definecolor{currentfill}{rgb}{0.754268,0.565033,0.211761}%
\pgfsetfillcolor{currentfill}%
\pgfsetlinewidth{0.000000pt}%
\definecolor{currentstroke}{rgb}{0.000000,0.000000,0.000000}%
\pgfsetstrokecolor{currentstroke}%
\pgfsetstrokeopacity{0.000000}%
\pgfsetdash{}{0pt}%
\pgfpathmoveto{\pgfqpoint{5.611538in}{1.674279in}}%
\pgfpathlineto{\pgfqpoint{5.620291in}{1.674279in}}%
\pgfpathlineto{\pgfqpoint{5.620291in}{1.788489in}}%
\pgfpathlineto{\pgfqpoint{5.611538in}{1.788489in}}%
\pgfpathlineto{\pgfqpoint{5.611538in}{1.674279in}}%
\pgfpathclose%
\pgfusepath{fill}%
\end{pgfscope}%
\begin{pgfscope}%
\pgfpathrectangle{\pgfqpoint{3.776708in}{0.600000in}}{\pgfqpoint{2.573292in}{2.070576in}}%
\pgfusepath{clip}%
\pgfsetbuttcap%
\pgfsetmiterjoin%
\definecolor{currentfill}{rgb}{0.754268,0.565033,0.211761}%
\pgfsetfillcolor{currentfill}%
\pgfsetlinewidth{0.000000pt}%
\definecolor{currentstroke}{rgb}{0.000000,0.000000,0.000000}%
\pgfsetstrokecolor{currentstroke}%
\pgfsetstrokeopacity{0.000000}%
\pgfsetdash{}{0pt}%
\pgfpathmoveto{\pgfqpoint{5.622480in}{1.671208in}}%
\pgfpathlineto{\pgfqpoint{5.631233in}{1.671208in}}%
\pgfpathlineto{\pgfqpoint{5.631233in}{1.787258in}}%
\pgfpathlineto{\pgfqpoint{5.622480in}{1.787258in}}%
\pgfpathlineto{\pgfqpoint{5.622480in}{1.671208in}}%
\pgfpathclose%
\pgfusepath{fill}%
\end{pgfscope}%
\begin{pgfscope}%
\pgfpathrectangle{\pgfqpoint{3.776708in}{0.600000in}}{\pgfqpoint{2.573292in}{2.070576in}}%
\pgfusepath{clip}%
\pgfsetbuttcap%
\pgfsetmiterjoin%
\definecolor{currentfill}{rgb}{0.754268,0.565033,0.211761}%
\pgfsetfillcolor{currentfill}%
\pgfsetlinewidth{0.000000pt}%
\definecolor{currentstroke}{rgb}{0.000000,0.000000,0.000000}%
\pgfsetstrokecolor{currentstroke}%
\pgfsetstrokeopacity{0.000000}%
\pgfsetdash{}{0pt}%
\pgfpathmoveto{\pgfqpoint{5.633422in}{1.669522in}}%
\pgfpathlineto{\pgfqpoint{5.642175in}{1.669522in}}%
\pgfpathlineto{\pgfqpoint{5.642175in}{1.776753in}}%
\pgfpathlineto{\pgfqpoint{5.633422in}{1.776753in}}%
\pgfpathlineto{\pgfqpoint{5.633422in}{1.669522in}}%
\pgfpathclose%
\pgfusepath{fill}%
\end{pgfscope}%
\begin{pgfscope}%
\pgfpathrectangle{\pgfqpoint{3.776708in}{0.600000in}}{\pgfqpoint{2.573292in}{2.070576in}}%
\pgfusepath{clip}%
\pgfsetbuttcap%
\pgfsetmiterjoin%
\definecolor{currentfill}{rgb}{0.754268,0.565033,0.211761}%
\pgfsetfillcolor{currentfill}%
\pgfsetlinewidth{0.000000pt}%
\definecolor{currentstroke}{rgb}{0.000000,0.000000,0.000000}%
\pgfsetstrokecolor{currentstroke}%
\pgfsetstrokeopacity{0.000000}%
\pgfsetdash{}{0pt}%
\pgfpathmoveto{\pgfqpoint{5.644363in}{1.664850in}}%
\pgfpathlineto{\pgfqpoint{5.653117in}{1.664850in}}%
\pgfpathlineto{\pgfqpoint{5.653117in}{1.763667in}}%
\pgfpathlineto{\pgfqpoint{5.644363in}{1.763667in}}%
\pgfpathlineto{\pgfqpoint{5.644363in}{1.664850in}}%
\pgfpathclose%
\pgfusepath{fill}%
\end{pgfscope}%
\begin{pgfscope}%
\pgfpathrectangle{\pgfqpoint{3.776708in}{0.600000in}}{\pgfqpoint{2.573292in}{2.070576in}}%
\pgfusepath{clip}%
\pgfsetbuttcap%
\pgfsetmiterjoin%
\definecolor{currentfill}{rgb}{0.754268,0.565033,0.211761}%
\pgfsetfillcolor{currentfill}%
\pgfsetlinewidth{0.000000pt}%
\definecolor{currentstroke}{rgb}{0.000000,0.000000,0.000000}%
\pgfsetstrokecolor{currentstroke}%
\pgfsetstrokeopacity{0.000000}%
\pgfsetdash{}{0pt}%
\pgfpathmoveto{\pgfqpoint{5.655305in}{1.666871in}}%
\pgfpathlineto{\pgfqpoint{5.664059in}{1.666871in}}%
\pgfpathlineto{\pgfqpoint{5.664059in}{1.753334in}}%
\pgfpathlineto{\pgfqpoint{5.655305in}{1.753334in}}%
\pgfpathlineto{\pgfqpoint{5.655305in}{1.666871in}}%
\pgfpathclose%
\pgfusepath{fill}%
\end{pgfscope}%
\begin{pgfscope}%
\pgfpathrectangle{\pgfqpoint{3.776708in}{0.600000in}}{\pgfqpoint{2.573292in}{2.070576in}}%
\pgfusepath{clip}%
\pgfsetbuttcap%
\pgfsetmiterjoin%
\definecolor{currentfill}{rgb}{0.754268,0.565033,0.211761}%
\pgfsetfillcolor{currentfill}%
\pgfsetlinewidth{0.000000pt}%
\definecolor{currentstroke}{rgb}{0.000000,0.000000,0.000000}%
\pgfsetstrokecolor{currentstroke}%
\pgfsetstrokeopacity{0.000000}%
\pgfsetdash{}{0pt}%
\pgfpathmoveto{\pgfqpoint{5.666247in}{1.660248in}}%
\pgfpathlineto{\pgfqpoint{5.675000in}{1.660248in}}%
\pgfpathlineto{\pgfqpoint{5.675000in}{1.740842in}}%
\pgfpathlineto{\pgfqpoint{5.666247in}{1.740842in}}%
\pgfpathlineto{\pgfqpoint{5.666247in}{1.660248in}}%
\pgfpathclose%
\pgfusepath{fill}%
\end{pgfscope}%
\begin{pgfscope}%
\pgfpathrectangle{\pgfqpoint{3.776708in}{0.600000in}}{\pgfqpoint{2.573292in}{2.070576in}}%
\pgfusepath{clip}%
\pgfsetbuttcap%
\pgfsetmiterjoin%
\definecolor{currentfill}{rgb}{0.754268,0.565033,0.211761}%
\pgfsetfillcolor{currentfill}%
\pgfsetlinewidth{0.000000pt}%
\definecolor{currentstroke}{rgb}{0.000000,0.000000,0.000000}%
\pgfsetstrokecolor{currentstroke}%
\pgfsetstrokeopacity{0.000000}%
\pgfsetdash{}{0pt}%
\pgfpathmoveto{\pgfqpoint{5.677189in}{1.657128in}}%
\pgfpathlineto{\pgfqpoint{5.685942in}{1.657128in}}%
\pgfpathlineto{\pgfqpoint{5.685942in}{1.722525in}}%
\pgfpathlineto{\pgfqpoint{5.677189in}{1.722525in}}%
\pgfpathlineto{\pgfqpoint{5.677189in}{1.657128in}}%
\pgfpathclose%
\pgfusepath{fill}%
\end{pgfscope}%
\begin{pgfscope}%
\pgfpathrectangle{\pgfqpoint{3.776708in}{0.600000in}}{\pgfqpoint{2.573292in}{2.070576in}}%
\pgfusepath{clip}%
\pgfsetbuttcap%
\pgfsetmiterjoin%
\definecolor{currentfill}{rgb}{0.754268,0.565033,0.211761}%
\pgfsetfillcolor{currentfill}%
\pgfsetlinewidth{0.000000pt}%
\definecolor{currentstroke}{rgb}{0.000000,0.000000,0.000000}%
\pgfsetstrokecolor{currentstroke}%
\pgfsetstrokeopacity{0.000000}%
\pgfsetdash{}{0pt}%
\pgfpathmoveto{\pgfqpoint{5.688131in}{1.656995in}}%
\pgfpathlineto{\pgfqpoint{5.696884in}{1.656995in}}%
\pgfpathlineto{\pgfqpoint{5.696884in}{1.708885in}}%
\pgfpathlineto{\pgfqpoint{5.688131in}{1.708885in}}%
\pgfpathlineto{\pgfqpoint{5.688131in}{1.656995in}}%
\pgfpathclose%
\pgfusepath{fill}%
\end{pgfscope}%
\begin{pgfscope}%
\pgfpathrectangle{\pgfqpoint{3.776708in}{0.600000in}}{\pgfqpoint{2.573292in}{2.070576in}}%
\pgfusepath{clip}%
\pgfsetbuttcap%
\pgfsetmiterjoin%
\definecolor{currentfill}{rgb}{0.754268,0.565033,0.211761}%
\pgfsetfillcolor{currentfill}%
\pgfsetlinewidth{0.000000pt}%
\definecolor{currentstroke}{rgb}{0.000000,0.000000,0.000000}%
\pgfsetstrokecolor{currentstroke}%
\pgfsetstrokeopacity{0.000000}%
\pgfsetdash{}{0pt}%
\pgfpathmoveto{\pgfqpoint{5.699072in}{1.654179in}}%
\pgfpathlineto{\pgfqpoint{5.707826in}{1.654179in}}%
\pgfpathlineto{\pgfqpoint{5.707826in}{1.696572in}}%
\pgfpathlineto{\pgfqpoint{5.699072in}{1.696572in}}%
\pgfpathlineto{\pgfqpoint{5.699072in}{1.654179in}}%
\pgfpathclose%
\pgfusepath{fill}%
\end{pgfscope}%
\begin{pgfscope}%
\pgfpathrectangle{\pgfqpoint{3.776708in}{0.600000in}}{\pgfqpoint{2.573292in}{2.070576in}}%
\pgfusepath{clip}%
\pgfsetbuttcap%
\pgfsetmiterjoin%
\definecolor{currentfill}{rgb}{0.754268,0.565033,0.211761}%
\pgfsetfillcolor{currentfill}%
\pgfsetlinewidth{0.000000pt}%
\definecolor{currentstroke}{rgb}{0.000000,0.000000,0.000000}%
\pgfsetstrokecolor{currentstroke}%
\pgfsetstrokeopacity{0.000000}%
\pgfsetdash{}{0pt}%
\pgfpathmoveto{\pgfqpoint{5.710014in}{1.653626in}}%
\pgfpathlineto{\pgfqpoint{5.718768in}{1.653626in}}%
\pgfpathlineto{\pgfqpoint{5.718768in}{1.686270in}}%
\pgfpathlineto{\pgfqpoint{5.710014in}{1.686270in}}%
\pgfpathlineto{\pgfqpoint{5.710014in}{1.653626in}}%
\pgfpathclose%
\pgfusepath{fill}%
\end{pgfscope}%
\begin{pgfscope}%
\pgfpathrectangle{\pgfqpoint{3.776708in}{0.600000in}}{\pgfqpoint{2.573292in}{2.070576in}}%
\pgfusepath{clip}%
\pgfsetbuttcap%
\pgfsetmiterjoin%
\definecolor{currentfill}{rgb}{0.754268,0.565033,0.211761}%
\pgfsetfillcolor{currentfill}%
\pgfsetlinewidth{0.000000pt}%
\definecolor{currentstroke}{rgb}{0.000000,0.000000,0.000000}%
\pgfsetstrokecolor{currentstroke}%
\pgfsetstrokeopacity{0.000000}%
\pgfsetdash{}{0pt}%
\pgfpathmoveto{\pgfqpoint{5.720956in}{1.651147in}}%
\pgfpathlineto{\pgfqpoint{5.729709in}{1.651147in}}%
\pgfpathlineto{\pgfqpoint{5.729709in}{1.671627in}}%
\pgfpathlineto{\pgfqpoint{5.720956in}{1.671627in}}%
\pgfpathlineto{\pgfqpoint{5.720956in}{1.651147in}}%
\pgfpathclose%
\pgfusepath{fill}%
\end{pgfscope}%
\begin{pgfscope}%
\pgfpathrectangle{\pgfqpoint{3.776708in}{0.600000in}}{\pgfqpoint{2.573292in}{2.070576in}}%
\pgfusepath{clip}%
\pgfsetbuttcap%
\pgfsetmiterjoin%
\definecolor{currentfill}{rgb}{0.754268,0.565033,0.211761}%
\pgfsetfillcolor{currentfill}%
\pgfsetlinewidth{0.000000pt}%
\definecolor{currentstroke}{rgb}{0.000000,0.000000,0.000000}%
\pgfsetstrokecolor{currentstroke}%
\pgfsetstrokeopacity{0.000000}%
\pgfsetdash{}{0pt}%
\pgfpathmoveto{\pgfqpoint{5.731898in}{1.650857in}}%
\pgfpathlineto{\pgfqpoint{5.740651in}{1.650857in}}%
\pgfpathlineto{\pgfqpoint{5.740651in}{1.683252in}}%
\pgfpathlineto{\pgfqpoint{5.731898in}{1.683252in}}%
\pgfpathlineto{\pgfqpoint{5.731898in}{1.650857in}}%
\pgfpathclose%
\pgfusepath{fill}%
\end{pgfscope}%
\begin{pgfscope}%
\pgfpathrectangle{\pgfqpoint{3.776708in}{0.600000in}}{\pgfqpoint{2.573292in}{2.070576in}}%
\pgfusepath{clip}%
\pgfsetbuttcap%
\pgfsetmiterjoin%
\definecolor{currentfill}{rgb}{0.754268,0.565033,0.211761}%
\pgfsetfillcolor{currentfill}%
\pgfsetlinewidth{0.000000pt}%
\definecolor{currentstroke}{rgb}{0.000000,0.000000,0.000000}%
\pgfsetstrokecolor{currentstroke}%
\pgfsetstrokeopacity{0.000000}%
\pgfsetdash{}{0pt}%
\pgfpathmoveto{\pgfqpoint{5.742840in}{1.660875in}}%
\pgfpathlineto{\pgfqpoint{5.751593in}{1.660875in}}%
\pgfpathlineto{\pgfqpoint{5.751593in}{1.705897in}}%
\pgfpathlineto{\pgfqpoint{5.742840in}{1.705897in}}%
\pgfpathlineto{\pgfqpoint{5.742840in}{1.660875in}}%
\pgfpathclose%
\pgfusepath{fill}%
\end{pgfscope}%
\begin{pgfscope}%
\pgfpathrectangle{\pgfqpoint{3.776708in}{0.600000in}}{\pgfqpoint{2.573292in}{2.070576in}}%
\pgfusepath{clip}%
\pgfsetbuttcap%
\pgfsetmiterjoin%
\definecolor{currentfill}{rgb}{0.754268,0.565033,0.211761}%
\pgfsetfillcolor{currentfill}%
\pgfsetlinewidth{0.000000pt}%
\definecolor{currentstroke}{rgb}{0.000000,0.000000,0.000000}%
\pgfsetstrokecolor{currentstroke}%
\pgfsetstrokeopacity{0.000000}%
\pgfsetdash{}{0pt}%
\pgfpathmoveto{\pgfqpoint{5.753781in}{1.672179in}}%
\pgfpathlineto{\pgfqpoint{5.762535in}{1.672179in}}%
\pgfpathlineto{\pgfqpoint{5.762535in}{1.720804in}}%
\pgfpathlineto{\pgfqpoint{5.753781in}{1.720804in}}%
\pgfpathlineto{\pgfqpoint{5.753781in}{1.672179in}}%
\pgfpathclose%
\pgfusepath{fill}%
\end{pgfscope}%
\begin{pgfscope}%
\pgfpathrectangle{\pgfqpoint{3.776708in}{0.600000in}}{\pgfqpoint{2.573292in}{2.070576in}}%
\pgfusepath{clip}%
\pgfsetbuttcap%
\pgfsetmiterjoin%
\definecolor{currentfill}{rgb}{0.754268,0.565033,0.211761}%
\pgfsetfillcolor{currentfill}%
\pgfsetlinewidth{0.000000pt}%
\definecolor{currentstroke}{rgb}{0.000000,0.000000,0.000000}%
\pgfsetstrokecolor{currentstroke}%
\pgfsetstrokeopacity{0.000000}%
\pgfsetdash{}{0pt}%
\pgfpathmoveto{\pgfqpoint{5.764723in}{1.683656in}}%
\pgfpathlineto{\pgfqpoint{5.773477in}{1.683656in}}%
\pgfpathlineto{\pgfqpoint{5.773477in}{1.735535in}}%
\pgfpathlineto{\pgfqpoint{5.764723in}{1.735535in}}%
\pgfpathlineto{\pgfqpoint{5.764723in}{1.683656in}}%
\pgfpathclose%
\pgfusepath{fill}%
\end{pgfscope}%
\begin{pgfscope}%
\pgfpathrectangle{\pgfqpoint{3.776708in}{0.600000in}}{\pgfqpoint{2.573292in}{2.070576in}}%
\pgfusepath{clip}%
\pgfsetbuttcap%
\pgfsetmiterjoin%
\definecolor{currentfill}{rgb}{0.754268,0.565033,0.211761}%
\pgfsetfillcolor{currentfill}%
\pgfsetlinewidth{0.000000pt}%
\definecolor{currentstroke}{rgb}{0.000000,0.000000,0.000000}%
\pgfsetstrokecolor{currentstroke}%
\pgfsetstrokeopacity{0.000000}%
\pgfsetdash{}{0pt}%
\pgfpathmoveto{\pgfqpoint{5.775665in}{1.704292in}}%
\pgfpathlineto{\pgfqpoint{5.784418in}{1.704292in}}%
\pgfpathlineto{\pgfqpoint{5.784418in}{1.762207in}}%
\pgfpathlineto{\pgfqpoint{5.775665in}{1.762207in}}%
\pgfpathlineto{\pgfqpoint{5.775665in}{1.704292in}}%
\pgfpathclose%
\pgfusepath{fill}%
\end{pgfscope}%
\begin{pgfscope}%
\pgfpathrectangle{\pgfqpoint{3.776708in}{0.600000in}}{\pgfqpoint{2.573292in}{2.070576in}}%
\pgfusepath{clip}%
\pgfsetbuttcap%
\pgfsetmiterjoin%
\definecolor{currentfill}{rgb}{0.754268,0.565033,0.211761}%
\pgfsetfillcolor{currentfill}%
\pgfsetlinewidth{0.000000pt}%
\definecolor{currentstroke}{rgb}{0.000000,0.000000,0.000000}%
\pgfsetstrokecolor{currentstroke}%
\pgfsetstrokeopacity{0.000000}%
\pgfsetdash{}{0pt}%
\pgfpathmoveto{\pgfqpoint{5.786607in}{1.740526in}}%
\pgfpathlineto{\pgfqpoint{5.795360in}{1.740526in}}%
\pgfpathlineto{\pgfqpoint{5.795360in}{1.803361in}}%
\pgfpathlineto{\pgfqpoint{5.786607in}{1.803361in}}%
\pgfpathlineto{\pgfqpoint{5.786607in}{1.740526in}}%
\pgfpathclose%
\pgfusepath{fill}%
\end{pgfscope}%
\begin{pgfscope}%
\pgfpathrectangle{\pgfqpoint{3.776708in}{0.600000in}}{\pgfqpoint{2.573292in}{2.070576in}}%
\pgfusepath{clip}%
\pgfsetbuttcap%
\pgfsetmiterjoin%
\definecolor{currentfill}{rgb}{0.754268,0.565033,0.211761}%
\pgfsetfillcolor{currentfill}%
\pgfsetlinewidth{0.000000pt}%
\definecolor{currentstroke}{rgb}{0.000000,0.000000,0.000000}%
\pgfsetstrokecolor{currentstroke}%
\pgfsetstrokeopacity{0.000000}%
\pgfsetdash{}{0pt}%
\pgfpathmoveto{\pgfqpoint{5.797549in}{1.777068in}}%
\pgfpathlineto{\pgfqpoint{5.806302in}{1.777068in}}%
\pgfpathlineto{\pgfqpoint{5.806302in}{1.844642in}}%
\pgfpathlineto{\pgfqpoint{5.797549in}{1.844642in}}%
\pgfpathlineto{\pgfqpoint{5.797549in}{1.777068in}}%
\pgfpathclose%
\pgfusepath{fill}%
\end{pgfscope}%
\begin{pgfscope}%
\pgfpathrectangle{\pgfqpoint{3.776708in}{0.600000in}}{\pgfqpoint{2.573292in}{2.070576in}}%
\pgfusepath{clip}%
\pgfsetbuttcap%
\pgfsetmiterjoin%
\definecolor{currentfill}{rgb}{0.754268,0.565033,0.211761}%
\pgfsetfillcolor{currentfill}%
\pgfsetlinewidth{0.000000pt}%
\definecolor{currentstroke}{rgb}{0.000000,0.000000,0.000000}%
\pgfsetstrokecolor{currentstroke}%
\pgfsetstrokeopacity{0.000000}%
\pgfsetdash{}{0pt}%
\pgfpathmoveto{\pgfqpoint{5.808490in}{1.808837in}}%
\pgfpathlineto{\pgfqpoint{5.817244in}{1.808837in}}%
\pgfpathlineto{\pgfqpoint{5.817244in}{1.883956in}}%
\pgfpathlineto{\pgfqpoint{5.808490in}{1.883956in}}%
\pgfpathlineto{\pgfqpoint{5.808490in}{1.808837in}}%
\pgfpathclose%
\pgfusepath{fill}%
\end{pgfscope}%
\begin{pgfscope}%
\pgfpathrectangle{\pgfqpoint{3.776708in}{0.600000in}}{\pgfqpoint{2.573292in}{2.070576in}}%
\pgfusepath{clip}%
\pgfsetbuttcap%
\pgfsetmiterjoin%
\definecolor{currentfill}{rgb}{0.754268,0.565033,0.211761}%
\pgfsetfillcolor{currentfill}%
\pgfsetlinewidth{0.000000pt}%
\definecolor{currentstroke}{rgb}{0.000000,0.000000,0.000000}%
\pgfsetstrokecolor{currentstroke}%
\pgfsetstrokeopacity{0.000000}%
\pgfsetdash{}{0pt}%
\pgfpathmoveto{\pgfqpoint{5.819432in}{1.836858in}}%
\pgfpathlineto{\pgfqpoint{5.828186in}{1.836858in}}%
\pgfpathlineto{\pgfqpoint{5.828186in}{1.915478in}}%
\pgfpathlineto{\pgfqpoint{5.819432in}{1.915478in}}%
\pgfpathlineto{\pgfqpoint{5.819432in}{1.836858in}}%
\pgfpathclose%
\pgfusepath{fill}%
\end{pgfscope}%
\begin{pgfscope}%
\pgfpathrectangle{\pgfqpoint{3.776708in}{0.600000in}}{\pgfqpoint{2.573292in}{2.070576in}}%
\pgfusepath{clip}%
\pgfsetbuttcap%
\pgfsetmiterjoin%
\definecolor{currentfill}{rgb}{0.754268,0.565033,0.211761}%
\pgfsetfillcolor{currentfill}%
\pgfsetlinewidth{0.000000pt}%
\definecolor{currentstroke}{rgb}{0.000000,0.000000,0.000000}%
\pgfsetstrokecolor{currentstroke}%
\pgfsetstrokeopacity{0.000000}%
\pgfsetdash{}{0pt}%
\pgfpathmoveto{\pgfqpoint{5.830374in}{1.866069in}}%
\pgfpathlineto{\pgfqpoint{5.839127in}{1.866069in}}%
\pgfpathlineto{\pgfqpoint{5.839127in}{1.950185in}}%
\pgfpathlineto{\pgfqpoint{5.830374in}{1.950185in}}%
\pgfpathlineto{\pgfqpoint{5.830374in}{1.866069in}}%
\pgfpathclose%
\pgfusepath{fill}%
\end{pgfscope}%
\begin{pgfscope}%
\pgfpathrectangle{\pgfqpoint{3.776708in}{0.600000in}}{\pgfqpoint{2.573292in}{2.070576in}}%
\pgfusepath{clip}%
\pgfsetbuttcap%
\pgfsetmiterjoin%
\definecolor{currentfill}{rgb}{0.754268,0.565033,0.211761}%
\pgfsetfillcolor{currentfill}%
\pgfsetlinewidth{0.000000pt}%
\definecolor{currentstroke}{rgb}{0.000000,0.000000,0.000000}%
\pgfsetstrokecolor{currentstroke}%
\pgfsetstrokeopacity{0.000000}%
\pgfsetdash{}{0pt}%
\pgfpathmoveto{\pgfqpoint{5.841316in}{1.892429in}}%
\pgfpathlineto{\pgfqpoint{5.850069in}{1.892429in}}%
\pgfpathlineto{\pgfqpoint{5.850069in}{1.976873in}}%
\pgfpathlineto{\pgfqpoint{5.841316in}{1.976873in}}%
\pgfpathlineto{\pgfqpoint{5.841316in}{1.892429in}}%
\pgfpathclose%
\pgfusepath{fill}%
\end{pgfscope}%
\begin{pgfscope}%
\pgfpathrectangle{\pgfqpoint{3.776708in}{0.600000in}}{\pgfqpoint{2.573292in}{2.070576in}}%
\pgfusepath{clip}%
\pgfsetbuttcap%
\pgfsetmiterjoin%
\definecolor{currentfill}{rgb}{0.754268,0.565033,0.211761}%
\pgfsetfillcolor{currentfill}%
\pgfsetlinewidth{0.000000pt}%
\definecolor{currentstroke}{rgb}{0.000000,0.000000,0.000000}%
\pgfsetstrokecolor{currentstroke}%
\pgfsetstrokeopacity{0.000000}%
\pgfsetdash{}{0pt}%
\pgfpathmoveto{\pgfqpoint{5.852258in}{1.910056in}}%
\pgfpathlineto{\pgfqpoint{5.861011in}{1.910056in}}%
\pgfpathlineto{\pgfqpoint{5.861011in}{1.997372in}}%
\pgfpathlineto{\pgfqpoint{5.852258in}{1.997372in}}%
\pgfpathlineto{\pgfqpoint{5.852258in}{1.910056in}}%
\pgfpathclose%
\pgfusepath{fill}%
\end{pgfscope}%
\begin{pgfscope}%
\pgfpathrectangle{\pgfqpoint{3.776708in}{0.600000in}}{\pgfqpoint{2.573292in}{2.070576in}}%
\pgfusepath{clip}%
\pgfsetbuttcap%
\pgfsetmiterjoin%
\definecolor{currentfill}{rgb}{0.754268,0.565033,0.211761}%
\pgfsetfillcolor{currentfill}%
\pgfsetlinewidth{0.000000pt}%
\definecolor{currentstroke}{rgb}{0.000000,0.000000,0.000000}%
\pgfsetstrokecolor{currentstroke}%
\pgfsetstrokeopacity{0.000000}%
\pgfsetdash{}{0pt}%
\pgfpathmoveto{\pgfqpoint{5.863199in}{1.930085in}}%
\pgfpathlineto{\pgfqpoint{5.871953in}{1.930085in}}%
\pgfpathlineto{\pgfqpoint{5.871953in}{2.026979in}}%
\pgfpathlineto{\pgfqpoint{5.863199in}{2.026979in}}%
\pgfpathlineto{\pgfqpoint{5.863199in}{1.930085in}}%
\pgfpathclose%
\pgfusepath{fill}%
\end{pgfscope}%
\begin{pgfscope}%
\pgfpathrectangle{\pgfqpoint{3.776708in}{0.600000in}}{\pgfqpoint{2.573292in}{2.070576in}}%
\pgfusepath{clip}%
\pgfsetbuttcap%
\pgfsetmiterjoin%
\definecolor{currentfill}{rgb}{0.754268,0.565033,0.211761}%
\pgfsetfillcolor{currentfill}%
\pgfsetlinewidth{0.000000pt}%
\definecolor{currentstroke}{rgb}{0.000000,0.000000,0.000000}%
\pgfsetstrokecolor{currentstroke}%
\pgfsetstrokeopacity{0.000000}%
\pgfsetdash{}{0pt}%
\pgfpathmoveto{\pgfqpoint{5.874141in}{1.947692in}}%
\pgfpathlineto{\pgfqpoint{5.882895in}{1.947692in}}%
\pgfpathlineto{\pgfqpoint{5.882895in}{2.048155in}}%
\pgfpathlineto{\pgfqpoint{5.874141in}{2.048155in}}%
\pgfpathlineto{\pgfqpoint{5.874141in}{1.947692in}}%
\pgfpathclose%
\pgfusepath{fill}%
\end{pgfscope}%
\begin{pgfscope}%
\pgfpathrectangle{\pgfqpoint{3.776708in}{0.600000in}}{\pgfqpoint{2.573292in}{2.070576in}}%
\pgfusepath{clip}%
\pgfsetbuttcap%
\pgfsetmiterjoin%
\definecolor{currentfill}{rgb}{0.754268,0.565033,0.211761}%
\pgfsetfillcolor{currentfill}%
\pgfsetlinewidth{0.000000pt}%
\definecolor{currentstroke}{rgb}{0.000000,0.000000,0.000000}%
\pgfsetstrokecolor{currentstroke}%
\pgfsetstrokeopacity{0.000000}%
\pgfsetdash{}{0pt}%
\pgfpathmoveto{\pgfqpoint{5.885083in}{1.959122in}}%
\pgfpathlineto{\pgfqpoint{5.893836in}{1.959122in}}%
\pgfpathlineto{\pgfqpoint{5.893836in}{2.069882in}}%
\pgfpathlineto{\pgfqpoint{5.885083in}{2.069882in}}%
\pgfpathlineto{\pgfqpoint{5.885083in}{1.959122in}}%
\pgfpathclose%
\pgfusepath{fill}%
\end{pgfscope}%
\begin{pgfscope}%
\pgfpathrectangle{\pgfqpoint{3.776708in}{0.600000in}}{\pgfqpoint{2.573292in}{2.070576in}}%
\pgfusepath{clip}%
\pgfsetbuttcap%
\pgfsetmiterjoin%
\definecolor{currentfill}{rgb}{0.754268,0.565033,0.211761}%
\pgfsetfillcolor{currentfill}%
\pgfsetlinewidth{0.000000pt}%
\definecolor{currentstroke}{rgb}{0.000000,0.000000,0.000000}%
\pgfsetstrokecolor{currentstroke}%
\pgfsetstrokeopacity{0.000000}%
\pgfsetdash{}{0pt}%
\pgfpathmoveto{\pgfqpoint{5.896025in}{1.968661in}}%
\pgfpathlineto{\pgfqpoint{5.904778in}{1.968661in}}%
\pgfpathlineto{\pgfqpoint{5.904778in}{2.084009in}}%
\pgfpathlineto{\pgfqpoint{5.896025in}{2.084009in}}%
\pgfpathlineto{\pgfqpoint{5.896025in}{1.968661in}}%
\pgfpathclose%
\pgfusepath{fill}%
\end{pgfscope}%
\begin{pgfscope}%
\pgfpathrectangle{\pgfqpoint{3.776708in}{0.600000in}}{\pgfqpoint{2.573292in}{2.070576in}}%
\pgfusepath{clip}%
\pgfsetbuttcap%
\pgfsetmiterjoin%
\definecolor{currentfill}{rgb}{0.754268,0.565033,0.211761}%
\pgfsetfillcolor{currentfill}%
\pgfsetlinewidth{0.000000pt}%
\definecolor{currentstroke}{rgb}{0.000000,0.000000,0.000000}%
\pgfsetstrokecolor{currentstroke}%
\pgfsetstrokeopacity{0.000000}%
\pgfsetdash{}{0pt}%
\pgfpathmoveto{\pgfqpoint{5.906967in}{1.973143in}}%
\pgfpathlineto{\pgfqpoint{5.915720in}{1.973143in}}%
\pgfpathlineto{\pgfqpoint{5.915720in}{2.095031in}}%
\pgfpathlineto{\pgfqpoint{5.906967in}{2.095031in}}%
\pgfpathlineto{\pgfqpoint{5.906967in}{1.973143in}}%
\pgfpathclose%
\pgfusepath{fill}%
\end{pgfscope}%
\begin{pgfscope}%
\pgfpathrectangle{\pgfqpoint{3.776708in}{0.600000in}}{\pgfqpoint{2.573292in}{2.070576in}}%
\pgfusepath{clip}%
\pgfsetbuttcap%
\pgfsetmiterjoin%
\definecolor{currentfill}{rgb}{0.754268,0.565033,0.211761}%
\pgfsetfillcolor{currentfill}%
\pgfsetlinewidth{0.000000pt}%
\definecolor{currentstroke}{rgb}{0.000000,0.000000,0.000000}%
\pgfsetstrokecolor{currentstroke}%
\pgfsetstrokeopacity{0.000000}%
\pgfsetdash{}{0pt}%
\pgfpathmoveto{\pgfqpoint{5.917908in}{1.976428in}}%
\pgfpathlineto{\pgfqpoint{5.926662in}{1.976428in}}%
\pgfpathlineto{\pgfqpoint{5.926662in}{2.108958in}}%
\pgfpathlineto{\pgfqpoint{5.917908in}{2.108958in}}%
\pgfpathlineto{\pgfqpoint{5.917908in}{1.976428in}}%
\pgfpathclose%
\pgfusepath{fill}%
\end{pgfscope}%
\begin{pgfscope}%
\pgfpathrectangle{\pgfqpoint{3.776708in}{0.600000in}}{\pgfqpoint{2.573292in}{2.070576in}}%
\pgfusepath{clip}%
\pgfsetbuttcap%
\pgfsetmiterjoin%
\definecolor{currentfill}{rgb}{0.754268,0.565033,0.211761}%
\pgfsetfillcolor{currentfill}%
\pgfsetlinewidth{0.000000pt}%
\definecolor{currentstroke}{rgb}{0.000000,0.000000,0.000000}%
\pgfsetstrokecolor{currentstroke}%
\pgfsetstrokeopacity{0.000000}%
\pgfsetdash{}{0pt}%
\pgfpathmoveto{\pgfqpoint{5.928850in}{1.980451in}}%
\pgfpathlineto{\pgfqpoint{5.937604in}{1.980451in}}%
\pgfpathlineto{\pgfqpoint{5.937604in}{2.118098in}}%
\pgfpathlineto{\pgfqpoint{5.928850in}{2.118098in}}%
\pgfpathlineto{\pgfqpoint{5.928850in}{1.980451in}}%
\pgfpathclose%
\pgfusepath{fill}%
\end{pgfscope}%
\begin{pgfscope}%
\pgfpathrectangle{\pgfqpoint{3.776708in}{0.600000in}}{\pgfqpoint{2.573292in}{2.070576in}}%
\pgfusepath{clip}%
\pgfsetbuttcap%
\pgfsetmiterjoin%
\definecolor{currentfill}{rgb}{0.754268,0.565033,0.211761}%
\pgfsetfillcolor{currentfill}%
\pgfsetlinewidth{0.000000pt}%
\definecolor{currentstroke}{rgb}{0.000000,0.000000,0.000000}%
\pgfsetstrokecolor{currentstroke}%
\pgfsetstrokeopacity{0.000000}%
\pgfsetdash{}{0pt}%
\pgfpathmoveto{\pgfqpoint{5.939792in}{1.980467in}}%
\pgfpathlineto{\pgfqpoint{5.948545in}{1.980467in}}%
\pgfpathlineto{\pgfqpoint{5.948545in}{2.114394in}}%
\pgfpathlineto{\pgfqpoint{5.939792in}{2.114394in}}%
\pgfpathlineto{\pgfqpoint{5.939792in}{1.980467in}}%
\pgfpathclose%
\pgfusepath{fill}%
\end{pgfscope}%
\begin{pgfscope}%
\pgfpathrectangle{\pgfqpoint{3.776708in}{0.600000in}}{\pgfqpoint{2.573292in}{2.070576in}}%
\pgfusepath{clip}%
\pgfsetbuttcap%
\pgfsetmiterjoin%
\definecolor{currentfill}{rgb}{0.754268,0.565033,0.211761}%
\pgfsetfillcolor{currentfill}%
\pgfsetlinewidth{0.000000pt}%
\definecolor{currentstroke}{rgb}{0.000000,0.000000,0.000000}%
\pgfsetstrokecolor{currentstroke}%
\pgfsetstrokeopacity{0.000000}%
\pgfsetdash{}{0pt}%
\pgfpathmoveto{\pgfqpoint{5.950734in}{1.979121in}}%
\pgfpathlineto{\pgfqpoint{5.959487in}{1.979121in}}%
\pgfpathlineto{\pgfqpoint{5.959487in}{2.112691in}}%
\pgfpathlineto{\pgfqpoint{5.950734in}{2.112691in}}%
\pgfpathlineto{\pgfqpoint{5.950734in}{1.979121in}}%
\pgfpathclose%
\pgfusepath{fill}%
\end{pgfscope}%
\begin{pgfscope}%
\pgfpathrectangle{\pgfqpoint{3.776708in}{0.600000in}}{\pgfqpoint{2.573292in}{2.070576in}}%
\pgfusepath{clip}%
\pgfsetbuttcap%
\pgfsetmiterjoin%
\definecolor{currentfill}{rgb}{0.754268,0.565033,0.211761}%
\pgfsetfillcolor{currentfill}%
\pgfsetlinewidth{0.000000pt}%
\definecolor{currentstroke}{rgb}{0.000000,0.000000,0.000000}%
\pgfsetstrokecolor{currentstroke}%
\pgfsetstrokeopacity{0.000000}%
\pgfsetdash{}{0pt}%
\pgfpathmoveto{\pgfqpoint{5.961676in}{1.980276in}}%
\pgfpathlineto{\pgfqpoint{5.970429in}{1.980276in}}%
\pgfpathlineto{\pgfqpoint{5.970429in}{2.122233in}}%
\pgfpathlineto{\pgfqpoint{5.961676in}{2.122233in}}%
\pgfpathlineto{\pgfqpoint{5.961676in}{1.980276in}}%
\pgfpathclose%
\pgfusepath{fill}%
\end{pgfscope}%
\begin{pgfscope}%
\pgfpathrectangle{\pgfqpoint{3.776708in}{0.600000in}}{\pgfqpoint{2.573292in}{2.070576in}}%
\pgfusepath{clip}%
\pgfsetbuttcap%
\pgfsetmiterjoin%
\definecolor{currentfill}{rgb}{0.754268,0.565033,0.211761}%
\pgfsetfillcolor{currentfill}%
\pgfsetlinewidth{0.000000pt}%
\definecolor{currentstroke}{rgb}{0.000000,0.000000,0.000000}%
\pgfsetstrokecolor{currentstroke}%
\pgfsetstrokeopacity{0.000000}%
\pgfsetdash{}{0pt}%
\pgfpathmoveto{\pgfqpoint{5.972617in}{1.983382in}}%
\pgfpathlineto{\pgfqpoint{5.981371in}{1.983382in}}%
\pgfpathlineto{\pgfqpoint{5.981371in}{2.130848in}}%
\pgfpathlineto{\pgfqpoint{5.972617in}{2.130848in}}%
\pgfpathlineto{\pgfqpoint{5.972617in}{1.983382in}}%
\pgfpathclose%
\pgfusepath{fill}%
\end{pgfscope}%
\begin{pgfscope}%
\pgfpathrectangle{\pgfqpoint{3.776708in}{0.600000in}}{\pgfqpoint{2.573292in}{2.070576in}}%
\pgfusepath{clip}%
\pgfsetbuttcap%
\pgfsetmiterjoin%
\definecolor{currentfill}{rgb}{0.754268,0.565033,0.211761}%
\pgfsetfillcolor{currentfill}%
\pgfsetlinewidth{0.000000pt}%
\definecolor{currentstroke}{rgb}{0.000000,0.000000,0.000000}%
\pgfsetstrokecolor{currentstroke}%
\pgfsetstrokeopacity{0.000000}%
\pgfsetdash{}{0pt}%
\pgfpathmoveto{\pgfqpoint{5.983559in}{1.981129in}}%
\pgfpathlineto{\pgfqpoint{5.992313in}{1.981129in}}%
\pgfpathlineto{\pgfqpoint{5.992313in}{2.128416in}}%
\pgfpathlineto{\pgfqpoint{5.983559in}{2.128416in}}%
\pgfpathlineto{\pgfqpoint{5.983559in}{1.981129in}}%
\pgfpathclose%
\pgfusepath{fill}%
\end{pgfscope}%
\begin{pgfscope}%
\pgfpathrectangle{\pgfqpoint{3.776708in}{0.600000in}}{\pgfqpoint{2.573292in}{2.070576in}}%
\pgfusepath{clip}%
\pgfsetbuttcap%
\pgfsetmiterjoin%
\definecolor{currentfill}{rgb}{0.754268,0.565033,0.211761}%
\pgfsetfillcolor{currentfill}%
\pgfsetlinewidth{0.000000pt}%
\definecolor{currentstroke}{rgb}{0.000000,0.000000,0.000000}%
\pgfsetstrokecolor{currentstroke}%
\pgfsetstrokeopacity{0.000000}%
\pgfsetdash{}{0pt}%
\pgfpathmoveto{\pgfqpoint{5.994501in}{1.978236in}}%
\pgfpathlineto{\pgfqpoint{6.003254in}{1.978236in}}%
\pgfpathlineto{\pgfqpoint{6.003254in}{2.127056in}}%
\pgfpathlineto{\pgfqpoint{5.994501in}{2.127056in}}%
\pgfpathlineto{\pgfqpoint{5.994501in}{1.978236in}}%
\pgfpathclose%
\pgfusepath{fill}%
\end{pgfscope}%
\begin{pgfscope}%
\pgfpathrectangle{\pgfqpoint{3.776708in}{0.600000in}}{\pgfqpoint{2.573292in}{2.070576in}}%
\pgfusepath{clip}%
\pgfsetbuttcap%
\pgfsetmiterjoin%
\definecolor{currentfill}{rgb}{0.754268,0.565033,0.211761}%
\pgfsetfillcolor{currentfill}%
\pgfsetlinewidth{0.000000pt}%
\definecolor{currentstroke}{rgb}{0.000000,0.000000,0.000000}%
\pgfsetstrokecolor{currentstroke}%
\pgfsetstrokeopacity{0.000000}%
\pgfsetdash{}{0pt}%
\pgfpathmoveto{\pgfqpoint{6.005443in}{1.977570in}}%
\pgfpathlineto{\pgfqpoint{6.014196in}{1.977570in}}%
\pgfpathlineto{\pgfqpoint{6.014196in}{2.126061in}}%
\pgfpathlineto{\pgfqpoint{6.005443in}{2.126061in}}%
\pgfpathlineto{\pgfqpoint{6.005443in}{1.977570in}}%
\pgfpathclose%
\pgfusepath{fill}%
\end{pgfscope}%
\begin{pgfscope}%
\pgfpathrectangle{\pgfqpoint{3.776708in}{0.600000in}}{\pgfqpoint{2.573292in}{2.070576in}}%
\pgfusepath{clip}%
\pgfsetbuttcap%
\pgfsetmiterjoin%
\definecolor{currentfill}{rgb}{0.754268,0.565033,0.211761}%
\pgfsetfillcolor{currentfill}%
\pgfsetlinewidth{0.000000pt}%
\definecolor{currentstroke}{rgb}{0.000000,0.000000,0.000000}%
\pgfsetstrokecolor{currentstroke}%
\pgfsetstrokeopacity{0.000000}%
\pgfsetdash{}{0pt}%
\pgfpathmoveto{\pgfqpoint{6.016385in}{1.975209in}}%
\pgfpathlineto{\pgfqpoint{6.025138in}{1.975209in}}%
\pgfpathlineto{\pgfqpoint{6.025138in}{2.114059in}}%
\pgfpathlineto{\pgfqpoint{6.016385in}{2.114059in}}%
\pgfpathlineto{\pgfqpoint{6.016385in}{1.975209in}}%
\pgfpathclose%
\pgfusepath{fill}%
\end{pgfscope}%
\begin{pgfscope}%
\pgfpathrectangle{\pgfqpoint{3.776708in}{0.600000in}}{\pgfqpoint{2.573292in}{2.070576in}}%
\pgfusepath{clip}%
\pgfsetbuttcap%
\pgfsetmiterjoin%
\definecolor{currentfill}{rgb}{0.754268,0.565033,0.211761}%
\pgfsetfillcolor{currentfill}%
\pgfsetlinewidth{0.000000pt}%
\definecolor{currentstroke}{rgb}{0.000000,0.000000,0.000000}%
\pgfsetstrokecolor{currentstroke}%
\pgfsetstrokeopacity{0.000000}%
\pgfsetdash{}{0pt}%
\pgfpathmoveto{\pgfqpoint{6.027326in}{1.972868in}}%
\pgfpathlineto{\pgfqpoint{6.036080in}{1.972868in}}%
\pgfpathlineto{\pgfqpoint{6.036080in}{2.107808in}}%
\pgfpathlineto{\pgfqpoint{6.027326in}{2.107808in}}%
\pgfpathlineto{\pgfqpoint{6.027326in}{1.972868in}}%
\pgfpathclose%
\pgfusepath{fill}%
\end{pgfscope}%
\begin{pgfscope}%
\pgfpathrectangle{\pgfqpoint{3.776708in}{0.600000in}}{\pgfqpoint{2.573292in}{2.070576in}}%
\pgfusepath{clip}%
\pgfsetbuttcap%
\pgfsetmiterjoin%
\definecolor{currentfill}{rgb}{0.754268,0.565033,0.211761}%
\pgfsetfillcolor{currentfill}%
\pgfsetlinewidth{0.000000pt}%
\definecolor{currentstroke}{rgb}{0.000000,0.000000,0.000000}%
\pgfsetstrokecolor{currentstroke}%
\pgfsetstrokeopacity{0.000000}%
\pgfsetdash{}{0pt}%
\pgfpathmoveto{\pgfqpoint{6.038268in}{1.971539in}}%
\pgfpathlineto{\pgfqpoint{6.047022in}{1.971539in}}%
\pgfpathlineto{\pgfqpoint{6.047022in}{2.099478in}}%
\pgfpathlineto{\pgfqpoint{6.038268in}{2.099478in}}%
\pgfpathlineto{\pgfqpoint{6.038268in}{1.971539in}}%
\pgfpathclose%
\pgfusepath{fill}%
\end{pgfscope}%
\begin{pgfscope}%
\pgfpathrectangle{\pgfqpoint{3.776708in}{0.600000in}}{\pgfqpoint{2.573292in}{2.070576in}}%
\pgfusepath{clip}%
\pgfsetbuttcap%
\pgfsetmiterjoin%
\definecolor{currentfill}{rgb}{0.754268,0.565033,0.211761}%
\pgfsetfillcolor{currentfill}%
\pgfsetlinewidth{0.000000pt}%
\definecolor{currentstroke}{rgb}{0.000000,0.000000,0.000000}%
\pgfsetstrokecolor{currentstroke}%
\pgfsetstrokeopacity{0.000000}%
\pgfsetdash{}{0pt}%
\pgfpathmoveto{\pgfqpoint{6.049210in}{1.979791in}}%
\pgfpathlineto{\pgfqpoint{6.057963in}{1.979791in}}%
\pgfpathlineto{\pgfqpoint{6.057963in}{2.107556in}}%
\pgfpathlineto{\pgfqpoint{6.049210in}{2.107556in}}%
\pgfpathlineto{\pgfqpoint{6.049210in}{1.979791in}}%
\pgfpathclose%
\pgfusepath{fill}%
\end{pgfscope}%
\begin{pgfscope}%
\pgfpathrectangle{\pgfqpoint{3.776708in}{0.600000in}}{\pgfqpoint{2.573292in}{2.070576in}}%
\pgfusepath{clip}%
\pgfsetbuttcap%
\pgfsetmiterjoin%
\definecolor{currentfill}{rgb}{0.754268,0.565033,0.211761}%
\pgfsetfillcolor{currentfill}%
\pgfsetlinewidth{0.000000pt}%
\definecolor{currentstroke}{rgb}{0.000000,0.000000,0.000000}%
\pgfsetstrokecolor{currentstroke}%
\pgfsetstrokeopacity{0.000000}%
\pgfsetdash{}{0pt}%
\pgfpathmoveto{\pgfqpoint{6.060152in}{1.987154in}}%
\pgfpathlineto{\pgfqpoint{6.068905in}{1.987154in}}%
\pgfpathlineto{\pgfqpoint{6.068905in}{2.104883in}}%
\pgfpathlineto{\pgfqpoint{6.060152in}{2.104883in}}%
\pgfpathlineto{\pgfqpoint{6.060152in}{1.987154in}}%
\pgfpathclose%
\pgfusepath{fill}%
\end{pgfscope}%
\begin{pgfscope}%
\pgfpathrectangle{\pgfqpoint{3.776708in}{0.600000in}}{\pgfqpoint{2.573292in}{2.070576in}}%
\pgfusepath{clip}%
\pgfsetbuttcap%
\pgfsetmiterjoin%
\definecolor{currentfill}{rgb}{0.754268,0.565033,0.211761}%
\pgfsetfillcolor{currentfill}%
\pgfsetlinewidth{0.000000pt}%
\definecolor{currentstroke}{rgb}{0.000000,0.000000,0.000000}%
\pgfsetstrokecolor{currentstroke}%
\pgfsetstrokeopacity{0.000000}%
\pgfsetdash{}{0pt}%
\pgfpathmoveto{\pgfqpoint{6.071094in}{1.992652in}}%
\pgfpathlineto{\pgfqpoint{6.079847in}{1.992652in}}%
\pgfpathlineto{\pgfqpoint{6.079847in}{2.102802in}}%
\pgfpathlineto{\pgfqpoint{6.071094in}{2.102802in}}%
\pgfpathlineto{\pgfqpoint{6.071094in}{1.992652in}}%
\pgfpathclose%
\pgfusepath{fill}%
\end{pgfscope}%
\begin{pgfscope}%
\pgfpathrectangle{\pgfqpoint{3.776708in}{0.600000in}}{\pgfqpoint{2.573292in}{2.070576in}}%
\pgfusepath{clip}%
\pgfsetbuttcap%
\pgfsetmiterjoin%
\definecolor{currentfill}{rgb}{0.754268,0.565033,0.211761}%
\pgfsetfillcolor{currentfill}%
\pgfsetlinewidth{0.000000pt}%
\definecolor{currentstroke}{rgb}{0.000000,0.000000,0.000000}%
\pgfsetstrokecolor{currentstroke}%
\pgfsetstrokeopacity{0.000000}%
\pgfsetdash{}{0pt}%
\pgfpathmoveto{\pgfqpoint{6.082035in}{1.998369in}}%
\pgfpathlineto{\pgfqpoint{6.090789in}{1.998369in}}%
\pgfpathlineto{\pgfqpoint{6.090789in}{2.098296in}}%
\pgfpathlineto{\pgfqpoint{6.082035in}{2.098296in}}%
\pgfpathlineto{\pgfqpoint{6.082035in}{1.998369in}}%
\pgfpathclose%
\pgfusepath{fill}%
\end{pgfscope}%
\begin{pgfscope}%
\pgfpathrectangle{\pgfqpoint{3.776708in}{0.600000in}}{\pgfqpoint{2.573292in}{2.070576in}}%
\pgfusepath{clip}%
\pgfsetbuttcap%
\pgfsetmiterjoin%
\definecolor{currentfill}{rgb}{0.754268,0.565033,0.211761}%
\pgfsetfillcolor{currentfill}%
\pgfsetlinewidth{0.000000pt}%
\definecolor{currentstroke}{rgb}{0.000000,0.000000,0.000000}%
\pgfsetstrokecolor{currentstroke}%
\pgfsetstrokeopacity{0.000000}%
\pgfsetdash{}{0pt}%
\pgfpathmoveto{\pgfqpoint{6.092977in}{2.005265in}}%
\pgfpathlineto{\pgfqpoint{6.101731in}{2.005265in}}%
\pgfpathlineto{\pgfqpoint{6.101731in}{2.100685in}}%
\pgfpathlineto{\pgfqpoint{6.092977in}{2.100685in}}%
\pgfpathlineto{\pgfqpoint{6.092977in}{2.005265in}}%
\pgfpathclose%
\pgfusepath{fill}%
\end{pgfscope}%
\begin{pgfscope}%
\pgfpathrectangle{\pgfqpoint{3.776708in}{0.600000in}}{\pgfqpoint{2.573292in}{2.070576in}}%
\pgfusepath{clip}%
\pgfsetbuttcap%
\pgfsetmiterjoin%
\definecolor{currentfill}{rgb}{0.754268,0.565033,0.211761}%
\pgfsetfillcolor{currentfill}%
\pgfsetlinewidth{0.000000pt}%
\definecolor{currentstroke}{rgb}{0.000000,0.000000,0.000000}%
\pgfsetstrokecolor{currentstroke}%
\pgfsetstrokeopacity{0.000000}%
\pgfsetdash{}{0pt}%
\pgfpathmoveto{\pgfqpoint{6.103919in}{2.007011in}}%
\pgfpathlineto{\pgfqpoint{6.112672in}{2.007011in}}%
\pgfpathlineto{\pgfqpoint{6.112672in}{2.091086in}}%
\pgfpathlineto{\pgfqpoint{6.103919in}{2.091086in}}%
\pgfpathlineto{\pgfqpoint{6.103919in}{2.007011in}}%
\pgfpathclose%
\pgfusepath{fill}%
\end{pgfscope}%
\begin{pgfscope}%
\pgfpathrectangle{\pgfqpoint{3.776708in}{0.600000in}}{\pgfqpoint{2.573292in}{2.070576in}}%
\pgfusepath{clip}%
\pgfsetbuttcap%
\pgfsetmiterjoin%
\definecolor{currentfill}{rgb}{0.754268,0.565033,0.211761}%
\pgfsetfillcolor{currentfill}%
\pgfsetlinewidth{0.000000pt}%
\definecolor{currentstroke}{rgb}{0.000000,0.000000,0.000000}%
\pgfsetstrokecolor{currentstroke}%
\pgfsetstrokeopacity{0.000000}%
\pgfsetdash{}{0pt}%
\pgfpathmoveto{\pgfqpoint{6.114861in}{2.009813in}}%
\pgfpathlineto{\pgfqpoint{6.123614in}{2.009813in}}%
\pgfpathlineto{\pgfqpoint{6.123614in}{2.079994in}}%
\pgfpathlineto{\pgfqpoint{6.114861in}{2.079994in}}%
\pgfpathlineto{\pgfqpoint{6.114861in}{2.009813in}}%
\pgfpathclose%
\pgfusepath{fill}%
\end{pgfscope}%
\begin{pgfscope}%
\pgfpathrectangle{\pgfqpoint{3.776708in}{0.600000in}}{\pgfqpoint{2.573292in}{2.070576in}}%
\pgfusepath{clip}%
\pgfsetbuttcap%
\pgfsetmiterjoin%
\definecolor{currentfill}{rgb}{0.754268,0.565033,0.211761}%
\pgfsetfillcolor{currentfill}%
\pgfsetlinewidth{0.000000pt}%
\definecolor{currentstroke}{rgb}{0.000000,0.000000,0.000000}%
\pgfsetstrokecolor{currentstroke}%
\pgfsetstrokeopacity{0.000000}%
\pgfsetdash{}{0pt}%
\pgfpathmoveto{\pgfqpoint{6.125803in}{2.023679in}}%
\pgfpathlineto{\pgfqpoint{6.134556in}{2.023679in}}%
\pgfpathlineto{\pgfqpoint{6.134556in}{2.087963in}}%
\pgfpathlineto{\pgfqpoint{6.125803in}{2.087963in}}%
\pgfpathlineto{\pgfqpoint{6.125803in}{2.023679in}}%
\pgfpathclose%
\pgfusepath{fill}%
\end{pgfscope}%
\begin{pgfscope}%
\pgfpathrectangle{\pgfqpoint{3.776708in}{0.600000in}}{\pgfqpoint{2.573292in}{2.070576in}}%
\pgfusepath{clip}%
\pgfsetbuttcap%
\pgfsetmiterjoin%
\definecolor{currentfill}{rgb}{0.754268,0.565033,0.211761}%
\pgfsetfillcolor{currentfill}%
\pgfsetlinewidth{0.000000pt}%
\definecolor{currentstroke}{rgb}{0.000000,0.000000,0.000000}%
\pgfsetstrokecolor{currentstroke}%
\pgfsetstrokeopacity{0.000000}%
\pgfsetdash{}{0pt}%
\pgfpathmoveto{\pgfqpoint{6.136744in}{2.039491in}}%
\pgfpathlineto{\pgfqpoint{6.145498in}{2.039491in}}%
\pgfpathlineto{\pgfqpoint{6.145498in}{2.097791in}}%
\pgfpathlineto{\pgfqpoint{6.136744in}{2.097791in}}%
\pgfpathlineto{\pgfqpoint{6.136744in}{2.039491in}}%
\pgfpathclose%
\pgfusepath{fill}%
\end{pgfscope}%
\begin{pgfscope}%
\pgfpathrectangle{\pgfqpoint{3.776708in}{0.600000in}}{\pgfqpoint{2.573292in}{2.070576in}}%
\pgfusepath{clip}%
\pgfsetbuttcap%
\pgfsetmiterjoin%
\definecolor{currentfill}{rgb}{0.754268,0.565033,0.211761}%
\pgfsetfillcolor{currentfill}%
\pgfsetlinewidth{0.000000pt}%
\definecolor{currentstroke}{rgb}{0.000000,0.000000,0.000000}%
\pgfsetstrokecolor{currentstroke}%
\pgfsetstrokeopacity{0.000000}%
\pgfsetdash{}{0pt}%
\pgfpathmoveto{\pgfqpoint{6.147686in}{2.054565in}}%
\pgfpathlineto{\pgfqpoint{6.156440in}{2.054565in}}%
\pgfpathlineto{\pgfqpoint{6.156440in}{2.109995in}}%
\pgfpathlineto{\pgfqpoint{6.147686in}{2.109995in}}%
\pgfpathlineto{\pgfqpoint{6.147686in}{2.054565in}}%
\pgfpathclose%
\pgfusepath{fill}%
\end{pgfscope}%
\begin{pgfscope}%
\pgfpathrectangle{\pgfqpoint{3.776708in}{0.600000in}}{\pgfqpoint{2.573292in}{2.070576in}}%
\pgfusepath{clip}%
\pgfsetbuttcap%
\pgfsetmiterjoin%
\definecolor{currentfill}{rgb}{0.754268,0.565033,0.211761}%
\pgfsetfillcolor{currentfill}%
\pgfsetlinewidth{0.000000pt}%
\definecolor{currentstroke}{rgb}{0.000000,0.000000,0.000000}%
\pgfsetstrokecolor{currentstroke}%
\pgfsetstrokeopacity{0.000000}%
\pgfsetdash{}{0pt}%
\pgfpathmoveto{\pgfqpoint{6.158628in}{2.068040in}}%
\pgfpathlineto{\pgfqpoint{6.167381in}{2.068040in}}%
\pgfpathlineto{\pgfqpoint{6.167381in}{2.113936in}}%
\pgfpathlineto{\pgfqpoint{6.158628in}{2.113936in}}%
\pgfpathlineto{\pgfqpoint{6.158628in}{2.068040in}}%
\pgfpathclose%
\pgfusepath{fill}%
\end{pgfscope}%
\begin{pgfscope}%
\pgfpathrectangle{\pgfqpoint{3.776708in}{0.600000in}}{\pgfqpoint{2.573292in}{2.070576in}}%
\pgfusepath{clip}%
\pgfsetbuttcap%
\pgfsetmiterjoin%
\definecolor{currentfill}{rgb}{0.754268,0.565033,0.211761}%
\pgfsetfillcolor{currentfill}%
\pgfsetlinewidth{0.000000pt}%
\definecolor{currentstroke}{rgb}{0.000000,0.000000,0.000000}%
\pgfsetstrokecolor{currentstroke}%
\pgfsetstrokeopacity{0.000000}%
\pgfsetdash{}{0pt}%
\pgfpathmoveto{\pgfqpoint{6.169570in}{2.076509in}}%
\pgfpathlineto{\pgfqpoint{6.178323in}{2.076509in}}%
\pgfpathlineto{\pgfqpoint{6.178323in}{2.112128in}}%
\pgfpathlineto{\pgfqpoint{6.169570in}{2.112128in}}%
\pgfpathlineto{\pgfqpoint{6.169570in}{2.076509in}}%
\pgfpathclose%
\pgfusepath{fill}%
\end{pgfscope}%
\begin{pgfscope}%
\pgfpathrectangle{\pgfqpoint{3.776708in}{0.600000in}}{\pgfqpoint{2.573292in}{2.070576in}}%
\pgfusepath{clip}%
\pgfsetbuttcap%
\pgfsetmiterjoin%
\definecolor{currentfill}{rgb}{0.754268,0.565033,0.211761}%
\pgfsetfillcolor{currentfill}%
\pgfsetlinewidth{0.000000pt}%
\definecolor{currentstroke}{rgb}{0.000000,0.000000,0.000000}%
\pgfsetstrokecolor{currentstroke}%
\pgfsetstrokeopacity{0.000000}%
\pgfsetdash{}{0pt}%
\pgfpathmoveto{\pgfqpoint{6.180512in}{2.086150in}}%
\pgfpathlineto{\pgfqpoint{6.189265in}{2.086150in}}%
\pgfpathlineto{\pgfqpoint{6.189265in}{2.110591in}}%
\pgfpathlineto{\pgfqpoint{6.180512in}{2.110591in}}%
\pgfpathlineto{\pgfqpoint{6.180512in}{2.086150in}}%
\pgfpathclose%
\pgfusepath{fill}%
\end{pgfscope}%
\begin{pgfscope}%
\pgfpathrectangle{\pgfqpoint{3.776708in}{0.600000in}}{\pgfqpoint{2.573292in}{2.070576in}}%
\pgfusepath{clip}%
\pgfsetbuttcap%
\pgfsetmiterjoin%
\definecolor{currentfill}{rgb}{0.754268,0.565033,0.211761}%
\pgfsetfillcolor{currentfill}%
\pgfsetlinewidth{0.000000pt}%
\definecolor{currentstroke}{rgb}{0.000000,0.000000,0.000000}%
\pgfsetstrokecolor{currentstroke}%
\pgfsetstrokeopacity{0.000000}%
\pgfsetdash{}{0pt}%
\pgfpathmoveto{\pgfqpoint{6.191453in}{2.095758in}}%
\pgfpathlineto{\pgfqpoint{6.200207in}{2.095758in}}%
\pgfpathlineto{\pgfqpoint{6.200207in}{2.097617in}}%
\pgfpathlineto{\pgfqpoint{6.191453in}{2.097617in}}%
\pgfpathlineto{\pgfqpoint{6.191453in}{2.095758in}}%
\pgfpathclose%
\pgfusepath{fill}%
\end{pgfscope}%
\begin{pgfscope}%
\pgfpathrectangle{\pgfqpoint{3.776708in}{0.600000in}}{\pgfqpoint{2.573292in}{2.070576in}}%
\pgfusepath{clip}%
\pgfsetbuttcap%
\pgfsetmiterjoin%
\definecolor{currentfill}{rgb}{0.754268,0.565033,0.211761}%
\pgfsetfillcolor{currentfill}%
\pgfsetlinewidth{0.000000pt}%
\definecolor{currentstroke}{rgb}{0.000000,0.000000,0.000000}%
\pgfsetstrokecolor{currentstroke}%
\pgfsetstrokeopacity{0.000000}%
\pgfsetdash{}{0pt}%
\pgfpathmoveto{\pgfqpoint{6.202395in}{1.118516in}}%
\pgfpathlineto{\pgfqpoint{6.211149in}{1.118516in}}%
\pgfpathlineto{\pgfqpoint{6.211149in}{1.099240in}}%
\pgfpathlineto{\pgfqpoint{6.202395in}{1.099240in}}%
\pgfpathlineto{\pgfqpoint{6.202395in}{1.118516in}}%
\pgfpathclose%
\pgfusepath{fill}%
\end{pgfscope}%
\begin{pgfscope}%
\pgfpathrectangle{\pgfqpoint{3.776708in}{0.600000in}}{\pgfqpoint{2.573292in}{2.070576in}}%
\pgfusepath{clip}%
\pgfsetbuttcap%
\pgfsetmiterjoin%
\definecolor{currentfill}{rgb}{0.754268,0.565033,0.211761}%
\pgfsetfillcolor{currentfill}%
\pgfsetlinewidth{0.000000pt}%
\definecolor{currentstroke}{rgb}{0.000000,0.000000,0.000000}%
\pgfsetstrokecolor{currentstroke}%
\pgfsetstrokeopacity{0.000000}%
\pgfsetdash{}{0pt}%
\pgfpathmoveto{\pgfqpoint{6.213337in}{1.125321in}}%
\pgfpathlineto{\pgfqpoint{6.222090in}{1.125321in}}%
\pgfpathlineto{\pgfqpoint{6.222090in}{1.091244in}}%
\pgfpathlineto{\pgfqpoint{6.213337in}{1.091244in}}%
\pgfpathlineto{\pgfqpoint{6.213337in}{1.125321in}}%
\pgfpathclose%
\pgfusepath{fill}%
\end{pgfscope}%
\begin{pgfscope}%
\pgfpathrectangle{\pgfqpoint{3.776708in}{0.600000in}}{\pgfqpoint{2.573292in}{2.070576in}}%
\pgfusepath{clip}%
\pgfsetbuttcap%
\pgfsetmiterjoin%
\definecolor{currentfill}{rgb}{0.754268,0.565033,0.211761}%
\pgfsetfillcolor{currentfill}%
\pgfsetlinewidth{0.000000pt}%
\definecolor{currentstroke}{rgb}{0.000000,0.000000,0.000000}%
\pgfsetstrokecolor{currentstroke}%
\pgfsetstrokeopacity{0.000000}%
\pgfsetdash{}{0pt}%
\pgfpathmoveto{\pgfqpoint{6.224279in}{1.127807in}}%
\pgfpathlineto{\pgfqpoint{6.233032in}{1.127807in}}%
\pgfpathlineto{\pgfqpoint{6.233032in}{1.075523in}}%
\pgfpathlineto{\pgfqpoint{6.224279in}{1.075523in}}%
\pgfpathlineto{\pgfqpoint{6.224279in}{1.127807in}}%
\pgfpathclose%
\pgfusepath{fill}%
\end{pgfscope}%
\begin{pgfscope}%
\pgfsetbuttcap%
\pgfsetroundjoin%
\definecolor{currentfill}{rgb}{0.000000,0.000000,0.000000}%
\pgfsetfillcolor{currentfill}%
\pgfsetlinewidth{0.803000pt}%
\definecolor{currentstroke}{rgb}{0.000000,0.000000,0.000000}%
\pgfsetstrokecolor{currentstroke}%
\pgfsetdash{}{0pt}%
\pgfsys@defobject{currentmarker}{\pgfqpoint{0.000000in}{-0.048611in}}{\pgfqpoint{0.000000in}{0.000000in}}{%
\pgfpathmoveto{\pgfqpoint{0.000000in}{0.000000in}}%
\pgfpathlineto{\pgfqpoint{0.000000in}{-0.048611in}}%
\pgfusepath{stroke,fill}%
}%
\begin{pgfscope}%
\pgfsys@transformshift{4.423259in}{0.600000in}%
\pgfsys@useobject{currentmarker}{}%
\end{pgfscope}%
\end{pgfscope}%
\begin{pgfscope}%
\definecolor{textcolor}{rgb}{0.000000,0.000000,0.000000}%
\pgfsetstrokecolor{textcolor}%
\pgfsetfillcolor{textcolor}%
\pgftext[x=4.423259in,y=0.502778in,,top]{\color{textcolor}{\rmfamily\fontsize{10.000000}{12.000000}\selectfont\catcode`\^=\active\def^{\ifmmode\sp\else\^{}\fi}\catcode`\%=\active\def%{\%}1978}}%
\end{pgfscope}%
\begin{pgfscope}%
\pgfsetbuttcap%
\pgfsetroundjoin%
\definecolor{currentfill}{rgb}{0.000000,0.000000,0.000000}%
\pgfsetfillcolor{currentfill}%
\pgfsetlinewidth{0.803000pt}%
\definecolor{currentstroke}{rgb}{0.000000,0.000000,0.000000}%
\pgfsetstrokecolor{currentstroke}%
\pgfsetdash{}{0pt}%
\pgfsys@defobject{currentmarker}{\pgfqpoint{0.000000in}{-0.048611in}}{\pgfqpoint{0.000000in}{0.000000in}}{%
\pgfpathmoveto{\pgfqpoint{0.000000in}{0.000000in}}%
\pgfpathlineto{\pgfqpoint{0.000000in}{-0.048611in}}%
\pgfusepath{stroke,fill}%
}%
\begin{pgfscope}%
\pgfsys@transformshift{4.970349in}{0.600000in}%
\pgfsys@useobject{currentmarker}{}%
\end{pgfscope}%
\end{pgfscope}%
\begin{pgfscope}%
\definecolor{textcolor}{rgb}{0.000000,0.000000,0.000000}%
\pgfsetstrokecolor{textcolor}%
\pgfsetfillcolor{textcolor}%
\pgftext[x=4.970349in,y=0.502778in,,top]{\color{textcolor}{\rmfamily\fontsize{10.000000}{12.000000}\selectfont\catcode`\^=\active\def^{\ifmmode\sp\else\^{}\fi}\catcode`\%=\active\def%{\%}1991}}%
\end{pgfscope}%
\begin{pgfscope}%
\pgfsetbuttcap%
\pgfsetroundjoin%
\definecolor{currentfill}{rgb}{0.000000,0.000000,0.000000}%
\pgfsetfillcolor{currentfill}%
\pgfsetlinewidth{0.803000pt}%
\definecolor{currentstroke}{rgb}{0.000000,0.000000,0.000000}%
\pgfsetstrokecolor{currentstroke}%
\pgfsetdash{}{0pt}%
\pgfsys@defobject{currentmarker}{\pgfqpoint{0.000000in}{-0.048611in}}{\pgfqpoint{0.000000in}{0.000000in}}{%
\pgfpathmoveto{\pgfqpoint{0.000000in}{0.000000in}}%
\pgfpathlineto{\pgfqpoint{0.000000in}{-0.048611in}}%
\pgfusepath{stroke,fill}%
}%
\begin{pgfscope}%
\pgfsys@transformshift{5.517439in}{0.600000in}%
\pgfsys@useobject{currentmarker}{}%
\end{pgfscope}%
\end{pgfscope}%
\begin{pgfscope}%
\definecolor{textcolor}{rgb}{0.000000,0.000000,0.000000}%
\pgfsetstrokecolor{textcolor}%
\pgfsetfillcolor{textcolor}%
\pgftext[x=5.517439in,y=0.502778in,,top]{\color{textcolor}{\rmfamily\fontsize{10.000000}{12.000000}\selectfont\catcode`\^=\active\def^{\ifmmode\sp\else\^{}\fi}\catcode`\%=\active\def%{\%}2003}}%
\end{pgfscope}%
\begin{pgfscope}%
\pgfsetbuttcap%
\pgfsetroundjoin%
\definecolor{currentfill}{rgb}{0.000000,0.000000,0.000000}%
\pgfsetfillcolor{currentfill}%
\pgfsetlinewidth{0.803000pt}%
\definecolor{currentstroke}{rgb}{0.000000,0.000000,0.000000}%
\pgfsetstrokecolor{currentstroke}%
\pgfsetdash{}{0pt}%
\pgfsys@defobject{currentmarker}{\pgfqpoint{0.000000in}{-0.048611in}}{\pgfqpoint{0.000000in}{0.000000in}}{%
\pgfpathmoveto{\pgfqpoint{0.000000in}{0.000000in}}%
\pgfpathlineto{\pgfqpoint{0.000000in}{-0.048611in}}%
\pgfusepath{stroke,fill}%
}%
\begin{pgfscope}%
\pgfsys@transformshift{6.064528in}{0.600000in}%
\pgfsys@useobject{currentmarker}{}%
\end{pgfscope}%
\end{pgfscope}%
\begin{pgfscope}%
\definecolor{textcolor}{rgb}{0.000000,0.000000,0.000000}%
\pgfsetstrokecolor{textcolor}%
\pgfsetfillcolor{textcolor}%
\pgftext[x=6.064528in,y=0.502778in,,top]{\color{textcolor}{\rmfamily\fontsize{10.000000}{12.000000}\selectfont\catcode`\^=\active\def^{\ifmmode\sp\else\^{}\fi}\catcode`\%=\active\def%{\%}2016}}%
\end{pgfscope}%
\begin{pgfscope}%
\pgfsetbuttcap%
\pgfsetroundjoin%
\definecolor{currentfill}{rgb}{0.000000,0.000000,0.000000}%
\pgfsetfillcolor{currentfill}%
\pgfsetlinewidth{0.803000pt}%
\definecolor{currentstroke}{rgb}{0.000000,0.000000,0.000000}%
\pgfsetstrokecolor{currentstroke}%
\pgfsetdash{}{0pt}%
\pgfsys@defobject{currentmarker}{\pgfqpoint{-0.048611in}{0.000000in}}{\pgfqpoint{-0.000000in}{0.000000in}}{%
\pgfpathmoveto{\pgfqpoint{-0.000000in}{0.000000in}}%
\pgfpathlineto{\pgfqpoint{-0.048611in}{0.000000in}}%
\pgfusepath{stroke,fill}%
}%
\begin{pgfscope}%
\pgfsys@transformshift{3.776708in}{0.826710in}%
\pgfsys@useobject{currentmarker}{}%
\end{pgfscope}%
\end{pgfscope}%
\begin{pgfscope}%
\definecolor{textcolor}{rgb}{0.000000,0.000000,0.000000}%
\pgfsetstrokecolor{textcolor}%
\pgfsetfillcolor{textcolor}%
\pgftext[x=3.432571in, y=0.773948in, left, base]{\color{textcolor}{\rmfamily\fontsize{10.000000}{12.000000}\selectfont\catcode`\^=\active\def^{\ifmmode\sp\else\^{}\fi}\catcode`\%=\active\def%{\%}$\mathdefault{\ensuremath{-}40}$}}%
\end{pgfscope}%
\begin{pgfscope}%
\pgfsetbuttcap%
\pgfsetroundjoin%
\definecolor{currentfill}{rgb}{0.000000,0.000000,0.000000}%
\pgfsetfillcolor{currentfill}%
\pgfsetlinewidth{0.803000pt}%
\definecolor{currentstroke}{rgb}{0.000000,0.000000,0.000000}%
\pgfsetstrokecolor{currentstroke}%
\pgfsetdash{}{0pt}%
\pgfsys@defobject{currentmarker}{\pgfqpoint{-0.048611in}{0.000000in}}{\pgfqpoint{-0.000000in}{0.000000in}}{%
\pgfpathmoveto{\pgfqpoint{-0.000000in}{0.000000in}}%
\pgfpathlineto{\pgfqpoint{-0.048611in}{0.000000in}}%
\pgfusepath{stroke,fill}%
}%
\begin{pgfscope}%
\pgfsys@transformshift{3.776708in}{1.217953in}%
\pgfsys@useobject{currentmarker}{}%
\end{pgfscope}%
\end{pgfscope}%
\begin{pgfscope}%
\definecolor{textcolor}{rgb}{0.000000,0.000000,0.000000}%
\pgfsetstrokecolor{textcolor}%
\pgfsetfillcolor{textcolor}%
\pgftext[x=3.432571in, y=1.165191in, left, base]{\color{textcolor}{\rmfamily\fontsize{10.000000}{12.000000}\selectfont\catcode`\^=\active\def^{\ifmmode\sp\else\^{}\fi}\catcode`\%=\active\def%{\%}$\mathdefault{\ensuremath{-}20}$}}%
\end{pgfscope}%
\begin{pgfscope}%
\pgfsetbuttcap%
\pgfsetroundjoin%
\definecolor{currentfill}{rgb}{0.000000,0.000000,0.000000}%
\pgfsetfillcolor{currentfill}%
\pgfsetlinewidth{0.803000pt}%
\definecolor{currentstroke}{rgb}{0.000000,0.000000,0.000000}%
\pgfsetstrokecolor{currentstroke}%
\pgfsetdash{}{0pt}%
\pgfsys@defobject{currentmarker}{\pgfqpoint{-0.048611in}{0.000000in}}{\pgfqpoint{-0.000000in}{0.000000in}}{%
\pgfpathmoveto{\pgfqpoint{-0.000000in}{0.000000in}}%
\pgfpathlineto{\pgfqpoint{-0.048611in}{0.000000in}}%
\pgfusepath{stroke,fill}%
}%
\begin{pgfscope}%
\pgfsys@transformshift{3.776708in}{1.609196in}%
\pgfsys@useobject{currentmarker}{}%
\end{pgfscope}%
\end{pgfscope}%
\begin{pgfscope}%
\definecolor{textcolor}{rgb}{0.000000,0.000000,0.000000}%
\pgfsetstrokecolor{textcolor}%
\pgfsetfillcolor{textcolor}%
\pgftext[x=3.610041in, y=1.556434in, left, base]{\color{textcolor}{\rmfamily\fontsize{10.000000}{12.000000}\selectfont\catcode`\^=\active\def^{\ifmmode\sp\else\^{}\fi}\catcode`\%=\active\def%{\%}$\mathdefault{0}$}}%
\end{pgfscope}%
\begin{pgfscope}%
\pgfsetbuttcap%
\pgfsetroundjoin%
\definecolor{currentfill}{rgb}{0.000000,0.000000,0.000000}%
\pgfsetfillcolor{currentfill}%
\pgfsetlinewidth{0.803000pt}%
\definecolor{currentstroke}{rgb}{0.000000,0.000000,0.000000}%
\pgfsetstrokecolor{currentstroke}%
\pgfsetdash{}{0pt}%
\pgfsys@defobject{currentmarker}{\pgfqpoint{-0.048611in}{0.000000in}}{\pgfqpoint{-0.000000in}{0.000000in}}{%
\pgfpathmoveto{\pgfqpoint{-0.000000in}{0.000000in}}%
\pgfpathlineto{\pgfqpoint{-0.048611in}{0.000000in}}%
\pgfusepath{stroke,fill}%
}%
\begin{pgfscope}%
\pgfsys@transformshift{3.776708in}{2.000439in}%
\pgfsys@useobject{currentmarker}{}%
\end{pgfscope}%
\end{pgfscope}%
\begin{pgfscope}%
\definecolor{textcolor}{rgb}{0.000000,0.000000,0.000000}%
\pgfsetstrokecolor{textcolor}%
\pgfsetfillcolor{textcolor}%
\pgftext[x=3.540596in, y=1.947678in, left, base]{\color{textcolor}{\rmfamily\fontsize{10.000000}{12.000000}\selectfont\catcode`\^=\active\def^{\ifmmode\sp\else\^{}\fi}\catcode`\%=\active\def%{\%}$\mathdefault{20}$}}%
\end{pgfscope}%
\begin{pgfscope}%
\pgfsetbuttcap%
\pgfsetroundjoin%
\definecolor{currentfill}{rgb}{0.000000,0.000000,0.000000}%
\pgfsetfillcolor{currentfill}%
\pgfsetlinewidth{0.803000pt}%
\definecolor{currentstroke}{rgb}{0.000000,0.000000,0.000000}%
\pgfsetstrokecolor{currentstroke}%
\pgfsetdash{}{0pt}%
\pgfsys@defobject{currentmarker}{\pgfqpoint{-0.048611in}{0.000000in}}{\pgfqpoint{-0.000000in}{0.000000in}}{%
\pgfpathmoveto{\pgfqpoint{-0.000000in}{0.000000in}}%
\pgfpathlineto{\pgfqpoint{-0.048611in}{0.000000in}}%
\pgfusepath{stroke,fill}%
}%
\begin{pgfscope}%
\pgfsys@transformshift{3.776708in}{2.391682in}%
\pgfsys@useobject{currentmarker}{}%
\end{pgfscope}%
\end{pgfscope}%
\begin{pgfscope}%
\definecolor{textcolor}{rgb}{0.000000,0.000000,0.000000}%
\pgfsetstrokecolor{textcolor}%
\pgfsetfillcolor{textcolor}%
\pgftext[x=3.540596in, y=2.338921in, left, base]{\color{textcolor}{\rmfamily\fontsize{10.000000}{12.000000}\selectfont\catcode`\^=\active\def^{\ifmmode\sp\else\^{}\fi}\catcode`\%=\active\def%{\%}$\mathdefault{40}$}}%
\end{pgfscope}%
\begin{pgfscope}%
\pgfpathrectangle{\pgfqpoint{3.776708in}{0.600000in}}{\pgfqpoint{2.573292in}{2.070576in}}%
\pgfusepath{clip}%
\pgfsetrectcap%
\pgfsetroundjoin%
\pgfsetlinewidth{1.003750pt}%
\definecolor{currentstroke}{rgb}{0.000000,0.000000,0.000000}%
\pgfsetstrokecolor{currentstroke}%
\pgfsetdash{}{0pt}%
\pgfpathmoveto{\pgfqpoint{3.776708in}{1.609196in}}%
\pgfpathlineto{\pgfqpoint{6.350000in}{1.609196in}}%
\pgfusepath{stroke}%
\end{pgfscope}%
\begin{pgfscope}%
\pgfpathrectangle{\pgfqpoint{3.776708in}{0.600000in}}{\pgfqpoint{2.573292in}{2.070576in}}%
\pgfusepath{clip}%
\pgfsetrectcap%
\pgfsetroundjoin%
\pgfsetlinewidth{1.505625pt}%
\definecolor{currentstroke}{rgb}{0.000000,0.000000,0.000000}%
\pgfsetstrokecolor{currentstroke}%
\pgfsetdash{}{0pt}%
\pgfpathmoveto{\pgfqpoint{3.898052in}{1.754932in}}%
\pgfpathlineto{\pgfqpoint{3.908994in}{1.750981in}}%
\pgfpathlineto{\pgfqpoint{3.919936in}{1.744968in}}%
\pgfpathlineto{\pgfqpoint{3.930878in}{1.724150in}}%
\pgfpathlineto{\pgfqpoint{3.941819in}{1.732375in}}%
\pgfpathlineto{\pgfqpoint{3.952761in}{1.732975in}}%
\pgfpathlineto{\pgfqpoint{3.963703in}{1.728116in}}%
\pgfpathlineto{\pgfqpoint{3.974645in}{1.711045in}}%
\pgfpathlineto{\pgfqpoint{3.985587in}{1.712990in}}%
\pgfpathlineto{\pgfqpoint{4.007470in}{1.694968in}}%
\pgfpathlineto{\pgfqpoint{4.018412in}{1.673505in}}%
\pgfpathlineto{\pgfqpoint{4.040296in}{1.668166in}}%
\pgfpathlineto{\pgfqpoint{4.051237in}{1.659916in}}%
\pgfpathlineto{\pgfqpoint{4.062179in}{1.644134in}}%
\pgfpathlineto{\pgfqpoint{4.084063in}{1.644689in}}%
\pgfpathlineto{\pgfqpoint{4.095005in}{1.633769in}}%
\pgfpathlineto{\pgfqpoint{4.105946in}{1.627840in}}%
\pgfpathlineto{\pgfqpoint{4.116888in}{1.634411in}}%
\pgfpathlineto{\pgfqpoint{4.127830in}{1.637503in}}%
\pgfpathlineto{\pgfqpoint{4.149714in}{1.617094in}}%
\pgfpathlineto{\pgfqpoint{4.160655in}{1.614646in}}%
\pgfpathlineto{\pgfqpoint{4.171597in}{1.618075in}}%
\pgfpathlineto{\pgfqpoint{4.182539in}{1.617150in}}%
\pgfpathlineto{\pgfqpoint{4.193481in}{1.602145in}}%
\pgfpathlineto{\pgfqpoint{4.204423in}{1.589955in}}%
\pgfpathlineto{\pgfqpoint{4.215364in}{1.582804in}}%
\pgfpathlineto{\pgfqpoint{4.226306in}{1.572281in}}%
\pgfpathlineto{\pgfqpoint{4.237248in}{1.555316in}}%
\pgfpathlineto{\pgfqpoint{4.248190in}{1.542290in}}%
\pgfpathlineto{\pgfqpoint{4.259132in}{1.532885in}}%
\pgfpathlineto{\pgfqpoint{4.270073in}{1.532307in}}%
\pgfpathlineto{\pgfqpoint{4.281015in}{1.541044in}}%
\pgfpathlineto{\pgfqpoint{4.291957in}{1.544313in}}%
\pgfpathlineto{\pgfqpoint{4.302899in}{1.549633in}}%
\pgfpathlineto{\pgfqpoint{4.313841in}{1.558070in}}%
\pgfpathlineto{\pgfqpoint{4.324782in}{1.563042in}}%
\pgfpathlineto{\pgfqpoint{4.357608in}{1.558203in}}%
\pgfpathlineto{\pgfqpoint{4.368550in}{1.548886in}}%
\pgfpathlineto{\pgfqpoint{4.379491in}{1.553955in}}%
\pgfpathlineto{\pgfqpoint{4.390433in}{1.550783in}}%
\pgfpathlineto{\pgfqpoint{4.401375in}{1.550162in}}%
\pgfpathlineto{\pgfqpoint{4.412317in}{1.541884in}}%
\pgfpathlineto{\pgfqpoint{4.423259in}{1.541662in}}%
\pgfpathlineto{\pgfqpoint{4.434200in}{1.536203in}}%
\pgfpathlineto{\pgfqpoint{4.445142in}{1.525630in}}%
\pgfpathlineto{\pgfqpoint{4.456084in}{1.512417in}}%
\pgfpathlineto{\pgfqpoint{4.477968in}{1.503472in}}%
\pgfpathlineto{\pgfqpoint{4.488909in}{1.497246in}}%
\pgfpathlineto{\pgfqpoint{4.499851in}{1.487283in}}%
\pgfpathlineto{\pgfqpoint{4.510793in}{1.486370in}}%
\pgfpathlineto{\pgfqpoint{4.521735in}{1.479203in}}%
\pgfpathlineto{\pgfqpoint{4.532677in}{1.478037in}}%
\pgfpathlineto{\pgfqpoint{4.543618in}{1.465638in}}%
\pgfpathlineto{\pgfqpoint{4.554560in}{1.463398in}}%
\pgfpathlineto{\pgfqpoint{4.565502in}{1.463406in}}%
\pgfpathlineto{\pgfqpoint{4.576444in}{1.465779in}}%
\pgfpathlineto{\pgfqpoint{4.587386in}{1.461678in}}%
\pgfpathlineto{\pgfqpoint{4.598327in}{1.475959in}}%
\pgfpathlineto{\pgfqpoint{4.609269in}{1.487517in}}%
\pgfpathlineto{\pgfqpoint{4.620211in}{1.496676in}}%
\pgfpathlineto{\pgfqpoint{4.631153in}{1.515641in}}%
\pgfpathlineto{\pgfqpoint{4.642095in}{1.525094in}}%
\pgfpathlineto{\pgfqpoint{4.653036in}{1.526802in}}%
\pgfpathlineto{\pgfqpoint{4.685862in}{1.548013in}}%
\pgfpathlineto{\pgfqpoint{4.696804in}{1.565396in}}%
\pgfpathlineto{\pgfqpoint{4.707745in}{1.568346in}}%
\pgfpathlineto{\pgfqpoint{4.718687in}{1.577092in}}%
\pgfpathlineto{\pgfqpoint{4.729629in}{1.580997in}}%
\pgfpathlineto{\pgfqpoint{4.740571in}{1.603732in}}%
\pgfpathlineto{\pgfqpoint{4.751513in}{1.605320in}}%
\pgfpathlineto{\pgfqpoint{4.762454in}{1.614926in}}%
\pgfpathlineto{\pgfqpoint{4.773396in}{1.622060in}}%
\pgfpathlineto{\pgfqpoint{4.784338in}{1.633204in}}%
\pgfpathlineto{\pgfqpoint{4.795280in}{1.630968in}}%
\pgfpathlineto{\pgfqpoint{4.806222in}{1.634931in}}%
\pgfpathlineto{\pgfqpoint{4.817163in}{1.633568in}}%
\pgfpathlineto{\pgfqpoint{4.828105in}{1.639942in}}%
\pgfpathlineto{\pgfqpoint{4.849989in}{1.641946in}}%
\pgfpathlineto{\pgfqpoint{4.860931in}{1.640160in}}%
\pgfpathlineto{\pgfqpoint{4.871872in}{1.644730in}}%
\pgfpathlineto{\pgfqpoint{4.893756in}{1.642797in}}%
\pgfpathlineto{\pgfqpoint{4.904698in}{1.642785in}}%
\pgfpathlineto{\pgfqpoint{4.915640in}{1.648865in}}%
\pgfpathlineto{\pgfqpoint{4.926581in}{1.653492in}}%
\pgfpathlineto{\pgfqpoint{4.948465in}{1.658833in}}%
\pgfpathlineto{\pgfqpoint{4.959407in}{1.667790in}}%
\pgfpathlineto{\pgfqpoint{4.970349in}{1.670699in}}%
\pgfpathlineto{\pgfqpoint{4.981290in}{1.670683in}}%
\pgfpathlineto{\pgfqpoint{4.992232in}{1.677352in}}%
\pgfpathlineto{\pgfqpoint{5.003174in}{1.685423in}}%
\pgfpathlineto{\pgfqpoint{5.014116in}{1.687170in}}%
\pgfpathlineto{\pgfqpoint{5.025058in}{1.690421in}}%
\pgfpathlineto{\pgfqpoint{5.035999in}{1.690866in}}%
\pgfpathlineto{\pgfqpoint{5.046941in}{1.693943in}}%
\pgfpathlineto{\pgfqpoint{5.057883in}{1.690811in}}%
\pgfpathlineto{\pgfqpoint{5.068825in}{1.695203in}}%
\pgfpathlineto{\pgfqpoint{5.079767in}{1.692204in}}%
\pgfpathlineto{\pgfqpoint{5.090708in}{1.696344in}}%
\pgfpathlineto{\pgfqpoint{5.101650in}{1.691838in}}%
\pgfpathlineto{\pgfqpoint{5.112592in}{1.690410in}}%
\pgfpathlineto{\pgfqpoint{5.123534in}{1.685870in}}%
\pgfpathlineto{\pgfqpoint{5.134476in}{1.687597in}}%
\pgfpathlineto{\pgfqpoint{5.145417in}{1.684954in}}%
\pgfpathlineto{\pgfqpoint{5.156359in}{1.684857in}}%
\pgfpathlineto{\pgfqpoint{5.178243in}{1.670394in}}%
\pgfpathlineto{\pgfqpoint{5.189185in}{1.674194in}}%
\pgfpathlineto{\pgfqpoint{5.200126in}{1.669850in}}%
\pgfpathlineto{\pgfqpoint{5.211068in}{1.667599in}}%
\pgfpathlineto{\pgfqpoint{5.222010in}{1.667352in}}%
\pgfpathlineto{\pgfqpoint{5.232952in}{1.663250in}}%
\pgfpathlineto{\pgfqpoint{5.243894in}{1.655392in}}%
\pgfpathlineto{\pgfqpoint{5.254835in}{1.649946in}}%
\pgfpathlineto{\pgfqpoint{5.265777in}{1.649559in}}%
\pgfpathlineto{\pgfqpoint{5.276719in}{1.645962in}}%
\pgfpathlineto{\pgfqpoint{5.287661in}{1.638921in}}%
\pgfpathlineto{\pgfqpoint{5.298603in}{1.628336in}}%
\pgfpathlineto{\pgfqpoint{5.309544in}{1.628019in}}%
\pgfpathlineto{\pgfqpoint{5.320486in}{1.623231in}}%
\pgfpathlineto{\pgfqpoint{5.331428in}{1.613933in}}%
\pgfpathlineto{\pgfqpoint{5.342370in}{1.607082in}}%
\pgfpathlineto{\pgfqpoint{5.353312in}{1.607950in}}%
\pgfpathlineto{\pgfqpoint{5.364253in}{1.598057in}}%
\pgfpathlineto{\pgfqpoint{5.375195in}{1.581249in}}%
\pgfpathlineto{\pgfqpoint{5.397079in}{1.560842in}}%
\pgfpathlineto{\pgfqpoint{5.408021in}{1.560869in}}%
\pgfpathlineto{\pgfqpoint{5.418962in}{1.547717in}}%
\pgfpathlineto{\pgfqpoint{5.429904in}{1.546274in}}%
\pgfpathlineto{\pgfqpoint{5.440846in}{1.549631in}}%
\pgfpathlineto{\pgfqpoint{5.451788in}{1.546981in}}%
\pgfpathlineto{\pgfqpoint{5.462730in}{1.548833in}}%
\pgfpathlineto{\pgfqpoint{5.473671in}{1.548223in}}%
\pgfpathlineto{\pgfqpoint{5.484613in}{1.552749in}}%
\pgfpathlineto{\pgfqpoint{5.495555in}{1.548381in}}%
\pgfpathlineto{\pgfqpoint{5.506497in}{1.556101in}}%
\pgfpathlineto{\pgfqpoint{5.517439in}{1.555235in}}%
\pgfpathlineto{\pgfqpoint{5.528380in}{1.561138in}}%
\pgfpathlineto{\pgfqpoint{5.550264in}{1.560434in}}%
\pgfpathlineto{\pgfqpoint{5.561206in}{1.558041in}}%
\pgfpathlineto{\pgfqpoint{5.572148in}{1.562065in}}%
\pgfpathlineto{\pgfqpoint{5.583089in}{1.563447in}}%
\pgfpathlineto{\pgfqpoint{5.604973in}{1.552784in}}%
\pgfpathlineto{\pgfqpoint{5.615915in}{1.556810in}}%
\pgfpathlineto{\pgfqpoint{5.626857in}{1.559003in}}%
\pgfpathlineto{\pgfqpoint{5.637798in}{1.551417in}}%
\pgfpathlineto{\pgfqpoint{5.648740in}{1.546699in}}%
\pgfpathlineto{\pgfqpoint{5.659682in}{1.548382in}}%
\pgfpathlineto{\pgfqpoint{5.670624in}{1.546963in}}%
\pgfpathlineto{\pgfqpoint{5.681566in}{1.538393in}}%
\pgfpathlineto{\pgfqpoint{5.692507in}{1.537053in}}%
\pgfpathlineto{\pgfqpoint{5.714391in}{1.543415in}}%
\pgfpathlineto{\pgfqpoint{5.725333in}{1.537698in}}%
\pgfpathlineto{\pgfqpoint{5.736274in}{1.554450in}}%
\pgfpathlineto{\pgfqpoint{5.747216in}{1.578802in}}%
\pgfpathlineto{\pgfqpoint{5.769100in}{1.604960in}}%
\pgfpathlineto{\pgfqpoint{5.790983in}{1.621737in}}%
\pgfpathlineto{\pgfqpoint{5.801925in}{1.632211in}}%
\pgfpathlineto{\pgfqpoint{5.812867in}{1.639924in}}%
\pgfpathlineto{\pgfqpoint{5.823809in}{1.645480in}}%
\pgfpathlineto{\pgfqpoint{5.834751in}{1.652862in}}%
\pgfpathlineto{\pgfqpoint{5.845692in}{1.652775in}}%
\pgfpathlineto{\pgfqpoint{5.856634in}{1.646080in}}%
\pgfpathlineto{\pgfqpoint{5.867576in}{1.652046in}}%
\pgfpathlineto{\pgfqpoint{5.878518in}{1.659365in}}%
\pgfpathlineto{\pgfqpoint{5.889460in}{1.662649in}}%
\pgfpathlineto{\pgfqpoint{5.900401in}{1.662395in}}%
\pgfpathlineto{\pgfqpoint{5.911343in}{1.660101in}}%
\pgfpathlineto{\pgfqpoint{5.933227in}{1.664550in}}%
\pgfpathlineto{\pgfqpoint{5.955110in}{1.648162in}}%
\pgfpathlineto{\pgfqpoint{5.966052in}{1.651429in}}%
\pgfpathlineto{\pgfqpoint{5.976994in}{1.656058in}}%
\pgfpathlineto{\pgfqpoint{5.987936in}{1.648192in}}%
\pgfpathlineto{\pgfqpoint{5.998878in}{1.645348in}}%
\pgfpathlineto{\pgfqpoint{6.009819in}{1.647249in}}%
\pgfpathlineto{\pgfqpoint{6.020761in}{1.641773in}}%
\pgfpathlineto{\pgfqpoint{6.042645in}{1.625581in}}%
\pgfpathlineto{\pgfqpoint{6.053587in}{1.639742in}}%
\pgfpathlineto{\pgfqpoint{6.064528in}{1.642903in}}%
\pgfpathlineto{\pgfqpoint{6.075470in}{1.636583in}}%
\pgfpathlineto{\pgfqpoint{6.086412in}{1.634068in}}%
\pgfpathlineto{\pgfqpoint{6.097354in}{1.635551in}}%
\pgfpathlineto{\pgfqpoint{6.108296in}{1.623937in}}%
\pgfpathlineto{\pgfqpoint{6.119237in}{1.616784in}}%
\pgfpathlineto{\pgfqpoint{6.130179in}{1.618071in}}%
\pgfpathlineto{\pgfqpoint{6.141121in}{1.615019in}}%
\pgfpathlineto{\pgfqpoint{6.152063in}{1.620177in}}%
\pgfpathlineto{\pgfqpoint{6.163005in}{1.613458in}}%
\pgfpathlineto{\pgfqpoint{6.173946in}{1.612835in}}%
\pgfpathlineto{\pgfqpoint{6.184888in}{1.615143in}}%
\pgfpathlineto{\pgfqpoint{6.195830in}{1.609175in}}%
\pgfpathlineto{\pgfqpoint{6.206772in}{1.600928in}}%
\pgfpathlineto{\pgfqpoint{6.217714in}{1.608805in}}%
\pgfpathlineto{\pgfqpoint{6.228655in}{1.611670in}}%
\pgfpathlineto{\pgfqpoint{6.228655in}{1.611670in}}%
\pgfusepath{stroke}%
\end{pgfscope}%
\begin{pgfscope}%
\pgfsetrectcap%
\pgfsetmiterjoin%
\pgfsetlinewidth{0.803000pt}%
\definecolor{currentstroke}{rgb}{0.000000,0.000000,0.000000}%
\pgfsetstrokecolor{currentstroke}%
\pgfsetdash{}{0pt}%
\pgfpathmoveto{\pgfqpoint{3.776708in}{0.600000in}}%
\pgfpathlineto{\pgfqpoint{3.776708in}{2.670576in}}%
\pgfusepath{stroke}%
\end{pgfscope}%
\begin{pgfscope}%
\pgfsetrectcap%
\pgfsetmiterjoin%
\pgfsetlinewidth{0.803000pt}%
\definecolor{currentstroke}{rgb}{0.000000,0.000000,0.000000}%
\pgfsetstrokecolor{currentstroke}%
\pgfsetdash{}{0pt}%
\pgfpathmoveto{\pgfqpoint{6.350000in}{0.600000in}}%
\pgfpathlineto{\pgfqpoint{6.350000in}{2.670576in}}%
\pgfusepath{stroke}%
\end{pgfscope}%
\begin{pgfscope}%
\pgfsetrectcap%
\pgfsetmiterjoin%
\pgfsetlinewidth{0.803000pt}%
\definecolor{currentstroke}{rgb}{0.000000,0.000000,0.000000}%
\pgfsetstrokecolor{currentstroke}%
\pgfsetdash{}{0pt}%
\pgfpathmoveto{\pgfqpoint{3.776708in}{0.600000in}}%
\pgfpathlineto{\pgfqpoint{6.350000in}{0.600000in}}%
\pgfusepath{stroke}%
\end{pgfscope}%
\begin{pgfscope}%
\pgfsetrectcap%
\pgfsetmiterjoin%
\pgfsetlinewidth{0.803000pt}%
\definecolor{currentstroke}{rgb}{0.000000,0.000000,0.000000}%
\pgfsetstrokecolor{currentstroke}%
\pgfsetdash{}{0pt}%
\pgfpathmoveto{\pgfqpoint{3.776708in}{2.670576in}}%
\pgfpathlineto{\pgfqpoint{6.350000in}{2.670576in}}%
\pgfusepath{stroke}%
\end{pgfscope}%
\begin{pgfscope}%
\definecolor{textcolor}{rgb}{0.000000,0.000000,0.000000}%
\pgfsetstrokecolor{textcolor}%
\pgfsetfillcolor{textcolor}%
\pgftext[x=5.063354in,y=2.753910in,,base]{\color{textcolor}{\rmfamily\fontsize{10.000000}{12.000000}\selectfont\catcode`\^=\active\def^{\ifmmode\sp\else\^{}\fi}\catcode`\%=\active\def%{\%}Savings}}%
\end{pgfscope}%
\begin{pgfscope}%
\definecolor{textcolor}{rgb}{0.000000,0.000000,0.000000}%
\pgfsetstrokecolor{textcolor}%
\pgfsetfillcolor{textcolor}%
\pgftext[x=0.235523in, y=0.771037in, left, base,rotate=90.000000]{\color{textcolor}{\rmfamily\fontsize{10.000000}{12.000000}\selectfont\catcode`\^=\active\def^{\ifmmode\sp\else\^{}\fi}\catcode`\%=\active\def%{\%}\% Deviation from SS}}%
\end{pgfscope}%
\begin{pgfscope}%
\pgfsetbuttcap%
\pgfsetmiterjoin%
\definecolor{currentfill}{rgb}{0.066899,0.263188,0.377594}%
\pgfsetfillcolor{currentfill}%
\pgfsetlinewidth{0.000000pt}%
\definecolor{currentstroke}{rgb}{0.000000,0.000000,0.000000}%
\pgfsetstrokecolor{currentstroke}%
\pgfsetstrokeopacity{0.000000}%
\pgfsetdash{}{0pt}%
\pgfpathmoveto{\pgfqpoint{0.434821in}{0.155856in}}%
\pgfpathlineto{\pgfqpoint{0.712599in}{0.155856in}}%
\pgfpathlineto{\pgfqpoint{0.712599in}{0.253079in}}%
\pgfpathlineto{\pgfqpoint{0.434821in}{0.253079in}}%
\pgfpathlineto{\pgfqpoint{0.434821in}{0.155856in}}%
\pgfpathclose%
\pgfusepath{fill}%
\end{pgfscope}%
\begin{pgfscope}%
\definecolor{textcolor}{rgb}{0.000000,0.000000,0.000000}%
\pgfsetstrokecolor{textcolor}%
\pgfsetfillcolor{textcolor}%
\pgftext[x=0.823710in,y=0.155856in,left,base]{\color{textcolor}{\rmfamily\fontsize{10.000000}{12.000000}\selectfont\catcode`\^=\active\def^{\ifmmode\sp\else\^{}\fi}\catcode`\%=\active\def%{\%}Labor}}%
\end{pgfscope}%
\begin{pgfscope}%
\pgfsetbuttcap%
\pgfsetmiterjoin%
\definecolor{currentfill}{rgb}{0.133298,0.375282,0.379395}%
\pgfsetfillcolor{currentfill}%
\pgfsetlinewidth{0.000000pt}%
\definecolor{currentstroke}{rgb}{0.000000,0.000000,0.000000}%
\pgfsetstrokecolor{currentstroke}%
\pgfsetstrokeopacity{0.000000}%
\pgfsetdash{}{0pt}%
\pgfpathmoveto{\pgfqpoint{1.515442in}{0.155856in}}%
\pgfpathlineto{\pgfqpoint{1.793220in}{0.155856in}}%
\pgfpathlineto{\pgfqpoint{1.793220in}{0.253079in}}%
\pgfpathlineto{\pgfqpoint{1.515442in}{0.253079in}}%
\pgfpathlineto{\pgfqpoint{1.515442in}{0.155856in}}%
\pgfpathclose%
\pgfusepath{fill}%
\end{pgfscope}%
\begin{pgfscope}%
\definecolor{textcolor}{rgb}{0.000000,0.000000,0.000000}%
\pgfsetstrokecolor{textcolor}%
\pgfsetfillcolor{textcolor}%
\pgftext[x=1.904331in,y=0.155856in,left,base]{\color{textcolor}{\rmfamily\fontsize{10.000000}{12.000000}\selectfont\catcode`\^=\active\def^{\ifmmode\sp\else\^{}\fi}\catcode`\%=\active\def%{\%}Wage}}%
\end{pgfscope}%
\begin{pgfscope}%
\pgfsetbuttcap%
\pgfsetmiterjoin%
\definecolor{currentfill}{rgb}{0.302379,0.450282,0.300122}%
\pgfsetfillcolor{currentfill}%
\pgfsetlinewidth{0.000000pt}%
\definecolor{currentstroke}{rgb}{0.000000,0.000000,0.000000}%
\pgfsetstrokecolor{currentstroke}%
\pgfsetstrokeopacity{0.000000}%
\pgfsetdash{}{0pt}%
\pgfpathmoveto{\pgfqpoint{2.566698in}{0.155856in}}%
\pgfpathlineto{\pgfqpoint{2.844476in}{0.155856in}}%
\pgfpathlineto{\pgfqpoint{2.844476in}{0.253079in}}%
\pgfpathlineto{\pgfqpoint{2.566698in}{0.253079in}}%
\pgfpathlineto{\pgfqpoint{2.566698in}{0.155856in}}%
\pgfpathclose%
\pgfusepath{fill}%
\end{pgfscope}%
\begin{pgfscope}%
\definecolor{textcolor}{rgb}{0.000000,0.000000,0.000000}%
\pgfsetstrokecolor{textcolor}%
\pgfsetfillcolor{textcolor}%
\pgftext[x=2.955587in,y=0.155856in,left,base]{\color{textcolor}{\rmfamily\fontsize{10.000000}{12.000000}\selectfont\catcode`\^=\active\def^{\ifmmode\sp\else\^{}\fi}\catcode`\%=\active\def%{\%}Interest Rate}}%
\end{pgfscope}%
\begin{pgfscope}%
\pgfsetbuttcap%
\pgfsetmiterjoin%
\definecolor{currentfill}{rgb}{0.511253,0.510898,0.193296}%
\pgfsetfillcolor{currentfill}%
\pgfsetlinewidth{0.000000pt}%
\definecolor{currentstroke}{rgb}{0.000000,0.000000,0.000000}%
\pgfsetstrokecolor{currentstroke}%
\pgfsetstrokeopacity{0.000000}%
\pgfsetdash{}{0pt}%
\pgfpathmoveto{\pgfqpoint{4.164083in}{0.155856in}}%
\pgfpathlineto{\pgfqpoint{4.441861in}{0.155856in}}%
\pgfpathlineto{\pgfqpoint{4.441861in}{0.253079in}}%
\pgfpathlineto{\pgfqpoint{4.164083in}{0.253079in}}%
\pgfpathlineto{\pgfqpoint{4.164083in}{0.155856in}}%
\pgfpathclose%
\pgfusepath{fill}%
\end{pgfscope}%
\begin{pgfscope}%
\definecolor{textcolor}{rgb}{0.000000,0.000000,0.000000}%
\pgfsetstrokecolor{textcolor}%
\pgfsetfillcolor{textcolor}%
\pgftext[x=4.552972in,y=0.155856in,left,base]{\color{textcolor}{\rmfamily\fontsize{10.000000}{12.000000}\selectfont\catcode`\^=\active\def^{\ifmmode\sp\else\^{}\fi}\catcode`\%=\active\def%{\%}Transfer}}%
\end{pgfscope}%
\begin{pgfscope}%
\pgfsetbuttcap%
\pgfsetmiterjoin%
\definecolor{currentfill}{rgb}{0.754268,0.565033,0.211761}%
\pgfsetfillcolor{currentfill}%
\pgfsetlinewidth{0.000000pt}%
\definecolor{currentstroke}{rgb}{0.000000,0.000000,0.000000}%
\pgfsetstrokecolor{currentstroke}%
\pgfsetstrokeopacity{0.000000}%
\pgfsetdash{}{0pt}%
\pgfpathmoveto{\pgfqpoint{5.433302in}{0.155856in}}%
\pgfpathlineto{\pgfqpoint{5.711080in}{0.155856in}}%
\pgfpathlineto{\pgfqpoint{5.711080in}{0.253079in}}%
\pgfpathlineto{\pgfqpoint{5.433302in}{0.253079in}}%
\pgfpathlineto{\pgfqpoint{5.433302in}{0.155856in}}%
\pgfpathclose%
\pgfusepath{fill}%
\end{pgfscope}%
\begin{pgfscope}%
\definecolor{textcolor}{rgb}{0.000000,0.000000,0.000000}%
\pgfsetstrokecolor{textcolor}%
\pgfsetfillcolor{textcolor}%
\pgftext[x=5.822191in,y=0.155856in,left,base]{\color{textcolor}{\rmfamily\fontsize{10.000000}{12.000000}\selectfont\catcode`\^=\active\def^{\ifmmode\sp\else\^{}\fi}\catcode`\%=\active\def%{\%}Tax}}%
\end{pgfscope}%
\end{pgfpicture}%
\makeatother%
\endgroup%


    {\scriptsize \emph{Notes:} NBER-dated recessions highlighted in gray.}
    \label{fig:agg-endog-hist-decomp}
\end{figure}

Wages, interest rates, and taxes are the most important determinants of aggregate savings. Likely due to the $\rho_B$ parameter in the bond law of motion, the decomposed factors appear much smoother for savings than consumption, but the general shape of the savings decomposition is similar to that of the consumption decomposition, especially for the wage effects. Interest rates and taxes are far more important for savings than for consumption. The difference between the effects of taxes on consumption and savings suggests households in the model that were given tax breaks within the estimation window chose to save the extra money, not spend it. This could be because the largest tax breaks in the estimation window were given to higher wealth households, which have lower MPCs \autocite{auclert2023mpcs}.

The decomposition separated by income and wealth level is presented in Figure \ref{fig:cons-endog-hist-decomp}. The labor supply choice is, in general, the most important factor affecting household consumption decisions, especially for higher income or higher wealth households. For lower income and lower wealth households, direct household transfers and taxes are also important. Wages and interest rates are moderately important to all households except low and middle income 0th percentile households that are never impacted by the interest rate.

\begin{figure}[t!]
    \centering
    \caption{Historical Endogenous Decomposition: Household Consumption}
    \input{figures/endog_c_historical_decomp.pgf}

    {\scriptsize \emph{Notes:} Row labeled by wealth percentile. 12 quarter moving average applied. NBER-dated recessions highlighted in gray.}
    \label{fig:cons-endog-hist-decomp}
\end{figure}

Figure \ref{fig:sav-endog-hist-decomp} presents the same decomposition for household savings decisions. Again, low and middle income households at the 0th wealth percentile never choose to save. Interest rates are moderately important across all households, but most important to higher wealth households. In fact, interest rates determine almost all movement in savings decisions for 99th percentile low and middle income households. Households with either lower levels of wealth or higher income also face significant wage, labor, and tax effects. This suggests households choose to save more to consumption smooth after changes in income rather than to increase their future earnings when interest rates are higher.

\begin{figure}[t!]
    \centering
    \caption{Historical Endogenous Decomposition: Household Savings}
    \input{figures/endog_a_historical_decomp.pgf}

    {\scriptsize \emph{Notes:} Row labeled by wealth percentile. 12 quarter moving average applied. NBER-dated recessions highlighted in gray.}
    \label{fig:sav-endog-hist-decomp}
\end{figure}
