I model a discrete time, one-asset HANK with incomplete markets stemming from uninsurable, idiosyncratic income risks and nominal rigidities. The economy is composed of households, unions, firms, and a government. Within the model, there are shocks to total factor productivity (TFP) $A_t$, price markups $\psi_t$, wage markups $\psi_t^W$, government spending $g_t$, transfers to households $\eta_t$, tax progressivity $\tau^P_t$, and monetary policy $\xi_t$.

The household sector features a continuum of dynamically optimizing heterogeneous households that choose to consume and save. Households earn income from their wages, firm profits, and government transfers. Household productivity levels evolve idiosyncratically over time, which they self-insure against by investing in a risk-free government bond.

The union sector includes a labor packer and a continuum of unions. The labor packer aggregates the labor provided by the unions, which choose a homogenous level of labor to be supplied by households to maximize aggregate utility. Unions are subject to quadratic wage adjustment costs paid in utils following \textcite{auclert2023mpcs}.

The firm sector comprises a representative perfectly competitive final goods firm and a continuum of monopolistically competitive intermediate goods firms. The final goods firm aggregates production from the intermediate goods firms, who produce differentiated goods using labor supplied by unions. Following \textcite{rotemberg1982sticky}, intermediate goods firms face quadratic price adjustment costs, creating pricing frictions in the economy.

The government acts as the fiscal and monetary authority. As the fiscal authority, the government supplies a risk-free bond to households, spends exogenously, pays a lump-sum transfer amount to households, and imposes a progressive tax scheme to balance the budget. As the monetary authority, the government sets the interest rate according to a Taylor rule based on the levels of inflation and output.

In this section, I give the assumptions and key equations in the model. For a derivation of the equations and characterization of the model, see Appendix \ref{app:mod}.


\subsection{Households} \label{subsec:hh}

The model is populated by a unit continuum of infinitely lived households indexed $i \in [0, 1]$. Each period, households provide the amount of labor $\ell_{i, t}$ decided by the union and choose to consume $c_{i, t}$ and hold $b_{i, t}$ of a risk-free government bond which has gross real returns $R_t$ to maximize expected discounted utility. Households have constant relative risk aversion (CRRA) preferences given by 
\begin{equation*}
    \max_{\{c_{i,t}, \ell_{i, t}, b_{i, t}\}_{t=0}^\infty} \E \sum_{t = 0}^\infty \beta^t \left[ \frac{c_{i, t}^{1 - \gamma}}{1 - \gamma} - \phi \frac{\ell_{i, t}^{1 + \chi}}{1 + \chi} \right]
\end{equation*}
where $\beta$, $\gamma$, $\phi$, and $\chi$ represent the intertemporal discount rate, risk aversion level, relative disutility of labor, and inverse Frisch elasticity of labor supply.

Household productivity $z_{i, t}$ evolves stochastically over time subject to the log-AR(1) process
\begin{equation*}
    \log z_{i, t} = \rho_z \log z_{i, t-1} + \epsilon_{z, i, t}, \quad \epsilon_{z, i, t} \sim \mathcal{N} (0, \sigma_z^2) \label{eq:idio_prod}
\end{equation*}
where $\rho_z$ and $\sigma_z^2$ represent the persistence and variance of individual productivity shocks. Based on their productivity, labor supply, and the real wage $W_t$, households generate pre-tax labor income $W_t z_{i, t} \ell_{i, t}$. Additionally, dividends $D_t$ and transfers $\eta_t$ are evenly distributed across households from the profits of intermediate goods firms and exogenously by the government.

Following \textcite{mckay2016power}, the government imposes a progressive tax on productivity. Since productivity is exogenous, this acts like a lump sum tax and does not distort household decisions. The tax scheme is given by $\tau_t^L z_{i, t}^{\tau_t^P}$ where $\tau_t^L$ and $\tau_t^P$ measure the level and progressivity of the tax scheme respectively. Therefore, $\tau_t^P < 1$ creates a regressive tax scheme, $\tau_t^P = 1$ creates a proportional tax scheme, and $\tau_t^P > 1$ creates a progressive tax scheme.

Combined, this gets the household budget constraint
\begin{equation*}
    b_{i, t} + c_{i, t} = R_t b_{i, t-1} + W_t z_{i, t} \ell_{i, t} + D_t + \eta_t - \tau_t^L z_{i, t}^{\tau_t^P}.
\end{equation*}
Households are also subject to the borrowing constraint $b_{i, t} \geq \underline{b}$ which enforces a no-Ponzi condition for all households.

Because productivity $z_{i, t}$ follows and exogenous law of motion that is time invariant, the distribution of household productivity $\Gamma_t^z (z)$ is also fully exogenous and follows a time invariant process. Assuming the initial distribution for $\Gamma_t^z (z)$ is equal to the ergodic distribution of the AR-1 process, the overall distribution stays constant over time, even as individual households change state within it.

Each period, household's choices depend on their states $z_{i, t}$ and $b_{i, t - 1}$ entering the period. Given these states, households follow the decision rules
\begin{align*}
    b_t (b_{i, t-1}, z_{i, t}) &= b_{i, t} \\
    c_t (b_{i, t-1}, z_{i, t}) &= c_{i, t}.
\end{align*}
Therefore, the distribution of household states $\Gamma_t (b, z)$ evolves according to
\begin{equation*}
    \Gamma_{t + 1} (b', z') = \int_{\{(b, z) : b_t(b, z) = b'\}} \Pr (z' | z) d \Gamma_t (b, z)
\end{equation*}
which says the density of households with savings $b'$ and productivity $z'$ is equal to the density of households that choose to save $b'$ times the probability that their productivity ends up $z'$.


\subsection{Unions} \label{subsec:unions}

A single labor packer aggregates labor from a unit continuum of unions indexed $k \in [0, 1]$.

The labor packer aggregates labor supplied by each union $n_{k,t}$ into aggregate labot $N_t$ according to the Dixit-Stiglitz aggregator
\begin{equation*}
    N_t = \left(\int_0^1 n_{k, t}^\frac{1}{\psi_t^W}\right)^{\psi_t^W}
\end{equation*}
where $\frac{\psi_t^W}{\psi_t^W - 1}$ represents the elasticity of substitution for labor provided by each union. Profit maximization for the union gets the demand for labor provided by each union $k$
\begin{equation*}
    n_{k, t} = N_t \left(\frac{w_{k, t}}{W_t}\right)^\frac{\psi_t^W}{1 - \psi_t^W}
\end{equation*}
where $w_{k, t}$ is the real wage demanded by union $k$.

Unions choose a level of labor to demand uniformly from households $\ell_{k, t}$ and aggregates it according to
\begin{equation*}
    n_{k, t} = \int z \ell_{k, t} d \Gamma_t^z (z).
\end{equation*}
The uniform labor demand assumption follows \textcite{auclert2023mpcs}, and suggests that households supply the same level of labor to the union regardless of their productivity and wealth differences. This ignores household differences in willingness to work and does require that some households are required to work more than they would choose to \autocite{gerke2024household}. Alternative approaches would allow unions to vary the quantity of labor demanded or wage for different households, but add substantial mathematical and computational complexity to the model \autocite{gerke2024household}.

The union chooses $\ell_{k, t}$ to maximize household utility subject to quadratic adjustment $m_{k, t}^W$ costs
\begin{equation*}
    m_{k, t}^W = \frac{\psi_t^W}{\psi_t^W - 1} \frac{1}{2 \kappa^W} \log\left(\frac{w_{k, t}}{\overline{\pi}^W w_{k, t-1}}\right)^2
\end{equation*}
which is paid in utils where $\kappa^W$ denotes the responsiveness of wages to economic changes, $\pi_t^W = \frac{W_t}{W_{t-1}}$ is wage inflation, and the overline over a variable represents its steady state value. The aggregate utility maximization problem gets the wage Philips curve
\begin{equation*}
    \log\left(\frac{\pi_t^W}{\overline{\pi}^W}\right) = \kappa^W \left( \phi L_t^{1 + \chi} - \frac{1}{\psi_t^W} W_t L_t \int z c_t(b, z)^{-\gamma} d\Gamma_t(b, z)\right) + \beta \log\left(\frac{\pi_{t+1}^W}{\overline{\pi}^W}\right)
\end{equation*}
where $L_t$ is the amount of labor demanded from each household.


\subsection{Firms} \label{subsec:firms}

The model is populated by a representative, competitive final goods firm and a unit continuum of intermediate goods firms indexed $j \in [0, 1]$.

Like the labor packer, the final goods firm aggregates intermediate goods $y_{j, t}$ into output $Y_t$ according to the Dixit-Stiglitz aggregator
\begin{equation*}
    Y_t = \left( \int_0^1 y_{j, t}^\frac{1}{\psi_t} dj \right)^{\psi_t}
\end{equation*}
where $\frac{\psi_t}{\psi_t - 1}$ represents the elasticity of substitution for intermediate goods. Profit maximization for the final goods firm gets the demand for intermediate good $j$
\begin{equation*}
    y_{j, t} = Y_t \left(\frac{p_{j, t}}{P_t}\right)^\frac{\psi}{\psi - 1}
\end{equation*}
where $p_{j, t}$ is the price of intermediate good $j$ and $P_t$ is the overall price level of the economy given by
\begin{equation*}
    P_t = \left(\int_0^1 p_{j, t}^\frac{1}{1 - \psi_t} dj \right)^{1 - \psi_t}.
\end{equation*}

Intermediate goods firms use productive units of labor $n_{j, t}$ to produce their intermediate good according to
\begin{equation*}
    y_{j, t} = A_t n_{j, t}
\end{equation*}
where $A_t$ represents the overall productivity level of the economy.

Intermediate goods firms also choose prices subject to quadratic adjustment costs $m_{j, t}$ à la \textcite{rotemberg1982sticky} given by
\begin{equation*}
    m_{j, t} = \frac{\psi_t}{\psi_t - 1} \frac{1}{2 \kappa} \log\left(\frac{p_{j, t}}{\overline{\pi} p_{j, t - 1}}\right)^2 Y_t
\end{equation*}
where $\pi = \frac{P_t}{P_{t - 1}}$ is inflation and $\kappa$ determines the responsiveness of inflation to changes in output. Compared to the alternative \textcite{calvo1983staggered} rule, the Rotemberg price frictions have a couple advantages. First, price frictions under a Rotemberg rule are more consistent with real data \autocite{richter2016rotemberg}. Additionally, a Rotemberg rule has an analytically solvable Philips curve, which makes the model easier to solve. The Philips curve is
\begin{equation*}
    \log \left(\frac{\pi_t}{\overline{\pi}}\right) = \kappa \left(\frac{W_t}{A_t} - \frac{1}{\psi_t}\right) + R_{t+1} \frac{Y_{t+1}}{Y_t} \log\left(\frac{\pi_{t+1}}{\overline{\pi}}\right).
\end{equation*}

Finally, since intermediate goods firms are monopolistically competitive, they can make a profit. Profits will be paid out in the form of real dividends $d_{j, t}$ such that
\begin{equation*}
    d_{j, t} = \frac{p_{j, t}}{P_t} y_{i, t} - W_t n_{j, t} - m_{j, t}
\end{equation*}
where firms earn real revenue $\frac{p_{j, t}}{P_t} y_{i, t}$ and pay labor costs $W_t n_{j, t}$ and price adjustment costs. Aggregate dividends $D_t$ are
\begin{equation*}
    D_t = \int_0^1 d_{j, t} dj.
\end{equation*}


\subsection{Government}

In the economy, the government acts as both the fiscal and monetary authority.

As the fiscal authority, the government spends an exogenous fraction $g_t$ of output so that government spending $G_t$ follows
\begin{equation*}
    G_t = g_t Y_t.
\end{equation*}
The government also offers the risk-free bond $B_t$ and pays out transfers to households subject to the law of motion for bonds
\begin{equation*}
    B_t = \overline{B} + \rho_B \left(R_t B_{t-1} - \overline{R} \overline{B} + G_t - \overline{G} + \eta_t - \overline{\eta}\right)
\end{equation*}
following \textcite{auclert2024intertemporal} where $\rho_B$ represents how quickly the government pays back non-steady state levels of debt. In the steady state, this means the government holds a constant stock of debt which it pays all the interest on every period. However, increases in transfers $\eta_t$, the interest rate $R_t$, or government spending $G_t$ will be financed by taking on more debt and paying it back over time. To balance the budget, the government sets the tax level $\tau_L$ so that government spending equals government revenue
\begin{equation*}
    R_t B_{t - 1} + G_t + \eta_t = \tau_t^L \int z^{\tau_t^P} d \Gamma_t^Z (z) + B_t.
\end{equation*}

As the monetary authority, the government sets the interest rate $I_t$ according to the Taylor Rule
\begin{equation*}
    I_t = \overline{I} \hat{\pi}_t^{\omega_\pi} \hat{Y}_t^{\omega_Y} \xi_t
\end{equation*}
where $\omega_\pi$ and $\omega_Y$ represent the relative importance of inflation and output stabilization and $\xi_t$ is the monetary policy shock. The Fisher relation means
\begin{equation*}
    R_t = \frac{I_{t - 1}}{\pi_t}.
\end{equation*}


\subsection{Equilibrium}

For the economy to be in equilibrium, the labor, bond, and goods markets all need to clear. Labor market clearing requires unions to provide as much labor as firms demand so that
\begin{equation*}
    N_t = \int_0^1 n_{j, t} dj.
\end{equation*}
Bond market clearing requires the supply of bonds by the government to equal household savings
\begin{equation*}
    B_t = \int b_t (b, z) d \Gamma_t (b, z).
\end{equation*}
Finally, goods market clearing requires consumption, government spending, and price adjustment costs to equal output
\begin{equation*}
    Y_t = \int c_t (b, z) d \Gamma_t (b, z) + M_t + G_t
\end{equation*}
where $M_t = \int_0^1 m_{j, t} d j$.

Therefore, a solution to the model consists of sequences for prices $\{\pi_t, W_t, \pi_t^W, M_t, D_t, R_t, I_t, \tau_t^L\}_{t = 0}^\infty$, household decision rules $\{b_t, c_t\}_{t = 0}^\infty$ that solve the household utility maximization problem, the distribution of household states $\{\Gamma_t\}_{t = 0}^\infty$ that evolves following the policy rules, and macroeconomic aggregates $\{Y_t, N_t, L_t, B_t, G_t\}_{t = 0}^\infty$ all so that the labor, bond, and goods markets clear subject to exogenous, AR(1) processes for $\{A_t, \psi_t, \psi_t^W, g_t, \xi_t, \tau_t^P, \eta_t\}_{t = 0}^\infty$.


\subsection{Computational Methods}

I solve the model in the sequence-space following \textcite{auclert2021using}. This method has significant computational advantages over standard state-space methods like \textcite{reiter2009solving} or even dimensionality-reduced state-space methods like \textcite{bayer2018solving} since it removes household states, of which there can be thousands, from the system used to solve the model.

The first step to solve the model is to find the steady state. I discretize the household asset and productivity levels into a grid. Household transitions between productivity levels are modeled using a Rouwenhorst process \autocite{kopecky2010finite}. Following \textcite{reiter2009solving}, I add more asset gridpoints closer to the borrowing constraint $\underline{b}$ to address the nonlinearities in the decision rules near that point. I solve for household decision rules using the endogenous grid method \autocite{carroll2006method}. Then, following \textcite{young2010solving}, the distribution $\Gamma_t$ is represented as a histogram at each of the asset-productivity gridpoints, which households travel between based on the savings decision rule.

Shocks are modeled as linear perturbations around the steady state in the sequence space \autocite{auclert2021using}. I use the Python automatic differentiation library Jax to solve for derivatives of the aggregate conditions and the Fake News Algorithm with two-sided numerical differentiation to solve for derivatives of the heterogeneous agent block aggregates \autocites{auclert2021using}. To model the effect of shocks on individual policy rules, I use the disaggregated Fake News derivative and aggregate economic conditions to solve for the linearized effect of the shock on households.

The grid dimensions and sequence space truncation horizon are outlined in Table \ref{tab:comp-pars}. In Appendix \ref{app:comp}, I test the effect of the computational choices on my results, including using a high-dimensional model with substantially more gridpoints and different truncation horizons.

\begin{table}[t]
    \centering
    \caption{Computational Parameters}
    \begin{tabular}{ccl}
    \toprule
    \textbf{Parameter} & \textbf{Value} & \textbf{Description} \\
    \midrule
    $n_b$ & 501 & Number of asset gridpoints \\
    $\underline{b}$ & 0 & Borrowing constraint \\
    $\overline{b}$ & 50 & Maximum asset gridpoint \\
    $n_z$ & 7 & Number of productivity gridpoints \\
    $T$ & 500 & Sequence space perturbation time horizon \\
    \bottomrule
\end{tabular}
    \label{tab:comp-pars}
\end{table}
